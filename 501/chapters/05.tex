\ChapterAndMark{Special and Nexal Negation} 
\label{ch:5}
\is{nexal negation!and special negation|(}
\is{scope of negation|(}
\is{special negation!and nexal negation|(}

The negative notion may belong logically either to one definite idea or to the combination of two ideas (what is here called the nexus).

\is{special negation!defined}
The first, or special, negation may be expressed either by some modification of the word, generally a \is{prefixes!negative|(}prefix, as in\label{sec:neg-prefix}
\is{grammaticalization}  

\phantom{a}

\begin{tabular}{@{}l@{}}
\textit{\emph{n}ever} (etc., see p.~\pageref{para:neveretc})\\

\textit{\emph{un}happy}\\

\textit{\emph{im}possible}, \textit{\emph{in}human}, \textit{\emph{in}competent}\\

\textit{\emph{dis}order}\\

\textit{\emph{non}-belligerent}\\
\end{tabular}

\phantom{a}

\is{adverbs!negative|(}
\noindent (see on these prefixes \chapref{ch:13})\is{prefixes!negative|)}---or else by the addition of \il{English!not@\textit{not}|(}\textit{not} (\textit{not happy}) or \il{English!no@\textit{no}}\textit{no} (\il{English!no longer@\textit{no longer}}\textit{no longer}). Besides there seem to be some words with inherent negative meaning though positive in form: compare pairs like \is{inherently negative meaning}

\phantom{a}

\begin{tabular}{@{}ll@{}}
\textit{absent}& \textit{present}\\

\textit{fail}& \textit{succeed}\\

\textit{lack}& \textit{have}\\

\textit{forget}& \textit{remember}\\

\textit{exclude}& \textit{include}\\
\end{tabular}

\phantom{a}

But though we naturally look upon the former in each of these pairs as the negative (\textit{fail} = `not succeed'), nothing hinders us from logically inverting the order (\textit{succeed} = `not fail'). These words, therefore, cannot properly be classed with such formally negative words as \textit{unhappy}, etc.

A simple example of negatived nexus is \textit{he doesn't come}: it is the combination of the two positive ideas \textit{he} and \textit{coming} which is negatived. If we say \textit{he doesn't come today}, we negative the combination of the two ideas \textit{he} and \textit{coming today}; compare, on the other hand, \textit{he comes, but not today}, where it is only the temporal idea \textit{today} that is negatived.
\is{nexal negation!defined}

Though the distinction between special and nexal negation is clear enough in principle, it is not always easy in practice to distinguish the two kinds, which accounts for some phenomena to be discussed in detail below. In the sentence \textit{he doesn't smoke cigars} it seems natural to speak of a negative nexus, but if we add \textit{only cigarettes}, we see that it is possible to understand it as `he smokes, but not cigars, only cigarettes'.

Similarly, it seems to be of no importance whether we look upon one notion only or the whole nexus as being negatived in \textit{she is not happy} (`she is [positive] not-happy' or `she is not [negative nexus] happy'); %  PE: The (outer) parentheses are ours; the (inner) brackets are OJ's. He used ( ) parentheses. Didn't we agree that parentheses within parentheses would remain ( ) and not become [ ]? ... Though come to think of it, this is an editorial interpolation (by OJ), so [ ] would be more appropriate.
thus also \textit{it is not possible to see it}, etc. In these cases, there is a tendency to attract \textit{not} to the verb: \textit{she isn't happy}, \textit{it isn't possible to see it}, but there is scarcely any difference between these expressions and \textit{she is unhappy}, \textit{it is impossible to see it}, though the latter are somewhat stronger. If, however, we add a subjunct like \textit{very}, we see a great difference between \textit{she isn't very happy} and \textit{she is very unhappy}.

\is{quantifiers!negatived|(}
The nexus is negatived in \refp{ex:05-01}.

\ea \label{ex:05-01}
\textit{Many of us didn't want the war}, but many others did\hfill(news 1917)
\z

\noindent which rejects the combination of the two ideas \textit{many of us} and \textit{want the war} and thus predicates something (though something negative) about \textit{many of us}. But in \textit{Not many of us wanted the war} we have a special negative belonging to \textit{many of us} and making that into \textit{few of us}; and about these it is predicated that they wanted the war. Cf. p.~\pageref{08-not-all}ff (in \chapref{ch:8}) below % PE: OJ simply specifies chapter VIII; no page number
on \textit{not all}, \textit{all {\dots} not}.

Note also the difference between \textit{the disorder was perfect} (\textit{order} negatived) and \textit{the order was not perfect} (nexus negatived, which amounts to the same thing as: \textit{perfect} negatived).

In a sentence like \textit{he won't kill me} it is the nexus (between the subject \textit{he} and the predicate \textit{will kill me}) that is negatived, even though it is possible by laying extra emphasis on one of the words seemingly to negative the corresponding notion; for \textit{\textsc{he} won't kill me} is not `not-he will kill me', nor is \textit{he won't \textsc{kill} me} `he will do the reverse of killing me', etc.\footnote{Jespersen notes in his Addenda that on this page ``or in some other place combinations like \textit{he regretted that \textsc{more} Englishmen did \textsc{not} come here} (news 1917) should have been mentioned''. \eds} % PE: ??? It's not necessary to have both this footnote and the single-sentence paragraph starting "Combinations" that now appears below. Or if simply deleting this footnote is somehow unsatisfactory, then lets simplify it a little and attach it to the single-sentence paragraph.

Cf. also the following passage from \citet[\href{https://archive.org/details/elementarylesson00jevo/page/174/mode/2up?q=\%22curious+to+observe\%22&view=theater}{175}]{jevons1893elementary}:

\begin{quote}
It is curious to observe how many and various may be the meanings attributable to the same sentence according as emphasis is thrown upon one word or another. Thus the sentence ``The study of Logic is not supposed to communicate a knowledge of many useful facts,'' may be made to imply that the study of Logic \textit{does} communicate such a knowledge although it is not supposed to; or that it communicates a knowledge of a \textit{few} useful facts; or that it communicates a knowledge of many \textit{useless} facts. 
\end{quote}

\is{nexal negation!tendency towards}
\is{position of negative}
\label{para:nexal-negation-tendency}There is a general tendency to use nexal negation wherever it is possible (though we shall later on see another tendency that in many cases counteracts this one); and as the (finite) verb is the linguistic bearer of a nexus, at any rate in all complete sentences, we therefore always find a strong tendency to attract the negative to the verb. We see this in the prefixed \textit{ne} in French as well as in Old English, and also in the suffixed \is{grammaticalization}\is{negation, suffixes}\il{English!n't@\textit{-n't}}\textit{-n't} in Modern English, which will be dealt with in \chapref{ch:11}, and in the suffixed \textit{ikke} in modern Norwegian, as in \textit{Er ikke (erke) det fint?} (`Is not (archaic \textit{not}) that nice?') and \textit{Vil-ikke De komme?} (`Will-not you come?'), where Danish has the older word-order \textit{Er det ikke fint?} (`Is it not nice?') and \textit{Vil De ikke komme?} (`Will you not come?').

\is{auxiliary verbs}
In Modern English the use or non-use of the auxiliary \textit{do} serves in many, but not of course in all, cases to distinguish between nexal and special negation; thus we have special negation in \refp{ex:05-02}.\label{p:nexal-sp}

\ea \label{ex:05-02}
He seems \textit{not certain} of his way\hfill(\href{https://archive.org/details/mrswarrensprofes00shawuoft/page/160/mode/2up?q=%22seems+not+certain%22}{Shaw, \textit{Profession} 160})
\z

Combinations like \refp{ex:05-03} should also be mentioned.

\ea \label{ex:05-03}
He regretted that \textit{more} Englishmen \textit{did not} come here\hfill(news 1917) % PE: ??? It's not necessary to have both this little paragraph and the footnote "Jespersen notes in his Addenda that....").
\z\il{English!not@\textit{not}|)}
\is{quantifiers!negatived|)}

\is{partitive constructions|(}
In French we have a distinction which is somewhat analogous to that between nexal and special negation, namely that between \is{grammaticalization}\textit{pas de} (`not any') and \textit{pas du} (`not some'): \textit{je ne bois pas de vin} (`I do not drink (any) wine'); \textit{ceci n'est pas du vin, c'est du vinaigre} (`this is not (some) wine, it's (some) vinegar'), see the full treatment in \citet[\href{https://www.nb.no/items/b6e32092abfe229e8854840d79878e30?page=113}{p.~87ff}]{storm1911storre}. Good examples are found in \refp{ex:05-04}, but see % PE: "but" (by itself) is straight from OJ; "see" is my (feeble) addition, just to make it read less oddly. 
\refp{ex:05-05}. 

\ea \label{ex:05-04}
 \gll ce n' était \textit{plus de} la poésie, ce n' était \textit{pas} \textit{de la} prose, ce était de la poésie, mise en prose\\
 this not was {more of} the poetry this not was not {of the} prose this was of the poetry put into prose\\
 \glt `it was no longer poetry, nor was it prose, but poetry put into prose'\\\hfill(\href{https://www.gutenberg.org/cache/epub/62021/pg62021-images.html}{Rolland, \textit{Buisson} 192})
\z

\ea \label{ex:05-05}
 \gll Il n' y a \textit{pas} \textit{d'} amour, \textit{pas} \textit{de} haine, \textit{pas} \textit{d'} amis, \textit{pas} \textit{d'} ennemis, \textit{pas} \textit{de} foi, \textit{pas} \textit{de} passion, \textit{pas} \textit{de} bien, \textit{pas} \textit{de} mal.\\
 it not there has not of love not of hate not of friends not of enemies not of faith not of passion not of good not of evil\\
 \glt `There is no love, no hate, no friends, no enemies, no faith, no passion, no good, no evil.'\hfill(\href{https://www.gutenberg.org/cache/epub/62021/pg62021-images.html}{ibid 197})
\z

With the partitive force of \textit{pas} with \textit{de} should be compared the well-known use of the genitive for the object in Russian negative sentences and with \textit{nět} (`there is not'), etc., also the use of the partitive case for the subject of a negative sentence in Finnish. % ??? PE: I've moved this down from its previous position, in front of the last two quotations. 
\is{partitive constructions|)}

In the case of a contrast we have a special negation; hence the separation of \textit{is} (with comparatively strong stress) and \il{English!not@\textit{not}|(}\textit{not} in \refp{ex:05-06}. \il{English!do@\textit{do}|(}\textit{Do} is not used in such sentences as \refp{ex:05-07}.\is{auxiliary verbs|(}

\ea \label{ex:05-06}
the remedy is, not to remand him into his dungeon, but to accustom him to the rays of the sun\hfill(\href{https://archive.org/details/essaysonmiltona05macagoog/page/n128/mode/2up?view=theater&q=remand}{Macaulay, \textit{Milton} 1.41})
\z

\ea \label{ex:05-07}
\ea
I came not to send peace, but a sword\hfill(\href{https://www.kingjamesbibleonline.org/1611_Matthew-10-34/}{AV \textit{Matthew} 10.34})
\ex
my ruin came not from too great individualism of life, but from too little\hfill(\href{https://archive.org/details/deprofundiswilde00wildiala/page/104/mode/2up?q=%22great+individualism+of+life%22&view=theater}{Wilde, \textit{Profundis} 135})
\ex
We meet not in drawing-rooms, but in the hunting-field\\\hfill(\href{https://www.gutenberg.org/files/30432/30432-h/30432-h.htm}{Dickinson, \textit{Symposium} 14})
\z
\z

Even in such contrasted statements, however, the negative is very often attracted to the verb, which then takes \textit{do}, the latter part being then equivalent to \textit{but we meet in the hunting-field} \refp{ex:05-08}.\is{auxiliary verbs|)}

\ea \label{ex:05-08}
we do not meet in the drawing-room, but in the hunting-field 
\z

\ea \label{ex:05-09}
\ea
I do not complain of your words, but of the tone in which they were uttered
\ex
I do not admire her face, but [I do admire] her voice
\ex
He didn't say that it was a shame, but that it was a pity
\ex
I did not come to curse thee, Guinevere\\\hfill(\href{https://en.wikisource.org/wiki/Idylls_of_the_King/Guinevere}{Tennyson, \textit{Guinevere}}; contrast not expressed)
\z
\z\il{English!do@\textit{do}|)}\il{English!not@\textit{not}|)}

In such cases, the Old English verb naturally had no \il{English!Old English!ne@\textit{ne}|(}\textit{ne} before it, see e.g. \refp{ex:05-15}. The exception in \refp{ex:05-18} may be accounted for by the Latin word-order \textit{non veni pacem mittere, sed gladium}. But in Ælfric we have \refp{ex:05-19}, where the meaning is `it happened not-unprovidentially', as shown by the indicative \textit{wæs} and by the necessity of the repetition \textit{hit getimode}. Cf. also the Middle English version: % PE: As is, this reads strangely. Can we simply skip "edited by Paues 56:"?
%Brett: sure % PE Done.
\refp{ex:05-20}.

\ea \label{ex:05-15}
\ea\il{English!Old English!nalles@\textit{nalles}}
\gll wen ic þæt ge for wlenco nalles for wræcsiðum ac for higeþrymmum Hroðgar \textit{sohton}\\
 expect I that you for pride {not at all} for {misery} but for {high spirits} Hrothgar {sought}\\
\glt `I expect that you did not seek Hrothgar out of dire straits but out of boldness and strength of heart'\hfill(\href{http://ebeowulf.uky.edu/ebeo4.0/CD/main.html}{\textit{Beowulf} 338})
\ex
\gll ðæt he nalæs to idelnesse, swa sume oðre, ac to gewinne, in ðæt mynster \textit{eode}\\
 that he not to idleness as some others but to labour into that monastery went\\
\glt `that he did not go into the monastery for idleness, as some others, but to labour'\hfill(\href{https://archive.org/details/oldenglishversio02bede/page/264/mode/2up?q=%22+pcet+he+nales+to+idelnesse%2C+swa+aume+o%27Sre%22&view=theater}{\textit{Bede} 4.3}) % Linked-to version uses þ rather than ð, and also uses macrons (or what look like macrons). What to do?
%Brett: The old link was https://archive.org/details/anglosaxonreader00wyatuoft/page/52/mode/2up?q=%22to+idelnesse%22&view=theater. The new one still has þ, but at least it doesn't have the diacritics. % Peter: Thank you. This is good enough, I think.
\ex
\gll ðe ic lufode na for galnesse ac for wisdome\\
 whom I loved not for wantonness but for wisdom\\
\glt `whom I did not love for lust but for wisdom'\hfill(\href{https://archive.org/details/anglosaxonversi00thorgoog/page/n34/mode/2up?q=%22for+galnesse%22&view=theater}{\textit{Apollonius} 255})
\z
\z

\ea \label{ex:05-18}
\gll ne com ic sybbe to sendanne, ac swurd\\
 not come I peace to send but sword\\
\glt `I did not come to send peace, but a sword'\hfill(\href{https://books.google.co.jp/books?id=twINAAAAIAAJ&newbks=1&newbks_redir=0&printsec=frontcover&pg=PA48&dq=%22ic+sybbe+to+sendanne%22&hl=en&redir_esc=y#v=onepage&q=%22ic%20sybbe%20to%20sendanne%22&f=false}{WG \textit{Matthew} 10.34})
\z

\ea \label{ex:05-19}
\gll Ne getimode þam apostole Thome unforsceawodlice, þæt he ungleafful wæs Cristes æristes, ac hit getimode þurh Godes forsceawunge\\ % PE: Restored "Cristes æristes", whose meaning I'd GUESS is "Christ's resurrection" (see https://bosworthtoller.com/790 )
%Brett: looks good
 not happened {to the} apostle Thomas {by chance} that he unbelieving was Christ's resurrection but it happened through God's providence\\
\glt `It did not happen to the apostle Thomas by chance that he doubted Christ's resurrection, but it occurred through God's providence'\\\hfill(\href{https://archive.org/details/homiliesanglosa00thorgoog/page/234/mode/2up?q=%22apostole+Thome+unforsceawodlice%22&view=theater}{Ælfric, \textit{Homilies} 1.234}) % Peter: Can we change "was unbelieving in" to "doubted"?
%Brett: done
\z\il{English!Old English!ne@\textit{ne}|)}

\ea \label{ex:05-20}\il{English!Middle English!ne@\textit{ne}}\il{English!Middle English!noȝt@\textit{noȝt}}
\gll For Crist ne sende noȝt me {for to} baptyze, bote {for-to} preche þe gospel\\
 for Christ not sent not me to baptize but to preach the gospel\\
\glt `For Christ sent me not to baptize, but to preach the Gospel'\\\hfill(\href{https://archive.org/details/fourteenthcentur00pauerich/fourteenthcentur00pauerich/page/56/mode/2up?q=%22me+for+to+baptyze%22&view=theater}{MEV \textit{1 Corinthians} 1.17})
\z

\il{English!not@\textit{not}|(}Other examples of constructions in which \textit{not} is referred to the verb instead of some other word \refp{ex:05-21}.%Brett: OJ has single quotes

\ea \label{ex:05-21}
\ea
I did not step into the well-known boat Without a cordial greeting (`I stepped {\dots} not without')\\\hfill(\href{https://en.wikisource.org/wiki/The_Prelude_(Wordsworth)/Book_IV}{Wordsworth, \textit{Prelude} 4.16})
\ex
Don't pay only the arrears, pay all you can. (`Pay, not only')\hfill(\href{https://archive.org/details/quisantanovel00hopegoog/page/n144/mode/2up?q=%22pay+only+the+arrears%22&view=theater}{Hope, \textit{Quisanté} 132})
\ex
it doesn't only concern myself\hfill(\href{https://archive.org/details/freelands00galsrich/page/288/mode/2up?q=%22concern+myself%22&view=theater}{Galsworthy, \textit{Freelands} 332})
\z
\z

Note also \refp{ex:05-24}, where the sentence \textit{we aren't here} in itself is a contradiction in terms. (Differently in \refp{ex:05-25}, where \textit{not} belongs more closely to what follows.)

\ea 
\ea \label{ex:05-24}
We aren't here to talk nonsense, but to act
\ex \label{ex:05-25}
We are here, not to retire till compelled to do so 
\z
\z

\is{auxiliary verbs}
When the negation is attracted to the verb (in the form \textit{n't}), it occasions a cleaving of \il{English!never@\textit{never}}\textit{never}, \textit{ever} thus standing by itself. In writing the verbal form is sometimes separated in an unnatural way: (\ref{ex:05-26}, representing the spoken \textit{Can't she ever~{\dots}}); and thus we get seemingly \il{English!not ever@\textit{not ever}|(}\textit{not ever} (`never', \ref{ex:05-27}, different from the old \textit{not ever} as in \href{https://archive.org/details/utopiasirthomas00robigoog/page/n355/mode/2up?q=%22not+euer%22&view=theater}{More, \textit{Utopia} 244}, which meant `not always'). Compare the rare \il{English!not any@\textit{not any}}\textit{not any} as in \refp{ex:05-36}.

\ea \label{ex:05-26}
\textit{Can she not ever} write herself?\hfill(\href{https://archive.org/details/alfredlordtenny05tenngoog/page/n250/mode/2up?q=%22can+she+not+ever%22&view=theater}{Hallam, letter})
\z

\ea \label{ex:05-27}
\ea
You shan't \textit{touch} those hostels ever again. Ever.\hfill(\href{https://archive.org/details/wifeofsirisaacha00well/page/422/mode/2up?view=theater&q=%22touch+those+hostels%22}{Wells, \textit{Wife} 422})
\ex
I suppose you don't ever write to him?\hfill(\href{https://archive.org/details/dollydialogues00hope_0/page/62/mode/2up?view=theater&q=%22ever+write+to+him%22}{Hope, \textit{Dialogues} 40})
\ex
I can't ever see that man again.\hfill(\href{https://archive.org/details/marriageofwillia0000mrsh_i0u5/page/284/mode/2up?q=%22can%27t+ever+see+that+man+again%22&view=theater}{Ward, \textit{Marriage} 242})
\ex
Don't you ever go down beneath the surface of things?\\\hfill(\href{https://archive.org/details/septimus00unkngoog/page/n253/mode/2up?q=%22you+ever+go+down%22&view=theater}{Locke, \textit{Septimus} 26})
\ex
so don't you ever be troubled about that\hfill(\href{https://archive.org/details/prodigalson00caingoog/page/n222/mode/2up?view=theater&q=%22don%27t+you+ever+be+troubled%22}{Caine, \textit{Prodigal} 219})
\z
\z

\ea \label{ex:05-27a}
\ea
let not euer The soule of Nero enter this firme bosome\\\hfill(\href{https://internetshakespeare.uvic.ca/doc/Ham_F1/scene/3.2/index.html#tln-2260}{Shakespeare, \textit{Hml} 3.2.411})
\ex
A light around my steps which would not ever fade\\\hfill(\href{https://archive.org/details/completepoeticalshel/page/78/mode/2up?view=theater&q=%22light+around+my+steps%22}{Shelley, \textit{Revolt} 4.34})
\ex
Do you not ever go?\hfill(\href{https://archive.org/details/dukeschildrennov00troluoft/page/172/mode/2up?q=%22do+you+not+ever%22&view=theater}{Trollope, \textit{Children} 2.40})
\ex
you shall not---not ever\hfill(\href{https://archive.org/details/widowershousesun00shaw/page/42/mode/2up?q=%22you+shall+not%22&view=theater}{Shaw, \textit{Houses} 40})
\z
\z\il{English!not ever@\textit{not ever}|)}

\ea \label{ex:05-36}\il{English!no@\textit{no}}
``Had any gentleman heard of a dauphin killed by small-pox?'' No; \il{English!not any@\textit{not any}}\textit{not any} gentleman \textit{had} heard of such a case.\hfill(\href{https://archive.org/details/miscellaneousess00dequuoft/page/78/mode/2up?q=%22heard+of+a+dauphin%22&view=theater}{Quincey, \textit{Murder}}) % PE: Quincey italicizes "had"; OJ (among his Addenda) italicizes both "not any" and "had"
\z

\is{scope of negation!ambiguity of}
\is{scope of negation!cause or reason and|(}
A special case of frequent occurrence is the rejection of something as the cause of or reason for something real, expressed in a negative form: `he is happy, not on account of his riches, but on account of his good health' expressed in this form \textit{he is not (isn't) happy on account of his riches, but on account of his good health}. It will easily be seen that \textit{I didn't go because I was afraid} is ambiguous (`I went and was not afraid', or, `I did not go, and was afraid'), and sentences like this are generally avoided by good stylists. In \refp{ex:05-37}, the clause gives the reason for the speaker not wanting to be patronized. Similarly \refp{ex:05-38}.\is{ambiguity!related to reason}

\ea \label{ex:05-37}
Don't patronize \textit{me}, Ma, because I can take care of myself \\\hfill(\href{https://archive.org/details/ourmutualfriendc0000char/page/258/mode/2up?q=%22patronise+me%22&view=theater}{Dickens, \textit{Friend} 348}) % In this edition at least, "patronise" has an "s".
\ex \label{ex:05-38}
I have not drunk deep of life because I have been unathirst\\\hfill(\href{https://archive.org/details/bwb_P8-BLX-259/page/150/mode/2up?view=theater&q=unathirst}{Locke, \textit{Morals} 151})
\z

\is{scope of negation!intonation and}
In the spoken language a distinction will usually be made between the two kinds of sentences by the tone, which rises on \textit{call} in \textit{I didn't call because I wanted to see her} (but for some other reason), while it falls on \textit{call} in \textit{I didn't call because I wanted to avoid her} (the reason for not calling). In (\ref{ex:05-39} \& \ref{ex:05-40}), the clause indicates the reason for the prohibition.

\ea \label{ex:05-39}
You mustn't come whining back to me, because I won't have you\\\hfill(\href{https://archive.org/details/runningwater00masouoft/page/104/mode/2up?q=%22whining+back+to+me%22&view=theater}{Mason, \textit{Water} 95})
\ex\label{ex:05-40}
We have not gagged our Press because we disliked our freedom, but because to this extent the Prussian has triumphed\hfill(\textit{Parable}) % From Addenda.
\z\il{English!not@\textit{not}|)}
\is{adverbs!negative|)}
\is{scope of negation!cause or reason and|)}

In other languages we have corresponding phenomena. G. Brandes's \refp{ex:05-41} is ambiguous; and when Ernst Møller writes \refp{ex:05-42}, I suppose that most readers will misunderstand it as if \textit{opløses} were to be taken in a positive sense; it would have been made clearer by a transposition: \refp{ex:05-43}. Also, \refp{ex:05-44}.

\ea \label{ex:05-41}
\gll [Napoleon] handlede ikke saadan, fordi han trængte til sine generaler\\
[Napoleon] acted not thus because he needed to his generals\\
\glt `Napoleon did not act like this, as he needed his generals'\\
`Napoleon did not act thus because he needed his generals [but for some other reason]'\hfill(\href{https://archive.org/details/napoleonoggarib00brangoog/page/n33/mode/2up?q=%22handlede+ikke+saadan%22&view=theater}{\textit{Napoleon} 52}) % PE: OJ cites Brandes writing in Tilskueren. This is a journal; in March '24, the issue was not online at kb.dk. In the linked-to book, Brandes writes not "Napoleon" but "Han" (i.e. "He").
%% SG: The second alternative is not a possible translation of the Danish sentence. The two possible readings are: (a) 'Napoleon did not act like this, as he needed his generals'; and (b) 'Napoleon did not act thus because he needed his generals [but for some other reason]'

\ex \label{ex:05-42}
\gll Men retningens magt opløses, som alt fremhævet, ikke fordi dens argumenter og læresætninger eftergås og optrævles; dens magt vil blive stående\\
 but movement.\DEF.\POSS{} power {is dissolved} as already emphasized not because its arguments and doctrines {are examined} and {are unraveled} its power will remain standing\\
\glt 
`But, as already emphasized, the power of the movement will not be dissolved because its arguments and doctrines are examined and unraveled; its power will continue to stand'\\\hfill(\textit{Inderstyre} 249, in speaking of ``Christian Science'') % ??S When checked in March '24 and again in September '24, not online at kb.dk or archive.org. Is there a copy somewhere that we can link to?
\ex \label{ex:05-43}
\gll Men som alt fremhævet opløses retningens magt ikke~{\dots}\\
 but as already emphasized {is dissolved} movement.\DEF.\POSS{} power not\\
\glt `But, as already emphasized, the power of the movement is not dissolved~{\dots}'
\ex \label{ex:05-44}
\gll Jeg elsker ikke mit sprog, fordi det er eller har været herligt og {skjønt {\dots}} jeg elsker det, fordi det er mine fædres og mit folks sprog\\
 I love not my language because it is or has been glorious and beautiful I love it because it is my fathers.\POSS{} and my people.\POSS{} language\\
\glt `I do not love my language because it is or has been glorious and beautiful {\dots} I love it because it is the language of my ancestors and my people'
\hfill(\href{https://books.google.com/books?id=XAJJAQAAMAAJ&pg=RA3-PA90&lpg=RA3-PA90&dq=madvig+%22Jeg+elsker+ikke+mit+sprog%22&source=bl&ots=rZlO8Lq4of&sig=ACfU3U0BK7Im87JxQP6To6D2Ey22Jl04eg&hl=en&sa=X&ved=2ahUKEwjCn5X5gZOFAxV4nK8BHWEgCx8Q6AF6BAgIEAM#v=onepage&q=madvig%20%22Jeg%20elsker%20ikke%20mit%20sprog%22&f=false}{Madvig, \textit{Kjönnet} 90})% From Addenda. % OJ attributes this to "Madvig Program 1857. 90." Perhaps a mistake? Anyway, it's on p. 90 of Om Kjönnet i Sprogene.
\z

\phantomsection \label{p:48}\il{English!not@\textit{not}|(}
\is{auxiliary verbs|(}
\is{adverbs!negative|(}
Not unfrequently \textit{not} is attracted to the verb in such a way that an adverb, which belongs to the whole proposition, is more or less awkwardly placed between words which should not properly be separated, as in \refp{ex:05-45}. The tendency to draw the auxiliary and \textit{not} together has, on the other hand, been resisted in % ??? PE: Originally: "on the other hand, been resisted in the following passages"; immediately followed by six quotations. (Yes, "passages" is a bit of a stretch.) But in view of the new arrangement, "the following passages" seems superfluous at best. If we keep it, then let's move "The tendency to draw ... the following word" downwards, so that it immediately precedes "You will of course not meet him..." 
\refp{ex:05-49}. In most of these, \textit{not} evidently is a special negative, belonging to the following word.

\ea \label{ex:05-45}
\ea
you \textit{are not probably} aware {\dots}\\(`probably you are not aware', or: `you are probably not aware')\\\hfill(\href{https://archive.org/details/dukeschildrennov00troluoft/page/38/mode/2up?q=%22are+not+probably+aware%22&view=theater}{Trollope, \textit{Children} 1.76})
\ex
were he at that moment Home Secretary and in the cabinet, he \textit{would not probably} be reading it\hfill(\href{https://archive.org/details/marriageofwillia0000mrsh_i0u5/page/268/mode/2up?q=%22were+he+at+that+moment+Home+Secretary%22&view=theater}{Ward, \textit{Marriage} 228})
\ex
Edward Manisty, however, \textit{was not apparently} consoled by her remarks\hfill(\href{https://archive.org/details/cu31924013567130/page/2/mode/2up?q=%22not+apparently+consoled%22&view=theater}{Ward, \textit{Eleanor} 2}) % OJ has "M."; this is restored to "Manisty"
\ex
This is a strong expression. Yet it \textit{is not perhaps} exaggerated.\\\hfill(news 1917)
\z
\z
\is{adverbs!negative|)}

\ea \label{ex:05-49}
\ea
You \textit{will of course not} meet him until he has spoken to me\\\hfill(\href{https://archive.org/details/widowershousesun00shaw/page/28/mode/2up?q=%22you+will+of+course%22&view=theater}{Shaw, \textit{Houses} 27})
\ex
he \textit{is clearly not} a prosperous man\hfill(\href{https://archive.org/details/doctorsdilemmatr00shawuoft/page/20/mode/2up?q=%22prosperous+man%22&view=theater}{Shaw, \textit{Dilemma} 21})
\ex
they \textit{had clearly not} been unfavourable to him\hfill(\href{https://archive.org/details/strangeadventure00blac/page/268/mode/2up?view=theater&q=%22unfavorable+to+him%22}{Black, \textit{Phaeton} 280}) % The edition linked to actually spells it "unfavorable", but presumably British editions spell it "unfavourable".
\ex
a fashionable music-master, whose blood \textit{was certainly not} Christian\\\hfill(\href{https://archive.org/details/marriageofwillia0000mrsh_i0u5/page/154/mode/2up?q=%22whose+blood+was%22&view=theater}{Ward, \textit{Marriage} 133}) % "fashionable" restored
\ex
It'\textit{s simply not} fair to other people \phantom{x} (`is simply unfair')\\\hfill(\href{https://archive.org/details/silverboxcomedyi00gals/page/54/mode/2up?q=%22simply+not+fair+to+other+people%22&view=theater}{Galsworthy, \textit{Box} 55})
\ex
the smashing up of the Burnet family {\dots} \textit{was disagreeably not} in the picture of these suppositions\hfill(\href{https://archive.org/details/wifeofsirisaacha00well/page/120/mode/2up?view=theater&q=%22smashing+up+of%22}{Wells, \textit{Wife} 120}) % OJ had removed "by the International Stores"
\z
\z

\is{redundancy of expression|(}
It has sometimes been said that the combination \textit{he cannot possibly come} is illogical; \textit{not} is here taken to the verb \textit{can}, while in Danish and German the negative is referred to \textit{possibly}: \refp{ex:05-55}. There is nothing illogical in either expression, but only redundance: % PE: OJ wrote "redundance", not "redundancy". Uncommon a century ago and rare now, "redundance" seems a legit word.
the notion of possibility is expressed twice, in the verb and in the adverb, and it is immaterial to which of these the negative notion is attached.

\ea \label{ex:05-55}
\ea
\gll han kan umuligt komme [Danish]\\
 he can impossibly come\\
\glt `he can't possibly come'
\ex
\gll er kann unmöglich kommen [German]\\
 he can impossibly come\\
\glt `he can't possibly come'
\z
\z
 % ??? PE Pretty sure that this pair should either (A) be glossed, etc, and labelled as Danish and German respectively, or (B) moved back into the paragraph. The latter is the less cumbersome.
%% SG: Shall I add glosses?
\is{redundancy of expression|)}

\is{adverbs!negative|(}
When \textit{not} is taken with some special word, it becomes possible to use the adverb \textit{still}, which is only found in positive sentences: \refp{ex:05-56} is different from \textit{the officers were not yet friendly} (\textit{not yet} nexal negative) insofar as the latter presupposes a change having occurred after that time, which the former does not. Cf. also \refp{ex:05-57}.

\ea \label{ex:05-56}
The officers were still not friendly\hfill(news 1917) 
\z

\ea \label{ex:05-57}
\ea
Although I wrote to him a fortnight ago, I have still not heard from him\hfill(letter 1899)
\ex \il{English!no@\textit{no}}my head is still in no good order \phantom{x}(`is still bad', slightly different from `is not yet well')\hfill(\href{https://archive.org/details/journaltostellae00swifuoft/page/502/mode/2up?q=%22no+good+order%22&view=theater}{J. Swift, \textit{Journal} 503})
\z
\z

\textit{Yet not} is rare: \refp{ex:05-59}.

\ea \label{ex:05-59}
Pekuah was yet not satisfied {\dots}\hfill(\href{https://archive.org/details/historyrasselas01johngoog/page/n117/mode/2up?q=%22was+yet+not+satisfied%22&view=theater}{Johnson, \textit{Rasselas} 112}) % "P." expanded to "Pekuah"
\z
\is{auxiliary verbs|)}

\il{English!not a/one@\textit{not a}/\textit{one}|(}
\label{para:kind-of-stronger-no}\textit{Not a} or \textit{not one} before a substantive (very often \textit{word}) is a kind of stronger \il{English!no@\textit{no}}\textit{no}; at any rate, the two words may be treated as belonging closely together, i.e. as an instance of special negative, the verb consequently taking no auxiliary \textit{do}; cf. \citet[\href{https://archive.org/details/jespersen-1954-a-modern-english-grammar-on-historical-principles-part-ii-syntax-first-volume/page/426/mode/2up?q=\%22not+one+word\%22&view=theater}{16.73}]{jespersenMEG2}, where many examples are given; see further \refp{ex:05-60}.

\ea \label{ex:05-60}
\ea
Say not a word of it\hfill(\href{https://archive.org/details/mansfieldpark00aust_1/page/364/mode/2up?q=%22say+not+a+word%22&view=theater}{Austen, \textit{Mansfield} 395})
\ex
The Face seemed to smile, but answered not a word\\\hfill(\href{https://archive.org/details/snowimageothertw0000hawt/page/44/mode/2up?q=%22face+seemed+to+smile%22&view=theater}{Hawthorne, \textit{Image} 46}) % "The" and "Face" both capitalized, as in the book
\ex
he mentioned not a word\hfill(\href{https://archive.org/details/returnofthenativ00harduoft/page/270/mode/2up?q=%22he+mentioned+not%22&view=theater}{Hardy, \textit{Return} 356})
\ex
she said not a word about their interview\hfill(\href{https://archive.org/details/grandbabylonhote00bennuoft/page/82/mode/2up?view=theater&q=%22said+not+a+word%22}{Bennett, \textit{Babylon} 66}) % OJ has "that", but the book linked to has "their"
\ex
he lost not an hour in breaking entirely with the murderer\\\hfill(\href{https://archive.org/details/returnofsherlock0000acon/page/152/mode/2up?view=theater&q=%22lost+not+an+hour%22}{Doyle, \textit{Return} 5.230}) % OJ omits "entirely"
\z
\z\il{English!not a/one@\textit{not a}/\textit{one}|)}

\il{English!not the least\textit{/}the slightest@\textit{not the least}/\textit{the slightest}|(}
In a similar way \textit{not} is attracted to \textit{the least}, \textit{the slightest}, and in recent usage \textit{at all}, as shown by the absence of the auxiliary \textit{do} \refp{ex:05-65}. Cf. \refp{ex:05-71}.

\ea \label{ex:05-65}
\ea
his Majesty took not the least notice of us\hfill(\href{https://archive.org/details/bim_eighteenth-century_the-works-of-j-s-dd-_swift-jonathan_1735_3/page/200/mode/2up?view=theater&q=%22Maje%C5%BFty+took+not+the+lea%C5%BFt+Notice%22}{J. Swift, \textit{Travels} 200}) % "Notice" capitalized in the original
\ex
my resignation of the wardenship need offer not the slightest bar to its occupation by another person\hfill(\href{https://archive.org/details/warden0000anth_w6p5/page/228/mode/2up?q=%22need+offer+not%22&view=theater}{Trollope, \textit{Warden} 243})
\ex
He rested but two hours and slept not at all\hfill(\href{https://archive.org/details/themother00phil/page/350/mode/2up?q=%22but+two+hours%22&view=theater}{Phillpotts, \textit{Mother} 350}) % Starts the sentence
\ex
an urgency that helped him not at all\hfill(\href{https://archive.org/details/loveandmrlewisha00welluoft/page/64/mode/2up?view=theater&q=%22urgency+that+helped%22}{Wells, \textit{Love} 65})

\ex
this explanation enlightened the Commandant not at all\\\hfill(\href{https://archive.org/details/majorvigoureux00quil/page/58/mode/2up?q=%22enlightened%22&view=theater}{Quiller-Couch, \textit{Major} 59})
\il{English!not the least\textit{/}the slightest@\textit{not the least}/\textit{the slightest}|)}
\ex
they talked not at all for a long time\hfill(\href{https://archive.org/details/freelands00galsrich/page/184/mode/2up?q=%22not+at+all+for+a+long+time%22&view=theater}{Galsworthy, \textit{Freelands} 209})
\z
\z

\ea \label{ex:05-71}
he {\dots} cared not the snap of one of his thin, yellow fingers\hfill(\href{https://archive.org/details/freelands00galsrich/page/362/mode/2up?q=%22cared+not+the+snap%22&view=theater}{ibid 415}) % OJ deleted " had done it hundreds of times before and"
\z

\is{auxiliary verbs|(}
Where we have a verb connected with an infinitive, it is often of great importance whether the negation refers to the nexus (main verb) or to the infinitive. In the earlier stages of the language, this was not always clear: \textit{he tried not to look that way} was ambiguous; now the introduction of \il{English!do@\textit{do}|(}\textit{do} as the auxiliary of a negative nexus has rendered a differentiation possible: \textit{he did not try to look that way}; \textit{he tried not to look that way}; and the (not yet recognized) placing of \textit{not} after \textit{to} serves to make the latter sentence even more unambiguous: \textit{he tried to not look that way}. The distinction is clear in \refp{ex:05-72}.

\ea \label{ex:05-72}
She \textit{did not wish} to reflect; she strongly \textit{wished not to} reflect\\\hfill(\href{https://archive.org/details/cu31924013586940/page/470/mode/2up?q=%22did+not+wish+to+reflect%22&view=theater}{Bennett, \textit{Wives} 2.187})
\z\il{English!do@\textit{do}|)}\is{auxiliary verbs|)}

Other examples with \textit{not} belonging to an infinitive: \refp{ex:05-73}--\refp{ex:05-73c}.

\ea \label{ex:05-73}
\ea
\textit{Try not to do} it again\hfill(\href{https://archive.org/details/personalhistory05dickgoog/page/n53/mode/2up?q=%22try+not+to+do+it%22&view=theater}{Dickens, \textit{David} 112})
\ex
\textit{Try not to associate} bodily defects with mental\hfill(\href{https://archive.org/details/personalhistory05dickgoog/page/n191/mode/2up?q=%22bodily%22&view=theater}{ibid 432})
\ex
the more he \textit{endeavoured not to think}, the more he thought\\\hfill(\href{https://archive.org/details/christmascarol0000char_h5c8/page/40/mode/2up?q=%22endeavored%22&view=theater}{Dickens, \textit{Carol} 20}) % Edition linked to has "endeavored"; I (PE) presume that this was exclusively for the US market.
\ex
the fool {\dots} who \textit{resolved not to go} into the water till he had learnt to swim\hfill(\href{https://archive.org/details/essaysonmiltona05macagoog/page/n128/mode/2up?view=theater&q=resolved}{Macaulay, \textit{Milton} 1.41}) % "in the old story" has been cut
\ex
Tommy \textit{deserved not to be} hated\hfill(\href{https://archive.org/details/intrusionspeggy02hopegoog/page/n46/mode/2up?q=%22deserved+not+to+be%22&view=theater}{Hope, \textit{Intrusions} 38})
\ex
if one were to live always among those bright colours, one would \textit{get not to see} them\hfill(\href{https://archive.org/details/strangeadventure00blac/page/58/mode/2up?view=theater&q=%22among+those+bright+colors%22}{Black, \textit{Phaeton} 61}) % The edition linked to actually spells it "colors", but presumably British editions spell it "colours".
\ex
I soon \textit{got not to care}\hfill(\href{https://archive.org/details/in.ernet.dli.2015.260707/page/n79/mode/2up?q=%22soon+got+not+to+care%22&view=theater}{Galsworthy, \textit{Justice} 91})
\ex
I may \textit{come not to feel} such unbearable shame as I do now\\\hfill(\href{https://archive.org/details/lovescrosscurren00swinuoft/page/142/mode/2up?q=%22may+come+not+to+feel%22&view=theater}{Swinburne, \textit{Cross-currents} 158})
\ex
I knew he'd \textit{come not to care} about the book-selling\\\hfill(\href{https://archive.org/details/historydavidgri02wardgoog/page/458/mode/2up?q=%22come+not+to+care%22&view=theater}{Ward, \textit{David} 3.132})
\z
\z

\ea \label{ex:05-73a}
\ea
I beseech you before you go, not perhaps to return, once more to let me press the hand\hfill(\href{https://archive.org/details/vanityfairanove03thacgoog/page/n129/mode/2up?q=%22beseech+you+before%22&view=theater}{Thackeray, \textit{Vanity} 200})
\ex
the Prime Minister himself was personally too much absorbed in the zeal of his cause not sometimes to run counter to the feelings {\dots} of men less earnest\hfill(\href{https://books.google.co.jp/books?id=9PjnkXA5jOkC&pg=PP5&dq=%22justin+mccarthy%22+%22our+own+times%22&hl=en&newbks=1&newbks_redir=0&sa=X&ved=2ahUKEwiE2YnrmaCFAxVsoa8BHcesDZwQ6AF6BAgKEAI#v=onepage&q=%22much%20absorbed%22&f=false}{McCarthy, \textit{History} 2.521}) % Replaced an omission with dots, restored "personally" and "less earnest", etc. According to OJ, MacCarthy [sic] writes: "the Prime-minister was too much absorbed in the zeal of his cause not sometimes to run counter to the feelings of men".
\z
\z

\ea \label{ex:05-73b}
I wished to not treat you to more tears\hfill(\href{https://digital.library.upenn.edu/women/carlyle/jwclam/lam301.html#24}{J. Carlyle, \textit{Letters} 3.24})
\z

\ea \label{ex:05-73c}
``I might not have gone,'' I mused. ``I might easily not have gone.''\\\hfill(\href{https://archive.org/details/dollydialogues00hope_0/page/152/mode/2up?view=theater&q=%22might+not+have+gone%22}{Hope, \textit{Dialogues} 94}; cf. p.~\pageref{p:48} \hyperref[p:48]{above} % PE: Let's add page number
%Brett: go for it  % PE: Done
and p.~\pageref{must-and-may}ff (\chapref{ch:8}) below) % PE: Restoring "I mused". OJ merely says chapter VIII, not the page number(s) within this.
\z

When \textit{do} cannot be used, it is not always easy to see whether \textit{not} belongs to the main verb or the infinitive, as in \refp{ex:05-86},\footnote{Jespersen's Addenda include the example \textit{Sylvia was determined \textsc{not to be} disappointed} (\href{https://archive.org/details/runningwater00masouoft/page/114/mode/2up?view=theater}{Mason, \textit{Water} 104}). \eds} % Want to insert \href{https://archive.org/details/runningwater00masouoft/page/114/mode/2up?view=theater&q=%22Sylvia+was+determined%22}{Mason, \textit{Water} 104} within that pair of parentheses, but doing so triggers errors
%Brett: fixed by removing text search/highlighting. % Peter: Strange! I must try to remember this. 
where, however, the next line shows that what is meant is `it was not my purpose to have seen you here', and not `it was my purpose not to have {\dots}'. This paraphrase further serves to show that in some cases word-order may remove any doubt as to the belonging of the negative, thus very often with a predicative; cf. also such frequent cases as \refp{ex:05-87}. And in the spoken language the use of \textit{wasn't} [wɔznt] in one case, and unstressed \textit{was} [wəz] followed by a strongly stressed \textit{not} in the other, will at once make the meaning clear of such sentences as the one first quoted here.\is{stress!effect on negation of}

\ea \label{ex:05-86}
My purpose was not to haue seene you heere\\\hfill(\href{https://internetshakespeare.uvic.ca/doc/MV_F1/scene/3.2/index.html#tln-1575}{Shakespeare, \textit{Merch} 3.2.230}) % "seene" with three E s
\z

\ea \label{ex:05-87}
He was beginning not to despise the day of small things\\\hfill(\href{https://archive.org/details/septimus00unkngoog/page/n221/mode/2up?q=%22was+beginning+not%22&view=theater}{Locke, \textit{Septimus} 232})
\z

\textit{Don't let us} is the idiomatic expression, where logically it would be preferable to say \textit{let us} with \textit{not} to the infinitive (an injunction not to {\dots}): \refp{ex:05-88}.

\ea \label{ex:05-88}
Do not let us, however, be too prodigal of our pity upon Pegasus\\\hfill(\href{https://archive.org/details/dli.ministry.14127/page/343/mode/2up?q=%22be+too+prodigal%22&view=theater}{Thackeray, \textit{Pendennis} 2.213}) % "upon Pegasus" restored
\z

In the old construction without \textit{do} we see the same attraction of \textit{not} to \textit{let}: (\ref{ex:05-89}, though the last two quotations show \textit{not} placed with the infinitive).

\ea \label{ex:05-89}
\ea
let not vs rent it\hfill(\href{https://www.kingjamesbibleonline.org/1611_John-19-24/}{AV \textit{John} 19.24})
\ex
let not my behaviour seem rude\hfill(\href{https://archive.org/details/bim_eighteenth-century_epicne-or-the-silent-_jonson-ben_1776/page/n29/mode/2up?q=%22behaviour%22&view=theater}{Jonson, \textit{Epicœne} 3.183})
\ex
let not the prospect of worldly lucre carry us beyond your judgment\\\hfill(\href{https://archive.org/details/in.ernet.dli.2015.219151/page/n195/mode/2up?q=%22worldly+lucre%22&view=theater}{Congreve, \textit{Love} 255}) % Original is bristling with capitals
\ex

And let not those Londoners whose eyes have been accustomed to {\dots} suppose that {\dots}\hfill(\href{https://archive.org/details/lifeadventuresofdickrich/page/466/mode/2up?q=%22those+Londoners+whose%22&view=theater}{Dickens, \textit{Nicholas} 443}) % "Londoners" restored
\ex
let not another dare suspect it\hfill(\href{https://archive.org/details/evanharringtonno00mererich/page/194/mode/2up?q=%22dare+suspect%22&view=theater}{Meredith, \textit{Harrington} 219})
\ex
let us not add guilt to our misfortunes\hfill(\href{https://quod.lib.umich.edu/cgi/t/text/pageviewer-idx?c=ecco;cc=ecco;idno=004771299.0001.000;node=004771299.0001.000:6.1;seq=189;page=root;view=text}{Goldsmith, \textit{Good-natur'd} 5})
\ex
let us not imagine evils which we do not feel\hfill(\href{https://archive.org/details/historyrasselas01johngoog/page/n105/mode/2up?q=%22let+us+not+imagine%22&view=theater}{Johnson, \textit{Rasselas} 101}) % "Evil" corrected to "evils"
\z
\z

While now \textit{not} is always in natural language placed before the infinitive it belongs to, there is a poetic or archaic way of placing it after the infinitive, as in \refp{ex:05-96}.
\pagebreak

\ea \label{ex:05-96}
\ea
one object which you might pass by, Might see and \textit{notice not}\\\hfill(\href{https://archive.org/details/poemsofwilliamwo00wor/page/50/mode/2up?q=%22one+object+which+you+might+pass+by%22&view=theater}{Wordsworth, \textit{Michael}})
\ex
a continuance of enduring thought. Which then I can \textit{resist not}\\\hfill(\href{https://archive.org/details/manfreddramaticp06byro/page/n11/mode/2up?q=continuance&view=theater}{Byron, \textit{Manfred} 1.1})
\ex
God bless you, my son, {\dots} and when he smiles on you, may the frown of man \textit{affect you not}!\hfill(\href{https://archive.org/details/christianstory00cainrich/page/70/mode/2up?q=%22frown%22&view=theater}{Caine, \textit{Christian} 69}) % Little changes to accord with the printed book
\z
\z\il{English!not@\textit{not}|)}

\is{scope of negation!resolving ambiguity of|(}
In other languages, difficulties like those mentioned in English are obviated in different ways. Thus in Greek \textit{mē} is used to negative an infinitive, while \textit{ou} is used with a finite verb. In Danish, a certain number of combinations like \textit{jeg beklager ikke at kunne hjælpe Dem} (`I am sorry I cannot help you') may be ambiguous, though less so in the spoken than in the printed form; but in some instances the colloquial use of a preposition shows where \textit{ikke} belongs; instead of the literary \textit{prøv ikke at se derhen} (`try not to look over there') it is usual to say either \textit{prøv ikke på at se derhen} (`don't try to look over there') or \textit{prøv på ikke at se derhen}. There is another colloquial way out of the difficulty, by means of the verbal phrase \textit{lade være} or rather \textit{la vær} (literally `leave be'): % ??? PE: What's in single quotes is expected to be a more or less idiomatic translation; but a problem for me is that I don't understand "leave be" in these examples. "Leave her be", OK; ?"Leave her be to look", pretty strange; *"Leave be to look", ungrammatical (for me).
%% SG: It's a conventionalized expression: lad være at se derhen, lit. 'leave be to look over there' (= 'do not look over there')
\textit{prøv at} (\textit{å}) \textit{la vær at} (\textit{å}) \textit{se derhen} (`Avoid looking over there'). Thus also \textit{du skal lade vær at se derhen} (`You must refrain from looking over there'), different from \textit{du skal ikke se derhen} (`You must not look over there').

In Latin, the place of \textit{non} before the main verb or before the infinitive will generally suffice to make the meaning clear. Similarly in French: \textit{il ne tâche pas de regarder} (`he doesn't try to look'); \textit{il tâche de ne pas regarder} (`he tries not to look'); \textit{il ne peut pas entendre} (`he cannot hear'); \textit{il peut ne pas entendre} (`he can not hear (as he chooses)') --- whence the possibility of saying \textit{non potest non amare} (`he cannot not love'); \textit{il ne peut pas ne pas aimer} corresponding to Danish \textit{han kan ikke lade være at elske}, English \textit{he cannot but love}, \textit{cannot help loving} (\textit{cannot choose but love}). Cf. \chapref{ch:8} below. % PE: OJ writes "Cf. below ch. VIII.". We could insert a comma before "chapter 8" to show that it's in apposition, but I've switched the order instead. Feel free to revert.
\is{scope of negation!resolving ambiguity of|)}

\label{para:not-to-sing}In this connexion, I must mention an interesting phenomenon frequent in Russian; I take my examples from Holger \citet[12]{pedersen1916russisk}; \textit{a pět' už ne stal} `but sing now he not began' % Peter: The bunch of English translations in this paragraph are OJ's own, and putting them into ( ) would complicate matters.
% PE: I have now (Sept '04) rather changed my mind about this. I'm putting ‘must/should’ in ( ).
which is explained as standing for the logical `not-to-sing he began', i.e. `he ceased to sing'; \textit{ne vélěno étogo dělat'} `order is not given to do this' instead of the logical `order is given not to do this', i.e. `it is prohibited to do this'. Similarly with \textit{dolžen} (`must/should'). But how comes it that the negative \textit{ne} is in such expressions attached to the wrong word? There is another way of viewing these sentences, if we take the negative to mean not the contradictory, but the contrary term: \textit{ne stal} `did the opposite of beginning', i.e. `ceased'; \textit{ne velmo} `the opposite of order, i.e. prohibition, is given'. And in \citet[\href{https://archive.org/details/vergleichendesl00vondgoog/page/399/mode/2up?q=\%22mitunter+wird+der+begriff\%22&view=theater}{400}]{vondrak1908vergleichende}, I find:

\begin{quote}
mitunter wird der Begriff des Verbs nicht durch \textit{ne} aufgehoben, sondern in sein Gegenteil verwandelt: [altkirchenslawisch] nenaviděti ‘hassen’ ([böhmisch] náviděti ‘lieben’), [serbisch] nèstati ‘verschwinden’
 % The abbreviations are Vondrák’s; should we have somebody proficient in German expand them? Also, OJ writes "ins gegenteil" but V writes "in sein Gegenteil" (of course we're following OJ's non-capitalizing)
%Brett: The abbreviations in the text refer to different Slavic languages: aksl. - Altkirchenslavisch (Old Church Slavonic) b. - Bulgarisch (Bulgarian) s. - Serbisch (Serbian) % Peter: Ah yes, obvious now that I think of it. I've inserted "Bulgarian". But hang on -- isn't it a little odd to bring up Bulgarian? If "b." were also the abbreviation of a word meaning roughly "derived from", it would be a lot less surprising. 
%Brett: The abbreviation "b." does indeed appear to stand for "bei" or possibly "beziehungsweise" rather than "Bulgarisch" (Bulgarian). This is evident from the context: "aksl. nenaviditi 'hassen' (b. náviditi 'lieben')" Here, it's showing the contrast between the negated form (nenaviditi - to hate) and its positive counterpart (náviditi - to love).
% Peter: Accordingly, I've deleted "Bulgarian" from "Bulgarian náviděti". (However, I haven't provided any alternative rendering of "b.".)

`sometimes the concept of the verb is not negatived by \textit{ne}, but transformed into its opposite: in Old Church Slavonic, \textit{nenaviděti} means ``to hate'' (Czech náviděti ``to love''),, and in Serbian, \textit{nèstati} means ``to disappear''' % PE: I have (unenthusiastically) changed "negated" to "negatived".
\end{quote}

This closely resembles a Greek idiom, see:

\begin{quote}
Einzelne Begriffe werden besonders durch \textit{ou} aufgehoben, ja zuweilen ins Gegenteil verwandelt, wie \textit{oú phēmi} nego, verneine {\dots} \textit{ouk axiô} verlange dass nicht, \textit{ouk eô} veto, verwehre, widerrate (auch erlaube nicht)

`Individual concepts are especially negatived by \textit{ou}, and sometimes even transformed into the opposite, as in \textit{oú phēmi} (\textit{nego} in Latin, `I deny') {\dots} \textit{ouk axiô} (`I do not deem worthy'), \textit{ouk eô} (\textit{veto} in Latin, `I forbid, I refuse, I advise against'---also `I do not permit')'\hfill\citep[\href{https://archive.org/details/griechischesprac00kr/page/296/mode/2up?view=theater}{§67 1.a.2}]{kruger1875griechische}
% PE: How about: `Individual concepts are especially negated by \textit{ou}, and sometimes even transformed into the opposite, as in \textit{oú phēmi} (\textit{nego} (Latin), I deny) {\dots} \textit{ouk axiô} (I demand not), \textit{ouk eô} (\textit{veto} (Latin), I forbid, I refuse, I advise against (also I do not permit))'  How we deal with the Latin here should probably affect how we deal with it in the next quotation.
%Brett: Agreed. Done
%PE: I have (unenthusiastically) changed "negated" to "negatived".
\end{quote}
\is{litotes}
\begin{quote}
Eine ähnliche Litotes liegt vor, wenn \textit{phēmí} die Negation an sich zieht, die logisch richtiger beim abhängigen Infinitive stehen würde: \textit{oú phēmi toûto kalôs ékhein} nego hoc bene se habere % restored "Eine ähnliche"

\emergencystretch=3em
`A similar litotes occurs when \textit{phēmí} attracts the negation to itself, which logically would more correctly stand with the dependent infinitive: \textit{oú phēmi toûto kalôs ékhein} (\textit{nego hoc bene se habere} in Latin, `I deny that this is in a good state')'\phantom{.}\hfill\citep[\href{https://archive.org/details/p2ausfhrlichegra02khuoft/page/180/mode/2up?q=litotes&view=theater}{180}]{kuhner1904ausfuhrliche}
\end{quote}

\noindent This is explained as change into the contrary:

\begin{quote}
\textit{\emph{ouk eô} prohibeo {\dots} \emph{ou stérgō} odi {\dots} \emph{ou sumbouleúō} dissuadeo}

\emergencystretch=3em
`I prohibit, I hate, I dissuade'\hfill \citep[\href{https://archive.org/details/p2ausfhrlichegra02khuoft/page/182/mode/2up?q=litotes&view=theater}{182}]{kuhner1904ausfuhrliche}
\end{quote}

\is{position of negative|(}
\is{raising of negation|(}
As an ``accusative with an infinitive'' can be considered as a kind of dependent clause, the mention of Latin \textit{nego Gaium venisse} (`I say that Gaius has not come') naturally leads us to the strong tendency found in many languages to attract to the main verb a negative which should logically belong to the dependent nexus. In many cases, \textit{I don't think he has come} and similar sentences really mean `I think he has not come'; though \textit{I hope} (\textit{expect})\textit{ he won't come} is more usual than the less logical \textit{I do not hope} (\textit{expect})\textit{ he will come}, which is usual in Danish and German, and also, according to \citet[\href{https://archive.org/details/englishaswespeak00joycuoft/page/19/mode/2up?view=theater&q=\%22So+in+our+modern+speech\%22}{20}]{joyce1910english} among the Irish, who will say, e.g. \textit{It is not my wish that you should go to America at all}, by which is meant the positive assertion: `It is my wish that you should not go', --- as well as \textit{I didn't pretend to understand what he said} for `I pretended not to understand'.
\is{attraction!of negative to verb}

A few Scandinavian examples may be given of this tendency to insert the negative in the main sentence: \refp{ex:05-99}.

\ea \label{ex:05-99}
\ea
\gll saa vil jeg \textit{aldrig} ønske, at du maa blive gift\\
 then will I never wish that you may become married\\
\glt `{\dots} then I wish that you will never get married'\\\hfill(\href{https://tekster.kb.dk/text/adl-texts-hostrup01-shoot-workid54989#idm140699400034704}{Hostrup, \textit{Gjenboerne} 3.6}) % &&
%Brett: What's this? % Peter: I've no idea. Just a slip of the fingers, perhaps.
\ex
\gll Jeg tror \textit{ikke}, at mange har læst Brand og at færre har forstaaet den\\
 I believe not that many have read Brand and that fewer have understood it\\
\glt `I believe that not many have read Brand and that fewer have understood it'\hfill(Schandorff, news 1897; note here the continuation,\\\hfill which shows that what is meant is: \textit{tror at ikke mange {\dots}}) % It's not even clear to me (PE) which Schandorff this is.
\is{quantifiers!negatived}

\ex
\gll Men det lot {'o [= hun]} ikke, som 'o hørte\\
 but that pretended she not {as if} she heard\\
\glt `But she acted as if she hadn't heard that'\hfill(\href{https://www.nb.no/items/URN:NBN:no-nb_digibok_2008072810004?page=31&searchText=%22Men%20det%20lot%22}{Bjørnson, \textit{Guds} 21})
\ex
\gll Han trodde \textit{icke} presterna voro annat än examinerade studenter \textit{och} att deras besvärjelseord bara var mytologi\\
 he believed not priests.\DEF{} were other than examined students and that their incantations just were mythology\\
\glt `He believed that priests were nothing but graduated students, and that their incantations were mere mythology'\\\hfill(\href{https://litteraturbanken.se/f%C3%B6rfattare/StrindbergA/titlar/GiftasII1886/sida/134/faksimil}{Strindberg, \textit{Giftas} 2.134}; note also here the positive continuation) 
\z
\z

Compare from French \refp{ex:05-103}.

\ea \label{ex:05-103}
\gll il ne faut pas que tu meures\\
 it not need not that you die\\
\glt `you must not die'\hfill(\href{https://archive.org/details/vermischtebeitr04toblgoog/page/n197/mode/2up?view=theater&q=%22faut+pas+que+tu+meures%22}{Tobler, \textit{Beiträge} 1.164})
\z
\is{raising of negation|)}

\il{English!not@\textit{not}|(}
In English, we must note the distinction between \textit{I don't suppose} (\textit{I am not afraid}), where the main nexus is negatived, and \textit{I suppose not} (\textit{I am afraid not}) where the nexus is positive, but the object (a whole sentence understood) is negative; how old is this use of \textit{not} for a whole sentence? Examples: \refp{ex:05-104}.

\ea \label{ex:05-104}
\ea I'm afraid not\hfill(\href{https://archive.org/details/in.ernet.dli.2015.219151/page/n171/mode/2up?q=%22afraid+not%22&view=theater}{Congreve, \textit{Love} 121})
\ex `` {\dots} whether it ever came to my knowledge until this moment?'' --- ``I believe not directly'' {\dots}. --- ``Why, you know not''\hfill(\href{https://archive.org/details/personalhistory05dickgoog/page/n47/mode/2up?q=%22whether+it+ever%22&view=theater}{Dickens, \textit{David} 93}) % "until this moment" restored; CD's "Why" instead of OJ's "Well". (Incidentally, the original three paragraphs have other grammatical interest too.)
\ex ``I am afraid you can't learn it, my poor fellow.'' --- ``I am afraid not''\\\hfill(\href{https://archive.org/details/lifeadventuresofdickrich/page/330/mode/2up?q=%22you+can%27t+learn+it%22&view=theater}{Dickens, \textit{Nicholas} 311}) % "my poor fellow" restored
\ex ``can you bear the thought of that?'' --- {\dots} ``I should imagine not, indeed!''\hfill(\href{https://archive.org/details/lifeadventuresofdickrich/page/616/mode/2up?q=%22bear+the+thought%22&view=theater}{ibid 590}) % OJ mangles three paragraphs into one; I (PE) have replaced the second with dots and separated the other two.
\ex ``I should not mind'' {\dots}. --- ``I daresay not, because you have nothing particular to say.'' {\dots} --- ``But I have something particular to say.'' --- ``I hope not.'' --- ``Why should you hope not?''\hfill(\href{https://archive.org/details/dukeschildrennov00troluoft/page/194/mode/2up?q=%22I+should+not+mind%22&view=theater}{Trollope, \textit{Children} 2.81}) % repunctuated
\ex ``I'll tell the boys and they'll draw you like a badger.'' --- ``Please not, old man.''\hfill(\href{https://archive.org/details/lightthatfailed0000rudy_q6g8/page/228/mode/2up?q=%22please+not%2C+old+man%22&view=theater}{Kipling, \textit{Light} 217}) % "and they'll draw you like a badger" restored
\ex I believe I asked him to hold his tongue about them---he says not.\\\hfill(\href{https://babel.hathitrust.org/cgi/pt?id=hvd.hnpei8&seq=9&q1=hold+his+tongue}{Conway, \textit{Called} 1}) % "about them" restored. (This is a single sentence, spoken by one person.) 
\z
\z
\is{position of negative|)}

\is{adverbs!negative}
\is{proforms|(}
\is{subordinator, negative|(}
\label{not_that}Inversely, we have a negative adverb standing for a whole main sentence, \il{English!not that@\textit{not that}|(}\textit{not that} meaning `I do not say that' or `the reason is not that' as in \refp{ex:05-111}. We shall see in \chapref{ch:12} the use of \il{English!not but@\textit{not but}}\textit{not but} (\textit{that}) and \textit{not but what} in the same sense.

\ea \label{ex:05-111}
\ea Not that I lou'd Cæsar lesse, but that I lou'd Rome more\\\hfill(\href{https://internetshakespeare.uvic.ca/doc/JC_F1/scene/3.2/index.html#tln-1550}{Shakespeare, \textit{Cæs} 3.2.22}) % The uvic.ca page has not "Cæsar" but instead "Caesar". Is it simplifying, or is OJ indulging another minor eccentricity?
%Brett: the first folio has "Cæsar". We can link to it, but it's hard to search and hard to read https://archive.org/details/mrvvilliamshakes00shak/page/120/mode/2up?q=%22that+Friend+demand%2C+why+Brutus+role+againft+Ca%7E+far%2C+this+is+my+anfwer+%3A%22&view=theater % Peter: ⟨æ⟩ is a pain, but as it's spattered across the Danish that's so prominent in this book, and as changing those examples back to ⟨aa⟩ would be most anachronistic, let's tolerate it in English as well.
\ex Not that the heart can be good without knowledge\\\hfill(\href{https://archive.org/details/bunyanspilgrims00moffgoog/page/108/mode/2up?q=%22heart+can+be+good%22&view=theater}{Bunyan, \textit{Progress} 113})
\ex Not that I agree with everything I have said in this essay\\\hfill(\href{https://archive.org/details/intentions01wild/page/256/mode/2up?q=%22agree%22&view=theater}{Wilde, \textit{Intentions} 212})
\ex Not that he had forgotten them\hfill(\href{https://archive.org/details/wonderfulyear00lockuoft/page/326/mode/2up?q=%22not+that+he+had+forgotten%22&view=theater}{Locke, \textit{Year} 309})
\z
\z\il{English!not that@\textit{not that}|)}\il{English!not@\textit{not}|)}

In other languages correspondingly: \refp{ex:05-115}.

\ea \label{ex:05-115}
\ea
\gll Ikke at han havde (or: skulde ha) glemt dem\\
 not that he had {} should have forgotten them\\
\glt `Not that he had forgotten them'
\ex
\gll nicht dass er sie vergessen hätte\\
 not that he them forgotten would have\\
\glt `not that he would have forgotten them'
\ex
\gll Non pas qu'il parlât à personne\\
 not not {that he} spoke to anyone\\
\glt `Not that he spoke to anyone'\hfill(\href{https://www.gutenberg.org/cache/epub/61876/pg61876-images.html}{Rolland, \textit{Foire} 306})
\z
\z
\is{subordinator, negative|)}

\il{English!not@\textit{not}|(}
When we say (``He'll come back'') \textit{Not he!} % PE: I think what OJ is saying is "When we say (in response to 'He'll come back') 'Not he!', it is not really...." And so he deliberately puts "He'll come back" in quotation marks, which I've restored.
it is not really \textit{he} that is negatived, but the nexus, although the predicative part of it is unexpressed; the exclamation is a complete equivalent of \textit{He won't!} (with stress on \textit{won't}): \refp{ex:05-118}.

\ea \label{ex:05-118}
\ea Who, I rob? I a theefe? Not I.\hfill(\href{https://internetshakespeare.uvic.ca/doc/1H4_F1/scene/1.2/index.html#tln-240}{Shakespeare, \textit{H4A} 1.2.153}) 
\ex Please not, old man.\hfill(\href{https://archive.org/details/lightthatfailed0000rudy_q6g8/page/228/mode/2up?q=%22please+not%2C+old+man%22&view=theater}{Kipling, \textit{Light} 217})
\ex Were I a Steam-engine, wouldst thou take the trouble to tell lies of me? Not thou!\hfill(\href{https://archive.org/details/sartorresartus02unkngoog/page/222/mode/2up?view=theater&q=%22were+I+a+steam-engine%22}{T. Carlyle, \textit{Sartor} 169})
\ex Meg don't know what he likes. Not she!\hfill(\href{https://archive.org/details/chimes00dick/page/58/mode/2up?q=%22meg+don%27t+know+what%22&view=theater}{Dickens, \textit{Chimes} 30})
\ex They wouldn't have touched \textit{us} {\dots} Not they!\\\hfill(\href{https://archive.org/details/freelands00galsrich/page/222/mode/2up?q=%22not+they%21%22&view=theater}{Galsworthy, \textit{Freelands} 255}) % OJ has "wouldn't touch", but JG writes "wouldn't have touched"
\ex ``it'll perhaps rain cats and dogs to-morrow {\dots}'' {\dots} --- ``Not \textit{it}''\\\hfill(\href{https://archive.org/details/silasmarnerbygeo00elio/page/30/mode/2up?q=%22cats+and+dogs%22&view=theater}{Eliot, \textit{Silas} 44})
\ex ``Do you think it will last long?'' --- ``Not it!''\hfill(\href{https://archive.org/details/cu31924013586940/page/238/mode/2up?q=%22will+last+long%22&view=theater}{Bennett, \textit{Wives} 1.263})
\ex ``Bit late now, isn't it?'' --- ``Not it.''\hfill(\href{https://archive.org/details/cardstoryofadven00bennuoft/page/244/mode/2up?view=theater&q=%22bit+late+now%22}{Bennett, \textit{Card} 244}) % OJ merely says "Bennett Cd. 244": no example, no indication of what "Cd." is, etc.
\ex All sorts of things accumulate, sir.{\dots} Not \textit{you}, of course, in particular.\\\hfill(\href{https://archive.org/details/in.ernet.dli.2015.475865/page/n129/mode/2up?q=%22Not+you%2C+of+course%2C+in+particular.%22&view=theater}{Wells, \textit{Stories} 49}) % (i) Let's try to find a better copy than this one. (ii) OJ doesn't specify the example, only the page number.
%Brett: Is this a better copy? % Peter: It's not as bad a copy. I've resuscitated the emphasis on "you".
\z
\z
\is{proforms|)}

The following examples \refp{ex:05-127} show the accusative used as a modern (vulgar or half-vulgar) ``disjointed'' nominative.

\ea \label{ex:05-127}
\ea
We sha'n't hang up on any misunderstanding. Not us.\\\hfill(\href{https://archive.org/details/annveronicamoder0000hgwe/page/360/mode/2up?q=%22any+misunderstanding%22&view=theater}{Wells, \textit{Veronica} 338}) % OJ has "shan't hang upon", but that isn't what the Harper & Brothers 1909 edition says.
\ex ``you were all in the same room together, were not you?''\\\il{English!no@\textit{no}}``No, indeed, not us.''\hfill(\href{https://archive.org/details/sensesensibility00austrich/page/242/mode/2up?q=%22all+in+the+same+room%22&view=theater}{Austen, \textit{Sense} 269}) % PE: Adjusted JA's punctuation to accord with the edition linked to.
\z
\z\il{English!not@\textit{not}|)}

In Old English we have the corresponding \il{English!Old English!nic@\textit{nic}}\textit{nic} \refp{ex:05-129}. \textit{Nic} is spelt \textit{nîc} and \textit{nyc} (\href{https://archive.org/details/holygospelsinan01skeagoog/page/n75/mode/2up?view=theater&q=nyc}{\textit{John}, ed. Skeat, 1.21}), spelt \textit{nicc} and \textit{nicht} (\href{https://archive.org/details/holygospelsinan01skeagoog/page/n365/mode/2up?view=theater&q=nicht}{ibid 18.17}). This (with the positive counterpart \textit{I}, which is probably the origin of \textit{ay} (`yes'), and \textit{ye we} in \refp{ex:05-130}) closely resembles the French \textit{naje} `not I' (in the third person \textit{nenil}) and the positive \textit{oje} `hoc ego' (in the third person \textit{oïl, oui}), see Tobler (\citeyear[\href{https://www.jstor.org/stable/pdf/40845615.pdf}{423}]{tobler1877franzosische}; \citeyear[\href{https://archive.org/details/vermischtebeitr04toblgoog/page/n17/mode/2up?view=theater}{1}]{tobler1886vermischte}); \citet[\href{https://www.jstor.org/stable/45042774?seq=3}{465}]{paris1878periodiques}.

\ea \label{ex:05-129}
\gll Wilt þu fon sumne hwæl? --- Nic\\
 want you catch some whale {} not I\\
\glt `Do you want to catch a whale? --- No.'\hfill(\href{https://archive.org/details/anglosaxonoldeng01wriguoft/page/51/mode/2up?view=theater&q=%22wilt%22}{Wright \& Wülcker 1.94})
\z\il{English!Old English!nic@\textit{nic}|)}
%% SG: More idiomatically, the reply just translates to "no". It's a dedicated first-person negation
%BR: done

\ea \label{ex:05-130}
wille ye doo this {\dots} --- {\dots} ye we, lorde\hfill(\href{https://archive.org/details/TheHistoryOfReynardTheFoxArber/page/n87/mode/2up?q=%22wille+ye+doo+this%22&view=theater}{Caxton, \textit{Reynard} 58})
\z
\is{nexal negation!and special negation|)}
\is{scope of negation|)}
\is{special negation!and nexal negation|)}
