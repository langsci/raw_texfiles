\ChapterAndMark{Positive becomes Negative} 
\label{ch:3}
\is{positive becomes negative|(}
\is{semantic change|(}

\is{grammaticalization|(}
\il{French!pas@\textit{pas}|(}\il{French!ne@\textit{ne}|(}The best-known examples of a transition from positive to negative meaning are found in French. Through the phenomenon which \citet[\href{https://archive.org/details/in.ernet.dli.2015.261285/page/n265/mode/2up&view=theater}{200ff}]{breal1900semantics} aptly terms 
\is{contagion|(}``contagion'' words like \textit{pas}, \il{French!point@\textit{point}}\textit{point}, \il{French!jamais@\textit{jamais}}\textit{jamais}, \il{French!plus@\textit{plus}}\textit{plus}, \il{French!aucun@\textit{aucun}}\textit{aucun}, \il{French!personne@\textit{personne}}\textit{personne}, which were extremely frequent in sentences containing \textit{ne} with the verb, acquired a negative colouring, and gradually came to be looked upon as more essential to express the negative notion than the diminutive \textit{ne}. As this came to be used exclusively in immediate juxtaposition with a verb, the other words were in themselves sufficient to express the negative notion when there was no verb, at first perhaps in answers: ``\textit{Ne viendra-t-il jamais?}'' ``\emph{\textit{\textsc{Jamais}}}.''; ``\textit{Ne vois-tu personne?}'' ``\emph{\textit{\textsc{Personne}}}.'' Now we have everywhere quite regularly \refp{ex:03-01}, etc. In a somewhat different way \refp{ex:03-06}.\il{French!ne@\textit{ne}|)}

\ea \label{ex:03-01}
\ea
\gll \textit{Pas} de ça\\
 not of that\\
\glt `Not that!' 
\ex
\gll Pourquoi \textit{pas}?\\
 why not\\
\glt `Why not?'
\ex
\gll le compartiment des \textit{pas-fumeurs}\\
 the compartment {of the} not-smokers\\
\glt `the non-smokers' compartment' 
\ex
\il{French!rien@\textit{rien}|(}\gll Permettez-moi de lui dire un seul mot, \textit{rien} \textit{qu}'un seul.\\
 {allow me} to him {to say} a single word nothing {but a} single\\
\glt `Allow me to say just one word to him, just one.'\\\hfill(\href{https://archive.org/details/lesdeuxhritages02gogogoog/page/n39/mode/2up?q=%22Permettez-moi+de+lui+dire+un+seul+mot%22&view=theater}{Mérimée, \textit{Héritages} 1.8})
% ??? PE: Here and below, should we gloss "que" as "but", or as "that"?
\ex
\gll Il frissonnait \textit{rien} \textit{que} d'y penser\\
 he shivered just by {of it} think\\
\glt `He shivered just thinking about it'
\hfill(\href{https://archive.org/details/saphomursparisi00daudgoog/page/n145/mode/2up?q=%22frissonnait+rien+que+d%27y+penser%22&view=theater}{Daudet, \textit{Sapho} 134})
\ex\il{French!rien@\textit{rien}|)}
\il{French!plus@\textit{plus}|(}\gll une chambre et un cabinet {\dots} la chambre \textit{guère} que plus grande\\
 a bedroom and a cabinet {} the bedroom hardly but more large\\
\glt `a bedroom and a cabinet {\dots} the bedroom hardly larger'\\\hfill(\href{https://archive.org/details/numaroumestanmo00daud/page/130/mode/2up?q=%22la+chambre+gu%C3%A8re+plus+grande%22&view=theater}{Daudet, \textit{Numa} 105})
\z
\z

\ea \label{ex:03-06}
\ea
\gll Mais si vous croyez que Tartarin avait peur\textit{,} \textit{pas} \textit{plus}!\\
 but if you believe that Tartarin had fear no more\\
\glt `But if you think that Tartarin was scared, not at all!'\\
\hfill(\href{https://archive.org/details/tartarinsurlesa01daudgoog/page/n147/mode/2up?q=%22Mais+si+vous+croyez+que+Tartarin%22&view=theater}{Daudet, \textit{Tartarin} 252})
\ex
\gll et toute la ligne [`tous les enfants assis en ligne'] mangeait jusqu'à \textit{plus} faim [`jusqu'à ce qu'ils n'eussent plus faim']\\
 and entire the line all the children seated in line ate until {no more} hunger until it {that they} {not had} {no more} hunger\\
\glt `and the whole line ate until they were no longer hungry'\\
\hfill(\href{https://archive.org/details/contesdelabcas00maupuoft/page/170/mode/2up?q=%22mangeait+jusqu%E2%80%99%C3%A0+plus+faim%22&view=theater}{Maupassant, \textit{Bécasse} 201}) % ??? PE: I think that this is unnecessarily confusing. As we are anyway glossing (Late Modern) French in English, there's no obvious need for OJ to do so in French (let alone any need for us to gloss his gloss). But we, or anyway I, would like to be able to say that we haven't removed anything. So how about removing OJ's gloss from where he placed it, and instead placing it within a footnote? (Something like: "Jespersen glosses this: `et tous les enfants assis en ligne mangeait jusqu'à ce qu'ils n'eussent plus faim'.")
\z
\z\il{French!plus@\textit{plus}|)}

\il{French!ne@\textit{ne}|(}The next step is the leaving out of \textit{ne} even where there is a verb. This may have begun through \is{prosiopesis}prosiopesis in \is{interrogatives}interrogative and \is{imperatives}imperative sentences: (\textit{ne}) \textit{viens-tu \textsc{pas}?}; (\textit{ne}) \textit{dis \textsc{pas} ça!} Cf. also (\textit{Il ne}) \textit{faut \textsc{pas} dire ça!} It may have been a concomitant circumstance in favour of the omission that it is in many sentences impossible or difficult to hear \textit{ne} distinctly in rapid \is{pronunciation, influence of}pronunciation: \textit{on n'a pas} \texttt{\vert} \textit{on n'est pas} \texttt{\vert} \il{French!jamais@\textit{jamais}}\textit{on n'arrive jamais} \texttt{\vert} \il{French!rien@\textit{rien}}\textit{la bonne n'a rien} \texttt{\vert} \textit{je ne nie pas}, etc. Sentences without \textit{ne}, which may be heard any day in France, also among the educated, begin to creep into literature, as in \refp{ex:03-09}. (Similarly \textit{ne} is now often omitted in those cases in which ``correct grammar'' requires its use without any \textit{pas}, for instance \textit{de peur qu'il vienne}.) In the soldiers' conversations in René Benjamin's \textit{Gaspard}\footnote{A 1915 novel, winner of the Prix Goncourt, based on the author's experiences during months-long hospitalization following war injuries. \eds} there is scarcely a single \textit{ne} left.\il{French!ne@\textit{ne}|)} % ??? PE: I've broken a long paragraph of OJ's at this point, and moved what was its lower half below the set of examples. Hope this seems an improvement.

\ea \label{ex:03-09}
\ea
\gll c'est \textit{pas} ces gredins-là\\
 {it is} not these {scoundrels there}\\
\glt `it's not these scoundrels there'
\hfill(\href{https://archive.org/details/notesetsouvenir01halgoog/page/n105/mode/2up?q=%22c%27est+pas+ces+gredins-la%22&view=theater}{Halévy, \textit{Notes} 91}) % ??? PE: Or change the gloss to "{it's} not those {scoundrels there}", and skip the additional translation?
\ex
\gll J'ai \textit{pas} fini, qu'elle disait\\
 {I have} not finished {that she} said\\
\glt `I haven't finished, she said'
\hfill(\href{https://archive.org/details/notesetsouvenir01halgoog/page/n107/mode/2up?q=%22J%27ai+pas+fini%2C+qu%27elle+disait%22&view=theater}{ibid 92})
\ex
\gll C'est \textit{pas} de l' eau bénite\\
 it's not of the water blessed\\
\glt `It's not holy water'
\hfill(\href{https://archive.org/details/notesetsouvenir01halgoog/page/n107/mode/2up?q=%22C%27est+pas+de+l%27eau+b%C3%A9nite%22&view=theater}{ibid 93}) % OJ merely points to this example but doesn't provide it.
\ex
\gll C'est \textit{pas} les Prussiens {\dots} C'est \textit{pas} comme des idiots de provinciaux\\
 it's not the Prussians {} it's not like the idiots of provincials\\
\glt `It's not the Prussians {\dots} It's not like the idiot provincials'
\hfill(\href{https://archive.org/details/notesetsouvenir01halgoog/page/n255/mode/2up?q=%22C%27est+pas+les+Prussiens%22&view=theater}{ibid 240}) % OJ merely points to this example but doesn't provide it. He also points to one on p 239 but I (PE) see nothing there and so have cut the mention of it.
\ex
\gll Vaut-il \textit{pas} mieux accepter ce qui est?\\
 {is worth it} not better {to accept} that what is\\
\glt `Isn't it better to accept what is?'
\hfill(\href{https://archive.org/details/saphomoeursparis00dauduoft/page/232/mode/2up?q=%22Vaut-il+pas+mieux+accepter+ce+qui+est+%22&view=theater}{Daudet, \textit{Sapho} 207})
\ex
\gll As \textit{pas} peur!\\
 have not fear\\
\glt `Don't be afraid!'
\hfill(\href{https://archive.org/details/germinielacerte00gonc/page/200/mode/2up?q=%22as+pas+peur%22&view=theater}{Goncourt, \textit{Germinie} 200})
\ex
\il{French!rien@\textit{rien}}\gll une famille où l'argent comptait pour \textit{rien}\\
 a family where money counted for nothing\\
\glt `a family where money meant nothing'
\hfill(\href{https://fr.wikisource.org/wiki/Une_vie/VII}{Maupassant, \textit{Vie} 132}) 
\ex
\gll tu seras \textit{pas} mal dans quelque temps\\
 you {will be} not bad in some time\\
\glt `in time, you won't look bad'
\hfill(\href{https://archive.org/details/fortcommelamort00maupuoft/page/72/mode/2up?q=%22tu+seras+pas+mal+dans+quelque+temps%22&view=theater}{Maupassant, \textit{Fort} 68})
\ex
\gll Est-ce \textit{pas} votre avis, Corbelle?\\
 {is it} not your opinion Corbelle\\
\glt `Isn't that what you think, Corbelle?'
\hfill(\href{https://archive.org/details/fortcommelamort00maupuoft/page/74/mode/2up?q=%22est-ce+pas%22&view=theater}{ibid 69}) % OJ merely points to this example; he doesn't provide it.
\ex
\gll Voudrais-tu \textit{pas} que je reprisse la vieille devise de haine\\
 {would you} not that I {take up again} the old currency of hatred\\
\glt `Wouldn't you want me to take up the old currency of hatred again'
\\
\hfill(\href{https://archive.org/details/jeanchristop03roll/page/100/mode/2up?q=%22Voudrais-tu+pas+que+je+reprisse+la+vieille+devise+de+haine%22&view=theater}{Rolland, \textit{Maison} 96})
\z
\z
\il{French!plus@\textit{plus}|(}In the case of \textit{plus} this new development might lead to frequent ambiguity, if this had not been obviated in the popular \is{pronunciation, disambiguating}pronunciation, in which [j ãn a ply] means `there is no more of it' and [j ãn a plys] `there is more of it' (literary \textit{il n'y en a plus} and \textit{il y en a plus}). In \textit{plus de bruit} we have a negative, but in \textit{Plus de bruit que de mal} a positive expression, though here the pronunciation is always the same. Note the difference between \textit{Jean n'avait plus confiance} and \textit{Jean n'avait pas plus confiance} [\textit{que Pierre}]; cf. also \textit{Jean n'avait pas confiance, non plus} (`nor had {\dots}').\il{French!pas@\textit{pas}|)}

There is a curious consequence of this negative use of \textit{plus}, namely that \il{French!moins@\textit{moins}}\textit{moins} may occasionally appear as a kind of comparative of its etymological antithesis \refp{ex:03-19}.

\ea \label{ex:03-19}
\il{French!moins@\textit{moins}}\gll Plus d'écoles, plus d'asiles, plus de bienfaisance, encore moins de théologie\\
 {no more} {of schools} {no more} {of asylums} {no more} of charity even less of theology\\ % Peter: I'm puzzled by the inclusion of "no", three times. We're purporting to give the literal meaning, so delete all three.
 %Brett: I don't understand. Don't we need at least one? And since Mérimée didn't write "Plus d'écoles, d'asiles, de bienfaisance", then why should we elide two "no"s?
 % PE: I suppose the problem is one of just how literal our "literal" should be. Perhaps "literally" this should be "more of schools, more of asylums, more of charity, even less of theology". Of course each "plus" means "no more"; yes, it could be argued that for a century or longer it has "literally" meant this.
 %Brett: I think that's the point of the example.
\glt `No more schools, no more asylums, no more charity, and even less theology'
\hfill(\href{https://archive.org/details/lesdeuxhritages02gogogoog/page/n59/mode/2up?q=%22plus+d%27%C3%A9coles%2C+plus+d%27asiles%2C+Plus+de+bienfaisance%2C+encore+moins+de+theologie.%22&view=theater}{Mérimée, \textit{Héritages} 2.2})
\z\il{French!plus@\textit{plus}|)}

\is{lexical change|(}\il{French!ne@\textit{ne}|(}One final remark before we leave French. From a psychological point of view it is exactly the same process that leads to the omission of \textit{ne} in two sentences like \textit{il} (\textit{ne})\textit{ voit \il{French!nul@\textit{nul}}nul danger} and \textit{il} (\textit{ne})\textit{ voit \il{French!aucun@\textit{aucun}}aucun danger}; but etymologically they are opposites: in one an originally negative word keeps its value, in the other an originally positive word is finally changed into a negative word.\il{French!ne@\textit{ne}|)}


In Spanish we have some curious instances of positive words turned into negative ones: \il{Spanish!nada@\textit{nada}}\textit{nada} from Latin \il{Latin!nata@\textit{nata}}\textit{nata} \il{Latin!res nata@\textit{res nata}}(\textit{res nata}) means `nothing', and \il{Spanish!nadie@\textit{nadie}}\textit{nadie}, older \il{Spanish!nadien@\textit{nadien}}\textit{nadien} with the ending of \il{Spanish!quien@\textit{quien}}\textit{quien} instead of \il{Spanish!nado@\textit{nado}}\textit{nado} from \il{Latin!natus@\textit{natus}}\textit{natus}, means `nobody'. In both I imagine that the initial sound of \textit{n-} as in \textit{no} has favoured the change. Through the omission of \textit{no} some temporal phrases come to mean `never' as in \refp{ex:03-20}. Thus also \il{Spanish!absolutamente@\textit{absolutamente}}\textit{absolutamente} (`durchaus nicht', `not at all'), see \citet[\href{https://archive.org/details/spanischegrammat00hansuoft/page/204/mode/2up?q=absolutamente&view=theater}{§60, 5}]{hanssen1910spanische}.

\ea \label{ex:03-20}
\ea
\gll \textit{En} \textit{todo} \textit{el} \textit{día} Se ve apartar de la puerta\\
 in all the day one sees {moving away} from the door\\
\glt `All day long, one sees [it/someone] moving away from the door'\\
\hfill(\href{https://archive.org/details/AGuichot05012/page/48/mode/2up?q=%22en+todo+el+dia%22&view=theater}
{Calderón, \textit{Alcalde} 2.1})
%{\cite[II.1]{Calderon1876}}
\ex
\gll A pesar de tan buen ejemplo, \textit{en} \textit{mi} \textit{vida} me hubiera sometido a ejercer una profesión\\
 in spite of such good example in my life myself {would have} submitted to practising a profession\\
\glt `In spite of such a good example, I would never in my life have submitted myself to practising a profession'
\hfill(\href{https://archive.org/details/Donaperfecta/page/n57/mode/2up?q=%22A+pesar+de+tan+buen+ejemplo%2C+en+mi+vida+me+hubiera%22&view=theater}
%{\cite[68]{Galdos1919}}
{Pérez Galdós, \textit{Doña} 68}) % OJ's "á ejercer una profecion" corrected to "a ejercer una profesión"
\z
\z

In Old Norse several words and forms are changed from positive to negative, as already indicated above: the ending \il{Old Norse!\textit{-gi}@\textit{-gi}}\textit{-gi} \il{Old Norse!\textit{-ge}@\textit{-ge}}(\textit{-ge}) in \il{Old Norse!eigi@\textit{eigi}}\textit{eigi}, \il{Old Norse!einngi@\textit{einngi}}\textit{einngi} \il{Old Norse!engi@\textit{engi}}(\textit{engi}), \il{Old Norse!eittgi@\textit{eittgi}}\textit{eittgi} \il{Old Norse!etki@\textit{etki}}(\textit{etki}, \il{Old Norse!ekki@\textit{ekki}}\textit{ekki}), \il{Old Norse!hvarrgi@\textit{hvárrgi}}\textit{hvárrgi}, \il{Old Norse!manngi@\textit{manngi}}\textit{manngi}, \il{Old Norse!vaettki@\textit{vættki}}\textit{vættki}, \il{Old Norse!aldrigi@\textit{aldrigi}}\textit{aldrigi}, \il{Old Norse!aevagi@\textit{ævagi}}\textit{ævagi}, further the enclitic \il{Old Norse!-a@\textit{-a}}\textit{-a} and \il{Old Norse!-at@\textit{-at}}\textit{-at}.

In German must be mentioned \il{German!kein@\textit{kein}}\textit{kein} from Old High German \il{German!Old High German!dihhein@\textit{dihhein}}\textit{dihhein}, orig. `irgend einer' (\textit{dih} of unknown origin), though the really negative form \il{German!Old High German!nihhein@\textit{nihhein}}\textit{nihhein} has of course also contributed to the negative use of \il{German!kein@\textit{kein}}\textit{kein}; further \il{German!weder@\textit{weder}}\textit{weder} from Old High German \textit{ni-wedar} (\textit{wedar} corresponding to English \textit{whether}).

In English, we have \il{English!but@\textit{but}}\textit{but} from \textit{ne {\dots} but}, cf. \il{English!dialectal!northern England}northern dialect \textit{nobbut} (see p.~\pageref{12-nobbut} (\chapref{ch:12}) below), % PE: OJ merely specifies chapter XII; no page number
and a rare \il{English!more@\textit{more}|(}\textit{more} meaning `no more', a clear instance of \is{prosiopesis}prosiopesis, which, however, seems to be confined to the \il{English!dialectal}\il{English!dialectal!south-western England}South-Western part of England, see \refp{ex:03-22}. (Cf. with negative verb \refp{ex:03-26}.) Similarly \il{English!me either@\textit{me either}}\textit{me either} = `nor me either' \refp{ex:03-27}.

\ea \label{ex:03-22}
\ea
``not much of a scholar, I'm afraid.''\\``More am I''\hfill(\href{https://archive.org/details/themother00phil/page/28/mode/2up?q=%22scholar%22&view=theater}{Phillpotts, \textit{Mother} 29})
\ex
You're no longer a child {\dots} and more am I\hfill(\href{https://archive.org/details/themother00phil/page/144/mode/2up?q=%22You%27re+no+longer+a+child%22&view=theater}{ibid 144})
\ex
Couldn't suffer it---more could he.\hfill(\href{https://archive.org/details/themother00phil/page/12/mode/2up?q=%22Couldn%27t+suffer+it%E2%80%94more+could+he%22&view=theater}{ibid 12})
\ex
you meant that I couldn't expect that man to like me. More I do.\\\hfill(\href{https://archive.org/details/themother00phil/page/322/mode/2up?q=%22you+meant+that+I+couldn%27t+expect+that+man+to+like+me.+More+I+do%22&view=theater}{ibid 322})
\z
\z

\ea \label{ex:03-26}
he's a man that won't be choked off a thing---and more won't I.\hfill(\href{https://archive.org/details/themother00phil/page/308/mode/2up?q=%22he%27s+a+man+that+won%27t+be+choked%22&view=theater}{ibid 309})
\z\il{English!more@\textit{more}|)}

\noindent 

\il{English!me either@\textit{me either}|(}
\ea \label{ex:03-27}
\ea
``it so happens that I have no small change about me.'' --- ``Me either'', said Mrs. Treacher idiomatically\hfill(\href{https://archive.org/details/majorvigoureux00quil/page/110/mode/2up?q=%22Me+either%22&view=theater}{Quiller-Couch, \textit{Major} 111}) % "T." restored to "Treacher"
\ex
``I never could endure the instrument.'' --- ``Me either''\hfill(\href{https://archive.org/details/majorvigoureux00quil/page/180/mode/2up?q=%22Me+either%22&view=theater}{ibid 181}) % OJ merely points to page 181; he does not provide the example.
\z
\z\il{English!me either@\textit{me either}|)}

\is{directives}
\il{English!nautical!\textit{near}|(}\noindent Similarly the order to the helmsman when he is too near the wind \textit{Near!} is said to be shortened through \is{prosiopesis}prosiopesis (which is here also a kind of haplology) from \textit{No near!} (\textit{near} the old comparative meaning what is now called \textit{nearer}), see \href{https://archive.org/details/oed6barch/page/n887/mode/2up?view=theater}{\textit{NED}, \textit{Near} \textit{adv.} 1~d}.\il{English!nautical!\textit{near}|)} % PE: OJ says "adv. 1 c", but it's 1 d.

\is{contagion|)}
\is{lexical change|)}
\is{positive becomes negative|)}
\is{semantic change|)}
\is{grammaticalization|)}
