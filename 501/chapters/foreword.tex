\addchap{Foreword to this edition} \label{ch:foreword}

Otto Jespersen (1860--1943) was a renowned Danish linguist who made significant contributions to the study of English grammar and several other fields of linguistics. His monumental work \textit{A Modern English grammar on historical principles}, published in seven volumes from 1909 to 1949, remains a landmark in English grammatical description.

The present volume, \textit{Negation in English and other languages}, stems from Jespersen's research for volume III or IV of \textit{A Modern English grammar}; but, as Jespersen explains in his preface, World War I prevented him from publishing the continuation of that work. And so he decided to expand the scope of his study on negation to include observations on other languages, and publish it as a separate volume. For a detailed discussion of the work's historical context and scholarly significance, readers should consult Olli Silvennoinen's excellent introduction (p.~\pageref{ch:intro}).

This new edition aims to make Jespersen's important work more accessible to modern readers. The text has been entirely re-typeset. We have modernized the transcription of linguistic examples and added Leipzig-style glosses to non-English examples to clarify their structure and meaning. Bibliographical references have been reformatted and expanded, with hyperlinks to source materials added where available. And we have added section titles where doing so seemed helpful.

A significant challenge in preparing this edition was dealing with Jespersen's incomplete system of abbreviations and his occasionally imprecise approach to quotation. We have endeavored to supply missing information and restore quotes to their original form wherever possible. Most abbreviations have been expanded (e.g., OE → Old English), and certain typographical conventions have been regularized, such as the capitalization of German nouns. Jespersen's addenda have been integrated into the text, with some of his original body text relocated to footnotes for better readability.

However, we have maintained Jespersen's original analyses unchanged, even where subsequent research has led to different interpretations. Our goal has been to preserve the historical value of his work while making it more accessible to contemporary scholars.

Another challenge was Jespersen's citation practice. While he usually provides both the text and location of examples, in some cases he gives only page references (sometimes as bare “ibid” citations), leaving readers to track down the examples themselves. Where possible, we have restored these examples by consulting the cited works. In several cases where we could not locate the precise example Jespersen had in mind, we have explicitly marked this fact.

This edition, based on the scanned copy\footnote{\url{http://www.archive.org/details/cu31924026632947}} from the Internet Archive, began as Brett's exploration of large language models' potential in facilitating scholarly editing, with ChatGPT-4 and Claude 3 proving helpful but not without limitations in tasks such as OCR correction, {\LaTeX} formatting, translation, and indexing. Peter later joined the project and undertook the systematic organization of the text and, as far as was feasible, the crucial work of locating and linking to Jespersen’s source documents. 

For sources we could not locate online, we have linked to reprints or alternative editions. So for example, the text may cite \citep{jespersenMEG1} while its link goes to a 1961 reprint.

A note on datedness: Few readers in the 21st century are likely to take Jespersen's word on such matters as syntactic categorization, so we hardly need express any regret about, for example, his classing of \textit{but} as a ``relative pronoun''. % In chapter 12. PE: It would be good to attach an attributive to "but" (cf "concessive 'if'", in "If physically feeble, she remains mentally alert", but I can't now think of how to put it concisely.
However, his estimates of the currency of particular words should also be read with care. It is true that, as he asserts,  \textit{investigable} and \textit{invertible} have also meant `incapable of being investigated' and `incapable of being changed' respectively; but such uses of the pair are obsolete and seem to have been so even when he wrote. Yet we should hesitate before judging: his description of \textit{insubstantial} as having given way to \textit{unsubstantial} may surprise, but Google Books Ngram Viewer will confirm that this was true when he wrote it: the relative popularity of the two words was to reverse in the late 1940s. % All of these -- ?nfirm, ?nvestigable, ?nvertible, ?nsubstantial -- in chapter 13.

\bigskip
\hfill--- Brett Reynolds \& Peter Evans\\ \phantom{.} \hfill \today
