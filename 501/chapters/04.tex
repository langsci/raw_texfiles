\ChapterAndMark{Indirect and Incomplete Negation} 
\label{ch:4}

In this chapter we shall discuss a great many different ways of expressing negative ideas through indirect or roundabout means, and finally words that without being real negatives express approximately the same thing as the ordinary negative adverb.

\addsec{Indirect negation} \label{sec:indirect-negation}
\is{indirect negation|(}

\AddSubSection{Questions} \label{sec:questions}
\is{interrogatives|(}
\is{questions|(}
\is{grammaticalization|(}

Questions may be used implying a negative statement: (a) nexal question, e.g. \textit{Am I the guardian of my brother?} (`I am not\dots'); inversely a negative question means a positive assertion: \textit{Isn't he stupid} (`he is (very) stupid'); --- and (b) special question, e.g. \textit{Who knows?} (`I do not know', or even `No one knows'); \textit{And what should they know of England who only England know?} (`they know nothing', \href{https://archive.org/details/in.ernet.dli.2015.48402/page/n193/mode/2up?q=\%22know+of+england\%22&view=theater}{Kipling}); \textit{where shall I go?} (`I have nowhere to go').
 
 Examples of the first \refp{ex:04-01}.

\ea \label{ex:04-01}
\ea
``Would you know him again if you saw him?''\\``Shall I ever forget him!''
\hfill(\href{https://archive.org/details/in.ernet.dli.2015.469714/page/n33/mode/2up?q=\%22Shall+I+ever+forget+him%21\%22&view=theater}{Shaw, \textit{Arms} 1})
\ex
Could I see his face, I wept so \phantom{x} (`I wept so much that I could not see')\\
\hfill(\href{https://archive.org/details/auroraleighpoem00brow/page/358/mode/2up?q=\%22Could+I+see+his+face%2C+I+wept+so\%22&view=theater}{E. B. Browning, \textit{Aurora} 326})
\ex
Well, didn't I just get a wigging from the Sister now!\\\hfill(\href{https://archive.org/details/christianstory00cainrich/page/40/mode/2up?q=\%22get+a+wigging\%22&view=theater}{Caine, \textit{Christian} 34})
\ex
``Did you hit Rabbits-Eggs, Stalky?'' --- ``Did I jolly well not?''\\\hfill(\href{https://archive.org/details/stalkyandco015455mbp/page/n69/mode/2up?q=\%22hit+Rabbits-Eggs\%22&view=theater}{Kipling, \textit{Stalky} 72}) % Restoring ", Stalky?"
\z
\z

\textit{Must I not?} (`I must'), e.g. \refp{ex:04-05}.

\ea \label{ex:04-05}
\ea
Must I not die? \phantom{x} (`I must')\hfill(\href{https://archive.org/details/cainmystery01byro/page/10/mode/2up?q=\%22Must+I+not+die%3F\%22&view=theater}{Byron, \textit{Cain} 1.1})
\ex
It has been a wilderness from the Creation. Must it not be a wilderness for ever?\hfill(\href{https://archive.org/details/snowimageothertw0000hawt/page/52/mode/2up?q=\%22wilderness+from+the+creation\%22&view=theater}{Hawthorne, \textit{Image} 53})
\ex
Must I not have a voice in the matter, now I am your wife {\dots}?\\\hfill(\href{https://archive.org/details/returnofthenativ00harduoft/page/224/mode/2up?q=\%22Must+I+not+have+a+voice\%22&view=theater}{Hardy, \textit{Return} 292}) % Dots added for "and the sharer of your doom", which OJ cut
\z
\z

\textit{Won't I?} (`I will') \refp{ex:04-08}.

\ea \label{ex:04-08}
\ea
``And wilt thou?'' --- ``Will I not?'' \phantom{x}(`I will')\hfill(\href{https://archive.org/details/in.ernet.dli.2015.285363/page/n459/mode/2up?q=\%22And+wilt+thou%3F++Will+I+not\%22&view=theater}{Byron, \textit{Sardanapalus} 3.1})
\ex
Oh my eye, won't I give it to the boys!\hfill(\href{https://archive.org/details/lifeadventuresofdickrich/page/106/mode/2up?q=\%22won%27t+I+give+it+to+the+boys\%22&view=theater}{Dickens, \textit{Nicholas} 95})
\ex
There's Waddy---Sam Waddy making up to her; won't I cut him out?\\\hfill(\href{https://archive.org/details/professortale01bron/page/46/mode/2up?q=\%22sam+waddy+making+up\%22&view=theater}{Brontë, \textit{Professor} 24}) % Restoring the repetition
\ex
I say, if you went to school, wouldn't you get into rows!\\\hfill(\href{https://archive.org/details/ordealofrichardf01mere/page/118/mode/2up?q=\%22get+into+rows\%22&view=theater}{Meredith, \textit{Ordeal} 27}) % Exclamation mark restored
\ex
I never drank much Claret before. {\dots} Won't I now, though! Claret is my wine.\hfill(\href{https://archive.org/details/ordealofrichardf01mere/page/120/mode/2up?q=\%22Won%27t+I+now%2C+though%21\%22&view=theater}{ibid 27}) % dots added for "Ripon was off again". Capital "C" for "Claret".
\z
\z
 
The reply in \refp{ex:04-13} clearly shows that the other person rightly understood the first speaker's seeming question as a negative statement: `there never was {\dots}'.

\ea \label{ex:04-13}
``Was there ever a more mild-mannered, Sunday-school young man?'' --- ``It is true.''\hfill(\href{https://archive.org/details/returnofsherlock0000acon/page/48/mode/2up?view=theater&q=\%22was+there+ever%22}{Doyle, \textit{Return} 75}) % "Sunday-school" restored
\z
\is{interrogatives|)}
 
In the same way naturally in other languages as well. In Danish this form has the curious effect that after \textit{så sandelig} the same meaning may be expressed with and without \textit{ikke}, the word order being the same, only in the latter case we have the slight rising of the tone indicating a question:\is{intonation} \refp{ex:04-14}. In the same way in Norwegian and Swedish: \refp{ex:04-15}. (In none of these quotations, however, there is any question mark.)

\ea \label{ex:04-14}\il{Danish!ikke@\textit{ikke}}
\gll Ja, saa sandelig er det ikke ham! Og han kommer her til mig!\\
 yes so certainly is it not him and he comes here to me\\
\glt `Yes, certainly it is him! And he is coming here to me!'\hfill(\href{https://archive.org/details/guds-fred/page/62/mode/2up?q=\%22saa+sandelig+er+det+ikke\%22&view=theater}{Nansen, \textit{Fred} 62})
\z

\ea \label{ex:04-15}
\ea
\gll Jo så sandelig glemte jeg det ikke\\
 yes so certainly forgot I it not\\
\glt `Yes, I certainly have forgotten it'
\hfill(\href{https://archive.org/details/vildandenskuesp00ibsegoog/page/n69/mode/2up?q=\%22glemte+jeg+det+ikke\%22&view=theater}{Ibsen, \textit{Vildanden} 61})\il{Danish!ikke@\textit{ikke}}

%% SG: For Ibsen, you could also refer to the online edition of his works from the University of Oslo: https://www.ibsen.uio.no/DRVIT_Vi%7CViht.xhtml -- note that in this and the Swedish example, the negation is "superfluous" and not included in the idiomatic translation. The characters in fact do forget and see something

\ex\il{Swedish!\textit{ej}}\il{Swedish!\textit{sannerligen}}
\gll Nä sannerligen ser han ej något svart och stort komma\\
 well certainly sees he not something black and big come\\
\glt `Well, he certainly is seeing  something big and black coming'\\
\hfill(\href{https://archive.org/details/arkivkopia.se-littbank-LagerlofS_GostaBerling1/page/n61/mode/2up?q=\%22sannerligen+ser+han+ej+n%C3%A5got\%22&view=theater}{Lagerlöf, \textit{Saga} 1.153}) % ??S Visually, not pleasing. See if we can find a better alternative for this work.
%Brett: I'm fine with this
% PE: I meant that the page linked to looks a bit iffy. Yes, we can link to it. But if something preferable pops up before publication time, let's put it in.
\z
\z

\is{Mad magazing@\textit{Mad} magazine sentences|(}
\is{small clauses}
\is{incredulity expressions}
A variant of these nexal questions is the elliptical use of a subject and a (``loose'') %Brett: OJ has single quotes.
infinitive (see \cite[\href{https://archive.org/details/progressinlangua00jespuoft/page/204/mode/2up?view=theater}{§164f}]{jespersen1894progress}) % OJ's [ ] changed to ( ).
with a rising intonation, implying that it is quite impossible to combine the two ideas: \refp{ex:04-17}.

\ea \label{ex:04-17}
\ea
My owne flesh and blood to rebell\hfill(\href{https://internetshakespeare.uvic.ca/doc/MV_F1/scene/3.1/index.html#tln-1245}{Shakespeare, \textit{Merch} 3.1.37}) % Removing final exclamation point; it's not in the folio
\ex
``You make fat rascalls, Mistris Dol.'' --- ``I make them? Gluttonie and diseases make them, I make them not.''\hfill(\href{https://internetshakespeare.uvic.ca/doc/2H4_F1/scene/2.4/index.html#tln-1065}{Shakespeare, \textit{H4B} 2.4.45}) 
\ex
Oh la! a footman have the spleen!\hfill(\href{https://archive.org/details/beauxstratagema00fitzgoog/page/n81/mode/2up?q=\%22Oh+la%21+a+footman+have+the+spleen\%22&view=theater}{Farquhar, \textit{Stratagem} 341}) % Restoring final "!"
\ex
you amaze me. Such a girl as you want jewels!\hfill(\href{https://archive.org/details/shestoopstoconqu03gold/page/68/mode/2up?q=\%22Such+a+girl+as+you+want+jewels\%22&view=theater}{Goldsmith, \textit{Stoops}})
\ex
Why! they don't come down here to dine, you know, they only make believe to dine. \textit{They} dine here, Law bless you! They go to some of the swell clubs\hfill(\href{https://archive.org/details/dli.ministry.14127/page/285/mode/2up?q=\%22dine+you+know\%22&view=theater}{Thackeray, \textit{Pendennis} 2.130}) % Thackeray italicizes "They"
\ex
My son and heir marry a beggar's girl out of the gutter. D--- him, if he does, let him buy a broom and sweep a crossing.\\\hfill(\href{https://archive.org/details/vanityfairanove03thacgoog/page/n111/mode/2up?q=\%22My+son+and+heir+marry+a\%22&view=theater}{Thackeray, \textit{Vanity} 180}) % In the original: " {\dots}, if he does, let him buy a broom and sweep a crossing." It seems to me (PE) that "if he does" is less likely to be the prodosis for what precedes it than for what follows it. Shall we restore it?
%Brett: yes. % Peter: Done
\ex
``Gracious God!'' he cried out. ``My boy insult a gentleman at my table!''\hfill(\href{https://archive.org/details/newcomes00unkngoog/page/n164/mode/2up?q=\%22my+boy+insult+a+gentleman+at+my+table\%22&view=theater}{Thackeray, \textit{Newcomes} 163}) % OJ has this as one sentence; the version linked to at archive.org has it as two.
\ex
Me to sing to naked men!\hfill(\href{https://archive.org/details/sim_new-mcclures-magazine_1895-01_4_2/page/132/mode/2up?q=\%22Me+to+sing+to+naked+men\%22&view=theater}{Kipling, \textit{Second} 72}) % Visually unpleasant; can we get something better for this Kipling work?
%Brett: is this better? https://archive.org/details/sim_new-mcclures-magazine_1895-01_4_2/page/132/mode/2up?q=\%22Me+to+sing+to+naked+men\%22&view=theater % Peter; Good!
\ex
A man not know what he had on! No, no!\hfill(\href{https://archive.org/details/manofproperty00gals/page/8/mode/2up?q=\%22A+man+not+know+what+he+had+on\%22&view=theater}{Galsworthy, \textit{Man} 8})
\z
\z \is{Mad magazing@\textit{Mad} magazine sentences|)}

\is{interrogatives}
Examples of negative statements expressed by questions containing an interrogative pronoun: \refp{ex:04-26}.

\ea \label{ex:04-26}
\ea
What bootes it thee to call thyselfe a sunne?\hfill(\href{https://internetshakespeare.uvic.ca/doc/Tit_F1/scene/5.3/index.html#tln-2515}{Shakespeare, \textit{Tit} 5.3.18}) % OJ has "thy selfe"; uvic.ca has "thyselfe"
\ex
``Why she hath not writ to me?'' --- ``What need shee, When she hath made you write to your selfe?''\hfill(\href{https://internetshakespeare.uvic.ca/doc/TGV_F1/scene/2.1/index.html#tln-540}{Shakespeare, \textit{Gent} 2.1.158}) % "Valentine", "Why" and the question mark restored. (OJ just has a rather half-baked gloss of the question.)
\ex
Who cares? \phantom{x} (`no one cares', or `I don't care')
\z
\z

\is{conversion|(}
In this way \textit{what not}, especially after a long enumeration, comes to mean `everything' (double negation), as in \refp{ex:04-29}.

\ea \label{ex:04-29}
\ea
Marrie peace it boads, and loue, and quiet life, An awfull rule, and right supremicie: And to be short, what not, that's sweet and happie.\\\hfill(\href{https://internetshakespeare.uvic.ca/doc/Shr_F1/scene/5.2/index.html#tln-2660}{Shakespeare, \textit{Shr} 5.2.110}) % "boads" restored
\ex
 silver, gold, pearls, precious stones, and what not\\\hfill(\href{https://archive.org/details/bunyanspilgrims00moffgoog/page/116/mode/2up?q=\%22precious+stones\%22&view=theater}{Bunyan, \textit{Progress} 121}) % Dots because this is the ending of a longer list. OJ italicizes "what not" here, but doesn't do so for the other examples (and Bunyan doesn't do it), so I've removed the underlining 
\ex
Robin, who was butler, valet-de-chambre, footman, gardener, and what not\hfill(\href{https://archive.org/details/talesmylandlord09scotgoog/page/n180/mode/2up?q=\%22Robin%2C+who+was+butler\%22&view=theater}{Scott, \textit{Mortality} 68}) % "Valet-de-chambre", which OJ cut, restored.
\ex
As now we put our money into railways or what not? so then the keen man of business took shares in the new ship\\\hfill(\href{https://archive.org/details/ourcolonialexpan00seel/page/58/mode/2up?q=\%22what+not\%22&view=theater}{Seeley, \textit{Expansion} 111})
\ex
Whether Newfoundland, mastiff, bloodhound, or what not, it was impossible to say\hfill(\href{https://archive.org/details/farfrommaddingcr0000thom_h4k2/page/250/mode/2up?q=\%22what+not\%22&view=theater}{Hardy, \textit{Far} 314})
\ex
Talking of Exhibitions, World's Fairs, and what not\\\hfill(\href{https://archive.org/details/lifeslittleironi00harduoft/page/152/mode/2up?q=\%22what+not\%22&view=theater}{Hardy, \textit{Ironies} 179})
\ex
if I want five shillings for a charity or what not\hfill(\href{https://archive.org/details/joyplayonletteri00gals/page/106/mode/2up?q=\%22what+not\%22&view=theater}{Galsworthy, \textit{Joy} 1}) % Galsworthy italicizes the "I"; I suppose we have to ignore this.
\ex
whether he be Hindu or Mohammedan or what-not in religion\\\hfill(\href{https://archive.org/details/worldswork24gard/page/640/mode/2up?q=\%22or+what-not+in+religion\%22&view=theater}{news 1912})
\ex
he wont consent unless they send letters and invitations and congratulations and the deuce knows what not\hfill(\href{https://archive.org/details/widowershousesun00shaw/page/18/mode/2up?q=\%22+what+not\%22&view=theater}{Shaw, \textit{Houses} 14})
\ex
they would {\dots} give me what-not for to eat and drink\\\hfill(\href{https://archive.org/details/personalhistory05dickgoog/page/n239/mode/2up?q=\%22what-not\%22&view=theater}{Dickens, \textit{David} 544} (vulgar)) % Dots added to show omission
\z
\z

Hence \textit{a what-not} as a substantive, `piece of furniture with shelves for nick-nacks' \refp{ex:04-39}.

\ea \label{ex:04-39}
on a what-not at the door-side of the room another photograph stood.\\\hfill(\href{https://archive.org/details/christianstory00cainrich/page/476/mode/2up?q=\%22what-not\%22&view=theater}{Caine, \textit{Christian} 399}) % OJ has "whatnot"; Caine has "what-not".
\z

\textit{What not} is used as a verb and adjective in \refp{ex:04-40}.

\ea \label{ex:04-40}
\ea
Had been neglected, ill-used, and what not\hfill(\href{https://archive.org/details/workslordbyron10unkngoog/page/362/mode/2up?view=theater&q=\%22and+what+not%22}{Byron, \textit{Juan} 8.110})
\ex
the Government of London, or the Council, or the Commission, or what not other barbarous {\dots} body of fools\hfill(\href{https://archive.org/details/newsfromnowher00morr/page/62/mode/2up?q=\%22what+not\%22&view=theater}{Morris, \textit{News} 46}) % "of London" restored; "consul" corrected to "council". (Morris writes "other barbarous, half-hatched body", and capitalizes "Government", "Council", and "Commission".) 
\z
\z
\is{conversion|)}

\is{interrogatives|(}
\emergencystretch=1em
Pronominal questions implying a negative are, of course, frequent in all languages: Danish \textit{hvem veed?} French \textit{qui sait?} Spanish \textit{quién sabe?} (`no one knows'), etc.

Here belong also questions with \textit{why}: \textit{Why should he?} (`there is no reason why he should', `he should not'); \textit{Why shouldn't he?} (`he should').

Note the continuation in \refp{ex:04-42}.

\ea \label{ex:04-42}
Why should she, any more than I {\dots}?\hfill(\href{https://archive.org/details/septimus00unkngoog/page/n193/mode/2up?q=\%22Why+should+she%2C+any+more+than+I%3F\%22&view=theater}{Locke, \textit{Septimus} 197}) % Added dots, as this isn't the end of the sentence
\z

In the following two quotations the continuation \textit{and not} shows clearly that the negative questions are to be taken as positive statements \refp{ex:04-43}. % "As" here replaces OJ's "=". 

\ea \label{ex:04-43}
\ea
why should he not be accepted for what he is, and not for what he is not\hfill(\href{https://archive.org/details/compleatenglishg00defouoft/page/28/mode/2up?q=\%22he+not+be+accepted\%22&view=theater}{Defoe, \textit{Gentleman} 28})
\ex
Doesn't one develop, daddy, through one's passions, and not through one's renunciations?\hfill(\href{https://archive.org/details/arundel00bens/page/38/mode/2up?q=\%22and+not+through+one%27s+renunciations%3F\%22&view=theater}{E. F. Benson, \textit{Arundel} 40}) % Restored ", daddy,"
\z
\z

In colloquial Danish one hears pretty frequently questions containing \textit{næsten}, which is only justified logically if the sentence is transposed into the corresponding negative \refp{ex:04-45}.

\ea \label{ex:04-45}
\ea\il{Danish!ikke@\textit{ikke}}
\gll Kan du næsten se dærhenne? {} (`du kan visst næsten ikke se')\\
 can you almost see {over there} {} you can surely almost not see\\
\glt `Can you even see over there?' % ??? PE: OJ paraphrases late Modern Danish in late Modern Danish; and we gloss both the original and the paraphrase in ModE. I suggest that we relegate his paraphrase to a footnote.

\ex
\gll hvordan kan her næsten blive plads til os allesammen?\\
 how can here almost become space for us all\\
\glt `How is there even going to be room for all of us here?'

\ex
\gll Hvad skulde saadan een næsten forslaa tiden med \textit{---} andet end med det unaturlige!\\
 what should such one almost pass time.\DEF{} with {} other than with the unnatural\\
\glt `How else would such a person pass the time---other than with the unnatural!'
\hfill(\href{https://archive.org/details/ordbogoverdetdan15dansuoft/page/97/mode/2up?q=\%22Hvad+skulde+saadan+een+n%C3%A6sten+forslaa+tiden\%22&view=theater}{Knudsen, \textit{Urup} 104}) % ??? I can hardly understand what "passing the time with the unnatural" might mean. (A contrast with "spending/wasting one's time on the unnatural".) Let's ask an L1 Danish speaker. ??S Link goes to an example within a dictionary. We should look for some web version of the original work.
%Brett: can't find anything
%PE: Let's look again later. (I tried and failed in Sept '24.)
%% SG: The quote is sarcastic. It's from a novel about a schoolteacher (Urup) who is a devout Lutheran and very critical of the modern world. In this passage he is describing the behaviour of a lower-class villager who, according to Urup, cannot be expected to do anything other than unnatural things (it's left to the reader's imagination exactly what these unnatural things are) \\ %% PE: Excellent. Well, the translation seems accurate, and I suppose the sarcasm isn't essential to what OJ is trying to say.

\ex\il{Danish!ikke@\textit{ikke}}
\gll Tror jeg næsten ikke, det er første gang, solen skinner for mig paa denne egn.\\
 think I almost not it is first time sun.\DEF{} shines for me in this region\\
\glt `I think this must be the first time the sun is shining for me in this region.'
\hfill(\href{https://www.henrikpontoppidan.dk/text/kilder/boeger/landsbybilleder/vinterbillede.html}{Pontoppidan, \textit{Billeder} 162}) % Link is to some visually horrible newish thing; let's look for a better version of the work.
%Brett: how about https://www.henrikpontoppidan.dk/text/kilder/boeger/landsbybilleder/vinterbillede.html % Peter: Excellent; done! ??? PE: But again this is pragmatically bizarre. For "I almost do not believe", perhaps "I find it hard to believe"; "I can hardly believe"? But "the sun shines for me in this region"? Sounds to me like something out of a parody phrase book (cf https://en.wikipedia.org/wiki/My_postillion_has_been_struck_by_lightning ).

%% SG: The context in the short story is this: the character in question (a bishop) is visiting a parish during a very rainy period in the winter. After a dinner, he suddenly gets up and wonders at the sunshine outside. Formally, the sentence is a (rhetorical) question with a "superfluous" negation like some of Jespersen's earlier examples. The more idiomatic meaning is something like "I think this must be the first time..." or "I'm quite sure this is the first time..."
\z
\z

A similar phenomenon is the use of \textit{heller} (`either'), which is not common except with a negative, in \refp{ex:04-49}.
%% SG: I'd add the literal translation of _heller_ (`either') either in the text itself or a footnote. In the example, _Hvorledes ... heller_ rather translates to 'how else'
%%BR: done

\ea \label{ex:04-49}
\gll Hvorledes skulde de heller forstaa kæmper med lyst haar?\\
 how should they either understand giants with light hair\\
\glt `How else were they to understand giants with blond hair?'\\
\hfill(\href{https://archive.org/details/bren00jensgoog/page/n257/mode/2up?q=\%22Hvorledes+skulde+de+heller+forstaa+k%C3%A6mper+med\%22&view=theater}{Jensen, \textit{Bræen} 230})
\z
\is{interrogatives|)}
\is{questions|)}
\is{grammaticalization|)}

\AddSubSection{In a conditional clause} \label{sec:conditional}\is{conditional constructions|(}


\is{grammaticalization|(}
Another popular way of denying something is by putting it in a conditional clause with \textit{I am a villain} or something similar in the main clause \refp{ex:04-50}.

\ea \label{ex:04-50}
\ea If I understand thee, I am a villain\hfill(\href{https://archive.org/details/merrydevilofedmo00fabe/page/16/mode/2up?q=\%22If+I+understand+thee%2C+I+am+a+villain\%22&view=theater}{\textit{Devill} 534})
\ex I am a rogue if I drunke to day\hfill(\href{https://internetshakespeare.uvic.ca/doc/1H4_F1/scene/2.4/index.html#tln-1110}{Shakespeare, \textit{H4.A} 2.4.169}) 
\ex if I fought not with fiftie of them, I am a bunch of radish\hfill(\href{https://internetshakespeare.uvic.ca/doc/1H4_F1/scene/2.4/index.html#tln-1140}{ibid 2.4.205}) 
\ex I am a Iew if I serue the Iew anie longer\hfill(\href{https://internetshakespeare.uvic.ca/doc/MV_F1/scene/2.2/index.html#tln-670}{Shakespeare, \textit{Merch} 2.2.120})
\ex Don't you know it? No, I am a rook if I do.\hfill(\href{https://archive.org/details/bim_eighteenth-century_epicne-or-the-silent-_jonson-ben_1776/page/n37/mode/2up?q=\%22No%2C+I+am+a+rook+if+I+do\%22&view=theater}{Jonson, \textit{Epicœne} 3.195})
\z
\z

A variant is \textit{the devil take me} or \textit{I will be damned} etc. in the main clause, often with prosiopesis \textit{Be damned} or \textit{damned}; any substitute for \textit{damn} may of course be used: \refp{ex:04-55}.\largerpage[2]

\ea \label{ex:04-55}
\ea
You may converse with those two nymphs if you please, but the ------ take me if ever I do.\hfill(\href{https://archive.org/details/journaltostellae00swifuoft/page/428/mode/2up?q=\%22take+me\%22&view=theater}{J. Swift, \textit{Journal} 428}) % Surely the dash here stands not for an interruption in the thought train or similar but instead for a deleted word ("devil"). I've therefore doubled its length. (If LaTeX offers a way of constructing arbitrarily long dashes, I haven't found it.)
\ex
``We'll go into the Parks if you like.''\\``Be damned if I do''\hfill(\href{https://archive.org/details/lightthatfailed02kipluoft/page/106/mode/2up?q=\%22Be+damned+if+I+do\%22&view=theater}{Kipling, \textit{Light} 229}) % "Parks" capitalized
\ex
``Will you leave it to me, Mount?''\\``Be damned before I do!''\hfill(\href{https://archive.org/details/ordealofrichardf0000geor_q5z1/page/500/mode/2up?q=\%22Will+you+leave+it\%22&view=theater}{Meredith, \textit{Ordeal} 394}) % "Mount" restored
\ex
Darned if I know\hfill(\href{https://archive.org/details/pitepicofwheatde00norruoft/page/72/mode/2up?view=theater&q=\%22darned+if+I+know%22}{Norris, \textit{Pit} 90})
\ex
I'm dashed if I know\hfill(\href{https://archive.org/details/lightthatfailed01kipluoft/page/144/mode/2up?q=\%22I%27m+dashed+if+I+know\%22&view=theater}{Kipling, \textit{Light} 121})
\ex
I’m dashed if I’ll be adopted by Sinjon\hfill(\href{https://archive.org/details/doctorsdilemmage0000bern/page/342/mode/2up?q=\%22dashed\%22&view=theater}{Shaw, \textit{Married} 283}) % OJ ascribes an instance of "I'm dashed if" (or similar) to "Shaw D", which his bibliography explains is Shaw's The Doctor's Dilemma. The closest I (PE) have got for Shaw is this in Getting Married, a play that has appeared (e.g. https://archive.org/details/doctorsdilemmage0000bern/page/342/mode/2up?q=\%22dashed%22 ) in the same book as The Doctor's Dilemma. True, there's also "I'm dashed if I will" in The Philanderer https://archive.org/details/playspleasantunp01shawuoft/page/92/mode/2up?view=theater&q=\%22dashed+if+I%22 and very likely other examples by Shaw. But on p 74 (chapter 7) of the printed book, OJ attributes "you can hardly tell who anyone is" to Shaw's "D"; but it too appears not in The Doctor's Dilemma but in Getting Married.
\ex
Dashed if I know!\hfill(\href{https://archive.org/details/ourmutualfriendc0000char/page/254/mode/2up?q=\%22dashed+if+I+know\%22&view=theater}{Dickens, \textit{Friend} 343}; \href{https://archive.org/details/evanharringtonno00mererich/page/308/mode/2up?q=\%22dashed+if\%22&view=theater}{Meredith, \textit{Harrington} 346})
\ex
ding me if I remember\hfill(\href{https://archive.org/details/silasmarnerbygeo00elio/page/110/mode/2up?q=\%22ding+me\%22&view=theater}{Eliot, \textit{Silas} 158})
\ex
Dinged ef I oughtenter be plowin'\hfill(\href{https://archive.org/details/kentuckycolonel00readrich/page/14/mode/2up?q=\%22Dinged+ef+I+oughtenter+be+plowin\%22&view=theater}{Read, \textit{Colonel} 17})
\ex
be dazed if he who do marry the maid won't hae an uncommon picture {\dots} Be jown'd if I don't learn ten new songs\hfill(\href{https://archive.org/details/returnofthenativ00harduoft/page/44/mode/2up?q=\%22dazed+if+he+who+do+marry\%22&view=theater}{Hardy, \textit{Return} 56})
\ex
hang me if I can tell\hfill(\href{https://archive.org/details/frankfairleghors00smed/page/198/mode/2up?q=\%22hang+me+if+I+can+tell\%22&view=theater}{Smedley, \textit{Frank} 1.268})
\ex
``Give me credit for a little gumption.'' --- ``Be hanged if I do!'' --- ``\textit{Be} hanged then''\hfill(\href{https://archive.org/details/lightthatfailed01kipluoft/page/98/mode/2up?q=\%22Be+hanged+then\%22&view=theater}{Kipling, \textit{Light} 83}) % Kipling italicizes the second "Be"
\ex
Blame me if it didnt come into my head once or twyst that he must be horff 'is chump\hfill(\href{https://archive.org/details/candidamystery02shawuoft/page/120/mode/2up?q=\%22that+he+must+be+horff+%E2%80%99is+chump\%22&view=theater}{Shaw, \textit{Candida} 120})
\ex
I'll be shot if I am\hfill(\href{https://archive.org/details/dukeschildren01trolgoog/page/n60/mode/2up?q=\%22I%27ll+be+shot+if+I+am\%22&view=theater}{Trollope, \textit{Children} 1.50})
\ex
I'm shot if you do\hfill(\href{https://archive.org/details/joyousadventures00lockuoft/page/94/mode/2up?q=\%22I%27m+shot+if+you+do\%22&view=theater}{Locke, \textit{Adventure}})
\ex
It does you honour. I'm blest if it don't\hfill(\href{https://archive.org/details/Chuzzlewit11/page/n33/mode/2up?q=\%22I%E2%80%99m+blest+if+it+don%E2%80%99t\%22&view=theater}{Dickens, \textit{Martin} 280})
\ex
blest if you ain't the best old fellow ever was\hfill(\href{https://archive.org/details/tombrownsschoold00hugh4/page/226/mode/2up?q=\%22blest+if+you+ain%27t+the+best+old+fellow+ever+was\%22&view=theater}{Hughes, \textit{Days} 1.220})
\z
\z

With these last sentences containing \textit{blessed} may be compared the following indirect negatives: \refp{ex:04-72}.

\ea \label{ex:04-72}
\ea
God bless you, if you ha'n't taken snuff\hfill(\href{https://archive.org/details/cu31924013200898/page/n127/mode/2up?q=\%22if+you+ha%27n%27t+taken+snuff\%22&view=theater}{J. Swift, \textit{Conversation} 92})
\ex why, Lord love my heart alive, if it ain't a treat to look at him!\\\hfill(\href{https://archive.org/details/personalhistory05dickgoog/page/n63/mode/2up?q=\%22Lord+love+my+heart\%22&view=theater}{Dickens, \textit{David} 132})
\z
\z

We have \textit{but} (`if not') in \refp{ex:04-74}. Thus often in Shakespeare; \textit{but} here might be taken as corresponding to Latin \textit{sed}, as \textit{Beshrew me} is used as a single asseveration before a main sentence, e.g. \refp{ex:04-75}.

\ea \label{ex:04-74}
Beshrew me but I loue her heartily \phantom{x} (`damn me if I do not', thus `I do')\\\hfill(\href{https://internetshakespeare.uvic.ca/doc/MV_F1/scene/2.6/index.html#tln-950}{Shakespeare, \textit{Merch} 2.6.52})
\z

\ea \label{ex:04-75}
Beshrew me, the knights in admirable fooling.\footnote{Modern editions often amend this to ``Beshrew me, the knight's in admirable fooling," meaning `Curse me, the knight is engaged in wonderful foolishness.' The First Folio's use of \textit{knights} (plural) may be a typographical error. \eds}
\hfill(\href{https://internetshakespeare.uvic.ca/doc/TN_F1/scene/2.3/index.html#tln-775}{Shakespeare, \textit{Tw} 2.3.85}) % Peter: Yes, this is what the first folio says. Am I stupid for not understanding what it means (or how "the knights in admirable fooling" can be a sentence)?
%Brett: does the fn help? % PE Yes!
\z

A curious variant is found in \refp{ex:04-76}.

\ea \label{ex:04-76}
if that ben't fair, hang fair\hfill(\href{https://archive.org/details/cu31924013200898/page/n145/mode/2up?q=\%22hang+fair\%22&view=theater}{Swift, \textit{Conversation} 110})
\z

In Danish we have corresponding expressions, such as \refp{ex:04-77}, cf. \refp{ex:04-78}. In a slightly different way \refp{ex:04-80}.

\ea \label{ex:04-77}
\gll Du må kalde mig Mads, om jeg gør det\\
 you may call me Mads if I do it\\
\glt `I am never going to do that!' (lit. `You can call me Mads if I am going to do that')
\z
%% SG: I think it's better to give both an idiomatic and a more literal translation here, as the idiomatic meaning is not in any way transparent. (By the way, it is not clear to me why the finite verb in the main clause is italicized.)
%%BR: Thanks! Not in the original. Not sure how it crept in. Fixed now.

\ea \label{ex:04-78}
\ea\il{Danish!ikke@\textit{ikke}}
\gll Jeg er aldrig ærlig, om det ikke er min gamle cammerat Andreas\\
 I am never honest if it not is my old comrade Andreas\\
\glt `I'd be lying if that isn't my old comrade Andreas!'
\hfill(\href{https://tekster.kb.dk/text/adl-texts-holb04val-shoot-workid68476}{Holberg, \textit{Pulver} 1})

\ex
\gll Jeg vil aldrig døe som en honnet kone, naar jeg de to sidste maaneder har hørt tale om andet end om politik og italienerne.\\
 I will never die as an honorable wife when I the two last months have heard talk about other than about politics and Italians\\
\glt `I'd be lying if I said that for the last two months I've heard talk of anything but politics and the Italians.' (lit. `I will never die an honorable wife...')
\hfill(\href{https://www.kb.dk/e-mat/dod/130024712320-color.pdf}{Faber, \textit{Stegekjælderen} 33})
\z
\z
%% SG: Check against original, which has "end om -- Politik og Italienerne". Note also that the correct spelling is "honnet", not "hønnet" (never seen that before, must be a typo in Jespersen)

\ea \label{ex:04-80}\il{Danish!ikke@\textit{ikke}}
\gll En skielm, der nu har flere penger {\phantom{x}} (`jeg har ikke flere penger')\\
a rogue who now has more money {\phantom{x}} I have no more money\\
\glt `As if I had more money'\hfill(\href{https://books.google.com/books?id=s94jACPhoMoC&pg=PA15&lpg=PA15&dq=\%22skielm+der+nu+har+flere+penger%22+holberg&source=bl&ots=8Ei1uK2suF&sig=ACfU3U1_nI2dJe2UXyxsNpVAxkzqY1JHkw&hl=en&sa=X&ved=2ahUKEwiSr-S2lcWEAxVJk1YBHZDVDJQQ6AF6BAgIEAM#v=onepage&q=\%22skielm%20der%20nu%20har%20flere%20penger%22%20holberg&f=false}{Holberg, \textit{Jeppe} 1.6}) % ??? PE: Another place where I think OJ and we confuse matters by adding a paraphrase in Danish and then glossing both original and paraphrase. Again I suggest both `jeg har ikke flere penger' and `I have no more money' are relegated to a footnote.
\z

By a further development the main clause may be left out entirely, and an isolated \textit{if I ever heard} comes to mean `I never heard', and \textit{if it isn't a pity} comes to mean `it is a pity'. There is a parallel in French \textit{argot}, where \textit{tu parles s'il est venu} (`you bet he has come') is an emphatic way of saying \il{French!pas@\textit{pas}}``il n'est pas venu'' (`he has not come'). English examples: \refp{ex:04-81}.

\ea \label{ex:04-81}
\ea
as I am a lady, if he did not make me blush so that mine eyes stood a water \phantom{x} (`he made me blush')\hfill(\href{https://archive.org/details/representativee02unkngoog/page/444/mode/2up?q=\%22as+i+am+a+lady%2C+if\%22&view=theater}{\textit{Eastward} 444})
\ex
Mercy! if ever I heard the like from a lady\hfill(\href{https://archive.org/details/bim_eighteenth-century_sir-charles-grandison-_richardson-samuel_1780_1/page/52/mode/2up?q=\%22Mercy%21+if+ever+I+heard+the+like+from+a+lady\%22&view=theater}{Richardson, \textit{Grandison} 50})
\ex
I declare if it isn't a pity\hfill(\href{https://archive.org/details/lifeadventuresofdickrich/page/140/mode/2up?q=\%22declare+if+it+isn%27t+a+pity\%22&view=theater}{Dickens, \textit{Nicholas} 127})
\ex
If there isn't Captain Donnithorne and Mr Irwine a-coming into the yard!\hfill(\href{https://archive.org/details/adambede01eli/page/142/mode/2up?q=\%22If+there+isn%E2%80%99t+Captain+Donnithorne\%22&view=theater}{Eliot, \textit{Adam} 65}) % Restored "and Mr Irwine"
\ex
Why, Tess, if there isn't thy father riding hwome in a carriage\\\hfill(\href{https://archive.org/details/in.ernet.dli.2015.350984/page/n25/mode/2up?q=\%22Why%2C+Tess+Durbeyfield\%22&view=theater}{Hardy, \textit{Tess} 13}) % "hwome" isn't a typo
\ex
``Now if this isn't too bad!'' he exclaimed in a thick voice. ``If this isn't monstrously unkind!''\hfill(\href{https://archive.org/details/newgrubstreetnov02gissuoft/page/96/mode/2up?q=\%22%E2%80%98Now+if+this+isn%E2%80%99t+too+bad%21%E2%80%99\%22&view=theater}{Gissing, \textit{Grub} 196})
\ex
'Pon me word, if this ain't what comes of trusting a woman\\\hfill(\href{https://archive.org/details/lostpropertystor00ridgrich/page/106/mode/2up?q=\%22%27pon+me+word\%22&view=theater}{Ridge, \textit{Property} 252})
\ex
Well, I'm sure! if this is English manners!\hfill(\href{https://archive.org/details/johnbullsotheris0000shaw/page/102/mode/2up?q=\%22Well%2C+I%E2%80%99m+sure%21+if+this+is+English+manners%21\%22&view=theater}{Shaw, \textit{Island} 102})
\ex
If Dr. Davidson isna comin' up the near road\\\hfill(\href{https://archive.org/details/daysofauldlangsy0000ianm/page/112/mode/2up?q=\%22If+Doctor+Davidson+isna+comin%E2%80%99+up+the+near+road%21\%22&view=theater}{Maclaren, \textit{Days} 110}, also 47, 107, 169) % Should try to find those three additional examples. Additionally: (i) OJ's "MacLaren", with its capital L, is mistaken. (ii) The sentence neither ends here nor ends with an exclamation mark. (iii) OJ's "D." was the writer's "Davidson".
%Brett: No, I think the reader can look for those three additional examples if they wish. % Peter: I thought I'd try anyway. What OJ may have found on p47 of his copy may be what I find ("...if he didna stop...") on p50 of the copy at Internet Archive; but I can't find the other two. 
\ex
Well, if this don't lick cock-fighting!\hfill(\href{https://archive.org/details/TheStrandMagazineAnIllustratedMonthly/TheStrandMagazine1895aVol.IxJan-jun/page/n523/mode/2up?q=\%22if+this+don%27t+lick\%22&view=theater}{Doyle, \textit{How} 511}) 
\ex
My goodness! --- if I ain't all tired a'ready!\hfill(\href{https://archive.org/details/martineden00londiala/page/274/mode/2up?q=\%22My+goodness%21+%E2%80%94+if+I+ain%E2%80%99t+all+tired+a%E2%80%99ready%21\%22&view=theater}{London, \textit{Martin} 276})
\ex
Well, if I've hardly patience to lie in the same bed!\\\hfill(\href{https://archive.org/details/mrscaudlescurtai00jerruoft/page/74/mode/2up?q=\%22Well%2C+if+I%E2%80%99ve+hardly+patience+to+lie+in+the+same+bed%21\%22&view=theater}{Jerrold, \textit{Lectures} 56})
\z
\z

In Danish and Norwegian with \textit{om} very often preceded by some adverb of asseveration: \refp{ex:04-93}.

\ea \label{ex:04-93}
\ea
\gll Næ, om jeg gjorde det!\\
 no if I did it\\
\glt `No, as if I would do that!'

\ex\il{Danish!lidt@\textit{lidt}}
\gll ``De lovte før At spede lidt til.'' --- ``Nej, om jeg gør!''\\
 you promised before to contribute {a little} to {} no if I do\\
\glt `You promised before to contribute a little bit.' --- `Not like I will!'\\
\hfill(\href{https://archive.org/details/peergyntetdrama00hagegoog/page/n213/mode/2up?q=\%22spede+lidt\%22&view=theater}{Ibsen, \textit{Peer} 195})

\ex\il{Danish!ikke@\textit{ikke}}
\gll ``Kan du ikke mindes det nu længer?'' --- ``Nej, så sandelig om jeg kan!''\\
 can you not remember it now longer {} no so truly if I can\\
\glt `Can you no longer remember it?' --- `No, truly I can't!'\\% Peter: How about "Can you no longer remember it?"?
%Brett: yes, good. I've changed it.
\hfill(\href{https://archive.org/details/nrviddevgner00ibsegoog/page/n158/mode/2up?q=\%22ikke+mindes\%22&view=theater}{Ibsen, \textit{Når} 145})

\ex
\gll men nei saagu' om jeg ved, hvad jeg har gjort\\
 but no damn if I know what I have done\\
\glt `but no, I'd be damned if I knew what I've done'
\hfill(\href{https://books.google.co.jp/books?id=FrQNAAAAYAAJ&newbks=1&newbks_redir=0&printsec=frontcover&redir_esc=y#v=onepage&q=\%22hvad%20jeg%20har%20gjort\%22&f=false}{Kielland, \textit{Fortuna} 40})

\ex\il{Danish!ingen@\textit{ingen}}
\gll men ved gud! om jeg vilde undvære oppositionen, ingen af os vilde undvære den\\
 but by God if I would {do without} opposition.\DEF{} none of us would {do without} it\\
\glt `but by God, as if I would want to do without the opposition, none of us would want to do without it'
\hfill(\href{https://www.dansketaler.dk/tale/viggo-hoerups-grundlovstale-1889}{Hørup, speech 2.267})

%% The ref. here is almost certainly to vol. 2, p. 267 in "V. Hørup i Skrift og Tale" (https://search.worldcat.org/title/1452179130) - it seems that only vol. 1 has been digitized, unfortunately.

\ex\il{Danish!ikke@\textit{ikke}}
\gll Og ja, så min sæl, om jeg ikke også ser William sidde derovre\\
 and yes so my soul if I not also see William sit {over there}\\
\glt `And yes, by my soul, if I don't also see William sitting over there'\\
\hfill(\href{https://books.google.co.jp/books?id=7X0_AAAAIAAJ&newbks=1&newbks_redir=0&printsec=frontcover&dq=Niels+M%C3%B8ller.+Koglerier&hl=ja&redir_esc=y#v=onepage&q=\%22ser%20William%20sidde%20derovre\%22&f=false}{N. Møller, \textit{Koglerier} 297}) % ??? PE: If I understand correctly, the speaker does see William. If this is right, then not also seeing him is counterfactual. Change "don't" to "didn't"?

\ex
\gll Om det just er sundt at ligge og døse i saadan en hundekulde\\
 if it exactly is healthy to lie and doze in such a {freezing cold}\\
\glt `I wonder if it's really healthy to be napping in such a freezing cold'\\
\hfill(\href{https://books.google.com/books?id=RvQNAAAAYAAJ&printsec=frontcover&dq=herman+bang+h%C3%A5bl%C3%B8se+sl%C3%A6gter&hl=en&newbks=1&newbks_redir=0&sa=X&redir_esc=y#v=onepage&q=herman%20bang%20h%C3%A5bl%C3%B8se%20sl%C3%A6gter&f=false}{H. Bang, \textit{Slægter} 357})
\z
\z

In the same way in German \refp{ex:04-100}, and in Dutch \refp{ex:04-101}. Cf. French (with an oath) \refp{ex:04-102}.

\ea \label{ex:04-100}
\gll Ob ich das verstehen kann!\\
 if I that understand can\\
 \glt `As if I could understand that'
\z

\ea \label{ex:04-101}
\gll Of ik niet besta! Drommels goed.\\
 as if not exist devilishly good\\
\glt `As if I don't exist! Devilishly good.'\hfill(\href{}{van Eeden, \textit{Johannes} 115})\footnote{Jespersen perhaps made an editorial slip here. \textit{Of ik besta! Drommels goed!} (without \textit{niet}) is what appears in the editions of the book that we have consulted. \eds} % PE: Consulted: https://www.dbnl.org/tekst/_nie002nieu01_01/_nie002nieu01_01_0010.php and https://archive.org/details/dekleinejohannes0000eede_y1y2/page/94/mode/2up?q=Drommels&view=theater (if anyone were to ask). Friend and L1 Dutch speaker Ellen Van Goethem (Kyushu Univ) confirms that whether this is ±"niet" makes little or no difference to the meaning: Whether Van Eeden's sentence is misquoted or quoted correctly, it "would indeed lead to the same conclusion: 'of course I exist!' OJ's revised version (with 'niet') possibly is a tad bit of a stronger statement." 
\z

\ea \label{ex:04-102}
\gll Du diable si je me souviens de son nom\\
 the devil if I myself remember of his name\\
\glt `The devil if I remember his name'
\\\hfill(see~p.~\pageref{sec:the-devil} below on \textit{the devil}; \href{https://gallica.bnf.fr/ark:/12148/bpt6k64180x.pdf}{Droz, \textit{Monsieur} 3})% ??S This PDF is unwieldy; we should look for a better alternative. Also, text search within this PDF is poor; the sentence occurs at the very start of the main text.
\z

\textit{As if} is often used in the same way: \refp{ex:04-103}. In the same way in other languages: \refp{ex:04-104}.

\ea \label{ex:04-103}
``What college?''\\``As if you knew not!'' \phantom{x} (`of course you know')\hfill(\href{https://archive.org/details/in.ernet.dli.2015.46505/page/n5/mode/2up?q=\%22What+college%3F\%22&view=theater}{Jonson, \textit{Epicœne}})
\z

\ea \label{ex:04-104}
\ea\il{Danish!ikke@\textit{ikke}}
\gll Somom du ikke vidste det!\\
 {as if} you not knew it\\
\glt `As if you didn't know it!'

\ex\il{German!nicht@\textit{nicht}}
\gll Als ob du es nicht wüsstest!\\
 as if you it not knew\\
\glt `As if you didn't know it!'

\ex\il{French!ne@\textit{ne}}\il{French!pas@\textit{pas}}
\gll Comme si tu ne savais pas!\\
 as if you not knew not\\
\glt `As if you didn't know!' % ??? PE: Should we identify these as Danish, German, and French respectively? (Back in 1917, no; but a century later this book is likely to be read by Indians, Mongolians, and others; so I dunno.)
\z
\z
\is{conditional constructions|)}
\is{grammaticalization|)}


\AddSubSection{Imperatives} \label{sec:imperatives}
\is{imperatives|(}
\is{grammaticalization|(}

In \refp{ex:04-107}, \textit{let me see you play} means the same as `don't play'; a threatening `and I shall punish you' is left out after \textit{let me see}, etc. 

\ea \label{ex:04-107}
Hence both twaine. And let me see you play me such a part againe.\\\hfill(\href{https://archive.org/details/roisterdoister00udalgoog/page/n44/mode/2up?view=theater&q=\%22hence%2C+both%22}{\textit{Roister} 38})
\z

More often we have the imperative \textit{see} (or \textit{you see}) with an \textit{if}-clause: \textit{see if I don't} (`I shall'): \refp{ex:04-108}.

\ea \label{ex:04-108}
\ea
see if the fat villain haue not transform'd him ape\\\hfill(\href{https://internetshakespeare.uvic.ca/doc/2H4_F1/scene/2.2/index.html#tln-850}{Shakespeare, \textit{H4.B} 2.2.77})
\ex
I see such a fine girl sitting in the corner {\dots}; see if I don't get her for a partner in a jiffy!\hfill(\href{https://archive.org/details/professortale01bron/page/46/mode/2up?q=\%22I+see+such+a+fine+girl+sitting+in+the+corner\%22&view=theater}{Brontë, \textit{Professor} 27})
\ex
Make your fortune, see if you won't\hfill(\href{https://archive.org/details/newcomesmemoirso03thac/page/138/mode/2up?q=\%22Make+your+fortune%2C+see+if+you+won%E2%80%99t\%22&view=theater}{Thackeray, \textit{Newcomes} 529})
\ex
now I'll get the day fixed; you see if I don't\hfill(\href{https://archive.org/details/anoldmanslove00trolgoog/page/n226/mode/2up?q=\%22I%E2%80%99ll+get+the+day+fixed\%22&view=theater}{Trollope, \textit{Love} 137})
\ex
I shall rise to the occasion, see if I don't\hfill(\href{https://archive.org/details/newgrubstreetnov01gissuoft/page/132/mode/2up?q=\%22shall+rise+to+the+occasion%2C+see+if+I+don%E2%80%99t+\%22&view=theater}{Gissing, \textit{Grub}. 64})
\ex 
But I'll prove my case. You see if I don't.\hfill(\href{https://archive.org/details/loveandmrlewisha00welluoft/page/94/mode/2up?q=\%22see+if\%22&view=theater}{Wells, \textit{Love} 94})
\z
\z

Exactly the same phrase is usual in Danish, see, e.g. \refp{ex:04-114}, whence \refp{ex:04-115}.

\ea \label{ex:04-114}
\gll Stat op, her Ioen, och gach her-ud! See, om jeg gør! sagde Ioen\\
stand up Sir Ian and go here-out see if I will said Ian\\ % OJ uses quotation marks, but the cited source does not.
\glt ``Get up, Sir Ian, and get out here!'' --- ``See if I do!'' said Ian\hfill(\href{https://archive.org/details/danmarksgamlefo03denmgoog/page/52/mode/2up?view=theater}{\textit{Lave og Jon}}) % Seemingly a hybrid of verse 8 of version D (p.~59) of the ballad in the anthology that Jespersen cites (\textit{Danmarks gamle Folkeviser}) and verses 14--15 of version F (p.~62); however, Jespersen may instead be referring to some edition of the book that we have not examined.
\z

\ea \label{ex:04-115}
\ea
\gll ``Kom ud, ridder Rap, til den øvrige flok!'' --- ``Ja see, om jeg giør!'' sagde Rap\\
come out knight Rap to the remaining flock {} yes see if I do said Rap\\
\glt `Come out, Sir Rap, to the rest of the flock!' --- `Yes, see if I do!' said Rap \hfill(\href{https://archive.org/details/jensbaggesensda02unkngoog/page/n192/mode/2up?q=\%22Kom+ud%2C+Ridder+Rap%2C+til+den+%C3%B8vrige+Flok%21%E2%80%9C+%E2%80%94+%E2%80%9EJa+fee%2C+om+jeg+gj%C3%B8r%21\%22&view=theater}{Baggesen, \textit{Værker}}) % ??? PE: Quotation mark conundrum in the translation. Should this perhaps be ```Come out, knight Rap, to the rest of the flock!'' --- ``Yes, see if I do!'' said Rap' ?

\ex
\gll Du skal nok see, at bormester staaer paa pinde for dig\\
you shall indeed see that mayor stands on sticks for you\\
\glt `You will indeed see that the mayor is bending over backwards for you'
\hfill(\href{https://archive.org/details/holbergskomedie01martgoog/page/68/mode/2up?q=\%22Du+skal+nok+see%2C+at+bormester+staaer+paa+pinde+for+dig\%22&view=theater}{Holberg, \textit{Kandestøber} 5.1})

\ex
\gll Du skal nok see, at det er saa lyst klokken fire i januarii maaned\\
you shall indeed see that it is so bright {o'clock} four in January month\\
\glt `You will indeed see that it is so bright at four o'clock in January'\\
\hfill(\href{http://holbergsskrifter.dk/holberg-public/view?docId=skuespill%2FMascarade%2FMascarade_mod.page&toc.depth=1&brand=&chunk.id=act1&toc.id=act1}{Holberg, \textit{Mascarade} 1.1})
\z
\z
\is{imperatives|)}

\AddSubSection{\textit{You won't catch me doing it}} \label{sec:catch} \is{observation and non-observation|(}

A somewhat similar phrase is \textit{catch me doing it} (``you won't catch me doing it", `I shan't do it') \refp{ex:04-118}; also with \textit{at it}, \textit{at that}; in (\ref{ex:04-118}f) this is combined with the conditional way of expressing a negative.

\ea \label{ex:04-118}
\ea
Catch him at that, and hang him\hfill(\href{https://archive.org/details/cu31924013200898/page/n109/mode/2up?q=\%22hang+him\%22&view=theater}{J. Swift, \textit{Conversation} 74})
\ex
Catch you forgetting anything!\hfill(\href{https://archive.org/details/dombeyson00dick_0/page/174/mode/2up?q=\%22Catch+you+forgetting+anything\%22&view=theater}{Dickens, \textit{Dombey} 108})
\ex
Peggotty go away from you? I should like to catch her at it\\\hfill(\href{https://archive.org/details/personalhistory05dickgoog/page/n51/mode/2up?q=\%22Peggotty+go+away\%22&view=theater}{Dickens, \textit{David} 104})
\ex
Old Copas won't say a word---catch him\hfill(\href{https://archive.org/details/tombrownatoxford00hughiala/page/132/mode/2up?q=\%22Old+Copas+won%E2%80%99t+say+a+word+%E2%80%94+catch+him\%22&view=theater}{Hughes, \textit{Oxford} 127})
\ex
Catch him going down to collect his own rents! Not likely!\\\hfill(\href{https://archive.org/details/George-Bernard-Shaw-public/Widowers%27%20housesa%20play/page/34/mode/2up?q=\%22Catch+him+going+down+to+collect+his+own+rents%21+Not+likely%21\%22&view=theater}{Shaw, \textit{Houses} 34})
\ex
but if ever you catch me there again: for I was never so frightened in all my life\hfill(\href{https://archive.org/details/bim_eighteenth-century_the-history-of-tom-jones_fielding-henry_1780_5/page/78/mode/2up?q=\%22but+if+ever+you+catch+me+there+again\%22&view=theater}{Fielding, \textit{Tom} 5.526})
\z
\z

With this may be compared the Danish phrase with \textit{lur} (`watch, observe') \refp{ex:04-124}.

%% SG: "lur" literally 'watch' or 'observe' (though the verb is hardly ever used like this anymore). I've added the literal translation between brackets.

\ea \label{ex:04-124}
\ea
\gll Talen er det eneste, der adskiller os fra dyret; saa mangen fugl synger poesi; men luur den, om den kan holde en tale, men det kan jeg!\\
speech.\DEF{} is the only that separates us from animal.\DEF{} so {many a} bird sings poetry but watch it if it can hold a speech but that can I\\
\glt `Speech is the only thing that separates us from the animals; many a bird sings poetry; but you won't catch them making a speech, but I can!'
\hfill(\href{https://tekster.kb.dk/text/adl-texts-goldschmidt03-root}{Goldschmidt, \textit{Hjemløs} 2.767}) % PE: I've changed "animal" to "animals"

\ex
\gll bladet anmodede i fredags Hørup om at tænke resten. Men lur ham, om han gør\\
paper.\DEF{} requested on Friday Hørup about to think rest.\DEF{} but watch him if he does\\
\glt `The paper requested last Friday that Hørup should consider the rest. But watch if he does.'
\hfill(Hørup 2.105) % ??S I've looked for this but failed. ??? I don't understand "think the rest". ("think about the rest" [i.e. the other matters]? "think about a rest" [i.e. consider taking a rest]? Etc)

%% SG: 'resten' means 'the rest' in the sense 'remainder'. It's kinda strange as an object to 'think', but perhaps it means 'consider, think about' here. The reference is most likely to the second volume of Hørup's writings (https://search.worldcat.org/title/1452179130), cf. earlier comment
\z
\z
\is{observation and non-observation|)}

\AddSubSection{\textit{Excuse my doing}} \label{sec:excuse}
\is{permission, requesting|(}
\is{positive becomes negative|(}

\textit{Excuse my} (\textit{me})\textit{ doing} is sometimes used in the positive sense (`forgive me for doing'), but not unfrequently in the negative sense (`forgive me for not doing'). Examples of the latter (cf. \href{https://archive.org/details/oed03arch/page/n1147/mode/2up?view=theater}{\textit{NED}, \textit{Excuse} \textit{v.}~8}, only one example (1726) of \textit{-ing}): \refp{ex:04-126}.

\ea \label{ex:04-126}
\ea
Her mother came in and said, she hoped I should excuse Sarah's coming up\hfill(\href{https://archive.org/details/liberamorisornew00hazlrich/page/108/mode/2up?q=\%22her+mother+came+in+and+said\%22&view=theater}{Hazlitt, \textit{Liber} 108})
% OJ has simply "she said", very unlike what Hazlitt writes
\ex
you will excuse my saying any thing that will criminate myself\\\hfill(\href{https://archive.org/details/talesmylandlord09scotgoog/page/n198/mode/2up?q=\%22criminate\%22&view=theater}{Scott, \textit{Mortality} 76}) % Scott has "any thing", two words.
\ex
You must excuse my telling you  \phantom{x} (`I won't')\hfill(\href{https://archive.org/details/ourmutualfriendc0000char/page/20/mode/2up?q=\%22excuse+my+telling+you\%22&view=theater}{Dickens, \textit{Friend} 28})
\ex
Excuse my rising, gentlemen {\dots} but I am very weak\\\hfill(\href{https://archive.org/details/yeastaproblem01kinggoog/page/n113/mode/2up?q=\%22excuse+my+rising+gentlemen\%22&view=theater}{Kingsley, \textit{Yeast} 64})
\ex
you must excuse my saying anything more on the subject at the present moment\hfill(\href{https://archive.org/details/24180134.2382.emory.edu/page/n69/mode/2up?q=\%22you+must+excuse+my+saying+anything\%22&view=theater}{Philips, \textit{Glass} 64})
\z
\z
\is{permission, requesting|)}
\is{positive becomes negative|)}

\AddSubSection{Ironic incredulity} \label{sec:ironic-incredulity}
\is{imperatives}
\is{incredulity expressions|(}
\is{irony|(}

Ironical phrases implying incredulity (`I don't believe what you are just saying') are frequent in colloquial and jocular speech, thus \refp{ex:04-131}. Similarly in Danish \refp{ex:04-136}.

\ea \label{ex:04-131}
\ea
Go and tell the marines!
\ex
``That's my father,'' cried Winnie.\\``Go along!'' said cook incredulously.\hfill(\href{https://babel.hathitrust.org/cgi/pt?id=nyp.33433081588182&seq=304&q1=said+cook+incredulously}{Ridge, \textit{Garland} 291}) % (No, there's no "the" in front of "cook", which isn't capitalized.)
\ex
``Oh, get out,'' protested the broker\hfill(\href{https://archive.org/details/pitepicofwheatde00norruoft/page/68/mode/2up?view=theater&q=\%22protested+the+broker%22}{Norris, \textit{Pit} 84}) % Quotation marks readded
\ex
Oh, come now\hfill(\href{https://archive.org/details/pitepicofwheatde00norruoft/page/68/mode/2up?q=\%22Oh%2C+come+now%22}{ibid 86})
\ex
``Ah, go to bed,'' protested Hirsch.\hfill(\href{https://archive.org/details/pitepicofwheatde00norruoft/page/78/mode/2up?q=\%22Ah%2C+go+to+bed%22}{ibid 98}) % Norris writes "Hirsch"; OJ abbreviates this to "H".
\z
\z

\ea \label{ex:04-136}
\ea
\gll Gå væk!\\
 Go away!\\

\ex
\gll Den må du længere ud på landet med!\\
that must you farther out into countryside.\DEF{} with\\
\glt `You'll have to try that one on someone more gullible!' (lit. `You'll have to go farther into the countryside with that one')
\z
\z

\textit{Fiddlesticks} is used either by itself (`nonsense') or after a partial repetition of some words that one wants scornfully to reject \refp{ex:04-138}.

\ea \label{ex:04-138}
\ea
{\dots} twenty pounds. What did you say? \textit{Twenty fiddlesticks?} What?\\\hfill(\href{https://archive.org/details/mrscaudlescurtai00jerruoft/page/70/mode/2up?q=\%22Twenty+fiddlesticks\%22&view=theater}{Jerrold, \textit{Lectures} 53}) % "Twenty fiddlesticks?" italicized in the original. "What did you say?" restored, and "What?" added to the end, because OJ's cut version made little sense to me (PE). 
\ex
``Good men have gone out to the mission field, auntie.''\\``Mission fiddlesticks!''\hfill(\href{https://archive.org/details/christianstory00cainrich/page/418/mode/2up?q=\%22fiddlesticks%22}{Caine, \textit{Christian} 351})
\z % Completed and amended according to what Caine actually wrote
\z

Similar exclamations in other languages are French \textit{Des navets!} and German \textit{Blech!} In Danish \textit{en god støvle} is said either by itself or after the verb \refp{ex:04-140}.

\ea \label{ex:04-140}
\ea
\gll Vilhelm smilte og forsikkrede, at man maatte opfriskes lidt efter den megen læsning. --- ``Ja, De læser nok en god støvle!''\\
 Vilhelm smiled and asserted that one must {be refreshed} {a bit} after the much reading {} yes you read indeed a good boot\\
\glt `Vilhelm smiled and said that one needs a bit of refreshment after so much reading. --- ``Yeah, like you're reading a lot!"'
\\\hfill(\href{https://tekster.kb.dk/text/adl-texts-andersen04val-root}{Andersen, \textit{O. T.} 1.88}) % Andersen didn't write "Vilhelm forsikkrede"; he wrote "Vilhelm smilte og forsikkrede". And this spans two paragraphs.
\ex
\gll han ligner Themistokles {\dots} Pyt, {\dots} Themistokles, en god støvle!\\
 he resembles Themistocles {} {no way} {} Themistocles a good boot\\
\glt `he looks like Themistocles {\dots} No way, {\dots} Themistocles, good one!'\\ % Readded one ellipsis that OJ skips
\hfill(\href{https://tekster.kb.dk/text/adl-texts-jacobjp06val-root}{Jacobsen, \textit{Niels} 299})
\ex
\gll Det viser dog ``en ærlig og redelig vilje''. Det viser en god støvle, gør det\\
 it shows though an honest and upright will it shows a good boot does it\\
\glt `But it shows ``an honest and upright will''. My arse it does.'\\
\hfill(Hørup 2.228) % ??? I've changed "ass" to "arse" because we use the latter elsewhere and because we decided to use BrEng (not that I can get worked up about this). % ??S Can we find where this is from (and, preferably, a version online)?
%Brett: I've searched but to no avail.
% PE: Tried again on 14 Sep 24: nothing.

%% SG: See earlier comments on Hørup. The second vol. of his writings has not been digitized, but the Danish royal library has the first one: https://www.kb.dk/e-mat/dod/115208033358-color.pdf -- it might be worth asking them if they can also upload the second and third vols, they are probably in the public domain already
\z
\z

Among other rebuffs implying a negative may be mentioned Danish \textit{på det lag!} (`at that level'); \textit{snak om et ting!} (`speak about a thing!'); French \textit{plus souvent!} % OJ capitalizes the "P" of "plus"; I (PE) see no particular reason for doing so. Into lowercase?
%Brett: done
(`more often'; \href{https://archive.org/details/notesetsouvenir00halgoog/page/n261/mode/2up}{Halévy, \textit{Notes} 247}, frequent). Swift in the same sense uses a word which is now considered very low: \refp{ex:04-143}.\footnote{The low word is most likely \textit{arse}. \eds} Thus also formerly in Danish, see \refp{ex:04-145}.

\ea \label{ex:04-143}
\ea
they promise me letters to the two archbishops here; but mine a--- for it all\hfill(\href{https://archive.org/details/journaltostellae00swifuoft/page/56/mode/2up?q=\%22two+archbishops\%22&view=theater}{\textit{Journal} 57})
\ex In general you may be sometimes sure of things, as that about \textit{style}, because it is what I have frequently spoken of; but guessing is mine a---, and I defy mankind, if I please.\hfill(\href{https://archive.org/details/journaltostellae00swifuoft/page/60/mode/2up?q=\%22sometimes+sure\%22&view=theater}{ibid 61}) 
\z
\z

\ea \label{ex:04-145}
\gll Min fromme Knep, kan du mig kiende? --- O, kysz mig i min bagendel\\
 my pious trick can you me recognize {} oh kiss me in my backside\\
\glt `My good Knep, do you recognize me? --- Oh, kiss my arse.'\\ % PE: changed "ass" to "arse" (see above).
% PE: You don't talk to a trick; so surely (I thought) this is a cognate of English "knave". However, a large old Danish-English dictionary says that no it isn't: https://archive.org/details/adanishenglishd00reppgoog/page/n199/mode/2up?view=theater&q=Rnt%5E . OJ has chosen to capitalize "Knep"; could it be a character's nickname?
%% SG: It's the name of a character
\hfill(\href{https://archive.org/details/hieronymusjuste00rancgoog/page/322/mode/2up?view=theater&q=\%22min+fromme+knep%22}{Ranch, \textit{Niding} 322})
\z
\is{incredulity expressions|)}

\AddSubSection{Ironic \textit{much}} \label{sec:ironic-much}

A frequent ironical way of expressing a negative is by placing a word like \textit{much} in the beginning of a sentence \refp{ex:04-146}.

\ea \label{ex:04-146}
\ea
Much I care \phantom{x} (`I don't care (much)')\hfill(\href{https://archive.org/details/treasureisl00stev/page/18/mode/2up?q=\%22much+I+care\%22&view=theater}{Stevenson, \textit{Treasure} 27})
\ex
much he cares\hfill(\href{https://archive.org/details/ourmutualfriendc0000char/page/490/mode/2up?q=\%22much+he+cares\%22&view=theater}{Dickens, \textit{Friend} 659}) % OJ doesn't provide the example, merely the page number.
\ex
Much he cares\hfill(\href{https://archive.org/details/wifeofsirisaacha00well/page/122/mode/2up?q=\%22he+cares\%22&view=theater}{Wells, \textit{Wife} 122}) % OJ doesn't provide the example, merely the page number.
\ex
``Mr. Copperfield was teaching me---''\\(``Much he knew of it himself!'') said Miss Betsy in a parenthesis.\\\hfill(\href{https://archive.org/details/personalhistory05dickgoog/page/n11/mode/2up?q=\%22was+teaching+me\%22&view=theater}{Dickens, \textit{David} 8}) % I kid you not; this is what appears in the book.
\ex
you yawned---much my company is to you\hfill(\href{https://archive.org/details/wessextalesstran00hardrich/page/170/mode/2up?q=\%22you+yawned\%22&view=theater}{Hardy, \textit{Wessex} 224})
\ex
Much good that would have done\hfill(\href{https://archive.org/details/strifedraminthre00galsuoft/page/250/mode/2up?view=theater&q=\%22much+good%22}{Galsworthy, \textit{Strife} 96})
\ex
Much good your pity will do it [England]\hfill(\href{https://archive.org/details/john00shawbullsotherisrich/page/122/mode/2up?q=\%22Much+good+your+pity+will+do+it\%22&view=theater}{Shaw, \textit{Island} 114})
\ex
much good you are to wait up\hfill(\href{https://archive.org/details/forpur00shawthreeplaysrich/page/4/mode/2up?q=\%22much+good+you\%22&view=theater}{Shaw, \textit{Disciple} 1})
\ex
Much you can do to stop 'em, old fellow\hfill(\href{https://archive.org/details/in.ernet.dli.2015.53170/page/n49/mode/2up?q=\%22can+do+to+stop\%22&view=theater}{Hope, \textit{Rupert} 37})
\ex
A lot I should have cared whose fault it was\hfill(\href{https://archive.org/details/dli.ministry.15783/page/257/mode/2up?q=\%22lot+I+should+have\%22&view=theater}{Kipling, \textit{Jungle} 230})
\ex
Plucky lot she cared for idols when I kissed 'er where she stud!\\\hfill(\href{https://archive.org/details/in.ernet.dli.2015.48402/page/n71/mode/2up?q=\%22plucky+lot\%22&view=theater}{Kipling, \textit{Ballads} 58}) % OJ writes "her" but the original has "'er"
\ex
His brogue! A fat lot you know about brogues!\hfill(\href{https://archive.org/details/john00shawbullsotherisrich/page/14/mode/2up?q=\%22His+brogue%21+A+fat+lot+you+know+about+brogues\%22&view=theater}{Shaw, \textit{Island} 14})
\ex
Livingstone tossed her head, ``Fine he knows the heart of a lass {\dots}.''\\\hfill(\href{https://archive.org/details/queensquairorsi00hewlgoog/page/116/mode/2up?q=\%22tossed+her+head\%22&view=theater}{Hewlett, \textit{Quair} 117}) % OJ writes "She", but the original has "Livingstone".
\z
\z

Similarly in Danish, for instance \refp{ex:04-159}.

\ea \label{ex:04-159}
\ea
\gll {}[han] trak spottende paa skuldren og sagde: Naa, det skal vel \textit{stort} hjælpe!\\
 [he] pulled mockingly on shoulder.\DEF{} and said well that shall probably greatly help\\
\glt `[he] shrugged mockingly and said: Well, that's \textit{really} going to help a lot!'
\hfill(\href{https://archive.org/details/mitlivoglevnedso00fibi/page/236/mode/2up?q=spottende&view=theater}{Fibiger, \textit{Liv} 236}) % This does not start with "han" ('he')

\ex
\gll Det skulde \textit{stort} hjælpe, om jeg {\dots}\\
 it would greatly help if I\\
\glt `It would \textit{really} help a lot if I {\dots}'
\hfill(\href{https://archive.org/details/fruingertilstra00ibsegoog/page/n104/mode/2up?q=\%22stort+hj%C3%A6lpe\%22&view=theater}{Ibsen, \textit{Inger} 98}) % "skulle" corrected to what the book says: "skulde". Will the English translation be mistaken?
%Brett: No, it's good.
%% SG: The spelling 'skulde' is now obsolete, but I think it makes sense to cite it as it appears in the original publication

\ex
\gll Det skulde hjælpe \textit{fedt}\\
 it would help fat\\
\glt `A fat lot of good that would do me'
\hfill(\href{https://books.google.co.jp/books?id=7X0_AAAAIAAJ&newbks=1&newbks_redir=0&printsec=frontcover&dq=Niels+M%C3%B8ller.+Koglerier&hl=ja&redir_esc=y#v=onepage&q=\%22skulde%20hj%C3%A6lpe%20fedt\%22&f=false}{N. Møller, \textit{Koglerier} 235}) % "skulle" corrected to what the book says: "skulde". Will the English translation be mistaken?
%Brett: No, it's good.
%PE: I changed "fat load of good" (neither in my idiolect nor known to COCA) to "fat lot of good" (in my idiolect and with twenty tokens in COCA).
%% SG: "fedt" here doesn't literally mean 'fat'. I'd simply translate it "a lot" \\\ PE: Ah, but "a fat lot of good/use/etc" also doesn't have the normal meaning of "fat" (it merely intensifies "lot").

\ex
\gll men \textit{ligemeget} hjalp det\\
 but {as much} helped it\\
\glt `but it didn't help at all'
\hfill(\href{https://books.google.co.jp/books?id=0iDWtPUTi7kC&newbks=1&newbks_redir=0&printsec=frontcover&hl=en&redir_esc=y#v=onepage&q=\%22ligemeget%20hjalp\%22&f=false}{Matthiesen, \textit{Stjærner} 30}) % OJ writes "Stjerner" but the cited book says "Stjærner".
%% SG: the spelling "Stjærner" was already old-fashioned in Jespersen's day -- regarding the translation, I note that you have opted for a less direct one here. More literally it means something like "But it helped just the same" (i.e. not at all)
\z
\z

There is a curious use of \textit{fejl} as a negative, only with \textit{bryde sig om} \refp{ex:04-163}.

\ea \label{ex:04-163}
\gll Du bryder dig jo feil om eiermanden\\
 you care you indeed wrongly about owner.\DEF{}\\
\glt `You really don't care about the owner'
\hfill(\href{https://books.google.co.jp/books?id=Hm4AAAAAcAAJ&pg=PA142&lpg=PA142&dq=\%22Du+bryder+dig+jo+feil+om+eiermanden\%22&source=bl&ots=OGEV9PHatE&sig=ACfU3U2KQkFsep5hS2e4GQEaQuD8F6EYOw&hl=en&sa=X&ved=2ahUKEwjv9NSwu8-EAxV4hlYBHadyCAQQ6AF6BAgIEAM#v=onepage&q=\%22Du%20bryder%20dig%20jo%20feil%20om%20eiermanden\%22&f=false}{Paludan-Müller, \textit{Adam} 1.142})
\z

Among ironical expressions must also be mentioned English \textit{love} (`nothing'). This, I take it, originated in the phrase \textit{to marry for love, not for money}, whence the common antithesis \textit{for love or money}. Then it was used extensively in the world of games, where it is now the usual word in counting the score, in tennis, for instance, \textit{love fifteen}, meaning that one party has nothing to the other's 15, in football \textit{winning by two goals to love}, etc. In this sense the English word has become international in the terminology of some games.
\is{irony|)}

\AddSubSection{\textit{The devil}} \label{sec:the-devil}\is{devil@`devil', expressions meaning|(}
\is{grammaticalization|(}

\textit{The devil} (also without the article) is frequently used as an indirect negative; cf. from other languages \citet[\href{https://archive.org/details/bub_gb_NQwJAAAAQAAJ/page/n23/mode/2up?q=\%22+teufel+in+der+bette\%22&view=theater}{23f}]{grimm1856uber}. In English we have \textit{the devil} joined either to a verb, or to a substantive: \textit{the devil a word} (`not a word'); \textit{the devil a bit} (`nothing'). There is a well-known little verse \refp{ex:04-164} (sometimes quoted with \textit{a saint} instead of \textit{a monk}).

\ea \label{ex:04-164}
When the devil was ill, the devil a monk would be; \\
When the devil got well, \textit{the devil a monk} was he.
\z

\bigskip

The following may serve as an illustration of the natural way in which \textit{the devil} has come to play this part of a disguised negative \refp{ex:04-165}.

\ea \label{ex:04-165}
``Lady Rosamund is going to take a sketch of the luncheon-party.''\\``Let her take a sketch of the devil!'' said this very angry and inconsiderate papa.\hfill(\href{https://www.gutenberg.org/files/16217/16217-h/16217-h.htm}{Black, \textit{Fortunatus} 184}) % "luncheon-party" hyphenated
\z

Examples of \textit{devil}, etc. with a verb: \refp{ex:04-166}.

\ea \label{ex:04-166}
\ea
the devil she won't  \phantom{x} (`she will')\hfill(\href{https://archive.org/details/bim_eighteenth-century_the-history-of-tom-jones_fielding-henry_1768_4/page/174/mode/2up?q=\%22devil+%C5%BFhe\%22&view=theater}{Fielding, \textit{Tom} 4.174})
\ex
``Captain Absolute and Ensign Beverley are one and the same person.''\\``The devil they are''\hfill(\href{https://archive.org/details/playsofsheridanc00sheruoft/page/6/mode/2up?q=\%22captain+absolute+and\%22&view=theater}{Sheridan, \textit{Rivals} 1.1}) % The Sheridan examples cite an omnibus edition that I can't find. I've therefore specified the play, act, and scene.
\ex
``she's in the room now.'' --- ``The devil she is''\hfill(\href{https://archive.org/details/playsofsheridanc00sheruoft/page/140/mode/2up?q=\%22in+the+room+now\%22&view=theater}{Sheridan, \textit{School} 4.3})
\ex
``{[}You have not heard a word{]} of his being dangerously wounded?''\\``The devil he is!''\hfill(\href{https://archive.org/details/playsofsheridanc00sheruoft/page/150/mode/2up?q=\%22the+devil\%22&view=theater}{ibid 5.2}) % OJ merely provides the page number; I (PE) have added the example.
\ex
``I was up at that place at Richmond yesterday.''\\``The devil you were!''\hfill(\href{https://archive.org/details/dukeschildrennov00troluoft/page/178/mode/2up?q=\%22at+that+place+at+richmond\%22&view=theater}{Trollope, \textit{Children} 2.52}) % Restoring "up"
\ex
``I am going back.'' --- ``The devil you are''\hfill(\href{https://archive.org/details/anoldmanslove01trolgoog/page/n112/mode/2up?q=\%22devil+you+are\%22&view=theater}{Trollope, \textit{Love} 204})
\ex
``I can't give you the money,'' I went on.\\``The devil you can't!'' \phantom{x} (`you can')\hfill(\href{https://archive.org/details/manofmark00hope/page/182/mode/2up?q=\%22give+you+the+money\%22&view=theater}{Hope, \textit{Man} 102}) % Reinserted "I went on".
\z
\z

Examples of \textit{devil} + substantive (in Scotch also with pronouns): \refp{ex:04-173}.

\ea \label{ex:04-173}
\ea
My parents are al dead, and the diuel a peny they haue left me, but a bare pention\hfill(\href{https://babel.hathitrust.org/cgi/pt?id=uiuo.ark:/13960/t1pg73v3n&seq=132&q1=My+parents+are+al+dead}{Marlowe, \textit{Faustus} 766})
\ex
The diu'll a Puritane that hee is\hfill(\href{https://internetshakespeare.uvic.ca/doc/TN_F1/scene/2.3/index.html#tln-835}{Shakespeare, \textit{Tw} 2.3.159})
\ex
I have been out this whole afternoon and the devil a bird have I seen\\\hfill(\href{https://quod.lib.umich.edu/cgi/t/text/pageviewer-idx?cc=evans;c=evans;idno=n18038.0001.001;node=N18038.0001.001:10;seq=155;page=root;view=text}{Fielding, \textit{Joseph} 4.290}) % Restored "I have been out this whole afternoon": not strictly necessary; but without it, this seemed rather strange to me (PE)
\ex
But now-a-days the devil a thing of their own manufacture's about them, except their faces.\hfill(\href{https://books.google.co.jp/books?id=MCM2BEhQzZcC&pg=PA196&lpg=PA196&dq=\%22now-a-days+the+devil+a+thing\%22&source=bl&ots=hplEdUF9Gh&sig=ACfU3U0xbFjQH7Znke0VGqO2G9JEoQvlAA&hl=en&sa=X&ved=2ahUKEwiCjZ2syJGGAxWkjK8BHT_zAQ8Q6AF6BAggEAM#v=onepage&q=\%22now-a-days%20the%20devil%20a%20thing\%22&f=false}{Goldsmith, \textit{Good-natur'd}}) % Restored "manufacture's": OJ's dots made this hard for me to understand
\ex
``has nothing been heard---about me?''\\``Devil a bit''\hfill(\href{https://archive.org/details/lifeadventuresofdickrich/page/88/mode/2up?q=\%22has+nothing+been+heard\%22&view=theater}{Dickens, \textit{Nicholas} 76}) % "---about me" reintroduced
\ex
``If she did not tell you {\dots}''\\``Tell me? Devil a bit of it''\hfill(\href{https://archive.org/details/majorvigoureux00quil/page/210/mode/2up?q=\%22Devil+a+bit\%22&view=theater}{Quiller-Couch, \textit{Major} 210})
\ex 
it [the law-suit]'s been four times in afore the fifteen, and de'il ony thing the wisest o' them could make o't\hfill(\href{https://archive.org/details/antiquary20unkngoog/page/n36/mode/2up?q=\%22four+times%22}{Scott, \textit{Antiquary} 1.21}) % Not "deil" but "de'il"
\ex
the de'il a drap punch ye'se get here the day\hfill(\href{https://archive.org/details/antiquary20unkngoog/page/n46/mode/2up?q=\%22drap+punch\%22&view=theater}{ibid 1.30}) % Not "deil" but "de'il"
\ex 
the de'il ane wad hae stirr'd\hfill(\href{https://archive.org/details/antiquary20unkngoog/page/n48/mode/2up?q=\%22de%27il+ane\%22&view=theater}{ibid 1.31}) % OJ has "stirred" but WS has "stirr'd".
\ex
de'il ony o' them daur hurt a hair o' auld Edie's head\hfill(\href{https://archive.org/details/antiquary00scot_0/page/200/mode/2up?q=\%22daur+hurt\%22&view=theater}{ibid 1.341})
\z
\z

The following quotations exemplify more unusual employments (Irish?) of \textit{devil} as a negative: \refp{ex:04-183}.

\ea \label{ex:04-183}
\ea
Devil the other idea there is in your head this minute.\\(`there is no other idea') \hfill(\href{https://archive.org/details/advofdrwhitty00birmiala/page/n17/mode/2up?q=\%22Devil+the+other+idea+there\%22&view=theater}{Birmingham, \textit{Whitty} 6})
\ex
and devil the word I'll speak to Mr. Eccles on your behalf\hfill(\href{https://archive.org/details/advofdrwhitty00birmiala/page/34/mode/2up?q=\%22devil+the+word+I%27ll+speak%22}{ibid 34})
\ex
They're good anchors {\dots} Devil the better you'd see\hfill(\href{https://archive.org/details/advofdrwhitty00birmiala/page/184/mode/2up?q=%22They%27re+good+anchors%2C%22&view=theater}{ibid 185}) % Reinserted "said Michael Geraghty"
\z
\z

In Scotch there is an idiomatic use of \textit{deil} (or \textit{fient}) \textit{hae't} (`have it') in the sense of a negative: \refp{ex:04-186}.

\ea \label{ex:04-186}
\ea
For thae frank, rantin, ramblin billies, Fient haet o' them's \ob`not one of them is'\cb ~ill-hearted fellows\hfill(\href{https://www.scottishpoetrylibrary.org.uk/poem/the-twa-dogs/}{Burns, \textit{Dogs}})
\ex
Tho' deil-haet ails them \phantom{x} (`nothing') \hfill(\href{https://www.scottishpoetrylibrary.org.uk/poem/the-twa-dogs/}{ibid}) % PE added parentheses around 'nothing'
\ex
``And what do you expect now, {\dots} ?''\\``Deil haet do I expect''\hfill(\href{https://archive.org/details/cewaverleynovels03scotuoft/page/406/mode/2up?view=theater&q=\%22what+do+you+expect+now%22}{Scott, \textit{Antiquary} 2.348}) % Restoring "And"
\z
\z

This leads to a curious use of \textit{hae't} (`a bit, anything'): \textit{She has-na a haed left}; see \href{https://archive.org/details/newenglishdict05murrmiss/page/n137/mode/2up?view=theater}{\textit{NED}, \textit{Hate} $sb.^{2}$}.% ??? OJ has "sb 2", suggesting that it's the second meaning of "hate" as a single "substantive". But actually this refers to the second lexeme of "hate" as a substantive; NED makes the "2" a superscript. The problem now is that the reader might now look for our (of course nonexistent) second footnote.....
%Brett: I don't see a way around this.
% PE: We could simply cut the "2" (however formatted).

Instead of the word \textit{devil}, (\textit{the}) \textit{deuce} is very often used in the same way; the word probably is identical with \textit{deuce} from French \textit{deux}, Old French \textit{deus}, to indicate the lowest, and therefore most unlucky, throw at dice, but is now felt as a milder synonym of \textit{devil}.

Examples with the verb negatived: \refp{ex:04-189}.

\ea \label{ex:04-189}
\ea
``I heard what you said about Airey Newton.'' {\dots}\\``The deuce you did!''\hfill(\href{https://archive.org/details/intrusionspeggy02hopegoog/page/n148/mode/2up?q=\%22I+heard+what+you+said\%22&view=theater}{Hope, \textit{Intrusions}}) % OJ attributes this to "Houseman J." But it doesn't appear in Laurence Housman's King John of Jingalo. (Even "deuce" and "devil" don't appear.) It does appear, with a short intervening paragraph (thus the added dots), in this novel by an OJ favorite, Anthony Hope.
\ex
Deuce he has\hfill(\href{https://archive.org/details/ordealofrichardf0000geor_q5z1/page/376/mode/2up?q=\%22Deuce+he+has\%22&view=theater}{Meredith, \textit{Ordeal} 287})
\ex
``He lies in his room upstairs {\dots}.''\\``The deuce he does!''\hfill(\href{https://archive.org/details/prisonerzendabe01hopegoog/page/n180/mode/2up?q=\%22he+lies+in+his+room\%22&view=theater}{Hope, \textit{Zenda} 174}) % Punctuation altered to better accord with the original.
\z
\z

Examples with a substantive (or pronoun) negatived: \refp{ex:04-192}.\largerpage

\ea \label{ex:04-192}
\ea
Mr. Secretary and I {\dots} thought to have been very wise; but the deuce a bit, the company stayed\hfill(\href{https://archive.org/details/journaltostellae00swifuoft/page/130/mode/2up?q=\%22deuce+a+bit\%22&view=theater}{J. Swift, \textit{Journal} 130}) % This, unlike OJ's version, reflects what Swift writes.
\ex
the deuce of any other rule have I to govern myself by\\\hfill(\href{https://www.gutenberg.org/files/39270/39270-h/39270-h.htm#bookIV_chapX}{Sterne, \textit{Tristram} 4.10}) % Renumbered for volume and chapter
\ex
she did beguile me of my tears, but the deuce a one did she shed\\\hfill(\href{https://archive.org/details/liberamoris00hazlgoog/page/n114/mode/2up?q=beguile&view=theater}{Hazlitt, \textit{Liber} 38})
\ex
The deuce a bit more is there of it.\hfill(\href{https://archive.org/details/liberamoris00hazlgoog/page/n118/mode/2up?q=\%22deuce+a+bit+more\%22&view=theater}{ibid 40})
\ex
``Sit down, my good people, sit down!'' But the deuce a bit would they sit down.\hfill(\href{https://archive.org/details/returnofthenativ00harduoft/page/160/mode/2up?q=\%22my+good+people\%22&view=theater}{Hardy, \textit{Return} 209}) % Additional "sit down!" added, following the book.
\ex
if poor Harry should find me out, deuce a bit more home for me\\\hfill(\href{https://archive.org/details/evanharringtonno00mererich/page/424/mode/2up?q=\%22poor+Harry+should\%22&view=theater}{Meredith, \textit{Harrington} 468}) % OJ merely gives the page number, not the example. In this book, Meredith uses "deuce" a lot, but only once with this sense.
\ex
Jeuce a word I ever heard of it!\hfill(\href{https://archive.org/details/johnbullsotheris0000shaw_o3k8/page/52/mode/2up?q=jeuce}{Shaw, \textit{Island} 38}) % Restored the closing exclamation point
\ex
if you stay here, the deuce a man [`nobody'] in all Ruritania will doubt of it\hfill(\href{https://archive.org/details/prisonerzendabe01hopegoog/page/n42/mode/2up?q=\%22deuce+a+man\%22&view=theater}{Hope, \textit{Zenda} 37}) % Restored "in all Ruritania"
\z
\z

Occasionally other words may be used as substitutes for \textit{the devil} with negative purport \refp{ex:04-200}.

\ea \label{ex:04-200}
\ea
``You may give him up, mother. He'll not come here.''\\ ``Death give him up. He \textit{will} come here.''\hfill(\href{https://archive.org/details/dombeyson00dick_0/page/696/mode/2up?q=\%22You+may+give+him+up%2C+mother\%22&view=theater}{Dickens, \textit{Dombey} 447})
\ex
``But we're not mixed up in the party fight.''\\``The hell you're not!''\hfill(\href{https://books.google.com/books?id=8UbL0Ikt-HcC&pg=PA238&lpg=PA238&dq=\%22we%27re+not+mixed+up+in+the+party+fight%22}{Page, \textit{Southerner} 238}) % OJ attributes this to "Worth S." (explained in MEG vol 7 as Nicholas Worth, The Southerner) but I (PE) find no reason to think that it's by a Nicholas Worth, or even that Page used this as a pseudonym
\ex
but ne'er-be-licket could they find that was to their purpose\\\hfill(\href{https://archive.org/details/antiquary20unkngoog/page/n212/mode/2up?q=\%22could+they+find+that\%22&view=theater}{Scott, \textit{Antiquary} 1.45})
\z
\z

In Irish \textit{sorrow} (pronounced ``sorra", [sɔrə]) is used as a synonym of \textit{the devil} (cf. \cite[\href{https://archive.org/details/englishaswespeak00joycuoft/page/69/mode/2up?view=theater}{70}]{joyce1910english}), also as a negative, cf. the following quotations \refp{ex:04-203}.

\ea \label{ex:04-203}
\ea
when he had to cross the mountains on an empty stomach to say Mass, and sorra a bite of bread or sip of water to stay his stomach\\\hfill(\href{https://archive.org/details/fatheranthonyar00buchgoog/page/n122/mode/2up?q=sorra&view=theater}{Buchanan, \textit{Anthony} 110}) % OJ's "ship" corrected to "sip"
\ex
Anthony was all for books and book-learning; and sorra a colleen ever troubled the heart of him\hfill(\href{https://archive.org/details/fatheranthonyar00buchgoog/page/n122/mode/2up?q=sorra&view=theater}{ibid 111})
\ex
``Is there any more news of Master Michael?''\\``Sorra news, except that he's lying in the gaol''\hfill(\href{https://archive.org/details/fatheranthonyar00buchgoog/page/n128/mode/2up?q=sorra&view=theater}{ibid 114}) % Restoring "of Master Michael"
\ex
``Do you think the intention was to hit the car?''\\``Sorra doubt''\hfill(\href{https://archive.org/details/fatheranthonyar00buchgoog/page/n172/mode/2up?q=\%22intention+was+to+hit\%22&view=theater}{ibid 163})
\ex
Did one of them think that Rory himself would be on the car? Sorra one\hfill(\href{https://archive.org/details/fatheranthonyar00buchgoog/page/n182/mode/2up?q=sorra&view=theater}{ibid 172}) % Restored "that Rory himself would be on [sic] the car"
\ex
Sorra the man in the town we'd rather be listening to than yourself\\\hfill(\href{https://archive.org/details/advofdrwhitty00birmiala/page/308/mode/2up?q=\%22sorra+the+man+in+the+town\%22&view=theater}{Birmingham, \textit{Whitty} 308})
\ex
{[}Irish lady:{]} Master Sam tells me sorra a sowl goes nigh ut\\\hfill(\href{https://archive.org/details/astonishinghisto0000quil_t3y6/page/180/mode/2up?q=\%22sorra+a+sowl\%22&view=theater}{Quiller-Couch, \textit{Troy} 181}) % "Master" restored to the front
\ex
He gets rid of one wife and saddles himself with another---sorrow a bit will he stop at home for either of them!\hfill(\href{https://archive.org/details/historydavidgri02wardgoog/page/254/mode/2up?q=\%22gets+rid+of+one+wife\%22&view=theater}{Ward, \textit{David} 2.113}) % Restoring "!"
\ex
But sorrow a bit o' pity will you get out o' me, my boy---sorrow a bit\\\hfill(\href{https://archive.org/details/historyofdavidgri03ward/page/30/mode/2up?q=\%22But+sorrow+a+bit+o%27+pity+will+you+get+out+o%27+me%2C+my+boy\%22&view=theater}{ibid 3.30})
\z
\z

\il{Danish!fanden@\textit{fanden}|(}The corresponding use of Danish \textit{fanden} is extremely frequent in Holberg and later, see e.g. \refp{ex:04-212}. Similarly with the synonym \textit{djævelen} (`the devil') \refp{ex:04-219}. This is not usual nowadays.

\ea \label{ex:04-212}
\ea
\gll jeg vil bevise af den sunde logica, at I er en tyr. --- I skal bevise fanden\\
 I will prove by the sound logic that you are a bull {} you shall prove devil\\
\glt ``I will prove with sound logic that you are a bull.'' --- ``You shall prove nothing of the sort''
\hfill(\href{http://holbergsskrifter.dk/holberg-public/view?docId=skuespill%2FErasmus%2FErasmus.page&brand=&chunk.id=act4sc2&toc.id=act4&toc.depth=1}{Holberg, \textit{Erasmus} 4.2})

\ex\il{Danish!ikke@\textit{ikke}}
\gll Havde jeg ikke været en politicus, saa havde jeg skiøttet fanden derom\\
 had I not been a politician then had I cared devil thereof\\
\glt `If I hadn't been a politician, I would have cared nothing about it'\\
\hfill(\href{http://holbergsskrifter.dk/holberg-public/view?docId=skuespill%2FUlysses%2FUlysses.page&brand=&chunk.id=act2sc7&toc.id=act2&toc.depth=1}{Holberg, \textit{Ulysses} 2.7}) % PE: I removed "then" from in front of "I would have"
%% SG: I'd write "If I hadn't been..." in the English translation. The V1 conditional order definitely sounds more marked in English than in Danish

\ex\il{Danish!ikke@\textit{ikke}}
\gll ``Kan vi ikke sejle fra ham {{\dots}?'' ---} ``Fanden kan vi,'' svarte han\\
 can we not sail from him {} devil can we answered he\\
\glt `Can we not sail away from him?' --- `Not a chance, he answered'\\
\hfill(\href{https://books.google.com/books?id=8aeyxdW_f_oC&pg=PA120&lpg=PA120&dq=blicher+%22Fanden+kan+vi\%22&source=bl&ots=oMK334LyIt&sig=ACfU3U1mh55vPZ6ivVj_YBWpBx2dJEk1jQ&hl=en&sa=X&ved=2ahUKEwietaSn68-EAxU2ZvUHHUJ0A04Q6AF6BAgIEAM#v=onepage&q=blicher%20%22Fanden%20kan%20vi\%22&f=false}{Blicher, \textit{Dagbog} 1.43})

\ex
\gll Jeg vidste fanden hvad det var\\
 I knew devil what it was\\
\glt `I had no idea what it was'
\hfill(\href{https://tekster.kb.dk/text/adl-texts-andersen04val-root#idm140137151050288}{Andersen, \textit{O. T.} 1.67})

\ex
\gll Jeg bryder fanden mig om eiermanden\\
 I care devil about the owner\\
\glt `I don't give a damn about the owner'
\hfill(\href{https://books.google.co.jp/books?id=Hm4AAAAAcAAJ&pg=PA142&lpg=PA142&dq=\%22Du+bryder+dig+jo+feil+om+eiermanden\%22&source=bl&ots=OGEV9PHatE&sig=ACfU3U2KQkFsep5hS2e4GQEaQuD8F6EYOw&hl=en&sa=X&ved=2ahUKEwjv9NSwu8-EAxV4hlYBHadyCAQQ6AF6BAgIEAM#v=onepage&q=\%22Jeg%20bryder%20fanden\%22&f=false}{Paludan-Müller, \textit{Adam} 1.140}) 

\ex
\gll De er virkelig født kommentator! --- Jeg er fanden, er jeg\\
 you are truly born commentator {} I am devil am I\\
\glt `You really are a born commentator! --- The devil I am'\\
\hfill(\href{https://tekster.kb.dk/text/adl-texts-drachmann14val-root#s144}{Drachmann, \textit{Forskrevet} 1.195})

\ex 
\gll han {\dots} brydde sig fanden om sang og solskin\\
 he {} \ cared the devil about song and sunshine\\
\glt `he didn't give a damn about singing or sunshine'
\hfill(\href{https://www.nb.no/items/URN:NBN:no-nb_digibok_2008072810004?searchText=%22brydde%20sig%20fanden\%22&page=81}{Bjørnson, \textit{Guds} 71}) % Dots for an omission
\z
\z

%% SG: Note that most of these examples have capitalization of all nouns in the original, removed by OJ

\ea \label{ex:04-219}
\gll ``Jeg siger, I er en hane, og skal bevise {det {\dots}''} --- ``I skal bevise divelen''\\ % removing "at" (not in original)
 I say you are a rooster and shall prove it {} you shall prove devil.\DEF{}\\
\glt `I say you're a rooster, and I'll prove it {\dots}' --- `You'll prove no such thing'
\\\hfill(\href{http://holbergsskrifter.dk/holberg-public/view?docId=skuespill%2FErasmus%2FErasmus.page&brand=&chunk.id=act4sc2&toc.id=act4&toc.depth=1}{Holberg, \textit{Erasmus} 4.2}) % This is quite unlike what we read in http://holbergsskrifter.dk/holberg-public/view?docId=skuespill%2FErasmus%2FErasmus.page&brand=&chunk.id=act4sc2&toc.id=act4&toc.depth=1 , which instead says "Jeg siger I er en Hane, og skal bevise det saa klart, som 2 og 3 er 5" (which Google Translate renders as "I say you are a rooster, and must prove it as clearly as 2 and 3 are 5." A different text in a different edition of this play? Need to check.
%Brett: If you mean "as clearly as 2 and 3 are 5" I think it's just elided. If you mean "for all I care" the fault is in the translation. I've removed that. % Peter: OK
% PE: I've changed "You shall" to "You'll"
%% SG: Yes, this is just an elision
\z

\textit{Fanden} often stands for `not I': \refp{ex:04-220}. \textit{Fanden} (\textit{Satan})\textit{ heller} is also used in a negative sense (`I would rather have the devil'), thus \refp{ex:04-blicher-h}. % According to U Toronto's K-Lee, ["Blicher 3. 547"] is Samlede noveller og Skizzer. Page 547 of this can be viewed at https://books.google.co.jp/books?redir_esc=y&hl=ja&id=Y3wOAAAAYAAJ . It doesn't obviously contain the string, but it's in Fraktur and thus hard to read. Its running head says "Høstferierne". This can be read far more easily at https://tekster.kb.dk/text/adl-texts-blich05-shoot-workid66367 -- which also doesn't obviously show the string.
%Brett: p. 537 has it: "The devil seize me" https://books.google.co.jp/books?id=Y3wOAAAAYAAJ&hl=ja&pg=PA537#v=onepage&q=fanden&f=false
% PE: Thanks!
%% SG: "Fanden hakke mig!" on p. 537 literally means "(May) the devil chop me up". 
% PE: OJ writes "Goldschmidt Kol. 92".  Goldschmidt says it at https://archive.org/details/mgoldschmidtspo00unkngoog/page/n364/mode/2up?q=\%22fanden+heller\%22&view=theater but I can't see how the title of this work ("Ekko'et", here in a volume of the collected works) could be abbreviated to "Kol"
%% SG: "Kol." as an abbreviation might be "Kolonne"/'column', so OJ may have referred to an edition with column numbers. Other than that I have no idea what it might mean.
% PE: We're DELETING this (for the meantime).
\ea \label{ex:04-220}
\ea
\gll Gid nu fanden staae her {længer}, vi maae ogsaa have noget af byttet\\
 {I hope} now devil stand here longer we must also have some of loot.\DEF{}\\
\glt `Damned if I'll stand here any longer. We've also got to get a share of the loot.'
\hfill(\href{http://holbergsskrifter.dk/holberg-public/view?docId=skuespill%2FUlysses%2FUlysses.page&chunk.id=act3sc7&show.second=&toc.depth=1&toc.id=act3}{Holberg, \textit{Ulysses} 3.7}) % Act and scene numbers added. OJ has "maa" but the website has "maae"

\ex
\gll Fanden forstaa sig paa kvindfolk!\\
 devil understand himself on women\\
\glt `Not even the devil understands women!'
\hfill(\href{https://archive.org/details/samledepoetiske04dracgoog/page/206/mode/2up?q=\%22fanden+forstaa+sig+paa+kvindfolk\%22&view=theater}{Drachmann, \textit{Kitzwalde} 85})

\ex
\gll Fanden véd, om det holder.\\
 devil knows if it holds\\
\glt `Not even the devil knows if it will hold up.'
\\\hfill(\href{https://www.gutenberg.org/cache/epub/10829/pg10829-images.html}{H. Bang, \textit{Ludvigsbakke} 38})

\ex
\gll Satan forstaa sig paa havet.\\
 Satan understand himself on {the sea}\\
\glt `Not even Satan understands the sea.'
\hfill(\href{https://www.kb.dk/e-mat/dod/130004856751-bw.pdf}{Nexø, \textit{Pelle} 2.129})
\z
\z

\ea \label{ex:04-blicher-h}
\gll Fanden hakke mig! \\
 devil chop me \\
\glt `May the devil chop me up!'
\hfill(\href{https://books.google.co.jp/books?id=Y3wOAAAAYAAJ&hl=ja&pg=PA537#v=onepage&q=fanden&f=false}{Blicher, \textit{Høstferierne} 3.537})
\z

Sometimes \textit{fanden} is used simply to intensify an expressed negative \refp{ex:04-224}.

\ea \label{ex:04-224}
\ea
\gll ``Gaae du til {fanden {\dots}!''} Den anden Gik \textit{fanden} \textit{ei} til fanden\\
 go you to devil the other went devil not to devil\\
\glt ``Go to hell!'' The other didn't \textit{bloody well} go to hell'
\\\hfill(\href{https://books.google.co.jp/books?newbks=1&newbks_redir=0&hl=ja&id=9_-Z9j0M3z8C&dq=\%22ki%C3%B8rte+pokker%22+wessel&q=Gaae+du+til+fanden#v=snippet&q=\%22Gaae%20du%20til%20fanden\%22&f=false}{Wessel, \textit{Polser} 204}) % Marking an omission of four words
\ex\il{Danish!ikke@\textit{ikke}}
\gll og så véd jeg \textit{fanden} \textit{ikke}, hvordan det gik til\\
 and so know I devil not how it went to\\
\glt `and so I don't \textit{bloody well} know how it happened'
\\\hfill(\href{https://tekster.kb.dk/text/adl-texts-juelh01val-root#s186}{Juel-Hansen, \textit{Historie} 186}) % EJH writes "så", not "saa".
\z
\z\il{Danish!fanden@\textit{fanden}|)}

\label{para:negativesbefore}Two modern German examples of \textit{den Teufel} (`nicht') may suffice \refp{ex:04-226}. For older examples, see \citet{grimm1856uber}, quoted above (Section \ref{sec:the-devil}).

\ea \label{ex:04-226}
\ea
\gll Die fremden Weiber gingen mich den Teufel was an\\
 the foreign women go-\textsc{pst} me the devil something on\\
\glt `Those strange women didn't mean a damn thing to me'
\\\hfill(\href{https://archive.org/details/moriturifritzch00sudegoog/page/n109/mode/2up?q=\%22+Sie+ftemben+%C3%A4Beibet+gingen+micg+ba+ben+Xeufel+toas+an.\%22&view=theater}{Sudermann, \textit{Fritzchen}})

\ex
Im Theaterstück sagt ein Mann zu seiner stets keifenden, zanksüchtigen Frau: ``Ich weiss {ja doch}, dass ich einen sanften Engel zur Frau habe''---worauf sie mit ``artigem'' Widerspruch schreit: ``Den Teufel hast du'', wobei sie zunächst nur an Widerspruch denkt, als ob sie sagen wollte ``nein, gar nichts hast du''\\
`In the play, a man says to his constantly nagging, quarrelsome wife: ``I know indeed that I have a gentle angel as a wife''---to which she shouts in ``polite'' contradiction: ``The devil you have,'' where her first thought is simply of contradicting, as if she wanted to say ``no, you have nothing at all."'
\hfill\citep[\href{https://archive.org/details/psychologische-studien-zur-sprachgeschic/page/172/mode/2up\&view=theater}{172}]{bruchmann1888psychologische} % "artigem" in quotation marks. Oddly, changing "co.jp" in the link to ".com" brings a blank page.
\z\z

\is{pox@`pox', use of words meaning}
As \textit{pox} (originally the name of a disease) was popularly used as a kind of substitute for \textit{the devil} in imprecations, it can also be used in indirect negation, as in \refp{ex:04-228}.

\ea \label{ex:04-228}
The Dean friendly! the Dean be poxed \phantom{x} (`he is not')\hfill(\href{https://archive.org/details/journaltostellae00swifuoft/page/22/mode/2up?q=\%22dean+be+poxed\%22&view=theater}{J. Swift, \textit{Journal} 22})
\z

\il{Danish!pokker@\textit{pokker}|(}
In the same way Danish \textit{pokker} is used, as in \refp{ex:04-229}. Also with \textit{heller}; as above: \refp{ex:04-232}.

\ea \label{ex:04-229}
\ea\il{Danish!ikke@\textit{ikke}}
\gll I kiørte pokker, I! og ikke til majoren\\
you drove pox you and not to major.\DEF{}\\
\glt `Oh sure you drove off, but not to the major'
\hfill(\href{https://books.google.com/books?id=9_-Z9j0M3z8C&pg=PA4&dq=\%22ki%C3%B8rte+pokker%22+wessel&hl=en&newbks=1&newbks_redir=0&sa=X&ved=2ahUKEwj1-5_B282EAxXZia8BHSJYCUMQ6AF6BAgLEAI#v=onepage&q=\%22ki%C3%B8rte%20pokker%22%20wessel&f=false}{Wessel, \textit{Kierlighed} 4}) % ??? PE: I suppose you can drive to a major just as you can drive to a brigadier-general, but I've a hunch that this has some different meaning. But what meaning, I can't guess.
\ex
\gll Han tror vistnok, at han gør mig en hel glæde {\dots} Han gør pokker, gør han\\
 he believes surely that he does me a whole joy {} he does pox, does he\\
\glt `He surely thinks that he is doing me a great favour {\dots} He does nothing of the sort!'
\hfill(\href{https://tekster.kb.dk/text/adl-texts-topsoe01-root}{Topsøe, \textit{Skitseb.} 107}) % Vilhelm Topsøe writes this on page 141 of the book shown in ; but he does so with different (older?) orthography: "han gjør Pokker, gjør han", etc.
%Brett: No, it's there as quoted, just a little before the part you quote: "Han tror vistnok, at han gjør mig en hel Glæde ved at give mig det elendige Herredsfoged-Embede; han gjør Pokker, gjør han." % Peter: OK
\ex
\gll Han har pokker, har han!\\
 he has {the pox} has he\\
\glt `He has done nothing of the sort!'
\hfill(Hørup 2.173) % PE: "He has the pox, he does!" doesn't seem to have any negating sense of "pokker". Is this perhaps a mistranslation? (Or is this rather bleached of meaning: "He's fucked, he is!" "He's up shit creek, he is!" etc?)
%Brett: I assume the second.
% ??S Peter: Where is this from?
%% SG: "pokker" is used as an emphatic negation. See earlier comment on the source of Hørup's speeches
\z
\z
\largerpage

\ea \label{ex:04-232}
\gll ``Det retter sig med aarene.'' {---} ``Det gjør pokker heller''\\
 it corrects itself with {years.\DEF} {} that does pox either\\
\glt `It will sort itself out over the years.' --- `That will be the day!'\\
\hfill(\href{https://archive.org/details/alexanderlangek00kielgoog/page/n269/mode/2up?q=\%22Det+gjor+Pokker+heller%21\%22&view=theater}{Kielland, \textit{Jacob} 67})
\z
\il{Danish!pokker@\textit{pokker}|)}
\is{pox@`pox', use of words meaning}

\is{unknowability, figurative use of|(}
\textit{God} (or \textit{Heaven}) \textit{knows} is in all languages a usual way of saying `I don't know'; the underlying want of logic is brought out in \refp{ex:04-233}.

\ea \label{ex:04-233}
``wheres thy maister?'' --- ``God in heauen knowes.'' --- ``Why, dost not thou know?'' --- ``Yes I know, but that followes not.''\hfill(\href{https://babel.hathitrust.org/cgi/pt?id=uiuo.ark:/13960/t1pg73v3n&seq=84&q1=wheres+thy+maister}{Marlowe, \textit{Faustus} 200})
\z

But inversely \textit{Heaven knows} also serves as a strong asseveration, as in \refp{ex:04-234}.

\ea \label{ex:04-234}
``For we were very happy then, I think.''\\``Heaven knows we were!'' said I.\hfill(\href{https://archive.org/details/personalhistory05dickgoog/page/n341/mode/2up?q=\%22we+were+very+happy+then\%22&view=theater}{Dickens, \textit{David} 786}) % OJ writes "We were happy" but Dickens writes "For we were very happy".
\z

Elsewhere (\cite[\href{https://archive.org/details/maalogminne1910olse/page/n213/mode/2up?q=\%22gud+veed\%22&view=theater}{36}]{jespersen1911ombanden}), I have mentioned that in Danish \textit{gud veed} (`God knows') is used to express uncertainty, and \textit{det veed gud} (`God knows that'), certainty; cf. \textit{Gud må vide om han er dum} (`God must know if he is foolish'); uncertainty, but \textit{gud skal vide, han er dum} (`God shall know he is foolish'); certainty.
\is{unknowability, figurative use of}

\is{devil@`devil', expressions meaning|)}
\is{grammaticalization|)}

\AddSubSection{Hypothetical clauses} \label{sec:hypothetical}
\is{counterfactual clauses}
\is{hypothetical clauses}

Hypothetical clauses, like \textit{if I were rich} (nowadays also in the indicative:
\textit{if I was rich}) or \textit{if I had been rich} are often termed ``clauses of rejected condition'', but as it is not the condition that is rejected but that which is (or would be) dependent on the condition, (for instance, \textit{I should travel}, or \textit{I should have travelled}) a better name would be ``clauses of rejecting condition''. At any rate they express by the tense (and mood) that something is irreal, implying `I am not rich'. 

The negative idea may be strengthened in the same way as a pure negative, cf. \refp{ex:04-235}, implying: `she does not care a button for you'.

\ea \label{ex:04-235}
What your poor wife would do if she cared \textit{a button} for you, I don't know.\\\hfill(\href{https://archive.org/details/dollydialogues00hope_0/page/150/mode/2up?view=theater&q=\%22cared+a+button%22}{Hope, \textit{Dialogues} 202})
\z

\AddSubSection{Other more or less indirect ways} \label{sec:other-indirect}

There are other more or less indirect ways of expressing a negative, e.g. \refp{ex:04-236}.

\ea \label{ex:04-236}
\ea
recollections which were \textit{any thing rather than} agreeable\\\hfill(\href{https://archive.org/details/antiquary20unkngoog/page/n98/mode/2up?q=\%22rather+than+agreeable\%22&view=theater}{Scott, \textit{Antiquary} 1.65})
\ex
leaving her lover in \textit{anything but} a happy state of mind\\\hfill(\href{https://archive.org/details/warden0000anth_w6p5/page/80/mode/2up?q=\%22anything+but+a+happy\%22&view=theater}{Trollope, \textit{Warden} 85})
\ex
it is \textit{the reverse of} important to my position\hfill(\href{https://archive.org/details/ourmutualfriendc0000char/page/204/mode/2up?q=\%22reverse+of+important\%22&view=theater}{Dickens, \textit{Friend} 275})
\ex
the constitution of his mind made it \textit{the opposite} of natural for him to credit himself with {\dots}\hfill(\href{https://www.gutenberg.org/cache/epub/4526/pg4526-images.html}{Gissing, \textit{Born} 339})
\ex
I am \textit{at a loss} to understand it. % I don't think we should risk boring readers with more detail; but if we did want to do so, this exact sentence appears in Wilkie Collins, The Legacy of Cain https://www.gutenberg.org/files/1975/1975-h/1975-h.htm and Sydney George Fisher, The True Benjamin Franklin https://www.gutenberg.org/files/34193/34193-h/34193-h.htm . Neither writer is quoted elsewhere in this book; but Collins does appear in the list of works cited in MEG 7.
%Brett: I say we take a pass on this opportunity. % Peter: Agreed
\z
\z

Cf. Danish \refp{ex:04-241}. Below we shall see a further development of \textit{andet end}. % ??? PE: Where do we see this? My guess: Within chapter 7, a sizable passage that starts "In this connexion I must mention a Danish expression" (corresponding to pp 76-77 of the printed book). If you agree, let's add a link there from the word "below" here.

\ea \label{ex:04-241}
\gll Der havde været tidsafsnit, hvor han laa \textit{alt} \textit{andet} \textit{end} paa den lade side.\\
 there had been periods where he lay anything other than on the lazy side\\
\glt `There had been times when he was anything but lazy.'\\
\hfill(\href{https://tekster.kb.dk/text/adl-texts-drachmann14val-root#s517}{Drachmann, \textit{Forskrevet} 2.190}) % PE: I changed "periods" to "times"
\z

On the whole it may be said that words like \textit{other} (\textit{otherwise}, \textit{else}, \textit{different}) in all languages are used as negative terms; cf. also \textit{I had to decide upon the desirability or otherwise} (`or the undesirability') \textit{of leaving him there}.

Negation is also implied in expressions with \textit{too} (\textit{she is too poor to give us anything} (`she cannot' {\dots})) and in all second members of a comparison after a comparative (\textit{she is richer than you think} `you do not think that she is so rich as she really is'); % PE: Can we tinker with this and thereby avoid parentheses (or brackets) within parentheses?
%Brett: What about simply omitting the internal parens in this case? % PE: Done.
hence we understand the use of French \il{French!ne@\textit{ne}}\textit{ne} \refp{ex:04-242} and the development of negatives to signify `than', as in \refp{ex:04-243}, and often \textit{nor} as dialectal \refp{ex:04-244}. See \citet[\href{https://archive.org/details/indogermanischef32berluoft/page/338/mode/2up}{339}]{holthausen1913negation} 
and for Slavonic \citet[\href{https://archive.org/details/vergleichendesl00vondgoog/page/335/mode/2up?view=theater}{336}]{vondrak1908vergleichende}. 

\ea \label{ex:04-242}\il{French!ne@\textit{ne}}
elle est plus riche que vous ne croyez
\z

\ea \label{ex:04-243}
you are more used to it \textit{nor} I, as Mr. Raymond says\hfill(\href{https://archive.org/details/journaltostellae00swifuoft/page/498/mode/2up?q=\%22are+more+used+to+it\%22&view=theater}{J. Swift, \textit{Journal} 499}) % Swift italicizes "nor"
\z

\ea \label{ex:04-244}
\ea
a fine sight more schoolin' \textit{nor} I ever got\hfill(\href{https://archive.org/details/millonfloss0009geor/page/4/mode/2up?q=\%22fine+sight+more+schoolin%27\%22&view=theater}{Eliot, \textit{Mill} 1.6}) % OJ doesn't provide the example; I (PE) am pretty sure that this is what he had in mind.
%Brett: Looks good to me
\ex
I'd sooner be a dog \textit{nor} a trainer.\hfill(\href{https://archive.org/details/cashelbyronsprof00shawuoft/page/n95/mode/2up?q=\%22sooner+be+a+dog\%22&view=theater}{Shaw, \textit{Cashel} 69} (vulgar)) 
\z\z

\AddSubSection{Continuation with \textit{much less}}
\is{emphatic exclusion modifiers|(}

The indirect way of expressing the negative notion is responsible for a pretty frequent continuation with \textit{much less} (which is practically synonymous with \textit{not to speak of} and corresponds very nearly in many instances to Danish \textit{endsige}, German \textit{geschweige denn} to introduce a stronger expression), as in \refp{ex:04-246}.\largerpage[2]

\ea \label{ex:04-246}
\ea
How very long since I have thought Concerning---much less wished for---aught Beside the good of Italy \phantom{x} (`I have not long thought')\\\hfill(\href{https://archive.org/details/dli.ernet.212792/page/19/mode/2up?q=\%22aught+Beside+the+good+of+Italy\%22&view=theater}{R. Browning, \textit{Italian}})
\ex
it would need long years, not a few crowded months, to master the history of Venice, much less that of Italy, for the whole Middle Ages \phantom{x} (`it is impossible in the course of a few months')\hfill(\href{https://archive.org/details/in.ernet.dli.2015.179024/page/n85/mode/2up?q=\%22would+need+long+years\%22&view=theater}{Harrison, \textit{Ruskin} 73})
\ex
Why did he ever write, much less publish, his memoirs?\newline (`he should not have')\hfill(Harrison (on Mark Pattison))
\ex
Why were you so weak, mother, as to admit such an enemy to your house---one so obviously your evil genius---much less accept him as a husband?\hfill(\href{https://archive.org/details/lifeslittleironi00hard_0/page/38/mode/2up?q=\%22much+less+accept+him+as+a+husband%3F\%22&view=theater}{Hardy, \textit{Ironies} 46}) % OJ skips ", mother,". Reinstated. (OJ also skips the end of the sentence, ", after so long".)
\ex
a place of Dantesque gloom at this hour, which would have afforded secure hiding for a battery of artillery, much less a man and a child\newline (`where you could not see {\dots} much less')\hfill(\href{https://archive.org/details/lifeslittleironi00hard_0/page/172/mode/2up?q=\%22much+less+a+man+and+a+child\%22&view=theater}{ibid 201})
\ex
the child thought it was a marvellous feat to read it, much less know precisely how to chant it \phantom{x} (`it was not easy')\hfill(\href{https://archive.org/details/dreamersofghetto00zangiala/page/8/mode/2up?q=\%22the+child+thought+it+was+a+marvellous+feat+to+read+it%2C+much+less+know+precisely+how+to+chant+it\%22&view=theater}{Zangwill, \textit{Child}}) % (i) "was" restored. (ii) OJ's original reference ("Zangwill in Cosmopolis '97. 619") means, I think, that it should be in https://archive.org/details/sim_connoisseur-a-journal-of-music-and-the-fine-arts_1897-09_7_21/page/n297/mode/2up?view=theater&q=\%22marvellous%22 ; however, it isn't.
\ex
Is it right to entrust the mental development of a single child, much less a class of children, to a man who is ignorant of mental science?\\\hfill(news 1907)
\z
\z

Thus also in Danish, e.g. \refp{ex:04-253}.

\ea \label{ex:04-253}
\ea
\gll hvem skulde ta sig det nær, langt mindre blive hidsig\\
 who should take oneself it near far less become angry\\
\glt `who should take it to heart, let alone become angry'\\
\hfill(\href{https://danskestudier.dk/wp-content/uploads/2018/06/1909.pdf}{Gravlund, \textit{Kristrup} 86}) % Can we get something less unwieldy than this PDF? PS: Yes; got it; switched.
%Brett: How's this? Can't link to the page. % Peter: The Yumpu link https://www.yumpu.com/da/document/read/18259555/danske-studier-1909 works for me -- but it's a giant PDF. Another problem is that it's Yumpu, which I think will "publish" anything.
% PE: PS: The danskestudier.dk PDF should be fine.
\ex
\gll Det er vistnok første gang, at han overhovedet har været i Rømersgade {---} langt mindre talt der\\
 it is surely first time, that he {at all} has been in Rømersgade {} far less spoken there\\
\glt `It is surely the first time that he has been in Rømersgade at all---let alone spoken there'
\hfill(news 1915)
\z
\z

\is{confusion of negation|(}
In a similar way, we have \textit{impossible} followed by \textit{much less} (`much less possible'): \refp{ex:04-255}.\largerpage

\ea \label{ex:04-255}
\ea
It was impossible that this should be, much less in the labour ghetto south of Market.\hfill(\href{https://archive.org/details/in.ernet.dli.2015.65698/page/n265/mode/2up?q=\%22much+less+in+the+labour+ghetto+south+of+Market\%22&view=theater}{London, \textit{Martin} 314})
\ex
it is impossible for a Prime Minister to follow, far less to supervise, the work of individual Ministers\hfill(news 1914)
\ex
to make any extracts from it---still less to make any extracts which should do justice to it---is almost impracticable\hfill(\href{https://archive.org/details/henryfieldingeng0000aust/page/98/mode/2up?q=\%22make+any+extracts\%22&view=theater}{Dobson, \textit{Fielding} 105})
\z
\z

By a similar confusion, Carlyle uses \textit{much more}, because he is thinking of something like: `it is impossible for {\dots} to foster the growth of anything' \refp{ex:04-258}.

\ea \label{ex:04-258}
How can an inanimate, mechanical Gerund-grinder {\dots} foster the growth of any thing; much more of Mind, which grows {\dots} by mysterious contact of Spirit?\hfill(\href{https://archive.org/details/sartorresartus02unkngoog/page/94/mode/2up?view=theater&q=\%22gerund-grinder%22}{\textit{Sartor} 73}) % Restoring "mechanical", which OJ omits, and replacing OJ's "anything" with C's "any thing".
\z

\textit{Much more} would have been more apposite than \textit{much less} in \refp{ex:04-259}.

\ea \label{ex:04-259}
I loved you hard enough to melt the heart of a stone, much less the heart of the living, breathing woman you are.\hfill(\href{https://archive.org/details/martineden00lond/page/180/mode/2up?q=\%22melt+the+heart\%22&view=theater}{London, \textit{Martin} 181})
\z
\is{confusion of negation|)}
\is{emphatic exclusion modifiers|)}
\is{indirect negation|)}

\addsec{Incomplete negation}\label{sec:incomplete-negation}
\is{incomplete negation|(}
\is{negative polarity items|(}
\is{quantifiers!negative|(}
\is{grammaticalization|(}

\is{approximate negatives|(}
Among \textsc{approximate negatives}, we must first mention \textit{hardly}, which from signifying `with hardness', i.e. `with difficulty' comes to mean `almost not'; the negative import is shown by the possibility of strengthening \textit{hardly} by adding \textit{at all} (which is only found with negative expressions). In this sense, \textit{hardly} follows the general tendency to place negatives before the notion negatived (see above, p.~\pageref{para:naturaltendency}): \textit{I hardly know}. Cf. \citet[\href{https://archive.org/details/in.ernet.dli.2015.83994/page/n37/mode/2up?view=theater}{§1847}]{sweet1898new2} on the difference between \textit{I hardly think we want a fire} and \textit{to think hardly of a person}.

Corresponding words in other languages, like Danish \textit{vanskeligt}, German \il{German!schwerlich@\textit{schwerlich}}\textit{schwerlich}, French \il{French!a peine@\textit{à peine}|(}\textit{à peine}, also have approximately the value of a negative, though perhaps not quite so much as \textit{hardly}.

\textit{Scarcely} (obsolete adverb \textit{scarce}) also is what the \textit{NED} terms ``a restricted negative'' (`not quite'); in the same way Danish \il{Danish!knap@\textit{knap}}\textit{knap}, \il{Danish!næppe@\textit{næppe}}\textit{næppe}, \il{Danish!knebent@\textit{knebent}}\textit{knebent}, German \il{German!kaum@\textit{kaum}}\textit{kaum}.

Note the use after words meaning \textit{before} in \refp{ex:04-260}.\largerpage[2]

\ea \label{ex:04-260}
\ea
Recollection returned before I had scarcely written a line\\\hfill(\href{https://archive.org/details/oed8barch/page/n193/mode/2up?view=theater}{\textit{NED}, \textit{Scarcely} 2~b, quot. 1795}) % Is it worth adding that the NED attributes this to p. 2.158 of Fate of Sedley (1795), i.e. the anonymous The Fate of Sedley (written by Henry Summersett)?
%Brett: No. % Peter: Thank you for saving me ten minutes of my finite life!
\ex
\gll Avant de savoir {à peine} écrire ses lettres, il s'évertua à griffonner\\
 before {of to} know hardly write his letters he endeavoured to scribble\\
\glt `Even before he could properly write his letters, he strove to scribble'\\\nopagebreak
\hfill(\href{https://archive.org/details/jeanchristoph01roll/page/168/mode/2up?q=\%22Avant+de+savoir+%C3%A0+peine+%C3%A9crire+ses+lettres\%22&view=theater}{Rolland, \textit{Aube} 168})\il{French!a peine@\textit{à peine}|)}
\ex
\gll Og før han knap selv vidste deraf, gik Berg med en politiker i maven\\
 and before he scarcely himself knew thereof walked Berg with a politician in stomach.\DEF{}\\
\glt `And before he scarcely knew it himself, Berg was harbouring a politician within him'
\hfill(\href{https://books.google.com/books?id=r9YIEAAAQBAJ&pg=PT89&lpg=PT89&dq="gik+Berg+med+en+politiker+i+maven"&source=bl&ots=22TFOCNZP3&sig=ACfU3U2QM28iQTDnT9OF7lfuzEhogycZNQ&hl=en&sa=X&ved=2ahUKEwibuLqAwtWEAxXF0jQHHYXiDTsQ6AF6BAgJEAM#v=onepage&q="gik Berg med en politiker i maven"&f=false}{Henrichsen, \textit{Mændene} 108}) % ??S Try to find something more suitable than this edition of 2020
% PE: ??? For some reason that I can't articulate, "before he himself scarcely knew it" sounds wrong. How about "before he himself was scarcely aware of it"?
\z
\z

In English \textit{scarcely any} / \textit{scarcely ever} is % Peter: change "is" to "are"? Put "or" between the pair?
%Brett: Sure, or a slash?
% PE: Good, I've slashified what had been a comma.
generally preferred to the combinations \textit{almost no}, \textit{almost never}.

But \textit{almost} with \textit{no}, \textit{nothing}, \textit{never} is not quite so rare as most grammarians would have us think; it is perhaps more Scotch (and American) than British, hence Boswell in later editions changed \textit{I suppose there is almost no language} (\href{https://quod.lib.umich.edu/e/ecco/004839390.0001.002/1:66.1?rgn=div2;view=fulltext}{\textit{Life}\textsubscript{A}~2.222}) to \textit{we scarcely know of a language} (\href{https://archive.org/details/lifeofsamueljohn02boswuoft/page/784/mode/2up?view=theater&q=\%22we+scarcely+know+of%22}{\textit{Life}\textsubscript{B}~2.785}). % I (PE) have tinkered with the bit about Boswell's phrasing.
In the following quotations, I have separated British \refp{ex:04-263}, Scotch \refp{ex:04-268}, and American examples \refp{ex:04-273}.  % PE: How about "... and American examples: \refp{ex:04-263}, \refp{ex:04-268} and \refp{ex:04-273} respectively"
%Brett: Of course. Done.

\ea \label{ex:04-263}
\ea here is almost no fier\hfill(\href{https://archive.org/details/gammergurtonsnee00stiluoft/page/n29/mode/2up?view=theater}{\textit{Gammer} 104}) % This copy is hard to read, but yes, "almost no fier" appears on the page. (Much) newer editions spell it "fire".
\ex {}[unidentified] \hfill (Bacon, see \citet[{74}]{bogholm1906bacon}) % ??? PE: What should one do with this? (Ideally, persuade somebody within easy reach of a first-rate Danish library to look in the book.)
\ex I shall remember almost nothing of the matter\\\hfill(\href{https://www.gutenberg.org/cache/epub/47790/pg47790-images.html#Page_107}{Cowper, letter, 6 March 1782})
\ex she had {\dots} found almost nothing\hfill(\href{https://archive.org/details/mansfieldpark00aust_1/page/324/mode/2up?q=\%22she+had+hoped+much\%22&view=theater}{Austen, \textit{Mansfield} 362}) % JA: "she had hoped much, and found almost nothing"; OJ's version of this ("she has found almost nothing") is corrupt.
\ex almost nothing definite\hfill(\href{https://archive.org/details/historydavidgri02wardgoog/page/218/mode/2up?q=\%22almost+nothing+definite\%22&view=theater}{Ward, \textit{David} 2.51}; also see \citet[\href{https://archive.org/details/p2englischephilo01storuoft/page/942/mode/2up?view=theater}{942}]{storm1896englische})
\z
\ex \label{ex:04-268}
\ea rites {\dots} which are now rarely practised in Protestant countries, and almost never in Scotland\hfill(\href{https://archive.org/details/cewaverleynovels03scotuoft/page/248/mode/2up?view=theater&q=\%22rarely+practised%22}{Scott, \textit{Antiquary} 2.66}) % OJ has "rights", which is plain wrong -- and actually it's " rites to which in former times these walls were familiar, but which {\dots}"
\ex open to all, seen by almost none\hfill(\href{https://archive.org/details/heroesheroworshi00carl/page/74/mode/2up?q=\%22seen+by+almost+none\%22&view=theater}{T. Carlyle, \textit{Heroes} 76})
\ex Nothing, or almost nothing, is certain to me, except the Divine Infernal character of this universe\hfill(\href{https://archive.org/details/thomascarlylehis0000jame/page/n67/mode/2up?q=\%22almost+nothing\%22&view=theater}{T. Carlyle, \textit{Life} 3.62})
\ex what {\dots} could he look for there? Exasperated Tickets of Entry answer: Much, all. But cold Reason answers: \textit{Little, almost nothing}\\\hfill(\href{https://archive.org/details/gri_33125008092856/page/4/mode/2up?q=\%22cold+reason+answers\%22&view=theater}{T. Carlyle, \textit{Revolution} 406}) %TC omits ", on the whole."
\ex On first entering I could see almost nothing\hfill(\href{https://babel.hathitrust.org/cgi/pt?id=nyp.33433074931001&seq=117&q1=%22on+first+entering%22}{Buchanan, \textit{Anthony} 97})
\z
\ex \label{ex:04-273}
\ea He himself was almost never bored\hfill(\href{https://archive.org/details/american01jamegoog/page/n162/mode/2up?q=\%22almost+never+bored\%22&view=theater}{James, \textit{American} 1.265})
\ex the academies paid almost no attention whatever to English instruction\hfill(\href{https://archive.org/details/teachingofenglis02carp/page/44/mode/2up?q=\%22academies+paid\%22&view=theater}{Carpenter, \textit{Teaching} 44})
\z
\z
\largerpage[2]

\textit{Little} and \textit{few} are also incomplete negatives; note the frequent collocation with \textit{no}: \textit{there is \textsc{little or no} danger}; \textit{There have been \textsc{few or no} attempts at denial}; note also the rise of \textit{yet} in \refp{ex:04-275}. Other examples (the last with \textit{little} before a plural): \refp{ex:04-276}.

\ea \label{ex:04-275}
I have yet seen little of Florence\hfill(\href{https://archive.org/details/worksinversepros08sheluoft/page/130/mode/2up?view=theater&q=\%22little+of+florence%22}{Shelley, \textit{Prose} 2.299})
\ex \label{ex:04-276}
\ea There's few or none do know me\hfill(\href{https://internetshakespeare.uvic.ca/doc/Jn_F1/scene/4.3/index.html#tln-1995}{Shakespeare, \textit{John} 4.3.3})
\ex with few wise longings and but little love\hfill(\href{https://en.wikisource.org/wiki/The_Prelude_(Wordsworth)/Book_III}{Wordsworth, \textit{Prelude} 3.626})
\ex the situation showed little signs of speedy development\\\hfill(\href{https://archive.org/details/fatherstafford00hope/page/66/mode/2up?q=\%22signs+of+speedy+development\%22&view=theater}{Hope, \textit{Father} 38})
\z
\z

\label{para:little-a-little}The negative force of \textit{little} is seen very clearly when (like other negatives, see p.~\pageref{para:naturaltendency}) it is placed before the verb. ``This use is confined to the vbs. \textit{know}, \textit{think}, \textit{care}, and synonyms of these'' (\href{https://archive.org/details/oed6aarch/page/n365/mode/2up?view=theater}{\textit{NED}, \textit{Little} C \textit{adv.}~1~b} with examples so far back as 1200): \refp{ex:04-279}. It may be mentioned for the curiosity of the thing that \textit{little} and \textit{much} (see~p.~\pageref{sec:ironic-much}) mean exactly the same in \textit{Little} (\textit{much}) \textit{she cares what I say}.

\ea \label{ex:04-279}
\ea
I little thought, when I mounted him [John Gilpin] on my Pegasus, that he would become so famous\hfill(\href{https://www.gutenberg.org/cache/epub/47790/pg47790-images.html#Page_198}{Cowper, letter, 30 April 1785})
\ex
They little think what mischief is in hand\hfill(\href{https://archive.org/details/workslordbyron10unkngoog/page/218/mode/2up?view=theater&q=\%22little+think+what%22}{Byron, \textit{Juan} 5.1})
\ex
I little thought to have seen your honour here\hfill(\href{https://archive.org/details/antiquary20unkngoog/page/n34/mode/2up?q=\%22little+thought\%22&view=theater}{Scott, \textit{Antiquary} 1.21})
\ex
Little they thought, the brutes, how I was plotting for their amusement\hfill(\href{https://archive.org/details/hypatia00kinggoog/page/n298/mode/2up?q=\%22little+they+thought\%22&view=theater}{Kingsley, \textit{Hypatia} 236}) % OJ had removed "the brutes"
\ex
He little knew the cause of what he saw\hfill(\href{https://archive.org/details/in.ernet.dli.2015.53170/page/n265/mode/2up?q=\%22little+knew+the+cause\%22&view=theater}{Hope, \textit{Rupert} 205})
\z
\z

This negative \textit{little} is frequent with verbs and adjectives, but rarer with substantives; in \refp{ex:04-284} we have it with verbal substantives,\footnote{For example, \textit{little understanding} or \textit{little reduction in costs}. \eds} % Peter: Do you understand "verbal substantive"? I don't.
%Brett: Does the fn help?
% PE: So, "substantives derived from verbs"? (What we might today call "deverbal nouns"? Although even if yes, this is what it means, I'm not suggesting that we should describe it in this way.)
and \textit{or} in (\ref{ex:04-284}b) shows clearly the negative value of \textit{little}. % PE: I've replaced OJ's "the following quotations" with "\refp{ex:04-284}"

\ea \label{ex:04-284}
\ea reading in her two nieces' minds their
little approbation of a plan\\\hfill(\href{https://archive.org/details/mansfieldpark00aust_1/page/50/mode/2up?q=\%22reading+in+her+two\%22&view=theater}{Austen, \textit{Mansfield} 55}) % Correcting OJ's "their" to JA's "her two nieces'"
\ex as he or I had little interest in that\hfill(\href{https://archive.org/details/reminiscences0000thom_e9a0/page/232/mode/2up?q=\%22as+he+or+I\%22&view=theater}{T. Carlyle, \textit{Reminiscences} 294})
\z
\z

\is{a little@`a little', expressions meaning|(}
\is{articles, polarity affected by indirect|(}
While \textit{little} and \textit{few} are approximate negative, % ??? PE: Yes, the printed book has "approximate negative". But isn't "negative" here merely a typo for "negatives"? If indeed so, let's discreetly correct it.
\textit{a little} and \textit{a few} are positive expressions:\is{positive polarity items} \textit{he has little money} and \textit{he has few friends} express the opposite of \textit{much money} and\textit{ many friends} and therefore mean about the same thing as \textit{no money} and \textit{no friends}; but \textit{he has a little money} and \textit{he has a few friends}, generally with the verb stressed rather strongly, mean the opposite of \textit{no money} and \textit{no friends}, thus nearly the same thing as \textit{some money} and \textit{some friends}. \textit{Little} means `less than you would expect', \textit{a little} `more than you would expect' (\ref{ex:285}). Cf. p.~\pageref{ch8-not-a-little} below on \textit{not a little}, \textit{not a few}. Note in (\ref{ex:288}) the stress on \textit{are}.

\ea \label{ex:285}
\ea Unfortunately, little is left of the former splendour
\ex Fortunately, a little is still left of the former splendour
\ex Unfortunately, there are few who think clearly
\ex Fortunately, there are a few who think clearly\label{ex:288}
\z
\z

Shakespeare uses \textit{a few} in some cases % Peter: Shall we remove this comma?
%Brett: done
where now \textit{few} would be used without the article e.g. (\ref{ex:289}). The difference between \textit{a little} and \textit{little} is well brought out in (\ref{ex:290}). On the other hand, \textit{little} is positive in (\ref{ex:291}).

\ea \label{ex:289}
Loue all, trust a few, Doe wrong to none\\\hfill(\href{https://internetshakespeare.uvic.ca/doc/AWW_F1/index.html#tln-65}{\textit{All} 1.1.73}, see \citet[\href{https://www.perseus.tufts.edu/hopper/text\%3Fdoc\%3DPerseus\%3Atext\%3A1999.03.0079\%3Aalphabetic\%2Bletter\%3DF\%3Aentry\%2Bgroup\%3D7\%3Aentry\%3DFew}{\textit{Few}}]{schmidt1886}) % PE: Restoring "Doe" rather than OJ's "Do". 
%Brett: When a link with % or another suspect character doesn't work, put a backslash before it: \%  PE: Thanks for the timely reminder!

\ex when he is best, he is a little worse then a man, and when he is worst, he is little better then a beast\hfill(\href{https://internetshakespeare.uvic.ca/doc/MV_F1/scene/1.2/index.html#tln-275}{\textit{Merch} 1.2.95})\label{ex:290} % PE: not "than" but "then" 

\ex love me \textit{little} and love me long\\\hfill{(mentioned as a proverb as early as 1548, \href{https://archive.org/details/oed6aarch/page/n365/mode/2up?view=theater}{\textit{NED}, \textit{Little} C~\textit{adv.}~1})}\label{ex:291}
\z

Note the different idioms with the two synonyms \textit{but} and \textit{only}: \textit{there is \textsc{but little} difference} (`there is \emph{only a little} difference'); \textit{there are \textsc{but few} traces left} (`there are \emph{only a few} traces left'). See e.g. \refp{ex:292}. 

\ea\label{ex:292}
\ea
``How many gentlemen haue you lost in this action?'' --- ``But few of any sort, and none of name''\hfill(\href{https://internetshakespeare.uvic.ca/doc/Ado_F1/index.html#tln-5}{Shakespeare, \textit{Ado} 1.1.7}) % OJ here anomalously modernizes the spelling; I (PE) have gone back to the first folio, and also added quotation marks.
\ex\label{ex:293}
The fog has lifted only a little; only a few big landmarks are yet visible.\hfill(\href{https://archive.org/details/sim_new-republic_1917-01-27_9_117/page/340/mode/2up?q=\%22fog+has+lifted\%22&view=theater}{Lippmann, \textit{Speaks} 341})
\ex\label{ex:294}
For but few of them that begin to come hither do shew their face on these mountains\hfill(\href{https://archive.org/details/bunyanspilgrims00moffgoog/page/158/mode/2up?view=theater&q=\%22shew+their+face%22}{Bunyan, \textit{Progress} 156})
\ex\label{ex:295}
a passion such as a few only are capable of attaining\\\hfill(\href{https://archive.org/details/sowersnovel00merr/page/196/mode/2up?q=\%22passion+such+as\%22&view=theater}{Merriman, \textit{Sowers} 124})
\z
\z

In America, \textit{a little} is to such an extent felt as a positive term that it can be strengthened by \textit{quite}: \textit{quite a little} means nearly the same thing as `a good deal', and \textit{quite a few} as `a good many'. This is rare in England, see (\ref{ex:296}). 

\ea In quite a little time Mrs. Britling's mind had adapted itself\\\hfill(\href{https://archive.org/details/mrbritlingseesi02unkngoog/page/n282/mode/2up?q=\%22Mrs.+Britling%E2%80%99s+mind+had+adapted\%22&view=theater}{Wells, \textit{Britling} 264})\label{ex:296}
\z


Practically the same distinction as between \textit{little} and \textit{a little} is made between French \il{French!peu@\textit{peu}}\textit{peu} and \il{French!un peu@\textit{un peu}}\textit{un peu}, Italian and Spanish \il{Spanish!poco@\textit{poco}}\il{Italian!\textit{poco}}\textit{poco} and \il{Spanish!un poco@\textit{un poco}}\il{Italian!\textit{un poco}}\textit{un poco}, German (Middle High German) \il{German!Middle High German!wenig@\textit{wenig}}\textit{wenig} and \il{German!Middle High German!ein wenig@\textit{ein wenig}}\textit{ein wenig}. Has this developed independently in each language? In Danish, the corresponding differentiation has been effected in another way: \il{Danish!lidet@\textit{lidet}}\textit{lidet} (literary) or generally \il{Danish!kun lidt@\textit{kun lidt}}\textit{kun lidt} (`little'), \textit{lidt} or very often \il{Danish!en smule@\textit{en smule}}\textit{en smule} (`a little'). \is{articles, polarity affected by indirect|)}
\is{a little@`a little', expressions meaning|)}

\textit{Small} has not exactly the same negative force as its synonym \textit{little}, cf. however (\ref{ex:297}) where \textit{either} is due to the negative notion. Cf. also \textit{slight} in (\ref{ex:298}).

\ea Small thanks you get for it either\hfill(\href{https://archive.org/details/christianstory00cainrich/page/42/mode/2up?q=\%22small+thanks+you\%22&view=theater}{Caine, \textit{Christian} 36}) \label{ex:297}
\ex she had slight hope that any other caller would appear\hfill(\href{https://www.gutenberg.org/cache/epub/4526/pg4526-images.html}{Gissing, \textit{Born}})\label{ex:298} % OJ attributes this to Gissing's The Nether World, but it doesn't appear in that book.
\z

The comparative of \textit{little} has a negative meaning, especially in the old combination Old English % Peter: Yes, OJ writes "old combination OE". If "Early Old English" is "a thing", is this what he might mean? Or can/should we simply remove "old combination"?
%Brett: ¯\_(ツ)_/¯
\il{English!Old English!thy laes the@\textit{þy læs þe}}\textit{þy læs þe}, which has become \textit{lest} and is the equivalent of `that not'. (With a following \textit{not} it means the positive `in order that' as in \refp{ex:299}.) With this should be compared the Latin \textit{minus} in \textit{quo minus} and \textit{si minus}.

\ea \label{ex:299}
But least you should not vnderstand me well, {\dots} I would detaine you here some month or two\hfill(\href{https://internetshakespeare.uvic.ca/doc/MV_F1/scene/3.2/index.html#tln-1345}{Shakespeare, \textit{Merch} 3.2.7}) % OJ simply omits line 8 ("And yet a maiden hath no tongue, but thought,") between lines 7 and 9; I (PE) have put in dots. Also, not "lest' but "least"
\z
\is{approximate negatives|)}
\is{incomplete negation|)}
\is{negative polarity items|)}
\is{quantifiers!negative|)}
\is{grammaticalization|)}
