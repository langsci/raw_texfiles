\chapter{List of sources}
\markdouble{%
      \chapappifchapterprefix{\ }%
      \thechapter\autodot\enskip List of sources%
    }

The first edition of \textit{Negation} explains only a frequently used dozen among its many abbreviations for quoted sources, for the rest referring the reader to the lists in both volume II % PE: In his preface, OJ writes of "volume III or IV" of MEG; thus Roman numerals here too.
(1914) of \textit{A Modern English grammar} and (if it ever appeared) volume III. A typical entry in the former list provides author(s); title; publisher or place of publication (but not both); and year of publication. Volume III contains no list; but volume VII (1949) contains the forty-page ``Abbreviations and list of books quoted in vols  II--VII'', for whose items ``The original year of publication is often added in parenthesis.'' % PE: Yes, in "parenthesis" singular. I'm keen neither to pluralize this nor to apply "[sic]".
A number of abbreviations in \textit{Negation} for English-language sources appear in neither list, and, unsurprisingly, \textit{A Modern English grammar} does not cite from languages other than English; thus neither list explains such cryptic attributions in \textit{Negation} as ``Chr. Pedersen 4.~493'', ``Goldschm. Ravn. 65'', or ``Pontopp. Landsbyb. 155''.

Where we believe we can identify the edition to which Jespersen refers, we may augment the information he provides; otherwise we simply relay what he writes.

In his quotations, Jespersen is careful to retain the original spelling (although not the original capitalization); so we have treated titles in the same way: thus for example \textit{Gammer Gurtons nedle} rather than \textit{Gammer Gurton's needle}.

``Perf.'', ``pub.'', and ``wr.'' indicate ``first performed'', ``first published'', and ``written'' respectively. In order to save space, these gloss over numerous distinctions, such as old- versus new-style calendar and serial versus book publication. To the same end, we have also deleted most subtitles and the like. % The vague "and the like" is intended to encompass such truncations as that of "The Life and Opinions of Tristram Shandy, Gentleman" into simple "Tristram Shandy"

An abbreviation in brackets---such as ``[L]'' or ``[Ca.St]''---shows how to interpret the numbering provided for a quoted work: \textbf{A}, act; \textbf{Bk}, book; \textbf{Ca}, canto; \textbf{Ch}, chapter; \textbf{L}, line; \textbf{Let}, letter; \textbf{Ps}, psalm; \textbf{Sc}, scene; \textbf{St}, stanza; \textbf{V}, verse. % Combinations actually used: A, A.L, A.Sc, A.Sc.L, Bk.Ch, Bk.L, Ca, Ca.St, Ch.V, Gr.L, L, Let, Ps.V, St
Thus ``Milton, \textit{Samson} 210'' refers to line 210 of \textit{Samson agonistes} (marked ``[L]'' below); and ``Byron, \textit{Juan} 10.53'' to canto 10, stanza 53 of \textit{Don Juan} (``[Ca.St]''). Unless noted otherwise, % A few references to Shakespeare are to the first quarto.
the three numbers for a quotation from a play by Shakespeare specify the act, scene, and line in the first folio. For other works whose numbering is not explained within brackets below, interpret a single number as page and a pair as volume and page; thus ``Austen, \textit{Sense} 253'' means \textit{Sense and sensibility}, p.~253, and ``Scott, \textit{Antiquary} 2.36'' means \textit{The antiquary}, vol.~2, p.~36.

% \setlength{\extrarowheight}{4pt} % Adjust the 5pt as needed to increase the space
\noindent\begin{xltabular}{\textwidth}{ lQ }
\lsptoprule
Reference & Details \\
\midrule
\endfirsthead
\midrule
Reference & Details \\
\midrule
\endhead
\lspbottomrule\endlastfoot

\hangindent=1em \hangafter=1 Ade, \textit{Artie} & George Ade (b.~1866). \textit{Artie} (pub.~1896). Chicago, 1897. \\

\hangindent=1em \hangafter=1 Ælfric, \textit{Homilies} & Ælfric of Eynsham (b. c. 955). Homilies (wr. c. 990--995). \\

\hangindent=1em \hangafter=1 AKC, letter & Anna Katrine C. (b. c. 1832). Letter (wr. 1864). In Karl Larsen (ed.), \textit{Under vor sidste krig} (pub.~1897). Copenhagen, 1897. 124.\\

\hangindent=1em \hangafter=1 Alexander  & William Alexander (b.~1826)\\
\hspace{1em}\textit{Johnny} & \textit{Johnny Gibb of Gushet\-neuk} (pub.~1869). Edinburgh, 1897. \\

\hangindent=1em \hangafter=1 Allen & Grant Allen (b.~1848) \\
\hspace{1em}\textit{Æsthetics} & ``\textit{Physiological æsthetics} and \textit{Philistia}.'' In Walter Besant et al., \textit{My first book}, 43--52. London, 1894. \\
\hspace{1em}\textit{Woman} & \textit{The woman who did} (pub.~1895). Tauchnitz, 1895. \\

\hangindent=1em \hangafter=1 Andersen & Hans Christian Andersen (b.~1805) \\
\hspace{1em}\textit{Baronesser} & \textit{De to baronesser} (pub.~1848). \\
\hspace{1em}\hangindent=1em \hangafter=1 \textit{Improvisa\-toren} & \textit{Improvisatoren} (pub.~1835). \\ % ???
\hspace{1em}\textit{O. T.} & \textit{O. T.} (pub.~1836). \\ % The abbreviation "O. T." is the actual title.

\hangindent=1em \hangafter=1 \textit{Andreas} & \textit{Andreas}. Anonymous Old English poem, found in a manuscript dated to the second half of the 10th century. In George Philip Krapp (ed.), \textit{``Andreas'' and ``The fates of the apostles'': Two Anglo-Saxon narrative poems}. Boston, 1906. [L]\\  % "Andr.", cited together with ``Gu.'' 

\hangindent=1em \hangafter=1 \textit{Apollonius} & \textit{Apollonius of Tyre}. Anonymous Old English prose translation, surviving in an 11th-century manuscript, of the Latin \textit{Historia Apollonii regis Tyri}.\\

\hangindent=1em \hangafter=1 Arnskov, \textit{Nielsen} & L. Th. Arnskov. ``Anders Nielsen.'' \textit{Tilskueren} 1914. 29--44. \\ % ??? PE: Can we get his full name?

\hangindent=1em \hangafter=1 Austen & Jane Austen (b.~1775) \\
\hspace{1em}\textit{Emma} & \textit{Emma} (pub.~1815). Tauchnitz, 1877. \\
\hspace{1em}\textit{Mansfield} & \textit{Mansfield Park} (pub.~1814). London, 1897. \\
\hspace{1em}\textit{Pride} & \textit{Pride and prejudice} (pub.~1813). London, 1894. \\
\hspace{1em}\textit{Sense} & \textit{Sense and sensibility} (pub.~1811). London, n.d. \\

\hangindent=1em \hangafter=1 AV \textit{1~Corinthians} & \textit{First epistle to the Corinthians}. \textit{The Holy Bible} (pub.~1611). [Ch.V] \\
AV \textit{Job} & \textit{Book of Job}. \textit{The Holy Bible} (pub.~1611). [Ch.V] \\
AV \textit{John} & \textit{Gospel of John}. \textit{The Holy Bible} (pub.~1611). [Ch.V] \\
AV \textit{Matthew} & \textit{Gospel of Matthew}. \textit{The Holy Bible} (pub.~1611). [Ch.V] \\
AV \textit{Psalms} & \textit{Book of psalms}. \textit{The Holy Bible} (pub.~1611). [Ps.V] \\

\hangindent=1em \hangafter=1 Baggesen, \textit{Værker} & Jens Baggesen (b.~1764). \textit{Jens Baggesens danske værker}. Copenhagen, 1845. \\

\hangindent=1em \hangafter=1 G. Bang & Gustav Bang (b.~1871)\\
\hspace{1em}\textit{Husmanden} & ``Husmanden.'' \textit{Tilskueren}. 1902. 372--388. \\

\hangindent=1em \hangafter=1 H. Bang & Herman Bang (b.~1857)\\
\hspace{1em}\textit{Fædra} & \textit{Fædra. Brudstykker af et livs historie} (pub.~1883). \\
\hspace{1em}\textit{Ludvigs\-bakke} & \textit{Ludvigsbakke} (pub.~1896). \\
\hspace{1em}\textit{Slægter} & \textit{Haabløse slægter} (pub.~1880). \\

\hangindent=1em \hangafter=1 Barrie, \textit{Margaret} & J. M. Barrie (b.~1860). \textit{Margaret Ogilvy} (pub.~1896). Tauchnitz, 1897. \\

\hangindent=1em \hangafter=1 \textit{Bede} & Late 9th-century Old English translation, traditionally attributed to King Alfred, of Bede's 8th-century \textit{Historia ecclesiastica gentis Anglorum}. [Bk.Ch] \\

\hangindent=1em \hangafter=1 Behn, \textit{Mistake} & Aphra Behn (b.~1640). \textit{The lucky mistake} (pub.~1689). In Ernest A. Baker (ed.), \textit{The novels}. London, 1904. \\
% \textit{Bemærkninger} & & Oskar Siesbye, Kristian Mikkelsen \& Otto Jespersen. ``Bemærkninger til afhandlingen `En sproglig værdiforskydning'.'' \textit{Dania} 10 (1896). 239--258. \\ Moved to BibTeX

Bennett & Arnold Bennett (b.~1867) \\
\hspace{1em}\textit{Anna} & \textit{Anna of the five towns} (pub.~1902). London, 1912. \\
\hspace{1em}\textit{Babylon} & \textit{The Grand Babylon Hotel} (pub.~1902). London, 1912. \\ % MEG 7 garbles the title.
\hspace{1em}\textit{Card} & \textit{The card} (pub.~1911). London, 1913. \\
\hspace{1em}\textit{Clayhanger} & \textit{Clayhanger} (pub.~1910). Tauchnitz, 1912. \\
\hspace{1em}\textit{Hilda} & \textit{Hilda Lessways} (pub.~1911). Tauchnitz, 1912. \\
\hspace{1em}\textit{Twain} & \textit{These twain} (pub.~1915). London, 1916. \\
\hspace{1em}\textit{Wives} & \textit{The old wives' tale} (pub.~1908). Tauchnitz, 1909. \\

\hangindent=1em \hangafter=1 E. F. Benson & E. F. Benson (b.~1867) \\
\hspace{1em}\textit{Arundel} & \textit{Arundel} (pub.~1914). London, 1915. \\
\hspace{1em}\textit{Dodo} & \textit{Dodo: A detail of the day} (pub.~1893). Tauchnitz, 1894.\\
\hspace{1em}\textit{Judgment} & \textit{The judgment books} (pub.~1895). London, 1895. \\
\hspace{1em}\textit{Second} & \textit{Dodo the second} (pub.~1913). Tauchnitz. \\ 

\hangindent=1em \hangafter=1 R. H. Benson  & Robert Hugh Benson (b.~1871)\\
\hspace{1em}\textit{None} & \textit{None other gods} (pub.~1911). London, n.d. \\

\hangindent=1em \hangafter=1 \textit{Beowulf} & \textit{Beowulf}. Old English epic poem, surviving in a manuscript dated to c.~1000 CE. [L] \\

\hangindent=1em \hangafter=1 Bersezio, \textit{Bolla} & Vittorio Bersezio (b.~1828). \textit{Una bolla di sapone} (pub.~1870). Milano, 1870. [A.Sc] \\

\hangindent=1em \hangafter=1 Birmingham  & George A. Birmingham (b.~1865)\\
\hspace{1em}\textit{Whitty} & \textit{The adventures of Dr. Whitty} (pub.~1913). London, 1913.\\ % MEG 7 says 1915; a mistake?

\hangindent=1em \hangafter=1 Bjørnson & Bjørnstjerne Bjørnson (b.~1832) \\
\hspace{1em}\textit{Flager} & \textit{Det flager i byen og på havnen} (pub.~1884).\\
\hspace{1em}\textit{Guds} & \textit{På guds veje} (pub.~1889). \\

\hangindent=1em \hangafter=1 Black & William Black (b.~1841) \\
\hspace{1em}\textit{Fortunatus} & \textit{The new Prince Fortunatus} (pub.~1890). Tauchnitz, 1890. \\
\hspace{1em}\textit{Phaeton} & \textit{The strange adventures of a phaeton} (pub.~1872). London, n.d. \\

\hangindent=1em \hangafter=1 Blicher & Steen Steensen Blicher (b.~1782) \\
\hspace{1em}\textit{Bindstouw} & \textit{E bindstouw: Fortællinger og digte i jydske mundarter} (pub.~1842). \\
\hspace{1em}\textit{Dagbog} & \textit{Brudstykker af en landsbydegns dagbog} (pub.~1824). \\ % PE: OJ refers to this (probably in a volume of Samlede noveller og Skizzer that I haven't seen) as "Blicher"
\hspace{1em}\textit{Høstferierne} & ``Høstferierne'' (pub.~1840). In \textit{Samlede noveller og skizzer}. Copenhagen, 1882. \\ % OJ refers to this as "Blicher"

\hangindent=1em \hangafter=1 Boethius, \textit{Orpheus} & ``The story of Orpheus.'' Within a 9th-century translation (attributed to King Alfred) of Boethius's \textit{De consolatione philosophiae}. \\

\hangindent=1em \hangafter=1 Bøgholm, \textit{Anglia} & Niels Bøgholm. Contribution to \textit{Anglia} n. f. 26. \\ % ??? Title of the article is still unknown. This is cited once, in chapter 7.
% Not to be confused with Bøgholm, \textit{Bacon og Shakespeare} & Niels Bøgholm. \textit{Bacon og Shakespeare: En sproglig sammenligning}. Copenhagen, 1906.\\ Now handled by BibTeX

\hangindent=1em \hangafter=1 Boswell & James Boswell (b.~1740) \\
\hspace{1em}\textit{Life}\textsubscript{A} & \textit{Life of Samuel Johnson, LL.D.} (pub.~1791).\\
\hspace{1em}\textit{Life}\textsubscript{B} & \textit{Life of Samuel Johnson, LL.D.} (pub.~1793 or later). Roger Ingpen (ed.). London, 1907.\\ % PE: One edition of Boswell's Life of SJ appears as a BibTeX reference; I haven't yet checked to see if this is the same as either of these two editions. Incidentally, the second is not a very satisfying reference (given the context); something published during Boswell's lifetime would be preferable. I've searched, but I haven't found.

\hangindent=1em \hangafter=1 Bradley, \textit{Tragedy} & A. C. Bradley (b.~1851). \textit{Shakespearean tragedy} (pub.~1904). London, 1904. \\

\hangindent=1em \hangafter=1 E. Brandes & Edvard Brandes (b.~1847)\\
\hspace{1em}\textit{Lykkens} & \textit{Lykkens blændværk fortælling}. 1898. \\

\hangindent=1em \hangafter=1 G. Brandes & Georg Brandes (b.~1842)\\
\hspace{1em}\textit{Napoleon} & ``Napoleon.'' \textit{Tilskueren} 1915. 32--60. \\ % Tilskueren is a journal.

\pagebreak
\hangindent=1em \hangafter=1 Brontë & Charlotte Brontë (b.~1816) \\
\hspace{1em}\textit{Jane} & \textit{Jane Eyre} (pub.~1847). London, 1847. \\
\hspace{1em}\textit{Professor} & \textit{The professor} (wr. 1846, pub. 1857). London, 1867. \\

\hangindent=1em \hangafter=1 Brooke, \textit{Poems} & Rupert Brooke (b.~1887). \textit{Poems}. London, 1911. \\ 

\hangindent=1em \hangafter=1 E. B. Browning & Elizabeth Barrett Browning (b.~1806)\\
\hspace{1em}\textit{Aurora} & \textit{Aurora Leigh} (pub.~1856). Tauchnitz, n.d. \\ % OJ refers to her as "Mrs. Browning"

\hangindent=1em \hangafter=1 R. Browning & Robert Browning (b.~1812) \\
\hspace{1em}\textit{Andrea} & ``Andrea del Sarto'' (pub.~1855). \\ 
\hspace{1em}\textit{Italian} & ``The Italian in England'' (pub.~1845). \\ 
\hspace{1em}\textit{Lippo} & ``Fra Lippo Lippi'' (pub.~1855). \\ 
\hspace{1em}\textit{Pompilia} & ``Pompilia.'' In \textit{The ring and the book} (pub.~1868--69). \\ 
\hspace{1em}\textit{Rabbi} & ``Rabbi Ben Ezra'' (pub.~1864). [St] \\
\hspace{1em}\textit{Ride} & ``The last ride together'' (pub.~1855). \\

\hangindent=1em \hangafter=1 Buchanan & Robert Buchanan (b.~1841)\\
\hspace{1em}\textit{Anthony} & \textit{Father Anthony} (pub.~1898). New York, 1900. \\ % MEG 7 says London, n.d.

\hangindent=1em \hangafter=1 Bunyan & John Bunyan (b.~1628) \\
\hspace{1em}\textit{Grace} & \textit{Grace abounding to the chief of sinners} (wr. 1660s--1672, pub. 1666). John Brown (ed.). Cambridge, 1907. \\
\hspace{1em}\textit{Progress} & \textit{The pilgrim's progress} (wr. 1660s--1677, pub. 1678). London, 1678. \\

\hangindent=1em \hangafter=1 Burns & Robert Burns (b.~1759) \\
\hspace{1em}\textit{Dogs} & ``The twa dogs'' (pub.~1786). \\
\hspace{1em}\textit{Man} & ``A man's a man for all that'' (pub.~1795). \\

\hangindent=1em \hangafter=1 Byron & George Byron (b.~1788) \\
\hspace{1em}\textit{Cain} & \textit{Cain: A mystery} (pub.~1821). [A.Sc] \\
\hspace{1em}\textit{Childe Harold} & \textit{Childe Harold's pilgrimage} (pub.~1812--18). [Ca] \\
\hspace{1em}\textit{Juan} & \textit{Don Juan} (pub.~1819--24). [Ca.St] \\
\hspace{1em}\textit{Manfred} & \textit{Manfred: A dramatic poem} (pub.~1817). [A.Sc] \\
\hspace{1em}\textit{Mazeppa} & \textit{Mazeppa} (pub.~1819). [St] \\
\hspace{1em}\textit{Sardana\-palus} & \textit{Sardanapalus: A tragedy} (pub.~1821). [A.Sc] \\

\hangindent=1em \hangafter=1 Caine & Hall Caine (b.~1853) \\
\hspace{1em}\textit{Christian} & \textit{The Christian} (pub.~1897). London, 1897.  \\
\hspace{1em}\textit{City} & \textit{The eternal city} (pub.~1901). London, 1901. \\
\hspace{1em}\textit{Manxman} & \textit{The Manxman} (pub.~1894). London, 1894. \\
\hspace{1em}\textit{Prodigal} & \textit{The prodigal son} (pub.~1904). London, 1904. \\

\hangindent=1em \hangafter=1 Calderón, \textit{Alcalde} & Pedro Calderón de la Barca (b.~1600). \textit{El alcalde de Zalamea} (perf. 1636). [A.Sc] \\

\hangindent=1em \hangafter=1 J. Carlyle, \textit{Letters} & Jane Carlyle (b.~1801). In James Anthony Froude (ed.), \textit{Letters and memorials of Jane Welsh Carlyle}. London, 1883. \\ % OJ refers to her/this as "Mrs. Carlyle F."

\hangindent=1em \hangafter=1 T. Carlyle & Thomas Carlyle (b.~1795) \\
\hspace{1em}\textit{Heroes} & \textit{On heroes, hero-worship, and the heroic in history} (pub.~1841). London, 1890. \\
\hspace{1em}\textit{Life} & In James Anthony Froude, \textit{Life of Carlyle}. (Vols 1 \& 2: \textit{Thomas Carlyle: A history of the first forty years of his life.} Vols 3 \& 4: \textit{Thomas Carlyle: A history of his life in London.}) London, 1882/1884. \\
\hspace{1em}\textit{Reminiscences} & Thomas Carlyle. \textit{Reminis\-cences} (wr. 1832--, pub. 1881). James Anthony Froude (ed.). London, 1881. \\
\hspace{1em}\textit{Revolution} & \textit{The French revolution} (pub.~1837). London. \\
\hspace{1em}\textit{Sartor} & \textit{Sartor resartus} (wr. 1831--, pub. 1833--34). London, n.d. \\

\hangindent=1em \hangafter=1 Carpenter & \\
\hspace{1em}\textit{Teaching}& George R. Carpenter, Franklin T. Baker, \& Fred N. Scott. \textit{The teaching of English in the elementary and the secondary school}. New York, 1913.\\

\hangindent=1em \hangafter=1 Cauer  & Paul Cauer (b.~1854)\\
\hspace{1em}\textit{Grammatica} & \textit{Grammatica militans: Erfahrungen und Wünsche im Gebiete des lateinischen und griechischen Unterrichtes}. Berlin, 1903. \\ % PE: This is also (properly) cited via BibTeX; but as it's cited as the source of an example or two I suppose that it has to be cited in both ways. 

\hangindent=1em \hangafter=1 Caxton, \textit{Reynard} & William Caxton (b. c. 1422) (trans.). \textit{Historye of Reynart the foxe} (pub.~1481). In Edward Arber (ed.), \textit{The history of Reynard the fox}. London, 1880.  \\

\hangindent=1em \hangafter=1 Chaucer & Geoffrey Chaucer (b. c. 1343) \\
\hspace{1em}\textit{Clerkes} & ``The clerkes tale'' (wr. 1387--1400). In Walter W. Skeat (ed.), \textit{The Canterbury tales: Text}. Oxford, 1894. [Gr.L] \\ % OJ refers to this as "E". ??? PE: How about adding "(\textit{CTT})" immediately after "Text}", and then replacing every instance below of "Walter W. Skeat (ed.), \textit{The Canterbury tales: Text}. Oxford, 1894" with "\textit{CTT}"? (If this seems good, I could do something similar for Rolland, for the Shelley poetry collection, for Sheridan's play collection and for Tennyson's Poetical works.)
\hspace{1em}\textit{Duchesse} & ``The book of the duchesse'' (wr. 1368--72). In Walter W. Skeat (ed.), \textit{The minor poems}. Oxford, 1896. [L] \\ % OJ refers to this as "Duch"
\hspace{1em}\textit{Lawe} & ``The tale of the man of lawe'' (wr. c. 1387). In Walter W. Skeat (ed.), \textit{The Canterbury tales: Text}. Oxford, 1894. [Gr.L] \\ % OJ refers to this as "B"
\hspace{1em}\textit{Melibeus} & ``The tale of Melibeus'' (wr. 1387--1400). In Walter W. Skeat (ed.), \textit{The Canterbury tales: Text}. Oxford, 1894. [Gr.L] \\ % OJ refers to this as "B"
\hspace{1em}\hangindent=1em \hangafter=1 \textit{Miller's prologue} & ``The miller's prologue'' (wr. 1387--1400). In Walter W. Skeat (ed.), \textit{The Canterbury tales: Text}. Oxford, 1894. [Gr.L] \\ % J refers to this as "A"
\hspace{1em}\textit{Pardoners} & ``The pardoners tale'' (wr. 1387--1400). In Walter W. Skeat (ed.), \textit{The Canterbury tales: Text}. Oxford, 1894. [Gr.L] \\ % OJ refers to this as "C"
\hspace{1em}\textit{Prologue} & ``The prologue'' (wr. 1387--1400). In Walter W. Skeat (ed.), \textit{The Canterbury tales: Text}. Oxford, 1894. [Gr.L] \\ % OJ refers to this as "A"

\hangindent=1em \hangafter=1 Christiansen & Einar Christiansen (b.~1861)\\
\hspace{1em}\textit{Fædreland} & \textit{Fædreland} (pub.~1910).\\ % ??? PE: Not yet seen; so this is a guessed interpretation of OJ's cryptic "Christiansen Fædrel."

\hangindent=1em \hangafter=1 Cicero, \textit{De oratore} & Marcus Tullius Cicero. \textit{De oratore} (wr. 55 BCE). \\

\hangindent=1em \hangafter=1 Coleridge, letter & Samuel Taylor Coleridge (b.~1772). Letter of 21 April 1800. \\

\hangindent=1em \hangafter=1 Collitz, \textit{Präteritum} & Hermann Collitz. \textit{Das schwache Präteritum und seine Vorgeschichte}. Göttingen, 1912.\\

\hangindent=1em \hangafter=1 Congreve & William Congreve (b.~1670) \\
\hspace{1em}\textit{Dealer} & \textit{The double-dealer} (perf. 1693). In \textit{Works}, 6th edn., vol.~1. The Hague, 1753.\\
\hspace{1em}\textit{Love} & \textit{Love for love} (perf. 1695). \\

\hangindent=1em \hangafter=1 Conway, \textit{Called} & Hugh Conway (b.~1847). \textit{Called back} (pub.~1883). Tauchnitz, 1884.\\ 

\hangindent=1em \hangafter=1 Cowper & William Cowper (b.~1731) \\
\hspace{1em}letter & Letter. In J.~G. Frazer (ed.), \textit{Letters}. London, 1912.\\ % More than one are referred to; each reference is dated
\hspace{1em}\textit{Task} & \textit{The task} (pub.~1785). In \textit{Poetical works}. London, 1889. \\

\hangindent=1em \hangafter=1 Darwin & Charles Darwin (b.~1809) \\
\hspace{1em}\textit{Expression} & \textit{The expression of the emotions in man and animals} (pub.~1872). London, 1872. \\
\hspace{1em}\textit{Life} & In Francis Darwin (ed.), \textit{The life and letters of Charles Darwin}. London, 1887.  \\

\hangindent=1em \hangafter=1 Daudet & Alphonse Daudet (b.~1840) \\
\hspace{1em}\textit{Numa} & \textit{Numa Roumestan} (pub.~1881). \\
\hspace{1em}\textit{Sapho} & \textit{Sapho} (pub.~1881). Paris, 1884. \\
\hspace{1em}\textit{Tartarin} & \textit{Tartarin sur les Alpes} (pub.~1885). \\

\hangindent=1em \hangafter=1 Defoe & Daniel Defoe (b. c. 1660) \\
\hspace{1em}\textit{Farther} & \textit{The farther adventures of Robinson Crusoe} (pub.~1719). London, 1719. \\
\hspace{1em}\textit{Gentleman} & \textit{The compleat English gentleman} (wr. 1720s, pub. 1890). Karl D. Bülbring (ed.). London, 1890. \\ 
\hspace{1em}\textit{Journal} & \textit{A journal of the plague year} (pub.~1722). E.~W. Brayley (ed.). London, n.d. \\
\hspace{1em}\textit{Robinson} & \textit{Robinson Crusoe} (pub.~1719). Facsimile edn. London, 1883. \\

\hangindent=1em \hangafter=1 Dekker, \textit{Sinnes} & Thomas Dekker (b. c. 1572). \textit{The seuen deadly sinnes of London} (pub.~1606). Edward Arber (ed.). London, 1879. \\ % "Dekker" is how the surname is I think more commonly spelt (for this writer) and how OJ spells it. But "Decker" is how Arber spells it.

\hangindent=1em \hangafter=1 Delbrück, \textit{Syntax} & Berthold Delbrück. \textit{Germanische Syntax} I: \textit{Zu den negativen Sätzen}. Leipzig, 1911. \\

\hangindent=1em \hangafter=1 Deutschbein & Max Deutschbein\\
\hspace{1em}\textit{System} &  \textit{System der neuenglischen Syntax}. Cöthen, 1917. \\

\hangindent=1em \hangafter=1 \textit{Devill} & \textit{The merry devill of Edmonton} (perf. c. 1602, pub. 1608). In Charles Mills Gayley (ed.), \textit{Representative English comedies}, vol.~2, \textit{Later contemporaries of Shakespeare}. New York, 1913. \\ % OJ refers to this as "Devil E" or "Devil Edm.". LCS is vol 2 of the 3-vol REC.

\hangindent=1em \hangafter=1 Dewey, \textit{School} & John Dewey (b.~1859). \textit{The school and society} (pub.~1899). \\

\hangindent=1em \hangafter=1 Dickens & Charles Dickens (b.~1812). \\
\hspace{1em}\textit{Carol} & \textit{A Christmas carol} (pub.~1843). In \textit{Christmas books}. London, 1892. \\ % OJ refers to this as "X"
\hspace{1em}\textit{Chimes} & \textit{The chimes} (pub.~1844). In \textit{Christmas books}. London, 1892. \\ % OJ refers to this as "X"
\hspace{1em}\textit{Cricket} & \textit{The cricket on the hearth} (pub.~1845). In \textit{Christmas books}. London, 1892. \\
\hspace{1em}\textit{David} & \textit{David Copperfield} (pub.~1849--50). London, 1897. \\
\hspace{1em}\textit{Dombey} & \textit{Dombey and son} (pub.~1848). London, 1887. \\
\hspace{1em}\textit{Friend} & \textit{Our mutual friend} (pub.~1865). London, 1912. \\
\hspace{1em}\textit{Martin} & \textit{The life and adventures of Martin Chuzzlewit} (pub.~1843). London, n.d. \\
\hspace{1em}\textit{Nicholas} & \textit{The life and adventures of Nicholas Nickleby} (pub.~1839). London, 1900. \\

\hangindent=1em \hangafter=1 Dickinson & G. Lowes Dickinson (b.~1862) \\
\hspace{1em}\textit{Letters} & \textit{Letters from John Chinaman} (pub.~1901). London, 1904. \\
\hspace{1em}\textit{Symposium} & \textit{A modern symposium} (pub.~1905). London, 1906. \\
\hspace{1em}\textit{War} & \textit{After the war} (pub.~1915). \\

\hangindent=1em \hangafter=1 Disraeli, \textit{Lothair} & Benjamin Disraeli (b.~1804). \textit{Lothair} (pub.~1870). London, n.d. \\ % OJ refers to him/this as "Beaconsfield L."; I (PE) think that as an author he's more commonly known as Disraeli

\hangindent=1em \hangafter=1 Dobson, \textit{Fielding} & Austin Dobson (b.~1840). \textit{Fielding} (pub.~1883). London, 1889.  \\ % In the example I (PE) looked at and link to (which isn't the same edition as the one OJ quoted: their page numbers are different), the title page says, simply, "Fielding", but the front cover says "Henry Fielding".

\hangindent=1em \hangafter=1 Doyle & Arthur Conan Doyle (b.~1859) \\
\hspace{1em}\textit{Hound} & \textit{The hound of the Baskervilles} (pub.~1901). Tauchnitz, 1902. \\
\hspace{1em}\textit{How} & ``How the King held the brigadier.'' \textit{The exploits of Brigadier Gerard}. \textit{The Strand Magazine} 9 (1895). 501--514. \\ % OJ refers to this as "Doyle NP. 1895"
\hspace{1em}\textit{Letters} & \textit{The Stark Munro letters} (pub.~1895). Tauchnitz, 1895.  \\
\hspace{1em}\textit{Memoirs} & \textit{The memoirs of Sherlock Holmes} (pub.~1894). Tauchnitz. \\ % OJ refers to this as "S. 4"
\hspace{1em}\textit{Return} & \textit{The return of Sherlock Holmes} (pub.~1905). Tauchnitz. \\ % OJ refers to this as "S. 5"

\hangindent=1em \hangafter=1 Drachmann & Holger Drachmann (b.~1846) \\
\hspace{1em}\textit{Forskrevet} & \textit{Forskrevet} (pub.~1890). \\
\hspace{1em}\textit{Kitzwalde} & \textit{Kitzwalde: En lille munter ridderroman} (pub.~1895). \\

\hangindent=1em \hangafter=1 Droz, \textit{Monsieur} & Gustave Droz (b.~1832). \textit{Monsieur, madame et bébé} (pub.~1866). Paris, 1882.\\

\hangindent=1em \hangafter=1 Dryden & John Dryden (b.~1631) \\
\hspace{1em}\textit{All} & \textit{All for love} (perf. 1677). [A] \\
\hspace{1em}\textit{Aureng-Zebe} & \textit{Aureng-Zebe} (perf. 1675, pub. 1676). [A] \\

\hangindent=1em \hangafter=1 \textit{Eastward} & George Chapman, Ben Jonson \& John Marston. \textit{Eastward hoe} (perf. 1605, pub. 1605). In Charles Mills Gayley (ed.), \textit{Representative English comedies}, vol.~2, \textit{The later contemporaries of Shakespeare}. New York, 1913. \\ % LCS is vol 2 of the 3-vol REC.

\hangindent=1em \hangafter=1 Egerton, \textit{Keynotes} & George Egerton (b.~1859). \textit{Keynotes} (pub.~1893). London, 1893. \\ 

\hangindent=1em \hangafter=1 Eliot & George Eliot (b.~1819) \\
\hspace{1em}\textit{Adam} & \textit{Adam Bede} (pub.~1859). London, 1900. \\
\hspace{1em}\textit{Mill} & \textit{The mill on the floss} (pub.~1860). Tauchnitz. \\
\hspace{1em}\textit{Silas} & \textit{Silas Marner} (pub.~1861). Tauchnitz. \\


\hangindent=1em \hangafter=1 Ellis, \textit{Address} & Alexander J. Ellis (b.~1814) \textit{First annual address of the president to the Philological Society} (pub.~1874). In \textit{Transactions of the Philological Society}, 1--34.\\


\hangindent=1em \hangafter=1 Emerson, \textit{Traits} & Ralph Waldo Emerson (b.~1803). \textit{English traits} (pub.~1856). \\
% \textit{English Dialect Dictionary} & Joseph Wright. \textit{The English Dialect Dictionary}. London, 1898--1905. \\ Converted to BibTeX (under "Wright")

\hangindent=1em \hangafter=1 Faber & Peter Faber (b.~1810)\\
\hspace{1em}\textit{Stegekjælderen} &  ``Stegekjælderen eller Den fine verden'' (pub.~1849). Copenhagen, 1883. \\ % This is an individual song

\hangindent=1em \hangafter=1 Farmer \& Henley & John S. Farmer \& W. E. Henley. \textit{Slang and its analogues past and present}. 1890–1904. \\ % PE: Currently cited in both ways; I suppose that this must continue.

\hangindent=1em \hangafter=1 Farquhar & George Farquhar (b. c. 1677)\\
\hspace{1em}\textit{Stratagem} & \textit{The beaux' stratagem} (perf. 1707). In \textit{Restoration plays}. 1912.  \\

\hangindent=1em \hangafter=1 Feilberg, \textit{Snaps} & Henning Frederik Feilberg (b.~1831). ``Den fattige mands snaps.'' \textit{Dania} 5 (1898) 88--123. \\ % A different work by Feilberg is cited via BibTeX

\hangindent=1em \hangafter=1 Fibiger, \textit{Liv} & Johannes Fibiger (b.~1867). \textit{Mit liv og levned, som jeg selv har forstaaet det}. Karl Gjellerup (ed.). Copenhagen, 1898. \\

\hangindent=1em \hangafter=1 Fielding & Henry Fielding (b.~1707)\\
\hspace{1em}\textit{Jonathan} & \textit{The history of the life of the late Mr. Jonathan Wild the great} (pub.~1743). \\ % ??? Is this the right form of the title?
\hspace{1em}\textit{Joseph} & \textit{Joseph Andrews} (pub.~1742). \\
\hspace{1em}\textit{Tom} & \textit{The history of Tom Jones, a foundling} (pub.~1749). London, 1782. \\
\hspace{1em}\textit{Tragedy} & \textit{The tragedy of tragedies} (pub.~1731). \\

\hangindent=1em \hangafter=1 Franklin & \\
\hspace{1em} \textit{Autobiography} & \textit{The autobiography} (wr. 1771–1790). In William MacDonald (ed.), \textit{The autobiography of Benjamin Franklin}. London, 1905. \\

\hangindent=1em \hangafter=1 Friis, \textit{Politiken} & Aage Friis (b.~1870). In \textit{Politiken}, 1906. \\ % PE: "Aage Friis Politiken 6. 2. 06" is what OJ writes (in chapter 6). My guess is that this is Aage Friis, writing in either the 6 Feb '06 or the 2 June '06 issue of the newspaper Politiken

\hangindent=1em \hangafter=1 Galsworthy & John Galsworthy (b.~1867)\\
\hspace{1em}\textit{Box} & \textit{The silver box: A comedy in three acts} (perf. 1906). London, 1910. \\
\hspace{1em}\textit{Flower} & \textit{The dark flower} (pub.~1913). Tauchnitz, 1913.  \\
\hspace{1em}\textit{Freelands} & \textit{The Freelands} (pub.~1915). London, 1916. \\
\hspace{1em}\textit{Joy} & \textit{Joy: A play on the letter ``I''} (perf. 1907). [A] \\ % OJ refers to this as "P 2" 
\hspace{1em}\textit{Justice} & \textit{Justice: A tragedy in four acts} (perf. 1910). \\ % OJ refers to this as "P 6" 
\hspace{1em}\textit{Man} & \textit{The man of property} (pub.~1906). \\
\hspace{1em}\textit{Motley} & \textit{A motley} (pub.~1910). Tauchnitz, 1910. \\
\hspace{1em}\textit{Strife} & \textit{Strife: A drama in three acts} (perf. 1909). \\ % OJ refers to this as "P 3" 

\hangindent=1em \hangafter=1 \textit{Gammer} & \textit{Gammer Gurtons nedle} (pub.~1575). In John Matthews Manly (ed.), \textit{Specimens of the pre-Shaksperean drama}. Boston, 1897. \\ % Yes, "Shaksperean" is how it's spelt.

\hangindent=1em \hangafter=1 Garborg & Arne Garborg (b.~1851) \\
\hspace{1em}\textit{Bondestudentar} & \textit{Bondestudentar} (pub.~1883). Bergen, 1883.\\

\hangindent=1em \hangafter=1 Giellerup & Karl Gjellerup (b.~1857) \\
\hspace{1em}\textit{Minna} & \textit{Minna} (pub.~1889). Copenhagen, 1889. \\ % OJ calls him Giellerup; everyone else seems to call him Gjellerup. Even this 1889 edition of this book calls him Gjellerup.
\hspace{1em}\textit{Romulus} & \textit{Romulus} (pub.~1889). \\ % ??? PE: 1884 in German and Swedish but 1889 in Danish? (So says Worldcat.)

\hangindent=1em \hangafter=1 Gilbert, \textit{Charity} & W. S. Gilbert (b.~1836). \textit{Charity} (perf. 1874). In \textit{Original plays}. London, 1884. \\

\hangindent=1em \hangafter=1 Gissing & George Gissing (b.~1857) \\
\hspace{1em}\textit{Born} & \textit{Born in exile} (pub.~1892). London, 1892. \\
\hspace{1em}\textit{Grub} & \textit{New Grub Street} (pub.~1891). London, 1908. \\
\hspace{1em}\textit{Henry} & \textit{The private papers of Henry Ryecroft} (pub.~1903). London, 1912. \\

\hangindent=1em \hangafter=1 Goethe & Johann Wolfgang von Goethe (b.~1749) \\
\hspace{1em}\textit{Faust}~I & \textit{Faust. Der Tragödie erster Teil} (pub.~1808). \\
\hspace{1em}\textit{Lehrjahre} & \textit{Wilhelm Meisters Lehrjahre} (pub.~1795–96). \\
\hspace{1em}letter & Letter (21/5936) to Behrendt, 21 March 1810. \\ % Must the number be explained?

\hangindent=1em \hangafter=1 Goldschmidt & Meïr Aron Goldschmidt (b.~1819) \\
\hspace{1em}\textit{Hjemløs} & \textit{Hjemløs} (pub.~1853). \\ %\hspace{1em}Kol. 92 & [unidentified] \\ % ??? In chapter 4 (p.35 of the printed book); intentionally DELETED from this new edition as being an unspecified example in a not-yet-identified source
\hspace{1em}\textit{Levi} & ``Levi og Ibald.'' \textit{M. Goldschmidts poetiske skrifter, udgivene af hans søn}. vol. 8. Copenhagen, 1898. \\ 
\hspace{1em}\textit{Ravnen} & \textit{Ravnen}. Copenhagen, 1867. \\ % OJ refers to both this edition and a different one. Since I (PE) can't identify the other but can identify (and link to) this one, I've standardized on this one.

\hangindent=1em \hangafter=1 Goldsmith & Oliver Goldsmith (b.~1728) \\
\hspace{1em}\textit{Citizen} & \textit{The citizen of the world} (pub.~1760--62). [Let] \\
\hspace{1em}\textit{Good-natur'd} & \textit{The good-natur'd man} (perf. 1768). \\
\hspace{1em}\textit{Stoops} & \textit{She stoops to conquer} (perf. 1773). \\
\hspace{1em}\textit{Vicar} & \textit{The vicar of Wakefield} (wr. 1761--62, pub. 1766). Facsimile edn. London, 1885. \\

\hangindent=1em \hangafter=1 Goncourt, \textit{Germinie} & Edmond de Goncourt\\&(b.~1822) \& Jules de Goncourt (b.~1830). \textit{Germinie Lacerteux}. Paris, 1864. \\ % fr:Wikipedia says that "date de parution" of this novel was 1865, but the title page says 1864

\hangindent=1em \hangafter=1 Gosse, \textit{History} & Edmund Gosse (b.~1849). \textit{A short history of modern English literature} (pub.~1897). London, 1908. \\

\hangindent=1em \hangafter=1 Gravlund, \textit{Kristrup} & Thorkild Gravlund (b.~1879). ``Kristrup ved Randers.'' \textit{Danske Studier} (1909). 85--103. \\ % There's no volume number. OJ refers to this as "Da. Studier 1909" (and doesn't name the author)

\pagebreak
\hangindent=1em \hangafter=1 Grundtvig & Svend Grundtvig (b.~1824)\\
\hspace{1em}\textit{Folkeæventyr} & \textit{Danske folkeæventyr} (pub.~1876–83). \\

\hangindent=1em \hangafter=1 \textit{Gu.} & [unidentified] \\ % ??? PE: UNIDENTIFIED Occurs only once, in chapter 10 (penultimate line of p.108 of the printed book); see my comments in 10.tex on "Guthlac" (my only guess so far about what this is short for)

\hangindent=1em \hangafter=1 Halévy, \textit{Notes} & Ludovic Halévy (b.~1834). \textit{Notes et souvenirs 1871--1872} (pub.~1889). Paris, 1889. \\

\hangindent=1em \hangafter=1 Hallam, letter & Henry Hallam (b.~1777). Letter (wr. 1845) to Alfred Tennyson, in Hallam Tennyson, \textit{Alfred Lord Tennyson: A memoir by his son}. \\ % OJ refers to this as "Tennyson L"

\hangindent=1em \hangafter=1 Hardy & Thomas Hardy (b.~1840) \\
\hspace{1em}\textit{Far} & \textit{Far from the madding crowd} (pub.~1874). London, 1906. \\
\hspace{1em}\textit{Ironies} & \textit{Life's little ironies} (pub.~1894). London, 1908. \\
\hspace{1em}\textit{Return} & \textit{The return of the native} (pub.~1878). London, 1912. \\
\hspace{1em}\textit{Tess} &\textit{Tess of the d'Urbervilles} (pub.~1891). London, 1892. \\
\hspace{1em}\textit{Wessex} & \textit{Wessex tales} (pub.~1888).  London, 1889. \\

\hangindent=1em \hangafter=1 Harraden & Beatrice Harraden (b.~1864) \\
\hspace{1em}\textit{Fowler} & \textit{The fowler} (pub.~1899). London, 1899.  \\
\hspace{1em}\textit{Ships} & \textit{Ships that pass in the night} (pub.~1893). London.  \\

\hangindent=1em \hangafter=1 Harrison & Frederic Harrison (b.~1831) \\
\hangindent=1em \hangafter=1 \hspace{1em}on Mark Pattison & [unidentified] \\ % ??? UNIDENTIFIED
\hspace{1em}\textit{Ruskin} & \textit{John Ruskin} (pub.~1902). London, 1902. \\

\hangindent=1em \hangafter=1 Hawthorne & Nathaniel Hawthorne (b.~1804) \\
\hspace{1em}\textit{Image} & \textit{The snow image and other twice-told tales} (pub.~1851). New York, n.d. \\
\hspace{1em}\textit{Wonder} & \textit{A wonder book for girls and boys} (pub.~1851). \\ % OJ calls this “T”, misattributing the sole example to Tanglewood Tales

\hangindent=1em \hangafter=1 Hay, \textit{Breadwinners} & John Hay (b.~1838). \textit{The breadwinners} (pub.~1883). Tauchnitz, 1883. \\ % Also titled "The Bread-winners" (with hyphen).

\hangindent=1em \hangafter=1 Hazlitt, \textit{Liber} & William Hazlitt (b.~1778). \textit{Liber amoris} (pub.~1823). London, 1823. \\

\hangindent=1em \hangafter=1 Henley, \textit{Beau} & W. E. Henley (b.~1849) \& Robert Louis Stevenson (b.~1850). \textit{Beau Austin} (perf. 1884). London, 1897. \\

\hangindent=1em \hangafter=1 Henley, \textit{Burns} & William E. Henley (b.~1849) \& Thomas F. Henderson (b.~1844). \textit{The poetry of Robert Burns} (pub.~1897). Edinburgh, 1897. \\

\hangindent=1em \hangafter=1 Henrichsen & Erik Henrichsen (b.~1865)\\
\hspace{1em}\textit{Mændene} & \textit{Mændene fra forfatningskampen} (pub.~1913). \\

\hangindent=1em \hangafter=1 Herrick, \textit{Memoirs} & Robert Herrick (b.~1868). \textit{The memoirs of an American citizen} (pub.~1905). New York, 1905. \\

\hangindent=1em \hangafter=1 Hewlett, \textit{Quair} & Maurice Hewlett (b.~1861). \textit{The Queen's quair} (pub.~1904). London, 1904. \\

\hangindent=1em \hangafter=1 Høffding, \textit{Humor} & Harald Høffding (b.~1843). \textit{Den store humor} (pub.~1916). Copenhagen, 1916. \\

\hangindent=1em \hangafter=1 Holberg & Ludvig Holberg (b.~1684) \\
\hspace{1em}\textit{Erasmus} & \textit{Erasmus Montanus} (wr. 1722, pub.  1731, perf. 1747). [A.Sc] \\ % Leave "Montanus" capitalized
\hspace{1em}\textit{Jeppe} & \textit{Jeppe paa bierget} (pub.~1722). [A.Sc] \\
\hspace{1em}\textit{Kande\-støber} & \textit{Den politiske kandestøber} (pub.~1722). [A.Sc] \\
\hspace{1em}\textit{Mascarade} & \textit{Mascarade} (pub.~1724). [A.Sc] \\
\hspace{1em}\textit{Peder} & \textit{Peder Paars} (pub.~1720). [Bk.Ca] \\
\hspace{1em}\textit{Pulver} & \textit{Det arabiske pulver} (pub.~1724). [Sc] \\
\hspace{1em}\textit{Ulysses} & \textit{Ulysses von Ithacia} (pub.~1723). [A.Sc] \\
% Homer, \textit{Odyssey} & Homer. \textit{Odyssey}.  In chapter 1, OJ refers to this as Homer's Odyssey; in chapter 7, he refers to it with no mention of Homer. This list of sources places it not here but instead under "Odyssey".

\hangindent=1em \hangafter=1 Hope & Anthony Hope (b.~1863) \\
\hspace{1em}\textit{Change} & \textit{A change of air} (pub.~1893). Tauchnitz, 1893. \\
\hspace{1em}\textit{Courtship} & \textit{Comedies of courtship} (pub.~1896). \\
\hspace{1em}\textit{Dialogues} & \textit{The Dolly dialogues} (pub.~1894). London, 1894. \\ % Dolly is a character's name; leave capitalized
\hspace{1em}\textit{Father} & \textit{Father Stafford} (pub.~1891). London, 1900. \\
\hspace{1em}\textit{Intrusions} & \textit{The intrusions of Peggy} (pub.~1902). London, 1907. \\
\hspace{1em}\textit{Man} & \textit{A man of mark} (pub.~1890). London. \\
\hspace{1em}\textit{Quisanté} & \textit{Quisanté} (pub.~1900). London. \\
\hspace{1em}\textit{Rupert} & \textit{Rupert of Hentzau} (pub.~1898). Tauchnitz, 1898. \\
\hspace{1em}\textit{Zenda} & \textit{The prisoner of Zenda} (pub.~1894). London, 1894. \\

\hangindent=1em \hangafter=1 Horace & Quintus Horatius Flaccus \\
\hspace{1em}\textit{Epistola} & ``Epistola I. ad Mæcenatem''. \textit{Epistolarum liber primus} (wr. c.~20 BCE).\\
\hspace{1em}\textit{Odes} & \textit{Carmina} (wr. 23 BCE).\\

\hangindent=1em \hangafter=1 Hørup & Viggo Hørup (b.~1841). \textit{V. Hørup i skrift og tale, udvalgte artikler og taler}. Copenhagen, 1902--1905. \\

\hangindent=1em \hangafter=1 Hostrup, \textit{Gjenboerne} & Jens Christian Hostrup (b.~1818). \textit{Gjenboerne} (pub.~1844). [A.Sc] \\

\hangindent=1em \hangafter=1 Housman, \textit{John} & Laurence Housman (b.~1865). \textit{John of Jingalo} (pub.~1912). London, 1912. \\

\hangindent=1em \hangafter=1 Howells, \textit{Rise} & William Dean Howells (b.~1837). \textit{The rise of Silas Lapham} (pub.~1885). Tauchnitz. \\

\pagebreak
\hangindent=1em \hangafter=1 Hughes & Thomas Hughes (b.~1822)\\
\hspace{1em}\textit{Days} & \textit{Tom Brown's school days} (pub.~1861). London, 1886. \\ % OJ refers to this as "T. 1"
\hspace{1em}\textit{Oxford} & \textit{Tom Brown at Oxford} (pub.~1856). London, 1886. \\ % OJ refers to this as "T. 2"

\hangindent=1em \hangafter=1 Huxley, letter & Thomas Henry Huxley (b.~1825). Letter (wr. 1854), in Leonard Huxley. \textit{Life and letters of Thomas Henry Huxley} (London, 1900), 1.118.\\

\hangindent=1em \hangafter=1 Ibsen & Henrik Ibsen (b.~1828) \\
\hspace{1em}\textit{Inger} & \textit{Fru Inger til Østråt} (perf. 1855). Copenhagen, 1874. \\
\hspace{1em}\textit{Når} & \textit{Når vi døde vågner} (pub.~1899). Copenhagen, 1899. \\
\hspace{1em}\textit{Peer} & \textit{Peer Gynt} (pub.~1867, perf. 1876). Copenhagen, 1891. \\
\hspace{1em}\textit{Solness} & \textit{Bygmester Solness} (pub.~1892, perf. 1893). Copenhagen, 1892. \\
\hspace{1em}\textit{Vildanden} & \textit{Vildanden} (pub.~1884, perf. 1885). Copenhagen, 1884. \\

\hangindent=1em \hangafter=1 Jacobs, \textit{Lady} & W. W. Jacobs (b.~1863). \textit{The lady of the barge} (pub.~1902). London, 1902. \\

\hangindent=1em \hangafter=1 Jacobsen & J. P. Jacobsen (b.~1847) \\
\hspace{1em}\textit{Fønss} & ``Fru Fønss'' (pub.~1882). \\
\hspace{1em}\textit{Niels} & \textit{Niels Lyhne} (pub.~1880). \\

\hangindent=1em \hangafter=1 James & Henry James (b.~1843) \\
\hspace{1em}\textit{American} & \textit{The American} (pub.~1877). Tauchnitz. \\
\hspace{1em}\textit{Side} & \textit{The soft side} (pub.~1900). London, 1900.  \\

\hangindent=1em \hangafter=1 Jensen, \textit{Bræen} & Johannes V. Jensen (b.~1873). \textit{Bræen} (pub.~1908). Copenhagen, 1908.\\

\hangindent=1em \hangafter=1 Jerrold, \textit{Lectures} & Douglas Jerrold (b.~1803). \textit{Mrs Caudle's curtain lectures} (pub.~1846). London. \\

\hangindent=1em \hangafter=1 Johannsen & W. Johannsen\\
\hspace{1em}\textit{Salmonsen} & In \textit{Salmonsens konversationsleksikon}.\\ % ??? PE: This depends on a fair amount of my guesswork. In chapter 6, OJ attributes something to "W. Johannsen Salmonsen 9. 184". I take that to mean wrote in in volume 9 of Salmonsens konversationsleksikon (a major encyclopedia). But volume 9 of the 2nd edition was only published in 1920 (says https://runeberg.org/salmonsen/2 ), and I haven't yet been able to consult volume 9 of the first.

\hangindent=1em \hangafter=1 Johnson, \textit{Rasselas} & Samuel Johnson (b.~1709). \textit{History of Rasselas, Prince of Abissinia} (pub.~1759). Oxford, 1887.  \\

\hangindent=1em \hangafter=1 Jonson & Ben Jonson (b.~1572) \\
\hspace{1em}\textit{Epicœne} & \textit{Epicœne} (perf. 1609/10). [A.L] \\ % Quoted by OJ (from "B. Jo. 3") with original spelling
\hspace{1em}\textit{Humour} & \textit{Every man in his humour} (perf. 1598). [A.Sc] \\ % Quoted by OJ (from "B. Jo. 1") with modern spelling

\hangindent=1em \hangafter=1 Juel-Hansen & Erna Juel-Hansen (b.~1845)\\
\hspace{1em} \textit{Historie} & \textit{En ung dames historie} (pub.~1888). Copenhagen, 1888. \\

\hangindent=1em \hangafter=1 Kielland & Alexander Kielland (b.~1849)\\
\hspace{1em}\textit{Fortuna} & \textit{Fortuna} (pub.~1884). Copenhagen, 1884.\\
\hspace{1em}\textit{Jacob} & \textit{Jacob} (pub.~1891). \\

\hangindent=1em \hangafter=1 Kierkegaard & Søren Kierkegaard (b.~1813) \\
\hspace{1em}\textit{Af\-handlinger} & \textit{Tvende ethisk-religieuse smaa-afhandlinger} (pub.~1849). \\
\hspace{1em}\textit{Øieblikket} & \textit{Øieblikket Nr. 1–9} (pub.~1855). 3rd edn. Copenhagen, 1895. \\
\hspace{1em}\textit{Stadier} & \textit{Stadier paa livets vei} (pub.~1845). \\

\hangindent=1em \hangafter=1 Kinglake, \textit{Eothen} & A. W. Kinglake (b.~1809). \textit{Eothen} (pub.~1844). D.~G. Hogarth \& Vere H. Collins (eds.). Oxford, 1914. \\

\hangindent=1em \hangafter=1 Kingsley & Charles Kingsley (b.~1819) \\
\hspace{1em}\textit{Hypatia} & \textit{Hypatia} (pub.~1853). London, n.d. \\
\hspace{1em}\textit{Yeast} & \textit{Yeast: A problem} (pub.~1848). \\

\hangindent=1em \hangafter=1 Kipling & Rudyard Kipling (b.~1865) \\
\hspace{1em}\textit{Ballads} & \textit{Barrack-room ballads} (pub.~1890). 1892. \\
\hspace{1em}\textit{Jungle} & \textit{The jungle book} (pub.~1894). 1897. \\
\hspace{1em}\textit{Kim} & \textit{Kim} (pub.~1901). 1908. \\
\hspace{1em}\textit{Light} & \textit{The light that failed} (pub.~1890). \\
\hspace{1em}\textit{Seas} & \textit{The seven seas} (pub.~1896). \\
\hspace{1em}\textit{Second} & \textit{The second jungle book} (pub.~1895). Tauchnitz, 1897. \\
\hspace{1em}\textit{Stalky} & \textit{Stalky \& Co} (pub.~1899). Tauchnitz.  \\

\hangindent=1em \hangafter=1 Knudsen, \textit{Urup} & Jakob Knudsen (b.~1858). \textit{Lærer Urup} (pub.~1909). \\

\hangindent=1em \hangafter=1 Kyd, \textit{Spanish} & Thomas Kyd (b.~1558). \textit{The Spanish tragedie} (wr. 1580s). In Frederick Boas (ed.), \textit{The works of Thomas Kyd}. London, 1901. [A.Sc.L] \\

\hangindent=1em \hangafter=1 La Bruyère & Jean de La Bruyère (b.~1645)\\
\hspace{1em}\textit{Caractères} & \textit{Les Caractères ou Les Mœurs de ce siècle} (pub.~1688). \\

\hangindent=1em \hangafter=1 Lagerlöf, \textit{Saga} & Selma Lagerlöf (b.~1858). \textit{Gösta Berlings saga} (pub.~1891). \\

\hangindent=1em \hangafter=1 Larsen, \textit{Punkt} & Karl Larsen (b.~1860). \textit{Det springende punkt} (pub.~1911). Copenhagen, 1911.\\

\hangindent=1em \hangafter=1 \textit{Lave og Jon} & ``Lave og Jon.'' No. 390 in Svend Grundtvig \& Axel Olrik (eds.), \textit{Danmarks gamle folkeviser}. Copenhagen, 1899--1904. 7.52--65. \\

\hangindent=1em \hangafter=1 Lawrence, \textit{Abolition} & C. E. Lawrence. ``The abolition of death.'' \textit{Fortnightly Review} 101 (no. 602, 1917). 326--331. \\

\hangindent=1em \hangafter=1 Leonora Christina & Leonora Christina Ulfeldt (b.~1621). \textit{Jammers minde} (wr. 1674, pub. 1869). \\

\pagebreak
\hangindent=1em \hangafter=1 Lessing & Gotthold Ephraim Lessing (b.~1729) \\
\hspace{1em}\textit{Emilia} & \textit{Emilia Galotti} (perf. 1772). [A.Sc]\\
\hspace{1em}\textit{Nathan} & \textit{Nathan der weise} (pub.~1779, perf. 1783). [A.Sc] \\

\hangindent=1em \hangafter=1 Lie, \textit{Sol} & Jonas Lie (b.~1833). \textit{Naar sol gaar ned}. Copenhagen, 1895.\\

\hangindent=1em \hangafter=1 Lindsey, \textit{Beast} & Ben B. Lindsey (b.~1869). ``The beast and the jungle.'' \textit{Everybody's Magazine} 21 (1909), 779. \\ % As-yet unidentified issue, some time from September to December of 1909. Also in chapter 8 (``At work with the children'') of Ben B. Lindsey \& Harvey J. O'Higgins, \textit{The Beast} (New York, 1910).

\hangindent=1em \hangafter=1 Lippmann, \textit{Speaks} & [Walter Lippmann (b.~1889)]. ``America speaks.'' \textit{The New Republic}, 27 January 1917. 340--342. \\

\hangindent=1em \hangafter=1 Locke & William J. Locke (b.~1863) \\
\hspace{1em}\textit{Adventure} & ``The adventure of the kind Mr. Smith'' (pub.~1911). \\
\hspace{1em}\textit{Morals} & \textit{The morals of Marcus Ordeyne} (pub.~1905). New York, 1907. \\ % ??? Or London, 1906?
\hspace{1em}\textit{Septimus} & \textit{Septimus} (pub.~1909). London, 1916. \\
\hspace{1em}\textit{Vagabond} & \textit{The belovéd vagabond} (pub.~1906). London, 1906. \\ % Yes, not a grave but an acute accent.
\hspace{1em}\textit{Year} & \textit{The wonderful year} (pub.~1916). London, 1916. \\

\hangindent=1em \hangafter=1 \textit{Lokasenna} & \textit{Lokasenna}. Old Norse narrative poem, known from a 13th-century manuscript. \\

\hangindent=1em \hangafter=1 London & Jack London (b.~1876) \\
\hspace{1em}\textit{Martin} & \textit{Martin Eden} (pub.~1909). London, 1915. \\
\hspace{1em}\textit{Valley} & \textit{The valley of the Moon} (pub.~1913). London, 1914. \\

\hangindent=1em \hangafter=1 Luther, \textit{Bücher} & Martin Luther (b.~1483). \textit{Alle Bücher und Schrifften: Vom XXII Jar an, bis auff den christlichen vnd seligen Abschied aus diesem Leben des hochlöblichen Herrn Friderichen, Hertzogen vnd Churfürst. zu Sachssen, im Meien des XXV. Jars}. Jena, 1558. \\ % PE: Surely this title can and should be abridged, but I don't know how. 

\hangindent=1em \hangafter=1 Macaulay & Thomas Babington Macaulay (b.~1800) \\
\hspace{1em}\textit{Clive} & ``Lord Clive'' (pub.~1840, review of John Malcolm, \textit{The life of Robert Lord Clive}). \\
\hspace{1em}\textit{Milton} & ``Essay on Milton'' (pub.~1825, review of Milton, \textit{A treatise on Christian doctrine}, trans. Charles R. Sumner). \\ % It's not a translated review, but a very long essay that's a review of a translation of Milton's Latin work.

\hangindent=1em \hangafter=1 MacDonald & William MacDonald (b.~1863)\\
\hspace{1em}\textit{Account} & ``Some account of Franklin's later life.'' In William MacDonald (ed.), \textit{The autobiography of Benjamin Franklin}. 205--314. London, 1905. \\

\hangindent=1em \hangafter=1 Maclaren, \textit{Days} & Ian Maclaren (b.~1850). \textit{The days of auld lang syne} (pub.~1895). London, 1896. \\ % Correcting OJ's "MacLaren"

% Madvig, \textit{Græsk ordføjningslære} & Johan Nikolai Madvig. \textit{Græsk Ordføjningslære, især for den attiske Sprogform}. Copenhagen. \\ % There was an 1846 (1st) and an 1857 (2nd) edition.  Now cited via BibTeX
\hangindent=1em \hangafter=1 Madvig, \textit{Kjönnet} & Johan Nikolai Madvig (b.~1804). \textit{Om kjönnet i sprogene især i Sanskrit, Latin og Græsk}. Copenhagen, 1835. \\

\hangindent=1em \hangafter=1 Malory, \textit{Morte} & Thomas Malory. \textit{Le morte Darthur} (wr. c. 1470, pub. 1485). H.~Oskar Sommer (ed.). London, 1889. \\

\hangindent=1em \hangafter=1 Marlowe & Christopher Marlowe (b.~1564) \\
\hspace{1em}\textit{Edward} & \textit{Edward the second} (pub.~1594). Tucker Brooke (ed.). Oxford, 1910. \\
\hspace{1em}\textit{Faustus} & \textit{Doctor Faustus} (wr. c. 1592). \textit{Marlowes Werke, historische-kritische Ausgabe}, vol.~2. Hermann Breymann \& Albrecht Wagner (eds.). Heilbronn, 1889. \\
\hspace{1em}\textit{Tambur\-laine} & \textit{Tamburlaine the great} (wr. c. 1587). \textit{Marlowes Werke, historische-kritische Ausgabe}, vol.~1. Hermann Breymann \& Albrecht Wagner (eds.). Heilbronn, 1885. \\

\hangindent=1em \hangafter=1 Marryat, \textit{Peter} & Frederick Marryat (b.~1792). \textit{Peter Simple} (pub.~1834). London, 1834. \\

\hangindent=1em \hangafter=1 Masefield, \textit{Mercy} & John Masefield (b.~1878). \textit{The everlasting mercy} (pub.~1911). London, 1912. \\

\hangindent=1em \hangafter=1 Mason, \textit{Water} & A. E. W. Mason (b.~1865). \textit{Running water} (pub.~1907). London. \\

\hangindent=1em \hangafter=1 Matthews, \textit{Son} & Brander Matthews (b.~1852). \textit{His father's son} (pub.~1895). New York, 1896. \\

\hangindent=1em \hangafter=1 Matthiesen  & Oscar Matthiesen (b.~1861)\\
\hspace{1em}\textit{Stjærner} & \textit{Stjærner og striber}. Copenhagen, 1874. \\ % OJ writes "Stjerner" but the cited book says "Stjærner".

\hangindent=1em \hangafter=1 \textit{Maundevile} & \textit{The voiage and travaile of Sir John Maundevile, Kt} (wr. 1357--71). J.~O. Halliwell (ed.). London, 1883. \\

\hangindent=1em \hangafter=1 Maupassant & Guy de Maupassant (b.~1850) \\
\hspace{1em}\textit{Bécasse} & \textit{Contes de la bécasse} (pub.~1887). \\
\hspace{1em}\textit{Fort} & \textit{Fort comme la mort} (pub.~1889). \\
\hspace{1em}\textit{Vie} & \textit{Une vie} (pub.~1883, also titled \textit{L'Humble Vérité}). \\

\hangindent=1em \hangafter=1 McCarthy & Justin McCarthy (b.~1830) \\
\hspace{1em}\textit{History} & \textit{A history of our own times} (pub.~1879--). New York, 1880. \\ % OJ misspells the surname as MacCarthy.
\hspace{1em}\textit{Short history} & \textit{A short history of our own times}. \\ % OJ misspells the surname as MacCarthy. Probably everything in the Short History is also in the (full) History. But both are multivolume works, and there are problems of availability on the web.

%Meillet, \textit{Langue Grecque} & Antoine Meillet. \textit{Aperçu d'une histoire de la langue grecque}. Paris, 1913. \\  Now cited via BibTeX
\hangindent=1em \hangafter=1 Meillet, \textit{Caractères} & A. Meillet (b.~1866). \textit{Caractères généraux des langues germaniques} (pub.~1917). Paris, 1917. \\ 

\hangindent=1em \hangafter=1 Meredith & George Meredith (b.~1828) \\
\hspace{1em}\textit{Harrington} & \textit{Evan Harrington} (pub.~1861). London, 1889. \\
\hspace{1em}\textit{Ordeal} & \textit{The ordeal of Richard Feverel} (pub.~1859). London, 1895. \\

\hangindent=1em \hangafter=1 Mérimée, \textit{Héritages} & Prosper Mérimée (b.~1803). \textit{Les Deux Héritages ou Don Quichotte} (pub.~1850). In \textit{Les Deux Héritages: suivis de L'Inspecteur général et des Débuts d'un aventurier}. Paris, 1853. [A.Sc] \\

\hangindent=1em \hangafter=1 Merriman & Henry Seton Merriman (b.~1862) \\
\hspace{1em}\textit{Sowers} & \textit{The sowers} (pub.~1896). London, 1905. \\
\hspace{1em}\textit{Vultures} & \textit{The vultures} (pub.~1890). London, 1902. \\

\hangindent=1em \hangafter=1 Milton & John Milton (b.~1608) \\
\hspace{1em}\textit{Areo\-pagitica} & \textit{Areopagitica} (pub.~1644). \\
\hspace{1em}\textit{Lost} & \textit{Paradise lost} (pub.~1667). [Bk.L] \\
\hspace{1em}\textit{Samson} & \textit{Samson agonistes} (pub.~1671). [L] \\

\hangindent=1em \hangafter=1 Molbech, letter & Christian K. F. Molbech (b.~1821). Letter (wr. 1855). In Harald Høffding (ed.), \textit{Hans Brøchner og Christian K. F. Molbech: en brevvexling (1845--1875)}. Copenhagen, 1902. \\

\hangindent=1em \hangafter=1 E. Møller, \textit{Inderstyre} & Ernst Møller (b.~1860). \textit{Inderstyre: Veje og midler for herredömmet over de indre kræfter og for sjæleligt fremarbejde} (pub.~1914). 1914. \\

\hangindent=1em \hangafter=1 N. Møller, \textit{Koglerier} & Niels Møller (b.~1859). \textit{Koglerier} (pub.~1895). Copenhagen, 1895. \\

\hangindent=1em \hangafter=1 P. Møller, letter & Poul Møller (b.~1794). Letter of 25 June 1820. \\

\hangindent=1em \hangafter=1 More, \textit{Utopia} & Thomas More. \textit{Utopia} (pub.~1551). Ralph Robynson (b.~1520) (trans.). In J.~H. Lupton (ed.), \textit{The Utopia of Sir Thomas More}. Oxford, 1895. \\

\hangindent=1em \hangafter=1 Morris, \textit{News} & William Morris (b.~1834). \textit{News from nowhere} (pub.~1890). London, 1908. \\

\hangindent=1em \hangafter=1 Mulock, \textit{Halifax} & Dinah Maria Mulock (b.~1826). \textit{John Halifax, gentleman} (pub.~1858). Tauchnitz. \\

\hangindent=1em \hangafter=1 Nansen, \textit{Fred} & Peter Nansen (b.~1861). \textit{Guds fred} (pub.~1895). \\

\hangindent=1em \hangafter=1 \textit{NED} & \textit{A new English dictionary on historical principles}. Oxford, 1884--.\footnote{\textit{NED} was published in installments from 1884 to 1933 (when a corrected edition was retitled \textit{The Oxford English dictionary}); by 1917 its coverage extended from A to Sh. \eds} \\

\hangindent=1em \hangafter=1 news & newspaper or periodical\footnote{Originally ``NP''; \citet[xxii]{jespersenMEG2} explained this as ``Newspaper (or periodical; among those most frequently quoted are \textit{The Times, Daily News, Daily Chronicle, Westminster Gazette, The Tribune; New York Times, Evening News; Everyman, Public Opinion; The Outlook; The Bookman, Review of Reviews, The World's Work}).'' Where we have found the particular source, we have specified it rather than cite ``news''. \eds} \\

\hangindent=1em \hangafter=1 Nexø, \textit{Pelle} & Martin Andersen Nexø (b.~1869). \textit{Pelle erobreren} (pub.~1906--10). Copenhagen, 1906--10. \\

\hangindent=1em \hangafter=1 Nielsen & [unidentified] \\ % ??? PE: Which appearance of OJ's sole quotation of L. C. Nielsen should we refer to -- OJ's own (which I can't find), or a different one (which I can find)? See the note within 06.tex. (Simply search within it for "Nielsen".)

\hangindent=1em \hangafter=1 \textit{Noah} & \textit{Noah and the ark} (wr. mid-16th century). In George England (ed.), \textit{The Towneley plays}. London, 1897. [L] \\

\hangindent=1em \hangafter=1 Norris & Frank Norris (b.~1870) \\
\hspace{1em}\textit{Octopus} & \textit{The octopus: A story of California} (pub.~1901). London, 1908. \\
\hspace{1em}\textit{Pit} & \textit{The pit: A story of Chicago} (pub.~1903). London, 1908. \\ 

\hangindent=1em \hangafter=1 \textit{Odyssey} & Hómēros. \textit{Odýsseia}. (Homer. \textit{The Odyssey}. Wr. 8th/\linebreak[2]7th c.~BCE.) \\

\hangindent=1em \hangafter=1 Oppenheim & E. Phillips Oppenheim (b.~1866)\\
\hspace{1em}\textit{Millionaire} &  \textit{A millionaire of yesterday} (pub.~1900). \\

\hangindent=1em \hangafter=1 Otway, \textit{Venice} & Thomas Otway (b.~1652). \textit{Venice preserv’d; or, A plot discover’d} (pub.~1682). In Charles F. McClumpha (ed.), \textit{The orphan} and \textit{Venice preserved}. Boston, 1908. \\ % The spelling of the original is "preserv'd"; the spelling of the 1908 title is "preserved". (How irritating....)

\hangindent=1em \hangafter=1 T. N. Page, \textit{Marvel} & Thomas Nelson Page (b.~1853). \textit{John Marvel, assistant} (pub.~1907). New York, 1909. \\

\hangindent=1em \hangafter=1 W. H. Page & Walter Hines Page (b.~1855)\\
\hspace{1em}\textit{Southerner} & \textit{The southerner} (pub.~1909). \\

\hangindent=1em \hangafter=1 Paludan-Müller & Frederik Paludan-Müller (b.~1809)\\
\hspace{1em}\textit{Adam} & \textit{Adam Homo, et digt} (pub.~1841--48). Copenhagen, 1857. \\

\hangindent=1em \hangafter=1 \textit{Parable} & ``A parable of the war.'' \textit{Times Literary Supplement}. 3 August 1917. \\ % https://scholarship.law.vanderbilt.edu/cgi/viewcontent.cgi?article=1525&context=vjtl says 2 August and also identifies the author.

\hangindent=1em \hangafter=1 Parker, \textit{Right} & Gilbert Parker (b.~1862). \textit{The right of way} (pub.~1901). London, 1906. \\

\hangindent=1em \hangafter=1 Paton, \textit{Tower} & Lucy Allen Paton (b.~1865). ``The story of Vortigern's tower: An analysis.'' \textit{Radcliffe College Monographs} 15 (1910). 13--23. \\

\hangindent=1em \hangafter=1 Pedersen, \textit{Skrifter} & Christiern Pedersen (b. c. 1480). In \textit{Christiern Pedersens danske skrifter}. C.~J. Brandt (ed.). Copenhagen, 1854. \\ % If for some reason H. Pedersen's Russisk læsebog reverts from BibTeX to this style of referencing, NB the author of Danske skrifter should be "C. Pedersen".

\hangindent=1em \hangafter=1 Pérez Galdós, \textit{Doña} & Benito Pérez Galdós (b.~1843). \textit{Doña perfecta} (pub.~1876). \\

\hangindent=1em \hangafter=1 \textit{Pericles} & William Shakespeare \& George Wilkins (attributed). \textit{Pericles, Prince of Tyre} (pub.~1609). [A] \\

\hangindent=1em \hangafter=1 Petersen, \textit{Uddrag} & Niels Matthias Petersen (b.~1791). \textit{Nogle uddrag af forelæsninger, vedkommende de nordiske sprog}. \textit{Samlede afhandlinger} 4.~107--294. Copenhagen, 1861. \\ 

\hangindent=1em \hangafter=1 Philips, \textit{Glass} & F. C. Philips (b.~1849). \textit{As in a looking glass} (pub.~1885). Tauchnitz, 1886.  \\

\hangindent=1em \hangafter=1 Phillpotts, \textit{Mother} & Eden Philpotts (b.~1862). \textit{The mother} (pub.~1908). London, 1908.  \\

\hangindent=1em \hangafter=1 \textit{Pilgrime} & Various authors. \textit{The passionate pilgrime} (pub.~1599). [L] \\

\hangindent=1em \hangafter=1 Pinero & Arthur W. Pinero (b.~1855) \\
\hspace{1em}\textit{Benefit} & \textit{The benefit of the doubt} (perf. 1895). London, 1895.  \\
\hspace{1em}\textit{Magistrate} & \textit{The magistrate} (perf. 1885). London, 1897.  \\
\hspace{1em}\textit{Quex} & \textit{The gay Lord Quex} (perf. 1899). London, 1900.  \\
\hspace{1em}\textit{Second} & \textit{The second Mrs. Tanqueray} (perf. 1893). London, 1895.  \\

\hangindent=1em \hangafter=1 Pollard, \textit{Prevention} & A. F. Pollard (b.~1869). ``The prevention of war.'' \textit{Times Literary Supplement}, 5 July 1917. \\ % OJ says 6 July but this is a mistake

\hangindent=1em \hangafter=1 Pontoppidan & Henrik Pontoppidan (b.~1857)\\
\hspace{1em}\textit{Billeder} & \textit{Landsbybilleder} (pub.~1883). \\

\hangindent=1em \hangafter=1 Pope, \textit{Rape} & Alexander Pope (b.~1688). \textit{The rape of the lock} (pub.~1714). In \textit{Poetical works}. London, 1892. \\

\hangindent=1em \hangafter=1 Quiller-Couch & Arthur Quiller-Couch (b.~1863) \\
\hspace{1em}\textit{Major} & \textit{Major Vigoureux} (pub.~1907). London, 1907.  \\
\hspace{1em}\textit{Troy} & \textit{The astonishing history of Troy town} (pub.~1888). London, {[}1888{]}. \\

\hangindent=1em \hangafter=1 Quincey & Thomas De Quincey (b.~1785) \\
\hspace{1em}\textit{Confessions} & \textit{Confessions of an English opium-eater} (pub.~1821). London, 1901. \\
\hspace{1em}\textit{Mail-coach} & ``The English mail-coach'' (pub.~1849). \\
\hspace{1em}\textit{Murder} & ``On murder considered as one of the fine arts'' (pub.~1827). \\

\hangindent=1em \hangafter=1 Raleigh  & Walter Raleigh (b.~1861)\\
\hspace{1em}\textit{Shakespeare} & \textit{Shakespeare} (pub.~1907). London, 1907. \\

\hangindent=1em \hangafter=1 Ranch, \textit{Niding} & Hieronymus Justesen Ranch (b.~1539). \textit{Karrig Niding} (wr. c. 1600). In S. Birket Smith (ed.), \textit{Hieronymus Justesen Ranch's danske skuespil og fuglevise}. Copenhagen, 1876--77. \\ % "Niding" is a proper name, so leave capitalized

\hangindent=1em \hangafter=1 Rask & Rasmus Kristian Rask (b.~1787)\\
\hspace{1em}\textit{Undersögelse} &  \textit{Undersögelse om det gamle nordiske eller islandske sprogs oprindelse} (pub.~1818). Copenhagen, 1818.\\

\pagebreak
\hangindent=1em \hangafter=1 Read & Opie Read (b.~1852) \\
\hspace{1em}\textit{Colonel} & \textit{A Kentucky colonel} (pub.~1890). \\
\hspace{1em}\textit{Toothpick} & \textit{Toothpick tales} (pub.~1892). Chicago, 1892. \\

\hangindent=1em \hangafter=1 \textit{Recluse} & ``The recluse'' (wr. 1375--1400). Excerpted in A. C. Paues, ``A XIVth-century version of the \textit{Ancren riwle}.'' \textit{Englische Studien} 30 (1902). 344--346. \\ % OJ gives the page number; this source does not. So OJ probably used a later and fuller reproduction of this item in the MS

\hangindent=1em \hangafter=1 Richardson  & Samuel Richardson (b.~1689)\\
\hspace{1em}\textit{Grandison} & \textit{The history of Sir Charles Grandison} (pub.~1753). London, 1764. \\

\hangindent=1em \hangafter=1 Ridge & W. Pett Ridge (b.~1859) \\
\hspace{1em}\textit{Garland} & \textit{Name of Garland} (pub.~1907). Tauchnitz. \\
\hspace{1em}\textit{Property} & \textit{Lost property} (pub.~1902). London, 1902. \\
\hspace{1em}\textit{Son} & \textit{A son of the state} (pub.~1899). London, n.d. \\

\hangindent=1em \hangafter=1 \textit{Roister} & Nicholas Udall. \textit{Roister Doister} (wr. c. 1552, pub. 1567). Edward Arber (ed.). Birmingham, 1869. \\

\hangindent=1em \hangafter=1 Rolland & Romain Rolland (b.~1866) \\
\hspace{1em}\textit{Amies} & \textit{Les Amies} (pub.~1910, \textit{Jean-Christophe}, vol.~8). \\ % OJ refers to this as "JChr. 8"
\hspace{1em}\textit{Aube} & \textit{L'Aube} (pub.~1904, \textit{Jean-Christophe}, vol.~1). \\ % OJ refers to this as "JChr. 1"
\hspace{1em}\textit{Buisson} & \textit{Le Buisson ardent} (pub.~1911, \textit{Jean-Christophe}, vol.~9). \\ % OJ refers to this as "JChr. 9"
\hspace{1em}\textit{Foire} & \textit{La Foire sur la place} (pub.~1908, \textit{Jean-Christophe}, vol.~5). \\ % OJ refers to this as "JChr. 5"
\hspace{1em}\textit{Journée} & \textit{La Nouvelle Journée} (pub.~1912, \textit{Jean-Christophe}, vol.~10). \\ % OJ refers to this as "JChr. 10"
\hspace{1em}\textit{Maison} & \textit{Dans la maison} (pub.~1908, \textit{Jean-Christophe}, vol.~7). \\ % OJ refers to this as "JChr. 7"
\hspace{1em}\textit{Révolte} & \textit{La Révolte} (pub.~1905, \textit{Jean-Christophe}, vol.~4). \\ % OJ refers to this as "JChr. 4"

\hangindent=1em \hangafter=1 Ruskin & John Ruskin (b.~1819) \\
\hspace{1em}\textit{Crown} & \textit{The crown of wild olive} (pub.~1866). London, 1904.  \\
\hspace{1em}\textit{Fors} & \textit{Fors clavigera} (pub.~1871--84). London, 1902.  \\
\hspace{1em}\textit{Præterita} & \textit{Præterita} (pub.~1885--89). London, 1902.  \\
\hspace{1em}\textit{Selections} & \textit{Selections}. London, 1893. \\ % I (PE) haven't been able to find this edition on the web. There are a number of editions titled "Selections" and those I've found have been quite independent of the one that OJ cites. I've therefore left in the "Selections" attribution but linked elsewhere.
\hspace{1em}\textit{Sesame} & \textit{Sesame and lilies} (pub.~1865). London, 1904.  \\
\hspace{1em}\textit{Things} & ``Things to be studied.'' Appendix to \textit{The elements of drawing} (pub.~1857). \\
\hspace{1em}\textit{Time} & \textit{Time and tide} (pub.~1867). London, 1904. \\

\hangindent=1em \hangafter=1 Rutebeuf, \textit{Pharisian} & Rutebeuf. ``Du Pharisian'', ou ``C'est d'ypocrisie'' (wr. 1257). \\ % PE: Wikipedia says "He was born in the first half of the 13th century": so vague that repeating it here would effectively add next to nothing to "(wr. 1257)".

\hangindent=1em \hangafter=1 Sand, \textit{Consuelo} & George Sand (b.~1804). \textit{Consuelo} (pub.~1842--43). \\

\hangindent=1em \hangafter=1 Schiller & Friedrich Schiller (b.~1759) \\
\hspace{1em}\textit{Messina} & \textit{Die Braut von Messina} (perf. 1803). [A.Sc] \\
\hspace{1em}\textit{Tod} & \textit{Wallensteins Tod} (perf. 1799). [A.Sc] \\

\hangindent=1em \hangafter=1 Schreiner & Olive Schreiner (b.~1855) \\
\hspace{1em}\textit{Peter Halket} & \textit{Trooper Peter Halket of Mashonaland} (pub.~1897). London, 1897.\\

\hangindent=1em \hangafter=1 Scott & Walter Scott (b.~1771) \\
\hspace{1em}\textit{Antiquary} & \textit{The antiquary} (pub.~1816). Edinburgh, 1821.  \\
\hspace{1em}\textit{Chronicles} & \textit{Chronicles of the Canongate}, 2nd series (pub.~1828). \\
\hspace{1em}\textit{Ivanhoe} & \textit{Ivanhoe} (pub.~1819). London. \\
\hspace{1em}\textit{Mortality} & \textit{Old mortality} (pub.~1816). Oxford, 1906. \\ % OJ refers to this as either "O" or "OM"

\hangindent=1em \hangafter=1 Seeley, \textit{Expansion} & J. R. Seeley (b.~1834). \textit{The expansion of England} (pub.~1883). London, 1883. \\

\hangindent=1em \hangafter=1 Shakespeare & William Shakespeare (b.~1564).\footnote{Unless noted otherwise, all quotations from plays by Shakespeare are from the text of the first folio (\textit{Mr. William Shakespeares comedies, histories, \& tragedies}, 1623), and are specified [A.Sc.L]. \eds}\\
\hspace{1em}\textit{Ado} & \textit{Much adoe about nothing} (wr. 1598--99). \\
\hspace{1em}\textit{Alls} & \textit{All's well, that ends well} (wr. 1598--1608). \\
\hspace{1em}\textit{As} & \textit{As you like it} (wr. 1599). \\
\hspace{1em}\textit{Cæs} & \textit{The tragedie of Ivlivs Cæsar} (perf. 1599).\\
\hspace{1em}\textit{Cor} & \textit{The tragedy of Coriolanus} (wr. 1605--08).\\
\hspace{1em}\textit{Cymb} & \textit{The tragedie of Cymbeline} (perf. 1611).\\
\hspace{1em}\textit{Err} & \textit{The comedie of errors} (perf. 1594).\\
\hspace{1em}\textit{Gent} & \textit{The two gentlemen of Uerona} (wr. c. 1590). \\
\hspace{1em}\textit{Hml} & \textit{The tragedie of Hamlet, Prince of Denmarke} (wr. 1599--1601). \\
\hspace{1em}\textit{H4A} & \textit{The first part of Henry the Fourth} (wr. 1596--97). \\
\hspace{1em}\textit{H4B} & \textit{The second part of Henry the Fourth} (wr. 1596--99). \\
\hspace{1em}\textit{H5} & \textit{The life of Henry the fift} (wr. c. 1599). \\ % No "h" at the end of "Fift"!
\hspace{1em}\textit{H6A} & \textit{The first part of Henry the sixt} (wr. 1591--92). \\ % No "h" at the end of "Sixt"!
\hspace{1em}\textit{H8} & \textit{The life of King Henry the eight} (wr. 1613). \\ % Sic.( Not "Eighth" but "Eight".)
\hspace{1em}\textit{John} & \textit{The life and death of King Iohn} (wr. 1590s, pub. 1623). \\
\hspace{1em}\textit{LLL} & \textit{Loues labour's lost} (wr. 1595--97). \\
\hspace{1em}\textit{Lr} & \textit{The tragedie of King Lear} (perf. 1606). \\
\hspace{1em}\textit{Lucr} & \textit{Lvcrece} (pub.~1594). Quarto: London, 1594. [L] \\ % This is the title on the title page. (No mention there of "Rape".)
\hspace{1em}\textit{Mcb} & \textit{The tragedie of Macbeth} (wr. 1596--98). \\
\hspace{1em}\textit{Meas} & \textit{Measvre, for measure} (perf. 1604). \\ % Yes, this is the way it's spelt and comma'd in the first folio.
\hspace{1em}\textit{Merch} & \textit{The merchant of Venice} (wr. 1596--98). \\
\hspace{1em}\textit{Mids} & \textit{A midsommer nights dreame}  (wr. c. 1595). \\ % Yes, really, so spelt.
\hspace{1em}\textit{Oth} & \mbox{\textit{The tragedie of Othello, the moore of Venice} (wr.\,c.\,1603).}\\
\hspace{1em}\textit{R2} & \textit{The life and death of Richard the second} (perf. 1595). \\
\hspace{1em}\textit{R3} & \textit{The tragedy of Richard the third} (wr. 1592--94).  \\
\hspace{1em}\textit{Rom} & \textit{The tragedie of Romeo and Ivliet} (wr. 1591--95). \\
\hspace{1em}\textit{Shr} & \textit{The taming of the shrew} (wr. 1590--92). \\
\hspace{1em}\textit{Tit} & \textit{Titus Andronicus} (wr. c. 1590). \\
\hspace{1em}\textit{Tp} & \textit{The tempest} (wr. 1610--11). \\
\hspace{1em}\textit{Tw} & \textit{Twelfe night, or What you will} (wr. 1601). \\
\hspace{1em}\textit{Ven} & \mbox{\textit{Venvs and Adonis} (pub.~1593). Quarto: London, 1593. [L]}\\
\hspace{1em}\textit{Wint} & \textit{The winters tale} (wr. 1609--11, pub. 1623). \\
\hspace{1em}\textit{Wiv} & \textit{The merry wiues of Windsor} (pub.~1602). \\

\hangindent=1em \hangafter=1 Shaw & George Bernard Shaw (b.~1856) \\
\hspace{1em}\textit{Arms} & \textit{Arms and the man} (perf. 1894). In \textit{Plays pleasant}. London, 1898. [A] \\ 
\hspace{1em}\textit{Candida} & \textit{Candida: A mystery} (wr. 1894, pub. 1898). In \textit{Plays pleasant}. London, 1898. \\ 
\hspace{1em}\textit{Cashel} & \textit{Cashel Byron's profession} (wr. 1882). London, 1901. \\ 
\hspace{1em}\textit{Destiny} & \textit{The man of destiny: A trifle} (perf. 1897). In \textit{Plays pleasant}. London, 1898. \\
\hspace{1em}\textit{Dilemma} & \textit{The doctor's dilemma} (perf. 1906, pub. 1909). London, 1911. \\ 
\hspace{1em}\textit{Disciple} & \textit{The Devil's disciple} (perf. 1897). In \textit{Three plays for puritans}. London, 1901. [A] \\ 
\hspace{1em}\textit{First} & \textit{Fanny's first play} (perf. 1911). In \textit{Misalliance, The dark lady}, and \textit{Fanny's first play}. London, 1914. \\ 
\hspace{1em}\textit{Houses} & \textit{Widowers' houses} (perf. 1892). In \textit{Plays unpleasant}. London, 1898. \\ 
\hspace{1em}\textit{Ibsenism} & \textit{The quintessence of Ibsenism} (wr. 1891). London, 1891. \\ 
\hspace{1em}\textit{Island} & \textit{John Bull's other island} (wr. 1904). London, 1907. \\ 
\hspace{1em}\textit{Major} & \textit{Major Barbara} (perf. 1905, pub. 1907). \\ 
\hspace{1em}\textit{Married} & \textit{Getting married: A disquisitory play} (perf. 1908). \\
\hspace{1em}\textit{Never} & \textit{You never can tell} (wr. 1897, perf. 1899). In \textit{Plays pleasant}. London, 1898. \\ 
\hspace{1em}\textit{Philanderer} & \textit{The philanderer: A topical comedy} (wr. 1893, perf. 1902). In \textit{Plays unpleasant}. London, 1898. \\ 
\hspace{1em}\textit{Profession} & \textit{Mrs. Warren's profession}  (wr. 1893, perf. 1902). In \textit{Plays unpleasant}. London, 1898. \\

\pagebreak
\hangindent=1em \hangafter=1 Shelley & Percy Bysshe Shelley (b.~1792) \\
\hspace{1em}\textit{Epi\-psychidion} & \textit{Epipsychidion} (pub.~1821). In Thomas Hutchinson (ed.), \textit{The complete poetical works of Percy Bysshe Shelley}. London, 1914. \\
\hspace{1em}\textit{Letters} & Roger Ingpen (ed.), \textit{The letters of Percy Bysshe Shelley}. London, 1914. \\
\hspace{1em}\textit{Prometheus} & \textit{Prometheus unbound} (pub.~1820). In Thomas Hutchinson (ed.), \textit{The complete poetical works of Percy Bysshe Shelley}. London, 1914.  \\
\hspace{1em}\textit{Prose} & Richard Herne Shepherd (ed.), \textit{The prose works of Percy Bysshe Shelley}. London, 1914. \\ % OJ’s book says “Pr 295”. I (PE) can’t think what “Pr” can be short for in this context, other than “Prose”, and suspect that it’s this book, with its slightly different page numbering, that’s meant.
\hspace{1em}\textit{Revolt} & \textit{The revolt of Islam} (wr. 1817). In Thomas Hutchinson (ed.), \textit{The complete poetical works of Percy Bysshe Shelley}. London, 1914. \\

\hangindent=1em \hangafter=1 Shenstone & William Shenstone (b.~1714)\\
\hspace{1em}\textit{Schoolmistress} & ``The schoolmistress'' (wr. 1737).\\

\hangindent=1em \hangafter=1 Sheridan & Richard Brinsley Sheridan (b.~1751) \\
\hspace{1em}\textit{Critic} & \textit{The critic} (perf. 1779). In \textit{Dramatic works}. Tauchnitz. [A.Sc.L] \\ 
\hspace{1em}\textit{Duenna} & \textit{The duenna} (perf. 1775). In \textit{Dramatic works}. Tauchnitz. [A.Sc] \\ % I (PE) am guessing that this is what OJ is referring to by "D" ... though I have serious doubts.
\hspace{1em}\textit{Rivals} & \textit{The rivals} (perf. 1775). In \textit{Dramatic works}. Tauchnitz. [A.Sc] \\
\hspace{1em}\textit{School} & \textit{The school for scandal} (perf. 1777). In \textit{Dramatic works}. Tauchnitz. [A.Sc] \\

\hangindent=1em \hangafter=1 Sibbern, \textit{Breve} & Frederik Christian Sibbern (b.~1785). \textit{Gabrielis breve} (wr. 1813--14, pub. 1826). \\

\hangindent=1em \hangafter=1 Skeat, \textit{John} & In Walter W. Skeat (ed.), \textit{The gospel according to Saint John: In Anglo-Saxon and Northumbrian versions synoptically arranged}. Cambridge, 1878. \\ % I (PE) presume that this (at https://archive.org/details/holygospelsinan01skeagoog/page/n7/mode/2up?view=theater ) is what OJ refers to in the last paragraph on p 55 of the printed book.

\hangindent=1em \hangafter=1 Skram, \textit{Lucie} & Amalie Skram (b.~1846). \textit{Lucie} (pub.~1888). Copenhagen, 1888. \\

\hangindent=1em \hangafter=1 Smedley, \textit{Frank} & Frank E. Smedley (b.~1818). \textit{Frank Fairlegh} (pub.~1850). Tauchnitz. \\

\hangindent=1em \hangafter=1 Sörensen & Axel Sörensen (b.~1851)\\
\hspace{1em}\textit{Ariadnetråd} & \textit{En Ariadnetråd} (pub.~1902). Copenhagen, 1902.  \\ % OJ spells his surname “Sørensen”, but on the title page of the cited book it’s instead “Sörensen”. About language.

\hangindent=1em \hangafter=1 \textit{Spectator} & Joseph Addison (b.~1672) et al. \textit{The Spectator} (pub.~1711--14). Henry Morley (ed.). London, 1888. \\ % The Spectator is the title of the magazine, and for this reason perhaps "Spectator" should be capitalized.

\hangindent=1em \hangafter=1 Spencer & Herbert Spencer (b.~1820) \\
\hspace{1em}\hangindent=1em \hangafter=1 \textit{Autobiography} & \textit{An autobiography} (pub.~1904). London, 1904. \\
\hspace{1em}\textit{Education} & \textit{Education: Intellectual, moral, and physical} (pub.~1861). London, 1882. \\ %Brett: I futzed with this one to make Autobiography break and align correctly

\hangindent=1em \hangafter=1 Spenser & Edmund Spenser (b. c. 1552). \textit{The faerie queene} (pub.~1590/96). \\ % Peter: OJ refers to ES just once, in chap 10. He spells him "Spencer" and doesn't specify the work(s). Therefore we don't add "Queene" or any other short-form title here.

\hangindent=1em \hangafter=1 Stacpoole, \textit{Cottage} & Henry de Vere Stacpoole (b.~1863). \textit{The cottage on the fells} (pub.~1908). Toronto, n.d. \\

\hangindent=1em \hangafter=1 Stanley, \textit{Dark} & Henry M. Stanley (b.~1841). \textit{Through the dark continent} (pub.~1878). London, 1878. \\

\hangindent=1em \hangafter=1 Sterling, letter & John Sterling (b.~1806). Letter of 29 May 1835 to Thomas Carlyle. In Thomas Carlyle, \textit{The life of John Sterling}. \\

\hangindent=1em \hangafter=1 Sterne, \textit{Tristram} & Laurence Sterne (b.~1713). \textit{Tristram Shandy} (pub.~1759--67). In David Herbert (ed.), \textit{The complete works}. Edinburgh, 1885. \\ 

\hangindent=1em \hangafter=1 Stevenson & Robert Louis Stevenson (b.~1850) \\
\hspace{1em}\textit{Arrow} & \textit{The black arrow} (pub.~1883). London, 1904. \\
\hspace{1em}\textit{Art} & \textit{Essays in the art of writing} (pub.~1905). London, 1905. \\
\hspace{1em}\textit{House} & ``The house of Eld'' (pub.~1896). In \textit{The strange case of Dr. Jekyll and Mr. Hyde, with other fables}. London, 1896. \\ % OJ calls this "JHF". Don’t remove “with Other Fables”.
\hspace{1em}\textit{Jekyll} & \textit{The strange case of Dr. Jekyll and Mr. Hyde} (pub.~1886). In \textit{The strange case of Dr. Jekyll and Mr. Hyde, with other fables}. London, 1896. \\ % OJ calls this "JHF". Don’t remove “with Other Fables”.
\hspace{1em}\textit{Memories} & \textit{Memories and portraits} (pub.~1887). London, 1900. \\  % The list of sources in MEG vol 7 has this as “Memoirs and Portraits”: a mistake. (MEG vol 2 gets it right.)
\hspace{1em}\textit{Men} & \textit{Familiar studies of men and books} (pub.~1882). London, 1901. \\
\hspace{1em}\textit{Merry} & \textit{The merry men} (pub.~1887). London, 1896.  \\
\hspace{1em}\textit{Treasure} & \textit{Treasure island} (pub.~1881--82). Tauchnitz.  \\
\hspace{1em}\textit{Virginibus} & \textit{Virginibus puerisque and other papers} (pub.~1881). London, 1894.  \\

\hangindent=1em \hangafter=1 Strindberg & August Strindberg (b.~1849) \\
\hspace{1em}\textit{Giftas} & \textit{Giftas} (pub.~1884/86). Stockholm, 1886. \\
\hspace{1em}\textit{Röda} & \textit{Röda rummet} (pub.~1879). \\
\hspace{1em}\textit{Utopier} & \textit{Utopier i verkligheten} (pub.~1885). Stockholm, 1885. \\

\hangindent=1em \hangafter=1 Sudermann & Hermann Sudermann (b.~1857)\\
\hspace{1em}\textit{Fritzchen} & \textit{Fritzchen}. In \textit{Morituri. Teja, Fritzchen, Das Ewig-Männliche} (pub.~1896).\\

\hangindent=1em \hangafter=1 J. Swift & Jonathan Swift (b.~1667) \\
\hspace{1em}\textit{Conversa\-tion} & \textit{A compleat collection of genteel and ingenious conversation} (wr. 1731, pub. 1738). George Saintsbury (ed.). 1892. \\
\hspace{1em}\textit{Journal} & \textit{A journal to Stella} (wr. 1710--13, pub. 1766). George A. Aitken (ed.). London, 1901. \\
\hspace{1em}letter & Letter to Isaac Bickerstaff. \textit{The Tatler} no. 230. 28 September 1710. \\ % PE: [If it might matter for any later reformatting:] The issue of The Tatler is dated 28 September. (The letter is dated 27 September. A rather implausible date combination, to be sure.)
\hspace{1em}\textit{Travels} & \textit{Volume III of the author’s works. Containing travels into several remote nations of the world} (pub.~1726, \textit{Gulliver's travels}). Dublin, 1735. \\
\hspace{1em}\textit{Tub} & \textit{A tale of a tub} (wr. 1694--97, pub. 1704). London, 1760. \\

\hangindent=1em \hangafter=1 M. I. Swift & Morrison I. Swift (b.~1856) \\ % OJ identifies this as "NP. 1911"
\hspace{1em}\textit{Humanizing} &  ``Humanizing the prisons.'' \textit{The Atlantic Monthly}. August 1911. 170--179.\\

\hangindent=1em \hangafter=1 Swinburne & Algernon Charles Swinburne (b.~1837) \\
\hspace{1em}\textit{Cross-currents} & \textit{Love's cross-currents} (pub.~1877). As \textit{A year's letters}: Tauchnitz, 1905. \\ % OJ refers to this as "L". NB In MEG, OJ uses "Swinburne L." for an entirely different work (Swinburne's Locrine); or so says the list in MEG Part II.
\hspace{1em}\textit{Pilgrimage} & ``The last pilgrimage.''  Canto 8 of \textit{Tristram of Lyonesse} (pub.~1882). In \textit{Tristram of Lyonesse, and other poems}. London, 1884. \\ 
\hspace{1em}\textit{Shakespeare} & \textit{A study of Shakespeare} (pub.~1880). London, 1895. \\
\hspace{1em}\textit{Songs} & \textit{Songs before sunrise} (pub.~1871). London, 1903. \\
% Szinnyei, \textit{Ungarische Sprachlehre} & József Szinnyei (b.~1830). \textit{Ungarische Sprachlehre}. Berlin, 1912. \\ moved to BibTeX

\hangindent=1em \hangafter=1 Tennyson & Alfred Tennyson (b.~1809) \\
\hspace{1em}\textit{1852} & ``The third of February, 1852.'' \textit{Enoch Arden, and other poems}. In \textit{Poetical works}. London, 1894. \\
\hspace{1em}\textit{Coming} & ``The coming of Arthur'' (pub.~1869). \textit{Idylls of the king}. In \textit{Poetical works}. London, 1894. \\ % Is "pub. 1869" correct?
\hspace{1em}comment & Reported comment. In Hallam Tennyson, \textit{Alfred Lord Tennyson: A memoir by his son}.  \\ % OJ refers to this as "L"
\hspace{1em}diary & Diary entry (1872). In Hallam Tennyson, \textit{Alfred Lord Tennyson: A memoir by his son}. \\ % OJ refers to this as "L"
\hspace{1em}\textit{Enid} & ``Enid'' (pub.~1859). \textit{Idylls of the king}. In \textit{Poetical works}. London, 1894. \\
\hspace{1em}\textit{Guinevere} & ``Guinevere'' (pub.~1859). \textit{Idylls of the king}. In \textit{Poetical works}. London, 1894. \\
\hspace{1em}letter & Letter (1847). In Hallam Tennyson, \textit{Alfred Lord Tennyson: A memoir by his son}. \\  % OJ refers to this as "L"
\hspace{1em}\textit{Memoriam} & \textit{In memoriam A. H. H.} (pub.~1850). In \textit{Poetical works}. London, 1894. \\
\hspace{1em}\textit{Merlin} & ``Merlin and Vivien'' (wr. 1856). \textit{Idylls of the king}. In \textit{Poetical works}. London, 1894. \\
\hspace{1em}\textit{Princess} & \textit{The Princess: A medley} (pub.~1847). In \textit{Poetical works}. London, 1894. \\
\hspace{1em}\textit{Wellington} & ``Ode on the death of the Duke of Wellington.'' \textit{Maud, and other poems} (pub.~1855). \\

\hangindent=1em \hangafter=1 Thackeray & William Makepeace Thackeray (b.~1811) \\
\hspace{1em}\textit{Newcomes} & \textit{The Newcomes} (pub.~1854--55). London, 1901. \\
\hspace{1em}\textit{Pendennis} & \textit{The history of Pendennis} (wr. 1848--50). Tauchnitz. \\
\hspace{1em}\textit{Samuel} & \textit{The history of Samuel Titmarsh and the great Hoggarty diamond} (pub.~1841). London, 1878. \\
\hspace{1em}\textit{Snobs} & \textit{The book of snobs} (pub.~1846--47). London, 1900. \\
\hspace{1em}\textit{Vanity} & \textit{Vanity fair} (pub.~1847--48). London, 1848. \\

\hangindent=1em \hangafter=1 Tobler, \textit{Beiträge} & Adolf Tobler. \textit{Vermischte Beiträge zur französischen Grammatik}. Leipzig, 1886. \\ % Also cited via BibTeX

\hangindent=1em \hangafter=1 Topsøe, \textit{Skitseb}. & Vilhelm Topsøe (b.~1840). [unidentified] \\ % ??? OJ calls this "Skitseb."; it's still UNIDENTIFIED

\hangindent=1em \hangafter=1 \textit{Trifles} & [George Nugent-Bankes (b.~1860), anonymously.] \textit{Cambridge trifles, or splutterings from an undergraduate pen}. London, 1881. \\ 

\hangindent=1em \hangafter=1 Trollope & Anthony Trollope (b.~1815) \\
\hspace{1em}\textit{Barchester} & \textit{Barchester Towers} (pub.~1857). \\
\hspace{1em}\textit{Children} & \textit{The duke's children} (pub.~1879--80). Tauchnitz, 1880.  \\
\hspace{1em}\textit{Love} & \textit{An old man's love} (pub.~1884). Tauchnitz. \\
\hspace{1em}\textit{Warden} & \textit{The warden} (pub.~1855). London, 1913. \\

\hangindent=1em \hangafter=1 Twain & Mark Twain (b.~1835) \\
\hspace{1em}\textit{Huckleberry} & \textit{Adventures of Huckleberry Finn} (pub.~1884). Tauchnitz, n.d. \\
\hspace{1em}\textit{Mississippi} & \textit{Life on the Mississippi} (pub.~1883). London, 1887. \\

\pagebreak
\hangindent=1em \hangafter=1 van Eeden & Frederik van Eeden (b.~1860) \\
\hspace{1em}\textit{Johannes} & \textit{De kleine Johannes} (pub.~1885).\\

\hangindent=1em \hangafter=1 Villiers, \textit{Rehearsal} & George Villiers (b.~1628). \textit{The rehearsal} (perf. 1671). Edward Arber (ed.). London, 1895. \\

\hangindent=1em \hangafter=1 Virgil, \textit{Aeneid} & Publius Vergilius Maro. \textit{Aeneid} (wr. 29--19 BCE). \\

\hangindent=1em \hangafter=1 \textit{Vulgate John} & \textit{Jesu Christi evangelium secundum Joannem, Vulgate} (wr. 4th century). \\
\hangindent=1em \hangafter=1 \textit{Vulgate Matthew} & \textit{Jesu Christi evangelium secundum Matthæum, Vulgate} (wr. 4th century). \\

\hangindent=1em \hangafter=1 Wägner  & Elin Wägner (b.~1882) \\
\hspace{1em}\textit{Norrtullsligan} & \textit{Norrtullsligan} (pub.~1908).\\

\hangindent=1em \hangafter=1 Walton, \textit{Angler} & Izaak Walton (b.~1593). \textit{The compleat angler} (pub.~1653). London, 1653. \\

\hangindent=1em \hangafter=1 Ward & Mrs Humphry Ward (b.~1851) \\
\hspace{1em}\textit{David} & \textit{The history of David Grieve} (pub.~1892). Tauchnitz, 1892. \\
\hspace{1em}\textit{Eleanor} & \textit{Eleanor} (pub.~1900). London, 1900. \\
\hspace{1em}\textit{Marriage} & \textit{The marriage of William Ashe} (pub.~1904--05). London. \\

\hangindent=1em \hangafter=1 Wells & H. G. Wells (b.~1866) \\
\hspace{1em}\textit{Anticipa\-tions} & \textit{Anticipations} (pub.~1901). London, 1902. \\
\hspace{1em}\textit{Britling} & \textit{Mr. Britling sees it through} (pub.~1916). London, 1916. \\
\hspace{1em}\textit{Love} & \textit{Love and Mr. Lewisham} (pub.~1900). London, 1906. \\
\hspace{1em}\textit{Machiavelli} & \textit{The new Machiavelli} (pub.~1911). London, 1911. \\
\hspace{1em}\textit{Stories} & \textit{Twelve stories and a dream} (pub.~1903). London. \\
\hspace{1em}\textit{Utopia} & \textit{A modern utopia} (pub.~1904--05). London, 1905. \\
\hspace{1em}\textit{Veronica} & \textit{Ann Veronica} (pub.~1909). London, 1909. \\
\hspace{1em}\textit{Wife} & \textit{The wife of Sir Isaac Harman} (pub.~1914). London, 1914. \\
\hspace{1em}\textit{Worlds} & \textit{New worlds for old} (pub.~1908). London, 1908. \\

\hangindent=1em \hangafter=1 Wessel & Johan Herman Wessel (b.~1742) \\
\hspace{1em}\textit{Kierlighed} & \textit{Kierlighed uden strømper} (pub.~1772). In J. Levin (ed.), \textit{J. H. Wessels samlede digte}. Copenhagen, 1862. \\ % First word now conventionally written "Kærlighed"
\hspace{1em}\textit{Polser} & ``Mosters Polser.'' In J. Levin (ed.), \textit{J. H. Wessels samlede digte}. Copenhagen, 1862.\\

\hangindent=1em \hangafter=1 WG \textit{Matthew} & Gospel of St. Matthew. \textit{Wessex gospels}, or \textit{West-Saxon gospels} (wr. c. 995). \\ % PE: Previously we had "Đæt gódspell æfter Matheus gerecednysse, Đa Feower Cristes Béc on Engliscum Gereorde" but I'm unsure where this comes from, let alone which of these words is a proper noun (and so must be capitalized)

\hangindent=1em \hangafter=1 Whiteing, \textit{Five} & Richard Whiteing (b.~1840). \textit{No. 5 John Street} (pub.~1899). \\

\hangindent=1em \hangafter=1 Whitney, \textit{Studies} & William Dwight Whitney (b.~1827). \textit{Oriental and linguistic studies}. New York, 1873. \\ % Title may sound like that of a journal, but it's a book by Whitney

\hangindent=1em \hangafter=1 Wieland, \textit{Oberon} & Christoph Martin Wieland (b.~1733). \textit{Oberon: Ein Gedicht} (pub.~1780--96). \\

\hangindent=1em \hangafter=1 Wilde & Oscar Wilde (b.~1854) \\
\hspace{1em}\textit{Fan} & \textit{Lady Windermere’s fan} (perf. 1892). \\
\hspace{1em}\textit{Gaol} & \textit{The ballad of Reading gaol} (pub.~1898). London, 1898. \\
\hspace{1em}\textit{Importance} & \textit{The importance of being earnest} (perf. 1895). London, n.d. \\
\hspace{1em}\textit{Intentions} & \textit{Intentions} (pub.~1891). 1891. \\
\hspace{1em}\textit{Picture} & \textit{The picture of Dorian Gray} (pub.~1891). New York, n.d. \\
\hspace{1em}\textit{Profundis} & \textit{De profundis} (wr. 1897, pub. 1905). London, 1905. \\

\hangindent=1em \hangafter=1 Wilkins, \textit{Pericles} & George Wilkins. \textit{The painfull aduentures of Pericles prince of Tyre} (pub.~1608). In Tycho Mommsen (ed.), \textit{Pericles prince of Tyre}. Oldenburg, 1857. \\

\hangindent=1em \hangafter=1 Williamson & C. N. Williamson (b.~1859) \& A. M. Williamson (b.~1858) \\
\hspace{1em}\textit{Lightning} & \textit{The lightning conductor} (pub.~1902). London. \\ 
\hspace{1em}\textit{Powers} & \textit{The powers and Maxine} (pub.~1907). London. \\ 

\hangindent=1em \hangafter=1 Wimmer, \textit{Læsebog} & Ludvig F. A. Wimmer. \textit{Oldnordisk læsebog}. Copenhagen, 1870. \\

\hangindent=1em \hangafter=1 Wordsworth & William Wordsworth (b.~1770) \\
\hspace{1em}\textit{Michael} & ``Michael: A pastoral poem''  (pub.~1800). In \textit{Lyrical ballads}, etc. \\
\hspace{1em}\textit{Prelude} & \textit{The prelude} (wr. 1798--1850). [Bk.L] \\
\hspace{1em}\textit{Travelled} & ``I travelled among unknown men.'' (In \textit{Poems: In two volumes} (1807), etc.) \\

\hangindent=1em \hangafter=1 Wright \& Wülcker & Thomas Wright. \textit{Anglo-Saxon and Old English vocabularies}. Richard Paul Wülcker (ed.). London, 1884.\\ % Also cited via BibTeX

\hangindent=1em \hangafter=1 Zangwill & Israel Zangwill (b.~1864)\\
\hspace{1em}\textit{Child} & ``A child of the ghetto.'' \textit{Cosmopolis} 1897. (Also in Israel Zangwill, \textit{Dreamers of the ghetto}.) \\ % OJ refers to this as "Cosmopolis '97"
\hspace{1em}\textit{Mystery} & \textit{The big bow mystery} (pub.~1892). In \textit{The grey wig: Stories and novelettes}. London, 1903.\\

\pagebreak
\hangindent=1em \hangafter=1 Þorkelsson & Jón Þorkelsson (b.~1822)\\
\hspace{1em}\textit{Nekrolog} & ``Nekrolog öfver Guðbrandur Vigfússon.'' \textit{Arkiv för Nordisk Filologi} 6 (1890). 156--163.\\
\end{xltabular}
