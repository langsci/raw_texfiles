\addchap{List of sources}
\markdouble{%
      \chapappifchapterprefix{\ }%
      \thechapter\autodot\enskip List of sources%
    }
% \setlength{\extrarowheight}{4pt} % Adjust the 5pt as needed to increase the space
\begin{longtable}{p{0.3\textwidth} p{0.6\textwidth}}
%\caption{Abbreviations of cited sources}\label{tab:your_label_here} \\
\textbf{Reference} & \textbf{Details} \\
\endfirsthead

\multicolumn{2}{c}%
{} \\
\textbf{Reference} & \textbf{Details} \\
\endhead

\multicolumn{2}{r}{} \\
\endfoot


\endlastfoot

% \raggedright{Abbott, \textit{Shakespearian Grammar}} & Edwin Abbott. \textit{A Shakespearian Grammar: An Attempt to Illustrate Some of the Differences between Elizabethan and Modern English}. London, 1894. \\ Converted to BibTeX
Ade, \textit{Artie} & George Ade. \textit{Artie}. (1896) Chicago, 1897. \\

Ælfric, \textit{Homilies} & Ælfric of Eynsham. A collection of vernacular Old English homilies, written around 990 to 995. \\

Alexander, \textit{Johnny} & William Alexander. \textit{Johnny Gibb of Gushetneuk}. Edinburgh, 1897. \\ 
% \raggedright{Alford, \textit{Queen's English}} & Henry Alford. \textit{The Queen's English: A manual of idiom and usage}. 8th ed. London, 1889. \\ Converted to BibTeX

Allen, \textit{Æsthetics} & Grant Allen. ``\textit{Physiological æsthetics} and \textit{Philistia}.'' In Walter Besant et al., \textit{My first book}, 43--52. London, 1894. \\
Allen, \textit{Woman} & Grant Allen. \textit{The woman who did}. Tauchnitz, 1895. \\

Andersen, \textit{Baronesser} & Hans Christian Andersen. \textit{De to Baronesser}. (1848) \\
\raggedright{Andersen, \textit{Improvisatoren}} & Hans Christian Andersen. \textit{Improvisatoren}. (1835) \\
Andersen, \textit{O. T.} & Hans Christian Andersen. \textit{O. T.} (1836) \\ % The abbreviation "O. T." is the actual title.

\textit{Andreas} & \textit{Andreas}. Anonymous Old English poem, found in the Vercelli Book manuscript, dated to the second half of the 10th century. In George Philip Krapp (ed.), \textit{``Andreas'' and ``The fates of the apostles'': Two Anglo-Saxon narrative poems}. Boston, 1906.\\  % "Andr.", cited together with ``Gu.''
% \raggedright{Andresen, \textit{Sprachgebrauch und sprachrichtigkeit}} & Karl Andresen. \textit{Sprachgebrauch und Sprachrichtigkeit im Deutschen}. 7th ed. Leipzig, 1892. \\ Converted to BibTeX

\textit{Apollonius} & \textit{Apollonius of Tyre}, an anonymous Old English prose translation of the Latin \textit{Historia Apollonii regis Tyri}, surviving in the 11th-century manuscript Cambridge, Corpus Christi College, 201.\\

Arnskov, \textit{Nielsen} & L. Th. Arnskov. ``Anders Nielsen.'' \textit{Tilskueren} 1914. 29--44. \\

Austen, \textit{Emma} & Jane Austen. \textit{Emma}. (1816) Tauchnitz, 1877. \\
Austen, \textit{Mansfield} & Jane Austen. \textit{Mansfield Park}. (1814) London, 1897. \\
Austen, \textit{Pride} & Jane Austen. \textit{Pride and prejudice}. (1813) London, 1894. \\
Austen, \textit{Sense} & Jane Austen. \textit{Sense and sensibility}. (1811) London, n.d. \\

AV \textit{1 Corinthians} & \textit{First epistle to the Corinthians}. \textit{The holy bible}. (1611) \\
AV \textit{Job} & \textit{Book of Job}. \textit{The holy bible}. (1611) \\
AV \textit{John} & \textit{Gospel of John}. \textit{The holy bible}. (1611) \\
AV \textit{Matthew} & \textit{Gospel of Matthew}. \textit{The holy bible}. (1611) \\
AV \textit{Psalms} & \textit{Book of psalms}. \textit{The holy bible}. (1611) \\

Baggesen, \textit{Værker} & Jens Baggesen. \textit{Jens Baggesens danske værker}. Copenhagen, 1845. \\

G. Bang, \textit{Husmanden} & Gustav Bang. ``Husmanden.'' \textit{Tilskueren}. 1902. 372--388. \\

H. Bang, \textit{Fædra} &  Herman Bang. \textit{Fædra. Brudstykker af et livs historie}. (1883) \\
H. Bang, \textit{Ludvigsbakke} & Herman Bang. \textit{Ludvigsbakke}. (1896) \\
H. Bang, \textit{Slægter} & Herman Bang. \textit{Haabløse slægter}. (1880) \\

Barrie, \textit{Margaret} & James M. Barrie. \textit{Margaret Ogilvy}. Tauchnitz, 1897. \\

\textit{Bede} & The 9th-century Old English translation, traditionally attributed to King Alfred, of Bede's 8th-century work \textit{Historia ecclesiastica gentis Anglorum}. \\

Behn, \textit{Mistake} & Aphra Behn. \textit{The lucky mistake}. (1689) In Ernest A. Baker (ed.), \textit{The novels}. London, 1904. \\

\textit{Bemærkninger} & Oskar Siesbye, Kristian Mikkelsen \& Otto Jespersen. ``Bemærkninger til afhandlingen `En sproglig værdiforskydning'.'' \textit{Dania} 10 (1896). 239--258. \\

Bennett, \textit{Anna} & Arnold Bennett. \textit{Anna of the five towns}. (1902) London, 1912. \\
Bennett, \textit{Babylon} & Arnold Bennett. \textit{The Grand Babylon Hotel}. (1902) London, 1912. \\ % MEG 7 garbles the title.
Bennett, \textit{Card} & Arnold Bennett. \textit{The card}. (1911) London, 1913. \\
Bennett, \textit{Clayhanger} & Arnold Bennett. \textit{Clayhanger}. (1910) Tauchnitz, 1912. \\
Bennett, \textit{Twain} & Arnold Bennett. \textit{These twain}. London, 1916. \\
Bennett, \textit{Wives} & Arnold Bennett. \textit{The old wives' tale}. (1908) Tauchnitz, 1909. \\

E. F. Benson, \textit{Arundel} & E. F. Benson. \textit{Arundel}. (1914) London, 1915. \\
E. F. Benson, \textit{Dodo} & E. F. Benson. \textit{Dodo: A detail of the day}. (1893)  Tauchnitz, 1894.\\
E. F. Benson, \textit{Judgment} & E. F. Benson. \textit{The judgment books}. London, 1895. \\
E. F. Benson, \textit{Second} & E. F. Benson. \textit{Dodo the second}. (1913) Tauchnitz. \\ 

R. H. Benson, \textit{None} & Robert Hugh Benson. \textit{None other gods}. (1911) London, n.d. \\

\textit{Beowulf} & \textit{Beowulf}. An Old English epic poem, surviving in the manuscript London, British Library, Cotton Vitellius A.XV, dated to around 1000 CE. \\

Bersezio, \textit{Bolla} & Vittorio Bersezio. \textit{Una bolla di sapone}. Milano, 1870. \\

Birmingham, \textit{Whitty} & George A. Birmingham. \textit{The adventures of Dr. Whitty}. London, 1913.\\ % MEG 7 says 1915; a mistake?

Bjørnson, \textit{Flager} & Bjørnstjerne Bjørnson. \textit{Det flager i byen og på havnen}. (1884)\\
Bjørnson, \textit{Guds} & Bjørnstjerne Bjørnson. \textit{På guds veje}. (1889) \\

Black, \textit{Fortunatus} & William Black. \textit{The new Prince Fortunatus}. Tauchnitz, 1890. \\
Black, \textit{Phaeton} & William Black. \textit{The strange adventures of a phaeton}. (1872) London, n.d. \\

Blicher & Steen Steensen Blicher. \\ % ??? OJ refers to this as "Blicher"; it's still UNIDENTIFIED
Blicher, \textit{Bindstouw} & Steen Steensen Blicher. \textit{E Bindstouw: Fortællinger og digte i jydske mundarter}. (1842) \\

\textit{Boethius} & The 9th-century translation, traditionally attributed to King Alfred, of Boethius's \textit{De consolatione philosophiae} (On the consolation of philosophy). \\

Bøgholm, \textit{???} & Niels Bøgholm. ``???.'' \textit{Anglia} n. f. 26. \\ % ??? Title of the article is still unknown. This is cited once, in chapter 7.
% Not to be confused with \raggedright{Bøgholm, \textit{Bacon og Shakespeare}} & Niels Bøgholm. \textit{Bacon og Shakespeare: En sproglig sammenligning}. Copenhagen, 1906.\\ Now handled by BibTeX

Boswell, \textit{Life}\textsubscript{A} & James Boswell. \textit{The life of Samuel Johnson, LL.D.} 1740--1795.\\
Boswell, \textit{Life}\textsubscript{B} & James Boswell. \textit{The life of Samuel Johnson.} Edited by Roger Ingpen. London, 1907.\\ % PE: One edition of Boswell's Life of SJ appears as a BibTeX reference; I haven't yet checked to see if this is the same as either of these two editions. Incidentally, the second is not a very satisfying reference (given the context); something published during Boswell's lifetime would be preferable. I've searched, but I haven't found.

Bradley, \textit{Tragedy} & A. C. Bradley. \textit{Shakespearean tragedy}. London, 1904. \\

E. Brandes, \textit{Lykkens} & Edvard Brandes. \textit{Lykkens blændværk fortælling}. 1898. \\

G. Brandes, \textit{Napoleon} & Georg Brandes. ``Napoleon.'' \textit{Tilskueren} 1915. 32--60. \\ % Tilskueren is a journal.

Brazil, \textit{School} & Angela Brazil. \textit{For the sake of the school}. (1915) \\
%
% Bréal, \textit{Essai de semantique} & Michel Bréal. \textit{Essai de semantique}. Paris, 1897. \\ This isn't cited anywhere.
% Bréal, \textit{Semantics} & Michel Bréal. \textit{Semantics: Studies in the science of meaning}. New York, 1900. \\ Now cited via BibTeX

Brontë, \textit{Jane} & Charlotte Brontë. \textit{Jane Eyre}. London, 1847. \\
Brontë, \textit{Professor} & Charlotte Brontë. \textit{The professor}. London, 1867. \\

Brooke, \textit{Poems} & Rupert Brooke. \textit{Poems}. London, 1911. \\ 

E. B. Browning, \textit{Aurora} & Elizabeth Barrett Browning. \textit{Aurora Leigh}. (1856) Tauchnitz, n.d. \\ % OJ refers to her as "Mrs. Browning"

R. Browning, \textit{Andrea} & Robert Browning. ``Andrea del Sarto.'' (1855) \\
R. Browning, \textit{Italian} & Robert Browning. ``The Italian in England.'' (1845) \\
R. Browning, \textit{Lippo} & Robert Browning. ``Fra Lippo Lippi.'' (1855) \\
R. Browning, \textit{Pompilia} & Robert Browning. ``Pompilia.'' In \textit{The ring and the book}. (1869) \\
R. Browning, \textit{Rabbi} & Robert Browning. ``Rabbi Ben Ezra.'' (1864) \\
R. Browning, \textit{Ride} & Robert Browning. ``The last ride together.'' (1855) \\
%
% Bruchmann, \textit{Studien} & Kurt Bruchmann. \textit{Psychologische Studien zur Sprachgeschichte}. Leipzig, 1888. \\ % Now handled by BibTeX

Buchanan, \textit{Anthony} & Robert Buchanan. \textit{Father Anthony}. (1898) New York, 1900. \\ % MEG 7 says London, n.d.

Bunyan, \textit{Grace} & John Bunyan. \textit{Grace abounding to the chief of sinners}. (1666) Edited by John Brown. Cambridge, 1907. \\
Bunyan, \textit{Progress} & John Bunyan. \textit{The pilgrim's progress}. London, 1678. \\

Burns, \textit{Dogs} & Robert Burns. ``The twa dogs.'' (1786) \\
Burns, \textit{Man} & Robert Burns. ``A man's a man for all that.'' (1795) \\

Byron, \textit{Cain} & George Byron. \textit{Cain: A mystery}. (1821) \\
Byron, \textit{Childe Harold} & George Byron. \textit{Childe Harold's pilgrimage}. (1812--1818) \\
Byron, \textit{Juan} & George Byron. \textit{Don Juan}. (1819–1824) \\
Byron, \textit{Manfred} & George Byron. \textit{Manfred: A dramatic poem}. (1817) \\
Byron, \textit{Mazeppa} & George Byron. \textit{Mazeppa}. (1819) \\
Byron, \textit{Sardanapalus} & George Byron. \textit{Sardanapalus: A tragedy}. (1821) \\

Caine, \textit{Christian} & Hall Caine. \textit{The Christian}. London, 1897.  \\
Caine, \textit{City} & Hall Caine. \textit{The eternal city}. London, 1901. \\
Caine, \textit{Manxman} & Hall Caine. \textit{The Manxman}. London, 1894. \\
Caine, \textit{Prodigal} & Hall Caine. \textit{The prodigal son}. London, 1904. \\

Calderón, \textit{Alcalde} & Pedro Calderón de la Barca. \textit{El Alcalde de Zalamea}. (1636) \\

J. Carlyle, \textit{Letters} & James Anthony Froude (ed.), \textit{Letters and memorials of Jane Welsh Carlyle}. London, 1883. \\ % OJ refers to her/this as "Mrs. Carlyle F."

T. Carlyle, \textit{Heroes} & Thomas Carlyle. \textit{On heroes, hero-worship, and the heroic in history}. (1841) London, 1890. \\
T. Carlyle, \textit{Life} & Thomas Carlyle; in James Anthony Froude, \textit{Life of Carlyle}. (Vols 1 \& 2: \textit{Thomas Carlyle: A history of the first forty years of his life.} Vols 3 \& 4: \textit{Thomas Carlyle: A history of his life in London.}) London, 1882/1884. \\
\raggedright{T. Carlyle, \textit{Reminiscences}} & Thomas Carlyle. \textit{Reminiscences}. Edited by James Anthony Froude. London, 1881. \\
T. Carlyle, \textit{Revolution} & Thomas Carlyle. \textit{The French revolution}. (1837) London. \\
T. Carlyle, \textit{Sartor} & Thomas Carlyle. \textit{Sartor resartus}. (1839) London, n.d. \\

Carpenter, \textit{Teaching} & George R. Carpenter, Franklin T. Baker, \& Fred N. Scott. \textit{The teaching of English in the elementary and the secondary school}. New York, 1913.\\

Cauer, \textit{Grammatica militans} & Paul Cauer. \textit{Grammatica militans: Erfahrungen und Wünsche im Gebiete des lateinischen und griechischen Unterrichtes}. Berlin, 1903. \\ % PE: This is also (properly) cited via BibTeX; but as it's cited as the source of an example or two I suppose that it has to be cited in both ways. 

Caxton, \textit{Reynard} & William Caxton. \textit{The history of Reynard the fox}. (1481) Edited by Edward Arber. London, 1880.  \\

Chaucer, \textit{Clerkes} & Geoffrey Chaucer. ``The clerkes tale.'' (1387--1400) In Walter W. Skeat (ed.), \textit{The Canterbury tales: Text}. Oxford, 1894. \\ % OJ refers to this as "E"
Chaucer, \textit{Duchesse} & Geoffrey Chaucer. ``The book of the duchesse.'' (1368--72) In Walter W. Skeat (ed.), \textit{The minor poems}. Oxford, 1896. \\ % OJ refers to this as "Duch"
Chaucer, \textit{Lawe} & Geoffrey Chaucer. ``The tale of the man of lawe.'' (c.1387) In Walter W. Skeat (ed.), \textit{The Canterbury tales: Text}. Oxford, 1894. \\ % OJ refers to this as "B"
Chaucer, \textit{Melibeus} & Geoffrey Chaucer. ``The tale of Melibeus.'' (1387--1400) In Walter W. Skeat (ed.), \textit{The Canterbury tales: Text}. Oxford, 1894. \\ % OJ refers to this as "B"
\raggedright{Chaucer, \textit{Miller's prologue}} & Geoffrey Chaucer. ``The miller's prologue.'' (1387--1400) In Walter W. Skeat (ed.), \textit{The Canterbury tales: Text}. Oxford, 1894.  \\ % OJ refers to this as "A"
Chaucer, \textit{Pardoners} & Geoffrey Chaucer. ``The pardoners tale.'' (1387--1400) In Walter W. Skeat (ed.), \textit{The Canterbury tales: Text}. Oxford, 1894. \\ % OJ refers to this as "C"
Chaucer, \textit{Prologue} & Geoffrey Chaucer. ``The prologue.'' (1387--1400) In Walter W. Skeat (ed.), \textit{The Canterbury tales: Text}. Oxford, 1894. \\ % OJ refers to this as "A"

Cicero, \textit{De oratore} & Marcus Tullius Cicero. \textit{De oratore}. (55 BCE)\\

Coleridge, letter & Samuel Taylor Coleridge. Letter of 21 April 1800. \\

Collitz, \textit{Präteritum} & Hermann Collitz. \textit{Das schwache Präteritum und seine Vorgeschichte}. Göttingen, 1912.\\

Congreve, \textit{Dealer} & William Congreve. \textit{The double-dealer}. (1693) In \textit{Works}, 6th edn., vol.~1. The Hague, 1753.\\
Congreve, \textit{Love} & William Congreve. \textit{Love for love}. (1695) \\

Conway, \textit{Called} & Hugh Conway [F. J. Fargus]. \textit{Called back}. (1883) Tauchnitz, 1884.\\ %
% Cooper, \textit{Grammatica} & Christopher Cooper. \textit{Grammatica Linguæ Anglicanæ}. London, 1685. \\ Now cited via BibTeX

Cowper, letter & William Cowper. Letter. In  \textit{Letters}, edited by J.~G. Frazer. London, 1912.\\ % More than one are referred to; each reference is dated
Cowper, \textit{Task} & William Cowper. \textit{The task}. (1785) In \textit{Poetical works}. London, 1889. \\

Darwin, \textit{Expression} & Charles Darwin. \textit{The expression of the emotions in man and animals}. London, 1872.  \\
Darwin, \textit{Life} & Charles Darwin. In Francis Darwin (ed.), \textit{The life and letters of Charles Darwin}. London, 1887.  \\

Daudet, \textit{Numa} & Alphonse Daudet. \textit{Numa Roumestan}. (1880) \\
Daudet, \textit{Sapho} & Alphonse Daudet. \textit{Sapho}. Paris, 1884. \\
Daudet, \textit{Tartarin} & Alphonse Daudet. \textit{Tartarin sur les Alpes}. (1885) \\

Defoe, \textit{Farther} & Daniel Defoe. \textit{The farther adventures of Robinson Crusoe}.  London, 1719. \\
Defoe, \textit{Gentleman} & Daniel Defoe. \textit{The compleat English gentleman}. (Written 1720s.) Edited by Karl D. Bülbring. London, 1890.  \\ % First published in 1890
Defoe, \textit{Journal} & Daniel Defoe. \textit{A journal of the plague year}. (1722) Edited by E. W. Brayley. London, n.d. \\
Defoe, \textit{Robinson} & Daniel Defoe. \textit{Robinson Crusoe}. (1719) Facsimile edn. London, 1883. \\

Dekker, \textit{Sins} & Thomas Decker. \textit{The seven deadly sins of London}. (1606) Edited by Edward Arber. London, 1879. \\ % "Dekker" is how the surname is I think more commonly spelt (for this man) and how OJ spells it. But "Decker" is how Arber spells it.
%
% \raggedright {Delbrück, \textit{Germanische Syntax} I. \textit{Zu den negativen Sätzen}} & Berthold Delbrück. \textit{Germanische Syntax} I.~\textit{Zu den negativen Sätzen.} Abhandlungen der Königlich-Sächsischen Gesellschaft der Wissenschaften. Leipzig, 1910.\\ Now cited via BibTeX
%\raggedright {Delbrück, \textit{Vergleichende Syntax}}. & Berthold Delbrück. \textit{Vergleichende Syntax der Indogermanischen Sprachen}. Strassburg, 1893. \\ Now cited via BibTeX

\raggedright {Deutschbein, \textit{System der neuenglischen syntax}} & Max Deutschbein. \textit{System der neuenglischen Syntax}. Cöthen, 1917. \\

\textit{Devil} & \textit{The merry devil of Edmonton}. (1592--1604) In \textit{Representative English comedies}, vol.~2, \textit{Later contemporaries of Shakespeare}. Edited by Charles Mills Gayley. New York, 1913. \\ % OJ refers to this as "Devil E" or "Devil Edm.". LCS is vol 2 of the 3-vol REC.

Dewey, \textit{School} & John Dewey. \textit{The school and society}. (1899) \\

\textit{DgF} & \textit{Danmarks gamle folkeviser}. (Collection of old Danish popular ballads, already a multivolume work by 1917, and further incremented thereafter.) \\

Dickens, \textit{Carol} & Charles Dickens. \textit{A Christmas carol}. (1843) In \textit{Christmas books}. London, 1892. \\ % OJ refers to this as "X"
Dickens, \textit{Chimes} & Charles Dickens. \textit{The chimes}. (1844) In \textit{Christmas books}. London, 1892. \\ % OJ refers to this as "X"
Dickens, \textit{Cricket} & Charles Dickens. \textit{The cricket on the hearth}. (1845) In \textit{Christmas books}. London, 1892. \\
Dickens, \textit{David} & Charles Dickens. \textit{David Copperfield}. (1849--50) London, 1897. \\
Dickens, \textit{Dombey} & Charles Dickens. \textit{Dombey and son}. (1848) London, 1887. \\
Dickens, \textit{Friend} & Charles Dickens. \textit{Our mutual friend}. (1865) London, 1912. \\
Dickens, \textit{Martin} & Charles Dickens. \textit{The life and adventures of Martin Chuzzlewit}. (1843) London, n.d. \\
Dickens, \textit{Nicholas} & Charles Dickens. \textit{The life and adventures of Nicholas Nickleby}. (1839) London, 1900. \\

Dickinson, \textit{Letters} & G. Lowes Dickinson. \textit{Letters from John Chinaman}. (1901) London, 1904. \\
Dickinson, \textit{Symposium} & G. Lowes Dickinson. \textit{A modern symposium}. (1905) London, 1906. \\
Dickinson, \textit{War} & G. Lowes Dickinson. \textit{After the war}. (1915) \\

Disraeli, \textit{Lothair} & Benjamin Disraeli. \textit{Lothair}. (1870) London, n.d. \\ % OJ refers to him/this as "Beaconsfield L."; I (PE) think that as an author he's more commonly known as Disraeli

Dobson, \textit{Fielding} & Austin Dobson. \textit{Fielding}. (1883) London, 1889.  \\ % In the example I (PE) looked at and link to (which isn't the same edition as the one OJ quoted: their page numbers are different), the title page says, simply, "Fielding", but the front cover says "Henry Fielding".

Doyle, \textit{Hound} & Arthur Conan Doyle. \textit{The hound of the Baskervilles}. Tauchnitz, 1902. \\
Doyle, \textit{How} & Arthur Conan Doyle. ``How the King held the brigadier.'' \textit{The exploits of Brigadier Gerard}.  \textit{The Strand Magazine}. 9 (1895). 501--514. \\ % OJ refers to this as "Doyle NP. 1895"
Doyle, \textit{Letters} & Arthur Conan Doyle. \textit{The Stark Munro letters}. Tauchnitz, 1895.  \\
Doyle, \textit{Memoirs} & Arthur Conan Doyle. \textit{The memoirs of Sherlock Holmes}. (1894) Tauchnitz. \\ % OJ refers to this as "S. 4"
Doyle, \textit{Return} & Arthur Conan Doyle. \textit{The return of Sherlock Holmes}. (1905) Tauchnitz. \\ % OJ refers to this as "S. 5"

Drachmann, \textit{Forskrevet} & Holger Drachmann. \textit{Forskrevet}. (1890) \\
Drachmann, \textit{Kitzwalde} & Holger Drachmann. \textit{Kitzwalde}. (1895) \\

Droz, \textit{Monsieur} & Gustave Droz. \textit{Monsieur, madame et bébé}. (1866) Paris, 1882.\\

Dryden, \textit{All} & John Dryden. \textit{All for love}. (1677) \\
Dryden, \textit{Aureng-Zebe} & John Dryden. \textit{Aureng-Zebe}. (1675) \\
%
%\raggedright{Dyboski, \textit{Tennysons sprache}} & Roman Dyboski. \textit{Tennysons Sprache und Stil}. Vienna, 1907. \\ Now cited via BibTeX

\textit{Eastward} & George Chapman, Ben Jonson \& John Marston. \textit{Eastward hoe}. (1605) In Charles Mills Gayley (ed.), \textit{Representative English comedies}, vol.~2, \textit{The later contemporaries of Shakespeare}. New York, 1913. \\ % LCS is vol 2 of the 3-vol REC.

Egerton, \textit{Keynotes} & George Egerton [Mary Dunne Bright]. \textit{Keynotes}. London, 1893. \\
%
% \raggedright {Einenkel, \textit{Die englische Verbalnegation}} & Eugen Einenkel. ``Die englische Verbalnegation, ihre Entwickelung, ihre Gesetze und ihre zeitlich-örtliche Verwendung.'' \textit{Anglia. Zeitschrift für englische Philologie}. 35 (1911). 187--, 401--424. \\ % ??? From 187 to which page? Not seen, so we don't know. Now cited via BibTeX

Eliot, \textit{Adam} & George Eliot. \textit{Adam Bede}. (1859) London, 1900. \\
Eliot, \textit{Mill} & George Eliot. \textit{The mill on the floss}. (1860) Tauchnitz. \\
Eliot, \textit{Silas} & George Eliot. \textit{Silas Marner}. (1861) Tauchnitz. \\
%
% Ellis, \textit{Relation} & Alexander J. Ellis. ``On the Relation of Thought to Sound as the Pivot of Philological Research.'' \textit{Transactions of the Philological Society}, 1873--74. 3--34. \\ % Surprisingly, the volume number isn't visible in https://books.google.co.jp/books?id=77UKAAAAIAAJ  Now cited via BibTeX
%
% Elphinston, \textit{Principles} & James Elphinston. \textit{The Principles of the English Language Digested}. London, 1765. \\  Now cited via BibTeX

Emerson, \textit{Traits} & Ralph Waldo Emerson. \textit{English traits}. (1856) \\
%
% \raggedright{\textit{English Dialect Dictionary}} & Joseph Wright. \textit{The English Dialect Dictionary}. London, 1898--1905. \\ Converted to BibTeX (under "Wright")

Faber, \textit{Stegekjælderen} & Peter Faber. ``Stegekjælderen eller Den fine verden.'' (1849) Copenhagen, 1883. \\ % This is an individual song

Farmer \& Henley & John S. Farmer \& W. E. Henley. \textit{Slang and its analogues past and present}. 1890–1904. \\ % PE: Currently cited in both ways; I suppose that this must continue

Farquhar, \textit{Stratagem} & George Farquhar. \textit{The beaux' stratagem}. (1707) In \textit{Restoration plays}. 1912.  \\

Feilberg, \textit{Snaps} & Henning Frederik Feilberg. ``Den fattige mands snaps.'' \textit{Dania}. 5.88--123. \\ % A different work by Feilberg is cited via BibTeX

Fibiger, \textit{Liv} & Johannes Fibiger. \textit{Mit liv og levned, som jeg selv har forstaaet det}. Edited by Karl Gjellerup. Copenhagen, 1898. \\

Fielding, \textit{Jonathan} & Henry Fielding. \textit{The history of the life of the late Mr. Jonathan Wild the great}. (1743) \\ % ??? Is this the right form of the title?
Fielding, \textit{Joseph} & Henry Fielding. \textit{Joseph Andrews}. (1742) \\
Fielding, \textit{Tom} & Henry Fielding. \textit{The history of Tom Jones, a foundling}. (1749) London, 1782. \\
Fielding, \textit{Tragedy} & Henry Fielding. \textit{The tragedy of tragedies}. (1731) \\

\raggedright {Franklin, \textit{Autobiography}} & William MacDonald (ed.), \textit{The autobiography of Benjamin Franklin}. London, 1905. \\
%
% \raggedright {Gabelentz, \textit{Weiteres zur vergleichenden Syntax}} & Georg von der Gabelentz. ``Weiteres zur vergleichenden Syntax: Wort- und Satzstellung''. \textit{Zeitschrift für Völkerpsychologie und Sprachwissenschaft} 8 (1875). 129--165. \\ Converted to BibTeX (as "von der Gabelentz")

Galsworthy, \textit{Box} & John Galsworthy. \textit{The silver box: A comedy in three acts}. (1906) London, 1910. \\
Galsworthy, \textit{Flower} & John Galsworthy. \textit{The dark flower}. Tauchnitz, 1913.  \\
Galsworthy, \textit{Freelands} & John Galsworthy. \textit{The Freelands}. (1915) London, 1916. \\
Galsworthy, \textit{Joy} & John Galsworthy. \textit{Joy: A play on the letter ``I''}. (1907) \\ % OJ refers to this as "P 2" 
Galsworthy, \textit{Justice} & John Galsworthy. \textit{Justice: A tragedy in four acts}. (1910) \\ % OJ refers to this as "P 6" 
Galsworthy, \textit{Man} & John Galsworthy. \textit{The man of property}. (1906) \\
Galsworthy, \textit{Motley} & John Galsworthy. \textit{A motley}. Tauchnitz, 1910. \\
Galsworthy, \textit{Strife} & John Galsworthy. \textit{Strife: A drama in three acts}. (1909) \\ % OJ refers to this as "P 3" 

\textit{Gammer} & \textit{Gammer Gurtons nedle}. (1575) In John Matthews Manly (ed.), \textit{Specimens of the pre-Shaksperean drama}. Boston, 1897. \\ % Yes, "Shaksperean" is how it's spelt.

\raggedright{Garborg, \textit{Bondestudentar}} & Arne Garborg. \textit{Bondestudentar}. Bergen, 1883. \\

Giellerup, \textit{Minna} & Karl Gjellerup. \textit{Minna}. Copenhagen, 1889. \\ % OJ calls him Giellerup; everyone else seems to call him Gjellerup. Even this 1889 edition of this book calls him Gjellerup.
Giellerup, \textit{Romulus} & Karl Gjellerup. \textit{Romulus}. (1889) \\

Gilbert, \textit{Charity} & W. S. Gilbert. \textit{Charity}. (1874) In \textit{Original plays}. London, 1884. \\

Gissing, \textit{Born} & George Gissing. \textit{Born in exile}. London, 1892. \\
Gissing, \textit{Grub} & George Gissing. \textit{New Grub Street}. (1891) London, 1908. \\
Gissing, \textit{Henry} & George Gissing. \textit{The private papers of Henry Ryecroft}. (1903) London, 1912. \\

\raggedright {Goethe, \textit{Faust}~I} & Johann Wolfgang von Goethe. \textit{Faust. Der Tragödie erster Teil}. (1808) \\
Goethe, \textit{Lehrjahre} & Johann Wolfgang von Goethe. \textit{Wilhelm Meisters Lehrjahre}. (1795–96) \\
Goethe, letter & Johann Wolfgang von Goethe. Letter (21/5936) to Behrendt, 21 March 1810. \\ % Must the number be explained?

Goldschmidt, \textit{Hjemløs} & Meïr Aron Goldschmidt. \textit{Hjemløs}. (1853) \\
Goldschmidt Kol. 92. & ??? UNIDENTIFIED \\ % In chapter 4 (p.35 of the printed book)
Goldschmidt, \textit{Levi} & ``Levi og Ibald.'' \textit{M. Goldschmidts poetiske skrifter, udgivene af Hans Søn}. Vol. 8. Copenhagen, 1898. \\ 
Goldschmidt, \textit{Ravnen} & Meïr Aron Goldschmidt. \textit{Ravnen}. Copenhagen, 1867. \\ % OJ refers to both this edition and a different one. Since I (PE) can't identify the other but can identify (and link to) this one, I've standardized on this one.

Goldsmith, \textit{Citizen} & Oliver Goldsmith. \textit{The citizen of the world}. (1760--62) \\
\raggedright {Goldsmith, \textit{Good-natur'd}} & Oliver Goldsmith. \textit{The good-natur'd man}. (1768) \\
Goldsmith, \textit{Stoops} & Oliver Goldsmith. \textit{She stoops to conquer}. (1773) \\
Goldsmith, \textit{Vicar} & Oliver Goldsmith. \textit{The vicar of Wakefield}. (1766) Facsimile edn. London, 1885. \\

Goncourt, \textit{Germinie} & Edmond et Jules de Goncourt. \textit{Germinie Lacerteux}. Paris, 1864. \\ % fr:Wikipedia says that "date de parution" of this novel was 1865, but the title page says 1864

Gosse, \textit{History} & Edmund Gosse. \textit{A short history of modern English literature}. (1897) London, 1908. \\

Gravlund, \textit{Kristrup} & Thorkild Gravlund. ``Kristrup ved Randers.'' \textit{Danske Studier} (1909). 85--103. \\ % There's no volume number. OJ refers to this as "Da. Studier 1909" (and doesn't name the author)
% Grein, \textit{Sprachschatz} & Christian Wilhelm Michael Grein (with Ferdinand Holthausen). \textit{Sprachschatz der angelsächsischen Dichter}. Edited by Johann Jakob Köhler. Heidelberg, 1912. \\ Now cited via BibTeX
% \raggedright {Grimm, \textit{Personenwechsel in der Rede}} & Jacob Grimm. \textit{Über den Personenwechsel in der Rede}. Berlin, 1856.\\ Now cited via BibTeX

\raggedright {Grundtvig, \textit{Folkeæventyr}} & Svend Grundtvig. \textit{Danske folkeæventyr}. (1876–83) \\
%
% \raggedright{Guinness, \textit{Mosaic History and Gospel Story Epitomised in the Congo Language}} &  H. Grattan Guinness. \textit{Mosaic History and Gospel Story Epitomised in the Congo Language}. London, 1882. \\  Now cited via BibTeX

Halévy, \textit{Notes} & Ludovic Halévy. \textit{Notes et souvenirs 1871--1872}. Paris, 1889. \\
%
% Hall, \textit{Modern English} & Fitzedward Hall. \textit{Modern English}. New York, 1873. \\  Now cited via BibTeX

Hallam, letter & Henry Hallam, letter to Alfred Tennyson, in Hallam Tennyson, \textit{Alfred Lord Tennyson: A memoir by his son}. \\ % OJ refers to this as "Tennyson L"
%
%\raggedright {Hanssen, \textit{Spanische Grammatik}} & Friedrich Hanssen. \textit{Spanische Grammatik auf historischer Grundlage}. Halle, 1910. \\  Now cited via BibTeX

Hardy, \textit{Far} & Thomas Hardy. \textit{Far from the madding crowd}. (1874) London, 1906. \\
Hardy, \textit{Ironies} & Thomas Hardy. \textit{Life's little ironies}. (1894) London, 1908. \\
Hardy, \textit{Return} & Thomas Hardy. \textit{The return of the native}. (1878) London, 1912. \\
Hardy, \textit{Tess} & Thomas Hardy. \textit{Tess of the d'Urbervilles}. (1891) London, 1892. \\
Hardy, \textit{Wessex} & Thomas Hardy. \textit{Wessex tales}. London, 1889. \\

Harraden, \textit{Fowler} & Beatrice Harraden. \textit{The fowler}. London, 1899.  \\
Harraden, \textit{Ships} & Beatrice Harraden. \textit{Ships that pass in the night}. (1893) London.  \\

\raggedright{``Harrison {[}on Mark Pattison{]}''} & ??? UNIDENTIFIED\\

Harrison, \textit{Ruskin} & Frederic Harrison. \textit{John Ruskin}. London, 1902. \\

Hawthorne, \textit{Image} & Nathaniel Hawthorne. \textit{The snow image and other twice-told tales}. (1851) New York, n.d. \\
Hawthorne, \textit{Wonder} & Nathaniel Hawthorne. \textit{A wonder book for girls and boys}. (1851) \\ % OJ calls this “T”, misattributing the sole example to Tanglewood Tales

Hay, \textit{Breadwinners} & John Hay. \textit{The breadwinners}. Tauchnitz, 1883. \\ % Also titled "The Bread-winners" (with hyphen).

Hazlitt, \textit{Liber} & William Hazlitt. \textit{Liber amoris}. London, 1823. \\
%
% \raggedright {Hein, ``Über die bildliche verneinung''} & J. Hein. ``Über die bildliche Verneinung in der mittelenglischen Poesie.'' \textit{Anglia} 15 (1893). 41--186, 396--472. \\  Now cited via BibTeX

Henley, \textit{Beau} & W. E. Henley \& Robert Louis Stevenson. \textit{Beau Austin}. (1884) London, 1897. \\
Henley, \textit{Burns} & William E. Henley \& Thomas F. Henderson. \textit{The poetry of Robert Burns}. Edinburgh, 1897. \\

Henrichsen, \textit{Mændene} & Erik Henrichsen. \textit{Mændene fra forfatningskampen}. (1913--14) \\

Hewlett \textit{Quair} & Maurice Hewlett. \textit{The Queen's quair}. London, 1904. \\

Høffding, \textit{Humor} & Harald Høffding. \textit{Den store humor}. Copenhagen, 1916. \\

Holberg, \textit{Erasmus} & Ludvig Holberg. \textit{Erasmus Montanus}. (1723) \\ % Leave "Montanus" capitalized
Holberg, \textit{Jeppe} & Ludvig Holberg. \textit{Jeppe paa bierget}. (1722) \\
Holberg, \textit{Kandestøber} & Ludvig Holberg. \textit{Den politiske kandestøber}. (1722) \\
Holberg, \textit{Mascarade} & Ludvig Holberg. \textit{Mascarade}. (1724) \\
Holberg, \textit{Pulver} & Ludvig Holberg. \textit{Det arabiske pulver}. (1724) \\
Holberg, \textit{Ulysses} & Ludvig Holberg. \textit{Ulysses von Ithacia}. (1723) \\
%
% \raggedright{Holthausen, \textit{Negation}} & Ferdinand Holthausen. ``Negation statt Vergleichungspartikel beim Komparativ.'' \textit{Indogermanische Forschungen} 32 (1891). p.~333--339.\\  Now cited via BibTeX

Homer, \textit{Odyssey} & Homer. \textit{Odyssey}.  \\ % ??? OJ refers to this simply as "Od.". Which edition?

Hope, \textit{Change} & Anthony Hope. \textit{A change of air}. Tauchnitz, 1893. \\
Hope, \textit{Courtship} & Anthony Hope. \textit{Comedies of courtship}. (1896) \\
Hope, \textit{Dialogues} & Anthony Hope. \textit{The Dolly dialogues}. London, 1894. \\ Dolly is a character's name; leave capitalized
Hope, \textit{Father} & Anthony Hope. \textit{Father Stafford}. (1891) London, 1900. \\
Hope, \textit{Intrusions} & Anthony Hope. \textit{The intrusions of Peggy}. (1902) London, 1907. \\
Hope, \textit{Man} & Anthony Hope. \textit{A man of mark}. (1890) London. \\
Hope, \textit{Quisanté} & Anthony Hope. \textit{Quisanté}. (1900) London. \\
Hope, \textit{Rupert} & Anthony Hope. \textit{Rupert of Hentzau}. Tauchnitz, 1898. \\
Hope, \textit{Zenda} & Anthony Hope. \textit{The prisoner of Zenda}. London, 1894. \\

Horace, \textit{Epistola} & Quintus Horatius Flaccus. ``Epistola I. ad Mæcenatem''. \textit{Epistolarum liber primus}. (c.~20 BCE)\\
Horace, \textit{Odes} & Quintus Horatius Flaccus. \textit{Carmina}. (23 BCE)\\

Hørup & Viggo Hørup. \textit{V. Hørup i skrift og tale, udvalgte artikler og taler}. Copenhagen, 1902--1905. \\

Hostrup, \textit{Gjenboerne} & Jens Christian Hostrup. \textit{Gjenboerne}. (1844) \\

Housman, \textit{John} & Laurence Housman. \textit{John of Jingalo}. London, 1912. (Also titled \textit{King John of Jingalo}.) \\

Howells, \textit{Rise} & William Dean Howells. \textit{The rise of Silas Lapham}. (1884--85) Tauchnitz. \\

Hughes, \textit{Days} & Thomas Hughes. \textit{Tom Brown's school days}. (1861) London, 1886. \\ % OJ refers to this as "T. 1"
Hughes, \textit{Oxford} & Thomas Hughes. \textit{Tom Brown at Oxford}. (1856) London, 1886. \\ % OJ refers to this as "T. 2"

Huxley, letter & Thomas Henry Huxley. Letter, in Leonard Huxley. \textit{Life and letters of Thomas Henry Huxley} (London, 1900), 1.118.\\

Ibsen, \textit{Inger} & Henrik Ibsen. \textit{Fru Inger til Østråt}. (1854) Copenhagen, 1874. \\
Ibsen, \textit{Når} & Henrik Ibsen. \textit{Når vi døde vågner}. (1899) Copenhagen, 1899. \\
Ibsen, \textit{Peer} & Henrik Ibsen. \textit{Peer Gynt}. (1867) Copenhagen, 1891. \\
Ibsen, \textit{Solness} & Henrik Ibsen. \textit{Bygmester Solness}. (1892) Copenhagen, 1892. \\
Ibsen, \textit{Vildanden} & Henrik Ibsen. \textit{Vildanden}. (1884) Copenhagen, 1884. \\

Jacobs, \textit{Lady} & W. W. Jacobs. \textit{The lady of the barge}. London, 1902. \\

Jacobsen, \textit{Fønss} & J. P. Jacobsen. ``Fru Fønss.'' (1882) \\
Jacobsen, \textit{Niels} & J. P. Jacobsen. \textit{Niels Lyhne}. (1880) \\

James, \textit{American} & Henry James. \textit{The American}. (1877) Tauchnitz. \\
James, \textit{Side} & Henry James. \textit{The soft side}. London, 1900.  \\

Jensen, \textit{Bræen} & Johannes V. Jensen. \textit{Bræen}. Copenhagen, 1908.\\

Jerrold, \textit{Lectures} & Douglas Jerrold. \textit{Mrs Caudle's curtain lectures}. (1846) London. \\
%
% Jevons, \textit{Lessons} & Stanley Jevons. \textit{Elementary lessons in logic: Deductive and inductive}. (1871) London, 1893. \\  Now cited via BibTeX

Johnson, \textit{Rasselas} & Samuel Johnson. \textit{History of Rasselas, Prince of Abyssinia}. (1759) Oxford, 1887.  \\
%
%\raggedright{Jones, \textit{Pronouncing dictionary}} & Daniel Jones. \textit{English pronouncing dictionary}. London, 1917. \\ Now cited via BibTeX
% \raggedright{Jones \& Plaatje, \textit{Sechuana Reader}} & Daniel Jones \& Solomon Tshekisho Plaatje. \textit{A Sechuana Reader: In International Phonetic Orthography (with English Translations)}. London, 1916. \\ Now cited via BibTeX

Jonson, \textit{Epicœne} & Ben Jonson. \textit{Epicœne}. (1609--10)\\ % Quoted by OJ (from "B. Jo. 3") with original spelling
Jonson, \textit{Humour} & Ben Jonson. \textit{Every man in his humour}. (1598) \\ % Quoted by OJ (from "B. Jo. 1") with modern spelling
%
% \raggedright{Joyce, \textit{English as We Speak It in Ireland}} & P. W. Joyce. \textit{English as We Speak It in Ireland}. London, 1910. \\ Now cited via BibTeX

Juel-Hansen, \textit{Historie} & Erna Juel-Hansen. \textit{En ung dames historie}. Copenhagen, 1888. \\

Kielland, \textit{Fortuna} & Alexander Kielland. \textit{Fortuna}. Copenhagen, 1884.\\
Kielland, \textit{Jacob} & Alexander Kielland. \textit{Jacob}. (1891) \\

Kierkegaard, \textit{Øieblikket} & Søren Kierkegaard. \textit{Øieblikket Nr. 1–9}. (1855) 3rd ed. Copenhagen, 1895. \\
\raggedright{Kierkegaard, \textit{Smaa-afhandlinger}} & H. H. {[}Søren Kierkegaard{]}. \textit{Tvende ethisk-religieuse smaa-afhandlinger}. (1849) \\
Kierkegaard, \textit{Stadier} & Hilarius Bogbinder {[}Søren Kierkegaard{]}. \textit{Stadier paa livets vei}. (1845) \\

Kinglake, \textit{Eothen} & A. W. Kinglake. \textit{Eothen}. (1844) Edited by D.~G. Hogarth \& Vere H. Collins. Oxford, 1914. \\

Kingsley, \textit{Hypatia} & Charles Kingsley. \textit{Hypatia}. (1853) London, n.d. \\
Kingsley, \textit{Yeast} & Charles Kingsley. \textit{Yeast: A problem}. (1851) \\

Kipling, \textit{Ballads} & Rudyard Kipling. \textit{Barrack-room ballads}. (1890) 1892. \\
Kipling, \textit{Jungle} & Rudyard Kipling. \textit{The jungle book}. (1894) 1897. \\
Kipling, \textit{Kim} & Rudyard Kipling. \textit{Kim}. (1901) 1908. \\
Kipling, \textit{Light} & Rudyard Kipling. \textit{The light that failed}. (1890) \\
Kipling, \textit{Seas} & Rudyard Kipling. \textit{The seven seas}. (1896) \\
Kipling, \textit{Second} & Rudyard Kipling. \textit{The second jungle book}. (1895) Tauchnitz, 1897. \\
Kipling, \textit{Stalky} & Rudyard Kipling. \textit{Stalky \& Co}. (1899) Tauchnitz.  \\
%
%\raggedright {Knörk, \textit{Die Negation in der altenglischen Dichtung}} & M. Knörk. \textit{Die Negation in der altenglischen Dichtung}. Doctoral dissertation, Kiel, 1907. \\ Now cited via BibTeX

Knudsen, \textit{Urup} & Jakob Knudsen. \textit{Lærer Urup}. (1909) \\
%
% \raggedright {König, \textit{Vers in Shaksperes dramen}} & Goswin König. \textit{Der Vers in Shaksperes Dramen}. Strassburg, 1888. \\ Now cited via BibTeX
% \raggedright{Koppel, \textit{Verbesserungsvorschläge}} & Richard Koppel. \textit{Verbesserungsvorschläge den Erläuterungen und der Textlesung des „Lear“}. Berlin, 1899. \\ Now cited via BibTeX
% \raggedright {Krüger, \textit{Griechische sprachlehre}} & Karl Wilhelm Krüger. \textit{Griechische Sprachlehre für Schulen}. 5th ed. Edited by W. Pökel. Leipzig, 1875. \\ Now cited via BibTeX
% Krummacher, \textit{Notizen} & M. Krummacher. ``Notizen über den Sprachgebrauch Carlyle's.'' \textit{Englische Studien} 6 (1883). 352--397. \\ Now cited via BibTeX
% \raggedright{Kühner, \textit{Ausführliche grammatik der griechischen sprache}} & Raphael Kühner. \textit{Ausführliche Grammatik der griechischen Sprache}. Edited by Bernhard Gerth. Hannover, 1904. \\ Now cited via BibTeX

Kyd, \textit{Spanish} & Thomas Kyd. \textit{The Spanish tragedie}. (1592) In Frederick Boas (ed.), \textit{The works of Thomas Kyd}. London, 1901 \\

La Bruyère, \textit{Caractères} & Jean de La Bruyère. \textit{Les Caractères ou Les Mœurs de ce siècle}. (1688) \\

Lagerlöf, \textit{Berlings} & Selma Lagerlöf. \textit{Gösta Berlings saga}. (1891) \\

Larsen, \textit{Krig} & Karl Larsen. \textit{Under vor sidste krig}. Copenhagen, 1897.\\
Larsen, \textit{Punkt} & Karl Larsen. \textit{Det springende punkt}. Copenhagen, 1911.\\

Lawrence, \textit{Abolition} & C. E. Lawrence. ``The abolition of death.'' \textit{Fortnightly Review} 101 (no. 602, 1917). 326--331. \\

\raggedright{Leonora Christina, \textit{Jammers-minde}} & Leonora Christina Ulfeldt. \textit{Jammers-minde}. (Written 1674, first published 1869.) \\

Lessing, \textit{Emilia} & Gotthold Ephraim Lessing. \textit{Emilia Galotti}. (1772) \\
Lessing, \textit{Nathan} & Gotthold Ephraim Lessing. \textit{Nathan der weise}. (1779, first performed 1783) \\
%
% \raggedright{Levin, \textit{``Dagbladet'' som det danske sprogs ridder}} & Israel Levin. \textit{``Dagbladet'' som det danske Sprogs Ridder}. Copenhagen, 1861. \\  Now cited via BibTeX

Lie, \textit{Sol} & Jonas Lie. \textit{Naar sol gaar ned}. Copenhagen, 1895.\\

Lindsey, \textit{Beast} & Ben B. Lindsey. ``The beast and the jungle.'' \textit{Everybody's Magazine}, volume 21 (1909), 779. % As-yet unidentified issue, some time from September to December of 1909.
Also in chapter 8 (``At work with the children'') of Ben B. Lindsey \& Harvey J. O'Higgins, \textit{The Beast} (New York, 1910). \\

Lippmann, \textit{Speaks} & [Walter Lippmann]. ``America speaks.'' \textit{The New Republic}, 27 January 1917. 340--342. \\

Locke, \textit{Adventure} & William J. Locke. ``The adventure of the kind Mr. Smith.'' (1911) \\
Locke, \textit{Morals} & William J. Locke. \textit{The morals of Marcus Ordeyne}. (1905) New York, 1907. \\ % ??? Or London, 1906?
Locke, \textit{Septimus} & William J. Locke. \textit{Septimus}. (1909) London, 1916. \\
Locke, \textit{Year} & William J. Locke. \textit{The wonderful year}. London, 1916. \\

London, \textit{Martin} & Jack London. \textit{Martin Eden}. (1909) London, 1915. \\
London, \textit{Valley} & Jack London. \textit{The valley of the Moon}. (1913) London, 1914. \\

Luther, \textit{Bücher} & Martin Luther. \textit{Alle Bücher und Schrifften: Vom XXII Jar an, bis auff den Christlichen vnd seligen Abschied aus diesem Leben des Hochlöblichen Herrn Friderichen, Hertzogen vnd Churfürst. zu Sachssen, im Meien des XXV. Jars}. Jena, 1558. \\

Macaulay, \textit{Clive} & Thomas Babington Macaulay. ``Lord Clive.'' (1840 Review of \textit{The life of Robert Lord Clive}, by John Malcolm.) \\
Macaulay, \textit{Milton} & Thomas Babington Macaulay. ``Essay on Milton.'' (1825. Review of Milton, \textit{A treatise on Christian doctrine}, translated by Charles R. Sumner.) \\

Macdonald, \textit{Account} & William Macdonald. ``Some account of Frank\-lin's later life.'' In \textit{The autobiography of Benjamin Franklin}. 205--314. London, 1905. \\

Maclaren, \textit{Days} & Ian Maclaren. \textit{The days of auld lang syne}. London, 1896. \\ % Correcting OJ's "MacLaren"

% \raggedright {Madvig, \textit{Græsk ordføjningslære}} & Johan Nikolai Madvig. \textit{Græsk Ordføjningslære, især for den attiske Sprogform}. Copenhagen. \\ % There was an 1846 (1st) and an 1857 (2nd) edition.  Now cited via BibTeX
Madvig, \textit{Kjönnet} & Johan Nikolai Madvig. \textit{Om Kjönnet i Sprogene især i Sanskrit, Latin og Græsk}. Copenhagen, 1835. \\

Malory, \textit{Morte} & Thomas Malory. \textit{Le morte Darthur}. (Written circa 1470, first printed 1485.) Edited by H.~Oskar Sommer. London, 1889. \\

\textit{Mandeville} & J. O. Halliwell (ed.), \textit{The voiage and travaile of Sir John Maundevile, Kt}. (Written 1357--71.) London, 1883. \\

Marlowe, \textit{Edward} & Christopher Marlowe. \textit{Edward the second}. (c. 1592) Edited by Tucker Brooke. Oxford, 1910. \\
Marlowe, \textit{Faustus} & Christopher Marlowe. \textit{Doctor Faustus}. (c. 1588--92) \textit{Marlowes Werke, historische-kritische Ausgabe}, vol.~2. Edited by Hermann Breymann \& Albrecht Wagner. Heilbronn, 1889. \\
Marlowe, \textit{Tamburlaine} & Christopher Marlowe. \textit{Tamburlaine the great}. (c. 1587--88) \textit{Marlowes Werke, historische-kritische Ausgabe}, vol.~1. Edited by Hermann Breymann \& Albrecht Wagner. Heilbronn, 1885. \\

Marryat, \textit{Peter} & Frederick Marryat. \textit{Peter Simple}. London, 1834. \\

Masefield, \textit{Mercy} & John Masefield. \textit{The everlasting mercy}. (1911) London, 1912. \\

Mason, \textit{Water} & A. E. W. Mason. \textit{Running water}. (1907) London. \\

Matthews, \textit{Son} & Brander Matthews. \textit{His father's son}. New York, 1896. \\

Matthiesen, \textit{Stjærner} & Oscar Matthiesen. \textit{Stjærner og striber}. Copenhagen, 1874. \\ % OJ writes "Stjerner" but the cited book says "Stjærner".

Maupassant, \textit{Bécasse} & Guy de Maupassant. \textit{Contes de la Bécasse}. (1887) \\
Maupassant, \textit{Fort} & Guy de Maupassant. \textit{Fort comme la mort}. (1889) \\
Maupassant, \textit{Vie} & Guy de Maupassant. \textit{Une vie}. (1883. Also titled \textit{L'Humble Vérité}.) \\

McCarthy, \textit{History} & Justin McCarthy. \textit{A history of our own times}. New York, 1880. \\ % OJ misspells the surname as MacCarthy.
McCarthy, \textit{Short history} & Justin McCarthy. \textit{A short history of our own times}. \\ % OJ misspells the surname as MacCarthy. Probably everything in the Short History is also in the (full) History. But both are multivolume works, and there are problems of availability on the web.

%\raggedright {Meillet, \textit{Langue Grecque}} & Antoine Meillet. \textit{Aperçu d'une histoire de la langue grecque}. Paris, 1913. \\  Now cited via BibTeX
\raggedright {Meillet, \textit{Caractères généraux}} & Antoine Meillet. \textit{Caractères généraux des langues germaniques}. Paris, 1917. \\ 

Meredith, \textit{Harrington} & George Meredith. \textit{Evan Harrington}. (1861) London, 1889. \\
Meredith, \textit{Ordeal} & George Meredith. \textit{The ordeal of Richard Feverel}. (1859) London, 1895. \\

Mérimée, \textit{Héritages} & Prosper Mérimée. \textit{Les Deux Héritages ou Don Quichotte}. In \textit{Les Deux Héritages: Suivis de L'Inspecteur général et des Débuts d'un aventurier}. Paris, 1853.  \\

Merriman, \textit{Sowers} & Henry Seton Merriman. \textit{The sowers}. (1896) London, 1905. \\
Merriman, \textit{Vultures} & Henry Seton Merriman. \textit{The vultures}. London, 1902. \\
%
%\raggedright {Mikkelsen, \textit{Ordføjningslære}} & Kristian Mathias Mikkelsen. \textit{Dansk Ordföjningslære med sproghistoriske Tillæg: Håndbog for Viderekomne og Lærere}. Copenhagen, 1911. \\ % Yes, OJ writes "Ordføjnings..." but the title page has "Ordföjnings..."  Now cited via BibTeX
% \raggedright {Misteli, \textit{Sprachbau}} & Franz Misteli. \textit{Charakteristik der hauptsächlichsten Typen des Sprachbaues}. (Vol.~2 of Heymann Steinthal \& Franz Misteli, \textit{Abriss der Sprachwissenschaft}, Berlin 1893.) \\  Now cited via BibTeX
% \raggedright {\textit{Modern English Grammar}} & Otto Jespersen. \textit{A Modern English Grammar on Historical Principles}. Part 1. \textit{Sounds and Spellings}. Part 2. \textit{Syntax. First volume}. Heidelberg, 1909 and 1914. (Publication was completed, in seven volumes, in 1949.) \\

Molbech, letter & Christian K. F. Molbech. Letter. (1855) In Harald Høffding (ed.), \textit{Hans Brøchner og Christian K. F. Molbech: en brevvexling (1845--1875)}. Copenhagen, 1902. \\

E. Møller, \textit{Inderstyre} & Ernst Møller. \textit{Inderstyre: Veje og midler for herredömmet over de indre kræfter og for sjæleligt fremarbejde}. 1914. \\

N. Møller, \textit{Koglerier} & Niels Møller. \textit{Koglerier}. Copenhagen, 1895. \\

More, \textit{Utopia} & Thomas More. \textit{Utopia}. In Ralph Robynson (trans., 1551), J.~H. Lupton (ed.), \textit{The Utopia of Sir Thomas More}. Oxford, 1895. \\

Morris, \textit{News} & William Morris. \textit{News from nowhere}. London, 1908. \\

Mulock, \textit{Halifax} & Dinah Maria Mulock. \textit{John Halifax, gentleman}. (1858) Tauchnitz. \\

Nansen, \textit{Fred} & Peter Nansen. \textit{Guds fred}. (1895) \\
%
% \raggedright {Neckel, \textit{Zu den german\-ischen Negationen}} & G. Neckel. ``Zu den germanischen Negationen.'' \textit{Zeitschrift für vergleichende Sprachforschung auf dem Gebiete der Indogermanischen Sprachen} 45 (1912). 1--23. \\  Now cited via BibTeX
% \textit{New English Dictionary} & \textit{A New English Dictionary on Historical Principles}. 1884--. (In 1933 retitled \textit{Oxford English Dictionary}.) \\ % The references to the NED are currently (6 Aug 24) rather a mess: some are to "Murray" (and are BibTeX style), others are to "New English Dictionary" (and don't use BibTeX style).

Nexø, \textit{Pelle} & M. Andersen Nexø. \textit{Pelle erobreren}. Copenhagen, 1906--1910. \\

Nielsen ??? \\ % Which appearance of OJ's sole quotation of L. C. Nielsen should we refer to -- OJ's own (which I (PE) can't find), or a different one (which I can find)? See the note within 06.tex. (Simply search within it for "Nielsen".)

\textit{Noah} & \textit{Noah and the ark}. (mid-16th century) In George England (ed.), \textit{The Towneley plays}. London, 1897. \\
%
% Noreen, \textit{Vårt språk} & Adolf Noreen. \textit{Vårt språk. Nysvensk grammatik i utförlig framställning}. Lund, 1904. \\  Now cited via BibTeX

Norris, \textit{Octopus} & Frank Norris. \textit{The octopus}. (1901) London, 1908. \\
Norris, \textit{Pit} & Frank Norris. \textit{The pit: A story of Chicago}. (1903) London, 1908. \\ 
% Converted to BibTeX: \raggedright {\textit{Nutidssprog hos børn og voksne}} & Otto Jespersen. \textit{Nutidssprog hos Børn og Voksne}. Copenhagen, 1916. \\

\textit{Odyssey} & Hómēros. \textit{Odýsseia}. (Homer. \textit{The Odyssey}. 8th/\linebreak[2]7th c.~BCE) \\
% \raggedright{\textit{Om banden og sværgen}} & Otto Jespersen. ``Om Banden og Sværgen: Iagttagelser om edernes sproglære.'' In \textit{Festskrift til H. F. Feilberg fra nordiske sprog- og folkemindeforskere på 80 års dagen den 6. august 1911}, 33--40. Stockholm, 1911. \\ Converted to BibTeX

Oppenheim, \textit{Millionaire} & E. Phillips Oppenheim. \textit{A millionaire of yesterday}. (1900) \\

Otway, \textit{Venice} & Thomas Otway. \textit{Venice preserv’d; or, A plot discover’d}. (1682) In Charles F. McClumpha (ed.), \textit{The orphan} and \textit{Venice preserved}. Boston, 1908. \\ % The spelling of the original is "preserv'd"; the spelling of the 1908 title is "preserved". (How irritating....)

T. N. Page, \textit{Marvel} & Thomas Nelson Page. \textit{John Marvel, assistant}. New York, 1909. \\

W. H. Page, \textit{Southerner} & Walter Hines Page. \textit{The southerner}. (1909) \\

Paludan-Müller, \textit{Adam} & Frederik Paludan-Müller. \textit{Adam Homo, et digt}. (1841--48) Copenhagen, 1857. \\

\textit{Parable} & ``A parable of the war.'' \textit{Times Literary Supplement}. 3 August 1917. \\ % https://scholarship.law.vanderbilt.edu/cgi/viewcontent.cgi?article=1525&context=vjtl says 2 August and also identifies the author.
%
% Paris, \textit{Périodiques} & Gaston Paris. ``Périodiques.'' \textit{Romania} 7 (1878). 463–477. \\ Converted to BibTeX

Parker, \textit{Right} & Gilbert Parker. \textit{The right of way}. (1900) London, 1906. \\

Paton, \textit{Tower} & Lucy Allen Paton. ``The story of Vortigern's Tower: An analysis.'' \textit{Radcliffe College Monographs} 15 (1910). 13--23. \\

\raggedright {Paues, ed., Middle English version} & Anna C. Paues, ed. \textit{A fourteenth century English biblical version}. Cambridge, 1904. \\ % "Middle English version" deliberately in neither quotation marks nor italics; see the citation (chapter 5)
%
%  \raggedright {Paul, \textit{Mittelhoch\-deutsche Grammatik}} & Hermann Paul. \textit{Mittelhochdeutsche Grammatik}. 4th ed. Halle, 1894. \\ Converted to BibTeX
% \raggedright {Paul, \textit{Wörterbuch}} & Hermann Paul. \textit{Deutsches Wörterbuch}. 2nd ed. Halle, 1908. \\ Converted to BibTeX
% \raggedright {Payne, \textit{Word-list from East Alabama}} & Leonidas Warren Payne. ``A Word-list from East Alabama.'' \textit{Bulletin of the University of Texas}, no. 123, 1 May 1909. \\ Converted to BibTeX

Pedersen, \textit{Skrifter} & C. J. Brandt (ed.), \textit{Christiern Pedersens danske skrifter}. Copenhagen, 1854. \\ % If for some reason H. Pedersen's Russisk læsebog reverts from BibTeX to this style of referencing, NB the author of Danske skrifter should be "C. Pedersen".
% \raggedright {H. Pedersen, \textit{Russisk læsebog}} & Holger Pedersen. \textit{Russisk læsebog}. Copenhagen, 1916. \\

Pérez Galdós, \textit{Doña} & Benito Pérez Galdós. \textit{Doña perfecta}. (1876) \\

\textit{Pericles} & William Shakespeare (attributed). \textit{Pericles, Prince of Tyre}. (1609) \\

Petersen, \textit{Uddrag} & Niels Matthias Petersen. \textit{Nogle uddrag af forelæsninger, vedkommende de nordiske sprog}. \textit{Samlede afhandlinger}, volume 4, pp.~107--294. Copenhagen, 1861. \\ 

Philips, \textit{Glass} & F. C. Philips. \textit{As in a looking glass}. Tauchnitz, 1886.  \\

Phillpotts, \textit{Mother} & Eden Philpotts. \textit{The mother}. London, 1908.  \\

\textit{Pilgrim} & William Shakespeare (attributed). \textit{The passionate pilgrim}. (1500) \\

Pinero, \textit{Benefit} & Arthur W. Pinero. \textit{The benefit of the doubt}. London, 1895.  \\
Pinero, \textit{Magistrate} & Arthur W. Pinero. \textit{The magistrate}. (1885) London, 1897.  \\
Pinero, \textit{Quex} & Arthur W. Pinero. \textit{The gay Lord Quex}. (1899) London, 1900.  \\
Pinero, \textit{Second} & Arthur W. Pinero. \textit{The second Mrs. Tanqueray}. (1893) London, 1895.  \\
%
% \raggedright {Pogatscher, \textit{Etymologisches}} & Alois Pogatscher. ``Etymologisches.'' \textit{Anglia}. 14 \textit{Beiblatt} (1903). 181–185. \\ Converted to BibTeX

Pollard, \textit{Prevention} & A. F. Pollard. ``The prevention of war.'' \textit{Times Literary Supplement}, 5 July 1917. \\ % OJ says 6 July but this is a mistake
% 
% \raggedright {Polle, \textit{Wie denkt das volk über die sprache}} & Friedrich Polle. \textit{Wie denkt das Volk über die Sprache? Gemeinverständliche Beiträge zur Beantwortung dieser Frage}. Leipzig, 1889. \\ Converted to BibTeX

\raggedright {Pontoppidan, \textit{Landsbybilleder}} & Henrik Pontoppidan. \textit{Landsbybilleder} (1883)\\

Pope, \textit{Rape} & Alexander Pope. \textit{The rape of the lock}. (1712/14) In \textit{Poetical works}. London, 1892. \\
% \raggedright {\textit{Progress in Language}} & Otto Jespersen. \textit{Progress in Language}. London \& New York, 1894. \\ Converted to BibTeX
% \raggedright {Puttenham, \textit{Arte of English Poesie}} & George Puttenham. \textit{The Arte of English Poesie}. (1589) Edited by Edward Arber. Birmingham, 1869. \\ Converted to BibTeX

Quiller-Couch, \textit{Major} & Arthur Quiller-Couch. \textit{Major Vigoureux}. London, 1907.  \\
Quiller-Couch, \textit{Troy} & Arthur Quiller-Couch. \textit{The astonishing history of Troy town}. London, {[}1888{]}. \\

Quincey, \textit{Confessions} & Thomas De Quincey. \textit{Confessions of an English opium-eater}. (1821) London, 1901. \\
Quincey, \textit{Mail-coach} & Thomas De Quincey. ``The English mail-coach.'' (1849) \\
Quincey, \textit{Murder} & Thomas De Quincey. ``On murder considered as one of the fine arts.'' (1827) \\

Raleigh, \textit{Shakespeare} & Walter Raleigh. \textit{Shakespeare}. London, 1907. \\

Ranch, \textit{Niding} & Hieronymus Ranch. \textit{Karrig Niding}. (1599) In S. Birket Smith (ed.), \textit{Hieronymus Justesen Ranch's danske skuespil og fuglevise}. Copenhagen, 1876--1877. \\ % "Niding" is a proper name, so leave capitalized

Rask, \textit{Undersögelse} & Rasmus Kristian Rask. \textit{Undersögelse om det gamle nordiske eller islandske sprogs oprindelse}. Copenhagen, 1818. \\
%
% \raggedright {Rauert, \textit{Die Negation in den Werken Alfred's}} & Matthäus Rauert. \textit{Die Negation in den Werken Alfred’s.} Doctoral dissertation, Kiel, 1910. \\ Converted to BibTeX

Read, \textit{Colonel} & Opie Read. \textit{A Kentucky colonel}. (1890) \\
Read, \textit{Toothpick} & Opie Read. \textit{Toothpick tales}. Chicago, 1892. \\

\textit{Recluse} & ``The recluse.'' (1375--1400) Within MS Pepys 2498. Excerpted in A. C. Paues. ``A XIVth-century version of the \textit{Ancren riwle}.'' \textit{Englische Studien}. 30 (1902). 344--346. \\ % OJ gives the page number; this source does not. So OJ probably used a later and fuller reproduction of this item in the MS

Richardson, \textit{Grandison} & Samuel Richardson. \textit{The history of Sir Charles Grandison}. (1753) London, 1764. \\

Ridge, \textit{Garland} & W. Pett Ridge. \textit{Name of Garland}. (1907) Tauchnitz. \\
Ridge, \textit{Property} & W. Pett Ridge. \textit{Lost property}. London, 1902. \\
Ridge, \textit{Son} & W. Pett Ridge. \textit{A son of the state}. (1899) London, n.d. \\

\textit{Roister} & Nicholas Udall. \textit{Roister Doister}. (Written c. 1552, published 1567.) Edited by Edward Arber. Birmingham, 1869. \\

Rolland, \textit{Amies} & Romain Rolland. \textit{Les Amies}. (1910. \textit{Jean-Christophe}, vol.~8.) \\ % OJ refers to this as "JChr. 8"
Rolland, \textit{Aube} & Romain Rolland. \textit{L'Aube}. (1904. \textit{Jean-Christophe}, vol.~1.) \\ % OJ refers to this as "JChr. 1"
Rolland, \textit{Buisson} & Romain Rolland. \textit{Le Buisson ardent}. (1911. \textit{Jean-Christophe}, vol.~9.) \\ % OJ refers to this as "JChr. 9"
Rolland, \textit{Foire} & Romain Rolland. \textit{La Foire sur la place}. (1908. \textit{Jean-Christophe}, vol.~5.) \\ % OJ refers to this as "JChr. 5"
Rolland, \textit{Journée} & Romain Rolland. \textit{La Nouvelle Journée}. (1912. \textit{Jean-Christophe}, vol.~10.) \\ % OJ refers to this as "JChr. 10"
Rolland, \textit{Maison} & Romain Rolland. \textit{Dans la maison}. (1908. \textit{Jean-Christophe}, vol.~7.) \\ % OJ refers to this as "JChr. 7"
Rolland, \textit{Révolte} & Romain Rolland. \textit{La Révolte}. (1905. \textit{Jean-Christophe}, vol.~4.) \\ % OJ refers to this as "JChr. 4"

Ruskin, \textit{Crown} & John Ruskin. \textit{The crown of wild olive}. (1866) London, 1904.  \\
Ruskin, \textit{Fors} & John Ruskin. \textit{Fors clavigera}. (1871--84) London, 1902.  \\
Ruskin, \textit{Præterita} & John Ruskin. \textit{Præterita}. (1885--89) London, 1902.  \\
Ruskin, \textit{Selections} & John Ruskin. \textit{Selections}. London, 1893. \\ % I (PE) haven't been able to find this edition on the web. There are a number of editions titled "Selections" and those I've found have been quite independent of the one that OJ cites. I've therefore left in the "Selections" attribution but linked elsewhere.
Ruskin, \textit{Sesame} & John Ruskin. \textit{Sesame and lilies}. (1865) London, 1904.  \\
Ruskin, \textit{Things} & John Ruskin. ``Things to be studied.'' Appendix to \textit{The elements of drawing}. (1857) \\
Ruskin, \textit{Time} & John Ruskin. \textit{Time and tide}. (1867) London, 1904.  \\

Rutebeuf, \textit{Pharisian} & Rutebeuf. ``Du Pharisian'', ou ``C'est d'ypocrisie.'' (13th century) \\

Sand, \textit{Consuelo} & George Sand. \textit{Consuelo}. (1842--43) \\

Schiller, \textit{Messina} & Friedrich Schiller. \textit{Die Braut von Messina}. (1803) \\
Schiller, \textit{Tod} & Friedrich Schiller. \textit{Wallensteins Tod}. (1799) \\
%
% \raggedright {Schmidt, \textit{Shakespeare-lexicon}} & Alexander Schmidt. \textit{Shakespeare-Lexicon.} Berlin, 1886. \\ Converted to BibTeX

Schreiner, \textit{Peter Halket} & Olive Schreiner. \textit{Trooper Peter Halket of Mashonaland}. London, 1897. \\
%
% \raggedright {Schuchardt, ``An Aug. Leskien zum 4. juli 1894''} & Hugo Schuchardt. ``An August Leskien zum 4. juli 1894.'' Graz, 1894. (Reprinted within Hugo Schuchardt, \textit{Slawo-deutsches und Slawo-italienisches}. Munich, 1971.) \\ Converted to BibTeX

Scott, \textit{Antiquary} & Walter Scott. \textit{The antiquary}. (1816) Edinburgh, 1821.  \\
Scott, \textit{Mortality} & Walter Scott. \textit{Old mortality}. (1816) Oxford, 1906. \\ % OJ refers to this as either "O" or "OM"

Seeley, \textit{Expansion} & J. R. Seeley. \textit{The expansion of England}. London, 1883. \\

Shakespeare, \textit{Ado} & William Shakespeare. \textit{Much adoe about nothing}. (1598--99) \\
Shakespeare, \textit{Alls} & William Shakespeare. \textit{All's well, that ends well}. (1604--05) \\
Shakespeare, \textit{As} & William Shakespeare. \textit{As you like it}. (1599--1600) \\
Shakespeare, \textit{Cæs} & William Shakespeare. \textit{The tragedie of Ivlivs Cæsar}. (1599) \\
Shakespeare, \textit{Cor} & William Shakespeare. \textit{The tragedy of Coriolanus}. (1608) \\
Shakespeare, \textit{Cymb} & William Shakespeare. \textit{The tragedie of Cymbeline}. (1610) \\
Shakespeare, \textit{Err} & William Shakespeare. \textit{The comedie of errors}. (1594) \\
Shakespeare, \textit{Gent} & William Shakespeare. \textit{The two gentlemen of Uerona}. (1587--91) \\
Shakespeare, \textit{Hml} & William Shakespeare. \textit{The tragedie of Hamlet, Prince of Denmarke}. (1599--1601) \\
Shakespeare, \textit{H4A} & William Shakespeare. \textit{The first part of Henry the Fourth}. (1596--97) \\
Shakespeare, \textit{H4B} & William Shakespeare. \textit{The second part of Henry the Fourth}. (1597--98) \\
Shakespeare, \textit{H5} & William Shakespeare. \textit{The life of Henry the fift}. (1599) \\ % No "h" at the end of "Fift"!
Shakespeare, \textit{H6A} & William Shakespeare. \textit{The first part of Henry the sixt}. (1591--92) \\ % No "h" at the end of "Sixt"!
Shakespeare, \textit{H8} & William Shakespeare. \textit{The life of King Henry the eight}. (1612--13) \\ % Sic.( Not "Eighth" but "Eight".)
Shakespeare, \textit{John} & William Shakespeare. \textit{The life and death of King Iohn}.  (1596) \\
Shakespeare, \textit{LLL} & William Shakespeare. \textit{Loues labour's lost}. (1595--97) \\
Shakespeare, \textit{Lr} & William Shakespeare. \textit{The tragedie of King Lear}. (1605--06) \\
Shakespeare, \textit{Lucr} & William Shakespeare. \textit{Lvcrece}. (1594) \\ % This is the title on the quarto. (No mention of "Rape".)
Shakespeare, \textit{Mcb} & William Shakespeare. \textit{The tragedie of Macbeth}. (1606) \\
Shakespeare, \textit{Meas} & William Shakespeare. \textit{Measvre, for measure}. (1603--04) \\ % Yes, this is the way it's spelt and comma'd in the first folio.
Shakespeare, \textit{Merch} & William Shakespeare. \textit{The merchant of Venice}. (1596--97) \\
Shakespeare, \textit{Mids} & William Shakespeare. \textit{A midsommer nights dreame}.  (1595) \\ % Yes, really, so spelt.
Shakespeare, \textit{Oth} & William Shakespeare. \textit{The tragedie of Othello, the moore of Venice}. (1603--04) \\
Shakespeare, \textit{R2} & William Shakespeare. \textit{The life and death of Richard the second}. (1595) \\
Shakespeare, \textit{R3} & William Shakespeare. \textit{The tragedy of Richard the third}. (1592--93)  \\
Shakespeare, \textit{Rom} & William Shakespeare. \textit{The tragedie of Romeo and Ivliet}.  (1595) \\
Shakespeare, \textit{Shr} & William Shakespeare. \textit{The taming of the shrew}. (1591--92) \\
Shakespeare, \textit{Tit} & William Shakespeare. \textit{Titus Andronicus}. (1591--92) \\
Shakespeare, \textit{Tp} & William Shakespeare. \textit{The tempest}. (1610--11) \\
Shakespeare, \textit{Tw} & William Shakespeare. \textit{Twelfe night, or What you will}. (1601) \\
Shakespeare, \textit{Ven} & William Shakespeare. \textit{Venvs and Adonis}. (1593) \\
Shakespeare, \textit{Wint} & William Shakespeare. \textit{The winters tale}. (1609--11) \\
Shakespeare, \textit{Wiv} & William Shakespeare. \textit{The merry wiues of Windsor}. (1597) \\

Shaw, \textit{Arms} & George Bernard Shaw. \textit{Arms and the man}. (1894) In \textit{Plays pleasant}. London, 1898. \\ 
Shaw, \textit{Candida} & George Bernard Shaw. \textit{Candida: A mystery}. (Written 1894) In \textit{Plays pleasant}. London, 1898. \\ 
Shaw, \textit{Cashel} & George Bernard Shaw. \textit{Cashel Byron's profession}. (Written 1882) London, 1901. \\ 
Shaw, \textit{Destiny} & George Bernard Shaw. \textit{The man of destiny: A trifle}. (1897) In \textit{Plays pleasant}. London, 1898. \\
Shaw, \textit{Dilemma} & George Bernard Shaw. \textit{The doctor's dilemma}. (1906) London, 1911. \\ 
Shaw, \textit{Disciple} & George Bernard Shaw. \textit{The Devil's disciple}. (1897) In \textit{Three plays for puritans}. London, 1901. \\ 
Shaw, \textit{First} & George Bernard Shaw. \textit{Fanny's first play}. (1911) In \textit{Misalliance, The dark lady}, and \textit{Fanny's first play}. London, 1914. \\ 
Shaw, \textit{Houses} & George Bernard Shaw. \textit{Widowers' houses}. (1892) In \textit{Plays unpleasant}. London, 1898. \\ 
Shaw, \textit{Ibsenism} & George Bernard Shaw. \textit{The quintessence of Ibsenism}. London, 1891. \\ 
Shaw, \textit{Island} & George Bernard Shaw. \textit{John Bull's other island}. (1904) London, 1907. \\ 
Shaw, \textit{Major} & George Bernard Shaw. \textit{Major Barbara}. (1905) \\ 
Shaw, \textit{Married} & George Bernard Shaw. \textit{Getting married: A disquisitory play}. (1908) \\
Shaw, \textit{Never} & George Bernard Shaw. \textit{You never can tell}. (1897) In \textit{Plays pleasant}. London, 1898. \\ 
Shaw, \textit{Philanderer} & George Bernard Shaw. \textit{The philanderer: A topical comedy}. (Written 1893, first staged 1902.) In \textit{Plays unpleasant}. London, 1898. \\ 
Shaw, \textit{Profession} & George Bernard Shaw. \textit{Mrs. Warren's profession}. In \textit{Plays unpleasant}. (Written 1893, first staged 1902.) London, 1898. \\

Shelley, \textit{Epipsychidion} & Percy Bysshe Shelley. \textit{Epipsychidion}. (1821) In Thomas Hutchinson (ed.), \textit{The complete poetical works of Percy Bysshe Shelley} London, 1914. \\
Shelley, \textit{Letters} & Roger Ingpen (ed.), \textit{The letters of Percy Bysshe Shelley}. London, 1914. \\
Shelley, \textit{Prometheus} & Percy Bysshe Shelley. \textit{Prometheus unbound}. (1820) In Thomas Hutchinson (ed.), \textit{The complete poetical works of Percy Bysshe Shelley}. London, 1914.  \\
Shelley, \textit{Prose} & Richard Herne Shepherd (ed.), \textit{The prose works of Percy Bysshe Shelley}. London, 1914. \\ % OJ’s book says “Pr 295”. I (PE) can’t think what “Pr” can be short for in this context, other than “Prose”, and suspect that it’s this book, with its slightly different page numbering, that’s meant.
Shelley, \textit{Revolt} & Percy Bysshe Shelley. \textit{The revolt of Islam}. (1817) In Thomas Hutchinson (ed.), \textit{The complete poetical works of Percy Bysshe Shelley}. London, 1914. \\

Shenstone, \textit{Schoolmistress} & William Shenstone. ``The schoolmistress.'' (1737) \\

Sheridan, \textit{Critic} & Richard Brinsley Sheridan. \textit{The critic}. (1779) In \textit{Dramatic works}. Tauchnitz. \\ 
Sheridan, \textit{Duenna} & Richard Brinsley Sheridan. \textit{The duenna}. (1775) In \textit{Dramatic works}. Tauchnitz. \\ % I (PE) am guessing that this is what OJ is referring to by "D" ... though I have serious doubts.
Sheridan, \textit{Rivals} & Richard Brinsley Sheridan. \textit{The rivals}. (1775) In \textit{Dramatic works}. Tauchnitz. \\
Sheridan, \textit{School} & Richard Brinsley Sheridan. \textit{The school for scandal}. (1777) In \textit{Dramatic works}. Tauchnitz. \\

Sibbern, \textit{Breve} & Frederik Christian Sibbern. \textit{Gabrielis breve}. (Written 1813--14, published 1826.) \\

Siesbye, \textit{Kuriosa} & Oskar Siesbye. ``Fortsatte Bemærkninger hensyn til `sproglige Kuriosa'.'' \textit{Dania}, 10 (1903). 39--51. \\
Siesbye, \textit{Småting} & Oskar Siesbye. ``Småting''. In \textit{Opuscula philologica ad Ioannem Nicolaum Madvigium} = \textit{Lykønskningsskrift i Anledning af Johan Nicolai Madvigs}. 234--255. Copenhagen, 1876. \\ % Book titles in both Latin and Danish
\raggedright{Siesbye, \textit{Strøbemærkninger}} & Oskar Siesbye. ``Strøbemærkninger.'' \textit{Nordisk tidsskrift die filologi}, 3rd series. 8 (1899). 1--16. \\

Skeat, \textit{John} & Walter W. Skeat, ed. \textit{The gospel according to Saint John: In Anglo-Saxon and Northumbrian versions synoptically arranged}. Cambridge, 1878. \\ % I (PE) presume that this (at https://archive.org/details/holygospelsinan01skeagoog/page/n7/mode/2up?view=theater ) is what OJ refers to in the last paragraph on p 55 of the printed book.
% Skeat, \textit{Notes} & Walter W. Skeat. \textit{Notes to The Canterbury Tales}. Volume 5 of \textit{The Complete Works of Geoffrey Chaucer}. 2nd edn. Oxford, at the Clarendon Press, 1900. \\ Converted to BibTeX

Skram, \textit{Lucie} & Amalie Skram. \textit{Lucie}. Copenhagen, 1888. \\

Smedley, \textit{Frank} & Frank E. Smedley. \textit{Frank Fairlegh}. (1850) Tauchnitz. \\

Sörensen, \textit{Ariadnetråd} & Axel Sörensen. \textit{En Ariadnetråd}. Copenhagen, 1902.  \\ % OJ spells his surname “Sørensen”, but on the title page of the cited book it’s instead “Sörensen”. About language.

\textit{Spectator} & Joseph Addison et al. \textit{The Spectator}. (1711--14) Edited by Henry Morley. London 1888. \\ % The Spectator is the title of the magazine, and for this reason perhaps "Spectator" should be capitalized.

Spencer, \textit{Autobiography} & Herbert Spencer. \textit{An autobiography}. London, 1904. \\
Spencer, \textit{Education} & Herbert Spencer. \textit{Education: Intellectual, moral, and physical}. (1861) London, 1882. \\ 

Spenser & Edmund Spenser. \textit{The faerie queene}. (1590, 1596) \\

Stacpoole, \textit{Cottage} & Henry de Vere Stacpoole. \textit{The cottage on the fells}. Toronto, n.d. \\

Stanley, \textit{Dark} & Henry M. Stanley. \textit{Through the dark continent}. London, 1878. \\

Sterling, letter & John Sterling. Letter of 29 May 1835 to Thomas Carlyle. In Thomas Carlyle, \textit{The life of John Sterling}. \\

Sterne, \textit{Tristram} & Laurence Sterne. \textit{Tristram Shandy}. In David Herbert (ed.), \textit{The complete works}. Edinburgh, 1885. \\ 

Stevenson, \textit{Arrow} & Robert Louis Stevenson. \textit{The black arrow}. (1888) London, 1904.  \\
Stevenson, \textit{Art} & Robert Louis Stevenson. \textit{Essays in the art of writing}. London, 1905.  \\
Stevenson, \textit{Jekyll} & Robert Louis Stevenson. \textit{The strange case of Dr. Jekyll and Mr. Hyde, with other fables}. London, 1896. \\ % OJ calls this "JHF". Don’t remove “with Other Fables”. (Plenty of other editions lack these other fables, and OJ quotes from one of the other fables at least once.)
Stevenson, \textit{Memories} & Robert Louis Stevenson. \textit{Memories and portraits}. (1887) London, 1900. \\  % The list of sources in MEG vol 7 has this as “Memoirs and Portraits”: a mistake. (MEG vol 2 gets it right.)
Stevenson, \textit{Men} & Robert Louis Stevenson. \textit{Familiar studies of men and books}. (1882) London, 1901. \\
Stevenson, \textit{Merry} & Robert Louis Stevenson. \textit{The merry men}. (1887) London, 1896.  \\
Stevenson, \textit{Treasure} & Robert Louis Stevenson. \textit{Treasure island}. (1882) Tauchnitz.  \\
Stevenson, \textit{Virginibus} & Robert Louis Stevenson. \textit{Virginibus puerisque}. (1881) London 1894.  \\
%
% \raggedright{Stoffel, \textit{Studies in English}} & Cornelis Stoffel. \textit{Studies in English, Written and Spoken: For the Use of Continental Students}. Zutphen, 1894. \\ Converted to BibTeX
% \raggedright{Storm, \textit{Englische philologie}} & Johan Storm. \textit{Englische Philologie}. Leipzig, 1892–96. \\ Converted to BibTeX
% \raggedright{Storm, \textit{Større fransk syntax}} & Johan Storm. \textit{Større fransk syntax}. 1. \textit{Artiklerne}. Kristiana, 1911. \\ Converted to BibTeX

Strindberg, \textit{Giftas} & August Strindberg. \textit{Giftas}. Stockholm, 1886. \\
Strindberg, \textit{Röda} & August Strindberg. \textit{Röda rummet}. (1879) \\
Strindberg, \textit{Utopier} & August Strindberg. \textit{Utopier i verkligheten}. Stockholm, 1885. \\

Sudermann, \textit{Fritzchen} & Hermann Sudermann. \textit{Fritzchen}. In \textit{Morituri. Teja, Fritzchen, Das Ewig-Männliche}. (1876) \\
%
% Sully, \textit{Childhood} & James Sully. \textit{Studies of childhood}. London, 1903. \\ Converted to BibTeX
% \raggedright{Sweet, \textit{New English Grammar}} & Henry Sweet. \textit{A New English Grammar: Logical and Historical}. Oxford, 1892--1898. \\ Converted to BibTeX

J. Swift, \textit{Conversation} & Jonathan Swift. \textit{A compleat collection of genteel and ingenious conversation}. (Written 1731, published 1738.) Edited by George Saintsbury, 1892. \\
J. Swift, \textit{Journal} & Jonathan Swift. \textit{A journal to Stella}. (Written 1710--13, published 1766.) Edited by George A. Aitken. London, 1901. \\
\raggedright{J. Swift, ``in the \textit{Tatler} no. 230''} & Jonathan Swift. Letter to Isaac Bickerstaff. \textit{The Tatler} no. 230. 28 September 1710. \\
J. Swift, \textit{Travels} & [Jonathan Swift]. \textit{Volume III of the author’s works. Containing travels into several remote nations of the world}. (1726, \textit{Gulliver's travels}.) Dublin, 1735. \\
J. Swift, \textit{Tub} & Jonathan Swift. \textit{A tale of a tub}. (Written 1694--97, published 1704.) London, 1760. \\

M. I. Swift, \textit{Humanizing} & Morrison I. Swift. ``Humanizing the prisons.'' \textit{The Atlantic Monthly}. August 1911. 170--179. \\ % OJ identifies this as "NP. 1911"

\raggedright {Swinburne, \textit{Cross-currents}} & Algernon C. Swinburne. \textit{Love's cross-currents}. As \textit{A year's letters}: Tauchnitz, 1905.  \\ % OJ refers to this as "L". NB In MEG, OJ uses "Swinburne L." for an entirely different work (Swinburne's Locrine); or so says the list in MEG Part II.
Swinburne, \textit{Pilgrimage} & Algernon Charles Swinburne. ``The last pilgrimage.'' \textit{Tristram of Lyonesse, and other poems}. (1882) London, 1884.  \\
Swinburne, \textit{Shakespeare} & Algernon Charles Swinburne. \textit{A study of Shakespeare}. (1879) London, 1895. \\
Swinburne, \textit{Songs} & Algernon Charles Swinburne. \textit{Songs before sunrise}. (1871) London, 1903. \\

\raggedright{Szinnyei, \textit{Ungarische Sprachlehre}} & József Szinnyei. \textit{Ungarische Sprachlehre}. Berlin, 1912. \\

Tennyson, \textit{1852} & Alfred Tennyson. ``The third of February, 1852.'' \textit{Enoch Arden, and other poems}. In \textit{Poetical works}. London, 1894. \\
Tennyson, \textit{Coming} & Alfred Tennyson. ``The coming of Arthur.'' \textit{Idylls of the king}. In \textit{Poetical works}. London, 1894. \\
Tennyson, comment & Alfred Tennyson. Reported comment. In Hallam Tennyson, \textit{Alfred Lord Tennyson: A memoir by his son}.  \\ % OJ refers to this as "L"
Tennyson, diary & Alfred Tennyson. Diary entry. In Hallam Tennyson, \textit{Alfred Lord Tennyson: A memoir by his son}. \\ % OJ refers to this as "L"
Tennyson, \textit{Enid} & Alfred Tennyson. ``Enid.'' (1859) \textit{Idylls of the king}. In \textit{Poetical works}. London, 1894. \\
Tennyson, \textit{Guinevere} & Alfred Tennyson. ``Guinevere.'' (1859) \textit{Idylls of the king}. In \textit{Poetical works}. London, 1894. \\
Tennyson, letter & Alfred Tennyson. Letter. In Hallam Tennyson, \textit{Alfred Lord Tennyson: A memoir by his son}. \\  % OJ refers to this as "L"
Tennyson, \textit{Memoriam} & Alfred Tennyson. \textit{In memoriam A. H. H.}. (1850) In \textit{Poetical works}. London, 1894. \\
Tennyson, \textit{Merlin} & Alfred Tennyson. ``Merlin and Vivien.'' \textit{Idylls of the king}. In \textit{Poetical works}. London, 1894. \\
Tennyson, \textit{Princess} & Alfred Tennyson. \textit{The Princess: A medley}. (1847) In \textit{Poetical works}. London, 1894. \\
Tennyson, \textit{Wellington} & Alfred Tennyson. ``Ode on the death of the Duke of Wellington.'' \textit{Maud, and other poems}. In \textit{Poetical works}. London, 1894. \\

Thackeray, \textit{Newcomes} & William Makepeace Thackeray. \textit{The Newcomes}. (1855) London, 1901. \\
Thackeray, \textit{Pendennis} & William Makepeace Thackeray. \textit{The history of Pendennis}. (1848--50) Tauchnitz. \\
Thackeray, \textit{Samuel} & William Makepeace Thackeray. \textit{The history of Samuel Titmarsh and the great Hoggarty diamond}. London, 1878. \\
Thackeray, \textit{Snobs} & William Makepeace Thackeray. \textit{The book of snobs}. (1846--1847) London, 1900. \\
Thackeray, \textit{Vanity} & William Makepeace Thackeray. \textit{Vanity fair}. London, 1848. \\

%\raggedright{Tobler, \textit{Französische Etymologien}} & Adolf Tobler. ``Französische Etymologien.'' \textit{Zeitschrift für vergleichende Sprachforschung auf dem Gebiete der Indogermanischen Sprachen} 23 (4, 1877). 414–23. \\ Converted to BibTeX
\raggedright {Tobler, \textit{Vermischte Beiträge}} & Adolf Tobler. \textit{Vermischte Beiträge zur französischen Grammatik}. Leipzig, 1886. \\ % Also cited via BibTeX

Topsøe, \textit{Skitseb}. & Vilhelm Topsøe. \\ % ??? OJ calls this "Skitseb."; it's still UNIDENTIFIED

\textit{Trifles} & [George Nugent-Bankes.] \textit{Cambridge trifles, or splutterings from an undergraduate pen}. London, 1881. \\ % published anonymously

Trollope, \textit{Barchester} & Anthony Trollope. \textit{Barchester Towers}. (1857) \\
Trollope, \textit{Children} & Anthony Trollope. \textit{The duke's children}. (1880) Tauchnitz, 1880.  \\
Trollope, \textit{Love} & Anthony Trollope. \textit{An old man's love}. (1884) Tauchnitz. \\
Trollope, \textit{Warden} & Anthony Trollope. \textit{The warden}. (1855) London, 1913. \\

Twain, \textit{Huckleberry} & Mark Twain. \textit{Adventures of Huckleberry Finn}. (1884) Tauchnitz, n.d. \\
Twain, \textit{Mississippi} & Mark Twain. \textit{Life on the Mississippi}. (1883) London, 1887. \\
% \textit{Værdiforskydning} & Otto Jespersen. ``En sproglig Værdiforskydning: \textit{Og} = \textit{at}.'' \textit{Dania} 10 (1895). 145--182. \\
%
% \raggedright{van Dam, \textit{William Shakespeare}} & B. A. P. van Dam. \textit{William Shakespeare, Prosody and Text}. Leyden, 1900. \\ Converted to BibTeX

van Eeden, \textit{Johannes} & Frederik van Eeden. \textit{De kleine Johannes}. (1885) \\
%
% \raggedright {van Ginneken, \textit{Principes de linguistique psycho\-logique}} & Jacobus van Ginneken. \textit{Principes de linguistique psychologique}. Amsterdam, 1907. \\  Converted to BibTeX

Villiers, \textit{Rehearsal} & George Villiers. \textit{The rehearsal}. (1671) Edited by Edward Arber. London, 1895. \\

Virgil, \textit{Aeneid} & Publius Vergilius Maro. \textit{Aeneid}. (29--19 BCE)\\
%
% \raggedright{Vondrák, \textit{Vergleichende slavische Grammatik}} & Wenzel {[}Václav{]} Vondrák. \textit{Vergleichende slavische Grammatik}. Göttingen, 1908.\\ Converted to BibTeX

\textit{Vulgate John} & \textit{Jesu Christi evangelium secundum Joannem, Vulgate}. (4th century) \\
\textit{Vulgate Matthew} & \textit{Jesu Christi evangelium secundum Matthæum, Vulgate}. (4th century) \\

Wägner, \textit{Norrtullsligan} & Elin Wägner. \textit{Norrtullsligan}. (1908) \\

Walton, \textit{Angler} & Izaak Walton. \textit{The compleat angler}. London, 1653. \\

Ward, \textit{David} & Mrs Humphry Ward. \textit{The history of David Grieve}. Tauchnitz, 1892. \\
Ward, \textit{Eleanor} & Mrs Humphry Ward. \textit{Eleanor}. London, 1900. \\
Ward, \textit{Marriage} & Mrs Humphry Ward. \textit{The marriage of William Ashe}. (1905) London. \\
% \raggedright{Wellander, \textit{Om betydelseutvecklingens lagbundenhet}} & Erik Wellander. ``Om betydelseutvecklingens lagbundenhet.'' \textit{Språkvetenskapliga sällskapets i Uppsala Förhandlingar 1913--1915}. Uppsala, 1916. Now a BibTeX reference

Wells, \textit{Anticipations} & H. G. Wells. \textit{Anticipations}. London, 1902. \\
Wells, \textit{Britling} & H. G. Wells. \textit{Mr. Britling sees it through}. London, 1916. \\
Wells, \textit{Love} & H. G. Wells. \textit{Love and Mr. Lewisham}. London, 1906. \\
Wells, \textit{Machiavelli} & H. G. Wells. \textit{The new Machiavelli}. London, 1911. \\
Wells, \textit{Stories} & H. G. Wells. \textit{Twelve stories and a dream}. (1903) London. \\
Wells, \textit{Utopia} & H. G. Wells. \textit{A modern utopia}. London, 1905. \\
Wells, \textit{Veronica} & H. G. Wells. \textit{Ann Veronica}. London, 1909. \\
Wells, \textit{Wife} & H. G. Wells. \textit{The wife of Sir Isaac Harman}. London, 1914. \\
Wells, \textit{Worlds} & H. G. Wells. \textit{New worlds for old}. London, 1908. \\

Wessel & Johan Herman Wessel. J. Levin (ed.), \textit{J. H. Wessels samlede digte}. Copenhagen, 1862.\\

WG \textit{Matthew} & \textit{Đæt gódspell æfter Matheus gerecednysse}, \textit{Đa Feower Cristes Béc on Engliscum Gereorde}. (\textit{Wessex gospels} {[}or \textit{West-Saxon gospels}{]}, circa 995.) \\ % I've no idea which of these words is a proper noun (and so must be caplitalized)
%
% Whately, \textit{Remains} & \textit{Miscellaneous remains from the commonplace book of Richard Whately, D.D.} London, 1865. \\ Converted to BibTeX

Whiteing, \textit{Five} & Richard Whiteing. \textit{No. 5 John Street}. (1899) \\

Whitney, \textit{Studies} & William Dwight Whitney. \textit{Oriental and linguistic studies}. New York, 1873. \\ % Title may sound like that of a journal, but it's a book by Whitney

Wieland, \textit{Oberon} & Christoph Martin Wieland. \textit{Oberon: Ein Gedicht}. (1780) \\

Wilde, \textit{Fan} & Oscar Wilde. \textit{Lady Windermere’s fan}. (1892) \\
Wilde, \textit{Gaol} & Oscar Wilde. \textit{The ballad of Reading gaol}. London, 1898. \\
Wilde, \textit{Importance} & Oscar Wilde. \textit{The importance of being earnest}. (1895) London, n.d. \\
Wilde, \textit{Intentions} & Oscar Wilde. \textit{Intentions}. 1891. \\
Wilde, \textit{Picture} & Oscar Wilde. \textit{The picture of Dorian Gray}. (1891) New York, n.d. \\
Wilde, \textit{Profundis} & Oscar Wilde. \textit{De profundis}. (Written 1897, published 1905.) London, 1905. \\

Wilkins, \textit{Pericles} & George Wilkins. \textit{Pericles Prince of Tyre}. (1608) Edited by Tycho Mommsen. Oldenburg, 1857. \\
%
% \raggedright{Willert, ``Über bildliche verneinung''} & H. Willert ``Über bildliche Verneinung im Neu\-englischen.'' \textit{Archiv für das Studium der neueren Sprachen und Litteraturen} 105 (1900), 37--47. \\ Converted to BibTeX

Williamson, \textit{Lightning} & C. N. \& A. M. Williamson. \textit{The lightning conductor}. (1902) London. \\ 
Williamson, \textit{Powers} & C. N. \& A. M. Williamson. \textit{The Powers and Maxine}. (1907) London. \\ 

Wimmer, \textit{Læsebog} & Ludvig F. A. Wimmer. \textit{Oldnordisk læsebog}. Copenhagen, 1870. \\

Wordsworth, \textit{Michael} & William Wordsworth. ``Michael: A pastoral poem.''  (1800. In \textit{Lyrical ballads}, etc.)\\
Wordsworth, \textit{Prelude} & William Wordsworth. \textit{The prelude}. (Written 1798--1850.)\\
Wordsworth, \textit{Travelled} & William Wordsworth. ``I travelled among unknown men.'' (In \textit{Poems: In two volumes} (1807), etc.) \\

% \raggedright{Wright, \textit{Rustic Speech}} & Elizabeth Mary Wright. \textit{Rustic Speech and Folk-Lore}. Oxford, 1913. \\ converted to BibTeX
%
\raggedright{Wright \& Wülcker, \textit{Vocabularies}} & Thomas Wright. \textit{Anglo-Saxon and Old English vocabularies}. 2nd edn., by Richard Paul Wülcker. London, 1884. \\ % Also cited via BibTeX
%
%\raggedright{Wyld, \textit{Historical Study}} & Henry Cecil Wyld. \textit{A Historical Study of the Mother Tongue}. London, 1906. \\ converted to BibTeX
 
Zangwill, \textit{Child} & Israel Zangwill. ``A child of the ghetto.'' \textit{Cosmopolis} 1897. (Also in Israel Zangwill, \textit{Dreamers of the ghetto}.) \\ % OJ refers to this as "Cosmopolis '97"
Zangwill, \textit{Wig} & Israel Zangwill. \textit{The grey wig}. London, 1903.\\
%
% \raggedright{Ziemer, \textit{Junggrammatische streifzüge}} & Hermann Ziemer. \textit{Junggrammatische Streifzüge im Gebiete der Syntax}. 2nd ed. Colberg, 1883. \\ converted to BibTeX
    
Þorkelsson, \textit{Nekrolog} & Jón Þorkelsson. ``Nekrolog öfver Guðbrandur Vigfússon.'' \textit{Arkiv för Nordisk Filologi} 6 (1890). 156--163.\\
\end{longtable}
