\chapter{Negative Lexical Items} \label{ch:lexical}

%%%%%%%%%%%%
% Produced by ChatGPT after significant back and forth but not checked. Likely suffers from omissions and inclusion of false positives.
%%%%%%%%%%%%
\largerpage
\begin{itemize}[leftmargin=*]
    \item {Core English negators}
    \begin{itemize}
        \item Basic negators: \textit{not}, \textit{-n't}, \textit{never}, \textit{no}, \textit{none}, \textit{nothing}, \textit{nobody}, \textit{neither}, \textit{nor}
        \item Obsolete/variant forms: \textit{nought}, \textit{naught}, \textit{nay}, \textit{na}
    \end{itemize}

    \item {Variants and contractions}
    \begin{itemize}
        \item Contractions: \textit{don’t}, \textit{won’t}, \textit{can’t}, \textit{shan’t}, \textit{ain’t}
        \item Old/Middle English: \textit{ne}, \textit{nam} (for \textit{am not}), \textit{nis} (for \textit{is not}), \textit{nill} (for \textit{will not}), \textit{nolde} (for \textit{would not})
    \end{itemize}

    \item {Intensifiers and emphatic negators}
    \begin{itemize}
        \item Idiomatic intensifiers: \textit{not a bit}, \textit{not a jot}, \textit{not a scrap}, \textit{not a whit}, \textit{not a damn}, \textit{not a toss}, \textit{not a tinker’s curse}, \textit{zilch}, \textit{diddly-squat}
        \item Phrases: \textit{by no means}, \textit{not at all}, \textit{not in the least}, \textit{never a one}, \textit{never a word}, \textit{never a thought}, \textit{never a sound}
    \end{itemize}

    \item {Exclamatory and emphatic negation}
    \begin{itemize}
        \item Exclamations: \textit{damn}, \textit{damned}, \textit{blame}, \textit{blest}, \textit{be hanged}, \textit{be shot}
        \item Expressions: \textit{be damned if I do}, \textit{I’ll be shot if I am}, \textit{hang me if I can tell}
    \end{itemize}

    \item {Lexicalized negation with \textit{devil} and related terms}
    \begin{itemize}
        \item Devil and variants: \textit{devil}, \textit{de’il}, \textit{fanden}, \textit{djævelen} (Danish), \textit{the devil knows}, \textit{devil a bit}, \textit{devil a word}, \textit{the devil you say}, \textit{devil the other idea}
        \item Deuce: \textit{deuce a word}, \textit{the deuce he does}
        \item Irish English: \textit{sorra} (e.g., \textit{sorra a bit}, \textit{sorra a word})
    \end{itemize}

    \item {Negative prefixes and suffixes}
    \begin{itemize}
        \item Prefixes: \textit{un-}, \textit{in-}, \textit{dis-}, \textit{non-}, \textit{a-}
        \item Suffixes: \textit{-less} (e.g., \textit{hopeless}, \textit{careless})
    \end{itemize}

    \item {Mitigated or softened negation}
    \begin{itemize}
        \item Mitigated negatives: \textit{hardly}, \textit{scarcely}, \textit{barely}
        \item Comparative of \textit{little}: \textit{least} in \textit{lest} (e.g., \textit{lest you forget})
    \end{itemize}

    \item {Idiomatic and playful uses}
    \begin{itemize}
        \item Sports terminology: \textit{love} (in tennis, meaning \textit{nothing})
        \item Curses as negation: \textit{pox} (as in \textit{the pox take me})
    \end{itemize}

    \item {French negators}
    \begin{itemize}
        \item \textit{ne}, \textit{pas}, \textit{point}, \textit{mie}, \textit{jamais}, \textit{personne}, \textit{aucun}, \textit{rien}, \textit{guère}
    \end{itemize}

    \item {German negators}
    \begin{itemize}
        \item \textit{nicht}, \textit{kein}, \textit{nichts}, \textit{nie}
    \end{itemize}

    \item {Scandinavian/Danish/Norse negators}
    \begin{itemize}
        \item \textit{ikke}, \textit{ej}, \textit{ingen}, \textit{intet}, \textit{eigi}, \textit{ekki}
    \end{itemize}

    \item {Latin negators}
    \begin{itemize}
        \item \textit{ne}, \textit{non}, \textit{nullus}, \textit{nullus est}
    \end{itemize}

    \item {Greek negators}
    \begin{itemize}
        \item \textit{ouden}, \textit{oudeis}, \textit{ou gar}
    \end{itemize}

    \item {Scots and dialectal negation}
    \begin{itemize}
        \item \textit{fient} (e.g., \textit{fient haet}, meaning \textit{not a bit})
    \end{itemize}

    \item {Old and Middle English forms}
    \begin{itemize}
        \item Middle English: \textit{ne seye} (for \textit{say not}), \textit{ne can} (for \textit{cannot}), \textit{nese} (for \textit{no})
        \item Old English: \textit{ne}, \textit{nan} (for \textit{none}), \textit{nalles} (for \textit{not at all}), \textit{na} (for \textit{not})
        \item Compound forms: \textit{nawiht} (for \textit{nothing}), \textit{nawiht at all}
    \end{itemize}

    \item {Indirect or softened negation}
    \begin{itemize}
        \item Comparative and superlative negatives: \textit{none}, \textit{neither}, \textit{less}, \textit{least}
    \end{itemize}
    \largerpage[2]

    \item {Cross-linguistic and historical terms}
    \begin{itemize}
        \item \textit{ainata} (Old Norse for \textit{one thing} used in negation)
        \item \textit{ou gar} (Ancient Greek for \textit{not indeed})
    \end{itemize}

    \item {Idioms and figurative negation}
    \begin{itemize}
        \item Figurative negatives: \textit{not worth a fig}, \textit{not worth a straw}, \textit{not worth a farthing}, \textit{not worth a brass farthing}, \textit{not worth a curse}
    \end{itemize}
\end{itemize}
