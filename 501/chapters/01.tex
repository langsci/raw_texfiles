\ChapterAndMark{General Tendencies}
\label{ch:1}
\label{p:first}

\is{Jespersen's cycle|(}The history of negative expressions in various languages makes us witness the following curious fluctuation: the original negative adverb\is{adverbs!negative} is first weakened\is{adverbs!weakening of negative}, then found insufficient and therefore strengthened, generally through some additional word, and this in its turn may be felt as the negative proper and may then in course of time be subject to the same development as the original word.

Similar renewals of linguistic expressions may be found in other domains as well, but in this instance they are due not only to the general inconstancy of human habits, but to specific causes operating on these particular words. \is{contrast, function of|(}The negative adverb\is{adverbs!negative} very often is rather weakly \is{stress}stressed, because some other word in the same sentence receives the strong stress of contrast---\is{contradiction, function of|(}the chief use of a negative sentence being to contradict and to point a contrast. The negative notion, which is logically very important, is thus made to be accentually subordinate\is{phonological reduction} to some other notion; and as this happens constantly, the negative gradually becomes a mere \is{proclitic, negative}proclitic syllable (or even less than a syllable) prefixed\is{prefixes!negative} to some other word.\is{contrast, function of|)} The incongruity between the notional importance and the formal insignificance of the negative (often, perhaps, even the fear of the hearer failing to perceive it) may then cause the speaker to add something to make the sense perfectly clear to the hearer.\is{Jespersen's cycle|)}

\label{para:naturaltendency}\is{position of negative|(}\is{negative first principle}On the other hand, there is a natural tendency, also for the sake of clearness, to place the negative first, or at any rate as soon as possible, very often immediately before the particular word to be negatived (generally the verb, see below p.~\pageref{rapid_sketch}). At the very beginning of the sentence, it is found comparatively often in the early stages of some languages, thus \il{Greek!\textit{ou}|(}\textit{ou} (`not') in Homer (see, for instance, in \textit{Odyssey} \href{http://www.perseus.tufts.edu/hopper/text?doc=Hom.+Od.+6.33&fromdoc=Perseus%3Atext%3A1999.01.0135}{6.33}, 
57, 167, 201, 241, 279, 7.22, 32, 67, 73, 159, 205, 239, 293, 309, besides the frequent instances of \textit{ou gár} (`for not' or `indeed not'); \textit{ou} (`no one') is far less frequent in the middle of sentences).\il{Greek!\textit{ou}|)}% What do we do about this? I (PE) am not up for even the most superficial digestion of Greek.
%Brett: Link added.
% PE: ??? Well done; but elsewhere individual instances (of whatever) are linked; perhaps we should do the same here.
Readers of Icelandic sagas will similarly have noticed the numerous instances of \il{Icelandic!Old Icelandic!\textit{eigi}}\textit{eigi} and \il{Icelandic!Old Icelandic!\textit{eiki}}\textit{ekki} (`not') at the beginning of sentences, especially in dialogues. %% SG: I assume the "langmeta" field is for the index? In that case, note that the language referred to here is Old Norse (or Old Icelandic), not present-day Icelandic. Not sure if this is important though?
In later stages, this tendency, which to us seems to indicate a strong spirit of contradiction, is counterbalanced in various ways, thus very effectively by the habit of placing the subject of a sentence first. But it is still strong in the case of \is{prohibitions}prohibitions, where it is important to make the hearer realize as soon as possible that it is not a permission that is imparted; hence in Danish frequently such sentences as \il{Danish!ikke@\textit{ikke}|(}\textit{ikke spise det!} (`don't eat that') with the infinitive (which is chiefly or exclusively due to ``echoism'', see \citet[164]{jespersen1916nutidssprog} or \textit{ikke spis det!} with the imperative; cf. \refp{ex:01-01}. Further the German \textit{nicht hinauslehnen} (`don't lean out'), etc., corresponding to the first mentioned Danish form; and we might also include prohibitions in other languages, Latin \textit{noli putare} (`do not think'), etc.\is{contradiction, function of|)}\is{position of negative|)}

\ea \label{ex:01-01}
\ea
\gll Hys --- hys; ikke sig noget endnu.\\
hush {} hush not say.\IMP{} anything yet\\
\glt `Hush---hush; don't say anything yet.'\hfill(\href{https://archive.org/details/vildandenskuesp00ibsegoog/page/n87/mode/2up?q=%22ikke+sig+noget+endnu%22&view=theater}{Ibsen, \textit{Vildanden} 79})

\ex
\gll Men ikke fordærv øjnene! \phantom{x}\\
but not ruin.\IMP{} eyes.\DEF{}\\
\glt `But do not ruin your eyes!' \hfill(\href{https://archive.org/details/vildandenskuesp00ibsegoog/page/n113/mode/2up?q=%22Men+ikke+ford%C3%A6rv+%C3%B8jnene%22&view=theater}{ibid 105})
\z
\z\il{Danish!ikke@\textit{ikke}|)}

Now, when the negative begins a sentence, it is on account of that very position more liable than elsewhere to fall out, by the phenomenon for which I venture to coin the term of \textsc{\is{prosiopesis}prosiopesis} (the opposite of what has been termed of old \textit{\is{aposiopesis}aposiopesis}): the speaker begins to articulate, or thinks he begins to articulate, but produces no audible sound (either for want of expiration, or because he does not put his vocal chords in the proper position) till one or two syllables after the beginning of what he intended to say. The phenomenon is particularly frequent, and may become a regular speech-habit, in the case of certain set phrases, but may spread from these to other parts of the language.

Some examples of \textit{prosiopesis} outside the domain of negatives may be given here by way of illustration. Forms of salutation like English \textit{morning} for \textit{Good morning}, Danish (\textit{God}) \textit{dag}, German (\textit{Guten}) \textit{Tag} are frequent in many languages. Further colloquial English \refp{ex:01-03}, colloquial French \refp{ex:01-04}, Swedish \refp{ex:01-05}.\largerpage

%% SG: I've corrected the German word "tag" to "Tag", which would have been the correct spelling even in Jespersen's day

\ea \label{ex:01-03}
\ea (Do you) see?
\ex (Do you re)member that chap?
\ex (Will) that do?
\ex (I'm a)fraid \il{English!not@\textit{not}}not
\ex (The) fact is {\dots}
\ex (When you) come to think of it
\ex (I shall) see you again this afternoon
\ex (Have you) seen the Murrays lately?
\ex (Is) that you, John?
\ex (God) bless you. 
\z
\z

\begin{samepage}

\ea \label{ex:01-04}
\ea (na)turellement \phantom{x} (`naturally')
\ex (En)tends-tu? \phantom{x} (`Are you listening?')
\ex (Est-ce) convenu\textit{?} \phantom{x} (`Is it agreed?')
\ex (Par)faitement \phantom{x} (`perfectly')
\il{French!ne@\textit{ne}}\ex (Je ne me) rappelle plus. \phantom{x} (`I can't recall any more.')
\z
\z
\end{samepage}

\ea \label{ex:01-05}
(Öd)mjukaste tjenare \phantom{x} (`most humble servant')

\z

%% SG: No space in "(Öd)mjukaste", corrected

\is{Jespersen's cycle|(}
\addsec{A rapid sketch of the history of negatives in French, Scandinavian, and English}\label{rapid_sketch}
\is{history of negative expressions|(}
\is{grammaticalization|(}

\emergencystretch=3em % Adjust the value as needed
The interplay of these tendencies---weakening and strengthening, and \is{protraction|(}pro\-traction% Wiktionary says ``the lengthening of a short syllable'', but this isn't in Crystal's Dictionary of Linguistics \& Phonetics (6th ed) and it's a new word for me (PE). Explanation in a footnote? OJ seems to use it only twice; and the second time, it doesn't seem to mean the lengthening of a short syllable. \textbf{Brett:} I don't know it either, and he doesn't use it in \textit{MEG}. I think the reader's guess is as good as ours. % Peter: Let's just ignore it 
---will be seen to lead to curiously similar, though in some respects different developments in Latin with its continuation French, in Scandinavian, and in English. A rapid sketch of the history of negatives in these three languages may, therefore, be an appropriate introduction to the more specified investigations of the following chapters.

The starting point in all three languages is the old negative \il{French!Old French!ne@\textit{ne}}\il{Norse!\textit{ne}}\il{English!ne@\textit{ne}}\textit{ne}, which I take to be (together with the variant \textit{me}) a \is{interjections, primitive}primitive interjection of disgust, accompanied by the facial gesture of contracting the muscles of the nose (Danish \textit{rynke på nesen} (`wrinkle one's nose'), German \textit{die Nase rümpfen}, French \textit{froncer les narines}; the English \textit{to turn}, or \textit{to screw, up one's nose} is not so expressive). This natural origin will account for the fact that negatives beginning with nasals (\textit{n}, \textit{m}) are found in many languages outside the Indo-European family.

\il{Latin!ne@\textit{ne}|(}\label{para:negativebeforeverb}In Latin, % PE: OJ italicizes "Latin", I suppose in order to make it stand out; but as he doesn't similarly italicize "Old Norse" below in "If we turn to Old Norse", I've removed the italics.
then, we have at first sentences like\largerpage

\bigskip

\textbf{Stage 1:} \textit{ne dico} \phantom{x} (`do not say')

\bigskip

This persists with a few verbs only, \textit{nescio}, \textit{nequeo}, \textit{nolo}. \textit{Ne} also enters into the well-known combinations \textit{neque}, \textit{neuter}, \textit{numquam}, \textit{nemo}, \textit{ne {\dots} quidem}, \textit{quin}, etc., and is also used ``as a conjunction'' in subjunctival clauses; further as an ``interrogative particle'' in \textit{scis-ne?} (`you know, don't you?'). But otherwise \textit{ne} is felt to be too weak, and it is strengthened by the addition of \textit{oenum} (`one thing'); \il{Latin!non@\textit{non}|(}the resulting \textit{non} becomes the usual negative adverb\is{adverbs!negative} and like \textit{ne} is generally placed before the verb:\il{Latin!ne@\textit{ne}|)}

\bigskip

\textbf{Stage 2:} \textit{non dico} \phantom{x} (`I do not say')

\bigskip

In Old French, \textit{non}\il{Latin!non@\textit{non}|)} becomes \il{French!Old French!nen@\textit{nen}}\textit{nen}, as in \textit{nenil}, \textit{nenni}, properly `not he, not it', but more usually with further phonetic weakening \il{French!Old French!ne@\textit{ne}|(}\textit{ne}, and thus we get:

\bigskip

\textbf{Stage 3:} \textit{jeo ne di} \phantom{x} (`I do not say')
\il{French!Old French!ne@\textit{ne}|)}
\bigskip

\il{French!ne@\textit{ne}|(}This form of negative expression survives in literary French till our own days in a few combinations, \textit{je ne sais}, \textit{je ne saurais le dire}, \textit{je ne peux}, \textit{n'importe}; but in most cases, the second \textit{ne}, like the first, was felt to be too weak, and a strengthening was found to be necessary, though it is effected in a different way, namely by the addition after the verb, thus separated from \textit{ne}, of some such word as \il{French!mie@\textit{mie}}\textit{mie} (`a crumb'), \il{French!point@\textit{point}}\textit{point} (`a point'), or \il{French!pas@\textit{pas}|(}\textit{pas} (`a step'):

\bigskip

\il{French!n'@\textit{n'}|(}\textbf{Stage 4:} \textit{je ne dis pas} (or rather: \textit{je n' dis pas})

\bigskip

Everyday colloquial French does not stop here: the weak
\textit{ne},\il{French!ne@\textit{ne}|)} \textit{n'} disappears and we have as the provisionally final stage:\il{French!n'@\textit{n'}|)}

\bigskip

\textbf{Stage 5:} \textit{je dis pas}\il{French!pas@\textit{pas}}

\bigskip

If we turn to Old Norse, we first find some remnants of the old \textit{ne} before the verb, inherited from Old Arian, corresponding to Gothic \textit{ni}, Old Saxon and Old High German \il{German!Old High German!ni@\textit{ni}}\textit{ni}, Old English \textit{ne}; thus

\bigskip

%% SG: These examples are Old Norse, not Old English. I can add glosses if you think that would be helpful %% PE: Perhaps helpful to the reader, but probably hard to integrate with the "Stage" layout. So perhaps not.

\textbf{Stage 1:} \textit{Haraldr ne veit} (`Harald does not know'); cf. \textit{þú gefa ne skyldir} (`thou shouldst not give' (\textit{Lokasenna})).

\bigskip

This was strengthened in various ways, by adding \textit{at} (`one thing') corresponding to Gothic \textit{ainata}, or \il{Old Norse!-a@\textit{-a}|(}\textit{a}, which is generally explained as corresponding to Gothic \textit{aiw}, Latin \textit{ævum}, but may according to \ia{Kock, Axel}Kock % I (PE) suppose that this is Axel Kock, described in en:Wikipedia as a Swedish philologist. This is the only occurrence of "Kock" in the book. Can we provide an informed explanation in a footnote? 
%Brett: How's this? % PE: Good.
be merely a weakened form of \textit{at};\il{Old Norse!-a@\textit{-a}|)}\footnote{Likely Axel Kock (1851--1935), a Swedish philologist who made substantial contributions to the study of Old Norse and other Germanic languages. His work on sound changes and word formation in Nordic languages was particularly influential. \eds} both were placed after the verb and eventually became enclitic quasi-suffixes; the result being

\bigskip

\il{Old Norse!-at@\textit{-at}|(}\textbf{Stage 2:} \textit{Haraldr ne veit-at}; or, with a different word order, \textit{ne veit-at Haraldr}

\bigskip

In the latter combination, however, \textit{ne} was dropped through \is{prosiopesis}prosiopesis:

\bigskip

\textbf{Stage 3:} \textit{veit-at Haraldr}

\bigskip

\il{Old Norse!-a@\textit{-a}|(}This form, with \textit{-at}\il{Old Norse!-at@\textit{-at}|)} or \textit{-a} as the negative element, is frequent enough in poetry;\il{Old Norse!-a@\textit{-a}|)} in prose, however, another way of strengthening the negative was preferred as having ``more body'', namely by means of \il{Old Norse!eigi@\textit{eigi}}\il{Icelandic!Old Icelandic!\textit{eigi}}\textit{eigi} or \il{Old Norse!eiki@\textit{eiki}}\il{Icelandic!Old Icelandic!\textit{eiki}}\textit{ekki} after the verb; these also at first must have had a \il{Old Norse!ne@\textit{ne}|(}\textit{ne} before the verb as the bearer of the negative idea, as they are compounded of \textit{ei}, originally `always' like the corresponding Old English \textit{ā}, and Old Norse \textit{eitt} `one (neutr.)' % I (PE) am lost. I'd understand the concept of "neuter 'one'", but this looks like "neutral 'one'", which mystifies me. Or are "take" and "banana" (as opposed to "taking" or "bananas") "neutral" because they lack agreement or other morphs? (NB "neutr." appears elsewhere too. 
%Brett: No, I think this is neuter gender.
% PE: ??? If you're pretty certain, go ahead and expand it to "neuter".
%% SG: eitt is the neuter form of the numeral 'one' in Old Norse. Note that the only Old English form here is _ā_, the other forms are Old Norse
+ \textit{ge}, \textit{gi}, which was at first positive (it corresponds to Gothic \textit{hun}, having a voiced consonant in consequence of weak \is{stress|(}stress; see \citet{delbruck_negativen_1910} for relation to Sanskrit \textit{caná}) but acquired a negative signification through constant employment in negative sentences. This, then, becomes the usual negative in Scandinavian languages; e.g. \il{Danish!ej@\textit{ej}|(}Danish \textit{ej} (now chiefly poetical; colloquial only in a few more or less settled combinations like \textit{nej, jeg vil ej}\il{Danish!ej@\textit{ej}|)}) and \il{Danish!ikke@\textit{ikke}}\textit{ikke} (with regard to \textit{inte} see \hyperref[para:inte]{below}, p.~\pageref{para:inte}). The use of the original negative \textit{ne} with a verb has in these languages disappeared centuries ago, leaving as the only curious remnant the first sound of \il{Danish!nogen@\textit{nogen}}\textit{nogen}, which is, however, a positive pronoun `some, any', from \textit{ne veit}(\textit{ek})\textit{ hverr} `nescio quis'. Sic transit.{\dots}\il{Old Norse!ne@\textit{ne}|)}

\is{position of negative|(}\il{Danish!ikke@\textit{ikke}|(}The Danish \textit{ikke} shares with French colloquial \il{French!pas@\textit{pas}}\textit{pas} the disadvantage of being placed after the verb: \textit{jeg veed ikke} just as \il{French!pas@\textit{pas}}\textit{je sais pas}, even after the verb and subject in cases like \textit{det veed jeg ikke}; but in dependent clauses we have protraction\footnote{With this idiosyncratic term, Jespersen means movement of the negation to an earlier position in the clause -- in Danish, it follows the finite verb in main clauses (\textit{jeg ved ikke} `I know not'), but comes before it in most dependent clauses (\textit{fordi jeg ikke ved} `because I not know'). \eds} of \textit{ikke}: \textit{at jeg ikke veed}; \textit{fordi jeg ikke veed}; etc.\il{Danish!ikke@\textit{ikke}|)}
\is{protraction|)}
% If "protraction" does indeed mean "lengthening of a vowel" (as Wiktionary says), this isn't its meaning here: it seems to mean something like "elaboration".
%Brett: I read it as meaning 'pulling forward'. % Peter: Again, let's not attempt to gloss or annotate it, and instead just leave it unexplained.

%% SG: What Jespersen means here is movement of the negation to an earlier position in the clause -- in Danish, it follows the finite verb in main clauses (_jeg ved ikke_ 'I know not'), but comes before it in most dependent clauses (_fordi jeg ikke ved_ 'because I not know').
\bigskip
In English the development has been along similar lines, though with some interesting new results, due chiefly to changes that have taken place in the Modern English period. The starting point, as in the other languages, was

\bigskip

\il{English!Old English!ne@\textit{ne}|(}\textbf{Stage 1:} \textit{ic ne secge}

\bigskip

This is the prevalent form throughout the Old English period, though the stronger negatives which were used (and required) whenever there was no verb, \textit{na} (from \textit{ne}~+~\textit{a} corresponding to Gothic \textit{aiw}, Old Norse \textit{ei}), \textit{nalles} (`not at all'), and \textit{noht} (from \textit{nawiht}, \textit{nowiht}, originally meaning `nothing'), were by no means rare after the verb to strengthen the preceding \textit{ne}.\il{English!Old English!ne@\textit{ne}|)} The last was the word surviving in Standard English, and thus we get the typical Middle English form

\bigskip

\il{English!Middle English!not@\textit{not}|(}\textbf{Stage 2:} \textit{I ne seye not}

\bigskip

Here \textit{ne} was pronounced with so little stress that it was apt to disappear altogether, and \textit{not} becomes the regular negative in all cases:

\bigskip

\textbf{Stage 3:} \textit{I say not}

\bigskip

This point---the practical disappearance of \textit{ne} and the exclusive use of \textit{not}---was reached in the fifteenth century.\il{English!Middle English!not@\textit{not}|)} Thus far the English development presents an exact parallel to what had happened during the same period in German. Here also we find as the earliest stage (1) \il{German!Old High German!ni@\textit{ni}}\textit{ni} before the verb, then (2) \il{German!Middle High German!ne@\textit{ne}}\textit{ne}, often weakened into \il{German!Middle High German!n-@\textit{n-}}\textit{n-} or \textit{en} (which probably means syllabic \textit{n}) before and \il{German!Middle High German!niht@\textit{niht}}\textit{niht} after the verb; \textit{niht} of course is the compound that corresponds to English \textit{not}; and finally (3) \il{German!nicht@\textit{nicht}|(}\textit{nicht} alone. The rules given in \citet[§310ff]{paul1894mittelhochdeutsche} for the use of \textit{ne} alone and with \textit{niht} and of the latter alone might be applied to Middle English of about the same date with hardly any change except in the form of the words, so close is the correspondence. But German remains at the stage of development reached towards the end of the middle period, when the weak \textit{ne}, \textit{en} had been given up; and thus the negative continues in the awkward position after the verb. We saw the same thing in colloquial \il{French!pas@\textit{pas}}French \textit{pas} and in Danish \il{Danish!ikke@\textit{ikke}}\textit{ikke}; but these are never separated from the verb by so many words as is often the case in German, the result being that the hearer or reader is sometimes bewildered at first and thinks that the sentence is to be understood in a positive sense, till suddenly he comes upon the \textit{nicht}, which changes everything; see, for instance \refp{ex:01-06}. In dependent clauses \textit{nicht}, like other subjuncts, is placed before the verb: \textit{dass er nicht kommt} `that he is not coming'; \textit{wenn er nicht kommt} `if he does not come': \refp{ex:01-08}. I remember feeling the end of \refp{ex:01-07} as something like a shock when reading it in an article by \citet[\href{https://archive.org/details/zeitschrift-fur-volkerpsychologie-7-8/page/153/mode/2up?view=theater}{153}]{gabelentz1870weiteres}.\is{position of negative|)} 

\ea \label{ex:01-06}
Das Leben ist der Güter höchstes nicht\\
`Life is not the highest of goods'\hfill(\href{https://archive.org/details/bwb_W2-CNQ-459/page/146/mode/2up?q=%22Das+Leben+ijt+der+Gitter+Hichftes+nicht%22&view=theater}
{Schiller, \textit{Messina} 4.10})

\z

\ea \label{ex:01-07}
Man unterschätze den deutschen Stil der Zopfzeit, den der Canzleien des vorigen und vorvorigen Jahrhunderts nicht. \\ 
`One should not underestimate the German style of the wig period, or that of the chancelleries of the last and the second to last centuries.' 

\z

\ea \label{ex:01-08}
\ea
Denn der Rigveda kennt die Lautgruppe \emph{skh-}, die ganz den Eindruck einer aus dem Prakrit stammenden Lautverbindung macht, überhaupt nicht.\\
`For the Rigveda knows the sound group \textit{skh-}, which entirely gives the impression of being a sound combination originating from Prakrit, \emph{not at all}.'\hfill
(\href{https://archive.org/details/dasschwacheprt01colluoft/page/66/mode/2up?q=%22ber+Rigocba+fennt%22&view=theater}{Collitz, \textit{Präteritum} 67}) % ??? PE: I can't be sure whether OJ is providing these two quotations, from Collitz and Deutschbein (immediately above and below respectively), (A) as relevant observations about Sanskrit and Modern English, or (B) as German-language illustrations that just happen to be about language. If (A), then the Collitz reference should use BibTeX and the Deutschbein reference should only use BibTeX; if (B), then the Deutschbein reference should not use BibTeX.  

\ex
Das [Frühneuenglisch] hat die Neigung, das Objekt möglichst an das Verbum anzuschliessen, noch nicht.\\ % "frühneuengl." is OJ's own (partial) expansion; but I (PE) have corrected his "object" to Deutschbein's "objekt".
`Early Modern English had \emph{not yet} developed the tendency to place the object as close as possible to the verb.'
\hfill (\href{https://archive.org/details/systemderneueng00deutgoog/page/n45/mode/2up?q=%22hat+die+neigung%22&view=theater}{Deutschbein, \textit{System} 27}) \\ % ??? PE: There had here been two references -- one in the "abbreviations_3" style, one for BibTex -- in one place for the exact same quotation. Believing that Deutschbein is being quoted here not for the substance of what he's writing but instead for its phrasing, I'm plumping for the former. If you disagree, what I removed was: \citet[\href{https://archive.org/details/systemderneueng00deutgoog/page/n45/mode/2up?q=\%22hat+die+neigung\%22&view=theater}{27}]{deutschbein1917system})
\z
\z\il{German!nicht@\textit{nicht}|)}

In English, on the other hand, we witness a development that obviates this disadvantage. \is{do@\textit{do}, auxiliary|(}The Elizabethans began to use the auxiliary \textit{do} indiscriminately in all kinds of sentences, but gradually it was restricted to those sentences in which it served either the purpose of emphasis or a grammatical purpose. In those questions in which the subject is not an interrogatory pronoun, which has to stand first, \textit{do} effects a compromise between the interrogatory word-order (verb--subject) % PE: I've changed "verb hyphen subject" to "verb dash subject" 
and the universal tendency to have the subject before the verb (that is, the verb that means something) as in \textit{Did he come?} (See \citet[\href{https://archive.org/details/progressinlangua00jespuoft/page/92/mode/2up?view=theater}{93}]{jespersen1894progress} for parallels from other languages). And in sentences containing \il{English!not@\textit{not}|(}\textit{not} a similar compromise is achieved by the same means, \textit{not} retaining its place after the verb which indicates tense, number, and person, and yet being placed before the really important verb. Thus we get

\bigskip

\textbf{Stage 4:} \textit{I do not say}.

\bigskip

Note that we have a corresponding word-order in numerous sentences like \textit{I will not say}; \textit{I cannot say}; \textit{I have not said}; etc. But in this position, \textit{not} cannot keep up its strongly stressed pronunciation; and through its weakening, we arrive at the colloquial\il{English!not@\textit{not}|)}\is{stress|)}

\bigskip

\is{contraction of negatives|(}\il{English!n't@\textit{-n't}|(}\textbf{Stage 5:} \textit{I don't say}.\is{Jespersen's cycle|)}

\bigskip

In many combinations, even the sound [t] is often dropped here, and thus \textit{nowiht}, \il{English!nought@\textit{nought}}\textit{nought} has been finally reduced to a simple [n] tagged onto an auxiliary of no particular signification. If we contrast an extremely common pronunciation of the two opposite statements \textit{I can do it} and \textit{I cannot do it}, the negative notion will be found to be expressed by nothing else but a slight change of the vowel [aikæn duˑ it~\vert~aikaˑn duˑ it]. Note also the extreme reduction\is{phonological reduction} in a familiar pronunciation of \textit{I don't know} and \textit{I don't mind} as [ai dn-nou] or [ai d-nou] and [ai dm-maind] or [ai d-maind], where practically nothing is left of the original negative. It is possible that some new device of strengthening may at some future date be required to remedy such reductions.\il{English!n't@\textit{-n't}|)}\is{contraction of negatives|)}

It is interesting to observe that through the stages (4) and (5) the English language has acquired a negative construction that is closely similar to that found in Finnish, where we have a negative auxiliary, inflected in the various persons before an unchanged main verb: \il{Finnish!\textit{sido}|(}\textit{en sido} (`I do not bind'), \textit{et sido} (`thou dost not bind'), \textit{ei sido} (`he does not bind'), \textit{emme sido} (`we {\dots}'), \textit{ette sido} (`you (pl) {\dots}'), \textit{eivät sido} (`they do not bind').\il{Finnish!\textit{sido}|)} \il{Finnish!\textit{sito-}|(}There is, however, the important difference that in Finnish the tense is marked not in the auxiliary, but in the form of the main verb: \textit{en sitonut} (`I did not bind'), \textit{emme sitoneet} (`we did not bind') (\textit{sitonut}, plural \textit{sitoneet} is a participle).\il{Finnish!\textit{sito-}|)}
\is{grammaticalization|)}

\label{para:neveretc}A few things must be added here to supplement the brief sketch of the evolution of English negatives. \is{prefixes!negative|(}The old \textit{ne} in some frequently occurring combinations lost its vowel and was fused with the following word; thus we have the following pairs of positive and negative words:

\il{English!Middle English!be, have, will, would, obsolete forms of@\textit{be}, \textit{have}, \textit{will}, \textit{would}, obsolete forms of|(}
\is{auxiliary verbs|(}
\begin{exe}
\exi{(a)} verbs (given in late Middle English forms):\medskip\\
         \begin{tabular}{@{}ll@{}}
         am & nam\\
         art & nart\\
         is & nis\\
         has & nas\\
         had(de) & nad(de)\\
         was & nas\\
         were(n) & nere(n)\\
         will(e) & \il{English!Middle English!nill@\textit{nill}|(}nill(e)\\
         wolde & nolde\\
         \end{tabular}
\end{exe}

These had all become extinct before the Modern English period, except \textit{nill}, which is found rarely, e.g. \refp{ex:01-10}; twice in pseudo-Shakespearian passages: \refp{ex:01-11}. Shakespeare himself has it only in the combinations \textit{will you}, \textit{nill you} (\href{https://internetshakespeare.uvic.ca/doc/Shr_F1/scene/2.2/index.html#tln-1150}{\textit{Shr} 2.273}) and \textit{will he, nill he} (\href{https://internetshakespeare.uvic.ca/doc/Ham_F1/scene/5.1/index.html#tln-3205}{\textit{Hml} 5.1.19}); and the latter combination (or \textit{will I, nill I}; \textit{will ye, nill ye}, which all would yield the same phonetic result) survives in modern \il{English!nill, nilly@\textit{nill}, \textit{nilly}}\textit{willy-nilly}, rarely spelt as separate words, as in \refp{ex:01-13}, where both the person (\textit{he}) and the tense shows that the whole has really become one unanalyzed adverb. % PE: I sense that part of this refers to both examples (by Byron and Allen) of what's currently (10), but that part refers only to the second (by Allen). Also, "shows" looks at first glance like a mistake for "show"; but on further thought I wonder if OJ meant to say that "the combination of the person (he) and the tense shows...." I'm pretty sure that something is amiss but can't be sure of what it is.
%Brett: I think it's good as is. % PE: OK

\ea \label{ex:01-10}
I nill refuse\hfill(\href{https://archive.org/details/spanishtragedya00kydgoog/page/n68/mode/2up?q=%22I+nill+refuse%22&view=theater}{Kyd, \textit{Spanish} 1.4.7.})
\z

\ea \label{ex:01-11}
\ea
In scorne or friendship, nill I conster whether\hfill(\href{https://internetshakespeare.uvic.ca/doc/PP_O2/complete/index.html#tln-185}{\textit{Pilgrime} 188}) % "scorne" and "conster" in accordance with uvic.ca
\ex I nill relate\hfill(\href{https://internetshakespeare.uvic.ca/doc/Per_Q1/scene/3.0/index.html#tln-1105}{\textit{Pericles} 3.prol.55})
\z
\z

\ea \label{ex:01-13}
\ea
Will I---Nill I\hfill(rimes with \textit{silly}; \href{https://archive.org/details/workslordbyron10unkngoog/page/300/mode/2up?view=theater&q=%22nill+I%22}{Byron, \textit{Juan} 6.118})
\ex {}[other motives] would obtrude themselves, will he, nill he, upon him\\\hfill(\href{https://archive.org/details/womanwhodid00allerich/page/52/mode/2up?q=%22nill+he%2C+upon+him%22&view=theater}{Allen, \textit{Woman} 64})
\z
\z\il{English!Middle English!be, have, will, would, obsolete forms of@\textit{be}, \textit{have}, \textit{will}, \textit{would}, obsolete forms of|)}\il{English!Middle English!nill@\textit{nill}|)}

\begin{exe}
\exi{(b)} other words (given in Modern English forms):\medskip\\
\begin{tabular}{@{}ll@{}}
one, an, a (Old English ān) & \il{English!none@\textit{none}}none, \il{English!no@\textit{no}|(}no\\
aught, ought & \il{English!naught@\textit{naught}}naught, \il{English!nought@\textit{nought}}nought, \il{English!not@\textit{not}}not\\
either & \il{English!neither@\textit{neither}}neither\\
or & \il{English!nor@\textit{nor}}nor\\
ever & \il{English!never@\textit{never}}never\\
\end{tabular}\is{prefixes!negative|)}
\end{exe}

It should be remembered that \textit{no} represents two etymologically distinct combinations: Old English \textit{ne ān} (as in \textit{no man}, also in \textit{nobody}, \textit{nothing}), and Old English \textit{ne}~+~\textit{ā} (as in: \textit{are you ill? No}; also in \textit{nowhere})\il{English!no@\textit{no}|)}; cf. \citet[\ob1914\cb~\href{https://archive.org/details/jespersen-1954-a-modern-english-grammar-on-historical-principles-part-ii-syntax-first-volume/page/424/mode/2up?view=theater}{16.7}]{jespersenMEG2}. % PE: Very strange for a 1917 book to cite a 1954 book
%Brett: But it's a 2024/5 edition. Anyhow, how about "Jespersen (1954 [1914]: 16.7)" or "cf. Jespersen (1954: 16.7)[1914]". for the first option, I could simply add [1914] to the year in the bibtex. For the second, we could add it manually. % PE: Either would be an improvement; if the former doesn't trigger an error, let's adopt it.
%Brett: Sadly, the former does cause sorting problems in the list of references.
% ??? PE: The additional "[1914]" can now be cut, I think.

\il{English!Middle English!ne@\textit{ne}|(}The transition between stages 2 and 3 is seen, for instance, in \textit{Maundevile} (14th~c.),
%% SG: The received English spelling is "Mandeville", but I suppose this is included in the list of sources? %% PE: Yes, Maundevile is how it appears there. (OJ used "Mandeville". But elsewhere he was most pernickety about spellings.)
where \textit{ne} by itself is rare \refp{ex:01-15} but is more frequent with some other negative word \refp{ex:01-16}. But \textit{ne} is not required, see e.g. \refp{ex:01-21}. A late example of isolated \textit{ne} is \refp{ex:01-22}.

%% SG: Note that the language in the following examples is Middle English, not Old English (the Old English period ends around 1100 CE; the English version of Mandeville's Travels is from the .

\ea \label{ex:01-15} 
 \gll ȝif the snow ne were\\
 if the snow not were\\
 \trans `if the snow were not there'\hfill(\href{https://archive.org/details/voiageandtravai02hallgoog/page/n165/mode/2up?q=%22zif+the+snow+ne+were%22&view=theater}{130})
\z
%% SG: The first word in this example should be spelt <ȝif> with the character yogh (Unicode U+021C). Jespersen is quoting the Middle English source from a printed edition where a <z> was substituted for the yogh (a common practice in earlier editions because a yogh wasn't available in most typefaces)

\ea \label{ex:01-16}
\ea
 \gll it ne reynethe not\\
 it not rains not\\
 \trans `it does not rain'\hfill(\href{https://archive.org/details/voiageandtravai02hallgoog/page/n81/mode/2up?q=%22it+ne+reynethe+not%22&view=theater}{45})
 
\ex
 \gll ȝee ne schulle not suffre\\
 you not shall not suffer\\ % "yee" is OJ's mistake. It's "zee"
 %% SG: See my previous comment -- OJ definitely knew that the <z> in the edition actually represented a yogh, so he probably substituted a <y>, which better reflects the pronunciation
 \trans `you shall not suffer'\hfill(\href{https://archive.org/details/voiageandtravai02hallgoog/page/n87/mode/2up?q=%22zee+ne+schulle+not%22&view=theater}{51})

 
\ex
 \gll ne ben not\\
 not be not\\
 \trans `is not'\hfill(\href{https://archive.org/details/voiageandtravai02hallgoog/page/n87/mode/2up?q=%22ne+ben+not+so+grete%22&view=theater}{52})

\ex
 \gll there nys nouther mete for hors ne watre\\
 there is neither food for horse nor water\\
 \trans `there is neither food for horse nor water'\hfill(\href{https://archive.org/details/voiageandtravai02hallgoog/page/n93/mode/2up?q=%22there+nys+nouther+mete%22&view=theater}{58})

\ex
 \gll ne {\dots} nevere\\
 not {\dots} never\\
 \trans `never'\hfill(\href{https://archive.org/details/voiageandtravai02hallgoog/page/n217/mode/2up?q=%22ne+meeven+nevere%22&view=theater}{181})

\z
\z

\ea \label{ex:01-21}
thei may not enlargen it [sc Egypt] toward the Desert, for defaute of Watre. {\dots} For there it reyneth not but litylle\hfill(\href{https://archive.org/details/voiageandtravai02hallgoog/page/n81/mode/2up?q=%22thei+may+not+enlargen%22&view=theater}{45}) % OJ somehow managed to garble this.
\z

\ea \label{ex:01-22}
he ne can\hfill(\href{https://archive.org/details/gamFmergurtonsne00stiluoft/page/46/mode/2up?q=%22Because+for+lack+of+light+discern+him+he+ne+can.%22}{\textit{Gammer} 140}; the usual negative in that play is \textit{not})\il{English!Middle English!ne@\textit{ne}|)}
\z

Before the \textit{do}-construction was fully developed, there was a certain \is{position of negative}tendency to place \textit{not} before the verb, in all kinds of sentences, thus not only in dependent clauses (the difference in word-order between main sentences and dependent clauses, which we have alluded to in Scandinavian and German, was never carried through in English). The word-order in \textit{And if I not performe, God let me neuer thrive} for \textit{performe not} is considered by \citet[\href{https://quod.lib.umich.edu/e/eebo2/A68619.0001.001/1:6?rgn=div1;subview=detail;type=simple;view=fulltext;q1=if+I+not+performe}{262}]{puttenham1589arte} as a ``pardonable fault'' which ``many times giues a pretie grace vnto the speech''; it is pretty frequent in Shakespeare, see \citet[\href{https://www.perseus.tufts.edu/hopper/text?doc=Perseus\%3Atext\%3A1999.03.0079\%3Aalphabetic+letter\%3DN\%3Aentry+group\%3D8\%3Aentry\%3DNot}{779}]{schmidt1886}, but is rare after the seventeenth % PE: OJ has "17th", but I've spelled this out for consistency with the content of the following paragraph.
century. Examples: \refp{ex:01-23}.

\ea \label{ex:01-23}
\ea
it not appeares to me\hfill(\href{https://internetshakespeare.uvic.ca/doc/2H4_F1/scene/4.1/index.html#tln-1970}{Shakespeare, \textit{H4B} 4.1.107})
\ex
For who not needs, shall neuer lacke a frend\\\hfill(\href{https://internetshakespeare.uvic.ca/doc/Ham_F1/scene/3.2/index.html#tln-2075}{Shakespeare, \textit{Hml} 3.2.217})
\ex
I meruell our mild husband Not met vs on the way\\\hfill(\href{https://internetshakespeare.uvic.ca/doc/Lr_F1/scene/4.2/}{Shakespeare, \textit{Lr} 4.2.1}
\ex
It is the cowish terror of his spirit That dares not vndertake: Hee'l not feele wrongs Which tye him to an answer: our wishes on the way May proue effects. \hfill (both orders closely together, \href{https://internetshakespeare.uvic.ca/doc/Lr_F1/scene/4.2/index.html#tln-2280}{ibid 4.2.50}) % I (PE) don't understand either [abridged]
%Brett: Sh has "dare not" followed by "not feele". Seems fine. % PE: Got it. I've added the quote; I hope this is an improvement.
\ex
I not doubt\hfill(\href{https://internetshakespeare.uvic.ca/doc/Tmp_F1/scene/2.1/index.html#tln-790}{Shakespeare, \textit{Tp} 2.1.121})
\ex
if I not revenge {\dots} Thy sufferings\hfill(\href{https://archive.org/details/venicepreservdor00otwa/page/42/mode/2up?q=%22if+I+not+revenge%22&view=theater}{Otway, \textit{Venice} 4}) % OJ omits a whole line
\ex
the cups That cheer but not inebriate\hfill(\href{https://archive.org/details/in.ernet.dli.2015.186503/page/n29/mode/2up?q=%22cheer%22&view=theater}{Cowper, \textit{Task} 4.39})
\ex
Himself not lives, but is a thing that cries\hfill(\href{https://archive.org/details/poemsbyrupert00broorich/page/22/mode/2up?q=%22Himself+not+lives%2C+but+is+a+thing+that+cries%22}{Brooke, \textit{Poems} 23})
\z
\z

When \textit{do} became the ordinary accompaniment of \textit{not}, it was not at first extended to all verbs; besides the well-known instances with \textit{can}, \textit{may}, \textit{must}, \textit{will}, \textit{shall}, \textit{am}, \textit{have}, \textit{dare}, \textit{need}, \textit{ought} we must here mention \is{know@`know', grammar of words meaning|(}\textit{know}, which now takes \textit{do}, but was long used in the form \textit{know not}, thus pretty regularly in the seventeenth and often in the eighteenth and even in the first part of the nineteenth century. In poetry, forms without \textit{do} are by no means rare, but they are now felt as archaisms, and as such must also be considered those instances in which prose writers dispense with \textit{do}.\is{do@\textit{do}, auxiliary|)} In some instances, this is probably done in direct imitation of Biblical usage, thus in \refp{ex:01-30}; cf. \refp{ex:01-31}. Perhaps also in \refp{ex:01-32} --- this combination occurs in \href{https://archive.org/details/authorizedversio05wrig/page/130/mode/2up?q=%22and+they+vnderstood+not%22&view=theater}{\textit{Luke} 2.50} and elsewhere in the Bible.

\ea \label{ex:01-30}
Somehow, in a way that Darius comprehended not\\\hfill(\href{https://archive.org/details/clayhanger01benngoog/page/36/mode/2up?q=%22that+Darius%22&view=theater}{Bennett, \textit{Clayhanger} 1.47})
\z

\ea \label{ex:01-31} And the light shineth in darknesse, and the darknesse comprehended it not.\hfill(\href{https://archive.org/details/authorizedversio05wrig/page/200/mode/2up?q=%22And+the+light+shineth+in+darknesse%2C+and+the+darknesse+comprehended+it+not%22&view=theater}{AV \textit{John} 1.5})
\z

\ea \label{ex:01-32}
``Isn't Haddington staying here?'' --- ``I don't know. I understood not.''\\\hfill(\href{https://archive.org/details/fatherstafford00hope/page/78/mode/2up?q=%22+don%27t+know.+I+understood+not%22&view=theater}{Hope, \textit{Father} 43}) % The edition linked to has "staying here"; OJ, quoting a different edition, has "breaking up". Should we point out this discrepancy? (I don't think the choice between "staying here" and "breaking up" affects what OJ wants to say.)
%Brett: I don't think we should bother.
\z
\is{auxiliary verbs|)}

There is a curious agreement among different languages in the kind of verbs that tend to keep up an old type of negative construction after it has been abandoned in other verbs; cf. Latin \textit{nolo}, English \textit{nill}, \il{German!Middle High German!en@\textit{en}}Middle High German \textit{en will} and Latin \textit{ne scio}, French \textit{je ne sais}, Middle High German \textit{i-n weiz}, English \textit{I know not}. These syntactical correspondences must, of course, have developed independently in each language---in consequence of natural human tendencies on a common basis.\is{know@`know', grammar of words meaning|)} (But I do not believe in Miklosich's explanation which is accepted by \citet[\href{https://archive.org/details/grundrissderver01delbgoog/page/522/mode/2up?view=theater&q=Miklosich}{523}]{delbruck1897vergleichende}.)\footnote{It appears that the Slovene philologist Franz Miklosich (Franc Miklošič, 1813--1891) suggests that in some languages, including Slavic, negation is not merely an external element modifying the verb but becomes part of the verb's essence, effectively altering the verb's meaning. \eds}
\ia{Miklosich@Miklosich, Franz, aka Franc Miklošič}

\is{history of negative expressions|)}
