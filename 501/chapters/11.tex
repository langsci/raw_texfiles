\ChapterAndMark{English Verbal Forms in \textit{n't}} 
\label{ch:11}
\is{negation,suffixex|(}
\il{English!n't@\textit{-n't}|(}
\is{auxiliary verbs|(}

\is{word order}
\textit{Not} was attracted to the verb, even before it was reduced to \textit{n't} as an integral part of a coalesced verbal form; thus instead of \textit{will I not} we find 
\il{English!Middle English!wol@\textit{wol}}\textit{wol not I} as early as Chaucer (\href{https://archive.org/details/completeworksofg04chauuoft/completeworksofg04chauuoft/page/90/mode/2up?q=%22wol+nat+I%22&view=theater}{\textit{Miller's Prologue} A~3131}); both positions in (\ref{ex:11-01}).

\ea \label{ex:11-01}
\gll Wol nat oure lord yet leve his vanytee? Wol he nat wedde?\\
will not our lord yet leave his vanity will he not wed\\
\glt `Will our lord not yet abandon his vanity? Will he not marry?'\\\hfill(\href{https://archive.org/details/completeworksofg04chauuoft/completeworksofg04chauuoft/page/396/mode/2up?q=%22wol+nat+our%22&view=theater}{Chaucer, \textit{Clerkes} E~250})% The spelling in the source linked to differs; I haven't adjusted OJ's spelling
\footnote{This and the following two chapters deal exclusively with English grammar. (Jespersen)}
\z 

From Modern English times may be noted (\ref{ex:11-02}).
% Brett: This doesn't work the way it is. Peter, what do you think?
% Peter: We're going to encounter problems such as this rather often. Sometimes, ingredients may have to be shunted around. Here I suggest relegation to a footnote, whereby: QUOTE
% From Modern English times may be noted:
% (2) a-q 
% UNQUOTE. Yes, what's now (3) becomes (2)q. Attached to (2)p (i.e. the Austen quotation): QUOTE
% \footnote{Thus continually in conversations: \textit{is not he{\dots}}; \textit{will not you{\dots}}; \textit{could not he{\dots}}; etc (ibid). (Jespersen)}
% UNQUOTE The editorial preface should make it clear that, with the aim of ease of exposition, this new edition has made alterations such as the integration of addenda and the relegation of some of OJ's body text into footnotes.
%
% PE: How about if I do something like the following: (i) Remove from the paragraph immediately before (2): "; thus continually in conversations: is not he. ... will not you. ... could not he ..., etc.; ibid; 3". (ii) Append to (2)p the footnote "Thus continually in conversations: is not he. ... will not you. ... could not he ..., etc.; ibid. (Jespersen)" (iii) What is now (3) (Disraeli), turn into (2)q.
%Brett: OK, I've implemented this. The only thing I'd suggest is removing (Jespersen), which goes without saying.
% PE I've removed it.

\ea \label{ex:11-02}
\ea
art not thou Pryamus sone {\dots} art not thou one of the possessours\\\hfill(\href{https://archive.org/details/TheHistoryOfReynardTheFoxArber/page/n113/mode/2up?q=%22thou+Pryamus+sone%22&view=theater}{Caxton, \textit{Reynard} 84})
\ex
Will not ye, then will they\hfill(\href{https://archive.org/details/roisterdoister00udalgoog/page/n58/mode/2up?view=theater&q=%22will+not+ye%22}{\textit{Roister} 52})
\ex
Did not you make me a letter brother?\hfill(\href{https://archive.org/details/roisterdoister00udalgoog/page/n62/mode/2up?view=theater&q=%22make+me+a+letter%22}{ibid 56}) % Restored "brother?"
\ex
doe not ye. {\dots} be not ye. {\dots}\hfill(\href{https://archive.org/details/roisterdoister00udalgoog/page/n84/mode/2up?view=theater&q=%22doe+not+ye%22}{ibid 79})
\ex
Is not the causer {\dots}\hfill(\href{https://internetshakespeare.uvic.ca/doc/R3_F1/scene/1.2/index.html#tln-300}{Shakespeare, \textit{R3} 1.2.117})
\ex
So do not I\hfill(\href{https://internetshakespeare.uvic.ca/doc/R3_F1/scene/1.4/index.html#tln-1110}{ibid 1.4.286})
\ex
Cannot my Lord Stanley sleepe\hfill(\href{https://internetshakespeare.uvic.ca/doc/R3_F1/scene/3.2/index.html#tln-1800}{ibid 3.2.6}) % OJ has "Cannot thy master sleep". The first quarto has "Cannot thy Master sleepe", but I (PE) see no reason why OJ would have wanted to use it; so rather than cite it (with the complication of pointing out that it's from the quarto), I've just foliofied the quotation.https://internetshakespeare.uvic.ca/doc/R3_Q1/complete/index.html#tln-1990
\ex
Had not you come\hfill(\href{https://internetshakespeare.uvic.ca/doc/R3_Q1/complete/index.html#tln-1990}{ibid, quarto, 3.4.29})\footnote{The first folio instead has \textit{Had you not come}. \eds} % First folio: https://internetshakespeare.uvic.ca/doc/R3_F1/scene/3.4/index.html#tln-1990 (Not linked from the footnote, as linking from a footnote displeases the Overleaf software.)
\ex
Are not you the chief womã\hfill(\href{https://internetshakespeare.uvic.ca/doc/LLL_F1/scene/4.1/index.html#tln-1025}{Shakespeare, \textit{LLL} 4.1.51}) % "the chief womã?" restored to the end: without it, this seemed a bit undercontextualized.
\ex
Now will not I deliuer his letter\hfill(\href{https://internetshakespeare.uvic.ca/doc/TN_F1/scene/3.4/index.html#tln-1700}{Shakespeare, \textit{Tw} 3.4.202}) 
\ex
Am not I your Rosalind\hfill(\href{https://internetshakespeare.uvic.ca/doc/AYL_F1/scene/4.1/index.html#tln-2000}{Shakespeare, \textit{As} 4.1.89})
\ex
Doe not I hate them {\dots} and am not I grieued\hfill(\href{https://www.kingjamesbibleonline.org/1611_Psalms-139-21/}{AV \textit{Psalms} 139.21})
\ex
did not I execute the scheme, did not I run the whole risque? Should not I have suffered the whole punishment if I had been taken? and is not the labourer worthy of his hire?\hfill(\href{https://archive.org/details/bim_eighteenth-century_jonathan-wild-the-his_fielding-henry_1795/page/28/mode/2up?q=%22not+I+execute+the%22&view=theater}{Fielding, \textit{Jonathan} 3.431)} % Original has question mark after "taken"
\ex
were not these men of honour?\hfill(\href{https://archive.org/details/bim_eighteenth-century_jonathan-wild-the-his_fielding-henry_1795/page/46/mode/2up?q=%22were+not+these+men%22&view=theater}{ibid 448})
\ex
Had not you better sell them?\hfill(\href{https://archive.org/details/benjaminfrautobio00franrich/page/298/mode/2up?q=%22had+not+you+better+sell%22&view=theater}{Franklin, \textit{Autobiography} 159})
\ex
They are wanted in the farm, Mr. Bennet, are not they?\footnote{Thus continually in conversations: \textit{is not he{\dots}}; \textit{will not you{\dots}}; \textit{could not he{\dots}}; etc. (ibid). \mbox{(Jespersen)}}\\ \hfill(\href{https://archive.org/details/prideprejudice00aust/page/40/mode/2up?q=%22wanted+in+the+farm%22&view=theater}{Austen, \textit{Pride} 40})
\ex \label{ex:11-18}
 {\dots} had not he instinctively felt \dots\hfill(\href{https://archive.org/details/lothairb01disr/page/n27/mode/2up?q=%22had+not+he%22&view=theater}{Disraeli, \textit{Lothair} 7}) % OJ says "Beaconsfield L. 7" but his bibliography explains that "Beaconsfield" is Disraeli. Believing that the author is better known as "Disraeli", I (PE) have made the change here.
\z
\z

There is some vacillation between the two word-orders; for Shakespeare, see \xref{ex:11-19}. % PE: Previously "we have (3), but the first quarto has Doth she not.…"; which didn't seem satisfactory.
Swift in his \textit{Journal to Stella} generally has \textit{did not I}, \textit{should not I}, etc., but sometimes as \href{https://archive.org/details/journaltostellae00swifuoft/page/16/mode/2up?q=%22did+I+not+say%22&view=theater}{p.~17} \textit{Did I not say}; and the latter word-order is even nowadays affected by many writers, though \textit{Didn't I say} has now for generations been the only natural form in everyday speech.

\ea \label{ex:11-19}
\ea 
Doth not she thinke me an old murtherer\\\hfill(\href{https://internetshakespeare.uvic.ca/doc/Rom_F1/scene/3.3/index.html#tln-1910}{Shakespeare, \textit{Rom} 3.3 (first folio)}) % OJ mysteriously omits act, scene, line numbers and instead writes "1786"
\ex 
Doth she not thinke me an olde murderer\\\hfill(\href{https://internetshakespeare.uvic.ca/doc/Rom_Q1/scene/3.3/index.html#tln-1910}{Shakespeare, \textit{Rom} 3.3 (first quarto)})
\z
\z

\is{corruption of language, alleged|(}
The contracted forms seem to have come into use in speech, though not yet in writing, about the year 1600. In a few instances (extremely few) they may be inferred from the metre in Shakespeare, though the full form is written, thus (\ref{ex:11-20}) 

\il{English!cannot@\textit{cannot}}
\ea \label{ex:11-20}
\ea
``\emph{Are not} you a strumpet?'' --- ``No, as I am a Christian.''\hfill(\href{https://internetshakespeare.uvic.ca/doc/Oth_F1/scene/4.2/index.html#tln-2775}{\textit{Oth} 4.2.82}) % added punctuation
\ex \label{ex:WS11-1}
But neuer taynt my loue. I \emph{cannot} say whore\hfill(\href{https://internetshakespeare.uvic.ca/doc/Oth_F1/scene/4.2/index.html#tln-2875}{ibid 4.2.161}) % "Loue" and "Whore" are both capitalized in the original: no surprise. What is surprising is that OJ retains the capitalization of "Whore". I (PE) suggest that we regularize this as "whore".
%Brett: done
\z
\z

\noindent (but \textit{cant} in \href{https://internetshakespeare.uvic.ca/doc/AWW_F1/scene/1.3/index.html#tln-490}{Shakespeare, \textit{Alls} 1.3.171} % Why the "F"? (Perhaps to emphasize that this is in the folio? internetshakespeare.uvic.ca doesn't currently offer any quarto, so I (PE) haven't checked.) Suggest simply deleting the "F".
%Brett: done
stands for \textit{can it} [be]\footnote{The line is Helena's, ``So I were not his sister, cant no other.'' She is expressing a wish not to be perceived as Bertram's sister, rhetorically asking if there is no other possible way: `If only I were not seen as his sister---can it be no other way?' \eds}). % Even if one knows "cant" means "can it", the particular passage is hard to understand. I (PE) would attempt a very short footnote explaining it ... if only I could understand it myself.
%Brett: done   % PE: Excellent!

Van Dam's (\citeyear[\href{https://archive.org/details/williamshakespea00dambrich/page/154/mode/2up?view=theater}{155}]{vandam1900william}) examples 
% Title is "William Shakespeare, Prosody and Text" 
are most of them questionable, and some unquestionably wrong. \citet[\href{https://archive.org/details/der-vers-in-shaksperes-dramen/page/n47/mode/2up?view=theater}{39}]{konig1888vers} 
%König (\href{https://archive.org/details/der-vers-in-shaksperes-dramen/page/n47/mode/2up?view=theater}{\textit{Vers in Shaksperes dramen} 39}) 
has only the following instances: \textit{Othello} 4.2.161 (as \xref{ex:WS11-1} above), \href{https://internetshakespeare.uvic.ca/doc/1H6_F1/scene/2.2/index.html#tln-815}{\textit{I~Henry VI} 2.2.47} (\textit{may not}), \href{https://internetshakespeare.uvic.ca/doc/H5_F1/scene/4.5/index.html#tln-2460}{\textit{Henry~V} 4.5.6} (but the folio arranges the line: \textit{O meschante Fortune, do not runne away}---with \textit{do not} as two syllables), % I (PE) am not certain what OJ is saying here. It's not about the folio as opposed to the first quarto, as the first quarto (available at uvic.ca) skips this material
%Brett: I think, Jespersen is not comparing the folio to the First Quarto, but rather highlighting the metrical arrangement in the folio as evidence of the transitional state of negation forms.
\href{https://internetshakespeare.uvic.ca/doc/Err_F1/scene/2.1/index.html#tln-345}{\textit{Errors} 2.1.68} (\textit{know not}; line metrically doubtful).

In writing, the forms in \textit{n't} make their appearance about 1660 and are already frequent in Dryden's, Congreve's, and Farquhar's comedies. Addison in the \href{https://archive.org/details/spectatornewedre00addiuoft/page/202/mode/2up?q=%22very+much+untuned%22&view=theater}{\textit{Spectator} no.~135} speaks of \textit{mayn't}, \textit{cann't}, \textit{sha'n't}, \textit{won't}, and the like as having ``very much untuned our language, and clogged it with consonants''. % He capitalizes "Language" and "Consonants"
Swift also (in a \href{https://en.wikisource.org/wiki/Page%3AThe_Works_of_the_Rev._Jonathan_Swift%2C_Volume_5.djvu/202}{letter}) brands as examples of ``the continual corruption of our English tongue'' such forms as \textit{cou'dn't}, \textit{ha'n't}, \textit{can't}, \textit{shan't}; but nevertheless he uses some of them very often in his \textit{Journal to Stella}. 

Among the forms there are some that are so simple that they call for no remark, thus

\phantom{a}

\begin{tabular}{@{}lll@{}}
\textit{mayn't}& [meint]\\
\textit{hadn't}& [hædnt]\\
\textit{didn't}& [didnt]\\
\textit{couldn't}& [kudnt]\\
\textit{wouldn't}& [wudnt]\\
\textit{shouldn't}& [ʃudnt]\\
\textit{mightn't}& [maitnt]\\
\textit{daren't}& [dɛˑənt]\\
\textit{mustn't}& [mʌsnt] with natural dropping of [t]\\ 
\end{tabular}\\
\jambox*{\citep[\href{https://archive.org/details/a-modern-english-grammar-on-historical-principles.-jespersen-otto-1860-1943-hais/page/224/mode/2up?q=mustn\%27t\&view=theater}{7.73}]{jespersenMEG1}}

\phantom{a}

Thus also

\phantom{a}

\begin{tabular}{@{}ll@{}}
\textit{hasn't}& [hæznt]\\
\textit{isn't}& [iznt]\\
\textit{doesn't}& [dʌznt]\\
\textit{haven't}& [hævnt]\\
\textit{aren't}& [aˑnt]
\il{English!aren't@\textit{aren't}}
\end{tabular}

\phantom{a}

\noindent are simple enough, but it should be noted that these are recent restitutions after \textit{has}, \textit{is}, etc., which have succeeded, partially at least, in ousting other forms developed formerly through phonetic shortening, see below. % PE: The description of other, ousted forms starts very soon and takes up a number of pages; therefore I think that simple "below", with no page number(s), is enough.


\il{English!can't@\textit{can't}}
\il{English!cannot@\textit{cannot}}
\textit{Cannot} [kæn(n)ɔt] becomes \textit{can't} with a different vowel, long [aˑ]; Otway (\href{https://archive.org/details/venicepreservdor00otwa/page/68/mode/2up?q=%22cannot%22&view=theater}{\textit{Venice preserv'd}}) writes \textit{cannot}, but pronounces it in one syllable. Congreve (\href{https://archive.org/details/in.ernet.dli.2015.219151/page/n207/mode/2up?q=%22can%27t%22&view=theater}{\textit{Love for love} 4.1}) has \textit{can't}. In the same way, with additional dropping of [l], \textit{shall not} becomes [ʃaˑnt]. The spelling was not, and is not yet, settled; \href{https://archive.org/details/in.ernet.dli.2015.271834/page/n611/mode/2up?view=theater}{\textit{NED}, \textit{Shall} \textit{v.} 6~b} records \textit{sha'nt} from 1664, 
\il{English!sha'n't@\textit{sha'n't}}\textit{shan't} from 1675, \textit{shann't} from 1682 (besides Dryden's \textit{shan'not} 1668); now both 
\il{English!shan't@\textit{shan't}}\textit{shan't} and 
\il{English!sha'n't@\textit{sha'n't}}\textit{sha'n't} are in use. % There are two lemmas for "shall"; the first is for a "substantive".
For the long [aˑ] in these see \citet[300 \href{https://archive.org/details/a-modern-english-grammar-on-historical-principles.-jespersen-otto-1860-1943-hais/page/300/mode/2up?q=sha\%27n\%27t&view=theater}{\S10.552}]{jespersenMEG1}.

\bigskip

\il{English!am not@\textit{am not}}
\il{English!an't@\textit{an't}|(}
In a similar way, I take it that \textit{am not} has become [aˑnt] with lengthening of the vowel and dropping of [m]. This may have been the actual pronunciation meant by the spelling \textit{an't} (cf. \textit{can't}, \textit{shan't}) in earlier times, see e.g. (\ref{ex:11-22}).

\ea \label{ex:11-22}
\ea
I a'n't well\hfill(\href{https://archive.org/details/cu31924013200898/page/n125/mode/2up?q=%22a%27n%27t+well%22&view=theater}{J. Swift, \textit{Conversation} 90}; also \href{https://archive.org/details/cu31924013200898/page/n131/mode/2up?q=%22a%27n%27t+well%22&view=theater}{ibid 97}) 
\ex
I an't vexed.\hfill(\href{https://archive.org/details/journaltostellae00swifuoft/page/74/mode/2up?q=%22an%27t+vexed%22&view=theater}{J. Swift, \textit{Journal} 75})
\ex
I an't sleepy\hfill(\href{https://archive.org/details/journaltostellae00swifuoft/page/82/mode/2up?q=%22an%27t+sleepy%22&view=theater}{ibid 83})
\ex
an't I a reasonable creature?\hfill(\href{https://archive.org/details/journaltostellae00swifuoft/page/152/mode/2up?q=%22an%27t+I%22&view=theater}{ibid 152}) % "a reasonable creature?" restored
\ex
I an't to be a trades-man; I am to be a gentleman: I an't to go to school.\hfill(\href{https://books.google.co.jp/books?redir_esc=y&hl=ja&id=WopaAAAAMAAJ&q=am+to+be+a+gentleman#v=snippet&q=trades-man&f=false}{Defoe, \textit{Gentleman} 98}) % Hyphen restored within "trades-man"
\ex
I an't deaf\hfill(\href{https://archive.org/details/in.ernet.dli.2015.219151/page/n187/mode/2up?q=%22an%27t+deaf%22&view=theater}{Congreve, \textit{Love} 3.3})
\ex
I an't calf enough\hfill(\href{https://archive.org/details/in.ernet.dli.2015.219151/page/n191/mode/2up?q=%22calf+enough%22&view=theater}{ibid 251}) % Original has "Calf" capitalized
\ex 
\ob Sir Oliver:] an't I rather too smartly dressed\hfill(\href{https://archive.org/details/schoolforscandal/page/n49/mode/2up?q=an%27t&view=theater}{Sheridan, \textit{School} 3.1})
\ex 
{}[Sir Peter:] an't I to be in a good humour\hfill(\href{https://archive.org/details/schoolforscandal/page/n51/mode/2up?q=an%27t&view=theater}{ibid 3.1})
\ex
I an't the least astonished at it\hfill(\href{https://archive.org/details/sensesensibility00austrich/page/252/mode/2up?q=%22least+astonished+at+it%22&view=theater}{Austen, \textit{Sense} 280})
\ex
I an't so fond of his company\hfill(\href{https://archive.org/details/christmascarol0000char_h5c8/page/98/mode/2up?q=%22an%27t+so+fond%22&view=theater}{Dickens, \textit{Carol} 59} (vulgar))
\ex
An't I good enough?\hfill(\href{https://upload.wikimedia.org/wikipedia/commons/1/1f/The_old_wives%27_tale%2C_a_novel_of_life_%28IA_cu31924014151611%29.pdf}{Bennett, \textit{Wives} 1.152}) % ??S Bulky PDF; we should look for a better alternative. NB although a plus of this PDF is that it has "an't": by contrast, most editions (including Project Gutenberg's) instead have "Ain't I good enough?".
\ex
You are what my wife calls intellectual. I an't, a bit.\\\hfill(\href{https://ia601301.us.archive.org/14/items/cu31924014338226/cu31924014338226.pdf}{James, \textit{American} 1.37}) % ??S Bulky PDF; we should look for a better alternative. NB https://archive.org/details/american01jamegoog/page/n26/mode/2up?q=%22you+are+what+my+wife%22&view=theater (a page within an edition also linked to within chapter 4) has "I ain't, a bit", as does the Project Gutenberg file.
\z
\z
\noindent Cf. below (p.~\pageref{ch11-aint}), on \textit{ain't}.\il{English!ain't@\textit{ain't}}

\citet[\href{https://archive.org/details/bim_eighteenth-century_the-principles-of-the-en_elphinston-james_1765_1/page/134/mode/2up?view=theater&q=sink}{1.134}]{elphinston1765principles} 
% OJ has "Elphinstone 1765", but (unusually) there's no "e" at the end of this author's surname and 1765 is merely the year of publication. 
mentions \textit{an't} for \textit{am not} with ``sinking'' of \textit{m} and \textit{o} but does not specify the vowel sound.
\il{English!an't@\textit{an't}|)}


\il{English!aren't@\textit{aren't}|(}
Nowadays [aˑnt] is frequently heard, especially in tag-questions: \textit{I'm a bad boy,} [aˑnt ai] \textit{?}; but when authors want to write it, they are naturally induced to write \textit{aren't}, as \textit{r} has become mute in such combinations, and the form then looks as if it originated in a mistaken use of the plural instead of the singular (which is in itself absurd, as no one would think of using [aˑnt it] or [aˑnt hiˑ]). I find the spelling \textit{aren't I} or \textit{arn't I} pretty frequently in George Eliot (\ref{ex:11-35}), but only to represent vulgar or dialectal speech.

\ea \label{ex:11-35}
\ea
I'm no reader, I aren't.\hfill(\href{https://archive.org/details/millonfloss0009geor/page/28/mode/2up?q=%22aren%27t%22&view=theater}{\textit{Mill} 1.34})
\ex Aren't I a good brother to you?\hfill(\href{https://archive.org/details/millonfloss0009geor/page/36/mode/2up?q=%22aren%27t%22&view=theater}{ibid 1.43})
\ex I aren't frighted\hfill(\href{https://archive.org/details/millonfloss0009geor/page/52/mode/2up?q=%22aren%27t%22&view=theater}{ibid 1.63})
\ex {[}unidentified{]} \hfill(\href{}{ibid 2.164}) % ??? I (PE) can't find anything for this. What shall we do here?
\ex I aren't like a bird-clapper\hfill(\href{https://archive.org/details/adambede00eliouoft/page/n463/mode/2up?q=%22aren%27t%22&view=theater}{\textit{Adam} 441})
\ex Am I a gardener as knows his business, or aren't I, Mills?\hfill(\href{https://archive.org/details/adambede00eliouoft/page/n473/mode/2up?q=%22aren%27t%22&view=theater}{ibid 451})
\ex ``I aren't a-going to try and 'bate your price.{\dots}'' --- ``{\dots} I aren't a turn-tail cur.''\hfill(\href{https://archive.org/details/silasmarnerbygeo00elio/page/58/mode/2up?q=%22aren%27t%22&view=theater}{\textit{Silas} 84})
\ex mother 'ull be in trouble as I aren't there\hfill(\href{https://archive.org/details/silasmarnerbygeo00elio/page/156/mode/2up?q=%22aren%27t%22&view=theater}{ibid 226})
\z \z
\is{corruption of language, alleged|)}

\is{child language}
In the younger generation of writers, however, it is also found as belonging to educated speakers: (\ref{ex:11-36}).

\ea \label{ex:11-36}
\ea
I am always smart. Aren't I, Mr. Worthing?\hfill(\href{https://archive.org/details/bub_gb_4HIWAAAAYAAJ/page/n35/mode/2up?q=%22I+am+always+smart%22&view=theater}{Wilde, \textit{Importance} 10}) % "Mr. Worthing" restored. Some editions instead have "Am I not, Mr. Worthing?". An example: https://archive.org/details/importanceofbein00wildiala/page/24/mode/2up?view=theater&q=%22i+am+always+smart%22
\ex
Aren't I a wise woman, Jack?\hfill(\href{https://archive.org/details/dododetailofday00bensuoft/page/134/mode/2up?view=theater&q=%22wise+woman%22}{E. F. Benson, \textit{Dodo} 126}) % Restored "Jack"
\ex
I am a very wonderful woman, aren't I\hfill(\href{https://archive.org/details/dodosecond00bensiala/page/178/mode/2up?view=theater&q=%22very+wonderful+woman%22}{E. F. Benson, \textit{Second} 192}) % Removed "?": this actually ends with a comma, and the sentence ends with a period.
\ex
{}[aristocrat:] I'm a first-class ass, aren't I?\hfill(\href{https://archive.org/details/noneothergods00bens/page/310/mode/2up?view=theater&q=%22first-class+ass%22}{R. H. Benson, \textit{None} 319})
% Don't remove {}. (Doing so produces a syntax error, as the compiler takes "aristocrat:] to be an argument of \ex.)
\ex
``I think you're precious lucky to get such a girl.'' --- ``Yes, aren't I?''\\\hfill(\href{https://archive.org/details/comediesofcourts00hopeuoft/page/120/mode/2up?q=%22precious+lucky%22&view=theater}{Hope, \textit{Courtship} 100}) % Restored according to the novel.
\ex
Well, aren't I, my lord?\hfill(\href{https://archive.org/details/gaylordquexcomed00pine/page/168/mode/2up?q=%22aren%27t+I%2C+my+lord%22&view=theater}{Pinero, \textit{Quex} 203})
\ex
{}[an M.P.:] Aren't I in a net?\hfill(\href{https://archive.org/details/newmachiavelli00welluoft/page/476/mode/2up?view=theater&q=%22in+a+net%22}{Wells, \textit{Machiavelli} 513})
% See above for the reason for {}.
\ex
Aren't I always at your service?\hfill(\href{https://archive.org/details/wifeofsirisaacha00well/page/40/mode/2up?view=theater&q=%22aren%27t+i%22}{Wells, \textit{Wife} 41}) % OJ doesn't provide the example, merely the place where it occurs.
\ex
{}[Ann Veronica:] Aren’t I asking---asking plainly now?\\\hfill(\href{https://archive.org/details/annveronicamoder0000hgwe/page/260/mode/2up?q=%22aren%27t+i%22&view=theater}{Wells, \textit{Veronica} 245})
\ex
I \emph{am} pretty, aren't I?\hfill(\href{https://archive.org/details/dramaticworksofs02hank/page/154/mode/2up?q=%22aren%27t+I%22&view=theater}{Hankin, \textit{Works} 2.154}) % OJ got the volume and page numbers wrong; "am" originally in italics 
\ex
Are n't I going to get you to do your frock, Miss Joy?\\\hfill(\href{https://archive.org/details/joyplayonletteri00gals/page/130/mode/2up?q=%22going+to+get+you+to%22&view=theater}{Galsworthy, \textit{Joy} 2}) % "Miss Joy" restored; "aren't" is split into two, according to the printed play
\ex {[}unidentified{]} 
\hfill(ibid 73) % ??? If this is really on p73 of the second volume of the set of Galsworthy's plays, then it should appear on or around p144 of the edition at https://archive.org/details/joyplayonletteri00gals . But neither archive.org nor I (PE) can find either "aren't I" or "are n't I" there. Time permitting, I could look in other of Galsworthy's plays.
\ex
I'm always right, aren't I, Stephen?\hfill(\href{https://archive.org/details/thesetwain0000arno_p9h0/page/50/mode/2up?q=%22always+right%22&view=theater}{Bennett, \textit{Twain} 53}) % "Stephen" restored
\ex
I'm only sixteen, aren't I\hfill(\href{https://archive.org/details/clayhanger01benngoog/page/94/mode/2up?q=%22aren%27t+I%22&view=theater}{Bennett, \textit{Clayhanger} 1.113}) % OJ merely points to something on this page that he doesn't specify
\ex
aren't I lucky?\hfill(\href{https://www.gutenberg.org/cache/epub/1878/pg1878-images.html#link2HCH0030}{Oppenheim, \textit{Millionaire} 180}) % This doesn't start a sentence.
\z
\z

This form is mixed up with other forms in (\ref{ex:11-51}).

\il{English!ain't@\textit{ain't}|(}
\ea \label{ex:11-51}
That's a wall, ain't et? An' I'm a preacher, arn't I? An' you be worms, bain't 'ee?\hfill(\href{https://archive.org/details/astonishinghisto0000quil_t3y6/page/112/mode/2up?q=%22that%27s+a+wall%22&view=theater}{Quiller-Couch, \textit{Troy} 113})
\z

The form [aˑnt ai] is found convenient and corresponds to the other \textit{n't}-forms; it obviates the clumsy \textit{am I not} and the unpronounceable \textit{amn't I}, which I find written in \href{https://archive.org/details/trooperpeterhalk00schriala/page/100/mode/2up?q=%22am+n%27t%22&view=theater}{Olive Schreiner's \textit{Peter Halket} 202}.

But as [aˑnt] may be taken as developed from \textit{aren't}, it may sometimes in children's speech lead to the substitution of \textit{are} for \textit{am} in positive sentences, as when one of Darwin's little boys remarked, ``I are an extraordinary grass-finder'' (\href{https://www.darwinproject.ac.uk/people/about-darwin/family-life/darwin-s-observations-his-children}{\mbox{Darwin}, \textit{Life} 1.116}).
\il{English!aren't@\textit{aren't}|)}

\bigskip
\textit{Are not} becomes [arnt], which regularly becomes [aˑnt]; we find spellings like \textit{ar'n't you sorry} (\href{https://archive.org/details/cu31924013200898/page/n125/mode/2up?q=%22ar%27n%27t+you%22&view=theater}{J. Swift, \textit{Conversation} 90}), \textit{ar'n't you asham'd?} (\href{https://archive.org/details/cu31924013200898/page/n129/mode/2up?q=%22ar%27n%27t+you%22&view=theater}{ibid 94}).

Thus frequently in the 19th century.

\label{ch11-aint}But there is also another frequent form, which \textit{may} have developed phonetically from the older alternate form with long Middle English [aˑ],
% PE: What does OJ mean by bracketing in | | ? (Perhaps just a compositor's error for [  ] ?)
%Brett: I agree. I've changed it.
see \citet[4.432]{jespersenMEG1}, %Brett: this was in here as 1.4.432
and dropping of \textit{r} (7.79)%\href{https://archive.org/details/a-modern-english-grammar-on-historical-principles.-jespersen-otto-1860-1943-hais/page/228/mode/2up?view=theater}{ibid 7.79})
; this gives the result [eint]; compare the spellings in \refp{ex:11eint} and \refp{ex:11einta}. 
\il{English!an't@\textit{an't}}\textit{An't} in Trollope's \textit{Barchester Towers} also in the speech of educated people, e.g. (\ref{ex:11An't}). % PE: OJ says where the Trollope pair may be found but doesn't supply them.

\ea \label{ex:11eint}
\ea
an't you an impudent lying slut\hfill(\href{https://archive.org/details/journaltostellae00swifuoft/page/80/mode/2up?q=%22impudent+lying+slut%22&view=theater}{J. Swift, \textit{Journal} 81})
\ex an't you an impudent slut\hfill(\href{https://archive.org/details/journaltostellae00swifuoft/page/92/mode/2up?q=%22impudent+slut%2C+to+expect%22&view=theater}{ibid 93}) 
\ex an't you ashamed\hfill(\href{https://archive.org/details/journaltostellae00swifuoft/page/130/mode/2up?q=%22an%27t+you+ashamed%22&view=theater}{ibid 131}) 
\ex
An't you rich\hfill(\href{https://books.google.co.jp/books?redir_esc=y&hl=ja&id=WopaAAAAMAAJ&q=An%27t%20you%20rich&f=false#v=onepage&q=%22you%20rich%22&f=false}{Defoe, \textit{Gentleman} 129}) 
\ex
{}[Mrs. Honour:] if two people who loves one another a'n't happy\\\hfill(3rd person pl.; \href{https://archive.org/details/bim_eighteenth-century_the-history-of-tom-jones_fielding-henry_1768_4/page/98/mode/2up?q=%22a%27n%27t%22&view=theater}{Fielding, \textit{Tom} 4.98}) % "99" is OJ's mistake; it really is 98. I (PE) have added a bit of context. OK to delete "3rd person pl.;"?
\ex
you an't half a man\hfill(\href{https://archive.org/details/bim_eighteenth-century_the-history-of-tom-jones_fielding-henry_1750_1_0/page/76/mode/2up?q=%22you+an%27t%22&view=theater}{ibid 1.86}) % Added a little context
\ex
you ant the first gentleman {\dots}\hfill(\href{https://archive.org/details/bim_eighteenth-century_the-history-of-tom-jones_fielding-henry_1768_4/page/256/mode/2up?q=%22+you+ant+the+fir%C5%BFt%22&view=theater}{ibid 4.256}) % Added a little context
\ex
{[lady:]} they are very pretty, ma'am---an't they?\hfill(\href{https://archive.org/details/sensesensibility00austrich/page/206/mode/2up?q=%22they+are+very+pretty%22&view=theater}{Austen, \textit{Sense} 234}) % restored "ma'am"
\ex
you an't well\hfill(\href{https://archive.org/details/sensesensibility00austrich/page/210/mode/2up?q=%22you+an%27t+well%22&view=theater}{ibid 237})

\ex\label{ex:11An't}
\ea
you always forget that I an’t a young man\\\hfill(\href{https://archive.org/details/barchestertowers0000anth_w2h4/page/344/mode/2up?q=%22an%27t%22&view=theater}{Trollope, \textit{Barchester} 411}) 
\ex
I an’t cross {\dots} No, I an’t angry\hfill(\href{https://archive.org/details/barchestertowers0000anth_w2h4/page/438/mode/2up?q=%22an%27t%22&view=theater}{ibid 483})
\z
\z
\z

\ea \label{ex:11einta}
\ea {[old lady:]} Mind me, now, if they ain't married by Midsummer.\\\hfill(\href{https://archive.org/details/sensesensibility00austrich/page/168/mode/2up?q=%22married+by+Midsummer%22&view=theater}{Austen, \textit{Sense} 196})
\ex youre joking, aint you?\hfill(\href{https://archive.org/details/cashelbyronsprof00shawuoft/page/n143/mode/2up?q=%22youre+joking%22&view=theater}{Shaw, \textit{Cashel} 116})
\ex Ain't you glad you aren't short of wheat these days?\hfill(\href{https://archive.org/details/pitepicofwheatde00norruoft/page/194/mode/2up?view=theater&q=%22Ain%27t+you+glad%22}{Norris, \textit{Pit} 245}) % "these days" restored
\z
\z

\textit{Ain't} in the first person singular probably has arisen through morphological analogy, as nowhere else the persons were distinguished in the \textit{n't}-forms. %  Peter: Shall we insert an apostrophe ("-n't")?
%Brett: done
Examples: (\ref{ex:11ain't-I}). It is probable that some at least of the 19th century quotations above for \textit{an't I} are meant as [eint ai].

\ea \label{ex:11ain't-I}
\ea Ain't I a beast for not answering you before?\hfill(\href{https://archive.org/details/alfredlordtenny05tenngoog/page/n286/mode/2up?q=%22beast+for+not+answering%22&view=theater}{Tennyson, letter})
\ex {}[young lord:] I ain't a diplomatist\hfill(\href{https://archive.org/details/evanharringtonno00mererich/page/308/mode/2up?q=%22ain%27t+a+diplomatist%22&view=theater}{Meredith, \textit{Harrington} 346})
\z
\z


\il{English!have not@\textit{have not}}
\il{English!han't@\textit{han't}}
\textit{Have not} became [heint]; note the older pronunciation of \textit{have} as [heiv], also [hei], written so often \textit{ha'} (\ref{ex:11ha'}); the spelling \textit{han't} or \textit{ha'n't} is frequent, e.g. (\ref{ex:11han't}), etc., (\ref{ex:11han't2}).

\ea \label{ex:11ha'}
Ha' not [2 syllables] you seene Camillo\hfill(\href{https://internetshakespeare.uvic.ca/doc/WT_F1/scene/1.2/index.html#tln-355}{Shakespeare, \textit{Wint} 1.2.267})
\z

\ea \label{ex:11han't}
\ea
han't you four thousand pound\hfill(\href{https://archive.org/details/in.ernet.dli.2015.219151/page/n163/mode/2up?q=%22thousand%22&view=theater}{Congreve, \textit{Love} 230}) % "Pound" in singular form, following Congreve
\ex Of these spellings the publick will meet with many examples in the following book. For instance, \textit{can't}, \textit{han't}, \textit{sha'nt}, \textit{didn't}, \textit{coodn't}, \textit{woodn't}, \textit{is~n't}, \textit{e'n't}, with many more\hfill(\href{https://archive.org/details/cu31924013200898/page/n67/mode/2up?q=%22the+following+Book%22&view=theater}{J. Swift, \textit{Conversation} 32})
\ex you ha'n't taken snuff\hfill(\href{https://archive.org/details/cu31924013200898/page/n127/mode/2up?q=%22you+ha%27n%27t%22&view=theater}{ibid 92})
\ex I han't eaten it\hfill(\href{https://archive.org/details/cu31924013200898/page/n189/mode/2up?q=%22my+Knife%22&view=theater}{ibid 155}) % "eaten it" restored
\ex Han't I brought myself\hfill(\href{https://archive.org/details/journaltostellae00swifuoft/page/22/mode/2up?q=%22I+brought+myself%22&view=theater}{J. Swift, \textit{Journal} 22})% "brought myself" restored
\ex Han't I said\hfill(\href{https://archive.org/details/journaltostellae00swifuoft/page/40/mode/2up?q=%22han%27t%22&view=theater}{ibid 40})% "said" restored
\ex I han't time to say more\hfill(\href{https://archive.org/details/journaltostellae00swifuoft/page/42/mode/2up?q=%22han%27t%22&view=theater}{ibid 43})
\ex Han't you done things like that\hfill(\href{https://archive.org/details/journaltostellae00swifuoft/page/62/mode/2up?q=%22han%27t%22&view=theater}{ibid 63})% "done things like that" restored
\z
\z

\ea \label{ex:11han't2}
\ea I han't the impudence\hfill(\href{https://archive.org/details/fartheradventure00defo/page/152/mode/2up?q=%22I+han%27t%22&view=theater}{Defoe, \textit{Farther} 153}) % ??? OJ says p.164. I (PE) find no example there. I do find one on p. 153 and therefore have substituted that, together with a minimum of context.
\ex Han't you a vast estate?\hfill(\href{https://books.google.co.jp/books?id=WopaAAAAMAAJ&printsec=frontcover&hl=ja&source=gbs_ge_summary_r&cad=0#v=onepage&q=vast%20estate&f=false}{Defoe, \textit{Gentleman} 129}) 
\ex I han't words\hfill(\href{https://ia800900.us.archive.org/29/items/compleatenglishg00deforich/compleatenglishg00deforich.pdf}{ibid 132}) % ??S OJ gives the page number but doesn't say what's there. Here it is, with a minimum of context. As noted above, this is a bulky PDF
\ex han't you heard\hfill(\href{https://archive.org/details/bim_eighteenth-century_the-tragedy-of-tragedies_fielding-henry_1765/page/22/mode/2up?q=%22han%27t+you+heard%22&view=theater}{Fielding, \textit{Tragedy} 1.377})
\ex I ha'n't a moment to lose\hfill(\href{https://archive.org/details/criticoratraged00aitkgoog/page/n70/mode/2up?q=%22moment+to+lose%22&view=theater}{Sheridan, \textit{Critic} 1.2.401})
\ex I han't been here for so long\hfill(\href{https://archive.org/details/returnofthenativ00harduoft/page/26/mode/2up?q=%22I+han%27t+been%22&view=theater}{Hardy, \textit{Return} 34}) % Restored a bit of context
\ex Ha'n't I nussed her?\hfill(\href{https://archive.org/details/lifeslittleironi00harduoft/page/172/mode/2up?q=%22nussed+her%22&view=theater}{Hardy, \textit{Ironies} 201}) % OJ has "mussed". That's a mistake. It's "nussed" (presumably for "nursed").
\z
\z

Instead of \textit{han't} the spelling \textit{ain't} also occurs as a vulgarism (\textit{h} dropped).


\il{English!don't@\textit{don't}}
\il{English!do not@\textit{do not}}
\textit{Do not} becomes \textit{don't} [dount], which is found, e.g. in (\ref{ex:11don't}), etc., (\ref{ex:11don'tb}), and innumerable times since then.

\ea \label{ex:11don't} Pray don't let that puppy Parvisol sell him\hfill(\href{https://archive.org/details/journaltostellae00swifuoft/page/16/mode/2up?q=%22don%27t%22&view=theater}{J. Swift, \textit{Journal} 17}) % OJ points to where this example may be found, but doesn't provide it.
\z

\ea \label{ex:11don'tb}
\ea we don't care to do it\hfill(\href{https://ia800900.us.archive.org/29/items/compleatenglishg00deforich/compleatenglishg00deforich.pdf}{Defoe, \textit{Gentleman} 12}) % ??S OJ points to where this example may be found, but doesn't provide it. Bulky PDF
\ex I don't know\hfill(\href{https://books.google.co.jp/books?hl=ja&id=WopaAAAAMAAJ&q=12#v=onepage&q=%22we%20don't%20care%20to%20do%22&f=false}{ibid 45}) % OJ points to where this example may be found, but doesn't provide it.
\ex your son will {\dots} delight in them too, perhaps, tho' you don't\hfill(\href{https://books.google.co.jp/books?hl=ja&id=WopaAAAAMAAJ&q=137#v=onepage&q=%22delight%20in%20them%22&f=false}{ibid 137}) % OJ points to where this example may be found, but doesn't provide it. 
\z
\z
\il{English!ain't@\textit{ain't}|)}


\il{English!will not@\textit{will not}}
\il{English!won't@\textit{won't}}
For \textit{will not} we have \textit{won't} [wount], developed (through \textit{wonnot}, found in Dryden and other writers of that time) from the Middle English form 
\il{English!Middle English!wol@\textit{wol}}\textit{wol}. It is written \textit{wont} in Defoe (R. 2. 166), % ??? OJ uses "R. 2." for The Farther Adventures of Robinson Crusoe. I don't find "wont" (or "won't") on p. 166 of that book. FWIW, archive.org's search engine doesn't find "wont" anywhere in the book. It does find "won't" several times; among these, on page 155: https://archive.org/details/fartheradventure00defo/page/154/mode/2up?q=%22won%27t%22&view=theater . Could there have been a double mistake here; "wont" on 166 for "won't" on 155? (Incidentally, I don't find "wont" -- other than as the adjective still used today -- in The Adventures, in A Journal, or in The Compleat Gentleman.) I've yet to look for/in other editions of The Farther Adventures.
%Brett: ¯\_(ツ)_/¯
but generally \textit{won't}, thus (\ref{ex:11won't}), etc., etc.

\ea \label{ex:11won't}
\ea
that won't do\hfill(\href{https://archive.org/details/rehearsalwithil00arbegoog/page/n45/mode/2up?q=%22that+won%27t%22&view=theater}{Villiers, \textit{Rehearsal} 41}) % OJ points to where this example may be found, but doesn't provide it. 
\ex
``she won’t let him come near her'' {\dots} --- ``I won’t be seen in’t''\\\hfill(\href{https://archive.org/details/in.ernet.dli.2015.219151/page/n171/mode/2up?q=%22won%27t%22&view=theater}{Congreve, \textit{Love} 237}) % As can be seen in https://archive.org/details/in.ernet.dli.2015.219151/page/n163/mode/2up?q=%22won%27t%22&view=theater , Congreve has many examples of "won't". OJ points to where this example may be found, but doesn't provide it. 
\ex
the Justices won't give their own servants a bad example\\\hfill(\href{https://archive.org/details/beauxstratagema00fitzgoog/page/n79/mode/2up?q=%22won%27t%22&view=theater}{Farquhar, \textit{Stratagem} 335}) % OJ points to where this example may be found, but doesn't provide it. 
\ex
we won't dispute\hfill(\href{https://books.google.co.jp/books?hl=ja&id=WopaAAAAMAAJ&q=48#v=snippet&q=%22we%20won't%22&f=false}{Defoe, \textit{Gentleman} 48}) % OJ points to where this example may be found, but doesn't provide it. 
\ex
I won't lessen it\hfill(\href{https://books.google.co.jp/books?hl=ja&id=WopaAAAAMAAJ&q=66#v=onepage&q=%22warrant%20I%20won't%22&f=false}{ibid 66}) % OJ points to where this example may be found, but doesn't provide it. 
\ex
I won't give it up neither\hfill(\href{https://archive.org/details/bim_eighteenth-century_the-history-of-tom-jones_fielding-henry_1750_1_0/page/240/mode/2up?q=%22won%27t%22&view=theater}{Fielding, \textit{Tom} 1.237}) % OJ points to where this example may be found, but doesn't provide it. 
\z
\z

\il{English!ain't@\textit{ain't}|(}
The [z]\footnote{The original edition has ``[s]'', likely an error. \eds}
% PE: Did OJ perhaps mean to write either [z] or "s"?
%Brett: changed and footnoted.
was frequently dropped in \textit{isn't}, \textit{wasn't}, \textit{doesn't}, (thus expressly \cite[\href{https://archive.org/details/bim_eighteenth-century_the-principles-of-the-en_elphinston-james_1765_1/page/134/mode/2up?view=theater&q=sink}{1.134}]{elphinston1765principles}) % OJ has "Elphinstone 1765", but (unusually) there's no "e" at the end of this author's surname and 1765 is merely the year of publication.
and this gives rise to various forms of interest. For \textit{isn't} we find \textit{'ent} (``facilitatis causa'', \cite[79]{cooper1685grammatica}) and in the 18th century the form \textit{i'n't}, which \citet[236]{hall1873modern} % OJ writes here "Fitzedward Hall"; I (PE) don't know why, and have deleted the personal name so that the entry for the book may be found more easily.
quotes from Foote, Richardson, and Miss Burney. But the vowel is unstable; \href{https://archive.org/details/cu31924013200898/page/n67/mode/2up?q=%22following+Book%22&view=theater}{J. Swift, \textit{Conversation} 32} writes \textit{e'n't}; and if we imagine a lowering and lengthening of the vowel (corresponding pretty exactly to what happened in \textit{don't}, \textit{won't}, and really also in \textit{can't}, etc.), this would result in a pronunciation [eint]; 
\il{English!an't@\textit{an't}}now this must be written \textit{an't} or \textit{ain't}, and would fall together with the form mentioned above as possibly developed from 
\il{English!aren't@\textit{aren't}}\textit{aren't}. \textit{An't} is found in the third person as early as (\ref{ex:11:aren't}). In the 19th century, \textit{an't} and \textit{ain't} are frequent for \textit{is not} in representations of vulgar speech; see quotations in \citet[\href{https://archive.org/details/p2englischephilo01storuoft/page/708/mode/2up?view=theater}{709}]{storm1896englische} and \citet[\href{https://archive.org/details/slangitsanalogue01farm/page/50/mode/2up?view=theater}{50}]{farmer1909slang}, also e.g. (\ref{ex:11-63}).

\ea \label{ex:11:aren't}
\ea Presto is plaguy silly to-night, an't he?\hfill(\href{https://archive.org/details/journaltostellae00swifuoft/page/104/mode/2up?q=%22Presto+is+plaguy%22&view=theater}{J. Swift, \textit{Journal} 105})
\ex An't that right now?\hfill(\href{https://archive.org/details/journaltostellae00swifuoft/page/146/mode/2up?q=%22An%27t+that+right%22&view=theater}{ibid 147})
\ex it an't my fault\hfill(\href{https://archive.org/details/journaltostellae00swifuoft/page/178/mode/2up?q=%22an%27t+my+fault%22&view=theater}{ibid 179})
\ex no, her name an't Hannah\hfill(\href{https://archive.org/details/journaltostellae00swifuoft/page/272/mode/2up?q=%22an%27t+Hannah%22&view=theater}{ibid 273}) % OJ points to the page, but doesn't say what's on it.
\z
\z

\ea \label{ex:11-63}
\ea
I don't pretend to say that there ain't.\hfill(\href{https://books.google.co.jp/books?id=W1kVAAAAYAAJ&pg=PA102&lpg=PA102&dq=%22sense+and+sensibility%22+%22pretend+to+say+that+there+ain%27t%22&source=bl&ots=5SCX-yP9KA&sig=ACfU3U3mnKs2V0Gx-o1aaxDRIx5FTrn7Gw&hl=en&sa=X&ved=2ahUKEwiZ6LnBzsCGAxU0qFYBHQ6aAPUQ6AF6BAgJEAM#v=onepage&q=%22sense%20and%20sensibility%22%20%22pretend%20to%20say%20that%20there%20ain't%22&f=false}{Austen, \textit{Sense} 125}) % Other editions, such as https://archive.org/details/sensesensibility00austrich/page/108/mode/2up?q=%22pretend+to+say+that%22&view=theater , have "an't" instead of "ain't"
\ex
What an ill-natured woman his mother is, an't she?\hfill(\href{https://archive.org/details/sensesensibility00austrich/page/242/mode/2up?q=%22what+an+ill-natured%22&view=theater}{ibid 270}) 
\ex
if Lucy an't there\hfill(\href{https://archive.org/details/sensesensibility00austrich/page/258/mode/2up?q=%22Lucy+an%27t+there%22&view=theater}{ibid 287})
\z
\z

But now it is not felt as so vulgar as formerly; Dean \citet[\href{https://archive.org/details/queensenglishstr00alfo/page/70/mode/2up?q=\%22highly\%20educated\%20persons\%22&view=theater}{71}]{alford1888queens} says: ``\textit{I ain't certain.} % Not "It" but "I".
\textit{I ain't going} {\dots} . are % Adding "are" to OJ's version; "are" is in Alford's original.
very frequently used, even by highly educated persons.'' And in Anthony Hope (\ref{ex:11-66}) people of the best society are represented as saying \textit{it ain't} and \textit{ain't it}. Dr. Furnivall,\footnote{Frederick James Furnivall (1825--1910), a co-founder of \textit{A new English dictionary}. \eds} to mention only one man, was particularly fond of using this form. 

\ea \label{ex:11-66}
\ea
That vow of his is all nonsense, ain't it?\hfill(\href{https://archive.org/details/fatherstafford00hope/page/72/mode/2up?q=%22ain%27t%22&view=theater}{\textit{Father} 40})

\ex
I ain't to have any birds unless I field at long-leg.\hfill(\href{https://archive.org/details/fatherstafford00hope/page/82/mode/2up?q=%22ain%27t%22&view=theater}{ibid 45})

\ex
It ain't over and above flattering to him, though.\hfill(\href{https://archive.org/details/comediesofcourts00hopeuoft/page/62/mode/2up?q=ain%27t&view=theater}{\textit{Courtship} 57}) 
\z \z
\il{English!ain't@\textit{ain't}|)}


\il{English!wa'n't@\textit{wa'n't}}
\il{English!wa'nt@\textit{wa'nt}}
The form \textit{wa'nt} or \textit{wa'n't} for \textit{was not} is pretty frequent in Defoe, e.g. (\ref{ex:11-67}).

\ea \label{ex:11-67}
\ea
You was nurst abroad, wa'nt you?\hfill(\href{https://ia800900.us.archive.org/29/items/compleatenglishg00deforich/compleatenglishg00deforich.pdf}{\textit{Gentleman} 51}) % ??S  Bulky PDF. "nurst abroad" reintroduced: it's so short.
\ex
I warrant you were frighted, wa'n't you\hfill(\href{https://archive.org/details/lifeandstranges00dobsgoog/page/n33/mode/2up?q=%22warrant+you+were%22&view=theater}{\textit{Robinson} 8})
\z \z

I find the same form frequently in American writers: (\ref{ex:11-68}).

\ea \label{ex:11-68}
\ea
we wa'n't ragged\hfill(\href{https://archive.org/details/riseofsilaslapha0000will_r0t6/page/6/mode/2up?view=theater&q=%22ragged%22}{Howells, \textit{Rise} 10})
\ex
I wa'n't \phantom{x} (often, in all persons)\hfill(\href{https://archive.org/details/riseofsilaslapha0000will_r0t6/page/10/mode/2up?view=theater&q=%22wa%27n%27t%22}{ibid 15})
\ex
he wa'n't all greenhorn\hfill(\href{https://archive.org/details/valleyofmoon00londrich/page/n337/mode/2up?view=theater&q=%22he+wa%27n%27t%22}{London, \textit{Valley} 329}) % "all greenhorn" restored
\ex
I wa'n't after no money. {\dots} I wa'n't after you. {\dots} 'T wa'n't me.\\\hfill(\href{https://archive.org/details/johnmarvelassistant00pageiala/page/80/mode/2up?q=%22wa%27n%27t+after+no+money%22&view=theater}{T. N. Page, \textit{Marvel} 350} (vulgar)) % "I wa'n't after you" restored
\z
\z


\il{English!warn't@\textit{warn't}}
A variant is written \textit{warn't}, where \textit{r} of course is mute, the sound represented being [wɔˑnt]; it is frequent vulgarly in Dickens, e.g. (\ref{ex:11-72}).

\ea \label{ex:11-72}
    \ea
        If I warn't a man on a small annuity\hfill(\href{https://archive.org/details/dombeyson00dick_0/page/126/mode/2up?q=%22if+I+warn%27t+a+man%22&view=theater}{\textit{Dombey} 77})
    \ex
        it warn't him\hfill(\href{https://archive.org/details/dombeyson00dick_0/page/352/mode/2up?q=%22It+warn%27t+him%22&view=theater}{ibid 223} (vulgar))
    \ex
        Look and see if he warn't.\hfill(\href{https://archive.org/details/ourmutualfriendc0000char/page/18/mode/2up?q=%22See+if+he+warn%E2%80%99t%22&view=theater}{\textit{Friend} 24}) % Restored "Look and"
    \ex
        It warn't their duty to take my wife {\dots} that 'adn't done nothing.\\\hfill(\href{https://archive.org/details/silverboxcomedyi00gals/page/86/mode/2up?q=%22warn%27t%22&view=theater}{Galsworthy, \textit{Box} 86}) % OJ merely points to the example; he doesn't present it. Here it is.
    \z
\z

\il{English!don't@\textit{don't}|(}
\il{English!does not@\textit{does not}}
\textit{Don't} for \textit{does not} is generally explained from a substitution of some other person for the third person; but as this is not a habitual process,---as \textit{do} in the third person singular is found only in some few dialects, but not in standard English, and as the tendency is rather in the reverse direction of using the verb form in \textit{s} with subjects of the other persons (\textit{says I}, \textit{they talks}, etc.), the inference is natural that we have rather a phonetic process, \textit{s} being absorbed before \textit{n't} as in \textit{i\textsc{sn't}}, etc., above. The vowel in [dount] must have developed in the same way as in \textit{do not}, if we admit that the mutescence of \textit{s} took place before the vowel in \textit{does} was changed into [ʌ]. \textit{Don't} in the third person is found in (\ref{ex:11-74}) and very frequently in the 19th century. Byron uses it repeatedly in the colloquial verse of \textit{Don Juan} (\ref{ex:11-77}), where \textit{doesn't} is probably never found, though \textit{does not} and \textit{doth not} are found. Dickens has it constantly in his dialogues, chiefly, but not exclusively, in representing the speech of vulgar people (see e.g. \ref{ex:11-78}); and he sometimes even uses it in his own name (as \ref{ex:11-79}). The form is used constantly in the conversations in such books as Hughes's \textit{Tom Brown}. Kingsley makes a well-bred man say \textit{She don't care for me} (\href{https://archive.org/details/hypatia00kinggoog/page/n110/mode/2up?q=%22she+don%27t+care%22&view=theater}{\textit{Hypatia} 76}% "for me" restored
; cf. \ref{ex:11-81}), and similarly Meredith has an MP say (\ref{ex:11-82}), Philips a perfect gentleman (\ref{ex:11-83}), Egerton a lady (\ref{ex:11-84}).

\ea \label{ex:11-74}
\ea
``He don't belong to our gang.'' --- {\dots} ``he don't belong to our fraternity''\\\hfill(\href{https://archive.org/details/beauxstratagema00fitzgoog/page/n41/mode/2up?q=%22he+don%27t%22&view=theater}{Farquhar, \textit{Stratagem} 321}) % OJ points to this example but doesn't provide it.
\ex
my brother don't kno'\hfill(\href{https://books.google.co.jp/books?hl=ja&id=WopaAAAAMAAJ&q=%22my%20brother%20don%27t%22&f=false#v=snippet&q=%22my%20brother%20don't%22&f=false}{Defoe, \textit{Gentleman} 47}) 
\ex
why don't my girl play me such a trick\hfill(\href{https://archive.org/details/bim_eighteenth-century_the-duenna-a-comic-oper_sheridan-richard-brinsl_1794/page/30/mode/2up?q=don%27t&view=theater}{Sheridan, \textit{Duenna} 2.1}) % OJ doesn't provide the example. He says "Sheridan D. 277". This is the only example in this book that's attributed to "Sheridan D.", an abbreviation that isn't explained in MEG. Perhaps wrongly, I (PE) guessed that it meant Sheridan's The Duenna (or more strictly his libretto for an opera of that title) ... though I wonder how any libretto could run to 277 pages. Anyway, there seems to be only one 3sg "don't" in the book. (Another possibility might be that "D." is just some printing error. Without it, the reference would point to Sheridan's The Critic. However, The Critic doesn't seem to have any 3sg use of "don't".)
\z
\z

\ea \label{ex:11-77}
\ea
it don't ask much to mar\hfill(\href{https://archive.org/details/workslordbyron10unkngoog/page/146/mode/2up?view=theater&q=%22don%27t%22}{Byron, \textit{Juan} 3.10})
\ex
she don't pin men's limbs in\hfill(\href{https://archive.org/details/workslordbyron10unkngoog/page/386/mode/2up?view=theater&q=%22don%27t%22}{ibid 9.44})
\ex
she don't forget the infant girl\hfill(\href{https://archive.org/details/workslordbyron10unkngoog/page/414/mode/2up?view=theater&q=%22don%27t%22}{ibid 10.51})
\ex
Indifference, certes, don't produce distress\hfill(\href{https://archive.org/details/workslordbyron10unkngoog/page/490/mode/2up?view=theater&q=%22don%27t%22}{ibid 13.35})
\ex
that fair clime which don't depend on climate\hfill(\href{https://archive.org/details/workslordbyron10unkngoog/page/522/mode/2up?view=theater&q=%22don%27t%22}{ibid 14.29})
\z \z

\ea \label{ex:11-78}
\ea
don't she, my dear?\hfill(\href{https://archive.org/details/dombeyson00dick_0/page/28/mode/2up?q=%22don%27t+she%2C+my+dear%22&view=theater}{Dickens, \textit{Dombey} 13}) 
\ex
``She don't worry me'' --- ``that don't matter {\dots} it don't follow {\dots}'' --- ``Well, it don't matter''\hfill(\href{https://archive.org/details/dombeyson00dick_0/page/32/mode/2up?q=%22she+don%27t+worry+me%22&view=theater}{ibid 16}) 
\ex
{[}unidentified{]} \hfill(\href{}{ibid 22}) % ??? It's not obvious what this refers to. Perhaps to "He don't want me. He don't want me!" https://archive.org/details/dombeyson00dick_0/page/38/mode/2up?q=%22don%27t%22&view=theater , a couple of pages earlier than I'd expect. Alternatively we might be bold and just have this area point to https://archive.org/details/dombeyson00dick_0/page/32/mode/2up?q=%22don%27t%22&view=theater , in which double-page spread there are five examples of "don't" with a 3sg subject (as well as three examples that are grammatical in Standard E).
\ex
She don't gain on her papa in the least.\hfill(\href{https://archive.org/details/dombeyson00dick_0/page/56/mode/2up?q=%22don%27t+gain+on%22&view=theater}{ibid 31}) 
\ex
I think he knows it, though he pretends he don't.\hfill(\href{https://archive.org/details/personalhistory05dickgoog/page/n43/mode/2up?q=%22don%27t%22&view=theater}{Dickens, \textit{David} 84}) 
\ex
Mr. Dick is his name here, and everywhere else, now---if he ever went anywhere else, which he don't.\hfill(\href{https://archive.org/details/personalhistory05dickgoog/page/n87/mode/2up?q=%22mr.+dick+is+his%22&view=theater}{ibid 188}) 
\ex
it don't signify\hfill(\href{https://archive.org/details/personalhistory05dickgoog/page/n89/mode/2up?q=%22it+don%27t%22&view=theater}{ibid 191})
\ex
It don't matter\hfill(\href{https://archive.org/details/personalhistory05dickgoog/page/n89/mode/2up?q=%22it+don%27t%22&view=theater}{ibid 376})
\ex
It don't matter how much\hfill(\href{https://archive.org/details/personalhistory05dickgoog/page/n209/mode/2up?q=%22a+certain+property%22&view=theater}{ibid 476})
\ex
he'll think she don't like him any more\hfill(\href{https://archive.org/details/personalhistory05dickgoog/page/n259/mode/2up?q=%22he%27ll+think+she+don%27t%22&view=theater}{ibid 590}) 
\ex
{}[educated young man:] He don't do any good with it. He don't make himself comfortable with it. {\dots} He don't lose much of a dinner.\\\hfill(\href{https://archive.org/details/christmascarol0000char_h5c8/page/82/mode/2up?q=%22don%27t+do+any+good%22&view=theater}{Dickens, \textit{Carol} 45})
\z \z

\ea \label{ex:11-79}
\ea How Susan does it, she don't know\hfill(\href{https://archive.org/details/dombeyson00dick_0/page/776/mode/2up?q=%22how+susan+does+it%22&view=theater}{Dickens, \textit{Dombey} 500})
\ex he don't appear to break his heart\hfill(\href{https://archive.org/details/dombeyson00dick_0/page/842/mode/2up?q=%22he+don%27t+appear%22&view=theater}{ibid 541}) 
\z \z

\ea \label{ex:11-81}
Why don't she marry some hero?\hfill(\href{https://archive.org/details/hypatia00kinggoog/page/n192/mode/2up?q=%22why+don%27t+she%22&view=theater}{ibid 146}) % OJ merely points to this example but doesn't provide it.
\z

\ea \label{ex:11-82}
my stomach don't like cold bathing\hfill(\href{https://archive.org/details/evanharringtonno00mererich/page/444/mode/2up?q=%22my+stomach+don%27t%22&view=theater}{Meredith, \textit{Harrington} 489})
\z

\ea \label{ex:11-83}
a man don't like waiting for his breakfast\hfill(\href{https://archive.org/details/asinalookinggla00philgoog/page/n230/mode/2up?q=%22man+don%27t+like+waiting%22&view=theater}{Philips, \textit{Glass} 226})
\z

\ea \label{ex:11-84}
that don't matter\hfill(\href{https://archive.org/details/b29012612/page/108/mode/2up?q=%22that+don%27t+matter%22&view=theater}{Egerton, \textit{Keynotes} 101})
\z

A characteristic illustration of the way in which educated people look upon \textit{don't} in the third person singular is found in the conversation in \href{https://archive.org/details/martineden00lond/page/62/mode/2up?q=%22But+it+don%27t+mean%22&view=theater}{Jack London's \textit{Martin Eden}, p.~64f}.\footnote{This remark is from Jespersen's Addenda; London's conversation itself is our addition. \eds} % Peter: When I added this, I was confident that it was a suitable and even welcome addition (something that OJ would have added if not pressed for space and perhaps also concerned about copyright). But now (Jul '24) I really wonder. Retain? Cut?
%Brett: Retain.

\begin{quote}``There's something else I noticed in your speech. You say `don't' when you shouldn't. `Don't' is a contraction and stands for two words. Do you know them?''

He thought a moment, then answered, ``‘Do not'.''

She nodded her head, and said, ``And you use `don't' when you mean `does not'.''

He was puzzled over this, and did not get it so quickly.

``Give me an illustration,'' he asked.

``Well---'' She puckered her brows and pursed up her mouth as she thought, while he looked on and decided that her expression was most adorable. ``‘It don't do to be hasty.' Change `don't' to `do not', and it reads, `It do not do to be hasty', which is perfectly absurd.''

He turned it over in his mind and considered.

``Doesn't it jar on your ear?'' she suggested.

``Can't say that it does,'' he replied judicially. % PE: This looked to me like a mere malapropism for "judiciously" (and better equipped with "[sic]"); but I learn from Wiktionary's entry for "judicial" that no, I was mistaken.

``Why didn't you say, `Can't say that it do'?'' she queried.

``That sounds wrong,'' he said slowly.\end{quote} % OJ points to this but doesn't provide it. It is rather bulky; but I (PE) think that it's worth the space. Of course, even we retain it we could shorten it somewhat.
%Brett: I think it should stay in its entirety.

That this use of \textit{don't} could not by any means be called a vulgarism nowadays, however much schoolmasters may object to it, will also appear from the following quotations: (\ref{ex:11-185}) (the two last American (\ref{ex:11-91}, \ref{ex:11-herrick})).

\ea \label{ex:11-185}
\ea
I have just heard from Peacock, saying that he don't think that my tragedy will do, and that he don't much like it.\hfill(\href{https://archive.org/details/worksinversepros08sheluoft/page/132/mode/2up?q=%22have+just+heard+from+Peacock%22&view=theater}{Shelley, letter})
\ex
It don't signify talking\hfill(\href{https://archive.org/details/sensesensibility00austrich/page/166/mode/2up?q=%22signify+talking%22&view=theater}{Austen, \textit{Sense} 193})
\ex
{}[a lord:] Well, it don't matter\hfill(\href{https://archive.org/details/historydavidgri02wardgoog/page/258/mode/2up?q=%22Well%2C+it+don%27t+matter%22&view=theater}{Ward, \textit{David} 184}) % OJ attributes this to Mrs HW's work "F". He doesn't explain what "F" is, but anyway this doesn't appear in the book Fenwick's Career.
\ex
{}[a celebrated traveller:] But that don't matter\hfill(\href{https://archive.org/details/marriageofwillia0000mrsh_i0u5/page/96/mode/2up?q=%22that+don%27t+matter%22&view=theater}{Ward, \textit{Marriage} 86}) % Restoring "But"
\ex
{}[a young diplomatist:] It don't sound much\hfill(\href{https://archive.org/details/cu31924013567130/page/80/mode/2up?q=%22it+don%27t+sound+much%22&view=theater}{Ward, \textit{Eleanor} 64})
\ex
he don't take Manisty at his own valuation\hfill(\href{https://archive.org/details/cu31924013567130/page/82/mode/2up?q=%22he+don%27t+take+manisty%22&view=theater}{ibid 65})
\ex
{}[an ambassador:]  That don't count.\hfill(\href{https://archive.org/details/cu31924013567130/page/316/mode/2up?q=%22that+don%27t+count%22&view=theater}{ibid 254})
\ex
{}[a lady:] He don't care\hfill(\href{https://archive.org/details/cu31924013567130/page/322/mode/2up?q=%22don%27t+care%22&view=theater}{ibid 258}
\ex
{}[Mr. Lewisham:] It don't matter a bit\hfill(\href{https://archive.org/details/loveandmrlewisha00welluoft/page/n27/mode/2up?q=%22matter+a+bit%22&view=theater}{Wells, \textit{Love} 19})
\ex
{}[Sir Patrick:] Why dont he live for it to some purpose?\\\hfill(\href{https://archive.org/details/doctorsdilemmatr00shawuoft/page/92/mode/2up?q=%22he+live+for+it%22&view=theater}{Shaw, \textit{Dilemma} 93})
\ex
it don't matter\hfill(\href{https://archive.org/details/playspleasantunp01shaw/page/4/mode/2up?q=don%27t&view=theater}{Shaw, \textit{Houses} 4})

\ex
he looks cheerful, don't he?\hfill(\href{https://archive.org/details/playspleasantunp01shaw/page/174/mode/2up?q=don%27t&view=theater}{Shaw, \textit{Profession} 174})

\ex
it don't matter\hfill(\href{https://archive.org/details/playspleasantunp01shaw/page/180/mode/2up?q=don%27t&view=theater}{ibid 178})

\ex
it don't matter whether they're father and son, husband and wife, brother and sister\hfill(\href{https://archive.org/details/playspleasantunp01shaw/page/210/mode/2up?q=don%27t&view=theater}{ibid 203})

\ex
Don't it make your flesh creep ever so little?\hfill(\href{https://archive.org/details/playspleasantunp01shaw/page/212/mode/2up?q=don%27t&view=theater}{ibid 204})
% Actually OJ writes "compare Shaw 1. 4, 174, 178, 179, 203, 204, etc.", seemingly referring to an edition of Plays Unpleasant that has slightly different pagination than that linked to here. I (PE) have, I hope, found five of OJ's six.

\ex \label{ex:11-91}
it stands to reason, don't it\hfill(\href{https://archive.org/details/octopusstoryofca00norruoft/page/230/mode/2up?view=theater&q=%22stands+to+reason%22}{Norris, \textit{Octopus} 231}) % Removed "?". (There is one, but only after a longish string of words.)
\ex \label{ex:11-herrick}
It don't make any difference\hfill(\href{https://archive.org/details/memoirsofamerica0000robe/page/186/mode/2up?view=theater&q=%22make+any+difference%22}{Herrick, \textit{Memoirs} 187})
\z
\z
\il{English!don't@\textit{don't}|)}

Here, as with \textit{ain't}, the distinction of person and number has been obliterated in the negative forms.\largerpage[2]


\il{English!daren't@\textit{daren't}}
\textit{Daren't} stands for both \textit{dare not} (\textit{dares not}) and \textit{dared not}, the latter through a natural phonetic development (\cite[\href{https://archive.org/details/a-modern-english-grammar-on-historical-principles.-jespersen-otto-1860-1943-hais/page/222/mode/2up?view=theater}{7.72}]{jespersenMEG1}; cf. also \cite[\href{https://www.google.com/books/edition/Englische_Studien/ELM_AQAAMAAJ?hl=en&gbpv=1&pg=PA461}{461}]{jespersen1896dare}). The use in the present needs no exemplification (\ref{ex:11-96}); in the preterite we have, e.g. (\ref{ex:11-97}).

\ea \label{ex:11-96}
I darent talk about such things\hfill(\href{https://archive.org/details/mrswarrensprofes00shawuoft/page/198/mode/2up?q=%22talk+about+such+things%22&view=theater}{Shaw, \textit{Profession} 198})
\ex \label{ex:11-97}
\ea
Her restlessness wakened her little bedfellows more than once. She daren't read more of ``Walter Lorraine'': father was at home and would suffer no light.\hfill(\href{https://archive.org/details/dli.ministry.14127/page/493/mode/2up?q=%22restlessness+wakened%22&view=theater}{Thackeray, \textit{Pendennis} 3.83}) % "little" restored; (oddly positioned) quotes restored; comma removed.
\ex
Her spirit failed her a little. She daren't climb after him in the dark.\\\hfill(\href{https://archive.org/details/historydavidgri02wardgoog/page/56/mode/2up?q=%22her+spirit+failed+her%22&view=theater}{Ward, \textit{David} 1.99})
\ex
the ship's charts were in pieces and our skipper daren't run south\\\hfill(\href{https://archive.org/details/lightthatfailed0000rudy_q6g8/page/132/mode/2up?q=%22charts+were+in+pieces%22&view=theater}{Kipling, \textit{Light} 126}) % Not "ships" but "skipper"
\ex
you know you darent have given the order to charge the bridge if you hadnt seen us on the other side.\hfill(\href{https://archive.org/details/manofdestinytri00shaw/page/194/mode/2up?q=%22have+given+the+order%22&view=theater}{Shaw, \textit{Destiny} 195}) % Parts of this restored after OJ had cut them.
\ex
otherwise {\dots} I darent have brought you here.\hfill(\href{https://archive.org/details/cashelbyronsprof00shawuoft/page/n141/mode/2up?q=%22darent+have%22&view=theater}{Shaw, \textit{Cashel} 114})
\ex
We were halted before I could see. And I daren't look round\\\hfill(\href{https://archive.org/details/thesetwain0000arno_p9h0/page/330/mode/2up?q=%22halted+before%22&view=theater}{Bennett, \textit{Twain} 326})
\z
\z


\il{English!dare not@\textit{dare not}}
\il{English!dursn’t@\textit{dursn’t}}
\textit{Dare not} is often written as a preterite, even by authors who do not use \textit{dare} (without \textit{not}) as a preterite;\footnote{Jespersen names ``Tennyson, Doyle, Kipling, Shaw, Hall Caine, Parker'' as six such authors but does not provide quotations. The selection of quotations for four of these authors is ours. We have not found examples for Tennyson or Kipling. \eds} this, of course, represents a spoken [dɛˑənt]:
\ea
\ea
I dare not strike a light, so I felt about in the darkness until my hand came upon something wet, which I knew to be his head. \\\hfill (\href{https://archive.org/details/TheStrandMagazineAnIllustratedMonthly/TheStrandMagazine1895aVol.IxJan-jun/page/n519/mode/2up?q=%22dare+not+strike%22&view=theater}{Doyle, \textit{How} 506})
\ex
He dare not have done it if I had been with him \hfill (\href{https://archive.org/details/devilsdisciplea00shawgoog/page/n16/mode/2up?q=%22dare+not%22&view=theater}{Shaw, \textit{Disciple}~1})
\ex
His simple face wore a strange expression of joy and fear, as if he wished to smile and dare not. \hfill (\href{https://archive.org/details/christianstory00cainrich/page/390/mode/2up?q=%22dare+not%22&view=theater}{Caine, \textit{Christian}})
\ex
{}[He was] the man who had saved him from hanging, to whom he owed a debt he dare not acknowledge. \hfill (\href{https://archive.org/details/bwb_UE-390-059/page/84/mode/2up?view=theater&q=%22dare+not+acknowledge%22}{Parker, \textit{Right}})
\z
\z

There is a negative form of the (obsolescent) preterite \textit{durst}, in which the first \textit{t} is often omitted; it is sometimes used as a present (thus  \ref{ex:11-103}). Recent examples, (\ref{ex:11-105}), to which are added some dialectal forms: (\ref{ex:11-107}).

\ea \label{ex:11-103}
\ea
{}[a Norfolk speaker:] We dursn't do it.\hfill(\href{https://archive.org/details/personalhistory05dickgoog/page/n181/mode/2up?q=%22dursn%27t%22&view=theater}{Dickens, \textit{David} 407}) % OJ doesn't provide the words for the example.
\ex
{}[Captain Cuttle:] Dursn't do it, Wal'r\hfill(\href{https://archive.org/details/dombeyson00dick_0/page/122/mode/2up?q=%22dursn%27t%22&view=theater}{Dickens, \textit{Dombey} 75}) % OJ doesn't provide the words for the example.
\z
\ex \label{ex:11-105}
\ea
they dursn't do it\hfill(\href{https://archive.org/details/sevenseas00kipl/page/144/mode/2up?view=theater&q=%22do+it%22}{Kipling, \textit{Seas} 166})
\ex
They dussent ave nothink to do with me\hfill(\href{https://archive.org/details/candidamystery02shawuoft/page/90/mode/2up?q=dussent&view=theater}{Shaw, \textit{Candida} 91})
\z
\ex \label{ex:11-107}
\ea
I durn't.\hfill(\href{https://archive.org/details/dli.ernet.2911/page/33/mode/2up?q=%22I+durn%27t%22&view=theater}{Masefield, \textit{Mercy} 39})
\ex
Even the potatoes daurna look like potatoes.\hfill(\href{https://archive.org/details/margaretogilvy00barr/page/76/mode/2up?q=daurna&view=theater}{Barrie, \textit{Margaret} 100}) % OJ merely writes "daurnd". I (PE) looked in several editions and the word ends not with D but with A. And I've restored its immediate context.
\ex
I dasn't scratch it\hfill(\href{https://archive.org/details/adventureshuckle00twaiiala/page/22/mode/2up?q=%22dasn%27t+scratch%22&view=theater}{Twain, \textit{Huckleberry} 1.17})
\z
\z

The sound [t] is also left out in the colloquial form [juˑsnt] for \textit{used not}; an American lady told me that this was childish: ``no grown-up person in America would say so'', but in England, it is very often heard, and also often written, see (\ref{ex:11-110}).\largerpage

\ea \label{ex:11-110}
\ea
my face is covered with little shadows that usen't to be there\\\hfill(\href{https://archive.org/details/thesecondmrstanq00pineuoft/page/188/mode/2up?q=%22my+face+is+covered%22&view=theater}{Pinero, \textit{Second} 189})
\ex
``I am not one of her admirers.'' --- ``I usen't to be, but I am now.''\\\hfill(\href{https://archive.org/details/ladywindermeresf00wildrich/page/86/mode/2up?q=%22one+of+her+admirers%22&view=theater}{Wilde, \textit{Fan} 37}) % OJ quoted this very loosely.
\ex
Usent it to be a lark?\hfill(\href{https://archive.org/details/cashelbyronsprof00shawuoft/page/n37/mode/2up?q=%22usent+it+to+be+a+lark%22&view=theater}{Shaw, \textit{Cashel} 11})
\ex
I'm blest if I usent to have to put him up\hfill(\href{https://archive.org/details/cashelbyronsprof00shawuoft/page/n219/mode/2up?q=%22blest+if+i+usent%22&view=theater}{ibid 193})% Jespersen here points to one more example in \textit{John Bull's Other Island} and two more in \textit{Man and Superman}. We have been unable to find any of the three. Certainly Shaw did use it elsewhere: among these, ``he usent to drink'' (\textit{Major Barbara}), and ``They usent to mean anything real to me'' and ``But youre not the larky sort. At least you usent to be'' (\textit{Fanny's First Play}). \eds}% OJ: "(Shaw J. 255, M. 192, 202)"

\ex Usen't we to be taught that it was our duty to love our enemies?\\\hfill(\href{https://archive.org/details/dramaticworksofs02hank/page/32/mode/2up?q=%22usen%27t+we%22&view=theater}{Hankin, \textit{Works} 2.47})
\ex Usen't the monks to keep peas in their boots {\dots}?\\\hfill(\href{https://archive.org/details/dodosecond00bensiala/page/266/mode/2up?view=theater&q=%22monks+to+keep+peas%22}{E. F. Benson, \textit{Second} 288}) % added dots
\z
\z

\bigskip

\il{English!ben't@\textit{ben't}}
\il{English!bain't@\textit{bain't}}
\textit{Ben't} seems now extinct except in dialects (\textit{bain't}); it was heard in educated society in Swift's time, see (\ref{ex:11-116}). 

\ea \label{ex:11-116}
\ea
So will you be, if you ben't hang'd when you're young.\\\hfill(\href{https://archive.org/details/cu31924013200898/page/n139/mode/2up?q=%22ben%27t+hang%27d%22&view=theater}{J. Swift, \textit{Conversation} 105}) % Context restored.
\ex
Well, Tom, if that ben't fair, hang fair.\hfill(\href{https://archive.org/details/cu31924013200898/page/n145/mode/2up?q=%22hang+fair%22&view=theater}{ibid 110}) % "Well, Tom," restored
\z
\z

Dialectal \textit{n't}-forms for the second person singular occur, for instance in Fielding, \textit{Tom Jones} (Squire Western): \textit{shatunt} `(thou) shalt not', \textit{wout unt} = `wouldst not', \textit{o'n't} or \textit{at unt} `art not', and others. 


For \textit{needn't} I find an abbreviated American form several times in Opie Read's \textit{Toothpick tales}, e.g. (\ref{ex:11-118}).

\ea \label{ex:11-118}
Yer neenter fly off'n the handle\hfill(\href{https://archive.org/details/toothpicktales00readgoog/page/n114/mode/2up?view=theater&q=%22yer+neenter%22}{108})
\z

There is a curious American form \textit{whyn't} = `why didn't' or `why don't' \citep{payne1909wordlist}; in T. N. Page's \textit{John Marvel, assistant}, a negro asks: (\ref{ex:11-119}).

\ea \label{ex:11-119}
Whyn't you stay where you is? \hfill (\href{https://archive.org/details/johnmarvelassist00page_0/page/56/mode/2up?view=theater&q=%22you+stay+where%22}{57})
% "where you is" restored
\z

\is{child language}
In children's speech, there is a negative form corresponding to \textit{you better do that} (from \textit{you'd better}), namely \textit{Bettern't you} (`had you not better') \citep[\href{https://archive.org/details/studiesofchildho02sull/page/176/mode/2up?view=theater\&q=\%22bettern\%27t\%22}{177}]{sully1903studies}.



\bigskip
\is{spoken v. written language}
The \textit{n't} forms are colloquial, but may be heard in university lectures, etc. They are not, however, used much in \textit{reading}, and it sounds hyper-colloquial, in some cases even with a comical tinge, when too many 
\il{English!don't@\textit{don't}}\textit{don't},\textit{ isn't} are substituted for 
\il{English!do not@\textit{do not}}\textit{do not},\textit{ is not}, etc., in reading serious prose aloud. In poetry, the contracted forms are justified only where other colloquial forms are allowed, e.g. (\ref{ex:11-120}).

\ea \label{ex:11-120}
They vow to amend their lives, and yet they don't; Because if drowned, they can't---if spared, they won't.\hfill(\href{https://archive.org/details/workslordbyron10unkngoog/page/220/mode/2up?view=theater&q=%22amend+their+lives%22}{Byron, \textit{Juan} 5.6})
\z

\is{stress}
Naturally, the full forms admit of greater emphasis on the negative element than the contracted forms; [kænɔt] is hardly ever heard in colloquial speech unless exceptionally stressed, and then the second syllable may have even stronger stress than the first (cf. the emphasis % Jespersen: ``cf. the italics.'' \eds} % Peter: I added this footnote back when I was extraordinarily reluctant to make any change to OJ's words, other than clarifying sources and spelling out abbreviations. Let's scrap this very trivial footnote.
%Brett: please, do
in \href{https://archive.org/details/personalhistory05dickgoog/page/n109/mode/2up?q=%22I+cannot+say%22&view=theater}{Dickens, \textit{David Copperfield} 241} \textit{I cannot say --- I really can\textsc{not} say}). In Byron's \textit{Don Juan} a distinction seems to be carried through between \textit{cannot} when the stress is on \textit{can}, and \textit{can not} when it is on \textit{not}. 
\il{English!will not@\textit{will not}}\textit{Will not} is more emphatic than \textit{won't} in \textit{I won't have it! I will not have it!} (Ridge, \textit{Garland} 219) % Checked against snippet view at Google Books; isn't there a good version we can link to?
But this does not apply to the two forms in \href{https://archive.org/details/gaylordquexcomed00pine/page/174/mode/2up?q=%22it%27s+not+true%22&view=theater}{Pinero, \textit{Quex} 213} \textit{It's not true! it isn't true!} 

The difference between the full and the contracted form is sometimes that between a special and a general negative (see chapter \ref{ch:5}); cf. \citet[§366]{sweet1892new1}:%\href{https://archive.org/details/newenglishgramma01sweeuoft/page/126/mode/2up?view=theater}{Sweet, \textit{New English Grammar} §366}: .... 
% PE: Previously "\citet[126 §366]{sweet1892new1}". Yes it's on page 126 (not all of which is devoted to §366), yes it's in §366 (not all of which is on page 126); but to provide both is unnecessary and looks odd.

\is{adjective-pronouns, negative}
\begin{quote}
In fact such sentences [as \textit{he is not a fool}] have in the spoken language two forms (hij iznt ə fuwl) and (hijz not ə fuwl). In the former the negation being attached specially to an unmeaning form-word must necessarily logically modify the whole sentence, just as in \textit{I do not think so} (ai dount þiŋk sou), so that the sentence is equivalent to `I deny that he is a fool'. In the other form of the sentence the \textit{not} is detached from the verb, and is thus at liberty to modify the following noun, so that the sentence is felt to be equivalent to \textit{he is no fool}, where there can be no doubt that the negative adjective-pronoun \textit{no} modifies the noun, so that (hijz not ə fuwl) is almost equivalent to `I assert that he is the opposite of a fool'.
\end{quote} 

On the distinction between \textit{may not} and \textit{mayn't}, \textit{must not} and \textit{mustn't} in some cases see p.~\pageref{must-and-may}ff.

The contracted forms are very often used in tag questions (\textit{He is old, isn't he? you know her, don't you?} etc.), and in such questions as are hardly questions at all, but another form of putting a positive assertion (\ref{ex:11-positive-assertion}).

\ea \label{ex:11-positive-assertion}
\ea Isn't he old? \\ (`he is very old': you cannot disagree with me on that point)
\ex Don't you know? \phantom{x} (`you surely must know') 
\z \z

\is{refinement, false|(}
\is{hypercorrection|(}
\noindent In a real question, therefore, it is preferable to say and write, for instance (\ref{ex:11-121}).

\ea \label{ex:11-121}
Did I meet the lady when I was with you? If not, \emph{did you not} know her at that time?
\z

\noindent because \textit{didn't you know her?} would seem to admit of only one reply.

With regard to the standing of the contracted forms and the way in which they are regarded by the phonetician as opposed to many laymen, there is a characteristic passage in \citet[379]{wyld1906historical}:
%\href{https://archive.org/details/historicalstudyo00wylduoft/page/378/mode/2up?view=theater}{Wyld's \textit{Historical Study}, p.~379}:

\begin{quote}
We occasionally hear peculiarly flagrant breaches of polite usage, such as (iz nɔt it) for (iznt it) or (æm nɔt ai), for the now rather old-fashioned, but still commendable, (ɛint ai) or the more usual and familiar (aˑnt ai), or, in Ireland, (æmnt ai). These forms, which can only be based upon an uneasy and nervous stumbling after `correctness', are perfectly indefensible, for no one ever uttered them naturally and spontaneously. They are struck out by the individual, in a painful gasp of false refinement.
% Above, Wyld provides his own pronunciations, according to a convention that differs from OJ's, such that for example "aren't" is ⟨ānt⟩. OJ modifies this, so that "aren't" is instead ⟨aˑnt⟩; but leaves the phonetic script in Wyld's ( ) rather than in his own [ ]
\end{quote}
\is{hypercorrection|)}
\is{refinement, false|)}


\il{English!Old English!na@\textit{nā}}
\il{Scottish!\textit{-na}}
\il{English!dialectal!northern!\textit{-na}}
In Northern English and Scottish, we have an enclitic \textit{-na} (from Old English \textit{nā}); thus frequently in \href{https://www.gutenberg.org/cache/epub/507/pg507-images.html}{George Eliot's \textit{Adam Bede}} \textit{donna}, \textit{mustna}, \textit{wasna}, \textit{wonna}, \textit{thee artna}, \textit{ye arena}; in Burns \textit{dinna}, \textit{winna}, \textit{wadna}, \textit{wasna}, etc.

\textit{Canna} is used by Goldsmith (\ref{ex:11-canna}) as vulgar, not as specifically Scottish.

\ea \label{ex:11-canna}
I'm sure it canna be mine \hfill(\href{https://archive.org/details/shestoopstoconqu03gold/page/30/mode/2up?q=%22canna%22&view=theater}{Goldsmith, \textit{Stoops} 560})
\z 
\is{negation, suffixes|)}
\il{English!n't@\textit{-n't}|)}

\is{auxiliary verbs|)}
