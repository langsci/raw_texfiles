\ChapterAndMark{Negative Attraction} 
\label{ch:6}
\is{attraction!of negative other than to verb|(}
\is{position of negative|(}
\is{special negation!instead of nexal negation|(}

While the preceding chapter has shown the universal tendency to attract the negative to the verb even where it logically belongs to some other word, there is another tendency to attract the negative notion to any word that can easily be made negative.\label{p:made-neg}\is{colloquial syntax v. literary syntax}\is{literary syntax v. colloquial syntax} In colloquial language the former is the stronger tendency, but in literary English the latter often predominates because it yields a more elegant expression. Thus to the colloquial \textit{we didn't meet anybody} corresponds a more literary \textit{we met nobody}. Cf. also \textit{union won't be an easy matter} and \textit{union will be no easy matter}. 

In the following sentences \refp{ex:06-01} the negative really belongs to the nexus and should therefore be placed with the verb; note especially the tag question (\textit{have we}? as after a negative \textit{we haven't got}) in \refp{ex:06-got-have-we}. % PE: I've rearranged the content of this sentence.

\ea \label{ex:06-01}\il{English!nothing@\textit{nothing}|(}
\ea those of thy tribe give nothing for nothing\newline (`don't give anything for nothing')\hfill(\href{https://archive.org/details/scottsivanhoeedi0000amar/page/80/mode/2up?q=%22those+of+thy+tribe%22&view=theater}{Scott, \textit{Ivanhoe} 89})
\ex She was aware of having done nothing wrong\hfill(\href{https://archive.org/details/breadwinnersaso04haygoog/page/n72/mode/2up?view=theater&q=%22aware+of+having+done%22}{Hay, \textit{Breadwinners} 68})
\ex she found that she could count certainly upon nobody\\\hfill(\href{https://archive.org/details/queensquairorsi00hewlgoog/page/50/mode/2up?q=%22found+that+she+could+count%22&view=theater}{Hewlett, \textit{Quair} 50})
\ex we ask him to do nothing against his cousin. We ask only his silence\\\hfill(\href{https://archive.org/details/in.ernet.dli.2015.53170/page/n299/mode/2up?q=%22against+his+cousin%22&view=theater}{Hope, \textit{Rupert} 230})
\ex she loves you so well that she has the heart to thwart you in nothing\\\hfill(\href{https://archive.org/details/originalplays00gilbgoog/page/n140/mode/2up?q=%22heart+to+thwart%22&view=theater}{Gilbert, \textit{Charity} 1})
\ex \label{ex:06-got-have-we} we've got a glass o' nothing in the house, have we, Bessy?\\\hfill(\href{https://archive.org/details/millonfloss0009geor/page/378/mode/2up?q=%22got+a+glass%22&view=theater}{Eliot, \textit{Mill} 2.114}) % not "of" but "o'"; adding "Bessy"
\ex \label{ex:6-or-any-part} 'tis none of my business, or any part of my design \hfill (\href{https://archive.org/details/fartheradventure00defo/page/278/mode/2up?q=%22any+part+of+my%22&view=theater}{Defoe, \textit{Farther} 299}) 
\z
\z

\noindent (The continuation of \refp{ex:6-or-any-part} with \textit{or any} shows that the beginning is felt to correspond to `it isn't any {\dots}'.) Cf. also the
examples in \textit{Modern English grammar} \citep[\href{https://archive.org/details/jespersen-1954-a-modern-english-grammar-on-historical-principles-part-ii-syntax-first-volume/page/426/mode/2up?view=theater}{16.74}]{jespersenMEG2}. % PE: "be =" rendered as "correspond to" ??? Delete the title of MEG?

This is particularly frequent with \textit{need} \refp{ex:06-08}. Cf. with a comparative: \refp{ex:06-11}.

\ea \label{ex:06-08}
\ea Of ladders I need say nothing\hfill(\href{https://en.wikisource.org/wiki/Page:The_Works_of_the_Rev._Jonathan_Swift,_Volume_2.djvu/126}{J. Swift, \textit{Tub} 26})
\ex you need be under no uneasiness\hfill(\href{https://archive.org/details/TheVicarOfWakefield/page/n123/mode/2up?q=%22under+no+uneasiness%22&view=theater}{Goldsmith, \textit{Vicar}}; and \href{https://archive.org/details/TheVicarOfWakefield/page/n145/mode/2up?q=%22under+no+uneasiness%22&view=theater}{ibid}, \href{https://archive.org/details/TheVicarOfWakefield/page/n269/mode/2up?q=%22under+no+uneasiness%22&view=theater}{ibid}) % OJ's sole example has become my (PE's) three. Yes, the exact same string occurs three times. I could do the simple algebraic exercise needed to deduce which of the three OJ had in mind, but life's too short....
\ex ye need say nothing about that foolish story\hfill(\href{https://archive.org/details/cewaverleynovels03scotuoft/page/42/mode/2up?q=%22need+say+nothing%22&view=theater}{Scott, \textit{Antiquary} 1.63})
\z
\z

\ea \label{ex:06-11}
\ea I need tell you no more\hfill(\href{https://archive.org/details/journaltostellae00swifuoft/page/460/mode/2up?q=%22need+tell+you+no+more%22&view=theater}{J. Swift, \textit{Journal} 461})
\ex We need detain you no longer\hfill(\href{https://archive.org/details/lifeadventuresofdickrich/page/138/mode/2up?q=%22detain+you+no+longer%22&view=theater}{Dickens, \textit{Nicholas} 125})
\z
\z

\is{quantifiers!interaction of negation and other|(}
A curious example is \refp{ex:06-13}, where \textit{hardly any} is used as a mitigated \textit{no}; the logical expression would be: \textit{I have hardly succeeded in throwing any light}.

\ea \label{ex:06-13}
the whole subject {\dots} is so obscure, that I have succeeded in throwing hardly any light on it\hfill(\href{https://archive.org/details/expressionofemot1872darw/page/92/mode/2up?view=theater&q=%22whole+subject%22}{Darwin, \textit{Expression} 93}) % Dots added to show a major cut by OJ
\z

Note also \refp{ex:06-14}.

\ea \label{ex:06-14}
\ea to be able to do nothing [`unable to do anything'] without hurting someone\hfill(\href{https://archive.org/details/darkflower0000john_y2w0/page/104/mode/2up?q=%22without+hurting+someone%22&view=theater}{Galsworthy, \textit{Flower} 101})
\ex you and I will go to the smoking-room, and talk about nothing at all subtle \phantom{x} (`something that is not subtle')\hfill(\href{https://archive.org/details/dododetailofday00bensuoft/page/54/mode/2up?view=theater&q=%22you+and+I+will+go%22}{E. F. Benson, \textit{Dodo} 50})\il{English!nothing@\textit{nothing}|)}
\ex I'm no Bear any longer \phantom{x} (`am a Bear no longer')\hfill(\href{https://archive.org/details/pitepicofwheatde00norruoft/page/144/mode/2up?view=theater&q=%22I%27m+no+Bear%22}{Norris, \textit{Pit} 183})
\z
\z

\citet[\href{https://archive.org/details/p2englischephilo01storuoft/page/694/mode/2up?q=\%22had+no+right+to+be+treated+like+common+soldiers\%22&view=theater}{694}]{storm1896englische} has a few curious quotations like \refp{ex:06-17}.

\ea \label{ex:06-17}
O'Brien {\dots} stated that we were officers, and \textit{had no right} to be treated like common soldiers \phantom{x} (`and had a right not to be treated').\\\hfill(\href{https://archive.org/details/petersimple01mar/page/324/mode/2up?q=\%22stated+that\%22&view=theater}{Marryat, \textit{Peter}, 1.324}) % OJ doesn't specify the title, let alone the edition or page number. Incidentally, it's Storm who surreptitiously cuts several words; OJ faithfully follows Storm.
\z

This tendency leads to the use of combinations like \textit{he was no ordinary boy} in preference to the unidiomatic \textit{he was a not ordinary boy}; for examples see \textit{Modern English grammar} \citep[\href{https://archive.org/details/jespersen-1954-a-modern-english-grammar-on-historical-principles-part-ii-syntax-first-volume/page/426/mode/2up?view=theater}{16.751}]{jespersenMEG2}. % ??? PE: Again, delete the title of MEG?

Similarly in Spanish, \refp{ex:06-18}.

\ea \label{ex:06-18}
 \gll Era un santo varón piadoso y de \textit{no} \textit{común} saber\\
 was a holy man pious and of not common knowledge\\
\glt `He was a pious holy man with uncommon knowledge'\\\hfill(\href{https://archive.org/details/Donaperfecta/page/n39/mode/2up?q=%22var%C3%B3n+piadoso%22&view=theater}{Pérez Galdós, \textit{Doña} 39})
\z

\il{Danish!ingen@\textit{ingen}|(} \il{Danish!ingenting@\textit{ingenting}|(}The attraction of the negative element is the reason why a pronoun, like \textit{ingen},\textit{ ingenting}, \il{Danish!intet@\textit{intet}}\textit{intet} is very often in Danish placed in a position which would be impossible in the case of a positive pronoun, but is the one required for the adverb \textit{ikke}: \refp{ex:06-19}, or, more popularly, \refp{ex:06-22}, etc. Cf. also the following quotations \refp{ex:06-23}, the last two or three of which are, perhaps, not quite natural, though the attraction in them is easy to understand. Bjørnson's \refp{ex:06-30} % ??? PE: This was my helpful (?) addition. BB is described (in Wikipedia, etc) as Norwegian and as a writer of great Norwegian literature, and OJ contrasts this with "natural Danish". All in all "It's in Norwegian" seemed the obvious inference. But for all I know, it might better be described as Norwegian-dialect-inflected Danish. Ideally, we'd get a comment from an L1 Danish or (better) Norwegian speaker.
would in natural Danish be rather \textit{bærer ingen over med}.

\ea \label{ex:06-19}
\ea
 \gll det fører ingenting til\\
 that leads nothing to\\
\glt `that leads to nothing'
\ex
 \gll det er ingen skade til\\
 that is no harm to\\
\glt `there is no harm in that'
\ex
 \gll når man ingenting har\\ % The PDF appears to have "ingènting"; but I (PE) suspect that this is merely some kind of error.
 %% SG: definitely an error for "ingenting"
 when one nothing has\\
\glt `when one has nothing'
\z
\z

\ea \label{ex:06-22}
\gll når ingenting man har\\
    when nothing one has\\
\glt `When one has nothing'
% ??? PE: Needs a gloss, and probably a translation too
%% SG: Added. 
\z

\ea \label{ex:06-23}
\ea
 \gll Thi man må ingen gjøre uret\\
 for one may/must {no one} do injustice\\ % According to OJ, "Ti" and "gøre"; the book I (PE) link to has different spellings.
 %% SG: OJ just modernized these spellings. "Thi" is very archaic (def. more so than English "for" as a conjunction), and we usually use the old-fashioned spelling with <h> in contemporary Danish anyway
 %% SG: Note that "må" corresponds to either "may" or "must" depending on the context. I think the most idiomatic Eng. translation is to "dissolve" the negative pronoun, hence "may not + anyone"
\glt `For one may not commit injustice against anyone'\\\hfill(\href{https://archive.org/details/samledeafhandlin04peteuoft/page/122/mode/2up?q=%22thi+man+m%C3%A5+ingen+gore+uret%22&view=theater}{Petersen, \textit{Uddrag} 14}) % OJ credits "N. M. Petersen Afhdl. 4. 123", which I suppose means volume 4 of some periodical. I can't locate this. I'm substituting a book that says more or less the same.
%Brett: The title is "Samlede afhandlinger", volume 4. Not sure how you want to handle that in the citation and list of sources. % PE Peter: I've (rather messily) added information to the entry in the bibliography; but this will of course become a differently-formatted BibTeX entry.
\ex
 \gll Det franske sprog har ingen fordærvet, men den franske gouvernante har gjort det\\
 the French language has {no one} corrupted but the French governess has done it\\
\glt `The French language has corrupted no one, but the French governess has'\hfill(\href{https://books.google.co.jp/books?id=wolFAAAAcAAJ&pg=PA17&lpg=PA17&dq=%22Det+franske+sprog+har+ingen+ford%C3%A6rvet%22&source=bl&ots=yFxXLQduT9&sig=ACfU3U1CjhNS5gs6x629T5PjRf9PGxX_rg&hl=en&sa=X&ved=2ahUKEwjT0vD087aFAxWCePUHHWVOARAQ6AF6BAgJEAM#v=onepage&q=%22Det%20franske%20sprog%20har%20ingen%20ford%C3%A6rvet%22&f=false}{ibid 17})
\ex \il{Danish!intet@\textit{intet}|(}
 \gll lad pøblen intet mærke\\
 let crowd.\DEF{} nothing notice\\
\glt `don't let the crowd notice anything'\hfill(\href{https://tekster.kb.dk/text/adl-texts-goldschmidt03-root#idm139686918080032}{Goldschmidt, \textit{Hjemløs} 2.841})
%% SG: Or "Don't let the crowd notice anything"
\ex  \gll Tage mærkede imidlertid ingen kølighed til\\
 Tage noticed however no coldness to\\
\glt `However, Tage noticed no coldness'\hfill(\href{https://tekster.kb.dk/text/adl-texts-jacob03val-root#idm139892126928240}{Jacobsen, \textit{Fønss} 2.406})
\ex  \gll Den samme jordlod, som for 20 aar siden intet eller lidet udbytte gav, fordi der intet eller lidet arbejde var nedlagt i dens drift\\
 the same plot which for 20 years ago nothing or little yield gave because there nothing or little work was {laid down} in its cultivation\\
\glt `The same plot of land, which 20 years ago gave little or no yield because little or no work had been invested in its cultivation'\\\hfill(\href{https://archive.org/details/Tilskuerenmaaned1902vvgoog/page/386/mode/2up?q=%22Den+samme+jordlod%22&view=theater}{G. Bang, \textit{Husmanden} 386})
\ex 
 \gll Jeg veed ogsaa, at jeg intet af alt dette har gjort selv\\
 I know also that I nothing of all this have done myself\\
\glt `I also know that I have done none of this myself'\\\hfill(Johs. Jørgensen, news 1915) % Would this be Johannes Jørgensen https://en.wikipedia.org/wiki/Johannes_J%C3%B8rgensen ?
%% SG: Most likely
\il{Danish!intet@\textit{intet}|)}
\ex 
 \gll for at jeg ingenting andet skulde ha’ at hæfte mig ved\\
 {in order} that I nothing else should have to attach myself by\\
\glt `in order that I would have nothing else to cling to'\hfill(\href{https://archive.org/details/bygmestersolnes00ibsegoog/page/n218/mode/2up?view=theater&q=%22for+at+jeg+ingen%C2%ADting+andet%22}{Ibsen, \textit{Solness} 204})
\z
\z\il{Danish!ingenting@\textit{ingenting}|)}

\ea \label{ex:06-30}
 \gll de bærer over med ingen\\
 they bear over with {no one}\\
\glt `they tolerate no one'\hfill(\href{http://f9.no/ebok/filer/bjornson_det_flager_i_byen_og_paa_havnen.html}{Norwegian: \textit{Flager} 48})
\z

\label{join-to-first}Whenever there is logically a possibility of attracting the negative element to either of two words, there seems to be a universal tendency to join it to the first. We may say \il{English!no one@\textit{no one}}\textit{\textsc{no one ever} saw him angry} or \textit{\textsc{never} did \textsc{any one} see him angry}, but not \textit{\textsc{any one never} saw him angry} nor \textit{\textsc{ever} did \textsc{no one} see him angry}. In the same way in Danish \textit{\textsc{ingen} har \textsc{nogensinde} set ham vred} or \textit{\textsc{aldrig} har \textsc{nogen} set ham vred}, but not otherwise. Instead of \textit{\textsc{no} woman would \textsc{ever} think of that} it is impossible to say \textit{\textsc{any} woman would \textsc{never} think of that}, though it is possible to say \textit{\textsc{a} woman would \textsc{never} think of that}, because \textit{no} is not (now) felt to be a combination of the negative element and the indefinite article. % Peter: Now shorn of underlining to show emphasis-within-what's-already-italicized, this is hard to understand. But in this paragraph OJ makes liberal use of " ". The result isn't elegant, but it's easier to follow. Shall we do the same? 
% Brett: Now the emphasized elements seem de-emphasized. How about small caps?
% PE: Done
\il{Danish!ingen@\textit{ingen}|)}

\is{grammaticalization}
The negative is also attracted to the first word in well-known Latin combinations \il{Latin!neque quisquam@\textit{neque quisquam}}\textit{neque quisquam} `and no one' (not \il{Latin!et nemo@\textit{et nemo}}\textit{et nemo}), \il{Latin!neque ullus@\textit{neque ullus}}\textit{neque ullus} `and nothing', \il{Latin!nec unquam@\textit{nec unquam}}\textit{nec unquam} `and never'; thus also \il{Latin!ne quis@\textit{ne quis}}\textit{ne quis}, \il{Latin!ne quid@\textit{ne quid}}\textit{ne quid}, etc., in clauses of purpose. The same tendency is found in combinations like \textit{without any danger}, \textit{uden nogen fare}, \textit{sine ullo periculo}, where, however, English sometimes has \textit{with no danger} (\textit{to any one}); cf. \refp{ex:06-31}.\largerpage[2]

\ea \label{ex:06-31}
\ea it is a spot which has all the solemnity, with none of the savageness, of the Alps\hfill(\href{https://archive.org/details/selectionsfromwr0000rusk/page/4/mode/2up?q=%22spot+which+has+all+the+solemnity%22&view=theater}{Ruskin, \textit{Selections} 1.9})
\ex she went out, with not another word or look\hfill(\href{https://archive.org/details/powersmaxine00williala/page/250/mode/2up?q=%22she+went+out%2C+with+not+another+word+or+look%22&view=theater}{Williamson, \textit{Powers} 231})
\z
\z

\il{English!Old English!aefre@\textit{æfre}|(}\il{English!Old English!aenig@\textit{ænig}|(}It strikes one as contrary to this universal tendency to find in Old English poetry combinations in which \textit{æfre} or \textit{ænig} precedes a verb with prefixed \textit{ne}, as in \href{https://archive.org/details/andreasandfateso00cyneuoft/andreasandfateso00cyneuoft/mode/2up?ref=ol&view=theater}{\textit{Andreas}} \refp{ex:06-33}, and lines \href{https://archive.org/details/andreasandfateso00cyneuoft/andreasandfateso00cyneuoft/page/20/mode/2up?ref=ol&view=theater}{499}, \href{https://archive.org/details/andreasandfateso00cyneuoft/andreasandfateso00cyneuoft/page/22/mode/2up?ref=ol&view=theater}{553} etc. Both combined: \refp{ex:06-34}.

\ea \label{ex:06-33}
\ea
 \gll þær ænig þa git elþeodigra eðles ne mihte blædes brucan\\
 where any then yet {of foreigners} land not might prosperity enjoy\\
\glt `Where no foreigners could yet enjoy the land and prosperity'\\\hfill(\href{https://archive.org/details/andreasandfateso00cyneuoft/andreasandfateso00cyneuoft/page/2/mode/2up?ref=ol&view=theater}{15b–17a}) % The original is bristling with what look like macrons.
%Brett: Here's a different option: No macrons, but different characters https://archive.org/details/bibliothekderangelsachsischenpoesieb2h1/page/n13/mode/2up?q=%22wordum+writan+wundorcraefte%22&view=theater
% Peter: Pleading total ignorance of OE matters, I defer to you. Incidentally, I've changed "not any foreigners" to "no foreigners"
\ex \gll \textit{Æfre} ic \textit{ne} hyrde\\
 ever I not heard\\
\glt `never have I ever heard'\hfill(\href{https://archive.org/details/andreasandfateso00cyneuoft/andreasandfateso00cyneuoft/page/14/mode/2up?ref=ol&view=theater}{360b})
\ex \gll \textit{ænig} \textit{ne} wende, / þæt he lifgende land begete\\
 any not thought {} that he living land {would reach}\\
\glt `no one thought that he, while living, would reach land'\hfill(\href{https://archive.org/details/andreasandfateso00cyneuoft/andreasandfateso00cyneuoft/page/14/mode/2up?ref=ol&view=theater}{377b–378a})
\z
\z\il{English!Old English!aenig@\textit{ænig}|)}

\ea \label{ex:06-34}
 \gll swa ic \textit{æfre} \textit{ne} geseah \textit{ænigne} mann\\
 so I ever not saw any man\\
\glt `so I never saw any man'\hfill(\href{https://archive.org/details/andreasandfateso00cyneuoft/andreasandfateso00cyneuoft/page/20/mode/2up?ref=ol&view=theater}{ibid 493})
\z\il{English!Old English!aefre@\textit{æfre}|)}

\is{attraction!of negative to subject|(}
When the negative is attracted to the subject, the sentence is often continued in such a way that the positive counterpart of the first subject must be understood. In ordinary life, such sentences will cause no misunderstanding, and it is only the critical, or even hyper-critical, grammarian that sees anything wrong in them. Examples: \refp{ex:06-35}.

\ea \label{ex:06-35}\il{English!but@\textit{but}|(}
\ea Not one should scape, but perish by our swords \phantom{x} (`but all perish')\\\hfill(\href{https://quod.lib.umich.edu/e/eebo/A07004.0001.001/1:3.5.2?rgn=div3;view=fulltext}{Marlowe, \textit{Tamburlaine} 4.2})
\ex I pray him That none of you may liue his naturall age, But by some vnlook'd accident cut off\hfill(\href{https://internetshakespeare.uvic.ca/doc/R3_F1/scene/1.3/index.html#tln-680}{Shakespeare, \textit{R3} 1.3.213})
\ex none of them are hurtful, but loving and holy \phantom{x} (`but they are')\\\hfill(\href{https://archive.org/details/bunyanspilgrims00moffgoog/page/18/mode/2up?q=%22are+hurtful%22&view=theater}{Bunyan, \textit{Progress} 147}) % OJ misattributes this to "Bunyan G."; it's not there.
\ex no man may judge another by looking down upon him, but must needs descend into the crowd\hfill(\href{https://archive.org/details/vulturesnovel00merr/page/266/mode/2up?ref=ol&view=theater&q=%22judge+another+by+looking%22}{Merriman, \textit{Vultures} 265})
\ex Neither spoke, but lay silently listening \hfill(`both lay'; \href{https://archive.org/details/ladyofbarge00jaco/page/50/mode/2up?q=%22neither+spoke%22&view=theater}{Jacobs, \textit{Lady} 51})
\ex Don't let any of us go to bed to-night, but see the morning come\\\hfill(\href{https://archive.org/details/dodosecond00bensiala/page/120/mode/2up?view=theater&q=%22don%27t+let+any+of+us%22}{E. F. Benson, \textit{Second} 130})\il{English!but@\textit{but}|)}
\ex Nobody 'll get anything till eight, and then [they'll get]\textit{ only cold shoulder}\hfill(\href{https://archive.org/details/joyplayonletteri00gals/page/124/mode/2up?q=%22cold+shoulder%22&view=theater}{Galsworthy, \textit{Joy} 2}) % The original play has "Nobody [space] 'll"
\ex None of these versions throw any further light upon the original form, and are therefore not important for our analysis of it\\
(`These versions throw no {\dots}')\hfill(\href{https://babel.hathitrust.org/cgi/pt?id=hvd.hn8nyu&seq=39}{Paton, \textit{Tower} 23}) % "of it" restored
\z
\z

We find the same phenomenon with \textit{few}, as that, too, has a negative purport: \refp{ex:06-43}.

\ea \label{ex:06-43}
\ea few of the princes had any wish to enlarge their bounds, but passed their lives in full conviction that they had all\hfill(\href{https://archive.org/details/historyrasselas01johngoog/page/n45/mode/2up?q=%22few+of+the+princes%22&view=theater}{Johnson, \textit{Rasselas} 40}) % Might want to add "within their reach that art or nature could bestow" after "all"
\ex Few thought of Jessop,---only of themselves\newline (`they thought only of themselves')\hfill(\href{https://archive.org/details/johnhalifaxgentlcrai/page/492/mode/2up?q=%22few+thought+of+jessop%22&view=theater}{Mulock, \textit{Halifax} 2.152}) % yes, a comma+dash combination in the original
\z
\z
\is{quantifiers!interaction of negation and other|)}
\is{quantifiers!negative}

\emergencystretch=3em
Similarly in the following quotations \refp{ex:06-45}: \textit{forget} = `do not remember'; \textit{unfrequented}~= `frequented by (of) no one'.

\ea \label{ex:06-45}
\ea new made honor doth forget mens names\hfill(\href{https://internetshakespeare.uvic.ca/doc/Jn_F1/index.html#tln-195}{Shakespeare, \textit{John} 1.1.188}) % OJ asked readers to compare something by Shakespeare which he expected them to look up for themselves. I (PE) have spelled out the quote and rearranged this area; before I did so, it had become unrecognizable as a paragraph
\ex I forget, without looking back to some old letters {\dots}, whether it was my great-grandfather\hfill(\href{https://archive.org/details/lifeadventuresofdickrich/page/636/mode/2up?q=%22looking+back+to+some+old+letters%22&view=theater}{Dickens, \textit{Nicholas} 607}) % Dots added to mark a cut made by OJ; great-hyphen-grandfather
\ex I quite forget the details, only that I had a good deal of talk with him\\\hfill(\href{https://archive.org/details/reminiscences0000thom_e9a0/page/518/mode/2up?q=%22quite+forget+the+details%22&view=theater}{T. Carlyle, \textit{Reminiscences} 2.317})
\ex the house vnfrequented, onely of their owne housholde\\\hfill(\href{https://archive.org/details/cu31924013126390/page/n113/mode/2up?view=theater&q=%22house+vnfrequented%22}{Wilkins, \textit{Pericles} 67}) % Only one "e" in "housholde"
\ex it is idle to consider how much territory may come up for settlement, nor how it may be disposed of\hfill (idle
= `no use'; \href{https://archive.org/details/afterwar00dickrich/page/22/mode/2up?q=%22idle+to+consider+how+much+territory+may+come+up+for+settlement%22&view=theater}{Dickinson, \textit{War} 22})
\z
\z

Danish examples of sentences begun negatively and continued as if begun positively: \refp{ex:06-50}.

\ea \label{ex:06-50}
\ea\il{Danish!intet@\textit{intet}|(}
 \gll Intet af de finniske sprog adskiller kjøn, hvori de ligne grønlandsken, men have ellers en vidtløftig deklinering.\\
 none of the Finnish languages distinguishes gender wherein they resemble Greenlandic but have otherwise an extensive declension\\
\glt `None of the Finnish languages distinguish gender, in which they resemble Greenlandic, but they otherwise have an extensive declension.'\hfill(\href{https://archive.org/details/undersgelseomde00raskgoog/page/n111/mode/2up?q=%22intet+af+de%22&view=theater}{Rask, \textit{Undersögelse} 97}) % The original book capitalizes nouns; but it also uses "ö" wherever OJ uses "ø". ??? Does "otherwise have" mean "differ from it in having"?
%% SG: no, it just means that the Fin. lgs. have extensive declension despite lacking gender. The relative clause about Greenlandic is just a side note %% PE: Got it.
\ex \il{Danish!ingen@\textit{ingen}|(}
 \gll Ingen pil bliver længe hængende derved [ved hjertet], men flyver tvert igjennem\\
 no arrow remains long hanging thereby by heart.\DEF{} but flies across through\\
\glt `No arrow stays lodged by the heart for long, but flies straight through'\hfill(\href{https://archive.org/details/ChristianWintherEtLivsbillede/page/n209/mode/2up?view=theater&q=%22pil+bliver+l%C3%A6nge+h%C3%A6ngende%22}{P. Møller, letter}) % Jespersen quotes ``tvert igjennem''; Nicolaj Bøgh (\textit{Christian Winther: Et Livsbillede} 1.186) quotes ``tværsigjennem''.  % PE: OJ writes "in Vilh. Andersen 181". I guess that this would be in Poul (Martin) Møller's two-volume Udvalgte Skrifter, ed. Vilhelm Andersen (1895). But it doesn't seem to be in the second volume, and I can't find the first volume on the web.
%Brett: ¯\_(ツ)_/¯
% PE: Found a new source.
\ex 
 \gll ingen begivenhed havde interesse uden som del af hans indre historie eller fik kun ved denne sin rette farve\\
 no event had interest without as part of his inner story or got only by it its proper color\\
\glt `no event was of interest unless as a part of his inner story, or only through this it obtained its true color'\hfill(\href{https://tekster.kb.dk/text/adl-texts-goldschmidt04-root#idm140467282407024}{Goldschmidt, \textit{Hjemløs} 5.116}) % ??? (i) The edition linked to has "deel" and "denne" where OJ has "del" and "den". What to do? (ii) I don't understand the translation.
%% SG: OJ is evidently misquoting, "denne" is correct (and the sentence makes better sense that way). And "Deel/del" is just modernized by OJ. I've corrected "den" to "denne", I leave it to you whether you want to restore the capitalization
\ex 
 \gll Bare ingen vil skoptisere over mig, men lade mig have ro!\\
 if.only {no one} will scoff over me but let me have peace\\
\glt `If only no one will scoff at me, but leave me in peace!'\\\hfill(\href{https://archive.org/details/ravnenfortlling00goldgoog/page/332/mode/2up?q=%22mig+have+ro%22&view=theater}{Goldschmidt, \textit{Ravnen} 332}) % OJ attributes this to volume 7, page 507 of something I (PE) can't identify. But elsewhere in Negation, he refers to the edition I've linked to. And I've therefore standardized on it.
\ex
 \gll Intet betragtede han som tilfældigt, men som et led i den store kjæde\\
 nothing considered he as random but as a link in the great chain\\
\glt `He considered nothing as random, but as a link in the great chain'\\\hfill(\href{https://tekster.kb.dk/text/adl-texts-andersen02val-root#idm140563624535424}{Andersen, \textit{Baronesser} 2.66})
\ex 
 \gll jeg havde den tilfredsstillelse, at ikke en eneste af mine 10 tilhørere forlod mig, men holdt alle ud til den sidste time\\
 I had the satisfaction that not a single of my 10 listeners left me but held all out to the last hour\\
\glt `I had the satisfaction that not one of my ten listeners left me, but all endured until the last hour'\hfill(Molbech, letter 155)
\ex
 \gll Intet menneskeligt forhold kan have værdi i sig selv, men har kun værdi, naar det bevidst underordnes {\dots} uendelighedssynspunktet\\
 no human circumstance can have value in it self but has only value when it consciously {is subordinated} {} {infinity perspective.\DEF}\\
\glt `No human condition can have value in itself, but only has value when it is consciously subordinated to the perspective of infinity'\\\hfill(\href{https://archive.org/details/111408025488-bw_202406/page/104/mode/2up?q=%22Intet+menneskeligt+Forhold+kan+have+V%C3%A6rdi+i+sig+selv%2C+men+har+kun+V%C3%A6rdi%2C+naar+det+bevidst+underordnes+og+omdannes+ud+fra%22&view=theater}{Høffding, \textit{Humor} 104})
\il{Danish!intet@\textit{intet}|)}
\ex 
 \gll Når korn blev kørt hjem, drak ingen af sin egen flaske, men fik brændevin af manden\\
 when grain was driven home drank {no one} from his own bottle but got brandy from man.\DEF{}\\
\glt `When grain was brought home, no one drank from his own bottle, but received brandy from the man'\hfill(\href{https://archive.org/details/dania06samfgoog/page/n123/mode/2up?q=%22drak+ingen+af+sin+egen+flaske%22&view=theater}{Feilberg, \textit{Snaps} 117})
\ex 
 \gll jeg saa, at ingen elskede hende, men forførte hende og handlede ilde med hende\\
 I saw that {no one} loved her but seduced her and dealt badly with her\\
\glt `I saw that no one loved her; instead, they seduced her and treated her badly.'\hfill(Nielsen, \textit{Tilskueren} 1898. 694) % ??S PE: (i) OJ attributes this to "L. C. Nielsen Tilsk. ’98. 694". Clearly this means p.694 of Tilskueren for 1898: it may well be there, but I (PE) can't find Tilskueren for 1898 on the web. However, the sentence is also on p.26 of L. C. Nielsen's book Historier (Kjøbenhavn, 1907): https://archive.org/details/historier00nielgoog/page/n38/mode/2up?q=%22ingen+elskede+hende%22&view=theater (ii) Should this be understood as meaning "... but all seduced her ..."? (Taken literally, the English translation in its current form says the reverse.)
%Brett: changed the translation
\ex 
 \gll Ingen af dem [teorierne] kan siges at være fyldestgørende og forbigaas derfor her\\
 none of them theories.\DEF{} can {be said} to be comprehensive and {are omitted} therefore here\\
\glt `None of the theories can be said to be comprehensive, and they are therefore omitted here'\hfill(Johannsen, \textit{Salmonsen} 9.184) % ??S Guess: this is something that one W. Johannsen wrote in in volume 9 of Salmonsens Konversationsleksikon (a major encyclopedia). But volume 9 of the 2nd edition was only published in 1920 (says https://runeberg.org/salmonsen/2 ), and I (PE) haven't yet been able to find volume 9 of the first.
\il{Danish!ingen@\textit{ingen}|)}
\ex 
 \gll Ikke ên af hundrede læsere gör sig rede for hvorfor, og vil også have nogen vanskelighed ved at indse grunden\\ % Sörensen (as the cited book's title page spells his name) writes not "gør" but "gör"; OJ omits "for" and my (PE's) restoration of it may need an adjustment to the translation
 not one of {hundred} readers do themselves clarity for why and will also have some difficulty by to grasp reason\\
\glt `Not one out of a hundred readers considers why, and will also have quite some difficulty grasping the reason.'\hfill(\href{https://archive.org/details/enariadnetrdgen00srgoog/page/n63/mode/2up?view=theater&q=%22ikke+en+af%22}{Sörensen, \textit{Ariadnetråd} 52})\footnote{Negative continued as if positive: A reference has here unfortunately fallen out to \citet[\href{https://archive.org/details/nordisktidsskrif08kb/page/n17/mode/2up?view=theater}{8ff}]{siesbye1899stroebemerkninger} and \citet[\href{https://archive.org/details/dania00samfgoog/page/n51/mode/2up?view=theater}{44}]{siesbye1903kuriosa} (from Jespersen's Addenda). \eds} % Siesbye, \href{https://archive.org/details/nordisktidsskrif08kb/page/n17/mode/2up?view=theater}{\textit{Strøbemærkninger} 8ff} and \href{https://archive.org/details/dania00samfgoog/page/n51/mode/2up?view=theater}{\textit{Kuriosa} 44}
\z
\z

The following quotations are somewhat different: \refp{ex:06-61}.

\ea \label{ex:06-61}
\ea
 \gll Jeg kand skaffe attester fra hele byen, at jeg er ingen hane eller at nogen af mine forældre har været andet end christne mennisker\\ % Linked-to source has not "hele" but "helle"
 I can obtain certificates from whole town that I am no rooster or that either of my parents have been other than Christian people\\
\glt `I can provide certificates from the entire town, proving that I am no rooster or that neither of my parents has been anything other than a Christian person'\hfill(\href{http://holbergsskrifter.dk/holberg-public/view?docId=skuespill%2FErasmus%2FErasmus.page&brand=&chunk.id=act4sc2&toc.id=act4&toc.depth=1}{Holberg, \textit{Erasmus} 4.2}) % PE: (i) OJ writes "mennesker"; the web page has "Mennisker"; (ii) ??? "parents", or "ancestors"? (If "parents", then not "any" but "either" ... and even then, the result would I think be subtly ungrammatical.
%% SG: The reason is that the negative polarity item "nogen" is triggered by the negation in the previous clause "I am no rooster". I think the best solution is to write "neither" in the Eng. transl.
\ex
 \gll Langtfra alle vil samstemme med prof. Steenstrup {\dots} men vil hellere slutte sig til Bricka's beskedne tvivl \phantom{x} [mange vil ikke{\dots}]\\
 {far from} everyone will agree with Prof. Steenstrup {} but would rather join themselves to Bricka's modest doubt \phantom{x} [many will not]\\
\glt `Far from everyone will agree with Prof. Steenstrup {\dots} but will instead lean towards Bricka's modest doubt'\hfill(Friis, \textit{Politiken}) % ??? I (PE) suspect that this is the historian Aage Friis, writing in either the 6 Feb '06 or the 2 June '06 issue of the newspaper Politiken (OJ writes "6. 2. 06") ... except that although OJ sometimes specifies the author of an "NP." (i.e. newspaper or magazine) quote, for no other "NP" quote, I think, does he specify the date. So I really wonder....
\z
\z

\is{quantifiers!negative}
\il{Danish!de færreste@\textit{de færreste}|(}Thus also with Danish \textit{de færreste} (`the fewest', or `de fleste {\dots} ikke'): \refp{ex:06-63}.

\ea \label{ex:06-63}
\ea
 \gll de færreste af disse tropper er imidlertid bevæbnede med nye gode rifler, men nøjes med gamle flintebøsser\\
 the fewest of these troops are however armed with new good rifles but {make do} with old flintlocks\\
\glt `however, only very few of these troops are armed with new high-quality rifles, and most make do with old flintlocks'\hfill(news 1892)
%% SG: I hope the gloss and translation together make the Danish construction transparent. The point is that only few have new rifles, most have old flintlocks. However, there is no "most" in the Danish, it's understood from the context \\ %% PE: I think it works now.
\ex 
 \gll De færreste forstod meningen eller vilde ikke forstaa den\\
 the fewest understood meaning.\DEF{} or wanted not understand it\\
\glt `Only a few understood the meaning, and most did not want to understand it'\hfill(Arnskov, \textit{Nielsen} 29) % OJ attributes this to "Arnskov Tilsk. '14. 29". Unfortunately I (PE) can't find Tilskueren 1914 anywhere on the web. An image of the front cover of the July 1914 issue shows that L. Th. Arnskov's piece "Anders Nielsen" starts on p. 29. Incidentally, the sentence is quoted in a dictionary: https://archive.org/details/ordbogoverdetdan09dansuoft/page/47/mode/2up?q=%22De+f%C3%A6rreste+forstod+meningen+eller+vilde+ikke+forstaa+den%22&view=theater . 
\z
\z\il{Danish!de færreste@\textit{de færreste}|)}
\is{attraction!of negative to subject|)}

\il{English!and not@\textit{and not}|(}\textit{And} with a negative infinitive means the same thing as \textit{without -ing}. This is felt to be perfectly natural in positive sentences (\ref{ex:positive}), but there is a growing awkwardness about the construction in the following groups: negative sentences (\ref{ex:negative}), interrogative sentences, generally equivalent to negative statements (\ref{ex:interrogative}), and negative interrogative sentences (\ref{ex:neg-interrogative}); the sentence in (\ref{ex:unanalyzable}) is, strictly speaking, quite unanalyzable. In \textit{I couldn't see you, and not love you} (\ref{ex:negative-love}) \textit{couldn't} refers at the same time to \textit{see you}, and to \textit{not love you}, the latter in a way that would be quite unidiomatic if used by itself: \textit{I couldn't not love you} (cf. Latin \il{Latin!non possum non amare@\textit{non possum non amare}}\textit{non possum non amare}); we see that the expression is unimpeachable if we substitute: \il{English!impossible@\textit{impossible}}\textit{Impossible} (\textit{to see you and}) \textit{not to love you}. But it is difficult to apply the same test to all our quotations.

\ea\label{ex:positive}
 \ea Strangers and foes do sunder, and not kisse\hfill(\href{https://internetshakespeare.uvic.ca/doc/AWW_F1/scene/2.5/index.html#tln-1360}{Shakespeare, \textit{Alls} 2.5.91})
 \ex that glib and oylie art, To speak and purpose not\\\hfill(\href{https://internetshakespeare.uvic.ca/doc/Lr_F1/scene/index.html#tln-245}{Shakespeare, \textit{Lr} 1.1.228})
 \z
\ex \label{ex:negative}
 \ea \label{ex:negative-love} I couldn't see you, and not love you\hfill(\href{https://archive.org/details/personalhistory05dickgoog/page/n249/mode/2up?q=%22couldn%27t+see+you%22&view=theater}{Dickens, \textit{David} 570})
 \ex But he could not look at her and not be afraid of her\\\hfill(\href{https://archive.org/details/dombeyson00dick_0/page/734/mode/2up?q=%22But+he+could+not+look+at+her%22&view=theater}{Dickens, \textit{Dombey} 473}) % removed comma
 \ex I cannot love my lord and not his name\hfill(\href{https://archive.org/details/dli.bengal.10689.17599/page/n529/mode/2up?q=%22I+cannot+love+my+lord+and+not+his+name%22&view=theater}{Tennyson, \textit{Enid}})
 \ex I could not live in a house where such a thing was half conceivable, and not probe the matter home\hfill(\href{https://archive.org/details/merrymenothertal00stev/page/166/mode/2up?view=theater&q=%22i+could+not+live+in+a+house%22}{Stevenson, \textit{Merry} 179}) % Restoring "half"
 \ex \il{English!nothing@\textit{nothing}} what are we to do? {\dots} Can't bury the poor chap and say nothing about it\hfill(\href{https://archive.org/details/sowersnovel00merr/page/10/mode/2up?q=%22what+are+we+to+do%22&view=theater}{Merriman, \textit{Sowers} 13}) % Added dots
 \ex I could not live, and not be true with him\hfill(\href{https://archive.org/details/beauaustinadram00stevgoog/page/n36/mode/2up?view=theater&q=%22could+not+live+and+not+be%22}{Henley, \textit{Beau} 20}) % Comma restored
 \ex\il{English!nothing@\textit{nothing}} I must not stay here and do nothing\hfill(\href{https://archive.org/details/dli.bengal.10689.8131/page/n271/mode/2up?q=%22stay+here+and+do+nothing%22&view=theater}{Hardy, \textit{Wessex} 265})
 \ex no one can read it and not be moved\hfill(\href{https://archive.org/details/essaysinartofwri00stevuoft/page/84/mode/2up?view=theater&q=%22no+one+can+read%22}{Stevenson, \textit{Art} 84})
 \ex No one could have had such a splendid old father as I have, and not believe in them\hfill(\href{https://archive.org/details/fowler00harrgoog/page/n62/mode/2up?view=theater&q=%22no+one+could+have+had%22}{Harraden, \textit{Fowler} 54}) % OJ writes "the people" (and doesn't use brackets]; Harraden writes "them". I (PE) believe that the referent is (indefinite) old people.
 \z
\ex \label{ex:interrogative}
 \ea how then can I go back from this, and not be hanged as a traitor?\\\hfill(\href{https://archive.org/details/bunyanspilgrims00moffgoog/page/76/mode/2up?q=%22go+back+from+this%22&view=theater}{Bunyan, \textit{Progress} 68}) % Restoring "then"
 \ex Who can touch pitch, and not be defiled?\hfill(\href{https://archive.org/details/bim_eighteenth-century_sir-charles-grandison-_richardson-samuel_1780_1/page/30/mode/2up?q=%22not+be+defiled%22&view=theater}{Richardson, \textit{Grandison} 28}) % restoring comma
 \ex how Shall I descend, and perish not?\hfill(\href{https://archive.org/details/completepoeticalshel/page/408/mode/2up?view=theater&q=%22shall+I+descend%22}{Shelley, \textit{Epipsychidion}})
 \ex But oh!---what we can bear and not die!\hfill(\href{https://archive.org/details/cu31924013567130/page/304/mode/2up?q=%22and+not+die%22&view=theater}{Ward, \textit{Eleanor} 244})
 \z
\ex \label{ex:neg-interrogative} 
 \ea May not a man then trifle out an hour With a kind woman and not wrong his calling?\hfill(\href{https://archive.org/details/venicepreservdor00otwa/page/34/mode/2up?q=%22may+not+a+man+then%22&view=theater}{Otway, \textit{Venice} 3.2})
 \ex why can't you marry me, and live here with us, and not be a Methodist preacher any more?\hfill(\href{https://archive.org/details/dli.bengal.10689.8131/page/n275/mode/2up?q=%22Methodist+preacher+any+more%22&view=theater}{Hardy, \textit{Wessex} 270})
 \z
\ex \label{ex:unanalyzable} I'm doing just as little as I can and not be punished\newline (`without being punished')\hfill(\href{https://cdn.theatlantic.com/media/archives/1911/08/108-2/132299447.pdf}{M. I. Swift, \textit{Humanizing} 172})
\z\il{English!and not@\textit{and not}|)}

\is{attraction!of negative to conditional conjunction}
\is{conjunctions!conditional}\is{conjunctions!negative}
Conditional conjunctions also have a strong attraction for the negative notion in many languages (cf. Latin \il{Latin!nisi@\textit{nisi}}\textit{nisi}, Danish colloquial \il{Danish!hvis ikke@\textit{hvis ikke}}\textit{hvis ikke} (\textit{at}) \textit{han kommer} instead of \textit{hvis han ikke kommer}). Thus we have in English the negative conjunction \is{grammaticalization}\il{English!unless@\textit{unless}}\textit{unless} (formerly \textit{onles, onles that}) meaning `if {\dots} not'; \il{English!lest@\textit{lest}}\textit{lest} (Old English \il{English!Old English!thy laes the@\textit{þy læs þe}}\textit{þy læs þe}) meaning `that {\dots} not'; \il{English!for fear@\textit{for fear}}\textit{for fear} often is equivalent to `(in order) that {\dots} not'; cf. also \il{English!but@\textit{but}}\textit{but} (\textit{but that, but what}), \chapref{ch:12}; Danish \il{Danish!medmindre@\textit{medmindre}}\textit{medmindre}; French \il{French!a moins que@\textit{à moins que}}\textit{à moins que}, Spanish \il{Spanish!a menos que@\textit{á menos que}}\textit{á menos que}.
\is{attraction!of negative other than to verb|)}


\is{position of negative|)}
\is{special negation!instead of nexal negation|)}
