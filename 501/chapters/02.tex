\ChapterAndMark{Strengthening of Negatives} 
\label{ch:2}
\is{strengthening of negatives|(}

\is{grammaticalization|(}
\label{para:non}\is{adverbs!negative}There are various ways of strengthening negatives. Sometimes it seems as if the essential thing were only to increase the phonetic bulk of the adverb by an addition of no particular meaning, as when in \il{Latin!non@\textit{non}|(}Latin \textit{non} was preferred to \il{Latin!ne@\textit{ne}}\textit{ne}, \textit{non} being according to the explanation generally accepted compounded of \textit{ne} and \il{Latin!oenum@\textit{oenum}}\textit{oenum} (or % Originally "(= unum)"; "corresponding to" seemed heavy-handed
\textit{unum}) `one' (neuter).\il{Latin!non@\textit{non}|)} % PE: Not for the only time, OJ uses "neutr." This doesn't appear in any of his lists of abbreviations that I've looked at, but pretty obviously it stands for "neutral". The problem is of what this means. In MEG part 2, "neutral" pertains to the neuter gender of OE (p409), or describes a word whose meaning is determined anaphorically or deictically (pp 413-414), particularly what we'd now call a pro-form (p411), or is a description OJ says he'd like to use for number-undetermined (but refrains, and uses "common number" instead, because of the other uses of "neutral; p136)
%Brett: seems to me this is "nueter". % PE: OK, "neuter" it is. (But how bizarre as an abbreviation for this!)
\is{paucal supplements|(}\is{lexical change|(}\il{English!bit@\textit{bit}|(}\is{small, words for something very|(}But in most cases the addition serves to make the negative more impressive as being more vivid or picturesque, generally through an exaggeration, as when substantives meaning something very small are used as subjuncts.\footnote{A \textsc{substantive} is a noun as distinguished from an adjective. In languages like Finnish where no distinction pertains, the term \textsc{noun} is used. Pronouns are not substantives. A \textsc{subjunct} is a modifier of a modifier. In \textit{extremely hot weather}, \textit{weather} is a \textsc{primary}, \textit{hot} is an \textsc{adjunct}, and \textit{extremely} is a subjunct. \eds} \is{etymological meaning, loss of|(}Some English examples will show how additions of this kind are often used more or less incongruously, no regard being taken to their etymological meaning \refp{ex:02-01}. Cf. (\refp{ex:02-12}).

\ea \label{ex:02-01}
\ea
She didn't know one \textit{bit} how to speak to a gentleman\\\hfill(\href{https://archive.org/details/adambede01eli/page/242/mode/2up?q=%22She+didn%27t+know+one+bit+how+to+speak+to+a+gentleman%22&view=theater}{Eliot, \textit{Adam} 173})
\ex
I don't believe it was Peppermint's fault a \textit{bit}\\\hfill(\href{https://archive.org/details/dukeschildren03trolgoog/page/n194/mode/2up?q=%22I+don%27t+believe+it+was+Peppermint%27s+fault+a+bit%22&view=theater}{Trollope, \textit{Children} 1.189}) % OJ Has "Pepperment's" but the source linked to has "Peppermint's"
\ex
\il{English!not@\textit{not}|(}{}[the Jackal] was not a \textit{bit} impressed\hfill(\href{https://archive.org/details/secondjungleboo03kiplgoog/page/n153/mode/2up?q=%22not+a+bit+impressed%22&view=theater}{Kipling, \textit{Second} 127}) % Not "he"
\ex
it's of not a \textit{bit} of use\hfill(\href{https://archive.org/details/personalhistory05dickgoog/page/n283/mode/2up?q=%22not+a+bit+of+use%22&view=theater}{Dickens, \textit{David} 649}) % The "of" before "not" restored
\ex\il{English!bit@\textit{bit}|)}
``an accomplice hid among them, I suppose.''\\``Not a \textit{jot}.'' \hfill(\href{https://archive.org/details/cewaverleynovels03scotuoft/page/220/mode/2up?q=%22Xot+a+jot%2C%22&view=theater}{Scott, \textit{Antiquary} 2.17})
\ex
\il{English!jot@\textit{jot}}Never got a \textit{sniff} of any ticket\hfill(\href{https://archive.org/details/stalkyandco015455mbp/page/n55/mode/2up?q=%22sniff+of+any+ticket%22&view=theater}{Kipling, \textit{Stalky} 58})
\ex
\il{English!scrap@\textit{scrap}}``Am I not to care at all?'' --- ``Not a \textit{scrap}'' \hfill(\href{https://archive.org/details/devilsdisciplea00shawgoog/page/n58/mode/2up?q=%22Am+I+not+to+care+at+all%3F%22%22Not+a+scrap.%22&view=theater}{Shaw, \textit{Disciple} 3})
\ex
``Were you tired?'' % OJ's unattributed original: ``Were you tired?'' 
\il{English!jot@\textit{jot}}--- ``Not a \textit{scrap}'' %\hfill (\href{https://archive.org/details/forsakeofschool0000ange/page/130/mode/2up?q=%22aren%27t+you+tired%22&view=theater}{Brazil, \textit{School}})% OJ doesn't give the source. I (PE) can't find any source for that. (Angela Brazil's novel For the Sake of the School (first published in 1915, according to https://en.wikipedia.org/wiki/Angela_Brazil ) has "Cuckoo aren't you tired?" -- "Not a scrap" https://archive.org/details/forsakeofschool0000ange/page/130/mode/2up?q=%22aren%27t+you+tired%22&view=theater . Shall we use this? A problem is that I can't imagine that any philologist of the time would have read this book ... though they might have caught sight of this nugget of dialogue when looking over a daughter's shoulder.
%Brett: I see no harm in it. Let's. 
% Peter: ??? I now have cold feet about adding it. (Incidentally, I know nothing about OJ's family, if he even had one.) But ... done.
\ex
he does not care a \textit{snap of his} strong \textit{fingers} whether he ever sees me again\hfill(\href{https://archive.org/details/24180134.2382.emory.edu/page/n99/mode/2up?q=%22a+snap+of+his+strong+fingers%22&view=theater}{Philips, \textit{Glass} 93})
\ex
\il{English!toss@\textit{toss}}he doesn't care a \textit{toss} about all that\hfill(\href{https://archive.org/details/cu31924013342781/page/n47/mode/2up?q=%22toss+about+all+that%22&view=theater}{Doyle, \textit{Letters} 29})
\ex
\il{English!tinker's damn/curse@\textit{tinker's damn}/\textit{curse}|(}the real world doesn't care a \textit{tinker's}---doesn't care a \textit{bit}\\\hfill(\href{https://archive.org/details/lightthatfailed01kipluoft/page/134/mode/2up?q=%22the+real+world+doesn%27t+care+a+tinker%27s%E2%80%94doesn%27t+care+a+bit%22&view=theater}{Kipling, \textit{Light} 112})
\z
\z

\ea \label{ex:02-12}
\ea
not worth a \textit{tinker's damn}/\textit{curse}, or \textit{curse}\hfill(\href{https://archive.org/details/slangitsanalogue07farmuoft/page/130/mode/2up?q=%22not+worth+a+tinker%E2%80%99s+damn%2C+or+curse%22&view=theater}{Farmer \& Henley})

\ex
Who now cares a \textit{tinker's curse} for Cheops? \hfill(\href{https://archive.org/details/sim_fortnightly_1917-02-01_101_602/page/328/mode/2up?q=%22curse+for+Cheops%22&view=theater}{Lawrence, \textit{Abolition} 328})
\il{English!tinker's damn/curse@\textit{tinker's damn}/\textit{curse}|)}

\ex
\il{English!don't give a blank@\textit{don't give a blank}}I don't give a \textit{blank} what you think\hfill(\href{https://archive.org/details/johnmarvelassistant00pageiala/page/252/mode/2up?q=%22I+don%27t+give+a+blank+what+you+think%22&view=theater}{T. N. Page, \textit{Marvel} 491})
\z
\z\is{small, words for something very|)}

\emergencystretch=3em % Adjust the value as needed
Collections of similar expressions have been made by \citet{hein1893ueber} and \citet{willert1900ober}. The term ``bildliche Verneinung'' \is{figurative negation}(`figurative negation'), by the way, does not seem a very happy one for these combinations, as it is not the negation itself that is expressed figuratively; the term would be more suitably applied to some of the instances I have collected below (p.~\pageref{sec:indirect-negation}) under the heading of ``Indirect negation''.

\is{cat@`cat', ≈ `nobody'}\il{English!cat@\textit{cat}|(}There  is a curious use of the word \textit{cat} in this connexion which is paralleled in Danish \refp{ex:02-15} (i.e. `nobody') in \refp{ex:02-16}, cf. the old \refp{ex:02-17}.

\il{Danish!ikke@\textit{ikke}}\il{Danish!kat@\textit{kat}}\ea \label{ex:02-15}
\gll der er ikke en kat der veed det\\
 there is not a cat that knows it\\
\glt `Nobody knows it at all.'
\z

\ea \label{ex:02-16}
there is not a cat he knows\hfill(\href{https://archive.org/details/asinalookinggla00philgoog/page/n288/mode/2up?q=%22there+is+not+a+cat+he+knows%22&view=theater}{Philips, \textit{Glass} 285})
\z

\ea \label{ex:02-17}
it shold not auaylle me a cattes tayl \hfill(\href{https://archive.org/details/TheHistoryOfReynardTheFoxArber/page/n79/mode/2up?q=%22not+auaylle+me%22&view=theater}{Caxton, \textit{Reynard} 50})
\z\il{English!not@\textit{not}|)}\il{English!cat@\textit{cat}|)}

\is{small, words for something very|(}To the same order belong, of course, the well-known French words already alluded to, \il{French!mie@\textit{mie}}\textit{mie} (obsolete), \il{French!goutte@\textit{goutte}}\textit{goutte}, \il{French!pas@\textit{pas}|(}\textit{pas}, \il{French!point@\textit{point}}\textit{point}. Originally \textit{pas} could only be used with a verb of motion, etc., but the etymological meaning of all these words was soon forgotten, and they came to be used with all kinds of verbs.\il{French!pas@\textit{pas}|)}\is{lexical change|)}

\is{lexical change}
\il{English!trace@\textit{trace}|(}
\is{trace@`trace', expressions meaning|(}
Similar supplements to negatives are frequent in all languages; I have noted, for instance, the Italian \refp{ex:02-18}. In Danish \il{Danish!spor@\textit{spor}|(}\textit{spor} (`trace') is the most usual addition: \refp{ex:02-19} etc., followed by partitive \textit{af} not only before substantives, as in \refp{ex:02-20}, but also before adjectives and verbs \refp{ex:02-21}. One may even hear \refp{ex:02-23}, where \textit{af} has no object. Another frequent combination is \il{Danish!ikke@\textit{ikke}|(}\textit{ikke skygge} (`not a shade').

\ea \label{ex:02-18}
\il{Italian!\textit{un fico secco}}\gll Non mi batterò un fico secco!\\
 not myself {will fight} a fig dry\\
\glt `I won't give a damn!'
\hfill(\href{https://archive.org/details/bub_gb_JUcxlkCjgNgC/page/n85/mode/2up?q=%22non+mi+batter%C3%B2+un+fico+secco%22&view=theater}{Bersezio, \textit{Bolla} 3.14})

\z

\ea \label{ex:02-19}
\gll han læser \textit{ikke} \textit{spor}\\
 he reads not trace\\
\glt `he does not read anything at all'


\z

\ea \label{ex:02-20}
\gll der var \textit{ikke} \textit{spor} \textit{af} aviser\\
 there was not trace of newspapers.\DEF{}\\
\glt `there was not a trace of the newspapers'

\z

\ea \label{ex:02-21}
\ea
\gll han er \textit{ikke} \textit{spor} \textit{af} \textit{bange}\\
 he is not trace of afraid\\
\glt `he is not at all afraid'

\ex 
\gll Han skulde \textit{ikke} \textit{fare} \textit{op}, \textit{ikke} \textit{spor} \textit{af} \textit{fare} \textit{op}\\
 he should not rush up not trace of rush up\\
\glt `He should not lose his temper, not lose his temper at all'\\ \hfill(\href{https://www.bokselskap.no/boker/lucie/xviii}{Skram, \textit{Lucie} 187})
\z
\z

\ea \label{ex:02-23}
 \gll Det forstår jeg mig \textit{ikke} \textit{spor} \textit{af} på\\
 that understand I me not trace of on\\
 \glt`I do not understand that at all'
\z\il{Danish!ikke@\textit{ikke}|)}\il{Danish!spor@\textit{spor}|)}\is{paucal supplements|)}
\is{small, words for something very|)}
\is{trace@`trace', expressions meaning|)}
\il{English!trace@\textit{trace}|)}

\label{para:inte}\is{nothing@`nothing', ≈ `not'}We must here also mention the extremely frequent instances in which \is{pronouns, negative}words meaning `nothing' come to mean simply \il{English!not@\textit{not}|(}`not'; these, of course, are closely related to \is{not a bit@`not a bit', expressions corresponding to}\textit{not a bit}, etc., meaning `not'. Thus Latin \il{Latin!nihil@\textit{nihil}|}\textit{nihil} (cf. also \textit{non}, see \hyperref[para:non]{above}), Greek \il{Greek!\textit{oudèn}}\textit{oudèn}, which has become the usual Modern Greek word for `not' \il{Greek!\textit{dèn}}\textit{dèn} (pronounced [ðen]), % PE: I've put this in [ ]: OJ doesn't, but it's his usual practice
English \textit{not}\il{English!not@\textit{not}|)} from \textit{nought}, \textit{nawiht}, \il{German!nicht@\textit{nicht}}German \textit{nicht} (cf. Old Norse \il{Old Norse!vaettki@\textit{vættki}}\textit{vættki}); further Old Norse \il{Old Norse!ekki@\textit{ekki}}\textit{ekki} from \textit{eittki}, Danish \il{Danish!ikke@\textit{ikke}}\textit{ikke}, Swedish \textit{icke}; also Danish and Swedish \il{Danish!inte@\textit{inte}}\textit{inte}, in Danish now obsolete in educated speech, though very frequent within living memory even in the highest classes; in dialects it survives in many forms, \textit{it}, \textit{et}, \textit{int}, etc. The expanded form \il{Danish!intet@\textit{intet}}\textit{intet} is still in use as the pronoun `nothing', chiefly however in literary style. Cf. adverbial \textit{none} in \citet[\href{https://archive.org/details/jespersen-1954-a-modern-english-grammar-on-historical-principles-part-ii-syntax-first-volume/page/424/mode/2up?view=theater}{16.69}]{jespersenMEG2}.\is{lexical change}

\is{pronominal adverbs, negative}Where the word for `nothing' becomes usual in the sense `not', a new word is frequently formed for the pronoun: thus (probably) Latin \il{Latin!nihil@\textit{nihil}|(}\textit{nihil}, when \il{Latin!non@\textit{non}}\textit{non} was degraded, English \textit{nothing} (besides \textit{nought}, the fuller form of \il{English!not@\textit{not}}\textit{not}), \il{Danish!ingenting@\textit{ingenting}}Danish \textit{ingenting}, German \il{German!nichts@\textit{nichts}}\textit{nichts}. But in its turn, the new word may be used as a subjunct meaning `not', thus \textit{nihil}\il{Latin!nihil@\textit{nihil}|)} (above), % PE: Referring to the previous paragraph, so requires no link
English \textit{nothing} as in \textit{nothing loth}, etc., see the full treatment in \citet[\href{https://archive.org/details/jespersen-1954-a-modern-english-grammar-on-historical-principles-part-ii-syntax-first-volume/page/424/mode/2up?view=theater}{17.36ff}]{jespersenMEG2}. 

\is{auxiliary verbs|(}
\is{temporal expressions in negation|(}\is{adverbs!negative|(}\is{never@`never', words meaning|(}
Another way of strengthening the negative is by using some word meaning `never' without its temporal signification. This is the case with Old English \il{English!Old English!na@\textit{nā}|(}\textit{nā} (\textit{ne}~+~\textit{ā}, corresponding to \il{Gothic!\textit{ni aiws}}Gothic \textit{ni aiws}, German \il{German!nie@\textit{nie}}\textit{nie}); this \il{English!Old English!na@\textit{nā}}\textit{nā} was very frequent in Old English and later as a rival of \textit{not}, and has prevailed in Scotch and the \il{English!dialectal!northern England}northern dialects, where it is attached to auxiliaries in the same way as -\textit{n't} in the South: \textit{canna}, \textit{dinna}, etc. In Standard English its rôle is more restricted; besides being used as a sentence-word in \is{answers, negative}answers it is found in combinations like \il{English!no@\textit{no}|(}\textit{whether or no}, \textit{no better}, \textit{no more}, see \citet[\href{https://archive.org/details/jespersen-1954-a-modern-english-grammar-on-historical-principles-part-ii-syntax-first-volume/page/424/mode/2up?view=theater}{16.8}]{jespersenMEG2}; sometimes it may be doubtful whether we have this original \is{pronominal adverbs, negative}adverb or the \is{adjective-pronouns, negative}pronominal adjective \textit{no}\il{English!no@\textit{no}|)} from Old English \il{English!Old English!nan@\textit{nān}}\textit{nān}, \textit{ne} + \textit{ān}, see also \href{https://archive.org/details/jespersen-1954-a-modern-english-grammar-on-historical-principles-part-ii-syntax-first-volume/page/424/mode/2up?view=theater}{ibid 16.7}.
% OJ's dash here is converted into a paragraph break.

The corresponding Old Norse \textit{nei} has given English \textit{nay} (on which see p.~\pageref{ch10-nay} below); another Old Norse compound of the same \textit{ei} is \il{Old Norse!eigi@\textit{eigi}}\textit{eigi}, which gradually loses its temporal signification and becomes the ordinary word for `not', see \citet[\href{https://archive.org/details/germanische-syntax-i-zu-den-negativen-s/page/40/mode/2up?q=eigi&view=theater}{40ff}]{delbruck_negativen_1910}
% PE: Here we should really say what by Delbrück, and where.
%Brett: done
and \citet[15ff]{neckel1912germanischen}.\is{never@`never', words meaning|)}\is{lexical change}
\il{English!Old English!na@\textit{nā}|)}
\is{auxiliary verbs|)}

\is{never@`never', ≈ `not'|(}\il{English!never@\textit{never}|(}English \textit{never} also in some connexions comes to mean merely `not' \refp{ex:02-24}. A transitional case is \refp{ex:02-26}.\is{lexical change}

\ea \label{ex:02-24}
\ea I never knew it was so chilly \phantom{x} (`didn't know')\hfill(\href{https://archive.org/details/lightthatfailed01kipluoft/page/130/mode/2up?q=%22I+never+knew+it+was+so+chilly%22&view=theater}{Kipling, \textit{Light} 109})
\ex He {\dots} knew that for a moment Brown never moved\hfill(\href{https://archive.org/details/softside00jamerich/page/n11/mode/2up?q=%22knew+that+for+a+minute+Brown+never+moved%22%22&view=theater}{James, \textit{Side} 6}) % Marking an omission by OJ
\z
\z

\ea \label{ex:02-26}
never once looking over his shoulder\hfill(\href{https://archive.org/details/dombeyson00dick_0/page/124/mode/2up?q=%22never+once+looking+over+his+shoulder%22&view=theater}{Dickens, \textit{Dombey} 76})
\z

\textit{Never} in this sense is especially frequent before \textit{the} (Old English \textit{þȳ})\footnote{\textit{Þȳ} % If we ever need to revert: ȳ (y with macron) can also be produced via $\bar{y}$
is the instrumental case form of \textit{sē} or \textit{þæt} `the'. \eds} with a comparative (as in \textit{nevertheless}) \refp{ex:02-27}, and in the combination \textit{never a} (`no'), which has become a kind of compound \is{pronominal adverbs, negative}(adjunct) \is{pronouns, negative}pronoun, used to a great extent in some dialects (see \citet[\href{https://archive.org/details/englishdialectdi04wrig/page/256/mode/2up?view=theater}{\textit{Never~a}}]{wright1905english4}), and very frequent in colloquial English, especially in the phrase \textit{never a word} \refp{ex:02-27a}.

\ea \label{ex:02-27}
then we be neuer the nearer\hfill(\href{https://archive.org/details/gammergvrtonsned0000mrsm/page/44/mode/2up?q=%22then+we+be+neuer+the+nearer%22&view=theater}{\textit{Gammer} 134})
\z

\ea \label{ex:02-27a}
\ea
it nedeth never-a-del\hfill(\href{https://archive.org/details/completeworksofg04chauuoft/completeworksofg04chauuoft/page/310/mode/2up?q=%22it+nedeth%22&view=theater}{Chaucer, \textit{Pardoners} C~670}) % Orthography adjusted to follow Skeat
\ex
to neuer a penny coste\hfill(\href{https://archive.org/details/utopiasirthomas00robigoog/page/n375/mode/2up?q=%22penny+coste%22&view=theater}{More, \textit{Utopia} 264})
\ex
he would {\dots} leaue you never a hen on-liue\hfill(\href{https://archive.org/details/gammergvrtonsned0000mrsm/page/46/mode/2up?q=%22And+leue+you+neuer+a+hen+on+liue%22%22&view=theater}{\textit{Gammer} 136})
\ex
Canst thou tell nere a one\hfill(\href{https://archive.org/details/representativee02unkngoog/page/482/mode/2up?q=%22Canst+thou+tell+nere+a+one%22&view=theater}{\textit{Eastward} 482})
\ex
\il{English!not@\textit{not}}thou canst not tell ne're a word on't\hfill(\href{https://babel.hathitrust.org/cgi/pt?id=uiuo.ark:/13960/t1pg73v3n&seq=137&q1=thou+canst+not+tell}{Marlowe, \textit{Faustus} 759})
\ex
you [\href{https://internetshakespeare.uvic.ca/doc/1H4_Q1/complete/index.html#tln-650}{quarto}: they] will allow vs ne're a iourden\\\hfill(\href{https://internetshakespeare.uvic.ca/doc/1H4_F1/scene/2.1/index.html#tln-650}{Shakespeare,~\textit{H4A} 2.1.21};\\\hfill note the difference from: \textit{they will never allow us a jourden}.) % Now follows the first folio more precisely
\ex
neuer a mans thought in the world, keepes the rode-way better then thine\hfill(\href{https://internetshakespeare.uvic.ca/doc/2H4_F1/scene/2.2/index.html#tln-835}{Shakespeare, \textit{H4B} 2.2.62}) % Now follows the first folio more precisely
\ex
the man answered never a word\hfill(\href{https://archive.org/details/pilgrimsprogress03bunyanj/page/144/mode/2up?q=%22the+man+answered+never+a+word%22&view=theater}{Bunyan, \textit{Progress} 232})
\ex
he bit his lip, and looked at her, and said never a word\\\hfill(\href{https://archive.org/details/ourmutualfriendc0000char/page/330/mode/2up?q=%22bit+his+lip%22&view=theater}{Dickens, \textit{Friend} 445}) % "and looked at her" reinstated
\ex
when you're married, and have got a three-legged stool to sit on, and never a blanket to cover you\hfill(\href{https://archive.org/details/adambede01eli/page/134/mode/2up?q=%22and+never+a+blanket+to+cover+you%22&view=theater}{Eliot, \textit{Adam} 62})
\ex
he answered never a word\hfill(\href{https://archive.org/details/dli.ministry.23383/page/39/mode/2up?q=%22answered+never+a+word%22&view=theater}{Stevenson, \textit{Jekyll} 39})
\ex
but never a word did Dick say of Maisie\hfill(\href{https://archive.org/details/lightthatfailed02kipluoft/page/94/mode/2up?q=%22but+never+a+word+did+Dick+say+of+Maisie%22&view=theater}{Kipling, \textit{Light} 218}) % RK writes "of Maisie or marriage"
\ex
but never a beast came to the shrine\hfill(\href{https://archive.org/details/secondjungleboo03kiplgoog/page/n65/mode/2up?q=%22beast+came+to+the+shrine%22&view=theater}{Kipling, \textit{Second} 53})
\ex
Blank slopes on either side, with never a sign of a decent beast\\ \hfill(\href{https://archive.org/details/twelvestoriesan00wellgoog/page/n74/mode/2up?q=%22with+never+a+sign+of+a+decent+beast%22&view=theater}{Wells, \textit{Stories} 21})
\z
\z


\il{Danish!aldrig@\textit{aldrig}|(}A Danish parallel is \refp{ex:02-42}.

\ea \label{ex:02-42}
\gll Jeg seer aldrig en smuk plet paa denne Helene.\\
 I see never a beautiful spot on this Helene\\
\glt `I don't see any beauty at all in this Helene.'
\hfill(\href{https://tekster.kb.dk/text/adl-texts-holb04val-shoot-workid100866}{Holberg, \textit{Ulysses} 1.7}) % ??? PE: Whether or not a comma should be inserted ("... this, Helene"), this is pragmatically bizarre. Let's find a Danish speaker we can ask.

%% SG: no comma here, the speaker is not talking to Helene, but about her. And the example is pragmatically perfectly fine in the context -- it means that the speaker doesn't see any beauty at all ("not even a beautiful spot") in this Helene person that the Greeks are going to war for // %% PE: I sit corrected!
\z

\textit{Never} is also used in surprised exclamations like \refp{ex:02-44}. In the same way in Danish \refp{ex:02-45}.\is{lexical change}

\ea \label{ex:02-44}
Why, it's never Bella!\hfill(\href{https://archive.org/details/ourmutualfriendc0000char/page/506/mode/2up?q=%22never+bella%22&view=theater}{Dickens, \textit{Friend} 680})
\ex Why, it's never No. 406!\hfill(\href{https://archive.org/details/fannysfirstplaye00shawrich/page/n55/mode/2up?q=%22Why%2C+it%27s+never+No.+406%21%22&view=theater}{Shaw, \textit{First} 203})
% OJ attributes this to "Shaw M.", i.e. G B Shaw's Man and Superman. It doesn't appear there. It does appear in Fanny's First Play
\z

\ea \label{ex:02-45}
\gll \textit{det} \textit{er} \textit{da} \textit{vel} \textit{aldrig} \textit{Bella!}\\
 that is then surely never Bella\\
\glt `Why, if that isn't Bella!'

\z\il{English!never@\textit{never}|)}
%% SG: the glosses in this ex are a bit tricky -- _da_ and _vel_ here are discourse particles without a good English translation; I suppose 'surely' is alright for _vel_, but I really wouldn't know how to translate _da_. _then_ is perhaps as good as it gets if you want to avoid making an additional grammatical abbreviation just for this examples \\ %% PE: Perhaps "da vel" isn't necessary for the point that OJ is making; and if it isn't, we can skip an explanatory footnote, especially as an explanatory footnote might have to be rather long.

Danish \textit{aldrig} also means `not' in the combination \textit{aldrig så snart} (`no sooner') as in \refp{ex:02-46}.

\ea \label{ex:02-46}
\gll \textit{Men} \textit{aldrig} \textit{saasnart} \textit{var} \textit{seiren} \textit{vunden,} \textit{før} \textit{den} \textit{hos} \textit{den} \textit{Seirende} \textit{vakte} \textit{den} \textit{dybeste} \textit{anger}\\
 but never {so soon} was victory.\DEF{} won before it at the victorious evoked the deepest regret\\
\glt `But no sooner was the victory won, than it caused the deepest regret in the victor'
\hfill(\href{https://tekster.kb.dk/text/adl-texts-goldschmidt03-root}{Goldschmidt, \textit{Hjemløs} 1.105}) % PE I changed "victorious one" to "victor"
\z\il{Danish!aldrig@\textit{aldrig}|)}
\is{never@`never', ≈ `not'|)}
\is{temporal expressions in negation|)}

\is{not at all@`not at all', expressions corresponding to|(}The frequent adverbial strengthenings of negatives as in \il{English!not at all@\textit{not at all}}\textit{not at all}, \il{French!pas du tout@\textit{pas du tout}}\textit{pas du tout}, \il{Danish!ikke@\textit{ikke}|(}\il{Danish!aldeles ikke@\textit{aldeles ikke}}\textit{aldeles ikke}, \il{Danish!slet ikke@\textit{slet ikke}}\textit{slet ikke}\il{Danish!ikke@\textit{ikke}|)}, \il{German!nicht@\textit{nicht}|(}\il{German!durchaus nicht@\textit{durchaus nicht}}\textit{durchaus nicht}, \il{German!gar nicht@\textit{gar nicht}}\textit{gar nicht}\il{German!nicht@\textit{nicht}|)}, etc., call for no remark here. It should be mentioned, however, that \textit{by no means} and corresponding expressions in other languages are very often used without any reference to what might really be called `means', in the same way as in the instances just referred to there is no reference to the time-element of `never'. In colloquial Danish one may sometimes hear sentences like \refp{ex:02-47} for `not the least'.

\ea \label{ex:02-47}
\il{Danish!ikke@\textit{ikke}}\gll Jeg synes, at brevet var \textit{ikke} \textit{ud} \textit{af} \textit{stedet} tørt\\
 I think that letter.\DEF{} was not out of place.\DEF{} dry\\
\glt `I think that the letter was not the least uninteresting'
\z
\is{not at all@`not at all', expressions corresponding to|)}
\is{adverbs!negative|)}
\is{etymological meaning, loss of|)}

\is{pronouns, negative}On the flux and reflux in Greek \il{Greek!\textit{oudeís}}\textit{oudeís} (`no one'), strengthened into \il{Greek!\textit{oude heîs}@\textit{oudè heîs}}\textit{oudè heîs} (`not even one'), soldered into \textit{outh'heîs}, which was weakened into \textit{outheís}, and replaced in its turn by \textit{oudeís}, see the interesting account in \citet[\href{https://archive.org/details/aperudunehisto00meil/page/198/mode/2up?q=n\%C3\%A9gation&view=theater}{290f}]{meillet1920aperu}. %  We show OJ, in 1917, citing a source published in 1920, three years in the future.
%Brett: shall we add [1st ed 1913]? % PE: Please do.
%Brett: I've sought Stefan's guidance, but he's away for two weeks.

On strengthening through repeated negation see \chapref{ch:7}.\is{lexical change}
\is{strengthening of negatives|)}
