\addchap{\lsPrefaceTitle} \label{ch:preface}
 
 The nucleus of the following disquisition is the material collected during many years for the chapter on Negatives in volume III or IV of my \textit{Modern English Grammar},
 % Removed "(abbreviated MEG)", as all instances are now spelled out 
 of which the first two volumes appeared in 1909 and 1914 respectively (Winter, Heidelberg).\footnote{It would be published in 1940 as ``Negation'', chapter 23 of volume~V. % PE: I've unenthusiastically expanded this from "It would be published in 1940 as chapter 23 of volume V". I wish OJ hadn't numbered both the "volumes" of "Syntax" and the "parts" of the entire work; but he did just that.
 % Second thoughts: I've reverted this from "chapter 23 of part~V (\textit{Syntax}, fourth volume)", which is correct but rather pointlessly confusing.
\eds} But as the war has prevented me (provisionally, I hope) from printing the continuation of my book, I have thought fit to enlarge the scope of this paper by including remarks on other languages so as to deal with the question of Negation in general as expressed in language. Though I am painfully conscious of the inadequacy of my studies, it is my hope that the following pages may be of some interest to the student of linguistic history, and that even a few of my paragraphs may be of some use to the logician. My work in some respects continues what \citet[\href{https://archive.org/details/grundrissderver01delbgoog/page/518/mode/2up?view=theater}{519ff}]{delbruck1897vergleichende} has written on negation in Indo-European languages, but while he was more interested in tracing things back to the ``ursprache'', I have taken more interest in recent developments and in questions of general psychology and logic.
 
With regard to the older stages of Teutonic or Germanic languages, I have learned much from \citet{delbruck_negativen_1910}, supplemented by \citet{neckel1912germanischen}. Of much less value are the treatments of the specially Old English negatives in \citet{knork1907negation} and \citet{rauert1910negation} as well as \citet{einenkel1911englische}. As in my \textit{Grammar}, my chief interest is in Modern English; a great many interesting problems can be best treated in connexion with a language that is accessible to us in everyday conversation as well as in an all\hyp comprehensive literature. Besides, much of what follows will be proof positive that the English language has not stagnated in the modern period, as \citeauthor{einenkel1911englische} would have us believe (\citeyear[234]{einenkel1911englische}, ``Bei Caxton ist der heutige Zustand bereits erreicht''). Further literature on the subject will be quoted below; here I shall mention only the suggestive remarks in \citet[199ff]{vanginneken1907principes}.
