\chapter{Parametermanipulationen}
\label{chap:04}

\section{Parameter im Task-Dynamic-Modell}
\label{sec:0401}

Im Task-Dynamic-Modell \citep{Browman1986} werden die Parameter Masse $m$ und Dämpfung $b$ für die Systemobjekte (\isi{Traktvariablen}) konstant definiert (vgl. Kapitel~\ref{sec:0101} und~\ref{sec:0102}). \isi{Steifheit} $k$ und \isi{Target} ${x}_{0}$ hingegen sind variabel, und auch die \isi{Phasenbeziehung} zwischen zwei Gesten (Koordination von dynamischen Zuständen zwischen Gesten, intergestural phasing) ist spezifizierbar. Somit bilden \isi{Steifheit}, \isi{Target} und Phase die Schlüsselparameter des Systems, mit deren Hilfe der Einfluss der prosodischen Struktur auf die phonetische Realisierung von \isi{Prominenz} und Position in ihren wesentlichen Grundzügen abbildbar sein sollte \citep{Kelso1987, Kelso1985, Munhall1985, Beckman1992, Hawkins1992, Harrington1995, Byrd2000a, Cho2002a, Cho2006}; für einen kritischen Überblick vgl. \citet{Fuchs2011}.

Abbildung~\ref{figure:0401} gibt ein Beispiel für relevante Landmarken im kinematischen Signal. Bei einer Verschlussbewegung sind \isi{Onset} (Anfangspunkt der Bewegung) und \isi{Target} (Zielpunkt der Bewegung) bestimmbar. An beiden Extrempositionen -- \isi{Onset} und \isi{Target} -- ist die Geschwindigkeit der Bewegung \emph{NULL}. Bewegt sich ein \isi{Artikulator} auf sein Ziel zu, so beschleunigt er zunächst, um die Höchstgeschwindigkeit (peak velocity, \isi{pVel}) für die Ausführung dieser Aufgabe zu erreichen. Die Höchstgeschwindigkeit stellt eine weitere relevante Landmarke im Signal dar. Nach Erreichen der Höchstgeschwindigkeit bremst die Bewegung vor dem Ziel wieder ab. Die Zeit zwischen \isi{Onset} und \isi{pVel} stellt die \isi{Beschleunigungsphase} (acceleration phase,  ${\Delta} _{{time}2{peak}}$) dar, und die Zeit zwischen \isi{pVel} und \isi{Target} die Abbremsphase (deceleration phase). Das Intervall zwischen \isi{Onset} und \isi{Target} ist  ${\Delta} _{{Dauer}}$ der Gesamtbewegung. Der Weg, den der \isi{Artikulator} von seiner Ausgangsposition zur \isi{Zielposition} zurücklegt, ist als  ${\Delta} _{{Amplitude}}$ der Bewegung (displacement) gekennzeichnet. 

\begin{figure}
	\includegraphics[width=\textwidth]{figures/4-1_Landmarken.png}
	\caption{Landmarken und Messintervalle im kinematischen Signal.}
	\label{figure:0401}
\end{figure}

Bei den vier gängigen Strategien zur Parametermanipulation im dynamischen System handelt es sich um die Modifikationen (a) \isi{Target}, (b) \isi{Steifheit}, (c) \isi{Reskalierung} als Kombination aus \isi{Target} und \isi{Steifheit} und (d) Gestischer Überlappungsgrad. Diese Strategien sind im Folgenden unter Berücksichtigung ihrer kinematischen Konsequenzen nach \citet{Beckman1992} und \citet{Cho2002a} in den Abbildungen~\ref{figure:0402}, \ref{figure:0403}, \ref{figure:0404} und \ref{figure:0405} schematisiert.

\begin{figure}[p]
	\includegraphics[width=\textwidth]{figures/4-2_Target.png}
	\caption{Manipulation des Targets in Anlehnung an \citet[71]{Beckman1992} und \citet[17]{Cho2002a}.}
	\label{figure:0402}
\end{figure}


\begin{figure}[p]
	\includegraphics[width=\textwidth]{figures/4-3_Steifheit.png}
	\caption{Manipulation der Steifheit in Anlehnung an \citet[71]{Beckman1992} und \citet[17]{Cho2002a}.}
	\label{figure:0403}
\end{figure}

\begin{figure}[p]
	\includegraphics[width=.8\textwidth]{figures/4-4_Reskalieren.png}
	\caption{Manipulation von Steifheit und Target als Reskalierung der Gesamtbewegung in Anlehnung an \citet[17]{Cho2002a}.}
	\label{figure:0404}
\end{figure}

\begin{figure}[p]
	\includegraphics[width=.8\textwidth]{figures/4-5_Trunkieren.png}
	\caption{Manipulation der Phase, die zur Trunkierung der Bewegung führt. Schema in Anlehnung an \citet[71]{Beckman1992} und \citet[17]{Cho2002a}.}
	\label{figure:0405}
\end{figure}


\begin{description}
	\item[\isi{Target}:] Das \isi{Target} ${x}_{0}$ ist ein räumlicher Parameter, der der \isi{Gleichgewichtslage der Feder} entspricht. Ist die Gleichgewichtslage erreicht, kommt die Feder zur Ruhe. Eine Veränderung des Targets stellt eine räumliche Modifikation dar. Im kinematischen Signal führt sie zu einer Veränderung von  ${\Delta} _{{Amplitude}}$ (\isi{Displacement}). Der Wert der ${\Delta}$ Amplitude im physikalischen Signal reflektiert im Feder-Masse-Modell die Differenz zwischen momentaner Objektposition und neuem \isi{Target}  $(x-x_{0})$. Targetmodifikationen lassen sich, sofern es das Messverfahren zulässt, auch über das Intervall zwischen Endpunkt der Bewegung (\isi{Target}) und Gaumenkontur des Sprechers ermitteln. Bei einem größeren \isi{Target} (oder einem Target-Overshoot) steigt  ${\Delta} _{{Amplitude}}$. Proportional zu  ${\Delta} _{{Amplitude}}$ steigt die Höchstgeschwindigkeit (\isi{pVel}), während  ${\Delta} _{{Dauer}}$ der Bewegung unverändert bleibt. Das Zeitintervall für die \isi{Beschleunigungsphase}  $(\Delta_{{time2peak}})$ und auch die mit ihr assoziierte \isi{Steifheit} der \isi{Geste} bleiben bei einer reinen Modifikation des Targets unverändert.
\end{description}

\begin{description}
	\item[\isi{Steifheit}:] Die \isi{Steifheit} $k$ ist ein sehr abstrakter Kontrollparameter, der eine Veränderung der Oszillationsfrequenz bewirkt. Im Rahmen des Feder-Masse-Modells bezieht sie sich direkt auf die Federsteifheit (vgl. Kapitel~\ref{chap:01}). Sie hat aber kein direktes Korrelat in der physikalischen Welt, d.\,h. sie muss aus der Messung abgeleitet werden. Ein Absenken der \isi{Steifheit} führt zum einen zu einer Verlangsamung der Bewegung und zum anderen beeinflusst es das Verhältnis zwischen Höchstgeschwindigkeit (\isi{pVel}) und  ${\Delta} _{{Amplitude}}$. Die Gesamtdauer der Bewegung  $\left({\Delta} _{{Dauer}}\right)$ und die Dauer der \isi{Beschleunigungsphase}  $\left({\Delta} _{{time}2{peak}}\right)$ werden größer, während  ${\Delta} _{{Target}}$ unverändert bleibt. Wird jedoch die \isi{Steifheit} herabgesetzt, aber die Dauer des gesturalen Aktivierungsintervalls  $\left({\Delta} _{{Dauer}}\right)$ beibehalten, so kann es zu einem \isi{Target} undershoot kommen. \citet{Beckman1992} geben beispielsweise geringere \isi{Steifheit} als relevanten Kontrollparameter für die Unterscheidung von akzentuierten und nicht-akzentuierten Silben im Englischen an. Es gibt unterschiedliche Messungen im physikalischen Signal, die mit \isi{Steifheit} assoziiert sind. Nach \citet{Munhall1985} lässt sich \isi{Steifheit} $k$ als Verhältnis zwischen \isi{Maximalgeschwindigkeit} (\isi{pVel}) und  ${\Delta} _{{Amplitude}}$ berechnen (vgl. auch \citealt{Hawkins1992, Beckman1992, Roon2007}). Als Wert ergibt sich eine Annäherung an die \isi{Eigenfrequenz} oder Eigenperiode der \isi{Geste}.
\end{description}

\begin{equation}
\label{eq:diff03}
\text{Steifheit}(k)=\frac{{pVel}\left(\frac{{mm}}{{ms}}\right)}{{Amp}\left({mm}\right)}
\end{equation}

%Eine andere Möglichkeit, die sich in der Literatur findet, ist die \isi{Steifheit} mit Messungen der \isi{Beschleunigungsphase} $\left({\Delta}_{{time}2{peak}}\right)$ des gestischen Aktivierungsintervalls als reinen Zeitparameter gleichzusetzen.
Eine weitere Möglichkeit ist, die \isi{Steifheit} mit Messungen der \isi{Beschleunigungsphase} $\left({\Delta}_{{time}2{peak}}\right)$ des gestischen Aktivierungsintervalls als reinen Zeitparameter gleichzusetzen.
Bei einer Verlangsamung der Bewegung vergrößert sich ${\Delta}_{{time}2{peak}}$ \citep[vgl.][]{Cho2002a, Cho2006, Byrd1998}. Diese Form der Berechnung ist jedoch keine \enquote{echte} \isi{Steifheitsberechnung} im Sinne eines Feder-Masse-Modells, weil sie den räumlichen Parameter nicht mit einbezieht. Insbesondere in der prosodischen Literatur hat sie sich jedoch bewährt, weil längere Beschleunigungsphasen häufig mit einer lokalen Verlangsamung der Artikulationsbewegungen einhergehen (localized hyperarticulation, vgl. Kapitel~\ref{chap:05}).

\citet{Fuchs2011} weisen darauf hin, dass Steifheitsberechnungen aus dem physikalischen Signal problematisch sind. Die Berechnungen basieren auf der Annahme, dass \isi{Steifheit} und Dämpfung während der gesamten Aktivierungsdauer der \isi{Geste} konstant sind. Im physikalischen Signal hingegen finden sich insbesondere bei Langvokalen, Sibilanten oder Geminaten Plateaus \citep[][1068]{Fuchs2011}, die gegen diese Annahme sprechen, und sie schlussfolgern:

\begin{quote}
	(\dots{}) the relation between gesture duration and stiffness always becomes less strong when gesture duration increases. \citep[][1074]{Fuchs2011}
\end{quote}

Die Leistungsfähigkeit der traditionellen Steifheitsberechnungen \isi{Maximalgeschwindigkeit}/Amplitude \citep{Munhall1985} und Time-To-Peak-Intervall \citep{Cho2002a, Cho2006, Byrd1998} sind unterschiedlich und abhängig vom Datensatz. Insbesondere bei unsymmetrischen Gesten, d.\,h. wenn Beschleunigungs- und Abbremsphase unterschiedlich lang dauern, ist das Time-To-Peak-Intervall problematisch, weil es nur die \isi{Beschleunigungsphase} in die Messung miteinbezieht.


\begin{description}
	\item[\isi{Reskalierung}:] Die \isi{Reskalierung} (rescaling, resizing) einer Bewegung beinhaltet die Veränderungen der Parameter \isi{Target}
	und \isi{Steifheit}. Die Abbildung~\ref{figure:0404} zeigt ein Beispiel für eine lineare \isi{Reskalierung}, bei der jeweils \isi{Target} und \isi{Steifheit} proportional verändert werden. Diese Manipulation führt im physikalischen Signal zu einem proportionalen Anstieg von ${\Delta}_{{Amplitude}}$ und ${\Delta}_{{Dauer}}$. Mit dem Anstieg von ${\Delta}_{{Dauer}}$ der Gesamtbewegung steigt auch die Dauer für die \isi{Beschleunigungsphase} ${\Delta}_{{time}2{peak}}$ an. Die Höchstgeschwindigkeit \isi{pVel} bleibt jedoch unverändert \citep{Cho2002a, Cho2006, Byrd2000b}.
\end{description}


\begin{description}
	\item[Phase:] Die Manipulation der Phase beeinflusst die Koordination zwischen zwei Gesten und somit deren Überlappungsgrad. Wird beispielsweise in einer  ${C}_{1}{C}_{2}${}-Sequenz die \isi{Geste} für ${C}_{2}$ früher aktiviert, so löst diese die vorangehende Konstriktionsbewegung für ${C}_{1}$ früher ab (Ablösephase, vgl. auch \citealt{Kröger1998}). Es kommt zur \isi{Trunkierung} der Bewegung für ${C}_{1}$ durch ${C}_{2}$ (\isi{Target} undershoot). Die Parameter innerhalb der Aktivierung von ${C}_{1}$ bleiben unverändert: die \isi{Beschleunigungsphase} ${\Delta} _{{time}2{peak}}$ und die Höchstgeschwindigkeit \isi{pVel} werden nicht modifiziert. Jedoch erreicht ${C}_{1}$ durch eine verfrühte Ablösung durch ${C}_{2}$ sein \isi{Target} nicht mehr \citep{Beckman1992, Harrington1995}. Eine Ausnahme bilden Konstriktionsbewegungen, die ein Plateau beinhalten. Wird die Aktivierung einer \isi{Geste} nach dem Erreichen der Maximalposition für das Plateau durch eine andere \isi{Geste} abgelöst, so erreicht ${\Delta}_{{Amplitude}}$ ihren durch das \isi{Target} spezifizierten Maximalwert und lediglich ${\Delta}_{{Dauer}}$ des Signals sinkt \citep{Byrd2000b, Cho2002a, Cho2006}.
\end{description}
 
\clearpage  
\citet{Harrington1995} führen aus, dass die Strategien häufig nicht allein aufgrund einer kinematischen Datenbasis trennbar seien. So sind Modifikationen der Parameter Phase und Rescaling ähnlich in ihren kinematischen Konsequenzen. Sowohl bei einer \isi{Trunkierung} (Phase) als auch bei einem Shrinking (Rescaling) von Vokalen in unakzentuierter Position ergäbe sich eine Verkleinerung von ${\Delta}_{{Amplitude}}$ und ${\Delta}_{{Dauer}}$ bei gleichbleibenden Maximalgeschwindigkeiten. Unterscheidungen wären hier vermutlich nur über die \isi{Beschleunigungsphase} ${\Delta}_{{time}2{peak}}$ möglich.

\begin{quote}
	However, a major difficulty in establishing evidence for truncation, at least from jaw movement data, is in knowing what constitutes a truncated vowel. For example, although a truncated vowel is likely to be accompanied by a reduction in both duration and displacement, together with minimal changes in the peak velocity of movement, these articulatory characteristics are also compatible with making the vowel ``smaller'' by linear rescaling (analogous to looking at a movement waveform through a zoom lens, and zooming out, producing smaller durations and displacements, but maintaining the same overall shape, and therefore the same peak velocities). \citep[][307]{Harrington1995}
\end{quote}

\citet{Cho2006} zeigt in einer Studie zur Akzent- und Positionsmarkierung im Englischen, dass die durch das Task-Dynamic-Modell propagierten Parameter im kinematischen Signal kaum in reiner Form auftreten und die \isi{Enkodierung} prosodischer Struktur durch das supralaryngale System weitaus komplexer ist als angenommen:

\begin{quote}
	As was the case for Accent effect, the boundary-induced kinematic variations were not fully accounted for by any single dynamical parameter setting. (\dots{}) the results regarding movement kinematics suggest that speech mechanisms are more complex than has been assumed. \citep[][539, 545]{Cho2006}
\end{quote}

\citet{Mücke2014b} zeigen in Analysen zur supralaryngalen Fokusmarkierung im Deutschen, dass bei der Markierung der prosodischen Struktur von multiplen Parametern auszugehen ist (vgl. Kapitel~\ref{chap:06}).

\section[Modellierungen sprechmotorischer Parametervariationen]{Modellierungen sprechmotorischer Parametervariationen am Beispiel der Tiefen Hirnstimulation}
\label{sec:0402}

Im Folgenden wird ein Beispiel gegeben, wie artikulatorische Daten parametrisiert werden können. Das Beispiel stammt aus einer Studie zur Tiefen \isi{Hirnstimulation} von \citet{Mücke2014a}. Die Studie beschäftigt sich mit der Frage, wie sich unter der Tiefen \isi{Hirnstimulation} (deep brain stimulation, DBS) bei Patienten mit Essentiellem Tremor (ET) die artikulatorischen Muster verschlechtern und inwieweit diese Verschlechterungen quantifizierbar sind. Im Rahmen eines dynamischen Systems wie dem Task-Dynamic-Modell können die Störungen der \isi{Artikulation} mit einer Veränderung der Parameterspezifikationen des motorischen Systems unter \isi{Stimulation} verstanden werden, während die das kognitive System steuernden Parameter (die das System modulierenden Differenzialgleichungen) unverändert bleiben. Eine entsprechende akustische Analyse findet sich in \citet{Mücke2014a}.

\subsection{Hintergrundwissen zur Tiefen Hirnstimulation}
\label{subsec:040201}

Diese Studie ist in Zusammenarbeit mit der Abteilung Neurologie, Klinikum der Universität zu Köln entstanden (Forschungsgruppe Brain Modulation and Speech Motor Control, bestehend aus Kölner Neurologen und Phonetikern). Die ET-Patienten sind am Klinikum der Universität zu Köln neurochirurgisch behandelt worden. Sie erhielten Implantate für eine chronische Tiefe \isi{Hirnstimulation}, die Impulse an den Nucleus ventralis intermedius (VIM) abgeben. Der VIM gilt im Tremor Netzwerk als Relaisstation zwischen Zerebellum und Motorkortex und dient deshalb als klassische neuroanatomische Zielregion im Gehirn für DBS bei Essentiellem Tremor-Patienten \citep{Schnitzler2009}. VIM-DBS wird bei gegen Medikamente resistentem Tremor eingesetzt, vor allem bei ET und Parkinson (tremordominanter Typ), vgl. \citet{Benabid1996}.

Essentieller Tremor (ET) ist eine Bewegungsstörung, die sich häufig durch Haltetremor und/oder Intentionstremor der oberen Gliedmaßen wie den Händen ausdrückt, aber auch andere Körperregionen wie Kopf oder Stimme betreffen kann \citep{Deuschl2009}. Die genaue Pathophysiologie von ET wird noch in der einschlägigen Literatur debattiert \citep{Elble2013, Raethjen2012, Louis2009, Rajput2012}. ET ist vermutlich keine monosymptomatische Erkrankung \citep{Elble2013}, sondern wird häufig eng mit beispielsweise zerebellaren Symptomen in Verbindung gebracht, was vermuten lässt, dass sich hier auch Probleme bei der Koordination von sprachlichen und nicht-sprachlichen Aufgaben ergeben können.

Bei chronischer VIM-DBS werden Elektroden und ein Impulsgeber in Schädel und Brust implantiert. Durch \isi{Stimulation} (on-DBS) oder Deaktivierung (off-DBS) werden elektrische Impulse in einer variablen Stromfrequenz an das Hirnareal VIM abgegeben, vgl. die Abbildungen~\ref{figure:0407}. Während durch diese Behandlung der Tremor, der als Bewegungsstörung bei den Patienten auftritt, häufig erfolgreich unterdrückt wird, klagen viele Patienten über Nebenwirkungen in Form einer Verschlechterung der \isi{Sprechmotorik} \citep{Benabid1996, Krack2002}. Tatsächlich gilt die stimulationsinduzierte \isi{Dysarthrie} als häufiger Nebeneffekt der thalamischen/subthalamischen \isi{Stimulation} \citep{Flora2010, Krack2002} mit ernsten Folgen für Lebensqualität und soziale Zugehörigkeit, denn Patienten klagen, dass die auftretenden Sprechverschlechterungen denen von verwaschener Sprache unter Alkoholkonsum ähneln. Es sei hier angemerkt, dass \isi{Dysarthrie} mehr als ein einzelnes artikulatorisches Subsystem beeinträchtigen kann, d.\,h. Atmung, Stimmgebung und supralaryngeale Artikulationsmuster können gestört sein \citep{Victor2001, Raphael2011}.

\begin{figure}
	\begin{minipage}[b]{0.4\textwidth}
		\includegraphics[width=\textwidth]{figures/a05-img6.png}
	\end{minipage}
	\hfill
	\begin{minipage}[b]{0.4\textwidth}
		\includegraphics[width=\textwidth]{figures/a05-img7.png}
	\end{minipage}
	\caption{Röntgenaufnahmen der DBS Implantate bei Morbus-Parkinson-Patienten am Schädel (Sondenverlauf) und Thorax (Impulsgeberaggregate) von Hellerhof (lizensiert auf Wikimedia Commons, CC BY-SA 3.0, Hellerhoff).}
	\label{figure:0407}
\end{figure}

\subsection{Akustische Parameter}
\label{subsec:040202}

Zunächst einmal lässt sich die akustische Ebene bei den Patienten betrachten. Häufig werden in solchen Fällen schnelle Silbenwiederholungsaufgaben (orale Diadochokinese, DDK) mit Patienten und Kontrollsprechern aufgenommen. DDK bestehen aus schnellen Wiederholungen von CV-Silben auf einem Atemzug, bei denen meist Plosive und Vokale kombiniert werden wie /papapa/, /tatata/ oder /kakaka/. Die Idee dahinter ist, dass eine schlechte Koordination des glottalen und oralen Systems \citep{Kent1999, Weismer1984, Ackermann1991, Pützer2007} sowie unvollständige orale Verschlussbildungen/orale Okklusionen
\citep[vgl.][]{Ziegler1983, Kent1999, Logeman1981, Weismer1984, Ackermann1995, Kent1982, Schweitzer2005} Merkmale sind, die häufig bei \isi{Dysarthrie} auftreten. Diese pathologischen Merkmale sind gut messbar im akustischen Signal, beispielsweise durch Erfassen von Stimmbeteiligung oder Friktion während der intendierten stummen Verschlussphase bei stimmlosen Plosiven, und eröffnen somit die Möglichkeit zu testen, ob Störungen der \isi{Sprechmotorik} unter Anwendung von VIM-DBS vermehrt auftreten. 

\begin{quote}
	The precision of stop consonant production can be determined in part by measures of the acoustic energy during the intended occlusive phase, or stop gap [\dots]. In general, normal production of a voiceless stop consonant is associated with a virtually silent gap. But some dysarthric speakers [\dots] tend to produce energy during the gap. This energy is typically one of two forms: turbulence noise (spirantization) generated at the site of \isi{oral} constriction because of an incomplete occlusion, and voicing energy, which often occurs because of poor coordination between laryngeal and supralaryngeal actions. \citep[][157--158]{Kent1999}
\end{quote}

Bleibt man bei der Messung im Bereich der Silbendomäne, so kann beispielsweise als pathologisches Merkmal eine mögliche Verlangsamung der Sprechgeschwindigkeit als Hinweis auf eine \isi{Dysarthrie} untersucht werden. Hierbei werden Silbendauern als ein akustisches Korrelat für die Messung der \isi{Artikulationsrate} verwendet (vgl. \citealt{Crystal1990}). Das wurde beispielsweise in einer Studie von \citet{Kronenbuerger2009} für ET-Patienten mit VIM-DBS durchgeführt. Es wurden drei Patientengruppen untersucht: ET Patienten ohne zusätzliche zerebellare Einschränkungen, ET Patienten mit zusätzlichen zerebellaren Einschränkungen und ET Patienten mit Tiefer \isi{Hirnstimulation} im Thalamus. Sie fanden, dass Patienten mit zusätzlichen zerebellaren Dysfunktionen längere Silbendauern aufweisen als solche, die keine zerebellaren Einschränkungen aufweisen. Darüber hinaus wurde aber kein Einfluss der \isi{Stimulation} auf die \isi{Artikulationsrate} ermittelt. Sie folgern, dass die Tiefe \isi{Hirnstimulation} keinen Einfluss auf die \isi{Sprechmotorik} hat. Mit dieser Interpretation muss man jedoch vorsichtig sein. Bedeutet dieses Ergebnis wirklich, dass die \isi{Stimulation} generell keinen Effekt auf die \isi{Sprechmotorik} hat oder wurde nur nicht der passenden Parameter untersucht?

Tatsächlich ändern sich die Ergebnisse, wenn Messungen im subphonemische Bereich miteinbezogen werden. \citet{Pützer2007} untersuchten Effekte der Tiefen \isi{Hirnstimulation} (VIM-DBS) für Multiple-Sklerose-Patienten (MS). In ihrer Studie schlossen sie lokale Parameter der Konsonantenproduktion in die Analyse mit ein. Hier zeigten sich bei Sequenzen wie /papapa/, /tatata/ und /kakaka/ erkennbare Stimulationseffekte, die zu dem Eindruck von „verwaschener Sprache“ führen. Sie fanden eine stimulationsinduzierte Defizienz in der Produktion von Plosiven, u.a. verifizierbar durch \isi{aperiodische Energie} während der konsonantischen Konstriktion, die aus einem unvollständigen oralen Verschluss resultiert. Auch \citet{Mücke2014a}, die mit Hilfe eines vergleichbaren Untersuchungsparadigmas den Effekt von VIM-DBS auf die \isi{Sprechmotorik} bei Essentiellen Tremor Patienten untersuchten, fanden bei schnellen Silbenwiederholungsaufgaben Probleme bei der Konsonantenproduktion. So zeigten auch hier unter \isi{Stimulation} die intendierten stimmlosen Verschlussphasen der Plosive /pa/, /ta/ und /ka/ vermehrt \isi{aperiodische Energie}, die auf einen unvollständigen Verschluss im Mundraum deuten. Der Effekt ist dramatisch, denn Friktion trat nahezu doppelt so oft in der on-DBS-Kondition verglichen mit off-DBS auf, und führt zu einer kritischen Verschlechterung in der Produktion von Plosiven. 

Die folgenden beiden Abbildungen veranschaulichen die Probleme im subphonemischen Bereich. Sie zeigen Beispiele für eine /ka/-Produktion im akustischen Signal mit und ohne artikulatorische Störungen. In Abbildung~\ref{figure:0408} ist während des konsonantischen Vollverschlusses der Luftstrom wie zu erwarten zeitweilig unterbrochen, was sich in Form eines steilen Energieabfalls (stummer Schall) manifestiert. Der Verschlussphase folgen die Verschlusslösung und \isi{Aspiration}, dann setzt der \isi{Vokal} ein.

\begin{figure}
	\includegraphics[width=.95\textwidth]{figures/4-7_Plosic_typisch.png}
	\caption{Beispiel für einen Silbenzyklus /ka/ mit akustischer Wellenform (oben) und Spektrogramm (unten) ohne artikulatorische Störung. Abbildung von \citet{Mücke2014a} adaptiert.}
	\label{figure:0408}
\end{figure}

Die Abbildung~\ref{figure:0409} zeigt das Beispiel einer Silbenproduktion von /ka/ mit artikulatorischen Störungen während der Konsonantenproduktion. Hier liegt Friktion in Form von aperiodischer Energie während der Konstriktion vor. Während ein vollständiger oraler Verschluss zu einer stummen Phase auf der akustischen Oberfläche führen sollte, sind die aerodynamischen Konsequenzen eines undichten Verschlusses geräuschverursachende Turbulenzen des nur unvollständig blockierten Luftstroms.

\begin{figure}
	\includegraphics[width=.95\textwidth]{figures/4-8_Plosiv_atypisch.png}
	\caption{Beispiel für einen Silbenzyklus /ka/ mit akustischer Wellenform (oben) und Spektrogramm (unten) mit artikulatorischer Störung. Abbildung von \citet{Mücke2014a} adaptiert. Es tritt aperiodische Energie im Verschluss auf.}
	\label{figure:0409}
\end{figure}

\clearpage 
\subsection{Artikulatorische Parameter}
\label{subsec:040203}

Besser noch kann die \isi{Sprechmotorik} direkt auf artikulatorischer Ebene untersucht werden. Im Folgenden soll am Beispiel eines Sprechers mit ET und VIM-DBS gezeigt werde, wie unvollständige konsonantische Verschlüsse im kinematischen Signal erfasst und mittels artikulatorischer Parameter abgebildet werden können. Die Aufnahmen wurden mittels 3D-Elektromagnetischer \isi{Artikulographie} am I\textit{f}L-Phonetik an der Universität zu Köln durchgeführt. Es handelt sich um ein Verfahren zur Erfassung von Bewegungsmustern der Sprechwerkzeuge wie Unter- und Oberlippe, Kiefer und Zunge. Das Verfahren gibt Aufschluss über Hohlraumkonfigurationen im \isi{Sprechtrakt} bei der Produktion erlernter Lautmuster, die auf akustischer Oberfläche nicht direkt ableitbar sind. Im Kopfbereich des Sprechers wird dabei ein inhomogenes Magnetfeld erzeugt, innerhalb dessen die Sensoren beim Sprechen lokalisierbar sind, vgl. Abbildung~\ref{figure:0410}.

\begin{figure}
	\includegraphics[width=\textwidth]{figures/a05-img10.jpg}
	\caption{Beispiel für das Befestigen der Sensoren mit Hilfe von medizinischem Gewebekleber, Foto von Fabian Stürtz 2014.}
	\label{figure:0410}
\end{figure}

Es wurden im Rahmen dieser Studie insgesamt 12 ET Patienten mit VIM-DBS und 12 gesunde Kontrollsprecher im selben Alter aufgenommen. Die Aufnahmen wurden mit einem 3-D Carstens AG501 (16 Kanäle, vgl. Abbildung~\ref{figure:0410}) durchgeführt. Dem Sprecher wurden Sensoren auf Unter- und Oberlippe, Kiefer, Zungenspitze, -blatt und –rücken platziert. Darüber hinaus wurden drei Sensoren an der Nasenwurzel und hinter den Ohren als Referenz verwendet, um Kopfbewegungen aus dem Gesamtdatensatz herauszurechnen, die insbesondere bei Patienten mit Essentiellem Tremor bei ausgeschalteter \isi{Stimulation} stark vorhanden sein können. Als Sprachmaterial wurden orale DDK-Aufgaben verwendet, bei denen die Patienten schnelle Silbenwiederholungen von Plosiv-Vokal-Sequenzen auf einem egressiven Atemzug produzieren sollten. Wie bei \citet{Pützer2007} wurden stimmlose Plosive mit stimmhaften Vokalen alterniert, um die Koordination des glottal-oralen Systems innerhalb der Silbenzyklen zu testen. Hinzu kommt eine Variation der Artikulationsstelle zur Alternierung der Beteiligung von Mundlippen, Zungenspitze und -rücken als primäre Konstriktoren zwischen den Sprachaufgaben in /papapa/, /tatata/ und /kakaka/. Es sei jedoch angemerkt, dass DDK-Aufgaben nicht direkt mit natürlicher Satzproduktion verglichen werden können, da dich die Patienten bei DDK an eine neue motorische Aufgabe, die \isi{schnelle Silbenwiederholung}, adaptieren müssen \citep{Staiger2016}.

Die artikulographische Aufnahmen der Patienten bestanden jeweils aus zwei aufeinanderfolgenden Sessions, einmal mit eingeschalteter (DBS-ON) und einmal mit ausgeschalteter \isi{Stimulation} (DBS-OFF) in randomisierter Form. Zwischen dem Ein- bzw. Ausschalten der \isi{Stimulation} und dem Beginn der zweiten Session wurde 20 Minuten gewartet, damit der Patient sich an den jeweiligen Zustand gewöhnen konnte. Die ganze Zeit, also vom Beginn der ersten Session bis zum Ende der zweiten Session, blieben die Sensoren an den Artikulatoren befestigt, um eine Vergleichbarkeit der Daten zu gewährleisten.

Die aufgezeichneten artikulographischen Positionsdaten wurden auf ein zweidimensionales kartesisches Koordinatensystem abgebildet \citep[vgl.][]{Hoole1999}, aus dem wiederum die Artikulationsbewegungen des Sprechers auf mittsagittaler Ebene ableitbar sind. Die Daten wurden bei der Aufnahme mit 1250 Hz gesampelt und bei der Datenaufbereitung dann zwecks Glättung auf 250 Hz downgesampelt sowie mit einem 40Hz Tiefpassfilter gefiltert, vgl. auch \citet{Mücke2014a}. Abbildung~\ref{figure:0411} veranschaulicht die Bewegungstrajektorien für die Unterlippe bei der Produktion von /papapa/ eines Sprechers. Es handelt sich bei der gestrichelten Kontur um zehn Silbenwiederholungen von /pa/ im DBS-OFF Zustand (\isi{Stimulation} ist ausgeschaltet) und bei der schwarzen Kontur zehn Silbenwiederholungen im DBS-ON Zustand (\isi{Stimulation} ist angeschaltet). Auf der x-Achse ist die Zeit in $ms$ und auf der y-Achse die \isi{Bewegungsauslenkung} (displacement) aufgetragen. Große Werte auf der y-Achse indizieren, dass die Lippen während des Plosives /p/ verschlossen sind, niedrige Werte indizieren eine Öffnung der Lippen während der Vokalproduktion.

\begin{figure}
	\includegraphics[width=.9\textwidth]{figures/4-10_Patient_papapa.png}
	\caption{Bewegungstrajektorien der Unterlippe während der Produktion von /papapa/ Silben im DBS-OFF (gestrichelte Linien) und DBS-ON Zustand (schwarze Linien) eines ET Patienten mit Tiefer Hirnstimulation.}
	\label{figure:0411}
\end{figure}

Die hier gezeigten Bewegungsabläufe veranschaulichen, dass die Silbenzyklen sowohl im DBS-OFF, als auch im DBS-ON unregelmäßig verlaufen, d.\,h. hier findet sich vermutlich ein erster Hinweis auf eine durch die \isi{zerebellare Dysfunktion} induzierte \isi{Dysarthrie}. Die Bewegungsabläufe verschlechtern sich jedoch deutlich unter der \isi{Stimulation}. Im DBS-ON Zustand haben die Silbenzyklen längere Dauern und kleinere Amplituden (Displacements) und teilweise sind die Silbenzyklen sind nicht mehr klar voneinander zu trennen. Um die Störung in der Silbenwiederholung zu verdeutlichen, zeigt Abbildung~\ref{figure:0412} zehn schnelle Silbenwiederholungen von /papapa/ eines gesunden Sprechers, der in demselben Alter wie der Patient ist. Die Silbenwiederholungen sind nicht nur deutlich schneller, sondern auch deutlich regelmäßiger in der räumlichen und zeitlichen Ausführung.

\begin{figure}
	\includegraphics[width=.9\textwidth]{figures/4-11_Kontroll_papapa2.png}
	\caption{Bewegungstrajektorien der Unterlippe während der Produktion von /papapa/ eines gesunden Kontrollsprechers.}
	\label{figure:0412}
\end{figure}

    
Es sei jedoch angemerkt, dass nicht alle Patienten eine so starke Störung der \isi{Artikulation} aufweisen, wie oben in der Abbildung dargestellt. Die Beeinträchtigung der Lippen und Zungenbewegungen variiert zwischen den Sprechern. Im Folgenden soll anhand eines Fallbeispiels gezeigt werden, wie die Artikulationsbewegungen parametrisiert werden können. Es handelt sich um einen 73jährigen ET Patienten mit VIM-DBS, der mehrere Monate vor der Aufnahme eine bilaterale VIM-DBS-\isi{Stimulation} erhalten hatte, d.\,h. es wurden ihm zwei Elektroden implantiert. Er zeigte zum Zeitpunkt der Aufnahme bereits zusätzliche zerebellare Dysfunktionen, die häufig mit Essentiellem Tremor einhergehen. Der Sprecher ist Muttersprachler des Deutschen mit westfälischem Regiolekt.

Zunächst werden die konsonantischen Verschlussgesten für /p/, /t/ und /k/ in den /papapa/, /tatata/, /kakaka/ Sequenzen annotiert. Die Lippengeste bezeichnet die Bewegung der Unterlippe von der maximalen Öffnung des vorangehenden Vokals bis zum maximalen Verschluss im initialen Konsonanten. Start- und Zielpunkt (\isi{Onset} und \isi{Target}) der Lippenbewegung sind anhand von Nulldurchgängen der Geschwindigkeitskurve (erste Ableitung der Positionskurve) bestimmbar. Beim Start und beim Ziel des Aktivierungsintervalls für die \isi{Öffnungsgeste} beträgt die Geschwindigkeit der Lippenbewegung Null. Eine weitere Landmarke ist die \isi{Maximalgeschwindigkeit} der Verschlussgeste (peak velocity, \isi{pVel}). Diese ist anhand der Beschleunigungskurve (zweite Ableitung der Positionskurve) bestimmbar. Ist die \isi{Maximalgeschwindigkeit} der Bewegung erreicht, beträgt deren Beschleunigung Null. Abbildung~\ref{figure:0413} zeigt ein Annotationsschema für die konsonantische Verschlussgeste mit Bestimmung von \isi{Onset}, \isi{Maximalgeschwindigkeit} und \isi{Target} des Verschlusses.

\begin{figure}[t]
	\includegraphics[width=.8\textwidth]{figures/4-12_Annotationsschema.png}
	\caption{Annotationsschema für die konsonantische Verschlussgeste mit Bestimmung von Onset, Maximalgeschwindigkeit und Target des Verschlusses.}
	\label{figure:0413}
\end{figure}


Basierend auf den Landmarken \isi{Onset}, \isi{pVel} und \isi{Target} werden die folgenden vier Messvariablen bestimmt, um die zeitlichen und räumlichen Eigenschaften der \isi{Öffnungsgeste} zu erfassen, vgl. Kapitel~\ref{sec:0401}.

\begin{enumerate}[(a)]
	\item Die Dauer der Verschlussgeste vom \isi{Onset} bis zum \isi{Target} der Bewegung, in ms.
	\item Die Amplitude (maximale \isi{Auslenkung}, auch \isi{Displacement}) der Verschlussgeste vom \isi{Onset} bis zum \isi{Target} der Bewegung, in mm.
	\item Das \isi{Maximalgeschwindigkeit} (\isi{pVel}) der Verschlussgeste, in mm/s.
	\item Die \isi{Steifheit} \[\left(\frac{pVel}{Displacement}\right)\] der Verschlussgeste als Verhältnis zwischen \isi{Maximalgeschwindigkeit} und Amplitude \citep[nach][]{Munhall1985}.
\end{enumerate}

Die Ergebnisse sind in Tabelle~\ref{table:0401} in Form von Mittelwerten und zugehörigen Standardabweichungen dargestellt. Die Abbildungen~\ref{figure:0414}~(a-d) zeigen die entsprechenden Balkendiagramme für die einzelnen Messungen (a) Dauer der \isi{Geste}, (b) Amplitude, (c) \isi{Maximalgeschwindigkeit} und (d) \isi{Steifheit}, separat nach DBS-OFF (ausgeschalteter \isi{Stimulation}) und DBS-ON Kondition (eingeschaltete \isi{Stimulation}). Da es sich nur um einen Sprecher mit relativ wenigen Datenpunkten handelt, wird hier eine rein deskriptive Statistik verwendet.

\begin{table}[htpb] 
	\begin{tabularx}{\textwidth}{Xcccc} \lsptoprule
		& \textbf{DBS} & \textbf{/pa/} & \textbf{/ta/} & \textbf{/ka/}\\ \midrule
\multirow{2}{*}{\textbf{Dauer} ($ms$)} & OFF & 97 (5,9) & 103 (14,9) & 115 (40,5)\\
		& ON &  173 (28,6) & 174 (19,3) & 174 (36,5)\\ \cmidrule(lr){1-5}
\multirow{2}{*}{\textbf{Amplitude} ($mm$)} & OFF &  7 (1,7) &  4 (1,0) &  4 (1,3)\\
		& ON &  6 (1,5) &  6 (1,4) &  5 (2,0)\\ \cmidrule(lr){1-5}
\multirow{2}{*}{\textbf{Max. Geschwindigkeit} ($mm/s$)} & OFF &  117 (29,0) &  72 (22,8) &  64 (17,3)\\
		& ON &  66 (22,3) &  75 (15,6) &  55 (26,7)\\ \cmidrule(lr){1-5}
\multirow{2}{*}{\textbf{Stiffness}} & OFF &  17,3 (1,3) &  19,3 (2,4) &  15,9 (2,2)\\
		& ON &  10,8 (2,1) &  13,3 (3,4) &  10,8 (2,8)\\ \lspbottomrule
	\end{tabularx} 
	\caption{Mittelwerte und zugehörige Standardabweichungen (in Klammern) für die Messungen (a) Dauer der Geste, (b) Amplitude, (c) Maximalgeschwindigkeit und (d) Steifheit, getrennt nach DBS-OFF und DBS-ON (1 Sprecher).}
	\label{table:0401}
\end{table}

\begin{figure}
	\includegraphics[width=\textwidth]{figures/4-13_Barplos_einSprecher.png}
	\caption{Mittelwerte und zugehörige Standardabweichungen in Klammern für die Messungen (a) Dauer der Geste, (b) Displacement, (c) Maximalgeschwindigkeit und (d) Steifheit, getrennt nach DBS-OFF und DBS-ON (1 Sprecher).}
	\label{figure:0414}
\end{figure}

Es zeigt sich, dass die Dauer der Verschlussgeste unter \isi{Stimulation} deutlich zunimmt. Für bilabiale Plosive /p/ verlängern sich die Dauern um 76 ms, für alveolare Plosive /t/ um 71 ms und für velare Plosive /k/ um 59 ms. Gleichzeitig verringert sich die \isi{Steifheit} (die relative Geschwindigkeit) für alle drei Artikulationsstellen. Sie sinkt für /p/ um 6,5, für /t/ um 6 und für /k/ um 5,1 ab. Die Gründe hierfür sind jedoch in den labialen und lingualen Subsystemen unterschiedlich. So verhalten sich beispielsweise /p/ und /t/ gegenläufig: Bei /p/ verringert sich die Amplitude (\isi{Displacement}) leicht, d.\,h. die Unterlippe legt tendenziell einen kürzeren Weg von durchschnittlich 1 mm unter \isi{Stimulation} zurück (da es sich nur um die \isi{Trajektorie} der Unterlippe handelt, kann nicht geklärt werden, ob die Oberlippe kompensiert). Dagegen sinkt aber die \isi{Maximalgeschwindigkeit} der Unterlippe um 51 mm/sec dramatisch ab. Dadurch, dass die \isi{Maximalgeschwindigkeit} nichtproportional zur Amplitude (\isi{Displacement}) abnimmt, sinkt die mit der \isi{Eigenfrequenz} der Bewegung assoziierte \isi{Steifheit}. Der Weg ist relativ zur erreichten \isi{Maximalgeschwindigkeit} größer geworden, und es braucht eine längere Zeit, um das Ziel zu erreichen. 

Bei der Zungenspitzenbewegung in /t/ findet sich ein nicht-proportionaler Anstieg von Amplitude (\isi{Displacement}) relativ zur \isi{Maximalgeschwindigkeit} im DBS-ON; der Weg, den die Zungenspitze vom Verschlussbeginn bis zum Ziel zurücklegt ist im DBS-ON durchschnittlich 2 mm länger, aber die Bewegung in ihrer \isi{Maximalgeschwindigkeit} nur 3 mm/sec schneller. Auch dieses Verhältnis führt zu einer Absenkung der \isi{Steifheit}, denn die \isi{Maximalgeschwindigkeit} hat nicht genügend zugenommen, um das weiter entfernte Ziel in gleicher Zeit zu erreichen und somit dauert die gesamte Bewegungsausführung länger im DBS-ON als im DBS-OFF. Beim Zungenrücken in /k/ nimmt die Amplitude (\isi{Displacement}) um 1 mm zu und die Geschwindigkeit um 9 mm/sec ab. Auch hier sinkt die \isi{Steifheit} und mit ihr steigt die Gesamtdauer der \isi{Bewegungsauslenkung}, denn der Weg ist nicht nur weiter, sondern auch die \isi{Maximalgeschwindigkeit} geringer geworden. 

Wir haben es hier vermutlich mit multiplen Parametermodifikationen zu tun, die sich innerhalb der artikulatorischen Subsysteme unterschiedlich ausdrücken. Wenngleich Gesamtdauern der konsonantischen Verschlussgesten stets zunehmen und die damit verbundene \isi{Steifheit} abnimmt, zeigen sich doch deutliche Unterschiede. So werden nicht alle Bewegungseinheiten in ihrem Aufwand reduziert. Vielmehr vergrößern sich teilweise Amplituden und Maximalgeschwindigkeiten im lingualen System, was zunächst mit einem größeren Artikulationsaufwand verbunden ist und vermutlich einer größeren Öffnung des Vokaltraktes während der transvokalischen Öffnung geschuldet ist (die transvokalische Öffnung des vorangehenden Vokals ist ja der Ausgangspunkt des nun folgenden konsonantischen Verschlusses). Dennoch steigt die Präzision in der Konsonantenproduktion nicht an, und es kommt zur Friktion auf akustischer Oberfläche während der intendierten stummen Phase in /p/, /t/ und /k/. Vermutlich liegt die Ursache darin, dass schnelle Silbenwiederholungen (orale Diadochokinese) keine natürliche Sprache darstellen. Es handelt sich vielmehr um neuartige Bewegungsaufgaben, an deren sprechmotorische Anforderungen sich der Sprecher erst adaptieren muss \citep{Ziegler2002}. Diese Adaption gelingt den Patienten nur bedingt und es kommt insbesondere unter \isi{Stimulation} zu einer unökonomischen Ausführung der sprechmotorischen Aufgabe mit erhöhtem Artikulationsaufwand bei gleichzeitig abnehmender Präzision.

\section{π-Geste als „artikulationsloser“ Parameter}
\label{sec:0403}

Neben Konstriktionsgesten, die linguistisch relevante Verschlüsse innerhalb des Sprechtrakts für die Produktion von Konsonanten und Vokalen festlegen, gibt es auch „artikulationslose“ Gesten. Diese artikulationslosen Gesten werden beispielsweise verwendet, um Effekte der prosodischen Struktur auf die textuelle Schicht abbilden zu können. Der Aufbau der prosodischen Struktur wird in Kapitel~\ref{chap:05} genauer dargelegt. Es sei hier aber bereits angemerkt, dass bei einer solchen Struktur u.a. Grenzen zwischen linguistischen Konstituenten wie \isi{Silbe}, Wort oder Phrase angenommen werden. Diese Grenzen sind unterschiedlich stark. Während zwischen zwei Silben innerhalb eines Wortes eine kleine Grenze besteht, wird zwischen zwei Phrasen eine große Grenze angenommen. Solche Grenzen haben einen Einfluss auf die temporalen Eigenschaften von Konstriktionsgesten und werden mit Hilfe von prosodischen Gesten (π-\isi{Geste}) modelliert
\citep{Saltzman1999, Byrd2000b, Byrd2003a, Byrd2003b}.

Die π-\isi{Geste} ist eine artikulationslose \isi{Geste} ohne eigene \isi{Traktvariable} \citep{Byrd2003a, Byrd2003b}. Sie hat ähnlich wie Konstriktionsgesten ein \isi{intrinsisch} definiertes \isi{Aktivierungsintervall} und kann mit Konstriktionsgesten aller Art -- darunter auch Intonationsgesten -- überlappen. Sie markiert die prosodischen Grenzen, an denen sie implementiert ist. Dabei bewirkt die Verlangsamung der Taktung einer Äußerung (clock slowing), dass die lokalen Ausführungsgeschwindigkeiten von Konstriktionen vor einer Grenze verlangsamt und nach einer Grenze wieder beschleunigt werden. Dabei kann sie nur solche Konstriktionen modifizieren, die in ihr \isi{Aktivierungsintervall} fallen, d.\,h. mit denen sie überlappt.

Abbildung~\ref{figure:0415} gibt ein Beispiel für die Modifikation einer ${C}_{1}$\#${C}_{2}$ Sequenz durch die π-\isi{Geste}. Es handelt sich um das Beispiel /m\#z/ in der \il{Deutsch}deutschen Äußerung „Es wurde warm. Sonne schien durchs Fenster.“ Während dem Konstriktionslevel konkrete \isi{Traktvariablen} (LA = Lip Aperture; TT = Tongue Tip) zur Ausführung der Bewegungsaufgaben zugeordnet sind, bleibt die π-\isi{Geste} artikulationslos. Ihr Einfluss auf den Konstriktionslevel ist während ihrer maximalen Aktivierung an der Grenze am stärksten. In der abgebildeten \isi{Gestenpartitur} beeinflusst sie am stärksten die Lösung von ${C}_{1}$ und die Verschlussbildung von ${C}_{2}$. 

\begin{figure}[b]
	\includegraphics[width=.8\textwidth]{figures/4-14_pi.png}
	\caption{Schematische Abbildung für den Einfluss der π-Geste auf Konstriktionsgesten vor und nach einer Grenze ${C}_{1}$\#${C}_{2}$, adaptiert von \citet[][160]{Byrd2003a}).}
	\label{figure:0415}
\end{figure}

Wie stark ist der Einfluss der π-\isi{Geste} an der jeweiligen Grenze? Das hängt von der maximalen \isi{Auslenkung} der π-\isi{Geste} ab. Diese \isi{Auslenkung} kann nach dem Modell aus der Stärke der prosodischen Grenze abgeleitet werden und möglicherweise damit auch die Domänengröße der π-\isi{Geste}. 

\newpage 
\begin{quote}
	The π-gesture’s maximum level of activation is determined by the prosodic boundary strength (boundary strength could, for example, be viewed as the number of aligned domain edges). \citep[][160]{Byrd2003a}
\end{quote}

Es werden nicht mehrere π-Gesten für die Erhöhung der Stärke übereinander gelagert, sondern nur eine π-\isi{Geste} pro Grenze verwendet. Im Falle einer Pause zwischen zwei Phrasen werden zwei aufeinanderfolgende π-Gesten benötigt, um sowohl die Finalität der ersten Grenze als auch den Beginn der nachfolgenden Grenze über die Pause hinweg modellieren zu können. Die π-\isi{Geste} kann für eine konkrete \isi{Modellierung} in Form, Stärke und \isi{Alignierung} modifiziert werden.

Die π-\isi{Geste} beeinflusst den Takt einer Äußerung und bewirkt damit per Definition ausschließlich \textit{temporale} Veränderungen im Sinne von längeren Gesten. Folglich überlappen Konstriktionsbewegungen weniger und werden somit nicht trunkiert. Spatiale Modifikationen jeglicher Art entstehen also indirekt aus den temporalen Modifikationen, hier dem geringeren Überlappungsgrad zwischen Gesten.

\citet[][161--163]{Byrd2003a} machen die folgenden Annahmen für die Implementierung der π-\isi{Geste} und deren Effekte auf koaktive Konstriktionsgesten:


\begin{description}
	\item[π-\isi{Geste} (1):] Konstriktionsgesten aller Art werden verlangsamt, wenn sie während des Zeitraums der Aktivierung der π-\isi{Geste} durch \isi{Überlappung} aktiv sind.
	\item[π-\isi{Geste} (2):] Der Grad der Verlangsamung ist am höchsten, wenn die π-\isi{Geste} ihre \isi{maximale Aktivierung} an der prosodischen Grenze erreicht hat.
	\item[π-\isi{Geste} (3):] Effekte sind auf die Konstriktionsgesten beschränkt, die sich nahe einer prosodischen Grenze finden.
	\item[π-\isi{Geste} (4):] Die π-\isi{Geste} hat prinzipiell dieselben dynamischen Effekte auf die Konstriktionsgesten der linken und rechten Seite einer prosodischen Domäne. Je nach konkreter Koordination der π-\isi{Geste} mit den Konstriktionsgesten können sich aber die Stärke des Effekts und die kinematischen Charakteristika unterscheiden.
\end{description}

Da die Ebene des Taktgebers und die der Konstriktionsgesten miteinander bidirektional gekoppelt sind, handelt es sich bei den zeitlichen Modifikationen nicht um Effekte einer externen globalen Taktung. Vielmehr handelt es sich um ein \isi{intrinsisch} basiertes Modell, bei dem die prosodische Ebene als Einflussnehmer auf die lokale Taktfrequenz und die Ebene der Konstriktionsgesten mit intrinsischen Dauerverhältnissen ein einziges dynamisches Kollektiv höherer Ordnung bilden \citep[][156]{Byrd2003a}. Die Verlangsamung des Äußerungstakts (clock slowing) ist nach diesem Modell nicht durch ein unidirektionales Herabsetzen der \isi{Steifheit} von Konstriktionsgesten innerhalb einer Domäne oder durch die Implementierung eines \isi{prosodisch} gesteuerten Fensters im Sinne eines Zeitlupeneffekts determiniert, sondern vielmehr multidirektionaler Art.

Die genaue Domäne der π-\isi{Geste} ist jedoch unklar. Sie kann in ihrer Ausdehnung stark variieren \citep{Byrd2006}. \citet[543--544]{Cho2006} zeigt auf, dass auch Vokale vor und nach einer Grenze modifiziert werden können. In einer ${C}_{1}{V}_{1}$\# ${C}_{2}{V}_{2}${}-Sequenz können beispielsweise alle konsonantischen und vokalischen Gesten dieser Sequenz von der Taktverlangsamung betroffen sein (vgl. Kapitel~\ref{chap:05}). Des Weiteren stellt \citet{Cho2006} die Frage nach spatialen Modifikationen, die im π-\isi{Gestenmodell} nur aus einer Reduktion des Überlappungsgrades ableitbar sind. Diese spatialen Modifikationen finden aber regelhaft statt und werden möglicherweise im Zusammenhang mit weiteren prosodischen Strukturen auch direkt angesteuert.

\il{Spanisch!peruanisch}
\il{Spanisch!argentinisch}
\citet{Kim2010} und \citet{Kim2011} haben die π-\isi{Geste} adaptiert und erstmals auf lexikalische Wortbetonung angewendet. Sie haben Onset-Nukleus-Relationen im argentinischen und peruanischen Spanisch untersucht. In unbetonten Silben sind die Dauerverhältnisse in CV-Silben vergleichbar, während systematische Unterschiede in betonten CV-Silben auftreten. Unter Wortbetonung sind auf artikulatorischer Ebene die Konstriktionsbewegungen für C im peruanischen im Vergleich zum \isi{Vokal} von längerer Ausführungsdauer als im argentinischen Spanisch. Auf akustischer Oberfläche führt das zu unterschiedlichen Segmentdauern: ${C}_{1}>{C}_{2}$ und ${V}_{1}<{V}_{2}$ (vergleiche Abbildung~\ref{figure:0416}).

\begin{figure}
	\includegraphics[width=\textwidth]{figures/4-15_pi_gesture_Miran_Kim.png}
	\caption{Einfluss der π-Geste auf die Onset-Nukleus-Relation im peruanischen und argentinischen Spanisch; computergestützte Simulation in der artikulatorischen Synthese TADA (Task Dynamics Application) aus \citet[151]{Kim2011}.}
	\label{figure:0416}
\end{figure}

Diese systematischen Unterschiede können mit Hilfe der π-\isi{Geste} modelliert werden, die in lexikalisch betonten Silben mit den oralen Konstriktionsgesten gekoppelt wird. Die π-\isi{Geste} hat dabei unterschiedliche Formen: Im peruanischen Spanisch ist der Anstieg der π-\isi{Geste} schneller (höhere \isi{Steifheit}). Der maximale Level der Aktivierung ist noch während des Konsonanten erreicht und bewirkt eine temporale Dehnung des Konsonanten (die Effekte der π-\isi{Geste} auf die Verlangsamung des Taktes ist während ihrer maximalen Aktivierung am stärksten). Im argentinischen Spanisch hingegen steigt die π-\isi{Geste} weniger steil an (weniger steif), sodass ihre \isi{maximale Aktivierung} auf den \isi{Vokal} und nicht auf den vorangehenden Konsonanten fällt.

Auch \citet{Bombien2010} vermuten, dass die π-\isi{Geste} auf durch prominenzinduzierte Effekte der prosodischen Stärke angewendet werden könnte (vgl. Kapitel~\ref{chap:05}). In einer EPG-Studie zu wortinitialen  ${C}_{1}{C}_{2}$-Sequenzen im Deutschen zeigen sie, dass ${C}_{1}$ stärker durch die Grenze und ${C}_{2}$ stärker durch die Wortbetonung gelängt werden (die Effekte der Wortbetonung waren in ihren Daten schwächer und weniger konsistent als die der Position). Sie zeigen aber auch, dass die segmentale Komposition der Clustersequenz bereits einen starken Einfluss auf die Koordination zwischen Gesten haben. So überlappen unabhängig von der prosodischen Struktur die konsonantischen Plateaus in /kl/-Sequenzen stärker als /kn/-Sequenzen. Ein Vergleich von prosodischer Stärke darf demnach nicht zwischen unterschiedlichen Clustertypen durchgeführt werden.