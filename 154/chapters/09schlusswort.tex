\chapter{{Schlusswort}}
\label{chap:09}

Die \isi{Artikulatorische Phonologie} stellt eine wichtige Alternative zu segmentalen Ansätzen dar. Als dynamisches Model, das mit Attraktoren statt mit festen Kategoriengrenzen arbeitet, kann sie natürliche \isi{Variabilität} abbilden. Diese natürliche \isi{Variabilität} ist wichtig, will man das Verhalten eines Sprachsystems verstehen. Neben kontextuellen Einflüssen kann auch die Prosodie eine Vielfalt von systematischer \isi{Variabilität} generieren, die weniger als Rauschen in den Daten, sondern mehr ein Fenster zur linguistischen Struktur darstellt.

In den letzten Jahren hat es zunehmend Forschung im Bereich der Wechselwirkung von \isi{Artikulation} und prosodischer Struktur gegeben. Hier sind beispielsweise Markierungen von \isi{Prominenz} zu nennen (Hyperatikulation, \isi{Assimilation}, Reduktionsformen, VOT-Variationen), aber auch die Interaktion von tonaler und segmentaler Signatur in der der phonetischen Substanz. Diese Studien verbindet die Auffassung, dass es keine starren Kategoriengrenzen gibt. So kann \isi{Assimilation} als \isi{Überlappung} von Gesten verstanden werden, die unterschiedlichste Realisierungsformen erlauben, statt das vollständige Angleichen eines Segments an ein anderes. Auch in der klinischen Linguistik ist dieser Ansatz wichtig. Gerade in Dysarthrien kommt es häufig nicht zur Substitution von Segmenten (beispielsweise Spirantisierung), sondern vielmehr zu graduellen Variationen (eine artikulatorische \isi{Zielposition} wird nicht erreicht), die sich auch in der Schwere einer \isi{Dysarthrie} widerspiegeln kann.

Obwohl die \isi{segmentale Ebene} inzwischen sehr gut durch die \isi{Artikulatorische Phonologie} beschrieben ist, gibt es noch Baustellen im Bereich der Implementierung von prosodischer Struktur. So ist noch unklar, wie eine tonale \isi{Geste} im System integriert sein kann (beispielsweise sind tonale Gesten letztendlich aufgrund eines akustischen F0-Verlaufs und nicht artiukulatorisch definiert) und es bleibt auch noch offen, wie die Markierung der prosodischen Grenzen als lokale Modifikatoren im räumlichen und zeitlichen Bereich an die Kopplungsgraphen eines Netzwerks von gesturalen Triggern angebunden sind.

Die \isi{Artikulatorische Phonologie} ist sicher kein Ersatz für segmentale Ansätze. Aber sie bildet eine Erweiterung insbesondere im Hinblick auf das Erfassen von \isi{Variabilität}, die wir nicht mehr mittels künstlicher Schwellenwerte und starren Kategoriengrenzen aus Datensätzen herausrechnen sollen. Sie sollte ein wichtiger Bestandteil nicht nur in der experimentellen Forschung, sondern auch in der sprachwissenschaftlich ausgelegten Lehre sein, will man die Beziehung von Signal und kognitiver Verarbeitung verstehen.