\chapter{Studien zur Prominenz-Markierung}
\label{chap:06}

Das folgende Kapitel behandelt zwei Studien zur Fokusmarkierung im Deutschen, die jeweils unterschiedliche Konzepte verfolgen und unterschiedliche Messverfahren einsetzen \citep[vgl.][]{Mücke2008c, Mücke2014b}. Es baut auf den Grundlagen der prosodischen Analyse in Kapitel~\ref{chap:05} auf. In der ersten Studie, die mit Hilfe von Elektropalatograpie durchgeführt worden ist, geht es um den Grad der \isi{Assimilation} in Partikelverben im Deutschen in nicht-akzentuierter versus akzentuierter Position \citep{Mücke2008c}. Es zeigt sich, dass im engen und kontrastiven \isi{Fokus} der Grad der gesturalen \isi{Überlappung} in den Zielsilben weniger stark ist als im Hintergrund. Hier führt die \isi{prosodische Stärkung} zu einem geringeren Grad von \isi{Assimilation}. Es zeigt sich auch, dass die \isi{Assimilation} graduell und nicht kategorial verläuft.  

Die zweite Studie, die elektromagnetische \isi{Artikulographie} als Messverfahren nutzt, basiert auf der Analyse von \citet{Mücke2014b}, legt jedoch diesmal das Hauptaugenmerk auf den methodischen Teil, indem ein Vergleich unterschiedlicher Messvariablen durchgeführt wird. In dieser Studie geht es um die Frage, ob die phonologische Spezifikation der Akzentuierung eines Wortes tatsächlich dessen artikulatorische Hervorhebung bedingt. So stellt im Autosegmental-Metrischen Modell die Akzentuierung eine diskrete Kategorie dar. Ein Wort ist entweder akzentuiert oder nicht. Es hat sich aber in der vorliegenden Studie gezeigt, dass es sich bei Akzentuierung und artikulatorischer Markierung um Faktoren handelt, die bis zu einem gewissen Grad unabhängig voneinander betrachtet werden können. Zielwörter im weiten \isi{Fokus}, beispielsweise, erhalten einen \isi{Tonakzent}, werden aber nicht artikulatorisch markiert. In dieser Studie gehen wir einen Schritt weiter und vergleichen neben artikulatorischen Muster auch tonale Muster. Das ist wichtig, weil Fokusmarkierung ja primär \isi{tonal} erfolgt und beide Ebenen, die tonale und die artikulatorische, ineinander spielen. Deshalb sollen die Ebenen hier nicht, wie das traditionell oft der Fall ist, isoliert betrachtet werden.

Da beide Studien sich mit Fokusmarkierung beschäftigen, wird zunächst auf die Elizitation von Sprachdaten innerhalb solcher Experimente eingegangen. Zu prosodischen Grundlagen und Begriffsbestimmungen vgl. Kapitel~\ref{chap:05}. 

\section{Elizitation: Was ist fokuskontrollierender Kontext?}
\label{sec:0601}

Als \isi{Fokus} wird der Teil einer Äußerung bezeichnet, der inhaltlich im Vordergrund steht, also kommunikativ relevante Information enthält \citep[vgl.][]{Lambrecht1994, Uhmann1991, Krifka2008}. Dem steht der nicht-fokussierte Teil einer Äußerung gegenüber, der in der Regel aus dem Kontext ableitbar ist.

Hierbei eignen sich beispielsweise Frage-Antwort-Paare als fokuskontrollierender Kontext für Sprachaufnahmen \citep[u.a.][]{Uhmann1991, Wagner2012, Krifka2008, Culicover1983, Büring2003}. Den \isi{Fokus} bildet jeweils derjenige Teil der Äußerung, der die Frage beantwortet. Bei den folgenden Beispielen, die für das Deutsche adaptiert sind, werden die akzentuierten Silben durch Großbuchstaben gekennzeichnet \citep{Mücke2014b}.

In Beispiel~\ref{ex:0601} bildet <Ich treffe mich mit> den Hintergrund, <Mary> bildet den \isi{Fokus}. Liegt der \isi{Fokus} nur auf einer einzelnen \isi{Konstituente}, wird er als \textit{enger Fokus} bezeichnet \citep[vgl.][]{Ladd1980}.

%%(6.1)
\begin{exe}
	\ex Enger \isi{Fokus}:\label{ex:0601}
	\sn F: Mit wem triffst du dich?
	\sn A: [Ich treffe mich mit]\textsubscript{Hintergrund}[MAry]\textsubscript{Fokus}
\end{exe}

Beispiel~\ref{ex:0602} zeigt exemplarisch einen \textit{kontrastiven Fokus}. Dieser kontrastive \isi{Fokus} beinhaltet hier einen expliziten Kontrast und eine Korrektur des Vorerwähnten \citep[vgl.][]{Wagner2012}. Kontrastiver \isi{Fokus} zählt im Vergleich zu engem oder weitem \isi{Fokus} als eigener Fokustyp.

%%(6.2)
\begin{exe}
	\ex Kontrastiver \isi{Fokus}:\label{ex:0602}
	\sn F: Triffst du dich mit John?
	\sn A: Nein, [ich treffe mich mit]\textsubscript{Hintergrund}[MAry]\textsubscript{Fokus}
\end{exe}

In den Fällen der Beispiele~\ref{ex:0601} und \ref{ex:0602} erhält das Zielwort <Mary> einen \isi{Tonakzent}, der mit der lexikalisch betonten \isi{Silbe} assoziiert ist. In Beispiel~\ref{ex:0603} ist jedoch ein Fall von \textit{weitem Fokus} dargestellt, bei dem die Fokusdomäne größer als eine \isi{Konstituente} sein kann \citep[vgl.][]{Ladd1980}. Hier ist die ganze Antwort fokussiert. Dieses Phänomen wird als Fokusprojektion bezeichnet \citep[vgl.][]{Büring2003, Welby2003}. Der \isi{Tonakzent} auf <Mary> markiert eine größere Domäne, d.\,h. das Wort, das den \isi{Tonakzent} zugewiesen bekommt, stimmt nicht mit der Größe der Fokusdomäne, wie in Beispiel~\ref{ex:0601} und \ref{ex:0602}, überein \citep[vgl.][]{Uhmann1991, Fery1993}.

\newpage 
%%(6.3)
\begin{exe}
	\ex Weiter \isi{Fokus}:\label{ex:0603}
	\sn F: Was gibt´s Neues?
	\sn A: [Ich treffe mich mit MAry]\textsubscript{Fokus}
\end{exe}

In kurzen Dialogen kann also die Frage als fokuskontrollierender Kontext verwendet werden, um eine bestimmte \isi{Fokusstruktur} in der Antwort zu elizitieren, beispielsweise ob das Zielwort im engen \isi{Fokus}, weiten \isi{Fokus}, oder kontrastivem \isi{Fokus} steht. Es kann auch außerhalb der Fokusdomäne stehen und den Hintergrund bilden, dann erhält es keinen Akzent, vgl. Beispiel~\ref{ex:0604}.

%%(6.4)
\begin{exe} 
	\ex Hintergrund:\label{ex:0604}
	\sn F: Trifft sich Martin mit Mary?
	\sn A: [ICH]\textsubscript{Fokus} [treffe mich mit Mary]\textsubscript{Hintergrund}
\end{exe}

Im Autosegmental-Metrischen-Modell stellt die Akzentuierung eine diskrete Kategorie dar. Entweder ist ein Wort akzentuiert (im \isi{Fokus}) oder nicht (im Hintergrund). 

\section{Assimilation und Fokus: Eine EPG-Studie}
\label{sec:0602}

\subsection{Assimilation als graduelles Phänomen}
\label{subsec:060201}

Verschiedene Studien zur Messung des Assimilationsgrades mit Hilfe von Elektropalatographie und elektromagnetischer \isi{Artikulographie} haben gezeigt, dass Assimilationen häufig partiell auftreten (vgl. Kapitel~\ref{subsec:020202} zur gesturalen \isi{Modellierung} von \isi{Assimilation}). Anhand von alveolar-velaren Sequenzen konnte nachgewiesen werden, dass die alveolare \isi{Geste} häufig nicht substituiert, sondern zeitlich und räumlich abgeschwächt wird (\citealt{Barry1991} und \citealt{Ellis2002} zum Englischen; 
\citealt{Kohler1995}, \citealt{Jaeger2007}, \citealt{Bergmann2008} und \citealt{Mücke2008c} zum Deutschen).
Das bedeutet, dass die alveolare \isi{Geste} im kinematischen Signal in alveolar-velaren Sequenz wie /n\#g/ in <hiNGeben> meist noch nachweisbar, wenngleich auf akustischer Oberfläche der Nasal an die velare Assimilationsstelle angeglichen scheint. Der Grad der \isi{Assimilation} kann stark variieren. Es kann zwischen (a) vollständiger Ausführung der Gesten (keine \isi{Assimilation}), (b) abgeschwächter Ausführung der Gesten (partielle \isi{Assimilation}) und (c) Ausfall der Gestenausführung (segmentale Substitution) unterschieden werden. In (a) findet keine \isi{Assimilation} statt und die alveolare \isi{Geste} wird vollständig realisiert. In (b) wird eine partielle \isi{Assimilation} vorgenommen, bei der die alveolare \isi{Geste} räumlich und/oder zeitlich geschwächt ist und in (c) eine Substitution, bei der keine alveolare \isi{Geste} realisiert wird. Diese drei Strategien werden am Beispiel elektropalatographischer Messungen veranschaulicht.

Bei der Elektropalatographie handelt es sich um ein Messverfahren, mit dem der Zungen-Gaumen-Kontakt im Mundraum erfasst wird. Bei einer solchen Messung trägt der Sprecher einen künstlichen Gaumen, in dem acht Reihen von insgesamt 62 Elektroden eingelassen sind. Die Veränderungen der Zungen-Gaumen\-kontakte können dynamisch über die Zeit im Abstand von 10 ms pro Frame gemessen werden. Abbildung~\ref{figure:0601} gibt ein Beispiel für einen alveolaren und velaren Vollverschluss im Mundraum, /n/ und /g/ (Reading EPG unter Verwendung der Articulate Assistant\textsuperscript{\textcopyright} Software). Bei solchen Kontaktprofilen wird die 
Reihe 1--2 in der Regel als alveolare Zone klassifiziert, 
Reihe 3--4 als postalveolare Zone, 
Reihe 5--7 als palatale Zone und Reihe 8 als velare Zone \citep{Gibbon1999}. Es können jedoch sprecherspezifische Abweichungen auftreten \citep{Byrd1996b}, so dass für eine Annotation von EPG-Daten sprecherspezifische Kontaktprofile angelegt werden sollten. Das betrifft u.a. velare Konsonanten, bei denen der velare Komplettverschluss häufig erst hinter dem künstlichen Gaumen stattfindet. 


\begin{figure}
	\includegraphics[width=.6\textwidth]{figures/6-1_EPG_Schema.png}
	\caption{Zungen-Gaumen-Kontakte für /n / und /g/.}
	\label{figure:0601}
\end{figure}



\begin{figure}[p]
	\includegraphics[width=\textwidth]{figures/6-2_keineAssimilation.png}
	\caption{Zeitlicher Verlauf der Zungen-Gaumenkontakte in <hingeben>. Der alveolare Verschluss für die Sequenz /n\#g/ ist vollständig ausgeführt.}
	\label{figure:0602}
\end{figure}


\begin{figure}[p]
	\includegraphics[width=\textwidth]{figures/6-3_partielle_Assimilation.png}
	\caption{Zungen-Gaumenkontakte in <hingeben>. Es finden sich Kontakte oben an den Seiten der vorderen Reihen, die auf eine residuale alveolare Geste verweisen.}
	\label{figure:0603}
\end{figure}

\begin{figure}[t]
	\includegraphics[width=\textwidth]{figures/6-4_Substitution.png}
	\caption{Zungen-Gaumenkontakte in <hingeben>. Es ist kein alveolarer Verschluss für die /n\#g/-Sequenz aufgebaut worden.}
	\label{figure:0604}
\end{figure}

Die Abbildungen~\ref{figure:0602}, \ref{figure:0603} und \ref{figure:0604} schematisieren die verschiedenen Strategien der \isi{Assimilation} am Beispiel der /n\#g/-Sequenz im \il{Deutsch}deutschen Partikelverb <hiNGeben> mit den oralen Konstriktionsgesten {TT \textit{closure alveolar}} für /n/ und {TB \textit{closure velar}} für /g/. In Abbildung~\ref{figure:0602} findet keine \isi{Assimilation} statt, und die alveolare \isi{Geste} wird vollständig realisiert. Für /n/ kann ein alveolares Plateau von Frame 173--200 in der oberen Reihe lokalisiert werden. Es folgt in der unteren Reihe ein velares Plateau für /g/ (Frame 201-230), das während des alveolaren Verschlusses aufgebaut wird (ab Frame 195). Die velare Reihe ist nicht vollständig geschlossen, da der Laut weiter hinten -- außerhalb des EPG Messbereichs -- gebildet wird.



In Abbildung~\ref{figure:0603} wird eine partielle \isi{Assimilation} vorgenommen, bei der die alveolare \isi{Geste} in der /n\#g/-Sequenz geschwächt ist. Es findet kein vollständiger alveolarer Kontakt statt. Dennoch zeigt sich ein erhöhter lateraler Kontakt in den vorderen drei Reihen (Frame 344--367), der als Residuum für einen intendierten alveolaren Verschluss interpretiert werden kann \citep[][381]{Ellis2002}. Diese residuale \isi{Geste} für /n/ überlappt stark mit dem velaren Verschluss für /g/ (für /g/ vgl. Frame 347-375).


\largerpage
Abbildung~\ref{figure:0604} zeigt eine Realisierung ohne alveolare \isi{Geste} für /n/. Es findet sich nur ein velarer Verschluss für die gesamte /n\#g/-Sequenz (Frame 284--310). Diese Realisierung kann als vollständige \isi{Assimilation} (Substitution) gewertet werden. Es sei hier jedoch kritisch angemerkt, dass es sich auch um ein Artefakt des Messverfahrens handeln kann. So kann das EPG nur Vollkontakte und keine Annäherungen für Bewegungsausführungen erfassen. Eine intendierte alveolare \isi{Geste}, die stark hypoartikuliert würde, könnte durch das Raster hindurchfallen. Hierzu sind elektromagnetische Aufnahmen aufschlussreich \citep{Jaeger2007}.




In den aufgeführten Fällen führt eine Zunahme des Überlappungsgrades zwischen Gesten zu den unterschiedlichen Reduktions- und Assimilationsformen. Die \isi{Modellierung} ist quantitativer Natur und kann vielfältige Aussprachevariationen ohne zusätzlichen Regelapparat generieren.

\subsection{Gesturale Überlappung und Fokusstruktur}
\label{subsec:060202}

\citet{Mücke2008c} untersuchen in ihrer EPG-Studie den Einfluss von Akzent und der damit verbundenen \isi{Fokusstruktur} auf den Grad der \isi{Assimilation} im Deutschen. Eine vergleichbare Studie mit ähnlicher Ausrichtung findet sich bei \citet{Bergmann2012}.

Bei den Sprechern in \citet{Mücke2008c} handelt es sich um vier weibliche Sprecher (KA, JM, AH, DM) nördlich und einem männlichen Sprecher (PB) südlich der Benrather Isoglosse, im Alter zwischen 23 und 38 Jahren. Das Sprachmaterial bestand aus heterosilbischen N\#C-Sequenzen in Partikelverben wie in <hiNGeben> und <hiNKommen> mit einer Morphem- und Silbengrenze zwischen den beiden Konsonanten. In den Stimuli entsprach N einem alveolaren Nasal und C einem velaren Plosiv (N\#g, N\#k).

Um zu elizitieren, welche Teile einer Äußerung fokussiert sind und welche nicht, wurden Frage-Antwort-Paare als fokuskontrollierender Kontext verwendet \citep[vgl.][]{Uhmann1991, Wagner2012, Krifka2008, Culicover1983, Büring2003}. Zielwörter waren entweder im Hintergrund der Äußerung und nicht akzentuiert, oder im Vordergrund der Äußerung und akzentuiert. Dabei wurden zwei Akzent-Konditionen getestet: enger \isi{Fokus} und kontrastiver \isi{Fokus}. Die Studie umfasst 1440 Mini-Dialoge, von denen ein Auszug aus der Analyse von 300 Teststimuli im Folgenden für Partikelverben mit /n\#g/-Sequenzen gezeigt wird (5 Sprecher x 2 Targetwörter x 10 Wiederholungen x 3 Fokusstrukturen). Beginn und Ende der konsonantischen Plateaus in den /n\#g/ wurden unter Verwendung sprecherspezifischer Kontaktprofile nach \citet{Byrd1996b} annotiert.

Drei Messungen werden im Folgenden im Zusammenhang mit der Frage nach der artikulatorischen Markierung der \isi{Fokusstruktur} in den N\#C-Sequenzen diskutiert:
\begin{enumerate}[(i)]
	\item alveolare Plateaudauer als Intervall zwischen Start des alveolaren Plateaus für N und Offset,
	\item velare Plateaudauer als Intervall zwischen Start des velaren Plateaus für C und Offset und
	\item das Überlappungsintervall zwischen N und C, wobei der Offset des alveolaren Plateaus vom Start des velaren Plateaus subtrahiert wird (negative Werte zeigen an, dass es sich um eine zeitliche \isi{Überlappung} der Bewegungsausführungen handelt).
\end{enumerate}

Abbildung~\ref{figure:0605} zeigt separat für die fünf Sprecher die Ergebnisse für die Plateaudauermessungen. Auf der x-Achse ist die Dauer des alveolaren und velaren Plateaus und auf der y-Achse die verschiedenen Fokusstrukturen Hintergrund (HG), enger \isi{Fokus} (eF) und kontrastiver \isi{Fokus} (kF) abgetragen. In Tabelle~\ref{table:0601} sind die zugehörigen Mittelwerte für die alveolaren Plateaudauern und die \isi{Überlappung} zwischen alveolarem und velarem Plateau aufgelistet.


\begin{figure}[t]
	\includegraphics[width=\textwidth]{figures/6-5_Barplots_AssiEPGfokus.png}
	\caption{Koordination des alveolaren (weiß) und velaren Plateaus (grau) für N\#C-Sequenzen separat für die einzelnen Sprecher und Fokusstrukturen (HG = Hintergrund, eF = enger Fokus, kF = kontrastiver Fokus).}
	\label{figure:0605}
\end{figure}



\begin{table}[b]
\small
  \begin{tabularx}{\textwidth}{Q R{1.5cm}R{1cm}R{2cm} R{1.5cm}R{1cm}R{2cm}c} 
  \lsptoprule
	  & \multicolumn{3}{c}{\textbf{Alveolare Plateaudauer} (${ms}$)} & \multicolumn{3}{c}{ \textbf{Überlappungsintervall} (${ms}$)}\\ 
	  & \textbf{Hintergrund} & \textbf{Enger Fokus} & \textbf{Kontrastiver Fokus} & \textbf{Hintergrund} & \textbf{Enger Fokus} & \textbf{Kontrastiver Fokus}\\
	  \midrule
	  \textbf{DM} & 0 & 0 & 114 & NA & NA & 7 \\
	  \textbf{AH} & 85 & 95 & 101 & -8 & -22 & -14 \\
	  \textbf{JM} & 55 & 78 & 88 & -7 & -26 & -24 \\
	  \textbf{KA} & 37 & 86 & 85 & -25 & -45 & -40 \\
	  \textbf{PB} & 74 & 95 & 99 & -29 & -46 & -49 \\
	  \lspbottomrule
  \end{tabularx} 
  \caption{Mittelwerte für alveolare Plateaudauern und Überlappungsintervalle der alveolaren und velaren Plateaus in den N\#C-Sequenzen, separat für die verschiedenen Fokusstrukturen und Sprecher (NA = not available, fehlender Wert).}
  \label{table:0601}		
\end{table}

Die letzte Sprecherin (DM) zeigt eine Stärkung der alveolaren \isi{Geste} nur im kontrastiv-korrektiven \isi{Fokus}. Hier führt sie eine volle alveolare Konstriktion mit einer gemittelten Plateaudauer von $114$\,ms aus. In den Konditionen enger \isi{Fokus} und Hintergrund substituiert sie die alveolare \isi{Geste} für N durch einen velaren Verschluss und es gibt in diesen Fällen keine alveolaren Plateaudauern. 


Die ersten vier Sprecher -- KA, JM, AH und PB -- hingegen zeigen eine Stärkung der alveolaren \isi{Geste} in \isi{prosodisch} starken Positionen. Trägt das Zielwort den \isi{Nuklearakzent} (enger und kontrastiv \isi{Fokus} zusammen), so ist die Realisierung des alveolaren Plateaus im Mittel für diese vier Sprecher um  länger als in nichtakzentuierter Position (Hintergrund). Gleichzeitig verkürzt sich in den \isi{prosodisch} starken Positionen enger \isi{Fokus} und kontrastiver \isi{Fokus} (eF und kF) die Dauer des Überlappungsintervalls um $16{ms}$. Im Vergleich zur nichtakzentuierten Position (Hintergrund) werden in \isi{prosodisch} starker Position (enger \isi{Fokus} und kontrastiver \isi{Fokus} zusammen) die alveolaren Plateaudauern vergrößert und der Grad der \isi{Überlappung} zwischen Zungenspitzengeste und Zungenrückengeste verringert. 



Ein Vergleich zwischen engem und kontrastivem \isi{Fokus} zeigt jedoch keine weiteren Abstufungen. Es ist möglich, dass beide Fokustypen sehr prominent sind und auf ähnliche Weise artikulatorisch markiert werden. Da wir die artikulatorischen Kontaktmuster untersucht haben, sind die Modifikationen hauptsächlich der Strategie der \isi{Hyperartikulation} zuzuordnen: In prominenter Position werden die Konsonanten stärker verschliffen. Sie überlappen stärker miteinander und es zeigt sich eine Hypoartikulation.

\section{Prominenzgrade: Eine EMA-Studie}
\label{sec:0603}

Die folgenden Daten basieren auf einer EMA-Studie von \citet{Mücke2014b}. Dabei sind \isi{Bewegungsmuster} von Zielwörtern in vier unterschiedlichen Fokuskonditionen (Hintergrund, weiter \isi{Fokus}, enger \isi{Fokus}, kontrastiver \isi{Fokus}) untersucht worden. Hier ist also ein weniger prominenter Akzent, der weite \isi{Fokus}, in die Studie mit einbezogen worden. Anders als bei \citet{Mücke2014b} liegt im Folgenden das Hauptaugenmerk auf dem methodischen Teil. So werden zunächst die Hauptergebnisse der Studie unter Verwendung von Akustik und Lippenkinematik referiert. Es folgt ein Vergleich von unterschiedlichen Steifheitsberechnungen sowie der Darlegung des Problems der Messung der \isi{Zwischenlippendistanz} (Lip Aperture) bei gerundeten Vokalen.

Die Studie verdeutlicht in Anlehnung an \citet{Mücke2014b}, dass die \isi{Fokusstruktur} zu tonalen und artikulatorischen Markierungen führen kann. Ein weniger prominentes aber akzentuiertes Wort (z.\,B. weiter \isi{Fokus}, bei dem die Fokusdomäne größer als das Zielwort ist), beinhaltet laryngale Modifikationen (es wird ein \isi{Tonakzent} platziert), aber nicht notwendigerweise orale Modifikationen (z.\,B. der Lippen oder des Kiefers). Prosodische Stärkung kann verschiedene Prominenzstufen ausdrücken, die mit unterschiedlichen Fokusstrukturen einhergehen \citep[fokale \isi{Prominenz}, vgl.][]{Beckman2010} statt einfach -- wie bislang angenommen --  konkomitant zur Akzentuierung zu sein. Zusätzlich zeigt sich, dass bei gerundeten Vokalen die \isi{Zwischenlippendistanz} weniger gut die Modifikationen auf der horizontalen Ebene, das Vorstülpen der Lippen, erfasst und Positionskurven hier weiterhelfen können. Außerdem kann der Parameter \isi{Steifheit}, der mit der Oszillationsfrequenz der Bewegung im Feder-Masse-Modell assoziiert ist, unterschiedlich gemessen werden (nur temporal als Beschleunigungsintervall oder räumlich-zeitlich als Verhältnis von \isi{Maximalgeschwindigkeit} und \isi{Auslenkung}). Die Messungen haben eine unterschiedliche Granularität. So scheint die rein zeitliche Messung weniger sensitiv zu sein, ist aber andererseits bei der Implementierung der Ergebnisse im Feder-Masse-Modell insofern von Vorteil, als dass sie Trunkierungen einer \isi{Geste} durch eine andere erfassen kann und diese Trunkierungsformen gerade im Hintergrund (also in nicht-akzentuierter Position) eine große Rolle spielen.

\subsection{Methode der EMA-Studie}
\label{subsec:060301}
\largerpage
Für das Experiment wurden fünf Sprecher des Standarddeutschen im Alter zwischen 22 und 37 Jahren aufgenommen, darunter drei weibliche (F1, F2 und F3) und zwei männliche Sprecher (M1 und M2). Drei Sprecher stammen gebürtig aus dem fränkischen Sprachgebiet (F1, F3 und M1) und zwei Sprecher (F2 und M1) aus der westniederdeutschen Region. 

Alle akustischen und kinematischen Aufnahmen sind im Labor des I\textit{f}L Phonetik an der Universität zu Köln durchgeführt worden. Die kinematischen Daten wurden mittels eines 2-D elektromagnetischen Artikulographen (Carstens AG100, 10 Kanäle) mit einer Abtastrate von 500 Hz aufgezeichnet. Für die Erfassung der Artikulationsbewegungen wurden zwei Sensoren auf dem oberen und unteren Lippenrand auf mittsagittaler Ebene platziert. Weitere Sensoren wurden auf Zunge (Zungenspitze, -blatt und -rücken) und am Kinn befestigt. Zwei Sensoren (Nasenwurzel, obere Schneidezähne) dienten als Referenz, um Kopfbewegungen zu erfassen und aus dem Gesamtdatensatz herauszurechnen. Für die Rotation der mittsagittalen Ebene wurde am Ende jeder Aufnahmesitzung eine Messung der Bissebene durchgeführt. Um die kinematischen Daten zu glätten, wurden sie auf 200 Hz heruntergerechnet und mit einem 40 Hz Tiefpassfilter gefiltert. Aus den Trajektorien der Mundlippensensoren (obere und untere Lippe) wurde für die spätere Analyse die \isi{Zwischenlippendistanz} berechnet \citep[Lip-Aperture-Index,][]{Byrd2000a}.

Es wurden vier unterschiedliche Fokusstrukturen mittels Frage-Antwort-Paa\-ren getestet (vgl. die Beispiele in~\ref{ex:0611}, \ref{ex:0612}, \ref{ex:0613} und \ref{ex:0614}). Dabei kam das Zielwort entweder im Hintergrund (nicht-fokussiert) oder im weiten, engen bzw. kontrastiven \isi{Fokus} vor. Bei den Zielwörtern handelte es sich um zweisilbische, fiktive Namen nach Dr. (/dɔktɐ/), die jeweils einen der folgenden vier gespannten Langvokale des Deutschen in der betonten \isi{Silbe} enthält /iː, aː, oː, uː/ (<Bieber, Bahber, Bohber, Buhber>). Insgesamt wurden 560 Zielwörter (4 Zielwörter x 7 Wiederholungen x 4 Fokuskonditionen x 5 Sprecher) aufgezeichnet. Die Analyse beschränkt sich auf die Zielwörter mit den Vokalen /iː, aː, oː/. Zielwörter mit /uː/ wurden von der Analyse ausgenommen, da die lokalen Wendepunkte in der Lippenkinematik aufgrund der geringen Lippenöffnung nur schwer identifizierbar waren.

%%(6.11)
\begin{exe}
	\ex Weiter \isi{Fokus}:\label{ex:0611}
	\sn F: Was gibt´s Neues?
	\sn A: [Melanie will Dr. Bahber treffen]\textsubscript{Fokus}
\end{exe}

%%(6.12)
\begin{exe}
	\ex Enger \isi{Fokus}:\label{ex:0612}
	\sn F: Wen will Melanie treffen?
	\sn A: Melanie will [Dr. Bahber]\textsubscript{ Fokus} treffen.
\end{exe}

%%(6.13)
\begin{exe}
	\ex Kontrastiver \isi{Fokus}:\label{ex:0613}
	\sn F: Will Melanie Dr. Werner treffen?
	\sn A: Melanie will [Dr. Bahber]\textsubscript{ Fokus} treffen.
\end{exe}

%%(6.14)
\begin{exe}
	\ex Hintergrund:\label{ex:0614}
	\sn F: Will Norbert Dr. Barber treffen?
	\sn A: [Melanie]\textsubscript{ Fokus} will Dr. Bahber treffen.
\end{exe}

Die Annotation der tonalen Daten fand unabhängig von der Annotation der segmentalen Daten statt. Es wurde die Software PRAAT \citep{praat2010} verwendet. Bei den tonalen Landmarken wurden die akustischen Wellenformen (Oszillogramm) und die zugehörigen F0-Konturen verwendet. Die akzentuierten Zielwörter wurden im Hinblick auf drei verschiedene Tonakzente nach GToBI Standard, H+!H*, H* and L+H*, klassifiziert (\citealt{Grice2005}; Richtlinien befinden sich online unter \url{http://www.gtobi.uni-koeln.de}). Die Abbildungen~\ref{figure:0606}, \ref{figure:0607} und \ref{figure:0608} zeigen Screenshots einzelner Sprecher für die Tonakzente H!H*, H* und L+H*.

\begin{figure}[p]
	\includegraphics[width=.9\textwidth]{figures/6-6_MelanieF0.png}
	\caption{Screenshot mit (von oben nach unten) Oszillogramm und F0-Verlauf für die Äußerung <Melanie will Dr. Bieber treffen> von Sprecher M1. Das Zielwort <BIEber> ist im weiten Fokus und der nukleare Tonakzent als H!H* klassifiziert.}
	\label{figure:0606}
\end{figure}


\begin{figure}[p]
	\includegraphics[width=.9\textwidth]{figures/6-7_Melanie_F0_2.png}
	\caption{Screenshot mit (von oben nach unten) Oszillogramm und F0-Verlauf für die Äußerung <Melanie will Dr. Bieber treffen> von Sprecherin F3. Das Zielwort <BIEber> ist im engen Fokus und der nukleare Tonakzent als H* klassifiziert.}
	\label{figure:0607}
\end{figure}


\begin{figure}
	\includegraphics[width=.9\textwidth]{figures/6-8_Melanie_F0_3.png}
	\caption{Screenshot mit (von oben nach unten) Oszillogramm und F0-Verlauf für die Äußerung <Melanie will Dr. Bieber treffen> von Sprecherin M1. Das Zielwort <BIEber> ist im engen Fokus und der nukleare Tonakzent als L+H* klassifiziert.}
	\label{figure:0608}
\end{figure}


In allen Fällen folgte dem nuklearen \isi{Tonakzent} ein Grenzton, L-L\%. Zielwörter, die nicht akzentuiert waren, wurden mit ‘Ø’ markiert. 

Die segmentalen Daten wurden in der Software EMU \citep{Cassidy2001} auf akustischer und kinematischer Ebene manuell annotiert. Auf akustischer Ebene wurden jeweils das Oszillogramm und zugehöriges Breitbandsonagramm verwendet. Es wurden jeweils Anfang und Ende der Zielwörter (<Bieber>, <Bahber>, <Bohber> und Zielsilben (/bi:/, /ba:/ und /bo:/) bestimmt und dann die zugehörigen akustischen Wort- und Silbendauern berechnet. Die Wortdauer entspricht in diesen Fällen den Fußdauern. Für die kinematische Analyse wurden die Bewegungen der Lippen annotiert. Hierzu wurde nach \citet[6]{Byrd2000a} der Index für die \isi{Zwischenlippendistanz} berechnet (Lip-Aperture-Index), bei dem die euklidische Distanz zwischen unterer und oberer Lippe kalkuliert wird, vgl. Gleichung~\ref{eqn:06_17}.

%(6.17)  
\begin{align}
\label{eqn:06_17}
\text{Lippenöffnung} = \sqrt{\left(x_{\text{Lippenöffnung}}\right)^2 + \left(y_{\text{Lippenöffnung}}\right)^2}\\ \cline{1-2}
x_{\text{Lippenöffnung}} = x_{\text{Oberlippe}} - x_{\text{Unterlippe}} \nonumber \\
y_{\text{Lippenöffnung}} = y_{\text{Oberlippe}} - y_{\text{Unterlippe}} \nonumber%
\end{align}

\begin{figure}[b]
	\includegraphics[width=\textwidth]{figures/6-9_Screenshot_Label_Arti.png}
	\caption{Screenshot mit (von oben nach unten) Oszillogramm, Geschwindigkeits- und Positionskurve für die Zwischenlippendistanz. Niedrige Werte in der Positionskurve verweisen auf geschlossene Lippen im Konsonanten und hohe Werte auf geöffnete Lippen im Vokal. Das Aktivierungsintervall der Öffnungsgeste ist rosa hervorgehoben mit den Landmarken Onset, Maximalgeschwindigkeit (pVel) und Target.}
	\label{figure:0609}
\end{figure}

Es wurden Landmarken für die \isi{Öffnungsgeste} der Lippen in den Zielsilben /ba:/, /bi:/ and /bo:/ gesetzt (vgl. auch Kapitel~\ref{sec:0402}). Die Lippenöffnungsgeste bezeichnet die Bewegung der Lippen vom maximalen Verschluss im initialen Konsonanten bis zur maximalen Öffnung im folgenden \isi{Vokal}. Start- und Zielpunkt (\isi{Onset} und \isi{Target}) der Bewegung sind anhand der Geschwindigkeitskurve (erste Ableitung der Positionskurve) bestimmbar, vgl. Abbildung~\ref{figure:0609}. Beim Start und beim Ziel des Aktivierungsintervalls für die \isi{Öffnungsgeste} (farblich hervorgehobene Box) beträgt die Geschwindigkeit der Lippenbewegung Null. Eine weitere Landmarke ist die \isi{Maximalgeschwindigkeit} der \isi{Öffnungsgeste} (peak velocity, \isi{pVel}). Diese ist anhand der Beschleunigungskurve (zweite Ableitung der Positionskurve) bestimmbar. Ist die \isi{Maximalgeschwindigkeit} der Bewegung erreicht, beträgt deren Beschleunigung Null.



Basierend auf den Landmarken \isi{Onset}, \isi{pVel} und \isi{Target} wurden die folgenden fünf Messvariablen (a-d) verwendet:

\begin{enumerate}[(a)]
	\item Die maximale Dauer der \isi{Öffnungsgeste} vom \isi{Onset} bis zum \isi{Target} der Bewegung, in ms.
	\item Die Amplitude (maximale \isi{Auslenkung}) der \isi{Öffnungsgeste} vom \isi{Onset} bis zum \isi{Target} der Bewegung, in mm.
	\item Die \isi{Maximalgeschwindigkeit} (\isi{pVel}) der \isi{Öffnungsgeste}, in mm/s.
	\item Die \isi{Steifheit} (Time-To-Peak-Intervall) der \isi{Öffnungsgeste} als zeitliches Intervall vom Start der Bewegung bis zur \isi{Maximalgeschwindigkeit} \citep{Cho2002a, Cho2006, Byrd1998}.
\end{enumerate}
  

 
\subsection{Ergebnisse der tonalen Analyse}
\label{subsec:060302}

Die Abbildung~\ref{figure:0610} zeigt die \isi{Verteilung der Tonakzente} im Hinblick auf die unterschiedlichen Fokusstrukturen für alle Sprecher. Diese drei Fokuskonditionen weiter, enger und kontrastiver \isi{Fokus} werden mit einem \isi{Tonakzent} markiert, die Kondition Hintergrund dagegen nicht.


\begin{figure}
	\includegraphics[width=\textwidth]{figures/6-10_Barplots_Akzente.png}
	\caption{Verteilung der Tonakzente im Hinblick auf die verschiedenen Fokuskonditionen, gemittelt über alle Sprecher.}
	\label{figure:0610}
\end{figure}


Es zeigt sich darüber hinaus ein Zusammenhang zwischen dem verwendeten \isi{Tonakzent} und der zu markierenden \isi{Fokusstruktur} \citep{Mücke2016}. So tritt L+H* überwiegend im weiten \isi{Fokus} auf. H+!H* tritt im weiten und engen, jedoch nicht im kontrastiven \isi{Fokus} auf. H* hingegen wird über alle drei Fokuskonditionen verwendet.

Die Sprecher nutzen jedoch unterschiedliche Strategien. So zeigt die Tabelle~\ref{table:0602} die \isi{Verteilung der Tonakzente} für die einzelnen Sprecher. Sprecher F1 und F2 beispielsweise verwenden unterschiedliche Tonakzente, um zwischen weitem (H+!H*) und kontrastivem \isi{Fokus} (L+H*) zu unterscheiden. Sprecher F3 und M2 verhalten sich jedoch anders. Beide verwenden H* stärker über die verschiedenen Fokuskonditionen hinweg und variieren nicht so stark in der Auswahl der verwendeten Tonakzente. So verwendet Sprecherin F3 den H*-Akzent auch im kontrastiven \isi{Fokus} und verzichtet ganz auf eine Markierung durch L+H*.


\begin{table} 
		\begin{tabularx}{\textwidth}{lcSSSS} \lsptoprule
			\multicolumn{2}{c}{} & \textbf{Hintergrund} & \textbf{Weit} & \textbf{Eng} & \textbf{Kontrastiv} \\ \midrule
			\multirow{4}{*}{\textbf{F1}} & ‘Ø’ & 100\% &  &  & \\
										& H+!H* &  & 95\% & 14\% & \\
										& H* &  & 5\% & 24\% & 5\% \\
										& L+H* &  &  & 62\% & 95\% \\ \cline{1-6}
			\multirow{4}{*}{\textbf{F2}} & ‘Ø’ & 100\% &  &  & \\
										& H+!H* &  & 100\% & 50\% & \\
										& H* &  &  & 20\% & \\
										& L+H* &  &  & 30\% & 100\% \\ \cline{1-6}
			\multirow{4}{*}{\textbf{F3}} & ‘Ø’ & 100\% &  &  & \\
										& H+!H* &  & 10\% &  & \\
										& H* &  & 90\% & 100\% & 100\% \\
										& L+H* &  &  &  & \\ \cline{1-6}
			\multirow{4}{*}{\textbf{M1}} & ‘Ø’ & 100\% &  &  & \\
										& H+!H* &  & 95\% &  & \\
										& H* &  & 5\% & 24\% & 1\% \\
										& L+H* &  &  & 76\% & 81\% \\ \cline{1-6}
			\multirow{4}{*}{\textbf{M2}} & ‘Ø’ & 100\% &  &  & \\
										& H+!H* &  &  &  & \\
										& H* &  & 100\% & 61\% & 11\% \\
										& L+H* &  &  & 39\% & 89\% \\ \lspbottomrule
		\end{tabularx} 
		\caption{Sprecherspezifische Verteilung der Tonakzente im Hinblick auf die verschiedenen Fokuskonditionen (Hintergrund, weiter Fokus, enger Fokus, kontrastiver Fokus), alle Angaben in Prozent (leere Zellen = 0~\%); ‘Ø’ = kein Akzent.}
		\label{table:0602}
\end{table}

\begin{figure}
% 	\includegraphics[width=.8\textwidth]{figures/6-11_F0_Umfang.png}
	  \begin{tikzpicture}
    \begin{axis}[
	xlabel={Fokusstruktur},  
	ylabel={Halbtöne}, 
	ymax=13,
	axis lines*=left, 
        width  = \textwidth,
	height = .3\textheight,
    	nodes near coords, 
	xtick=data,
	ytick={4,8,12},
	x tick label style={},  
	ymin=0,
	symbolic x coords={weit,eng,kontrastiv}
	]
	\addplot+[ybar,lsRichGreen!80!black,fill=lsRichGreen] plot coordinates {
	(weit,8.3)
	(eng,9.0)
	(kontrastiv,10.5)
	}; 
    \end{axis} 
  \end{tikzpicture}  
	\caption{Tonhöhenumfang in Halbtönen vom nuklearen Gipfel, H*, bis zur tiefen Grenze, L\%, für Sprecherin F3 \citep[adaptiert von][]{Krüger2009}.}
	\label{figure:0610a}
\end{figure}



Eine detaillierte Analyse der tonalen Korrelate findet sich in \citet{Krüger2009} und \citet{Grice2017}. Es sei hier aber angemerkt, dass einige Sprecher, wie Sprecherin F3, hauptsächlich einen Tonakzenttypen über die unterschiedlichen Fokusbedingungen produzieren. Betrachtet man jedoch die akustischen Korrelate wie \isi{Tonhöhenumfang} und zeitliche \isi{Alignierung} (Synchronisation von Sprechmelodie und Text) genauer, so finden sich systematische Variationen in Abhängigkeit von der \isi{Fokusstruktur} auch innerhalb der Akzentkategorien. Die Abbildungen~\ref{figure:0610a} und~\ref{figure:0610b} zeigen die Korrelate für Sprecherin F3, die ausschließlich den pragmatisch neutralen Akzenttyp H* über die unterschiedlichen Fokusstrukturen hinweg produziert hat. Die erste Abbildung~\ref{figure:0610a} zeigt den \isi{Tonhöhenumfang} (peak height) vom nuklearen Gipfel bis zur tiefen Grenze. Es zeigt sich, dass der \isi{Tonhöhenumfang} systematisch von weitem und engen \isi{Fokus} hin zum kontrastiven \isi{Fokus} zunimmt (t-Testung: weiter \isi{Fokus} = enger \isi{Fokus} < kontrastive \isi{Fokus}, $p\;\leq\;0.05$). Im kontrastiven \isi{Fokus} ist der \isi{Tonhöhenumfang} um 2,2 Halbtöne größer als im weiten \isi{Fokus}.




Die Abbildung~\ref{figure:0610b} zeigt das zeitliche Alignierungsmuster für den nuklearen Gipfel, H*, relativ zum Beginn des akustischen Vokalbeginns. Solche Alignierungsmuster zählen zur segmentalen Ankerhypothese, die davon ausgeht, dass Wendepunkte in der F0-Kontur zeitgleich mit akustischen Segmentgrenzen auftreten, weil sie dort verankert sind (vgl. Kapitel~\ref{chap:07}). 

\begin{figure} 
% 	\includegraphics[width=.8\textwidth]{figures/6-12_F0_Alignierung.png}
		  \begin{tikzpicture}
    \begin{axis}[
	xbar,
	xlabel={ms},  
	ylabel={},  
	axis lines*=left, 
        width  = \textwidth,
	height = .3\textheight,  
	ytick=data,
    	nodes near coords, 
	y tick label style={},  
	xmin=0,
	symbolic y coords={weit,eng,kontrastiv}
	]
	\addplot+[xbar,lsRichGreen!80!black,fill=lsRichGreen] plot coordinates {
	(27.1,weit)
	(59.4,eng)
	(78.0,kontrastiv)
	}; 
    \end{axis} 
  \end{tikzpicture}  
	\caption{Zeitliches Alignment für den nuklearen Gipfel (H*) relativ zum Beginn des Vokals in der Akzentsilbe für Sprecherin F3\\ \citep[adaptiert von][]{Krüger2009}.}
	\label{figure:0610b}
\end{figure}


Positive Werte indizieren, dass in allen Konditionen der Gipfel (gemessen als F0-Wendepunkt) zeitlich nach dem Beginn des akzentuierten Vokals auftritt, das sogenannte \enquote{peak delay}. Im Hinblick auf die \isi{Fokusstruktur} zeigt sich, dass der Gipfel durchschnittlich 32 ms später im engen als im weiten \isi{Fokus} und 19 ms später im kontrastiven als im engen \isi{Fokus} auftritt (t-Testung: weiter \isi{Fokus} < enger \isi{Fokus} < kontrastiver \isi{Fokus}, $p\;\leq\;0.05$). Sowohl höhere als auch spätere Gipfel (gemessen als Peak Height und Peak Delay) werden in der Regel von Hörern als besonders prominent wahrgenommen \citep{Gussenhoven2004}.

\subsection{Ergebnisse der supralaryngalen Analyse}
\label{subsec:060303}

Für die statistische Analyse wurde die Statistiksoftware \textit{R} \citep{Rcite} verwendet. Es wurden Varianzanalysen mit Messwiederholung über alle Datenpunkte durchgeführt, die auf Zellenmittelwerten basierten (overall ANOVAs; repeated measures). Als unabhängige Variablen wurden \isi{Fokusstruktur} (Hintergrund, weiter \isi{Fokus}, enger \isi{Fokus}, kontrastiver \isi{Fokus}) sowie Vokalqualität (/aː/, /iː/, /oː/) mit den einzelnen Sprechern als Zufallsfaktor getestet. Als abhängige Variable dienten jeweils die verschiedenen akustischen und artikulatorischen Messungen (akustisch: Fuß- und Silbendauern; artikulatorisch: Dauer, Amplitude, \isi{Maximalgeschwindigkeit}, \isi{Steifheit} der \isi{Öffnungsgeste}). Für die Post-Hoc-Testung wurde eine Reihe von Tukeys HSD verwendet. Bei der artikulatorischen Steifheitsmessung wird zunächst auf das Time-to-Peak Intervall genommen. In Kapitel~\ref{subsec:060304} werden dann die beiden gängigen Steifheitsmessungen (vgl. Kapitel~\ref{chap:04}), das Time-to-Peak Intervall und die \isi{Steifheitsberechnung} nach \citet{Munhall1985} auf ihre Leistungsfähigkeit hin verglichen.


\begin{figure}[b]
	\includegraphics[width=\textwidth]{figures/6-13_Results_akustik_fokus.png}
	\caption{Mittelwerte und zugehörige Standardfehler für die akustischen Messungen (a) Wortdauern und (b) Silbendauern für die Fokuskonditionen Hintergrund, weiter Fokus, enger Fokus, kontrastiver Fokus und die Vokalkonditionen  B/aː/ber, B/iː/ber, B/oː/ber, gemittelt über alle Sprecher.}
	\label{figure:0611}
\end{figure}

Zunächst werden die akustischen Dauern dargestellt. Abbildung~\ref{figure:0611} zeigt Mittelwerte und Standardfehler für die Wort- und Silbendauern auf akustischer Ebene. Die Analyse zeigt einen systematischen Effekt des Faktors \isi{Fokusstruktur} auf Wortdauern [F(3; 12) = 20,44; $p\;\leq\;0.05$] und Silbendauern [F(3; 12) = 19,05;  $p\;\leq\;0.05$]. Mit steigendem Grad der \isi{Prominenz} werden die Wort- und Silbendauern länger. Vergleicht man Hintergrund mit den anderen Fokuskonditionen weiter \isi{Fokus}, enger \isi{Fokus} und kontrastiver \isi{Fokus}, so zeigt sich ein systematischer Unterschied zwischen den maximal divergierenden Fokusstrukturen: zwischen Hintergrund und kontrastivem \isi{Fokus} steigen die Wortdauern um durchschnittlich 47 ms und die Silbendauern um 31 ms an. Darüber hinaus steigen die durchschnittlichen Fußdauern zwischen Hintergrund und engem \isi{Fokus} um 27 ms und die Silbendauern um 17 ms an. Der Vergleich von weniger stark divergierenden Fokusstrukturen führt jedoch zu gänzlich anderen Resultaten: Zwischen Hintergrund und weitem \isi{Fokus} finden sich keine Effekte der \isi{Fokusstruktur} auf die Messungen von Wort- und Silbendauern (post-hoc Testung: Hintergrund = weiter \isi{Fokus} < enger \isi{Fokus} < kontrastiver \isi{Fokus}, $p\;\leq\;0.05$). Vergleicht man die akzentuierten Konditionen untereinander, so zeigt sich ein systematischer Anstieg von weitem \isi{Fokus} über engen \isi{Fokus} bis hin zum kontrastiven \isi{Fokus} (post-hoc Testung: weiter \isi{Fokus} < enger \isi{Fokus} < kontrastiver \isi{Fokus}, $p\;\leq\;0.05$).

\begin{figure}[t]
	\includegraphics[width=\textwidth]{figures/6-14_Results_artiku_fokus.png}
	\caption{Mittelwerte und zugehörige Standardfehler für die kinematischen Messungen der Öffnungsgeste (a) maximale Dauer, (b) Amplitude, (c) Maximalgeschwindigkeit und (d) Steifheit (Time-To-Peak-Intervall) für die Fokuskonditionen Hintergrund, weiter Fokus, enger Fokus, kontrastiver Fokus und die Vokalkonditionen B/aː/ber, B/iː/ber, B/oː/ber, gemittelt über alle Sprecher.}
	\label{figure:0612}
\end{figure}


Im Folgenden werden die artikulatorischen EMA-Daten betrachtet, also eine kinematische Analyse hinzugezogen. Abbildung~\ref{figure:0612} zeigt Mittelwerte und Standardfehler für die kinematischen Messungen, gemittelt über alle Sprecher. Es zeigt sich ein signifikanter Haupteffekt der \isi{Fokusstruktur} auf alle vier Messungen der \isi{Öffnungsgeste}: Dauer [F(3, 12) = 16,49; p < 0,001]; Amplitude [F(3, 12) = 9,445; p < 0,01], \isi{Maximalgeschwindigkeit} [F(3, 12) = 5,441; p < 0,05] und \isi{Steifheit} (zunächst gemessen als Time-To-Peak-Intervall: [F(3, 12) = 5,756; p < 0,05]). Vergleicht man Hintergrund mit den anderen Fokuskonditionen weiter \isi{Fokus}, enger \isi{Fokus} und kontrastiver \isi{Fokus}, so zeigen sich für die maximal divergierenden Fokusstrukturen längere, größere und schnellere Bewegungen. Zwischen Hintergrund und kontrastivem \isi{Fokus} findet ein systematischer Anstieg in der Dauer um durchschnittlich 27 ms, in der Amplitude um 4 mm und in der \isi{Maximalgeschwindigkeit} um 52 mm/s statt, während die \isi{Steifheit} hingegen unverändert bleibt. Beim Vergleich von Hintergrund und engem \isi{Fokus} finden sich ebenfalls längere und größere Bewegungen (die Dauern steigen im Durchschnitt um 11 ms und die Auslenkungen um 2 mm an), jedoch bleiben \isi{Steifheit} (gemessen als Time-To-Peak-Intervall) und \isi{Maximalgeschwindigkeit} gleich. Anders verhält es sich beim Vergleich weniger stark divergierender Fokusstrukturen: Zwischen Hintergrund und weitem \isi{Fokus} erreicht keine der kinematischen Messungen Signifikanz ($p\;\geq\;0.05$, ns).

 
Vergleicht man die Fokuskonditionen (weiter \isi{Fokus}, enger \isi{Fokus} und kontrastiver \isi{Fokus}, siehe Abbildung~\ref{figure:0612} untereinander, so finden sich ebenfalls unterschiedliche Effekte der \isi{Fokusstruktur} auf die kinematischen Messungen. So steigen Dauern und Amplitude systematisch in allen Fällen vom weiten über engen bis hin zum kontrastiven \isi{Fokus} ($p\;\leq\;0.05$; post-hoc Testung: weiter \isi{Fokus} < enger \isi{Fokus} < kontrastiver \isi{Fokus}). Die \isi{Maximalgeschwindigkeit} hingegen steigt nur dann an, wenn weiter und kontrastiver \isi{Fokus} verglichen werden, aber nicht beim Vergleich von engem \isi{Fokus} mit den jeweils benachbarten Fokusstrukturen, weiter und kontrastiver \isi{Fokus} ($p\;\leq\;0.05$). Die \isi{Steifheit} steigt beim Vergleich von weitem und kontrastivem \isi{Fokus} ($p\;=\;0,0394$), jedoch nicht zwischen den anderen Fokuskonditionen (weiter versus enger \isi{Fokus}; enger versus kontrastiver \isi{Fokus}, $p\;\leq\;0.05$).

\clearpage 
Abbildung~\ref{figure:0613} zeigt gemittelte Bewegungstrajektorien für die Lippenöff\-nungs\-bewegung, aufgeschlüsselt nach den einzelnen Sprechern (F1, F2, F3, M1, M2), Zielwort (B/aː/ber, B/iː/ber, B/oː/ber) und Fokuskondition. Alle Trajektorien sind mit dem akustischen Anfang des Zielwortes aligniert. Die x-Achse kodiert die Zeit in ms und die y-Achse kodiert die \isi{Bewegungsauslenkung} (\textit{displacement}). Große Werte auf der y-Achse indizieren, dass die Lippen während des Vokals geöffnet sind. Die Bewegungsverläufe zeigen nicht nur, dass alle Sprecher ihre supralaryngale \isi{Artikulation} in Abhängigkeit der Fokuskondition modifizieren, sondern auch, dass es sprecherspezifische Unterschiede gibt. F1 und M1 zeigen im Vergleich zu Sprecher F2, F3 und M2 durchweg stärkere Anpassungen in der räumlichen Dimension, d.\,h. größere Auslenkungen der Bewegungsamplituden im Hinblick auf die Markierung der Fokuskonditionen. Während die Unterschiede im Grad der räumlichen \isi{Bewegungsauslenkung} tendenziell am größten in B/aː/ber und am kleinsten in B/oː/ber sind, zeigen alle Zielwörter vergleichbare Dauermodifikationen (Gesamtdauer der \isi{Öffnungsgeste} vom Start bis zum \isi{Target}).



\begin{figure}
	\includegraphics[width=\textwidth]{figures/6-15_Averaged_Trajcetories.png}
	\caption{Gemittelte Bewegungstrajektorien (Background = Hintergrund, broad = weiter Fokus, narrow = enger Fokus, contrastive = kontrastiver Fokus).}
	\label{figure:0613}
\end{figure}


Die stärksten Unterschiede in den Trajektorien zeigen sich, wie erwartet, zwischen den maximal divergierenden Fokusstrukturen, Hintergrund und kontrastiver \isi{Fokus}. Im kontrastiven \isi{Fokus} produzieren vor allem Sprecher F1, F2 und M1 längere, größere und schnellere Bewegungen. Die anderen beiden Sprecher, F3 und M2, zeigen über die Konditionen hinweg durchweg weniger konsistente Artikulationsmuster. Betrachtet man die Trajektorien bei weniger stark divergierenden Fokusstrukturen wie Hintergrund und weiter \isi{Fokus}, so finden sich kaum Unterschiede in den artikulatorischen Mustern. Das ist interessant, denn in der Kondition „weiter \isi{Fokus}“ sind die Zielwörter akzentuiert und im Hintergrund nicht. Ähnlich verhält es sich, wenn man die Trajektorien von Hintergrund und engem \isi{Fokus} vergleicht, wo zumindest drei von fünf Sprechern (F2, F3 und M2) keine Unterschiede in der Bewegungsausführung zeigen.

 
Eine detaillierte Analyse zum Einfluss von sprecherspezifischer Variation und Vokalqualität auf die \isi{artikulatorische Markierung} findet sich in \citet{Mücke2014b}. Hier reicht zunächst die Feststellung aus, dass:

\begin{enumerate}[(i)]
	\item offene Vokale stärker markiert sind als die geschlossenen Vokale, da sie eine geringe \isi{koartikulatorische Resistenz} aufweisen \citep[vgl. auch][]{Harrington2000, Tabain2003a, Tabain2003b}. Es gibt zwar einen Einfluss des segmentalen Materials auf die prosodische Markierung von Vokalen, aber auch geschlossenen Vokale werden -- wenn auch in einem geringeren Masse -- markiert.
	\item Alle Sprecher markieren artikulatorisch, aber der Grad der Markierung sowie die Strategie variieren. Einige Sprecher wie F1 und M1 machen räumliche und zeitliche Modifikationen in den Lippenbewegungen, um unterschiedliche Grade der \isi{Prominenz} abzubilden, während andere wie F2, F3 und M2 nur zeitliche Modifikationen vornehmen.
	\item Zwischen Hintergrund und weitem \isi{Fokus} finden kaum Modifikationen in der \isi{Artikulation} statt, obwohl die letztere Kondition akzentuiert ist und die erstere nicht. Demgegenüber zeigen stark divergierende Fokusstrukturen wie Hintergrund versus kontrastiver \isi{Fokus} oder Hintergrund versus enger \isi{Fokus} klare Unterschiede in der Lippenöffnung.
	\item Da wir die Lippenkinematik bereits betrachtet haben, sind die Modifikationen hauptsächlich der \isi{Sonoritätsexpansion} und somit der Stärkung des phonologischen Merkmals [+sonorant] zuzuordnen \citep[vgl.][]{DeJong1995, Harrington2000, Cho2005a}.
\end{enumerate}


\subsection{Vergleich unterschiedlicher Steifheitsberechnungen}
\label{subsec:060304}

Wie in Kapitel~\ref{sec:0401} aufgeführt gibt es unterschiedliche Berechnungen, um von dem physiologischen Signal auf den abstrakten Steifheitsparameter rückschließen zu können. In der Studie wurde zunächst das Time-to-Peak-Intervall verwendet, das als reiner Zeitparameter die \isi{Beschleunigungsphase} der gestischen Aktivierung beschreibt. Eine weitere \isi{Steifheitsberechnung}, die häufig in der Literatur verwendet wird, basiert auf einer spatio-temporalen Messung nach \citet{Munhall1985}. Hier wird das Verhältnis zwischen \isi{Maximalgeschwindigkeit} (\textit{pVel}) und Amplitude wie folgt berechnet: 

%%(6. 26)     
\begin{align}
\label{eqn:06_26}
\text{Steifheit} \;k\; = \frac{\text{pVel}\left(\frac{mm}{ms}\right)}{\text{Amp}\left(mm\right)}
\end{align}

Im Folgenden wird der Einfluss der Fokusstrukur auf die beiden unterschiedlichen Steifheitsberechnungen Time-To-Peak-Intervall und \isi{pVel}/Amplitude verglichen. Die folgende Analyse konzentriert sich auf die über alle Sprecher gemittelten Werte. Bei der Varianzanalyse aller Datenpunkte (Varianzanalyse mit Messwiederholung basierend auf Zellmittelwerten mit den unabhängigen Faktoren \isi{Fokusstruktur} und Vokalqualität und Sprecher als Zufallsfaktor) zeigt sich ein Haupteffekt der \isi{Fokusstruktur} auf beide Steifheitsberechnungen (Time-To-Peak-Intervall [F(3; 12) = 5,756; $p\;\leq\;0.05$]; \isi{pVel}/Amplitude [F(3; 60) = 8,674; $p\;\leq\;0.001$]). In Abbildung~\ref{figure:0614} sind Mittelwerte und Standardfehler gemittelt über alle Sprecher graphisch dargestellt. Bei der Darstellung der \isi{Steifheit} als Time-to-Peak Intervall in (a) ist die y-Achse so belassen wie in Abbildung~\ref{figure:0612}, um eine Vergleichbarkeit mit den vorherigen Analysen zu gewährleisten.



\begin{figure}
	\includegraphics[width=\textwidth]{figures/6-16_Steifheitsberechnungen.png}
	\caption{Mittelwerte und zugehörige Standardfehler für die zwei unterschiedlichen Steifheitsberechnungen der Öffnungsgeste (a) Time-To-Peak-Velocity (in $ms$) und (b) pVel/Amplitude, für die Fokuskonditionen Hintergrund, weiter Fokus, enger Fokus, kontrastiver Fokus und die Vokalkonditionen B/aː/ber, B/iː/ber, B/oː/ber), gemittelt über alle Sprecher.}
	\label{figure:0614}
\end{figure}


Beim Vergleich der maximal divergierenden Fokusstrukturen (Hintergrund und kontrastiver \isi{Fokus}) findet sich ein Fokusstruktureffekt auf die \isi{Steifheitsberechnung} \isi{pVel}/Amplitude, ($p\;\leq\;0,001$), die vom Hintergrund zum kontrastivem \isi{Fokus} um 1,9 abnimmt. Auch zwischen Hintergrund und engem \isi{Fokus} nimmt die \isi{Steifheit} \isi{pVel}/Amplitude ab ($p\;\leq\;0,05$); hier sinkt sie um 1,3 vom Hintergrund zum engen \isi{Fokus}. Beim Vergleich von Hintergrund und weitem \isi{Fokus} hingegen bleibt sie unverändert ($p\;\geq\;0,05$). Anders verhält es sich beim rein temporalen Steifheitsintervall (Time-To-Peak), das in allen Konditionen des Akzentvergleichs (Hintergrund und kontrastiver \isi{Fokus}, Hintergrund und enger \isi{Fokus}, Hintergrund und weiter \isi{Fokus}) gleichbleibt (Time-To-Peak;$p\;\geq\;0,05$).

Beim Vergleich zwischen weitem, engem und kontrastivem \isi{Fokus} zeigt sich ein systematischer Abfall der \isi{Steifheit} \isi{pVel}/Amplitude für den Vergleich sowohl von weitem und kontrastivem \isi{Fokus} ($p\;\leq\;0,05$), als auch von weitem und engem \isi{Fokus} ($p\;\leq\;0,05$), d.\,h. die \isi{Steifheit} sinkt um 1,9 vom weiten zum kontrastiven \isi{Fokus} und um 1,3 vom weiten zum engen \isi{Fokus}. Zwischen engem und kontrastivem \isi{Fokus} jedoch bleibt die \isi{Eigenfrequenz} unverändert ($p\;\geq\;0,05$). Für die rein temporale Steifheitsmessung, dem Time-To-Peak-Intervall, ergeben sich systematische Unterschiede beim Vergleich von weitem und kontrastivem \isi{Fokus} ($p\;\leq\;0,05$), d.\,h. hier vergrößert sich das Time-To-Peak-Intervall um durchschnittlich 6~ms vom weiten zum kontrastiven \isi{Fokus}. Beim Vergleich der benachbarten Fokuskonditionen hingegen (enger und weiter \isi{Fokus} sowie enger und kontrastiver \isi{Fokus}) bleibt das Time-To-Peak-Intervall gleich ($p\;\geq\;0,05$).

 
Es zeigt sich für die analysierten Daten, dass es sich bei den beiden Steifheitsberechnungen um unterschiedlich sensitive Messtechniken handelt. Die \isi{Steifheit} \isi{pVel}/Amplitude deckt mehr Unterschiede zwischen den einzelnen Prominenzgraden auf. Sie hat den Vorteil, dass sie neben der zeitlichen Ebene auch die räumlichen Modifikationen berücksichtigt. Die \isi{Steifheit} Time-To-Peak-Intervall hingegen bezieht nur die zeitliche Ebene mit ein, indem sie die Dauer der \isi{Beschleunigungsphase} der Bewegung zu Grunde legt (vgl. \citealt{Cho2002a, Cho2006, Byrd1998}). Beiden Messungen liegt die Annahme zu Grunde, dass ein Absenken der \isi{Steifheit} -- also ein Absenken der Oszillationsfrequenz im Feder-Masse-Modell -- im physiologischen Signal -- mit oder ohne Einbeziehung der räumlichen Komponente zu langsameren Bewegungen führt. Es gibt jedoch einen Vorteil bei der \isi{Modellierung}, wenn man die Messung Time-To-Peak-Intervall verwendet, auch wenn sie in den Daten weniger sensitiv war. Das Time-To-Peak-Intervall kann die \isi{Trunkierung} einer \isi{Geste} durch eine andere aufdecken. Ändert sich die Phase zwischen zwei Gesten und somit deren gestischer Überlappungsgrad, kann es zur früheren Ablösung einer \isi{Geste} durch eine andere kommen. Als Konsequenz ändern sich im physiologischen Signal zwar die Gesamtdauern und Amplituden der \isi{Geste}, nicht aber deren \isi{Maximalgeschwindigkeit} und Beschleunigungsintervall (Time-To-Peak). Eine Ausnahme bilden starke Reduktionen, bei denen eine \isi{Geste} bereits vor dem Erreichen ihrer \isi{Maximalgeschwindigkeit}, also noch während der \isi{Beschleunigungsphase}, trunkiert wird. Demgegenüber gibt die Steifheitsmessung \isi{pVel}/Amplitude keinen Aufschluss darüber, ob die gefundenen Variationen aus längeren und größeren Bewegungsauslenkungen auf unterschiedliche \isi{Koordinationsmuster} zwischen Gesten (mit zunehmender prosodischer Stärke nehmen Trunkierungsphänomene ab) oder auf Reskalierungsprozesse innerhalb einer \isi{Geste} (mit zunehmender prosodischer Stärke werden Bewegungen räumlich und zeitlich größer skaliert) rückführbar ist.

\subsection{Der Parameter Lippenrundung bei Zielwörtern mit /oː/}
\label{subsec:060305}

Die \isi{Zwischenlippendistanz} (Lip-Aperture-Index) basiert auf der Berechnung des Euklidischen Abstands zwischen den beiden Sensoren auf der Unter- und Oberlippe; diese Abstandsberechnung bezieht die vertikalen (y-Position) und horizontalen (x-Position) Lippenbewegungen in einem zweidimensionalen Raum ein. Es hat sich im Datensatz gezeigt, dass Zielsilben mit gerundeten Vokalen geringere räumliche Modifikationen aufweisen als solche mit ungerundeten Vokalen, vgl. Abbildung~\ref{figure:0613}.

Bei gerundeten Vokalen drückt sich das phonologische Merkmal [+gerundet] nur indirekt in der \isi{Zwischenlippendistanz} aus. Für die Lippenrundung findet aktiv eine horizontale Vorwärtsbewegung der Ober- und Unterlippe statt. Deshalb kann für die gerundeten Vokale eine getrennte Betrachtung der Dimensionen x-Position (Lippenrundung) und y-Position sinnvoll sein: eventuelle Differenzierungen in artikulatorischen Mustern könnten bei einer relativen Abstandsmessung wie der \isi{Zwischenlippendistanz} weniger klar sichtbar sein als bei der Betrachtung der absoluten Positionen.

\begin{figure}[p]
	\includegraphics[width=.8\textwidth]{figures/6-17_xy_Plots.png}
	\caption{Gemittelte Trajektorien der Unterlippe für die Öffnungsgeste in /boː/, einzeln für Sprecher und Fokuskonditionen, zeitnormalisert über 20 gleichabständige Datenpunkte pro Trajektorie. x-Achse = horizontale Position (niedrige Werte indizieren eine stärkere Vorwärtsbewegung); y-Achse = vertikale Position (niedrige Werte indizieren eine stärkere Öffnungsbewegung).}
	\label{figure:0615}
\end{figure}

Im Folgenden werden für das Zielwort B/oː/ber die Trajektorien der unteren Lippe nach horizontalen und vertikalen Bewegungsmustern aufgeschlüsselt. In Abbildung~\ref{figure:0615}~(a-d) sind die gemittelten Trajektorien der unteren Lippe für die \isi{Öffnungsgeste} einzeln für jeden Sprecher und jede Fokuskondition dargestellt. Der Start der \isi{Trajektorie} ist mit dem kinematischen Beginn der \isi{Öffnungsgeste} in /boː/ (maximaler Verschluss im initialen Konsonanten) und das Ende der \isi{Trajektorie} mit dem \isi{Target} der \isi{Öffnungsgeste} (maximale Öffnung im \isi{Vokal}) aligniert. Jede \isi{Trajektorie} ist zeitnormalisiert über 20 gleichabständige Datenpunkte (vgl. auch \citealt{Tabain2003b}). Auf der y-Achse ist die vertikale Position der Unterlippe abgebildet (oben/unten). Dabei indizieren hohe Werte den labialen Verschluss im initialen Konsonanten und niedrige Werte die transvokalische Öffnung im \isi{Vokal}. Die x-Achse zeigt die horizontale Position der Unterlippe (vorne/hinten). Niedrige Werte indizieren eine stärkere Vorwärtsbewegung der Unterlippe, die mit einer stärkeren Rundung gleichgesetzt werden kann, während hohe Werte auf eine Rückwärtsbewegung verweisen.




Es handelt sich bei den Graphiken um eine starke Vergrößerung der Zielregion der Lippenrundungsgeste: Während der gesamten dargestellten \isi{Trajektorie} der transvokalischen \isi{Öffnungsgeste} in /boː/ ist die Unterlippe -- wenn auch in einem unterschiedlichen Grad -- bereits vorwärts gestülpt. Tatsächlich ist bei den Sprechern die Vorwärtsbewegung der Unterlippe am stärksten zu Beginn der \isi{Öffnungsgeste} (im konsonantischen Verschluss) und nimmt dann zur maximalen Vokalöffnung hin leicht ab. Dieses Phänomen deutet auf antizipatorische \isi{Koartikulation} hin, denn das Maximum für die Vorwärtsbewegung der Unterlippe in /oː/ wird bereits im vorangehenden initialen labialen Plosiv /b/ erreicht.



Sprecher F1 und M1 zeigen bezüglich der supralaryngalen Differenzierung der Fokusstrukturen große Unterschiede in der Öffnung (y-Achse) und der Vorwärtsbewegung der Unterlippe (x-Achse). So ist die Unterlippe im kontrastiven \isi{Fokus} deutlich weiter offen (y-Achse) und vorne als im Hintergrund (x-Achse). Des Weiteren produziert Sprecherin F1 als Zwischenkategorie engen und weiten \isi{Fokus}, die jedoch bezüglich der erreichten Lippenrundung zusammenfallen. Sprecher M1 produziert für engen \isi{Fokus} Vorwärtsbewegungen, die zwischen Hintergrund und kontrastivem \isi{Fokus} liegen, während weiter \isi{Fokus} reduziertere Bewegungen als Hintergrund aufweist. Diese Ergebnisse zeigen, dass die Sprecher F1 und M1 distinkte Artikulationsmuster zur Markierung der \isi{Fokusstruktur} nutzen.

Ähnlich verhält es sich bei Sprecherin F2, die jedoch tendenziell kleinere Unterschiede in der Vorwärtsbewegung der Unterlippe zur Markierung der \isi{Fokusstruktur} macht. Dabei überschneiden sich bei ihr jeweils die Trajektorien für Hintergrund und weiten \isi{Fokus} sowie für engen und kontrastiven \isi{Fokus}. 

Anders verhält es sich bei den Sprechern F3 und M2, die bei der Betrachtung der absoluten Positionskurven mehr Unterschiede bezüglich der Markierung der \isi{Fokusstruktur} zeigen als dies bei einer reinen Betrachtung der \isi{Zwischenlippendistanz} der Fall wäre (vgl. \citealt{Mücke2014b}). Sprecherin F3 produziert eine deutlich größere Lippenöffnung im kontrastiven und engen \isi{Fokus} verglichen mit Hintergrund oder weitem \isi{Fokus}, aber die Vorwärtsbewegung der Unterlippe ist entgegen der Erwartung stärker im Hintergrund und weitem \isi{Fokus} als im engen und kontrastiven \isi{Fokus}. Sprecher M2 produziert kleinere, aber dennoch gleichmäßige Unterschiede in der \isi{Bewegungsauslenkung} der Unterlippe auf vertikaler und horizontaler Achse für weiten, engen und kontrastiven \isi{Fokus}, während die \isi{Trajektorie} für die Hintergrundkondition starke Überschneidungen mit den anderen Trajektorien aufweist. Wie bei F3 führt bei M1 eine Betrachtung der absoluten vertikalen und horizontalen Positionskurven zu klaren Ergebnissen, um die supralaryngale Markierung der \isi{Fokusstruktur} bei Zielwörtern mit gerundeten Vokalen zu erfassen. 

Es zeigt sich, dass Unterschiede in der Markierung der \isi{Fokusstruktur} bei der relativen Abstandberechnung der \isi{Zwischenlippendistanz} verdeckt sein können. So können bei gerundeten Vokalen mehr Differenzierungen in den Positionskurven (vertikale und horizontale Bewegung der Unterlippe) auftreten als in der \isi{Zwischenlippendistanz}. Diese Beobachtung stimmt mit den Eingabevoraussetzungen im Task-Dynamic-Modell überein, bei dem die \isi{Traktvariablen} des oralen Systems in Paaren auf zwei Beschreibungsdimensionen des virtuellen vertikal-horizontalen Vokaltraktes aufgeteilt sind und in LA (Lip Aperture) und LP (Lip Protrusion) unterschieden werden, vgl. Kapitel~\ref{subsec:010201}. Es empfiehlt sich bei gerundeten Vokalen, neben der \isi{Zwischenlippendistanz} eine Analyse der absoluten Positionskurven durchzuführen, um Prosodische Stärke zu erfassen.

\subsection{Diskussion und Implementierung}
\label{subsec:060306}

Es zeigt sich, wie in \citet{Mücke2014b} dargelegt, dass die Markierung der prosodischen Struktur nicht auf die Dichotomie akzentuiert -- unakzentuiert beschränkt ist, sondern sich weitaus komplexer gestaltet. Die prosodische Stärke ist direkt an die \isi{Fokusstruktur} selbst und somit an \isi{Prominenz} gebunden. Das wird erst deutlich, wenn sich der Vergleich der supralaryngalen Markierung nicht nur auf maximal divergierender Fokusstrukturen (Hintergrund und kontrastiver \isi{Fokus}) beschränkt, sondern auch geringere Prominenzgrade wie weiter \isi{Fokus} in die Analyse mit einbezieht.

Für die Akzentuierung in Form der tonalen Markierung zeigen die Daten wie erwartet, dass alle Sprecher Tonakzente in weitem, engen und kontrastivem \isi{Fokus} produzieren, während nicht-fokussierte Zielwörter ohne \isi{Tonakzent} produziert werden. Aus Sicht der Intonation gibt es eine klare Distinktion zwischen fokussierten (akzentuierten) und nicht-fokussierten (nicht akzentuierten) Konditionen und bei der Betrachtung der Tonakzente über die verschiedenen fokussierten Konditionen zeigt sich auch ein Unterschied in ihrer Verteilung. So finden sich L+H*-Akzente überwiegend in kontrastivem \isi{Fokus} und H+!H* in weitem \isi{Fokus}. Dennoch zeigen sich hier auch sprecherspezifische Strategien, und einige Sprecher produzieren H* auch in weitem oder kontrastivem \isi{Fokus} und verzichten beispielsweise auf die Markierung von weitem \isi{Fokus} durch einen L+H*. Bei näherer Betrachtung wird jedoch deutlich, dass sich die Realisierungen innerhalb einer Akzentkategorie quantitativ unterscheiden, beispielsweise bezüglich \isi{Tonhöhenumfang} und zeitlichem Alignment (vgl. auch \citealt{Grice2017}).

Für die \isi{artikulatorische Markierung} (supralaryngale \isi{Artikulation}) gibt es keine klare Trennung zwischen fokussierten (akzentuierten) und nicht-fokussierten (nichtakzentuierten) Zielwörtern. Die supralaryngale \isi{Artikulation} ist nicht einfach konkomitant mit der Akzentuierung, sondern drückt vielmehr direkt die \isi{Fokusstruktur} -- und somit \isi{Prominenz} -- aus.

Sowohl für die akustischen Messungen (Wort- und Silbendauern) als auch für die kinematischen Parameter (Dauer, Amplitude, \isi{Maximalgeschwindigkeit}) zeigt sich eine klare Distinktion, wenn stark divergierende Fokuskonditionen wie Hintergrund und kontrastiver \isi{Fokus} untersucht werden. Diese Ergebnisse stimmen mit den Ergebnissen vieler kinematischer und akustischer Studien überein \citep[u.a.][]{Beckman1992, DeJong1993, DeJong1995, Harrington1995, Harrington2000, Cho2005a, Cho2006, Dohen2005, Dohen2006, Baumann2006, Avesani2007, Hermes2008a, Kügler2008}.

Ein vergleichbares Ergebnis stellt sich für den Vergleich von Hintergrund und engem \isi{Fokus} ein, wenngleich hier keine klaren Unterschiede bezüglich des Parameters \isi{Maximalgeschwindigkeit} der Lippenbewegung feststellbar waren. Beim Vergleich von Hintergrund und weitem \isi{Fokus} wird deutlich, dass es hier keine Markierung durch das supralaryngale System gibt. Wenngleich weiter \isi{Fokus} \isi{tonal} markiert ist und die Hintergrund-Kondition nicht, finden sich aus segmentaler Perspektive weder auf akustischer noch auf artikulatorischer Ebene Unterschiede zwischen Hintergrund und weitem \isi{Fokus}. Demnach findet keine direkte Anpassung des supralaryngalen Systems an die Akzentkondition statt.

Beim Vergleich zwischen weitem und engem \isi{Fokus} zeigt sich ein systematischer Anstieg in den akustischen Wort- und Silbendauern sowie in den kinematischen Parametern der \isi{Öffnungsgeste} Dauer und Amplitude. Beim Vergleich von engem \isi{Fokus} mit den benachbarten Fokuskonditionen (enger versus weiter \isi{Fokus} und enger versus kontrastiver \isi{Fokus}) sind die Ergebnisse ähnlich wie in der EPG-Studie in Kapitel~\ref{sec:0602} weit weniger klar. Insbesondere hier zeigt sich ein sehr hohes Maß an sprecherspezifischer \isi{Variabilität}, die auf keine klare Trennung in der Realisierung zwischen den beiden phonologischen Kategorien hinweist. Diese \isi{sprecherspezifische Variation} reflektiert die unterschiedlichen und teilweise widersprüchlichen Ergebnisse in der Forschungsliteratur. So fanden beispielsweise \citet{Eady1986} beim Vergleich von engem \isi{Fokus} mit weitem \isi{Fokus} im Englischen keine akustischen Unterschiede. Derselbe Vergleich, weiter und enger \isi{Fokus}, führte jedoch bei \citet{Breen2010} zu systematischen Wortdauer- und Intensitätsunterschieden im Englischen und bei \citealt{Fery2008,Kügler2008}, sowie \citet{Baumann2006} zu längeren Wortdauern im Deutschen. Des Weiteren lässt sich keine klare Trennung zwischen engem und kontrastivem \isi{Fokus} ziehen, weil angenommen werden kann, dass enger \isi{Fokus} einen Kontrast mit alternativen Formen impliziere \citep{Krifka2008}.

Die Unterschiede in der supralaryngalen \isi{Artikulation} können im Hinblick auf die Markierung der verschiedenen Fokuskonditionen wie folgt als Parameter in einem Feder-Masse-Modell modelliert werden: Beim Vergleich von engem \isi{Fokus} mit Hintergrund finden sich Unterschiede in Dauer und Amplitude, während \isi{Maximalgeschwindigkeit} und \isi{Steifheit} (gemessen als Time-To-Peak-Intervall) unverändert bleiben. Diese Variation (längere und größere, aber nicht schnellere oder weniger steife Bewegungen deutet auf Veränderungen im Grad der koartikulatorischen \isi{Überlappung} zweier Gesten hin, und weniger auf die \isi{Reskalierung} innerhalb einer \isi{Geste}. Im Hintergrund ist die \isi{Öffnungsgeste} durch eine frühere Aktivierung der Verschlussgeste trunkiert (Abbildung~\ref{figure:0616}). \isi{Trunkierung} wird von \citealt[][305]{Harrington1995} als eine wichtige Strategie im Australischen Englisch in der Realisierung unakzentierter Zielwörter versus kontrastiver \isi{Fokus}. In ihrer Untersuchung haben sie die Trajektorien für die Öffnungs- und die Verschlussgeste des Kiefers in /bVb/ Silben gemessen. Für die Klärung, ob es sich jeweils um \isi{Reskalierung} oder \isi{Trunkierung} der Kiefergesten handelt, ziehen sie weitere Parameter hinzu, die die Form der \isi{Trajektorie} zwischen den beiden Maximalgeschwindigkeiten der Öffnung und des Verschlusses in die Analyse mit einbeziehen. Diese Form verläuft im Falle der \isi{Trunkierung} spitzer (die \isi{Öffnungsgeste} wird ja von der Verschlussgeste „abgeschnitten“) als bei der \isi{Reskalierung}. Sie stellen abschließend fest: ‘This study shows that the accented/unaccented differences are more appropriately modelled as a consequence of truncation, than linear rescaling.’


\begin{figure}
	\includegraphics[width=.8\textwidth]{figures/6-18_schema_modifikation.png}
	\caption{Trunkierung der Bewegung, d.\,h. frühere Ablösung einer Geste durch eine andere Geste. Schematische Darstellung in Anlehnung an \citet[71]{Beckman1992} und \citet[17]{Cho2002a}.}
	\label{figure:0616}
\end{figure}


Beim Vergleich von engem \isi{Fokus} mit weitem und kontrastivem \isi{Fokus} zeigt sich, dass enger \isi{Fokus} im Hinblick auf die supralayrngalen Parameter eine Zwischenposition einnimmt. Gesten im engen \isi{Fokus} zeigen weniger koartikulatorische \isi{Überlappung} als im weiten \isi{Fokus} und mehr als im kontrastiven \isi{Fokus}. Vergleicht man jedoch Hintergrund mit kontrastivem \isi{Fokus} finden sich nicht nur Unterschiede in Dauer und Amplitude sondern auch in der \isi{Maximalgeschwindigkeit}. Diese Variationen lassen sich im Feder-Masse-Modell jedoch nicht mehr auf eine einzelne Parametervariation zurückführen, sondern verweisen auf multiple Parametermodifikationen (Abbildung~\ref{figure:0617}). Eine Veränderung im Grad der koartikulatorischen \isi{Überlappung} (vgl. Kapitel~\ref{sec:0401}) kann nicht den Anstieg in der \isi{Maximalgeschwindigkeit} erklären. Es ist wahrscheinlich, dass die erhöhte \isi{Maximalgeschwindigkeit} auf eine zusätzliche Modifikation der Zielspezifikation (einfache \isi{Parametermodifikation} des Targets) zurückzuführen ist. So hat \citet{DeJong1995} für Vokale im Englischen gezeigt, dass Veränderungen im zugrundeliegenden \isi{Target} die artikulatorischen Modifikationen plausibel erklären können (Anstieg in \isi{Maximalgeschwindigkeit} und Amplitude). Dies stimmt mit den Beobachtungen von \citet{Cho2006} überein, der für \isi{prosodische Stärkung} der supralaryngalen \isi{Artikulation} multiple Parametermanipulationen annimmt.

\begin{figure} 
	\includegraphics[width=.8\textwidth]{figures/6-19_Schema_Modfikation_2.png}
	\caption{Targetmodifikation wird bei manchen Sprechern mit der Trunkierung (siehe Abbildung~\ref{figure:0616}) kombiniert, um bei kontrastivem Fokus versus Hintergrund eine zusätzliche räumliche Modifikation als Ausdruck von Emphase bzw. prosodischer Prominenz zu erreichen. Schematische Darstellung in Anlehnung an \citealt[71]{Beckman1992} und \citealt[17]{Cho2002a}.}
	\label{figure:0617}
\end{figure}


Das zeigt sich beispielsweise in den Trajektorien der Sprecherinnen F1 und F2 im Zielwort  <Bahber> im Hintergurnd versus kontrastivem \isi{Fokus}. Die räumlichen Auslenkungen der Trajektorien von F1 im kontrastiven \isi{Fokus} in der \isi{Silbe} /ba:/ sind deutlich größer als bei Sprecherin F2 und können nur mit einer zusätzlichen Targetmodifikation als Ausdruck von Emphase abgebildet werden (Abbildung~\ref{figure:0618}).


\begin{figure} 
	\includegraphics[width=.8\textwidth]{figures/6-20_Averaged_trajectories.png}
	\caption{Trajektorien der Zwischenlippendistanz für Sprecherin F1 (Änderungen im Überlappungsgrad der transvokalischen Öffnungs- und Schließgesten mit zusätzlicher Targetmodifikation) und F2 (Änderungen im Überlappungsgrad der transvokalischen Öffnungs- und Schließgesten ohne zusätzliche Targetmodifikation) verdeutlichen die sprecherspezifische Variation; kontrastiver Fokus versus Hintergrund (out of focus).}
	\label{figure:0618}
\end{figure}


Zusammenfassend lässt sich bezüglich der Parametervariationen sagen, dass zwischen Hintergrund und engem \isi{Fokus} die intergesturale Koordination als wichtigste \isi{Parametermodifikation} auszureichend scheint. Im Hintergrund überlappen Gesten stärker als im engen \isi{Fokus}. Das stimmt auch mit den Beobachtungen der zuvor aufgeführten EPG-Studie im Deutschen überein (Kapitel~\ref{sec:0602}), bei der im Hintergrund konsonantische Gesten meist stark überlappen bzw. stark assimiliert sind. Bei Hintergrund versus kontrastivem \isi{Fokus} kann der Prominenzgrad jedoch so stark ansteigen, dass bei einigen Sprechern eine zusätzliche Targetmodifikation stattfindet, die zu einer deutlich größeren räumlichen \isi{Auslenkung} der Bewegung als Ausdruck von Emphase führt. Das scheint jedoch \isi{sprecherspezifisch} bedingt zu sein und findet sich nicht in den Realisierungen aller Sprecher wieder.