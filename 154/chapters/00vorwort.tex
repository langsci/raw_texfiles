\addchap{{Vorwort}}
\label{chap:00}
Die Artikulatorische Phonologie wurde als Alternative zu segmentalen Ansätzen entwickelt. So nimmt die segmentale Phonologie an, dass nur distinkte Information gespeichert wird, die dann mit Hilfe von Regeln und Rechenprinzipien von der kategorialen Welt der Symbole in die kontinuierliche Welt der physikalischen Repräsentation übersetzt wird (\citealt{Ohala1990}; \citealt[][2]{Gafos2006}; \citealt{Mücke2016}).
In segmentalen Ansätzen wird mittels einer Schnittstelle versucht, von der abstrakten symbolischen Repräsentation zum konkreten artikulatorischen und/oder akustischen Output eines Sprechers zu gelangen, d.h. es werden zwei unterschiedliche formalen Sprachen der Mathematik verwendet, von der jeweils eine der Phonologie und die andere der Phonetik zugeordnet wird. Dies führt jedoch zu Mehrdeutigkeiten in der Theoriebildung, vor allem was die Granularität der phonologischen Beschreibung angeht (\citealt{Trubeckoj1939}; \citealt{DeSaussure1916}; \citealt{Rischel1990}; \citealt{Pierrehumbert1990}; \citealt[][321]{Keating1990a}). 

Die Artikulatorische Phonologie hingegen nimmt an, dass auch kontinuierliche Information wie beispielsweise sprecher- oder situationsbezogen Variationen als Teil des Sprachsystems gespeichert werden. Natürliche Variabilität wird hier als Teil des linguistischen Systems betrachtet, das konkret Aufschluss über zugrundeliegende Strukturen gibt. Im Modell der Artikulatorischen Phonologie wird Sprache als dynamisches System betrachtet und somit phonetische und phonologische Information integriert (u.a. \citealt{Browman1986}, \citealt{Browman1988}, \citealt{Browman1991a}, \citealt{Fowler1977}; \citealt{Fowler1980}; \citealt{Saltzman1986}, \citealt{Browman1986}; \citealt{Saltzman1987};  \citealt{Kugler1987}; \citealt{Saltzman1989}; \citealt{Kelso1995}; \citealt{Gafos2006}). 

Die Grundeinheiten der Artikulatorischen Phonologie sind nicht Segmente oder Merkmale einer Sprache, sondern artikulatorische Gesten. Diese legen linguistische relevante Konstriktionen wie beispielsweise ein Vollverschluss der Zungenspitze an den Alveolen sowie eine glottale Öffnungsgeste für Stimmlosigkeit bei der Produktion von /t/ für ein definiertes Zeitintervall fest. Die Einbeziehung der zeitlichen Domäne ermöglicht im Gegensatz zu segmentalen Ansätzen die Abbildung natürlicher Variabilität. Sie kann beispielsweise im Falle von /t/ der Grad der Aspiration direkt aus der zeitlichen Anordnung der glottalen und oralen Geste abgeleitet werden: Ist die glottale Geste länger als die Zungenspitzengeste aktiviert, so entsteht auf akustischer Oberfläche Aspiration. Artikulatorische Gesten enkodieren darüber hinaus den kontextuellen Einfluss (Koartikulation in Form von Synergien zwischen Organgruppen) und können direkt den Einfluss höhere linguistischer Strukturen wie der prosodischen Hierarchie abbilden (\citealt{Shaw2011}, \citealt{Mücke2017}). So fällt der Grad der Aspiration von Plosiven in Sprachen wie dem Deutschen in prosodisch starken Positionen stärker aus als in schwachen Positionen, um diesen Äußerungsteil neben der tonalen Markierung durch einen Tonakzent auch artikulatorisch Prominenz zu verleihen. Es handelt sich dabei um ein komplexes Wechselspiel zwischen Artikulation und Prosodie, ein neues Forschungsfeld, dem man am besten mit einer quantitativen Modellierung in Form von dynamischen Systemen gerecht wird. 

Das vorliegende Buch stellt eine Einführung in die Artikulatorische Phonologie dar. Es richtet sich an Leser und Leserinnen, die phonetische Grundkenntnisse besitzen und sich mit der Artikulatorischen Phonologie beschäftigen möchten. Darüber hinaus werden neben einer Einführung in das Model auch neuere Arbeiten und aktuelle Weiterentwicklungen aufgezeigt, insbesondere die Implementierung prosodischer Aspekte in die Artikulatorische Phonologie betreffend. Somit eignet sich das Buch auch für Leser und Leserinnen, die bereits mit der Artikulatorischen Phonologie in Kontakt gekommen sind, aber ihr Wissen vertiefen möchten. Zur Veranschaulichung des Models werden Beispiele aus verschiedenen Sprachen gegeben, darunter \ili{Deutsch}, \ili{Katalanisch}, \ili{Italienisch}, \ili{Polnisch}, \ili{Mandarin} und \ili{Tashlhiyt} Berber.

Die ersten vier Kapitel vermitteln Grundlagen der Artikulatorischen Phonologie und der prosodischen Analyse. Es werden Artikulatorische Gesten auf der Basis des dynamischen Modells der Task Dynamics definiert (Kapitel~\ref{chap:01}). Anhand von Gestenpartituren werden verschiedene Bildungsformen für lexikalische Kontraste in der Artikulatorischen Phonologie exemplifiziert, sowie die grundlegenden Ordnungsprinzipien für die gestische Organisation vorgestellt, um Prozesse wie Reduktion, Assimilation und Tilgung quantitativ abbilden zu können (Kapitel~\ref{chap:02}). Des Weiteren werden gesturale Strukturen als Modell der Selbstorganisation vorgestellt. Mit Hilfe eines multiplen Netzwerks zeitlicher Triggern -- dem Modell der nichtlinearen paarweise gekoppelten Oszillatoren -- formieren sich Gesten als dynamisches System zu prosodischen Einheiten wie der Silbe (Kapitel~\ref{chap:03}). Es folgt eine Einführung in die Modellierungsparameter, die in experimentellen Studien im Rahmen der Artikulatorischen Phonologie verwendet werden (Kapitel~\ref{chap:04}). Diese werden anhand eines Beispiels eines Vergleichs von Artikulationsmustern mit ein- und ausgeschalteter Tiefenhirnstimulation in der klinischen Linguistik veranschaulicht. Es schließt sich eine Einführung in die prosodische Analyse mit Schwerpunkt auf der Markierung von Prominenz in der phonetischen Substanz an (Kapitel~\ref{chap:05}).

Es folgen zwei Anwendungsbereiche, die Artikulation und prosodische Struktur miteinander verbinden. Hier ist einmal die artikulatorische und tonale Markierung von Prominenz zu nennen (Kapitel~\ref{chap:06}). Zum anderen wird im Bereich der tonalen Alignierungsforschung aufgezeigt, wie Tonakzente mit artikulatorischen Gesten koordiniert sind (Kapitel~\ref{chap:07} ).Das Buch schließt mit einer englischen Zusammenfassung (Kapitel~\ref{chap:08}) und einer kritischen Diskussion des Models der Artikulatorischen Phonologie und dessen Verankerung in Forschung und Lehre (Kapitel~\ref{chap:09}).

\vspace*{2em}
\noindent \textbf{Funding Acknowledgements:}\\%
Diese Arbeit wurde unterstützt und gefördert von der Deutschen Forschungsgemeinschaft (DFG) im Rahmen des Sonderforschungsbereichs (SFB) 1252 \enquote{Prominenz in Sprache} (Projekt A04 \enquote{Dynamische Modellierung prosodischer Prominenz}) an der Universität zu Köln.