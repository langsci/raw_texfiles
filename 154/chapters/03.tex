\chapter{Selbstorganisation}
\label{chap:03}

Die Produktion von artikulatorischen Mustern verläuft nicht statisch. Vielmehr induzieren Faktoren wie \isi{prosodische Struktur} und segmentaler Kontext natürliche \isi{Variabilität}, die wiederum im Spannungsfeld zur Notwendigkeit von gesturaler Kohärenz für die Abbildung phonologischer Muster steht. Gesturale Kohärenz ist am stärksten innerhalb von lexikalischen Einheiten, d.\,h. sie sind innerhalb dieser Einheiten stabiler und weniger variabel koordiniert als zwischen verschiedenen Einheiten (vgl. \citealt{Saltzman2000}; \citealt{Goldstein2006}; \citealt{Goldstein2009}; \citealt{Yoon2011}). Die Prinzipien für gesturale Koordination können per Regel in der jeweiligen \isi{Gestenpartitur} festgelegt werden. Dabei werden meist invariante punktuelle Alignierungen zwischen unterschiedlichen Phasen der gesturalen Aktivierung festgelegt, beispielsweise einer Onset-zu-\isi{Target} Beziehung. Eine solche Onset-zu-\isi{Target} Beziehung meint das Interval zwischen dem Beginn einer \isi{Geste} und dem \isi{Target} einer anderen \isi{Geste} \citealt{Saltzman2000}). 

Möchte man jedoch natürliche \isi{Variabilität}, beispielsweise aufgrund von prosodischen Faktoren, \isi{Artikulationsrate} oder Perturbation, abbilden, so können relative Phasenbeziehungen zwischen Gesten nicht starr definiert sein. Vielmehr gilt es, eine beschränkte \isi{Variabilität} bei \isi{Modellierung} abzubilden \citep[vgl.][]{Saltzman2000,Goldstein2006,Nam2009a}. So gehen beispielsweise \citet{Byrd1996a} und \citet[][54]{Saltzman2000} statt von punktuellen (starren) Werten für eine \isi{Phasenbeziehung}, z.\,B. 220°, von einem Phasenfenster (phase window) aus, z.\,B. 220°-245°, in welches die relativen Phasenbeziehungen fallen können. Dieses Fenster bildet dann die Variabilitätsbeschränkung (variability constraint). Gesten, die zu einer lexikalischen Einheit gehören, haben kleinere Phasenfenster als solche, die zu unterschiedlichen Einheiten gehören. Diese Eigenschaften lassen sich mit den Eigenschaften gekoppelter Oszillatoren adäquat modellieren.

Einer statischen Punkt-zu-Punkt \isi{Alignierung} ist demnach eine dynamische selbstorganisierende \isi{Modellierung} vorzuziehen. Bei einer dynamischen \isi{Modellierung} wird die Verbindung von relativen Phasenbeziehungen zwischen Gesten unter Verwendung von nichtlinearen paarweise gekoppelten Oszillatoren abgebildet \citep[vgl.][]{Saltzman2000,Goldstein2006,Goldstein2009,Nam2009a,Nam2009b,Pouplier2011a}. Hierbei wird Kohärenz von Gesten durch Variabilitätsbeschränkungen, die aus den Schwingungseigenschaften gekoppelter Oszillatoren resultieren, impliziert und somit dem Spannungsfeld aus Stabilität/Kohärenz und Flexibilität/\isi{Variabilität} genüge getan. Diese Vorgehensweise stützt sich in ihren Grundlagen auf die Beobachtungen aus nicht-sprachlichen Bewegungsmustern von Gliedmaßen (Koordination von Armen, Beinen und Händen) und überführt diese als zeitliche Trigger (paarweise gekoppelte nichtlineare Oszillatoren) in die sprachlichen Task Dynamics. In der Artikulatorischen Phonologie finden sich diese dynamischen Koordinationsprinzipien bereits in frühester Stufe sprechmotorischer Planung. So enthält das Linguistische \isi{Gestenmodell} der Artikulatorischen Phonologie bereits strukturelle Informationen über dynamische Kopplungen zwischen Gesten und berücksichtigt diese bei der Generierung der Gestenpartituren einer Äußerung. 

\section{Bimanuelle Koordination}
\label{sec:0301}

Motorische Experimente zur Koordination von Gliedmaßen wie Fingern, Armen oder Beinen haben gezeigt, dass es zwei intrinsische Modi gibt, die Versuchspersonen ohne Lernen anwenden können: In-Phase und Anti-Phase \citep{Kelso1981,Kelso1984b,Turvey1990,Blaufuß2001}. Werden Versuchspersonen beispielsweise gebeten, mit den beiden Zeigefingern ihrer Hände rhythmisch hin und her zu wackeln, so finden sich zwei grundlegende Muster: Entweder werden die beiden Zeigefinger gleichzeitig/symmetrisch nach rechts und links bewegt, d.\,h. in einem In-Phase Modus (0° \isi{relative Phase}). Oder die Zeigefinger werden in entgegengesetzte Richtungen/asymmetrisch hin und her bewegt, d.\,h. in einem Anti-Phase-Modus (180° \isi{relative Phase}, Out-of-Phase). Weitere Phasenbeziehungen können durch nachhaltiges Training erlernt werden. Der In-Phase-Modus beinhaltet eine gleichzeitige Aktivität von homologen Muskeln beider Zeigefinger (Extensor und Flexor), während beim Anti-Phase-Modus die Kontraktion der Muskeln zwischen den Zeigefingern der rechten und liken Hand alternieren. Wird bei einem Anti-Phase-Modus durch die Verwendung eines Metronoms die Bewegungsfrequenz in einem Experiment erhöht, so findet bei einer kritischen Frequenz eine abrupte \isi{Phasenverschiebung} vom Anti-Phase-Modus zum In-Phase-Modus statt. Umgekehrt findet unter Geschwindigkeitserhöhung keine Transition vom In-Phase- zum Anti-Phase-Modus statt. Aufgrund dieser Beobachtungen gilt der In-Phase-Modus als stabiler. 

Die Beobachtungen zu intrinsischen Phasenmodi bei oszillierenden Handbewegungen wurden von \citet{Haken1985} (kurz HKB) in ein mathematisches Modell überführt. Es handelt sich beim HKB-Modell um ein nieder-dimensionales \isi{dynamisches System} der Selbstorganisation, das einfache Potentialfunktionen und Attraktoren verwendet \citep{Nam2009b}. In einem dynamischen System stellt ein \isi{Attraktor} einen stabilen Zustand dar, auf den sich das System über die Zeit hin zubewegt, nachdem es in Schwingung versetzt worden ist. Es handelt sich bei einem \isi{Attraktor} beispielsweise um einen bestimmter Wert (z.\,B. einen Phasenwert), der im jeweiligen Phasenraum definiert ist und dem sich die Systemvariablen annähern. Im Falle des HKB-Modells werden der In-Phase \isi{Attraktor} (0° \isi{relative Phase}) und der Anti-Phase \isi{Attraktor} (180° \isi{relative Phase}) spezifiziert. Auf Basis dieser Attraktoren bildet das HKB-Modell die folgenden vier Eigenschaften ab \citep[][348]{Haken1985}:

\begin{description}
	\item[Eigenschaft (1):] Es existieren nur zwei stabile Phasen (Attraktoren) zwischen den Händen (mit den Systemvoraussetzungen, dass im Experiment die Versuchspersonen gebeten werden, entweder mit der In-Phase oder der Anti-Phase Bewegung zu beginnen).
	\item[Eigenschaft (2):] Die abrupte \isi{Phasenverschiebung} von einem \isi{Attraktor} zu einem anderen findet bei einer kritischen Frequenz statt.
	\item[Eigenschaft (3):] Nach einer \isi{Phasenverschiebung} beobachtet man nur noch einen In-Phase-Modus.
	\item[Eigenschaft (4):] Bei Reduzierung der Bewegungsfrequenz verbleibt das System im In-Phase-Modus und kehrt nicht zu seinem Ausgangszustand zurück.
\end{description}

Bei der einfachsten Darstellung mittels Potentialfunktionen addiert man zwei Grundfunktionen, siehe die Gleichung in~\ref{eq:hkb01}. Es handelt sich um zwei Kosinusfunktionen der relativen Phase $\left(\psi =\phi_2-\phi_1\right)$, von der eine Funktion die doppelte Geschwindigkeit der anderen hat. Die Faktoren $a$ und  $b$ sind Gewichtungskoeffizienten. 

\begin{equation}
\label{eq:hkb01}
V(\psi)=-a\cos(\psi)-b\cos(2\psi);(\psi ={\phi} _{2}-{\phi} _{1})
\end{equation}

Die Abbildung~\ref{figure:0301} zeigt eine Auswahl für den Verlauf der jeweiligen Potentiallandschaften unter Berücksichtigung unterschiedlicher Gewichtungskoeffizienten für  $b/a=1$ sowie  $b/a=0,5$ und  $b/a=0,25$; die Auswahl für diese Beispiele orientiert sich an \citet{Nam2009b}. In Abbildung~\ref{figure:0301}~(a) ist die Potentiallandschaft für $b/a=1$ dargestellt. Es finden sich zwei potentielle Minima, einmal für $0{}^{\circ}$ (In-Phase) und einmal für  $180{}^{\circ}$ ($\pi$, Anti-Phase). Die schwarzen Kugeln markieren die jeweiligen Startbedingungen im Experiment, die gefüllten schwarzen Kugeln den In-Phase-Modus und die Ungefüllten den Anti-Phase-Modus. Für die Anti-Phase gibt es zwei Startbedingungen, einmal bei $\pi$ und einmal bei $-\pi$, je nachdem mit welcher Hand die \isi{Bewegungsaufgabe} begonnen wurde. 

Die potentiellen Minima bilden jeweils die Attraktoren für die Stabilisierung des Systems, wobei der In-Phase-Modus ein größeres Bassin hat, welches die Kugel zu seinem lokalen Minimum zieht und somit prinzipiell den stärkeren \isi{Attraktor} darstellt. 

\begin{quote}
	There are two potential minima at 0 and 180 degrees and, depending on initial conditions, relative phasing can stabilize at either of the two minima, making them attractors. However, the valley associated with the in-phase minimum is both deeper and broader. Thus, it technically has a larger basin because there is a larger range of initial values for  $\psi $ that will eventually settle into that minimum. \citep[][304]{Nam2009b}
\end{quote}

\begin{figure}[b]
	\includegraphics[width=\textwidth]{figures/3-1_Potentiallandschaft.png}
	\caption{Potentiallandschaften für unterschiedliche Werte von $b/{a}$. Die schwarze gefüllte Kugel markiert die Startbedingung für den In-Phase-Modus, die beiden ungefüllten Kugeln die für die Anti-Phase. Der Attraktor für die Anti-Phase wird in (b) schwächer und verschwindet schließlich in (c). Die Pfeile in (c) markieren den abrupten Wechsel vom Anti-Phase- zum In-Phase-Modus.}
	\label{figure:0301}
\end{figure}

Eine Absenkung des Wertes $b/{a}$ wirkt sich direkt auf die Attraktoren aus. In Abbildung~\ref{figure:0301}~(b) sinkt der Wert für $b/{a}$ auf $0,5$ und das Bassin für den Anti-Phase-Modus wird dabei deutlich flacher. Eine weitere Absenkung des Wertes $b/{a}$ auf $0,25$ in Abbildung~\ref{figure:0301}~(c) führt schließlich dazu, dass der \isi{Attraktor} für den Anti-Phase-Modus verschwindet und die Kugel zum In-Phase-\isi{Attraktor} gezogen wird. Würde der Wert für $b/{a}$ wieder erhöht, so verbliebe die Kugel beim In-Phase-\isi{Attraktor} und würde nicht zurück zum Anti-Phase-\isi{Attraktor} gezogen werden.



Die Ergebnisse der Experimente legen nahe, dass es sich bei der Bewegungsfrequenz um einen Kontrollparameter handelt. Im HKB-Modell kann das Amplitudenverhältnis $b/{a}$ als inverse Funktion der Oszillationsfrequenz aufgefasst werden. Der Wert für $b/{a}$ modelliert die Form der Potentiallandschaft und kontrolliert die Attraktoren. Mit einer Absenkung des Wertes steigt reziprok die Frequenz und schwächt den Anti-Phase-\isi{Attraktor}. Ab einem kritischen Wert ($b/{a}=0,25$) verschwindet der \isi{Attraktor} ganz und es findet eine abrupte Transition vom Anti-Phase- zum In-Phase-Modus statt. Dieser kritische Wert bzw. die damit wechselseitig verbundene Frequenz ist \isi{intrinsisch} definiert. Das HKB-Modell ist auch als Phasenverriegelungsmodell (phase locking model) bekannt. 

Die wechselseitige Beziehung von Amplitude und Frequenz in Form eines einzelnen Kontrollparameters stellt jedoch eine Einschränkung im HKB-Modell dar, und weicht von den Beobachtungen in den Experimenten zur Handmotorik von \citet{Blaufuß2001} ab. Für die Abbildung einer sprachlichen \isi{Bewegungsaufgabe} sollte das System eine Trennung von Phase und Amplitude als Äquivalenz zu Position und Geschwindigkeit vorsehen \citep{Saltzman2000}.

\section{Modell der nichtlinearen gekoppelten Oszillatoren} 
\label{sec:0302}

Die Beschreibung von Phasenbeziehungen (In-Phase, Anti-Phase) zwischen Gesten wurden schon früh im Modell der Artikulatorischen Phonologie aufgegriffen, um die Selbstorganisation von Gesten abbilden zu können \citep{Browman1992a}. Später wurden sie dann im Rahmen der Theorie der nicht-linear gekoppelten Oszillatoren vertieft.

Bei dem Modell der nichtlinearen gekoppelten Oszillatoren handelt es sich um ein dynamisches Modell der Selbstorganisation. Dabei ist jede artikulatorische \isi{Geste} mit einem nichtlinearen Planungsoszillator assoziiert. Dieser \isi{Oszillator} ist ein zeitlicher Trigger. Bei einer bestimmten Phase, die zumeist bei $0{}^{\circ}$ liegt, löst er die Aktivierung der mit ihm assoziierten \isi{Geste} aus \citep{Nam2009b,Browman2000,Goldstein2009,Nam2003}. 

Zu Beginn der sprechmotorischen Planung einer Äußerung befinden sich die Oszillatoren in arbiträren Phasen. Dann werden die Oszillatoren paarweise eingekoppelt und in Schwingung versetzt \citep{Saltzman2000}. Sie erreichen aufgrund der Kopplungskräfte über die Zeit eine stabile Phase zueinander, wobei das Erreichen einer stabilen Phase dem physikalischen Prinzip der Phasenverriegelung (phase-locking, HKB-Model in Kapitel~\ref{sec:0301}) folgt. Es gibt unterschiedliche Modi der Synchronisation, die miteinander \enquote{verriegelt} werden können: Frequenz-, Phasen- und Amplitudenverriegelung \citep{Goldstein2010}. Im Folgenden wird nur die Phasenverriegelung als relevante Attraktor-Eigenschaft besprochen (vgl. Kapitel~\ref{sec:0301}).

Nachdem also die Oszillatoren durch die Phasenverriegelung einen stabilen Zustand erreicht haben, aktivieren sie die jeweils mit ihnen assoziierte artikulatorische \isi{Geste}. Diese Aktivierungen werden in der Regel bei  $0{}^{\circ}$ der Phase des jeweiligen Oszillators ausgelöst. Ein \isi{Oszillator} ist somit der \isi{Taktgeber} (clock) für eine artikulatorische \isi{Geste}. Da die Oszillatoren paarweise gekoppelt sind und stabile Schwingungsmuster erreichen, sichern sie als \isi{Taktgeber} die stabile Koordination zwischen Gesten.

In der \isi{Sprechmotorik} unterscheidet man die In-Phase ($0{}^{\circ}$) und die Anti-Phase ($180{}^{\circ}$, HKB-Model), beides intrinsische Modi, die ohne Lernen zugänglich sind. Dabei ist die In-Phase stabiler als die Anti-Phase. Des Weiteren gibt es die nicht-intrinsischen Modi, die von \citet{Goldstein2010} als exzentrische Phasen (eccentric Phase) bezeichnet werden.

\begin{quote}
	While in-phase and anti-phase \isi{coupling} can be performed without any learning, other eccentric \isi{coupling} modes (arbitrary relative phases) can be learned, in order to perform more difficult coordination tasks, such as juggling or drumming. The syllable structure model hypothesizes that eccentric \isi{coupling} is used to coordinate consonant gestures in an onset or a coda cluster (and in some cases across syllables). This can be one of the reasons that clusters are acquired relatively late by children [\dots], and why they are relatively marked typologically. The particular eccentric \isi{coupling} employed may differ from language to language [\dots]. \citep[][5]{Goldstein2011}
\end{quote}

Solche besonderen Modi müssen erst erlernt werden und können in Abhängigkeit der jeweiligen Sprache unterschiedliche Phasenwerte annehmen, z.\,B. $90{}^{\circ}$ (shingled). In der Regel wird in der Literatur bei der \isi{Modellierung} von Silbenstruktur auf In-Phase und Anti-Phase zurückgegriffen, um das Modell einfach zu halten. Das wird auch im Folgenden so gehandhabt. 


\begin{table}[htbp]
	\resizebox{\textwidth}{!}{
		\begin{tabularx}{\textwidth}{XXX} \lsptoprule
			\textbf{In-Phase} & \textbf{Anti-Phase} & \textbf{exzentrische Phase}\\ \midrule
			\myrule[line width = 1.5mm]{}{} & \myrule[line width = 1.5mm,dashed]{}{} & \myrule[line width = 1.5mm,color=lightgray]{}{} \\
			synchron,  $0{}^{\circ}$ & sequentiell,  $180{}^{\circ}$ & partielle \isi{Überlappung} (schuppenartig, z.\,B. $90{}^{\circ}$)\\
			kein Lernen & kein Lernen, aber weniger stabil & Lernen von bestimmten Kombinationen\\ \lspbottomrule
		\end{tabularx}
	}
	\caption{Attraktoren für die Phasenverriegelung, adaptiert von \citet{Goldstein2010}; Form und Farbe der Pfeile zeigt die graphische Verwendung im Kopplungsgraph.}
	\label{table:0301}
\end{table}


In der Literatur gibt es unterschiedliche Evidenzen für die Annahme zweier intrinsischer Modi \citep[vgl.][]{Nam2009b}. So konnten \citet{Saltzman1998} zeigen, dass das System eingekoppelter Oszillatoren durch die Phasenverriegelung so stabil ist, dass es auch nach Perturbation in diesen Zustand zurückkehrt (phase resetting). Wie bei der Handmotorik gilt auch bei der \isi{Sprechmotorik}, dass der In-Phase-Modus im Vergleich zum Anti-Phase-Modus stabiler ist. So zeigen kinematische Studien zu Versprechern im Englischen, dass es wie beim HKB-Modell systematische Phasenverschiebungen vom weniger stabilen Anti-Phase- zum In-Phase-Modus gibt \citep{Pouplier2005,Pouplier2007,Goldstein2007}. In \il{Englisch}englischen Äußerungen wie <top cop> alternieren wortinitial Konstriktionsgesten der Zungenspitze {TT closure alveolar} für /t/ und des Zungenrückens {TB closure velar} für /k/. Bei Wiederholungen dieser Äußerungen tendieren die Sprecher dazu, die dorsale \isi{Geste} nicht mehr alternierend, sondern synchron zur alveolaren \isi{Geste} zu produzieren (eine \isi{Geste} wird zu Gunsten des stabileren In-Phase-Modus in Form einer Intrusion eingefügt). Wenn die Amplitude für die dorsale \isi{Geste} steigt, hört man einen Versprecher in Form eines diskreten Wechsels von einem Segment /t/ zu einem anderen /k/. Artikulatorisch hingegen handelt es sich um ein graduelles Phänomen von gleichzeitig produzierten Konstriktionen {TT alveolar closure} und {TB velar closure} mit unterschiedlichen Amplituden. Diesem Phänomen liegt aber zunächst die abrupte Transition von einem Anti-Phase Modus zu einem In-Phase Modus gesturaler Aktivität zugrunde.

Bei der Verwendung nichtlinearer Oszillatoren für die \isi{Modellierung} der Koordination in der \isi{Sprechmotorik} lassen sich im Prinzip zwei unterschiedliche Systemarchitekturen annehmen: Entweder sind die Oszillatoren jeweils paarweise zu einem multiplen Netzwerk eingekoppelt. Hierbei gäbe es viele \isi{Taktgeber} (clocks), die dezentral miteinander verbunden sind und ein sehr flexibles Triggern der artikulatorischen Gesten erlauben. Oder es sind alle relevanten Oszillatoren mit einem Mastertaktgeber (master clock) verbunden, der zentral über Frequenzveränderung das gesamte System steuert. Es hat sich jedoch gezeigt, dass ein System mit einem Mastertaktgeber die prosodischen Struktureigenschaften höherer Einheiten nicht direkt abbilden kann \citep{Goldstein2009}. Vielmehr lassen sich die Struktureigenschaften größerer prosodischer Einheiten wie beispielsweise der \isi{Silbe} oder dem Fuß über ein multiples Netzwerk gekoppelter Oszillatoren direkt modellieren \citep{Saltzman2000, Nam2003, Goldstein2009, Nam2009a}.

Das multiple Netzwerk der paarweise gekoppelten Oszillatoren wird in der Artikulatorischen Phonologie mittels \isi{Kopplungsgraph} dargestellt. Ein \isi{Kopplungsgraph} (\isi{coupling} graph) repräsentiert einen Teil des phonologischen Wissens eines Sprechers über bestimmte Wortformen \citep{Goldstein2007a} und enthält konkrete Informationen über die gestische Struktur der jeweiligen Äußerung. Abbildung~\ref{figure:0302} zeigt Kopplungsgraphen für die \il{Englisch}englische Äußerung <bud> und <dub>, die sich im \isi{Phasing} der Gesten zueinander unterscheiden. 


\begin{figure}[ht]
	\includegraphics[width=.8\textwidth]{figures/3-2_Kopplung_buddub.png}
	\caption{Kopplungsgraphen für <bud> und <dub> im Englischen. Graue Verbindungslinien markieren die In-Phase ( $0{}^{\circ}$) und schwarze, gestrichelte Pfeile die Anti-Phase ( $180{}^{\circ}$). LA = Lippenöffnung, TB = Zungenrücken, TT = Zungenspitze.}
	\label{figure:0302}
\end{figure}

Die Knoten (nodes) des Kopplungsgraphen spezifizieren die drei artikulatorischen Gesten mit ihren assoziierten Oszillatoren. Die Ränder (edges) spezifizieren den bevorzugten Kopplungsmodus der jeweils paarweise gekoppelten Oszillatoren. In <bud> ist die orale Konstriktionsgeste für den \isi{Vokal} {TB wide uvular} mit der labialen \isi{Geste} {LA closure labial} In-Phase und mit der Zungenspitzengeste {TT closure alveolar} in Anti-Phase gekoppelt. In <dub> hingegen bestehen die umgekehrten Phasenbeziehungen zwischen den einzelnen vokalischen und konsonantischen Gesten. Hier ist der \isi{Vokal} mit der Zungenspitzengeste In-Phase und mit der labialen \isi{Geste} in Anti-Phase gekoppelt.

Gestische Kopplungsgraphen zeigen Ähnlichkeiten zum Aufbau eines Moleküls. In Analogie zur Chemie können Gesten mit \enquote{Atomen} verglichen werden, die durch \enquote{Bindungen} in größeren lexikalischen Einheiten -- den \enquote{Molekülen} -- zusammengehalten werden. Solche lexikalischen Einheiten können beispielsweise Silben oder Wörter sein. Die Bindung -- d.\,h. die Koordination -- zwischen Gesten trägt linguistisch relevante Information und kann mit den Eigenschaften eines Moleküls als zwei- oder mehratomiges Teilchen verglichen werden: So gehen Gesten in einem multiplen Netzwerk miteinander Bindungen in Form von zeitlichen Triggern (Taktgebern) ein. Dabei ist zu beachten, dass es unterschiedliche Stärken der Bindung (bonding strength) gibt. Besonders starke Bindungen finden sich bei Gesten innerhalb eines Segments (beispielsweise die Bindung zwischen Velumsgeste und Zungenspitzengeste für die Bildung von /n/), während die Bindung zwischen Konsonant und \isi{Vokal} weniger stark ist \citep{Goldstein2003}. Auch sind Bindungen im Silbenonset stärker als in der Koda.

Im Folgenden werden Kopplungsmuster im Hinblick auf die Analyse von Silbenstruktur aufgezeigt. In Kapitel~\ref{chap:07} werden dann Kopplungen im Hinblick auf die Ton-Text Assoziation diskutiert.

\section{Silbenkopplungshypothese}
\label{sec:0303}

Eine Reihe kinematischer Studien hat sich mit der gestischen Koordination von Konsonanten und Vokalen innerhalb der \isi{Silbe} beschäftigt und dabei \isi{Bewegungsmuster} im Hinblick auf Regularitäten und Unterschiede in den einzelnen Silbendomänen untersucht \citep[vgl.][]{Browman1988, Honorof1995, Byrd1995, Bombien2010, Goldstein2007a, Goldstein2009, Hermes2008b, Marin2010, Nam2007a, Nam2009b, Shaw2009, Mücke2010b, Geng2010, Hermes2011b, Hermes2011a, Hermes2013}. Die Ergebnisse solcher Studien wurden mit Hilfe eines Netzwerkes nicht-linearer paarweise gekoppelter Oszillatoren modelliert \citep[vgl.][]{Browman2000, Saltzman2000, Nam2003, Nam2007a, Goldstein2007a, Goldstein2009}. Auf dieser Basis entstand die Silben-Kopplungshypothese (\isi{coupling} hypothesis of syllable structure), wie sie auch in \citet{Hermes2016} beschrieben ist.

\begin{figure}[t]
% 	\includegraphics[width=.66\textwidth]{figures/a04-img3.png}
	
\begin{forest} baseline, where n children=0{tier=word}{} ,for tree={align=center,base=bottom}
[σ 
  [Onset
    [C\\      /t,name=0303t1
    ]
    [C\\     ʁ,name=0303ʁ
    ]
  ]
  [Reim
    [Nukleus
      [V\\	oː,name=0303o
      ]
    ]
    [Koda
      [C\\	  s,name=0303s
	]
      [C\\	  t/,name=0303t2
      ]
    ]
  ]
]
\draw[
    thick,
    decoration={
        brace,
        mirror,
        raise=.7cm
    },
    decorate
] (0303t1.west) -- (0303o.east) ;
\draw[
    thick,
    decoration={
        brace,
        mirror,
        raise=15mm
    },
    decorate
] (0303o.west) -- (0303t2.east) ;
\node [below=3mm of 0303ʁ] {Onset -- Nukleus}; 
\node [below=10mm of 0303s,xshift=-2mm] {Nukleus -- Koda}; 
\end{forest}
	
	\caption{Strukturbaum für <Trost> als CCVCC-Sequenz mit subsilbischen Konstituenten im Rahmen des Autosegmental-Metrischen Modells.}
	\label{figure:0303}
\end{figure}
Die Organisation von lautlichen Einheiten in Silben und deren Konstituenten \isi{Onset} (Silbenanlaut), Nukleus (Silbenkern) und Koda (Silbenauslaut) ist eine grundlegende Eigenschaft phonologischer Systeme vieler Sprachen \citep[u.a.][]{Lenerz2002}. Abbildung~\ref{figure:0303} zeigt im Rahmen des Autosegmental Metrischen Modells einen Strukturbaum für das Wort <Trost>, hier mit den Konstituenten \isi{Onset} (O), Reim (R), Nukleus (N) und Koda (K). Im dargestellten Fall handelt es sich um eine rechtsverzweigende Onset-Reim Struktur mit einem Reim, der sich weiter in Nukleus und Koda verzweigt \citep[vgl.][]{Lowenstamm1979, Lass1984}; es sind je nach Theorie auch andere Domänenspezifikationen möglich. Der \isi{Vokal} /o:/ bildet als sonorstes Element den Nukleus und somit das Zentrum der \isi{Silbe}. Die Konsonanten gruppieren sich um den Nukleus, das initiale Cluster /tʁ/ als ein verzweigender \isi{Onset} und finales /st/ als komplexe Koda. In dieser Analyse zählen alle konsonantischen und vokalischen Elemente zur \isi{Silbe}; weder phonotaktische Beschränkungen noch die Sonoritätshierarchie werden bei <Trost> im Deutschen verletzt, sofern innerhalb der jeweiligen Theorie angenommen wird, dass Plosive weniger sonor sind als Frikative. 



Solche Silbendomänen können mit Hilfe der Silbenkopplungshypothese in die physikalischen Prinzipien der gesturalen Organisation übersetzt werden \citep{Goldstein2007a, Nam2009b}. Dabei geht man bei der Silbenkopplungshypothese davon aus, dass konsonantische und vokalische Konstriktionsgesten paarweise über gekoppelte Oszillatoren verbunden sind. Zwischen diesen Oszillatoren gibt es unterschiedliche Kopplungsstärken, die sich aus den unterschiedlichen Target-Spezifikationen der relativen Phasen ergeben. Diese Target-Spezifikationen sind direkt aus der Domänenzugehörigkeit der oralen \isi{Geste} innerhalb der \isi{Silbe} ableitbar.


\subsection{CV und VC Silben}
\label{subsec:030301}

\begin{figure}[t]
	\includegraphics[width=.8\textwidth]{figures/3-4_paArp.png}
	\caption{Kopplungsgraphen für CV <pa> und VC <Arp>. Graue Verbindungslinien markieren die In-Phase ( $0{}^{\circ}$) und schwarze gestrichelte Pfeile die Anti-Phase ( $180{}^{\circ}$) Zielspezifikationen. GLO =Glottis, LP = Lippenöffnung, TB = Zungenrücken.}
	\label{figure:0304}
\end{figure}
Für die Onset-Nukleus-Relation wird ein stabiler In-Phase-Modus ($0{}^{\circ}$) angenommen. Abbildung~\ref{figure:0304}~(links) zeigt einen Kopplungsgraphen für die CV-\isi{Silbe} der \il{Englisch}englischen Äußerung <pa> \citep[vgl.][]{Mücke2012}.  Die Oszillatoren, die mit den oralen Konstriktionsgesten für den Konsonanten und \isi{Vokal} assoziiert sind, sind in-phase gekoppelt. Da beide Oszillatoren bei $0{}^{\circ}$ auch die Aktivität der artikulatorischen \isi{Geste} auslösen, werden die Gesten gleichzeitig (synchron) initiiert. Abbildung~\ref{figure:0304}~(rechts) hingegen zeigt den \isi{Kopplungsgraph} für die \il{Englisch}englische VC-\isi{Silbe} <Arp>. Hier wird von einem Anti-Phase-Modus ausgegangen. Die mit den oralen Gesten für C und V assoziierten Oszillatoren sind sequentiell ($180{}^{\circ}$) gekoppelt und aktivieren die Gesten somit phasenverschoben (Out-of-Phase). In beiden Fällen, CV und VC, gibt es keinen Wettbewerb zwischen den Kopplungen der Oszillatoren (competition, \citealt{Goldstein2007a}) und somit entspricht die spezifizierte Zielphase auch der resultierenden finalen Phase. Das bedeutet, dass eine $0{}^{\circ}$ Zielphase bei CV auch einer synchronen Aktivierung und eine $180{}^{\circ}$ Zielphase einer entsprechenden sequentiellen Aktivierung der assoziierten Gesten entspricht.




CV und VC Silben bedienen sich intrinsischer Modi, die nicht gelernt werden müssen. Dabei ist der In-Phase-Modus stabiler als der Anti-Phase-Modus. Aus diesen physikalischen Eigenschaften von gekoppelten Oszillatoren ist der unmarkierte Status von CV-Silben ableitbar \citep{Goldstein2007a, Nam2009b, Goldstein2010}: Der In-Phase-Modus hat den stärksten \isi{Attraktor} und ist am leichtesten in der Sprachplanung zugänglich. In vielen Sprachen können in einfachen CV-Onsets Konsonanten ohne Restriktionen mit Vokalen kombiniert werden. Bei gegebener Synchronität sollte jede artikulatorische Aktionseinheit mit einer anderen aufgrund der Stärke des Attraktors ohne Lernen kombinierbar sein. Bei komplexen Onsets, komplexen Kodas und zwischen Nukleus und Koda hingegen bestehen mehr Restriktionen. 

\begin{figure}[t]
% 	\includegraphics[width=.8\textwidth]{figures/3-5_KopplungCV_VC.png}
	
	\begin{tikzpicture} 
\node[minimum width=3cm, minimum height=.7cm, inner sep=0pt, draw, fill=black!10] (rect1) {V};
   
\node[minimum width=3cm, minimum height=.7cm, inner sep=0pt, draw, fill=black!10] (rect2) [below=of rect1] {V};
 

\node[minimum width=1.4cm, minimum height=.5cm, inner sep=0pt, draw, fill=white] (inset1) [below=of rect1,yshift=1.25cm,xshift=-.8cm] {C};
\node[minimum width=1.4cm, minimum height=.5cm, inner sep=0pt, draw, fill=white] (inset1) [below=of rect2,yshift=1.25cm,xshift=1.5cm] {C};
 
 
\node[circle,draw, minimum size=3mm] (C1) [right=of rect1,xshift=5mm] {C};
\node[circle,draw, minimum size=3mm] (V1) [right=of C1   ] {V};

\node[circle,draw, minimum size=3mm] (V2) [right=of rect2,xshift=5mm]   {V};
\node[circle,draw, minimum size=3mm] (C2) [right=of V2]  {C};
 

\draw 	      (C1) -- (V1);
\draw[dashed] (V2) -- (C2); 

\node (Gestenpartitur) [above = of rect1,yshift=-.8cm] {Gestenpartitur};
\node (Kopplungsgraph) [right = of Gestenpartitur,xshift=5mm] {Kopplungsgraph};

\node () [left = of rect1] {(a)};
\node () [left = of rect2] {(b)}; 

\end{tikzpicture}
	
	
	\caption{Vereinfachte Kopplungsgraphen (rechts) und zugehörige Gestenpartituren (link) für CV und VC Silben.}
	\label{figure:0305}
\end{figure}

Abbildung~\ref{figure:0305} veranschaulicht die In-Phase \isi{Kopplung} in CV Silben und die Anti-Phase \isi{Kopplung} in VC Silben noch einmal in vereinfachter Form. Bei dieser Darstellung werden C und V nur als strukturelle Elemente angegeben und keine konkreten Gesten spezifiziert. Rechts in der Abbildung ist jeweils die \isi{Kopplung} darstellt (In-Phase als durchgezogene und Anti-Phase als gestrichelte Linie). Links in der Abbildung findet sich die zugehörige \isi{Gestenpartitur} als Response auf den Kopplungsgraphen. Bei einer CV \isi{Silbe} (Abbildung~\ref{figure:0305}~(a) starten C und V gleichzeitig. V hat jedoch eine geringere \isi{Eigenfrequenz} und wird deshalb als \isi{Bewegungseinheit} langsamer ausgeführt als C. Auf akustischer Oberfläche entsteht der Eindruck einer Abfolge von C und V, obwohl sie sich artikulatorisch komplett überlagern. Bei einer VC \isi{Silbe} (Abbildung~\ref{figure:0305}~(b)) werden V und C sequentiell aktiviert. Erst startet der \isi{Vokal}, und dann der Konsonant.


\largerpage
\subsection{CCV und VCC Silben}
\label{subsec:030302}

Bei komplexen Onsets, die aus zwei Konsonanten bestehen, ist anzunehmen, dass beide Konsonanten in-phase mit dem \isi{Vokal} assoziiert sind. Das würde aber bedeuten, dass beide Konsonanten synchron starteten und einer den anderen perzeptiv verdeckte. Aus diesem Grund sind die Onset-Konsonanten zusätzlich miteinander out-of-phase gekoppelt. Es entsteht somit ein Wettbewerb zwischen den verschiedenen Kopplungsstärken (In-Phase der Cs mit dem \isi{Vokal} und \isi{exzentrische Phase} der Cs untereinander), die zu Beginn der Planung in das System eingegeben werden. Am Ende der Planung stellt die finale Phase einen Kompromiss zwischen den konkurrierenden Kräften der Zielphasen dar \citep{Goldstein2007a}. Als Konsequenz ergibt sich folgender Timing-Unterschied: Wenn ein Konsonant zum \isi{Onset} hinzugefügt wird, bewegt sich ${C}_{1}$ nach links weg vom \isi{Vokal} (leftward shift) und ${C}_{2}$ nach rechts hin zum \isi{Vokal} (rightward shift). Dieses Muster ist als \isi{C-Center Effekt} bekannt \citep[vgl.][]{Browman1988, Browman2000, Goldstein2007a, Goldstein2009, Hermes2008a, Marin2010, Nam2007a, Nam2009b, Shaw2009}.

\begin{figure} 
	\includegraphics[width=.8\textwidth]{figures/3-6_spaASP.png}
	\caption{Kopplungsgraphen für CCV <spa> und VCC <asp>. Graue Verbindungslinien markieren die In-Phase und schwarze Pfeile die exzentrische Target-Spezifikationen. GLO = Glottis, TT = Zungenspitze, TB = Zungenrücken, LP = Lippenöffnung.}
	\label{figure:0306}
\end{figure}

Abbildung~\ref{figure:0306}~(links) zeigt den \isi{Kopplungsgraph} für die \il{Englisch}englische Äußerung <spa> mit komplexem \isi{Onset}. /s/ und /p/ sind beide In-Phase mit dem \isi{Vokal} und Anti-Phase (eventuell auch in exzentrischer Phase) zueinander gekoppelt (competitive \isi{coupling}). Als Konsequenz ergibt sich ein C-Center-Effekt, der statt Synchronizität ein früheres Starten von ${C}_{1}$ und ein späteres Starten von ${C}_{2}$ relativ zum \isi{Vokal} beinhaltet.


Komplexe Kodas sind in vielen Sprachen durch Out-of-Phase-Kopplungen gekennzeichnet, woraus sich relativ lose Bindungen und erhöhte \isi{Variabilität} im Signal ergeben. Abbildung~\ref{figure:0306}~(rechts) zeigt einen Kopplungsgraphen für die \il{Englisch}englische Äußerung <asp> mit komplexer Koda (VCC). Nur ${C}_{1}$ ist direkt mit dem \isi{Vokal} assoziiert. Hierbei handelt es sich um eine Anti-Phase \isi{Kopplung} (eventuell auch exzentrische \isi{Kopplung}), die bewirkt, dass der \isi{Vokal} und ${C}_{1}$ nacheinander gestartet werden. ${C}_{2}$ ist nur mit ${C}_{1}$ aber nicht mehr direkt mit dem \isi{Vokal} gekoppelt. 


\largerpage
Abbildung~\ref{figure:0307} veranschaulicht die \isi{Kopplung} in CCV und VCC Silben noch einmal in vereinfachter Weise, analog zu Abbildung~\ref{figure:0305}. Bei CCV Silben (Abbildung~\ref{figure:0306}~(a)) sind zwei Kopplungen zu sehen, die miteinander konkurrieren, weil sie erst einmal inkompatibel sind. Beide Cs sind gleichzeitig zum V getriggert, aber sequentiell zueinander. Der am \isi{Vokal} angrenzende Konsonant bewegt sich nach rechts auf den \isi{Vokal} zu, um Platz für den hinzukommenden Konsonanten zu machen. Es ergibt sich ein komplexes \isi{Koordinationsmuster}. Dieses Muster könnte man auch mit folgendem, sehr vereinfachtem Bild beschreiben: Man stelle sich vor, die Konsonanten wären Menschen, die sich einen Stuhl teilen sollen. Ein Mensch alleine kann auf einem Stuhl mittig Platz nehmen. Kommt ein weiterer Mensch hinzu, so muss der erste von der Mitte wegrutschen und sich seitlich auf den Stuhl setzen, um dem anderen Platz zu machen. Der andere kann sich dann auf die andere Seite setzen, zusammen ragen sie rechts und links über den Stuhlrand hinaus. 


\begin{figure}
% 	\includegraphics[width=.8\textwidth]{figures/3-7_KopplungCCV_VCC.png}
	\begin{tikzpicture}  

\node[minimum width=3cm, minimum height=.7cm, inner sep=0pt, draw, fill=black!10] (rect3) [] {~~~~~~~~~~V};
\node[minimum width=3cm, minimum height=.7cm, inner sep=0pt, draw, fill=black!10] (rect4) [below =of rect3,yshift=-5mm] {V};


\node[minimum width=1.4cm, minimum height=.5cm, inner sep=0pt, draw, fill=white] (inset1) [below=of rect3,yshift=1.5cm,xshift=-1.5cm] {C};
\node[minimum width=1.4cm, minimum height=.5cm, inner sep=0pt, draw, fill=white] (inset2) [below=of rect3,yshift=1.2cm,xshift=-.5cm] {C};

\node[minimum width=1.4cm, minimum height=.5cm, inner sep=0pt, draw, fill=white] (inset3) [below=of rect4,yshift=1.25cm,xshift=1cm] {C};
\node[minimum width=1.4cm, minimum height=.5cm, inner sep=0pt, draw, fill=white] (inset4) [right=of inset3,xshift=-1cm] {C};

\node[circle,draw, minimum size=3mm] (C3)  [right=of rect3,xshift=25mm]  {C};
\node[circle,draw, minimum size=3mm] (C4)  [right=of C3   ]  {C};
\node (V3) [circle,draw, minimum size=3mm,below = of $(C3)!0.5!(C4)$,yshift=4mm]  {V};


\node[circle,draw, minimum size=3mm] (V5)  [right=of rect4,xshift=16mm]  {V};
\node[circle,draw, minimum size=3mm] (C5)  [right=of V5   ]  {C};
\node[circle,draw, minimum size=3mm] (C6)  [right=of C5   ]  {C};

\draw[dashed] (C3) -- (C4);
\draw         (V3) -- (C3);
\draw         (V3) -- (C4);

\draw[dashed]         (V5) -- (C5);
\draw[dashed]         (C5) -- (C6);

\node (Gestenpartitur) [above = of rect1,yshift=-.8cm] {Gestenpartitur};
\node (Kopplungsgraph) [right = of Gestenpartitur,xshift=25mm] {Kopplungsgraph};

\node () [left = of rect3] {(a)};
\node () [left = of rect4] {(b)}; 

\end{tikzpicture}

	\caption{Vereinfachte Kopplungsgraphen (rechts) und zugehörige Gestenpartituren (link) für CCV und VCC Silben.}
	\label{figure:0307}
\end{figure}


Abbildung~\ref{figure:0307}~(b) veranschaulicht das weniger komplexe \isi{Koordinationsmuster} bei VCC Silben. In der Koda sind die Konsonanten einfach sequentiell angeordnet, nur der am \isi{Vokal} angrenzende Konsonant ist mit diesem gekoppelt. Die Konsonanten stellen eine lose Reihe dar, jeder hat -- nehmen wir das Bild mit den Menschen und den Stühlen wieder auf -- seinen eigenen Platz.

 \largerpage
\section{Empirische Evidenz für Silbenstruktur im \ili{Polnisch}en}
\label{sec:0304}

Kinematische Messungen können diagnostisch verwendet werden, um die Zugehörigkeit eines Segments zu einer Silbendomäne zu klären. So gibt es Sprachen, die aus phonologischer Sicht mehrere Konsonanten im Silbenonsets zulassen (komplexe Onsets), und solche, die nur Onsets mit maximal einem Konsonanten generieren (simple Onsets). Es gibt bereits eine größere Anzahl von Studien zu verschiedenen Sprachen mit simplen und komplexen Onsetparse, die die silbenstrukturelle Organisation in wortanlautenden Konsonantenclustern auf der Basis artikulatorischer Daten untersucht haben. Es sollte jedoch angemerkt werden, dass die silbeninterne Koordination nicht unabhängig von segmentalen Einflüssen \citep{Brunner2014, Marin2013, Marin2014, Pouplier2012}, Einflüssen der prosodischen Struktur \citep{Shaw2009, Shaw2011, Gafos2014, Hermes2017} oder sprecherspezifischer Variation ist. So beziehen sich artikulatorischen Messungen zur Erfassung der Silbenstruktur meist auf Grad der \isi{Überlappung} zwischen Konsonanten und einem folgenden Silbenanker. Der Grad der \isi{Überlappung} unterliegt aber neben dem phonologischen Silbenparse auch natürlicher Variationen \citep{Goldstein2007a}. Im Deutschen, beispielsweise zeigt das Cluster /kl/ wie in <Claudia> (/’klaʊdia/) einen größeren Grad an \isi{Überlappung} zwischen den Konsonanten als /kn/ wie in <Kneipe> (/’knaɪpə/). Vermutlich ist der Überlappungsgrad bei /kn/ geringer, damit die nasale Lösung auditiv nicht maskiert ist. Solche \isi{segmental} bedingte Unterschiede wie beispielsweise zwischen /kl/ und /kn/ haben einen Einfluss auf die gesturale Koordination in höhere Konstituenten wie der \isi{Silbe}. Betrachtet man also die einzelnen Studien zur Silbenkoordination, so finden sich in den Messungen immer wieder auch Muster, die dem zugrundliegenden Silbentrigger zu widersprechen scheinen. So zeigen \citet{Brunner2014} beispielsweise, dass das Cluster /sk/ eine Koordination für komplexe Onsets besitzt, das Cluster /pl/ jedoch nicht. Diese Ergebnisse bieten Raum für Mehrdeutigkeiten in der Interpretation der phonologischen Zuordnung. Für eine ausführliche Diskussion zum Problem der artikulatorischen Messungen für die Ermittlung von Silbenzugehörigkeiten von Konsonantensequenzen aufgrund segmentaler und prosodischer Einflüsse sei auf \citet{Hermes2017} verwiesen.

  
Wenngleich die oben genannten Probleme bei den Verfahren zur Silbenmessung auftreten können und bei der Interpretation Berücksichtigung finden müssen, gibt es doch eine Reihe aussagekräftiger Ergebnisse in der Literatur. Evidenzen, die für eine Organisation von Konsonantenverbindungen als komplexe Onsets sprechen, finden sich in den Sprachen \il{Englisch!amerikanisch}Amerikanisches Englisch \citep{Browman2000, Marin2010}, \ili{Französisch} \citep{Kühnert2006}, \ili{Georgisch} \citep{Goldstein2007a} und {\ili{Polnisch} \citep{Mücke2010b}. Evidenzen für Sprachen, die keine komplexen Onsets haben, finden sich für \ili{Tashlhiyt} Berber \citep{Goldstein2007a, Hermes2011b, Hermes2011a} und \il{Arabisch!marokkanisch}Marokkanisches Arabisch \citep{Shaw2009}. Eine Ausnahme bildet \ili{Italienisch}. Hier konnten \citet{Hermes2008b} zeigen, dass es wortanlautend zwei distinktive \isi{Koordinationsmuster} innerhalb einer Sprache gibt. Konsonantensequenzen ohne anlautenden Sibilanten zeigen die sprechmotorische Organisation komplexer Onsets, während Sibilanten in Sibilant-plus-Konsonant-Verbindungen (s-impura; unreines-s) morphosyntaktische Alternationen auslösen und das Parsen von komplexen Onsets nicht erlauben. Eine detaillierte Übersicht über Organisationen von Silbenonsets in verschiedenen Sprachen befindet sich in \citet{Hermes2017}.

\largerpage[2]
Im Folgenden wird die Methodik am Beispiel der Silbenorganisation im \ili{Polnisch}en aus einer Studie von \citet{Mücke2010b} veranschaulicht. Mittels elektromagnetischer \isi{Artikulographie} (EMMA, AG 100) wurden in dieser Studie die \isi{Bewegungsmuster} für verschiedene Obstruent-Sonorant-Verbindungen in wort-initialer und wortfinaler Position untersucht. Es wurden drei Sprecher (zwei Frauen und ein Mann) aufgenommen, die jeweils Zielwörter im folgenden Trägersätz produzierten: 

%%(3.10)
\begin{exe}
	\label{ex:0310}
	\ex Ona mówi \longrule{} aktualnie.
	\sn lit.: \textit{Sie sagt \longrule{} jetzt.}
\end{exe}

\newpage 
Jedes Zielwort wurde im Korpus siebenfach wiederholt. Die Struktur der Zielwörter bildete jeweils die folgenden Triaden: 

%%(3.11)
\begin{exe}
	\label{ex:0311}
	\ex \textit{Wortinitial:} /\textbf{kr}/asić, /\textbf{r}/abin, /\textbf{k}/adisz
	\sn \textit{(abschmelzen, Rabbiner, Kaddisch)}
\end{exe}

%%(3.12)
\begin{exe}
	\label{ex:0312}
	\ex \textit{Wortfinal:} WI/\textbf{kr}/, ti/\textbf{r}/, ti/\textbf{k}/
	\sn \textit{(WIKR (als Akronym), Laster, Tick)}
\end{exe}

Um die \isi{Bewegungsmuster} zu erfassen, wurden die EMMA-Trajektorien für Zungenrücken und -spitze ausgewertet. Dabei wurden Targets für Konsonanten und Vokale annotiert; lokale Maxima und Minima bei Positionsbestimmungen in der vertikalen Dimension entsprechen hierbei Nulldurchgängen in der Geschwindigkeitskurve. Für die Auswertung der Daten wurden jeweils die zeitlichen Abstände zwischen Konsonanten- und Vokaltargets im kinematischen Signal gemessen. 

\begin{figure}[b]
% 	\includegraphics[width=.8\textwidth]{figures/3-8_CV_CCV_Kopplung.png}
\begin{tikzpicture}	



\node[circle,draw, minimum size=3mm] (C1)  []  {C};
\node[circle,draw, minimum size=3mm] (V1)  [below=of C1,yshift=8mm  ]  {V};

\node[circle,draw, minimum size=3mm] (C2)  [right=of C1,xshift=20mm]  {C};
\node[circle,draw, minimum size=3mm] (C3)  [right=of C2]  {C};
\node (V2) [circle,draw, minimum size=3mm,below = of $(C2)!0.5!(C3)$,yshift=4mm]  {V};

\draw         (C1) -- (V1);
\draw         (C2) -- (V2);
\draw         (C3) -- (V2);
\draw[dashed] (C2) -- (C3);

\node[above= of V1]{CV};
\node[above= of V2]{CCV};

\node(text1)[right=of V1,yshift=4mm,xshift=-3mm] {\parbox{18mm}{Bewegung nach links}};
\node(text2)[right=of V2,yshift=4mm] {\parbox{18mm}{Bewegung nach rechts}};

\end{tikzpicture}

	\caption{Hypothetische Kopplungsgraphen für CV und CCV im  {Polnisch}en. Durchgezogene Linien markieren die Zielspezifikation In-Phase und gestrichelten Linien Out-of-Phase.}
	\label{figure:0308}
\end{figure}

\citet{Mücke2010b} überprüften die folgende Annahme: Wortinitiale Obstruent-Sonorant-Verbindungen sind im \ili{Polnisch}en als komplexe Onsets organisiert und die entsprechenden wortfinalen Cluster zeigen silbenstrukturelle Eigenschaften von komplexen Kodas. Mit komplexer Koda ist hier gemeint, dass mehrere Konsonanten in der Koda zugelassen sind, es wird jedoch nicht wie im Silbenonset ein \isi{C-Center Effekt} erwartet. Es ergeben sich damit unterschiedliche Kopplungsgraphen für die dynamische Selbstorganisation der Silbenstruktur: Für den \isi{Onset} werden konkurrierende Zielspezifikationen angenommen (competitive \isi{coupling}, Abbildung~\ref{figure:0308}), während in der Koda die Zielspezifikationen den finalen Phasen einer Out-of-Phase-\isi{Kopplung}, hier mit Anti-Phase gekennzeichnet, entsprechen sollten (Abbildung~\ref{figure:0309}). Die Konsequenzen der Kopplungsgraphen sollten sich im kinematischen Signal als Bewegungen beider Konsonanten in CCV versus CV zeigen (${C}_{1}$ bewegt sich nach links und ${C}_{2}$ nach rechts), während in VCC versus VC keine Bewegung des vokalangrenzenden Konsonanten (${C}_{1}$) messbar sein sollte.

\begin{figure}
	\includegraphics[width=.8\textwidth]{figures/3-9_VC_VCC_Kopplung.png}
	\caption{Hypothetische Kopplungsgraphen für VC und VCC. Gestrichelte Linien markieren die Out-of-Phase Zielspezifikationen.}
	\label{figure:0309}
\end{figure}

Im Folgenden sind die Ergebnisse für Konsonantenverbindungen in wortinitaler und -finaler Position aufgeführt.


\subsection{Onset-Messung im  {Polnisch}en für CV und CCV}
\label{subsec:030401}

Für die Onset-Messungen werden die Triaden /kr/asić, /r/abin, /k/adisz untersucht (vgl. Beispiel 3.11). Dabei wird jeweils der zeitliche Abstand der konsonantischen Targets relativ zum Folgevokal berechnet. Diese Abstände werden für /k/ und /r/ in /kr/asić mit /k/ und /r/ in /r/abin und /k/adisz/ vergleichen. Die Messvariable sind in Abbildung~\ref{figure:0310} veranschaulicht. Auf der linken Seite der Graphik ist die Messung für die „Leftmost-C Variable“ und auf der rechten Seite für den „Rightmost-C Variable“ initialer Konsonanten abgebildet. Bei der „Leftmost-C Variable“ geht man davon aus, dass sich der Abstand von /k/ relativ zum Folgevokal vergrößern müsste, wenn man /k/V und /kr/V vergleicht. So shiftet /k/ nach links weg vom \isi{Vokal} und der Grad der \isi{Überlappung} zwischen /k/ und V ab. Umgekehrt verhält es sich bei der „Rightmost-C Variable“. Hier geht man davon aus, dass sich im Falle einer komplexen Onset-Struktur der Abstand von /r/ relativ zum Folgevokal verringern müsste, wenn man /r/V und /kr/V vergleicht. Hier shiftet /r/ nach rechts in Richtung Folgevokal und der Grad der \isi{Überlappung} zwischen /r/ und V nimmt zu.

\begin{figure}
	\includegraphics[width=.8\textwidth]{figures/3-10_Ueberlappungsschema_janeOnset_.png}
	\caption{Variablen für die Onset-Messungen „Leftmost-C“ (linke Abbildung) und „Rightmost-C“ (rechte Abbildung) für CV versus CCV.}
	\label{figure:0310}
\end{figure}

Die Abbildung~\ref{figure:0311} zeigt die Ergebnisse für die zeitlichen Abstandsmessungen für wortinitiale Konsonantenverbindungen für die drei Sprecher (JSf, JSm, NL) und den Mittwelwert aller drei Sprecher (mean). Es handelt sich um eine Messung des Konsonanten am rechten Clusterrand (right-edge, rightmost C) relativ zum folgenden \isi{Vokal}. Die grauen Balken zeigen die zeitlichen Abstände für \isi{Target} /r/ relativ zum \isi{Target} /a/ in </r/abin>. Die schwarzen Balken zeigen die zeitlichen Abstände von /r/ relativ zu /a/ in </kr/asić>.

\begin{figure}
	\includegraphics[width=.8\textwidth]{figures/3-11_RightmostCtoV.png}
	\caption{Durchschnittliche Werte (Latenzen in ms) für das Erreichen des konsonantischen Targets /r/ relativ zum nachfolgenden Vokal in <rabin> (schwarze Balken) und <krasić> getrennt nach Sprechern.}
	\label{figure:0311}
\end{figure}

\newpage 
Es zeigt sich für alle drei Sprecher ein \isi{Bewegungsmuster}, das typisch für die Organisation komplexer Onsets, CCV, im Vergleich zu einfachen Onsets, CV, ist. Die Latenzen zwischen dem rechten Konsonanten und dem folgenden \isi{Vokal} verringern sich deutlich, d.\,h. der Konsonant bewegt sich auf den nachfolgenden \isi{Vokal} zu. Wird ein Konsonant hinzugefügt, verschiebt sich also der vokalangrenzende Konsonant (rightmost C, rightward shift) nach rechts auf das Vokaltarget zu. Es gibt keine Stabilität des rechten Clusterrandes. Vielmehr scheint eine Stabilität des Clusterzentrums relativ zum \isi{Vokal} vorzuliegen. 

Abbildung~\ref{figure:0312} veranschaulicht diese Zentrumsorganisation des Konsonantenclusters (center stability). Die schwarzen Balken zeigen die Verschiebung des rechten Konsonanten auf den \isi{Vokal} zu, die grauen Balken die Verschiebung des linken Konsonanten vom \isi{Vokal} weg, in CCV versus CV. Dabei wurden die jeweiligen zeitlichen Abstände für die Targets des Konsonanten relativ zum \isi{Vokal} voneinander subtrahiert. Verkleinert sich der Abstand zwischen C und V, handelt es sich um eine Rechtsverschiebung (positive Werte) und vergrößert sich der Abstand, um eine Linksverschiebung (negative Werte).

\begin{figure}
	\includegraphics[width=.8\textwidth]{figures/3-12_onset_KR_polish.png}
	\caption{Durchschnittliche Werte (Latenzen in ms) für die Rechtsbewegung von /r/ in <rabin>/ versus <krasić> (schwarze Balken) und die Linksbewegung von /k/ in <krasić> versus <kadisz> (graue Balken).}
	\label{figure:0312}
\end{figure}

Für die \isi{Rechtsbewegung} wurde die CV-Latenz für /r/ relativ zu /a/ in <krasić> und <rabin> subtrahiert. Für die Linksbewegung wurde die CV Latenz für /k/ relativ zu /a/ in <krasić> und <kadisz> subtrahiert. Für /r/ zu /a/ ergibt sich eine Verkürzung der Abstände und somit eine Bewegung nach rechts auf den \isi{Vokal} zu (Rechtsverschiebung um $43{ms}$ für alle Sprecher gemittelt, wie bereits in Abbildung~\ref{figure:0310} beschrieben). Für /k/ zu /a/ ergibt sich entsprechend eine Vergrößerung der Latenzen und somit eine Linksbewegung weg vom \isi{Vokal} um $-44{ms}$ für alle Sprecher gemittelt.


\subsection{Kodamessung im  {Polnisch}en für VC und VCC}
\label{subsec:030402}

Für die Koda-Messungen werden die Triaden WI/kr/, ti/r/ und ti/k/ untersucht (vgl. Beispiel 3.12). Dabei wird jeweils der zeitliche Abstand der konsonantischen Targets relativ zum vorhergehenden \isi{Vokal} berechnet. Verglichen werden die Abstände für /k/ und /r/ in WI/kr/ mit /k/ und /r/ in ti/r/ und ti/k/. Abbildung~\ref{figure:0313} veranschaulicht die entsprechenden Messvariablen, die sich diesmal auf die Koda beziehen. Auf der linken Seite der Graphik ist die Messung für die „Rightmost-C Variable“ und auf der rechten Seite für den „Leftmost-C Variable“ finaler Konsonanten abgebildet. Bei der „Rightmost-C Variable“ geht man davon aus, dass der Abstand von /k/ relativ zum vorhergehenden \isi{Vokal} stabil bleiben müsste, wenn man V/k/ und V/kr/ vergleicht. Im Gegensatz zum \isi{Onset} sollte hier keine C-Center Organisation bestehen. Vergleicht man /r/ relativ zum vorhergehenden \isi{Vokal} V/r/ und V/kr/, so müsste /r/ nach rechts weg vom vorhergehenden \isi{Vokal} shiften. Der zeitliche Abstand zwischen /r/ und dem vorhergehendem \isi{Vokal} sollte größer werden, da der Konsonant an die Sequenz einfach angehängt wird. 

\begin{figure}
	\includegraphics[width=.8\textwidth]{figures/3-13_Ueberlappunsmessung_JaneKoda.png}
	\caption{Variablen für die Koda-Messungen „Leftmost-C“ (linke Abbildung) und „Rightmost-C“ (rechte Abbildung) für VC versus VCC.}
	\label{figure:0313}
\end{figure}

Für Konsonantenverbindungen in wortfinaler Position wurden Alignierungsabstände (zeitliche Distanzen) vom \isi{Target} des Konsonanten am linken Rand des Clusters (leftmost C) relativ zum vorangehenden Vokaltarget erfasst. Die Ergebnisse sind in dem Balkendiagramm in Abbildung~\ref{figure:0314} dargestellt.

\begin{figure}
	\includegraphics[width=.8\textwidth]{figures/3-14_Leftmost.png}
	\caption{Durchschnittliche Werte (Latenzen in $ms$) für das Erreichen des konsonantischen Targets /k/ relativ zum vorangehenden Vokal in <tik> (schwarze Balken) und <krasić> getrennt nach Sprechern.}
	\label{figure:0314}
\end{figure}

Es zeigt sich für alle drei Sprecher ein \isi{Bewegungsmuster}, das von dem im \isi{Onset} abweicht. Die Latenzen zwischen dem linken Konsonantenrand der Koda und dem vorangehenden \isi{Vokal} zeigen bei einer Testung von VC versus VCC keine signifikanten Unterschiede (${p}\leq0.05$). So verändern sich die Mittelwerte in <tik> versus <WIKR> nur geringfügig. Wird ein Konsonant wortfinal hinzugefügt (VCC versus VC), so bleibt der linke Clusterrand relativ zum vorangehenden \isi{Vokal} stabil (left edge stability, leftmost C). 

In Abbildung~\ref{figure:0315} ist diese Stabilität des linken Randes der \isi{Konsonantenverbindung} noch einmal veranschaulicht. Die grauen Balken zeigen die Stabilität von ${C}_{1}$ (/k/) relativ zum vorangehenden \isi{Vokal}. Es wurden die zeitlichen Abstände von /k/ relativ zum \isi{Vokal} in <WIKR> und <tik> subtrahiert. Die zeitlichen Muster unterscheiden sich nur geringfügig zwischen VCC und VC, um durchschnittlich um $10{ms}$ für alle Sprecher gemittelt. 

\begin{figure}
	\includegraphics[width=.8\textwidth]{figures/3-15_coda_KR_polish.png}
	\caption{Durchschnittliche Werte (Latenzen in $ms$) für die Stabilität von /k/ in <tik> versus <WIKR> (graue Balken) und die starke Rechtsbewegung von /r/ in <tir > versus <WIKR> (schwarze Balken).}
	\label{figure:0315}
\end{figure}

Für die \isi{Rechtsbewegung} von ${C}_{2}$ (/r/) wurden die zeitlichen Abstände in <WIKR> und <tik> substrahiert. Positive Werte markieren die Verschiebung nach rechts weg vom vorangehenden \isi{Vokal}. Die Verschiebung ist stark und zeigt im Durchschnitt eine Zunahme der Latenz um $96{ms}$ für alle Sprecher gemittelt. Wird ein Konsonant hinzugefügt, so verändert er das Verhältnis zwischen VC nicht.


\subsection{Interpretation der Daten}
\label{subsec:030403}

Abbildung~\ref{figure:0316} zeigt schematisch die Organisation von konsonantischen und vokalischen Gesten für wortinitiale (links) und wortfinale (rechts) Cluster im \ili{Polnisch}en. Die Abbildung basiert auf den zeitlichen Abstandsmessungen vom \isi{Target} des Konsonanten relativ zum \isi{Target} des Vokals. 

\begin{figure}
	\includegraphics[width=.8\textwidth]{figures/3-16_schemaVerschiebung.png}
	\caption{Schematische Repräsentation der Organisation von Konsonanten und Vokalen, links für eine Zentrumstabilität (center-stability) in wortinitialer Position und rechts für eine Stabilität des linken Randes (left-edge stability) in wortfinaler Position.}
	\label{figure:0316}
\end{figure}

In wortinitialer Position CCV versus CV findet zugunsten einer Stabilität des Zentrums relativ zum nachfolgenden \isi{Vokal} eine Verschiebung von  ${C}_{1}$ und  ${C}_{2}$ relativ zum folgenden \isi{Vokal} statt. Diese Verschiebung kann als Evidenz für konkurrierende Zielspezifikationen im \isi{Kopplungsgraph} der mit den Gesten assoziierten Oszillatoren gewertet werden. Im Vergleich zu Synchronizität in CV-Silben startet  ${C}_{1}$ früher (leftward shift) und  ${C}_{2}$ später (rightward shift). Im Vergleich CV zu CCV verkürzt sich der Abstand des rechten Konsonanten (rightmost C, right edge, hier /r/) relativ zum \isi{Vokal}, während er sich für den linken Konsonanten (leftmost C, left edge, hier /k/) vergrößert. Es liegt eine Stabilität für die Koordination der \isi{Konsonantenverbindung} als Einheit mit einem einzelnen Zentrum (center stability) vor. Insbesondere die Messung für C am rechten Rand der \isi{Konsonantenverbindung} ist ein wichtiger Indikator für die Organisation des Clusters. Veränderte sich dieser Abstand am rechten Rand zwischen CV und CCV nicht, so wäre von einer Stabilität des rechten Randes auszugehen (right edge stability, ${C}_{2}$ und V starten auch in einer CCV-Verbindung synchron), die gegen eine Annahme komplexer Onsets spräche.

Bei wortfinalen Clustern findet sich eine Stabilität des linken Cluster-Randes relativ zum vorangehenden \isi{Vokal} in VC versus VCC (left edge stability). Zwischen VC und V ${C}_{1}{C}_{2}$ findet sich keine Verschiebung von  ${C}_{1}$ hin zu dem vorangehenden \isi{Vokal}. Dieses sind Evidenzen für die Annahme, dass im \isi{Kopplungsgraph} der mit den Gesten assoziierten Oszillatoren keine konkurrierenden Zielspezifikationen vorliegen. Nur  ${C}_{1}$ ist direkt mit dem \isi{Vokal} in einer Out-of-Phase-Spezifikation gekoppelt. Über die genaue Art der \isi{Kopplung} von  ${C}_{2}$ mit  ${C}_{1}$ lassen sich aufgrund der Daten jedoch keine konkreten Aussagen machen, d.\,h. die genaue Art der Out-of-Phase-Spezifikation lässt sich nicht ableiten. Damit kann die Silbenzugehörigkeit von  ${C}_{2}$ in der wortfinalen Obstruent-Sonorant-Verbindung (es liegt zumindest eine Sonoritätsverletzung vor) nicht direkt mit kinematischen Bewegungsmustern geklärt werden.