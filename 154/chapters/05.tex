\chapter{Prosodische Analyse}
\label{chap:05}

Um die Wechselwirkung zwischen \isi{Artikulation} und prosodischer Struktur zu verstehen, wird in diesem Kapitel eine Einführung in die Grundlagen der prosodischen Stärkung mit den entsprechenden Definitionen gegeben. Die \isi{Enkodierung} der prosodischen Struktur geschieht auf verschiedenen Ebenen. Dabei besteht in vielen Sprachen, und hier vor allem in westgermanischen Sprachen, eine enge Verflechtung u.a. zwischen der Markierung von Akzent bzw. \isi{Tonakzent} auf Äußerungsebene (accent), Wortakzent auf lexikalischer Ebene (stress) und Phrasengrenzen (boundary). \isi{Tonakzent} und Wortakzent bestimmen die \isi{Prominenz} (prominence) und Phrasengrenzen die Position (position) einer \isi{Konstituente} in einer Phrase.

Die Unterscheidungen von \isi{Prominenz} und Position basieren auf der Auffassung des Autosegmental-Metrischen-Modells, dass sich kleinere prosodische Konstituenten, wie die \isi{Silbe} oder Wörter, in größere übergeordnete Einheiten wie die \isi{Intonationsphrase} gruppieren \citep[vgl. Übersicht in][]{Shattuck1996, Beckman1996, Gussenhoven2004, Keating2004}, denen tonale Ereignisse (Tonakzente, Grenztöne) zugewiesen werden. Gemeinsam ergeben sie die \isi{prosodische Struktur} einer bestimmten Sprache.


\begin{figure}[b]
% 	\includegraphics[height=35mm]{figures/a06-img1.png}
	\caption{Vereinfachte prosodische Hierarchie \citep[nach][779]{Grice2006}.}
	\label{figure:0501}
\small
\begin{forest}
[
[Äußerung, no edge
  [Intonationsphrase, no edge
    [Kleine Phrase, no edge
      [Phonologisches Wort, no edge
	[Fuß, no edge 
	  [Silbe, no edge]
	]
      ]
    ]
  ]
]
[U, no edge
  [IP [XP [W [F 
	  [{\parbox[t]{2mm}{~}}, no edge]
	  [s]
	  [{\parbox[t]{2mm}{{\dots}}}, no edge]
  ]]]]
  [IP
    [XP
      [W
	[F
	  [s]
	  [s]
	]
	[F
	  [s]
	  [s]
	]
      ] 
      [W
	[F 
	  [{\dots}, no edge]	
	]
	[F]
      ]
    ]    
    [XP
      [W] 
      [W]
    ]
  ] 
]
]  
\end{forest}

	
\end{figure}

Der genaue Aufbau einer prosodischen Hierarchie ist nicht nur sprach- sondern auch theoriespezifisch, insbesondere die Annahmen von Konstituenten auf Höhe eines intermediären, unterhalb der \isi{Intonationsphrase} befindlichen Levels. \citet{Grice2006} skizziert eine vereinfachte (theorie- und sprachübergreifende) \isi{prosodische Hierarchie}, vgl. Abbildung~\ref{figure:0501}, und beschreibt den Aufbau dieser Hierarchie wie folgt:

\begin{itemize}
	\item Eine Äußerung (U; Utterance) setzt sich aus mindestens einer \isi{Intonationsphrase} (IP; Intonational Phrase) zusammen.
	\item Eine IP setzt sich aus mindestens einer kleinen Phrase zusammen (XP; Smaller Phrase).
	\item Eine XP setzt sich aus mindestens einem phonologischen Wort (W; Word) zusammen.
	\item Ein phonologisches Wort setzt sich aus mindestens einer \isi{Silbe} (s; Syllable) zusammen.
\end{itemize}

Es sei hier kurz angemerkt, dass auf die kleine Phrase in dem Schema nach \citet{Grice2006} mit XP verwiesen wird, weil dieser Level unterschiedliche Strukturen zeigen kann, die beispielsweise einer Phonologischen Phrase, Intermediärphrase oder Akzentphrase entsprechen.

Abbildung~\ref{figure:0502} zeigt ein Beispiel für eine Oberflächenrepräsentation der \il{Englisch}englischen Äußerung <Too many cooks spoil the broth> aus \citet{Gussenhoven2004}. Gussenhoven nimmt an, dass sich die Äußerung in zwei Intonationsphrasen aufteilt. In seiner Analyse bekommt jede \isi{Intonationsphrase} initiale und finale Grenztöne zur Markierung der Grenzen zugewiesen, sowie Akzenttöne zur Markierung von prosodischen Köpfen in der \isi{Intonationsphrase}.

\begin{figure}[b]
% 	\includegraphics[height=.3\textheight]{figures/a06-img2.png}
\small
\begin{forest}    
[
[Äußerung, no edge
  [Intonationsphrase, no edge
    [Phonologische Phrase, no edge
      [Phonologisches Wort, no edge
	[Fuß, no edge 
	  [Silbe, no edge
	    [Segmentale Struktur, no edge
	      [Tonale Struktur, no edge]
	    ]
	  ]
	]
      ]
    ]
  ]
]  
[υ, no edge
  [ι
    [φ
      [ω
	[F
	  [σ, name=52sigma1
	    [tuː, roof
	      [H*, name=52hstar1, no edge]
	    ]
	  ]
	]
      ]
      [ω
	[F  
	  [σ
	    [mɛ, roof]
	  ]
	  [σ
	    [niː, roof]
	  ]
	]
      ]
      [ω
	[F
	  [σ, name=52sigma2
	    [kʊks, roof 
	      [L*+H, name=52lstarh, no edge]
	    ]
	  ]
	]
      ]  
    ]
  ] 
  [ι, name=52iota2
    [φ
      [ω
	[F
	  [σ, name=52nosigma3 
	    [spɔɪl, roof 
	      [H*, name=52hstar2, no edge]
	    ]
	  ]
	]
      ]   	  
    ]
    [φ, name=52phi3
      [~, no edge
	[~, no edge
	  [σ, no edge, name=52pseudo
	    [ðə, roof
	      [~, no edge]
	    ]
	  ]
	]
      ]
      [ω
	[F
	  [σ, name=52sigma4
	    [bɹɔθ,roof
	      [~,no edge]
	      [H*+L,name=52hstarl, no edge]
	      [L$_\iota$, name=52liota, no edge]
	    ]
	  ]
	]
      ]
    ]      
  ] 
]  
]
\draw (52sigma1)   to[out=222, in=north west] (52hstar1);
\draw (52sigma2)   to[out=222, in=north west] (52lstarh);
\draw (52nosigma3) to[out=222, in=north west] (52hstar2);
\draw (52sigma4)   to[out=222, in=north west] (52hstarl);
\draw (52iota2)    to[out=340, in=north] (52liota);
\draw (52phi3)     -- (52pseudo);
\end{forest}

	
	
	\caption{Oberflächenrepräsentation der \il{Englisch}englischen Äußerung <Too many cooks spoil the broth> \citep[nach][124]{Gussenhoven2004}.}
	\label{figure:0502}
\end{figure}

Die \isi{prosodische Struktur} wird nicht nur auf tonaler Ebene durch suprasegmentale Merkmale wie die Grundfrequenz, sondern auch auf nicht-tonaler Ebene durch Modifikationen des supralaryngalen Systems markiert. Die \isi{Enkodierung} durch das supralaryngale System wird als \isi{prosodische Stärkung} bezeichnet (prosodic \isi{strengthening}) und ist von \citet{Cho2006} wie folgt definiert:

\begin{quotation}
	(\dots) robust phonetic phenomena in the vicinity of prosodic boundaries have led to a growing awareness that it is no longer fruitful to describe the sound properties of a language without adequately taking into account the interface between prosodic structure and phonetics. Accordingly, the focus of recent laboratory work has been on more diverse prosodic locations, including domain-initial and -final positions, as well as manifestation of prosodic structure in articulatory variation, as well as stressed (pitch-accented) syllables (de Jong1991, 1995, Cho2002). These three positions have been shown to give rise to some type of \isi{strengthening} of articulatory properties of features or gestures (also known as prosodic \isi{strengthening}), which is taken to be an articulatory signature of prosodic structure \citep[vgl.][520--521]{Cho2006}.
\end{quotation}

Die \isi{prosodische Stärkung} umfasst die Beschreibung temporaler und spatialer Modifikationen in der \isi{Artikulation}. Eine Teilstrategie ist die prosodische Längung (prosodic lengthening), die ausschließlich temporale Modifikationen berücksichtigt (meist an Grenzen; bei Akzenten auch als accentual lengthening bezeichnet). Darüber hinaus hat die \isi{prosodische Stärkung} eine phonologische Basis und bezieht sich in ihrem Wirkungsbereich überwiegend auf die Stärkung von Merkmalen (feature enhancement, \citealt[][3867]{Cho2005a} und \citealt[][521]{Cho2006}) oder gesturalen Deskriptoren in \isi{prosodisch} starken Positionen.

Prosodische Stärkung dient der Kontrastbildung auf syntagmatischer (\isi{Enkodierung} prosodischer Struktur) und paradigmatischer Achse (\isi{Enkodierung} lexikalischer Kontraste in kommunikativ „wichtigen“ Positionen, die fokussierte und/oder neue Informationen enthalten). Für die Stärkung werden auf Ebene der \isi{Artikulation} Eigenschaften von Merkmalen und Gesten in Abhängigkeit des phonologischen Systems räumlich oder zeitlich modifiziert, wenn auch in unterschiedlichem Ausmaß (\citet[][3867]{Cho2005a}, \citet{Cho2005b} und \citet[][521]{Cho2006}). Es werden beispielsweise bei Vokalen nicht alle Ortsmerkmale verstärkt. So findet sich bei \citet{DeJong1993}, \citet{Harrington2000} und \citet{Cho2005a} für Akzentmarkierung bei kontrastivem \isi{Fokus} im Englischen für geschlossene Vokale eine Stärkung des Ortsmerkmals [±hinten], aber nicht von [+hoch]. Auch bei Konsonanten, z.\,B. bei Plosiven, können die verstärkten Parameter variieren. So zeigen sich systematische, durch die Phonologie abgeleitete kinematische Muster für die Markierung stimmloser Plosive [±spread glottis] durch Variationen der \isi{VOT} (Voice \isi{Onset} Time, Stimmeinsatzzeit) an \isi{prosodisch} starken Grenzen im Englischen und Niederländischen \citep{Cho2005b}. Im Englischen vergrößert sich die \isi{VOT} [+spread glottis] an \isi{prosodisch} starken Positionen, während sie sich im Niederländischen verringert [-spread glottis]. \citet{Kuzla2007} zeigen jedoch, dass es Unterschiede zwischen den Sprachen in der \isi{Enkodierung} der prosodischen Struktur durch die Hervorhebung konsonantischer Merkmale gibt. In einer Studie zu Plosiven im Deutschen führe die Stärkung der Verschlussdauer und nicht die der \isi{VOT} zu konsistenten Mustern. Es sei hier kurz angemerkt, dass das Merkmal [±spread glottis] auch mit [±aspiriert] übersetzt werden kann, aber nicht mit dem Merkmal für Stimmbeteiligung [±stimmhaft] gleichzusetzen ist. So bezieht sich [±spread glottis] auf das Vorhandensein einer Aspirationsphase nach der Verschlusslösung (die Stimmlippen werden hierfür gespreizt) und nicht auf das prinzipielle Vorhandensein von Stimmhaftigkeit während des Verschlusses \citep[vgl.][]{Hall2011}.

Prosodische Stärkung kann jedoch nur dann realisiert werden, wenn es die Segmente zulassen. So gibt es Segmente, die \isi{koartikulatorisch} resistent sind und keinen großen Spielraum für prosodische Modifikationen erlauben \citep[coarticulatory resistance, vgl.][]{Hardcastle1999, Tabain2001, Recasens2009, Iskarous2010}. Bestimmte Segmente sind anfällig für kontextuell bedingte Variation, während andere Laute den Einfluss adjazenter Laute blocken. Insbesondere die Laute mit einem hohen Grad an koartikulatorischer Resistenz verfügen meist auch über einen hohen Grad an koartikulatorischer Aggressivität (coarticulatory aggressiveness), d.\,h. sie beeinflussen die benachbarten Laute statt selbst beeinflusst zu werden. 

\citet{Recasens2009} haben gezeigt, dass für den Grad der koartikulatorischen Resistenz verschiedene artikulatorische Beschränkungen verantwortlich sind (articulatory constraints; DAC Model: degree of articulatory constraint). Dabei spielen Ort und Art der Konstriktion eine Rolle. So zeigen \citet{Recasens2009} für Zungenbewegungen im Katalanischen, dass Palatalität und Friktion bei intervokalischen Konsonanten zu einem höheren Grad an koartikulatorischer Resistenz führen als beispielsweise Lateralität. Der Grad der koartikulatorischen Resistenz wird mit Hilfe des Grades der koartikulatorischen \isi{Überlappung} zwischen zwei Gesten \isi{kinematisch} ermittelt \citep[coarticulatory overlap, vgl.][]{Iskarous2010}. Um den Einfluss der prosodischen Struktur auf die supralaryngale \isi{Artikulation} zu messen, muss demnach der Grad der koartikulatorischen Resistenz beachtet werden: ein offener \isi{Vokal} /a/ wird in vergleichbarer prosodischer Position vermutlich mehr Variation als ein geschlossener \isi{Vokal} /i/ zeigen.

Des Weiteren stellt für die \isi{Modellierung} der unterschiedlichen prosodischen Faktoren die Erfassung der unterschiedlichen Domänengrößen der Modifikationen eine besondere Herausforderung dar. Diese ist für die \isi{prosodische Grenze} eng definiert und wirkt sich nur auf die grenz-adjazente Kinematik aus (π-\isi{Geste}, vgl. Kapitel~\ref{sec:0403}). Bei der Akzentmarkierung hingegen scheint -- vermutlich in Abhängigkeit der \isi{Fokusstruktur} -- die Domäne wesentlich größer und variabler zu sein. \citet{Cho2005b} zeigen für das Niederländische auf, dass sie mehr als einen Fuß umfassen kann. Während die Grenzmarkierung durch Anwendung der π-\isi{Geste} (prosodische \isi{Geste}) gut abbildbar ist, ist die \isi{Modellierung} von Akzenten und zugehöriger \isi{Fokusstruktur} noch weitgehend ungelöst (vgl. Kapitel~\ref{chap:06}).

\section{Akzentinduzierte Stärkung}
\label{sec:0501}

In der Literatur wird die \isi{artikulatorische Markierung} von Wortakzent und \isi{Tonakzent} häufig nicht getrennt. Die Strategien sind vergleichbar, und zumeist sind es die lexikalisch starken Silben, die den \isi{Tonakzent} tragen. Während der Wortakzent auf die artikulatorische Stärkung der lexikalisch starken \isi{Silbe} und die Schwächung der lexikalisch schwachen \isi{Silbe} begrenzt ist, kann sich der \isi{Tonakzent} in seiner artikulatorischen Stärkung über mehrere starke und schwache Silben ausbreiten und mit dem Wortakzent auf intra-gesturaler und inter-gesturaler Ebene interagieren
\citep[vgl.][]{Turk1997, Turk1999, DeJong2004, Cho2005a, Cho2005b, Cho2006, Cho2009}.

\subsection{Hyperartikulation und Sonoritätsexpansion}
\label{subsec:050101}

In dem H\&H-Modell führt \citet{Lindblom1990} das Konzept der \isi{Hyperartikulation} ein und verweist auf den adaptiven Charakter gesprochener Sprache. Je nach Sprechstil und Einschätzung des Hörers durch den Sprecher nutzt der Sprecher ein Kontinuum von Hypo- zur \isi{Hyperartikulation} \citep{Liberman1985, Farnetani2010}. Der Sprecher bringt in Abhängigkeit des kommunikativen Nutzens ein unterschiedliches Maß an artikulatorischem Aufwand auf. Bei sehr sorgfältigem Sprechen ist der Aufwand hoch und die Sprache hyperartikuliert. Dabei nimmt der Grad an \isi{Überlappung} zwischen den Gesten ab und das Maß an \isi{Koartikulation} wird reduziert (\isi{Hyperartikulation}). Bei verschliffenem Sprechen hingegen ist der Aufwand niedrig und die Sprache hypoartikuliert. Der Überlappungsgrad zwischen zwei Gesten nimmt zu, und somit treten mehr Reduktions- und Assimilationserscheinungen auf der akustischen Oberfläche auf \citep[Hypoartikulation; vgl. auch][]{DeJong1993,Kröger1998}.

Während das physiologische Kontrollsystem aus Gründen der Artikulationsökonomie eine Minimierung des artikulatorischen Aufwandes anstrebt, kann dagegen die \isi{linguistische Struktur} einen höheren Aufwand erfordern. Aus diesem Spannungsfeld entsteht eine enorme Spannweite an konkreten Realisierungsformen, die in Form von Parametermodifikationen im Rahmen eines Feder-Masse-Modells abbildbar sind (vgl. Kapitel~\ref{chap:04}). Verändert man die die Werte nur eines gestischen Parameters, so verändern sich die damit verbundenen zeitlichen und räumlichen Eigenschaften des artikulatorischen und akustischen Signals \citep[vgl.][]{Saltzman1986, Saltzman1987, Saltzman1989, Browman1989, Browman1992a}. Entsprechende Parametermodifikationen wären beispielsweise Veränderungen im \isi{Target} einer \isi{Geste} (extremere oder präzisere Zielpositionen für \isi{Hyperartikulation}), \isi{Steifheit} (Verringerung der relativen Geschwindigkeit für \isi{Hyperartikulation}), \isi{Reskalierung} der \isi{Geste} (Kombination aus extremerer \isi{Zielposition} und Verringerung der \isi{Ausführungsgeschwindigkeit} für \isi{Hyperartikulation}) und \isi{Phasing} (weniger \isi{Überlappung} zwischen zwei Gesten für \isi{Hyperartikulation}). 

Die akzentinduzierte Stärkung verwendet -- abgesehen von der Platzierung eines Tonakzents -- unterschiedliche artikulatorische Strategien: die lokalisierte \isi{Hyperartikulation} und die \isi{Sonoritätsexpansion}. Hierzu gibt es unterschiedliche Studien, u.a. zu Vokalproduktion im \ili{Englisch}en
\citep[vgl.][]{Beckman1992,DeJong1993,Harrington2000,Erickson2002,Cho2005a}, 
\ili{Italienisch}en \citep{Avesani2007} und 
\ili{Französisch}en \citep{Dohen2006}. Die lokalisierte \isi{Hyperartikulation} nimmt primär Einfluss auf die Kontrastbildung der paradigmatischen Achse (Stärkung von Ortsmerkmalen). Die \isi{Sonoritätsexpansion} hingegen agiert auf der syntagmatischen Achse, um beispielsweise den vokalischen Nukleus vom Silbenrand oder eine starke von einer schwachen \isi{Silbe} abzugrenzen \citep[Stärkung der intrinsischen \isi{Sonorität}, vgl.][]{DeJong1993,Harrington2000}. Häufig wird die Hyperatikulation mit unterschiedlichen Zungenkonfigurationen (z.\,B. extreme Vokalpositionen) in Verbindung gebracht, während für die \isi{Sonoritätsexpansion} die Lippenbewegungen untersucht werden (eine größere Lippenöffnung führt dazu, dass mehr Energie vom Mundraum abgestrahlt werden kann). Bei genauerer Betrachtung existiert jedoch keine klare Abgrenzung zwischen den Strategien. So betrifft erstens bei Labialen die \isi{Hyperartikulation} auch direkt die Lippen. Zweitens wirkt die \isi{Hyperartikulation} aufgrund ihres Wechsels mit Hypoartikulation auch auf der syntagmatischen Achse und kodiert somit \isi{prosodische Struktur} wie die Hervorhebung des Kopfes einer prosodischen Phrase \citep{Dohen2006}. Drittens kann \isi{Sonorität} als ein weiteres distinktives Merkmal [+sonorant] aufgefasst werden, das hyperartikuliert wird \citep[vgl.][]{DeJong1995,Harrington2000,Cho2005a}. 

\begin{figure}[hptb]
	\includegraphics[height=.4\textheight]{figures/5-3_Hyper.png}
	\caption{Schaubild über die Wirkungsfelder von Hyperartikulation und Sonoritätsexpansion.}
	\label{figure:0503}
\end{figure}

Der Wechsel von Hyper- und Hypoartikulation kann jedoch nicht nur dem globalen Sprechstil zugeordnet werden, sondern auch lokal innerhalb einer Äußerung stattfinden. So markiert der Sprecher für den Hörer wichtige Information beispielsweise bei der Akzentuierung eines Wortes und verstärkt durch lokalisierte \isi{Hyperartikulation} den lexikalischen Kontrast \citep{DeJong1993,DeJong1995}. Die lokalisierte \isi{Hyperartikulation} wird im Folgenden am Beispiel der Produktion von offenen Vokalen im \ili{Tashlhiyt} Berber verdeutlicht, eine Sprache, die im Süden von Marokko gesprochen wird. Das Besondere an dieser Sprache ist, dass es keine Restriktionen für den Silbennukleus gibt. Vielmehr können alle Konsonanten -- und somit auch Obstruenten -- silbische Funktion übernehmen \citep[vgl.][]{Dell1985, Ridouane2008, Hermes2011a}. Dabei kommt es zu vokallosen Äußerungen, die beispielsweise nur aus Obstruenten bestehen können. Die folgenden Daten stammen aus einer Studie von \citet{Diercks2011}.

\citet{Diercks2011} untersucht die Produktion von /a/ im Targetwort <inna> in akzentuierter und unakzentuierter Testbedingung. Insgesamt enthält die Studie 233 Tokens eines männlichen Sprechers, die einen offenen \isi{Vokal} /a/ in akzentuierter und unakzentuierter Position der prosodischen Hierarchie enthalten. Das jeweilige Targetwort ist Bestandteil des Trägersatzes <Inna \longrule bahra.>, vgl. Beispiel~\ref{ex:0504} und~\ref{ex:0505}. In Abhängigkeit vom nachfolgenden Wort trägt <inna> einen Akzent oder ist unakzentuiert. Hat das Folgewort einen vokalischen bzw. sonorantischen Nukleus, so trägt <inna> bei der dargestellten Äußerungsstruktur keinen Akzent. Ist der Nukleus jedoch nicht-sonorantisch, so fällt der Akzent auf <inna> \citep[vgl. die tonale Analyse von][]{Grice2011}.

%%(5.4)
\begin{exe}
	\ex Unakzentuierte Testbedingung:\label{ex:0504}
	\sn <Inna GZMT bahra.>
	\sn (Er sagte oft: Zerreiße es.)
\end{exe}

%%(5.5)
\protectedex{
\begin{exe}
	\ex Akzentuierte Testbedingung:\label{ex:0505}
	\sn <INNA tbdgt bahra.>
	\sn (Er sagte oft: Du bist nass.)
\end{exe}%
}

Abbildung~\ref{figure:0504} zeigt Ergebnisse für die akustischen Messungen des Zielwortes <inna>. Es handelt sich um eine Formantkarte mit dem ersten und zweiten Formanten für den \isi{Vokal} /a/ (in Hz). In akzentuierter Position zeigt sich für /a/ eine Anhebung des ersten Formanten (F1) um durchschnittlich  $58$~Hz (vgl. schwarz durchgezogene Ellipse). Eine solche Anhebung von F1 korreliert mit einem größeren Öffnungsgrad, d.\,h. der \isi{Vokal} wird offener realisiert. Der zweite Formant (F2) hingegen verändert sich im Mittel kaum, d.\,h. der \isi{Vokal} zeigt akustisch keine Rück- oder Vorverlagerung der Artikulationsstelle.


\begin{figure}
	\includegraphics[height=.3\textheight]{figures/5-4_F1F2_Tash_Kopie.png}
	\caption{Formantkarte F1 und F2 (in Hz) für die Produktion von /a/ in <inna> in akzentuierter (durchgezogene Ellipse) und unakzentuierter (gestrichelte Elipse) Position in Tashliyt Berber \citep[nach][]{Diercks2011}.}
	\label{figure:0504}
\end{figure}

\begin{figure}
	\includegraphics[height=.3\textheight]{figures/5-5_TrajektTash_Kopie.png}
	\caption{Gemittelte Trajektorien für den Zungenrücken in <inna>, vertikale Position. Bei hohen Werten ist der Zungenrücken angehoben, bei niedrigen Werten abgesenkt \citep[nach][]{Diercks2011}. Die gestrichelte Trajektorie bildet den Bewegungsverlauf des Zungenrückens in unakzentuierter Position ab, die durchgezogene Trajektorie in akzentuierter Postion.}
	\label{figure:0505}
\end{figure}


Abbildung~\ref{figure:0505} veranschaulicht die zugehörigen kinematischen Daten (elektomagnetische \isi{Artikulographie}) für den Zungenrücken bei der Produktion des offenen Vokals /a/. Es handelt sich um gemittelte Trajektorien, die die vertikale Bewegung des Zungenrückens vom /i/ zum /a/ in <inna> abbildet. Es zeigt sich, dass in akzentuierter Position (gestrichelte \isi{Trajektorie}) der Zungenrücken weiter abgesenkt wird und eine größere \isi{Bewegungsauslenkung} stattfindet. Der \isi{Vokal} /a/ wird in prominenter Position offener realisiert. 



Die dargestellten Modifikationen von /a/ stehen im Einklang mit der Strategie der lokalisierten \isi{Hyperartikulation}. Fällt der \isi{Tonakzent} auf <inna>, so wird die Spezifikation des Ortsmerkmals [+tief] von /a/ verstärkt. Der \isi{Vokal} wird mehr in Richtung Peripherie des Vokalraums artikuliert, um den Kontrast zu anderen Vokalen, die potentiell im selben Kontext vorkommen könnten, zu vergrößern \citep[][41]{Harrington2000}. Die Stärkung des lexikalischen Kontrastes hilft dem Hörer, eine \isi{Silbe} als Kopf einer \isi{Intonationsphrase} zu interpretieren und ein wichtiges Wort wahrzunehmen \citep[][210]{DeJong1993}.

Die dargestellten Modifikationen für /a/ stehen aber auch im Einklang mit der \isi{Sonoritätsexpansion}. So wird eine Stärkung des Öffnungsgrades der Zunge zumeist von einer größeren Kieferöffnung und einer größeren Öffnung zwischen den Lippen begleitet. Diese Strategie führt dann auch zu einer Erhöhung der abgestrahlten Schallfülle und somit zu einer Erhöhung der intrinsischen \isi{Sonorität} \citep{Sievers1876} des jeweiligen Lautes \citep{Beckman1992}. Durch Modifikationen des supralaryngalen Systems wird der Vokaltrakt stärker und länger geöffnet, um einen lauteren und besser hörbaren \isi{Vokal} zu produzieren \citep{Harrington2000}. Umgekehrt wird bei Konsonanten der Vokaltrakt stärker geschlossen, um Konsonanten von den nachfolgenden Vokalen abzuheben \citep{Cho2005a}; vgl. auch \citet{Vennemann1988} zur konsonantischen Stärke als reziproke Skala zur \isi{Sonorität}. Die größere Öffnung bei /a/ verstärkt also den Kontrast zwischen \isi{Vokal} und vorhergehendem Konsonanten und somit deren beider Funktionen innerhalb der \isi{Silbe} als Nukleus und Silbenrand. Außerdem verstärkt sie den Kontrast zwischen einer \isi{Akzentsilbe} und den angrenzenden unakzentuierten Silben.

\subsection{Hyperartikulation und Sonoritätsexpansion im Konflikt}
\label{subsec:050102}

Insbesondere bei geschlossenen Vokalen können \isi{Hyperartikulation} und \isi{Sonoritätsexpansion} konfligieren. So würde beim \isi{Vokal} /i/ die Ortsspezifikation [+vorn] und [+hoch] eine Verengung des Sprechtraktes bedeuten, um den paradigmatischen Kontrast durch \isi{Hyperartikulation} zu verstärken. Gleichzeitig würde die Stärkung der intrinsischen \isi{Sonorität} des Nukleus /i/ zum Silbenrand eine stärkere Öffnung des Vokaltraktes bedeuten und antagonistisch zur \isi{Hyperartikulation} wirken \citep{Harrington2000}. Für die Frage nach dem Ausmaß der Realisierung der Strategien macht es also durchaus Sinn neben offenen Vokalen (kumulative Effekte) auch geschlossene Vokale zu untersuchen \citep[Konflikt der Strategien,][]{Cho2005a}.

\citet[][210]{DeJong1993} untersuchen die die Zungenkonfigurationen des geschlossenen Hinterzungenvokals /ʊ/ im Zielwort <put> für das Englische. Der \isi{Vokal} /ʊ/ ist als [+hoch] und [+hinten] spezifiziert. Sie beziehen drei Testkonditionen für die Phrasenposition des Targetwortes in die Untersuchung mit ein: unakzentuiert, akzentuiert pränuklear und akzentuiert \isi{nuklear}. Die Autoren finden eine systematische Rückverlagerung des Zungenrückens in akzentuierter versus unakzentuierter Position für die \isi{Hyperartikulation} des vokalischen Merkmals [+hinten]. Aber auch innerhalb des Akzents gibt es graduelle Abstufungen. So ist die Rückverlagerung des Zungenkörpers stärker in nuklearer als in pränuklearer Position der prosodischen Hierarchie \citep{Shattuck1996}. Der Nukleus der \isi{Intonationsphrase} würde somit am stärksten hyperartikuliert. Es findet demnach nicht nur eine Stärkung des lexikalischen Kontrastes statt, sondern auch auf der syntagmatischen Achse eine \isi{Enkodierung} der Akzentstärken. In einer weiteren Studie untersucht \citet{DeJong1995} Zungenpositionen und Kieferbewegungen, auch für /ʊ/ in <put>. Diesmal beschreibt er jedoch nur \isi{Nuklearakzent} versus Hintergrund. Er findet in Nuklearakzenten (kontrastiver \isi{Fokus}) eine Rückverlagerung des Zungenrückens für die \isi{Hyperartikulation} von [+hinten] bei gleichzeitiger Maximierung der Kieferöffnung für die Expansion der \isi{Sonorität}. An diesem Beispiel lässt sich noch ein weiterer Aspekt verdeutlichen. So werden geschlossene Vokale nicht mit einer Verstärkung des Merkmals [+hoch] realisiert, sondern mit einer Rückverlagerung des Zungenrückens [+hinten]. Das liegt vermutlich daran, dass es bei einer erhöhten Zungenposition zu einer geräuschverursachenden Engebildung käme und der \isi{Vokal} zum Frikativ würde.

\citet{Harrington2000} untersuchen offene und geschlossene Vokale /a/ und /i/ im Australischen Englisch. Sie testen die Stimuli <Dr. Barber> und <Dr. Beaber> in \isi{nuklear} akzentuierter Position (kontrastiver \isi{Fokus}) und in postnuklear deakzentuierter Position. Für /a/ zeigt sich in prominenter Position die erwartete stärkere Absenkung von Kiefer und Zungenrücken als kumulativer Effekt. Für den \isi{Vokal} /i/ zeigen sich starke sprecherspezifische Variationen. Ein Sprecher konnte \isi{Hyperartikulation} und \isi{Sonoritätsexpansion} so vereinbaren, dass die jeweiligen Targets nacheinander erreicht wurden. Zunächst wurde für /i/ in kontrastivem \isi{Fokus} der Kiefer für die Sonoritätsmarkierung stärker abgesenkt und dann die Zunge für die \isi{Hyperartikulation} des Merkmals [+vorn] vorverlagert.

Die Beobachtungen von \citet{DeJong1993} und \citet{Harrington2000} zur \isi{Sonoritätsexpansion} und \isi{Hyperartikulation} von geschlossenen Vokalen werden von \citet{Cho2005a} für das Englische bestätigt. So wird /i/ weiter vorn, aber nicht weiter geschlossen realisiert. Er zieht das Fazit, dass durch akzentinduzierte Stärkung nur die Ortsmerkmale verstärkt werden, die nicht direkt mit der \isi{Sonoritätsexpansion} konfligieren.

Bei der Untersuchung von kontrastivem \isi{Fokus} im Französischen untersuchen \citet{Dohen2006} Lippen- und Kieferbewegungen bei der Produktion gerundeter Vokale. Ihrer Studie nach ist die Lippenrundung (lip protrusion, Lippenvorstülpung) der stärkste hyperartikulierte Parameter, um fokale Positionen von prä- und postfokalen Positionen zu unterscheiden. Sie beschreiben die Verwendung zweier Strategien, die \isi{sprecherspezifisch} sind: absolut und differentiell. Bei der absoluten Strategie ist nur die fokale Position hyperartikuliert, während die prä- und postfokale Position eine „neutrale“ \isi{Artikulation} zeigen. Bei der differentiellen Strategie ist die fokale Position hyperartikuliert und die postfokale Position hypoartikuliert. Bei genauerer Betrachtung kann dieses Phänomen auch als postfokale Kompression auf nicht-tonaler Ebene (vgl. auch \citealt{Xu2011} zur tonalen postfokalen Kompression) zur Markierung der prosodischen Struktur eingeordnet werden, die auf syntagmatischer Achse operiert. 

\section{Grenzinduzierte Stärkung}
\label{sec:0502}

An prosodischen Grenzen wird die \isi{Artikulation} vor (domäneninitial) und nach (domänenfinal) der Grenze verstärkt. Die Domäne der Grenzmarkierung ist in ihrer phonetischen Manifestation lokal an die Grenze selbst gebunden. Je stärker die Grenze ist, desto stärker ist ihr Effekt auf die Realisierung der grenzbenachbarten Konstriktionen (vgl. \citealt{Fougeron1997} und \citealt{Cho2009} für kumulative Effekte bei der Stärkung domäneninitialer Konsonanten).

Neuere Studien zeigen, dass die \isi{prosodische Stärkung} neben temporalen auch spatiale Modifikationen beinhalten kann. Diese Modifikationen betreffen häufig sowohl die domäneninitiale als auch die domänenfinale Position, beispielsweise bei der V\#C-\isi{Artikulation} im Englischen. Dennoch variieren die Angaben über den Grad und die Domäne der Stärkung. Bei einer  ${C}_{1}{V}_{1}${}-Sequenz herrscht Konsens, dass in jedem Fall ${V}_{1}$ und ${C}_{2}$ von einer temporalen und eventuell auch räumlichen Stärkung betroffen sind. Je nach Untersuchungsgegenstand können diese Modifikationen dann auch ${C}_{1}$ und ${V}_{2}$ betreffen \citep[u.a.][]{Beckman1990,Edwards1991,Beckman1992,Fougeron1997,Fougeron2001,Byrd2003a,Tabain2003a,Tabain2003b,Cho2005a,Cho2006,Kuzla2007,Cho2009}.

\subsection{Domäneninitiale Stärkung}
\label{subsec:050201}

Konsonanten am linken Rand einer Domäne werden \isi{prosodisch} verstärkt, jedoch ist die Realisierung einer solchen Stärkung vermutlich sprachspezifisch. So haben \citet{Fougeron1997} für das Englische, \citealt{Fougeron2001} für das Französische und \citet{Keating2004} für das Koreanische und Taiwanesische systematisch größere Bewegungsauslenkungen in Form von größeren Zungen-Gaumen-Kontakten im EPG (Elektropalatographie) gefunden. In phrasen-initialer Position werden Konsonanten mit stärkeren Bewegungsauslenkungen realisiert (längere Dauer der konsonantischen Plateaus häufig begleitet von stärkeren räumlichen Kontakten zwischen Zunge und Gaumen) als in phrasenmedialer Position. Für wortinitiale Konsonantencluster im Deutschen hingegen finden \citet{Bombien2010} zwar einen Zusammenhang zwischen der Grenzstärke und der Dauer der Plateaus; sie finden aber keine Hinweise auf größere, räumliche Bewegungsauslenkungen im EPG. In einer ${C}_{1}{C}_{2}$ V-Sequenz wird ${C}_{1}$ stärker von der Position (Grenze) und ${C}_{2}$ stärker von der \isi{Prominenz} (lexikalischer Wortakzent) beeinflusst.

Vergleichende Studien zum Niederländischen und Englischen haben gezeigt, dass die Stärkung eine phonologische Basis hat \citep{Cho2005b}. So wurde die \isi{VOT} im Englischen an starken Grenzen länger und im Niederländischen kürzer. Die systematisch längere \isi{VOT} im Englischen entspricht der Stärkung des laryngalen Merkmals [+spread glottis] an \isi{prosodisch} starken Grenzen, während die gegenläufige Verkürzung der \isi{VOT} im Niederländischen der Stärkung des Merkmals [-spread glottis] entspricht. In beiden Sprachen wurden die Plosive mit längeren Verschlussdauern an \isi{prosodisch} starken Grenzen realisiert.

\citet{Kuzla2007} untersuchen für das Deutsche die prosodische Verstärkung von Plosiven an Grenzen. Sie finden eine systematische Längung der Verschlussdauern von grenzinitialen Plosiven. Die \isi{VOT} jedoch nimmt, obwohl die meisten Varianten des Deutschen für den fortis-lenis-Kontrast zwischen Plosiven laryngal mit [+spread glottis] spezifiziert sein sollten, an starken Grenzen ab. Sie diskutieren ihre Ergebnisse im Hinblick auf die Frage, ob es sich dabei um Merkmalshervorhebung (feature enhancement) oder Fortisierung (fortition) handelt. Die Fortisierung ist die gegenläufige Strategie zur Lenisierung (lenition) und beinhaltet die Idee, einen Plosiv in seiner Artikulationsart stärker hervorzuheben. So können beispielweise im Zuge von Assimilationsprozessen Frikative als Plosive realisiert werden. \citet{Kuzla2007} finden keine einheitliche Antwort, um welche Strategie es sich handelt. Vielmehr können sie in ihrer Studie zeigen, dass bei der domäneninitialen Verstärkung von Konsonanten unterschiedliche phonetische Messungen in Betracht gezogen werden sollten, da es nicht unbedingt laryngale Merkmale sind, die \isi{prosodisch} verstärkt werden. Es sei hier jedoch angemerkt, dass die Untersuchungen von Verschlussdauern grenzinitialer Plosive häufig methodisch problematisch sind. An großen Grenzen wie einer \isi{Intonationsphrase} ist der Beginn des Verschlusses häufig nicht im akustischen Signal bestimmbar. So ist in diesen Fällen nicht klar, ob der stumme Schall der Verschlussphase dem Plosiv oder einer eventuell vorangehenden Pause zuzuordnen ist. Annotationskriterien wie „die zweite Hälfte der stummen Phase zählt zum Plosiv, die erste Hälfte zur vorangehenden Pause“ sind stark hypothetisch und wenig reliabel.

Auch Vokale können nach einer Grenze \isi{prosodisch} gestärkt werden. So finden \citet{Tabain2003a} und \citet{Tabain2003b} für /a/ an \isi{prosodisch} starken Grenzen temporale und spatiale Modifikationen in Form von größeren Bewegungsamplituden der Zunge und des Kiefers in domänenfinaler Position. Für /i/ finden sie vor allem eine \isi{temporale Modifikation}.

Ähnliches bestätigt sich für Vokale im Französischen. So untersuchen \citet{Georgeton2014} die Akustik und Lippenkinematik von zehn Vokalen im Französischen, /i, e, ɛ, a, y, ø, oe, u, o, ɔ/, jeweils an kleinen und großen Grenzen (wortinitial und Intonationsphraseninitial). Es zeigt sich ein systematischer Unterschied im Öffnungsgrad der Lippen: So zeigen alle Vokale größere Lippenöffnungen an großen Grenzen im Vergleich zu kleinen Grenzen. Der Effekt ist jedoch stärker und robuster für die ungerundeten als für die gerundeten Vokale, wodurch nach \citet{Georgeton2014} der Kontrast zwischen dem Merkmal [±gerundet] \isi{prosodisch} gestärkt wird.

\subsection{Domänenfinale Stärkung}
\label{subsec:050202}

\citet{Beckman1990}, \citet{Edwards1991} und \citet{Beckman1992} beschreiben das Phänomen der finalen Längung (final lengthening, preboundary lengthening) prosodischer Konstituenten im Englischen am Beispiel von CVC Silben. Mit Hilfe von Messungen der Kieferbewegungen (transvokalische Öffnung des Kiefers und anschließender Verschluss in geschlossenen Silben wie /pɑp/) kommen sie zu dem Schluss, dass in finaler Position überwiegend temporale Modifikationen auftreten, die sich parametrisch in Steifheitsvariationen der \isi{Geste} ausdrücken können \citep[vgl. auch][]{Wightman1992}. So werden final häufig Gesten gelängt, aber es finden sich keine zusätzlichen räumlichen Modifikationen. Eine geringere \isi{Steifheit} der entsprechenden \isi{Geste} führt genau zu diesem Effekt: Langsamere \isi{Artikulation} mit gleichbleibender \isi{Zielposition}. Dieser Effekt wird in der Literatur auch schon mal als domänenfinale Schwächung (statt Stärkung) bezeichnet, da spatiale Modifikationen auszubleiben scheinen.

Es ist jedoch unklar, ob finale Grenzmarkierung nicht doch auch Modifikationen spatialer Art beinhalten kann. \citet{Cho2005a} untersucht beispielsweise akzentuierte und nicht-akzentuierte CV-Silben im Englischen vor prosodischen Grenzen. Für /ɑ/ in Silben wie /bɑ/ findet er sowohl temporale als auch spatiale Modifikationen in Form von größeren Bewegungsauslenkungen für den Zungenrücken und die Lippenöffnung. Der Kiefer hingegen wird nur durch den Akzent, nicht aber durch die Grenze modifiziert. Er schlussfolgert, dass in finaler Position nicht nur eine temporale, sondern auch systematisch eine spatiale Stärkung im Sinne der \isi{Sonoritätsexpansion} und der lokalisierten \isi{Hyperartikulation} von Ortsmerkmalen stattfindet. Dies würde erst sichtbar, wenn über den Kiefer hinaus das labiale und linguale System in die Untersuchung miteinbezogen würden. So sei der Kiefer vermutlich weniger sensitiv für die Markierung prosodischer Grenzen. 

\begin{quote}
	(\dots) the results of the present study do indicate that domain-final articulation is marked not only by temporal expansion but also by the enhancement of sonority and place features at a higher prosodic boundary—that is, by \isi{strengthening} as well as lengthening. (\dots) The data presented in this paper show that among the three articulators that were examined, it is only the jaw maxima that do not show expanded articulation for domain-final /ɑ/ (\dots), showing less sensitivity of the jaw to prosodic boundaries. \citep[3876]{Cho2005a}
\end{quote}

Weitere Evidenzen für die Kombination aus zeitlicher und räumlicher Stärkung domänenfinaler Vokale im Englischen stammen von \citet{Fougeron1997}. Sie finden für finale Vokale und die folgenden domäneninitialen Konsonanten eine extremere linguale \isi{Artikulation}. Diese Stärkung von Vokalen in ihrer gesamten \isi{Bewegungsauslenkung} führt grenzübergreifend zu einer deutlicheren V-zu-C-\isi{Artikulation}. \citet[][3728]{Fougeron1997} führen an, dass prosodische Stärke als ein alternativer Ansatz zur \isi{Deklination}, wie sie in anderen Studien beschrieben ist, betrachtet werden könnte.

\citet{DiNapoli2012} untersucht Glottalisierungen in wortfinalen Vokalen im \ili{Italienisch}en (Nord- und Zentralitalien). Sie zeigt in ihrer Studie, dass Glottalisierungen besonders häufig vor prosodischen Grenzen auftreten und als Grenzmarker benutzt werden. Dabei handelt es sich um eine Stärkung vor einer prosodischen Grenze, weil mit der Glottalisierung das Merkmal [+constricted glottis] vor der Grenze eingefügt werde. 

Abbildung~\ref{figure:0506} zeigt ein Beispiel für die Insertion einer Glottalisierung vor einer Phrasengrenze. Die Realisierung stammt von einem zentralitalienischen Sprecher aus der Studie von \citet{DiNapoli2012}. Das Zielwort <faro> (Leuchtturm) steht in phrasenfinaler Position (vgl. Beispiel~\ref{ex:0508}). Der Trägersatz gliedert sich in zwei Phrasen und lautet mit eingebettetem Zielwort:

%%(5.8)
\begin{exe}
	\ex Zielwort <faro> in phrasenfinaler Position:\label{ex:0508}
	\sn{[Dico la parola faro] [a Gianni]}
	\sn (Ich sage das Wort Leuchtturm zu Gianni).
\end{exe}

\begin{figure}
	\includegraphics[width=\textwidth]{figures/5-6_faroJessica.png}
	\caption{Realisierung der italienischen Äußerung <Dico la parola faro a Gianni> (Ich sage das Wort Leuchtturm zu Gianni). Vor der Phrasengrenze bzw. nach dem Zielwort <faro> tritt als prosodische Stärkung eine Glottalisierung auf /ʔ/ \citep[aus][]{DiNapoli2012}.}
	\label{figure:0506}
\end{figure}

Im Oszillogramm tritt nach der Realisierung von <faro> eine Glottalisierung in Form von irregulären Schwingungen auf, die hier als Glottalverschluss /ʔ/ annotiert sind.

Im Gegensatz zur phrasenfinalen Position treten in phrasenmedialer Position diese Glottalisierungen nicht auf. Im folgenden Trägersatz befindet sich das Zielwort <faro> medial in einer Phrase: 

%%(5.10)
\begin{exe}
	\ex Zielwort <faro> in phrasenmedialer Position:\label{ex:0510}
	\sn{[Il faro illuminò tutti gli scogli]}
	\sn (Der Leuchtturm bescheint die felsige Küste).
\end{exe}

Die Abbildung~\ref{figure:0507} zeigt das zugehörige Oszillogramm für die Realisierung des Zielworts <faro> in phrasenmedialer Position \citep[aus][]{DiNapoli2012}. Es handelt sich um denselben zentralitalienischen Sprecher wie in der Abbildung~\ref{figure:0506}. Nach <faro> treten hier keine unregelmäßigen Schwingungen auf. Der Übergang von far[o] zu [i]lluminò ist durchgehend quasiperiodisch.

\begin{figure}
	\includegraphics[width=\textwidth]{figures/5-7_faro_illumiJessica.png}
	\caption{Realisierung der italienischen Äußerung <Il faro illuminò tutti gli scogli.> (Der Leuchtturm bescheint die felsige Küste). Phrasenmedial tritt keine Glottalisierung nach dem Zielwort <faro> auf \citep[aus][]{DiNapoli2012}.}
	\label{figure:0507}
\end{figure}

Bei der Merkmalseinfügung im \ili{Italienisch}en handelt es sich um eine räumliche Modifikation, die nicht mit einer Verlangsamung der Bewegungen vor einer Grenze oder einer geringeren \isi{Überlappung} zwischen Gesten erklärt werden kann. In diesem Fall würde ein \isi{Taktgeber}, wie die π-\isi{Geste} (siehe Kapitel~\ref{chap:04}), keine ausreichende Lösung bieten. Die Ergebnisse von \citet{DiNapoli2012} zeigen, dass die \isi{prosodische Stärkung} vor Grenzen im Rahmen des Gestenmodells noch nicht ausreichend modellierbar ist.

\section{Deklination}
\label{sec:0503}

\subsection{F0-Deklination}
\label{subsec:050301}

Der Begriff der \isi{Deklination} wurde vor über vierzig Jahren im Rahmen der IPO Theorie (Institute für Perception Research, Eindhoven) entwickelt. Er geht auf \citet{Cohen1967} und \citet{tHart1990} zurück und beschreibt ein Phänomen aus der Intonationsforschung (\isi{F0} declination), bei dem der Grundfrequenzverlauf (\isi{F0}) vom Beginn einer Äußerung hin zum Ende stetig abgesenkt wird (downward trend of \isi{F0}), vgl. Abbildung~\ref{figure:0508}. Wird eine neue Äußerung begonnen, so wird die Intonationskontur als Grundfrequenzverlauf zurückgesetzt (\isi{F0} reset, declination reset). Allgemein bezieht sich die \isi{Deklination} auf \isi{prosodisch} starke Grenzen. Dabei muss es sich nicht um eine Äußerungsgrenze handeln. \isi{Deklination} findet sich häufig auch auf Ebene der Intonationsphrasen, wobei der F0-Verlauf dann jeweils beim Beginn einer neuen \isi{Intonationsphrase} zurückgesetzt würde. Deklinationseffekte können aber sogar auch auf Ebene der Intermediärphrase beobachtet werden.

\begin{figure}
	\includegraphics[width=.8\textwidth]{figures/5-8_deklination.png}
	\caption{F0-Deklination adaptiert von \citet[][76]{Ladd2008}}
	\label{figure:0508}
\end{figure}

Im IPO-Modell wird die \isi{Deklination} als globaler Effekt betrachtet, der physiologische Ursachen hat. Der Luftdruck nimmt zum Ende einer Äußerung beständig ab und beeinflusst damit die Frequenz der Stimmlippenschwingung. Dennoch handelt es sich beim Absinken des Luftdrucks nicht um ein passives Nebenprodukt der Sprechphysiologie, da der Druck beim passiven Ausströmen der Luft aus der Lunge schneller abfallen müsste als es beim Sprechen der Fall ist. Vielmehr handelt es sich um die dynamische Kontrolle der respiratorischen Muskulatur, um den Luftdruck über eine längere Äußerung hin möglichst zu stabilisieren \citep[][66]{Gelfer1987}.


\begin{figure}[p]
	\includegraphics[width=\textwidth]{figures/5-9_declination_unterrichtsthemenStefan.png}
	\caption{Abfallender Grundfrequenzverlauf über die Dauer einer Äußerung im Deutschen <Seine Unterrichtsthemen sind viel moderner und interessanter als die der anderen Lehrer>.}
	\label{figure:0509}
\end{figure}


\begin{figure}[p]
	\includegraphics[width=\textwidth]{figures/5-10_declination_messeStefan.png}
	\caption{Abfallender Grundfrequenzverlauf über die Dauer einer Intonationsphrase im Deutschen <Die Messe findet alle zwei Jahre statt und gilt als bedeutendste dieser Art weltweit (\dots)>.}
	\label{figure:0510}
\end{figure}


Abbildung~\ref{figure:0509} und~\ref{figure:0510} geben Beispiele für abfallende F0-Verläufe in Intonationsphrasen des Deutschen (Laborsprache), die nach dem IPO-Modell als \isi{Deklination} im Sinne einer physiologisch bedingten abfallenden oberen und unteren Bezugslinie (topline, baseline) zum F0-Verlauf über die Phrase hinweg betrachtet werden. Die Bilder sind mit Hilfe der Software Praat entstanden \citep{praat2010}.

\largerpage
Das Konzept der \isi{Deklination} ist umstritten. Obwohl sie als universales Phänomen gilt, kann sie nicht in jeder Äußerung bzw. Phrase beobachtet werden \citep{Gilles2008} und tritt überwiegend in Deklarativsätzen auf. Bei Interrogativsätzen im Niederländischen hingegen zeigt sich, dass die obere Bezugslinie des F0-Verlaufs vom Beginn der Äußerung bis hin zum Ende ansteigt statt abfällt \citep{vanHeuven2000}. Für Entscheidungsfragen im Griechischen finden \citet[][66]{Arvaniti2009} keine Evidenzen für \isi{Deklination}. Zwischen langen versus kurzen Entscheidungsfragen finden sich keine physiologisch bedingten Skalierungsunterschiede. Ein weiteres Problem besteht darin, dass \isi{Deklination} in Spontansprache seltener als in Laborsprache zu beobachten ist \citep{Vaissière1983}.


\newpage
Es ist strittig, ob das globale Phänomen der \isi{Deklination} überhaupt existiert oder lokale, strukturell-linguistische Eigenschaften zur Partitionierung einer Äußerung reflektiert \citep[u.a.][]{Pierrehumbert1979,Pierrehumbert1980,Ladd1984,Ladd2008,Möbius1993,Grabe1998,Arvaniti2009}. So kann eine abfallende F0-Bewegung auch auf Downstep (tonaler Abfall von einer hohen zu einer mittleren Stimmlage) und Final Lowering zurückgeführt werden. Unter einem Downstep versteht man bei Tonakzenttypen den tonalen Abfall von einer z.\,B. hohen zu einer mittleren Stimmlage. Ein Downstep kann an den verschiedensten Positionen der prosodischen Struktur auftreten und ist nicht an das Phrasenende gebunden. Unter Final Lowering hingegen versteht man ein lokales Phänomen, das auf die letzte \isi{Silbe} innerhalb einer Phrase beschränkt ist. In Abbildung~\ref{figure:0509} wäre der starke F0-Abfall zu Beginn der Äußerung nicht der \isi{Deklination}, sondern vielmehr einem Downstep zuzuschreiben.

Vermutlich liegen aber bei solchen „Deklinationseffekten“ häufig beide Faktoren -- linguistische und physiologische -- vor \citep[][44]{Arvaniti2009}. So ist beispielsweise denkbar, dass die linguistische Funktion sich physiologische Eigenschaften menschlicher \isi{Artikulation} und Perzeption zu Nutze macht, um diese dann zu phonologisieren. 

\subsection{Supralaryngale Deklination}
\label{subsec:050302}

Das Konzept der F0-\isi{Deklination} wurde auf die \isi{Artikulation} des gesamten supralaryngalen Systems übertragen. So zeigen Studien für das \ili{Italienisch}e \citep{Vayra1992}, dass Bewegungsauslenkungen vom Beginn einer Äußerung bis hin zum Ende schwächer werden. \citet{Vayra1992} führen dabei Formantmessungen bei italienischen Eckvokalen, u.a. [a] und [i], in wortbetonter Position durch. In einer akustischen Studie mit einem toskanischen Sprecher fanden sie Evidenzen, dass Vokale zum Äußerungsende hin weniger peripher artikuliert werden. In einem weiteren Experiment mit zwei Sprechern des Norditalienischen (Standardvarietät) führten sie auch kinematische Messungen durch. Bei den Zielwörtern handelte es sich um Logatome wie <bababa> oder <bibabi>, die in Trägersätze eingebettet wurden. Sie konnten nur für offene Vokale, [a], zeigen, dass der Grad der \isi{kinematisch} gemessenen Kieferöffnung gegen Phrasenende abnimmt \citep[supralaryngeal weakening,][]{Vayra1992}. Eine Modifikation des Kiefers für [i] konnten sie nicht nachweisen. Die deklinationsbedingten Modifikationen des supralaryngalen Systems betrafen in ihrer Studie nur Vokale in lexikalisch betonten Positionen (während bei der F0-\isi{Deklination} der Grundfrequenzverlauf auch bei lexikalisch schwachen Silben abfiel). Interessant ist, dass im Gegensatz zur F0-\isi{Deklination} die oralen Modifikationen innerhalb eines Wortes relativ schwach waren, so dass davon auszugehen ist, dass deklinationsbedingte Schwächung der \isi{Artikulation} vorwiegend auf Phrasen- und nicht auf Wortebene stattfindet. Würde jedes Wort am Ende seiner Domäne geschwächt, müssten die Unterschiede zwischen wortinitialen und -finalen Silben jeweils größer sein. Bei einer phrasenbedingten Schwächung ist hingegen davon auszugehen, dass die \isi{Artikulation} vom Anfang der Phrase bis zu ihrem Ende beständig leicht abnimmt.

\citet{Krakow1995} untersuchten supralaryngale \isi{Deklination} am Beispiel von silben-initialen Obstruenten, [s] und [t], im Englischen. Für die Plosive führten sie zusätzlich kinematische Messungen zur Velumshöhe an Reiterationen des Wortes <ten> an Sätzen mit unterschiedlichen Silbenwiederholungen durch.

\begin{table}[hbtp]
	\resizebox{\textwidth}{!}{
	\begin{tabularx}{\textwidth}{lXc} \lsptoprule
	\textbf{Targetsatz} & \textbf{Iteration} & \textbf{Anzahl ${\sigma}$}\\ \midrule
	Sue saw Sid. & ten ten ten & 3\\
	Suzy saw Sid. & ten ten ten ten & 4\\
	Suzy saw sad Sid. & ten ten ten ten ten & 5\\
	Suzy saw sexy Sid & ten ten ten ten ten ten & 6\\
	Suzy saw sad sexy Sid & ten ten ten ten ten ten ten & 7\\
	Suzy saw sexy sassy Sid & ten ten ten ten ten ten ten ten & 8\\
	Suzy saw sad sexy sassy Sid & ten ten ten ten ten ten ten ten ten & 9\\ \lspbottomrule
	\end{tabularx}
	}
	\caption{Zieläußerungen und Iterationen auf der Silbe <ten> nach \citet[][337]{Krakow1995}.}
	\label{table:0501}
\end{table}

Die Autoren stellten fest, dass die Höhe des Velums vom Äußerungsanfang zum Äußerungsende abnimmt (Schwächung der Velumshöhe, velische \isi{Deklination}). Die Differenz zwischen der ersten und der letzten \isi{Silbe} vergrößert sich dabei mit zunehmender Satzlänge, was von den Autoren als Evidenz für eine globale \isi{Deklination} der supralaryngalen \isi{Artikulation} gedeutet wurde. Darüber hinaus fanden sie eine Interaktion zwischen Wortbetonung und Satzposition. Bei Betrachtung der Ergebnisse aus heutiger Sicht ist jedoch bemerkenswert, dass die Velumshöhe von äußerungsinitialer zur medialen Position nicht kontinuierlich, sondern sprunghaft abnimmt (vgl. Abbildung~\ref{figure:0511}). Dieser Unterschied könnte auch auf domäneninitiale Stärkung der ersten \isi{Silbe} an einer großen Grenze rückführbar sein und wäre somit strukturell statt physiologisch bedingt.

\begin{figure}[pthb]
	\includegraphics[width=\textwidth]{figures/5-11_sue.png}
	\caption{Velumsbewegungen (Velumshöhe) für Iterationen auf der Silbe <ten> mit drei bis neun Wiederholungen aus \citet[][343]{Krakow1995}}
	\label{figure:0511}
\end{figure}

Beim Vergleich der aufgeführten Studien stellt sich die Frage, ob supralaryngale \isi{Deklination} tatsächlich ein globales Phänomen ist \citep[vgl. die Diskussion in][]{Fougeron1997}. Man kann \isi{Deklination} als globale Schwächung begreifen und -- sofern man lokale finale Dehnung ebenfalls als Schwächung begreift -- beide Strategien derselben Quelle zuordnen. In diesem Falle gäbe es einen einfachen Zusammenhang zwischen \isi{Deklination} und Grenzmarkierung: Vor prosodischen Grenzen würde aus Gründen der physiologischen \isi{Deklination} und der strukturell bedingten finalen Dehnung die \isi{Artikulation} verlangsamt \citep{Krakow1989}. Dem steht aber die Beobachtung entgegen, dass \isi{prosodisch} bedingte Stärkung (oder hier Schwächung) vom strukturellen Aufbau einer Äußerung bestimmt wird, also hierarchisch operiert.

\citet{Fougeron1997} bieten demnach eine strukturell motivierte Auslegung der \isi{Deklination} an. Sie betrachten \isi{Deklination} als Schwächung der \isi{Artikulation} auf Ebene der prosodischen Domänen. Demnach würde vor einer Domäne die Äußerung lokal geschwächt (domain-final weakening) und danach lokal gestärkt werden (domain-initial \isi{strengthening}). In ihrer Studie untersuchen sie Zungen-Gaumenkontakte in Iterationen \il{Englisch}englischer Sprecher (EPG) auf der \isi{Silbe} <no> und finden keine Evidenz für globale Deklinationseffekte. Vielmehr sprechen ihre Daten gegen eine kontinuierliche Schwächung vom Beginn der Äußerung bis hin zu ihrem Ende und für eine lokale Schwächung, die prosodisch-strukturell bedingt ist.

\begin{quotation}
	We found that the articulation of consonants and vowels varies as a function of their position in long sentences. This variation appears to be a localized effect at prosodic domain edges, i.e., a \isi{strengthening} of initial consonants and final vowels, and not a global declining trend. \citep[][3736]{Fougeron1997}
\end{quotation}

Die Annahme einer physiologisch bedingten \isi{Deklination} steht auch im Widerspruch zu Studien wie \citet{Cho2005a} und \citet{DiNapoli2012}, die in finaler Position auch spatiale Modifikationen in Form einer Stärkung, also in Form einer Erhöhung des Artikulationsaufwandes, gefunden haben.