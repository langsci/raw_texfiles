%add all your local new commands to this file

\newcommand{\smiley}{:)}

\renewbibmacro*{index:name}[5]{%
  \usebibmacro{index:entry}{#1}
    {\iffieldundef{usera}{}{\thefield{usera}\actualoperator}\mkbibindexname{#2}{#3}{#4}{#5}}}

% \newcommand{\noop}[1]{}

\makeatletter
\def\blx@maxline{77}
\makeatother

\newcommand{\appref}[1]{Appendix \ref{#1}}
\newcommand{\fnref}[1]{Appendix \ref{#1}}
\newcommand{\regel}[1]{#1}
\newcommand{\vernacular}[1]{\emph{#1}}
\newcommand{\gloss}[1]{#1}


% \newcommand{\textstyleIPAZchn}[0]{}
\newcommand{\doris}[1]{\todo[color=green!40,size=\scriptsize]{\textbf{doris:\\} #1}}
\newcommand{\bastian}[1]{\todo[color=blue!40,size=\scriptsize]{\textbf{bastian:\\} #1}}
%linien in tabellen-zellen
\newcommand{\myrule} [3] []{
%	\begin{center}
		\begin{tikzpicture}
		\draw[#2-#3, ultra thick, #1] (0,0) to (0.55\linewidth,0);
		\end{tikzpicture}
%	\end{center}
}


\newenvironment{stylecaption}{}{}
\newenvironment{styleStandai}{}{}

\newcommand{\textstyleBildTabellenbeschreibungZchn}[1]{#1}
\newcommand{\textstyleFunoteZchn}[1]{#1}
\newcommand{\textstyleGToBIZchn}[1]{#1}
\newcommand{\textstyleIPAZchn}[1]{{#1}}
\newcommand{\textstyleMerkmalZchn}[1]{\texttt{#1}}
\newcommand{\textstyleStrong}[1]{#1}
\newcommand{\textstyleTextkrperZchn}[1]{#1}
\newcommand{\textstyleTonZchn}[1]{#1}
\newcommand{\textstyleZahlEinheitZchn}[1]{#1}
\newcommand{\textstyleZitatquelleZchn}[1]{#1}
\newcommand{\textcyrillic}[1]{#1}

\newcommand{\myacute}{'}

\newenvironment{styleBildTabellenbeschreibung}{}{}
\newenvironment{stylefootnotetext}{}{}
\newcommand{\nocaption}{{\color{red} Please provide a caption!}}

\newcounter{lips@count}

\renewcommand{\lsNameIndexTitle}{Autorenregister}
\renewcommand{\lsSubjectIndexTitle}{Sachregister}
\renewcommand{\lsLanguageIndexTitle}{Sprachregister}