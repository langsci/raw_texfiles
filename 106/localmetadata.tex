\title{Language technologies for a multilingual Europe}  
\subtitle{TC3 III}
\author{Georg Rehm\and Felix Sasaki\and Daniel Stein\lastand Andreas Witt} 
\renewcommand{\lsSeriesNumber}{5}  
% \renewcommand{\lsCoverTitleFont}[1]{\sffamily\addfontfeatures{Scale=MatchUppercase}\fontsize{38pt}{12.75mm}\selectfont #1}

\renewcommand{\lsISBNdigital}{978-3-946234-73-9}   
\renewcommand{\lsISBNhardcover}{978-3-946234-77-7}
\renewcommand{\lsSeries}{tmnlp} 
\renewcommand{\lsID}{106} %
\BookDOI{10.5281/zenodo.1291947}

\typesetter{Felix Kopecky, Florian Stuhlmann, Iana Stefanova, Sebastian Nordhoff, Stefanie Hegele}
\proofreader{Ahmet Bilal Özdemir,
Alessia Battisti,
Alexis Michaud,
Amr Zawawy,
Anne Kilgus,
Brett Reynolds,
Benedikt Singpiel,
David Lukeš,
Eleni Koutso,
Eran Asoulin,
Ikmi Nur Oktavianti,
Jeroen van de Weijer,
Matthew Weber,
Lea Schäfer,
Rosetta Berger,
Stathis Selimis
}

\BackBody{This volume of the series “Translation and Multilingual Natural Language Processing” includes most of the papers presented at the Workshop “Language Technology for a Multilingual Europe”, held at the University of Hamburg on September 27, 2011 in the framework of the conference GSCL 2011 with the topic “Multilingual Resources and Multilingual Applications”, along with several additional contributions. In addition to an overview article on Machine Translation and two contributions on the European initiatives META-NET and Multilingual Web, the volume includes six full research articles. Our intention with this workshop was to bring together various groups concerned with the umbrella topics of multilingualism and language technology, especially multilingual technologies. This encompassed, on the one hand, representatives from research and development in the field of language technologies, and, on the other hand, users from diverse areas such as, among others, industry, administration and funding agencies. The Workshop “Language Technology for a Multilingual Europe” was co-organised by the two GSCL working groups “Text Technology” and “Machine Translation” (\url{http://gscl.info}) as well as by META-NET (\url{http://www.meta-net.eu}).}