\addchap{Preface}
% \begin{refsection}

%\title{Editorial}
%\subtitle{Special issue on technologies for a multilingual europe}
%\author{Georg Rehm\and Felix Sasaki\and Daniel Stein\lastand Andreas Witt}

This book, Language Technologies for a Multilingual Europe, is a reissue of the Special Issue on Technologies for a Multilingual Europe, which was originally published as Vol. 3, No. 1 of the Open Access online journal Translation: Computation, Corpora, Cognition (TC3). After the editors of TC3 had decided to transition the journal into a different format -- into the Open Access book series Translation and Multilingual Natural Language Processing -- they invited us to prepare a reissue of our compilation, originally published in 2013. While several smaller typos in the original manuscripts have been fixed, the papers in this collection have not been substantially modified with regard to the original publication, which is still, for archival reasons, available at \url{http://www.blogs.uni-mainz.de/fb06-tc3/vol-3-no-1-2013/}.

Since the original publication, Multilingual Europe has made several important steps forward. A new set of EU projects on multilingual technologies and machine translation was funded in 2015, e.g., QT21, HimL and CRACKER. The Cracking the Language Barrier federation (\url{http://www.cracking-the-language-barrier.eu}) was established as an umbrella organisation for all projects and organisations working on technologies for a multilingual Europe. At the time of writing META-NET is organising the next META-FORUM conference, which is to take place in Brussels on 13/14 November 2017. One of the key topics of this conference is the Human Language Project -- a large, coordinated funding programme spanning from education to research to innovation, which aims at bringing about the much needed boost in research and a paradigm shift in processing language automatically. First steps towards the Human Language Project were discussed at a workshop in the European Parliament in early 2017. Moreover, the Common Language Resources and Technology Infrastructure (CLARIN), founded in 2012 by ten European countries, doubled its number of members since then.

The editors of this special issue would like to thank the series editors, especially Oliver Culo, for the opportunity to publish a reissue of our original compilation.
\\
\\
\noindent Georg Rehm, Felix Sasaki, Daniel Stein, Andreas Witt \hfill
August 28, 2017

%\printbibliography[heading=subbibliography]
% \end{refsection}
