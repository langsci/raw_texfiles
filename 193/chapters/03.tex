\chapter {Comparative Deletion} \label{ch:3}
\section {Introduction} \label{sec:3introduction}
The aim of this chapter is to provide an account for \isi{Comparative Deletion} and to reduce the cross-linguistic differences attested in connection with it to minimal \isi{feature} differences in the relevant operators. On the one hand, the advantage of the proposal lies in the fact that \isi{Comparative Deletion} does not have to be treated as a parameter distinguishing between languages that have it and ones that do not. On the other hand, the feature-based account is apt for handling language-internal variation as well, since the difference is ultimately not between individual languages but rather between operators that do or do not trigger \isi{Comparative Deletion}. In order to see in what way my claim is radically new, I will first review some of the most important analyses concerning \isi{Comparative Deletion}, also showing the problems that arise in connection with them and that they can be fully eliminated using the feature-based approach proposed here.

\section {Earlier accounts} \label{sec:3earlier}
\subsection{The problems to be discussed} \label{sec:3problems}
The phenomenon of \isi{Comparative Deletion} (CD) traditionally denotes the absence of an adjectival or \isi{nominal expression} from the comparative \isi{subclause}. Consider the following examples:

\ea \label{ch3cd}
\ea Ralph is more qualified than Jason is \sout{\textbf{x-qualified}}.\label{predicative}
\ex Ralph has more qualifications than Jason has \sout{\textbf{x-many qualifications}}.\label{nominal}
\ex Ralph has better qualifications than Jason has \sout{\textbf{x-good qualifications}}.\label{attributive}
\z
\z

In all of the examples above, \textit{x} denotes a certain \isi{degree} or quantity as to which a certain entity is qualified, good, etc. (that is, the standard value). This is an \isi{operator} that has no phonological content (cf., for example, \citealt{chomsky1977}). As can be seen, in (\ref{predicative}) an \isi{adjectival expression} is deleted: this type is referred to as the predicative comparative since the quantified \isi{adjectival expression} functions as a \isi{predicate} in the \isi{subclause}. By \isi{contrast}, in both (\ref{nominal}) and (\ref{attributive}) a \isi{nominal expression} is deleted; structures like (\ref{nominal}) are nominal comparatives, where a \isi{nominal expression} bears quantification, while (\ref{attributive}) is an example of attributive comparatives, where the quantified \isi{adjectival expression} is an attributive \isi{modifier} within a \isi{nominal expression}.

Therefore, one of the most important questions to be answered in connection with \isi{Comparative Deletion} is how the fact that different constituents seem to be deleted by \isi{Comparative Deletion} can be accounted for. Moreover, this \isi{deletion} process seems to be obligatory inasmuch as the presence of the quantified expressions in (\ref{ch3cd}) would lead to ungrammatical constructions; thus a proper analysis of \isi{Comparative Deletion} must also address the issue of why it seems to be obligatory.

Additionally, the role of \isi{information structure} underlying \isi{Comparative Deletion} has to be taken into consideration as well. In \isi{subcomparative} structures, an adjectival or nominal element may be left overt in the \isi{subclause}; as opposed to the examples in (\ref{ch3cd}), these elements are not logically identical to an antecedent in the \isi{matrix clause}:

\ea \label{subcompparadigm}
\ea The table is longer than the desk is \textbf{wide}.\label{ch3subcomp}
\ex Ralph has more books than Jason has \textbf{manuscripts}.
\ex Ralph wrote a longer book than Jason did \textbf{a manuscript}.
\z
\z

The main question is of course whether such examples are to be treated as being exempt from \isi{Comparative Deletion} or whether \isi{Comparative Deletion} still applies in these cases.

Strongly connected to this, the exact site of \isi{Comparative Deletion} has to be investigated, for which there are two main candidates: the \isi{base position} of the quantified element, and an \isi{operator position} in the \isi{left periphery} of the \isi{subclause}. Interestingly, it seems that an \isi{operator} can in certain cases be visible even in \ili{English} (cf. \citealt{chomsky1977}):

\ea \% Ralph is more qualified than \textbf{what} Jason is.\label{thanwhat}
\z

This raises the question of how examples such as (\ref{thanwhat}) relate to the ones given in (\ref{ch3cd}) in terms of \isi{Comparative Deletion}; more specifically, whether constructions like (\ref{ch3cd}) also involve the \isi{movement} of the quantified expression and, on the other hand, whether \isi{Comparative Deletion} takes place in (\ref{thanwhat}) as well.

Moreover, apart from instances like (\ref{thanwhat}), in some languages full \isi{degree} expressions – i.e. when the \isi{degree element} is combined with a lexical \isi{AP} or an \isi{NP} -- can be attested at the \isi{left periphery} of the \isi{subclause} (cf. \citealt{kenesei1992}). The following examples are from \ili{Hungarian}:

\ea \label{hungarian}
\ea \gll Mari magasabb, mint \textbf{amilyen} \textbf{magas} P\'eter.\label{hungpred}\\
Mary taller than how tall Peter\\
\glt `Mary is taller than Peter.'
\ex \gll Marinak t\"{o}bb macsk\'aja van, mint \textbf{ah\'any} \textbf{macsk\'aja} P\'eternek van.\\
Mary.\textsc{dat} more cat.\textsc{poss.3sg} is than how.many cat.\textsc{poss.3sg} Peter.\textsc{dat} is\\
\glt `Mary has more cats than Peter has.'
\ex \gll Marinak nagyobb macsk\'aja van, mint \textbf{amilyen} \textbf{nagy} \textbf{macsk\'aja} P\'eternek van.\\
Mary.\textsc{dat} bigger cat.\textsc{poss.3sg} is than how big	cat.\textsc{poss.3sg} Peter.\textsc{dat} is\\
\glt `Mary has a bigger cat than Peter has.'
\z
\z

As can be seen, \ili{Hungarian} allows the overt presence of the \isi{degree} elements, which shows that \isi{Comparative Deletion} must be subject to parametric variation. The question is how this variation can be accounted for; that is, what licenses the overt presence of these elements in \ili{Hungarian} but not in \ili{English}. Conversely, a satisfactory answer to this question should also explain what underlies \isi{Comparative Deletion} in \ili{English} (and other languages that behave in the same way as \ili{English}).

Strongly related to this, the question arises to what extent the internal structure of the \isi{degree expression} plays a role and whether there is any difference among the individual operators. In Standard \ili{English}, as shown in (\ref{ch3subcomp}), the \isi{adjective} that remains overt in the \isi{subclause} is found in its \isi{base position} without an \isi{overt operator}. The \ili{Hungarian} \isi{operator} \textit{amilyen} `how' may appear together with the \isi{adjective}, as in (\ref{hungpred}), though the \isi{adjective} may not be stranded, as shown by the ungrammaticality of (\ref{amilyen}):

\ea [*]{\gll Mari magasabb, mint \textbf{amilyen} P\'eter \textbf{magas}. \label{amilyen}\\
Mary taller than how Peter tall\\
\glt `Mary is taller than Peter.'}
\z 

On the other hand, \ili{Hungarian} has another \isi{operator}, \textit{amennyire} `how much', see (\ref{amennyire}), which allows both options for the \isi{adjective}.

\ea \label{amennyire}
\ea \gll 	Mari magasabb, mint \textbf{amennyire} \textbf{magas} P\'eter.\\
Mary taller than how.much tall Peter\\
\glt `Mary is taller than Peter.'
\ex \gll Mari magasabb, mint \textbf{amennyire} P\'eter \textbf{magas}.\label{amennyirestranded}\\
Mary taller than how.much Peter tall\\
\glt `Mary is taller than Peter.'
\z
\z

Note that though the availability of (\ref{amennyirestranded}) with \textit{amennyire} may at first sight suggest that it is a VP-\isi{modifier}, it will be shown later on that such a claim could not be maintained and that \textit{amennyire} is hence an \isi{operator} within the extended \isi{degree expression}.

In addition, it has to be mentioned that \ili{Hungarian} requires the presence of some \isi{operator} if the \isi{adjective} is overt (note, however, that it is allowed for the \isi{adjective} and the \isi{operator} to be non-overt at the same time). This is illustrated by (\ref{zero}):

\ea \label{zero}
\ea \gll Mari magasabb, mint \textbf{(*magas)} P\'eter.\\
Mary taller than \phantom{\textbf{(*}}tall Peter\\
\glt `Mary is taller than Peter.'
\ex \gll Mari magasabb, mint P\'eter \textbf{(*magas)}.\\
Mary taller than Peter \phantom{\textbf{(*}}tall\\
\glt `Mary is taller than Peter.'
\z
\z

Therefore, a sound analysis of \isi{Comparative Deletion} must also take into account that languages differ with respect to the presence/absence of the \isi{operator} in a more intricate way than one that could be formulated on a +/-- basis.

In the following, I am going to present three approaches to \isi{Comparative Deletion}. The first one is that of \citet{bresnan1973}, which can be regarded as the first description and analysis of \isi{Comparative Deletion} as such. Second, I am going to present the proposal made by \citet{lechner1999diss, lechner2004}, which is interesting especially because it takes a \isi{deletion} in situ approach, which is not typical for the literature on \isi{Comparative Deletion}. Finally, I am going to deal with the analysis of \citet{kennedy2002}, which adopts the more traditional view of \textit{wh}-\isi{movement} in comparative subclauses, strongly relying on the literature since \citet{bresnan1973} and at the same time approaching the question of \isi{Comparative Deletion} in a more formalistic way than previous proposals.

\subsection{Comparative Deletion and identity -- \citet{bresnan1973}} \label{sec:3comparativedeletion}
\citet[316]{bresnan1973} assumes that something in the comparative \isi{subclause} ``is always deleted under `identity with' (nondistinctness from) the head''. This operation is referred to as \isi{Comparative Deletion} (\citealt[317]{bresnan1973}). Consider the examples in (\ref{neverseen3}), taken from \citet[316, ex. 242]{bresnan1973}:

\ea \label{neverseen3}
\ea I've never seen a man \textbf{taller} than my father.\label{bresnan1}
\ex	I've never seen \textbf{a taller man} than my father.\label{bresnan2}
\ex	I've never seen a man \textbf{taller} than my mother.\label{bresnan3}
\ex	I've never seen \textbf{a taller man} than my mother.\label{bresnan4}
\z
\z

In both (\ref{bresnan2}) and (\ref{bresnan4}) the quantified \isi{nominal expression} in the \isi{matrix clause} is \textit{a taller man}, which has a parallel in the \isi{subclause}; the analysis by \citet[317, ex. 245]{bresnan1973} is as follows:

\ea \label{bresnantree} \upshape
\scalebox{0.9}{
\begin{forest} baseline, qtree, for tree={align=center} 
[Pred\\\isi{NP}
	[Pred\\\isi{NP}
		[AP
			[QP [QP [Det [-er]] [Q [much]]]]
			[AP [A [tall]]]
		]
		[NP
			[Det [a]]
			[N [man]]
		]
	]
	[S
		[COMP
			[than]
		]
		[S
			[NP [my father\\my mother,roof ]]
			[VP [Cop [is]] [Pred\\\isi{NP} [AP [QP [Det [x]] [Q [much]]] [AP [A [tall]]]] [NP [Det [a]] [N [man]]]]]
		]
	]
]
\end{forest}}
\z

Disregarding now the apparent \isi{word order} problems (e.g. how the string \mbox{-\textit{er}} \textit{much tall a man} ultimately gives the surface string \textit{a taller man}), the primary importance of the particular representation for \citet{bresnan1973} is that it explains why (\ref{bresnan2}) is unproblematic while (\ref{bresnan4}) is semantically awkward: the reconstructed (underlying) structure of the \isi{subclause} contains the \isi{predicate} \textit{an x-much tall man} as a \isi{predicate}, which is fully acceptable with a subject such as \textit{my father} but is normally unavailable for a subject such as \textit{my mother} since there is a gender mismatch in the latter case.

As for \isi{Comparative Deletion} itself, what happens in a structure like (\ref{bresnantree}) is that \isi{Comparative Deletion} eliminates the \isi{predicate} in the \isi{subclause}, which in this case is an \isi{NP}. By \isi{contrast}, in structures such as (\ref{bresnan1}) and (\ref{bresnan3}) above, \citet[319, ex. 251]{bresnan1973} suggests that we have the following structure:

\ea \label{bresnantree2} \upshape
\begin{forest} baseline, qtree, for tree={align=center}
[AP
	[AP
		[QP
			[QP [Det [-er]] [Q [much]]]
		]
		[AP
			[A [tall]]
		]
	]
	[S
		[COMP
			[than]
		]
		[S
			[NP [my father\\my mother,roof ]]
			[VP [Cop [is]] [Pred\\\isi{AP} [QP [QP [Det [x]] [Q [much]]]] [AP [A [tall]]]]]
		]
	]
]
\end{forest}
\z

As can be seen, in this case the \isi{degree expression} in the \isi{subclause} is a \isi{predicate} on its own; consequently, the sentences in (\ref{bresnan1}) and (\ref{bresnan3}) are both felicitous because there is no gender mismatch in either case.

What happens in both (\ref{bresnantree}) and (\ref{bresnantree2}) is that the \isi{predicate} of the \isi{subclause} is deleted under identity with its matrix \isi{clausal} antecedent: crucially, this identity holds in terms of syntactic structure as well. Disregarding now the problem of how \isi{Comparative Deletion} exactly deletes this material, the point of the argument is that deleted material must be recoverable, and it seems that the most straightforward way of recovering elided material is that a structurally identical string is reconstructed. This is crucially important when trying to account for certain mismatches. Consider the following examples (cf. \citealt[320, ex. 254]{bresnan1973}):

\ea \label{solution}
\ea John wants to find \textbf{a better solution} than Christine did.\label{solutionaux}
\ex John wants to find \textbf{a better solution} than Christine's.\label{solutiongen}
\z
\z

In this case, both constructions are grammatical: the elided element in the \isi{subclause} is the \isi{nominal expression} \textit{an x-much good solution}, which may occur both as the object of the \isi{verb}, as in (\ref{solutionaux}), and as the \isi{predicate}, as in (\ref{solutiongen}); cf. \citet[319--320]{bresnan1973}. These cases correspond to the representation given in (\ref{bresnantree}). By \isi{contrast}, if the elided element has an antecedent that is not a nominal \isi{modifier}, a construction like (\ref{solutionaux}) is ruled out:

\ea \label{christine}
\ea	[*]{John wants to find a solution \textbf{better} than Christine did.} \label{christineaux}
\ex	[]{John wants to find a solution \textbf{better} than Christine's.}
\z
\z

As pointed out by \citet[320]{bresnan1973}, the problem with (\ref{christineaux}) is that the head of the comparative is an \isi{AP}; that is, the \isi{degree expression} in the \isi{matrix clause} (\textit{better}) is not an attribute but a \isi{predicate}. Thus, the structure corresponds to the one in (\ref{bresnantree2}) and the \isi{degree expression} in the \isi{subclause} should also be a complement of the \isi{verb} as such, which is ruled out in (\ref{christineaux}): the \isi{AP} cannot be the object of the \isi{verb}.

Such differences also hold if the \isi{degree expression} is a verbal \isi{modifier}, as in (\ref{jackcaviar3}) below (\citealt[320, ex. 256]{bresnan1973}):

\ea \label{jackcaviar3}
\ea []{Jack eats caviar \textbf{more} than he eats mush.} \label{caviar1}
\ex	[]{Jack eats \textbf{more caviar} than he eats mush.} \label{caviar2}
\ex	[]{Jack eats caviar \textbf{more} than he sleeps.} \label{caviar3}
\ex	[*]{Jack eats \textbf{more caviar} than he sleeps.} \label{caviar4}
\z
\z

As indicated, in the case of (\ref{caviar1}) and (\ref{caviar3}), the \isi{degree expression} is \textit{more}, which is a VP-\isi{modifier} and as such is available in both constructions. By \isi{contrast}, in (\ref{caviar2}) and (\ref{caviar4}) the \isi{degree expression} is \textit{more caviar}, which can have a corresponding element in the \isi{subclause} in the former (i.e. \textit{x-much mush}) but not in the latter: in (\ref{caviar4}) there is no (reconstructed) nominal element in the \isi{subclause} in which a \isi{degree element} could appear as an attribute.

Similar examples could be cited but the basic assumption made by \citet{bresnan1973} should be clear now: \isi{Comparative Deletion} eliminates something from the \isi{subclause} that is in some way identical to its matrix \isi{clausal} antecedent; this element may be a \isi{predicate} \isi{AP}, as in (\ref{bresnantree}) and (\ref{christine}), a \isi{predicate} \isi{NP}, as in (\ref{bresnantree2}) and (\ref{solution}), the \isi{degree expression} within a \isi{predicate} \isi{NP}, as in (\ref{caviar2}), or a verbal \isi{modifier}, as in (\ref{caviar1}) and (\ref{caviar3}).

Furthermore, there are instances where only part of a \isi{predicate} \isi{AP} is deleted, as in the following example (\citealt[322, ex. 262]{bresnan1973}):

\ea The table is \textbf{longer} than the door is \textbf{wide}. \label{tabledesk}
\z

According to \citet[322--324]{bresnan1973}, the clause given in (\ref{tabledesk}) should have the structure given in (\ref{bresnantree2}): the \isi{predicate} \isi{AP} in the \isi{subclause} is then \textit{x-much wide}, and \isi{deletion} affects the \isi{QP} \isi{modifier} \textit{x-much} but leaves the \isi{adjective} itself (\textit{wide}) intact.

Although the observations made by \citet{bresnan1973} on the phenomenon of \isi{Comparative Deletion} are crucially important, it has to be stressed that they can be regarded as a description of certain problems rather than the analysis thereof. First of all, it is left entirely unexplained what the mechanism of \isi{Comparative Deletion} actually is: \citet{bresnan1973} convincingly shows that -- in order to get the right interpretations -- the elements undergoing \isi{Comparative Deletion} have to be present in the structure at some point in the derivation but that they later also have to be eliminated in order to produce grammatical configurations. However, it is not clear why these elements cannot remain overt in the first place.

Second, \citet{bresnan1973} does not elaborate on how exactly the \isi{deletion} process is carried out: it seems that the elements in question are elided in their \isi{base position} (though the \isi{subclause} itself is claimed to be extraposed) but it remains unaddressed how the mechanism of \isi{Comparative Deletion} can detect what the \isi{deletion} site in each case is. At this point, it seems that \isi{Comparative Deletion} is assigned considerable power in the sense that it has the ability to actually decide how much of structure must and may be elided. Again, this is undesirable because it leads to circularity, that is, we know what \isi{Comparative Deletion} has to elide on the basis of the data but then the data are claimed to be such precisely because \isi{Comparative Deletion} applies in such a way. Therefore, instead of having a mechanism that can potentially elide anything, it would be desirable to have a well-defined rule or rather rules interacting with each other, which would operate in a more restricted way. In addition, a minimalist account should also clearly state which operations take place in overt syntax and which belong to \isi{PF}.

Third, if one were to assume that \isi{Comparative Deletion} takes place in the \isi{base position} of the arguments, the question arises how to account for constructions that involve \textit{wh}-\isi{movement} even in \ili{English} and to what extent evidence for \textit{wh}-\isi{movement} in any comparative \isi{subclause} can be disregarded when trying to provide an explanation for \isi{Comparative Deletion}. Strongly connected to this, the last problem with \citet{bresnan1973} is that she does not take cross-linguistic data into consideration: if \isi{Comparative Deletion} is taken to be an obligatory operation, this very definition of \isi{Comparative Deletion} proves to be untenable in the light of cross-linguistic data clearly contradicting the assumption that \isi{Comparative Deletion} would always be obligatory in the way it seems to be in \ili{English}.

\subsection{Comparative Deletion and coordination -- \citet{lechner1999diss, lechner2004}} \label{sec:3coordination}
As formulated by \citet[9]{lechner2004}, the view concerning \isi{Comparative Deletion} in the generative literature since \citet{bresnan1973, bresnan1975, bresnan1977} has been that \isi{Comparative Deletion} is ``an obligatory operation which removes the gradable property from the \isi{comparative complement} (\textit{than}-XP), accounting for the observation that comparatives in \ili{English} and in related languages characteristically contain a gap which cannot be lexically filled.''

\citet[9]{lechner2004} considers \isi{Comparative Deletion} to be an instance of syntactic \isi{ellipsis} and tries to account for it by way of the AP-Raising Hypothesis, contrary to \citet{lernerpinkal1992, lernerpinkal1995} and \citet{kennedy1997diss, kennedy1999}, who fundamentally build on the assumption that the \isi{ellipsis} site is recovered at the semantic component. The chief argument against a fully semantics-based analysis stems from the fact that if \isi{Comparative Deletion} is an \isi{LF} operation, then ``the principles which operate only on syntactic representations (overt syntax or \isi{LF})'' should be ``blind to the content of'' \isi{Comparative Deletion} (\citealt[14]{lechner2004}).

\citet[14--21]{lechner2004} presents two major arguments in favour of treating \isi{Comparative Deletion} as a process operating in syntax: disjoint reference effects and ATB \isi{extraction}. Examining first the issue of disjoint reference effects, let us consider the examples in (\ref{disjoint3}) containing the \isi{adjective} \textit{proud} (\citealt[14, ex.20]{lechner2004}):

\ea \label{disjoint3}
\ea	Mary is prouder of John than Bill is \underline{\hspace{1cm}} of Sally.\\\normalfont{(\underline{\hspace{1cm}} = x-proud)} \label{xproud}
\ex	Mary is prouder of John than Bill is \underline{\hspace{1cm}} .\\\normalfont{(\underline{\hspace{1cm}} = x-proud of John)}
\z
\z

As can be seen, the \isi{adjective} \textit{proud} may take a \isi{PP} complement and \isi{deletion} may affect either the \isi{adjective} head alone or the \isi{adjective} and the \isi{PP} together (\citealt[14]{lechner2004}). As argued for by \citet[15--16]{lechner2004}, based on similar analyses in coordination such as \citet{jayaseelan1990}, \citet{johnson1997} and \citet{lasnik1995}, in both cases the \isi{AP} is eliminated by \isi{Comparative Deletion}: the difference stems from the fact that in (\ref{xproud}) the \isi{PP} moves out of the \isi{AP} and is thus not affected by \isi{deletion}.

The importance of this becomes straightforward when considering examples such as (\ref{proudjohn}) below (\citealt[16, ex. 24]{lechner2004}):

\ea [*]{Mary is prouder of John\textsubscript{i} than he\textsubscript{i} is \underline{\hspace{1cm}} .\\\normalfont{(\underline{\hspace{1cm}} = x-proud of John\textsubscript{i})}} \label{proudjohn}
\z

As \citet[16]{lechner2004} argues, this example ``lacks a reading in which \textit{John} and \textit{he} are construed as coreferential, attesting to a Principle C violation.'' Given that ``Principle C is operative in syntax, the object \isi{PP} accordingly has to be present at least by \isi{LF}''; furthermore, because the \isi{PP} is part of the site \isi{Comparative Deletion}, one may conclude that the site ``has been restored already during the syntactic computation, i.e. prior to semantics'' (\citealt[16]{lechner2004}). In the light of this, consider the following example (\citealt[16, ex. 25]{lechner2004}):

\ea	Mary is prouder of John\textsubscript{i} than he\textsubscript{i} believes that I am \underline{\hspace{1cm}} .\\\normalfont{(\underline{\hspace{1cm}} = x-proud of John\textsubscript{i})} \label{proudof}
\z

In this case the Principle C effect is obviated; what happens is that ``Binding Theory treats the name inside'' the site of \isi{Comparative Deletion} ``as a \isi{pronoun}, and not as an R-expression'' (\citealt[16]{lechner2004}). Thus, the reconstruction into site of \isi{Comparative Deletion} ``for Principle C is subject to \textit{Vehicle Change} (in the sense of \citealt{fiengomay1994}) from R-expressions to pronouns'' (\citealt[16]{lechner2004}). As pointed out by \citet[16, ex. 26]{lechner2004}, a similar difference between (\ref{proudjohn}) and (\ref{proudof}) can also be observed in coordination, as shown by (\ref{maryjohn3}):

\ea \label{maryjohn3}
\ea	[*]{Mary is proud of John\textsubscript{i} and he\textsubscript{i} is \underline{\hspace{1cm}} , too.\\\normalfont{(\underline{\hspace{1cm}} = proud of John\textsubscript{i})}}
\ex	[]{Mary is proud of John\textsubscript{i} and he\textsubscript{i} believes that I am \underline{\hspace{1cm}} , too.\\\normalfont{(\underline{\hspace{1cm}} = proud of John\textsubscript{i})}}
\z
\z

Since \textit{Vehicle Change} implies that there is material present in the syntax before \isi{LF}, and since there is a strong resemblance to the kind of \isi{ellipsis} observed in coordination, which is treated as syntactic \isi{deletion}, there is reason to believe that \isi{Comparative Deletion} is indeed an instance of syntactic \isi{deletion}, too.

On the other hand, comparatives seem to allow ATB \isi{extraction}, in structures such as (\ref{moreproudof}) below (\citealt[19, ex. 35]{lechner2004}):

\ea	a person \textbf{who}\textsubscript{i} Mary is [more proud of \textbf{\emph{t}\textsubscript{i}}] than Peter is \underline{\hspace{1cm}}\\\normalfont{(\underline{\hspace{1cm}} = x-proud of \emph{t}\textsubscript{i})} \label{moreproudof}
\z

As \citet[19]{lechner2004} notes, this might at first sight resemble parasitic gap constructions, such as (\ref{paragap}), cf. \citet[19, ex. 36]{lechner2004}:

\ea \textbf{a book\textsubscript{i}} which you filed \textbf{\emph{t}\textsubscript{i}} [before reading \textbf{\emph{t}\textsubscript{i}}] \label{paragap}
\z

There is, however, a crucial difference between parasitic gap constructions and comparatives in that the former but not the latter tolerate asymmetric \isi{extraction} out of the \isi{matrix clause} (\citealt[19]{lechner2004}). Consider the example in (\ref{booknewspaper3}), taken from \citet[19, ex. 37]{lechner2004}:

\ea \textbf{a book}\textsubscript{i} which you filed \textbf{\emph{t}\textsubscript{i}} [before reading the newspaper] \label{booknewspaper3}
\z

However, the same is not available in comparatives (see \citealt[19, ex. 38]{lechner2004}):

\ea \label{compgap}
\ea	[*] {a person \textbf{who\textsubscript{i}} Mary is [more proud of \textbf{\emph{t}\textsubscript{i}}] than Peter is \underline{\hspace{1cm}} of John\textsubscript{k}\\\normalfont{(\underline{\hspace{1cm}} = x-proud of \emph{t}\textsubscript{k})}}
\ex [*] {a person \textbf{who\textsubscript{i}} Mary is [more proud of John] than Peter is \underline{\hspace{1cm}}\\\normalfont{(\underline{\hspace{1cm}} = x-proud of \emph{t}\textsubscript{i})}}
\z
\z

In this respect, comparatives seem to resemble coordination; consider the following examples (\citealt[19, exx. 39--40]{lechner2004}):

\ea \label{compcoord}
\ea [] {a person \textbf{who\textsubscript{i}} [\textsubscript{IP} Mary is proud of \textbf{\emph{t}\textsubscript{i}}] and [\textsubscript{IP} Peter is proud of \textbf{\emph{t}\textsubscript{i}}]}
\ex [*] {a person \textbf{who\textsubscript{i}} [\textsubscript{IP} Mary is proud of \textbf{\emph{t}\textsubscript{i}}] and [\textsubscript{IP} Peter is proud of John]} \label{compcoordb}
\ex	[*] {a person \textbf{who\textsubscript{i}} [\textsubscript{IP} Mary is proud of John] and [\textsubscript{IP} Peter is proud of \textbf{\emph{t}\textsubscript{i}}]} \label{compcoordc}
\z
\z

In line with previous proposals (see \citealt{pinkham1982diss, napoli1983, mccawley1988, moltmann1992diss, corver1993}), \citet{lechner2004}, as well as \citet{lechner1999diss}, builds his analysis on the apparent parallelism between coordination and comparative structures. What is relevant for us here is that in structures such as (\ref{compcoord}), \isi{extraction} out of only one of the conjuncts, as in (\ref{compcoordb}) and (\ref{compcoordc}), is prohibited by the Coordinate Structure Constraint (CSC), which is syntactic in nature (\citealt[19--20]{lechner2004}). Thus, if there is a similar phenomenon observed in comparatives, as in (\ref{compgap}), it is presumably also due to syntactic constraints. Furthermore, since these constraints have to apply to the \isi{degree expression} in the \isi{subclause}, the \isi{degree expression} itself must be present in the syntactic derivation. It logically follows that \isi{Comparative Deletion} involves some kind of syntactic \isi{deletion} and is not merely an \isi{LF} constraint (\citealt[21]{lechner2004}).

\citet[38--50]{lechner2004} proposes that \isi{Comparative Deletion} is in fact AP-rai\-sing, which involves the overt \isi{movement} of the \isi{AP} in the \isi{subclause} (located in the [Spec,\isi{DegP}] position) to the \isi{matrix clause} (likewise to the [Spec,\isi{DegP}] position). In the case of nominal or attributive comparatives, the \isi{NP} and the \isi{AP} are treated as a single constituent undergoing the same \isi{movement}, to the exclusion of the Deg head itself.

Therefore, for a string such as \textit{Mary knows younger authors than Peter knows}, the representation would be as follows (based on \citealt[41, ex. 90]{lechner2004}):

\ea \label{lechnerdptree} \upshape
\begin{forest} baseline, qtree, for tree={align=center}
[DP
	[D]
	[DegP
		[AP\textsubscript{i}
			[AP [young-er,roof]]
			[NP [authors,roof]]
		]
		[Deg$'$
			[Deg\\\textsubscript{{[}+compr{]}}]
			[\textit{than}-XP [than] [CP [OP\textsubscript{j}] [\phantom{XP} [\phantom{XXXXX}Peter knows,name=peter] [DP,name=dp [D] [DegP [AP\textsubscript{i} [AP [young,roof]] [NP [authors,roof]]] [Deg$'$ [Deg] [t\textsubscript{j}]]]]]]]
		]
	]
]
\draw (peter.north) -- (dp.north);
\end{forest}
\z

\newpage 
As can be seen in (\ref{lechnerdptree}), AP-Raising constitutes the upward \isi{movement} of the \isi{AP} in the \isi{subclause} to the \isi{matrix clause}, from a [Spec,\isi{DegP}] position into another [Spec,\isi{DegP}] position (\citealt[40--41]{lechner2004}). This kind of \isi{movement} is supposed to leave a semantically interpretable \isi{copy} in its \isi{base position} (in the \isi{subclause}) and so both copies are claimed to be visible at \isi{LF} (\citealt[42--43]{lechner2004}). The chief difference between the two \isi{DegP} projections is that while the one in the \isi{matrix clause} is equipped with a [+comparative] \isi{feature}, the one in the \isi{subclause} is not; consequently, only the higher \isi{DegP} is interpreted as [+comparative], see \citet[41]{lechner2004}. Note that the \isi{movement} of the \isi{comparative operator} to the [Spec,\isi{CP}] position happens independently from AP-Raising (\citealt[41]{lechner2004}).

In other words, by separating the identity that holds between the two APs and the non-identity that is maintained between the two DegPs, the analysis aims at accounting for one of the most important issues in terms of comparatives; that is, how far identity is required to hold between the two \isi{degree} expressions. Since the two Deg heads are clearly distinct from each other, there is nothing to require identity between them. However, as far as the APs are concerned, \isi{movement} by definition ensures that these have to be identical since they are two copies of one and the same syntactic object. Movement itself is motivated by the presence of the [+comparative] \isi{feature} on the Deg head in the \isi{matrix clause}: this \isi{feature} is claimed to be uninterpretable on the Deg head and it can be checked off by moving an \isi{AP} to the specifier of the \isi{DegP}.

Although certain points in the analysis may seem to be advantageous, it also raises a number of rather serious problems. First, it builds on a strong identity between the two APs and is therefore unable to account for \isi{subcomparative} structures; that is, where the \isi{AP} in the \isi{matrix clause} differs from the one in the \isi{subclause}, as in (\ref{subcompparadigm}). One might suppose that in these cases there is an \isi{AP} base-generated in the \isi{matrix clause} and the [+comparative] \isi{feature} of the Deg head can be checked off without the \isi{movement} of the subclausal \isi{AP}. In turn, the \isi{AP} in the \isi{subclause} would remain overt as it would not qualify as a lower \isi{copy}. However, this also raises the question of why base-generation is not an available option even if the two APs are identical, especially as the fact that both copies are to be interpreted by \isi{LF} at the same time seems to require an extra condition anyway; moreover, base-generation would in fact be more economical than \isi{movement}. Strongly related to this, the syntactic motivation behind AP-Raising is unclear in itself.

Second, the analysis of \isi{degree} expressions and of DPs containing \isi{degree} expressions is problematic, as should be clear from the discussion in \chapref{ch:2}. I will return to the issue of where \isi{degree} expressions are located within the \isi{DP} in \chapref{ch:4}; for the time being, suffice it to say that treating the \isi{NP} as part of the \isi{AP} is at least counterintuitive as the sequence of an \isi{AP} and an \isi{NP} is more likely to be treated as a \isi{nominal expression} by syntax. More importantly, however, the representation in (\ref{lechnerdptree}) fails to account for cases when the D head itself is filled by a \isi{determiner} (e.g. \textit{a younger author than Peter knows}): according to \citet{lechner2004}, only the \isi{AP} (containing the \isi{NP}) moves out, which has two implications. On the one hand, the D head in the \isi{matrix clause} should contain a base-generated \isi{determiner}, which again raises the question of why there is no base-generation available for the entire \isi{AP}. On the other hand, the D head in the \isi{matrix clause} should be deleted by some stipulated \isi{deletion} process targeting only this D head, which is obviously rather problematic. Alternatively, one may stipulate that the D head cannot be filled in the \isi{subclause} but this idea is again refuted by \isi{subcomparative} structures (e.g. in structures like \textit{Mary wrote a longer poem than Peter did a play}).

Third, the analysis clearly fails to account for cases where \isi{Comparative Deletion} does not seem to be obligatory, see the examples in (\ref{hungarian}) from \ili{Hungarian}. In these cases the \isi{AP} in the \isi{subclause} does remain overt even if it is identical to its counterpart in the \isi{matrix clause}: this would be ruled out by \citet{lechner2004}, whose analysis predicts that the elimination of the lower \isi{AP} happens regularly.

Moreover, there is a yet more serious problem, which is the separation of AP-\isi{movement} from \isi{operator movement}, at least in the form proposed by \citet{lechner2004}. While in Standard \ili{English} the separation of the \isi{zero operator} from the \isi{AP} may seem to be unproblematic, in languages such as \ili{Hungarian} it is obvious that the \isi{operator} can and in some cases must move together with the \isi{AP}, provided that the \isi{AP} is overtly present in the structure: see the examples in (\ref{hungarian}), (\ref{amilyen}), (\ref{amennyire}) and (\ref{zero}). This not only indicates that the structure of \isi{degree} expressions adopted by \citet{lechner2004} is flawed but also that there is no separate AP-Raising as such: the \isi{AP} either moves together with the \isi{operator} (that is, as part of the entire \isi{QP}, or as part of the entire \isi{DP} containing such a \isi{QP}), or it may stay in its \isi{base position}.

This latter distinction points to a further gap in the theory presented by \citet{lechner2004}, namely that comparative operators seem to differ with respect to whether they require overt APs and whether these APs may then be stranded or not. Since all of the problems enumerated here are crucial in terms of identifying what \isi{Comparative Deletion} is, especially in cross-linguistic terms, it should be clear that \citet{lechner2004} fails to provide a sound explanation for \isi{Comparative Deletion}, and hence an alternative should be sought.

\subsection{Comparative Deletion and movement -- \citet{kennedy2002}} \label{sec:3movement}
Before turning to the discussion of my analysis for \isi{Comparative Deletion}, let me briefly discuss one more proposal, namely that of \citet{kennedy2002}, which is crucially important in that it acknowledges that there is \isi{movement} in comparative subclauses and in that it builds the explanation on this fundamental assumption.

The core part of the analysis relies on the distinction between \isi{Comparative Deletion} (CD) structures and \isi{Comparative Subdeletion} (CSD) structures (\citealt[553--554]{kennedy2002}). The crucial difference between the two is that while in the case of \isi{Comparative Subdeletion} ``an amount or \isi{degree} term must be omitted from the constituent that provides the point of comparison with the morphologically marked phrase in the \isi{matrix clause}'', in \isi{Comparative Deletion} ``the lexical content must be omitted from the compared constituent as well'' (\citealt[554]{kennedy2002}).

Note, however, that even if the compared constituent is logically identical to its counterpart in the \isi{matrix clause}, it may remain overt if it bears contrastive focus (\citealt[555]{kennedy2002}). Consider the example in (\ref{contrastgiven}), taken from \citet[555, ex. 5a]{kennedy2002}, quoting \citet{chomsky1977}:

\begin{exe}
\ex \label{contrastgiven}
\begin{xlist} 
\exi{A:} This desk is \textbf{higher} than that one is \textbf{wide}.
\exi{B:} What is more, this desk is \textbf{higher} than that one is \textbf{HIGH}.
\end{xlist}
\end{exe}

As noted by \citet[555]{kennedy2002}, ``most analyses of comparatives in \ili{English} have hypothesized that CSD structures are basic, and that the omission of additional material in CD can be derived from general principles of redundancy reduction'' (cf. for example \citealt{lees1961}). In other words, such views assume that \isi{Comparative Deletion} and \isi{Comparative Subdeletion} have different syntactic derivations, in that the former but not the latter involves a \isi{deletion} process. Contrary to this, \citet[555--556]{kennedy2002} proposes that both structures involve the \isi{movement} of the compared constituent to the lower [Spec,\isi{CP}] position: however, while in \isi{Comparative Deletion} structures this \isi{movement} is overt, in \isi{Comparative Subdeletion} it is claimed to be covert. Consequently, the two types are essentially identical at \isi{LF} but differ at \isi{PF}; that is, there is \isi{deletion} taking place in the case of \isi{Comparative Deletion} (\citealt[556]{kennedy2002}).

Evidence for there being \isi{movement} in both structures comes from the fact that both constructions are ill-formed when the gap is within an \isi{extraction} island (\citealt[557--558]{kennedy2002}, based on \citealt{ross1967diss}, \citealt{huddleston1967}, \citealt{chomsky1977} and \citealt{postal1998}). This is indeed attested in various types of \isi{extraction} islands (complex \isi{NP} islands, Wh-islands, adjunct islands and sentential subjects); consider the examples in (\ref{michaeltattoo3}) involving complex \isi{NP} islands (\citealt[558, ex. 9]{kennedy2002}):

\largerpage[1]
\ea \label{michaeltattoo3}
\ea	*Michael has more scoring titles than Dennis is a guy who has.
\ex	*Michael has more scoring titles than Dennis is a guy who has tattoos.
\z
\z

In both cases there is a complex \isi{NP} in the comparative \isi{subclause} (\textit{a guy who has} and \textit{a guy who has tattoos}); the sentences are ungrammatical precisely because \isi{movement} of a \isi{degree expression} takes place from within these complex NPs.

Apart from island sensitivity, both \isi{Comparative Deletion} and \isi{Comparative Subdeletion} constructions show crossover effects (\citealt[558--559]{kennedy2002}) and as far as the interpretation of these structures is concerned, they have the same type of truth conditions (\citealt[559]{kennedy2002}). Admittedly, there are some differences as well; most importantly, there seems to be a problem with extracting the \isi{DegP} on its own in subcomparatives (\citealt[563--564]{kennedy2002}). Consider the examples in (\ref{michaeldennis3}), taken from \citet[564, ex. 32]{kennedy2002}:

\ea \label{michaeldennis3}
\ea	Michael has more scoring titles than [\textsubscript{CP} \normalfont{Op} Dennis has [\textsubscript{DP} \sout{\normalfont{Op}} tattoos]].
\ex	The shapes are longer than [\textsubscript{CP} \normalfont{Op} they are [\textsubscript{DegP} \sout{\normalfont{Op}} thick]].
\z
\z

As can be seen, the \isi{operator} moves out on its own and the lower \isi{copy} gets deleted; this is problematic, however, if the \isi{operator} has actual phonological content, as then we clearly have violations of the Left Branch Constraint (in the sense of \citealt{ross1967diss}), as pointed out by \citet[564]{kennedy2002}. Consider the example in (\ref{howmanytattoos3}), taken from \citet[564, ex. 33]{kennedy2002}:

\ea \label{howmanytattoos3}
\ea	*How many does Dennis have [\textsubscript{DP} \sout{how many tattoos}]?
\ex	*[\textsubscript{CP} How were the shapes [\textsubscript{DegP} \sout{how thick}]]?
\z
\z
	
The conclusion drawn by \citet[570]{kennedy2002} is that \isi{Comparative Deletion} and \isi{Comparative Subdeletion} ``are the same in their basic syntactic properties''; that is, both involve ``the same functional vocabulary and are subject to the same syntactic operations'' but they ``differ in the level of representation at which these operations apply''. Therefore, while the two types ``have structurally identical \isi{LF} representations'', they have ``structurally distinct \isi{PF} representations'' (\citealt[571]{kennedy2002}).

As shown by \citet[571--574]{kennedy2002}, \isi{Comparative Deletion} and \isi{Comparative Subdeletion} structures have essentially the same semantics, in addition to syntactic similarities (see above); ``the \isi{comparative clause} is interpreted as a description of a maximal amount, and supplies the standard of comparison for the \isi{comparative morpheme}'' (\citealt[574]{kennedy2002}). In either case, the compared constituent has to move at \isi{LF} (or before) ``because the quantificational force of the \isi{comparative clause} (the maximality \isi{operator}) is introduced by the \isi{degree} morphology on the compared constituent, not by a higher \isi{operator}'' and hence ``to generate the right interpretation of the \isi{comparative clause} [\ldots] the compared constituent must take scope over the rest of the clause'' (\citealt[574--575]{kennedy2002}).

Note that in both types of comparatives the entire compared constituent is assumed to move; as pointed out by \citet[581--582]{kennedy2002}, partial \isi{movement} would in certain cases lead to Left Branch Constraint violations. Consider the examples in (\ref{michaelmore3}), taken from \citet[581, ex. 79]{kennedy2002}:

\ea \label{michaelmore3}
\ea	Michael has more scoring titles than Dennis has \textbf{(tattoos)}.
\ex	Michael’s hands are wider than your feet are \textbf{(long)}.
\z
\z

As can be seen, it is grammatical to have a \isi{DP} or an \isi{AP} in the comparative \isi{subclause} without an overt \isi{degree marker}, which may lead one to the conclusion that in such cases the \isi{degree} \isi{operator} moves out on its own. However, comparatives then should have an analogous structure to the questions in (\ref{lbeint}), which are ungrammatical (\citealt[581, ex. 80]{kennedy2002}):

\ea \label{lbeint}
\ea	*\textbf{How many} does Dennis have \textbf{tattoos}?
\ex	*\textbf{How (much)} are your feet \textbf{long}?
\z
\z
	
Instead, the claim made by \citet{kennedy2002} is that (sub)comparatives are analogous to the structures in (\ref{nolbeint}), see \citet[581, ex. 81]{kennedy2002}:

\ea \label{nolbeint}
\ea	\textbf{How many tattoos} does Dennis have?
\ex	\textbf{How long} are your feet?
\z
\z

The claim that in subcomparatives the entire compared constituent moves implies for \citet{kennedy2002} that this \isi{movement} is covert, since the overt \isi{copy} of the compared constituent remains in its \isi{base position}.

Essentially, \citet[582--583]{kennedy2002} claims that in the case of \isi{Comparative Deletion}, that is, when the compared constituent is identical to its counterpart in the \isi{matrix clause}, both \isi{movement} and \isi{deletion} take place, whereas in CSD neither \isi{deletion} nor \isi{movement} happens. In his analysis, this is formulated in an optimality-theoretic approach, in that \isi{deletion} is claimed to be favourable to overt \isi{movement} (\citealt[583]{kennedy2002}). Since I do not adopt the framework of optimality theory, I will not provide further details of his analysis here.

Instead, let me point out some problems that, despite the merits of the analysis given by \citet{kennedy2002}, make it necessary to continue investigating the issues in question. First of all, while it is obvious that \isi{movement} takes place in both \isi{Comparative Deletion} and \isi{Comparative Subdeletion} structures, it is not straightforward how syntax should decide on which \isi{degree} expressions in the \isi{subclause} have to move before \isi{spell-out} and which cannot: taking identity as such into account would require semantic interpretation but \isi{movement}, at least in the case of \isi{Comparative Deletion}, takes place before that. Moreover, identity is not a satisfactory criterion in itself: as demonstrated by the grammaticality of examples like (\ref{contrastgiven}), the \isi{degree expression} may remain overt even if it is identical to its matrix \isi{clausal} counterpart.

This leads to the second problem, which is the following: while it is true that \isi{recoverability} is a prerequisite for material to be deleted, it is certainly not true that recoverable material falls under obligatory \isi{deletion}. Such a stance would be untenable in general but is also immediately refuted by languages such as \ili{Hungarian}, where there is no obligatory \isi{Comparative Deletion}. In other words, while \citet[554]{kennedy2002} notes that the requirement on the obligatory nature of \isi{Comparative Deletion} ``is important, as it distinguishes CD from other \isi{deletion} operations in \ili{English}, such as \isi{ellipsis}, which is optional'', his analysis clearly does not account for cross-linguistic variation.

Third, the distinction between \isi{Comparative Deletion} and \isi{Comparative Subdeletion} on the basis of whether they contain overt or \isi{covert movement} is highly questionable, too. As demonstrated by languages lacking \isi{Comparative Deletion}, such as \ili{Hungarian}, the \isi{degree expression} in the \isi{subclause} moves up in both types of constructions to a [Spec,\isi{CP}] position. Note that I assume that the two CP-layers in comparatives are available in \ili{Hungarian} as well, the higher headed by complementisers and the specifier of the lower one hosting relative operators: see \citet{kantor2008even}, \citet{bacskaiatkari2010even}. Consider the examples in (\ref{maritaller3}):

\ea \label{maritaller3}
\ea \gll Mari	magasabb,	mint \textbf{amilyen} \textbf{magas} P\'eter.\\
Mary taller	than how tall	Peter\\
\glt `Mary is taller than Peter.'
\ex \gll	Az asztal hosszabb,	mint \textbf{amilyen} \textbf{széles}	az iroda.\\
the	table	longer than	how	wide the office\\
\glt `The table is longer than the office is wide.'
\z
\z

On the other hand, in languages such as \ili{English}, it is always the lower \isi{copy} that remains overt, even if it happens to be identical to its matrix \isi{clausal} counterpart, see (\ref{contrastgiven}). Therefore, the chief distinction seems to be one that holds between languages and not one that can be observed between the two constructions.

Fourth, the analysis presented by \citet{kennedy2002} does not consider examples which show that overt material below \textit{than} may be overt in \ili{English} as well: the \isi{overtness} of \textit{what} was shown in (\ref{thanwhat}), but constructions with \textit{how} are also possible in certain dialects, as demonstrated in (\ref{disjoint3}):

\ea \label{how3}
\ea [\%]{Ralph is taller than \textbf{how tall} Peter is.} \label{howtall}
\ex [\%]{The desk is longer than \textbf{how wide} the office is.} \label{howwide}
\z
\z

The \isi{overtness} of \textit{how tall} in (\ref{howtall}) and \textit{how wide} in (\ref{howwide}) contradicts the assumption that \isi{movement} to [Spec,\isi{CP}] necessarily involves \isi{deletion} in \ili{English}; moreover, the availability of (\ref{howtall}) refutes the implied claim that only contrasted \isi{degree} expressions may remain overt in the \isi{subclause}.

In sum, it seems that the analysis provided by \citet{kennedy2002} does not take into consideration a number of phenomena that would be important for gaining a better understanding of how \isi{Comparative Deletion} works; and, most importantly, it is not explained why it should take place at all when it does.

\section{Constraints on deletion} \label{sec:3constraints}
In order to provide an account for \isi{Comparative Deletion}, let me first briefly summarise the most important issues concerning \isi{deletion} mechanisms in general. One such general constraint is that of \textsc{given}ness. Roughly speaking, elements can be \textsc{given} or focus-marked (F-marked), see \citet{selkirk1996, selkirk2005}; \citet{schwarzschild1999}; \citet{merchant2001}; \citet{buering2006}. Consider the examples in (\ref{ralphnovel3}):

\ea \label{ralphnovel3}
\ea	[]{Ralph was reading a novel and Peter \sout{was reading} an epic.} \label{readepic}
\ex	[*]{Ralph was reading a novel and Peter \sout{was writing} an epic.} \label{writeepic}
\z
\z

The sentence in (\ref{readepic}) is grammatical: the elided \isi{verb} in the second conjunct is \textit{read}, which is \textsc{given}, and can be deleted. As opposed to this, in (\ref{writeepic}) \textit{write} is F-marked as \textit{read} in the \isi{matrix clause} is not a salient antecedent for it: consequently, it is ungrammatical to elide it. This is fundamentally a \isi{recoverability} condition on \isi{deletion}: a constituent $\alpha$ can be deleted iff $\alpha$ is e-\textsc{given} (ellipsis-\textsc{given}, see \citealt[38]{merchant2001}), hence $\alpha$ must have a salient antecedent in the discourse.

It is worth mentioning that optional \isi{deletion} processes may save a given construction from ungrammaticality. This is true for \isi{sluicing}, which, as shown by the grammaticality of (\ref{nosluicewho}), is optional:

\ea \label{sluice3}
\ea	They want to hire someone who speaks a \ili{Slavic} language, but I don't remember who they want to hire. \label{nosluicewho}
\ex	They want to hire someone who speaks a \ili{Slavic} language, but I don't remember who \sout{they want to hire}. \label{sluicewho}
\z
\z

Sluicing, as can be seen in (\ref{sluicewho}), deletes the string after a \textit{wh}-\isi{pronoun} (\textit{who}) that has moved to the [Spec,\isi{CP}] position, which in this case has moved from within the elided \isi{subclause} (see \citealt{merchant2001}; \citealt{vancraenenbroeckliptak2006}). Since (\ref{nosluicewho}) is grammatical and in (\ref{sluicewho}) \isi{sluicing} takes place regularly, the sentences in (\ref{sluice3}) are fundamentally equivalent to each other.

The situation is different when the underlying structure is ungrammatical. The examples in (\ref{hire3}) contain island violations (based on \citealt[114, ex. 15]{merchant2001}):

\ea \label{hire3}
\ea	[*]{They want to hire someone who speaks a \ili{Slavic} language, but I don’t remember which they want to hire someone [who speaks].} \label{nosluicewhich}
\ex	[]{They want to hire someone who speaks a \ili{Slavic} language, but I don’t remember which \sout{they want to hire someone [who speaks]}.} \label{sluicewhich}
\z
\z

In both cases, \textit{which} moves up from within the   \isi{subclause} (\textit{who speaks}), which is a \textit{wh}-\isi{island violation}. In (\ref{nosluicewhich}), the sentence is therefore not grammatical, while in (\ref{sluicewhich}), where \isi{sluicing} takes place, the result is fully grammatical. Obviously, \isi{sluicing} in this case not only deletes the \isi{subclause} responsible for ungrammaticality but a larger chunk as well, since \isi{sluicing} by definition can only delete the entire complement of a \isi{functional head} (C) equipped with an [E] \isi{feature}. Since in this case the fully overt construction is ill-formed, while \isi{sluicing} deletes precisely the part causing ill-formedness at \isi{PF}, only the sentence containing \isi{deletion} will converge out of the two options. Thus, optional \isi{deletion} processes are able to save structures from ungrammaticality, without having to suppose that these processes would be obligatory. This conclusion will be important later when considering certain \isi{deletion} mechanisms.

In the exact mechanism of \isi{sluicing}, \isi{deletion} itself takes place at \isi{PF}; however, \isi{deletion} is licensed by an [E] \isi{feature} inserted in syntax, see \citet{merchant2001}. The significance of this is partly that optional \isi{deletion} processes can be handled in the syntax: while the insertion of an [E] \isi{feature} requiring \isi{deletion} is optional in the syntactic component, the presence or the absence of the \isi{feature} contains unequivocal information for \isi{PF} in terms of whether \isi{deletion} should take place. This is because while the prohibition of deleting F-marked elements is an axiom, \textsc{given} elements are not necessarily deleted. Hence \textsc{given}ness in itself is not unequivocal information for \isi{PF}; in turn, \isi{PF} is responsible for the linear structure and does not produce syntactic and/or semantic features.

\section{An analysis of Comparative Deletion} \label{sec:3analysis}
\subsection{General considerations} \label{sec:3general}
Recall that, descriptively, \isi{Comparative Deletion} (CD) is a process which eliminates the \isi{QP} or the quantified \isi{DP} from the \isi{subclause}, if it is logically identical to its antecedent in the \isi{matrix clause} (\citealt{bacskaiatkari2010even}, \citealt{bacskaiatkari2012elte}) in examples such as (\ref{ch3cd}), repeated here for the sake of convenience as (\ref{cdrepeated}):

\ea \label{cdrepeated}
\ea Ralph is more qualified than Jason is \sout{\textbf{x-qualified}}.\label{predicativerep}
\ex Ralph has more qualifications than Jason has \sout{\textbf{x-many qualifications}}.\label{nominalrep}
\ex Ralph has better qualifications than Jason has \sout{\textbf{x-good qualifications}}.\label{attributiverep}
\z
\z

Comparative subclauses exhibit regular \isi{operator movement} (see \citealt{chomsky1977}, \citealt{kennedy2002}) to a [Spec,\isi{CP}] position. This is illustrated in (\ref{treeopmovement}):

\ea \label{treeopmovement} \upshape
\begin{forest} baseline, qtree, for tree={align=center}
[CP
	[C$'$
		[C
			[than]
		]
		[CP
			[Op.]
			[C$'$ [C] [IP]]
		]
	]
]
\end{forest}
\z

As for the structure of the \isi{left periphery}, I assume that the \isi{CP} can be split (cf. \citealt[297]{rizzi1997}, \citealt[1]{rizzi1999}, \citealt[237--238]{rizzi2004}), even though I do not adopt a cartographic approach and the various CPs should rather be regarded as instances of an iterated \isi{CP}. Rizzi's split \isi{CP} is illustrated in (\ref{rizzisplitcp3}):

\ea \upshape {}[CP [TopP* [FocP [TopP* [CP]]]]] \label{rizzisplitcp3}
\z

Rizzi assumes that multiple TopPs and a designated FocP may appear between the two \isi{CP} projections; however, this is irrelevant for the present analysis, and therefore I will neither include them in the representations, nor will I discuss possible arguments against a strict cartographic approach. Further, \citet{rizzi1997, rizzi1999, rizzi2004} attributes different functions to the two CPs: he assumes that the higher C head is responsible for the ``illocutionary'' Force of the clause, while the lower is responsible for Finiteness.

The term ``illocutionary Force'' is fundamentally used to cover clause types, that is, categories such as declarative, interrogative, relative, comparative, etc.; it is terminologically unfortunate to involve the concept of illocution since the kind of illocution discussed by Rizzi has little to do with how \citet{austin1962} and \citet{searle1969} introduced the term, the sentence types in question not being performative. In addition, the distinction between Force and Finiteness is problematic as well because, though the relative position of a given C head in a combination (that is, whether it is a lower or a higher one) is straightforward, it is hard to disentangle the various functions in cases where a single C head marks both. Due to these reasons, I will henceforth not mark the Force/Finiteness distinction.

\subsection{Predicative versus attributive and nominal structures} \label{sec:3predicativeattributive}
Turning back to the representation given in (\ref{treeopmovement}), the \isi{complementiser} head of the comparative \isi{subclause} (\textit{than}) occupies the higher C position, while the \isi{comparative operator} (Op.) moves to the specifier of the lower \isi{CP}.

In predicative structures, such as (\ref{predicativerep}), the \isi{QP} containing the \isi{AP} is headed by a phonologically empty \isi{operator} (\textsl{x}), and the entire \isi{QP} moves up to the specifier of the \isi{CP}, where it is deleted. By \isi{contrast}, in nominal and attributive structures, such as (\ref{nominalrep}) and (\ref{attributiverep}), respectively, the \isi{QP} is an adjunct within the \isi{DP} (\citealt{kennedymerchant2000}, \citealt{kantor2008}) and thus the entire \isi{DP} moves up and is deleted. This is because the \isi{QP} cannot be extracted from the \isi{DP} due to the DP-island constraint (cf. \citealt{kayne1983}, \citealt{ross1986}, \citealt[217]{izvorski1995}, \citealt{grebenyova2004}, \citealt{boskovic2005}).

Movement in predicative structures is represented in (\ref{treepredicative3}), based on (\ref{predicativerep}):

\ea  \upshape \label{treepredicative3}
\begin{forest} baseline, qtree, for tree={align=center}
[CP
	[C$'$
		[C
			[than]
		]
		[CP
			[QP\textsubscript{i} [x-qualified,roof]]
			[C$'$ [C[$\emptyset$]] [IP [Jason is t\textsubscript{i},roof]]]
		]
	]
]
\end{forest}
\z

Movement in attributive and nominal structures is represented in (\ref{treeattributive3}), based on the examples given in (\ref{nominalrep}) and (\ref{attributiverep}):

\ea  \upshape \label{treeattributive3}
\begin{forest} baseline, qtree, for tree={align=center}
[CP
	[C$'$
		[C
			[than]
		]
		[CP
			[DP\textsubscript{i} [x-many qualifications\\x-good qualifications,roof]]
			[C$'$ [C[$\emptyset$]] [IP [Jason has t\textsubscript{i},roof]]]
		]
	]
]
\end{forest}
\z

All this can be derived from more general rules and is hence not specific for comparative subclauses, as similar phenomena can be observed in other constructions containing operators (cf. \citealt[7]{kennedymerchant1997}). Consider the examples given in (\ref{howpatterns3}):

\ea \label{howpatterns3}
\ea	[*]{\textbf{How} is Ralph \textbf{qualified}?} \label{howqualifiedungr}
\ex	[]{\textbf{How qualified} is Ralph?} \label{ch3howqualified}
\ex	[*]{\textbf{How big} did Ralph see \textbf{cats}?}
\ex	[]{\textbf{How big cats} did Ralph see?}
\ex	[*]{\textbf{How many} did Ralph see \textbf{cats}?}
\ex	[]{\textbf{How many cats} did Ralph see?}
\z
\z

As can be seen, the \isi{QP} \textit{how qualified} and the \isi{DP} \textit{how big cats} or \textit{how many cats} can be moved only as a whole: neither the Q head may be extracted from the \isi{QP} nor the \isi{QP} from the \isi{DP}. I will return to the issue later, also casting light upon how it varies cross-linguistically. At this point, suffice it to say that in cases such as (\ref{howqualifiedungr}) and (\ref{ch3howqualified}) above the Q head cannot be extracted because then it would have to occupy a phrase position in the lower [Spec,\isi{CP}] as a head. On the other hand, as I will show later, in some languages the \isi{quantifier} may also be realised as a \isi{QP} \isi{modifier} within the \isi{QP} heading the \isi{adjective} in question, and it can in such cases be extracted, cf. \citet{kantor2008}. Similarly, the \isi{extraction} of the \isi{QP} out of the \isi{DP} is highly dependent on the parametric settings of a given language: while \ili{English}, \ili{Bulgarian} and \ili{Greek} prohibit it, it is allowed in \ili{Polish} and \ili{Czech} (\citealt{kennedymerchant2000}); these questions will be addressed in\chapref{ch:4} in detail.

Turning back to comparative subclauses in \ili{English}, it is important to investigate the issue of copies. In our case, there are only two copies to consider: the lower one in the \isi{base position} of the \isi{QP} or the \isi{DP} and the higher one in the lower [Spec,\isi{CP}] as a result of \isi{movement}. The higher \isi{copy}, as has already been seen, is deleted by \isi{Comparative Deletion}; note that this is independent from whether the \isi{AP} or \isi{NP} is identical to the one in the \isi{matrix clause}. The lower \isi{copy} is regularly deleted by \isi{PF} (cf. \citealt{bobaljik2002}, \citealt{chomsky2008}, \citealt[44--48]{boskovicnunes2007}), which is possible because the \isi{QP} or \isi{DP} in question is e-\textsc{given}. The \isi{deletion} processes taking place in (\ref{cdrepeated}) are shown in (\ref{cdprocesses}):

\ea \label{cdprocesses}
\ea	Ralph is more qualified [\textsubscript{CP} than [\textsubscript{CP} \sout{[\textsubscript{QP} x-qualified]} Jason is \sout{[\textsubscript{QP} x-qualified]}]]. \label{cdprocesspred}
\ex	Ralph has more qualifications [\textsubscript{CP} than [\textsubscript{CP} \sout{[\textsubscript{DP} x-many qualifications]} Jason has \sout{[\textsubscript{DP} x-many qualifications]}]].
\ex	Ralph has better qualifications [\textsubscript{CP} than [\textsubscript{CP} \sout{[\textsubscript{DP} x-good qualifications]} Jason has \sout{[\textsubscript{DP} x-good qualifications]}]].
\z
\z

As should be obvious, \isi{Comparative Deletion} takes place in all structures, hence there is no difference between predicative and attributive/nominal constructions: the fact that the entire \isi{DP} has to be eliminated in the latter is due to different, independent constraints.

One obvious advantage of this approach is that it accounts for the \isi{deletion} of QPs and DPs without having to resort to extra mechanisms: \isi{Comparative Deletion} takes place in the lower [Spec,\isi{CP}] position and it deletes any material that is there. In turn, differences in terms of what phrases are found there arise simply out of \isi{movement} constraints. I will return to the issue of why \isi{Comparative Deletion} has to take place at all later, also accounting for the differences found between languages and varieties. At this point, suffice it to say that a \isi{movement} analysis claiming that the entire \isi{QP} or \isi{DP} moves (and not only the \isi{operator}) can successfully account for the elimination of both copies by assuming that \isi{Comparative Deletion} obligatorily takes place in the lower [Spec,\isi{CP}], eliding the higher \isi{copy}, and that lower copies are regularly deleted at \isi{PF}. Though it is a prerequisite that deleted material has to be e-\textsc{given}, the fact that obligatory \isi{deletion} takes place is not directly linked to these elements being recoverable, contrary to \citet{kennedy2002}: rather, it is associated with a syntactic position where it happens independently of whether the material there is e-\textsc{given} or not.

\subsection{Comparative Subdeletion} \label{sec:3comparativesubdeletion}
The case of \isi{Comparative Subdeletion}, as found in subcomparatives, may at first sight seem to be a counterexample for what has been established for \isi{Comparative Deletion}. In these (predicative) structures, as was mentioned at the beginning of this chapter, the \isi{QP} in the \isi{subclause} remains overt:

\ea	The table is longer than the desk is \textbf{wide}. \label{subcompenglish}
\z

However, even in such cases \isi{Comparative Deletion} takes place regularly in the [Spec,\isi{CP}] position: if \isi{Comparative Deletion} did not occur, then the higher \isi{copy} should remain (cf. \citealt{bacskaiatkari2010even}). On the other hand, the lower \isi{copy} cannot be eliminated since it is F-marked: it contrasts with the \isi{AP} (\textit{long}) in the \isi{matrix clause}. As pointed out by \citet[48]{boskovicnunes2007}, lower copies may remain overt if the pronunciation of the higher \isi{copy} would make the derivation crash at \isi{PF}. Thus, the following happens in (\ref{subcompenglish}):

\ea	The table is longer [\textsubscript{CP} than [\textsubscript{CP} \sout{[\textsubscript{QP} x-wide]\textsubscript{F}} the desk is [\textsubscript{QP} x-wide]\textsubscript{F}]]. \label{subcompderivation}
\z

As can be seen, the higher \isi{copy} of the \isi{QP} is deleted by \isi{Comparative Deletion} exactly the same way as in (\ref{cdprocesspred}) and the two clauses differ in fact only with respect to whether the lower \isi{copy} remains; however, this difference can be derived from \isi{recoverability}. This all indicates that subcomparatives are not exceptional in terms of \isi{Comparative Deletion}, and thus there is no separate \isi{Comparative Subdeletion} process.

In this way, the relation between \isi{Comparative Deletion} and \isi{Comparative Subdeletion} can be easily handled, without having to resort to distinguishing the two on the basis of whether they include overt or \isi{covert movement}, as was seen in connection with \citet{kennedy2002}. Again, the role of \isi{information structure} is not directly related to \isi{Comparative Deletion} itself: \isi{Comparative Deletion} is treated as a mechanical process eliminating material from the lower [Spec,\isi{CP}] position and the fact that the lower \isi{copy} of the \isi{QP} can remain overt is due to F-marking.

Note that being F-marked is not identical to not being e-\textsc{given}; it is rather intended to express some kind of \isi{contrast}. For instance, the \isi{QP} \textit{x-wide} in (\ref{subcompderivation}) is in \isi{contrast} with the \isi{QP} \textit{longer} of the \isi{matrix clause}. Also, this \isi{QP} appears in a clause-final position, which is the canonical position for foci and/or contrasted elements in \ili{English}: see \citet{selkirk1984, selkirk1986}, \citet{nesporvogel1986}, \citet{mccarthyprince1993}. This \isi{QP} expresses the main \isi{contrast} involved in comparison and it follows logically that it appears in a position where it can bear main \isi{sentential stress}.

As far as the overt lower \isi{copy} of an e-\textsc{given} \isi{AP} is concerned, it is usually ungrammatical because it should regularly be eliminated as a lower \isi{copy} and it should not appear in a contrastive position. However, if there is a context in which it can be interpreted as a contrasted element  even though it is \textsc{given}, it may remain overt: see also (\ref{contrastgiven}). The difference is illustrated in (\ref{givenap}) below:

\begin{exe}
\ex \label{givenap}
\begin{xlist} 
\ex ??/*The table is longer than the desk is \textbf{long}. \label{givenapnocontrast}
\ex \label{givenapcontrast}
\begin{xlist}
\exi{A:} The table is longer than the desk is wide.
\exi{B:} No, the table is longer than the desk is \textbf{LONG}.
\end{xlist}
\end{xlist}
\end{exe}
	
In both cases the \isi{subclause} contains an overt lower \isi{copy} of the \isi{QP} that is identical to the one in the \isi{matrix clause}. However, in (\ref{givenapnocontrast}) it should have been eliminated as there is no additional instruction for \isi{PF} to preserve the lower \isi{copy}. As opposed to this, (\ref{givenapcontrast}) is grammatical because the \isi{QP} in question is contrasted: this \isi{contrast} holds not with the \isi{QP} in the \isi{matrix clause} but with the one in the preceding sentence.

It can be concluded that subdeletion constructions also include \isi{Comparative Deletion} in the regular way, and the fact that the lower \isi{copy} remains overt stems from constraints independent from the mechanism of \isi{Comparative Deletion}.

\section{The structure of degree expressions revisited} \label{sec:3structure}
\subsection{On Hungarian operators} \label{sec:3hungarian}
In order to understand the mechanism of \isi{Comparative Deletion}, let us first consider a language where it does not operate. In \ili{Hungarian}, as has been mentioned, the quantified \isi{AP} may remain overt, that is, both the \isi{comparative operator} and the lexical \isi{AP} can be visible in the [Spec,\isi{CP}] position. However, there are differences between the available operators in this respect.

The canonical \isi{comparative operator} is \textit{amilyen} `how', which is shown in (\ref{amilyenpatterns}):

\ea \label{amilyenpatterns}
\ea	[]{\gll Mari	magasabb,	mint \textbf{amilyen} \textbf{magas} Péter volt. \label{amilyenap}\\
Mary taller	than how	tall Peter was.\textsc{3sg}\\
\glt `Mary is taller than Peter was.'}
\ex	[*]{\gll Mari magasabb, mint \textbf{amilyen} Péter volt \textbf{magas}. \label{amilyenstranding}\\
Mary taller	than how Peter was.\textsc{3sg} tall\\
\glt `Mary is taller than Peter was.'}
\z
\z

As can be seen, the \isi{operator} \textit{amilyen} is inseparable from the lexical \isi{AP} (\textit{magas} `tall'): it is grammatical to have them both overtly in the lower [Spec,\isi{CP}] position, as in (\ref{amilyenap}), but the \isi{AP} cannot be stranded and left behind in its \isi{base position}, as in (\ref{amilyenstranding}).

\ili{Hungarian} also has the \isi{operator} \textit{amennyire} `how much': this can otherwise modify VPs and it may modify APs as well, though there is some variation among speakers with respect to the availability of this \isi{operator} as an \isi{AP} \isi{modifier}. Still, if it appears in comparatives, it behaves differently from \textit{amilyen}, as shown by (\ref{amennyirepatterns}):

\ea \label{amennyirepatterns}
\ea	\gll Mari	magasabb,	mint \textbf{amennyire} \textbf{magas} Péter volt. \label{amennyireap}\\
Mary taller	than how.much	tall Peter was.\textsc{3sg}\\
\glt `Mary is taller than Peter was.'
\ex	\gll Mari magasabb, mint \textbf{amennyire} Péter volt \textbf{magas}. \label{amennyirestranding}\\
Mary taller	than how.much Peter was.\textsc{3sg} tall\\
\glt `Mary is taller than Peter was.'
\z
\z

Unlike \textit{amilyen}, \textit{amennyire} may appear both together with the lexical \isi{AP} in [Spec,\isi{CP}], as in (\ref{amennyireap}), and it may also allow the stranding of the \isi{AP}, as in (\ref{amennyirestranding}). Thus, \textit{amennyire} is separable from the lexical \isi{AP}.

There are reasons to believe that the \isi{operator} \textit{amennyire} in structures like (\ref{amennyirepatterns}) is indeed base-generated within the \isi{degree expression} and is not a VP-\isi{modifier}. First, if it were a VP-modifying \isi{operator}, then it could not move together with the lexical \isi{AP} to the lower [Spec,\isi{CP}] position, as, for instance, in (\ref{amennyireap}), because then they would not form one constituent. Second, if an \isi{adverb} modifies the \isi{verb}, then the \isi{verb} must be overt, whereas if the structure is simply predicative, the present-\isi{tense} 3rd singular \isi{copula} is not overt, as demonstrated by the pattern in (\ref{maricopula3}):

\ea \label{maricopula3}
\ea \gll Mari	jól	*(van).\\
Mary well \phantom{*(}is\\
\glt `Mary is well.'
\ex \gll Mari fáradt (*van).\\
Mary tired \phantom{*(}is\\
\glt `Mary is tired.'
\z
\z

As far as comparatives containing \textit{amennyire} are concerned, the \isi{copula} \textit{van} is not permitted to appear overtly in structures like (\ref{novan}):

\ea \gll	Mari magasabb, mint	amennyire	Péter	(*van) magas. \label{novan}\\
Mary taller	than how.much	Peter \phantom{*(}is tall\\
\glt `Mary is taller than Peter.'
\z
	
If \textit{amennyire} modified the \isi{verb}, then the presence of \textit{van} `is' would be required, which is not the case, and therefore \textit{amennyire} cannot be a VP-modifying \isi{adverb} in comparative subclauses.

Before turning to the further examination of the difference between \textit{amilyen} and \textit{amennyire}, note that \ili{Hungarian} has no zero comparative operators, and constructions like (\ref{hungzero}) are ungrammatical:

\ea \label{hungzero}
\ea [*]{\gll Mari	magasabb,	mint \textbf{magas}	Péter	volt.\\
Mary	taller	than	tall	Peter	was.\textsc{3sg}\\
\glt `Mary is taller than Peter was.'}
\ex	[*]{\gll Mari	magasabb,	mint Péter volt	\textbf{magas}.\\
Mary taller	than Peter was.\textsc{3sg} tall\\
\glt `Mary is taller than Peter was.'}
\z
\z

As shown above, it is impossible to have an overt \isi{AP} without an \isi{overt operator} (i.e. one with actual phonological content) either in the [Spec,\isi{CP}] or in its \isi{base position}.

It has to be stressed that the differences between \textit{amilyen} and \textit{amennyire}, as well as the impossibility of zero operators, are not dependent on whether the \isi{AP} is e-\textsc{given} or F-marked.  The \isi{operator} \textit{amilyen} cannot be separated from F-marked APs either, as shown in (\ref{huntablellonger3}):

\ea \label{huntablellonger3}
\ea	[]{\gll Az asztal	hosszabb,	mint \textbf{amilyen} \textbf{széles}	az iroda.\\
the	desk longer	than how wide	the	office\\
\glt `The desk is longer than the office is wide.'}
\ex	[*]{\gll Az	asztal hosszabb, mint	\textbf{amilyen} az iroda \textbf{széles}.\\
the	desk longer	than how the office	wide\\
\glt `The desk is longer than the office is wide.'}
\z
\z

By \isi{contrast}, \textit{amennyire} tolerates both positions of the \isi{AP}, as shown in (\ref{hungtablelongeramennyire3}):\footnote{Though the sentences marked as grammatical are all indeed grammatical, it must be mentioned that the degrees of acceptability may show individual differences, and there are structures that are clearly preferable. I will return to this question later on in the last section.}

\ea \label{hungtablelongeramennyire3}
\ea	\gll Az asztal	hosszabb,	mint \textbf{amennyire} \textbf{széles}	az iroda.\\
the	desk longer	than how.much wide	the	office\\
\glt `The desk is longer than the office is wide.'
\ex	\gll Az	asztal hosszabb, mint	\textbf{amennyire} az iroda \textbf{széles}.\\
the	desk longer	than how.much the office	wide\\
\glt `The desk is longer than the office is wide.'
\z
\z

Finally, just as in (\ref{hungzero}), zero operators are not allowed with F-marked APs either, as shown in (\ref{hungzerocontrast}):

\ea \label{hungzerocontrast}
\ea	[*]{\gll Az asztal	hosszabb,	mint \textbf{széles} az iroda.\\
the	desk longer	than wide the	office\\
\glt `The desk is longer than the office is wide.'}
\ex	[*]{\gll Az	asztal hosszabb, mint	az iroda \textbf{széles}.\\
the	desk longer	than the office	wide\\
\glt `The desk is longer than the office is wide.'}
\z
\z

The data shown in this section clearly demonstrate that \ili{Hungarian} has no \isi{Comparative Deletion}. In addition, there seem to be two types of operators. On the one hand, \textit{amilyen} is an \isi{operator} that must move together with the \isi{AP}: in this case, the higher \isi{copy} of the entire \isi{degree expression} is overt in the [Spec,\isi{CP}] position, and the lower \isi{copy} of the entire \isi{degree expression} is deleted regularly. On the other hand, the \isi{operator} \textit{amennyire} can move out on its own; it is by no means obligatory for it to do so, and if it does not, it behaves exactly the same way as \textit{amilyen}. However, if it moves out on its own, the higher \isi{copy} of \textit{amennyire} appears overtly in the [Spec,\isi{CP}] position but without any \isi{AP} there; in turn, the lower \isi{copy} of \textit{amennyire} is deleted regularly and the \isi{AP} itself remains overt in situ.

The difference between the two types of operators is also attested in interrogative operators. As shown in (\ref{hungintdegree3}), the \isi{operator} \textit{milyen} `how' does not allow the stranding of the \isi{AP}:

\ea \label{hungintdegree3}
\ea []{\gll \textbf{Milyen} \textbf{magas} volt	Péter?\\
how	tall was.\textsc{3sg}	Peter\\
\glt `How tall was Peter?'}
\ex	[*]{\gll \textbf{Milyen} volt Péter \textbf{magas}?\\
how	was.\textsc{3sg} Peter tall\\
\glt `How tall was Peter?'}
\z
\z

By \isi{contrast}, as shown by (\ref{hungintdegreemennyire3}), the \isi{operator} \textit{mennyire} `how much' may be separated from the \isi{AP}:\footnote{Again, \textit{mennyire} `how much' is a \isi{degree element} in the \isi{degree expression} just like its relative counterpart \textit{amennyire} `how much' for exactly the same reasons.} 

\ea \label{hungintdegreemennyire3}
\ea \gll \textbf{Mennyire} \textbf{magas} volt	Péter?\\
how.much	tall was.\textsc{3sg}	Peter\\
\glt `How tall was Peter?'
\ex	\gll \textbf{Mennyire} volt Péter \textbf{magas}?\\
how.much	was.\textsc{3sg} Peter tall\\
\glt `How tall was Peter?'
\z
\z

As can be seen, the interrogative operators \textit{milyen} and \textit{mennyire} have exactly the same distributions as their \isi{relative operator} counterparts, \textit{amilyen} and \textit{amennyire}, respectively. Since the difference seems to hold systematically, it presumably has to do with structural differences between the two types of operators.

\subsection{Operator positions} \label{sec:3operatorpositions}
In \chapref{ch:2}, I proposed a unified analysis for the structure of \isi{degree} expressions, concentrating primarily on the \isi{degree expression} in the \isi{matrix clause} of comparatives expressing \isi{inequality}. Recall that, for a string like \textit{far more intelligent than Peter is}, the representation in (\ref{treematrixqp}) was established:

\ea \label{treematrixqp} \upshape 
\begin{forest} baseline, qtree, for tree={align=center}
[QP
	[QP
		[far,roof]
	]
	[Q$'$
		[Q
			[-er\textsubscript{i} + much,name=er]
		]
		[DegP
			[AP [intelligent,roof]]
			[Deg$'$ [Deg [t\textsubscript{i},name=trace1]] [CP [than Peter is,roof]]]
		]
	]
]
\draw[->] (trace1) to[out=south west,in=south,looseness=1.5] (er);
\end{forest}
\z

Since this was and is intended to be a unified analysis for \isi{degree} expressions, I claim that the same structure is present in subclausal QPs, too. This has two main aspects: the difference between operators that cannot be extracted and ones that cannot, and the availability of operators as proforms standing for the entire \isi{degree expression}.

Let us first examine the general structure underlying \isi{degree} expressions. This is given in (\ref{treeop}), showing also the possible positions for operators (Op.):

\ea \label{treeop} \upshape 
\begin{forest} baseline, qtree, for tree={align=center}
[QP
	[QP
		[Op.,roof]
	]
	[Q$'$
		[Q
			[Op.\textsubscript{i},name=er]
		]
		[DegP
			[AP [\phantom{clever},roof]]
			[Deg$'$ [Deg [t\textsubscript{i},name=trace1]] [G]]
		]
	]
]
\draw[->] (trace1) to[out=south west,in=south,looseness=1.5] (er);
\end{forest}
\z

Recall that the \isi{DegP} is headed by the \isi{degree head}, which takes two arguments and which projects a \isi{QP} layer. The arguments of the \isi{degree head} are the lexical \isi{AP} itself (cf. \citealt{lechner2004}) and the Grade argument (\textit{G}), which expresses the standard value (cf. \citealt{lechner2004}). In matrix \isi{clausal} \isi{degree} expressions it is typically the \isi{subordinate clause} itself but it may also remain covert if it is recoverable from the context. Consider:

\begin{exe}
\ex \label{grade}
\begin{xlist} 
\exi{A:} Mary is as tall as Peter.
\exi{B:} No, she is taller.
\end{xlist}
\end{exe}
	
In (\ref{grade}), the Grade argument of \textit{taller} remains implicit as it is recoverable from the previous utterance. As far as the Grade argument of subclausal \isi{degree} expressions is concerned, it is also implicit but it relates the \isi{degree} in question to a certain point on a scale.

The \isi{QP} layer, as was seen in \chapref{ch:2}, is projected above the \isi{DegP} and the Deg head moves up to Q: the Q head itself is one of the possible positions for comparative operators. The specifier of the \isi{QP} may host other \isi{QP} modifiers; this is the other position that comparative operators may occupy. Note that these positions are \isi{operator} positions inasmuch as they may host operators; however, it is not necessary for them to be filled by operators: for instance, in matrix \isi{clausal} QPs such as (\ref{treematrixqp}), they obviously contain non-\isi{operator} elements.

Accordingly, the \isi{degree expression} in the \isi{subclause} has a structure conforming to (\ref{treeop}); \textit{amilyen} is a Deg head, and it ultimately occupies the Q position:

\ea \label{treeamilyen} \upshape 
\begin{forest} baseline, qtree, for tree={align=center}
[QP
	[Q$'$
		[Q
			[amilyen\textsubscript{i},name=er]
		]
		[DegP
			[AP [magas,roof]]
			[Deg$'$ [Deg [t\textsubscript{i},name=trace1]] [G]]
		]
	]
]
\draw[->] (trace1) to[out=south west,in=south,looseness=1.5] (er);
\end{forest}
\z

\largerpage[1]
By \isi{contrast}, \textit{amennyire} is a \isi{QP} \isi{modifier}:

\ea \label{treeamennyire} \upshape 
\begin{forest} baseline, qtree, for tree={align=center}
[QP
	[QP
		[amennyire,roof]
	]
	[Q$'$
		[Q
			[$\emptyset$\textsubscript{i},name=er]
		]
		[DegP
			[AP [magas,roof]]
			[Deg$'$ [Deg [t\textsubscript{i},name=trace1]] [G]]
		]
	]
]
\draw[->] (trace1) to[out=south west,in=south,looseness=1.5] (er);
\end{forest}
\z

Note that this applies to cases where the \isi{operator} \textit{amennyire} is used together with an \isi{adjective}. Interestingly, if it modifies an \isi{adverb}, it seems to be a Deg head, hence conforming to the structure in (\ref{treeamilyen}):

\largerpage[-2]
\ea
\ea []{\gll Mari jobban	tudja	a	verset,	mint \textbf{amennyire} \textbf{jól} Péter tudja a verset.\\
Mary better	knows	the	poem.\textsc{acc} than how.much	well Peter knows the poem.\textsc{acc}\\
\glt `Mary knows the poem better than Peter does.'}
\ex [*]{\gll Mari	jobban tudja a verset, mint	\textbf{amennyire}	Péter	tudja \textbf{jól}	a	verset.\\
Mary better	knows	the	poem.\textsc{acc}	than how.much	Peter	knows well the poem.\textsc{acc}\\
\glt `Mary knows the poem better than Peter does.'}
\z
\z

As can be seen, in these cases \textit{amennyire} has to move together with the \isi{adverb}, just as was seen for \textit{amilyen} `how' with adjectives, which suggests that \textit{amennyire} has been grammaticalised into a Deg head with adverbs. Since the main focus here is not to provide an account for this difference, I will not venture to analyse this issue any further.

Due to the fact that both \textit{amilyen} and \textit{amennyire} are operators, they cannot be co-present: only one [+rel] \isi{operator} is licensed in the clause, which then moves to a [Spec,\isi{CP}] position and checks off the [+rel] \isi{feature} there. The zero element in (\ref{treeamennyire}) is merely a \isi{degree marker}, not an \isi{operator}, and thus the \isi{operator} in [Spec,\isi{QP}] is necessary for the construction to survive.

The structural difference between \textit{amilyen} and \textit{amennyire} accounts for their different behaviour. While \textit{amennyire} is a \isi{QP} \isi{modifier} that thus may be extracted out of the entire \isi{degree expression} on its own, \textit{amilyen} is the head of that \isi{degree expression} itself, and therefore it cannot be extracted and naturally cannot move to the [Spec,\isi{CP}] position (a phrase position) as a single head.

\subsection{Proforms} \label{sec:3proforms}
Given the structural difference between individual operators described above, it is expected that further asymmetries should arise. This is indeed the case, as will be shown in connection with proforms. So far I have been dealing with \isi{degree} expressions containing a lexical \isi{AP}. However, this is not always necessary; for instance, \textit{amilyen} `how' may appear without a lexical \isi{AP}, as shown in (\ref{hungamilyenproform3}):

\ea \gll Mari	magasabb,	mint \textbf{amilyen} Péter volt. \label{hungamilyenproform3}\\
Mary taller	than how Peter was.\textsc{3sg}\\
\glt `Mary is taller than Peter was.'
\z
	
This is in line with the representation given in (\ref{treeamilyen}) for \textit{amilyen}: as a Deg head it may not require an overt \isi{AP} to be present in the structure but may stand for the entire \isi{degree expression} overtly.

The expectation is that \textit{amennyire} `how much' should behave differently in this respect, since the \isi{QP} \isi{modifier} then should be attached to a \isi{QP} that has no phonological content. This is indeed the case, as demonstrated by the ungrammaticality of (\ref{proformamennyire}):

\ea [*]{\gll Mari	magasabb, mint \textbf{amennyire}	Péter	volt. \label{proformamennyire}\\
Mary taller	than how.much Peter was.\textsc{3sg}\\
\glt `Mary is taller than Peter was.'}
\z

As can be seen, \textit{amennyire} is not allowed to appear as a \isi{comparative operator} on its own: the reason behind this is that, conforming to the representation given in (\ref{treeamennyire}), \textit{amennyire} is a \isi{QP} \isi{modifier} that should be attached to a \isi{QP} with some phonological content. Since the co-presence of Q heads and \isi{QP} modifiers is ruled out, the only way would be to have an overt lexical \isi{AP}, which is not the case in the example (\ref{proformamennyire}).\footnote{As was mentioned earlier, \textit{amennyire} `how much' seems to behave as a Deg head with adverbs but not with adjectives. If this is indeed so, then the expectation is that \textit{amennyire} should be able to function as a \isi{proform} with adverbs. Consider:

\ea \gll Mari	jobban tudja a verset, mint	amennyire	Péter	tudja	a verset. \label{proformamennyireadverb}\\
Mary better	knows	the	poem.\textsc{acc} than how.much	Peter	knows	the poem.\textsc{acc}\\
\glt `Mary knows the poem better than Peter does.'
\z
	
The grammaticality of (\ref{proformamennyireadverb}) above shows that this is indeed so, and thus there is a difference that holds between \textit{amennyire} as an adjectival \isi{modifier} and \textit{amennyire} as an adverbial \isi{modifier}.}

\section{Operators in English} \label{sec:3operatorsinenglish}
Having established all this, let us now return to \ili{English} comparative operators. Altogether, there are three candidates: the zero, \textit{how} and \textit{what} (see \citealt{chomsky1977} on treating \textit{what} as an \isi{operator} in comparatives).

Let us start with the element \textit{what}, which prohibits the co-presence of an overt \isi{AP} in [Spec,\isi{CP}] but not in its \isi{base position}. This is demonstrated by (\ref{whatcomp}):\footnote{I owe many thanks to Craig Thiersch for the discussion of the data with \textit{what}.}

\ea \label{whatcomp}
\ea	[\%]{Mary is taller than \textbf{what} Peter is.}
\ex	[*]{Mary is taller than \textbf{what tall} Peter is.} \label{whattall}
\ex	[*]{The desk is longer than \textbf{what wide} the office is.} \label{whatwide}
\ex	[\%]{The desk is longer than \textbf{what} the office is \textbf{wide}.} \label{whathead}
\z
\z

As shown by (\ref{whatcomp}), \textit{what} may appear below \textit{than}, at least in certain non-standard dialects of \ili{English}. However, it is not allowed to co-occur with a lexical \isi{AP}, as in (\ref{whattall}) and (\ref{whatwide}). Yet it is possible to have an overt, contrastive \isi{AP} in its \isi{base position}, as in (\ref{whathead}). If \textit{what} were an \isi{operator} taking a lexical \isi{AP} in [Spec,\isi{DegP}], then (\ref{whattall}) and (\ref{whatwide}) should be grammatical. On the other hand, if \textit{what} were a \isi{proform} \isi{operator} never allowing the \isi{gradable predicate} to be overt, then (\ref{whathead}) should be ruled out. Therefore, it seems that \textit{what} in comparatives is actually a lower C head, and the \isi{comparative operator} is zero, just as in Standard \ili{English} (note that the presence of an \isi{operator} is required by \isi{degree} semantics). In fact, there are a number of languages allowing the lexicalisation of the lower C head in comparatives, as will be shown later on for \ili{German}; see \citet{bacskaiatkari2014dia} on \ili{German} and \ili{Hungarian}, cf. \citet{jaeger2010} on \ili{German}, and \citet{bacskaiatkari2016alh} on \ili{Slavic}.

Turning now to \textit{how}, it must be noted that \textit{how} as a \isi{comparative operator} again shows dialectal variation. Consider the examples in (\ref{howsubclause3}):

\ea \label{howsubclause3}
\ea	[\%]{Mary is taller than \textbf{how tall} Peter is.} \label{howtallgrammatical}
\ex	[*]{Mary is taller than \textbf{how} Peter is \textbf{tall}.} \label{howstrandedtall}
\ex	[*]{Mary is taller than \textbf{how} Peter is.} \label{hownoap}
\ex	[\%]{The desk is longer than \textbf{how wide} the office is.} \label{howwidegrammatical}
\ex	[*]{The desk is longer than \textbf{how} the office is \textbf{wide}.} \label{howstrandedwide}
\z
\z

The only acceptable configurations with \textit{how} as a \isi{comparative operator} are given in (\ref{howtallgrammatical}) and (\ref{howwidegrammatical}). As indicated, these are completely well formed for some speakers, while for others they are ungrammatical. However, constructions such as (\ref{howstrandedtall}) and (\ref{howstrandedwide}), where the \isi{AP} is stranded, are ungrammatical even for those who would accept (\ref{howtallgrammatical}) and (\ref{howwidegrammatical}), which suggests that \textit{how} is a Deg head that cannot be extracted out of the \isi{degree expression}. Unlike \ili{Hungarian} \textit{amilyen} `how', \textit{how} is a Deg head that requires the presence of an overt \isi{AP}, as indicated by the ungrammaticality of (\ref{hownoap}).

Note that whether a given Deg head may combine with a lexical \isi{AP} is independent from whether the \isi{AP} is e-\textsc{given} or not: Deg heads that must take APs take them in either case.

Finally, let us turn to the zero \isi{comparative operator}, which is acceptable for all \ili{English} speakers. This is a Deg head that cannot move out on its own. Observe the difference in (\ref{zeropatterns3}):

\ea \label{zeropatterns3}
\ea	[??/*]{Mary is taller than Peter is \textbf{tall}.} \label{zerotall}
\ex	[]{The desk is longer than the office is \textbf{wide}.} \label{zerowide}
\z
\z

If the zero were a \isi{QP} \isi{modifier}, then it should be able to move out to the [Spec,\isi{CP}] on its own and (\ref{zerotall}) should be acceptable, just like (\ref{zerowide}); however, (\ref{zerotall}) is clearly unacceptable to an extent that cannot be attributed merely to the redundancy of the \isi{AP}. On the other hand, the fact that the zero can co-occur with a lexical \isi{AP} in cases such as (\ref{zerowide}) implies that in canonical \isi{Comparative Deletion} constructions, where an e-\textsc{given} \isi{AP} is eliminated, there is indeed \isi{deletion} at hand: as has been said, the Deg head imposes restrictions on the presence or the absence of any \isi{AP} irrespectively of whether that \isi{AP} is e-\textsc{given} or not.

\section{Operators cross-linguistically} \label{sec:3operatorscrosslinguistically}
From the discussion above, it should be clear that comparative operators may differ from each other in two respects: \isi{overtness} and extractability. Since these criteria are independent from each other, this leaves one with four logical possibilities for comparative operators. The operators I have dealt with so far (that is, the ones in \ili{English} and \ili{Hungarian}) can be grouped according to Table \ref{tableopenghun}.

\begin{table}
\begin{tabular}{ccc}
\lsptoprule
{} & overt & covert\\
\midrule
Deg head & {\textit{how} (\ili{English})} & {zero (\ili{English})}\\
{} & {\textit{amilyen} (\ili{Hungarian})} & {}\\
{} & {} & {}\\
\isi{QP} \isi{modifier} & {\textit{amennyire} (\ili{Hungarian})} & {}\\
\lspbottomrule
\end{tabular}
\caption{Comparative operators in English and Hungarian}
\label{tableopenghun}
\end{table}

The question is of course how operators from other languages fit into this scheme: more precisely, whether there are other overt \isi{QP} \isi{modifier} operators and whether there are covert \isi{QP} \isi{modifier} operators at all.

Let us first examine the case of \ili{Czech}.\footnote{For his indispensable help with the \ili{Czech} data, I owe many thanks to Radek Šimík.} As shown in (\ref{czechint3}), \ili{Czech} has the \isi{operator} \textit{jak} `how' that may appear in interrogative clauses:

\ea \label{czechint3}
\ea	\gll \textbf{Jak} \textbf{vysoký} je	Karel? \label{czechjakap}\\
how	tall	is	Charles\\
\glt	`How tall is Charles?'
\ex \gll \textbf{Jak} je	Karel	\textbf{vysoký}? \label{czechjakstranded}\\
how	is Charles tall\\
\glt `How tall is Charles?'
\z
\z

As indicated, \textit{jak} can appear together with the \isi{AP}, as in (\ref{czechjakap}), but the \isi{AP} may also be stranded, as in (\ref{czechjakstranded}). This shows that \textit{jak} is a \isi{QP} \isi{modifier}. Note that if \textit{jak} were a VP-\isi{modifier} and base-generated independently from the \isi{AP}, then (\ref{czechjakap}) should not be possible because the \isi{AP} would not undergo \textit{wh}-\isi{movement} in itself.

The expectation is that the same can be observed in comparative subclauses. This is indeed the case, as shown by (\ref{czechcomp3}):

\ea \label{czechcomp3}
\ea [??]{\gll Marie	je vyšší,	než	\textbf{jak} \textbf{vysoký} je Karel.\\
Mary	is taller	than	how	tall is Charles\\
\glt `Mary is taller than Charles.'}
\ex	[?]{\gll Marie je	vyšší, než	\textbf{jak} je	\textbf{vysoký}	Karel.\\
Mary	is taller	than how is	tall Charles\\
\glt `Mary is taller than Charles.'}
\ex [??]{\gll Ten stůl je delší, než	\textbf{jak} \textbf{široká} je	ta kancelář.\\
that desk	is longer	than how wide is that	office\\
\glt `The desk is longer than the office is wide.'}
\ex	[]{\gll Ten	stůl je	delší, než \textbf{jak} je ta kancelář \textbf{široká}.\\
that desk	is longer	than how is that office wide\\
\glt `The desk is longer than the office is wide.'}
\z
\z

The slight \isi{markedness} of the examples above stems from two factors: positional preferences (that is, the \isi{AP} is preferably stranded instead of moving together with the \isi{operator} as high as the [Spec,\isi{CP}] position), and redundancy in the case of an e-\textsc{given} \isi{AP}. I will return to the positional preferences later; what is important here is that these are all possible structures, indicating that \textit{jak} behaves in the same way as in interrogatives, in that it may be separated from the lexical \isi{AP}.

On the other hand, \ili{Czech} does not have a zero \isi{comparative operator}. Consider the examples in (\ref{czechnozero3}):

\ea \label{czechnozero3}
\ea [*]{\gll Marie	je vyšší,	než	\textbf{vysoký} je Karel.\\
Mary	is taller	than tall is Charles\\
\glt `Mary is taller than Charles.'}
\ex	[*]{\gll Marie je	vyšší, než je	\textbf{vysoký}	Karel.\\
Mary	is taller	than is	tall Charles\\
\glt `Mary is taller than Charles.'}
\ex [*]{\gll Ten stůl je delší, než	\textbf{široká} je	ta kancelář.\\
that desk	is longer	than wide is that	office\\
\glt `The desk is longer than the office is wide.'}
\ex	[*]{\gll Ten	stůl je	delší, než je ta kancelář \textbf{široká}.\\
that desk	is longer	than is that office wide\\
\glt `The desk is longer than the office is wide.'}
\z
\z

We can conclude that the \isi{comparative operator} is invariably an overt \isi{QP} \isi{modifier} in \ili{Czech}.

Let us now turn to \ili{Dutch},\footnote{I owe many thanks to Jos Tellings for all his help with the \ili{Dutch} data.} where the \isi{interrogative operator} \textit{hoe} `how' is non-separable from the \isi{AP}, as can be seen in (\ref{dutchhoe3}):

\ea \label{dutchhoe3}
\ea []{\gll \textbf{Hoe} \textbf{groot}	is Jan?\\
how	tall is	John\\
\glt `How tall is John?'}
\ex	[*]{\gll \textbf{Hoe} is	Jan	\textbf{groot}? \label{hoestranding}\\
how	is John	tall\\
\glt `How tall is John?'}
\z
\z
		
Since \textit{hoe} does not allow the stranding of the \isi{AP}, as demonstrated by (\ref{hoestranding}), it can be concluded that it is a Deg head. Accordingly, \textit{hoe} as a comparative \isi{relative operator} is also a Deg head (for speakers who find \textit{hoe} acceptable as a \isi{comparative operator}).\footnote{Note that the acceptability of \textit{hoe} `how' in comparatives varies among dialects and speakers, similarly to what was attested for \textit{how} in \ili{English}. I conducted a short online survey in August–September 2013 with 70 native participants (many thanks go to Laura Bos and Marlies Kluck for their help in distributing the survey), in which informants were asked to rate sentences on a scale from 1 (bad) to 5 (good). The sentence given in (\ref{howgroot}) here was accepted as fully grammatical (5) by 16\% of the participants, while the sentence given in (\ref{hoebreed}) by 27\%. This shows that even if \textit{hoe} as a \isi{comparative operator} is not acceptable for all speakers, its acceptability is still significant. Since my aim here is not to investigate comparatives in \ili{Dutch} but rather to give a cross-linguistic survey, I will not venture to analyse and describe the results of the online survey here.} Consider the examples in (\ref{dutchsubclausehoe3}):

\ea \label{dutchsubclausehoe3}
\ea [\%]{\gll Maria	is groter	dan	\textbf{hoe} \textbf{groot} Jan	is. \label{howgroot}\\
Mary is	taller than	how	tall John	is\\
\glt `Mary is taller than John.'}
\ex [*]{\gll Maria	is groter	dan	\textbf{hoe}	Jan	\textbf{groot}	is.\\
Mary is	taller than	how	John	tall	is\\
\glt `Mary is taller than John.'}
\ex	[\%]{\gll De tafel is	langer dan \textbf{hoe} \textbf{breed} het kantoor is. \label{hoebreed}\\
the	table	is longer than how wide	the.\textsc{n} office is\\
\glt `The table is longer than the office is wide.'}
\ex	[*]{\gll De	tafel	is langer	dan	\textbf{hoe}	het	kantoor \textbf{breed} is.\\
the	table	is longer	than how the.\textsc{n}	office wide is\\
\glt `The table is longer than the office is wide.'}
\z
\z

In addition, it is worth mentioning that \textit{hoe} cannot be a \isi{proform}, as shown by the ungrammaticality of (\ref{danhoe3}):

\ea	[*]{\gll Maria is	groter dan \textbf{hoe} Jan is. \label{danhoe3}\\
Mary is	taller than	how	John is\\
\glt `Mary is taller than John.'}
\z

Thus, \textit{hoe} behaves in the same way as \textit{how} does in \ili{English} (that is, for speakers who accept it as a \isi{comparative operator}).

In addition to \textit{hoe}, \ili{Dutch} also has a covert \isi{comparative operator}; this, however, behaves differently from the \isi{zero operator} observed in \ili{English}.\footnote{Again, there is considerable variation among speakers but (\ref{dutchzerogiven}) was judged by 10\% to be fully acceptable (5) and by 21\% to be acceptable (4) in the online survey mentioned before. On the other hand, (\ref{dutchzerocontrast}) was fully acceptable (5) for 81\% and acceptable (4) for 11\%.} Consider the examples in (\ref{dutchzero}):

\ea \label{dutchzero}
\ea [?]{\gll Maria is	groter dan Jan \textbf{groot} is. \label{dutchzerogiven}\\
Mary is	taller than	John tall is\\
\glt `Mary is taller than John.'}
\ex []{\gll De tafel is langer dan het kantoor \textbf{breed}	is. \label{dutchzerocontrast}\\
the	table	is longer	than the.\textsc{n} office wide	is\\
\glt `The table is longer than the office is wide.'}
\z
\z

If \ili{Dutch} had no \isi{zero operator}, then the sentences in (\ref{dutchzero}) would be ungrammatical, as in \ili{Hungarian}: see (\ref{hungzero}) and (\ref{hungzerocontrast}). On the other hand, if the \ili{Dutch} \isi{zero operator} were a Deg head like the one in \ili{English}, then (\ref{dutchzerogiven}) should be ungrammatical, which is not the case: though marked because of redundancy, (\ref{dutchzerogiven}) is still acceptable, in \isi{contrast} to (\ref{zerotall}). This leaves only one option: namely that the \isi{zero operator} in \ili{Dutch} is a \isi{QP} \isi{modifier}. Of course, this also means that the \isi{AP} may in principle move together with the \isi{operator} to [Spec,\isi{CP}]: in this case, just like in \ili{English}, it is deleted by \isi{Comparative Deletion}.

The same is true for the \isi{zero operator} in \ili{German}, as shown by (\ref{germanzero3}):

\ea \label{germanzero3}
\ea	[?]{\gll Maria ist größer	als	Johann \textbf{groß} ist. \label{germanzerogiven}\\
Mary is	taller than	John tall	is\\
\glt `Mary is taller than John.'}
\ex	[]{\gll Der	Tisch	ist	länger	als	das	Büro \textbf{breit} ist.\\
the.\textsc{m} table is	longer than	the.\textsc{n} office	wide is\\
\glt `The table is longer than the office is wide.'}
\z
\z

Again, the \isi{markedness} of (\ref{germanzerogiven}) is due to redundancy, as opposed to (\ref{zerotall}) in \ili{English}. Therefore, the \isi{AP} is indeed available in a stranded position in \ili{German}.

Note that the other logically possible candidate for the \isi{comparative operator} in \ili{German} is not available as an \isi{operator} in comparative subclauses. The \isi{interrogative operator} \textit{wie} `how' is a Deg head, as can be seen in (\ref{wieint3}):

\ea \label{wieint3}
\ea []{\gll \textbf{Wie} \textbf{groß} ist Johann?\\
how	tall is	John\\
\glt `How tall is John?'}
\ex	[*]{\gll \textbf{Wie}	ist	Johann \textbf{groß}?\\
how	is John	tall\\
\glt `How tall is John?'}
\z
\z

This suggests that \textit{wie} should appear together in the [Spec,\isi{CP}] position in comparatives. However, this is not the case, as demonstrated by (\ref{wienoop3}):

\ea \label{wienoop3}
\ea	[*]{\gll Maria	ist	größer	als	\textbf{wie}	\textbf{groß} Johann	ist.\\
Mary is	taller	than	how	tall	John	is\\
\glt `Mary is taller than John.'}
\ex	[*]{\gll Der	Tisch	ist	länger als	\textbf{wie}	\textbf{breit}	das Büro ist.\\
the.\textsc{m}	table	is	longer than	how	wide	the.\textsc{n} office is\\ 
\glt `The table is longer than the office is wide.'}
\z
\z

The data indicate that \textit{wie} is not a \isi{comparative operator} in \ili{German}: it is in fact a grammaticalised (lower) C head. The discussion of this question falls outside the scope of the present investigation; see the arguments in \citet{bacskaiatkari2014dia} and \citet[223--226]{bacskaiatkari2014diss}, following \citet{jaeger2010}. What matters for us here is that \ili{German} has only a zero \isi{QP} \isi{modifier} \isi{operator}.

The same asymmetry can be observed in \ili{Italian}, too (the discussion below applies to Northern dialects). Consider the examples in (\ref{quanto3}):

\ea \label{quanto3}
\ea []{\gll	\textbf{Quanto} \textbf{alta}	è	Maria? \label{quantoint}\\
how	tall.\textsc{f}	is	Mary\\
\glt `How tall is Mary?'}
\ex	[*]{\gll \textbf{Quanto}	è	Maria	\textbf{alta}?\\
how	is Mary	tall.\textsc{f}\\
\glt `How tall is Mary?'}
\z
\z

As can be seen, \textit{quanto} `how' is also a Deg head, and the \isi{AP} cannot be stranded. However, the grammatical interrogative configuration in (\ref{quantoint}) has no matching counterpart in the comparative \isi{subclause}, as demonstrated in (\ref{quantosubclause3}):

\ea \label{quantosubclause3}
\ea []{\gll Maria	è	più	alta	di	quanto	Giovanni sia alto. \label{quantostranding}\\
Mary is	more tall.\textsc{f} of	how	John be.\textsc{sbjv.3sg} tall.\textsc{m}\\
\glt `Mary is taller than John.'}
\ex	[*]{\gll Maria	è	più	alta	di	quanto	alto Giovanni sia. \label{quantonostranding}\\
Mary is	more	tall.\textsc{f}	of	how	tall.\textsc{m} John be.\textsc{sbjv.3sg}\\
\glt `Mary is taller than John.'}
\ex	[]{\gll La	tavola	è	più	lunga	di quanto l'ufficio	sia	largo. \label{quantostrandingcontrast}\\
the.\textsc{f} table is	more long.\textsc{f} of how the.office be.\textsc{sbjv.3sg} wide.\textsc{m}\\
\glt `The table is longer than the desk is wide.'}
\ex	[*]{\gll La	tavola	è	più	lunga	di quanto largo l'ufficio	sia. \label{quantonostrandingcontrast}\\
the.\textsc{f} table is	more long.\textsc{f} of how wide.\textsc{m} the.office be.\textsc{sbjv.3sg}\\
\glt `The table is longer than the desk is wide.'}
\z
\z

Again, the issue of grammaticalisation in \ili{Italian} comparatives cannot be addressed here (see \citealt[226--228]{bacskaiatkari2014dia}, \citealt{bacskaiatkari2014diss}); what is important here is that \textit{quanto} cannot be interpreted here as a Deg head, otherwise (\ref{quantonostranding}) and (\ref{quantonostrandingcontrast}) should be grammatical and (\ref{quantostranding}) and (\ref{quantostrandingcontrast}) should be ruled out. In other words, \textit{quanto} is not the \isi{comparative operator}. On the other hand, the grammaticality of (\ref{quantostranding}) and (\ref{quantostrandingcontrast}), showing APs in their base positions, indicates that the \isi{degree} expressions containing these APs have a \isi{QP} \isi{modifier} \isi{zero operator}.

Naturally, several other languages could be examined in this respect; however, the point here is not to provide a fully-fledged comparative analysis of several languages but rather to show how \isi{overtness} and extractability interact. This allows for an update in the representation shown in Table \ref{tableopenghun}, given in Table \ref{tablecrosslinguistic}.

\begin{table}
\begin{tabular}{ccc}
\lsptoprule
{} & {overt} & {covert}\\
\midrule
{Deg head} & {\textit{how} (\ili{English})} & {zero (\ili{English})}\\
{} & {\textit{amilyen} (\ili{Hungarian})} & {}\\
{} & {\textit{hoe} (\ili{Dutch})} & {}\\
{} & {} & {}\\
{\isi{QP} modifier} & {\textit{amennyire} (\ili{Hungarian})} & {zero (\ili{German})}\\
{} & {\textit{jak} (\ili{Czech})} & {zero (\ili{Dutch})}\\
{} & {} & {zero (\ili{Italian})}\\
\lspbottomrule
\end{tabular}
\caption{Comparative operators cross-linguistically}
\label{tablecrosslinguistic}
\end{table}

As shown, there are indeed covert \isi{QP} \isi{modifier} operators and other types are also more widely attested. In addition, it has to be stressed that a given language may have several operators and these do not necessarily fall into the same slot.

While the availability of both an overt and a \isi{covert operator} in a given language seems to be a straightforward option, the case of \ili{Hungarian} with two overt operators seems to be special. The availability of these operators is also due to the fact that \ili{Hungarian} developed a rich system of operators in Late Old \ili{Hungarian} and Early Middle \ili{Hungarian}, and there are several \isi{degree} operators (cf. \citealt[525]{gvarga1992}, \citealt{bacskaiatkari2013atoh, bacskaiatkari2014dia}). This means that at some point there were distinct operators for diverse functions, yet grammaticalisation processes may affect the system. For instance, VP-adverbs may grammaticalise into \isi{QP} modifiers within \isi{degree} expressions, and quantifiers may grammaticalise into \isi{degree} heads (which seems to be a common process, cf. \citealt{doetjes2008}), and \isi{degree} operators may also grammaticalise into C heads in comparative subclauses (but naturally not in interrogatives), as shown by \citet{bacskaiatkari2014dia}, \citet[175--228]{bacskaiatkari2014diss}. Still, the question arises whether \ili{Hungarian} is unique in having both an overt Deg head \isi{operator} and an overt \isi{QP} \isi{modifier} \isi{operator} in interrogative and relative structures. Interestingly, \ili{Estonian} exhibits a similar distinction between \textit{kui} `how' and \textit{kuivõrd} `how much' in interrogatives.\footnote{I owe many thanks to Nele Salveste for the data.} Consider the examples given in (\ref{estonian3}):

\ea \label{estonian3}
\ea []{\gll	\textbf{Kui} \textbf{pikk} on Peter?\\
how	tall is Peter\\
\glt `How tall is Peter?'}
\ex [*]{\gll \textbf{Kui} on	Peter	\textbf{pikk}?\\
how	is	Peter	tall\\
\glt `How tall is Peter?'}
\ex []{\gll	\textbf{Kuivõrd} \textbf{pikk} on	Peter?\\
how.much	tall	is	Peter\\
\glt `How tall is Peter?'}
\ex []{\gll \textbf{Kuivõrd} on Peter	\textbf{pikk}?\\
how.much	is	Peter	tall\\
\glt `How tall is Peter?'}
\z
\z
	
As can be seen, the \isi{operator} \textit{kui} is not separable from the lexical \isi{AP} while the \isi{operator} \textit{kuivõrd} is, demonstrating essentially the same difference that holds between \ili{Hungarian} \textit{milyen} `how' and \textit{mennyire} `how much'. As shown in (\ref{polishparadigm3}), \ili{Polish} shows the same phenomenon with the operators \textit{jak} `how' and \textit{jaki} `how' (cf. \citealt[81]{borsleyjaworska1981}):

\ea \label{polishparadigm3}
\ea []{\gll \textbf{Jak} \textbf{wysoki} jest Karol?\\
how tall is Charles\\
\glt `How tall is Charles?'}
\ex [*]{\gll \textbf{Jak} jest Karol \textbf{wysoki}?\\
how is Charles tall\\
\glt `How tall is Charles?'}
\ex []{\gll \textbf{Jaki} \textbf{wysoki} jest Karol?\\
how tall is Charles\\
\glt `How tall is Charles?'}
\ex []{\gll \textbf{Jaki} jest Karol \textbf{wysoki}?\\
how is Charles tall\\
\glt `How tall is Charles?'}
\z
\z

However, the same difference cannot be traced in comparative subclauses in either \ili{Estonian} or \ili{Polish}, since the interrogative operators in question are not available as comparative operators. Since the investigation of this issue would lead further than necessary here, I will leave this question open; what is important for us is that the availability of a Deg head \isi{operator} and a \isi{QP} \isi{modifier} \isi{operator} in \isi{degree} expressions is attested in languages other than \ili{Hungarian}, too.

\section{The Overtness Requirement} \label{sec:3overtnessrequirement}
Observing Table \ref{tablecrosslinguistic}, the answer to \isi{Comparative Deletion} is quite straightforward. \isi{Comparative Deletion}, that is, the obligatory elimination of the quantified expression in the [Spec,\isi{CP}] position, is always attested if the \isi{comparative operator} is a covert Deg head (as in Standard \ili{English}) and it may take place if the \isi{comparative operator} is a covert \isi{QP} \isi{modifier}, provided that the lexical \isi{AP} moves up together with the \isi{operator} (e.g. in \ili{Dutch}). That is, \isi{Comparative Deletion} takes place if (and only if) there is a \isi{covert operator} taking a lexical \isi{AP} in the relevant [Spec,\isi{CP}] position.

Essentially, then, \isi{Comparative Deletion} takes place because otherwise a requirement on certain \isi{operator} elements would be violated. I propose that this \isi{Overtness Requirement} states that a phonologically visible lexical XP may appear in an \isi{operator position} only if it appears together with a phonologically visible \isi{operator}.

Let us elaborate on this in more detail. A \isi{QP} or a \isi{DP} containing a \isi{QP} qualifies as [+rel] if there is a \isi{relative operator} that either heads the \isi{QP} or percolates this \isi{feature} up to the \isi{DP}. Phrases equipped with a [+rel] \isi{feature} must move up to the [Spec,\isi{CP}] position because of their [\textsc{edge}] \isi{feature}: unlike, for instance, [+wh], there is no relative-in-situ (at least in the languages under scrutiny). However, in a [+rel] position only material that is overtly marked as [+rel] may appear overtly: in the case of a zero \isi{comparative operator}, this condition is clearly not satisfied. 

Note that, as far as the CP-domain is concerned, the \isi{overtness} requirement is meant to capture the impossibility of \isi{lexical material} without operators in CPs that are indeed \isi{operator} positions, in this case [+rel]. \ili{German} is known to have V2 in main clauses and this is generally attributed to the fact that the \isi{verb} moves up to a C head and a phrase-sized constituent (most typically the subject) moves to the specifier of the same \isi{CP} (see e.g. \citealt{fanselow2004}). In these cases, the presence of lexical phrases is allowed without an overt (relative) \isi{operator} but this is so because these CPs are not [+rel] and the \isi{Overtness Requirement} simply does not apply to them.

The overt realisation of lower copies does not have to face this problem, hence the grammaticality of subcomparatives in \ili{English}. On the other hand, if the \isi{operator} is overt, irrespectively of whether it has a lexical phrase alongside it, no \isi{deletion} takes place. Similarly, if there is a \isi{zero operator} that moves on its own, there is no need for \isi{deletion}: all material in the [Spec,\isi{CP}] is already covert.

The proposal, based on cross-linguistic data, is strongly built on the formal characteristics of comparative operators and does not try to link \isi{Comparative Deletion} directly to the information-structural properties of the \isi{AP}. In other words, defining \isi{Comparative Deletion} as an operation eliminating the \textsc{given} \isi{AP} would be fundamentally flawed as \isi{Comparative Deletion} is essentially about the properties of the \isi{operator}.

There are basically three independent factors here interacting with each other. First, the \isi{overtness} of the \isi{operator} defines whether \isi{Comparative Deletion} is required to take place in [Spec,\isi{CP}]. Second, the position of the \isi{operator} in the \isi{degree expression} decides whether the \isi{AP} is separable or not. Third, the informa\-tion-structural properties define the preferred position of the \isi{AP}. I will return to the last criterion in the next section; for now, let us concentrate on the properties of operators.

There is considerable variation with respect to the acceptability of operators as comparative operators. In \ili{English}, for instance, \textit{how} is only marginally or dialectally acceptable, while the \isi{zero operator} is fully grammatical. Naturally, any such candidate has to qualify as a \isi{degree element} (either interpreted as the Deg head of the entire \isi{degree expression} or as a \isi{QP} \isi{modifier}), otherwise it cannot be interpreted as a \isi{comparative operator}. 

In addition, a \isi{comparative operator} is equipped with comparative and relative features, that is, [+compr] and [+rel]. The separation of [\textpm compr] and [\textpm rel] is justified: a \isi{feature} matrix of these two binary features gives four logical possibilities, all of which are attested. Consider Table \ref{tablerelcompr} showing examples from \ili{English}.

\begin{table}
\begin{tabular}{ccc}
\lsptoprule
{} & {[+rel]} & {[--rel]}\\
\midrule
{[+compr]} & {zero} & {-\textit{er}}\\
{} & {\% \textit{how}} & {}\\
{} & {} & {}\\
{[--compr]} & {\textit{which}} & {zero \isi{absolute degree} marker}\\
\lspbottomrule
\end{tabular}
\caption{The features [\textpm compr] and [\textpm rel]}
\label{tablerelcompr}
\end{table}

\largerpage[1]
The acceptability of individual elements as comparative operators fundamentally depends on whether they are equipped with both a [+rel] and a [+compr] \isi{feature}. This may vary depending on the dialect and/or the speakers, hence the differences attested between dialects and individual speakers.

\section{The role of information structure} \label{sec:3role}
Finally, let us briefly revisit the issue of \isi{information structure} as attested in \isi{comparative clause} formation. As has been established, \isi{Comparative Deletion} is not the same as the elimination of a \textsc{given} \isi{AP}. It is nevertheless true that once \isi{Comparative Deletion} takes place in the relevant [Spec,\isi{CP}] position, the lower \isi{copy} of an e-\textsc{given} \isi{AP} is preferably eliminated: that is, unless there is some \isi{contrast} expressed by this \isi{AP}, it undergoes \isi{deletion} regularly as a lower \isi{copy}, as in (\ref{zerotall}), repeated here as (\ref{zerotallrepeated}):

\ea	[??/*]{Mary is taller than Peter is \textbf{tall}.} \label{zerotallrepeated}
\z

On the other hand, if there is no \isi{Comparative Deletion}, then the APs may remain overt irrespectively of their information-structural status.

It is expected, though, that certain positional differences between e-\textsc{given} and F-marked APs may arise: more precisely, contrastive elements are expected to prefer \isi{contrast} positions, while non-contrastive elements are presumably more likely to appear in neutral positions.

Naturally, the question makes sense only in the case of separable operators, that is, \isi{QP} modifiers. If the \isi{operator} is a Deg head, there is no choice in the positions for an overt \isi{AP}: if the \isi{operator} itself is overt, such as \textit{how} in certain \ili{English} dialects or \textit{amilyen} in \ili{Hungarian}, invariably the higher \isi{copy} is realised - if the \isi{operator} is zero, as in standard \ili{English}, it is always the lower \isi{copy} of the \isi{AP} that is realised, the higher one being regularly deleted due to the \isi{Overtness Requirement}. However, if the \isi{operator} is separable, it is expected that \textsc{given} APs will typically appear in neutral positions and F-marked APs will appear in stress positions.

Let us first have a look at \ili{Czech}, where the \isi{operator} \textit{jak} `how' is a separable \isi{QP} \isi{modifier}. If it is combined with an e-\textsc{given} \isi{AP}, the pattern in (\ref{czechgiven3}) arises:

\ea \label{czechgiven3}
\ea	[??]{\gll Marie je vyšší, než	\textbf{jak}	\textbf{vysoký} je	Karel. \label{czechgivencp}\\
Mary	is taller than how	tall	is	Charles\\
\glt `Mary is taller than Charles.'}
\ex	[?]{\gll Marie je	vyšší, než	\textbf{jak}	je	\textbf{vysoký}	Karel. \label{czechgiveninternal}\\
Mary	is	taller	than	how	is	tall	Charles\\
\glt `Mary is taller than Charles.'}
\ex	[\#]{\gll Marie	je	vyšší,	než	\textbf{jak}	je	Karel	\textbf{vysoký}. \label{czechgivenfinal}\\
Mary	is	taller	than	how	is	Charles	tall\\
\glt `Mary is taller than Charles.'}
\z
\z
		
The differences in the acceptability of these examples can be explained via considering the basic information structural properties of \ili{Czech} clauses (Radek Šimík, p.c.). The most preferable position for a \textsc{given} \isi{AP} is the one in (\ref{czechgiveninternal}), where it is in a position for \textsc{given} elements; this is even preferable to the [Spec,\isi{CP}] position, which is by definition not reserved for either \textsc{given} or F-marked elements. Finally, (\ref{czechgivenfinal}) is infelicitous because the \isi{AP} appears in the canonical \isi{contrast position}, that is, clause-finally; however, in (\ref{czechgivenfinal}) the \isi{AP} does not carry \isi{contrast} at all, and the main \isi{contrast} is expressed by the \isi{DP} \textit{Karel}, which should appear clause-finally, as in (\ref{czechgivencp}) and (\ref{czechgiveninternal}).

Turning now to F-marked APs, the opposite pattern is attested, as illustrated in (\ref{czechcontrast3}): the infelicitous configuration arises when the AP is clause-internal, not when it is clause-final.

\ea \label{czechcontrast3}
\ea	[??]{\gll Ten	stůl	je	delší, než	\textbf{jak}	\textbf{široká}	je	ta kancelář.\\
that desk	is longer	than how	wide is	that office\\
\glt `The desk is longer than the office is wide.'}
\ex	[\#]{\gll Ten	stůl	je	delší,	než	\textbf{jak}	je	\textbf{široká}	ta kancelář. \label{czechcontrastinternal}\\
that desk	is longer	than how is	wide that office\\
\glt `The desk is longer than the office is wide.'}
\ex []{\gll Ten	stůl	je	delší,	než	\textbf{jak}	je	ta	kancelář \textbf{široká}. \label{czechcontrastfinal}\\
that desk	is longer	than wide	is that	office wide\\
\glt `The desk is longer than the office is wide.'}
\z
\z

Again, the [Spec,\isi{CP}] position, which is not specified in terms of informational-structural content, is less preferred than the most natural one, which is the clause-final position (the canonical \isi{contrast position}), as shown in (\ref{czechcontrastfinal}). Just as expected, the appearance of the F-marked \isi{AP} in a position maintained for neutral elements is infelicitous, see (\ref{czechcontrastinternal}): since the main \isi{contrast} is expressed by this \isi{AP}, it should appear clause-finally, where it can bear \isi{sentential stress}.

Similarly to \ili{Czech}, \ili{Hungarian} also shows a predictable correlation between the information-structural properties of the APs and their preferred positions; obviously, this is attested only in the case of \textit{amennyire} `how much', which is a \isi{QP} \isi{modifier}. Consider the following examples:

\ea \label{hunggiven}
\ea []{\gll	Mari	magasabb,	mint	\textbf{amennyire}	\textbf{magas}	Péter volt. \label{hunggivencp}\\
Mary	taller	than	how.much	tall	Peter was.\textsc{3sg}\\
\glt `Mary is taller than Peter was.'}
\ex	[\#]{\gll Mari	magasabb,	mint	\textbf{amennyire}	Péter	\textbf{magas} volt. \label{hunggivenfp}\\
Mary	taller	than	how.much	Peter	tall was.\textsc{3sg}\\
\glt `Mary is taller than Peter was.'}
\ex	[??]{\gll Mari	magasabb,	mint	\textbf{amennyire}	Péter	volt \textbf{magas}. \label{hunggivenfinal}\\
Mary	taller	than	how.much	Peter	was.\textsc{3sg} tall\\
\glt `Mary is taller than Peter was.'}
\z
\z

As shown, \textsc{given} APs are preferably located in the [Spec,\isi{CP}] position, as in (\ref{hunggivencp}); note that this is an unmarked position in the sense that it is not reserved either for \textsc{given} or F-marked elements. Less typically, they can appear clause-finally, but this position would prefer either total de-accenting or secondary focus, hence the slight \isi{markedness} of (\ref{hunggivenfinal}); still, as this particular position is not a \isi{contrast position} either, (\ref{hunggivenfinal}) is possible. However, (\ref{hunggivenfp}) is infelicitous because \textit{magas} `tall' is located in the \isi{preverbal position}, which is the canonical \isi{contrast position} where focussed elements move (cf. \citealt{brody1990, brody1995}, \citealt{ekiss2002}).

On the other hand, the following pattern can be established for F-marked APs:

\ea \label{hungcontrast}
\ea	[?]{\gll A	macska	kövérebb,	mint	amennyire	széles	a macskaajtó	volt. \label{hungcontrastcp}\\
the	cat	fatter	than	how.much	wide	the cat.flap was.\textsc{3sg}\\
\glt `The cat is fatter than the cat flap was wide.'}
\ex	[]{\gll A	macska	kövérebb,	mint	amennyire	a macskaajtó	széles	volt. \label{hungcontrastfp}\\
the	cat	fatter	than	how.much	the cat.flap wide was.\textsc{3sg}\\
\glt `The cat is fatter than the cat flap was wide.'}
\ex	[?]{\gll A	macska	kövérebb,	mint	amennyire	a macskaajtó	volt	széles. \label{hungcontrastfinal}\\
the	cat	fatter	than	how.much	the cat.flap was.\textsc{3sg} wide\\
\glt `The cat is fatter than the cat flap was wide.'}
\z
\z

The most preferred position is the preverbal \isi{contrast position} in (\ref{hungcontrastfp}) and the other two possibilities are less preferred, as shown by (\ref{hungcontrastcp}) and (\ref{hungcontrastfinal}); this is so because the main \isi{contrast} is expressed by the \isi{AP} itself and therefore it should appear in the focus position.  Note that, while there is a canonical \isi{contrast position}, there is no canonical non-\isi{contrast position}, as opposed to \ili{Czech}: hence the asymmetry between the patterns in (\ref{hunggiven}) and (\ref{hungcontrast}), in that an F-marked \isi{AP} is not infelicitous even in the less preferred positions, contrary to what was attested in \ili{Czech}.

Again, it has to be stressed that individual judgements may differ but for the vast majority of my informants (\ref{hungcontrastfp}) was perfectly acceptable while the other two options were both marked, though to different degrees. It seems that once the contrastive \isi{AP} is stranded, then it should appear in the \isi{preverbal position}, as it expresses the main \isi{contrast} involved in the comparison. Since this places a requirement on the \isi{AP} to appear in the \isi{preverbal position} but does not affect the position of the other elements, it should be possible to reverse the positions of the \isi{AP} \textit{széles} `wide' and the \isi{DP} \textit{a macskaajtó} `the cat flap' in (\ref{hungcontrastcp}), which is indeed the case:

\ea \gll A macska kövérebb, mint	\textbf{amennyire}	\textbf{széles}	volt a macskaajtó. \label{hungfatcat3}\\
the	cat	fatter	than	how.much	wide	was.\textsc{3sg}	the	cat.flap\\
\glt `The cat is fatter than the cat flap was wide.'
\z
	
In (\ref{hungfatcat3}), the contrastive \isi{AP} is in the focus position and since the \isi{postverbal position} is available both for \textsc{given} and F-marked elements, the contrastive \isi{DP} \textit{a macskaajtó} can appear there.

The reason why there are altogether three available positions for APs in a \ili{Hungarian} comparative \isi{subclause} is due to the fact that the \isi{QP} undergoes cyclic \isi{movement}: first from within the \isi{VP} to the \isi{edge} of the \isi{FP}, and subsequently from the \isi{FP} to the lower [Spec,\isi{CP}]. The \isi{AP} can be stranded either in its \isi{base position} or in the \isi{FP}, in addition to being able to move up as high as the [Spec,\isi{CP}].

I will not venture to examine the issue of positional differences here, since this is not my primary concern and would go far beyond the scope of the present investigation. What is important now is that though the information-structural properties of the lexical \isi{AP} obviously play a crucial role in the formation of the comparative \isi{subclause}, they do not have a bearing on whether \isi{Comparative Deletion} happens or not. \isi{Comparative Deletion} is a phenomenon linked to a specific syntactic position and is predictable from the formal properties of the \isi{comparative operator}.
