\chapter{Ellipsis without Comparative Deletion} \label{ch:6}
\section{Introduction} \label{sec:6introduction}
The last chapter of this book is devoted to the examination of \isi{ellipsis} phenomena which, alongside the phenomenon of \isi{Comparative Deletion}, are also responsible for the derivation of comparative subclauses. So far, I have been dealing with the elimination of the \isi{degree expression} itself, that is, a \isi{QP} or a \isi{DP} containing a \isi{QP} in the \isi{subclause} (though \chapref{ch:4} also examined the case of VP-\isi{ellipsis} in \ili{English} to a limited extent). The importance of taking other \isi{deletion} phenomena into consideration is that comparative subclauses tend to be highly elliptical, resulting in there being only one overt constituent following the \isi{complementiser}. In languages like \ili{English}, this means that in addition to \isi{Comparative Deletion}, \isi{ellipsis} also takes place. In other languages, such as \ili{Hungarian}, where \isi{Comparative Deletion} is not attested since comparative operators are overt, the question arises how the \isi{degree expression} is covered by ellipsis. As I will show, there is no \isi{movement} to the lower [Spec,\isi{CP}] position in such cases, and hence the well-formedness of the construction can be repaired only via \isi{ellipsis}, which in turn eliminates larger units than the quantified expression itself.

\section{Ellipsis in English} \label{sec:6ellipsis}
\subsection{VP-ellipsis revisited} \label{sec:6vpellipsis}
First of all, let us consider \isi{ellipsis} phenomena in \ili{English} comparatives, which operate in addition to the \isi{Overtness Requirement}, and which are responsible for the formation of typical comparative constructions that tend to overtly involve only contrastive elements in the comparative \isi{subclause}. Though these processes are typically instances of VP-\isi{ellipsis}, I will show that the \isi{ellipsis domain} can also be larger, even though the mechanism of \isi{ellipsis} is essentially the same. 
As was already seen in \chapref{ch:4}, comparative subclauses may involve VP-\isi{ellipsis}, which is an optional \isi{deletion} operation. To gain a fuller picture of its role in the formation of comparatives, let us first have a look at the data from various subtypes of comparatives.

As far as predicative structures are concerned, the following pattern arises:

\ea \label{predcomps}
\ea	The table is longer than the office is wide. \label{predsumcomp}
\ex	Ralph is more enthusiastic than Jason is. \label{ch6predfull}
\ex	Ralph is more enthusiastic than Jason. \label{predellipsis}
\z
\z

The full string is represented in (\ref{predsumcomp}), where the lower \isi{copy} of the \isi{QP} (\textit{wide}) remains overt, since it is contrastive. As there is no contrastive \isi{QP} in the \isi{subclause} in (\ref{ch6predfull}), the lower \isi{copy} is regularly eliminated but there is no VP-\isi{ellipsis}; finally, in (\ref{predellipsis}) the \isi{verb} is eliminated since it is recoverable.

The picture is slightly more complicated in nominal comparatives:

\ea \label{nomcomps}
\ea	Ralph bought more houses than Michael bought flats. \label{nomsubcomp}
\ex	Ralph bought more houses than Michael did flats. \label{ch6nomfull}
\ex	Ralph bought more houses than Michael did. \label{ch6nomgap}
\ex	Ralph bought more houses than Michael. \label{ch6nomellipsis}
\z
\z

As was shown in \chapref{ch:4}, nominal comparatives allow a full structure to appear in the \isi{subclause}, as in (\ref{nomsubcomp}); the \isi{lexical verb} may be eliminated, as in (\ref{ch6nomfull}), resulting in a \isi{gapping} construction. By \isi{contrast}, in (\ref{ch6nomgap}) not only the \isi{lexical verb} but also the \isi{nominal expression} is eliminated, under identity with its matrix \isi{clausal} antecedent; the same is true for (\ref{ch6nomellipsis}), where the auxiliary is also absent and there is only one overt \isi{DP} (\textit{Michael}).

Finally, attributive comparatives show the following distribution:

\ea \label{attrcomps}
\ea	Ralph bought a bigger house than Michael did a flat. \label{attrsubcomp}
\ex	Ralph bought a bigger house than Michael did. \label{attrfull}
\ex	Ralph bought a bigger house than Michael. \label{attrellipsis}
\z
\z

The structure parallel to the one in (\ref{nomsubcomp}) is not allowed, as shown in \chapref{ch:4}, and hence the \isi{lexical verb} must be eliminated, as in (\ref{attrsubcomp}). If the \isi{NP} is \textsc{given}, as in (\ref{attrfull}) and (\ref{attrellipsis}), then it is also deleted: this may result in an overt auxiliary, as in (\ref{attrfull}) or just a single overt \isi{DP} (\textit{Michael}), as in (\ref{attrellipsis}).

Note that I consider all the cases containing a single \isi{DP} following \textit{than} (see (\ref{predellipsis}), (\ref{ch6nomellipsis}) and (\ref{attrellipsis}) above) to be instances of reduced \isi{clausal} comparatives and not of phrasal comparatives. The term ``\isi{phrasal comparative}'' is often used in the literature on comparatives either to mean reduced clauses or genuine phrases (see \citealt{merchant2009}, \citealt{bhatttakahashi2011}) but to avoid confusion, I will refrain from treating reduced \isi{clausal} comparatives as phrasal comparatives. Originally, \citet{hankamer1973} proposed that there are two distinct \textit{than} elements in \ili{English}: a \isi{complementiser} taking clauses and a \isi{preposition} taking nominal expressions as its complement. There are a number of counterarguments to this (see also the discussion below): for instance, as \citet{bhatttakahashi2011} argue, it is expected that there are semantic differences between \isi{clausal} arguments and individual (DP-sized) arguments, but this is not attested for \ili{English}.\footnote{Neither are differences attested in island sensitivity, see \citet[160--161]{merchant2009}.} In particular, the ``Reduction Analysis'' (see, for instance, \citealt{lechner1999diss}, \citealt{lechner2001}, \citealt{lechner2004}; \citealt{merchant2009}) correctly predicts the behaviour of binding patterns, while the ``Direct Analysis'' (see, for instance, \citealt{heim1985}) does not (\citealt[586--590]{bhatttakahashi2011}). Of course, this does not mean that there are no languages with true phrasal comparatives: as \citet[590--613]{bhatttakahashi2011} show, \ili{Hindi-Urdu} and \ili{Japanese} have cases exactly like that, and therefore the category of the \isi{comparative complement} and the properties of the \isi{degree head} are subject to cross-linguistic variation.\footnote{It is very likely that cross-linguistic variation is fined-grained and that even comparatives that are invariably phrasal on the surface may have a more complex structure. As shown convincingly by \citet{merchant2009}, one such instance is that of \ili{Greek} comparatives with the \isi{preposition} \textit{apo} `from', which is always followed by an accusative-marked \isi{DP}. However, these constructions show island sensitivity, which is unexpected if there is no \isi{movement} in the structure, that is, if one would take a surfacist approach and assume that there is merely a \isi{DP} complement. \citet[151--160]{merchant2009} argues that there is either overt or \isi{covert movement}; if one adopts the analysis that the \isi{DP} moves overtly, it lands higher than the \isi{CP} it is base-generated in and ends in the complement of the prepositions \textit{apo}, and is assigned \isi{accusative case} (see \citealt[151--156]{merchant2009} for details). In \citet{bacskaiatkari2017icgl}, I adopted this view for \ili{Greek} comparatives with \textit{apo} and showed that this approach correctly predicts that \textit{apo}-comparatives pattern with \isi{clausal} comparatives and not with ordinary phrasal comparatives in certain ambiguous constructions. I cannot discuss the issue of \isi{clausal} and phrasal comparatives here in more detail, but it should be obvious that a \isi{clausal} analysis for reduced comparative clauses is altogether favourable for \ili{English} and also for similar constructions from other languages to be dealt with in this chapter.}

I claim that, in all of the elliptical cases in (\ref{predcomps})--(\ref{attrcomps}) above, the mechanism responsible for eliding the \isi{verb} is ordinary \isi{ellipsis} (either VP-\isi{ellipsis} or TP-\isi{ellipsis}). In other words, there is no specific \isi{ellipsis} mechanism applying in comparative subclauses, that is, there is no need for a separate process such as Comparative Ellipsis (see also the analyses of \citealt{kennedy2002}, \citealt{lechner2004}, \citealt{merchant2009}). Based on \citet{merchant2001}, I assume that \isi{ellipsis} is carried out via an [E] \isi{feature} that is present on a \isi{functional head}, which instructs \isi{PF} to eliminate its complement, that is, the head itself remains intact. Moreover, based on the discussion given in \chapref{ch:4}, I also assume that the presence of an F-marked constituent may stop the \isi{linear deletion} process; note that this is prosodically licensed if the constituent is also aligned to the \isi{right edge} of an \isi{Intonational Phrase} (cf. \citealt{szendroi2001diss}, based on \citealt{selkirk1984, selkirk1986}, \citealt{nesporvogel1986}, \citealt{chen1987}, \citealt{inkelas1989diss}, \citealt{mccarthyprince1993}, \citealt{neelemanweerman1999}, \citealt{truckenbrodt1999} among others).

In the case of VP-\isi{ellipsis}, the [E] \isi{feature} can be located in a v node above the \isi{VP} (or any thematic \isi{vP}). This is schematically represented in (\ref{treevp6}):

\ea \upshape \label{treevp6}
\begin{forest} baseline, qtree, for tree={align=center}
[vP
	[v$'$
		[v
			[{[}E{]}]
		]
		[VP
			[V$'$ [V] [XP]]
		]
	]
]
\end{forest}
\z

As can be seen, the [E] \isi{feature} is located in the v head above the \isi{VP}, and \isi{deletion} at \isi{PF} affects the complement of v, that is, the \isi{VP} (the domain of \isi{ellipsis}), as shown in (\ref{treevpellipsis6}):

\ea \upshape \label{treevpellipsis6}
\begin{forest} baseline, qtree, for tree={align=center}
[vP
	[v$'$
		[v
			[{[}E{]}]
		]
		[VP, name=vp
			[V$'$ [V] [XP]]
		]
	]
]
\path [draw, thick] (vp.north west) -- (vp.south east);
\path [draw, thick] (vp.north east) -- (vp.south west);
\end{forest}
\z

This leaves the v head itself intact and deletes the following \isi{VP}, including the V head and the XP in the complement position of the V head. However, if the XP (or a constituent thereof) is contrastive, then it can withstand \isi{deletion}, as demonstrated in (\ref{withstand6}):

\ea \label{withstand6}
\ea	v\textsubscript{[E]} \sout{[\textsubscript{VP} V [XP]]}
\ex v\textsubscript{[E]} \sout{[\textsubscript{VP} V }[XP]\textsubscript{F}]
\z
\z
	
Note that the presence of the [E] \isi{feature} is optional in itself; on the other hand, if the XP is a non-contrastive lower \isi{copy}, it is regularly deleted as a lower \isi{copy}, independently of the [E] \isi{feature} and hence VP-\isi{ellipsis}.

Let us turn to the cases outlined at the beginning of this section. In predicative structures like (\ref{predcomps}), the v node is headed by the \isi{copula} (\textit{is}) and the XP is a \isi{QP}:

\ea \upshape \label{predicativetree}
\begin{forest} baseline, qtree, for tree={align=center}
[vP
	[v$'$
		[v
			[is\\{[}E{]}]
		]
		[VP, name=vp
			[V$'$ [V] [QP [{[}x-enthusiastic{]}\\{[}x-wide{]}\textsubscript{{[}F{]}},roof]]]
		]
	]
]
\path [draw, thick] (vp.north west) -- (vp.south east);
\path [draw, thick] (vp.north east) -- (vp.south west);
\end{forest}
\z

Obviously, if the [E] \isi{feature} is not present on the v node, then the \isi{QP} \textit{x-enthusiastic} is deleted regularly as a lower \isi{copy}, resulting in the structure in (\ref{ch6predfull}), while the \isi{QP} \textit{x-wide} is realised overtly since it is F-marked, as in (\ref{predsumcomp}). The representation in (\ref{predicativetree}) shows the case when the [E] \isi{feature} is present on the v head: the \isi{ellipsis domain} is the \isi{VP} and the v head itself (\textit{is}) remains intact. This means that a non-contrastive \isi{QP} such as \textit{x-enthusiastic} is deleted, resulting in a configuration like (\ref{ch6predfull}), but note that the same configuration emerges even without the presence of the [E] \isi{feature}. On the other hand, a contrastive \isi{QP} such as \textit{x-wide} is not deleted, resulting in a configuration like (\ref{predsumcomp}), which is again the same output that emerges without the presence of the [E] \isi{feature}.

At this point, it may seem that the presence of the [E] \isi{feature} does not make any difference as far as the final structure is concerned; furthermore, it does not seem to matter either whether the \isi{verb} itself is \textsc{given} or not, since the [E] \isi{feature} does not delete the \isi{verb}. The importance of the [E] \isi{feature} will become clearer when considering attributive and nominative structures; before turning to them, however, let me briefly discuss one issue related to predicative structures that is crucial in understanding the importance of where the [E] \isi{feature} is located.

Let us suppose that the [E] \isi{feature} can be located on a node immediately dominating the \isi{vP} headed by \textit{is}, that is, the I/T head. This would mean that the I/T head is not affected by \isi{deletion} but its complement is, and the \isi{ellipsis domain} would be the entire \isi{vP} and not just the \isi{VP}, as in (\ref{predicativetree}). Following the analysis given above, this would mean that if the \isi{QP} is \textsc{given}, the entire \isi{vP} is deleted, while a contrastive \isi{QP} is overt (but the v head is not). In other words, this would predict that both examples in (\ref{et6}) should be grammatical, which is not the case:

\ea \label{et6}
\ea	[]{Ralph is more enthusiastic than Jason \sout{is enthusiastic}.} \label{etpred}
\ex	[*]{The table is longer than the office \sout{is} wide.} \label{etsub}
\z
\z

The ungrammaticality of (\ref{etsub}) shows that the [E] \isi{feature} cannot be located above the \isi{vP} node containing \textit{is}. On the other hand, the grammaticality of (\ref{etpred}) raises the question how the auxiliary can still be eliminated. 

One possibility has to do with phase theory. Note that the \isi{vP} in question is a phase boundary and hence in structures like (\ref{etpred}) the entire vP-phase is eliminated at \isi{PF}. This is perfectly possible because it affects only recoverable material, while this option would render an invalid construction in the case of (\ref{etsub}). I adopt the view that phases can be split: that is, material is transferred only to \isi{PF} or \isi{LF} (\citealt[129--130]{marusic2005}, based on \citealt{felser2004}, \citealt{bobaljikwurmbrand2005}; cf. also \citealt{marusiczaucer2006}). In the particular case, this means that the \isi{vP} headed by \textit{is} in (\ref{etpred}) is not transferred to \isi{PF} but only to LF; therefore its absence is actually due to the lack of PF-transfer and not to PF-\isi{deletion}.

The other possibility has to do with the \isi{FP} \citet{merchant2001} assumes for \isi{ellipsis} structures: the head of this functional projection hosts the \isi{feature} [E] and the remnant is moved to the specifier of the \isi{FP} and thus escapes \isi{deletion}. One advantage of assuming the availability of the \isi{FP} is that this projection is not tied to the notion of \isi{tense} and can appear in tenseless clauses as well. Consider the examples in (\ref{thanime6}) showing pronominal remnants:

\ea \label{thanime6}
\ea [?]{Ralph is taller than \textbf{I}.} \label{thani}
\ex []{Ralph is taller than \textbf{me}.} \label{thanme}
\z
\z

The remnant is preferably in the \isi{accusative case}, possibly also to phonological reasons; note also that in \ili{English}, the default case is the accusative (see \citealt{schuetze2001}). At the same time, as was also pointed out by \citet[618]{bhatttakahashi2011}, the nominative remnant is not excluded either, indicating that \textit{than} is not a \isi{preposition} assigning \isi{accusative case} to the \isi{pronoun} (contrary to \citealt{hankamer1973}). The appearance of the \isi{accusative case} on the remnant is rather due to the absence of the \isi{TP} projection in the \isi{subclause}, yet the gradable argument cannot be realised even if it is contrastive. Thus, some kind of \isi{ellipsis} still applies, even though it is not TP-\isi{ellipsis} (see \citealt{bacskaiatkari2014alh} for an analysis).

On the other hand, the \isi{FP} analysis cannot handle \isi{Attributive Comparative Deletion} without assuming \isi{rightward movement} of the remnant, since \isi{gapping} and pseudogapping constructions are different from \isi{sluicing} in that the second remnant cannot be located at the \isi{left periphery}. Since this chapter is primarily devoted to \isi{gapping} constructions, I will continue to use an FP-less analysis. However, note that I do not wish to theoretically exclude the availability of the \isi{FP}. On the contrary, I would like to stress that in cases where there is only a single remnant, the FP-style \isi{sluicing} analysis is possible and \ili{English} very probably has several ways of deriving the same surface structure.

Related to this, it is worth mentioning that the present chapter will deal with the derivation of elliptical tensed clauses, that is, the type given in (\ref{thani}) but not that in (\ref{thanme}). The availability of (\ref{thani}) shows that full, tensed comparative clauses can be reduced to a single remnant in \ili{English} (apart from examples where the remnant XP is not even a \isi{DP} but an \isi{AP} or a \isi{PP}), and I would like to show that these structures can be derived without either requiring a designated ``Comparative Ellipsis'' process or by assuming that comparative subclauses are coordinated at some point in the derivation, as is done by \citet{lechner1999diss, lechner2004}. If the \isi{DP} remnant is not pronominal in \ili{English}, it is case ambiguous, and could potentially be derived on a par with (\ref{thanme}), see the details given in \citet{bacskaiatkari2014alh}. For reasons of space, I cannot discuss these constructions here; the point here is to show how \isi{clausal} \isi{ellipsis} in comparatives works.

Let us now turn to nominal comparatives, as given in (\ref{nomcomps}). In these structures, the v node either is zero or is headed by the auxiliary \textit{do}; the XP is a \isi{DP} functioning as the object. The case with no overt auxiliary is illustrated in (\ref{treenoaux6}):

\ea \upshape \label{treenoaux6}
\begin{forest} baseline, qtree, for tree={align=center}
[vP
	[v$'$
		[v
			[$\emptyset$\\{[}E{]}]
		]
		[VP, name=vp
			[V$'$ [V [bought]] [DP [{[}x-houses{]}\\{[}x-flats{]}\textsubscript{{[}F{]}},roof]]]
		]
	]
]
\path [draw, thick] (vp.north west) -- (vp.south east);
\path [draw, thick] (vp.north east) -- (vp.south west);
\end{forest}
\z

If there is an [E] \isi{feature}, then it is on the v head. If there is no [E] \isi{feature}, the \isi{VP} is not elided; however, since the \isi{DP} is actually a lower \isi{copy}, it is deleted regularly as a lower \isi{copy} (and hence \textit{houses} is not overt), while it remains overt if it is contrastive (as in the case of \textit{flats}). The picture is slightly different if the [E] \isi{feature} is present, since in that case the \isi{VP} is the \isi{ellipsis domain}: if the \isi{DP} is not contrastive, this eliminates the entire \isi{VP} but if the \isi{DP} is contrastive, it stays overt.

Naturally, it is also possible that the v head is filled by the auxiliary \textit{do}:

\ea \label{treedid} \upshape 
\begin{forest} baseline, qtree, for tree={align=center}
[vP
	[v$'$
		[v
			[did\\{[}E{]}]
		]
		[VP, name=vp
			[V$'$ [V [buy]] [DP [{[}x-houses{]}\\{[}x-flats{]}\textsubscript{{[}F{]}},roof]]]
		]
	]
]
\path [draw, thick] (vp.north west) -- (vp.south east);
\path [draw, thick] (vp.north east) -- (vp.south west);
\end{forest}
\z

The insertion of the auxiliary is motivated because in the absence of the \isi{lexical verb} the \isi{tense} morpheme could not be spelt out: in other words, the \isi{dummy auxiliary} appears when there is an [E] \isi{feature} on the v head but not otherwise, since the overt co-occurrence of \textit{did} and \textit{buy} in structures like (\ref{treedid}) would violate general rules of economy. The \isi{ellipsis domain} is the \isi{VP}, and only contrastive elements, such as the \isi{DP} \textit{flats}, can withstand \isi{linear deletion}, as in (\ref{ch6nomfull}); otherwise the entire \isi{VP} is eliminated at \isi{PF}, as in (\ref{ch6nomgap}).

Finally, let us consider what happens in attributive comparatives. The crucial difference from nominal comparatives lies in the presence of a functional \isi{FP} layer above the \isi{NumP} (see \chapref{ch:4}), the specifier of which hosts the \isi{QP}. Again, the v head may or may not be filled by the \isi{dummy auxiliary}. In (\ref{treeattrlex}) below, there is no overt v head:

\ea \label{treeattrlex} \upshape 
\begin{forest} baseline, qtree, for tree={align=center}
[vP
	[v$'$
		[v
			[$\emptyset$\\{[}E{]}]
		]
		[VP, name=vp
			[V$'$ [V [bought]] [FP [QP [x-big,roof]] [F$'$ [F [$\emptyset$]] [NumP [{[}a house{]}\\{[}a flat{]}\textsubscript{{[}F{]}},roof]]]]]
		]
	]
]
\path [draw, thick] (vp.north west) -- (vp.south east);
\path [draw, thick] (vp.north east) -- (vp.south west);
\end{forest}
\z

If the entire \isi{FP} is \textsc{given}, then it can be deleted regularly as a lower \isi{copy}. In these cases, if there is no [E] \isi{feature} on the v head, the lexical V may remain overt in the structure. Otherwise, as was argued for in \chapref{ch:4} in detail, the presence of the [E] \isi{feature} on the v head is necessary, since the \isi{QP} must be eliminated. Recall also that this requirement is not sensitive to whether the lexical \isi{AP} is contrastive or not but stems from an \isi{Overtness Requirement} on operators moving to a left-peripheral position, and therefore a contrastive \isi{AP} would render an ungrammatical configuration.

\largerpage[-1]
It follows that the presence of the [E] \isi{feature} on the v head causes the elimination of the lexical V and also of the \isi{QP}, and only a contrastive \isi{NumP} (such as \textit{a flat}) may remain overt. If the \isi{NumP} is not contrastive, the entire \isi{FP} can be eliminated as a lower \isi{copy}.

The situation is essentially the same if the v node contains the \isi{dummy auxiliary}, as in (\ref{treedummy6}):

\ea \upshape \label{treedummy6}
\begin{forest} baseline, qtree, for tree={align=center}
[vP
	[v$'$
		[v
			[did\\{[}E{]}]
		]
		[VP, name=vp
			[V$'$ [V [buy]] [FP [QP [x-big,roof]] [F$'$ [F [$\emptyset$]] [NumP [{[}a house{]}\\{[}a flat{]}\textsubscript{{[}F{]}},roof]]]]]
		]
	]
]
\path [draw, thick] (vp.north west) -- (vp.south east);
\path [draw, thick] (vp.north east) -- (vp.south west);
\end{forest}
\z

Again, the \isi{lexical verb} and the \isi{QP} are deleted, and the \isi{dummy auxiliary} \textit{did} remains overt; the \isi{NumP} is elided if it is not contrastive, as in (\ref{attrfull}), and withstands \isi{linear deletion} if it is contrastive, as in (\ref{attrsubcomp}).

\subsection{Different domains of ellipsis and syntactic ambiguity} \label{sec:6differentdomains}
The analysis presented so far provides a unified account for \isi{ellipsis} phenomena in \ili{English} comparatives. Elliptical clauses tend to overtly contain only a single contrasted constituent. In the examples considered so far this constituent was invariably the subject, but this is not necessarily the case, as shown by (\ref{moregirls}) below:

\ea	More girls ate sandwiches than hamburgers. \label{moregirls}
\z

In this case, the remaining \isi{DP} constituent in the \isi{subclause} is \textit{hamburgers}, which is an object. The derivation of the \isi{subclause} is outlined in (\ref{moregirlsderivation}):

\ea	{}[\textsubscript{CP} than [\textsubscript{CP} \sout{[\textsubscript{DP} x-many girls]} \sout{[\textsubscript{IP} [\textsubscript{DP} x-many girls] [\textsubscript{VP} ate} [\textsubscript{DP} hamburgers]\textsubscript{F}]]]] \label{moregirlsderivation}
\z

As can be seen, the highest \isi{copy} of the quantified \isi{DP} (\textit{x-many girls}) in the [Spec,\isi{CP}] position is eliminated due to the \isi{Overtness Requirement}. In addition, the rest of the clause is elided except for the object \isi{DP} (\textit{hamburgers}), which is possible because the lower copies of the quantified \isi{DP} and the \isi{lexical verb} are not contrastive.

Since the lower \isi{copy} of the subject \isi{DP} is regularly eliminated, and VP-\isi{ellipsis} can take place independently, there is no reason to suppose that the two processes are connected. However, this is not necessarily so in cases when the subject and the quantified expression are independent. Consider:

\ea	Mary drank ale more often than sherry.
\z

The derivation is shown in (\ref{derivationdrink6}):

\ea	{}[\textsubscript{CP} than [\textsubscript{CP} \sout{[\textsubscript{QP} x-often]} \sout{[\textsubscript{IP} [\textsubscript{DP} Mary] [\textsubscript{VP} drank} [\textsubscript{DP} sherry]\textsubscript{F}] \sout{[\textsubscript{QP} x-often]]]]} \label{derivationdrink6}
\z

In this case, the lower \isi{copy} of the quantified \isi{adverb} (\textit{x-often}) is elided in its \isi{base position}; the subject \isi{DP} \textit{Mary} and the \isi{lexical verb} are deleted together. This is possible if the [E] \isi{feature} is located on the C head and \isi{ellipsis} in this case naturally affects both the subject and the \isi{verb}, neither of which are F-marked.

As shown by \citet{merchant2001}, \isi{ellipsis} is carried out via an [E] \isi{feature} that is present on a \isi{functional head}: in \isi{sluicing}, for instance, this \isi{functional head} is a C in \ili{English}. In cases like (\ref{moregirlsderivation}), then, if the [E] \isi{feature} is located on the lower C head, \isi{ellipsis} affects the non-contrastive \isi{lexical verb}. This option is preferable to locating the [E] \isi{feature} on the v head because it is preferable to elide the maximal unit (cf. \citealt{merchant2008}). Similarly to the case of VP-\isi{ellipsis}, an F-marked constituent blocks the \isi{linear deletion} process and hence the \isi{DP} \textit{hamburgers} remains overt.

The availability of [E] on both C and v is responsible for certain structural ambiguities. Consider the examples in (\ref{ambiguity6}):

\ea \label{ambiguity6}
\ea	I love you more than Peter. \label{ambigpeter}
\ex	I'm a linguist. I like ambiguity more than most people.
\z
\z
	
In both cases, the \isi{DP} following \textit{than} can be interpreted either as the subject or the object, as there is no overt case distinction. Taking the example in (\ref{ambigpeter}), the \isi{ellipsis domain} for a structure containing \textit{Peter} as the subject is shown below:

\ea \upshape \label{treeambigpetersubject}
\begin{forest} baseline, qtree, for tree={align=center}
[CP
	[C$'$
		[C
			[than]
		]
		[CP
			[Op.\textsubscript{i}]
			[C$'$ [C] [IP [DP\textsubscript{j} [Peter,roof]] [I$'$ [I] [vP [v$'$ [v [$\emptyset$\\{[}E{]}]] [vP, name=vp [t\textsubscript{j} loves you t\textsubscript{i},roof]]]]]]]]
		]
	]
]
\path [draw, thick] (vp.north west) -- (vp.south east);
\path [draw, thick] (vp.north east) -- (vp.south west);
\end{forest}
\z

As can be seen, in this case the contrastive element is the subject, and the [E] \isi{feature} can only be located on a v head and not on C; since there is no contrastive element in the \isi{ellipsis domain}, the entire \isi{vP} is eliminated.

By \isi{contrast}, if \textit{Peter} in (\ref{ambigpeter}) is an object, the the \isi{ellipsis domain} is the complement of a C head equipped with the [E] \isi{feature}, as illustrated in (\ref{treeellipsiscp6}):

\ea \upshape \label{treeellipsiscp6}
\begin{forest} baseline, qtree, for tree={align=center}
[CP
	[C$'$
		[C
			[than]
		]
		[CP
			[Op.\textsubscript{i}]
			[C$'$ [C [$\emptyset$\\{[}E{]}]] [IP, name=vp [DP\textsubscript{j} [\phantom{i}I\phantom{i},roof]] [I$'$ [I] [vP [v$'$ [v] [vP [t\textsubscript{j} love {[}Peter{]}\textsubscript{F} t\textsubscript{i},roof]]]]]]]]
		]
	]
]
\path [draw, thick] (vp.north west) -- (vp.south east);
\path [draw, thick] (vp.north east) -- (vp.south west);
\end{forest}
\z

Ellipsis hence affects the entire IP and stops only at the F-marked \isi{DP} \textit{Peter}. If the [E] \isi{feature} were located on a v head (as in (\ref{treeambigpetersubject})), then the subject would also remain overt and (\ref{illicitgapping}) should be grammatical, which is not the case:

\ea	*I love you more than I Peter. \label{illicitgapping}
\z

The fact that the \isi{feature} [E] is located as high as possible in the structure is essentially in line with economy requirements. The elimination of a larger unit is thus the result of a single process but the possibility of [E] appearing on a lower \isi{functional head} is not excluded either.

Thus, ambiguity may result from there being two possible underlying structures, with respect to the position of a remnant \isi{DP} in the \isi{subclause}. On the other hand, ambiguity may also be the result of which projection is responsible for withstanding \isi{deletion}. Consider the following example:\footnote{I am indebted to Jenneke van der Wal for calling my attention to this particular example.}

\ea More people die each year from falling coconuts than sharks. \label{ambigcoconuts}
\z

The sentence in (\ref{ambigcoconuts}) is ambiguous and the three possible readings are paraphrased with fuller structures in (\ref{disambigcoconuts}) below:

\ea \label{disambigcoconuts}
\ea	More people die each year from falling coconuts than sharks do. \label{disambigcoconutssharksdo}
\ex	More people die each year from falling coconuts than from sharks. \label{disambigcoconutsfromsharks}
\ex	More people die each year from falling coconuts than from falling sharks. \label{disambigcoconutsfallingsharks}
\z
\z

The most plausible meaning is the one given in (\ref{disambigcoconutsfromsharks}) but the other two meanings are also available and congruent. To derive the sentence in (\ref{ambigcoconuts}) with the meaning of (\ref{disambigcoconutssharksdo}), the [E] \isi{feature} has to be present on the v head and the entire \isi{VP} is elided; the subject \isi{DP} \textit{sharks} is left intact because it falls outside of the \isi{ellipsis domain}. Note that the \isi{DP} \textit{x-many sharks} is the quantified expression in the \isi{subclause} and hence it is moved from the \isi{VP} first into the [Spec,IP] position as the subject and then further to the [Spec,\isi{CP}] position as the quantified expression. The \isi{copy} in the [Spec,\isi{CP}] position is eliminated due to the \isi{Overtness Requirement} and the lowest \isi{copy} would have to be regularly deleted anyway. For the sake of convenience, I do not include these copies in the representation in (\ref{vpellipsiscoconuts6}), which shows exclusively the effect of VP-\isi{ellipsis}:

\ea	{}[\textsubscript{CP} than [\textsubscript{IP} [\textsubscript{DP} x-many sharks]\textsubscript{F} [\textsubscript{vP} \sout{[\textsubscript{VP} die [\textsubscript{PP} from falling coconuts]]]]]} \label{vpellipsiscoconuts6}
\z

Naturally, the [E] \isi{feature} cannot be located on the C head since then the contrastive \isi{DP} (\textit{sharks}) would stop the \isi{deletion} process from applying further. 

By \isi{contrast}, in order to derive the readings in (\ref{disambigcoconutsfromsharks}) and (\ref{disambigcoconutsfallingsharks}), the [E] \isi{feature} is present on the C head and it stops only at the clause-final contrastive element, which is either a full \isi{DP} or a part thereof. In the case of the reading given in (\ref{disambigcoconutsfromsharks}), ellipsis is as illustrated in (\ref{sluicingcoconuts6}):

\ea	{}[\textsubscript{CP} than \sout{[\textsubscript{IP} [\textsubscript{DP} x-many people] [\textsubscript{vP} [\textsubscript{VP} die [\textsubscript{PP} from }[\textsubscript{DP} sharks]\textsubscript{F}]]]]] \label{sluicingcoconuts6}
\z

By \isi{contrast}, for a reading such as the one in (\ref{disambigcoconutsfallingsharks}), there has to be an \isi{AP} present within the \isi{DP} as well, which is also elided, as shown in (\ref{coconutsfallingsharks6}):

\ea	{}[\textsubscript{CP} than \sout{[\textsubscript{IP} [\textsubscript{DP} x-many people] [\textsubscript{vP} [\textsubscript{VP} die [\textsubscript{PP} from [\textsubscript{DP} [\textsubscript{AP} falling] }[\textsubscript{NP} sharks]\textsubscript{F}]]]]] \label{coconutsfallingsharks6}
\z

Note also that the \isi{PP} may be able to withstand \isi{deletion} too, as in (\ref{disambigcoconutsfromsharks}), which is then derived as given in (\ref{ellipsisppcoconuts6}):

\ea	{}[\textsubscript{CP} than \sout{[\textsubscript{IP} [\textsubscript{DP} x-many people] [\textsubscript{vP} [\textsubscript{VP} die }[\textsubscript{PP} from [\textsubscript{DP} sharks]]\textsubscript{F}]]]] \label{ellipsisppcoconuts6}
\z

This is possible because F-marking may affect either the entire \isi{PP} or the \isi{DP}: just as an F-marked lexical element (the \isi{noun} \textit{sharks}) can project this property up to the \isi{DP} level, the same may be projected up to the \isi{PP}.

Naturally, the \isi{ellipsis} processes described above could be examined in other constructions as well; however, since the aim of the present investigation is not to provide a unified account for \isi{ellipsis} but to investigate the structure of comparative constructions, I will leave such questions open here. The advantage of the analysis presented in this section lies in the fact that it provides a unified framework for the various outputs, which are hence the results of otherwise optional processes and general requirements on \textsc{given}/contrastive lower copies. While the presence of an [E] \isi{feature} on a \isi{functional head} is in itself optional, once this option is taken, the way \isi{ellipsis} applies is predictable. In other words, all \isi{deletion} rules applying in \ili{English} comparatives can be reduced to general principles. Again, note that the analysis presented here applies only to tensed clauses; see \citet{bacskaiatkari2014alh} on tenseless clauses in \ili{English}.

\section{Ellipsis in Hungarian} \label{sec:6ellipsisinh}
\subsection{Sluicing and VP-ellipsis} \label{sec:6sluicing}
The question arises whether and to what extent the analysis given in \sectref{sec:6ellipsis} can be applied to other languages, such as \ili{Hungarian}; in \ili{English}, the higher \isi{copy} of the quantified expression is regularly eliminated in the [Spec,\isi{CP}] position due to the \isi{Overtness Requirement} but this is not so in languages that have overt comparative operators. Yet, the final linear structure of comparative subclauses in \ili{Hungarian} tends to be strikingly similar to their \ili{English} counterparts: that is, only contrastive elements are preserved and the quantified expression is not visible either.

First of all, let us consider examples containing a \textsc{given} \isi{verb}: in all of these cases there is a synonymous pair of sentences where one contains a full \isi{subclause} and the other shows the result of \isi{ellipsis}. I will argue that in these elliptical examples the [E] \isi{feature} is located on a \isi{functional head} right above the \isi{TP}/IP and lower than the \isi{CP}, and the effects are essentially similar to \isi{sluicing} and VP-\isi{ellipsis} in \ili{English}.

Consider the following examples for predicative structures:

\ea \label{ch6hungpred}
\ea	\gll Mari	magasabb	volt,	mint	\textbf{amilyen}	\textbf{magas}	Péter \textbf{volt}. \label{ch6hungpredfull}\\
Mary	taller	was.\textsc{3sg}	than	how	tall	Peter was.\textsc{3sg}\\
\glt `Mary was taller than Peter.'
\ex \gll	Mari	magasabb	volt,	mint	Péter. \label{hungpredellipsis}\\
Mary	taller	was.\textsc{3sg}	than	Peter\\
\glt `Mary was taller than Peter.'
\z
\z
		
The sentence in (\ref{ch6hungpredfull}) represents the full structure of a predicative comparative \isi{subclause}, that is, the \isi{subclause} where no \isi{ellipsis} has taken place. By \isi{contrast}, the one in (\ref{hungpredellipsis}) is the result of \isi{ellipsis}, since only a contrastive \isi{DP} (\textit{Péter}) remains overt and both the \isi{finite verb} (\textit{volt}) and the quantified expression (\textit{amilyen magas}) are elided. As far as their semantics is concerned, the two sentences are equivalent. The question that arises is how the quantified expression is deleted since \isi{Comparative Deletion} is not applicable (the \isi{operator} being visible); furthermore, (\ref{ch6hungpredfull}) suggests that the quantified expression and the \isi{finite verb} are not even adjacent.

Before attempting to provide an answer to this, let us see some examples for nominal comparatives:

\ea \label{hungnom}
\ea \gll Mari	több macskát vett, mint	\textbf{ahány} \textbf{macskát} Péter \textbf{vett}. \label{hungnomfull}\\
Mary more	cat.\textsc{acc} bought.\textsc{3sg} than	how.many cat.\textsc{acc} Peter bought.\textsc{3sg}\\
\glt `Mary bought more cats than Peter did.'
\ex \gll Mari	több macskát vett, mint	Péter. \label{hungnomellipsis}\\
Mary more	cat.\textsc{acc} bought.\textsc{3sg} than Peter\\
\glt `Mary bought more cats than Peter did.'
\z
\z
		
The (more) complete string is given in (\ref{hungnomfull}) and the one in (\ref{hungnomellipsis}) is the result of \isi{ellipsis} affecting the quantified \isi{DP} (\textit{ahány macskát}) and the \isi{lexical verb} (\textit{vett}). The picture is similar in the case of attributive structures:

\ea \label{hungattr}
\ea	\gll Mari nagyobb macskát vett, mint \textbf{amilyen} \textbf{nagy} \textbf{macskát} Péter \textbf{vett}. \label{hungattrfull}\\
Mary bigger	cat.\textsc{acc} bought.\textsc{3sg} than how big cat.\textsc{acc} Peter bought.\textsc{3sg}\\
\glt `Mary bought a bigger cat than Peter did.'
\ex \gll Mari nagyobb	macskát	vett,	mint Péter. \label{hungattrellipsis}\\
Mary bigger	cat.\textsc{acc} bought.\textsc{3sg} than	Peter\\
\glt `Mary bought a bigger cat than Peter did.'
\z
\z

Again, (\ref{hungattrfull}) shows the complete string containing the \isi{finite verb} and the quantified \isi{DP} (\textit{amilyen nagy macskát}), while in (\ref{hungattrellipsis}) these elements have been elided from the \isi{subordinate clause}. Both in (\ref{hungnom}) and in (\ref{hungattr}), it seems that the quantified expression and the \isi{finite verb} are not adjacent. Therefore, the question posed in connection with (\ref{ch6hungpred}) remains: that is, how both of these elements can be elided if the elements are not adjacent. There are two basic possibilities. First, there might be two different processes involved (even though the elimination of the highest \isi{copy} of the quantified expression cannot be the result of \isi{Comparative Deletion}). Second, there may be a single process that is able to affect both elements that are adjacent at some point. In what follows, I will argue for the latter.

That there is indeed a correlation between the \isi{deletion} of the quantified expression and the \isi{finite verb} is shown by the phenomenon descriptively termed Comparative Verb Gapping by \citet{bacskaiatkarikantor2012}. This is the observation ``that if the \isi{operator} is deleted, the \isi{finite verb} must also be deleted'' (\citealt[49]{bacskaiatkarikantor2012}). In other words, while examples (\ref{ch6hungpred})–(\ref{hungattr}) clearly show that structures containing both the quantified expression and the \isi{finite verb} are grammatical and so are ones where both of these elements are elided, the absence of an overt quantified expression seems to require the \isi{deletion} of the \isi{finite verb}. This is demonstrated by the ungrammaticality of the following sentences (see also the examples given in \citealt[54--56]{bacskaiatkarikantor2012}):

\ea \label{hungungrammaticalnoop}
\ea	[*]{\gll Mari	magasabb	volt,	mint	Péter \textbf{volt}. \label{hungungrammaticalnooppred}\\
Mary	taller	was.\textsc{3sg}	than	Peter was.\textsc{3sg}\\
\glt `Mary was taller than Peter.'}
\ex [*]{\gll Mari	több macskát vett, mint	Péter \textbf{vett}.\\
Mary more	cat.\textsc{acc} bought.\textsc{3sg} than	Peter bought.\textsc{3sg}\\
\glt `Mary bought more cats than Peter did.'}
\ex	[*]{\gll Mari nagyobb macskát vett, mint Péter \textbf{vett}.\\
Mary bigger	cat.\textsc{acc} bought.\textsc{3sg} than Peter bought.\textsc{3sg}\\
\glt `Mary bought a bigger cat than Peter did.'}
\z
\z

The ungrammaticality of the examples in (\ref{hungungrammaticalnoop}) shows that the \isi{deletion} of the quantified expression should affect the \textsc{given} \isi{finite verb} as well.

The core argument of \citet[56--59]{bacskaiatkarikantor2012} is that when there is no overt quantified expression in a \ili{Hungarian} comparative \isi{subclause}, it is so because the \isi{operator} failed to undergo \isi{movement} to the [Spec,\isi{CP}] position before \isi{spell-out} to \isi{PF}. However, it is ungrammatical to have a phrase containing a \isi{relative operator} in its \isi{base position}; more precisely, there is an unchecked [+rel] \isi{feature} on the \isi{operator}, and the construction can be saved only by \isi{deletion} (\citealt[58]{bacskaiatkarikantor2012}).

Interestingly, the phenomenon is not restricted to comparative subclauses but can be found in certain relative clauses as well; consider the following set of examples (based on \citealt[59, ex. 32]{bacskaiatkarikantor2012}):

\ea \label{hunrel}
\ea []{\gll Ugyanazt	a	könyvet	olvasom,	mint \textbf{amit}	Péter	\textbf{olvas}. \label{hunrelfull}\\
that.same.\textsc{acc}	the	book.\textsc{acc}	read.\textsc{1sg} as what.\textsc{acc} Peter reads\\
\glt `I am reading the same book that Peter is reading.'}
\ex	[]{\gll Ugyanazt	a	könyvet	olvasom,	mint \textbf{amit}	Péter. \label{hunrelop}\\
that.same.\textsc{acc}	the	book.\textsc{acc}	read.\textsc{1sg} as what.\textsc{acc} Peter\\		
\glt `I am reading the same book that Peter is reading.'}
\ex	[*]{\gll Ugyanazt	a	könyvet	olvasom,	mint	Péter \textbf{olvas}. \label{hunrelverb}\\
that.same.\textsc{acc}	the	book.\textsc{acc}	read.\textsc{1sg} as  Peter reads\\
\glt `I am reading the same book that Peter is reading.'}
\ex	[]{\gll Ugyanazt	a	könyvet	olvasom,	mint	Péter. \label{hunrelellipsis}\\
that.same.\textsc{acc}	the	book.\textsc{acc}	read.\textsc{1sg} as Peter\\
\glt `I am reading the same book that Peter is reading.'}
\z
\z

The relative clauses in (\ref{hunrel}) differ from ordinary relative clauses in that they also contain the \isi{complementiser} \textit{mint} `as'. However, the structure is not comparative: only non-\isi{degree} equation is involved in the sense that a given entity is identified with another one, but note that there are no \isi{degree} expressions either in the \isi{matrix clause} or in the \isi{subordinate clause}. The point is that since there is an overt \isi{complementiser} at the \isi{left periphery}, the \isi{relative operator} \textit{amit} `what' may be deleted, which would not be possible otherwise. Since \ili{Hungarian} lacks zero relative operators, the absence of an overt \isi{relative operator} from a \isi{relative clause} can only be the result of \isi{deletion}.

The full version is given in (\ref{hunrelfull}), containing both the \isi{operator} and the \isi{finite verb} (\textit{olvas}); note that the \isi{verb} can be elided even if the \isi{operator} is overt, as shown by (\ref{hunrelop}) and the same would be true for comparatives as well (cf. \citealt[59]{bacskaiatkarikantor2012}). The ungrammatical configuration in (\ref{hunrelverb}) lacks an \isi{overt operator} but the \isi{finite verb} is present; finally, the construction in which both the \isi{operator} and the \isi{finite verb} are deleted is again grammatical, as in (\ref{hunrelellipsis}). This reinforces the hypothesis that the absence of the \isi{operator} (or of the phrase containing the \isi{operator}) is due to some \isi{ellipsis} process that takes place in the verbal domain: that is, when the \isi{operator} fails to move up to the [Spec,\isi{CP}] position.

It has to be highlighted that ordinary relative clauses in \ili{Hungarian} do not contain the \isi{complementiser} \textit{mint} `as': they are introduced by a zero \isi{complementiser} and contain overt relative operators, there being no zero relative operators in \ili{Hungarian}. If, however, there is an overt \textit{mint} in the \isi{subclause}, the \isi{relative operator} is licensed to be absent (under the conditions discussed in connection with (\ref{hunrel}) above) since there is an overt marker introducing the \isi{subordinate clause}. In other words, the sentences in (\ref{hunrelfull}) and (\ref{hunrelop}) would be grammatical without \textit{mint} as well. It is also worth mentioning that the \isi{pronoun} in the \isi{matrix clause} is a composite of the prefix \textit{ugyan}- `same' and the \isi{pronoun} \textit{azt} `that.\textsc{acc}' but it could appear in the simple form of \textit{azt} as well; however, for most of my informants, the constructions sound more natural with the emphatic version given in (\ref{hunrel}). Since the \isi{pronoun} is also marked for case, the \isi{DP} containing the lexical \isi{noun} (\textit{a könyvet} `the book') can also be left out. The variations concerning \isi{relative clause} constructions containing the matrix pronominal element \textit{ugyanazt} and the overt \isi{relative pronoun} \textit{amit} are summarised in (\ref{ugyanazt}) below:

\ea \gll	Ugyanazt	(a	könyvet)	olvasom,	(mint)	amit	Péter (olvas). \label{ugyanazt}\\
that.same.\textsc{acc}	\phantom{(}the	book.\textsc{acc}	read.\textsc{1sg} \phantom{(}as	what.\textsc{acc} Peter \phantom{(}reads\\
\glt `I am reading the same (book) that Peter is reading.'
\z

Interestingly, the same options are available for comparatives expressing \isi{equality}; these contain the matrix \isi{clausal} \isi{pronoun} \textit{olyan} `so' or \textit{ugyanolyan} `self-same' and if there is an overt \isi{comparative operator} in the \isi{subclause}, the \isi{complementiser} \textit{mint} can be left out, as in (\ref{samebook6}):

\ea \gll	Ugyanolyan	könyvet	olvasok,	(mint)	amilyet	Péter	(olvas). \label{samebook6}\\
self.same	book.\textsc{acc}	read.\textsc{1sg} \phantom{(}as	how.\textsc{acc} Peter \phantom{(}reads\\
\glt `The book I am reading is like the one Peter is reading.'
\z
	
Again, the \isi{noun} can be left out of the \isi{matrix clause}, provided that the \isi{pronoun} \textit{ugyanolyan} takes the relevant case endings (this of course results in a change in the meaning), as given in (\ref{same6}):

\ea \gll	Ugyanolyat	olvasok,	(mint)	amilyet	Péter	(olvas). \label{same6}\\
self.same.\textsc{acc}	read.\textsc{1sg} \phantom{(}as	how.\textsc{acc} Peter \phantom{(}reads\\
\glt `What I am reading is like what Peter is reading.'
\z
	
The same option is available in ordinary comparative \isi{subclause} expressing \isi{equality}, as illustrated in (\ref{equative6}):

\ea \gll	Mari	olyan	magas,	(mint)	amilyen	az	anyja. \label{equative6}\\
Mary	as	tall \phantom{(}as	how	the	mother.\textsc{poss}\\
\glt `Mary is as tall as her mother.'
\z
	
However, this is not possible in comparatives expressing \isi{inequality}, as demonstrated by (\ref{compungrammatical6}):

\ea \gll	Mari magasabb, *(mint)	amilyen	az	anyja. \label{compungrammatical6}\\
Mary	taller \phantom{*(}than	how	the	mother.\textsc{poss}\\
\glt `Mary is taller than her mother.'
\z
	
This shows that there is a difference in the selectional restrictions between the two types: while the \isi{degree element} \textit{olyan} may select for a comparative \isi{subclause} introduced by \textit{mint} or by zero, the \isi{degree element} -\textit{bb} `-er' selects exclusively for \textit{mint} as a C head. The requirement to have an overt \isi{relative operator} in the \isi{subordinate clause} in the absence of \textit{mint} is a requirement that holds in the \isi{subclause} and is essentially one that makes the presence of some overt clause-type marker necessary: in this respect, a \isi{relative operator} is sufficient because it is also equipped with the [+rel] \isi{feature} and in comparatives also with a [+compr] \isi{feature}. I will not venture to investigate the difference between \textit{olyan} and -\textit{bb} in this respect, especially as the phenomenon is attested cross-linguistically and the requirement to have overt complementisers in comparative subclauses expressing \isi{inequality} seems to be universally applicable (see \citealt{bacskaiatkari2016alh} for a detailed analysis).

Let us now return to the clauses in (\ref{hungpredellipsis}), (\ref{hungnomellipsis}) and (\ref{hungattrellipsis}), repeated here for the sake of convenience in (\ref{hungellipsis5}):

\ea \label{hungellipsis5}
\ea \gll	Mari	magasabb	volt,	mint	Péter. \label{hungpredellipsisrepeat5}\\
Mary	taller	was.\textsc{3sg}	than	Peter\\
\glt `Mary was taller than Peter.'
\ex \gll Mari	több macskát vett, mint	Péter. \label{hungnomellipsisrepeat5}\\
Mary more	cat.\textsc{acc} bought.\textsc{3sg} than Peter\\
\glt `Mary bought more cats than Peter did.'
\ex \gll Mari nagyobb	macskát	vett,	mint Péter. \label{hungattrellipsisrepeat5}\\
Mary bigger	cat.\textsc{acc} bought.\textsc{3sg} than	Peter\\
\glt `Mary bought a bigger cat than Peter did.'
\z
\z

\citet[57--59]{bacskaiatkarikantor2012} argue in these constructions containing \isi{ellipsis} \isi{sluicing} takes place: therefore, they all contain an [E] \isi{feature} on the F (focus) head and thus the complement of the F head is elided, as given in (\ref{treehungariantpellipsis}):\footnote{\citet{bacskaiatkarikantor2012} identify the \isi{FP} as a FocP, and no \isi{TP} below it, just a functional \isi{vP}, in line with more traditional analyses of the \ili{Hungarian} clause. I follow \citet{ekiss2008li} in assuming the presence of a \isi{TP} below the \isi{FP}, but I differ from her analysis in that I do not treat the \isi{FP} as a designated Focus projection. This has the advantage that the ability of the F head to host an [E] \isi{feature} follows naturally from its status as a left-peripheral \isi{functional head}.}

\ea \label{treehungariantpellipsis} \upshape 
\begin{forest} baseline, qtree, for tree={align=center}
[FP
	[DP\textsubscript{i}
		[P\'eter,roof]
	]
	[F$'$
		[F
			[$\emptyset$\\{[}E{]}]
		]
		[TP, name=tp
			[t\textsubscript{i} volt {[}\textsubscript{QP} amilyen magas{]}\\t\textsubscript{i} vett {[}\textsubscript{DP} ahány macskát{]}\\t\textsubscript{i} vett {[}\textsubscript{DP} amilyen nagy macskát{]},roof]
		]
	]
]
\path [draw, thick] (tp.north west) -- (tp.south east);
\path [draw, thick] (tp.north east) -- (tp.south west);
\end{forest}
\z

The analysis follows \citet{vancraenenbroeckliptak2006}, who claim that \isi{sluicing} in \ili{Hungarian} is carried out by an [E] \isi{feature} on the F head; on the other hand, it is also a fairly standard assumption that \isi{PF} may save a construction via \isi{deletion}, which eliminates some \isi{uninterpretable feature} (see the discussion in \chapref{ch:4} and also \citealt[131]{kennedymerchant2000} and \citealt[145--151]{merchant2009}).

As \chapref{ch:3} argued in connection with extractable \isi{degree} operators, the quantified expression moves up first to the \isi{edge} of the verbal domain and subsequently to [Spec,\isi{CP}], and a contrastive lexical \isi{AP} is preferably stranded in [Spec,\isi{FP}]. In the elliptical examples, however, there is obviously no \isi{movement} to the \isi{edge} of the verbal domain since then the lexical element in the quantified expression should precede the focussed \isi{DP} \textit{Péter}, which is not the case. Alternatively, the \isi{ellipsis domain} could be larger by way of placing the [E] \isi{feature} on a C head, but then the entire \isi{subclause} would have to be elided. Since the \isi{FP} is a \isi{functional projection} above the \isi{TP}, the \isi{ellipsis} process is indeed similar to \ili{English} \isi{sluicing}. On the other hand, since the \isi{FP} is the highest projection to which the \isi{verb} may move and the constituent located in [Spec,\isi{TP}] regularly moves up to the [Spec,\isi{FP}] position (cf. \citealt{ekiss2008li}), locating the [E] \isi{feature} on F is also similar to VP-\isi{ellipsis} in \ili{English}. The point is that if there is an [E] feaure on the F head, the \isi{verb} does not move up to F, unlike in non-elliptical constructions.

On the other hand, note that the informational structural status of the quantified expression does not affect the \isi{ellipsis} processes as in \ili{English}. The reason behind this is that they are not the lower copies of a moved constituent that may be realised overtly under special conditions, but are the only copies since \isi{movement} has not taken place. Furthermore, due to the presence of an \isi{overt operator} with unchecked features, they are ungrammatical in their \isi{base position} irrespectively of whether they are contrastive or not. Consequently, elliptical comparatives of the type in (\ref{treehungariantpellipsis}) are only possible if the quantified expression is \textsc{given} since the elimination of an F-marked phrase would violate \isi{recoverability}.

\subsection{Contrastive verbs in predicative structures} \label{sec:6contrastiveverbs}
The question arises what happens if the \isi{verb} is not \textsc{given}. Consider the following example of a predicative structure (see also \citealt[55]{bacskaiatkarikantor2012}):

\ea \gll	Mari	magasabb,	mint	Péter	\textbf{volt}. \label{hungariancontrastivecopula}\\
Mary	taller	than	Peter	was.\textsc{3sg}\\
\glt `Mary is taller than Peter was.'
\z

As can be seen, the \isi{finite verb} (\textit{volt}) is overt in the \isi{subclause} but no overt quantified expression is present. However, unlike the sentences in (\ref{hungungrammaticalnoop}), it is still grammatical. The difference is that \textit{volt} in (\ref{hungariancontrastivecopula}) is not \textsc{given} since the zero \isi{copula} in the \isi{matrix clause} is in the present \isi{tense}, and hence the past \isi{tense} of \textit{volt} expresses new information that would not be recoverable if the \isi{verb} were elided.

This is possible if the [E] \isi{feature} is located on a \isi{functional head} lower than the F head; this was proposed by \citet{bacskaiatkarikantor2011} and the relevant \isi{functional projection} was identified as the AspP (aspectual phrase). I would like to take a slightly different approach here, as far as the projection is concerned, which I identify as a functional \isi{vP} below the \isi{TP}, identical to the \isi{PredP} proposed by \citet{ekiss2008li}. Essentially, the projection is reminiscent of the functional \isi{vP} hosting the \isi{copula} in \ili{English} predicative structures. The [E] \isi{feature} cannot be located on this v head in (\ref{hungungrammaticalnooppred}) because the [E] \isi{feature} has to be located as high as possible. In (\ref{hungungrammaticalnooppred}), the \isi{copula} does not carry new information, and should therefore be deleted. This is enabled if the [E] is on the F head, as in (\ref{hungpredellipsis}).

Note that even though the \isi{copula} remains overt and carries new information in (\ref{hungariancontrastivecopula}), the \isi{main stress} still falls on the focussed \isi{DP} \textit{Péter}, just as in (\ref{hungpredellipsis}). This is not surprising since the \isi{copula} is a function word: as stated by the Lexical Category Condition (\citealt[226]{truckenbrodt1999}) and the Principle of Categorial Invisibility of Function Words (\citealt[226]{selkirk1984}), function words are to be treated as invisible with respect to constraints holding at the syntax-phonology mapping, and they do not receive \isi{main stress} but are rather phonologically dependent on another element. In (\ref{hungariancontrastivecopula}), it is the preceding \isi{DP} \textit{Péter} (cf. \citealt[74]{ekiss2002}). On the other hand, \isi{main stress} is assigned to the \isi{DP} because Intonational Phrases are left-headed in \ili{Hungarian}: \isi{main stress} falls on the focussed constituent in a focus construction and not on the main (lexical) \isi{verb} (cf. \citealt[50--53]{szendroi2001diss}).\footnote{In other words, \isi{nuclear stress} falls on the leftmost element that may bear \isi{nuclear stress}. This excludes topics from being assigned \isi{main stress} as topics are treated as extrametrical (cf. \citealt[49]{szendroi2001diss}, based on \citealt{truckenbrodt1999}). On the other hand, the \isi{complementiser} is not to be stressed either, as shown by \citet{keneseivogel1989, keneseivogel1995}. Complementisers belong to the same \isi{Intonational Phrase} as the entire \isi{subclause}. However, based on the Lexical Category Condition (\citealt[226]{truckenbrodt1999}) and the Principle of Categorial Invisibility of Function Words (\citealt[226]{selkirk1984}), function words are to be treated as invisible with respect to constraints holding at the syntax–phonology mapping: consequently, the \isi{complementiser} \textit{mint} `than' may not receive strong stress. \citet{satodobashi2012} show for \ili{English} that complementisers are phonologically dependent on the word that immediately follows them. Since the present investigation is not centred on prosody and the mapping rules between syntax and prosodic structure, I will not elaborate on these issues any further here; for a recent discussion on the syntax-prosody mapping in \ili{Hungarian} comparatives, see \citet{bacskaiatkari2013vllxx}.}

\largerpage[1]
The derivation of (\ref{hungariancontrastivecopula}) is schematically given in (\ref{treehungariancontrastivecopula}):

\ea \label{treehungariancontrastivecopula} \upshape 
\begin{forest} baseline, qtree, for tree={align=center}
[FP
	[DP\textsubscript{i}
		[P\'eter,roof]
	]
	[F$'$
		[F
			[volt\textsubscript{j}]
		]
		[TP
			[t\textsubscript{i}]
			[T$'$ [T [t\textsubscript{j}]] [PredP [t\textsubscript{i}] [Pred$'$ [Pred [t\textsubscript{j}\\{[}E{]}]] [VP, name=tp [t\textsubscript{j} t\textsubscript{i} {[}\textsubscript{QP} amilyen magas{]},roof]]]]]
		]
	]
]
\path [draw, thick] (tp.north west) -- (tp.south east);
\path [draw, thick] (tp.north east) -- (tp.south west);
\end{forest}
\z

As described by \citet{ekiss2008li}, the constituent landing in the specifier of \isi{FP} (her FocP) first moves to the specifier of a Predicative Phrase (PredP) and to the specifier of the \isi{TP}; the \isi{lexical verb} moves along. In (\ref{treehungariancontrastivecopula}), the \isi{verb} is a \isi{copula} and is base-generated in Pred, just as the \ili{English} \isi{copula} is base-generated in a functional v head instead of V. The difference from (\ref{treehungariantpellipsis}) lies in the fact that here the [E] \isi{feature} is located on the Pred head, not the F head. This is possible because the Pred is a \isi{functional head} (essentially, a functional v head). If the [E] \isi{feature} were located on F, then \isi{verb movement} again would not take place, just as in (\ref{treehungariantpellipsis}). However, it is possible to base-generate the \isi{copula} on a head containing [E]; in this case, the \isi{copula} can again move higher to the F head. As the \isi{copula} is contrastive, locating the [E] \isi{feature} on Pred is in line with the requirement to delete the maximal \textsc{given} constituent. The point is that the \isi{copula} can be inserted into Pred irrespectively of whether there is an [E] \isi{feature} or not. The same does not hold for the F head, which does not trigger \isi{verb movement} if the [E] \isi{feature} is present, and \isi{verb movement} is not obligatory either if the clause is not finite (see \citealt{ekiss2008li}).

\subsection{Contrastive verbs in attributive and nominal structures} \label{sec:6contrastiveverbsattr}
The importance of all this becomes obvious when considering nominal and attributive comparative examples such as (\ref{huncontrastnomattr}), where the \isi{verb} that carries new information is a lexical one and as such is actually F-marked:

\ea \label{huncontrastnomattr}
\ea [?]{\gll Mari	több	macskát	vett,	mint	Péter \textbf{látott}.\\
Mary	more	cat.\textsc{acc}	bought.\textsc{3sg}	than	Peter saw.\textsc{3sg}\\
\glt `Mary bought more cats than Peter saw.'}
\ex	[?]{\gll	Mari	nagyobb	macskát	vett,	mint	Péter \textbf{látott}.\\
Mary	bigger	cat.\textsc{acc}	bought.\textsc{3sg}	than	Peter saw.\textsc{3sg}\\
\glt `Mary bought a bigger cat than Peter saw.'}
\z
\z

As can be seen, the \isi{lexical verb} in the \isi{subclause} (\textit{látott}) is different from the one in the \isi{matrix clause} (\textit{vett}); the sentences are acceptable but marked (the individual ratings of my informants differed as far as the \isi{degree} of \isi{markedness} is concerned).

Since the present investigation is not particularly concerned with the theory of focus in general, I do not attempt to address the issue of verbs and focus in detail. As shown by \citet{kenesei2006}, instances where the \isi{verb} seems to be focussed do not involve the focussing of the V head as such but either the \isi{VP} or the entire proposition is focussed. This is actually in line with my analysis here and the examples in this section clearly demonstrate that it is not merely a \isi{verb} in the \isi{subclause} that is contrasted with the one in the \isi{matrix clause} but rather an entire proposition: there are other elements that are contrastive, such as the subject \isi{DP} in the examples in (\ref{huncontrastnomattr}). However, since in the construction under scrutiny, contrastive elements are located above the \isi{VP} (all thematic \isi{vP} layers) and elements that are left in the thematic verbal domain are non-contrastive, what really matters to us here is indeed the status of the lexical verbal head. This behaves differently with respect to the \isi{ellipsis domain} depending on whether there is propositional \isi{contrast} or not.

\largerpage[1]
In (\ref{treehungariantpellipsis}), the \isi{lexical verb} (\textit{vett}) moves up from a functional v head (assuming a layered analysis of the \ili{Hungarian} \isi{verb} phrase, see \citealt{ekiss2008}, \citealt{ekiss2009}) to T but not beyond, since the F head contains an [E] \isi{feature}. This means that, in order to derive the constructions in (\ref{huncontrastnomattr}), the \isi{lexical verb} has to move up to F despite the presence of the [E] \isi{feature}, which regularly does not require \isi{verb movement} to F. Since the quantified DPs are not present overtly, \isi{ellipsis} must have taken place, as should be obvious from the discussion in the previous subsection. Since the \isi{lexical verb} has to undergo a \isi{movement} operation that it would not take otherwise, the construction is marked.

Moreover, in (\ref{huncontrastnomattr}) the \isi{main stress} has to fall on the \isi{verb} in the \isi{subordinate clause}; this follows from the fact that there are two propositions compared in (\ref{huncontrastnomattr}), and the \isi{contrast} is expressed by the main \isi{verb}. However, this would not be possible if the \isi{DP} \textit{Péter} were located in [Spec,\isi{FP}] because then the \isi{main stress} would be assigned to that constituent. Therefore, the \isi{DP} \textit{Péter} in (\ref{huncontrastnomattr}) has to move to a \isi{topic position} in order to escape both \isi{ellipsis} and \isi{main stress} (see \citealt{szendroi2001diss} on the extrametricality of topics).

As was mentioned before, contrastive verbs involve the \isi{contrast} between entire propositions and not merely verbal heads; in (\ref{huncontrastnomattr}), for instance, the subject DPs in the two clauses are also different. Though contrastive elements tend to appear preverbally, it is also possible to have contrastive elements that follow the \isi{verb}. Consider the following example:

\ea \gll	Nagyobb	macskát	vettem	\textbf{ma},	mint	amekkorát	láttam \textbf{tegnap}. \label{hungadverbcontrast}\\
bigger	cat.\textsc{acc}	bought.\textsc{1sg}	today	than	how.big.\textsc{acc}	saw.\textsc{1sg} yesterday\\
\glt `I bought a bigger cat today than the one I saw yesterday.'
\z
	
In this case, the adjuncts \textit{ma} `today' and \textit{tegnap} `yesterday' are also contrasted and they are phonologically prominent. In this position, as pointed out by \citet[53--55]{szendroi2001diss}, elements receive extra stress by an additional prosodic rule and not by the \isi{nuclear stress} rule. By default, it is more economical to move a phrase to the \isi{FP} for stress assignment than to leave it in the \isi{VP} but the \isi{verb} in (\ref{hungadverbcontrast}) above is also contrastive and would not receive \isi{main stress} by default if there were an element in the [Spec,\isi{FP}] position. This is in line with the analysis given by \citet{kenesei2006}, in that in the case of VP-focus or propositional focus the contrastive elements following the \isi{verb} are assigned focal stress. Note that this construction is not possible if the \isi{verb} is not contrastive, that is, when there is no propositional \isi{contrast}, as shown by (\ref{nocontrastungrammatical6}):

\ea [*]{\gll	Nagyobb	macskát	vettem	\textbf{ma},	mint	amekkorát	vettem \textbf{tegnap}. \label{nocontrastungrammatical6}\\
bigger	cat.\textsc{acc}	bought.\textsc{1sg}	today	than	how.big.\textsc{acc}	bought.\textsc{1sg} yesterday\\
\glt `I bought a bigger cat today than yesterday.'}
\z
	
In this case the \isi{adverb} \textit{tegnap} should move up to the [Spec,\isi{FP}] position in order to give a felicitous construction.

Turning back to (\ref{huncontrastnomattr}), there are a number of arguments in favour of the analysis, regarding both of the \isi{movement} of the \isi{lexical verb} and the non-focussed nature of the subject \isi{DP}. Evidence comes from constructions involving a verbal \isi{particle}; consider first the following examples that do not involve \isi{ellipsis}:

\ea \label{hungprtverb}
\ea \gll	Mari	több	macskát	vett,	mint	\textbf{ahány} \textbf{macskát}	Péter	\textbf{meglátott}.\\
Mary	more	cat.\textsc{acc}	bought.\textsc{3sg}	than	how.many cat.\textsc{acc} Peter \textsc{prt}.saw.\textsc{3sg}\\
\glt `Mary bought more cats than Peter noticed.'
\ex \gll	Mari	nagyobb	macskát	vett,	mint	\textbf{amekkora} \textbf{macskát}	Péter	\textbf{meglátott}.\\
Mary	more	cat.\textsc{acc}	bought.\textsc{3sg}	than	how.big cat.\textsc{acc} Peter \textsc{prt}.saw.\textsc{3sg}\\
\glt `Mary bought a bigger cat than Peter noticed.'
\z
\z

In (\ref{hungprtverb}), the comparative \isi{subclause} contains the verbal \isi{particle} \textit{meg}, which precedes the \isi{lexical verb}: adopting the analysis given by \citet{ekiss2008}, this is because the \isi{particle} moves to the specifier of \isi{PredP}. In other words, a verbal \isi{particle} preceding the \isi{lexical verb} is in complementary distribution with a focussed constituent that would also move to this position (before moving further up to the specifier of the \isi{TP} and of the \isi{FP}) and the verbal \isi{particle} + \isi{verb} order is indicative of the fact that there is no focussed constituent in the [Spec,\isi{FP}] position and the \isi{DP} \textit{Péter} in (\ref{hungprtverb}) is a \isi{topic} (though contrastive). The main \isi{sentential stress} in (\ref{hungprtverb}) hence falls on the leftmost element of the \isi{Intonational Phrase}, which is the verbal \isi{particle} \textit{meg} (cf.  \citealt{szendroi2001diss}).

The structure of the subclauses in (\ref{hungprtverb}) is represented in (\ref{treehungarianfullcp}):\footnote{I follow generally accepted views regarding the structure of a \ili{Hungarian} finite (subordinate) clause, see for instance in \citet{ekiss2002}, in that a focussed constituent may be preceded by topics and topics are immediately below the CP-layer. Of course, there are other possible functional projections that can otherwise occur but since my examples contain none of them, I am not particularly concerned with whether they are underlyingly present even when they are not overtly filled. Note also that, since I am using a non-cartographic approach, the distinction between various functional projections in the syntax is less important here than usually assumed in the literature.}

\ea \label{treehungarianfullcp} \upshape 
\begin{forest} baseline, qtree, for tree={align=center}
[CP
	[C$'$
		[C
			[mint]
		]
		[CP
			[DP\textsubscript{k} [ahány macskát\\amekkora macskát,roof]]
			[C$'$ [C] [TopP [DP\textsubscript{j} [P\'eter,roof]] [Top$'$ [Top] [TP [meg\textsubscript{i}] [T$'$ [T [l\'atott\textsubscript{l}]] [PredP [t\textsubscript{l} t\textsubscript{i} t\textsubscript{j} t\textsubscript{k},roof]]]]]]]
		]
	]
]
\end{forest}
\z

Since the quantified \isi{DP} moves up to the [Spec,\isi{CP}] position, there is no \isi{ellipsis} taking place. As there is no \isi{FP} (there being no focussed constituent) and [Spec,\isi{TP}] is filled by the verbal \isi{particle} \textit{meg}, main \isi{sentential stress} falls on this element, which renders a felicitous sentence since the main \isi{contrast} involved in the comparison is expressed by the \isi{verb}.

When the \isi{verb} is not contrastive, the situation is different:

\largerpage[2]
\ea \label{hungverbprt}
\ea \gll	Mari	több	macskát	látott,	mint	\textbf{ahány} \textbf{macskát}	Péter	\textbf{látott}	\textbf{meg}.\\
Mary	more	cat.\textsc{acc} saw.\textsc{3sg} than	how.many cat.\textsc{acc} Peter saw.\textsc{3sg} \textsc{prt}\\
\glt `Mary saw more cats than Peter noticed.'
\ex \gll	Mari	nagyobb	macskát	látott,	mint	\textbf{amekkora} \textbf{macskát}	Péter	\textbf{látott}	\textbf{meg}.\\
Mary	bigger	cat.\textsc{acc} saw.\textsc{3sg} than	how.big cat.\textsc{acc} Peter saw.\textsc{3sg} \textsc{prt}\\
\glt `Mary saw a bigger cat than Peter noticed.'
\z
\z

The subclauses in (\ref{hungverbprt}) contain the \isi{DP} \textit{Péter} as the focussed constituent; following \citet{ekiss2008li}, this \isi{DP} is located in the specifier of the \isi{FP}, and the verbal \isi{particle} does not move up, resulting in the non-neutral \isi{verb} + verbal \isi{particle} order. Since the leftmost constituent in this case is the focussed \isi{DP}, \isi{main stress} will fall on this constituent; this again renders a felicitous structure as the main \isi{contrast} in (\ref{hungverbprt}) is expressed by the \isi{DP}. The structure of the \isi{subclause} is shown in (\ref{treehungarianverbprt}):

\ea \label{treehungarianverbprt} \upshape 
\begin{forest} baseline, qtree, for tree={align=center}
[CP
	[C$'$
		[C
			[mint]
		]
		[CP
			[DP\textsubscript{k} [ahány macskát\\amekkora macskát,roof]]
			[C$'$ [C] [FP [DP\textsubscript{j} [P\'eter,roof]] [F$'$ [F [l\'atott\textsubscript{i}]] [TP [t\textsubscript{i} meg t\textsubscript{j} t\textsubscript{k},roof]]]]]
		]
	]
]
\end{forest}
\z

One of the chief differences between (\ref{treehungarianfullcp}) and (\ref{treehungarianverbprt}) is that there is no topicalised constituent in (\ref{treehungarianverbprt}) since the \isi{DP} \textit{Péter} moves to the \isi{FP}; on the other hand, as has been said, the presence of the \isi{DP} in the [Spec,\isi{FP}] position excludes the possibility of the verbal \isi{particle} (\textit{meg}) also moving out of the \isi{VP}, since both constituents cannot land in [Spec,\isi{PredP}]. The structures in (\ref{treehungarianfullcp}) and (\ref{treehungarianverbprt}) are not interchangeable: that is, whether the \isi{lexical verb} is contrastive or not determines what constituent may move to [Spec,\isi{PredP}] to result in a felicitous structure. On the other hand, the position of the verbal \isi{particle} (its relative position to the \isi{verb}) is indicative of which constituent has moved out.

Naturally, it is also possible to have full comparative subclauses without verbal particles:

\largerpage[-2]
\ea \label{hungfullnoprt}
\ea \gll	Mari	több	macskát	vett,	mint	\textbf{ahány} \textbf{macskát}	Péter	\textbf{látott}. \label{hunnoprtmorecatsbought}\\
Mary	more	cat.\textsc{acc} bought.\textsc{3sg}	than	how.many cat.\textsc{acc} Peter saw.\textsc{3sg}\\
\glt `Mary bought more cats than Peter saw.'
\ex \gll	Mari	nagyobb	macskát	vett,	mint	\textbf{amekkora} \textbf{macskát}	Péter	\textbf{látott}. \label{hunnoprtbiggercatbought}\\
Mary	bigger	cat.\textsc{acc} bought.\textsc{3sg}	than	how.big cat.\textsc{acc} Peter saw.\textsc{3sg}\\
\glt `Mary bought a bigger cat than Peter saw.'
\ex \gll	Mari	több	macskát	látott,	mint	\textbf{ahány} \textbf{macskát}	Péter	\textbf{látott}. \label{hunnoprtmorecatssaw}\\
Mary	more	cat.\textsc{acc} saw.\textsc{3sg}	than	how.many cat.\textsc{acc} Peter saw.\textsc{3sg}\\
\glt `Mary saw more cats than Peter did.'
\ex \gll	Mari	nagyobb	macskát	látott,	mint	\textbf{amekkora} \textbf{macskát}	Péter	\textbf{látott}. \label{hunnoprtbiggercatsaw}\\
Mary	bigger	cat.\textsc{acc} saw.\textsc{3sg}	than	how.big cat.\textsc{acc} Peter saw.\textsc{3sg}\\
\glt `Mary saw a bigger cat than Peter did.'
\z
\z

In these cases, the surface \isi{word order} in itself is not indicative of the underlying syntactic differences; however, the \isi{main stress} falls on the \isi{lexical verb} (\textit{látott}) in (\ref{hunnoprtmorecatsbought}) and (\ref{hunnoprtbiggercatbought}), while it falls on the \isi{DP} \textit{Péter} in (\ref{hunnoprtmorecatssaw}) and (\ref{hunnoprtbiggercatsaw}).

The structure of the subclauses in (\ref{hunnoprtmorecatsbought}) and (\ref{hunnoprtbiggercatbought}) is given in (\ref{treehungariancatbought}) below:

\ea \label{treehungariancatbought} \upshape 
\begin{forest} baseline, qtree, for tree={align=center}
[CP
	[C$'$
		[C
			[mint]
		]
		[CP
			[DP\textsubscript{k} [ahány macskát\\amekkora macskát,roof]]
			[C$'$ [C] [TopP [DP\textsubscript{j} [P\'eter,roof]] [Top$'$ [Top] [TP [T$'$ [T [l\'atott\textsubscript{i}]] [PredP [t\textsubscript{i} t\textsubscript{j} t\textsubscript{k},roof]]]]]]]
		]
	]
]
\end{forest}
\z

In line with \citet{ekiss2008}, there is no \isi{FP} in these cases: the \isi{main stress} falls on the \isi{lexical verb} anyway and the \isi{DP} moves to a \isi{topic position},\footnote{Note that topics may also be contrastive, and therefore the \isi{DP} \textit{Péter} can be located in a \isi{topic position} even though it is contrasted with the \isi{DP} \textit{Mari} in the \isi{matrix clause}. As described for instance by \citet[72--78]{ekiss2007}, in case a clause contains multiple contrastive elements, contrastive topics always precede the focus, which is essentially in line with the assumption that the focus is at the \isi{left edge} of the \isi{FP} and hence receives \isi{main stress}, while elements located above it cannot be interpreted as foci. Note also that contrastive elements may occasionally also appear in a \isi{postverbal position}, in which case they receive extra stress by an additional prosodic rule, as was described before (and see \citealt{szendroi2001diss}).} and thus there is nothing that could potentially be located in the [Spec,\isi{FP}] position.

By \isi{contrast}, the structure of (\ref{hunnoprtmorecatssaw}) and (\ref{hunnoprtbiggercatsaw}) is as given in (\ref{treecats6}):

\ea \upshape \label{treecats6}
\begin{forest} baseline, qtree, for tree={align=center}
[CP
	[C$'$
		[C
			[mint]
		]
		[CP
			[DP\textsubscript{k} [ahány macskát\\amekkora macskát,roof]]
			[C$'$ [C] [FP [DP\textsubscript{j} [P\'eter,roof]] [F$'$ [F [l\'atott\textsubscript{i}]] [TP [T$'$ [T [t\textsubscript{i}]] [PredP [t\textsubscript{i} t\textsubscript{j} t\textsubscript{k},roof]]]]]]]
		]
	]
]
\end{forest}
\z

The only difference from (\ref{treehungarianverbprt}) is that there is no verbal \isi{particle} but, since the verbal \isi{particle} is located within the thematic \isi{PredP} in (\ref{treehungarianverbprt}) as well, this makes no difference as far as the \isi{FP} is concerned.

To conclude, it should be obvious that if the \isi{verb} is contrastive, then the \isi{FP} is not generated, since anything in [Spec,\isi{FP}] would have to bear \isi{main stress}. Main stress is thus assigned to the \isi{verb} in T, and if there is a verbal \isi{particle}, which phonologically attaches itself to the \isi{verb}, \isi{main stress} falls on the \isi{particle} (which thus constitutes the first syllable of a phonological word). Otherwise, \isi{main stress} would fall on a constituent distinct from the \isi{verb} and the sentence would not be felicitous. In these cases, a contrastive \isi{DP} is topicalised, while it appears in the [Spec,\isi{FP}] position if the \isi{verb} is \textsc{given}.

Based on all this, the structure of the subclauses in (\ref{huncontrastnomattr}) is the following:

\ea \label{treehungcontrastnomattrpredp} \upshape 
\begin{forest} baseline, qtree, for tree={align=center}
[CP
	[C$'$
		[C
			[mint]
		]
		[TopP
			[DP\textsubscript{j} [P\'eter,roof]]
			[Top$'$ [Top] [TP [T$'$ [T [l\'atott\textsubscript{i}]] [PredP [Pred$'$ [Pred [t\textsubscript{i}\\{[}E{]}]] [VP, name=tp [t\textsubscript{i} t\textsubscript{j} {[}\textsubscript{DP} ahány/amekkora macskát{]},roof]]]]]]]
		]
	]
]
\path [draw, thick] (tp.north west) -- (tp.south east);
\path [draw, thick] (tp.north east) -- (tp.south west);
\end{forest}
\z

In (\ref{huncontrastnomattr}), the \isi{verb} must move out of the \isi{VP} in order to escape \isi{deletion}. This is similar to what was observed in connection with (\ref{hungariancontrastivecopula}), yet there is a crucial difference as well. In (\ref{hungariancontrastivecopula}), the \isi{copula} is base-generated in Pred, and the [E] \isi{feature} is added to Pred that happens to contain \isi{lexical material} already. In (\ref{huncontrastnomattr}), however, the \isi{lexical verb} is generated in the \isi{VP} and must move to the Pred head, which already contains an [E] \isi{feature}, and, as was argued previously, the \isi{movement} of the \isi{verb} is normally not triggered by a head that contains the [E] \isi{feature}. While the violation of this constraint does not result in ungrammaticality, the structure is marked due to an extra \isi{movement} step, hence the \isi{markedness} of the sentences in (\ref{huncontrastnomattr}). As there is no focussed constituent, the \isi{FP} is not generated and the \isi{verb} moves only as far as T. The \isi{lexical verb} can thus be assigned main \isi{sentential stress} in the prosody: the \isi{DP} \textit{Péter} is topicalised and falls outside the domain of \isi{nuclear stress}.

One might wonder whether the \isi{DP} \textit{Péter} is a focus, hence a constituent located in a [Spec,\isi{FP}] position. However, taking such a stance would be problematic for various reasons; first, it would contradict the data shown by comparatives with a full \isi{subclause}, as in (\ref{hungprtverb}), (\ref{hungverbprt}) and (\ref{hungfullnoprt}). Second, it would wrongly predict that if there is a verbal \isi{particle}, then it should follow the \isi{verb}, which is not the case:

\ea \label{huncontrastnomattrprt}
\ea [*]{\gll Mari	több	macskát	vett,	mint	Péter \textbf{látott} \textbf{meg}.\\
Mary	more	cat.\textsc{acc}	bought.\textsc{3sg}	than	Peter saw.\textsc{3sg} \textsc{prt}\\
\glt `Mary bought more cats than Peter noticed.'}
\ex	[*]{\gll	Mari	nagyobb	macskát	vett,	mint	Péter \textbf{látott} \textbf{meg}.\\
Mary	bigger	cat.\textsc{acc}	bought.\textsc{3sg}	than	Peter saw.\textsc{3sg} \textsc{prt}\\
\glt `Mary bought a bigger cat than Peter noticed.'}
\z
\z

The reason why the focus status of the \isi{DP} \textit{Péter} would trigger the \isi{verb} + verbal \isi{particle} order is that the [Spec,\isi{FP}] would then be occupied by this \isi{DP} and the \isi{verb} would move up to F, leaving the \isi{particle} behind, and thus the \isi{particle} would necessarily follow the \isi{verb}. However, the sentences in (\ref{huncontrastnomattrprt}) are not acceptable, which indicates that the \isi{DP} \textit{Péter} cannot be in [Spec,\isi{FP}]. If both a preverbal \isi{DP} and the \isi{verb} are contrastive (that is, there is VP-focus or propositional focus), then it is the \isi{verb} that should bear the pitch accent and not the \isi{DP}. Similarly, as was also mentioned before, if there are two contrastive DPs, then the first one is a contrastive \isi{topic} and the second one a focus, the latter bearing \isi{nuclear stress}. In order to ensure that the \isi{DP} does not get \isi{nuclear stress}, it has to move to a \isi{topic position} and since topics can also be contrastive, this does not result in semantic incongruence either. The correct intonation pattern can be assigned to the overt elements in the \isi{subclause} in (\ref{huncontrastnomattr}) but not in (\ref{huncontrastnomattrprt}), where the \isi{postverbal position} of the verbal \isi{particle} clearly indicates that the [Spec,\isi{FP}] is filled by another element (the \isi{DP} \textit{Péter}).

Moreover, if there is a verbal \isi{particle}, it moves to the [Spec,\isi{PredP}] and subsequently to the [Spec,\isi{TP}] position and the \isi{verb} moves to Pred and subsequently to T, resulting in the verbal \isi{particle} + \isi{verb} order, as indicated by the acceptability of the following examples:

\ea \label{huncontrastnomattrprtgrammatical}
\ea [?]{\gll Mari	több	macskát	vett,	mint	Péter \textbf{meglátott}.\\
Mary	more	cat.\textsc{acc}	bought.\textsc{3sg}	than	Peter \textsc{prt}.saw.\textsc{3sg}\\
\glt `Mary bought more cats than Peter noticed.'}
\ex	[?]{\gll	Mari	nagyobb	macskát	vett,	mint	Péter \textbf{meglátott}.\\
Mary	bigger	cat.\textsc{acc}	bought.\textsc{3sg}	than	Peter \textsc{prt}.saw.\textsc{3sg}\\
\glt `Mary bought a bigger cat than Peter noticed.'}
\z
\z

The acceptability of the sentences in (\ref{huncontrastnomattrprtgrammatical}) is similar to that of the ones in (\ref{huncontrastnomattr}); again, individual ratings may differ with respect to the \isi{degree} of \isi{markedness}. This is not surprising since the structure should be essentially the same as the one given in (\ref{treehungcontrastnomattrpredp}), except that there is a verbal \isi{particle} in the [Spec,\isi{PredP}] position:

\ea \label{treehungcontrastnomattrpredpprt} \upshape 
\begin{forest} baseline, qtree, for tree={align=center}
[CP
	[C$'$
		[C
			[mint]
		]
		[TopP
			[DP\textsubscript{j} [P\'eter,roof]]
			[Top$'$ [Top] [TP [meg\textsubscript{k}] [T$'$ [T [l\'atott\textsubscript{i}]] [PredP [t\textsubscript{k}] [Pred$'$ [Pred [t\textsubscript{i}\\{[}E{]}]] [VP, name=tp [t\textsubscript{i} t\textsubscript{k} t\textsubscript{j} {[}\textsubscript{DP} ahány/amekkora macskát{]},roof]]]]]]]
		]
	]
]
\path [draw, thick] (tp.north west) -- (tp.south east);
\path [draw, thick] (tp.north east) -- (tp.south west);
\end{forest}
\z

Just as in (\ref{treehungcontrastnomattrpredp}), the [E] \isi{feature} is located on the Pred head and hence the domain of \isi{ellipsis} is the \isi{VP}; again, the \isi{lexical verb} moves up to the Pred head and subsequently to T to escape \isi{deletion}, which involves an extra step in the derivation since the Pred head contains an [E] \isi{feature}; consequently, the structure is marked. The reason why there is no other option is that the [E] \isi{feature} cannot be located on a lexical v/V head. Therefore, either \isi{ellipsis} would not take place, leaving the uninterpretable quantified \isi{DP} overt, or \isi{ellipsis} would affect the contrastive \isi{verb} too: in either case, the structure would not converge.

To sum up, \ili{Hungarian} comparative subclauses that contain an overt, F-marked \isi{finite verb} but no overt quantified expression differ in a predictable way from ones that prohibit the presence of a \textsc{given} \isi{verb}. That is, the \isi{ellipsis} mechanisms are essentially the same in both cases: it is invariably an [E] \isi{feature} located on a \isi{functional head} that causes the complement to be elided at \isi{PF}. On the other hand, the derivation of ones containing a contrastive \isi{verb} involves an extra \isi{movement} step, which results in degraded acceptability.

\subsection{More on cross-linguistic differences} \label{sec:6moreon}
As has been seen, Standard \ili{English} and \ili{Hungarian} represent two rather different patterns in terms of comparative \isi{subclause} formation. Yet, as I also demonstrated, these differences can be reduced to general requirements and they do not result from construction-specific rules.

First, the reason why (Standard) \ili{English} exhibits \isi{Comparative Deletion} but \ili{Hungarian} does not is that the \isi{Overtness Requirement} requires relative operators (including comparative operators) to be phonologically visible if there is other overt material in [Spec,\isi{CP}]. Since Standard \ili{English} has no overt comparative operators, \isi{deletion} is required; by contrast, \ili{Hungarian} has only overt comparative operators and \isi{deletion} does not (and cannot) take place in [Spec,\isi{CP}]. In this way, the \isi{Overtness Requirement} is responsible for the difference between the \ili{English} example in (\ref{englishpattern}) and the \ili{Hungarian} one in (\ref{hungarianpattern}):

\ea \label{englishhungdiff6}
\ea	Mary bought more cats than Peter bought. \label{englishpattern}
\ex \gll Mari	több	macskát	vett,	mint ahány macskát	Péter	vett. \label{hungarianpattern}\\
Mary more	cat.\textsc{acc} bought.\textsc{3sg}	than	how.many cat.\textsc{acc} Peter bought.\textsc{3sg}\\
\glt `Mary bought more cats than Peter did.'
\z
\z

The difference in (\ref{englishhungdiff6}) can be detected in maximally non-elliptical clauses; however, clauses that are derived via \isi{ellipsis} tend to look the same:

\ea
\ea Mary bought more cats than Peter (did).
\ex	\gll Mari	több	macskát	vett,	mint	Péter. \label{hungarianelliptical}\\
Mary more	cat.\textsc{acc} bought.\textsc{3sg} than	Peter\\
\glt `Mary bought more cats than Peter did.'
\z
\z
		
As can be seen, the \isi{lexical verb} and the quantified \isi{DP} are missing in both cases. I showed in the previous sections that the \isi{deletion} mechanism is essentially the same in the two languages: that is, an [E] \isi{feature} is located on a \isi{functional head} and the complement of that \isi{functional head} is the domain of \isi{ellipsis}. In \ili{English}, the functional \isi{vP} can be headed by the \isi{dummy auxiliary} (here: \textit{did}), which is not an option in \ili{Hungarian} as \ili{Hungarian} does not have such auxiliaries. Apart from that, it seems that \isi{ellipsis} works in the same way in the two languages; yet, there are some questions to be clarified.

One point of difference concerns structures such as (\ref{englishdpremnant}) below:

\ea	Mary is taller than Peter. \label{englishdpremnant}
\z

I argued in \sectref{sec:6ellipsis} that either the \isi{vP} domain (as a phase) is not spelt out at \isi{PF} in such constructions, or there is no underlying \isi{TP} (see also \citealt{bacskaiatkari2014alh}); in either case, no \isi{ellipsis} takes place here. This option was not attested for \ili{Hungarian} and the reason behind this is that while in \ili{English} a subject \isi{DP}, as \textit{Peter} in (\ref{englishdpremnant}), normally moves to the specifier of the \isi{TP}/IP, the head of which cannot host the [E] \isi{feature}, in \ili{Hungarian} a contrastive \isi{DP} like \textit{Péter} in (\ref{hungarianelliptical}) moves to the [Spec,\isi{FP}] position, the head of which is a proper \isi{functional head} that can host the [E] \isi{feature}. In other words, \ili{English} has to resort to either the lack of spellout or to certain structural changes in order to derive at least a subset of comparative constructions. This is not the case in \ili{Hungarian} due to the peculiar properties of the \ili{Hungarian} clause which can generally be observed (and which are not construction-specific). If there are multiple contrastive elements, or when the \isi{verb} itself is contrastive, then it is possible for a contrastive element to be a \isi{topic} in \ili{Hungarian}. However, topics are essentially adjoined and prosodically count as extrametrical elements; as a consequence, there must be some overt element between the \isi{VP} and the \isi{topic} field so that main \isi{sentential stress} may be assigned. To conclude, if there is a single overt \isi{DP} in the \ili{Hungarian} comparative \isi{subclause}, as in (\ref{hungarianelliptical}), it is in the [Spec,\isi{FP}] position, as opposed to \ili{English}.

The second important difference concerns the way \isi{ellipsis} seems to operate in the two languages: whereas contrastive elements in \ili{English} were shown to be able to withstand linear, left to right \isi{deletion}, this is not attested in \ili{Hungarian}. The reason behind this is quite simple: in \ili{Hungarian}, contrastive elements move to the left and are hence located above the \isi{functional head} responsible for \isi{ellipsis} and thus there is simply no element that could withstand \isi{ellipsis} following the \isi{deletion} of a string of non-contrastive elements. In \ili{English}, however, contrastive elements appear clause-finally and, since \isi{ellipsis} works in a left to right fashion, the only way to have both \isi{ellipsis} and overt contrastive elements is precisely the one described as F-marked elements stopping the \isi{linear deletion} process. Furthermore, the difference between \ili{English} and \ili{Hungarian} in this respect follows from the way \isi{sentential stress} is assigned: in \ili{English}, stress falls on the rightmost constituent in the \isi{Intonational Phrase}, while in \ili{Hungarian} it falls on the leftmost constituent (see \citealt{szendroi2001diss}). In this way, \isi{ellipsis} in comparative subclauses (and in other constructions) can be directly linked to the way the syntax-prosody mapping operates in a given language. Since the detailed examination of this issue would clearly fall outside the scope of the present book, I will not venture to investigate it any further here.

Finally, let me highlight an important aspect of the analysis proposed here: this concerns the location of the [E] \isi{feature} on functional heads and the directionality of \isi{ellipsis}. Based on \citet{merchant2001}, the [E] \isi{feature} is located on some \isi{functional head}, such as a C or a v head, and \isi{ellipsis} affects the complement of that \isi{functional head}, which is located to the right. As I showed, this does not exclude the possibility of contrastive elements appearing clause-finally but \isi{ellipsis} still operates in a strictly left to right fashion. This predicts that \isi{ellipsis} can operate in a certain domain only if the \isi{functional head} precedes its complement: that is, if the projection in question (a \isi{CP} or a \isi{vP}) is head-initial. A head-final \isi{functional projection} is not able to license \isi{ellipsis} because in that case \isi{ellipsis} would have to apply retrospectively.

The difference between head-initial and head-final projections is attested in \ili{German}. In \ili{German}, the \isi{CP} is head-initial and, as also pointed out by \citet{merchant2004, merchant2013}, \isi{sluicing} is attested as in \ili{English}: that is, carried out by an [E] \isi{feature} on a C head. Compare the examples in (\ref{englishgermancompare6}) from \ili{English} and \ili{German}:

\ea \label{englishgermancompare6}
\ea	Ralph saw someone, but I don't know \textbf{who} \sout{he saw}.
\ex	\gll Ralf	hat	jemanden	gesehen,	aber	ich	weiß nicht,	\textbf{wen} \sout{er} \sout{gesehen} \sout{hat}.\\
Ralph	has	someone\textsc{acc}	seen	but	I	know.\textsc{1sg} not who.\textsc{acc} he seen has\\
\glt `Ralph saw someone but I don’t know who.'
\z
\z

In both cases, there is a \textit{wh}-\isi{pronoun} located in a [Spec,\isi{CP}] position and the complement of the C, equipped with an [E] \isi{feature}, is elided.

\largerpage[2]
However, in \ili{German}, the \isi{VP} and all vPs are head-final (cf. \citealt[34]{haider1985}), and VP-\isi{ellipsis} is not attested in the way it is in \ili{English} (see the discussion in \chapref{ch:4} for more details). As was pointed out in \chapref{ch:4}, this is responsible for the difference in the acceptability\footnote{See \citealt{kennedymerchant2000} regarding the grammaticality of the English data. My informants have differing judgements regarding the acceptability of the gapping structure, as opposed to the pseudogapping structure: while for most of them it is grammatical, some of them find it rather marked. There seems to be even more variation in German regarding the version of (\ref{germanvpellipsis5}) where the finite verb is not elided, whereby the differences are apparently not regional. These questions cannot be discussed here in more detail, but should be clarified by experimental studies.} of the examples for comparatives in (\ref{compellipsis6}):

\ea \label{compellipsis6}
\ea []{Ralph has a bigger flat than Michael \sout{has} a house.} \label{englishvpellipsis}
\ex	[*]{\gll Ralf	hat	eine	größere	Wohnung	als Michael	ein	Haus \sout{hat}. \label{germanvpellipsis5}\\
Ralph	has	a.\textsc{f} bigger.\textsc{f} flat	than Michael a.\textsc{f} house has\\
\glt `Ralph has a bigger flat than Michael a house.'}
\z
\z

The reason why (\ref{englishvpellipsis}) is possible is that the [E] \isi{feature} can be located on a v head in \ili{English} since the complement follows that v head: hence, the \isi{lexical verb} (\textit{has}) is elided and the object \isi{DP} stops \isi{deletion}. However, this is not possible in \ili{German} because the v head taking the \isi{VP} (\textit{ein Haus hat}) as its complement follows the \isi{VP} and even if an [E] \isi{feature} were located on this v, that would not (and could not) carry out \isi{ellipsis}.

The advantage of this analysis is that it connects the lack of the availability of the [E] \isi{feature} on a given \isi{functional head} to the relative position of that head, contrary to \citet{merchant2013}, who proposes that this is a lexical difference, in that \ili{English} has both an E\textsubscript{S} and an E\textsubscript{V} \isi{feature}, while \ili{German} lacks the E\textsubscript{V} \isi{feature} and has no VP-\isi{ellipsis} (but has \isi{sluicing}). Though the proposal of \citet{merchant2013} in this respect is descriptively adequate, it fails to link this property to some other, more general property of the grammar.

I propose that the reason why a head-final \isi{functional projection} cannot license \isi{ellipsis} is not because of a lexical difference from head-initial projections that would ban the appearance of an [E] \isi{feature} on a head-final \isi{vP}: it is simply that the \isi{PF} mechanism defined by the [E] \isi{feature} does not (and cannot) operate backwards. This also implies that there is essentially no restriction on the appearance of the [E] \isi{feature}: in principle, it can appear on the head of a head-final projection but it will have no effect on the final structure.

Again, I cannot examine these issues any further since it would necessarily involve constructions other than comparatives. What is important for us here is that \isi{ellipsis} in comparative subclauses seems to operate in a principled way, in that it is carried out by similar mechanisms in various languages, irrespective of whether these languages show \isi{Comparative Deletion} or not. The differences that do arise can be attributed to general requirements that follow either from the way syntax-prosody mapping works in the given language or from whether functional projections are head-initial or head-final.
