\title{A grammar of Tuatschin}
\subtitle{A Sursilvan Romansh dialect} 
\BackBody{This book is the first descriptive grammar of Tuatschin, a Sursilvan Romansh dialect spoken by approximately 800 people in the westernmost part of the Romansh territory, in the canton of Grisons in southeastern Switzerland. The description is mainly based on narratives and elicitation, collected during fieldwork conducted between 2016 and 2020. Besides the grammatical description, it also offers a variety of narratives produced by female and male native speakers between thirty and eighty years of age.}
% \dedication{Change dedication in localmetadata.tex}
\typesetter{Philippe Maurer-Cecchini, Sebastian Nordhoff}
\proofreader{Amir Ghorbanpour,
Andreas Hölzl,
Aviva Shimelman,
Christopher Straughn,
Craevschi Alexandru,
Jaime Peña,
Jeroen van de Weijer,
Konstantinos Sampanis,
Lachlan Mackenzie,
Melanie Röthlisberger,
Madeline Myers,
Russell Barlow,
Sandra Auderset,
Tihomir Rangelov,
Tom Bossuyt,
Yvonne Treis
}
\author{Philippe Maurer-Cecchini}
\BookDOI{10.5281/zenodo.5137647}
\renewcommand{\lsISBNdigital}{978-3-96110-318-8}
\renewcommand{\lsISBNhardcover}{978-3-98554-014-3}
\renewcommand{\lsSeries}{cogl}
\renewcommand{\lsSeriesNumber}{3}
\renewcommand{\lsID}{308}
 



