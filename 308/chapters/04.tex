\chapter{Verb phrase}


\section{The verb}\label{sec:4.1}
Tuatschin possesses \isi{impersonal}, \isi{intransitive} (\ref{ex:intrans1}), mono-, and \isi{ditransitive verbs}. \isi{Impersonal verb}s require an \isi{expletive pronoun} in \isi{subject} position (\ref{ex:impers1}). \isi{Monotransitive verbs} usually have a \isi{direct object} (\ref{ex:trans1} and \ref{ex:trans2}), but in rare cases they may have an \isi{indirect object} (\ref{ex:trans:indir})\footnote{\textit{udí da} is a calque from German. \textit{udí} `hear' is \textit{hören} in German, and belong is \textit{gehören}, which triggers dative case.}. Ditransitive verbs (\ref{ex:ditrans1} and \ref{ex:ditrans2}) have a direct and an \isi{indirect object}, the latter being marked by \textit{da/dad}.\footnote{As shown in \sectref{sec:3.2.1.2} and \sectref{sec:3.6.1}, the dative articles \textit{li} (see (\ref{ex:ditrans2}) as well as \textit{di} are obsolescent, and the \isi{dative marker} \textit{a}, which is used in rare cases, is a loan from \ili{Standard Sursilvan}.}. An exception is the verb \isi{\textit{dumandá}}, which has two direct objects (\ref{ex:trans:2DO}) (see \sectref{sec:4.2.2}). Note that if the two objects are pronominal, the object of asking is usually not mentioned (\ref{ex:trans:2DO}). The \isi{indirect object} usually precedes the \isi{direct object} (\ref{ex:ditrans2}).

\ea
\label{ex:impers1}
\gll Ad \textbf{i} \textbf{èr}’ è bitga úsit tg’ ins mava a scùlèta.\\
and  \textsc{expl} \textsc{cop.impf.3sg} also \textsc{neg} usage.\textsc{m.sg}  \textsc{rel} \textsc{gnr} go.\textsc{impf.3sg} to nursery\_school.\textsc{f.sg}\\
\glt `And it was not usual that one attended nursery school.' (Sadrún, m4, \sectref{sec:8.3})
\z

\ea
\label{ex:intrans1}
\gll  Quèl \textbf{durméva} sc’ in tajs.  \\
\textsc{dem.m.sg} sleep.\textsc{impf.3sg} like \textsc{indef.m.sg} badger\\
\glt `He used to sleep like a log.' (Sadrún, m4, \sectref{sec:8.3})
\z

\ea\label{ex:trans1}
\gll    In da Méjdal \textbf{vèv'} {\ob}\textbf{in}' \textbf{ura} \textbf{da} \textbf{Schwarzwald} \textbf{tg}' \textbf{èra} \textbf{rùta}{\cb}.\\
     one.\textsc{m.sg} of \textsc{pn} have.\textsc{impf.3sg} \textsc{\db{indef.f.sg}} clock of \textsc{pn} \textsc{rel} \textsc{cop.impf.3sg} break.\textsc{ptcp.f.sg}\\
\glt `An inhabitant of Medel had a clock from the Black Forest that was broken.' (Sadrún, \citealt[106]{Büchli1966})
\z

\ea\label{ex:trans2}
\gll    In dé […] \textbf{partgirava} in buép {\ob}\textbf{laṣ} \textbf{vacas}{\cb} sén Vòns.\\
     \textsc{indef.m.sg} day {} mind.\textsc{impf.3sg} \textsc{indef.m.sg} boy \textsc{\db{def.f.pl}} cow.\textsc{pl} up \textsc{pn}\\
\glt `One day [...] a boy was taking care of the cows at Vons.' (Sadrún, \citealt[103]{Büchli1966})
\z

\ea
\label{ex:trans:indir}
\gll    A quaj gang \textbf{udéva} {\ob}\textbf{dad} \textbf{òmaṣdús}{\cb}.\\
and \textsc{dem.m.sg} corridor belong.\textsc{impf.3sg} \textsc{\db{dat}} both.\textsc{m.pl}\\
\glt `And this corridor belonged to both [families].' (Sadrún, m1, \sectref{sec:8.2})
\z

\ea
\label{ex:ditrans1}
\gll    [...] {\ob}\textbf{mia} \textbf{carèzja}{\cb} tgu stù \textbf{dá} {\ob}\textbf{da} \textbf{quèlas}{\cb}.\\
{} \textsc{\db{poss.1sg.f.sg}} love \textsc{rel.1sg} must.\textsc{prs.1sg} give.\textsc{inf} \textsc{\db{dat}} \textsc{dem.f.pl}\\
\glt `[...] my love that I have to give them.' (Camischùlas, f6, \sectref{sec:8.4})
\z

\ea\label{ex:ditrans2}
\gll    […] Nossadùna lèva \textbf{dá} {\ob}\textbf{li} \textbf{gjuven} \textbf{préjr}{\cb} {\ob}\textbf{ina} \textbf{rancùnusciantscha} pal \textbf{survètsch}{\cb}.\\
   {}  Our\_Lady.\textsc{f.sg} want.\textsc{impf.3sg} give \textsc{\db{dat}} young.\textsc{m.sg} priest \textsc{\db{indef.f.sg}} mark\_of\_gratitude.\textsc{f.sg} for.\textsc{def.m.sg} favour \\
\glt `[…] the Holy Virgin wanted to give the young priest a mark of gratitude for the favour [he had done her].' (Bugnaj, \citealt[145]{Büchli1966})
\z

\ea
\label{ex:trans:2DO}
\gll  [...]  ina zagríndara […] \textbf{ò} \textbf{dumandau} {\ob}\textbf{la} \textbf{mùma} \textbf{da} \textbf{tgèsa}{\cb} {\ob}\textbf{in} \textbf{tgavégl} \textbf{da} \textbf{sia} \textbf{buéba}{\cb}. \\
{} \textsc{indef.f.sg} Yenish {} have.\textsc{prs.3sg}   ask.\textsc{ptcp.unm} \textsc{\db{def.f.sg}} mother of house.\textsc{f} \db{one} hair of \textsc{poss.3sg.f.sg} girl \\
\glt `[…] a Yenish woman [...] asked the mother of the house for one hair from her daughter.' (Bugnaj, \citealt[131]{Büchli1966})
\z


\subsection{Verbal morphology}\label{sec:4.1.1}
According to the ending of their infinitives, verbs can be divided into five classes:

\begin{itemize}

\item \textit{-á} (\textit{anflá} `find')
\item \textit{-è} (\textit{catschè} `hunt')
\item \textit{-ar} (\textit{métar} `put')
\item \textit{-í} (\textit{fugí} `flee')
\item  \textit{-aj} (\textit{tanaj} `hold')
\end{itemize}

The verbs ending in -\textit{aj} like \textit{savaj} `know', \textit{vulaj} `want', or \textit{tanaj} `hold'  are all irregular and well be presented in \sectref{sec:4.1.1.4}.

From a diachronic point of view, the \textit{-è}-class is a subclass of the \textit{-a}-class due to the general rule that \textit{a} becomes \textit{è} after a palatal consonant or glide.

The small number of regular verbs ending in \textit{-í} all have the stem extension \textit{-èsch} (\tabref{tab:reg.verb-i}). Some verbs of other conjugation classes also have this stem extension; in the oral corpus, this only concerns two verbs that end in \textit{-á} and one that ends in \textit{-è}: \textit{datá} `date' (\textit{ju/èla datèscha} `I/she dates'), \textit{discusjuná} `discuss' (\textit{ju/èla discusjunèscha} `I/she discusses'), and \textit{aprazjè} `appreciate' (\textit{ju/èla aprazjè\-scha} `I/she appreciates').

Tuatschin has three \isi{non-finite categories}: \isi{infinitive}, \isi{past participle}, and \isi{gerund}, whereby the \is{gerund}ger\-und is not in use in current speech.

Within the \isi{finite categories}, the language differentiates tense, aspect, and modal categories as well as simple,  compound, and doubly-compound categories.

The simple categories are \isi{present indicative}, \isi{present subjunctive}, \isi{imperfect indicative}, \isi{imperfect subjunctive}, \isi{direct conditional}, \isi{indirect conditional},\footnote{\textit{Direct} and \textit{indirect conditional} are terms used in Sursilvan Romansh grammars to refer to a \isi{conditional} which is used in direct in indirect speech, respectively  (\sectref{sec:4.1.2.2.10}).} and imperative.

The compound tenses are \isi{perfect indicative}, \isi{perfect subjunctive}, \isi{pluperfect indicative},  \isi{pluperfect subjunctive}, and \isi{future}. The compound tenses are formed with an \isi{auxiliary verb} (either \textit{èssar} `be', \textit{vaj} `have', or \textit{vagní} `come') and the \isi{past participle} or the \isi{infinitive}. 

The \isi{doubly-compound tenses} correspond to the perfect and the pluperfect, but with two past participles instead of one.

The \isi{personal ending} for the first person singular \isi{present} and \isi{imperfect indicative} is \textit{-a}, as in \textit{ju cònta} `I sing' and \textit{ju cantava} `I used to sing', but some irregular verbs lack this ending, as in  \textit{ju détsch} `I say', \textit{ju dùn} `I give', \textit{ju fétsch} `I do', \textit{ju végn} `I come', or \textit{ju vòn / ju mòn} `I go'. For further examples see \sectref{sec:4.1.1.4}.

\isi{Reflexive verbs} are built with the prefix \textit{sa-} in all persons, tenses, moods, and \isi{non-finite categories} and use the \isi{auxiliary verb} \textit{èssar} `be' for compound tenses: \textit{salavá} (infinitive) `wash (oneself)', \textit{ju salava} (present) `I wash', \textit{té salavassas} (direct conditional) `you (\textsc{sg}) would wash', \textit{nus èssan salavaj} (perfect) `we have washed', \textit{vus vagnís a salavá} (future) `you (\textsc{pl}) will wash', \textit{èlṣ èran salavaj} (pluperfect) `they had washed'.

According to the \DRG{1}{568}, the choice of \textit{esser} as \isi{auxiliary verb} for reflexives in \ili{Standard Sursilvan} is due to the prescriptive demand of Sursilvan grammarians since the 18th century. Nowadays speakers seek to conform to this rule, but in spoken Sursilvan, one still can find \textit{haver} as \isi{auxiliary} for \isi{reflexive} verbs.

The \isi{reflexive} verbs will not be treated this chapter, but their use will be presented in \sectref{sec:5.5.1} on \isi{reflexive voice}.

Verb forms that end in \textit{-n} in the first and third person singular \isi{present indicative} take a \isi{euphonic} \textit{-d} before /j/ (\ref{ex:euph1} and \ref{ex:euph2}) or a vowel (\ref{ex:euph3}).

\ea
\label{ex:euph1}
\gll  Api sjantar \textbf{sùnd} \textbf{ju} saṣjuṣ gjù [...].\\
and after be.\textsc{prs.1sg} \textsc{1sg} sit.\textsc{ptcp.m.sg} down\\
\glt `And then I sat down [...].' (Sadrún, m8, \sectref{sec:8.12})
\z

\ea
\label{ex:euph2}
\gll   \textbf{Sùnd} \textbf{juṣ} ah gjù Surajn [...]. \\
be.\textsc{prs.1sg} go.\textsc{ptcp.m.sg} eh down \textsc{pn}\\
\glt `[I] went eh down to Surrein [...].' (Sadrún, m4, \sectref{sec:8.3})
\z

\ea
\label{ex:euph3}
\gll  Api \textbf{prènd}’ \textbf{al} \textsc{PN} in cuntí ò da sac [...].\\
and take.\textsc{prs.3sg} \textsc{def.m.sg} \textsc{pn} \textsc{indef.m.sg} knife out of pocket.\textsc{m.sg}\\
\glt `And then \textsc{PN} takes a knife out of his pocket [...].' (Surajn, m7, \sectref{sec:8.17})
\z


\subsubsection{Auxiliary verbs}\label{sec:4.1.1.1}
The \isi{auxiliary verbs} \isi{\textit{èssar}} `be' (\tabref{tab:aux:èssar}) and \isi{\textit{vaj}} `have' (\tabref{tab:aux:vaj}) are used for \isi{compound tenses} (and \textit{vaj} also for \isi{doubly-compound} tenses), whereas \textit{vagní} `come' (\tabref{tab:aux:vagní}) is used for \isi{future}. In the following tables, only one \isi{compound tense} will be listed, the perfect; as for the \isi{doubly-compound tenses}, they are formed with the perfect or the imperfect of the auxiliary verb \textit{vaj}, the participle of \textit{vaj}, and the participle of the main verb and need not be listed. Examples will be given in \sectref{sec:4.1.2.2.6}.

\begin{table}
	\caption{Auxiliary verb \textit{èssar} `be'}
	\label{tab:aux:èssar}
	\begin{tabular}{llllll}
		\lsptoprule
		& \textsc{inf}  & \textsc{ptcp.m}  & \textsc{ptcp.f}  &  \textsc{ger}\\
		\midrule
		&\textit{èssar} &\textit{stauṣ}, \textit{staj}  & \textit{stada}, \textit{stadaṣ} & \textit{èssè̱n}\\
		\lsptoprule
	&\textsc{prs.ind}  &\textsc{impf.ind} & \textsc{prf.ind} & \textsc{fut}\\
		\midrule
		\textsc{1sg} &\textit{sùn} & \textit{èra} &\textit{sùn stauṣ/stada} &\textit{végn ád èssar}\\
		\textsc{2sg} &\textit{ajṣ} &\textit{èraṣ} &\textit{ajs stauṣ/stada} & \textit{végnaṣ ád èssar}\\
		\textsc{3sg} &\textit{è} & \textit{èra} & \textit{è stauṣ/stada} &\textit{végn ád èssar}\\
		\textsc{1pl} &\textit{èssan} &\textit{èran} & \textit{èssan staj/stadaṣ} &\textit{vagnín ád èssar}\\
		\textsc{2pl} &\textit{èssaṣ} & \textit{èraṣ} & \textit{èssas staj/stadaṣ} &\textit{vagníṣ ád èssar}\\
		\textsc{3pl}& \textit{èn} & \textit{èran} & \textit{èn staj/stadaṣ} & \textit{végnan ád èssar}\\
		\lsptoprule
		&\textsc{prs.sbjv} & \textsc{impf.sbjv}  &\textsc{cond.direct} & \textsc{cond.indirect}\\
		\midrule
		\textsc{1sg} & \textit{séjgi/ségi} & \textit{èri} & \textit{fùṣ} & \textit{fùssi}\\
		\textsc{2sg} & \textit{sé̱jgiaṣ/sé̱gias} & \textit{è̱riaṣ} & \textit{fùssaṣ} & \textit{fù̱ssiaṣ}\\
		\textsc{3sg} & \textit{séjgi/séj/ségi} & \textit{èri} & \textit{fùṣ} & \textit{fùssi}\\
		\textsc{1pl} & \textit{sé̱jgian, sé̱gian} & \textit{è̱rian} & \textit{fùssan} & \textit{fù̱ssian}\\
		\textsc{2pl} &  \textit{sé̱jgiaṣ, sé̱giaṣ} &  \textit{è̱riaṣ} & \textit{fùssaṣ} & \textit{fù̱ssiaṣ}\\
		\textsc{3pl} & \textit{sé̱jgian, sé̱gian} & \textit{è̱rian} & \textit{fùssan} & \textit{fù̱ssian}\\
		\lspbottomrule
	\end{tabular}
\end{table}

In the third person singular and plural present and \isi{imperfect}, the verb \textit{èssar} `be' has a special form when there is \isi{subject inversion} (which includes polar interrogatives): \textit{ásaj}, \textit{ṣaj} \textit{ṣè}, and \textit{ṣèn}, as well as \textit{ṣèra} and \textit{ṣèran} (\ref{ex:se1}--\ref{ex:se3}). These forms go back to \ili{Standard Sursilvan}, where \textit{igl ei} `\textsc{expl} + \textsc{cop.prs.3sg}' is realised as \textit{eiṣ ei} `is it' with \isi{subject inversion} (/ajzaj/ > /azaj/ > /zaj/ < /zɛ/). In contrast to \ili{Standard Sursilvan}, the form of the copula does not include an \isi{expletive pronoun} in (\ref{ex:se1}--\ref{ex:se3}).

\ea
\label{ex:se1}
\gll    Òz \textbf{ṣè} quaj ah, òz \textbf{ṣèni} schòn autar, òz \textbf{ṣèn} \textbf{aj} ... la stradún. \\
today \textsc{cop.prs.3sg} \textsc{dem.unm} eh today \textsc{cop.prs.3pl.3pl} in\_fact different today \textsc{cop.prs.3pl} \textsc{3pl} {} \textsc{def.f.sg} street.\textsc{m.sg.augm} \\
\glt `Nowadays this is, eh, as a matter of fact they are different, nowadays they are [called] ... the «big street».' (Ruèras, m2, \sectref{sec:8.13})
\z

\ea
\label{ex:se2}
\gll    A Cazis \textbf{ṣèra} \textbf{quaj} al madèm.\\
in \textsc{pn} \textsc{cop.impf.3sg} \textsc{dem.unm} \textsc{def.m.sg} same\\
\glt `In Cazas this was the same thing.' (Camischùlas, f6, \sectref{sec:8.4})
\z

\ea
\label{ex:se3}
\gll \textbf{Ṣè} quaj usché?\\
\textsc{cop.prs.3sg} \textsc{dem.unm} so\\
\glt `Is this so?' (Sadrún, m5)
\z

Children and very occasionally also older people generalise this form and use it without \isi{subject inversion} (\ref{ex:se4}--\ref{ex:se6}).

\ea
\label{ex:se4}
\gll \textbf{I} \textbf{ṣè} vit.\\
\textsc{expl} \textsc{cop.prs.3sg} empty.\textsc{adj.unm}\\
\glt `It is empty.' (Sadrún, m8)
\z

\ea
\label{ex:se5}
\gll [...] \textbf{i} \textbf{ṣèra} òns nùca tg’ èra aua [...].\\
{} \textsc{expl} \textsc{exist.impf.3sg} year.\textsc{m.pl} where \textsc{rel} \textsc{exist.impf.3sg} water \\
\glt `[...] there were years with rain [...].' (Sadrún, m5, \sectref{sec:8.9})
\z

\ea
\label{ex:se6}
\gll [...] anadas \textbf{tga} \textbf{ṣèn} uschéa [...].\\
{} age\_group.\textsc{f.pl} \textsc{rel} \textsc{cop.prs.3pl} so\\
\glt `[...] age groups which are like that [...].' (Sadrún, m9, \sectref{sec:8.15})
\z

In (\ref{ex:copexpl}) the form of the copula \textit{ṣè} does include an \isi{expletive pronoun} as in the \ili{Standard Sursilvan} form noted above. 

\ea
\label{ex:copexpl}
\gll Basta, ju sùn id’ ál trèn, tòcan gjù Sògn Gagl \textbf{ṣè} bigja da fá bjè falju [...].   \\
enough \textsc{1sg} be.\textsc{prs.1sg} go.\textsc{ptcp.f.sg} to.\textsc{def.m.sg} train until down \textsc{pn} {} \textsc{cop.prs.3sg.expl} \textsc{neg} \textsc{comp} make.\textsc{inf} much wrong.\textsc{adj.unm}\\
\glt `Enough. I went to the train, to St. Gallen there is not much you could do wrong [...].' (Ruèras, f7, \sectref{sec:8.14})
\z

\begin{table}
\caption{Auxiliary verb \textit{vaj} `have'}
\label{tab:aux:vaj}
 \begin{tabular}{llllll}
 
  \lsptoprule
& \textsc{inf}  & \textsc{ptcp.m.unm} \\
  \midrule
&  \textit{vaj} &\textit{gju} \\
     
  \lsptoprule
&\textsc{prs.ind}  &\textsc{impf.ind} & \textsc{prf.ind} & \textsc{fut}\\
   \midrule
\textsc{1sg} &\textit{a} & \textit{vèva} & \textit{a gju} & \textit{végn a vaj}\\
\textsc{2sg} &\textit{aṣ} & \textit{vèvaṣ} & \textit{aṣ gju} & \textit{végnaṣ a vaj}\\
\textsc{3sg} &\textit{ò} & \textit{vèva} & \textit{ò gju} &\textit{végn a vaj}\\
\textsc{1pl} &\textit{vajn} & \textit{vèvan} & \textit{vajn gju} &\textit{vagnín a vaj}\\
\textsc{2pl} &\textit{vajṣ} & \textit{vèvaṣ} & \textit{vajṣ gju} &\textit{vagníṣ a vaj}\\
\textsc{3pl}& \textit{òn} & \textit{vèvan} & \textit{òn gju} &\textit{végnan a vaj}\\

 \lsptoprule
&\textsc{prs.sbjv} & \textsc{impf.sbjv}  &\textsc{cond.direct} & \textsc{cond.indirect} & \textsc{imp} \\
\midrule
\textsc{1sg} & \textit{vagi}& \textit{vèvi} & \textit{vèṣ} & \textit{vèssi}\\
\textsc{2sg} & \textit{vájaṣ}& \textit{vèviaṣ} & \textit{vèssaṣ} & \textit{vè̱ssiaṣ} &  \textit{vajaṣ}\\
\textsc{3sg} & \textit{vagi} & \textit{vèvi} & \textit{vèṣ} & \textit{vèssi}\\
\textsc{1pl} & \textit{vájan}& \textit{vèvian} & \textit{vèssan} & \textit{vè̱ssian}\\
\textsc{2pl} & \textit{vájaṣ}& \textit{vèviaṣ} & \textit{vèssaṣ}& \textit{vèssiaṣ} &  \textit{vajaṣ}\\
\textsc{3pl} & \textit{vájan}& \textit{vèvian} & \textit{vèssan} & \textit{vè̱ssian}\\
  \lspbottomrule
 \end{tabular}
\end{table}

\begin{table}
\caption{Auxiliary verb \textit{vagní} `come'}
\label{tab:aux:vagní}
\fittable{
 \begin{tabular}{llllll} 
  \lsptoprule
& \textsc{inf}  & \textsc{ptcp.m}  & \textsc{ptcp.f}\\
  \midrule
&\textit{vagní} &\textit{vagnúṣ}, \textit{vagní} & \textit{vagnida}, \textit{vagnidaṣ}\\
   
  \lsptoprule
& \textsc{prs.ind}  &\textsc{impf.ind} & \textsc{prf.ind} & \textsc{fut}\\
   \midrule
\textsc{1sg} & \textit{végn} & \textit{vagnéva} &\textit{sùn vagnúṣ/vagnida} &\textit{végn a vagní}\\
\textsc{2sg} &\textit{végnaṣ} &\textit{vagnévaṣ} & \textit{ajṣ vagnúṣ/vagnida} & \textit{végnaṣ a vagní}\\
\textsc{3sg} &\textit{végn} & \textit{vagnéva}\footnote{\textit{Vignéva} is also used, but not frequent.} &\textit{è vagnúṣ/vagnida} &\textit{végn a vagní}\\
\textsc{1pl} &\textit{vagnín} &\textit{vagnévan} &\textit{èssan vagní/vagnidaṣ} &\textit{vagnín a vagní}\\
\textsc{2pl} &\textit{vagníṣ} & \textit{vagnévaṣ} &\textit{èssaṣ vagní/vagnidaṣ} &\textit{vagníṣ a vagní}\\
\textsc{3pl}& \textit{végnan} & \textit{vagnévan} &\textit{èn vagní/vagnidaṣ} &\textit{végnan a vagní}\\

 \lsptoprule
&\textsc{prs.sbjv} & \textsc{impf.sbjv}  &\textsc{cond.direct} & \textsc{cond.indirect} & \textsc{imp}\\
\midrule
\textsc{1sg} & \textit{végni}& \textit{vagnévi} & \textit{vagnéṣ}& \textit{vagnéssi}\\
\textsc{2sg} & \textit{vé̱gniaṣ} & \textit{vagné̱viaṣ} & \textit{vagnéssaṣ} & \textit{vagné̱ssiaṣ} & \textit{nò}\\
\textsc{3sg} & \textit{végni} & \textit{vagnévi} & \textit{vagnéṣ} & \textit{vagnéssi}\\
\textsc{1pl} & \textit{vé̱gnian} & \textit{vagné̱vian} & \textit{vagnéssan} & \textit{vagné̱ssian}\\
\textsc{2pl} & \textit{vé̱gniaṣ}& \textit{vagné̱viaṣ} & \textit{vagnéssaṣ} & \textit{vagné̱ssiaṣ} & \textit{vagní}\\
\textsc{3pl} & \textit{vé̱gnian} & \textit{vagné̱vian}& \textit{vagnéssan} & \textit{vagné̱ssian}\\
  \lspbottomrule
 \end{tabular}
 }
\end{table}


\subsubsection{Regular verbs}\label{4.1.1.2}
Future tense is always built with the \isi{auxiliary verb} \isi{\textit{vagní}} `come', but it is not used in normal daily speech, where almost exclusively present tense is used for \isi{future} reference. Therefore \isi{future} tense will not be listed in the table of the regular verbs (as well as of the irregular verbs). The same holds for the \isi{gerund}, which was used by traditional story tellers until some decades ago, but which is not in use any more (for examples see \sectref{sec:4.1.2.1.2}).

As mentioned above, the \textit{è--}conjugation has split from the original \textit{á}--conjuga\-tion (< Latin \textsc{-are})  because of the presence of a preceding palatal consonant. \tabref{tab:èconj} lists some examples of \textit{è}--verbs with their Standard Sursilvan counterparts. Note that the final \textit{--r} of the infinitives in \ili{Standard Sursilvan} orthography is not pronounced in any Sursilvan variety.

\begin{table}
\caption{Tuatschin verbs ending in \textit{-è} with their Standard Sursilvan equivalents}
\label{tab:èconj}
 \begin{tabular}{llll}
 \lsptoprule
&\textsc{Tuatschin}  & \textsc{Sursilvan}  & \textsc{English} \\
  \midrule
 ʎ & \textit{magliè} &\textit{magliar}& `eat' \\
ʥ&\textit{cargè}&\textit{cargar}&`carry'\\
ʨ&\textit{spatgè}&\textit{spitgar}&`wait'\\
ʧ&\textit{catschè}&\textit{catschar}&`hunt'\\
ʃ&\textit{schè}&\textit{schar}&`let'\\
j&\textit{sijè}&\textit{segar}&`mow'\\   
 \lspbottomrule
 \end{tabular}
\end{table}

\subsubsubsection{Suffixes of the regular finite verb forms}\label{sec:4.1.1.2.1}
The suffixes of the regular verbs (\tabref{suffixesfiniteverbs}) occur in the following order : tense{\slash}aspect – mood – person. Since there are different zero-marked categories, this order is only realised in  the imperfect subjunctive (with the exception of first and third person singular, which are zero-marked).

\begin{sidewaystable}
	\caption{Tense/Aspect, mood and personal suffixes of the finite regular verb forms}
    \label{suffixesfiniteverbs}. 

	\begin{tabularx}{\textwidth}{p{4cm}lllllllllll}
		\lsptoprule
	& \textit{infinitive}	 &\textsc{tense/aspect} &\textsc{mood} &\textsc{person} \\
	& & & & \textsc{1sg} & \textsc{2sg} & \textsc{3sg} & \textsc{1pl} & \textsc{2pl} & \textsc{3pl} \\
		\midrule
		present indicative & all & \textit{-Ø} & \textit{-Ø} & \textit{-a}, & \textit{-as}, & \textit{-a,} & \textit{-ájn}, & \textit{-ájs}, & -\textit{an}\\
		present subjunctive & all & \textit{-Ø} & \textit{-i} & \textit{-Ø},& \textit{-as}, & \textit{-Ø}, & \textit{-an}, & \textit{-a}s, & -\textit{an}\\
		imperfect indicative &\textit{ á, -è} & \textit{-áv} & \textit{-Ø} & \textit{-a}, & \textit{-as}, & \textit{-a}, & \textit{-an}, & \textit{-as},  & \textit{-an}\\
		& \textit{ar, í} & -\textit{év} & \textit{-Ø} & \textit{-a}, & \textit{-as}, & -\textit{a}, & \textit{-an}, & \textit{-as}, & -\textit{an}\\
		imperfect subjunctive &\textit{ á, è} & \textit{-áv} & \textit{-i} & \textit{-Ø}, & \textit{-as}, & \textit{-Ø}, & \textit{-an}, & \textit{-as}, & \textit{-an}\\
		& \textit{ar, í} &\textit{-év} & \textit{-i} & \textit{-Ø}, & \textit{-as}, & \textit{-Ø}, & \textit{-an}, & \textit{-as}, & \textit{-an}\\
		direct conditional & \textit{á, è} & \textit{-Ø} & \textit{-ás} & \textit{-Ø}, &\textit{-as}, & \textit{-Ø}, & \textit{-an}, & -\textit{as}, & -\textit{an}\\
       &  \textit{ar, í} & \textit{-Ø} & \textit{-és} & \textit{-Ø}, &\textit{-as}, & \textit{-Ø}, & \textit{-an}, & -\textit{as}, & -\textit{an}\\
		indirect conditional & \textit{á, è} & \textit{-Ø} & \textit{-áss-i} &\textit{-Ø}, & \textit{-as}, &\textit{-Ø}, & \textit{-an}, & \textit{-as}, & \textit{-an}\\
		& \textit{ar, í} & \textit{-Ø} & \textit{-éss-i} & \textit{-Ø}, & \textit{-és}, & \textit{-Ø}, & -\textit{an}, & \textit{-as}, & \textit{-an}\\
	
				\lspbottomrule
	\end{tabularx}
\end{sidewaystable} 

Aspect is not marked independently from tense ; the imperfect suffix \textit{-áv/}\textit{-év} is a portmanteau category which conflates past tense and imperfective aspect. Perfective aspect is expressed by the compound tenses perfect and pluperfect ; a preterit does not or does not exist any more in Tuatschin.

Morphologically, the indirect conditional represents a combination of the suffix of the conditional \textit{-as-/-és-} with the suffix of the subjunctive \textit{-i-}. The indirect conditional is used above all in indirect speech, which  triggers the use of subjunctive mood. For examples, see \sectref{sec:4.1.2.2.10}.

The stem extension \textit{-èsch-} precedes the the tense/aspect, mood, and person markers (see \tabref{tab:reg.verb-i} for examples).

\newpage
\tabref{suffixesfiniteverbs} shows different cases of syncretisms :

 \begin{itemize}
\item second person singular and third person plural are marked in the same way in all cases
\item first and second person plural are marked in the same way in all categories except for present indicative
\end{itemize}

There are also cases of zero-marked categories:

\begin{itemize}
\item present indicative is not marked for tense nor for mood
\item present subjunctive, direct conditional, and indirect conditional are not marked for tense
\item first and third person singular are not marked in subjunctive and conditional mood

\end{itemize}

Regarding the \isi{imperfect}, both indicative and subjunctive, as well as the direct and indirect conditional, the \textit{-á}-conjugation patterns with the \textit{-è}-conjugation, and, inversely, the \textit{-ar}-conjugation patterns with the \textit{-í}-conjugation. In these tenses and moods,  the verbs \isi{ending} in \textit{-á} and \textit{-è} have an \textit{-a} whereas the verbs ending in \textit{-ar} and in \textit{í} have an \textit{-é}. Many irregular verbs have  an \textit{-è} in these forms, as for instance \textit{ju dèva} 'I gave' (\textit{dá} `give') or \textit{ju lès} 'I would like' (\textit{vulaj}) `want' (see \sectref{sec:4.1.1.4})

\subsubsubsection{Paradigms of the regular verbs}\label{4.1.1.2.2}
\tabref{conja} and \tabref{conjè} illustrate the two conjugations deriving from the Latin first conjugation. The difference between the two conjugations concerns above all the \isi{infinitive} and the feminine \isi{past participle}, which ends in \textit{-èda} (vs \textit{-ada} in the \textit{a-}conjugation). The ending of the masculine form is the same in both conjugations (\textit{-au}). The \isi{imperfect} morpheme of the \textit{è}-conjugation is normally \textit{-áv-}, but some verbs have \textit{-èv-}, for example \textit{astgè} `be allowed', as in \textit{nuṣ \textbf{astgèvan} fá} `we were allowed to do' (\sectref{sec:8.4}) or \textit{schè} `let', as in \textit{a \textbf{schèvan} dá quèlas scúaṣ a da quaj} `and [we] would let these brooms and so on fall down' (\sectref{sec:8.6}). The choice of the ending \textit{-èv-} vs \textit{-av-} depends on the verb.



\begin{table}
	\caption{Regular verbs ending in \textit{-á}}
	\label{conja}
	\begin{tabularx}{.7\textwidth}{llll}
		
		\lsptoprule
		\textsc{inf} & & \textsc{ptcp.m}  & \textsc{ptcp.f}\\
		\midrule
		\textit{gidá} & `help' & \textit{gidau}, \textit{gidauṣ} & \textit{gidada}, \textit{gidadaṣ}\\
		\lspbottomrule  
	\end{tabularx}
	
	\medskip
	
	\begin{tabularx}{\textwidth}{p{2cm}lllll}
		\lsptoprule
		\textsc{sbj.pron} &\textsc{prs.ind} &\textsc{prs.sbjv} &\textsc{impf.ind} & \textsc{impf.sbjv} &\textsc{prf.ind}\\
		\midrule
		\textsc{1sg} & \textit{gida} & \textit{gidi} & \textit{gidava} & \textit{gidavi} & \textit{a gidau}  \\
		\textsc{2sg} & \textit{gidaṣ} & \textit{gídiaṣ} & \textit{gidavaṣ} & \textit{gidáviaṣ} & \textit{aṣ gidau}\\
		\textsc{3sg} & \textit{gida} & \textit{gidi} & \textit{gidava} & \textit{gidavi}  & \textit{ò gidau} \\
		\textsc{1pl} & \textit{gidajn} & \textit{gídian} & \textit{gidavan} & \textit{gidávian} & \textit{vajn gidau} \\
		\textsc{2pl} & \textit{gidajṣ} & \textit{gídiaṣ} & \textit{gidavaṣ}  & \textit{gidáviaṣ} & \textit{vajṣ gidau} \\
		\textsc{3pl} & \textit{gidan}  & \textit{gídian} & \textit{gidavan} & \textit{gidávian} & \textit{òn gidau}\\
		\lspbottomrule
	\end{tabularx}
	
	\medskip
	
	\begin{tabularx} {\textwidth}{p{2cm}XXXX}
		\lsptoprule
		\textsc{sbj.pron} &\textsc{cond.direct} &  \textsc{cond.indirect} & \textsc{imp}\\
		\midrule
		\textsc{1sg} & \textit{gidáṣ} & \textit{gidassi}\\
		\textsc{2sg} & \textit{gidassaṣ} & \textit{gidássiaṣ} & \textit{gida}\\
		\textsc{3sg} & \textit{gidáṣ} &  \textit{gidassi}\\
		\textsc{1pl} & \textit{gidassan} & \textit{gidássian}\\
		\textsc{2pl} & \textit{gidassaṣ} & \textit{gidássiaṣ}  & \textit{gidaj}\\
		\textsc{3pl} & \textit{gidassan}  & \textit{gidássian} \\
		\lspbottomrule
	\end{tabularx} 
\end{table}



\begin{table}
\caption{Regular verbs ending in \textit{-è}}
\label{conjè}
 \begin{tabularx}{\textwidth}{XXXl}
 
  \lsptoprule
  \textsc{inf}  && \textsc{ptcp.m}  & \textsc{ptcp.f}\\
  \midrule
  \textit{\textbf{catschè}} & `hunt' & \textit{catschau}, \textit{catschaus} & \textit{\textbf{catschèda}, \textit{\textbf{catschèdas}}}\\
  \lspbottomrule  
  \end{tabularx}
  
  \medskip
  
 \begin{tabularx}{\textwidth}{Xlllll}
  \lsptoprule
& \textsc{prs.ind} & \textsc{prs.sbjv} & \textsc{impf.ind} & \textsc{impf.sbjv} &\textsc{prf.ind}\\
 \midrule
\textsc{1sg} &\textit{catscha} & \textit{catschi}&\textit{catschava} &\textit{catschavi} & \textit{a catschau}\\
\textsc{2sg} &\textit{catschaṣ} & \textit{cátschiaṣ}&\textit{catschavaṣ} &\textit{catscháviaṣ} &\textit{aṣ catschau}\\
\textsc{3sg}  & \textit{catscha} & \textit{catschi} & \textit{catschava} & \textit{catschavi} & \textit{ò catschau}\\
\textsc{1pl} & \textit{catschajn} & \textit{cátschian} & \textit{catschavan} &\textit{catschávian} & \textit {vajn catschau}\\
\textsc{2pl} &\textit{catschajṣ} &\textit{cátschiaṣ} & \textit{catschavaṣ} &\textit{catscháviaṣ} &\textit{vajṣ catschau}\\
\textsc{3pl} & \textit{catschan} &\textit{cátschian} &\textit{catschavan} &\textit{catschávian} &\textit{òn catschau}\\

  \lspbottomrule
\end{tabularx}

\medskip

\begin{tabularx} {\textwidth}{XXXl}
 \lsptoprule
  &\textsc{cond.direct} &  \textsc{cond.indirect}  &\textsc{imp}\\
\midrule
\textsc{1sg} & \textit{catscháṣ}&\textit{catschassi} &\\
\textsc{2sg}  & \textit{catschassaṣ}&\textit{catschássiaṣ} &\textit{catscha}\\
\textsc{3sg} & \textit{catscháṣ} &\textit{catschassi} &\\
\textsc{1pl} & \textit{catschassan} & \textit{catschássian} &\\
\textsc{2pl} & \textit{catschassaṣ} & \textit{catschássiaṣ} &\textit{catschaj}\\
\textsc{3pl} & \textit{catschassan} & \textit{catschássian}&\\

  \lspbottomrule
 \end{tabularx} 
\end{table}

The ending in -\textit{a} of the first person singular present and \isi{imperfect indicative} is typical of Tuatschin Sursilvan. The standard ending in Sursilvan is -\textit{el} (\textit{jeu giavisch-el} `I wish'); in the DRG, however, I found one example of \textit{-a} in the variety of Riein, a village situated in the Lumnezia valley (\ref{rieina}).

\ea\label{rieina}
\gll  Jeu giavisch-\textbf{a} bien cunfiert.  \\
     \textsc{1sg} wish-\textsc{prs.1sg} good consolation.\textsc{m.sg}\\
\glt `My heartfelt sympathy.' (Sursilvan, Riein, \DRGoK{4}{327})
\z

This form was current in this local variety of Sursilvan for first person singular present and \isi{imperfect}, as the following forms show: \textit{jeu astga} `I am allowed to', \textit{jeu gneva} `I used to come', \textit{jeu era} `I was', \textit{jeu suna} `I play (an instrument)', \textit{jeu sunava} `I used to play', and so on (examples taken out of the\textit{ Questiunari principal} of the DRG, recorded between 1900 and 1920; Ursin Lutz p.c., 2017/04/19).


\tabref{conjar1} lists the verbs ending in \textit{-ar}. Many verbs of this conjugation are built as in \tabref{conjar1} but have one irregular form: the \isi{past participle}. Some examples are \textit{árdar} `burn', \textit{árvar} `open', and \textit{bétar} `throw', whose participles are \textit{ars/arsa}, \textit{aviart/avjarta}, and \textit{béz/béza}. A list of irregular verbs ending in \textit{-ˈar} is given in \tabref{stemaltvar}.


\begin{table}
	\caption{Regular verbs ending in \textit{-ar}}
	\label{conjar1}
	\begin{tabularx}{\textwidth}{XXXl}
		
		\lsptoprule
		\textsc{inf} & & \textsc{ptcp.m}  & \textsc{ptcp.f}\\
		\midrule
		\textit{bátar} & `beat' & \textit{batju}, \textit{batjuṣ} & \textit{batida}, \textit{batidaṣ}\\
		\lspbottomrule  
	\end{tabularx}
	
	\medskip
	
	\begin{tabularx}{\textwidth}{Xllllll}
		\lsptoprule
		&\textsc{prs.ind} &\textsc{prs.sbjv} &\textsc{impf.ind} & \textsc{impf.sbjv} &\textsc{prf.ind}\\
		\midrule
		\textsc{1sg} & \textit{bata} & \textit{bati} & \textit{batéva} & \textit{batévi} &  \textit{a batju}\\
		\textsc{2sg} & \textit{bataṣ} & \textit{bátiaṣ} & \textit{batévaṣ} & \textit{baté̱viaṣ} & \textit{aṣ batju}\\
		\textsc{3sg} & \textit{bata} & \textit{bati} & \textit{batéva}  & \textit{batévi} & \textit{ò batju}\\
		\textsc{1pl} & \textit{batín} & \textit{bátian} & \textit{batévan} & \textit{baté̱vian} & \textit{vajn batju}\\
		\textsc{2pl} & \textit{batíṣ} & \textit{bátiaṣ} &  \textit{batévaṣ} & \textit{baté̱viaṣ} & \textit{vajṣ batju} \\
		\textsc{3pl} & \textit{batan} & \textit{bátian} & \textit{batévan} & \textit{baté̱vian} & \textit{òn batju} \\
		\lspbottomrule
	\end{tabularx}
	
	\medskip
	
	\begin{tabularx} {\textwidth}{Xlllll}
		\lsptoprule
		&\textsc{cond.direct} &  \textsc{cond.indirect}    &\textsc{imp}\\
		\midrule
		\textsc{1sg} & \textit{batéṣ} & \textit{batéssi} \\
		\textsc{2sg} & \textit{batéssaṣ} & \textit{baté̱ssiaṣ} & \textit{bata}\\
		\textsc{3sg} & \textit{batéṣ} & \textit{batéssi}\\
		\textsc{1pl} & \textit{batéssan} & \textit{baté̱ssian} \\
		\textsc{2pl} & \textit{batéssaṣ} & \textit{baté̱ssiaṣ} & \textit{batí} \\
		\textsc{3pl} & \textit{batéssan} & \textit{baté̱ssian}\\
		\lspbottomrule
	\end{tabularx} 
\end{table}

As mentioned above, verbs ending in \textit{-í} which are conjugated regularly  take \textit{-èsch} in the forms of the present tense when the stem is stressed (\tabref{tab:reg.verb-i}).


\begin{table}
	\caption{Regular verbs ending in \textit{-í} with  \textit{-èsch}}
	\label{tab:reg.verb-i}
	\begin{tabularx}{.7\textwidth}{llll}
		
		\lsptoprule
		\textsc{inf} & & \textsc{ptcp.m}  & \textsc{ptcp.f}  \\
		\midrule
		\textit{finí} & `finish' & \textit{finju}, \textit{finjus} & \textit{finida}, \textit{finidas} \\
		\lspbottomrule  
	\end{tabularx}
	
	\medskip
	
	\begin{tabularx}{\textwidth}{p{1,7cm}lllll}
		\lsptoprule
		&\textsc{prs.ind} &\textsc{prs.sbjv} &\textsc{impf.ind} & \textsc{impf.sbjv} &\textsc{prf.ind}\\
		\midrule
		\textsc{1sg} & \textit{finèscha} & \textit{finèschi} & \textit{finéva} & \textit{finévi} & \textit{a finju} \\
		\textsc{2sg} & \textit{finèschaṣ} & \textit{finèschiaṣ} & \textit{finévaṣ} & \textit{finéviaṣ} & \textit{aṣ finju}\\
		\textsc{3sg} & \textit{finèscha}  & \textit{finèschi} & \textit{finéva} & \textit{finévi} & \textit{ò finju}\\
		\textsc{1pl} & \textit{finín} & \textit{finían} & \textit{finévan} & \textit{finévian} &\textit{vajn finju}\\
		\textsc{2pl} & \textit{finíṣ} & \textit{finíaṣ} & \textit{finévaṣ} & \textit{finé̱viaṣ} & \textit{vajṣ finju} \\
		\textsc{3pl} & \textit{finèschan}  & \textit{finèschian} & \textit{finévan} & \textit{finévian} & \textit{òn finju}\\
		\lspbottomrule
	\end{tabularx}
	
	\medskip
	
	\begin{tabularx} {\textwidth}{p{2cm}XXXX}
		\lsptoprule
		&\textsc{cond.direct} &  \textsc{cond.indirect} &\textsc{imp}\\
		\midrule
		\textsc{1sg} & \textit{finéṣ} & \textit{finéssi} \\
		\textsc{2sg} & \textit{finéssaṣ} &\textit{finéssiaṣ}  &  \textit{finèscha}\\
		\textsc{3sg} & \textit{finéṣ}  & \textit{finéssi}\\
		\textsc{1pl} & \textit{finéssan} &  \textit{finéssian}\\
		\textsc{2pl} &  \textit{finéssaṣ} & \textit{finéssiaṣ} & \textit{finí}\\
		\textsc{3pl} & \textit{finéssan} & \textit{finéssian}\\
		\lspbottomrule
	\end{tabularx} 
\end{table}


\subsubsection{Verbs with stem alternations}\label{4.1.1.3}
The lists presented in this section contain some verbs ending in \textit{-á}, \textit{-è}, \textit{-ˈar}, and \textit{-í} which display a change in their stem. The alternation depends on whether the stem is stressed or not. Most verbs ending in \textit{-aj} are irregular and will therefore be presented in \sectref{sec:4.1.1.4}.

For \isi{present indicative} and subjunctive, only the first person singular and plural will be indicated; for \isi{imperfect indicative} and subjunctive as well as for \isi{conditional}, only the first person singular will be noted. For the verbs ending in \textit{-á} the \isi{imperfect indicative} and subjunctive will not be listed since these forms are regular, and since the \isi{future}, the \isi{imperfect subjunctive}, and the \isi{gerund} are not used or only rarely used, they will not be mentioned as well.

For reasons of space the imperative will not be indicated in the following tables, but the second person singular imperative corresponds to the third person singular \isi{present indicative}, and the second person plural imperative corresponds to the second person plural \isi{present indicative} without its final \textit{-s}: \textit{èla cònta} `she sings' vs \textit{Cònta!} `Sing (sg)!', \textit{vus cantájs} `you (pl) sing' vs \textit{Cantaj!} `Sing (pl)!'.

The verbs ending in \textit{-á} listed in \tabref{stemalta1}--\tabref{stemalta3} show the following vocalic stem alternations:

\begin{itemize}
	\item a → aj (\textit{zavrá} `separate, sort out' → \textit{zajvra}), → au (\textit{ruassá} `rest' → \textit{ruaussa}), → ja (\textit{anzardá} `aerate'→ \textit{anzjarda}), → éj (\textit{lavá sé} `get up' → \textit{léjva sé}), → ia (\textit{samjá} `dream' → \textit{siamja}), → ò (\textit{sahaná},  `appreciate' → \textit{sahòna}), → u (\textit{cugljaná} `cheat' → \textit{cugljuna})
	\item i → aj (\textit{piná} `prepare' → \textit{pajna}), → é (\textit{cudizá} `provoke' → \textit{cudéza})
	\item u → au (\textit{antupá} `meet' → \textit{antaupa}), → íu (\textit{suá} `sweat' → \textit{síua}) → ò (\textit{dustá} `keep away' → \textit{dòsta}), → ù (\textit{angulá} `steal' → \textit{angùla}), → ué (\textit{cuzá} `last' → \textit{cuéza})
	\item u ... a → a ... ò (\textit{cumandá} `order' → \textit{camònda}), u ... a → a ... ù (\textit{scurṣalá} `sledge' → \textit{scarṣùla}), u ... a → a ... u (\textit{rumplaná} `rumble' → \textit{rampluna})
\end{itemize}

Metathesis occurs with \textit{r} in the following cases, with or without change in the vowel:

\begin{itemize}
	\item ar → ra (\textit{barsá} `roast' → \textit{brassa}), ar ... réj → (\textit{fardá} `smell' → \textit{fréjda})
	\item ur → rù (\textit{curdá} `fall' → \textit{crùda})
\end{itemize}


\begin{sidewaystable}
	\caption{Verbs ending in \textit{-á}, first part}
	\label{stemalta1}
	\begin{tabularx}{\textwidth}{lllllll}
		\lsptoprule
		\textsc{\textbf{inf}} & \textbf{translation} & \textsc{\textbf{prs.ind.1sg}} & \textsc{\textbf{prs.ind.1pl}} & \textsc{\textbf{prs.sbjv.1sg}} & \textsc{\textbf{prs.sbjv.1pl}} & \textsc{\textbf{ptcp}} \\
		\midrule
		\textit{ampruá} & `try' & \textit{amprùva} & \textit{ampruajn} &  \textit{amprùvi} & \textit{amprú̱vian} & \textit{ampruau} \\
		\textit{ampustá} & `order' & \textit{ampòsta} & \textit{ampustajn} & \textit{ampòsti} & \textit{ampò̱stian} & \textit{ampustau}\\
		\textit{anf(a)rá} & ‘shoe a horse’ & \textit{anfjara} & \textit{anf(a)rajn} &  \textit{anfjari} & \textit{anfja̱rian} & \textit{anf(a)rau}\\
		\textit{angulá} & `steal' & \textit{angùla} & \textit{angulajn} & \textit{angùli} & \textit{angù̱lian} & \textit{angulau}\\
		\textit{antupá} & `meet' & \textit{antaupa} & \textit{antupajn} & \textit{antaupi} & \textit{anta̱u̱pian} & \textit{antupau}\\
		\textit{anzardá} & `aerate (hay)' & \textit{anzjarda} & \textit{anzardajn} & \textit{anzjardi} & \textit{anzja̱rdian} & \textit{anzardau}\\
		\textit{barsá} & ‘roast’ & \textit{brassa} & \textit{barsajn} & \textit{brassi}& \textit{bra̱ssian} & \textit{barsau}\\
		\textit{cantá} & ‘sing’ & \textit{cònta} & \textit{cantajn} & \textit{cònti} & \textit{cò̱ntian} & \textit{cantau}\\
		\textit{cudizá} & ‘provoke’ & \textit{cudéza} & \textit{cudizajn} & \textit{cudéci} & \textit{cudé̱cian} & \textit{cudizau}\\
		\textit{cugljaná} & ‘cheat’ & \textit{cugljuna} & \textit{cugljanajn} & \textit{cugljuni} & \textit{cuglju̱nian} & \textit{cugljanau}\\
		\textit{cumandá} & ‘order’ & \textit{camònda} & \textit{cumandajn} & \textit{camòndi} & \textit{camò̱ndian} & \textit{cumandau}\\
		\textit{curdá} & ‘fall’ & \textit{crùda} & \textit{curdajn} & \textit{cròdi} & \textit{crò̱dian} & \textit{curdau}\\
		\textit{custá} & `cost' & \textit{cuésta} & \textit{custajn} & \textit{cuésti} & \textit{cué̱stian} & \textit{custau}\\
		\textit{cuzá} & ‘last’ & \textit{cuéza} & --- & \textit{cuéci} & --- & \textit{cuzau}\\
		\textit{digrá} & ‘drip’ & \textit{daghira} & \textit{digrajn} & \textit{daghiri} & \textit{daghi̱rian} & \textit{digrau}\\
		\textit{dustá} & `keep away' & \textit{dòsta} & \textit{dustajn} & \textit{dòsti} & \textit{dò̱stian} & \textit{dustau}\\
		\textit{duvrá} & ‘use’ & \textit{dùvra} & \textit{duvrajn} & \textit{drùvi} & \textit{drò̱vian} & \textit{duvrau}\\
		\textit{druvá} & `use' & \textit{drùva} & \textit{druvajn} & \textit{drùvi} & \textit{drò̱vian} & \textit{druvau}\\
		\textit{fardá} & `smell' & \textit{fréjda} & \textit{fardajn} & \textit{fréjdi} & \textit{fré̱jdian} & \textit{fardau}\\
		\lspbottomrule&
	\end{tabularx}
\end{sidewaystable}



\begin{sidewaystable}
	\caption{Verbs ending in \textit{-á}, second part}
	\label{stemalta2}
	\begin{tabularx}{\textwidth}{lllllll}
		\lsptoprule
		\textsc{\textbf{inf}} & \textbf{translation} & \textsc{\textbf{prs.ind.1sg}} & \textsc{\textbf{prs.ind.1pl}} & \textsc{\textbf{prs.sbjv.1sg}} & \textsc{\textbf{prs.sbjv.1pl}} &  \textbf{\textsc{ptcp}}\\
		\midrule
		\textit{fimá} & `smoke' & \textit{féma} & \textit{fimajn} & \textit{fémi} & \textit{fé̱mian} & \textit{fimau}\\
		\textit{furá} & ‘pierce’ & \textit{fùra} & \textit{furajn} & \textit{fùri} & \textit{fù̱rian} & \textit{furau}\\
		\textit{furṣchá} & `rub' & \textit{fruṣcha} & \textit{furṣchajn} & \textit{fruṣchi} & \textit{fru̱ṣchian} & \textit{furṣchau}\\
		\textit{gizá} & `sharpen' & \textit{géza} & \textit{gizajn} & \textit{géci} & \textit{gé̱cian} & \textit{gizau}\\
		\textit{lavá sé} & `stand up' & \textit{léjva} & \textit{lavajn} & \textit{léjvi} & \textit{lé̱jvian} & \textit{lavau}\\
		\textit{luá} & `melt' & \textit{líua} & \textit{luajn} & \textit{líui} & \textit{líuian}  & \textit{luau}\\
		\textit{ludá} & `praise' & \textit{lauda} & \textit{ludajn} & \textit{laudi} & \textit{la̱u̱dian} & \textit{ludau}\\
		\textit{manizá} & `chop' & \textit{manéza} & \textit{manizajn} & \textit{manéci} & \textit{mané̱cian} & \textit{manizau}\\
		\textit{mulá} & `grind' & \textit{mùla} & \textit{mulajn} &  \textit{mùli} & \textit{mù̱lian} & \textit{mulau}\\
		\textit{munglá}\footnote{Nowadays it is only the conditional \textit{munglás} which is used.} & ‘should’ & \textit{maungla} & \textit{munglajn} & --- & --- & ---\\
		\textit{mussá} & ‘show’ & \textit{mùssa} & \textit{musssajn} & \textit{mùssi} & \textit{mù̱ssian} & \textit{mussau}\\
		\textit{piná} & `prepare' & \textit{pajna} & \textit{pinajn} & \textit{pajni} & \textit{pa̱j̱nian} & \textit{pinau}\\
		\textit{quitá} & `think, find' & \textit{quéta} & \textit{quitajn} & \textit{quéti} & \textit{qu̱é̱tian} & \textit{quitau}\\
		\textit{ruassá} &`rest' & \textit{ruaussa} & \textit{ruassajn} & \textit{ruaussi} & \textit{rua̱u̱ssian} & \textit{ruassau}\\
		\textit{ruclá} & `roll, fall' & \textit{rùcla} & \textit{ruclajn} & \textit{rùcli} & \textit{rù̱clian} & \textit{ruclau}\\
		\textit{rupá} & `burp' & \textit{raupa} & \textit{rupajn} & \textit{raupi} & \textit{ra̱u̱pian} & \textit{rupau}\\
		\textit{rumplaná} & `rumble' & \textit{rampluna} & \textit{rumplanajn} & \textit{rampluni} & \textit{ramplúnian} & \textit{rumplanau} \\
		\textit{sacantá} & ‘dry’ & \textit{sacjanta} & \textit{sacantajn} & \textit{sacjanti} & \textit{sacj̱a̱ntian} & \textit{sacantau}\\
		\textit{sadapurtá} & ‘behave’ & \textit{sadapòrta} & \textit{sadapurtajn} & \textit{sadapòrti} & \textit{sadapò̱rtian} & \textit{sadapurtau}\\
			\lspbottomrule
		\end{tabularx}
\end{sidewaystable}


\begin{sidewaystable}
	\caption{Verbs ending in \textit{-á}, third part}
	\label{stemalta3}
	\begin{tabularx}{\textwidth}{lllllll}
		\lsptoprule
		\textsc{\textbf{inf}} & \textbf{translation} & \textsc{\textbf{prs.ind.1sg}} & \textsc{\textbf{prs.ind.1pl}} & \textsc{\textbf{prs.sbjv.1sg}} & \textsc{\textbf{prs.sbjv.1pl}} & \textsc{\textbf{ptcp}} \\
		\midrule
		\textit{sadrizá} & `address' & \textit{sadréza} & \textit{sadrizajn} & \textit{sadréci} & \textit{sadré̱cian} & \textit{sadrizau}\\
		\textit{sahaná} & ‘(not) appreciate’ & \textit{sahòna} & \textit{sahanajn} & \textit{sahòni} & \textit{sahò̱nian} & \textit{sahanau}\\
		\textit{samjá} & ‘dream’ & \textit{siamja} & \textit{samjajn} & \textit{siami} &  \textit{sj̱a̱mian} & \textit{samjau}\\
		\textit{sanudá} & `swim' & \textit{sanùda} & \textit{sanudajn} & \textit{sanùdi} & \textit{sanù̱dian} & \textit{sanudau} \\
		\textit{sará} & `close' & \textit{sjara} & \textit{sarajn} & \textit{sjari} & \textit{sja̲rian} & \textit{sarau} \\
		\textit{saragurdá} & `remember' & \textit{saragòrda} & \textit{saragurdajn} & \textit{saragòrdi} & \textit{saragò̱rdian} & \textit{saragurdau}\\
		\textit{satrá} & `bury' & \textit{satjara} & \textit{satrajn} & \textit{satjari} & \textit{satjárian} & \textit{satrau}\\
		\textit{scadá} & `heat' & \textit{scauda} & \textit{scadajn} & \textit{scaudi} & \textit{sca̱u̱dian} & \textit{scadau}\\
		\textit{schlupá} & `burst' & \textit{schlòpa} & \textit{schlupajn} & \textit{schlòpi} & \textit{schlò̱pian} & \textit{schlupau}\\
		\textit{scumandá} & `prohibit' & \textit{scamònda} & \textit{scumandajn} & \textit{scamòndi} & \textit{scamò̱ndian} & \textit{scumandau} \\
		\textit{scurṣalá} & `sledge' & \textit{scarṣùla} & \textit{scurṣalajn} & \textit{scarṣùli} & \textit{scarṣù̱lian} & \textit{scursalau}\\
		\textit{sitá} & ‘shoot’ & \textit{siéta} & \textit{sitajn} & \textit{siéti} & \textit{sié̱tian} & \textit{sitau}\\
		\textit{splaná} & ‘plane’ & \textit{splauna} & \textit{splanajn} & \textit{splauni} & \textit{spla̱u̱nian} & \textit{splanau}\\
		\textit{stizá} & `turn off' & \textit{stéza} & \textit{stizajn} & \textit{stéci} & \textit{sté̱cian} & \textit{stizau}\\
		\textit{suá} & `sweat' & \textit{síua} & \textit{suajn} & \textit{síui}& \textit{síuian} & \textit{suáu}\\
		\textit{sutá} & `dance' & \textit{sauta} & \textit{sutajn} & \textit{sauti} & \textit{sa̱u̱tian} & \textit{sutau}\\
		\textit{turná} & `return' & \textit{tùrna} & \textit{turnajn} & \textit{tùrni} & \textit{tù̱rnian} & \textit{turnau}\\
		\textit{uzá} & ‘lift’ & \textit{auza} & \textit{uzajn} & \textit{auci} & \textit{a̱u̱cian} & \textit{uzau} \\
		\textit{zavrá} & `separate, sort out' & \textit{zajvra} & \textit{zavrajn} & \textit{zajvri} & \textit{za̱j̱vrian} & \textit{zavrau}\\
		\textit{zulá} & ‘roll out’ & \textit{zùla} & \textit{zulajn} & \textit{zùli} & \textit{zù̱lian} & \textit{zulau} \\
		\lspbottomrule
	\end{tabularx}
\end{sidewaystable}


\clearpage
The verbs \textit{digrá} `drip' and \textit{satrá} `bury' are a different case. What looks like metathesis (\textit{daghira} `(s/he) drips' vs \textit{digrajn} `(we) drip' and \textit{satjara} `(s/he) buries vs \textit{satrajn} `(we) bury') is due to the dropping of the \isi{reduced vowel} /ɐ/ between \textit{g--r} and \textit{t--r} or, in other words, between a stop and a trill. According to \citet[311 and 973]{Decurtins2012}, \textit{digrá} is derived from the mixture of Latin \textsc{decu̱rrere} `flow off' and \textsc{cu̱rare} `sieve', whereas \textit{satrá} is derived from Middle Latin \textsc{subte̱rrare} `bury'.


Verbs ending in \textit{-è} that show stem alternation are listed in \tabref{stemalte}. The following stem alternations occur:

\begin{itemize}
	\item a → è (\textit{bagagè} `build' → \textit{ju baghègja, nus bagagjajn}), → é (\textit{pardagè} `preach' → \textit{ju pardégja, nus pardagjajn}), → ò (\textit{cumpagnè} `accompany' → \textit{ju cumpò\-gna, nus cumpagnajn})
	\item i → aj (\textit{piè} `pay' → \textit{paja}), → é (\textit{bitschè} `kiss' → \textit{bétscha})
	\item u → au (\textit{stuschè} `push' → \textit{stauscha}), → ò (\textit{bugnè} `give water' → \textit{bògna})
	\item u ... a → a ... ò (\textit{dumagnè} `cope with' → \textit{damògna})
	\end{itemize}
	
The \isi{imperfect} of the verbs ending in \textit{-agè} is sometimes realised as -\textit{java} instead of -\textit{agjava}, as in \textit{pardjavan} `they used to preach' vs \textit{pardagjavan} `idem', or \textit{schabjava} `used to happen' instead of \textit{schabagjava} `idem'. The variation between these two forms is free.
	
Metathesis occurs with \textit{r} in the following cases, with or without change in the vowel:

\begin{itemize}
	\item ar → ra (\textit{tartgè} `think' → \textit{tratga}), → rè (\textit{mudargè} `torment' → \textit{mudrègja}), → ri (\textit{barṣchè} `burn' → \textit{briṣcha}), → rò (\textit{fufargnè} `rummage' → \textit{fufrògna})
\end{itemize}

\begin{sidewaystable} 
	\caption{Verbs ending in \textit{-è}}
	\label{stemalte}
	\begin{tabularx}{\textwidth}{lllllllll} 
		\lsptoprule
		\textsc{\textbf{inf}} & \textbf{translation} & \textsc{\textbf{prs.ind.1/3sg}} & \textsc{\textbf{prs.ind.1pl}} & \textsc{\textbf{impf.ind.1/3sg}} & \textsc{\textbf{prs.sbjv.1/3sg}} & \textsc{\textbf{prs.sbv.1pl}} \\
		\midrule
		\textit{bagagè} & ‘build’ & \textit{baghègja} & \textit{bagagjajn} & \textit{bagagjava} & \textit{baghègi} & \textit{baghè̱gian}\\
		\textit{barṣchè} & ‘burn’ & \textit{briṣcha} & \textit{barṣchajn} & \textit{barṣchava} & \textit{briṣchi} & \textit{bri̱ṣchian}\\
		\textit{bitschè} & ‘kiss’ & \textit{bétscha} & \textit{bitschajn} & \textit{bitschava} & \textit{bétschi} & \textit{bé̱tschian}\\
		\textit{bugnè} & ‘give water’ & \textit{bògna} & \textit{bugnajn} & \textit{bugnava} & \textit{bògni} & \textit{bò̱gnian}\\
		\textit{cumpagnjè} & ‘accompany’ & \textit{cumpògna} & \textit{cumpagnajn} & \textit{cumpagnava} & \textit{cumpògni} & \textit{cumpò̱gnian}\\
		\textit{dumagnè} & `cope with' & \textit{damògna} & \textit{dumagnajn} & \textit{dumagnava} & \textit{damògni} & \textit{damò̱gnian}\\
		\textit{fufargnè} & ‘rummage’ & \textit{fufrògna} & \textit{fufargnajn} & \textit{fufargnava} & \textit{fufrògni} & \textit{fufrò̱gnian}\\
		\textit{lahargnè} & `giggle' & \textit{lahrògna} & \textit{lahargnajn}& \textit{lahargnava} & \textit{lahgrògni} & \textit{lahgrò̱gninan}\\
		\textit{mudargè} & `torment' & \textit{mudrègja} & \textit{mudargjajn} & \textit{mudargjava} & \textit{mudrègi} & \textit{mudrè̱gian} \\
		\textit{pardagè} & ‘preach’ & \textit{pardègja} & \textit{pardagjajn} & \textit{pardagèva} & \textit{pardègi} & \textit{pardè̱gian}\\
		\textit{piè} & ‘pay’ & \textit{paja} & \textit{piajn} & \textit{pièvan} & \textit{paji} & \textit{pájan}\\
		\textit{sampatschè} & ‘interfere’ & \textit{sampatscha} & \textit{sampatschajn} & \textit{sampatschava} & \textit{sampatschi} & \textit{sampa̱tschian}\\
		\textit{schabagè} & `happen' & \textit{schabègja} & --- & \textit{schabagjava} & \textit{schabègi} &  ---\\
		\textit{schagè} & `taste' & \textit{schagja} & \textit{schagjajn} & \textit{schagjava} & \textit{schagi} & \textit{scha̱gian}\\
		\textit{ṣchubargè} & `clean' & \textit{ṣchubrègja} & \textit{ṣchubargjajn} & \textit{ṣchubargjava} & \textit{ṣchubrègi} & \textit{ṣchubrè̱gian} \\
		\textit{scumbagljè} & `confuse' & \textit{scumbèglja} & \textit{scumbagljajn} & \textit{scumbagljava} & \textit{scumbègli} & \textit{scumbè̱glian}\\
		\textit{siè} & `mow' & \textit{sia} & \textit{siajn} & \textit{sièva} & \textit{sii} & \textit{sian}\\
		\textit{spatgè} & `wait' & \textit{spètga} & \textit{spatgajn} & \textit{spatgava} & \textit{spètgi} & \textit{spè̱tgian}\\
		\textit{stuschè} & `push' & \textit{stauscha} & \textit{stuschajn} & \textit{stuschava} & \textit{stauschi} & \textit{sta̱u̱schian}\\
		\textit{tartgè} & `think' & \textit{tratga} & \textit{tartgajn} & \textit{tartgava} & \textit{tratgi} & \textit{tra̱tgian}\\
		\textit{tgiè} & `shit' & \textit{tgaja} & \textit{tgiajn} & \textit{tgiava} & \textit{tgaji} & \textit{tgajan}\\
		\textit{tschitschè} & `suck' & \textit{tschétscha} & \textit{tschitschajn} & \textit{tschitschava} & \textit{tschétschi} & \textit{tsché̱tschian} \\
		
		\lspbottomrule
	\end{tabularx} 
\end{sidewaystable}

The verbs ending in \textit{-ar} that show stem alternation are listed in \tabref{stemaltvar}. The following stem alternations occur:
 
 \begin{itemize}
 	\item aj → u (\textit{bajbar} `drink' → \textit{bubín})
 	\item è → a (\textit{crèschar} `grow' → \textit{carschín})
 	\item é → a (\textit{curégjar} `correct' → \textit{curagín})
 	\item éj → a (\textit{séjṣar} `sit' → \textit{sasín})
 	\item ja → a (\textit{pjardar} `lose' → \textit{pardín})
 	\item ò → u (\textit{còschar} `keep quiet' → \textit{cuschín})
 	\item ué → u (\textit{laguétar} `swallow' → \textit{lagutín})
 	 \end{itemize}

\begin{sidewaystable} 
	\caption{Verbs ending in \textit{-ar}}
	\label{stemaltvar}
	\small
	\begin{tabularx}{\textwidth}{llllllll}
		\lsptoprule
		\textsc{\textbf{inf}} & \textbf{translation} & \textsc{\textbf{prs.ind.1sg}} & \textsc{\textbf{prs.ind.1pl}} & \textsc{\textbf{impf.ind.1sg}} & \textsc{\textbf{prs.sbjv.1sg}} & \textsc{\textbf{prs.sbjv.1pl}} & \textsc{\textbf{ptcp}}\\
		\midrule
		\textit{antschajvar} & `begin' & \textit{antschajva} & \textit{antschavín} & \textit{antschavéva} &  \textit{antschajvi} & \textit{antscha̱j̱vian} & \textit{antschiat}\\
		\textit{bajbar} & `drink' & \textit{bajba} & \textit{bubín} & \textit{buéva} & \textit{bajbi} & \textit{ba̲jbian} & \textit{bubju}\\
		\textit{còschar} & `keep quiet' & \textit{còsch} & \textit{cuschín} & \textit{cuschéva} & \textit{còschi} & \textit{cò̱schian} & \textit{cuschju}\\
		\textit{crèschar} & `grow' & \textit{crèscha} & \textit{carschín} & \textit{carschéva} & \textit{crèschi} & \textit{crè̱schian}  & \textit{carschju}\\
		\textit{curégjar} & `correct' & \textit{curégja} & \textit{curagín} & \textit{curagéva} & \textit{curégi} & \textit{curé̱gian} & \textit{curagjú}\\
		\textit{cuviarar} & `cover' & \textit{cuviara} & \textit{cuvrín} & \textit{cuvréva} & \textit{cuviari} & \textit{cuvj̱a̱rian} & \textit{cuvrétg}\\
		\textit{dapèndar} & `depend' & \textit{dapjanda} & \textit{dapandín} & \textit{dapandévan} & \textit{dapjandi} & \textit{dapj̱a̱ndian} & \textit{dapandju}\\
		\textit{dèrgjar} & `spill' & \textit{dèrgja} & \textit{dargín} & \textit{dargèvan} & \textit{dèrgi} & \textit{dè̱rgian}
		 & \textit{dèrs}\\
		\textit{laguétar} & `swallow' & \textit{laguéta} & \textit{lagutín} & \textit{lagutéva} & \textit{laguéti} & \textit{lagué̱tian} & \textit{lagutju}\\
		\textit{légjar} & `read' & \textit{légja} & \textit{lagín} & \textit{lagéva} & \textit{léjgja} & \textit{lé̱gian} & \textit{lagju}\\
		\textit{métar} & `put' & \textit{méta} & \textit{matajn} & \textit{matéva} & \textit{méti} & \textit{mé̱tian} & \textit{méz/mèz}\\
		\textit{mòrdar} & `bite' & \textit{mòrda} & \textit{murdín} & \textit{murdéva} & \textit{mòrdi} & \textit{mò̱rdian} & \textit{murdju}\\
		\textit{pjardar} & `lose' & \textit{pjarda} & \textit{pardín} & \textit{pardéva} & \textit{pjardi} & \textit{pj̱árdian} & \textit{pjars}\\
		\textit{rúmpar} & `break' & \textit{rùmpa} & \textit{rumpajn} & \textit{rumpéva} & \textit{rùmpi} & \textit{rù̱mpian} & \textit{rùt}\\
		\textit{séjṣar} & `sit' & \textit{séjṣa}  &	\textit{saṣín} & \textit{saṣéva} & \textit{séiṣi} & \textit{sé̱j̱ṣian} & \textit{saṣju}\\
		\textit{sòlvar} & `have breakfast' & \textit{sòlva} & \textit{sulvín} & \textit{sulvéva} & \textit{sòlvi} & \textit{sò̱lvian} & \textit{sjut}\\
		\lspbottomrule
	\end{tabularx} 
\end{sidewaystable}

Verbs ending in \textit{-í} that show stem alternation are presented in  \tabref{stemalti}. The following alternations occur:

\begin{itemize}
	\item a → aj (\textit{amplaní} `fill' → \textit{amplajna}), → é (\textit{saglí} `run' → \textit{séglja}), → ja (\textit{santí} `feel' → \textit{sjanta})
\end{itemize}

Metathesis occurs in \textit{bargí} `cry' vs \textit{bragja}.

\begin{sidewaystable} 
	\caption{Verbs ending in \textit{-í}}
	\label{stemalti}
	\begin{tabularx}{\textwidth}{llllllll} 
		\lsptoprule
		\textsc{\textbf{inf}} & \textbf{translation} & \textsc{\textbf{prs.ind.1sg}} & \textsc{\textbf{prs.ind.1pl}} & \textsc{\textbf{impf.ind.1sg}} & \textsc{\textbf{prs.sbjv.1sg}} & \textsc{\textbf{prs.sbjv.1pl}} & \textsc{\textbf{ptcp}}\\
		\midrule
		\textit{amplaní} & `fill' & \textit{amplajna} & \textit{amplanín} & \textit{amplanéva} & \textit{amplajni} & \textit{ampla̱j̱nian} & \textit{amplanju}\\
		\textit{ancurí} & `look for' & \textit{anquéra} & \textit{ancurín} &\textit{ancuréva} & \textit{anquéri} & \textit{anqu̱é̱rian} & \textit{ancurétg}\\
		\textit{bargí} & `cry' & \textit{bragja}  & \textit{bargín} & \textit{bargéva} & \textit{bragi} & \textit{brágian} & \textit{bargjú}\\
		\textit{durmí} & `sleep' & \textit{dòrma} & \textit{durmín} & \textit{durméva} & \textit{dòrmi} & \textit{dò̱rmian} & \textit{durmju}\\
		\textit{murí} & `die' & \textit{mùra} & \textit{murín} & \textit{muréva} & \textit{mùri} & \textit{mù̱rian} & \textit{mòrts}\\
		\textit{saglí} & `run' & \textit{séglja} & \textit{saglín} & \textit{sagljéva} & \textit{ségli} & \textit{sé̱glian} & \textit{sagljú}\\
		\textit{santí} & `feel' & \textit{sjanta} & \textit{santín} & \textit{santéva} & \textit{sjanti} & \textit{sj̱a̱ntian} & \textit{santju}\\
		\lspbottomrule
	\end{tabularx} 
\end{sidewaystable}


\subsubsection{Irregular verbs}\label{sec:4.1.1.4}
The irregular verbs \textit{èssar} `be', \textit{vay} `have', and \textit{vagní} `come',  which also function as \isi{auxiliary verbs}, have been presented in \sectref{sec:4.1.1.1}. The most important other irregular verbs will be presented in \tabref{tab:sadastada} -- \tabref{tab:fugi}.

\begin{table}
	\caption{\textit{\textit{sadastadá}} `wake up'}
\label{tab:sadastada}
	\begin{tabular}{lll}
		\lsptoprule
		\textsc{inf} & \isi{\textit{\textbf{sadastadá}}}\\
		\midrule
		\textsc{prs.ind} & \textit{sadadèsta}, \textit{sadastadajn} \\
		\textsc{prf.ind} & \textit{sùn sadastadauṣ, sadastadada}\\
		\textsc{impf.ind} & \textit{sadastadèvan}\\
		\textsc{cond} & \textit{sadastadá̱ṣ}\\
		\textsc{prs.sbjv} & \textit{sadadèsti}, \textit{sadadè̱stian}\\
		\textsc{imp} & \textit{sadadèsta}, \textit{sadastadaj}\\
		\lspbottomrule
	\end{tabular}
\end{table}



\begin{table}
	\caption{\textit{dá} `give' and \textit{stá} `stay'}

	\begin{tabular}{lll}
		\lsptoprule
		\textsc{inf} & \isi{\textit{\textbf{dá}}} & \isi{\textit{\textbf{stá}}}\\
		\midrule
		
		\textsc{prs.ind} & \textit{dùn, daṣ/dataṣ, dá} & \textit{stùn, staṣ/stataṣ, stat}\\
		& \textit{dajn, dajṣ, datan} & \textit{stajn, stajṣ, statan}\\
		\textsc{prf.ind} & \textit{a dau} & \textit{sùn stada/stauṣ}\\
		\textsc{impf.ind} & \textit{dèva} & \textit{stèva}\\
		\textsc{cond} & \textit{dèṣ} & \textit{stèṣ}\\
		\textsc{prs.sbjv}	& \textit{dèti}\footnote{The form \textit{dètschi} is used by an older consultant and is the only form given in the \DRG{5}{65}. The whole paradigm of the \isi{present subjunctive} given in the DRG is \textit{dètschi, dètschias, dètschi, dajan, dajas, dètschian.}}, \textit{dètian} & \textit{stèti, stètian}\\
		\textsc{imp} & \textit{dá, daj} & \textit{stá, staj}\\
		\lspbottomrule
	\end{tabular}
\end{table}


\begin{table}
	\caption{\textit{fá} `do' and \textit{trá} `pull'}

	\begin{tabular}{lll}
		\lsptoprule
		\textsc{inf} & \isi{\textit{\textbf{fá}}} & \isi{\textit{\textbf{trá}}}\\
		\midrule
		\textsc{prs.ind} & \textit{fétsch, faṣ, fò} & \textit{tila, tilaṣ, tila}\\
		& \textit{fagjajn, fagjajṣ, fòn} & \textit{trajn, trajṣ, tilan}\\
		\textsc{prf.ind} & \textit{a fatg} & \textit{a tratg}\\
		\textsc{impf.ind} & \textit{fagèva/fièva} & \textit{trèva} \\
		\textsc{cond} & \textit{fagèṣ} & \textit{trèṣ}\\ 
		\textsc{prs.sbjv} & \textit{fétschi}, \textit{fétschian} & \textit{tili, ti̱lian}\\
		\textsc{imp} & \textit{fò, fagjaj} & \textit{tila, traj}\\
		\lspbottomrule
	\end{tabular}
\end{table}

\begin{table}
\small
	\caption{\textit{craj} `believe' and \textit{duaj} `must'}

	\begin{tabular}{lll}
		\lsptoprule
		\textsc{inf} & \isi{\textit{\textbf{craj}}} `believe' & \isi{\textit{\textbf{duaj}}} `must'\\
		\midrule
		\textsc{prs.ind} & \textit{craj}, \textit{crajaṣ}, \textit{craj} & \textit{duaj/daj},\footnote{The \DRG{4}{370} offers for Sedrun \textit{déi, déjǝs} etc.; the modern forms \textit{daj, dajas} etc. are not accepted by all speakers that were consulted.} \textit{duajṣ/dajaṣ, duaj/daj,}\\
		& \textit{cartín}, \textit{cartíṣ}, \textit{crajan} & \textit{duajn/dajan, duajṣ/dajaṣ, duajn/dajan}\\
		\textsc{prf.ind} & \textit{a cartjú} & \textit{a dujú}\\
		\textsc{impf.ind} & \textit{cartéva} & \textit{duèva} \\
		\textsc{cond} & \textit{cartéṣ} & \textit{duèṣ}\\
		\textsc{prs.sbjv} & \textit{craji}, \textit{crajaṣ}, \textit{craji} & --- \\
		&\textit{crajan, crajaṣ, crajan} & --- \\
		\textsc{imp} & \textit{craj, cartí} & --- \\
		\lspbottomrule
	\end{tabular}
\end{table}

\begin{table}
\small
	\caption{\textit{gudaj} `enjoy' and \textit{pudaj} `can, be able'}

	\begin{tabular}{llll}
		\lsptoprule
		\textsc{inf} & \isi{\textbf{\textit{gudaj}}} & \isi{\textit{\textbf{pudaj}}} \\
		\midrule
		\textsc{prs.ind} & \textit{gauda, gaudaṣ, gauda} & \textit{pùṣ, pùṣ, pù}\\
		& \textit{gudajn, gudajṣ, gaudan} & \textit{pudajn, pudajṣ, pùn}\\
		\textsc{prf.ind} & \textit{a gudju} & \textit{a pudju}\\
		\textsc{impf.ind} & \textit{gudéva} & \textit{pudèva}\\
		\textsc{cond} & \textit{gudéṣ} & \textit{pudèṣ}\\
		\textsc{prs.sbjv} &\textit{gaudi, ga̱u̱dian} & \textit{pùssi, pù̱ssian}\\
		\textsc{imp} & \textit{gauda, gudí} & ---\\
		\lspbottomrule
	\end{tabular}
\end{table}


\begin{table}
\small
	\caption{\textit{savaj} `know' and \textit{ṣchaj} `lie'}

	\begin{tabular}{lll}
		\lsptoprule
		\textsc{inf} & \isi{\textbf{\textit{savaj}}} & \isi{\textbf{\textit{ṣchaj}}} \\
		\midrule
		\textsc{prs.ind} & \textit{sa, saṣ, sò} &  \textit{ṣchaj, ṣchajaṣ, ṣchaja} \\
		& \textit{savajn, savajṣ, sòn} & ṣ\textit{chiajn, ṣchiajṣ, ṣchajan} \\
		\textsc{prf.ind} & \textit{a savju}\footnote{The short form \textit{sju} is also used in rapid speech.} & \textit{sùn ṣchjus/ṣchida}\\
		\textsc{impf.ind} & \textit{savèva} & \textit{ṣchièva}\\
		\textsc{cond} & \textit{savèṣ} & \textit{ṣchièṣ}\\
		\textsc{prs.sbjv} & \textit{sapi, sápian} & \textit{ṣchaji, ṣchajan}\\
		\textsc{imp} & --- & \textit{ṣchaj, ṣchijí}\\
		\lspbottomrule
	\end{tabular}
\end{table}



\begin{table}
\small
	\caption{\textit{stuaj} `must' and \textit{tanaj} `hold'}

	\begin{tabular}{lll}
		\lsptoprule
		\textsc{inf} & \isi{\textit{\textbf{stuaj}}} & \isi{\textbf{\textit{tanaj}}}\\
		\midrule
		\textsc{prs.ind} & \textit{stù, stùṣ, stù} & \textit{tégn, tégnaṣ, tégn}\\
		& \textit{stuajn, stuajṣ, stùn} & \textit{tanín, taníṣ,tégnan}\\
		\textsc{prf.ind} & \textit{a stavjú/stujú/stju} & \textit{a tanju}\\
		\textsc{impf.ind} & \textit{stuèva/stèva/stavèva} & \textit{tanéva}\\
		\textsc{cond} & \textit{stuèṣ} & \textit{tanéṣ}\\
		\textsc{prs.sbjv} & \textit{stùpi, stù̱pian} & \textit{tégni}, \textit{té̱gnian}\\
		\textsc{imp} & --- & \textit{tégn, taní}\\
		\lspbottomrule
	\end{tabular}
\end{table}

\begin{table}
\small
	\caption{\textit{tumaj} `fear'  and \textit{vulaj} `want'}

	\begin{tabular}{lll}
		\lsptoprule
		\textsc{inf} & \isi{\textit{\textbf{tumaj}}} & \isi{\textit{\textbf{vulaj}}}\\
		\midrule
		\textsc{prs.ind} & \textit{téma, témaṣ, téma,} & \textit{vi, vutaṣ, vut,\footnote{In combination with \textit{dí} `say', the form \textit{vuta}, as in \textit{vuta di} `wants to say', i.e. `means', is used by some consultants. Most consultatns, however, reject this form.}}\\
		& \textit{tumajn, tumajṣ, téman} & \textit{lajn, lajṣ, vutan} \\
		\textsc{prf.ind} & \textit{a tumjú} & \textit{a vulju}\footnote{The form \textit{valju} also occurs.}\\
		\textsc{impf.ind} & \textit{tuméva} & \textit{lèva}\\
		\textsc{cond} & \textit{tuméṣ} & \textit{lèṣ}\\
		\textsc{prs.sbjv} & \textit{tèmi, tè̱mian} & ---\\
		\textsc{imp} & \textit{tèma, tumaj} & ---\\
		\lspbottomrule
	\end{tabular}
\end{table}


\begin{table}
\small
	\caption{\textit{vasaj} `see' and \textit{parvaj} `feed'}

	\begin{tabular}{lll}
		\lsptoprule
		\textsc{inf} & \isi{\textbf{\textit{vasaj}}} & \isi{\textbf{\textit{parvaj}}}\\
		\midrule
		\textsc{prs.ind} & \textit{vèza}, \textit{vèzaṣ}, \textit{vèza}, & \textit{parvaj}, \textit{parvajaṣ}, \textit{parvaj},\\
		& \textit{vasajn}, \textit{vasajṣ}, \textit{vèzan} & \textit{parvasín}, \textit{parvasíṣ}, \textit{parvajan}\\
		\textsc{prf.ind} & \textit{a vju} & \textit{a parvaṣjú}\\
		\textsc{impf.ind} & \textit{vasèva/vaséva} & \textit{parvasèva}\\
		\textsc{cond} &\textit{vasèṣ} & \textit{parvaséṣ}\\
		\textsc{prs.sbjv}	& \textit{vèci, vè̱cian} & \textit{parvaji, parvajan}\\
		\textsc{imp} & --- & \textit{parvaj}, \textit{parvasí}\\
		\lspbottomrule
	\end{tabular}
\end{table}


\begin{table}
	\caption{\textit{prèndar} `take' and \textit{schè/schá} `let, have do'}

	\begin{tabular}{llll}
		\lsptoprule
		\textsc{inf} & \isi{\textbf{\textit{prèndar}}} & \textbf{\textit{\isi{schè}/schá}}\footnote{\textit{Schá} is the \ili{Standard Sursilvan} form, which is commonly used in Tuatschin. The \DRG{10}{499} only notes \textit{schè} for Tuatschin; according to the \DRG{10}{502} the forms of the \isi{present subjunctive} \textsc{1pl} and \textsc{2pl} are \textit{schajan} and \textit{schajas}.}\\
		\midrule
		\textsc{prs.ind} & \textit{prèn, prèndaṣ, prèn} & \textit{lasch, lajaṣ, laj},\\
		& \textit{prandín, prandíṣ, prèndan} & \textit{schajn, schajṣ, lajan}\\
		\textsc{prf.ind} & \textit{a príu} & \textit{a schau}\\
		\textsc{impf.ind} & \textit{prandèva} & \textit{schèva}\\
		\textsc{cond} & \textit{prandèṣ} & \textit{schaṣ}\\
		\textsc{prs.sbjv} & \textit{prèndi, prè̱ndiaṣ, prèndi} & \textit{laschi, laschjas, laschi,}\\
		& \textit{prè̱ndian, prè̱ndiaṣ, prè̱ndian} & \textit{laschjan, laschjaṣ, laschjan}\\
		\textsc{imp} & \textit{prèn}, \textit{prandí} & \textit{lá, schaj}\\
		\lspbottomrule
	\end{tabular}
\end{table}


\begin{table}
	\caption{\textit{ira/ir/í} `go' and \textit{dí} `say'}

	\begin{tabular}{llll}
		\lsptoprule
		\textsc{inf} & \textbf{\textit{\isi{ira}/\isi{ir}/\isi{í}}} & \isi{\textbf{\textit{dí}}}\\
		\midrule
		\textsc{prs.ind} & \textit{vòn/mòn, vaṣ, vò} & \textit{détsch, diaṣ, di} \\
		& \textit{majn, majṣ, vòn} & \textit{ṣchajn, ṣchajṣ, dian}\\
		\textsc{prf.ind} & \textit{sùnd juṣ/ida} & \textit{a détg}\\
		\textsc{impf.ind} & \textit{mava} & \textit{ṣchèva}\\
		\textsc{cond} & \textit{maṣ} & \textit{ṣchèṣ}\\
		\textsc{prs.sbjv} & \textit{mòndi,\footnote{An old form is \textit{ju vòmi}.} mò̱ndiaṣ, mòndi}, & \textit{détschi}, \textit{dé̱tschiaṣ}, \textit{détschi},\\
&	\textit{vòndi, vò̱ndiaṣ, vòndi}\\
		& \textit{mò̱ndian, mò̱ndiaṣ, mò̱ndian} & \textit{dé̱tschian}, \textit{dé̱tschiaṣ}, \textit{dé̱tschian} \\
	&	\textit{vò̱ndian, vò̱ndiaṣ, vò̱ndian}\\
		\textsc{imp} & \textit{vò}, \textit{maj} & \textit{dí, ṣchaj}\\
		\lspbottomrule
	\end{tabular}
\end{table}

\clearpage

\begin{table}
\small
	\caption{\textit{fugí} `lie' and \textit{rí} `laugh'}
\label{tab:fugi}
	\begin{tabular}{lll}
		\lsptoprule
		\textsc{inf} & \isi{\textbf{\textit{fugí}}} & \isi{\textbf{\textit{rí}}}\\
		\midrule
		\textsc{prs.ind} & \textit{fuétsch, fujaṣ, fuj} & \textit{ri, riaṣ ri}\\
		&  \textit{fugín, fugíṣ, fujan} & \textit{riajn, riajṣ, rian}\\
		\textsc{prf.ind} & \textit{sùn fugjúṣ} & \textit{a riṣ}\\
		\textsc{impf.ind} & \textit{fugéva} & \textit{rièva}\\
		\textsc{cond} & \textit{fugéṣ} & \textit{rièṣ}\\
		\textsc{prs.sbjv} & \textit{fu̱é̱tschi, fu̱é̱tschiaṣ, fu̱é̱tschi,} & \textit{rii, riaṣ, rii,}\\
		& \textit{fu̱é̱tschian}, \textit{fu̱é̱tschiaṣ}, \textit{fu̱é̱tschian} & \textit{rian, riaṣ, rian}\\
		\textsc{imp} & \textit{fuj, fugí} & \textit{ri, riaj}\\
		\lspbottomrule
	\end{tabular}
\end{table}



\subsection{Usage of non-finite and finite verbal categories}\label{sec:4.1.2}
In this section the usage of the \isi{non-finite categories} \isi{past participle}, \isi{infinitive}, and \isi{gerund} as well as the \isi{finite categories} will be analysed, with the exception of the imperative, which will be treated in \sectref{sec:5.3}.

\subsubsection{Non-finite categories}\label{sec:4.1.2.1}

 \subsubsubsection{Past participle}\label{sec:4.1.2.1.1}
The \isi{past participle} is used to form compound (\ref{ex:ptcp1}) and \isi{doubly-compound tenses} (\ref{ex:dcomp:3}) as well as \isi{passive voice} (\ref{ex:ptcp2}); it is furthermore used attributively and predicatively and may also be nominalised (\ref{ex:ptcp3} and \ref{ex:ptcp4}), usually in its feminine form. If the \isi{auxiliary verb} is \textit{èssar} `be', the participle agrees with the \isi{subject} (\ref{ex:ptcp1}).

\ea
\label{ex:ptcp1}
\gll    [...] ju \textbf{sùn} \textbf{stauṣ} al davùs purtgè … da Sadrún~[...].\\
{}  \textsc{1sg} be.\textsc{prs.1sg} \textsc{cop.ptcp.m.sg} \textsc{def.m.sg} last swineherd {} of \textsc{pn} \\
\glt `[...] I was the last swineherd … of Sedrun [...].' (Sadrún, m6, \sectref{sec:8.11})
\z

\ea\label{ex:ptcp2}
\gll  A … tgu \textbf{sùn} \textbf{vagnida} \textbf{panṣjunada} scha … ju fagèva zuar schòn avaun majnadistríct~[...].\\
and {} when.\textsc{1sg} be.\textsc{prs.1sg} \textsc{pass.aux.ptcp.f.sg} pension\_off.\textsc{ptcp.f.sg} \textsc{corr} {} \textsc{1sg} do.\textsc{impf.1sg} although already before head\_of\_district.\textsc{m.sg}\\
\glt `And … when I got pensioned off … as a matter of fact, I had already worked as head of district before [...].' (Sadrún, f3, \sectref{sec:8.1})
\z

\ea\label{ex:ptcp3}
\gll [...] gljèz èra magari è léjgar  tgé \textbf{cuṣchinadas} èl [fagèva] [...].\\
{} \textsc{dem.unm} \textsc{cop.impf.3sg} sometimes also funny.\textsc{adj.unm} what cook.\textsc{ptcp.f.pl}  \textsc{3sg.m} [make.\textsc{impf.3sg}] \\
\glt `[...] it was sometimes also funny [to see] what he cooked [...].' (Sadrún, m4, \sectref{sec:8.3})
\z

\ea\label{ex:ptcp4}
\gll [...] i dat ina fòtògrafia tgu sùn sé cun mju còlè̱ga al dé da \textbf{la} \textbf{scargèda} [...].\\
{} \textsc{expl} \textsc{exist.prs.3sg}  \textsc{indef.f.sg} photograph \textsc{rel.1sg}  \textsc{cop.prs.1sg} on with  \textsc{poss.1sg.m.sg} mate \textsc{def.m.sg} day of \textsc{def.f.sg}  drove.\textsc{ptcp.f.sg} \\
\glt `[...] there is a photograph in which I am with my mate the day of the pig droving [...].' (Sadrún, m6, l. \sectref{sec:8.11})
\z

\ea
\label{ex:dcomp:3}
\gll Nuṣ \textbf{vajn} adina \textbf{gju} \textbf{fatg} parada.   \\
\textsc{1pl} have.\textsc{prs.1pl} always have.\textsc{ptcp.unm} make.\textsc{ptcp.unm} parade.\textsc{f.sg}\\
\glt `We always held a parade.' (Sadrún, m9, \sectref{sec:8.15})
\z

If the \isi{past participle} is used in compound tenses with the \isi{auxiliary} \isi{\textit{èssar}} `be' or in \isi{passive} constructions, it is treated like an \isi{adjective}, which means that (a) it agrees with the \isi{subject} of the \isi{verb} (\ref{ex:ptcp:agr1}), thus if the \isi{subject} has no \isi{gender}, the participle  takes its \isi{unmarked} form (\ref{ex:ptcp:agr2}), and (b) if in a \isi{passive} construction the \isi{subject} follows the participle, it does not agree with it (\ref{ex:ptcp:agr3} and \ref{ex:ptcp:agr5}) (see also \sectref{sec:5.5.4}).

\ea\label{ex:ptcp:agr1}
\gll [...] al cantún ò circa trènta da quèls majnadistrícts, \textbf{quèls} èn \textbf{partí} \textbf{ajn}\footnotemark{} ajn ragjúns [...].  \\
{} \textsc{def.m.sg} canton  have.\textsc{prs.3sg} about thirty of \textsc{dem.m.pl} head\_of\_district.\textsc{pl} \textsc{dem.m.pl} \textsc{pass.aux.prs.3pl} divide.\textsc{ptcp.m.pl} in in region.\textsc{f.pl}\\
\glt `[...] the canton has about thirty of these heads of district, these are divided in regions [...].' (Sadrún, f3; \sectref{sec:8.1})\footnotetext{\textit{Pártar ajn} is a particle verb meaning `divide'.}
\z

\ea\label{ex:ptcp:agr2}
\gll    Òh \textbf{gl’} \textbf{ampréndar} \textbf{tudèstg} è \textbf{stau}, l’ antschata ṣè \textbf{quaj} schòn \textbf{stau} in téc curjù̱s.\\
oh \textsc{def.m.sg} learn.\textsc{inf} German.\textsc{m.sg} be.\textsc{prs.3sg}  \textsc{cop.ptcp.unm} \textsc{def.f.sg} beginning be.\textsc{prs.3sg} \textsc{dem.unm} indeed \textsc{cop.ptcp.unm} \textsc{indef.m.sg} bit strange.\textsc{adj.unm}\\
\glt `Oh, to learn German was, at the beginning this was indeed a little bit strange.' (Zarcúns, m2; \sectref{sec:8.13})
\z

\ea\label{ex:ptcp:agr3}
\gll  Quaj è pròpi in ljuc ... nù tg’ \textbf{i} \textbf{vagnéva} \textbf{schau} tùt \textbf{la} \textbf{munizjun} tg’ i vèva, sigir.\\
\textsc{dem.unm} \textsc{cop.prs.3sg} exactly \textsc{indef.m.sg} place {} where \textsc{rel} \textsc{expl} \textsc{pass.aux.impf.3sg} leave.\textsc{ptcp.unm} all \textsc{def.f.sg} munition \textsc{rel} \textsc{expl} \textsc{exist.impf.3sg} sure.\textsc{adj.unm}\\
\glt `This is exactly a place ... where all the munition was stored, for sure.' (Sadrún, f3; \sectref{sec:8.1})
\z

\ea\label{ex:ptcp:agr5}
\gll    A lò … sén quaj intènt ségi è \textbf{vagnú} \textbf{bagagjau} \textbf{quèla} \textbf{caplùta}.\\
and there {} upon \textsc{dem.m.sg} undertaking be.\textsc{prs.sbjv.3sg} also \textsc{pass.aux.ptcp.unm} build.\textsc{ptcp.unm} \textsc{dem.f.sg} chapel \\
\glt `And there … after this undertaking this chapel was built.' (Sadrún, m5, \sectref{sec:8.8})
\z

If the \isi{past participle} is used attributively, the masculine singular form does not take the \isi{predicative} \textit{-s} if it has no complements as in \textit{in ùm panṣjunau} `a retired man'; if it has complements, the participle is treated like a \isi{predicative adjective} and takes \textit{-s} and can thus be considered an elliptic \isi{relative clause} (\ref{ex:ptcp:pred2}).

\ea\label{ex:ptcp:pred2}
\gll  Al \textbf{tètg} da duaṣ alas \textbf{fatg-} \textbf{s} \textbf{cun} \textbf{ajssas} \textbf{bétga} \textbf{splanadas} […] aj sén latas.\\
\textsc{def.m.sg} roof of two.\textsc{f} side.\textsc{pl} make.\textsc{ptcp-} \textsc{m.sg} with plank.\textsc{f.pl} \textsc{neg}  plane.\textsc{ptcp.f.pl} {} \textsc{cop.prs.3sg} on slat.\textsc{f.pl}\\
\glt `The two-sided roof made of planks that haven't been planed […] are on slats.' (Camischùlas, \DRGoK{3}{583})
\z

The negator \textit{bétga} and some \isi{temporal adverbs} may intervene between the \isi{auxiliary verb} and the \isi{past participle} (\ref{ex:betgasynt1}--\ref{ex:betgasynt5}).

\ea\label{ex:betgasynt1}
\gll  Álṣò ju a \textbf{bigja} fatg aj agrèssíf [...].  \\
well \textsc{1sg} have.\textsc{prs.1sg} \textsc{neg} make.\textsc{ptcp.unm} \textsc{3sg} aggressive.\textsc{adj.unm}\\
\glt `Well, I didn’t do it in an aggressive way [...].' (Sadrún, m8, \sectref{sec:8.12})
\z

\ea\label{ex:betgasynt2}
\gll    A lu sjantar vau \textbf{adina} fatg al pur, ábar ju sùn \textbf{ùṣ} bigja staus … in dals fétg buns purs.\\
and then after have.\textsc{prs.1sg.1sg} always do.\textsc{ptcp.unm} \textsc{def.m.sg} farmer but  \textsc{1sg} be.\textsc{prs.1sg} now \textsc{neg} \textsc{cop.ptcp.m.sg} {} one.\textsc{m.sg} of.\textsc{def.m.sg} very good.\textsc{m.pl} farmer.\textsc{pl}\\
\glt `And after this I have always worked as a farmer, but I’ve never been … one of the very good farmers.' (Ruèras, m1, \sectref{sec:8.2})
\z

\ea
\label{ex:betgasynt3}
\gll Èl è \textbf{grad} arivaus sé da Cuéra.\\
\textsc{3sg.m} be.\textsc{prs.3sg} just arrive.\textsc{ptcp.m.sg} up from \textsc{pn}\\
\glt `He has just arrived from Chur.' (Sadrún, m5)
\z

\ea\label{ex:betgasynt4}
\gll Ùssa, quèl da la quajda ṣaj \textbf{puspè} stauṣ ajn cul dét, miraj tschò!\\
   now \textsc{dem.m.sg} of \textsc{def.f.sg} desire be.\textsc{prs.3sg} again \textsc{cop.ptcp.m.sg} in with finger.\textsc{m.sg} look.\textsc{imp.2pl} here\\
\glt `Now the sweet-toothed one has again stuck his finger into it, look here!' (\DRGoK{4}{304})
\z

\ea
\label{ex:betgasynt5}
\gll    [...] avaun nus èra sagir al tgavrè \textbf{era} \textbf{schòn} jus culas tgauras, lèz mava lu èra.\\
{} before \textsc{1pl} be.\textsc{impf.3sg} sure \textsc{def.m.sg} goatherd also already go.\textsc{ptcp.m.sg} with.\textsc{def.f.pl} goat.\textsc{pl} \textsc{dem.m.sg} go.\textsc{impf.3sg} then also\\
\glt `[...] before us the goatherd had certainly already gone with the goats, he also used to go.' (Sadrún, m6, \sectref{sec:8.11})
\z

If two clauses which both contain a a compound verb form are conjoined, either the \isi{subject} (\ref{ex:diff:aux1}) or the \isi{subject} and the \isi{auxiliary} of the second or third verb may be omitted (\ref{ex:diff:aux3}). Note that the \isi{subject} and the \isi{auxiliary verb} may also be omitted if the second or third verb requires another \isi{auxiliary} as in the first clause. An example is (\ref{ex:diff:aux2}), where \textit{lavá} `get up' requires \textit{èssar} `be' and \textit{mirá} `look' \textit{vaj} `have'.

\ea\label{ex:diff:aux1}
\gll Ju \textbf{sùn} \textbf{juṣ} avaun nuégl ad \textbf{a} \textbf{grju} a bargju […].\\
\textsc{1sg}  be.\textsc{prs.1sg}  go.\textsc{ptcp.m.sg} before barn.\textsc{m.sg} and have.\textsc{prs.1sg} shout.\textsc{ptcp.unm} and cry.\textsc{ptcp.unm}\\
\glt `I went in front of the barn and shouted and cried.' (Ruèras, \citealt[69]{Büchli1966})
\z

\ea\label{ex:diff:aux3}
\gll Èl \textbf{ò} \textbf{príu} las duas sadjalas groma ad \textbf{è} \textbf{juṣ} òd tégja a \textbf{ṣvanjus}.\\
\textsc{3sg}  have.\textsc{prs.3sg} take.\textsc{ptcp.unm} \textsc{def.f.pl} two.\textsc{f} bucket.\textsc{pl}  cream.\textsc{f.sg} and be.\textsc{prs.3sg} go.\textsc{ptcp.m.sg} out\_of hut.\textsc{f.sg} and disappear.\textsc{ptcp.m.sg}\\
\glt `He [the devil] took the two buckets full of cream and left the hut and disappeared.' (Sèlva, \citealt[47]{Büchli1966})
\z

\ea\label{ex:diff:aux2}
\gll La damaun \textbf{èssan} aun \textbf{lavaj} baud a \textbf{mirau} da nòs tiars.\\
\textsc{def.f.sg}  morning be.\textsc{prs.1pl} still get\_up.\textsc{ptcp.m.pl} early and look.\textsc{ptcp.unm} of \textsc{poss.1pl.m.pl} animal.\textsc{pl}\\
\glt `In the morning we got up early and looked after our animals.' (Ruèras, \citealt[68]{Büchli1966})
\z

In narrative sequences where the perfect is used for storyline events, the \isi{auxiliary  verbs} may be omitted. In (\ref{ex:omit:1}), \textit{préndar} `take', \textit{magljè} `eat', and \textit{métar} `put' would take \textit{vaj} `have', in contrast to \textit{turná} `go back' and \textit{ira} `go', which would take \textit{èssar} `be'.

\ea\label{ex:omit:1}
\gll    A tschèls èran ajn stiva a dèvan tròcas né jass … a nus \textbf{príu} quèls … pinau tiar, quèls puschégns, quèlas … tablas cun sé tgarn a dal tùt … \textbf{príu} quaj ad \textbf{i} gjù ajn in clavau a \textbf{magljau} a sjantar \textbf{turnaj} sé cul cul … cun la vaschala vita a \textbf{méz} lò puspé api \textbf{i}.\\
and \textsc{dem.m.pl} \textsc{cop.impf.3pl} in living\_room.\textsc{f.sg} and give.\textsc{impf.3pl} k.o.card\_game.\textsc{f.pl} or k.o.card\_game.\textsc{m.sg} {} and \textsc{1pl} take.\textsc{ptcp.unm} \textsc{dem.m.pl} {} prepare.\textsc{ptcp.unm} by \textsc{dem.m.pl} snack.\textsc{m.pl}  \textsc{dem.f.pl} {} tray.\textsc{pl} with up meat.\textsc{f.sg} and of.\textsc{def.m.sg} all {} take.\textsc{ptcp.unm} \textsc{dem.unm} and go.\textsc{ptcp.m.pl} down in \textsc{indef.m.sg} hay\_barn and eat.\textsc{ptcp.unm} and after go\_back.\textsc{ptcp.m.pl} up with.\textsc{def.m.sg} with.\textsc{def.m.sg} {} with \textsc{def.f.sg} dishes.\textsc{f.sg} empty.\textsc{f.sg} and put.\textsc{ptcp.unm} there again and go.\textsc{ptcp.m.pl}\\
\glt `And the others were in the living room and were playing card games … and we took these … prepared, these snacks, these … trays with meat and all on it … we took this and went down into the hay barn and ate it and after we went up back with … with the empty dishes, put them there again and went away.' (Zarcúns, m2, \sectref{sec:8.13})
\z

\subsubsubsection{Gerund}\label{sec:4.1.2.1.2}

According to my consultants, the \isi{gerund} is not used any more in spoken Tuatschin. There is no occurrence of this category in the oral corpus, but it was used by the traditional story tellers whose legends were published in \citet{Büchli1966}.

The \isi{gerund} was used as a complement of a verb of perception and is introduced by \textit{a/ad} (\ref{ex:ger1} and \ref{ex:ger2}).

\ea\label{ex:ger1}
\gll    Als pástars udévan adina \textbf{a} \textbf{vagnèn} tiars.\\
   \textsc{def.m.pl} herdsman.\textsc{pl} hear.\textsc{impf.3pl} always \textsc{comp} come.\textsc{ger}  animal.\textsc{m.pl}\\
\glt `The herdsmen were always hearing cattle coming […].' (Surajn, \citealt[53]{Büchli1966})
\z

\ea\label{ex:ger2}
\gll   [...] ina sèra […] ò’ ‘ls pástars vju \textbf{ad} \textbf{èn} las vacas.\\
{} \textsc{indef.f.sg} evening {} have.\textsc{prs.3sg} \textsc{def.m.pl} herdsman.\textsc{pl} see.\textsc{ptcp.unm} \textsc{comp} go.\textsc{ger} \textsc{def.f.pl} cow.\textsc{pl} \\
\glt `[…] one evening [...] the herdsmen saw the cows going away.' (Sèlva, \citealt[28]{Büchli1966})
\z

The \isi{gerund} also introduced a non-finite \isi{causal} or \isi{temporal subordinate clause} (\ref{ex:ger3} and \ref{ex:ger4}).

\ea\label{ex:ger3}
\gll   \textbf{Raturnòn} gl' ùm bétg anavùṣ da mjadṣdé, ò’ las zarclunzas tumju […].\\
come\_back.\textsc{ger} \textsc{def.m.sg} man \textsc{neg} back of noon have.\textsc{prs.3pl} \textsc{def.f.pl} weeder\_woman.\textsc{pl} be\_afraid.\textsc{ptcp.unm}\\
\glt `Since the man hadn’t come back by noon, the weeder women got afraid […].' (Camischùlas, \citealt[82]{Büchli1966})
\z

\ea\label{ex:ger4}
\gll    \textbf{Mònd} spèl’ aua da Ségnas sé òn èlṣ udju da tschèla vard anzatgí […].\\
     go.\textsc{ger} next\_to.\textsc{def.f.sg} water of  \textsc{pn} up have.\textsc{prs.3pl} \textsc{3pl.m} hear.\textsc{ptcp.unm} of \textsc{dem.f.sg} side somebody \\
\glt `When walking along the creek from Segnas up, they heard somebody on the other side […].' (Camischùlas, \citealt[88]{Büchli1966})
\z


\subsubsubsection{Infinitive}\label{sec:4.1.2.1.3}
The \isi{infinitive} functions either as citation form of the verb or occurs in a \isi{non-finite verb phrase}. In the latter case, it may occur as the complement of a \isi{modal verb}  (\ref{ex:inf:1} and \ref{ex:inf:2}), or it in a \isi{verb phrase} introduced by the complementiser \textit{da/dad} (\ref{ex:inf:3a}).

\ea\label{ex:inf:1}
\gll  Qu’ è adina aviart a lu \textbf{saṣ} \textbf{í} \textbf{ajn} [...].\\
\textsc{dem.unm} \textsc{cop.prs.3sg} always open.\textsc{unm} and then can.\textsc{prs.2sg.gnr} go.\textsc{inf} in\\
\glt `This is always open, and then you can step in [...].' (Sadrún, m5, \sectref{sec:8.8})
\z

\ea\label{ex:inf:2}
\gll    Quèl \textbf{vès} lu aun \textbf{da} \textbf{pijè} da té al pustrètsch dal piartg tga té vèvas partgirau.\\
\textsc{dem.m.sg} have.\textsc{cond.3sg} then still to pay.\textsc{inf} \textsc{dat} \textsc{2sg} \textsc{def.m.sg} money of.\textsc{def.m.sg} pig \textsc{rel} \textsc{2sg} have.\textsc{impf.2sg} look\_after.\textsc{ptcp.unm}\\
\glt `This one should still pay you the money for the pig you had looked after.' (Sadrún, m6, \sectref{sec:8.11})
\z

\ea\label{ex:inf:3a}
\gll  «Ju \textbf{cala} \textbf{dad} \textbf{í} a \textbf{scùlèta}, ju pùs bitg í plé.»\\
\textsc{1sg} stop.\textsc{prs.1sg} \textsc{comp} go.\textsc{inf} to nursery\_school.\textsc{f.sg} \textsc{1sg} can.\textsc{prs.1sg} \textsc{neg} go.\textsc{inf} any\_more  \\
\glt `I’ll stop going to nursery school, I can’t stand it any longer.' (Sadrún, m4, \sectref{sec:8.3})
\z


The \isi{infinitive} is used in \isi{purposive clauses}, be it after a \isi{verb of movement} followed by the subordinator \textit{a} (\ref{ex:inf:3}) or after the subordinator \textit{ pr̩ /par} (\ref{ex:inf:4}).

\ea\label{ex:inf:3}
\gll    A lu, agl aucségnar … da Sadrún … è saméz sén via \textbf{par} \textbf{í} ajnta Ruèras \textbf{a} \textbf{purtá} agit a \textbf{dá} sògn jéli [...].\\
and then \textsc{def.m.sg} priest {} of \textsc{pn} {} be.\textsc{prs.3sg} \textsc{refl.}put.\textsc{ptcp.m.sg} on way.\textsc{f.sg} \textsc{subord} go.\textsc{inf} into \textsc{pn}  \textsc{subord} bring.\textsc{inf} help.\textsc{m.sg} and give.\textsc{inf} holy.\textsc{m.sg} oil\\
\glt `And then, the priest … of Sedrun … set off in order to go to Rueras and bring help and administer the sacrament of anointing [...].' (Sadrún, m6, \sectref{sec:8.5})
\z

\ea\label{ex:inf:4}
\gll [...] èl duvrava quaj mél \textbf{pr̩} \textbf{trá,} \textbf{pr̩} \textbf{trá} lèna sé da Cavòrgja.\\
{} \textsc{3sg.m} use.\textsc{impf.3sg} \textsc{dem.m.sg} mule \textsc{subord} pull.\textsc{inf} \textsc{subord} pull.\textsc{inf} wood.\textsc{coll} up from \textsc{pn}  \\
\glt `[...] he used that mule for transporting wood up from Cavorgia.' (Ruèras, m10, \sectref{sec:8.7})
\z

If a verb is fronted in order to \isi{topicalise} it, it occurs nominalised, i.e. as an \isi{infinitive}. The finite verb form remains in its initial position, but the \isi{subject} is moved after the finite verb (\ref{ex:inf:7}).

\ea\label{ex:inf:7}
\gll Na na, a \textbf{durmí} durmévan \textbf{nus} cò. \\
no no and sleep.\textsc{inf} sleep.\textsc{impf.1pl} \textsc{1pl} here\\
\glt `No, no, and as for sleeping, we would sleep here.' (Surajn, f5, \sectref{sec:8.10})
\z

In \isi{subject} sentences the \isi{infinitive} is either modified by the definite masculine singular \isi{article} (\ref{ex:inf:8}) or not (\ref{ex:inf:9}), without difference in meaning.
 
\ea
\label{ex:inf:8}
\gll \textbf{Al} \textbf{dèrgjar} \textbf{gjù} aj lu aun mal. Al Vagéli Mòn aj vagnús ṣut in caschnè.\\
\textsc{def.m.sg} demolish.\textsc{inf} down \textsc{cop.prs.3sg} then still bad.\textsc{unm} \textsc{def.m.sg} \textsc{pn} \textsc{pn} be.\textsc{prs.3sg} come.\textsc{ptcp.m.sg} under \textsc{indef.m.sg} hayrack\\
\glt `Demolishing [a hayrack] is indeed dangerous. Vigeli Monn came under a hayrack.' (Camischùlas, \DRGoK{3}{584})
\z

\ea
\label{ex:inf:9}
\gll \textbf{Dèrgjar} \textbf{gjù} in caschnè è prigulús.\\
demolish.\textsc{inf} down \textsc{indef.m.sg} hayrack \textsc{cop.prs.3sg} bad.\textsc{adj.unm}\\
\glt `Demolishing a hayrack is dangerous.' (Sadrún, m5)
\z

\subsubsection{Finite categories}\label{sec:4.1.2.2}

\subsubsubsection{Present indicative}\label{sec:4.1.2.2.1}
Present tense is formed with the verb stem and the personal endings, which means that it is a zero-marked form, in contrast to, for instance, the \isi{imperfect}, which is characterised by the suffix \textit-{áv/-év/-èv}.

Present tense is used with all verbs that refer to an event that includes the moment of speech, independently of the aktionsart of the \isi{verb}. In (\ref{ex:prs:1}) the present tense refers to an ongoing activity, in (\ref{ex:prs:2}) to a \isi{temporary state}, and in (\ref{ex:prs:3}) to a \isi{permanent state}.
                     
\ea\label{ex:prs:1}
\gll Tatlaj! Las òndas, las òlmas \textbf{dian} ... rusari gjùn basèlgja. \\
listen.\textsc{imp.2pl} \textsc{def.f.pl} aunt.\textsc{pl} \textsc{def.f.pl} soul.\textsc{pl} say.\textsc{prs.3pl} {} rosary.\textsc{m.sg} down\_in church.\textsc{f.sg}\\
\glt `Listen! The aunts, the spirits are saying ... a rosary down in the church.' (Sèlva, f2, \sectref{sec:8.6})
\z

\ea\label{ex:prs:2}
\gll Gè, \textbf{sùnd} ju ajn tju taritòri, \textbf{distùrb’} ju té?   \\
yes \textsc{cop.prs.1sg} \textsc{1sg} in \textsc{poss.2sg.m.sg} territory disturb.\textsc{prs.1sg} \textsc{1sg} \textsc{2sg} \\
\glt `Yes, am I in your territory, do I disturb you?' (Sadrún, m8, \sectref{sec:8.12})
\z

\ea\label{ex:prs:3}
\gll    La Plata dl Barlòt \textbf{è} sé Caschlè.\\
\textsc{def.f.sg} slab of.\textsc{def.m.sg} sorcery \textsc{cop.prs.3sg} up \textsc{pn}\\
\glt `The sorcery slab is at Caschlè.' (Sadrún, m6, \sectref{sec:8.5})
\z

Present tense also fulfils the function of \isi{habitual} (\ref{ex:prs:4}) or refers to other discontinuous activities (\ref{ex:prs:5}).
 
\ea\label{ex:prs:4}
\gll    A Cazis èr’ ju ajn tgòmbra, álṣò qu’ èra tgòmbras da trajs, a lu qu’ \textbf{è} adina, ina \textbf{è} gè \textbf{adina} pr̩sula, a nus trajs vèvan ábar … súpar!\\
in \textsc{pn} \textsc{cop.impf.1sg}	\textsc{1sg} in room.\textsc{f.sg} well \textsc{dem.unm} \textsc{cop.impf.3sg} room.\textsc{f.pl} of three and then \textsc{dem.unm} \textsc{cop.prs.3sg} always  one.\textsc{f.sg} \textsc{cop.prs.3sg} of\_course always alone.\textsc{f.sg} and \textsc{1pl} three have.\textsc{impf.3sg} but {} super\\
\glt `In Cazas I was in a room, well these were rooms for three, and then this was always, one [of the three] is always alone, of course, but the three of us, we had … a great time.' (Camischùlas, f6, \sectref{sec:8.4})
\z

\ea\label{ex:prs:5}
\gll    Ina \textbf{studègja} … a a Winterthur [...].\\
one.\textsc{f.sg} study.\textsc{prs.3sg} {} in in \textsc{pn}\\
\glt `One studies … in in Winterthur [...].' (Ruèras, m1, \sectref{sec:8.2})
\z

Present tense also refers to an imminent \isi{future} (\ref{ex:prs:6} and \ref{ex:prs:7}). 

\ea\label{ex:prs:6}
\gll  Ju \textbf{cala} dad \isi{í} a scùlèta, ju \textbf{pùs} bitg \isi{í} plé.\\
\textsc{1sg} stop.\textsc{prs.1sg} \textsc{comp} go.\textsc{inf} to nursery\_school.\textsc{f.sg} \textsc{1sg} can.\textsc{prs.1sg} \textsc{neg} go.\textsc{inf} any\_more  \\
\glt `I’ll stop going to nursery school, I can’t stand it any longer.' (Sadrún, m4, \sectref{sec:8.3})
\z

\ea\label{ex:prs:7}
\gll   Ju \textbf{raquénta} da mia lavur tga ju a fatg als davù̱s òns. \\
\textsc{1sg} tell.\textsc{prs.1sg} of \textsc{poss.1sg.f.sg} job \textsc{rel} \textsc{1sg}  have.\textsc{prs.1sg} do.\textsc{ptcp.unm} \textsc{def.m.pl} last.\textsc{pl} year.\textsc{pl}\\
\glt `I’ll tell [you] about the job I've done for the past few years.' (Sadrún, f3, \sectref{sec:8.1})
\z

Present tense is the usual way to refer to \isi{future} situations of any type (\ref{ex:prs:8}).

\ea\label{ex:prs:8}
\gll   \textbf{Damaun} / \textbf{Ajn} \textbf{duṣ} \textbf{òns} fagjajn nus quaj. \\
tomorrow {} in two.\textsc{m} year.\textsc{pl} do.\textsc{prs.1pl} \textsc{1pl} \textsc{dem.unm}\\
\glt `Tomorrow / In two years we'll do that.' (Sadrún, m10)
\z

There are also instances of narrative present whose function is to render the story more vivid (\ref{ex:narr:prs}).

\ea\label{ex:narr:prs}
\gll    A lu, agl aucségnar … da Sadrún … \textbf{è} \textbf{saméz} sén via par í ajnta Ruèras a purtá agit a dá sògn jéli né al davùs sacramèn tga \textbf{dèvan} da quels … mòribù̱nds, basta, agl aucségnar \textbf{végn} atrás … Zarcúns a lu \textbf{auda} `l las stréjas sé cò séssum la val da Lòndadusa \textbf{òni} \textbf{clumau}:\\
and then \textsc{def.m.sg} priest {} of \textsc{pn} {} be.\textsc{prs.3sg}  \textsc{refl.}put.\textsc{ptcp.m.sg} on way.\textsc{f.sg} \textsc{subord} go.\textsc{inf} into \textsc{pn}  \textsc{subord} bring.\textsc{inf} help.\textsc{m.sg} and give.\textsc{inf} holy.\textsc{m.sg} oil or \textsc{def.m.sg} last sacrament \textsc{rel} give.\textsc{impf.3pl} \textsc{dat} \textsc{dem.m.pl} {} dying.\textsc{pl} enough \textsc{def.m.sg} priest come.\textsc{prs.3sg} through {} \textsc{pn} and then hear.\textsc{prs.3sg} \textsc{3sg.m} \textsc{def.f.pl} witch.\textsc{pl} up here uppermost \textsc{def.f.sg} valley of \textsc{pn} have.\textsc{prs.3pl.3pl} call.\textsc{ptcp.unm}\\
\glt `And then, the priest … of Sedrun … set off in order to go to Rueras and bring help and administer the sacrament of anointing or the Holy Sacrament they would give to those … dying people. Well, the priest comes through Zarcuns and then he hears the witches up there, they called from the uppermost part of the Londadusa valley:' (Sadrún, m6, \sectref{sec:8.5})
\z

In this example, the first verb referring to story line events is modified by the perfect tense (\textit{è saméz sén via}), the two verbs that follow are modified by the present tense (\textit{végn} and \textit{auda}); the last one (\textit{òni clumau}) is again modified by the perfect tense.

\subsubsubsection{Imperfect indicative}\label{sec:4.1.2.2.2}
The \isi{imperfect indicative} is formed by the suffix \textit{-áv/-év/-èv}. The distribution of the allomorphs is as follows. \textit{-áv} is used with all verbs ending in \textit{--á} and with most verbs ending in \textit{-è},  \textit{-év} is used with all verbs ending in \textit{-ar} and \textit{-í}, and \textit{-èv} is used with some verbs ending in \textit{-è}, with most irregular verbs ending in \textit{-aj}, and with some other irregular verbs.

The basic functions of the \isi{imperfect indicative} are to refer to imperfective aspect in the past with all types of lexical aspect (\ref{ex:impf:1} and \ref{ex:impf:2}),\footnote{But see (\ref{ex:prf:inch:1} -- \ref{ex:prf:inch:3}).} to past \isi{habitual} (\ref{ex:impf:3} and \ref{ex:impf:4}), or to an unspecified repetition of actions in the past (\ref{ex:impf:5}).
	
\ea\label{ex:impf:1}
\gll Api grad \textbf{ajn} \textbf{quèl} \textbf{mumèn} \textbf{vagnév’} in cégn … gròn ni– vi datiar ad èra lò usché in téc dòmina̱nt.\\
and exactly in \textsc{dem.m.sg} moment come.\textsc{impf.3sg} \textsc{indef.m.sg} swan {} big.\textsc{m.sg.unm} or over next\_to and \textsc{cop.impf.3sg} there so \textsc{indef.m.sg} bit dominant.\textsc{adj.unm}\\
\glt `And precisely at that moment a big swan … was coming to the place where I was, a bit a dominant one. (Sadrún, m8, \sectref{sec:8.12})'
\z

\ea\label{ex:impf:2}
\gll    Ju \textbf{lèv’} amprè̱ndar da majstar [...].\\
\textsc{1sg} want.\textsc{impf.1sg} learn.\textsc{inf} of joiner.\textsc{m.sg}\\
\glt `I wanted to become a joiner [...].' (Ruèras, m1, \sectref{sec:8.2})
\z

\ea\label{ex:impf:3}
\gll    Nus \textbf{mavan} la damaun api \textbf{vagnévan} la sèra. \\
\textsc{1pl} go.\textsc{impf.1pl}  \textsc{def.f.sg} morning and come.\textsc{impf.1pl} \textsc{def.f.sg} evening \\
\glt `We would go in the morning and come back in the evening.' (Surajn, f5, \sectref{sec:8.10})
\z

\ea\label{ex:impf:4}
\gll    A nus \textbf{mavan} culs pòrs sé Valtgèva, \textbf{mintga} \textbf{dé} sé a gjù, ju savès raquintá da té quaj.\\
and \textsc{1pl}  go.\textsc{impf.1pl} with.\textsc{def.m.pl} pig.\textsc{pl} up \textsc{pn} every day.\textsc{m.sg} up and down  \textsc{1sg}  can.\textsc{cond.1sg}  tell.\textsc{inf}  \textsc{dat}  \textsc{2sg} \textsc{dem.unm}\\
\glt `And we would go up to Valtgeva with the pigs, every day up and down, I could tell you about that.' (Sadrún, m6, \sectref{sec:8.11})
\z

\ea\label{ex:impf:5}
\gll  Api lura … ju \textbf{mava} è \textbf{mintgataun} cun èl a dá culur las sèndas [...].\\
and then {} \textsc{1sg} go.\textsc{impf.1sg} also sometimes with \textsc{3sg.m} \textsc{subord} give.\textsc{inf} colour.\textsc{f.sg} \textsc{def.f.pl} trail.\textsc{pl}\\
\glt `And then … from time to time I would go with him to give colour [to the stones indicating] the trails [...].' (Sadrún, f3, \sectref{sec:8.1})
\z


\subsubsubsection{Perfect indicative}\label{sec:4.1.2.2.3}
The perfect is formed with the \isi{auxiliary verbs} \isi{\textit{èssar}} `be' or \isi{\textit{vaj}} `have' and the \isi{past participle}. If the verb is conjugated with \textit{èssar}, the participle agrees with the \isi{subject} in \isi{gender} and \isi{number}.

The following verbs are conjugated with \textit{èssar}:

\begin{itemize}
	
	\item \isi{intransitive motion verbs}: \textit{curdá} `fall', \textit{dá gjù} `fall down', \textit{í} `go', \textit{mitschá} `escape', \textit{ruclá} `fall down', \textit{saglí} `run', \textit{scapá} `escape', \textit{séjṣar gjù} `sit down', \textit{ṣgulá} `fly', \textit{ṣvaní} `disappear', and \textit{vagní} `come'
	\item verbs of state: \textit{èssar} `be', \textit{rastá} `remain', and \textit{vívar} `live'
	\item change-of-state verbs: \textit{capitá} `happen',  \textit{crèschar} `grow', \textit{maridá} `get married', \textit{murí} `die', \textit{néschar} `be born', and \textit{schabagjá} `happen'
	\item \isi{reflexive} verbs
	\item \isi{passive} verbs.
	\end{itemize}

The main function of the perfect is to express perfective aspect, i.e. to refer to the whole situation with beginning, middle, and end, with or without a relation to the present (\ref{ex:perf1}--\ref{ex:perf3}).

\ea\label{ex:perf1}
\gll    Ju \textbf{sùnd} \textbf{jus} sé Culmatsch ina dumèngja. \\
\textsc{1sg} be.\textsc{prs.1sg} go.\textsc{ptpc.m.sg} up \textsc{pn} \textsc{indef.f.sg} Sunday\\
\glt `One Sunday I went up to Culmatsch.' (Surajn, \citealt[128]{Büchli1966})
\z

\ea
\label{ex:perf2}
\gll  [...] api \textbf{vòu} \textbf{anflau} in bi ljuc [...].\\
 {} and have.\textsc{prs.1sg.1sg} find.\textsc{ptcp.m.unm} \textsc{indef.m.sg} beautiful.\textsc{m.sg} place\\
\glt `[...] and then I found a nice place [...].' (Sadrún, m8, \sectref{sec:8.12})
\z

\ea\label{ex:perf3}
\gll    A lu, agl aucségnar … da Sadrún … \textbf{è} \textbf{saméz} sén via par \isi{í} ajnta Ruèras [...].\\
and then \textsc{def.m.sg} priest {} of \textsc{pn} {} be.\textsc{prs.3sg} \textsc{refl.}put.\textsc{ptcp.m.sg} on way.\textsc{f.sg} \textsc{subord} go.\textsc{inf} into \textsc{pn}\\
\glt `And then, the priest … of Sedrun … set off in order to go to Rueras [...].' (Sadrún, m6,\sectref{sec:8.5})
\z

In Romance languages like French, when the perfective tenses modify a \isi{stative verb} like \textit{connaître} `know' or \textit{savoir} `know', it usually has an \isi{inchoative} meaning: \textit{J'ai connu Michel à une fête.} `I met Michel at a party', or \textit{J'ai su qu'elle était malade.} `I was told that she was ill'. But in Tuatschin, the perfect is used with these stative verbs (which take the form \textit{ancanùschar} and \textit{savaj}) without an \isi{inchoative} meaning (\ref{ex:prf:inch:1}--\ref{ex:prf:inch:2}). In other words, in these cases the verbs refer imperfectively to the situation, which is underlined by the use of the adverb \textit{schòn} `already' in (\ref{ex:prf:schon}).

\ea\label{ex:prf:inch:1}
\gll    Èl \textbf{ò} \textbf{ancanùschju} la familja, mù maj détg òra tgi èri. \\
     \textsc{3sg.m} have.\textsc{prs.3sg} know.\textsc{ptcp.unm} \textsc{def.f.sg} family but never tell.\textsc{ptcp.unm} out who \textsc{cop.impf.sbjv.3sg}\\
\glt `He knew the family, but never said who they were.' (Bugnaj, \citealt[139]{Büchli1966})
\z

\ea\label{ex:prf:schon}
\gll   Al buéb \textbf{ò} schòn \textbf{ancanuschju} èlas. \\
     \textsc{def.m.sg} boy have.\textsc{prs.3sg} already know.\textsc{ptcp.unm} \textsc{3pl.f}\\
\glt `The boy already knew them [= the girls].' (Sadrún, \citealt[103]{Büchli1966})
\z

\ea\label{ex:prf:inch:2}
\gll    La fuméglja d’ alp \textbf{ò} \textbf{savju} nuét.\\
     \textsc{def.f.sg} farmhand.\textsc{coll} of alp have.\textsc{prs.3sg} know.\textsc{ptcp.unm} nothing\\
\glt `The alp shepherds didn’t know anything.' (Cavòrgja, \citealt[53]{Büchli1966})
\z

To get the \isi{inchoative} meaning, Tuatschin uses \textit{amprèndar} \textit{d'} \textit{ancanùschar} (\ref{ex:prf:inch1}), literally `learn to know', and \textit{udí} (\ref{ex:prf:inch2}) `hear'.

\ea
\label{ex:prf:inch1}
\gll  Api, ah, quaj ah fascinava pròpi mè, ju vèṣ ah gè ju vèṣ è ugèn \textbf{ampríu} \textbf{d’} \textbf{ancanùschar} quaj mél, ábar ju ... sùn halt naschjus mèmja tart. \\
and eh \textsc{dem.unm} eh fascinate.\textsc{impf.3sg} really \textsc{1sg}  \textsc{1sg} have.\textsc{cond.1sg} ah yes \textsc{1sg} have.\textsc{cond.1sg} also with\_pleasure learn.\textsc{ptcp.m.unm} \textsc{comp} know.\textsc{inf} \textsc{dem.m.sg} mule but \textsc{1sg} {} be.\textsc{prs.1sg} just be\_born.\textsc{ptcp.m.sg} too late\\
\glt `And, eh, this really fascinated me, I would have eh yes I would have very much liked to get to know this mule, but I ... was just born too late.' (Ruèras, 10, \sectref{sec:8.7})
\z

\ea
\label{ex:prf:inch2}
\gll Ju a \textbf{udju} tg' èl \textbf{ségi} mazauns.\\
\textsc{1sg} have.\textsc{prs.1sg} hear.\textsc{ptcp.unm} \textsc{comp} \textsc{3sg.m} \textsc{cop.prs.sbjv.3sg} ill.\textsc{m.sg}\\
\glt `I was told that he is ill.' (Ruèras, m10)
\z

As for \isi{\textit{vaj}} `have', there is no difference between the use of the perfect or the \isi{imperfect}, at least according to the native speakers I have consulted. Both the perfect in (\ref{ex:prf:inch:3}) and the \isi{imperfect} in (\ref{ex:prf:inch:4}) could be interpreted as \isi{inchoative} or as a \isi{permanent state}.

\ea
\label{ex:prf:inch:3}
\gll Quaj è stau ina ... fétg grònda familja, èlṣ \textbf{òn} \textbf{gju} indiṣch ufauns [...].\\
\textsc{dem.unm}  be.\textsc{prs.3sg}  \textsc{cop.ptcp.m.unm}  \textsc{indef.f.sg} {} very big family \textsc{3pl.m} have.\textsc{prs.3pl} have.\textsc{ptcp.m.unm} eleven child.\textsc{m.pl}\\
\glt `This was a ... very big family, they had eleven children [...].' (Ruèras, m4, \sectref{sec:8.3})
\z

\ea
\label {ex:prf:inch:4}
\gll Èls \textbf{vèvan} indiṣch ufauns.\\
\textsc{3pl.m} have.\textsc{impf.3pl} eleven child.\textsc{m.pl}\\
\glt `They had eleven children.' (Sadrún, m5)
\z

As seen in (\ref{ex:omit:1}) in \sectref{sec:4.1.2.1.1}, story-line events can also be referred to only with the \isi{past participle}, without the \isi{auxiliary verbs} \textit{èssar} or \textit{vaj}.


\subsubsubsection{Pluperfect indicative}\label{sec:4.1.2.2.4}
The pluperfect fulfils the function of indicating the perfective aspect of a situation that is situated before another situation in the past (\ref{ex:plupf1}--\ref{ex:plupf3}).

\ea
\label{ex:plupf1}
\gll  Agl Andréòli \textbf{vèva} \textbf{finju} … las ... figuras … \textbf{avaun} \textbf{ca} la caplùta \textbf{èra} \textbf{stada} \textbf{finida}.\\
\textsc{def.m.sg} \textsc{pn} have.\textsc{impf.3sg} finish.\textsc{ptcp.unm} {} \textsc{def.f.pl} {} figure.\textsc{pl} {} before \textsc{rel} \textsc{def.f.sg} chapel be.\textsc{impf.3sg} \textsc{pass.ptcp.f.sg} finish.\textsc{ptcp.f.sg}\hspace*{-2mm}\\
\glt `Andreoli had finished ... the ... figures ... before the chapel was finished.' (Sadrún, m5, \sectref{sec:8.8})
\z

\ea
\label{ex:plupf2}
\gll  Qu' è lu ju quèluisa tgé ca nuṣ \textbf{èssan} \textbf{vagní} vidòra, \textbf{turnaj} ò da Pardatsch, tg' èssan nus staj ajn lò fòrsa … quátar tschun jamnas. Scha \textbf{vèva} `l \textbf{fatg} ṣchùber nuét. Quèla fascha \textbf{èra}  \textbf{satratg’} ansjaman [...]. \\
\textsc{dem.unm} be.\textsc{prs.3sg} then go.\textsc{ptcp.unm} such\_way \textsc{comp} when \textsc{1pl} be.\textsc{impf.1pl} come.\textsc{ptcp.m.pl} out return.\textsc{ptcp.m.pl} out of \textsc{pn} \textsc{comp} be.\textsc{prs.1pl} \textsc{1pl} \textsc{cop.ptcp.m.pl} in there maybe {} four five week.\textsc{f.pl} but have.\textsc{impf.3sg} \textsc{3sg.m} do.\textsc{ptcp.unm} clean.\textsc{adj.unm} nothing \textsc{dem.f.sg} bandage be.\textsc{impf.3sg} \textsc{refl.}contract.\textsc{impf.3sg} together \\
\glt `This happened in such a way that when we returned down [to Surrein] from Pardatsch, then we had stayed there maybe … four or five weeks. But he hadn’t done anything at all. That bandage had contracted [...].' (Sadrún, m4, \sectref{sec:8.3})
\z

\ea
\label{ex:plupf3}
\gll   A nuṣ \textbf{vajn} \textbf{gju} schi súpar. Ju èr' ùs, ah, ju \textbf{vèva} \textbf{gju} ajnsasèz al clégj dad èssar ajn tgòmbra cun ròmò̱ntschas.\\
and \textsc{1pl} have.\textsc{prs.1pl} have.\textsc{ptcp.unm} so super \textsc{1sf} \textsc{cop.impf.1sg} now ah \textsc{1sg} have.\textsc{impf.1sg} have.\textsc{ptcp.m.unm} in\_fact \textsc{def.m.sg} luck \textsc{comp} \textsc{cop.inf} in room.\textsc{f.sg} with Romansh.\textsc{f.pl}\\
\glt `And we had such a wonderful time. I was now, eh, in fact I had been lucky to share the room with Romansh girls.' (Camischùlas, f6, \sectref{sec:8.4})
\z

\subsubsubsection{Future}\label{sec:4.1.2.2.5}
According to my consultants, the \isi{future} is almost never used; in order to refer to a \isi{future} situation, present tense is used. The only example of the \isi{future} in the oral corpus is (\ref{ex:fut1}).

\ea\label{ex:fut1}
\gll Ad ana d' òtgòntasjat vajn nus gju ina vòtazjun fadarala ṣur da las sèndas, sch' i \textbf{végn} \textbf{a} \textbf{prèndar} \textbf{ajn} quaj né bétg.   \\
and year.\textsc{f.sg} of eighty-seven have.\textsc{prs.1pl} \textsc{1pl} have.\textsc{ptcp.m.unm} \textsc{indef.f.sg} vote federal over of \textsc{def.f.pl} trail.\textsc{pl} whether \textsc{expl} \textsc{fut.aux.3sg} \textsc{comp} take.\textsc{inf} in \textsc{dem.unm} or \textsc{neg}  \\
\glt `And in 1987 we had a federal vote about the trails, [about] whether it would be adopted or not.' (Sadrún, f3, \sectref{sec:8.1})
\z

\subsubsubsection{Doubly-compound tenses}\label{sec:4.1.2.2.6}
There are two \isi{doubly-compound tenses}: perfect (\ref{ex:dcomp:1} and \ref{ex:dcomp:2}) and pluperfect (\ref{ex:dcomp:4}). They usually fulfil the same functions as the simple compound tenses, but they express a longer temporal distance in the past. Note that in (\ref{ex:dcomp:4}), the function of the \isi{doubly-compound pluperfect} is to express the \isi{habitual}, a function which is usually fulfilled by the \isi{imperfect}.

\ea
\label{ex:dcomp:1}
\gll    Ábar tschaj è bi, ju \textbf{a} lu sjantar \textbf{gju} … \textbf{calau} da fá `l pur tgu vèva tgéj? … tschuncònt’ òns.\\
but \textsc{dem.unm} \textsc{cop.prs.3sg} nice.\textsc{adj.unm} \textsc{1sg} have.\textsc{prs.1sg} then after have.\textsc{ptcp.unm} {} stop.\textsc{ptcp.unm} \textsc{comp} do.\textsc{inf} \textsc{def.m.sg} farmer when.\textsc{rel.1sg} have.\textsc{impf.1sg} what {} fifty year.\textsc{m.pl}\\
\glt `But that is nice, I then had … stopped working as a farmer when I was … fifty years old.' (Ruèras, m1, \sectref{sec:8.2})
\z

\ea
\label{ex:dcomp:2}
\gll A gl òn ca tg’ \textbf{òn} \textbf{gju} \textbf{dépònju} quèlas ah figuras ò inṣ adina détg «la stiva dals gjadjus», ò quèla gju nùm sjantar.\\
and \textsc{def.m.sg} year \textsc{rel} \textsc{rel} have.\textsc{prs.3sg} have.\textsc{ptcp.m.unm} store.\textsc{ptcp.m.unm} \textsc{dem.f.pl} eh figure.\textsc{pl} have.\textsc{prs.3sg} \textsc{gnr} always say.\textsc{ptcp.m.unm} \textsc{def.f.sg} living\_room of.\textsc{def.m.pl} Jew.\textsc{pl} have.\textsc{prs.3sg} \textsc{dem.f.sg} have.\textsc{ptcp.m.unm} name after\\
\glt `And [since] the year they stored these eh figures one has always said «the living room of the Jews», has it been called since.' (Sadrún, m5, \sectref{sec:8.8})
\z

\ea
\label{ex:dcomp:4}
\gll    A … ad in òn, sa ju aun bégn, lu \textbf{vèvan} nus lu \textbf{gju} \textbf{fatg} in tòc humòrístic  da la músic’ anòra.\\
and {} and  \textsc{indef.m.sg} year know.\textsc{prs.1sg} \textsc{1sg} still well then have.\textsc{impf.1pl} \textsc{1pl} then have.\textsc{ptcp.unm} do.\textsc{ptcp.unm} \textsc{indef.m.sg} prank funny from \textsc{def.f.sg} music out\\
\glt `And … and one year, I still know very well, we from the music had played a funny prank.' (Zarcúns, m2, \sectref{sec:8.13})
\z

\subsubsubsection{Progressive aspect}\label{sec:4.1.2.2.7}
The \isi{progressive aspect} is formed with the copula \textit{èssar}, the preposition \textit{vid(a)} ‘at’, with (\ref{ex:prog:with}) or without (\ref{ex:prog:without}) the masculine singular \isi{definite article}, and the \isi{infinitive}.

\ea\label{ex:prog:with}
\gll    Duas zarclunzas \textbf{èran} \textbf{vid} `\textbf{l} \textbf{zarclá} [...].\\
     two.\textsc{f} weeder\_woman.\textsc{pl} \textsc{cop.impf.3pl} \textsc{prog} \textsc{def.m.sg} weed.\textsc{inf}\\
\glt `Two weeder women were weeding […].' (Bugnaj, \citealt[132]{Büchli1966})
\z

\ea\label{ex:prog:without}
\gll  Api quèls da la vischnaunca \textbf{èran} grad vida `l, \textbf{vida} \textbf{zaná} al bògn [...].  \\
and \textsc{dem.m.pl} of \textsc{def.f.sg} municipality \textsc{cop.impf.3pl} just \textsc{prog} \textsc{def.m.sg} \textsc{prog} renovate.\textsc{inf} \textsc{def.m.sg} bath\\
\glt `And the municipal employees were just renovating the swimming pool [...].' (Sadrún, f3, \sectref{sec:8.1})
\z

\subsubsubsection{Present and perfect subjunctive}\label{sec:4.1.2.2.8}
{Subjunctive mood}, be it present, perfect, or imperfect, is characterised by the suffix \textit{-i(-)}\footnote{See \sectref{sec:4.1.1.2.1}.}

Subjunctive mood mostly occurs in some types of \isi{object clauses} and in \isi{adjunct clauses} introduced by \textit{avaun tga} `before', \textit{par tga} `in order to', \textit{tòca tga} `until', or \textit{sènza tga} `without that'.\footnote{The most thorough analysis of mood in \ili{Standard Sursilvan} is \citet{Grünert2003}.} In the corpus, \isi{subjunctive mood} occurs in three tenses: present, perfect, and imperfect. Subjunctive imperfect will be treated in the next section.

The most important subjunctive triggers occurring in the corpus are

\begin{itemize}
\item (a) \isi{verbs of speaking}: \textit{dí} `say', \textit{dumandá} `ask', \textit{raquintá} `tell', and \textit{udí} `hear, be told';
\item (b) \isi{verbs of opinion}: \textit{craj} `believe, think', \textit{paraj} `seem', \textit{tanaj} `think, hold', and \textit{tartgè} `think';
\item (c) \isi{directive speech act} verbs and desiderative verbs: \textit{fá stém} and \textit{mirá}, both `make sure', \textit{rujè} `ask', and \textit{vulaj} `want';
\item (d) \isi{purposive} subordinators: \textit{par tga} and \textit{tga}, both `in order to';
\item (e) the conjunctions \textit{avaun (ca) tga} `before', \textit{sènza tga} `without', and \textit{tòcan} `until'.
\end{itemize}

In object clauses the complementiser is often absent (\ref{ex:subj1} and \ref{ex:subj8}). Examples (\ref{ex:subj1}--\ref{ex:subj4}) illustrate the use of subjunctive mood, present and perfect, with \isi{verbs of speaking}.

\ea
\label{ex:subj1}
\gll  A lu \textbf{ò} `l \textbf{détg} {\longrule} èl \textbf{sapi} bigja vagní da luòra, ju, èl {\longrule} \textbf{stètgi} mal, èl {\longrule} \textbf{mòndi} da via òra [...].  \\
and then have.\textsc{prs.3sg} \textsc{3sg.m} say.\textsc{ptcp.unm} {} \textsc{3sg.m} can.\textsc{prs.sbjv.3sg} \textsc{neg} come.\textsc{inf} from there\_out \textsc{1sg} \textsc{3sg.m} {} stay.\textsc{prs.sbjv.3sg} bad \textsc{3sg.m} {} go.\textsc{prs.sbjv.3sg} from road.\textsc{f.sg} out\\
\glt `And then he said he couldn’t walk on that path, that I - that he was sorry, [but] that he would walk on the road [...].' (Ruèras, m10, \sectref{sec:8.7})
\z

\ea
\label{ex:subj2}
\gll Lu dumandavan nuṣ èl, \textbf{vèvan} \textbf{dumandau} núa èl \textbf{ségi} stauṣ ajn plaza [...].\\
then ask.\textsc{impf.1pl} \textsc{1pl} \textsc{3sg.m} have.\textsc{impf.3sg}  ask.\textsc{ptcp.unm} where \textsc{3sg.m} be.\textsc{prs.sbjv.3sg} \textsc{cop.ptcp.m.sg} in job.\textsc{f.sg}\\
\glt `Then we would ask him, we had asked [him] where he had been working [...].' (Sadrún, m4, \sectref{sec:8.3})
\z

\ea
\label{ex:subj3}
\gll    A la détga \textbf{raquénta} … tga las stréjas dl Caschlè tg’ èran sé cò a fijèvan barlòt \textbf{vágian} \textbf{trans-pòrtau} quèla plata sin\footnotemark{} in fil-sajda [...].\\
and \textsc{def.f.sg} legend tell.\textsc{prs.3sg} {} \textsc{comp} \textsc{def.f.pl} witch.\textsc{pl} \textsc{def.m.sg} \textsc{pn} \textsc{rel} \textsc{cop.impf.3pl} up here and do.\textsc{impf.3pl} sorcery have.\textsc{prs.sbjv.3pl} carry.\textsc{ptcp.unm} \textsc{dem.f.sg} slab on \textsc{indef.m.sg} thread-silk\\
\glt `And the legend says … that the witches of the Caschlè which were up there and used to do sorcery had carried this slab on a … silk thread [...].'\footnotetext{\textit{sin} instead of \textit{sén}} (Sadrún, m6, \sectref{sec:8.5})
\z

\ea
\label{ex:subj4}
\gll Ju a \textbf{udju} tg' èl \textbf{ségi} mazauns.\\
\textsc{1sg} have.\textsc{prs.1sg} hear.\textsc{ptcp.unm} \textsc{comp} \textsc{3sg.m} \textsc{cop.prs.sbjv.3sg} ill.\textsc{m.sg}\\
\glt `I was told that he was ill.' (Ruèras, m10)
\z

Subjunctive mood is also used in \isi{free indirect speech}, which is characterised by the lack of an introductory verb (\ref{ex:free.ind1}).

\ea
\label{ex:free.ind1}
\gll  Èl \textbf{èri} avaun caplùta a \textbf{vagi} \textbf{vju} tga quèls méls èn saspuantaj, api \textbf{vagi} èl \textbf{tartgau} ... dad í vi ajn via ... a tanaj sé èls.  \\
\textsc{3sg.m} \textsc{cop.impf.sbjv.3sg} in\_front chapel and have.\textsc{prs.sbjv.3sg} see.\textsc{ptcp.unm} \textsc{comp} \textsc{dem.m.pl} mule.\textsc{pl} be.\textsc{prs.3pl} \textsc{refl.}frighten.\textsc{ptcp.m.pl} and have.\textsc{prs.sbjv.3sg} \textsc{3sg.m} think.\textsc{ptcp.unm} {} \textsc{comp} go.\textsc{inf} over on road.\textsc{f.sg} {} and hold.\textsc{inf} up \textsc{3pl.m}\\
\glt `He was in front of the chapel and had seen that these mules ran away and he thought ... that he would go on the road ... and stop them.' (Ruèras, m10, \sectref{sec:8.7})
\z

In (\ref{ex:free.ind2}) and (\ref{ex:free.ind3}), the sentence starts with a \isi{verb} in \isi{indicative mood}, which represents the words of the narrator, and then goes on in \isi{subjunctive mood}, which represents the words of the army.

\ea
\label{ex:free.ind2}
\gll  L’ autar dé va ju gju la lubiantscha dad í vidajn, ábar \textbf{stòpi} prèndar malitèr cun mè, tga \textbf{vajan} … fùnc a \textbf{sápian} prèndar ah, dí cu nus \textbf{vajan} da … ir davùṣ in cuélm.\\
\textsc{def.m.sg} other day have.\textsc{prs.1sg} \textsc{1sg} have.\textsc{ptcp.unm} \textsc{def.f.sg} permission \textsc{attr} go.\textsc{inf} in but must.\textsc{prs.sbjv.1sg}  take.\textsc{inf} military.\textsc{m.sg} with \textsc{1sg} \textsc{rel} have.\textsc{prs.sbjv.3pl} {} radio.\textsc{m.sg} and can.\textsc{prs.sbjv.3pl} take.\textsc{inf} eh say.\textsc{inf} when \textsc{1pl} have.\textsc{prs.sbjv.1pl} to {} go.\textsc{inf} behind \textsc{indef.m.sg} mountain\\
\glt `The day after I got permission to go there, but I needed to take with me some soldiers that had a radio and would say when we should … go behind a mountain [to protect ourselves].' (Sadrún, f3, \sectref{sec:8.1})
\z

\ea
\label{ex:free.ind3}
\gll [...] a mintga vaschnaunca \textbf{ò} lu \textbf{stavju} dá ajn tùt tgé ca la \textbf{vagi}, nùca la \textbf{vagi} lògans cun mussavias, a las sèndas, tùt.\\
{} and every municipality.\textsc{f.sg} have.\textsc{prs.3sg} then must.\textsc{ptcp.unm} give.\textsc{inf} in all what \textsc{rel} \textsc{3sg.f} have.\textsc{prs.sbjv.3sg} where \textsc{3sg.f} have.\textsc{prs.sbjv.3sg} place.\textsc{m.pl} with signpost.\textsc{f.pl} and \textsc{def.f.pl} trail.\textsc{pl} all \\
\glt `[...] and every municipality had then to inform about everything they had, where they had places with signposts and trails, everything.' (Sadrún, f3, \sectref{sec:8.1})
\z

Examples (\ref{ex:subj5}--\ref{ex:subj8}) illustrate the use of \isi{subjunctive mood} with \isi{verbs of opinion}.

\ea
\label{ex:subj5}
\gll  Da mé \textbf{par}' \textbf{aj} tg' al ajfar-piast da véjdar \textbf{èri} bétga schi lads.\\
\textsc{dat} \textsc{1sg} seem.\textsc{prs.3sg} \textsc{expl} \textsc{comp} \textsc{def.m.sg} hayrack\_post  of old.\textsc{adj.unm} \textsc{cop.impf.sbjv.3sg} \textsc{neg} so wide.\textsc{m.sg} \\
\glt `It seems to me that the hayrack posts of earlier times were not so wide.' (\DRGoK{3}{582})
\z

\ea
\label{ex:subj6}
\gll  Ju \textbf{tégn} tga quaj \textbf{végni} fatg pauc.  \\
\textsc{1sg} hold.\textsc{prs.1sg} \textsc{comp} \textsc{dem.unm} \textsc{pass.aux.prs.sbjv.3sg} make.\textsc{ptcp.unm} little\\
\glt `I think that this is not often done.' (Ruèras, \DRGoK{1}{393})
\z

\ea
\label{ex:subj8}
\gll    Api sjantar \textbf{vajn} nus \textbf{tartgau} {\longrule} nus \textbf{sápian} durmí òra [...].\\
and after have.\textsc{prs.1pl} \textsc{1pl} think.\textsc{ptcp.unm} {} \textsc{1pl}  can.\textsc{prs.sbjv.1pl} sleep.\textsc{inf} out\\
\glt `And then we thought we would have a good sleep [...].' (Camischùlas, f6, \sectref{sec:8.4})
\z

Examples (\ref{ex:subj9}--\ref{ex:subj12}) show the use of \isi{subjunctive mood} with \isi{directive speech act} verbs.

\ea
\label{ex:subj9}
\gll  \textbf{Mira} tga quaj lò \textbf{davjanti} bétg. \\
look.\textsc{imp.2sg} \textsc{comp} \textsc{dem.unm} there become.\textsc{prs.sbjv.3sg} \textsc{neg}\\
\glt `Make sure that this does not going to happen.' (\DRGoK{5}{535})
\z

\ea
\label{ex:subj10}
\gll  Té \textbf{mira} lu tg’ al tat \textbf{fétschi} lu mintga dé, \textbf{préndi} gjù quaj a \textbf{ṣchubrègi} a \textbf{fétschi} sé da néjf.\\
\textsc{2sg} look.\textsc{imp.2sg} then \textsc{comp} \textsc{def.m.sg} grandfather do.\textsc{prs.sbjv.3sg} then every day.\textsc{m.sg} take.\textsc{prs.sbjv.3sg} down \textsc{dem.unm} and clean.\textsc{prs.sbjv.3sg} and do.\textsc{prs.sbjv.3sg} up of new.\textsc{adj.unm} \\
\glt `And you, make sure that your grandfather does it every day, that he takes them off, that he cleans them and puts them on again.' (Sadrún, m4, \sectref{sec:8.3})
\z

\ea
\label{ex:subj11}
\gll [...] lò végni [...] \textbf{rujau} tgé Nòssadùna \textbf{laschi} madirá bégn al graun ajn Tujétsch.\\
{} there \textsc{pass.aux.prs.3sg.expl} {} ask.\textsc{ptcp.unm} \textsc{comp} Our\_Lady.\textsc{f.sg} let.\textsc{prs.sbjv.3sg} ripen.\textsc{inf} well \textsc{def.m.sg} cereals in \textsc{pn}\\
\glt `[...] there they pray that the Virgin Mary let grow well the cereals in the Tujetsch Valley.' (Camischùlas, \citealt[94]{Büchli1966})
\z

\ea
\label{ex:subj12}
\gll Ju \textbf{vi} bétg tga la tgèsa \textbf{ardi}.\\
\textsc{1sg} want.\textsc{prs.1sg} \textsc{neg} \textsc{comp} \textsc{def.f.sg} house burn.\textsc{prs.sbjv.3sg}\\
\glt `I don't want the house to burn.' (Sadrún, m5)
\z

In purposive clauses, the conjunctions \textit{par tga} or \textit{tga} are used (\ref{ex:subj13}--\ref{ex:subj15}).

\ea
\label{ex:subj13}
\gll  A la sèra \textbf{par} \textbf{tga} \textbf{briṣchi} bétg … vagnéva quaj, quaj mava `l ajnagjù cul maun èra sènza … [vòns] a trèva vidò̱ còtgla gjù sé sél plantschju.\\
and \textsc{def.f.sg} evening \textsc{subord} \textsc{comp} burn.\textsc{prs.sbjv.3sg} \textsc{neg} {} \textsc{pass.aux.impf.3sg} \textsc{dem.unm} \textsc{dem.unm} go.\textsc{impf.3sg} \textsc{3sg.m} in\_down with.\textsc{def.m.sg} hand also without {} [glove.\textsc{m.pl}] and pull.\textsc{impf.3sg} out charcoal.\textsc{coll} down up on.\textsc{def.m.sg} floor  \\
\glt `And in the evening, to avoid it burning … was that, there he went into [the fire] with one hand, also without [gloves], and pulled out charcoal from down there up to the floor.' (Sadrún, m4, \sectref{sec:8.3})
\z

\ea
\label{ex:subj14}
	\gll    A quaj stèvnṣ èssar … pulits-pulits l’ jamna … {\longrule} \textbf{tg}’ al bap \textbf{dètschi} in frang a miaz.\\
	and \textsc{dem.unm} must.\textsc{impf.1pl.1pl} \textsc{cop.inf} {} \textsc{red}\textasciitilde{well\_behaved}.\textsc{m.pl} \textsc{def.f.sg} week {} {}  \textsc{comp} \textsc{def.m.sg} father  give.\textsc{prs.sbjv.3sg} one.\textsc{m.sg} franc and half.\textsc{m.sg}\\
\glt `And we had to be … very well-behaved during the week … so that my father would give [us] one and a half francs.' (Ruèras, m1, \sectref{sec:8.2})
\z

\ea
\label{ex:subj15}
\gll  “[…] i ò dau las sjat.” “Lu cuschaj {\longrule} \textbf{tg}’ inṣ  \textbf{audi}.”\\
     {} \textsc{expl} have.\textsc{prs.3sg} give.\textsc{ptcp.unm} \textsc{def.f.pl} seven then be\_quiet.\textsc{imp.2pl} {}  \textsc{subord} \textsc{gnr} hear.\textsc{prs.sbjv.3sg}\\
\glt `“[…] It has struck seven o’clock.” “Then be quiet so we can hear.”' (\citealt[87]{Gadola1935})
\z

The subordinator \textit{avaun} `before' occurs as \textit{avaun tga} (\ref{ex:subj16}), \textit{avaun ca} (\ref{ex:subj17}), and \textit{avaun ca tga} (\ref{ex:subj18}). In (\ref{ex:subj18}) \isi{subjunctive mood} is used, in contrast to (\ref{ex:subj16}) and (\ref{ex:subj17}) where \isi{indicative mood} is used.

\ea
\label{ex:subj16}
\gll   \textbf{Avaun} \textbf{tgi} \textbf{végn} malaura isan las vacas ṣgarṣchajval.\\
before \textsc{comp.expl} come.\textsc{prs.ind.3sg} bad\_weather.\textsc{f.sg} run\_back\_and\_forth.\textsc{prs.3sg} \textsc{def.f.pl} cow.\textsc{pl} terrible.\textsc{adj.unm} \\
\glt `Before bad weather comes, the cows run back and forth like mad.' (\DRGoK{5}{777})
\z

\ea
\label{ex:subj17}
\gll  Agl Andréòli vèva finju … las ... figuras … \textbf{avaun} \textbf{ca} la caplùta \textbf{èra} \textbf{stada} finida.\\
\textsc{def.m.sg} \textsc{pn} have.\textsc{impf.3sg} finish.\textsc{ptcp.unm} {} \textsc{def.f.pl} {} figure.\textsc{pl} {} before \textsc{rel} \textsc{def.f.sg} chapel be.\textsc{impf.3sg} \textsc{pass.ptcp.f.sg} finish.\textsc{ptcp.f.sg} \\
\glt `Andreoli had finished ... the ... figures ... before the chapel was finished.' (Ruèras, m5, \sectref{sec:8.8})
\z

\ea
\label{ex:subj18}
\gll    Qu’ è stau … matʰaj … gl òn \textbf{avaun} \textbf{ca} \textbf{tgu} \textbf{mòndi} … ál’ ampréma classa.\\
\textsc{dem.unm} be.\textsc{prs.3sg} \textsc{cop.ptcp.unm} {} probably {} \textsc{def.m.sg} year before \textsc{rel} \textsc{rel.1sg} go.\textsc{prs.sbjv.1sg} {} to.\textsc{def.f.sg} first form\\
\glt `This was … probably… the year before I attended … the first form [of primary school].' (Sadrún, m6, \sectref{sec:8.5})
\z


A similar hesitation between indicative and subjunctive can be observed with \textit{tòca} or \textit{tòca tga} `until'. In (\ref{ex:toca:subj}) \textit{tòca} triggers subjunctive and in (\ref{ex:toca:ind}) \textit{tòca tga} triggers indicative.

\ea
\label{ex:toca:subj}
\gll    Api èra la sòra òra uschéja … avaun niaṣ ésch ad ò spatgau a spatgau \textbf{tòca} la \textbf{audi} anzatgéj [...].\\
and \textsc{cop.impf.3sg} \textsc{def.f.sg} nun out so {} in\_front\_of \textsc{poss.1pl.m.sg} door and have.\textsc{prs.3sg} wait.\textsc{ptcp.unm} and wait.\textsc{ptcp.unm} until \textsc{3sg.f} hear.\textsc{prs.sbjv.3sg} something\\
\glt `And then the nun was out [in the corridor] like this ... in front of our door, waiting and waiting until she would hear something [...].' (Sadrún, f6, \sectref{sec:8.4})
\z

\ea
\label{ex:toca:ind}
\gll  Anqual jèda vagnéva lu al pás[tar] … né usché cu aj vasévan a gidavan \textbf{tòca} \textbf{tg}’ inṣ \textbf{èr}’ ajn … ajn «ṣchwung» [...].\\
some time.\textsc{f.sg} \textsc{come.impf.3sg} then \textsc{def.m.sg} herdsman {} or so when \textsc{3pl} see.\textsc{impf.3pl} and help.\textsc{impf.3pl} until \textsc{subord} \textsc{gnr} \textsc{cop.impf.3sg} in {} in momentum.\textsc{m.sg}\\
\glt `Sometimes the herdsman would come ... or so, when they saw and they would help until one was again in momentum [...].' (Ruèras, m3, \sectref{sec:8.16})
\z

In the corpus, \textit{sènza tga} `without' only occurs with subjunctive (\ref{ex:senzatga1} and \ref{ex:senzatga2}).

\ea
\label{ex:senzatga1}
\gll  [...] èla savèv’ è í vidò gljunsch a paj \textbf{sènza} \textbf{tgu} \textbf{stòpi} tumaj tga la mòndi a funs. \\
{} \textsc{3sg.f} can.\textsc{impf.3sg} also go.\textsc{inf} over\_out far on foot.\textsc{m.sg} without \textsc{subord.1sg} must.\textsc{prs.sbjv.1sg} fear.\textsc{inf} \textsc{comp} \textsc{3sg.f} go.\textsc{prs.sbjv.3sg} to ground.\textsc{m.sg}\\
\glt `[...] she could go far on foot without me having to be afraid that she could drown.' (Ruèras, f7, \sectref{sec:8.14})
\z

\ea
\label{ex:senzatga2}
\gll Èla végn sjantar \textbf{sènza} tga nus \textbf{lajan}.\\
\textsc{3sg.f} come.\textsc{prs.3sg} after without \textsc{subord} \textsc{1pl} want.\textsc{prs.sbjv.1pl}\\
\glt `She follows us without us wanting [it].' (Bugnaj, \citealt[132]{Büchli1966})
\z

If a subordinate clause depends on a clause whose verb occurs in \isi{subjunctive mood}, the clause which normally does not take subjunctive takes it by syntactic attraction. An example is (\ref{ex:subj:synt.attr}), where the subjunctive occurs in the \isi{relative clause} which normally requires indicative.

\ea
\label{ex:subj:synt.attr}
	\gll    [...] quaj \textbf{fagèva} las gjufnas lu schòn \textbf{stém} \textbf{sch}’ i \textbf{vajan} sé la nègla tg’ èla \textbf{vaj} dau né bétg.\\
{} \textsc{dem.unm} do.\textsc{impf.3sg} \textsc{def.f.pl} young\_woman.\textsc{pl} then in\_fact attention.\textsc{m.sg} if \textsc{3pl}  have.\textsc{sbjv.prs.3pl} up \textsc{indef.f.sg} carnation \textsc{rel} \textsc{3sg.f} have.\textsc{sbjv.prs.3sg}  give.\textsc{ptcp.unm} or \textsc{neg} \\
\glt `[...] the young women would pay close attention to whether they had put on the hat the carnation she had given them or not.' (Zarcúns, m2, \sectref{sec:8.13})
\z

There are some cases where \isi{conditional} is used instead of \isi{subjunctive} (\ref{ex:condforsubj}).

\ea
\label{ex:condforsubj}
\gll Quèl lèva bétga \textbf{craj} tga `ls tiars \textbf{raṣdassan} da Nadal-nòtg durònt mèssa [...].\\
\textsc{dem.m.sg} want.\textsc{impf.3sg} \textsc{neg} believe.\textsc{inf} \textsc{comp} \textsc{def.m.pl} animal.\textsc{pl} speak.\textsc{cond.3pl} of Christmas-night.\textsc{f.sg} during mass.\textsc{f.sg}\\
\glt `He didn't want to believe that the animals speak during mass on Christmas Eve [...].' (Tschamùt, \citealt[132]{Büchli1966})
\z

There are two cases of the use of the subjunctive which I could not explain. I therefore asked the specialist of the use of mood in Sursilvan, Matthias Grünert, if he could explain these cases.

The first case is the subjunctive in an \isi{object clause} which depends on the verb \textit{savaj} `know' (\ref{savajsubj1}).

\ea
\label{savajsubj1}
\gll    A las sòras \textbf{savèvan} tga nus trajs nus \textbf{vágian} adina u-léjgar, a nus \textbf{mò̱ndian} bugèn cò gjù à scùla, a nus \textbf{fè̱tschian} filistùcas [...].\\
and \textsc{def.f.pl} nun.\textsc{pl} know.\textsc{impf.3pl} \textsc{comp} \textsc{1pl} three \textsc{1pl} have.\textsc{prs.sbjv.1pl} always \textsc{elat}-funny.\textsc{adj.unm} and \textsc{1pl} go.\textsc{prs.sbjv.1pl} with\_pleasure here down to school.\textsc{f.sg} and \textsc{1pl} do.\textsc{prs.sbjv.1pl} prank.\textsc{pl}\\
\glt `And the nuns knew that the three of us, we always had fun, and that we liked to come to school down here, and that we used to play pranks [...].' (Sadrún, f6, \sectref{sec:8.4})
\z

Matthias Grünert (p.c. 2020/05/25) explains that the subjunctive after \textit{saver}, especially in the \isi{imperfect}, is well documented in \ili{Standard Sursilvan}. Grünert's explanation for such cases is that the situation referred to in the \isi{object clause} is presented from the perspective of the \isi{subject} of the \isi{object clause} and not of the \isi{subject} of \textit{saver}.


The second example concerns the use of the subjunctive in a \isi{conditional} sentence which depends on the particle verb \textit{anflá ò/òra} `find out' (\ref{ex:anflao}). 

\ea
\label{ex:anflao}
\gll   [...] api vèvan nuṣ anflau òra scha nus \textbf{mò̱ndian} a \textbf{sé̱jṣian} spèr la sòr’ Andréa, lèza savèva ròmòntsch.\\
{} and have.\textsc{impf.1pl} \textsc{1pl}  find.\textsc{ptcp.unm} out if \textsc{1pl} go.\textsc{prs.sbjv.1pl} and sit.\textsc{prs.sbjv.1pl} next  \textsc{def.f.sg} nun \textsc{pn} \textsc{dem.f.sg} know.\textsc{impf.3sg} Romansh.\textsc{m.sg}\\
\glt `[...] and then we had found out that if we went to sit next to Sister Andrea, she knew Romansh.' (Sadrún, f6, \sectref{sec:8.4})
\z

In this case, Matthias Grünert states that the subjunctive in an \isi{object clause} depending on \textit{anflar ora} is also documented in \ili{Standard Sursilvan} (\ref{anflarorasubj1}).

\ea
\label{anflarorasubj1}
\gll Gia il pievel egipzian haveva \textbf{anflau} \textbf{ora} che mèl d’ aviuls \textbf{seigi} in remiedi.\\
already \textsc{def.m.sg} people Egyptian have.\textsc{impf.3sg} find.\textsc{ptcp.unm} out \textsc{comp} honey of bee.\textsc{m.pl} be.\textsc{prs.sbjv.3sg} \textsc{indef.m.sg} drug\\
\glt `The Egyptian people had already found out that honey is a drug.' (\ili{Standard Sursilvan}, La Quotidiana 2018/05/04)
\z


However, (\ref{ex:anflao}) is highly elliptic and there is no \isi{object clause} depending on \textit{ anflá òra} `find out'; the subjunctive occurs in a \isi{conditional clause} instead of the \isi{direct conditional}. Therefore, (\ref{ex:anflao}) could be the opposite of (\ref{ex:condforsubj}), where the \isi{direct conditional} is used instead of the subjunctive. In any case the use of the subjunctive in (\ref{ex:anflao}) is not accepted by other consultants.

\subsubsubsection{Imperfect subjunctive}\label{sec:4.1.2.2.9}
\isi{Imperfect subjunctive} is very rare in the corpus, where it only occurs with verbs of speaking (\ref{ex:subjimpf1} and \ref{ex:subjimpf2}) and \isi{verbs of opinion} (\ref{ex:subjimpf3}).

\ea
\label{ex:subjimpf1}
\gll [...] la détga \textbf{di} tg’ {\textbf è̱rian} schindanajn tg' i \textbf{udé̱vian} c’ i \textbf{tucavi} da mjadṣdé ajnt Ruèras.\\
{} \textsc{def.f.sg} legend say.\textsc{prs.3sg}  \textsc{comp} \textsc{cop.impf.sbjv.3pl} so\_in \textsc{comp} \textsc{3pl} hear.\textsc{impf.sbjv.3pl} \textsc{comp} \textsc{expl} beat.\textsc{impf.sbjv.3sg} of noon.\textsc{m.sg} in \textsc{pn}\\
\glt `[...] the legend says that they were so deep in the cave that they heard the clock strike noon in Rueras.' (Sadrún, m4, \sectref{sec:8.3})
\z

\ea
\label{ex:subjimpf2}
\gll  Èl \textbf{èri} avaun caplùta a \textbf{vagi} \textbf{vju} tga quèls méls èn saspuantaj [...].  \\
\textsc{3sg.m} \textsc{cop.impf.sbjv.3sg} in\_front chapel and have.\textsc{prs.sbjv.3sg} see.\textsc{ptcp.unm} \textsc{comp} \textsc{dem.m.pl} mule.\textsc{pl} be.\textsc{prs.3pl} \textsc{refl.}frighten.\textsc{ptcp.m.pl}\\
\glt `[The priest said that] He was in front of the chapel and had seen that these mules ran away [...].' (Sadrún, m10, \sectref{sec:8.7})
\z

\ea
\label{ex:subjimpf3}
\gll Ju \textbf{craj} tgu \textbf{vèvi} òtg vacas [...].\\
\textsc{1sg} believe.\textsc{prs.1sg} \textsc{comp.1sg} have.\textsc{sbjv.impf.1sg} eight cow.\textsc{f.pl}\\
\glt `I think I had eight cows [...].' (Ruèras, m3, \sectref{sec:8.16})
\z

\subsubsubsection{Direct and indirect conditional}\label{sec:4.1.2.2.10}

The \isi{direct conditional} mostly occurs in \isi{conditional sentences}, in the protasis as well as in the apodosis. The protasis (\ref{ex:conddir3} and \ref{ex:conddir4}) and the apodosis (\ref{ex:conddir2}) are sometimes not expressed overtly. The \isi{direct conditional} has a simple and a compound form. The simple form expresses \isi{present counter-factuality} (\ref{ex:conddir1} and \ref{ex:conddir2}).

\ea\label{ex:conddir1}
\gll A lu Pr̩datsch … plénansé cò ancúntar Tgòm … ṣaj ina rùsna, quaj \textbf{fùṣ} è aun intarassant \textbf{sch’} ins \textbf{savés}, quaj datèscha da gl òn ju … a ussa bigja grat prèsèn, méli a sistschian a zatgéj.\\
and then \textsc{pn} {} more\_uphill here in\_direction \textsc{pn} {} \textsc{cop.prs.3sg} \textsc{indef.f.sg} hole \textsc{dem.unm} \textsc{cop.cond.3sg} also indeed interesting.\textsc{m.unm} if \textsc{gnr} know.\textsc{cond.3sg} \textsc{dem.unm} date.\textsc{prs.3sg} from \textsc{def.m.sg} year \textsc{1sg} {} have.\textsc{prs.1sg} now \textsc{neg} just present thousand and six\_hundred and something\\

\glt `And then Pardatsch … a bit more uphill here in direction of Tgom … there is a cave, it would indeed be interesting if one knew, this is dated, I … don't have it exactly in mind, sixteen hundred something.' (Sadrún, m4, \sectref{sec:8.3})
\z

\ea
\label{ex:conddir2}
\gll  Té \textbf{savèssaṣ} í cul tat ajn Pardatsch.\\
\textsc{2sg} can.\textsc{cond.2sg} go.\textsc{inf} with.\textsc{def.m.sg} grandfather up \textsc{pn}\\
\glt `You could go up to Pardatsch with your grandfather.' (Sadrún, m4, \sectref{sec:8.3})
\z

The compound form expresses \isi{past counter-factuality} (\ref{ex:conddir3} and \ref{ex:conddir4}).

\ea
\label{ex:conddir3}
\gll   [...] nuṣ astgèvan bégja raṣdá ròmòntsch, inṣ \textbf{vèṣ} gè \textbf{savju} dá la bùca ṣur dlas sòras.\\
{} \textsc{1pl} be\_allowed.\textsc{impf.1pl} \textsc{neg} speak.\textsc{inf} Romansh.\textsc{m.sg} \textsc{gnr}  have.\textsc{cond.3sg} after\_all can.\textsc{ptcp.unm} give.\textsc{inf} \textsc{def.f.sg} mouth over of.\textsc{def.f.pl} nun.\textsc{pl}\\
\glt `[...] we were not allowed to speak Romansh, as a matter of fact one could have made derisive remarks about the nuns.' (Camischùlas, f6, \sectref{sec:8.4})
\z

\ea
\label{ex:conddir4}
\gll  [...] ju \textbf{fùṣ} ina sèra maj \textbf{id’} ò da tgèṣa la sèra da stgir. \\
{} \textsc{1sg} be.\textsc{cond.1sg} \textsc{indef.f.sg} evening never go.\textsc{ptcp.f.sg} out of home.\textsc{f.sg} \textsc{def.f.sg} evening of dark.\textsc{m.unm}\\
\glt `[...] I would never have left home in the evening when it was dark.' (Sadrún, f2, \sectref{sec:8.6})
\z

As examples (\ref{ex:conddir1}--\ref{ex:conddir4}) show, the final \textit{-s} of the singular persons and of the second person plural of the \isi{direct conditional} is realised as [ṣ] if it is followed by a vowel without a pause as is the case with all forms of the verbal paradigms that end in \textit{-s}. Note, however, that if the \isi{conditional} is followed by the \isi{expletive pronoun} or the pronoun of the third person plural which is not marked for \isi{gender}, both \textit{i}, the ending of the \isi{conditional} is pronounced [i] as in \textit{duèss-i} `should.\textsc{cond.3sg}-\textsc{expl}' \sectref{sec:8.8}.

Examples (\ref{ex:cond.indir1}) and (\ref{ex:cond.indir2}) illustrate the \isi{indirect conditional}, which occurs in object clauses that are governed by a \isi{speech act verb} like \textit{dumandá} `ask' or \textit{dí} `say'. (\ref{ex:cond.indir3}) illustrates the compound \isi{indirect conditional}.

\ea
\label{ex:cond.indir1}
\gll  [...]  a … lu vajn nuṣ, va ju dumandau sch’ èl \textbf{prandèssi} mè tòcan … a Ruèras. \\
{} and {} then have.\textsc{prs.1pl} \textsc{1pl} have.\textsc{prs.1sg}  \textsc{1sg} ask.\textsc{ptcp.m.unm} if \textsc{3sg.m} take.\textsc{cond.indir.3sg} \textsc{1sg} until {} to \textsc{pn}\\
\glt `[...] and … then we, I asked whether he could take me down to Rueras.' (Ruèras, m10, \sectref{sec:8.7})
\z

\ea
\label{ex:cond.indir2}
\gll  [...] api lu va ju … tlafònau dad èl a détg, éba, mi' ùm ségi èba mòrts scù i sápian, ábar … ju \textbf{fagèssi} ugèn vinavaun quèla lavur [...].  \\
{} and then have.\textsc{prs.1sg} \textsc{1sg} {} call.\textsc{ptcp.m.unm} \textsc{dat} \textsc{3sg.m} and say.\textsc{ptcp.m.unm} exactly \textsc{poss.1sg.m.sg} man be.\textsc{prs.sbjv.3sg} precisely die.\textsc{ptcp.m.sg} as \textsc{3pl} know.\textsc{prs.sbjv.3pl} but {} \textsc{1sg} do.\textsc{cond.indir.1sg} with\_pleasure still \textsc{dem.f.sg} job\\ 
\glt `[...] and then I … phoned him and said that my husband had died as they knew, but … that I would like to keep doing this job [...].' (Sadrún, f3, \sectref{sec:8.1})
\z


\ea
\label{ex:cond.indir3}
\gll Èl ò détg tg' èl \textbf{vèvi} \textbf{vju} èls.\\
 \textsc{3sg.m} have.\textsc{prs.3sg} say\textsc{.ptcp.unm} \textsc{comp} \textsc{3sg.m} have.\textsc{cond.indir.3sg} see.\textsc{ptcp.unm} \textsc{3pl.m}\\
 \glt `He said that he had seen them.' (Sadrún, m6)
 \z


\subsubsubsection{Tense agreement}\label{sec:4.1.2.2.11}
In contrast to other Romance varieties, Tuatschin has no \isi{tense agreement}. In object clauses, it is always the tense that would occur in direct speech which is used. This is probably connected to the fact that Tuatschin, as well as \ili{Standard Sursilvan}, uses subjunctive after verbs of speaking or \isi{verbs of opinion}, be they affirmative or negated, also in contrast to other Romance varieties. An example is (\ref{ex:notenseagr}).

\ea
\label{ex:notenseagr}
\gll  [...] api lu va ju … tlafònau dad èl a détg, éba, mi' ùm \textbf{ségi} èba \textbf{mòrts} [...]. \\
{} and then have.\textsc{prs.1sg} \textsc{1sg} {} call.\textsc{ptcp.unm} \textsc{dat} \textsc{3sg.m} and say.\textsc{ptcp.unm} exactly \textsc{poss.1sg.m.sg} man be.\textsc{prs.sbjv.3sg} precisely die.\textsc{ptcp.m.sg} \\ 
\glt `[...] and then I … phoned him and said that my husband had died [...].' (Sadrún, f3, \sectref{sec:8.1})
\z

In this example, \isi{perfect subjunctive} is used (\textit{ségi mòrts}) and not \isi{pluperfect subjunctive} (\textit{*èri mòrts}), which does not occur in the corpus. In direct speech, \isi{perfect indicative} would be used: «Mi' ùm \textit{è mòrts}.» `My husband has died.'

\subsubsubsection{The construction \textit{vaj} \textit{tga} `have that'}\label{sec:4.1.2.2.12}
It has not been possible to determine the exact function of \textit{\isi{vaj tga}} `have that' (\ref{ex:vajtga1}), (\ref{ex:vajtga2}), and (\ref{ex:vajtga3}), or \textit{végn tga} `come that' (\ref{ex:vegntga1}), but the examples suggest that the construction focuses on the current relevance of the event the verb refers to (\ref{ex:vajtga1} and \ref{ex:vegntga1}) or on \isi{habituality} in the past (\ref{ex:vajtga2} and \ref{ex:vajtga3}).

\ea
\label{ex:vajtga1}
\gll  Ju \textbf{a} \textbf{tga} fò bléd.\\
\textsc{1sg} have.\textsc{prs.1sg} \textsc{comp} do.\textsc{prs.3sg} sick\\
\glt `I am feeling sick.' (Ruèras, \DRGoK{2}{397})
\z

\ea
\label{ex:vegntga1}
\gll I briṣcha la cazèta, i \textbf{végn} \textbf{tga} sufla. \\
\textsc{expl} burn.\textsc{prs.3sg} \textsc{def.f.sg} pot \textsc{expl} come.\textsc{prs.3sg} \textsc{comp} blow.\textsc{prs.3sg} \\
\glt `[The soot] on the pot is burning, it is getting stormy.' (\DRGoK{2}{215})
\z

\ea
\label{ex:vajtga2}
\gll    A lu \textbf{vèvan} nus lò \textbf{tga} nus astgèvan raṣdá ramò̱ntsch.\\
and then have.\textsc{impf.1pl} \textsc{1pl} there \textsc{comp} \textsc{1pl} be\_allowed.\textsc{impf.1pl} speak.\textsc{inf} Romansh.\textsc{m.sg} \\
\glt `And then we had the opportunity to be allowed to speak Romansh there.' (Camischùlas, f6, \sectref{sec:8.4})
\z

\ea
\label{ex:vajtga3}
\gll  [...] api sjantar mav’ ins lu ... ségi quaj ajn quèla bar né ajn tschèla … nùca tg’ i \textbf{vèva} lu \textbf{tga} trèva ... dad ira.  \\
{} and after go.\textsc{impf.3sg} \textsc{gnr} then {} \textsc{cop.prs.sbjv.3sg} \textsc{dem.unm} in \textsc{dem.f.sg} bar or in \textsc{dem.f.sg} {} where \textsc{rel} \textsc{expl} have.\textsc{impf.3sg} then \textsc{comp} pull.\textsc{impf.3sg} {} \textsc{comp} go.\textsc{inf}\\
\glt `[...] and then we would go into this bar or into that one ... wherever it drew us to go.' (Sadrún, m9, \sectref{sec:8.15})
\z

\subsection{Particle verbs}\label{sec:4.1.3}
A p\isi{article verb} is a verb that combines with an adverb to form a semantic unit. An example is \textit{fá} gjù, literally `make down', and which means `make an appointment'. The origin of such structures is controversial: they are considered either a genuine Romansh structure, a loan from German or Swiss German, or both. \textit{Fá gjù}, however, is a clear case of calquing from Swiss German. In Swiss German, `make an appointment' is [ˈˈabˈmaxə]. In this lexeme, the prefix \textit{ab-} has been interpreted as [ˈabə] `down', hence \textit{gjù}, and [ˈmaxə] means 'do, make', which leads to the particle verb \textit{fá gjù}.

There is an important difference between the German and the Romansh construction: In German, Standard or Swiss, the particle is a verbal prefix which in simple tenses is located at the end of the sentence, as in (\ref{ex:pcl:chg1}).

\ea\label{ex:pcl:chg1}
\gll  [ix \textbf{max} ˈjedə ta:g mit əm \textbf{ab}]\\
     \textsc{1sg}  make every day with him  \textsc{ptcl} \\
\glt `I make an appointment with him every day.' (Swiss German, own knowledge)
\z

In such cases, the particle follows the verb in Tuatschin (and other Sursilvan varieties) (\ref{ex:pcl1}).

\ea\label{ex:pcl1}
\gll  Ju \textbf{fétsch} \textbf{gjù} cun èl mintga dé.\\
     \textsc{1sg}  make.\textsc{prs.1sg}  down with \textsc{3sg.m}  every day.\textsc{m.sg}\\
\glt `I make an appointment with him every day.' (Sadrún, m4)
\z

However, in Tuatschin and other Sursilvan varieties, the particle is not immediately adjacent to the verb, since some elements may intervene between the verb and the particle. These elements are inverted subjects -- pronouns (\ref{ex:pv:1}) or full \isi{noun phrases} (\ref{ex:pv:2}) --, the negator \textit{bétga} and its variants (\ref{ex:pv:3}), as well as other adverbs like \textit{aun} `still', \textit{è/èra} (\ref{ex:pv:4}) `also', \textit{lu} `then' (\ref{ex:pv:4}), \textit{magari} `sometimes' (\ref{ex:pv:5}), \textit{maj} `never' (\ref{ex:pv:6}), \textit{pròpi} `exactly' (\ref{ex:pv:7}), \textit{puspè} `again' (\ref{ex:pv:8}), or \textit{schòn} `certainly' (\ref{ex:pv:9}).

\ea\label{ex:pv:1}
\gll   Damaun prèn \textbf{ju} sé èl.\\
     tomorrow take.\textsc{prs.1sg} \textsc{1sg} up \textsc{3sg.m}\\
\glt `Tomorrow I will lift him up.' (Sadrún, m6)
\z

\ea\label{ex:pv:2}
\gll  Té mù trafica usché vinavaun, api sièta \textbf{al} \textbf{malitèr} gjù té in dé.\\
     \textsc{2sg} just be\_up\_to.\textsc{imp.2sg} so further and shoot.\textsc{prs.3sg} \textsc{def.m.sg} army down \textsc{2sg} \textsc{indef.m.sg} day\\
\glt `You just go on behaving this way and the army will shoot you down one day.' (\citealt[91]{Gadola1935})
\z

\ea\label{ex:pv:3}
\gll   Damaun prèn ju \textbf{bégja} sé èl.\\
     tomorrow take.\textsc{prs.1sg} \textsc{1sg} \textsc{neg} up \textsc{3sg}\\
\glt `Tomorrow I won’t lift him up.' (Sadrún, m6)
\z

\ea\label{ex:pv:4}
\gll    [...] a lu dèvani \textbf{lu} \textbf{è} sé da, da scrívar tòcs [...].\\
{} and then give.\textsc{3pl.3pl} then also up \textsc{comp} \textsc{comp} write.\textsc{inf} play.\textsc{m.pl}\\
\glt `[...] and then they also gave [us homework] to write plays [...].' (Zarcúns, m2, \sectref{sec:8.13})
\z

\ea
\label{ex:pv:5}
\gll  Ju prèn \textbf{magari} sé èl.  \\
\textsc{1sg} take.\textsc{prs.1sg} sometimes up \textsc{3sg.m}\\
\glt `Sometimes I lift him up.' (Sadrún, m6)
\z

\ea
\label{ex:pv:6}
\gll  Cun quèl fétsch ju \textbf{maj} gjù.  \\
with \textsc{dem.m.sg} make.\textsc{prs.1sg} \textsc{1sg} never down\\
\glt `With this person I never make an appointment.' (Sadrún, m6)
\z

\ea
\label{ex:pv:7}
\gll   Ah, tgé ... prandèvan \textbf{pròpi} òra sa ins bégj' éxáct [...]. \\
ah what {} take.\textsc{impf.3pl} exactly out know.\textsc{prs.3sg} \textsc{gnr} \textsc{neg} exactly\\
\glt `Ah, what … they really mined one does not know exactly [...].' (Sadrún, m4, \sectref{sec:8.3})
\z

\ea
\label{ex:pv:8}
\gll  Prèn \textbf{puspè} sé quaj! \\
take.\textsc{imp.2sg} again up \textsc{dem.unm}  \\
\glt `Lift this again!' (Sadrún, m9)
\z

\ea
\label{ex:pv:9}
\gll  Al plé mal stùn ju pal bian cazè. Al pòlisch crèscha \textbf{schòn} ansjaman.\\
     \textsc{def.m.sg}  most bad stay.\textsc{prs.1sg}  \textsc{1sg} for.\textsc{def.m.sg} good shoe \textsc{def.m.sg} thumb grow.\textsc{prs.3sg}  certainly together\\
\glt `I am very sorry for the shoe of good quality. My big toe will certainly knit together again.' (\citealt[51]{Berther1998})
\z

The adverbs presented in (\ref{ex:pv:1} to \ref{ex:pv:9}) must stand between the verb and its particle;  the adverbs \textit{savèns} `often' (\ref{ex:pv:10} and \ref{ex:pv:11}) and \textit{spèrt}  (\ref{ex:pv:12}--\ref{ex:pv:15}) as well as \textit{dabòt} (\ref{ex:pv:16}), both `rapidly', may occur between the verb and its particle or may follow the particle.

\ea
\label{ex:pv:10}
\gll    Ju prèn \textbf{savèns} sé èl.\\
     \textsc{1sg} take.\textsc{prs.1sg} often up \textsc{3sg.m}\\
\glt `I often lift him up.' (Sadrún, m6)
\z

\ea
\label{ex:pv:11}
\gll    Quèl prènd ju sé \textbf{savèns}.\\
     \textsc{dem.m.sg} take.\textsc{prs.1sg} \textsc{1sg} up often\\
\glt `This one I often lift up.' (Sadrún, f1)
\z

\ea
\label{ex:pv:12}
\gll Èls prèndan \textbf{spèrt} sé als ufauns.   \\
\textsc{3pl.m} take.\textsc{prs.3pl} rapidly up \textsc{def.m.pl} child.\textsc{pl}  \\
\glt `They lift the children rapidly.' (Sadrún, m9)
\z

\ea
\label{ex:pv:13}
\gll Èls prèndan sé \textbf{spèrt} als ufauns.   \\
   \textsc{3pl} take.\textsc{prs.3pl} up rapidly \textsc{def.m.pl} child.\textsc{pl}  \\
\glt `They lift the children rapidly.' (Sadrún, m9)
\z

\ea
\label{ex:pv:14}
\gll Ju prèn \textbf{aun} sé \textbf{spèrt} agl ufaun.    \\
    \textsc{1sg} take.\textsc{prs.1sg} still up rapidly \textsc{def.m.sg} child \\
\glt `Right now, I’ll lift the child rapidly.' (Ruèras, m3)
\z

\ea
\label{ex:pv:15}
\gll Ju prèn \textbf{spèrt} \textbf{aun} sé agl ufaun.    \\
    \textsc{1sg} take.\textsc{prs.1sg} rapidly still up \textsc{def.m.sg} child \\
\glt `Right now, I’ll lift the child rapidly.' (Ruèras, m3)
\z

\ea
\label{ex:pv:16}
\gll Ju prèn \textbf{aun} \textbf{dabòt} sé agl ufaun.    \\
    \textsc{1sg} take.\textsc{prs.1sg} rapidly still up \textsc{def.m.sg} child \\
\glt `Right now, I’ll lift the child rapidly.' (Ruèras, f4)
\z

The adverb \textit{mintgataun} `sometimes', which is a synonym of \textit{magari}, may not occur between the verb and the particle (\ref{ex:pv:17}).

\ea
\label{ex:pv:17}
\gll   *Ju prèn \textbf{mintgataun} sé èl. \\
     \textsc{1sg} take.\textsc{prs.1sg} sometimes up \textsc{3sg.m}\\
\glt `Sometimes I lift him up.' (Sadrún, m6)
\z

In contrast to German, direct objects, be they pronominal or nominal, may not stand between the verb and the particle: \textit{bétar navèn quaj} `throw away this' vs. \textit{*bétar quaj navèn} `throw this away', or \textit{prèndar sé èl/agl ufaun} `lift up him/the child' vs \textit{*prèndar èl/agl ufaun sé} `lift him/the child up'. A further example of the position of the \isi{personal pronoun} with respect to the particle can be found in (\ref{ex:schao}).

\ea
\label{ex:schao}
	\gll    A nuṣ duṣ vèvan dad í a rimná quèls pòrs, prèndar òr, \textbf{schá} \textbf{ò} \textbf{èls}, ò da nuégl.\\
	and \textsc{1pl} two.\textsc{m} have.\textsc{impf.1pl} to go.\textsc{inf}  \textsc{subord} collect.\textsc{inf} \textsc{dem.m.pl} pig.\textsc{pl} take.\textsc{inf} out  let.\textsc{inf} out \textsc{3pl.m} out of barn.\textsc{m.sg}\\
\glt `And the two of us had to go and collect these pigs, take out, let them out, out of the barn.' (Sadrún, m6, \sectref{sec:8.11})
\z


\subsection{Copulative verbs}\label{sec:4.1.4}
Copulative verbs are \textit{èssar} `be', \textit{paraj dad èssar} `seem', and the change of state \isi{verb} \textit{vagní} `become'.

The copula \textit{èssar} `be' is a general copula which allows non-verbal elements to fulfil \isi{predicative} functions, e.g. nouns (\ref{ex:cop:1}), adjectives (\ref{ex:cop:2}), prepositional phrases in \isi{locative} (\ref{ex:cop:3}), \isi{temporal} (\ref{ex:cop:4}), or \isi{comitative} (\ref{ex:cop:6}) function, or adverbs (\ref{ex:cop:5}).

\ea
\label{ex:cop:1}
	\gll    A qu’ \textbf{è} pròpi \textbf{ina} … pulit grònda \textbf{plata}~[...].\\
	and \textsc{dem.unm} \textsc{cop.prs.3sg} precisely \textsc{indef.f.sg} {} very big slab\\
\glt `And this really is a … very big slab [...].' (Sadrún, m6, \sectref{sec:8.5})
\z

\ea
\label{ex:cop:2}
\gll Quaj \textbf{èra} in’ jèda ... \textbf{brutal} tiar nus, bèn-bèn.\\
\textsc{dem.unm} \textsc{cop.impf.3sg} one.\textsc{f.sg} time {} terrible.\textsc{adj.unm} among \textsc{1pl} \textsc{red}\textasciitilde{really}\\
\glt `Once it was terrible among us, really.' (Sèlva, f2, \sectref{sec:8.6}
\z

\ea
\label{ex:cop:3}
\gll  Èl \textbf{èri} \textbf{avaun} \textbf{caplùta} [...].\\
\textsc{3sg.m} \textsc{cop.impf.sbjv.3sg} in\_front chapel.\textsc{f.sg}\\
\glt `He was in front of the chapel [...].' (Ruèras, m10, \sectref{sec:8.7})
\z

\ea
\label{ex:cop:4}
\gll Quaj hanlégj da tiars, ál sén scù quaj tg' i \textbf{èra} \textbf{ál} \textbf{ṣchènivával} \textbf{tschantanè} [...].\\
	\textsc{dem.m.sg} business of animal.\textsc{m.pl} in.\textsc{def.m.sg} sense like \textsc{dem.unm} \textsc{rel} \textsc{expl} \textsc{cop.impf.3sg} in.\textsc{def.m.sg} nineteenth century.\textsc{m.sg}\\
\glt`This cattle business, in the sense of how it was in the nineteenth century [...].' (Sadrún, m5, \sectref{sec:8.9})
\z

\ea
\label{ex:cop:6}
\gll    Ajn tùta cas mia, mia mùma a la mùma da mju còlè̱ga tg’ \textbf{èra} è \textbf{cun} \textbf{mè} … vèvan stju gidá nus [...].\\
in every case.\textsc{m.sg} \textsc{poss.1sg.f.sg}  \textsc{poss.1sg.f.sg} mother and \textsc{def.f.sg} mother of  \textsc{poss.1sg.m.sg} mate \textsc{rel}  \textsc{cop.impf.3sg} also with \textsc{1sg} {} have.\textsc{impf.3pl} must.\textsc{ptcp.unm} help.\textsc{inf} \textsc{1pl}\\
\glt `Anyhow my, my mother and the mother of my mate who was with me … had had to help us [...].' (Sadrún, m6, \sectref{sec:8.11})
\z

\ea
\label{ex:cop:5}
\gll    Gè scù bjè autar è tg’ è samidau. Quaj \textbf{è} \textbf{usché}.\\
yes as a\_lot other.\textsc{adj.unm} also \textsc{rel} be.\textsc{prs.3sg} \textsc{refl.}change.\textsc{ptcp.unm}  \textsc{dem.unm} \textsc{cop.prs.3sg} so\\
\glt `Yes, as many other things that also have changed. That’s how things are.' (Surajn, f5, \sectref{sec:8.10})
\z

The following examples illustrate the functions of \textit{paraj dad èssar} `seem' (\ref{ex:cop:7}) and \textit{vagní} `become' (\ref{ex:cop:8} and \ref{ex:cop:9}).

\ea
\label{ex:cop:7}
\gll Èla \textbf{para} \textbf{dad} \textbf{èssar} stauncla.   \\
\textsc{3sg.f} seem.\textsc{prs.3sg} \textsc{comp} \textsc{cop.inf} tired.\textsc{f.sg} \\
\glt `She seems to be tired.' (Sadrún, m5)
\z

\ea
\label{ex:cop:8}
\gll    Té stùs stá cò tùt parsula ad i \textbf{végn} \textbf{unviarn} a \textbf{végn} \textbf{frajd} […].\\
    \textsc{2sg} must.\textsc{prs.2sg} stay.\textsc{inf} here completely alone.\textsc{f.sg} and \textsc{expl} become.\textsc{prs.3sg} winter.\textsc{m.sg} and become.\textsc{prs.3sg} cold.\textsc{adj.unm}  \\
\glt `You must stay here completely alone, and winter is coming and it is getting cold.' (Bugnaj, \citealt[145]{Büchli1966})
\z

\ea
\label{ex:cop:9}
\gll  [...] èl’ èra \textbf{vagnida} tùt \textbf{còtschna} [...].\\
{} \textsc{3sg.f} be.\textsc{impf.3sg} become.\textsc{ptcp.f.sg} completely red.\textsc{f.sg}\\
\glt `[...] she had turned completely red [...].' (Sadrún, m6, \sectref{sec:8.11})
\z

\subsection{Existential verbs}\label{sec:4.1.5}
Existential constructions are formed with the \isi{expletive pronoun} \textit{i}, less frequently with \textit{quaj} `this', in the \isi{subject} position and the verbs \textit{èssar} `be' (\ref{ex:exist.essar1}--\ref{ex:exist.essar6}) or \textit{dá} `give' (\ref{ex:exist.da1}--\ref{ex:exist.da6}).

As is the case with \textit{quaj} (see \sectref{sec:3.2.2.1}, examples (\ref{ex:quajagrwithsubj1}--\ref{ex:quajagrwithsubj4}), the \isi{existential verb} agrees with the \isi{expletive subject pronoun}, but not with the \isi{predicative noun}, which means that if the \isi{predicative noun} is plural, the verb form is singular (\ref{ex:exist.essar1}).

\ea
\label{ex:exist.essar1}
\gll [...] ancunt’ agl atún \textbf{èri} plé \textbf{paucs} \textbf{tiarṣ} [...].\\
 {} towards \textsc{def.m.sg} autumn \textsc{exist.impf.3sg.expl} more little.\textsc{m.pl} animal.\textsc{pl} \\
\glt `[...] towards autumn there were fewer animals [...].' (Ruèras, m3, \sectref{sec:8.16})
\z

\ea
\label{ex:exist.essar2}
\gll [...] lu \textbf{ṣè} aun dus trajs intarassants lòganṣ ajn cò [...].\\
{} then \textsc{exist.prs.3sg.expl} still two three interesting place.\textsc{m.pl} in here\\
\glt `[...] there are furthermore two or three interesting places up there [...].' (Sadrún, m4, \sectref{sec:8.3})
\z

\ea
\label{ex:exist.essar3}
\gll   [...] a mintgataun \textbf{èri} aun grépa tga stèv’ ò in téc … . \\
{} and sometimes  \textsc{exist.impf.3sg.expl} moreover rock.\textsc{coll} \textsc{rel}  stand.\textsc{impf.3sg}  out \textsc{indef.m.sg} bit\\
\glt `[...] and from time to time there were rocks protruding a bit ... .' (Ruèras, m10, \sectref{sec:8.7})
\z

\ea
\label{ex:exist.essar4}
\gll Ér \textbf{ṣaj} \textbf{stau} bjè turists sé lò.\\
yesterday be.\textsc{prs.3sg.expl} \textsc{exist.ptcp.unm} many tourist.\textsc{pl} up there\\
\glt `Yesterday there were many tourists up there.' (Sadrún, m4)
\z

\ea
\label{ex:exist.essar5}
\gll    Api plénèngjù, èri gl unviarn, \textbf{qu’} \textbf{èra} baghétgs aun, ad \textbf{èra} pr̩vasèdars~[...].\\
and more\_down \textsc{cop.impf.3sg.expl} \textsc{def.m.sg} winter \textsc{dem.unm} \textsc{exist.impf.3sg} building.\textsc{m.pl} still and \textsc{exist.impf.3sg} herdsman.\textsc{m.pl} \\
\glt `And down there, it was winter, there were still buildings [there], and there were men who would feed the animals [...].' (Sèlva, f2, \sectref{sec:8.6})
\z

\ea
\label{ex:exist.essar6}
\gll    A \textbf{quaj} \textbf{èra} mù in ganc tras.\\
and \textsc{dem.unm} \textsc{exist.impf.3sg} only one.\textsc{m.sg} corridor through\\
\glt `And there was only one corridor.' (Ruèras, m1, \sectref{sec:8.2})
\z

\ea
\label{ex:exist.da1}
\gll  [...] \textbf{i} \textbf{dat} aun bjè da quèlas détgas.  \\
{} \textsc{expl} \textsc{exist.prs.3sg} still many of  \textsc{dem.f.pl} legend.\textsc{pl}\\
\glt `[...] there still are many such legends.' (Sadrún, m4, \sectref{sec:8.3})
\z

\ea
\label{ex:exist.da2}
\gll    [...] \textbf{i} \textbf{dat} ina fòtògrafia tgu sùn sé cun mju còl{\`e̱}ga al dé da la scargèda [...].\\
{} \textsc{expl} \textsc{exist.prs.3sg}  \textsc{indef.f.sg} photograph \textsc{rel.1sg}  \textsc{cop.prs.1sg} on with \textsc{poss.1sg.m.sg} mate \textsc{def.m.sg} day of  \textsc{def.f.sg}  drove.\textsc{ptcp.f.sg}\\
\glt `[...] there is a photograph in which I am with my mate the day of the pig droving [...].' (Sadrún, m6, \sectref{sec:8.11})
\z

\ea
\label{ex:exist.da3}
\gll    Las nòtízjas sa ju bétg danù̱ndar als gjaniturs, als duṣ baps prandèvan aj, \textbf{i} \textbf{dèva} ajnta Ruèras, \textbf{dèv}’ \textbf{aj} in ca vèva rá̱djò. \\
\textsc{def.f.pl} news.\textsc{pl} know.\textsc{prs.1sg} \textsc{1sg} \textsc{neg} from\_where \textsc{def.m.pl} parents.\textsc{pl} \textsc{def.m.pl} two.\textsc{m} father.\textsc{pl} take.\textsc{impf.3pl} \textsc{3sg} \textsc{expl} \textsc{exist.impf.3sg} in \textsc{pn} \textsc{exist.impf.3sg} \textsc{expl}  one.\textsc{m.sg} \textsc{rel} have.\textsc{impf.3sg} radio.\textsc{m.sg}\\
\glt `I don’t know where my parents had the news from, the two fathers took them, there was in Rueras, there was [only] one who had a radio.' (Ruèras, m1, \sectref{sec:8.2})
\z

\ea
\label{ex:exist.da4}
\gll  Quaj sch’ \textbf{i} \textbf{dèva} rèsts scha vagnévi lu magari rimnau quaj dus trajṣ diṣ api méz tùt ajn ina … tùt ansjaman.  \\
well if \textsc{expl} \textsc{exist.impf.3sg} leftovers.\textsc{m.pl} \textsc{corr} \textsc{pass.aux.impf.3sg.expl} then sometimes collect.\textsc{ptcp.unm} \textsc{dem.unm} two.\textsc{m.pl} three day.\textsc{pl} and put.\textsc{ptcp.m.pl} all in  \textsc{indef.f.sg} {} all together\\
\glt `Well, when there were leftovers, they would sometimes be collected for two or three days and then put all together in a … all together.' (Sadrún, m4, \sectref{sec:8.3})
\z

\ea
\label{ex:exist.da5}
\gll    A lu al pròxim, in dls pròximṣ unvjarns ... \textbf{òi} \textbf{dau} ina grònda navada [...].\\
and then \textsc{def.m.sg} next one of.\textsc{def.m.pl} next.\textsc{pl} winter.\textsc{pl} {} have.\textsc{prs.3sg}.\textsc{expl} \textsc{exist.ptcp.unm} \textsc{indef.f.sg} big snowfall \\
\glt `And then the next, one of the next winters … there was a big snowfall [...].' (Sadrún, m6, \sectref{sec:8.5})
\z

\ea
\label{ex:exist.da6}
\gll    Quaj èra ju gjù ina grònda lavina … a vèva … déstruí ina grònda part dl vitg ajntadém Ruèras … a \textbf{vèv’} è \textbf{dau} mòrts [...].\\
\textsc{dem.unm} be.\textsc{impf.3sg} go.\textsc{ptcp.unm} down \textsc{indef.f.sg} big avalanche {} and have.\textsc{impf.3sg} {} destroy.\textsc{ptcp.unm} \textsc{indef.f.sg} huge part of.\textsc{def.m.sg} village uppermost \textsc{pn} {} and have.\textsc{impf.3sg} also \textsc{exist.ptcp.unm} dead.\textsc{m.pl}\\
\glt `Then a huge avalanche went down … and … destroyed a big part of the village in the upper part of Rueras … and people died [...].' (Sadrún, m6, \sectref{sec:8.5})
\z

Examples (\ref{ex:exist:daessar1}) and (\ref{ex:exist:daessar2}) show the occurrence of \textit{dá} `give' and \textit{èssar} `be' in the same context. Furthermore, (\ref{ex:exist:daessar1}) also contains two examples with the \isi{expletive pronoun} \textit{i} and two examples without it.

\ea
\label{ex:exist:daessar1}
\gll Álṣò \textbf{i} \textbf{dèva} òns nùca tga gudignavan ... nùndétg, ad \textbf{i} \textbf{ṣèra} òns nùca tg’ \textbf{èra} aua, ad \textbf{èra} òns nùca tga spardévan.\\
well \textsc{expl} \textsc{exist.impf.3sg} year.\textsc{m.pl} where \textsc{rel} earn.\textsc{impf.3pl} {} incredibly and \textsc{expl} \textsc{exist.impf.3sg} year.\textsc{m.pl} where \textsc{rel} \textsc{exist.impf.3sg} water and \textsc{exist.impf.3sg} year.\textsc{m.pl} where \textsc{rel} lose.\textsc{impf.3pl} \\
\glt `Well, there were years when they earned ... a lot of money, and there were years with rain, and years when they would lose money.' (Sadrún, m5, \sectref{sec:8.9})
\z

\ea
\label{ex:exist:daessar2}
\gll [...] a lu \textbf{èri} è quèls prígals tga \textbf{dèva} [...] sén ira, naturálmajn cu i vagnévan anavùs [...].\\
{} and then \textsc{exist.impf.3sg.expl} also \textsc{dem.m.pl} danger \textsc{rel} \textsc{exist.impf.3sg} {} on go.\textsc{inf} natural.\textsc{f.sg.adv} when \textsc{3pl} come.\textsc{impf.3pl} back\\
\glt `[...] and then there were these dangers which [one encountered] when travelling, of course when they would come back [...].' (Sadrún, m5, \sectref{sec:8.9})
\z


\subsection{Modal verbs}\label{sec:4.1.6}
The following modal verbs occur in the corpus: \textit{astgè} `be allowed', \textit{duaj} `must, should', \textit{èssar da} `must, have to' \textit{munglá} `must', \textit{pudaj} `can, be able', \textit{savaj} `can', \textit{schè/schá} `let, allow', \textit{stuaj} `must, have to', \textit{vaj da} `have to', and \textit{vulaj} `want'.

Obligation is expressed by \textit{èssar da} `must, have to' (\ref{ex:essarda1}), \textit{duaj} `must, should' (\ref{ex:duaj1}), \textit{munglá} `must' (\ref{ex:mung1})\footnote{Nowadays \textit{munglá} is only used with \isi{conditional} mood.}, \textit{vaj da} `have to' (\ref{ex:vajda1}--\ref{ex:vajda3}), and \textit{stuaj} `must, have to' (\ref{ex:stuaj1} and \ref{ex:stuaj2}). Note that \textit{èssar da} is \isi{impersonal} and the complementiser \textit{da} does not have to be adjacent to \textit{vaj} (\ref{ex:vajda2} and \ref{ex:vajda3}). 

\ea
\label{ex:vajda2}
\gll  Vus vèssas \textbf{lu} \textbf{aun} da fá quèls bogns né mirá da la plaja.\\
\textsc{2sg.pol}  have.\textsc{cond.2pl} then still \textsc{comp} do.\textsc{inf} \textsc{dem.m.pl} bath.\textsc{pl} or look\_after.\textsc{inf} of \textsc{def.f.sg} wound\\
\glt `You should still take a bath or look after the wound.' (Sadrún, \sectref{sec:8.3})
\z

\ea
\label{ex:vajda3}
\gll A sch’ i èra malaura, scha \textbf{vèv}’ inṣ \textbf{a} \textbf{tgèsa} \textbf{da} \textbf{fá} [...].\\
and if \textsc{expl} \textsc{cop.impf.3sg} bad\_weather.\textsc{f.sg} \textsc{corr} have.\textsc{impf.3sg} \textsc{gnr} at house.\textsc{f.sg} to do.\textsc{inf} \\
\glt `And if the weather was bad, we [the women] had to work in the house [...].' (Ruèras, f4, \sectref{sec:8.16})
\z

\ea
\label{ex:essarda1}
	\gll [...] api \textbf{èri} \textbf{da} \textbf{fá} \textbf{fajn}.\\
{} and be.\textsc{impf.3sg.expl} to do.\textsc{inf} hay.\textsc{m.sg}\\
\glt `[...] and then one had to do hay.' (Ruèras, f4, \sectref{sec:8.16})
\z

\ea
\label{ex:duaj1}
\gll Gè grat uschéja, ábar ju, ina da la natira ad a adina … vulju fá mi' òbligazjun a finju, a tschèls \textbf{dajan} è \textbf{fá}.\\
yes exactly so but \textsc{1sg} one.\textsc{f.sg} of \textsc{def.f.sg} nature and have.\textsc{prs.3sg} always {} want.\textsc{ptcp.unm} do.\textsc{inf} \textsc{poss.1sg.f.sg} obligation and finish.\textsc{ptcp.unm} and \textsc{dem.m.pl} must.\textsc{prs.3pl} also do.\textsc{inf}\\
\glt `Yes, exactly like that. But I, a person who likes nature, and I have always … wanted to meet my obligations, and the other people should also do [the same].' (Sadrún, f3, \sectref{sec:8.1})
\z

\ea\label{ex:mung1}
\gll  Ju \textbf{munglaṣ} \textbf{ir}' a tgèsa.\\
\textsc{1sg}  must.\textsc{cond.1sg} go.\textsc{inf} to house.\textsc{f.sg} \\
\glt `I should go home.' (Cavòrgja, f1)
\z

\ea\label{ex:vajda1}
\gll    A tiar in pur … èssan nuṣ í, èr’ ju, \textbf{vèv}’ ju \textbf{gju} \textbf{dad} \textbf{incassá} quèls raps [...].\\ 
and by \textsc{indef.m.sg} farmer {} be.\textsc{prs.1sg} \textsc{1pl} go.\textsc{ptcp.m.pl}  be.\textsc{impf.1sg} \textsc{1sg}  have.\textsc{impf.1sg} \textsc{1sg} have.\textsc{ptcp.unm} to collect.\textsc{inf}  \textsc{dem.m.pl} cent.\textsc{pl}\\
\glt `And we went … to a farmer, I was, I had to collect this money [...]' (Sadrún, m6, \sectref{sec:8.11})
\z
\ea
\label{ex:stuaj1}
\gll  [...] l’ antschata da la parmavèra \textbf{stuèv}’ ins \textbf{schè} \textbf{ajn} als tiarṣ ajn nuégl [...]. \\
{} \textsc{def.f.sg} beginning of \textsc{def.f.sg} spring must.\textsc{impf.3sg} \textsc{gnr} let.\textsc{inf} in \textsc{def.m.pl} animal.\textsc{pl} in barn.\textsc{m.sg}\\
\glt `[...] at the beginning of spring one had to let the animals into the barn [...]' (Sadrún, m4, \sectref{sec:8.3})
\z

\ea
\label{ex:stuaj2}
\gll Api … ṣè quaj vagnú príu ajn, a lu ò la cònfadarazjun dau vi quaj da mintga cantún, a lèzs òn sèzs \textbf{stavju} lura … \textbf{métar} sén pajs quaj [...].\\
and {} be.\textsc{prs.3sg} \textsc{dem.unm} \textsc{pass.aux.ptcp.unm} take.\textsc{ptcp.unm} in and then have.\textsc{prs.3sg} \textsc{def.f.sg} confederation give.\textsc{ptcp.unm} over \textsc{dem.unm} \textsc{dat} every canton.\textsc{m.sg} and \textsc{dem.m.pl} have.\textsc{prs.3pl} self.\textsc{m.pl} must.\textsc{ptcp.unm} then {} put.\textsc{inf} on foot.\textsc{m.pl} \textsc{dem.unm}\\

\glt `Then … this has been adopted, and then the confederation handed it over to every canton, and these ... had  to get it off the ground themselves [...].' (Sadrún, f3, \sectref{sec:8.1})
\z

\textit{Astgè} `be allowed, can' (\ref{ex:astg1}) and \textit{schè/schá}\footnote{\textit{Schá} is the \ili{Standard Sursilvan} form.} (\ref{ex:astg2}) `let, allow' expresses \isi{permission}.

\ea
\label{ex:astg1}
\gll[...] api vajn nuṣ gju quèla gròndjuṣ’ idéa scha nuṣ \textbf{ástgian} \textbf{cuṣchiná}.\\
{} and have.\textsc{prs.1pl} \textsc{1sg} have.\textsc{ptcp.unm} \textsc{dem.f.sg} great.\textsc{f.sg} idea if \textsc{1pl} be\_allowed.\textsc{prs.sbjv.1pl} cook.\textsc{inf}\\
\glt `[...] and then we had that great idea [to ask] whether we were allowed to cook.' (Sadrún, f6, \sectref{sec:8.4})
\z

\ea
\label{ex:astg2}
\gll  In autar òn \textbf{schajṣ} vuṣ èra \textbf{ir}' ad alp mè.\\
\textsc{indef.m.sg} other year let.\textsc{prs.2pl} \textsc{2pl} also go.\textsc{inf} to alp \textsc{1sg}\\
\glt `Another year you will also let me go to alp.' (\citealt[85]{Gadola1935})
\z

\textit{Astgè} is used in \isi{polite requests}, as in  (\ref{ex:astg3}).

\ea
\label{ex:astg3}
	\gll  \textbf{Astg}’ \textbf{ju} aun \isi{dá} in glaṣ aua [...] ?\\
be\_allowed.\textsc{prs.1sg} \textsc{1sg} in\_addition give.\textsc{inf} \textsc{indef.m.sg} cup water\\
\glt `May I give [you] another glass of water [...] ?' (Ruèras, f4, \sectref{sec:8.16})
\z

Ability is expressed by \textit{pudaj} `can, be able' and \textit{savaj} `can'. \textit{Pudaj} refers to non-learned participant-internal \isi{ability} (\ref{ex:pudaj1}--\ref{ex:pudaj3}) or to \isi{permission} (\ref{ex:pudaj4}).

\ea
\label{ex:pudaj1}
\gll    A las sòras savèvan tga [...] nus fé̱tschian filistùcas, ad èlas \textbf{pudévan} maj tiar nus.\\
and \textsc{def.f.pl} nun.\textsc{pl} know.\textsc{impf.3pl} \textsc{comp} {} \textsc{1pl} do.\textsc{prs.sbjv.1pl} prank.\textsc.{pl} and \textsc{3pl.f} can.\textsc{impf.3pl} never to \textsc{1pl}\\
\glt `And the nuns knew that [...] we used to play pranks, and that they would never be able to prove anything against us.' (Camischùlas, f6, \sectref{sec:8.4})
\z

\ea
\label{ex:pudaj2}
\gll    Ad ad in piartg … ah … vèva \textbf{pudju} \textbf{scapá} sé Valtgèva tras la sajf, qu’ er’ ina, la saiv èri, vèva rùt in palé, quaj vèva \textbf{pudju} atrás [...].\\
and and \textsc{indef.m.sg} pig {} eh {}  have.\textsc{impf.3sg} can.\textsc{ptcp.unm} escape.\textsc{inf} up \textsc{pn} through \textsc{def.f.sg} fence  \textsc{dem.unm} \textsc{cop.impf.3sg}  \textsc{indef.f.sg} \textsc{def.f.sg} fence \textsc{cop.impf.3sg.expl} have.\textsc{impf.3sg} break.\textsc{ptcp.unm}  \textsc{indef.m.sg} post \textsc{dem.unm}  have.\textsc{impf.3sg} can.\textsc{ptcp.unm} through \\
\glt `And and a pig … eh … had been able to escape in Valtgeva through the fence, that was a, the fence was, had a broken post, so it had been able to go through [...].' (Sadrún, m6, \sectref{sec:8.11})
\z

\ea
\label{ex:pudaj3}
\gll  Ju cala dad í a scùlèta, ju \textbf{pùs} bitg' \textbf{í} plé.\\
\textsc{1sg} stop.\textsc{prs.1sg} \textsc{comp} go.\textsc{inf} to nursery\_school.\textsc{f.sg} \textsc{1sg} can.\textsc{prs.1sg} \textsc{neg} go.\textsc{inf} any\_more  \\
\glt `I’ll stop going to nursery school, I can’t stand it any longer.' (Sadrún, m4, \sectref{sec:8.3})
\z

\ea
\label{ex:pudaj4}
\gll  Api, quaj èra … quajnassé in pòst da survigilònza, bétga schanza da \textbf{pudaj} atrás, quaj dé òni bigja schau í nuṣ atrás.\\
and \textsc{dem.unm} \textsc{cop.impf.3sg} {} \textsc{dem.unm}.in\_up \textsc{indef.m.sg} guard of vigilance.\textsc{f.sg} \textsc{neg} chance.\textsc{f.sg} \textsc{attr} can.\textsc{inf} through \textsc{dem.m.sg} day have.\textsc{prs.3pl.3pl} \textsc{neg} let.\textsc{ptcp.unm} go.\textsc{inf} \textsc{1pl} through\\
\glt `And there was … up there a vigilance guard, no way to go through, that day they didn’t let us go through [that sentry].' (Sadrún, f3, \sectref{sec:8.1})
\z

\textit{Savaj} refers to participant-external (\ref{ex:savaj1}--\ref{ex:savaj4}) or learned participant-internal \isi{ability} (\ref{ex:savaj5} and \ref{ex:savaj6}). 

\ea
\label{ex:savaj1}
\gll  Pi ò èla dét[g]: «Té \textbf{savèssaṣ} \textbf{í} cul tat ajn Pardatsch.»  \\
and have.\textsc{prs.3sg} \textsc{3sg} say.\textsc{ptcp.unm} \textsc{2sg} can.\textsc{cond.2sg} go.\textsc{inf} with.\textsc{def.m.sg} grandfather up \textsc{pn}  \\
\glt `Then she said: «You could go up to Pardatsch with your grandfather.' (Sadrún, m4, \sectref{sec:8.3})
\z

\ea
\label{ex:savaj2}
\gll    Api sjantar vajn nus tartgau nus \textbf{sápian} \textbf{durmí} òra [...].\\
and after have.\textsc{prs.1pl} \textsc{1pl} think.\textsc{ptcp.unm} \textsc{1pl}  can.\textsc{prs.sbjv.1pl} sleep.\textsc{inf} out\\
\glt `And then we thought we would have a good sleep [...].' (Camischùlas, f6, \sectref{sec:8.4})
\z

\ea
\label{ex:savaj3}
\gll  [...] álṣò òr dal grép òni fatg ina pintga …  sènda tg’ ins \textbf{sò} \textbf{ira} ah a paj flòt.\\
{} this\_is\_to\_say out of.\textsc{def.m.sg} rock have.\textsc{prs.3pl.3pl} make.\textsc{ptcp.unm} \textsc{indef.f.sg} small {} path \textsc{rel} \textsc{gnr} can.\textsc{prs.3sg} go.\textsc{inf} eh on foot.\textsc{m.sg} easy.\textsc{adj.unm} \\
\glt `[...] this is to say out of the rock they made a small … path through which one could easily go eh on foot.' (Ruèras, m10, \sectref{sec:8.7})
\z

\ea
\label{ex:savaj4}
\gll  Api sjantar sùnd ju sasjuṣ gjù, api vau tartgau gè ábar ah, api \textbf{sau} bigj’ ajfach \textbf{í} ál’ aua.\\
and after be.\textsc{prs.1sg} \textsc{1sg} sit.\textsc{ptcp.m.sg} down and have.\textsc{prs.1sg.1sg} think.\textsc{ptcp.unm} yes but eh and can.\textsc{prs.1sg.1sg} \textsc{neg} simply go.\textsc{inf} into.\textsc{def.f.sg} water\\
\glt `And then I sat down and thought yes, but, eh, after all I should, I cannot simply jump into the water.' (Sadrún, m8, \sectref{sec:8.12})
\z                                       

\ea
\label{ex:savaj5}
\gll  Quaj crajs bé, l’ antschata cu ju a surpríu quaj, èri, èri da quèls tgé … mataj\footnotemark{} tg’ ina fèmna \textbf{sapi} \textbf{fá} da quaj.\\
\textsc{dem.unm} believe.\textsc{prs.2sg.gnr} \textsc{neg} \textsc{def.f.sg} beginning when \textsc{1sg} have.\textsc{prs.1sg} take\_on.\textsc{ptcp.unm} \textsc{dem.unm} \textsc{exist.impf.3sg.expl} \textsc{exist.impf.3sg.expl} of \textsc{dem.m.pl} \textsc{rel} {} probably \textsc{comp} \textsc{indef.f.sg} woman can.\textsc{prs.sbjv.3sg} do.\textsc{inf} of \textsc{dem.unm}\\
\glt `This you don’t believe, at the beginning when I took on this job, there were, there were some men who … [would say] that a woman is not able to do that.'\footnotetext{\textit{Mataj} means 'probably'; in this context, it is used ironically in the sense of 'impossibly'.} (Sadrún, f3, \sectref{sec:8.1})
\z

\ea
\label{ex:savaj6}
\gll Lu dumandavan nuṣ èl, vevan dumandau núa èl ségi stauṣ ajn plaza, èra `l staus zatgé vid Andermatt– a tudèstg \textbf{savèv’} ju è bigja– vèvan nuṣ dumandau in' jèda sch’ èl \textbf{sapi}, \textbf{savèva} `l lu schòn in téc tudèstg, \textbf{savèva} `l lu aun, quaj tg’ èra lu bigj' al cas tiar quèls végls aun.   \\
then ask.\textsc{impf.1pl} \textsc{1pl} \textsc{3sg.m} have.\textsc{impf.3sg}  ask.\textsc{ptcp.unm} where \textsc{3sg.m} be.\textsc{prs.sbjv.3sg} \textsc{cop.ptcp.m.sg} in job.\textsc{f.sg} be.\textsc{impf.3sg} \textsc{3sg.m}  \textsc{cop.ptcp.m.sg} something over \textsc{pn} and German know.\textsc{impf.1sg} \textsc{1sg} also \textsc{neg} have.\textsc{impf.1pl} \textsc{1pl} ask.\textsc{ptcp.unm} one.\textsc{f} time whether \textsc{3sg.m} can.\textsc{prs.sbjv.3sg} know.\textsc{impf.3sg} \textsc{3sg.m} then indeed \textsc{indef.m.sg} bit German know.\textsc{impf.3sg} \textsc{3sg.m} then really \textsc{dem.unm} \textsc{rel} \textsc{cop.impf.3sg} then \textsc{neg} \textsc{def.m.sg} case at \textsc{dem.m.pl} old.\textsc{pl} really \\
\glt `Then we would ask him, we had asked [him] where he had been working, he had been working for a certain time in Andermatt – and [that he knew] German I didn't know either – we had asked him whether he knew, he knew some German indeed, he really knew, which then was not the case with these old people.' (Sadrún, m4, \sectref{sec:8.3})
\z

If \textit{savaj} modifies a verb with complements, this verb is sometimes omitted, probably under the influence of Swiss German. In (\ref{ex:savaj7}), it is the verb \textit{í} `go' which is omitted.

\ea
\label{ex:savaj7}
	\gll [...] da nòs tjams salagravan nuṣ da vagní ò da scùla par è \textbf{savaj} {\longrule} {a} \textbf{la} \textbf{gjuvantétgna}.\\
 {} of \textsc{ poss.1pl.m.sg} time \textsc{refl}.appreciate.\textsc{impf.1pl} \textsc{1pl} \textsc{comp} come.\textsc{inf} out of school.\textsc{f.sg} \textsc{subord} also can.\textsc{inf} {} to \textsc{def.f.sg} youth\\
\glt `[...] when we were young we were happy to come out of school in order to also be able [to go] to the association of young people.' (Sadrún, m9, \sectref{sec:8.15})
\z

The opposition between \textit{pudaj} and \textit{savaj} is not always clear-cut. In (\ref{ex:pudaj6}), the modal verb refers to participant-external \isi{possibility} and one would expect \textit{savaj} instead of \textit{pudaj}.

\ea
\label{ex:pudaj6}
\gll   Cò \textbf{pùn} ins \textbf{cargè} tschuncònta vacas.\\
here can.\textsc{prs.3sg} \textsc{gnr} charge.\textsc{inf} fifty cow.\textsc{f.pl}\\
\glt `Here one can put to graze fifty cows.' (\DRGoK{3}{376})
\z
                               
Volition is expressed by \textit{vulaj} `want' (\ref{ex:vul1} and \ref{ex:vul2}). 

\ea
\label{ex:vul1}
\gll    «Gjòn, \textbf{vul}\footnotemark{} té bétga \textbf{gidá} mè da cargè quèla bùra?» «Bèn bèn, scù ju \textbf{pùs}, vi ju schòn gidá.»\\
\textsc{pn} want.\textsc{prs.2sg} \textsc{2sg} \textsc{neg} help.\textsc{inf} \textsc{1sg} \textsc{comp} carry.\textsc{inf} \textsc{dem.f.sg} block yes yes as \textsc{1sg} can.\textsc{prs.1sg} want.\textsc{prs.1sg} \textsc{1sg} certainly help.\textsc{inf}\\ 
\glt `«Gion, don’t you want to help me charge this block?» «Yes, sure, I will certainly help [you] as well as I can.»'\footnotetext{\textit{Vul} is \ili{Standard Sursilvan} for \textit{vutas}.} (Sadrún, \citealt[106]{Büchli1966})
\z

\ea
\label{ex:vul2}
\gll  [...] api vòu anflau in bi ljuc, api lu … \textbf{lèv’} ju \textbf{fá} \textbf{bògn} lò [...].\\
{} and have.\textsc{prs.1sg.1sg} find.\textsc{ptcp.unm} \textsc{indef.m.sg} beautiful.\textsc{m.sg} place and then {} want.\textsc{impf.1sg} \textsc{1sg} do.\textsc{inf} bath there\\
\glt `[...] and then I found a nice place, and then I wanted to take a bath there [...].' (Sadrún, m8, \sectref{sec:8.12})
\z

\textit{Vut dí} or \textit{vuta dí}, both `mean'  (literally `wants say'), is best considered a lexicalised expression (\ref{ex:vutdi1}).

\ea
\label{ex:vutdi1}
\gll Sas tgé quaj \textbf{vut} \textbf{dí}?   \\
know.\textsc{prs.2sg} what \textsc{dem.unm} want.\textsc{prs.3sg} say.\textsc{inf}\\
\glt `Do you know what this means?' (Ruèras, f4, \sectref{sec:8.16})
\z

\isi{Epistemic modality} is expressed by \textit{duaj} `should'  (\ref{ex:duaj2}), \textit{pudaj} `can'  (\ref{ex:pudaj5}), and \textit{pudaj èssar} `could be' (\ref{ex:pudsav1}) as well as \textit{savaj èssar} `could be' (\ref{ex:pudsav2}).

\ea
\label{ex:duaj2}
\gll   [...] sén quaj pas \textbf{duèssi} \textbf{èssar} ina samagljònta caplùta [...]. \\
{} on \textsc{dem.m.sg} pass should.\textsc{cond.3sg.expl} \textsc{cop.inf} \textsc{indef.f.sg} similar chapel\\
\glt `[...] on this pass there should be a similar chapel [...].' (Sadrún, m5, \sectref{sec:8.8})
\z

\ea
\label{ex:pudaj5}
\gll   [...] quèlṣ vèvan in purèsser plétò̱st … pin, tgé \textbf{pudévan} èlṣ vaj, déjsch quindiṣch armaulṣ gronṣ api lu aun tgauras [...]. \\
{} \textsc{dem.m.pl} have.\textsc{impf.3pl} \textsc{indef.m.sg} farm rather {} small what can.\textsc{impf.3pl} \textsc{3pl.m} have.\textsc{inf} ten fifteen animal.\textsc{m.pl} big.\textsc{pl} and then besides goat.\textsc{f.pl}\\
\glt `[...] they had a rather ... small farm, what could they have, maybe ten, fifteen big animals and then also goats [...].' (Sadrún, m4, \sectref{sec:8.3})
\z

\ea
\label{ex:pudsav1}
\gll Préndar ajn, \textbf{pù} schòn \textbf{èssar} tga samidav’ al grép [...].\\
take.\textsc{inf} in can.\textsc{prs.3sg} well be.\textsc{inf}  \textsc{comp} \textsc{refl}.change.\textsc{impf.3sg} \textsc{def.m.sg} rock\\
\glt `As for mining, it could well be that the rock changed [...].' (Sadrún, m4, \sectref{sec:8.3})
\z

\ea
\label{ex:pudsav2}
\gll Préndar ajn, \textbf{sò} schòn \textbf{èssar} tga samidav’ al grép [...].\\
take.\textsc{inf} in can.\textsc{prs.3sg} well be.\textsc{inf}  \textsc{comp} \textsc{refl}.change.\textsc{impf.3sg} \textsc{def.m.sg} rock\\
\glt `As for mining, it could well be that the rock changed [...].' (Sadrún, m5)
\z

Epistemic modality is also expressed by adverbs like \textit{fòrsa} (\sectref{sec:8.5}) / \textit{fòrza} (\sectref{sec:8.3}) `maybe', \textit{mataj} (\sectref{sec:8.11}) `probably', or \textit{sagir} (\sectref{sec:8.13}) `certainly'.

\largerpage
\section{Arguments of the verb}\label{sec:4.2}

\subsection{Subject}\label{sec:4.2.1}
The \isi{subject} is not marked morphologically but is defined by its position either before or after the verb according to the \isi{verb-second syntax} of Tuatschin (and more generally of Sursilvan). Subject inversion in general will be treated in \sectref{sec:5.1} about \isi{argument order}.

Singular \isi{subject nouns} which have a plural reference, such as \textit{gljut} `people' trigger the third person plural in the verb. Since this phenomenon is attested in the DRG (about 100 years ago) and in Büchli (\citeyear{Büchli1966}, at least 50 years ago), it can be assumed that it has already been in the language for a long time (\ref{ex:subjsing1}--\ref{ex:subjsing4}).

\ea
\label{ex:subjsing1}
\gll Cuélms a vals statan a la \textbf{gljut} \textbf{s'antaupan}.\\
mountain.\textsc{m.pl} and valley.\textsc{f.pl} stay.\textsc{prs.3pl} and \textsc{def.f.sg} people \textsc{refl}.meet.\textsc{prs.3pl}\\
\glt `Mountains and valleys stay, and people meet.' (\DRGoK{9}{575})
\z


\ea
\label{ex:subjsing2}
\gll \textbf{La} \textbf{gljut} tga \textbf{mavan} da quèla via ancùntar Bugnaj [...] \textbf{udévan} [...] ina vusch [...].\\
\textsc{def.f.sg} people \textsc{rel} go.\textsc{impf.3pl} from \textsc{dem.f.sg} way towards \textsc{pn} {} hear.\textsc{impf.3pl} {} \textsc{indef.f.sg} voice\\
\glt `The people who took that way towards Bugnei would hear a voice [...]. \citealt[142f.]{Büchli1966})
\z

\ea
\label{ex:subjsing3}
\gll  A bjè \textbf{gljut} \textbf{tumévan} è mju tat~[...].  \\
and many people.\textsc{f.sg} be.afraid.\textsc{impf.3pl} also \textsc{poss.1sg.m.sg} grandfather\\
\glt `And many people were afraid of my grandfather [...].' (Sadrún, m4, \sectref{sec:8.3})
\z

\ea
\label{ex:subjsing4}
\gll    [...] la \textbf{gjuvantétgna} … \textbf{fòn} parada.\\
{} \textsc{def.f.sg} youth {} do.\textsc{prs.3pl} parade.\textsc{f.s}g\\
\glt `[...] the association of young men … holds a parade.' (Zarcúns, m2)
\z

The following phenomenon is also attested in the DRG materials and in \citet{Büchli1966}. If there is \isi{subject inversion} and the \isi{subject} corresponds to a third person plural, the \isi{verb} form is in the singular (\ref{ex:subjsing5}--\ref{ex:subjsing12}), even if there is an element between the \isi{verb} and the inverted \isi{subject}. An example is (\ref{ex:3sg:e}), where \textit{è} `also' stands between the verb and the subject.


\ea
\label{ex:subjsing5}
\gll La salín \textbf{schava} `\textbf{lṣ} \textbf{utschalṣ} bétga stá ugèn.\\
\textsc{def.f.sg} wheat let.\textsc{impf.3sg} \textsc{def.m.pl} bird.\textsc{pl} \textsc{neg} stay.\textsc{inf} with\_pleasure\\
\glt `The birds didn't like to let the wheat be.' (Camischùlas, \DRGoK{3}{592})
\z

\ea
\label{ex:subjsing6}
\gll [...] ina sèra [...] \textbf{ò} `\textbf{ls} \textbf{pástars} vju ad èn las vacas.\\
{} \textsc{indef.f.sg} afternoon {} have.\textsc{prs.3sg} \textsc{def.m.pl} herdsman.\textsc{pl} see.\textsc{ptcp.unm} \textsc{comp} go.\textsc{ger} \textsc{def.f.pl} cow.\textsc{pl}\\
\glt `[...] one afternoon [...] the herdsmen saw the cows going.' (Sèlva, \citealt[28]{Büchli1966})
\z

\ea
\label{ex:subjsing7}
\gll [...] lu \textbf{vèva} \textbf{las} \textbf{fèmnas} da lavá ò la tgèsa [...].   \\
{} then have.\textsc{impf.3sg} \textsc{def.f.pl} woman.\textsc{pl} to wash.\textsc{inf} out \textsc{def.f.sg} house\\
\glt [...] then the women had to clean the house [...].' (Ruèras, f4, \sectref{sec:8.16})
\z

\ea
\label{ex:subjsing8}
\gll    [...] quaj \textbf{fagèva} \textbf{las} \textbf{gjufnas} lu schòn stém sch’ i vajan sé la nègla tg’ èla vaj dau né bétg.\\
{} \textsc{dem.unm} do.\textsc{impf.3sg} \textsc{def.f.pl} young\_woman.\textsc{pl} then in\_fact attention.\textsc{m.sg} if \textsc{3pl}  have.\textsc{sbjv.prs.3pl} up \textsc{indef.f.sg} carnation \textsc{rel} \textsc{3sg.f} have.\textsc{sbjv.prs.3sg}  give.\textsc{ptcp.unm} or \textsc{neg} \\
\glt `[...] the young women would pay close attention to whether they had put on the hat the carnation she [they] had given them or not.' (Zarcúns, m2, \sectref{sec:8.13})
\z

\ea
\label{ex:3sg:e}
\gll    Ad òz \textbf{fò} è \textbf{las} \textbf{gjufnas} … par tga … ségi avùnda.\\
and today do.\textsc{prs.3sg} also  \textsc{def.f.pl} young\_woman.\textsc{pl} {} \textsc{subord} \textsc{subord} {} \textsc{exist.prs.sbjv.3sg} enough\\
\glt `And today the young women also take part … so that … there are enough people.' (Zarcúns, m2, \sectref{sec:8.13})
\z

\ea
\label{ex:subjsing9}
\gll  [...] fòrsa scha ju ṣbaglja bitg \textbf{ṣè} \textbf{quèlas} \textbf{figuras} lu vagnidas trans-pòrtadaṣ a mézaṣ ajn quèla, ajn quaj sòntgè̱t.\\
{} maybe if \textsc{1sg} be\_wrong.\textsc{prs.1sg} \textsc{neg} \textsc{cop.prs.3sg} \textsc{dem.f.pl} figure.\textsc{pl} then \textsc{pass.aux.ptcp.f.pl} transport.\textsc{ptcp.f.pl} and put.\textsc{ptcp.f.pl} in \textsc{dem.f.sg} in \textsc{dem.m.sg} little\_chapel\\
\glt `[...] maybe, if I am not wrong, yes, when these figures were transported and put into this little chapel.' (Sadrún, m5, \sectref{sec:8.8})
\z

\ea
\label{ex:subjsing10}
\gll  [...] quaj è vagnú da bètòn’ ajn, a tanju \textbf{ò} \textbf{laṣ} aun adina. \\
{} \textsc{dem.unm} be.\textsc{prs.3sg} come.\textsc{ptcp.unm} \textsc{comp} concrete.\textsc{inf} in and hold.\textsc{ptcp.unm} have.\textsc{prs.3sg} \textsc{3pl.f} still always \\
\glt `[...] this has been concreted, and they still hold.' (Sadrún, f3, \sectref{sec:8.1})
\z

\ea
\label{ex:subjsing11}
\gll   A zatgéj \textbf{mava} \textbf{`lṣ} \textbf{aucs} mavan lu aun anzatgé … ád uáut.\\
and something go.\textsc{impf.3sg}  \textsc{def.m.pl} uncle.\textsc{pl}  go.\textsc{impf.3pl} then also something {} to forest.\textsc{m.sg} \\
\glt `And sometimes my uncles would also go sometimes ... to the forest.' (Sadrún, m4, \sectref{sec:8.3})
\z

\ea
\label{ex:subjsing12}
\gll  A ... cò \textbf{mava} \textbf{`lṣ} \textbf{buéts} la stad ad alp~[...].  \\
and {} here go.\textsc{impf.3sg} \textsc{def.m.pl} boy.\textsc{pl} \textsc{def.f.sg} summer to alp.\textsc{m.sg}\\
\glt `And ... here, during summer, the boys would go to the summer pastures [...].' (Cavòrgja, m7, \sectref{sec:8.17})
\z
 

\subsection{Direct object}\label{sec:4.2.2}
The \isi{direct object} is not marked morphologically, but is defined by its syntactic position, be it a pronoun or a full \isi{noun phrase}. With simple tenses, it is located after the \isi{verb} (\ref{ex:do1}) or after the subject in case of \isi{subject inversion} (\ref{ex:do2}), as well as after the negator \textit{bétga}, particles, and adverbs that have been treated in \sectref{sec:4.1.3} about particle verbs.

\ea
\label{ex:do1}
\gll  A… vagnéva mè̱ndar a mè̱ndar a dumagnavan \textbf{bigj}' {\ob}\textbf{èl}{\cb} ál, ál spital lèva  `l bitg \isi{í} né tiar miadis.  \\
and become.\textsc{impf.3sg} worse and worse and induce.\textsc{impf.3pl} \textsc{neg} \textsc{3sg.m} to.\textsc{def.m.sg} to.\textsc{def.m.sg} hospital want.\textsc{impf.3sg} \textsc{3sg.m} \textsc{neg} go.\textsc{inf} or to doctor.\textsc{m.pl} \\
\glt `And … it became worse and worse and they couldn’t induce [him] to go to the, to the hospital he didn’t want to go, nor to the doctors.' (Sadrún, m4, \sectref{sec:8.3})
\z

\ea\label{ex:do2}
\gll  Qu’ è adin' aviart a lu saṣ í ajn api vèzas {\ob}\textbf{té}{\cb} {\ob}\textbf{quèlas} ah... \textbf{figuraṣ}{\cb} ajn grondèzja da carstgaun.\\
\textsc{dem.unm} \textsc{cop.prs.3sg} always open.\textsc{adj.unm} and then can.\textsc{prs.2sg} go.\textsc{inf} in and see.\textsc{prs.2sg} \textsc{2sg} \textsc{dem.f.pl} ah figure.\textsc{pl} in size.\textsc{f.sg} of human\_being.\textsc{m.sg} \\
\glt `This is always open, and then you can step in and then you see these eh ... figures of the size of human beings.' (Sadrún, m5, \sectref{sec:8.8})
\z

With compound tenses, the \isi{direct object} is located after the participle (\ref{ex:do3}) or after the verbal particle if there is one (\ref{ex:do4}), but not after the inverted \isi{subject} or the negator, since these elements follow the finite \isi{verb}.

\ea\label{ex:do3}
\gll    Quaj ò \textbf{bégja} dau {\ob}\textbf{discusjun}{\cb}.\\
\textsc{dem.unm} have.\textsc{prs.3sg} \textsc{neg} \textsc{exist.ptcp.unm} discussion.\textsc{m.sg}\\
\glt `There was no discussion.' (Ruèras, m1, \sectref{sec:8.2})
\z

\ea
\label{ex:do4}
\gll  Ju prèn magari sé [\textbf{èl}].\\
\textsc{1sg} take.\textsc{prs.1sg} sometimes up \textsc{3sg.m}\\
\glt `Sometimes I lift him up.' (Sadrún, m6)
\z

One \isi{ditransitive verb}, \isi{\textit{dumandá}} `ask, ask for', has two direct objects (\ref{ex:doubledo1} and \ref{ex:doubledo2}).\footnote{This phenomenon is unusual in Romance languages and is probably due to German influence \textit{(jemanden etwas fragen}, literally `somebody (accusative) something (accusative) ask').}

\ea
\label{ex:doubledo1}
\gll  [...]  ina zagríndara […] ò dumandau {\ob}\textbf{la} \textbf{mùma} \textbf{da} \textbf{tgèsa}{\cb} {\ob}\textbf{in} \textbf{tgavégl} \textbf{da} \textbf{sia} \textbf{buéba}{\cb}. \\
{} \textsc{indef.f.sg} Yenish {} have.\textsc{prs.3sg}   ask.\textsc{ptcp.unm} \textsc{def.f.sg} mother.\textsc{f} of house one.\textsc{m} hair of \textsc{poss.3sg.f.sg} girl \\
\glt `[…] a Yenish woman [...] asked the mother of the house for one hair of her daughter.' (Bugnaj, \citealt[131]{Büchli1966})
\z

\ea
\label{ex:doubledo2}
\gll Èl ò dumandau {\ob}{\textbf{quaj}}{\cb} {\ob}{\textbf{la}} {\textbf{mùma}}{\cb}.\\
\textsc{3sg.m}  have\textsc{.prs.3sg} ask.\textsc{ptcp.unm} \textsc{dem.unm} \textsc{def.f.sg} mother\\
\glt `He asked his mother this.' (Sadrún, m4)
\z

With \textit{dumandá} 'ask', both direct objects may be passivised (\ref{ex:doubledo3} and \ref{ex:doubledo4}).

\ea
\label{ex:doubledo3}
\gll {\ob}\textbf{Quaj}{\cb} è vagnú dumandau la mùma.\\
 \textsc{dem.unm} be.\textsc{prs.3sg} \textsc{ pass.ptcp.unm} ask.\textsc{ptcp.unm} \textsc{def.f.sg} mother\\
\glt `This the mother was asked.' (Sadrún, m5)
\z

\ea
\label{ex:doubledo4}
\gll {\ob}\textbf{La} \textbf{mùma}{\cb} è vagnida dumandada quaj. \\
\textsc{def.f.sg} mother be.\textsc{prs.3sg} \textsc{pass.ptcp.f.sg} ask.\textsc{ptcp.f.sg} \textsc{dem.unm}\\
\glt `The mother was asked this.' (Sadrún, m6)
\z


Other cases could be \textit{dá fjuc la lèna} (m10) `light the firewood', literally `give fire the firewood',  \textit{dá culur las sèndas} `paint  the trails' (f3, \sectref{sec:8.1}), literally `give colour the trails', or \textit{dá culur al mir} (m4) `give colour the wall'.

But in contrast to \textit{dumandá}, only the \isi{\textsc{recipient}} of \textit{dá} may be passivised, in the sense that it is promoted to \isi{subject} position and that the \isi{past participle} agrees with it (\ref{miragr} and \ref{lènaagr}).

\ea
\label{miragr}
\gll  \textbf{Al} \textbf{mir} è bigja \textbf{vagnús} \textbf{daus} culur.\\
\textsc{def.m.sg} wall be.\textsc{prs.3sg} \textsc{neg} \textsc{pass.ptcp.m.sg} give\textsc{.ptcp.m.sg} colour.\textsc{f.sg}\\
\glt `The wall has not been painted.' (Sadrún, m4)
\z

\ea
\label{lènaagr}
\gll  \textbf{La} \textbf{lèna} è bigja \textbf{vagnida} \textbf{dada} fjuc.\\
\textsc{def.f.sg} firewood be.\textsc{prs.3sg} \textsc{neg} \textsc{pass.ptcp.f.sg} give.\textsc{ptcp.f.sg} fire.\textsc{m.sg}\\
\glt `The firewood has not been lit.' (Sadrún, m5)
\z


However, with a regular \isi{ditransitive  verb} like \textit{dá} `give' as in (\ref{dáact}), the \isi{\textsc{recipient}} may not be passivised (\ref{dápass}).

\ea
\label{dáact}
 \gll Ju a dau in bi schénghètg da la mùma.\\
 \textsc{1sg} have\textsc{.prs.1sg} give\textsc{.ptcp.unm} \textsc{indef.m.sg} beautiful.\textsc{unm} present \textsc{dat} \textsc{def.f.sg} mother\\
\glt  `I gave a beautiful present to my mother.' (Sadrún, m6)
 \z

\ea
\label{dápass}
\gll *La mùma è \textbf{vegnida} \textbf{dada} in bi schénghètg.\\
\textsc{def.f.sg} mother \textsc{cop.prs.3sg} \textsc{pass.ptcp.f.sg} give.\textsc{ptcp.f.sg} \textsc{def.m.sg} beautiful.\textsc{unm} present\\
\glt `Mother was given a beautiful present.' (Sadrún, m5)
\z

For these reasons, I conclude that \textit{mir} `wall' and `\textit{lèna}' `firewood' in (\ref{miragr} and \ref{lènaagr}) have to be viewed as direct objects and not as indirect objects.


As for the \isi{\textsc{theme}}, it cannot be passivised, in the sense that it cannot be promoted to \isi{subject} position and trigger the \isi{agreement} of the \isi{past participle} (\ref{culurpassnongram}).

\ea
\label{culurpassnongram}
\gll  \textbf{*Culur} è   bigja \textbf{*vagnida} \textbf{*dada} al mir.\\
colour.\textsc{f.sg} be.\textsc{prs.3sg} \textsc{neg} \textsc{pass.ptcp.f.sg} give.\textsc{ptcp.f.sg}  \textsc{def.m.sg} wall\\
\glt `Colour has not been given to the wall.' (literally `Colour has not been given given the wall'). (Sadrún, m6)
\z


Instead, an \isi{impersonal passive} with the \isi{expletive pronoun} \textit{i} in \isi{subject} position  as well as the \isi{past participle} in its \isi{unmarked} form must be used (\ref{culurimpers1}).

\ea
\label{culurimpers1}
\gll  \textbf{I} è   bigja \textbf{vagnú} \textbf{dau} \textbf{culur} al mir.\\
\textsc{expl} be.\textsc{.prs.3sg} \textsc{neg} \textsc{pass.ptcp.unm} give.\textsc{ptcp.unm}  colour.\textsc{f} \textsc{def.m.sg} wall\\
\glt `Colour has not been given to the wall.' (literally `It has not been given colour to the wall'). (Sadrún, m6)
\z


This holds for \textit{dá fjuc} as well.

Since \textit{culur} and \textit{fjuc} in \textit{dá culur la prajt} and \textit{dá fjuc la lèna}, cannot be passivised, they cannot be considered direct objects of \textit{dá} in these two constructions. Therefore I suggest that they form a unit with \textit{dá} so that \textit{dá culur} and \textit{dá fjuc} are two compound \isi{monotransitive} verbs with \textit{al mir} and \textit{la lèna} as their direct objects. However, in order to resolve this problem, more research is needed.

Some younger speakers do not accept  constructions like \textit{dá culur la prajt} and \textit{dá fjuc la lèna}, but prefer the \isi{\textsc{beneficiary}} to be marked by dative \textit{da}:\textit{ dá fjuc \textbf{da} la lèna}, \textit{dá culur \textbf{dal} mir}, whereas they do accept the construction with \textit{dumandá} `ask' as in (\ref{ex:doubledo2}).


With mono- or \isi{ditransitive verbs}, the \isi{direct object} may be omitted if it has been mentioned before (\ref{ex:noobj1}--\ref{ex:noobj3}).

\ea
\label{ex:noobj1}
\gll  A què èra schòn strètg, álṣò sch’ ju stèṣ aun fá in’ jèda \textbf{quaj}, figès ju bétga {\ob}\longrule{\cb}.  \\
and \textsc{dem.unm} \textsc{cop.impf.3sg} really narrow.\textsc{adj.unm} well if \textsc{1sg}  must.\textsc{cond.1sg} again do.\textsc{inf} one.\textsc{f.sg} time \textsc{dem.unm} do.\textsc{cond.1sg} \textsc{1sg} \textsc{neg} \textsc{do} \\
\glt `And this was really narrow, well, if I had to do it once again, I wouldn’t do it.' (Ruèras, m10, \sectref{sec:8.7})
\z

\ea
\label{ex:noobj2}
\gll    Api ò èla cò détg: Cool, ju mòn grad a raquénta {\ob}\textbf{dad} \textbf{èlas}{\cb} {\ob}\textbf{{\longrule}}{\cb}.\\
and have.\textsc{prs.3sg} \textsc{3sg.f} here say.\textsc{ptcp.unm} cool \textsc{1sg}  go.\textsc{prs.1sg} right\_away and tell.\textsc{prs.1sg} \textsc{dat} \textsc{3pl.f} \textsc{do}\\
\glt `And then she said there: Cool, I’ll just go and tell them.' (Camischùlas, f6, \sectref{sec:8.4})
\z

\ea\label{ex:noobj3}
\gll  A lura … nus, nuṣ ṣchajn {\ob}\textbf{dis} \textbf{vischnauncas}{\cb} {\ob}\textbf{{\longrule}}{\cb}, tarmètajn ábar tutina `eine Mängelmeldung'\footnotemark{} scù quaj ò nùm.  \\
and then {} \textsc{1pl} \textsc{1pl} tell.\textsc{prs.1pl} \textsc{def.dat.pl} municipality.\textsc{pl} \textsc{do} send.\textsc{prs.1pl} but nevertheless a report\_of\_damage as \textsc{dem.unm} have.\textsc{prs.3sg} name.\textsc{m.sg}\\
\glt `And then we tell it to the municipalities, but we nevertheless send `a report of damages' as this is called.'\footnotetext{Said in \ili{Standard German}.} (Sadrún, f3, \sectref{sec:8.1})
\z

The \isi{direct object} does not have to be immediately adjacent to the verb (\ref{ex:do5}).

\ea
\label{ex:do5}
	\gll Ju \textbf{vèv}’ als véntgatschún d’ avrél \textbf{nataléci} [...].   \\
\textsc{1sg} have.\textsc{impf.1sg} \textsc{def.m.pl} twenty-five of April.\textsc{m.sg} birthday.\textsc{m.sg} \\
\glt `I had my birthday on April 25 [...].' (Ruèras, f4, \sectref{sec:8.16})
\z


\subsection{Indirect object}\label{sec:4.2.3}
As mentioned in \sectref{sec:3.2.1.3} and \sectref{sec:3.6.1}, the nominal and pronominal definite \isi{indirect object} was introduced by \textit{di/dis} or \textit{li/lis}. Nowadays, the \isi{indirect object}, whether definite or not, is almost exclusively introduced by \textit{da} (but see (\ref{datart7}--\ref{datart11}) in \sectref{sec:3.2.1.3} about the speech of some older people).

In most cases, the \isi{indirect object} precedes the \isi{direct object} (\ref{ex:io:2} and \ref{ex:io:3}), but (\ref{ex:io:1}) shows that the inverse also occurs.
 
\ea\label{ex:io:2}
\gll  «Quèl vès lu aun da pijè {\ob}\textbf{da} \textbf{té}{\cb} {\ob}\textbf{al} \textbf{pustrètsch} \textbf{dal} \textbf{piertg} \textbf{tga} \textbf{té} \textbf{vèvas} \textbf{partgirau}{\cb}.» \\
\textsc{dem.m.sg} have.\textsc{cond.3sg} then still \textsc{comp} pay.\textsc{inf} \textsc{dat} \textsc{2sg} \textsc{def.m.sg} money of.\textsc{def.m.sg} pig \textsc{rel} \textsc{2sg} have.\textsc{impf.2sg} look\_after.\textsc{ptcp.unm}\\
\glt `This one should still pay you the money of the pig you had looked after.' (Sadrún, m6, \sectref{sec:8.11})
\z

\ea
\label{ex:io:3}
\gll    A nus mavan culs pòrs sé Valtgèva, mintga dé sé a gjù, ju savès raquintá {\ob}\textbf{da} \textbf{té}{\cb} {\ob}\textbf{quaj}{\cb}.\\
and \textsc{1pl}  go.\textsc{impf.1pl} with.\textsc{def.m.pl} pig.\textsc{pl} up \textsc{pn} every day.\textsc{m.sg} up and down  \textsc{1sg}  can.\textsc{cond.1sg}  tell.\textsc{inf}  \textsc{dat}  \textsc{2sg} \textsc{dem.unm}\\
\glt `And we would go up to Valtgeva with the pigs, every day up and down, I could tell you about that.' (Sadrún, m6, \sectref{sec:8.11})
\z

\ea\label{ex:io:1}
\gll Api … ṣè quaj vagnú príu ajn, a lu ò la cònfadarazjun dau vi {\ob}\textbf{quaj}{\cb} {\ob}\textbf{da} \textbf{mintga} \textbf{cantún}{\cb} [...].   \\
and {} be.\textsc{prs.3sg} \textsc{dem.unm} \textsc{pass.ptcp.unm} take.\textsc{ptcp.unm} in and then have.\textsc{prs.3sg} \textsc{def.f.sg} confederation give.\textsc{ptcp.unm} over \textsc{dem.unm} \textsc{dat} every canton.\textsc{m.sg}\\
\glt `Then … this has been adopted, and then the confederation handed it over to every canton [...].' (Sadrún, f3, \sectref{sec:8.1})
\z

The usual semantic role of an \isi{indirect object} is \isi{\textsc{recipient}} as in the examples above, but with verbs like \textit{plaṣchaj} `please' or \textit{fá plaṣchaj} `make pleasure', the semantic role is  \isi{\textsc{experiencer}} as in (\ref{ex:dat:exp:1} and \ref{ex:dat:exp:2}).

\ea\label{ex:dat:exp:1}
\gll A quaj \textbf{plaṣchéva} nuéta pròpi \textbf{da} \textbf{mé}. \\
and \textsc{dem.unm} please.\textsc{impf.3sg} nothing really \textsc{dat} \textsc{1sg}   \\
\glt `And I really didn’t like that.' (Sadrún, m4, \sectref{sec:8.3})
\z

\ea
\label{ex:dat:exp:2}
\gll  A quaj fò adina \textbf{da} \textbf{mé} … plaṣchaj.  \\
and \textsc{dem.unm} make.\textsc{prs.3sg} always \textsc{dat} \textsc{1sg} {} pleasure.\textsc{m.sg} \\
\glt `And this is always a … pleasure for me.' (Sadrún, f3, \sectref{sec:8.1})
\z

\textit{Da} as a \isi{dative marker} is also reported for the \ili{Surmiran} dialect of Marmorera (\ref{ex:marm1} and \ref{ex:marm2}).

\ea\label{ex:marm1}
\gll  Ja da detg \textbf{da} \textbf{mia} \textbf{sora} tgi la vegna no.\\
\textsc{1sg} have.\textsc{prs.1sg} say.\textsc{ptcp.unm} \textsc{dat} \textsc{poss.1sg.f.sg} sister \textsc{comp} \textsc{3sg.f} come.\textsc{prs.sbjv.3sg} here\\
\glt `I told my sister to come here.' (\ili{Surmiran}, Marmorera, \DRGoK{5}{19})
\z

\ea\label{ex:marm2}
\gll  Ja do(u)m in mail \textbf{da} \textbf{chel} \textbf{umfant}.\\
\textsc{1sg} give.\textsc{prs.1sg} \textsc{indef.m.sg} apple \textsc{dat} \textsc{dem.m.sg} child\\
\glt `I give an apple to this child.' (\ili{Surmiran}, Marmorera, \DRGoK{5}{19})
\z

\section{Adjuncts of the verb}\label{sec:4.3}

\subsection{Locative adjuncts}\label{sec:4.3.1}
Locative adjuncts are realised as adverbs or combinations of adverbs, as \isi{noun phrases}, or as \isi{adpositional phrases}. The latter are formed with simple or complex prepositions as well as with circumpositions.

Adverbs are either simple complex. Complex adverbs are combinations of adpositions, combinations of adpositions with adverbs, or multiple adverbs.

Simple adverbs include \textit{cò} `here', \textit{daspèras} `next to it', \textit{drétg} `right', \textit{gljunsch} `far away' (\ref{ex:glj1}), \textit{lò} `there', \textit{nagljú} `nowhere',  \textit{saniastar} `left'(\ref{ex:san1}), \textit{tschò} `here'\footnote{\textit{Tschò} `here' is usually used together with \textit{lò} `there'; see example (\ref{ex:tscholo}).}, and \textit{zanú/zanúa} `somewhere' (\ref{ex:zan1}).

\ea
\label{ex:glj1}
\gll  [...] al bùrdi stèva ò ualti \textbf{gljunsch}, quaj balantschava in téc.\\
{} \textsc{def.m.sg} load  stand.\textsc{impf.3sg} out quite far \textsc{dem.unm} roll.\textsc{impf.3sg}  \textsc{indef.m.sg} bit\\
\glt `[...] the load was sticking out quite a lot, it was rolling a bit.' (Ruèras, m10, \sectref{sec:8.7})
\z

\ea
\label{ex:san1}
\gll A lura … quaj è grat stau in téc, quèls mavan tùt \textbf{saniastar} sén via. \\
and then {} \textsc{dem.unm} be.\textsc{prs.3sg} just \textsc{cop.ptcp.unm} \textsc{indef.m.sg} bit \textsc{dem.m.pl} go.\textsc{impf.3pl} completely left.\textsc{adj.unm} on road.\textsc{f.sg}  \\
\glt `And then … this was just for a bit, they were walking on the very left side of the road.' (Ruèras, m10, \sectref{sec:8.7})
\z

\ea
\label{ex:tscholo}
\gll [...] ábar savènṣ è aun da quèls tg’ èran fumégl \textbf{tschò} né \textbf{lò}.\\
{} but often also in\_addition of \textsc{dem.m.pl} \textsc{rel} \textsc{cop.impf.3pl} farmhand.\textsc{m.sg} here or there\\
\glt `[...] but often also one of those that were farmhands here and there.' (Cavòrgja, m7, \sectref{sec:8.17})
\z

\ea
\label{ex:zan1}
\gll  A … api ṣè lu capitau … mù gè quaj cu `l vèva sjatòntanù̱v òns circa, ṣè `l \textbf{zanúa} para i è ruclaus.  \\
and {} and be.\textsc{prs.3sg} then happen.\textsc{ptcp.unm} {} but yes \textsc{dem.unm} when \textsc{3sg.m} have.\textsc{impf.3sg} seventy-nine year.\textsc{m.pl} around be.\textsc{prs.3sg} \textsc{3sg.m} somewhere  seem.\textsc{prs.3sg} \textsc{expl} also fall.\textsc{ptcp.m.sg}\\
\glt `And … and then it happened, ... well when he was about seventy-nine years old, it seems that he also fell down somewhere.' (Sadrún, m4, \sectref{sec:8.3})
\z

Combinations of prepositions or of prepositions with adverbs are e.g. \textit{ajndadájns} `inside' (< ` in + inside') (\ref{ex:advcomb1}), \textit{angjù} `down' (< `in + down'), \textit{ansé} `up' (< `in + up'), \textit{daváuntiar} `in front' (< `in front + towards'), \textit{òrdavaun} `in front' (< `out + before') (\ref{ex:advcomb2}), \textit{sédangjù} (< `up + down'), \textit{ṣurangjù} (`over + down') (\ref{ex:advcomb3}),  and \textit{vinavaun} `farther' (< `over + in front').

\ea
\label{ex:advcomb1}
\gll Ju spétga té \textbf{ajndadajns}, ajn stiva.\\
\textsc{1sg} wait\textsc{.prs.1sg} \textsc{2sg} inside in living room.\textsc{f.sg}\\
\glt `I wait for you inside, in the living-room.' (Cavòrgja, f1)
\z

\ea
\label{ex:advcomb2}
\gll    [...] a lu mava … in \textbf{òrdavaun} a lu mavan quèls pòrs tùt ajn còrda [...].\\
{} and then go.\textsc{impf.3sg} {} one.\textsc{m} in\_front and then go.\textsc{impf.3pl} \textsc{dem.m.pl} pig.\textsc{pl} all in single\_file.\textsc{f.sg} \\
\glt `[...] and then … one would move in front and then the other pigs would follow in single file [...].' (Sadrún, m6, \sectref{sec:8.11})
\z

\ea
\label{ex:advcomb3}
\gll A lu èssan nuṣ i sé api vagní \textbf{ṣurangjù} ad i gjù Tiefenbach.\\
and then be.\textsc{prs.1pl} \textsc{1pl} go.\textsc{ptcp.m.pl} up and come.\textsc{ptcp.m.pl} over\_down and go.\textsc{ptcp.m.pl} down \textsc{pn}\\
\glt `And then we went up and came down [from] over [the avalanche barriers] and went down to Tiefenbach.' (Ruèras, m10, \sectref{sec:8.7})
\z

Two special cases are \textit{gjùdém} `at the very bottom' and \textit{séssum} `at the very top'; the former is a combination of the prepositions \textit{gjù} and \textit{sé} along with the derivational morpheme \textit{-dém} `(down)most', the latter is a combination of the préposition \textit{sé} `up' with the derivational morpheme \textit{-sum} `(upper)most' (\ref{ex:sum1}).

\ea
\label{ex:sum1}
\gll [...] ad amplanju sé agl ésch cul … ròlas da pupí da tualèta tòcan \textbf{séssum} [...].\\
{} and fill.\textsc{ptcp.unm} up \textsc{def.m.sg} door with.\textsc{def.m.sg} {} roll.\textsc{f.pl} of paper.\textsc{m.sg} of toilet.\textsc{f.sg} until very\_top\\
\glt `[...] and filled up the doorway with the … rolls of toilet paper until the very top [...].' (Camischùlas, f6, \sectref{sec:8.4})
\z

Locative adverbs are also formed by combinations of adverbs with either the \isi{demonstrative} \textit{quaj} `this' (\ref{ex:quajnase}), the \isi{comparative} \textit{plé} `more (than)', or the \isi{consecutive} \textit{schi} `so (that)' (\ref{ex:schindanajn}). With \textit{plé} the compared element may be explicit (\ref{ex:plendanoragju}) or implicit (\ref{ex:plendanora}).

\ea
\label{ex:quajnase}
\gll  Api, quaj èra … \textbf{quajnassé} in pòst da survigilònza [...].\\
and \textsc{dem.unm} \textsc{cop.impf.3sg} {} \textsc{dem.unm}.in\_up \textsc{indef.m.sg} guard of vigilance.\textsc{f.sg}\\
\glt `And there was … up there a vigilance guard [...].' (Sadrún, f3, \sectref{sec:8.1})
\z

\ea
\label{ex:plendanoragju}
\gll  [...] quèls tg’ èran staj a mèssa èran schòn [...] \textbf{pléndanòragjù̱} \textbf{tga} quaj tga nuṣ èssan, tg’ als, tg’ als méls èn galòpaj.\\
{} \textsc{dem.m.pl} \textsc{rel} be.\textsc{impf.3pl} \textsc{cop.ptcp.m.pl} at mass.\textsc{f.sg} \textsc{cop.impf.3pl} already [...] more\_out\_down than \textsc{dem.unm} \textsc{rel} \textsc{1pl} be.\textsc{prs.1pl} \textsc{rel} \textsc{def.m.pl} \textsc{rel} \textsc{def.m.pl} mule.\textsc{pl} be.\textsc{prs.3pl} gallop.\textsc{ptcp.m.pl}\\
\glt `[...] those who had attended the mass were already [...] farther down than we were, that the, that the mules galloped.' (Ruèras, m10, \sectref{sec:8.7})
\z

\ea
\label{ex:plendanora}
\gll [...] lu ṣè aun dus trajs intarassants lògans ajn cò, ajn Burganèz, qu' è in tòc \textbf{pléndanòra} [...].\\
{}  then \textsc{exist.prs.3sg} still two.\textsc{m} three interesting.\textsc{pl} place.\textsc{pl} in here in \textsc{pn} \textsc{dem.unm} \textsc{cop.prs.3sg} \textsc{indef.m.sg} bit more\_out \\
\glt `[...] there are furthermore two or three interesting places up there, in Burganez, this is a little bit farther down the valley [...].' (Sadrún, m4, \sectref{sec:8.3})
\z

\ea
\label{ex:schindanajn}
\gll [...] la détga di tg’ è̱rian \textbf{schindanajn} tg' i udé̱vian c’ i tucavi da mjadṣ-dé ajnt Ruèras.\\
{} \textsc{def.f.sg} legend say.\textsc{prs.3sg}  \textsc{comp} \textsc{cop.impf.sbjv.3pl} so\_in \textsc{comp} \textsc{3pl} hear.\textsc{impf.sbjv.3pl} \textsc{comp} \textsc{expl} beat.\textsc{impf.sbjv.3sg} of noon.\textsc{m.sg} in \textsc{pn}\\
\glt `[...] the legend says that they were so deep in the cave that they heard the clock strike noon in Rueras.' (Sadrún, m4, \sectref{sec:8.3})
\z

Some combinations of adverbs include \textit{cò angjù} `down here' (\ref{ex:coang1}), \textit{gjù cò} `down here', and \textit{ò lò} `out there' (\ref{ex:olo1}).

\ea
\label{ex:coang1}
\gll  \textbf{Cò} \textbf{angjù̱} va ju la finala nagíns.\\
here in\_down have.\textsc{prs.1sg} \textsc{1sg} \textsc{def.f.sg} end no.\textsc{m.pl}  \\
\glt `In the end I don’t have any down here.' (Sadrún, f3, \sectref{sec:8.1})
\z

\ea
\label{ex:olo1}
\gll  \textbf{Ò} \textbf{lò} vòu fòrza schòn è survagnú in téc quajda d' í par crapa [...].\\
out there  have.\textsc{prs.1sg.1sg} maybe really also get.\textsc{ptcp.unm} \textsc{indef.m.sg} bit desire.\textsc{f.sg} \textsc{comp} go.\textsc{inf} for stone.\textsc{coll}\\
\glt `Out there I might have started enjoying looking for stones a bit [...].' (Sadrún, m4, \sectref{sec:8.3})
\z

The adverbs \textit{cò} `here' and \textit{lò} `there' combine with the prepositions \textit{sé} `up' and \textit{gjù} `down'. In the case of \textit{cò} and \textit{sé}, they combine either as \textit{sé cò} (\ref{ex:seco1}) or as \textit{cò sé} (\ref{ex:cose1}).\footnote{The syntactic status of \textit{sé} in the combination \textit{cò sé} is not clear to me, because it looks like a postposition. However, no other pospositions are attested in the domain of locatives, and in other domains they are very scarce.} If one says

\ea
\label{ex:seco1}
\gll Ju sùn \textbf{sé} \textbf{cò}.\\
\textsc{1sg} \textsc{cop.prs.1sg} up here\\
\glt `I am up here.' (Ruèras, m10)
\z

it implies that speaker and hearer are at the same place. If one says

\ea
\label{ex:cose1}
\gll Ju sùn \textbf{cò} \textbf{sé}.\\
\textsc{1sg} \textsc{cop.prs.1sg} here up\\
\glt `I am up here.' (Ruèras, m10)
\z

it implies that the speaker is at a higher place than the hearer. This semantic difference does not exist with \textit{lò sé} (\ref{ex:lose1}) and \textit{sé lò} (\ref{selo1}), both meaning `up there'. According to several consultants, both constructions refer to the same situation; the more common construction is \textit{sé lò}.

\ea
\label{ex:lose1}
\gll Nuṣ vèvan da partgirá als tiars, als buéts, a stèvan \textbf{lò} \textbf{sé}, sé majṣès.\\
\textsc{1pl} have.\textsc{impf.1pl} \textsc{comp} \textsc{mind.inf} \textsc{def.m.pl} animal.\textsc{pl} \textsc{def.m.pl} boy.\textsc{pl} and stay.\textsc{impf.1pl} there up up assembly\_of\_houses.\textsc{m.sg}\\
\glt `We had to mind the animals, we the boys, and stayed up there, at the \textit{majṣès}.' (Cavòrgja, m7, \sectref{sec:8.17})
\z

\ea
\label{selo1}
\gll Nuṣ stèvan \textbf{sé} \textbf{lò}, sé majṣès.\\
\textsc{1pl} stay.\textsc{impf.1pl} up there up assembly\_of\_houses.\textsc{m.sg}\\
\glt `We stayed up there, at the \textit{majṣès}.' (Cavòrgja, f5)
\z

In Tuatschin -- as well as in other Romansh varieties -- adpositions heading place names or names of important buildings like churches or schools take on a special meaning and will be treated below in this section (see \tabref{loc1}--\tabref{loc6}).

The following simple prepositions occur in the corpus: \textbf{a} `in, to' (\ref{ex:a1}), \textit{ajn} `in, into', \textit{ajnt/ajnta} `in, into' (\ref{ex:ajnta1}), \textit{amiaz} `in the middle of', \textit{ancùntar} `towards', \textit{antù̱rn} `around' (\ref{ex:anturn1}), \textit{avaun} `before' (\ref{ex:avaun1}), \textit{da} `from' (\ref{ex:da1}), \textit{dadajns} `inside', \textit{dadò} `outside', \textit{\textit{davùs}} `behind', \textit{gjù} `down', \textit{nùca} `by, next to' (\ref{ex:nuca1} and \ref{ex:nuca2}), \textit{ò/òra} `out', \textit{sé} `up', \textit{sén} `on', \textit{séssum} `on top of', \textit{spèr} `next to', \textit{ṣur/ṣu} `above', \textit{ṣut} `under', \textit{tòca/tòcan} `until', \textit{viars} `towards'.

\ea
\label{ex:a1}
\gll Ju vòn \textbf{a} \textbf{tgèsa}.\\
\textsc{1sg} go.\textsc{prs.1sg} to house.\textsc{f.sg}\\
\glt `I am going home.' (Cavòrgja, f1)
\z

\ea
\label{ex:ajnta1}
\gll A la nògj ṣè `l gjat vagnús \textbf{ajnta} \textbf{létg} […].\\
and \textsc{def.f.sg} night be.\textsc{prs.3sg} \textsc{def.m.sg} cat come.\textsc{ptcp.m.sg} into bed\\
\glt `And at night the cat came into [his] bed […].' (Cavòrgja, \citealt[121]{Büchli1966})
\z

\ea
\label{ex:anturn1}
\gll    [...] fatg antù̱rn in sujèt mataj a méz sé quaj \textbf{antù̱rn} \textbf{al} \textbf{vjantar} dals pòrs.\\
{} do.\textsc{ptcp.unm} around \textsc{indef.m.sg} rope probably and put.\textsc{ptcp.unm} up  \textsc{dem.unm} around \textsc{def.m.sg} belly of.\textsc{def.m.pl} pig.\textsc{pl} \\
\glt `[...] tied a rope around, and put them around the belly of the pigs.' (Sadrún, m6, \sectref{sec:8.11})
\z

\ea
\label{ex:avaun1}
\gll Ju spétga \textbf{avaun} \textbf{tgèsa}.\\
\textsc{1sg} wait\textsc{.prs.3sg} before house.\textsc{f.sg}\\
\glt `I'm waiting in front of the house.' (Cavòrgja, f1)
\z

\ea
\label{ex:da1}
\gll Èl èra \textbf{da} \textbf{Zarcúns}.\\
\textsc{3sg.m} \textsc{cop.impf.3sg} from \textsc{pn}\\
\glt `He was from Zarcuns.' (Zarcúns, m2)
\z


\ea
\label{ex:nuca1}
\gll [...] prandévan nòssa … sarvjèta, matévan sé \textbf{nùca} \textbf{la} \textbf{sòr}’ \textbf{Andréa}, a lu stgèvan nus raṣdá ròmòntsch.\\
{} take.\textsc{impf.1pl} \textsc{poss.1pl.f.sg} {} napkin put.\textsc{impf.1pl} up  where \textsc{def.f.sg} nun \textsc{pn} and then be\_allowed.\textsc{impf.1pl} \textsc{1pl} speak.\textsc{inf} Romansh.\textsc{m.sg}\\
\glt `[...] [we] would take our … napkin, would put it next to Sister Andrea, and then we were allowed to speak Romansh.' (Camischùlas, f6, \sectref{sec:8.4})
\z

\ea
\label{ex:nuca2}
\gll [...] api vèvanṣ da partgirá als tiars, mav’ ins lò sédòr ál sit, grat cò \textbf{nùca} \textbf{quèla} \textbf{rùsna} tgu a raquintau.\\
{} and have.\textsc{impf.1pl.1pl} \textsc{comp} mind.\textsc{inf} \textsc{def.m.pl} animal.\textsc{pl} go.\textsc{impf.3sg} \textsc{gnr} there up\_out in.\textsc{def.m.sg} south just there by \textsc{dem.f.sg} hole \textsc{rel.1sg} have.\textsc{prs.1sg} tell.\textsc{ptcp.unm}\\  
\glt `[...] and we also had to mind the animals; we would then go up to the south, just by that cave I have told about.' (Sadrún, m4, \sectref{sec:8.3})
\z

If the referent is known or can be inferred by the hearer, \isi{noun phrases} introduced by a preposition preclude the use of determiners, as in the phrases \textit{ajn cuṣchina, nuégl, tgòmbra} `in the kitchen, barn, bedroom', \textit{ajnta létg} `in bed', or \textit{avaun tgèsa} `in front of the house'.

But if a noun introduced by a preposition refers to an entity that is not known to the hearer, the noun must be modified by the \isi{indefinite article} (\ref{ex:ajnina1}).

\ea
\label{ex:ajnina1}
\gll  Èl è curdauṣ gjùdajn \textbf{ajn} \textbf{ina} \textbf{rùsna}.\\
\textsc{3sg.m} be.\textsc{prs.3sg} fall.\textsc{ptcp.m.sg} down\_into in \textsc{indef.f.sg} hole\\
\glt `[...] he fell in an hole [...].' (Sadrún, m4)
\z

The \isi{noun} may be modified by a \isi{demonstrative determiner} in \isi{anaphoric} function (\ref{ex:ajnquela1}).

\ea
\label{ex:ajnquela1}
\gll \textbf{Ajn} \textbf{quèla} \textbf{caplùta} ṣè ajn ina, la quarta stazjun da la via da la crusch.\\
in \textsc{dem.f.sg} chapel \textsc{cop.prs.3sg} in \textsc{indef.f.sg} \textsc{def.f.sg} fourth station of \textsc{def.f.sg} way of \textsc{def.f.sg} cross \\
\glt `In this chapel there is a, the fourth station of Christ’s way of the Cross.' (Sadrún, m5, \sectref{sec:8.8})
\z

If the \isi{noun phrase} is modified by an adjective, the \isi{definite article} must occur, as in the phrase \textit{ajn la tgèsa véglja} `in the old house' (\citealt[30]{Büchli1966}. Note that usually the preposition \textit{ajn} and the \isi{definite article} are fused: \textit{ajn al} → \textit{ajl/ál}, \textit{ajn la} → \textit{ajla/ála}.

Some additional examples of \isi{bare nouns} in prepositional phrases include \textit{ajn caplùta} `in the chapel' (\citealt[45]{Büchli1966}, \textit{ajn caplùta da Sòntgaclau} `in the chapel of St. Nicholas' (\citealt[45]{Büchli1966}), \textit{ajn stiva} `in the living room' (\sectref{sec:8.17}), \textit{ajn stizún} `in the shop' (\citep[123]{Büchli1966}, \textit{ajn tégja} `in the alpine hut' \citealt[122]{Büchli1966}), \textit{avaun tégja} `in front of the alpine hut' (\sectref{sec:8.17}).

Simple prepositions cannot stand alone, i.e. they cannot function as adverbs. In order to do so, they need a derivational morpheme, which is \textit{an}- in case of \textit{gjù} `down' (\ref{ex:angju}) and \textit{sé} `up'; and \textit{vid}- with \textit{ajn} `in(to)'  (\ref{ex:vidajn}) and \textit{ò/òr/òra} `out' (\ref{ex:vidora}). Another possibility to convert \textit{sé} into an adverb is to use \textit{òra} as in \textit{sédòra} `up' (\ref{ex:sedora}).\footnote{Note that in \textit{sédòra}, \textit{òra} does not mean `out of the valley', see  \tabref{loc5}.} The adverbial equivalent of \textit{spèr} `next to' is \textit{daspèras} (\ref{ex:dasperas}).

\ea
\label{ex:angju}
\gll A … la duméngja … alṣ ùmanṣ èn i \textbf{angjù̱} [...].\\
and {} \textsc{def.f.sg} Sunday {} \textsc{def.m.sg} man.\textsc{pl} be.\textsc{prs.3pl} go.\textsc{ptcp.m.pl} down\\
\glt `And ... on Sunday ... the men went down [...].' (Cavòrgja, m7, \sectref{sec:8.17})
\z

\ea
\label{ex:vidajn}
\gll A quèla tauna vò \textbf{vidajn} – quaj tgu sùn stauṣ ajn – vò lò \textbf{vidajn} circa véntgatschún mè̱tars, san\footnotemark {} ins í \textbf{vidajn} da quèla, api sasparti, vòi ajn duas.\\
and \textsc{dem.f.sg} cave go.\textsc{prs.3sg} into {}  \textsc{dem.unm} \textsc{rel.1sg} be.\textsc{prs.1sg} \textsc{cop.ptcp.m.sg} in {} go.\textsc{prs.3sg} there into about twenty-five metre.\textsc{m.pl} can.\textsc{prs.3sg} \textsc{gnr} go.\textsc{inf} into of \textsc{dem.f.sg} and \textsc{refl}.divide.\textsc{prs.3sg.expl} go.\textsc{prs.3sg.expl} in two.\textsc{f}  \\
\glt `And this cave – [judging from] where I have been in it – one can go into it about 25 metres, and then it splits in two.' (Sadrún, m4, \sectref{sec:8.3})\footnotetext{\ili{Standard Sursilvan} for \textit{sò inṣ dí}.}
\z


\ea
\label{ex:vidora}
\gll  A la sèra par tga briṣchi bétg … vagnéva quaj, quaj mava `l ajnagjù cul maun èra sènza … [vòns] a trèva \textbf{vidò̱} còtgla giù sé sél plantschju.  \\
and \textsc{def.f.sg} evening for \textsc{subord} burn.\textsc{prs.sbjv.3sg} \textsc{neg} {} \textsc{pass.aux.impf.3sg} \textsc{dem.unm} \textsc{dem.unm} go.\textsc{impf.3sg} \textsc{3sg.m} in\_down with.\textsc{def.m.sg} hand also without {} [glove.\textsc{m.pl}] and pull.\textsc{impf.3sg} out charcoal.\textsc{coll} down up on.\textsc{def.m.sg} floor  \\
\glt `And in the evening, to avoid it burning … was that, there he went into [the fire] with one hand, also without [gloves], and pulled out charcoal from down there up to the floor.' (Sadrún, m4, \sectref{sec:8.3})
\z

\ea
\label{ex:sedora}
\gll  [...] cun agid da … da la vischnaunca va ju … è aun … stju métar … nùvas pétgas … bétònaj,\footnotemark {} né, \textbf{sédòra} séssum als cuélms [...].\\
{} with help.\textsc{m.sg} of {} of  \textsc{def.f.sg} municipality have.\textsc{prs.1sg} \textsc{1sg} {} also moreover {} must.\textsc{ptcp.unm} put.\textsc{inf} {} new.\textsc{f.pl}  post.\textsc{pl} {} concrete.\textsc{ptcp.m.pl} right up\_out  on\_top \textsc{def.m.pl} peak.\textsc{pl}\\
\glt `[...] with the help of … of the municipality I had … also … had to put … new concreted ... posts, right, up there on top of the peaks [...].'\footnotetext{\textit{Bétònaj} is a performance error for \textit{bétònádas}.} (Sadrún, f3, \sectref{sec:8.1})
\z

\ea
\label{ex:dasperas}
\gll Ṣùtajn èri la tégja nùca `l caṣchav' èra, \textbf{daspèras} quèls dus nuégls [...].\\
under\_in \textsc{cop.impf.3sg.expl} \textsc{def.f.sg} alpine\_hut \textsc{rel} \textsc{3sg.m} make\_cheese.\textsc{impf.3sg} also next \textsc{dem.m.pl} two.\textsc{m} cow\_barn.\textsc{pl} \\
\glt `Below was the alpine hut where he would also make cheese, next to it those two cow barns [...].' (Sadrún, m4, \sectref{sec:8.3})
\z

However, \textit{sé} and \textit{gjù} may stand alone in \textit{sé a gjù} `up and down' (\ref{ex:seagju}), which parallels German \textit{auf und ab}.\footnote{In German, the prepositions \textit{auf} and \textit{ab} cannot stand alone in the meaning intend here; they must be combined with e.g. \textit{hin--}. \textit{Ich gehe *auf}  in the sense of `I am going up.' is not grammatical; one must say\textit{ Ich gehe \textbf{hin}auf}, as in Tuatschin\textit{ Ju mòn *sé vs Ju mòn \textbf{an}sé}.}

\ea
\label{ex:seagju}
\gll    A nus mavan culs pòrs \textbf{sé} \textbf{Valtgèva}, mintga dé \textbf{sé} a \textbf{gjù} [...].\\
and\textbf{} \textsc{1pl}  go.\textsc{impf.1pl} with.\textsc{def.m.pl} pig.\textsc{pl} up \textsc{pn} every day.\textsc{m.sg} up and down \\
\glt `And we would go up to Valtgeva with the pigs, every day up and down [...].' (Sadrún, m6, \sectref{sec:8.11})
\z

The preposition \textit{ancùntar} `towards' constitutes a special case in the sense that it triggers dative with \isi{human nouns} (\ref{ancuntar1} and \ref{ancuntar2})  but not with \isi{inanimate nouns} (\ref{ancuntar3} and \ref{ancuntar4}).

\ea
\label{ancuntar1}
\gll Cò ṣaj	vagnú ina fèmna \textbf{ancùntar} \textbf{li}	\textbf{quaj} \textbf{pur} […].\\
here be.\textsc{prs.3sg} come.\textsc{ptcp.unm} \textsc{indef.f.sg} woman towards \textsc{def.dat.sg} \textsc{dem.m.sg} farmer \\
\glt `At this moment a woman came towards this farmer […].' (Ruèras, \citealt[64]{Büchli1966})
\z

\ea
\label{ancuntar2}
\gll Èl è juṣ \textbf{ancùntar} \textbf{da} \textbf{la} \textbf{mùma}.\\
\textsc{3sg.m} be.\textsc{prs.3sg} go.\textsc{ptcp.m.sg} towards \textsc{dat} \textsc{def.f.sg} mother\\
\glt `He went towards his mother.' (Sadrún, m5)
\z

\ea
\label{ancuntar3}
\gll A lu ṣgulavan las còcas ò da la cazèta \textbf{ancunt}' \textbf{al} \textbf{tgamín}.\\
and then fly.\textsc{impf.3pl} \textsc{def.f.pl} small\_cake.\textsc{pl} out of \textsc{def.f.sg} pan towards \textsc{def.m.sg} chimney\\
\glt `And then the small cakes flew out of the pan towards the chimney.' (Sèlva, \citealt[26]{Büchli1966})
\z

\ea
\label{ancuntar4}
\gll    Ad ajnaquèla … ṣaj sadèrs … ina grònda lavina gjù da la val Lòndadusa gjù \textbf{ancùntar} \textbf{al} \textbf{vitg} [...].\\
and at\_that\_moment {} be.\textsc{prs.3sg.expl} \textsc{refl.}fall.\textsc{ptcp.unm} {} \textsc{indef.f.sg} huge avalanche down of \textsc{def.f.sg} valley \textsc{pn} down towards \textsc{def.m.sg} village\\
\glt `And precisely at that moment … a huge avalanche … came down from the Londadusa valley, down towards the village [...].' (Sadrún, m6, \sectref{sec:8.5})
\z

This is the only case where \isi{animacy} (human vs inanimate) plays a role in Tuatschin.

\textit{Via} instead of \textit{vi} `over' is rejected by some informants; it occurs, however, in \citet{Büchli1966} as well as in the oral corpus.

\ea
\label{ex:via1}
\gll A lu ségi la cúa ida \textbf{via} da Sòntg Antòni gjù a vagi suatju als zagríndars ò ṣùt Bugnaj.\\
and then be.\textsc{prs.sbjv.3sg} \textsc{def.f.sg} tail go.\textsc{ptcp.f.sg} over from holy.\textsc{m.sg} \textsc{pn} down and have.\textsc{prs.sbjv.3sg} catch\_up.\textsc{ptcp.unm} \textsc{def.m.pl} Yenish.\textsc{pl} out under \textsc{pn}\\
\glt `And then the tail went down by [the chapel of] Saint Anthony and caught up the Yenish beneath Bugnei.' (Bugnaj, \citealt[132]{Büchli1966})
\z

\ea
\label{ex:via2}
\gll I èr' in artg \textbf{ṣul} Rajn \textbf{via}.\\
\textsc{expl} \textsc{exist.impf.3sg} \textsc{indef.m.sg} rainbow on.\textsc{def.m.sg} \textsc{pn} over\\
\glt `There was a rainbow over the Rhine.' (Tschamùt, \citealt[15]{Büchli1966})
\z

\ea
\label{ex:via3}
\gll  [...] api mir’ al bab \textbf{via} sén mè:\\
{} and look.\textsc{prs.3sg} \textsc{def.m.sg} father over on \textsc{1sg} \\
\glt `[...] and then my father looks over to me:' (Cavòrgja, m7, \sectref{sec:8.17})
\z

Complex prepositions include \textit{damanajval da} `near'(\ref{damanajvalda1}), \textit{navèn da} `from' (\ref{navenda1}), \textit{ò da/òrd} `out of' (\ref{navenda1} and \ref{ord1}), \textit{òn} (< \textit{òra ajn} `out (in)to') (\ref{on1}), and \textit{vi da} `(over) to' (\ref{vida1}).

\ea
\label{damanajvalda1}
\gll  Immis, quaj è gjù, gjù tschò \textbf{damanajval} \textbf{da} … \textbf{Interlaken}.\\
\textsc{pn} \textsc{dem.unm} \textsc{cop.prs.3sg} down down there near of {} \textsc{pn} \\
\glt `Immis, that is down, down there, near … Interlaken.' (Ruèras, m10, \sectref{sec:8.7})
\z

\ea
\label{navenda1}
\gll  Tùts … \textbf{ò} \textbf{da} \textbf{scùla} a fumégl … a sjantar ad alp a \textbf{navèn} \textbf{dad} \textbf{alp} vagnévas pér al davùs mumèn a mavaṣ a scùla. \\
all.\textsc{m.pl} {} out of school.\textsc{f.sg} to farmhand.\textsc{m.sg} {} and after to alp.\textsc{f.sg} and away from alp come.\textsc{impf.2sg.gnr} only \textsc{def.m.sg} last moment and go.\textsc{impf.2sg.gnr} to school.\textsc{f.sg}\\
\glt `All ... out of school to farmhand ... and after this to the alpine pasture and you would only come away from the pasture at the last moment and then you would go to school.' (Ruèras, m3, \sectref{sec:8.16})
\z

\ea
\label{on1}
\gll    [...] api ṣè …  ina da nòssa tgòmbra id’ \textbf{òn} \textbf{tualèta} [...].\\
{} and \textsc{cop.prs.3sg} {} one.\textsc{f} of \textsc{poss.1pl.f.sg} room go.\textsc{ptcp.f.sg} out\_in toilet\\
\glt `[...] and then … one of our room went out to the toilet [...].' (Camischùlas, f6, \sectref{sec:8.4})
\z

\ea
\label{ord1}
\gll Las quátar \textbf{òrd} \textbf{létg} ad í a rimná las vacas ajn stával [...].\\
\textsc{def.f.pl} four out\_of bed.\textsc{m.sg} and go.\textsc{inf} \textsc{subord} collect.\textsc{inf} \textsc{def.f.pl} cow.\textsc{pl} in cowshed.\textsc{m.sg}\\
\glt `At four o'clock out of bed to go and gather the cows in the cowshed [...].' (Ruèras, m3, \sectref{sec:8.16})
\z

\ea
\label{vida1}
\gll Surajn è \textbf{vi} \textbf{da} \textbf{tschèla} \textbf{vart} dal Rajn.\\
\textsc{pn} \textsc{cop.prs.3sg} over of \textsc{dem.f.sg} side of.\textsc{def.m.sg} \textsc{pn} \\
\glt `Surrein is on the other side of the Rhine.' (Camischùlas, f6)
\z

The derived adverbs with {\textit{-dém}} (\ref{ajntadem1}  and \ref{odem1}) or \textit{-sum} (\ref{sessum1} and \ref{sessum2}) are also used as prepositions.

\ea
\label{ajntadem1}
\gll    Quaj èra ju gjù ina grònda lavina … a vèva … déstruí ina grònda part dl vitg \textbf{ajntadém} \textbf{Ruèras} [...].\\
\textsc{dem.unm} be.\textsc{impf.3sg} go.\textsc{ptcp.unm} down \textsc{indef.f.sg} big avalanche {} and have.\textsc{impf.3sg} {} destroy.\textsc{ptcp.unm} \textsc{indef.f.sg} huge part of.\textsc{def.m.sg} village uppermost \textsc{pn}\\
\glt `Then a huge avalanche went down … and  … destroyed a big part of the village in the upper part of Rueras [...].' (Camischùlas, m6, \sectref{sec:8.5})
\z

\ea
\label{odem1}
\gll A lu ṣchajnṣ adina al ... sòntgèt \textbf{òdém} \textbf{al} \textbf{vitg}.\\
and then say.\textsc{prs.1pl.1pl} always \textsc{def.m.sg} {} little\_chapel out\_most \textsc{def.m.sg} village\\
\glt `And then we always say the ... little chapel at the lowest part of the village.' (Sadrún, m5, \sectref{sec:8.8})
\z

\ea
\label{sessum1}
\gll    [...] a lu auda `l las stréjas sé cò, \textbf{séssum} \textbf{la} \textbf{val} \textbf{da} \textbf{Lòndadusa} òni clumau:\\
{} and then hear.\textsc{prs.3sg} \textsc{3sg.m} \textsc{def.f.pl} witch.\textsc{pl} up here uppermost \textsc{def.f.sg} valley of \textsc{pn} have.\textsc{prs.3pl.3pl} call.\textsc{ptcp.unm}\\
\glt `[...] and then he hears the witches up there, they called from the uppermost part of the Londadusa valley:' (Sadrún, m6, \sectref{sec:8.5})
\z

\ea
\label{sessum2}
\gll Quaj è ina, asch’ ina stazjun amiaz al pas circa né … strusch \textbf{séssum} \textbf{al} \textbf{pʰas}.   \\
\textsc{dem.unm} \textsc{cop.prs.3sg} \textsc{indef.f.sg} such \textsc{indef.f.sg} station amid \textsc{def.m.sg} pass around or {} almost on\_top \textsc{def.m.sg} pass\\
\glt `This is a, such a station in the middle [of the road to] the pass, approximately, or … almost on top of the pass.' (Ruèras, m10, \sectref{sec:8.7})
\z

Circumpositions are e.g. \textit{da {N} ajn} `through N into' (\ref{ex:da.path1}), \textit{da {N} ajnasé} `through N up' (\ref{ex:da.path2}),\textit{da {N} òra} `from N out' (\ref{ex:da.source1}), \textit{da {N} sé} `from N up', \textit{da {N} sédòra} `from N up' (\ref{ex:dasedora1}),  `through N out', \textit{par {N} antùrn} `around N', \textit{spèr {N} vi} `next to N over' (\ref{ex:spervi1}), \textit{ṣùt {N} ajn} `under N in(to)', and \textit{ṣùt \textsc{N} gjù} `under N down'.

In circumpositions, the preposed element \textit{da} often refers to \textsc{source} (`from') (\ref{ex:da.source1}) or to \textsc{path} (`through') (\ref{ex:da.path1}).

\ea
\label{ex:da.path1}
\gll […] a mava \textbf{da} la pòrta a \textbf{da} las rémas \textbf{ajn} ajn clavau.\\
{} and go.\textsc{impf.3sg} from \textsc{def.f.sg} door and from \textsc{def.f.pl} crack.\textsc{pl} in in barn \\
\glt `[…] and [the hay] came into the barn through the door and the cracks.' (Cavòrgja, \citealt[121]{Büchli1966})
\z

\ea
\label{ex:da.path2}
\gll  [...] sau bétg c’ ju sùn ida \textbf{da} la val Strém \textbf{ajnasé} [...].\\
{} know.\textsc{prs.1sg.1sg} \textsc{neg} when \textsc{1sg} be.\textsc{prs.1sg} go.\textsc{ptcp.f.sg} from \textsc{def.f.sg} valley \textsc{pn} in\_up\\
\glt `[...] I don’t know when I went up the Strem valley [...].' (Sadrún, f3, \sectref{sec:8.1})
\z

\ea
\label{ex:da.source1}
\gll Cu i òn purtau la bara ò da tgèsa, mirav’ èl \textbf{da} \textbf{fanèstr}’ \textbf{òra}.\\
when \textsc{3pl} have.\textsc{prs.3pl} carry.\textsc{ptcp.unm} \textsc{def.f.sg} coffin out of house.\textsc{f.sg}
look.\textsc{impf.3sg} \textsc{3sg.m} from window out\\
\glt `When they carried the coffin out of the house, he was looking out the window.' (Cavòrgja, \citealt[123]{Büchli1966})
\z

\ea
\label{ex:dasedora1}
\gll  Avaun, navèn da Realp essan nuṣ i dad ina … \textbf{dad} in … trùtg \textbf{sédòra} tòcan sésúr las lavinèras [...]. \\
before away from \textsc{pn} be.\textsc{prs.1pl} \textsc{1pl} go.\textsc{ptcp.m.pl} from \textsc{indef.f.sg} {} from \textsc{indef.m.sg} {} footpath up until up\_over \textsc{def.f.pl} avalanche\_barrier.\textsc{pl} \\
\glt `Before, from Realp we went on a footpath [that led us] above the avalanche barriers [...].' (Ruèras, m10, \sectref{sec:8.7})
\z

\ea
\label{ex:spervi1}
\gll Cu ‘l è vagnúṣ a tgèsa la sèra \textbf{spèr} ina gronda prajt-crap \textbf{vi}, ò `l schau dá la sagir \textbf{ṣu} la prajt-crap \textbf{gjù}~[…].\\
when \textsc{3sg.m} be.\textsc{prs.3sg} come.\textsc{ptcp.m.sg} to house.\textsc{f} \textsc{def.f.sg} afternoon next\_to \textsc{indef.f.sg} big rock\_face over have.\textsc{prs.3sg} \textsc{3sg.m} let.\textsc{ptcp.unm} give.\textsc{inf} \textsc{def.f.sg} saw under \textsc{def.f.sg} rock\_face down\\
\glt `When in the evening he came back home, passing a huge rock face, he let the saw fall down under the rock face […].' (Bugnaj, \citealt[135]{Büchli1966})
\z

One consultant uses the verb \textit{í} `go' without a preposition in the case of \textit{majṣès} (\ref{ex:irsenzaprep1} and \ref{ex:irsenzaprep2}). The others use either \textit{a} `to' or \textit{sé} `up' in such cases (\ref{ex:ircunprep1}).

\ea
\label{ex:irsenzaprep1}
\gll  [...] ábar avaun c’ al tjams dad alp èri lu aun dad \textbf{í} \textbf{majṣès} … culs tiars.  \\
{} but before \textsc{subord} \textsc{def.m.sg} time of alp be.\textsc{impf.3sg.expl} then in\_addition to go.\textsc{inf} assembly\_of\_houses {} with.\textsc{def.m.pl} animal.\textsc{pl}\\
\glt `[...] but before going to the summer pastures one had to go to the \textit{majṣès} with the animals.' (Cavòrgja, m7, \sectref{sec:8.17})
\z

\ea
\label{ex:irsenzaprep2}
\gll Scadín cas, quaj èra schòn … ah … in désidé̱ri, savaj \textbf{í} \textbf{majṣès} a durmí sé lò.\\
each.\textsc{m.sg} case dem.\textsc{unm} \textsc{cop.impf.3sg} indeed {} eh {} indef.\textsc{m.sg} longing can.\textsc{inf} go.\textsc{inf} assembly\_of\_houses and sleep.\textsc{inf} up there\\
\glt `In any case, this was indeed ... eh ... a longing, to be able to go to the \textit{majṣès} and sleep up there.' (Cavòrgja, m7, \sectref{sec:8.17})
\z

\ea
\label{ex:ircunprep1}
\gll A las fèmnas stèv’ ins aun gidá a tgèsa a cu `ls ùmans mavan lu a … mavan aj lu \textbf{a} \textbf{majṣès} [...].\\
and def.\textsc{f.pl} woman.\textsc{pl} must.\textsc{impf.3sg} \textsc{gnr} moreover help.\textsc{inf} at home.\textsc{f.sg} and when \textsc{def.m.pl} man.\textsc{pl} go.i\textsc{mpf.3pl} then and {} go.\textsc{impf.3pl} \textsc{3pl} then to assembly\_of\_houses.\textsc{m.sg}\\
\glt `And the women, one had to help at home and when the men would then go and ... would then go up to the \textit{majṣès} [...].' (Ruèras, f4, \sectref{sec:8.16})
\z

In Tuatschin -- as well as in other Romansh varieties -- \isi{locative adjuncts} referring to place names, and and other places are very important.\footnote{See \citep[4--126]{Ebneter1994} for \ili{Standard Sursilvan} and other Romansh varieties.} When  speakers are located in the Tujetsch valley, they must indicate whether they go down the valley (\textit{ò} `out'), up the valley (\textit{ajnta} `into'), outside the valley (\textit{ò} `out' or \textit{gjù} `down'), or over to a place (\textit{vi}), usually seen from the speech act place. If sthey are outside the valley and go into the valley, they must decide whether they use \textit{sé} (`up') or \textit{ajn} (`into').\footnote{The exact location of the place names mentioned in this section can be seen in \figref{fig:uppervalley} and \figref{fig:lowervalley}, \sectref{sec:1.3}.}

\tabref{loc1} shows the prepositions used when the speaker is located or moves within the Lower Valley (which starts in Bugnei and ends in Dieni), and \tabref{loc2} presents the prepositions used when going to the other side of the Rhine or to the Medel Valley.

\begin{table}
	\caption{Locatives I}
	\label{loc1}
	\begin{tabular}{lllllll}
		\lsptoprule
		& & Bugnaj & Sadrún & Camischùlas & Ruèras & Diani \\
		\midrule
		Bugnaj  & →& \longrule  & \textit{ajnta} & \textit{ajnta} & \textit{ajnt} & \textit{ajnta}\\
		Sadrún & → & \textit{ò}  & \longrule & \textit{ajnta} & \textit{ajnt} & \textit{ajnta}\\
		Camischùlas & →& \textit{ò} & \textit{ò} & \longrule & \textit{ajnt} & \textit{ajnta}\\
		Ruèras & →& \textit{ò} & \textit{ò} & \textit{ò} & \longrule & \textit{vi}\\
		Diani & →& \textit{ò} & \textit{ò} & \textit{ò} & \textit{ò} & \longrule\\
		\lspbottomrule
	\end{tabular}
\end{table}

\begin{table}
	\caption{Locatives II}
	\label{loc2}
	\begin{tabular}{lllllll}
		\lsptoprule
		& & Sadrún & Surajn & Cavòrgja & Méjdal & Curaglja\\ 
		\midrule
		Sadrún  &    →& \longrule & \textit{gjù/vi}  & \textit{gjùn}    & \textit{vin} & \textit{vi}\\
		Surajn  &   → &  \textit{sé/vi} & \longrule & \textit{gjùn}    & \textit{vin} & \textit{vi}   \\
		\lspbottomrule
	\end{tabular}
\end{table}

With \textit{Cavòrgja}, the preposition \textit{gjùn} is used, a combination of \textit{gjù} `down' and \textit{ajn} `in(to)',  because \textit{cavòrgja} means `canyon', hence `down' and `into'. Accordingly, \textit{vin}, a combination of \textit{vi} `over' and \textit{ajn} `in(to)', is used with \textit{Val Mejdal}, because \textit{Val Méjdal} is a valley, and in order to go there, one has to cross a pass, hence \textit{vi}, and go into the valley, hence \textit{ajn}.

\tabref{loc3} shows the prepositions used when going to or coming from the Upper Valley and the bordering canton of Uri with two villages with Romansh names, namely \textit{Ursèra} (\textit{Andermatt} in German) and \textit{Caṣchinùta} (\textit{Göschenen} in German).

\begin{table}
	\caption{Locatives III}
	\label{loc3}
	\begin{tabular}{lllllll}
		\lsptoprule
		& & Sèlva & Tschamùt & Ursèra & Caṣchinùta & Uri\\ 
		\midrule
		Sadrún  & → & \textit{ajnta} & \textit{sé/sén} & \textit{vid} & \textit{vi} & \textit{vi gl}\\
		Surajn  & → & \textit{sé} & \textit{sén}  & \textit{vid} & \textit{gjù} & \textit{vi gl}\\
		\lspbottomrule
	\end{tabular}
\end{table}

\begin{table}
	\caption{Locatives IV}
	\label{loc4}
	\begin{tabular}{lllllll}
		\lsptoprule
		& & Tschamùt & Sèlva & Ruèras & Sadrún & Bugnaj\\
		\midrule
		Tschamùt  & → & \longrule  & \textit{gjù}  & \textit{òragjù}     & \textit{òragjù}  & \textit{òragjù}\\
		Sèlva  & → & \textit{sén} & \longrule & \textit{òragjù}  & \textit{òragjù}   & \textit{òragjù} \\
		\lspbottomrule
	\end{tabular}
\end{table}

\tabref{loc4} shows the prepositions used inside the Upper Valley and going from the Upper Valley into the Lower valley.


\begin{table}
	\caption{locatives V}
	\label{loc5}
	\fittable{
	\begin{tabular}{lllllll}
		\lsptoprule
		&       & Mumpé Tujètsch & Ségnas & Mustajr & Surajn/Sumvitg & Trùn\\ 
		\midrule
		Sadrún  & →  & \textit{ò} & \textit{ò}  &    \textit{ò} & \textit{gjù} & \textit{gjù}\\
		Sadrún & ← & \textit{ajnta} & \textit{ajnta} & \textit{ajnta} & \textit{sé}  & \textit{sé} \\
		\lspbottomrule
	\end{tabular}
	}
\end{table}

\tabref{loc5} shows the prepositions used when going from Sedrun outside the Tujetsch valley or when going from outside into the valley to Sedrun. The first village outside the Tujetsch valley is Mompé Tujetsch, and the last considered here is Trun, which is still located in the Surselva. The villages until Mustér  are treated as if they still belonged to the Tujetsch valley (\ref{ex:omustajr1}).

\ea
\label{ex:omustajr1}
\gll Lu ṣèn quèls da Sadrún i \textbf{da} \textbf{las} \textbf{Cavòrgjas} \textbf{òra} a staj \textbf{ò} \textbf{Mustajr} avaun\_ca `ls zagríndars.\\
then be.\textsc{prs.3pl} \textsc{dem.m.pl} of \textsc{pn} go.\textsc{ptcp.m.pl} from \textsc{def.f.pl} \textsc{pn} out and \textsc{cop.ptcp.m.pl} out \textsc{pn} before {} \textsc{def.m.pl} Yenish.\textsc{pl}\\
\glt `Then the people of Sedrun passed  Cavorgia and were in Mustér before the Yenish.' (Bugnaj, \citealt[132]{Büchli1966})
\z

The combinations of \textit{cò} `here' with \textit{òra} and \textit{ajn} (realised as \textit{quòra} and \textit{quajn}, respectively) have both two meanings: either `here (down or up the valley)' or `outside (in direction down or up the valley)' (\ref{ex:quora1}--\ref{ex:quora3}).

\ea
\label{ex:quora1}
\gll \textbf{Quòra} ò Mustajr ṣaj bi.\\
here\_out out \textsc{pn} \textsc{cop.prs.3sg.expl} beautiful.\textsc{adj.unm}\\
\glt `Here in Mustér (down the valley) it is nice.' (Sadrún, m5)
\z

\ea
\label{ex:quora2}
\gll Quaj è \textbf{quòra}, òn Cavòrgja.\\
\textsc{dem} \textsc{cop.prs.3sg} here\_out, out\_in \textsc{pn}\\
\glt `This is here (in direction down the valley), in Cavorgia.' (Sadrún, m5)
\z

\ea
\label{ex:quora3}
\gll Ju spétga té \textbf{quòra}.\\
\textsc{1sg} wait.\textsc{prs.1sg} \textsc{2sg} here\_out\\
\glt `I'm waiting for you outside (the house) (in direction down the valley).' (Cavòrgja, f1)
\z

The villages and cities from Trun downwards are all modified by the preposition \textit{gjù} if they are relevant to the speakers, as are Chur or Zürich (\ref{ex:gjuturitg1}). If they are not relevant to them, the preposition \textit{a} `to, in' is used instead.

\ea
\label{ex:gjuturitg1}
\gll  Ju vòn \textbf{gjù} \textbf{Turitg}.\footnotemark\\
\textsc{1sg} go.\textsc{prs.1sg} down \textsc{pn}\\
\glt `I am going (down) to Zurich.' \footnotetext{The pan-Romansh form \textit{Turitg}, written \textit{Turich} in the Ladin varieties, derives either from its Swiss German form \textit{Züri} or, more probably, from its Latin form \textit{Turicum}.} (Cavòrgja, f1)
\z

Regarding the lateral valleys of the Tujetsch valley, most speakers use \textit{ajn} `in(to)' (\ref{ex:ajnval1}), but some use \textit{sé} `up' (\ref{ex:seval1}).

\ea
\label{ex:ajnval1}
\gll Ju mòn \textbf{ajn} Val Gjuf, \textbf{ajn} Val Val.\\
\textsc{1sg} go.\textsc{prs.1sg} into valley \textsc{pn} into valley \textsc{pn}\\
\glt `I go into the Gjuv valley, into the Val valley.' (Sadrún, m5)
\z

\ea
\label{ex:seval1}
\gll Ju mòn \textbf{sé} Val Gjuv, \textbf{sé} Val Val.\\
\textsc{1sg} go.\textsc{prs.1sg} up valley \textsc{pn} up valley \textsc{pn}\\
\glt `I go into the Gjuv valley, into the Val valley.' (Sadrún, m6)
\z

When coming from the lateral valleys, \textit{ajnagjù} or \textit{òragjù} is used (\ref{ex:ajnase1}).

\ea
\label{ex:ajnase1}
\gll [...] in … tga stèva ajnasé Gjuf ábar vaj gju da vagní \textbf{navèn} \textbf{da} \textbf{Gjuf} \textbf{tòcan} \textbf{òragjù} … \textbf{Zarcúns}  [...].\\
{} one.\textsc{m} {} \textsc{rel} live.\textsc{impf.3sg} in\_and\_up \textsc{pn} but  have.\textsc{sbjv.prs.3sg} have.\textsc{ptcp.unm} to come.\textsc{inf} from of \textsc{pn} until out\_down {} \textsc{pn} \\
\glt `[...] one … that lived up in Gjuf, but who had to come down from Gjuf to … Zarcuns [...].' (Zarcúns, m2, \sectref{sec:8.13})
\z

\textit{Dadajns} and \textit{dadò̱} in combination with a \isi{place name} or another reference point (e.g. a church, a school, or a bridge) mean that the \isi{subject} of the sentence is located outside the place or away from the reference point, in the direction up or down the valley (\ref{ex:dad1}--\ref{ex:dad6}).

\ea
\label{ex:dad1}
\gll Ju sùn \textbf{dadajns} \textbf{Ségnas}.\\
\textsc{1sg} \textsc{cop.prs.1sg} from\_in \textsc{pn}\\
\glt `I am outside Segnas, in the direction of the Tujetsch valley.' (Sadrún, m5)
\z

\ea
\label{ex:dad2}
\gll Ju sùn \textbf{dadò̱} \textbf{Ségnas}.\\
\textsc{1sg} \textsc{cop.prs.1sg} from\_out \textsc{pn}\\
\glt `I am outside Segnas, in the direction of Mustér.' (Sadrún, m5)
\z

\ea
\label{ex:dad3}
\gll Nus vèvan nòssa scùla ajn Sùtcrè̱stas, qu’ è \textbf{dadajns} \textbf{Sèlva} [...]. \\
\textsc{1pl} have.\textsc{impf.1pl} \textsc{poss.1pl.f.sg} school in \textsc{pn} \textsc{dem.unm} \textsc{cop.prs.3sg} from\_in \textsc{pn}\\
\glt `We had our school in Sutcrestas, this is outside Selva [in the direction up the valley] [...].' (Sèlva, f2, \sectref{sec:8.6})
\z

\ea
\label{ex:dad4}
\gll [...] Nacla, quaj é ... \textbf{dadajns} ... \textbf{Surajn}, fòrza végn minutas vidajn.\\
{} \textsc{pn} \textsc{dem.unm} \textsc{cop.prs.3sg} {} more\_back {} \textsc{pn} maybe twenty minute.\textsc{f.pl} into\\
\glt `[...] Nacla, this is … farther behind … Surrein, maybe twenty minutes farther behind.' (Sadrún, m4, l\sectref{sec:8.3})
\z

\ea
\label{ex:dad5}
\gll \ob{m3}\cb {} Sas tgé quaj vut dí? Quaj è ljung, quaj è sé Miléz, Ṣcharinas. \ob{f4}\cb {} Sé Miléz, \textbf{dadajns} \textbf{Miléz} ... . \\ 
{} know.\textsc{prs.2sg} what \textsc{dem.unm} want.\textsc{prs.3sg} say.\textsc{inf} \textsc{dem.unm} \textsc{cop.prs.3sg} long.\textsc{adj.unm} \textsc{dem.unm} \textsc{cop.prs.3sg} up \textsc{pn} \textsc{pn} {} up \textsc{pn} in \textsc{pn}\\
\glt  `\texttt{[m3]} Do you know what this means? This is a long way, this is up at Milez, Scharinas. \texttt{[f4]} Up at Milez, to the west of Milez ... .' (Ruèras, m3 and f4, l\sectref{sec:8.16})
\z

\ea
\label{ex:dad6}
\gll Lu ò ‘l vju \textbf{òragjù} \textbf{dadòr} \textbf{Camischùlas} sén in prau ina familja tga sùlvèva.\\
then have.\textsc{prs.3sg} \textsc{3sg.m} see.\textsc{ptcp.unm} out\_down outside \textsc{pn} on \textsc{indef.m.sg} field \textsc{indef.f.sg} family \textsc{rel} have\_breakfast.\textsc{impf.3sg}\\
\glt `Then he saw down there, outside Camischolas [in the direction down the valley], a family that was having breakfast in a field.' (Bugnaj, \citealt[139]{Büchli1966})
\z

As mentioned, the reference point does not have to be a village or a town; in (\ref{ex:dadajnspun}) it is the bridge over the Drun river. In order to explain to me the meaning of \textit{dadajns}, my consultant told me when we were in the Krüzli hotel:

\ea
\label{ex:dadajnspun}
\gll Nuṣ duṣ èssan \textbf{dadajns} \textbf{la} \textbf{pùn}.\\
\textsc{1pl} two.\textsc{m} \textsc{cop.prs.1pl} from\_in \textsc{def.f.sg} bridge\\
\glt `The two of us are away from the bridge [in the direction up the valley].' (Sadrún, m5)
\z


\tabref{loc6} shows the prepositions that are used with the neighbouring countries (Italy, Germany, Austria, France) or regions (Bavaria).

\begin{table}
	\caption{Locatives VI}
	\label{loc6}
	\begin{tabular}{lllllll}
		\lsptoprule
		& & Italja & Tjaratudèstga  & Baviara & Austrja & Fròntscha\\ 
		\midrule
		Tujétsch  &    →& \textit{gjù l'} &  \textit{ajn}   &  \textit{òn} & \textit{ò l'} & \textit{ajn}\\
		\lspbottomrule
	\end{tabular}
\end{table}

The adverbs follow the same rules as the prepositional phrases. In (\ref{ex:vidòr1}) the speech act participants are in Sedrun, and \textit{vidòr} `down the valley' is used because the hearer will go to Naclas (referred to as \textit{ajn lò} `up there') and the speaker wants the hearer to come back to Sedrun.

\ea
\label{ex:vidòr1}
\gll Té nò lu \textbf{vidòr} ùssa. Lò, quèsta sèra dòrma lu bigja \textbf{ajn} \textbf{lò}.\\
\textsc{2sg} come.\textsc{imp.2sg} then down now there  \textsc{dem.f.sg} evening sleep.\textsc{imp.2sg} then \textsc{neg} in there\\
\glt `Come down here now. Don’t sleep up there this evening.' (Ruèras, m4, \sectref{sec:8.3})
\z

Sometimes the combination of \isi{locative adverbs} does not refer to the direction up or down the valley. This is the case with \textit{sédòra} in (\ref{ex:sedora1}).

\ea
\label{ex:sedora1}
\gll    A sjantar c’ ins mava, sch’ mav’ ins sél Albṣu cul trèn, a quaj custav’ in franc dad í \textbf{sédòra}.\\
and after \textsc{subord} \textsc{gnr}  go.\textsc{impf.3sg} then go.\textsc{impf.3sg} \textsc{gnr} on.\textsc{def.m.sg} \textsc{pn} with.\textsc{def.m.sg} train and \textsc{dem.unm} cost.\textsc{impf.3sg} one.\textsc{m.sg}  franc \textsc{subord} go.\textsc{inf} up\_out \\
\glt `And after this, if one went, one would go up to the Alpsu [pass] by train and this cost one franc to go up there.' (Ruèras, m1, \sectref{sec:8.2})
\z

In this example \textit{sé} means `up'; however, \textit{òra} does not mean `down the valley' but `outside'. As a matter of fact, \textit{òra} refers to the fact that the Alpsu pass is not a village where one could be inside, but an open space. The opposite of \textit{sédòra} in such a context is \textit{sédajn} `up and into', as in (\ref{ex:sedajn1}).

\ea
\label{ex:sedajn1}
\gll Èls ajn i \textbf{sédajn} ajn tgèsa.\\
\textsc{3pl.m} be.\textsc{prs.3pl} go.\textsc{ptcp.unm} up\_into into house.\textsc{f.sg}\\
\glt `They went up into the house.' (Sadrún, m5)
\z

\subsection{Temporal adjuncts}\label{sec:4.3.2}
Temporal adjuncts consist of adverbs, \
noun phrases, prepositional phrases, and -- exceptionally -- of postpositional phrases. In this section are also included adjuncts that quantify the situation a verb refers to, like `every day', `many times', or `often'.

The \isi{calendar} is divided in the following way: \textit{tschantanè} `century', \textit{òn} `year', \textit{majns} `month', \textit{jamna} 'week', and \textit{dé} `day'.

The \isi{names of the months}, which are all masculine, are \textit{ṣchanè}, \textit{favrè}, \textit{mars}, \textit{avrél/avrèl}, \textit{matg}, \textit{zarcladur}, \textit{fanadur}\footnote{\textit{Zarcladur} and \textit{fanadur} are the only old \isi{names of the months} that still are in use; they are derived from the verbs \textit{zarclá} `weed' and \textit{faná} `hay'.}, \textit{uést} (2 syllables), \textit{satjámbar}, \textit{òctò̱bar}, \textit{nò\-vè̱m\-bar}, and \textit{dècè̱mbar}.

The \isi{names of the days} are \textit{gljéndiṣdís} (\textsc{m}), \textit{mardís} \textsc{(m}), \textit{maṣjamna} (\textsc{f}), \textit{géjvja} (\textsc{f}), \textit{vèndardís} (\textsc{m}), \textit{sònda} (\textsc{f}), and \textit{dumèngja} (\textsc{f}).

The day is divided as follows: \textit{damaun} (\textsc{f}) `morning', \textit{avaunmjadṣ-dé} \textsc{(m}) `morning' (literally `before noon'), \textit{mjadṣ-dé} (\textsc{m}) `noon', \textit{sjantarmjadṣdé} (\textsc{m}) `afternoon', \textit{sèra} (\textsc{f}) `afternoon, evening', \textit{nòtg} (\textsc{f}) `night', and \textit{mjasanòtg} (\textsc{f}) `midnight'.

Note that \textit{bian dé}, literally `good day', is used as a greeting until approximately twelve o'clock; after this, \textit{buna sèra} is used. \textit{Buna nòtg} `good night' is used when taking leave. 

When reference is being made to the day of speech, the \isi{parts of the day} are expressed as follows: \textit{òz andamaun} `this morning', \textit{òz sjantarmjadṣ-dé} `this afternoon', \textit{quèsta sèra} `this afternoon, this evening', \textit{quèsta nòtg} `tonight'.

Time adverbs include \textit{adina} `always', \textit{avaun} `before', \textit{baud} `early', \textit{daditg} `a long time ago', \textit{damaun} `tomorrow', \textit{dantaun} `meanwhile', \textit{ditg} `long, a long time', \textit{djantarájn} `in between', \textit{ér} `yesterday', \textit{grad} `just', \textit{magari} `sometimes' (\ref{ex:magari1}), \textit{maj} `never', \textit{mintgataun} `sometimes' (\ref{ex:mintgataun1}), \textit{òn} `last year', \textit{òrdavaun} `in advance', \textit{òz} `today', \textit{òzaldé} `nowadays', \textit{savèns} `often', \textit{stjarsas} `the day before yesterday', \textit{pusch\-maun} or \textit{surdamaun} `the day after tomorrow', \textit{tard} `late', \textit{uòn} `this year', \textit{ùs/ùssa} `now', \textit{vònzaj} `later' (\ref{ex:vonzaj1}), \textit{zacú/zacuras} `someday'.

\ea
\label{ex:magari1}
	\gll  Álṣò ins mava par èxè̱mpal è \textbf{magari} gjù Lucèrn, pr̩quaj tga nuṣ vèvan gjù Lucèrn còlègs tga stèvan gjù lò.\\
well \textsc{gnr} go.\textsc{impf.3sg} for example\textsc{.m.sg} also sometimes down \textsc{pn} because \textsc{subord} \textsc{1pl} have.\textsc{impf.1pl} down \textsc{pn} colleague.\textsc{m.pl} \textsc{rel} stay.\textsc{impf.3pl} down there\\
\glt `Well, we would also go for instance to Lucerne because in Lucerne we had friends who lived down there.' (Sadrún, m9, \sectref{sec:8.15})
\z

\ea
\label{ex:mintgataun1}
\gll \textbf{Mintgataun} mavan nuṣ èra … plas pitgògnas a cavá cristalas anstagl mirá dlas tgauras, pi vignévan nus halt \textbf{in} \textbf{téc} \textbf{tart}. \\
sometimes go.\textsc{impf.1pl} \textsc{1pl} also {} around.\textsc{def.f.pl}  steep\_slope.\textsc{pl} \textsc{subord} dig.\textsc{inf} crystal.\textsc{f.pl} instead look\_for.\textsc{inf} of.\textsc{def.f.pl} goat.\textsc{pl} and come.\textsc{impf.1pl} \textsc{1pl} simply \textsc{indef.m.sg} bit late \\
\glt `From time to time we would also … go farther up to extract crystals instead of looking after the goats, and then we would come back a bit late.' (Surajn, f5, \sectref{sec:8.10})
\z

\ea
\label{ex:vonzaj1}
\gll Api \textbf{vònzaj} vau tartgau, ah ju a in’ idéa [...].   \\
and later have.\textsc{prs.1sg.1sg} think.\textsc{ptcp.unm} eh \textsc{1sg} have.\textsc{prs.1sg} \textsc{indef.f.sg} idea\\
\glt `And after a while I thought, eh, I have an idea [...].' (Sadrún, m8, \sectref{sec:8.12})
\z

Time prepositions, simple or compound, include \textit{ancùntar} `towards' (\ref{ex:ancuntar1}), \textit{a\-vaun{\slash}avaun ca} `before, ago' (\ref{ex:avaunca1}), \textit{da} `at, during' (\ref{ex:daprep1}), \textit{durònt} `during', \textit{sjantar} `after' (\ref{ex:sjantar1}), and \textit{tòca/tòcan} `until' (\ref{ex:toca1}).

\ea
\label{ex:ancuntar1}
\gll Api \textbf{ancù̱ntar} \textbf{sèra}, las quátar, las tschun vagnévani sédò puspè ad usché vinavaun.\\
and towards evening.\textsc{f.sg} \textsc{def.f.pl} four \textsc{def.f.pl} five come.\textsc{impf.3pl.3pl} up\_out again and so further\\
\glt `And towards evening, at four o'clock, five o'clock, they would come up again and so on.' (Cavòrgja, m7, \sectref{sec:8.17})
\z

\ea
\label{ex:daprep1}
\gll A lu, nus cò sursilvanas, matévan adina \textbf{da} \textbf{pausa}, mavanṣ ajn ṣala da magljè [...].\\
and then \textsc{1pl} here Sursilvan.\textsc{f.pl} put.\textsc{impf.1pl} always during break.\textsc{f.sg} go.\textsc{impf.1pl.1pl} in hall.\textsc{f.sg} \textsc{comp} eat.\textsc{inf}\\
\glt `And then we, the Sursilvan students, would always place [the napkings] during the break, we would go into the dining hall [...].' (Camischùlas, f6, \sectref{sec:8.4})
\z

\ea
\label{ex:sjantar1}
\gll    A quaj è rastau da mé, a lu … \textbf{sjantar} \textbf{in} \textbf{pèr} \textbf{dis} … vajn nus savju dumigná quèls pòrs [...].\\
and \textsc{dem.unm} be.\textsc{prs.3sg} remain.\textsc{ptcp.unm}  \textsc{dat}  \textsc{1sg} and then {} after \textsc{indef.m.sg} couple  day.\textsc{m.pl} {} have.\textsc{prs.1pl} \textsc{1pl} can.\textsc{ptcp.unm} cope.\textsc{inf} \textsc{dem.m.pl} pig.\textsc{pl}\\
\glt `And I still remember this, and then … after a couple of days … we were able to cope with these pigs [...].' (Sadrún, m6, \sectref{sec:8.11})
\z

\ea
\label{ex:toca1}
\gll [...] a nuṣ vevan scùla \textbf{tòca} \textbf{gl} \textbf{avrél}.\\
{} and \textsc{1pl} have.\textsc{impf.1pl} school.\textsc{f.sg} until \textsc{def.m.sg} April\\
\glt `[...] and we had school until April.' (Ruèras, f4, \sectref{sec:8.16})
\z

\textit{Avaun ca} is very rare and only occurs in \citet{Büchli1966} (\ref{ex:avaunca1}).

\ea
\label{ex:avaunca1}
\gll [...] lu ṣèn quèls da Sadrún [...] staj ò Mustajr \textbf{avaun} \textbf{ca} `\textbf{ls} \textbf{zagríndars}.\\
{} then be.\textsc{prs.3pl} \textsc{dem.m.pl} of \textsc{pn} {} \textsc{cop.ptcp.m.pl} out \textsc{pn} before \textsc{subord} \textsc{def.m.pl} Yenish\\
\glt `[...] then those from Sedrun [...] were in Mustér before the Yenish.' (Bugnaj, \citealt[132]{Büchli1966})
\z

\isi{Postpositional phrases} are very rare in the domain of \isi{temporal adjuncts}. In the corpus, there are only two items: \textit{avaun} `before' and \textit{òra} `out'. They are used in very restricted contexts, such as \textit{al dé \textbf{avaun}} `the day before' (\sectref{sec:8.16}) or \textit{al dé \textbf{òra}} `the whole day' (\sectref{sec:8.3}, literally `the day out'). Note that the construction with \textit{avaun}, in contrast to the one with \textit{òra}, could be considered elliptic for \textit{al dé avaun quaj schabètg} `the day before that event'. According to this hypothesis, \textit{avaun} is a preposition rather than a postposition.

Noun phrases functioning as \isi{temporal adjuncts} can contain a \isi{determiner} or not. The \isi{definite article}, singular or plural, is used when the \isi{noun phrase} refers to a month, a week, or to a day of the week if the time span does not include the moment of speech (\ref{ex:tempdefart1}--\ref{ex:tempdefart3}).

\ea
\label{ex:tempdefart1}
\gll  La scùla finéva … \textbf{al} \textbf{matg} … \textbf{zarcladur} [...].  \\
\textsc{def.f.sg} school end.\textsc{impf.3sg} {} \textsc{def.m.sg} May {} June.\textsc{m.sg}\\
\glt `School ended in May ... June [...].' (Cavòrgja, m7, \sectref{sec:8.17})
\z

\ea
\label{ex:tempdefart2}
\gll    [...] tgi ca vagnéva traplaus stuèva \textbf{al} \textbf{vèndardís} \textbf{sèra} … stá lò, stgèvan bigj’ í a tgèsa, api stèvan nuṣ ṣchùbargè in’ ura zatgéj, durmí lò, api stèvan lu í pèr \textbf{la} \textbf{sònda} \textbf{andamaun} a tgèsa.\\
{} who \textsc{rel} \textsc{pass.aux.impf.3sg} catch.\textsc{ptcp.m.sg} must.\textsc{impf.3sg} \textsc{def.m.sg} Friday evening {} stay.\textsc{inf} there be\_allowed.\textsc{impf.3pl} \textsc{neg} go.\textsc{inf} to home.\textsc{f.sg} and stay.\textsc{impf.1pl} \textsc{1pl} clean.\textsc{inf} one.\textsc{f} hour something sleep.\textsc{inf} there and must.\textsc{impf.3pl} then go.\textsc{inf} only \textsc{def.f.sg} Saturday in\_morning to home.\textsc{f.sg}\\
\glt `[...] the person who got caught had to … remain there on Friday evening, they were not allowed to go home, and then they had to clean for more or less one hour, sleep there, and then could only go home on Saturday morning.' (Camischùlas, f6, \sectref{sec:8.4})
\z

\ea
\label{ex:tempdefart3}
\gll  Ajn quèla végljadé̱tgna, api al pròblèm èra, èra, al pròblèm èra \textbf{las} \textbf{sòndas} a \textbf{dumèngjas}.   \\
in  \textsc{dem.f.sg} age and \textsc{def.m.sg} problem \textsc{cop.impf.3sg} \textsc{cop.impf.3sg} \textsc{def.m.sg} problem \textsc{cop.impf.3sg} \textsc{def.f.pl} Saturday.\textsc{pl} and Sunday.\textsc{pl} \\
\glt `At that age, and the problem was, was, the problem was on Saturdays and Sundays.' (Sadrún, m4, \sectref{sec:8.3})
\z



Further examples include \textit{l'jamna vargjèda} `last week', \textit{al majns vargjau} `last month', \textit{als davùs òns} `during the last years' (\sectref{sec:8.1}) , \textit{al davùs mumèn} `at the last moment' (\sectref{sec:8.16}), \textit{la duméngja, sjantar vjaspras} `Sunday after vespers' (\sectref{sec:8.2}), \textit{la duméngja sjantarmjadṣ-dé} `Sunday afternoon' (\sectref{sec:8.2}), \textit{gl antiar sjantarmjadṣ-dé} `the whole afternoon' (\sectref{sec:8.2}).

If the moment of speech is included in the time reference, the \isi{demonstrative} \isi{determiner} of the \textit{quèst}-series is used (see \sectref{sec:3.2.2.3}) (\ref{ex:tempquest1}), except for `today', which is rendered by \textit{òz}.

\ea
\label{ex:tempquest1}
\gll Ùsa \textbf{quèst}’ \textbf{jamna} vau fatg gròndas turas … da … da sis sjat uras [...].  \\
now \textsc{dem.f.sg} week have.\textsc{prs.1sg.1sg} do.\textsc{ptcp.unm} big.\textsc{f.pl} tour.\textsc{pl} {} of {} of six seven hour.\textsc{f.pl}\\
\glt `Now this week I did long tours … of … of six seven hours [...].' (Sadrún, f3, \sectref{sec:8.1})
\z

The full hours of day are referred to by a \isi{noun phrase} with (\ref{ex:timeofday1}) or without (\ref{ex:timeofday2}) the preposition \textit{da}. The half hours are indicated without the \isi{article} (\ref{ex:timeofday1}), and the quarter hours with the \isi{indefinite article}, as in \textit{in quart avaun las trajs} `a quarter to three' and \textit{in quart vargjau las trajs} `a quarter past three'.

\ea
\label{ex:timeofday1}
\gll    [...] \textbf{da} \textbf{laṣ} \textbf{déjṣch} èri craj né \textbf{mjasa} \textbf{laṣ} \textbf{déjṣch} èri ruaus [...].\\
{} at \textsc{def.f.pl} ten \textsc{cop.impf.3sg.expl} believe.\textsc{prs.1sg} or half.\textsc{f.sg} \textsc{def.f.pl} ten \textsc{cop.impf.3sg.expl} quiet.\textsc{m.sg}\\
\glt `[...] at ten o’clock it had to be, I believe, or half past nine it had to be quiet [...].' (Camischùlas, f6, \sectref{sec:8.4})
\z

\ea
\label{ex:timeofday2}
\gll [...] par nuṣ vèvi adina nùm \textbf{laṣ} \textbf{òtg} lò.   \\
{} for \textsc{1pl} have.\textsc{impf.3sg.expl} always name.\textsc{m.sg} \textsc{def.f.pl} eight there\\
\glt `[...] for us it always meant at eight there.' (Sadrún, m9, \sectref{sec:8.15})
\z

Further \isi{temporal adjuncts} with determiners that are not articles are \textit{antiar dé} `the whole day' (\ref{ex:locwithoutdet1}), \textit{bjè jèdas} `many times' (\ref{ex:bjejedas1}), \textit{da quaj tjams} `at that time' (\sectref{sec:8.15}), \textit{mintga dé} `every day' (\sectref{sec:8.1}), and \textit{ana sjatònta} `in 1970' (\sectref{sec:8.2}).

\ea
\label{ex:locwithoutdet1}
\gll  Pr̩quaj tga quaj c’ inṣ vèva bigja grad da partgirá tiars sch’ èr’ inṣ \textbf{antiar} \textbf{dé} cun quèls [...].  \\
because \textsc{subord} \textsc{dem.unm} when \textsc{gnr} have.\textsc{impf.3sg}  \textsc{neg} just to mind.\textsc{inf} animal.\textsc{m.pl} then \textsc{cop.impf.3sg} \textsc{gnr} whole.\textsc{m.sg} day with \textsc{dem.m.pl}\\
\glt `Because when we didn’t just have to mind the animals, we were with them [the Italian workers] the whole day [...].' (Sadrún, m4, \sectref{sec:8.3})
\z

\ea
\label{ex:bjejedas1}
\gll [...] ábar inṣ èra … gè,  \textbf{bjè} \textbf{jèdas} trésts [...]\\
{} but \textsc{gnr} \textsc{cop.impf.3sg} {} yes many time.\textsc{f.pl} sad.\textsc{m.pl}\\
\glt `[...] but we were ... yes, many times sad [...].' (Ruèras, f4,\sectref{sec:8.16})
\z

When referring to the days of the week, \isi{bare noun phrases} are used (\ref{ex:dumengia1}).

\ea
\label{ex:dumengia1}
\gll Nus savasajn \textbf{gjéjvja} né \textbf{dumèngja}.\\
\textsc{1pl} \textsc{refl}.see.\textsc{prs.1pl} Thursday or Sunday\\
\glt `We'll see each other on Thursday or Sunday.' (Sadrún, m5)
\z

This is also the case when referring to years, where the dedicated nouns \textit{ana} (before consonant) or \textit{ana d'} (before vowel) (\ref{anad}) and \textit{ánò} are used (\ref{ano}). The regular \
noun for `year' is \textit{òn}.

\ea
\label{anad}
\gll Ad \textbf{ana} \textbf{d}' \textbf{òtgòntasjat} vajn nus gju ina vòtazjun fadarala [...].   \\
and year.\textsc{f.sg} of eighty-seven have.\textsc{prs.1pl} \textsc{1pl} have.\textsc{ptcp.unm} \textsc{indef.f.sg} vote federal\\
\glt `And in 1987 we had a federal vote [...].' (Sadrún, f3, \sectref{sec:8.1})
\z

\ea
\label{ano}
\gll  [...] lu ṣèn … in pèr bauns pins, gè, intarassant, gè, a méz in clutgè \textbf{ánò} \textbf{véntgò̱tg} tégn ju, gè.\\
{} then \textsc{exist.prs.3pl} {} \textsc{indef.m.sg} pair bench.\textsc{m.pl} small yes interesting.\textsc{adj.unm} yes and put.\textsc{ptcp.unm} \textsc{indef.m.sg} clock\_tower year twenty-eight hold.\textsc{prs.1sg} \textsc{1sg} yes \\
\glt `[...] then there are … some small benches, yes, interesting, yes, and also built a clock tower in 1928 I think, yes.' (Sadrún, m5, \sectref{sec:8.8})
\z

\subsection{Manner adjuncts}\label{sec:4.3.3}
Manner adjuncts are realised as adverbs (including adverbs derived from adjectives by the suffix -\textit{majn}), adjectives, and prepositional phrases.

Adverbs include \textit{bégn} `well',\footnote{In contrast to \ili{Standard Sursilvan}, Tuatschin distinguishes between \textit{bégn} `well' and \textit{bèn} `(negative) yes', which both are realised as \textit{bein} in \ili{Standard Sursilvan}. Tuatschin \textit{bégn} corresponds e.g. to  German \textit{doch} or to  French \textit{si}} \textit{mal} `badly' (\ref{ex:mal1}), \textit{nuídis} `reluctantly' (\ref{ex:nuidis1}), \textit{plaunsjú}\footnote{Forms like \textit{plaunmjú} or \textit{plauntjú}, which refer to first and second person singular, respectively, are not used in Tuatschin. \textit{Plaunsjú} is used for all persons.} `slowly', \textit{quèlui̱sa} `in such a way', \textit{tschèlui̱sa} `in that way' (\ref{ex:tscheluisa1}), \textit{ugèn} `gladly', \textit{usché/uschéja} `so', and \textit{zacù̱} `somehow' (\ref{ex:zacù1}).

\ea
\label{ex:mal1}
\gll Èl lavura \textbf{mal}.\\
\textsc{3sg.m} work.\textsc{prs.3sg} badly\\
\glt `He works badly.' (Sadrún, m4)
\z

\ea
\label{ex:nuidis1}
\gll A … la sò̱rvala dèv’ ju schòn \textbf{nuídis}.    \\
and {} \textsc{def.f.sg} cervelat give.\textsc{cond.1sg} \textsc{1sg} indeed reluctantly\\
\glt `And ... the cervelat I would only give away reluctantly.' (Cavòrgja, m7, \sectref{sec:8.17})
\z

\ea
\label{ex:tscheluisa1}
\gll  Qu' è lu ju \textbf{tschèlui̱sa} tga ca nuṣ èssan vagí vidòra, turnaj ò da Pardatsch, tg' èssan nus staj ajn lò fòrsa … quátar tschun jamnas.  \\
\textsc{dem.unm} be.\textsc{prs.3sg} then go.\textsc{ptcp.unm} such\_way \textsc{rel} \textsc{subord} \textsc{1pl} be.\textsc{impf.1pl} come.\textsc{ptcp.m.pl} over\_out return.\textsc{ptcp.m.pl} out of \textsc{pn} \textsc{corr} be.\textsc{cond.1pl} \textsc{1pl} \textsc{cop.ptcp.m.pl} in there maybe {} four five week.\textsc{f.pl}   \\
\glt `This happened in such a way that when we returned down [to Surrein] from Pardatsch, then we had stayed there maybe … four or five weeks.' (Sadrún, m4, \sectref{sec:8.3})
\z

\ea
\label{ex:zacù1}
\gll Ábar quaj è clar … òzaldé végni \textbf{zacù̱} è ...  purṣchju daplé.   \\
	but \textsc{dem.unm} \textsc{cop.prs.3sg} clear.\textsc{adj.unm} {} nowadays \textsc{pass.aux.prs.3sg.expl} somehow also ... offer.\textsc{ptcp.unm} more\\
\glt `But this is clear ... there are somehow also many more possibilities nowadays.' (Sadrún, m9, \sectref{sec:8.15})
\z

These adverbs can be modified by other adverbs or by adjectives used adverbially, as is the case of \textit{faruct} `crazy' in (\ref{ex:faruct1}).

\ea
\label{ex:faruct1}
\gll  Ábar … ju a fatg \textbf{faruct} \textbf{ugèn} quaj.\\
but {} \textsc{1sg} have.\textsc{prs.1sg} do.\textsc{ptcp.unm} crazy.\textsc{adj.unm} gladly \textsc{dem.unm} \\ 
\glt `But … I really loved to do that.' (Sadrún, f3, \sectref{sec:8.1})
\z

The \isi{comparative} of \textit{bégn} `well' is either \textit{plé bégn} (\ref{ex:plebegn1}) or \textit{mégljar}, and the \isi{comparative} of \textit{mal} `badly' is \textit{mèndar} (\ref{ex:mendar1}).

\ea
\label{ex:plebegn1}
\gll  I vò ònz \textbf{plé} \textbf{bégn}. \\
\textsc{expl} go.\textsc{prs.3sg} rather more well\\
\glt `I feel rather better.' (\DRGoK{1}{296})
\z

\ea
\label{ex:mendar1}
\gll Èl lavura \textbf{mèndar} tga té.\\
\textsc{3sg} work.\textsc{prs.3sg} worse than \textsc{2sg}\\
\glt `He works worse than you.' (Sadrún, m5)
\z

Derived adverbs with the suffix -\textit{majn} functioning as manner adverbs are relatively rare (\ref{ex:majn1} and \ref{ex:majn2}).

Adjectives are usually derived from their feminine form, as for example in \textit{bian}, \textit{buna} `good' → \textit{bunamajn} `almost', but adjectives ending in \textit{-al} or \textit{-ar} are derived from their masculine form (see \citet[494]{Spescha1989}, as in \textit{natural} `natural' → \textit{naturálmajn} `naturally, of course'.

\ea
\label{ex:majn1}
\gll Daváuntiar ṣaj sé la, l’ anada cur' i òn antschiat, ad inṣ vèz’ aun, quaj bagagjávani gjù ... cun ah, bjè \textbf{manuálmajn}.   \\
in\_front  \textsc{cop.prs.3sg} up \textsc{def.f.sg} \textsc{def.f.sg} year when \textsc{3pl} have.\textsc{prs.3pl} begin.\textsc{ptcp.unm} and \textsc{gnr}  see.\textsc{prs.3sg} still  \textsc{dem.unm} build.\textsc{impf.3pl.3pl} down {} with eh much manual.\textsc{adv}\\
\glt `In front of it there is, eh, the year when they started, one can still see, they used to mine this with, ah, a lot manually.' (Sadrún, m4, \sectref{sec:8.3})
\z

\ea
\label{ex:majn2}
\gll Gè, sch’ ju mir’ anavùs … quaj è fòrsa sjat, ògj òns … schòn mù gljèz … savèv’ ins í \textbf{patschíficamajn}.   \\
yes if \textsc{1sg} look.\textsc{prs.1sg} back {} \textsc{dem.unm} \textsc{cop.prs.3sg} maybe seven eight year.\textsc{m.pl} {} really only \textsc{dem.unm} {} can.\textsc{impf.3sg} \textsc{gnr} go.\textsc{inf} peaceful.\textsc{f.sg.adv}\\
\glt `Yes, if I look back ... that was maybe seven, eight years [ago] ... only that ... one could go peacefully.' (Sadrún, m9, \sectref{sec:8.15})
\z

Most of the adverbs that are derived from adjectives function as modal adverbs, which express the attitude of the speaker towards the propositional content of the sentence. Some examples are \textit{atgnamajn} `actually', \textit{bunamajn} `almost', \textit{naturálmajn} `naturally', \textit{vájramajn} `really', and \textit{símplamajn} `simply'.

The scarcity of derived adverbs is due to the fact that Tuatschin very often uses an \isi{adjective} in an adverbial function (see \sectref{sec:3.3.4}). An example of an \isi{adjective} functioning as a \isi{modal adverb} is \textit{pulit} instead of \textit{pulítamajn} in (\ref{ex:pulit1}).

\ea
\label{ex:pulit1}
\gll [...] a lu … sjantar in pèr dis … vajn nus savju dumigná quèls pòrs … tga mavan \textbf{pulit} [...].\\
{} and then {} after \textsc{indef.m.sg} couple day.\textsc{m.pl} {} have.\textsc{prs.1pl} \textsc{1pl} can.\textsc{ptcp.unm} cope.\textsc{inf} \textsc{dem.m.pl} pig.\textsc{pl} {} \textsc{rel}  go.\textsc{impf.3pl} proper.\textsc{adj.unm} \\
\glt `[...] and then … after a couple of days … we were able to cope with these pigs … which would move properly [...].' (Sadrún, m6, \sectref{sec:8.11})
\z

Prepositional phrases functioning as \isi{manner adjuncts} are rare; an example is (\ref{ex:apaj}).

\ea
\label{ex:apaj}
\gll Alṣ ùmans mavan cul latg gjù tgèsa a quaj èra … in' ur’ \textbf{a} \textbf{paj} bjabé̱gn.\\
\textsc{def.m.sg} man.\textsc{pl} go.\textsc{impf.3pl} with.\textsc{def.f.sg} milk down home.\textsc{f.sg} and \textsc{dem.unm} \textsc{cop.impf.3sg} {} one.\textsc{f} hour by foot around\\
\glt `The men would go home with the milk and that would take them about an hour on foot.' (Cavòrgja, m7, \sectref{sec:8.17})
\z

\subsection{Further adjuncts}\label{sec:4.3.4}

\isi{Beneficiary adjuncts} are introduced by the preposition \textit{par/pr̩} `for' (\ref{ex:par1}).

\ea
\label{ex:par1}
\gll   [...] i è stau in tjams tga mavan sél pas, mávani sé a métar najf \textbf{pala} ... \textbf{viafíar} \textbf{da} \textbf{la} \textbf{Fùrca-Albṣù̱}. \\
{} \textsc{expl} be.\textsc{prs.3sg} \textsc{cop.ptcp.unm} \textsc{indef.m.sg} time \textsc{rel} go.\textsc{impf.3pl} on.\textsc{def.m.sg} pass go.\textsc{impf.3pl.3pl} up \textsc{subord} put.\textsc{inf} snow for.\textsc{def.f.sg} {} railway of \textsc{def.f.sg} \textsc{pn}\\
\glt `[...] it was a time when they would go up to the pass, they used to go up to remove snow for the ... Furka-Alpsu railway line.' (Sadrún, m4, \sectref{sec:8.3})
\z

\isi{Causal adjuncts} are introduced by \textit{antrás} `through' (\ref{ex:antras}) or by \textit{parví da} `because of' (\ref{ex:parvida}).

\ea
\label{ex:antras}
\gll A lura … a lò quaj trùtg èra in … trùtgs \textbf{antrás} `\textbf{ls} \textbf{lavurs} c' i òn fatg la òvra èlèctrica.\\
and then {}  and there \textsc{dem.m.sg} footpath  \textsc{cop.impf.3sg} \textsc{indef.m.sg} {} footpath.\textsc{m.pl} through \textsc{def.f.pl}  work.\textsc{pl} when \textsc{3pl} have.\textsc{prs.3pl} make.\textsc{ptcp.unm} \textsc{def.f.sg} work electric \\
\glt `And then … and there, this footpath was a … paths were built when they built the electric power station.' (Ruèras, m10, \sectref{sec:8.7})
\z

\ea
\label{ex:parvida}
\gll    A lu, \textbf{parví} \textbf{da} \textbf{la} \textbf{pjaglja}, nuṣ vèvan lu dad í ad incassá.\\
and then because of \textsc{def.f.sg} wage \textsc{1pl} have.\textsc{impf.1pl} then to go.\textsc{inf}  \textsc{subord} collect.\textsc{inf} \\
\glt `And then, because of the wage, we had to go and collect [the money].' (Sadrún, m6, \sectref{sec:8.11})
\z

\isi{Comitative adjuncts} are expressed either by \textit{cun} `with' (\ref{ex:comcun}), by \textit{ansjaman/an\-zja\-man} `together' (\ref{ex:ansjaman}), or by \textit{ansjaman cun} `together with' if the argument of \textit{ansjaman} is mentioned (\ref{ex:ansjamancun}).

\ea
\label{ex:comcun}
\gll Té savèssaṣ í \textbf{cul} \textbf{tat} ajn Pardatsch.  \\
 \textsc{2sg} can.\textsc{cond.2sg} go.\textsc{inf} with.\textsc{def.m.sg} grandfather up \textsc{pn}  \\
\glt `You could go up to Pardatsch with your grandfather.' (Sadrún, m4, \sectref{sec:8.3})
\z

\ea
\label{ex:ansjaman}
\gll    Tgé! Quèlas taljánaras èran ampʰau \textbf{anzjaman} a las ròmòntschas né las tudèstgas, né è dal vitg matévani schòn in téc \textbf{anzjaman}.\\
what \textsc{dem.f.pl} Italian.\textsc{f.pl} \textsc{cop.impf.3pl} a\_bit together and \textsc{def\textbf{}.f.pl} Romansh.\textsc{f.pl} or  \textsc{def.f.pl} German.\textsc{pl} or also of.\textsc{def.m.sg} village put.\textsc{impf.3pl.3pl} in\_fact \textsc{indef.m.sg} bit together\\
\glt `Look! These Italians were a bit together, and the Romansh or the Germans, or they put them together even from the [same] village.' (Sadrún, f6, \sectref{sec:8.4})
\z

\ea
\label{ex:ansjamancun}
\gll   La règína végn cupanada; èla ṣgùla a vò a spaz \textbf{ansjaman} \textbf{cun} in grias […]. \\
    \textsc{def.f.sg} queen \textsc{pass.prs.3sg} fertilise.\textsc{ptcp.f.sg} \textsc{3sg.f} fly.\textsc{prs.3sg} and go.\textsc{prs.3sg} for walk together with \textsc{indef.m.sg} drone \\
\glt `The queen is fertilised; she flies away and goes for a trip with a drone […].' (Ruèras, \DRGoK{1}{602})
\z

Some speakers use the form \textit{cun} `with' with /u/ in all cases, and some speakers use /u/ with \textit{cun}, but /ʊ/ either in all cases or only with \isi{contracted} forms like \textit{cùl} `with the (\textsc{m.sg})' or \textit{cùlas} `with the \textsc{(f.pl})'.

\isi{Instrumental adjuncts} are introduced by \textit{cun} (\ref{ex:instr:cun}); if the adjunct refers to a material out of which something is made, \textit{òr da} `out of' is used (\ref{ex:instr:orda}).

\ea
\label{ex:instr:cun}
\gll Ábar èba, ju détsch adina ina dùna stù adina dá \textbf{cul} \textbf{pal}.\\
but precisely \textsc{1sg} say.\textsc{prs.1sg} always \textsc{indef.f.sg} woman must.\textsc{prs.3sg} always give.\textsc{inf} with.\textsc{def.m.sg} post\\
\glt `But in fact, I always say that a woman must always hit with a post.' (Sadrún, f3, \sectref{sec:8.1})
\z

\ea
\label{ex:instr:orda}
\gll  A lu vèvan, òni fatg \textbf{ò} \textbf{dal} \textbf{mir}, álṣò \textbf{òr} \textbf{dal} \textbf{grép} òni fatg ina pintga …  sènda [...].\\
and then have.\textsc{impf.3pl} have.\textsc{prs.3pl.3pl} make.\textsc{ptcp.unm} out of.\textsc{def.m.sg} rock\_face this\_is\_to\_say out of.\textsc{def.m.sg} rock have.\textsc{prs.3pl.3pl} make.\textsc{ptcp.unm} \textsc{indef.f.sg} small {} path\\
\glt `And then they made, out of the rock face, that is to say, out of the rock they made a small … path [...].' (Ruèras, m10, \sectref{sec:8.7})
\z 


\section{Negation}\label{sec:4.4}
Tuatschin possesses two \isi{verb phrase negators}: \textit{bétga} and its allomorphs \textit{bétg'/bé/ bigja/bgja/bigj'},\footnote{The form \textit{bé} is usually only used by children and younger people, but very rarely by older people.} which is the dedicated negator, and \textit{nuéta}, which is less frequently used than \textit{bétga}.

\textit{Bétga} is located after the finite verb (\ref{ex:negbetg1}), which means that for compound tenses, it is located after the auxiliary verb (\ref{ex:negbetg2}), and for modal verbs heading an infinitive clause, the negator is situated after the modal verb (\ref{ex:negbetg3}). 

\ea
\label{ex:negbetg1}
\gll Ju \textbf{sa} \textbf{bé} dacù̱.\\
\textsc{1sg} know.\textsc{prs.1sg} \textsc{neg} why \\
\glt `I don’t know why.' (Sadrún, m8, \sectref{sec:8.12})
\z

\ea
\label{ex:negbetg2}
\gll  Álṣò ju \textbf{a} \textbf{bigja} \textbf{fatg} aj agrèssíf [...]. \\
well \textsc{1sg} have.\textsc{prs.1sg} \textsc{neg} make.\textsc{ptcp.unm} \textsc{3sg} aggressive.\textsc{adj.unm}\\
\glt `Well, I didn’t do it in an aggressive way [...].' (Sadrún, m8, \sectref{sec:8.12}
\z

\ea\label{ex:negbetg3}
\gll  Ju \textbf{sa} \textbf{bigj’} \textbf{í} ál’ aua.\\
\textsc{1sg} can.\textsc{prs.1sg} \textsc{neg} go.\textsc{inf} into.\textsc{def.f.sg} water \\
\glt `I couldn’t jump into the water.' (Sadrún, m8, \sectref{sec:8.12})
\z

However, in a similar way as with particle verbs (see \sectref{sec:4.1.3}), the \isi{inverted subject} (\ref{ex:negbetg4}) and some adverbs like \textit{aun} `yet', \textit{è} `also', \textit{èba} `just', \textit{lu} `then', and \textit{ùs} `now', may intervene between the finite verb and the participle or the infinitive (\ref{ex:negbetg5}--\ref{ex:negbetg9}).

\ea
\label{ex:negbetg4}
\gll Las nòtízjas \textbf{sa} \textbf{ju} \textbf{bétg} danù̱ndar als gjaniturs, als duṣ baps prandèvan aj [...]. \\
\textsc{def.f.pl} news.\textsc{pl} know.\textsc{prs.1sg} \textsc{1sg} \textsc{neg} from\_where \textsc{def.m.pl} parents.\textsc{pl} \textsc{def.m.pl} two.\textsc{m} father.\textsc{pl} take.\textsc{impf.3pl} \textsc{3sg}\\
\glt `I don’t know where my parents had the news from, the two fathers took them [...].' (Ruèras, m1, \sectref{sec:8.2})
\z

\ea
\label{ex:negbetg5}
\gll  [...] avaun c’ ju sùn staus tial tat savèvu da quaj nuét a \textbf{vèṣ} \textbf{è} \textbf{bitga} safatg ajn zatgé spazjal.  \\
{} before \textsc{subord} \textsc{1sg} be.\textsc{prs.1sg} \textsc{cop.ptcp.m.sg} at.\textsc{def.m.sg} grandfather know.\textsc{impf.1sg.1sg} of \textsc{dem.unm} nothing and have.\textsc{cond.1sg} also \textsc{neg} \textsc{refl.}do.\textsc{ptcp.unm} in something special.\textsc{m.sg}\\
\glt `[...] before I stayed with my grandfather I didn’t know anything and I wouldn’t have noticed anything either.' (Sadrún, m4, \sectref{sec:8.3})
\z

\ea
\label{ex:negbetg6}
\gll    Ál’ antschata cu té \textbf{capèschaṣ} \textbf{aun} \textbf{bigja} quèls … curjòs plaids tg’ èls òn, stòs halt dumandá [...].\\
at.\textsc{def.f.sg} beginning when \textsc{2sg.gnr} understand.\textsc{prs.2sg.gnr} yet \textsc{neg} \textsc{dem.m.pl} {} strange.\textsc{pl} word.\textsc{pl} \textsc{rel} \textsc{3pl.m} have.\textsc{prs.3pl} must.\textsc{prs.2sg.gnr} just ask.\textsc{inf}\\
\glt `At the beginning when you don’t understand yet those … strange words they use, you must just ask [...].' (Camischùlas, f6, \sectref{sec:8.4})
\z

\ea
\label{ex:negbetg7}
\gll    A la sèra èssan nus, \textbf{stuèvan} \textbf{nuṣ} \textbf{èba} \textbf{bigj’} í ad uraṣ ajnta létg [...].\\
and \textsc{def.f.sg} evening be.\textsc{prs.1pl} \textsc{1pl} must.\textsc{impf.1pl} \textsc{1pl} just \textsc{neg} go.\textsc{inf} at hour.\textsc{f.pl} in  bed.\textsc{m.sg}\\
\glt `And in the evening, we went, we didn’t have to go to bed early [...].' (Sadrún, f6, \sectref{sec:8.4})
\z

\ea\label{ex:negbetg8}
\gll  [...] api sau \textbf{bigj’} ajfach \textsc{í} ál’ aua.\\
{} and can.\textsc{prs.1sg.1sg} \textsc{neg} simply go.\textsc{inf} into.\textsc{def.f.sg} water\\
\medskip
\glt `[...] I cannot simply jump into the water.' (Sadrún, m8, \sectref{sec:8.12})
\z

\ea
\label{ex:negbetg9}
\gll    Quaj è hald ina détga tgu sa, ábar plé gròn savès \textbf{ju} \textbf{lu} \textbf{è} \textbf{bétga} [...]\\
\textsc{dem.unm} \textsc{cop.prs.3sg} simply \textsc{indef.f.sg} legend \textsc{rel.1sg} know.\textsc{prs.1sg} but more big.\textsc{m.sg} can.\textsc{cond.1sg} \textsc{1sg} then also \textsc{neg}\\
\glt `This is a legend I know, but a longer one I would not be able [...].' (Zarcúns, m2, \sectref{sec:8.13})
\z

The two elements of \textit{bétga plé} `not any more' are not necessarily immediately adjacent to each other as in (\ref{betgaple1}) where a \isi{simple tense} is used.

\ea
\label{betgaple1}
\gll  Álṣò ùṣ è `l \textbf{bigja} \textbf{plé} òdém al vitg [...]. \\
well now  \textsc{cop.prs.3sg} \textsc{3sg.m} \textsc{neg} any\_more low\_most \textsc{def.m.sg} village\\
\glt `Well, now it is not at the lowest part of the village any more [...].' (Sadrún, m5, \sectref{sec:8.8})
\z

If a \isi{compound tense} is used, \textit{bétga} is situated before the participle and \textit{plé} follows it (\ref{betgaple2} and \ref{betgaple3}).

\ea
\label{betgaple2}
\gll   Èl è \textbf{bétga} jus \textbf{plé} cun quaj catschadur. \\
\textsc{3sg} be.\textsc{prs.3sg} \textsc{neg} go.\textsc{ptcp.m.sg} more with \textsc{dem.m.sg} hunter\\
\glt `[…] he didn’t go with this hunter any more.' (Tschamùt, \citealt[12]{Büchli1966})
\z

\ea\label{betgaple3}
\gll  A lu vès ju \textbf{bitga} ugagjau \textbf{plé} da, dad ira cun èls vinanavaun.\\
and then have.\textsc{cond.1sg} \textsc{1sg} \textsc{neg} dare.\textsc{ptcp.unm} more \textsc{comp} \textsc{comp} go.\textsc{inf} with \textsc{3pl.m} farther\\
\glt `And then I wouldn’t have dared to, to go farther with them any more.' (Ruèras, m10, \sectref{sec:8.7})
\z

The same holds for \isi{modal verbs} governing an \isi{infinitive phrase}, where \textit{bétga} precedes and \textit{plé} follows infinitive (\ref{betgaple4}) or the infinitive with its particle (\ref{betgaple5}).

\ea
\label{betgaple4}
\gll  Ju cala dad í a scùlèta, ju \textbf{pùs} \textbf{bitg} í \textbf{plé}.\\
\textsc{1sg} stop.\textsc{prs.1sg} \textsc{comp} go.\textsc{inf} to nursery\_school.\textsc{f.sg} \textsc{1sg} can.\textsc{prs.1sg} \textsc{neg} go.\textsc{inf} any\_more  \\
\glt `I’ll stop going to nursery school, I can’t stand it any longer.' (Sadrún, m4, \sectref{sec:8.3})
\z

\ea
\label{betgaple5}
\gll [...] òzaldé̱ … \textbf{sa} ju \textbf{bigja} mé̱tar avaun \textbf{plé} tg’ i fòn da gljèz, né?    \\
{} nowadays {} can.\textsc{prs.1sg} \textsc{1sg} \textsc{neg} put.\textsc{inf} before more \textsc{comp} \textsc{3pl} do.\textsc{prs.3pl} of \textsc{dem.unm} right \\
\glt `'[...] nowadays ... I cannot imagine any more that they play that, right?' (Sadrún, m9, \sectref{sec:8.15})
\z

In \isi{infinitive clauses} \textit{bétga} precedes the infinitive (\ref{betgaple10}).

\ea
\label{betgaple10}
\gll  [...] quaj èssan nuṣ i sé pr̩ … \textbf{pr̩} \textbf{bitga} \textbf{stuaj} \textbf{ira} sé la … sé la via dad autos, sé la via dal pas. \\
{} \textsc{dem.unm} be.\textsc{prs.1pl} \textsc{1pl} go.\textsc{ptcp.m.pl} up \textsc{subord} {} \textsc{subord} \textsc{neg} must.\textsc{inf} go.\textsc{inf} up \textsc{def.f.sg} {} up \textsc{def.f.sg} way of car.\textsc{m.pl} up \textsc{def.f.sg} way of.\textsc{def.m.sg} pass  \\
\glt `[...] there we went up in order to avoid the car road, the road of the pass.' (Ruèras, m10, \sectref{sec:8.7})
\z

Other elements may intervene between \textit{bétga} and \textit{plé}. (\ref{betgaple6}) shows that this is the case for the particle belonging to the verb \textit{tanaj} `hold' (\textit{sé}), a manner adjunct (\textit{bégn}), a nominal direct object (\textit{als praus}), and a locative adjunct (\textit{cò}).

\ea
\label{betgaple6}
\gll   I tégnan \textbf{bétga} sé schi bégn als praus cò \textbf{plé}. \\
\textsc{3pl} hold.\textsc{prs.3pl} \textsc{neg} up so well \textsc{def.m.pl} field.\textsc{pl} here any\_more \\
\glt `Here they don’t see to the fields well any more.' (\citealt[69]{Berther2007})
\z

Further elements are indirect interrogative clauses (\ref{betgaple7}) and object clauses (\ref{betgaple8} and \ref{betgaple9}).

\ea
\label{betgaple7}
\gll Ábar ju sa \textbf{bigja} tgé quaj è \textbf{plé}.\\
but \textsc{1sg} know.\textsc{prs.1sg} \textsc{neg} what \textsc{dem.unm} \textsc{cop.prs.3sg} more\\
\glt `But I don't know what this is any more.' (Sadrún, m4)
\z

\ea
\label{betgaple8}
\gll    Als pádars savèvan \textbf{bétga} spatgè plé ditg \textbf{plé}.\\
      \textsc{def.m.pl} Father.\textsc{pl} know.\textsc{impf.3pl} \textsc{neg} wait.\textsc{inf} more long any\_more\\
\glt `The fathers couldn’t wait any longer […].' (Bugnaj, \citealt[147]{Büchli1966})
\z

\ea
\label{betgaple9}
\gll    Lura lèvani \textbf{bétga} schè fá pástar gròn èla \textbf{plé} […].\\
     then want.\textsc{impf.3pl.3pl} \textsc{neg} let.\textsc{inf} make.\textsc{inf} shepherd big \textsc{3sg.f} any\_more\\
\glt `Then they didn’t want to let her be the main shepherdess […] any more.' (Cavòrgja, \citealt[119]{Büchli1966})
\z

Examples (\ref{ex:muple1} and \ref{ex:muple2}) illustrate \textit{mù … plé} `only ... more', which displays the same syntax as \textit{bétga plé}.

\ea
\label{ex:muple1}
\gll  [...] scha vagi èl \textbf{mù} in cazè \textbf{plé}.\\
{} then have.\textsc{prs.sbjv.3sg} \textsc{3sg.m} only one.\textsc{m} shoe any\_more \\
\glt `[…] then he would have only one shoe left.' (Tschamùt, \citealt[15]{Büchli1966})
\z

\ea
\label{ex:muple2}
\gll    [...] quaj piartg èra juṣ atráṣ a vèva rùt gjù al matg tga vèva \textbf{mù} la còrda \textbf{plé} antù̱rn.\\
{} \textsc{dem.m.sg} pig be.\textsc{impf.3sg} go.\textsc{ptcp.m.sg} through and have.\textsc{impf.3sg} break.\textsc{ptcp.unm} down \textsc{def.m.sg} bunch  \textsc{subord} have.\textsc{impf.3sg} only \textsc{def.f.sg} rope more around\\
\glt `[...] this pig had gone through and had broken the bunch of flowers so that he only had the rope around [his belly].' (Sadrún, m6, \sectref{sec:8.11})
\z

`Not yet' is rendered \textit{bétg aun} (\ref{betgaun1} and \ref{betgaun2}) or \textit{aun bétg} (\ref{aunbetg1}).

\ea
\label{betgaun1}
\gll  Ju vèva \textbf{bigja} \textbf{aun} ampríu, ajn scùlèta amprandèvan nus pauc.\\
\textsc{1sg} have.\textsc{impf.1sg} \textsc{neg} yet  learn.\textsc{ptcp.unm} in  nursery\_school.\textsc{f.sg} learn.\textsc{impf.1pl}  \textsc{1pl} little \\
\glt `I hadn’t learned yet, we didn’t learn much in nursery school.' (Sadrún, m6, \sectref{sec:8.4})
\z

\ea
\label{betgaun2}
\gll  Ábar lu èra quaj \textbf{bigja} \textbf{aun} schi bjè.  \\
but then \textsc{cop.impf.3sg} \textsc{dem.unm} \textsc{neg} yet so much  \\
\glt `But at that time this was not that much yet.' (Sadrún, f3, \sectref{sec:8.1})
\z

\ea
\label{aunbetg1}
\gll    [...] lu èri \textbf{aun} \textbf{bigja} turists [...]. \\
{} then \textsc{exist.impf.3sg.expl} yet \textsc{neg} tourist.\textsc{m.pl}\\
\glt `'[...]  then there weren’t tourists yet [...].' (Surajn, f5, \sectref{sec:8.10})
\z


\textit{Nuéta} as a \isi{verb phrase negator} is much less frequently used than \textit{bétga}, but it expresses a stronger \isi{negation} (\ref{ex:nueta1}--\ref{ex:nueta3}).

\ea
\label{ex:nueta1}
\gll A quaj plaṣchéva \textbf{nuéta} pròpi da mé. \\
and \textsc{dem.unm} please.\textsc{impf.3sg} \textsc{neg} really \textsc{dat} \textsc{1sg}\\
\glt `And I really didn’t  like that.' (Sadrún, m4, \sectref{sec:8.3})
\z

\ea
\label{ex:nueta2}
	\gll Ò quaj pùs té schòn, ùṣa quaj è \textbf{nuéta} schi nausch.\\
oh \textsc{dem.unm} can.\textsc{prs.2sg} \textsc{2sg} indeed now \textsc{dem.unm} \textsc{cop.prs.3sg} \textsc{neg} so bad.\textsc{adj.unm}\\
\glt `Oh, you are certainly able to do that, now this is not so bad.' (Ruèras, f4)
\z

\ea
\label{ex:nueta3}
\gll  Ábar èl vèva par clétg fétg bian saun al … ò `l lu \textbf{nuéta} gju còmplicazjuns.  \\
but  \textsc{3sg.m} have.\textsc{impf.3sg} for luck.\textsc{m.sg} very good.\textsc{m.sg} blood \textsc{def.m.sg} {} have.\textsc{prs.3sg} \textsc{3sg.m} then \textsc{neg} have.\textsc{ptcp.unm} complication.\textsc{f.sg}\\
\glt `But fortunately his blood was very good, the … he then hadn’t got any complications.' (Sadrún, m4, \sectref{sec:8.3})
\z

\isi{Negative adverbs} are \textit{maj} `never' (\ref{ex:maj1}), \textit{nagín/nagina} `nobody, no' (pronoun and determiner) (\ref{ex:nagin1}), \textit{nuét} `nothing' (\ref{ex:nuetpron1}), and \textit{nagljú} `nowhere' (\ref{ex:naglju1}). They never co-occur with \textit{bétga}.

\ea
\label{ex:maj1}
\gll Ju èr’ in gjuvanò̱tar, sùn \textbf{maj} staus fumégl [...].\\
\textsc{1sg} \textsc{cop.impf.1sg} \textsc{indef.m.sg} youngster be.\textsc{prs.1sg} never \textsc{cop.ptcp.m.sg} farmhand.\textsc{m.sg} \\
\glt `I was a youngster, I never was a farmhand [...].' (Ruèras, m3, \sectref{sec:8.16})
\z

\ea
\label{ex:nagin1}
\gll  Par í cu té vaṣ adina … ṣè quaj … \textbf{nagín} \textbf{pròblè̱m}.  \\
\textsc{subord} go.\textsc{inf} when \textsc{2sg.gnr} go.\textsc{prs.2sg.gnr} always {} \textsc{cop.prs.3sg} \textsc{dem.unm} {} no.\textsc{m.sg} problem \\
\glt `In order to go, if one always goes [there] ..., this is … no problem at all.' (Sadrún, f3, \sectref{sec:8.3})
\z

\ea
\label{ex:nuetpron1}
	\gll Al bap ò détg \textbf{nuét}.\\
\textsc{def.m.sg} father have.\textsc{prs.3sg} say.\textsc{ptcp.unm} nothing\\
\glt `My father didn't say anything.' (Cavòrgja, f7, \sectref{sec:8.17})
\z

Generally speaking, there is no double \isi{negation} in Tuatschin (\ref{ex:naglju1} and \ref{ex:senzaneg1}).

\ea
\label{ex:naglju1}
\gll Ju a vju \textbf{nagljú} \textbf{zatgéj}.\\
\textsc{1sg} have.\textsc{prs.1sg} see.\textsc{ptcp.unm} nowhere something\\
\glt `I haven't seen anything anywhere.' (Sadrún, m5)
\z

\ea
\label{ex:senzaneg1}
\gll    Al gjával ò stju í \textbf{sènza} savaj fá \textbf{zatgéj}.\\
      \textsc{def.m.sg} devil have.\textsc{prs.3sg} must.\textsc{ptcp.unm} go.\textsc{inf} without know.\textsc{inf} do.\textsc{inf} something \\
\glt `The devil had to leave without being able to do anything.' (Bugnaj, \citealt[147]{Büchli1966})
\z



