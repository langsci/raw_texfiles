\chapter{Morphological processes}

\section{Reduplication}\label{sec:7.1}
In Tuatschin, \isi{reduplication} has only an intensification function. Syntactic categories that may be reduplicated are attributive (\ref{ex:redadjattr1}) and predicative (\ref{ex:redadjpred1} and \ref{ex:redadjpred2})  adjectives, adjectives used adverbially (\ref{ex:redadjpred3}), as adverbs modifying adjectives (\ref{ex:redadv1}), as adverbs modifying verbs (\ref{ex:redadv2} and \ref{ex:redadv3}), or functioning as a discourse marker (\ref{ex:redadv4}).

\ea\label{ex:redadjattr1}
\gll  Ála \textbf{véglja-véglja} tgèsa-parvènda, né?\\
in.\textsc{def.f.sg} \textsc{red}\textasciitilde{old} presbytery right\\
\glt `At the very old presbytery, right?' (Sadrún, m5, \sectref{sec:8.9})
\z

\ea
\label{ex:redadjpred1}
\gll  […] i èra \textbf{sgtir-stgir} !\\
{}  \textsc{expl} \textsc{cop.impf.3sg} \textsc{red}\textasciitilde{dark}\\
\glt `[…] it was pitch-dark.' (Surajn, \citealt[128]{Büchli1966})
\z

\ea
\label{ex:redadjpred2}
\gll    A quaj stèvnṣ èssar … \textbf{pulits-pulits} l’ jamna … tg’ al bap dètschi in frang a miaz.\\
and \textsc{dem.unm} must.\textsc{impf.1pl.1pl} \textsc{cop.inf} {} \textsc{red}\textasciitilde{well\_behaved}.\textsc{m.pl} \textsc{def.f.sg} week {} \textsc{subord} \textsc{def.m.sg} father  give.\textsc{prs.sbjv.3sg} one.\textsc{m.sg} franc and half.\textsc{m.sg}\\
\glt `And we had to be … very well-behaved during the week … so that my father would give [us] one and a half francs.' (Ruèras, m1, \sectref{sec:8.2})
\z

\ea
\label{ex:redadjpred3}
\gll  Al tat èr’ ajn a durméva lò grat sc’ in tajṣ, vèv’ udju \textbf{ṣchùbar-ṣchùbar} nuét.  \\
\textsc{def.m.sg} grandfather \textsc{cop.impf.3sg} up and sleep.\textsc{impf.3sg} there precisely like \textsc{indef.m.sg} badger have.\textsc{impf.3sg} hear.\textsc{ptcp.unm} \textsc{red}\textasciitilde{clean}.\textsc{adj.unm} nothing\\
\glt `My grandfather was up there and was sleeping like a log, he hadn’t heard anything at all.' (Sadrún, m4, \sectref{sec:8.3})
\z


\ea
\label{ex:redadv1}
\gll  Ju a gju \textbf{fétg-fétg} bian cun èl. \\
\textsc{1sg} have.\textsc{prs.1sg} have.\textsc{ptcp.unm} \textsc{red}\textasciitilde{very} good.\textsc{unm} with \textsc{3sg.m}\\
\glt `I got along very well with him.' (Sadrún, m4)
\z

\ea
\label{ex:redadv2}
\gll El ò mirau \textbf{antùrn-antùrn} […].\\
\textsc{3sg.m} have.\textsc{prs.3sg} look.\textsc{ptcp.unm} \textsc{red}\textasciitilde{around} \\
\glt `He looked around and around [...].' (Tschamùt, \citealt[18]{Büchli1966})
\z

\ea
\label{ex:redadv3}
\gll    A sjantar surpríu acòrds \textbf{adin-adina}.\\
and after take\_over.\textsc{ptcp.unm} piecework.\textsc{m.pl} \textsc{red}\textasciitilde{always} \\
\glt `And afterwards [I] took over piecework, always.' (Ruèras, m1, \sectref{sec:8.2})
\z

\ea
\label{ex:redadv4}
\gll Quaj èra in’ jèda ... brutal tiar nus, \textbf{bèn-bèn}.   \\
\textsc{dem.unm} \textsc{cop.impf.3sg} one.\textsc{f} time {} terrible.\textsc{adj.unm} among \textsc{1pl} \textsc{red}\textasciitilde{really}\\
\glt `Once it was terrible among us, really.' (Sèlva, f2, \sectref{sec:8.6})
\z

A similar case is the repetition of words (\ref{ex:rep1} and \ref{ex:rep2}). It differs from the \isi{reduplication} insofar as the repeated items are separated by a short pause; they furthermore do not indicate intensification, but only repetition.

\ea
\label{ex:rep1}
\gll [...] a lu stuèv’ ju la sèra adin’ ir' a fá tschajna … a fá … \textbf{manèstra}, \textbf{manèstra}, \textbf{manèstra}, \textbf{manèstra}.\\
{} and then must.\textsc{impf.1sg} \textsc{1sg} \textsc{def.f.sg} evening always go.\textsc{inf} \textsc{subord} make.\textsc{inf} dinner.\textsc{f.sg} {} \textsc{subord} make.\textsc{inf} {}  pottage.\textsc{f.sg} pottage.\textsc{f.sg} pottage.\textsc{f.sg} pottage.\textsc{f.sg}\\
\glt `[...] and then in the evening I always had to go and prepare dinner ... and prepare pottage, pottage, pottage, pottage.' (Ruèras, f4, \sectref{sec:8.16})
\z
 
\ea
\label{ex:rep2}
\gll [...] a quèl’ èra da \textbf{trá} … \textbf{trá}, \textbf{trá} … tùt al latg cèntrifugau.\\
{} and \textsc{dem.f.sg} be.\textsc{impf.3sg} to pull.\textsc{inf} {} pull.\textsc{inf} pull.\textsc{inf} {} all \textsc{def.m.sg} milk centrifugate.\textsc{ptcp.unm}    \\
\glt `[...] and we had to pull ... pull, pull ... the whole centrifuged milk.' (Ruèras, m4, \sectref{sec:8.16})
\z

\section{Word formation}\label{sec:7.2}
Since Tuatschin is a spoken language, many derivational affixes and compound words which are used in Standard (written) Sursilvan\footnote{See \citet[163--194]{Spescha1989} for extensive lists.} do not occur.

\subsection{Compounding of nouns}\label{7.2.1}
\isi{Compounding} is achieved by joining two nouns with the preposition \textit{da} `of' (\ref{ex:tiartgèsa1}), and also by juxtaposition of two nouns (\ref{ex:clutgebaselgja1}), whereby the second noun modifies the first one as in \textit{baun-pégna} `oven bench', literally `bench-oven'. Which strategy is used depends on the compound, and in some cases the two strategies may apply to the same nouns with different meanings. This last point is best exemplified by (\ref{ex:tiartgèsa1}).

\ea\label{ex:tiartgèsa1}
\gll  In \textbf{tiar} \textbf{da} \textbf{tgèsa}, è `l gjat a `l tgaun, a \textbf{tiar-tgèsa} è sagir tùt quaj tga vò bigj’ ad alp. Als \textbf{tiars-tgèsa} èn atgnamajn cò, né sén majṣès, quaj è quèlas tgauras a nùrsas a pòrs.  \\
\textsc{indef.m.sg} animal of house.\textsc{f.sg} \textsc{cop.prs.3sg} \textsc{def.m.sg} cat and \textsc{def.m.sg} dog  and animal.\textsc{m.sg}-house.\textsc{f.sg} \textsc{cop.prs.3sg} for\_sure all \textsc{dem.unm} \textsc{rel} go.\textsc{prs.3sg} \textsc{neg} to alpine\_pasture.\textsc{m.sg} \textsc{def.m.pl} animal.\textsc{pl}-house.\textsc{f.sg} \textsc{cop.prs.3pl} actually here or on assembly\_of\_houses.\textsc{m.sg} \textsc{dem.gl} \textsc{cop.prs.3sg} \textsc{dem.f.pl} goat.\textsc{pl} and sheep.\textsc{f-pl} and pig.\textsc{m.pl}\\
\glt `A \textit{tiar da tgèsa}, these are cats and dogs, and \textit{tiar tgèsa} are of course all those that do not go to the alpine pastures. The \textit{tiars tgèsa} are actually here, or up in the assembly of houses, these are the goats, the sheep, and the pigs.' (Tuatschín, Cavòrgja, m7)
\z

\ea
\label{ex:clutgebaselgja1}
\gll[...] lò ani bagagjau gjù la … raquéntani … la crapa par bagagè al \textbf{clutgè-basèlgja}.\\
{} there have.\textsc{prs.3pl.3pl} build.\textsc{ptcp.unm} down \textsc{def.f.sg} {} tell.\textsc{prs.3pl.3pl} {} \textsc{def.f.sg} stone.\textsc{coll} \textsc{subord} build.\textsc{inf}  \textsc{def.m.sg} tower.\textsc{m.sg}-church.\textsc{f.sg}  \\
\glt `[...] there they removed, as they say, the stones used to build the church tower [of Sedrun].' (Sadrún, m4, \sectref{sec:8.3})
\z

The \isi{compounding of nouns} by juxtaposition is relatively frequent; further examples are \textit{carschèn-matg} `waxing moon of May', \textit{crusch-fiar} `iron cross', \textit{ésch-stiva} `door of the living-room', \textit{fil-sajda} `silk thread', \textit{lungatg-mùma} `mother-tongue', \textit{mòni-scúa} `broomstick', \textit{patrún-basèlgja} `Church Patron', \textit{pòrta-basèlgja} `church door', \textit{pòrta-clavau} `barn door', \textit{prajt-crap} `rock face', \textit{pròcèsjún-basèlgja} `religious procession', \textit{tètg-tégja} `roof of the alpine hut', and \textit{tgau-vitg} `head of the village'. \textit{Nadal-nòtg} `Christmas Eve' has different syntax: here, it is the first \isi{noun} that modifies the second one, probably under the influence of German \textit{Weihnachtsnacht}, literally `Christmas Night'.

\subsection{Derivation}\label{sec:7.2.2}
Some \isi{derivational morphemes} have already been treated: the non-finite verbal categories \isi{past participle} (\sectref{sec:4.1.2.1.1}), \isi{gerund} (\sectref{sec:4.1.2.1.2}), \isi{infinitive} (\sectref{sec:4.1.2.1.3}), the adverbialiser \textit{-majn} (\sectref{sec:4.3.3}), and the \isi{causative} -\textit{antá} (\sectref{sec:5.5.3}).


\subsubsection{Diminutive and augmentative}\label{sec:7.2.2.1}

The \isi{diminutive} of nouns is formed with the suffix \textit{-èt/-èta}. (\ref{ex:dim1}) shows that the use of the \isi{diminutive} does not preclude the use of \textit{pin} `small'.

\ea\label{ex:dim1}
\gll    Lò fùva in pin \textbf{laj-èt} cun pauc’ aua.\\
     there \textsc{exist.impf.3sg} \textsc{indef.m.sg} small lake-\textsc{dim} with little.\textsc{f.sg} water \\
\glt `There was a small lake with little water.' (Ruèras, \citealt[62]{Büchli1966})
\z

Further examples are \textit{buébèt} `little boy', \textit{fjuchèt} `little fire', and \textit{vitgèt} `small village'.

There are two \isi{augmentative} suffixes. One is \textit{-ún/-una} (\ref{ex:aum1} and \ref{ex:aum2}). 

\ea
\label{ex:aum1}
\gll  Quaj è \textbf{dòn-ún}. \\
\textsc{dem.unm} \textsc{cop.prs.3sg} pity.\textsc{m.sg-augm}\\
\glt `This is a real pity.' (Sadrún, m5)
\z

\ea
\label{ex:aum2}
\gll Sònda-dumèngja vagnévan quèls ò cò a fagjévan \textbf{fjast-unas}.\\
Saturday-Sunday come.\textsc{impf.3pl} \textsc{dem.m.pl} out here and do.\textsc{impf.3pl} party-\textsc{augm.f.pl}\\
\glt `On week-ends they would come here and have big parties.' (Sadrún, m4, \sectref{sec:8.3})
\z

Further examples are \textit{buébúna} `very tall girl', \textit{raubuna} `big assets', \textit{ùmún} `big man', and \textit{tgèsuna} 'big house'.

The other \isi{augmentative} is \textit{-az}, as in \textit{fòmaz} (> \textit{fòm} `hunger') (\ref{ex:aum3}).

\ea
\label{ex:aum3}
\gll Ju a \textbf{fòm-az}.\\
\textsc{1sg} have.\textsc{prs.1sg} hunger-\textsc{augm}\\
\glt `I am ravenous.' (Sadrún, m5)
\z

The suffix \textit{-ún/-una} is productive and can be added to almost all nouns; in contrast, the use of \textit{-az} is very reduced and seems to be restricted to \textit{fòmaz} in Tuatschin.

\subsubsection{Further nominal derivational morphemes}\label{sec:7.2.2.2}
The most common \isi{derivational suffix} in the corpus is -\textit{zjun}. It derives nouns from verbs like \textit{còmplicazjun} `complication' < \textit{cumplicar} `complicate'. In some cases -\textit{zjun} derives a \isi{noun} from a \isi{verb} that is not used or has another meaning in Tuatschin. An example is \textit{vòtazjun} `votation', which is derived from \ili{Standard Sursilvan} \textit{votar}, but the verb that is used for `vote' is \textit{vuṣchá}, in Tuatschin as well as in normal Sursilvan speech. Another example is \textit{tradizjun} `tradition', which is derived from \textit{tradí}, but \textit{tradí} means `betray' and not `transmit'. This means that some of the nouns that are derived by -\textit{zjun} are learned words.

Some more examples of nouns derived by -\textit{zjun} are \textit{afèczjun} `affection', \textit{confadarazjun} `confederation', \textit{dirèczjun} `direction', \textit{fòrmazjun} `formation', \textit{habitazjun} `appartment', \textit{munizjun} `munition', \textit{òbligazjun} `obligation', \textit{réaczjun} `reaction', and \textit{tussègazjun} `poisoning'.

Another suffix that derives nouns from verbs is the feminine ending of the \isi{past participle} \textit{-ada/-èda}: \textit{cuṣchinada} `mixture of food' (< \textit{cuṣchiná} `cook'), \textit{santupada} `meeting' (< \textit{santupá} `meet'), \textit{purṣchida} `offer' (<\textit{pòrṣcher} `offer'), \textit{satagljèda} `cut (to oneself)' (< \textit{satagljá} `cut oneself'), and \textit{scargjèda} `droving down the animals from the alps' (<\textit{scargè} `drove down'). A further example is \textit{antschata} `beginning', which is derived from \textit{antschajvar} `begin' the irregular participle of which is \textit{antschiat} (\textsc{unm}) / \textit{antschata} (\textsc{f}) `begun'. In the case of \textit{curnada} `push with the horns', the noun is derived from another noun, \textit{tgérn/còrns} `horn/horns'.

-\textit{ém} also derives nouns from verbs and emphasises the repetition of the action, as in \textit{samudargém} `constant torturing of oneself' (< \textit{samudargè} `torture oneself').

Note that in \textit{santupada}, \textit{satagljèda}, and \textit{samudargjém} the \isi{reflexive} prefix \textit{sa-} is maintained, which shows that it forms a tight unit with the verb.

The suffixes -\textit{dad}, \textit{-détgna/-tétgna}, -\textit{èzja}, and -\textit{ira} derive nouns from adjectives. Examples:

\begin{itemize}
	\item -\textit{dad}: \textit{paupradad} `poverty' (< \textit{paupra} `poor (\textsc{f.sg})), \textit{pussajvladad} `possibility' (< \textit{pussajvla} `possible' (\textsc{f.sg}))
	
	\item \textit{-adétgna/-tétgna}: \textit{gjuvantétgna} `youth' (< \textit{gjuvan} `young' \textsc{(m.sg})), \textit{marschadétgna} `laziness' (< \textit{marsch} `lazy' (\textsc{m.sg})) , \textit{stgiradétgna} `darkness' (< \textit{stgir} `dark' (\textsc{m.sg})), \textit{végljadétgna} `age' (< \textit{végl} `old' (\textsc{m.sg}))
	
	\item -\textit{èzja}: \textit{balèzja} `beauty' (< \textit{bials} `beautiful' (\textsc{m.sg.})), \textit{luṣchèzja} `proudness' (\textit{lùsch} `proud' \textsc{(m.sg})), \textit{scartèzja} `lack' (\textit{scart} `rare' (\textsc{m.sg}))
	
	\item -\textit{ira}: \textit{pupira} `poverty' (< \textit{paupar} `poor' \textsc{(m.sg})), and \textit{tupira} `stupidity' (< \textit{tup} `stupid' \textsc{(m.sg}))
\end{itemize}

The suffixes -\textit{am}, -\textit{èssar}\footnote{This suffix is calqued from German \textit{-wesen} `being, entity'.}, and -\textit{za} derive nouns from nouns. Examples are \textit{gaglinam} `flock of chickens' (< \textit{gaglina} `hen'), \textit{purèssar} `farming sector' (< \textit{pur} `farmer'), \textit{fòrèstalèssar} `forestry'(< \textit{fòrèstal} `forest ranger'), \textit{scòlarèssar} `school sector' (< \textit{scùla} `school'), and \textit{ufaunza} `childhood' (< \textit{ufaun} `child'). 

The suffix -\textit{aglja} derives nouns from verbs like in \textit{pjaglja} `wages' (< \textit{pijè} `pay') or from adjectives like in \textit{stgiraglja} `darkness' (< \textit{stgir} `dark'); in the case of \textit{panaglja} `butter tub', the derivation is not clear; it could be derived from \textit{pèn} `buttermilk' (\citet[777]{Decurtins2012}).

Prefixes are \textit{mal}- and \textit{anza-}/\textit{za}-.  \textit{mal}- usually modifies adjectives: \textit{malcuntjants} `unsatisfied', \textit{malsagidajvals} `ungainly', and \textit{malsagirs} `unsure', and \textit{anza-}/\textit{za}- (< \textit{ins sa} `one knows')  corresponds to English `any' or `some' with indefinite or other pronouns: \textit{zacù} `somehow', \textit{zacú} `somewhen', \textit{zanúa} `somewhere', \textit{anzatgéj}/\textit{zatgéj} `anything', and \textit{antzatgi}/\textit{zatgi} `anybody'.

