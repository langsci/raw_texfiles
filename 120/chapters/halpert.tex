\documentclass[output=paper,
modfonts
]{langscibook} 
% \bibliography{localbibliography}
\ChapterDOI{10.5281/zenodo.1251742}

\title{More on {have} and {need}}

\author{Claire Halpert\affiliation{University of Minnesota}\lastand Michael Diercks\affiliation{Pomona College}}


\usepackage{qtree}
\usepackage{tree-dvips}
% \newcommand{\node}[1]{{\color{red}#1}}

% \chapterDOI{} %will be filled in at production
% \epigram{}

\abstract{This paper addresses recent work on the cross-linguistic patterns involving {\it have} and {\it need} predicates, focusing on the debate surrounding the claim that all languages that lack a transitive {\it have} also lack transitive {\it need} \citep{Harves:2012}.  %with our additions in bold, along with those of \citet{Antonov:2014}. 
In this paper, we move the discussion beyond these surface patterns, first by presenting new syntactic diagnostics to demonstrate that the Bantu language counter-examples to the proposed generalization discussed by \citet{Antonov:2014} are true counter-examples to the original claim by \citet{Harves:2012}.  From this perspective, we evaluate the relevance of these conclusions for \citet{Harves:2012}'s lexical decomposition analysis of {\it need}.  %In particular, while we stand with \citet{Antonov:2014} in correcting the typological generalization, we show that 
We conclude that although Bantu languages form a straightforward counter-example to the proposed \citet{Harves:2012} typology, as \citet{Antonov:2014} noted, there are in fact some deep similarities between the Bantu patterns and the proposals of \citet{Harves:2012}.  In particular, we suggest that their observations about the role of case in the distribution of {\it have} and {\it need} verbs may in fact be amenable to the Bantu patterns, suggesting that their conclusions cannot yet be abandoned.}
% \keywords{Bantu languages, predicative possession, case, transitivity}


\begin{document}
\maketitle
\section{Overview of the Issues} \label{HKoverview}

%\todo{bib file missing in original submission}
 \citet{Harves:2012} survey a number of languages and propose an empirical generalization:  all languages that 
lack a lexical \isi{verb} of possession ({\it have}) likewise lack a
transitive lexical \isi{verb} {\it need}. Based on this apparent typological gap they
propose a lexical decomposition analysis of {\it need}. In response, \citet{Antonov:2014} provide a range of typological data showing
that the typological generalization that \citet{Harves:2012} rely on
is not in fact surface-true, a conclusion this paper supports. The chart in
\tabref{tab:halpert:newchart} summarizes both \citet{Harves:2012}'s original typological conclusions and the additions of \citet{Antonov:2014} and this paper, listed in bold. 



%\subsection{Introduction}



% We will demonstrate that the Bantu language
%counter-examples are true counter-examples, and evaluate the relevance
%of these conclusions for \citet{Harves:2012}'s proposals for a lexical decomposition
%analysis of {\it need}. We will show that despite the
%counter-examples, there are deep similarities between the Bantu
%patterns and the proposals of \citet{Harves:2012}, suggesting that their conclusions
%cannot yet be abandoned.





\begin{table} 
\begin{tabularx}{\textwidth}{p{1in}QQ}
\lsptoprule
& H-languages & B-languages\\
\midrule
Languages with transitive `need' & Czech, Slovak, Polish, Slovenian,
Croatian, Servian (dialects), Belorussian, \ili{English}, \ili{German}, Yiddish,
Luxemburgish, \ili{Dutch}, \ili{Swedish}, Norwegian, Icelandic, \ili{Spanish}, Catalan,
Basque, Paraguayan Guaran\'i, Pur\'epecha (Tarascan), Mapudungun &
\textbf{\ili{Zulu}, \ili{Setswana}, \ili{Kuria}, \ili{Swahili}, Otjiherero, Estonian, Moroccan
  \ili{Arabic}, Algerian \ili{Arabic}, Likpe, \ili{Ewe}, \ili{Ayacucho} Quechua}\\
&&\\
Languages without transitive `need' & \ili{Bulgarian}, Serbian (standard), Lithuanian, \ili{French}, \ili{Italian}, Bellinzonese, \ili{Portuguese}, Romanian, Farsi, \ili{Armenian}, Albanian, \ili{Latin}, Ancient \ili{Greek} & Russian, Latvian, Sakha, Bhojpuri, Bengali, \ili{Hindi}, Marathi, Irish, Welsh, Scots Gaelic, Georgian, \ili{Hungarian}, Turkish, Korean, Peruvian \ili{Quechua} (Cuzco, Cajamarca, Huallaga), Bolivian \ili{Quechua}, Yucatec \ili{Maya}, Tamil, \ili{Mohawk}, \ili{Amharic}\\
\lspbottomrule
\end{tabularx}

\caption{Revised typology of possession and need, with additions in bold}\label{tab:halpert:newchart} 
\end{table}

%The next section details the proposals of \citet{Harves:2012}, and
%provides counterexamples from five different Bantu languages.  We
%then focus on their discussion of how \ili{Finnish} necessitates a slightly
%more nuanced generalization than the one in (\ref{HKgeneralization}) 
%that is dependent on accusative case-licensing. In section
%\ref{\ili{Zulu}-Kuria}  we test the more nuanced generalization against
%additional  data from a subset of the languages %described above.
%Section \ref{Discuss}  then examines the questions raised for \citet{Harves:2012}'s
%analysis and point  toward some areas for future research with
%respect to the  Bantu examples. 

%I'm wondering, the paper might just be small enough to get rid of the
%roadmap, it's all very straightforward as you read through. ~MJKD

\subsection{An empirical correction to \citet{Harves:2012}}

\citet{Harves:2012} focus on what they claim is a significant typological gap in the cross-linguistic expression of possession and {\it need}, formalized in (\ref{HKgeneralization}):
%claiming that all languages with a transitive {\it need} \isi{verb} are also languages that express predicative possession via a {\it have} \isi{verb}:  
%CDH: I took this part out because we basically said it in the first paragraph and the generalization itself seems pretty transparent...


\begin{exe} 
\ex\label{HKgeneralization} \textbf{Harves-Kayne Generalization} (Strong version): \hfill \citep[1]{Harves:2012}\\
All languages that have a transitive \isi{verb} corresponding to {\it need} are H-languages.\hspace{1.75in}
\end{exe}
%\doublespacing

The gap in their data occurs when we compare languages that use a transitive \isi{verb} of possession, or H(ave)-languages, to languages that use a non-transitive strategy to express possession, or B(e)-languages.    While possession in H-languages looks straightforwardly
transitive, involving a nominative-accusative case pattern, possession
in B-languages  does not: in B-language possessors are typically
oblique, and possessees are nominative instead of accusative (unlike
possessees in H-languages).  H-languages may or may
not  have a transitive {\it need} \isi{verb},  but \citet{Harves:2012} crucially claim that B-languages
never do.% (as represented in table \ref{newchart}).

\newpage 
%\singlespacing
\begin{exe}
\ex \textbf{H-language with transitive need: Czech} \hfill \citep[4a, 5a]{Harves:2012}
\begin{xlist}
\ex \gll Maj\'i nov\'e auto.\\
have.3{\textsc{pl}} new car.\textsc{acc}\\
\glt `They have a new car.'
\ex\gll Tvoje d\v{e}ti t\v{e} pot\v{r}ebuj\'i.\\
your children.{\textsc{nom}} you.{\textsc{acc}} need.3\textsc{pl}\\
\glt `Your children need you.' 
\end{xlist}

\ex \textbf{H-language with non-transitive need: French} \hfill \citep[6a, 7a]{Harves:2012}

\begin{xlist}

\ex \gll J' ai une voiture.\\
I have.1{\textsc{sg}} a car\\
\glt `I have a car.'
\ex\gll J' ai besoin d' une voiture.\\
I have.1{\textsc{sg}} need of a car\\
\glt `I need a car.' \hfill 

\end{xlist}

\ex\textbf{B-language with non-transitive possession: Latvian} \hfill \citep[2b, 3c]{Harves:2012}

\begin{xlist}
\ex\gll Man ir velosip\=eds.\\
me.{\textsc{dat}} is bicycle.\textsc{nom}\\
\glt `I have a bicycle.'
\ex\gll Man vajag dak\v{s}u.\\
me.{\textsc{dat}} need.3{\textsc{sg}} fork.\textsc{gen}\\
\glt `I need a fork.' 

\end{xlist}

\end{exe}
%\doublespacing

\citet{Harves:2012} argue that this crucial gap -- the absence of B-languages with
transitive {\it need} -- follows directly from an \isi{incorporation}
account of transitive {\it need}: the derivation of the \isi{verb} {\it
  need} involves \isi{incorporation} of a nominal `need' into an
unpronounced (transitive, abstract) HAVE. Because `need' incorporates, it does
not require case \citep{Baker:1988c}, which allows HAVE to assign
accusative to the object. Languages that lack an overt {\it have} \isi{verb}
are assumed to lack abstract HAVE and are thus unable to do the necessary
\isi{incorporation} to create transitive {\it need}.  

%\singlespacing
\begin{exe} \label{NEEDtree}
\ex  \Tree  [.VP {N + V}\\{[\node{a}{\it need}_i + HAVE]} [.NP \node{b}t_i DP ] ]
\anodecurve[b]{b}[b]{a}{.25in}
\end{exe}

%\doublespacing

\newpage 
As noted by \citet{Antonov:2014}, a common pattern in Bantu languages contradicts the
generalization in (\ref{HKgeneralization}):  the following examples show 3 languages that  have transitive lexical
verbs for {\it need} %\footnote{These verbs are distinct from verbs of wanting/liking in the languages.} 
but construct predicative possession using a copular ({\it be})
construction followed by the \isi{preposition} {\it
  with}.\footnote{\citet{Antonov:2014} give parallel data to ours
  here in \ili{Swahili} and \ili{Zulu}.}%, along with a range of other languages outside of the Bantu family.}

%\singlespacing



\begin{exe}
\ex \textbf{\ili{Zulu}: {\it be-}possession and transitive {\it need}}
\begin{xlist}

\ex[]{\gll ngi- zo- ba ne- mali.\\
1\textsc{sg}- \textsc{fut}- be with.\textsc{aug}- 9money\\
\glt `I will have money.'}

\ex[]{\gll ngi- zo- dinga imali.\\   
1\textsc{sg}- \textsc{fut}- need \textsc{aug}.9money\\
\glt `I will need money.'}
\end{xlist}
\end{exe}

\begin{exe}
\ex \textbf{\ili{Swahili}: {\it be-}possession and transitive {\it need}}
\begin{xlist}

\ex[]{\gll ni- li- kuwa na nyumba.\\
1\textsc{sg}- \textsc{pst}- be with 9house\\
\glt `I had a house.'}

\ex[]{\gll ni- li- hitaji nyumba.\\
1\textsc{sg}- \textsc{pst}- need 9house\\
\glt `I needed a house.'}
\end{xlist}
\end{exe}

 
\begin{exe}
\ex \textbf{\ili{Kuria}: {\it be-}possession and transitive {\it need}} \label{KuriaBasic}
\begin{xlist}

\ex[]{\gll Gati n- a- a- re n- eng'ɔɔmbe.\\
1Gati \textsc{foc}- 1\textsc{s}- \textsc{rem}.\textsc{pst}- be with- 9cow\\
\glt `Gati had a cow.' (remote past)}

\ex[]{\gll Gati n- a- a- tun- ire eng'ɔɔmbe.\\
1Gati \textsc{foc}- 1\textsc{s}- \textsc{rem}.\textsc{pst}- need- {\textsc{rem}.\textsc{pst}} 9cow\\	
\glt `Gati needed a cow.' (remote past)}
\end{xlist}
\end{exe}

%\ili{Kuria}, \ili{Lubukusu}, and \ili{Shona} are all removed here bc they seem to allow
%their "need" verbs to mean "want" as well.  

%\begin{exe}
%\ex \textbf{\ili{Lubukusu}: {\it be-}possession and transitive {\it need}}
%\begin{xlist}
%
%\ex[]{\gll Wekesa a- a- ba ne(nde) enyungu.\\
%Wekesa 1\textsc{s}- \textsc{pst}- be with 9-pot\\
%\glt `Wekesa had a pot.'}
%
%\ex[]{\gll Wekesa e- enya enyungu.\\
%Wekesa 1\textsc{s}.\textsc{pst}- need 9-pot\\
%\glt `Wekesa needed a pot.'}
%\end{xlist}
%\end{exe}
%
%\begin{exe}
%\ex \textbf{\ili{Shona}: {\it be-}possession and transitive {\it need}}
%\begin{xlist}
%
%\ex[]{\gll Tendai a- i- va ne-sadza\\
%1Tendai 1\textsc{s}- \textsc{pst}- be with-sadza\\
%\glt `Tendai had (i.e. possessed) sadza.' (long ago)}
%
%\ex[]{\gll Tendai a- i- da sadza.\\
%1Tendai 1\textsc{s}- \textsc{pst}- need/want sadza\\
%\glt `Tendai needed/wanted/loved sadza.' (long ago)}
%\end{xlist}
%\end{exe}

%\doublespacing

In addition to the languages shown here,  initial evidence suggests that this pattern is well attested throughout the Bantu family.  Herero, for example,  expresses predicative possession using a {\it
    be (with)} construction, {\it na}, but has a transitive \isi{verb} of
  need, {\it hepa}, that is distinct from the \isi{verb} of wanting {\it
    vanga} \citep{Nguako:2013}.  \ili{Setswana} also uses a {\it
    be (with)} construction, {\it na (le)}, for predicative possession;\footnote{\citet{Creissels:2013} observes, though, that in \ili{Setswana} predicative possession patterns in some respects like a transitive \isi{verb}.  He remarks that this pattern is a departure from the general Bantu pattern in which predicative possession is completely indistinguishable from the comitative construction.}
  transitive {\it tlhoka} for `need'; and {\it batla} for `want'.
 % \citep[see][on the possessive predication]{Creissels:2013}. 
  We
  include these languages in table 1 on the basis of this preliminary
  evidence. Other languages, including \ili{Shona}, \ili{Lubukusu}, and Tiriki also lack a transitive \isi{verb}
  {\it have} and express `need' with a lexical \isi{verb};  these
  differ, however, in  that they seem to collapse {\it need} and  {\it want} (relying on circumlocutions in sentences contrasting `desiring' with
  `needing').\footnote{\ili{Kuria}, illustrated in (\ref{KuriaBasic}) and to which we do not return in this paper, appears at a glance to fall into this category: our consultant reports that the \isi{verb} {\it ugu-tuna} `{\sc{inf}}-need' in
   \ili{Kuria} can also have a reading of `want.' Despite this apparent lexical overlap,  %the consultant also reports that 
   it is
 possible to contrast {\it ugu-tuna} with an unambiguous  \isi{verb} of desire, {\it
   uku-igomba}, producing a  sentence like {\it Gati \textbf{naigombere} imburi,
   si bono \textbf{natunire} en'gombe}  `Gati desired a goat, but he needed a
 cow.' This kind of sentence would be unlikely if {\it ugu-tuna} was
lexically a \isi{verb} of `wanting' just as much as `needing' (cf. \ili{English}
\#{\it John desired a Porsche, but wanted a family sedan.}). We
suspect, therefore, that \ili{Kuria}'s {\it ugu-tuna} \isi{verb} is probably best
classified as a true `need' \isi{verb}, with metaphorical extensions to
notions of `wanting' (cf. \ili{English} {\it I need a beer right now}). On
this basis we include \ili{Kuria} in the languages added in \tabref{tab:halpert:newchart}. Due
to this complication, however, we restrict the core  examples
discussed in the paper to \ili{Swahili} and \ili{Zulu}.} %, which have clearer
%lexical distinctions between the verbs in question.} % (though it's worth noting
%that \ili{Kuria} behaves identically for all the relevant diagnostics).} 
 No Bantu language that we
  examined expressed predicative possession via a transitive {\it
    have}  \isi{verb}.


It is clear from these Bantu examples  that the
generalization in (\ref{HKgeneralization}) is not surface-true: these
languages all have lexical verbs for {\it need} but lack a
lexical \isi{verb} {\it have}. \citet{Antonov:2014} make this argument  based on data like these from \ili{Swahili} and \ili{Zulu}, as well as similar data from a typologically diverse set of
languages (including \ili{Arabic}, \ili{Quechua}, and Kwa languages). Our departure point is to
investigate the issue in more syntactic depth to determine whether
these apparent counter-examples hold up under further
investigation and, if so, what the consequences are for the \citet{Harves:2012}
analysis of {\it need}. 

We suggest in this paper that the resulting picture is more nuanced.  While the \ili{Swahili} and \ili{Zulu} patterns indeed constitute  true counter-examples to \citet{Harves:2012}'s generalization in
(\ref{HKgeneralization}), \citet{Harves:2012}'s revised generalization, discussed below in
(\ref{HKrevisedgeneralization}), and a more in-depth consideration of
structural licensing suggests that their core intuitions  may still have merit, at least with respect to the Bantu data. %not need to be abandoned, at least not on the basis of
%the Bantu data. 
This conclusion  contrasts with that of \citet{Antonov:2014}, who state on the basis of the 
typological evidence  that \citet{Harves:2012}'s
``hypothesis is thus unlikely to be valid as an absolute universal.''
While their conclusion may ultimately be correct, we suggest that a revised
conception of \citet{Harves:2012}'s relevant generalization based on the Bantu evidence
could potentially reveal a modified universal structural decomposition of {\it need}
verbs. This proposal makes useful predictions about the
structure of these predicates in the other languages in \citet{Antonov:2014}'s study, setting the stage for future investigation.
%that will can be investigated in the future. 

 
%%DO WE MAYBE WANT A SUB-SECTION BREAK HERE?

While \citet{Antonov:2014} establish a number of potential counter-examples to \citet{Harves:2012}'s proposed typology,\footnote{See \citet{Kayne:2014b}, though, for a discussion of some problems with the evidence \citet{Antonov:2014} give.}  \citet{Harves:2012} themselves discuss one language that does not straightforwardly
follow their generalization: \ili{Finnish} is canonically described as a
B-language but nonetheless has a transitive {\it need} \isi{verb} with a
\textsc{nom}-\textsc{acc}\ case pattern. \citet{Harves:2012} point out that while \ili{Finnish} uses the same
{\it be} \isi{verb} in existential, locative, predicational, and possessive
sentences, possessives differ from the other constructions in taking
an accusative -- rather than a nominative -- object:

\newpage 
%\singlespacing
\begin{exe}
\ex\label{finnish} \textbf{\ili{Finnish} predicational vs. possessive {\it be}} \hfill \citep[14c, 13]{Harves:2012}
\begin{xlist}
\ex\gll H\"an on vanha.\\
he.{\textsc{nom}}  be.3{\textsc{sg}} old.{\textsc{nom}}\\
\glt `He is old.' %\hfill \citet{Harves:2012} (14c)
\ex\gll Minu- lla on h\"ane-t.\\
I- \textsc{adess} be.3{\textsc{sg}} him-\textsc{acc}\\
\glt `I have him.' %\hfill \citet{Harves:2012} (13)
\end{xlist}
\end{exe}
%\doublespacing

\citet{Harves:2012} argue that the accusative case assignment in (\ref{finnish})
crucially distinguishes \ili{Finnish} from other B-languages: because \ili{Finnish}
expresses possession via an accusative-assigning (B)-\isi{verb}, {\it need}
may incorporate into accusative-assigning BE in this language to yield the transitive {\it
  need} pattern.  They thus revise their generalization to reflect the
importance of case-assignment patterns, as opposed to BE/HAVE distinctions:

%\singlespacing
\begin{exe}
\ex \label{HKrevisedgeneralization} \textbf{Harves-Kayne Generalization} (revised): \hfill \citep[15]{Harves:2012}\\
All languages that have a transitive \isi{verb} corresponding to {\it need}
are languages that have an accusative-case-assigning \isi{verb} of possession.%\hfill \citet{Harves:2012} (15)
\end{exe}
%\doublespacing

As we will see in the next section, even the revised 
generalization does not seem to capture the Bantu facts: the possessee
in \ili{Swahili} and \ili{Zulu} possessive predication does not behave like a normal
transitive direct object, but instead exhibits similar behavior to the
`objects' of copular, existential, and locative predication, which
also involve {\it be}.  In \sectref{RoleOfCase}, we return to an
aspect of the generalization in (\ref{HKrevisedgeneralization})
without a clear connection to the Bantu data -- case assignment
patterns.  We propose that the  the Bantu
exceptions to \citet{Harves:2012}'s generalization(s) may  in fact be linked to the exceptional behavior of Bantu languages with respect to syntactic case.  Based on recent work on case in Bantu \citep[e.g.][]{Diercks:2012,Halpert:2012}, we suggest that case-licensing of objects is independent of transitivity in these languages; transitive verbs and B-constructions have identical licensing properties.  Given this pattern, a version of (\ref{HKrevisedgeneralization}) that simply requires identical licensing properties between predicative possession and transitive verbs may be tenable.  %  we suggestproperties of the relevant Bantu languages.  

\section{The Bantu examples are true counter-examples} \label{Zulu-Swahili}

We have already established that the surface generalization in
(\ref{HKgeneralization}) cannot be upheld in the face of the patterns
in \ili{Swahili} and \ili{Zulu},\footnote{As well as the other Bantu languages discussed above. In addition, while we focus on \ili{Zulu} and \ili{Swahili} here, we note that \ili{Kuria} exhibits identical behavior on all relevant diagnostics.} which are both B-languages
that nonetheless have a lexical \isi{verb} {\it need}. % (and as noted above,
%evidence from Herero, \ili{Kuria}, and \ili{Setswana} suggests that
%they also follow this pattern). 
 In this section, we
demonstrate that possessees in these languages are not canonical
transitive objects, which rules out a \ili{Finnish}-style analysis for the
Bantu facts. %For reasons of space we will focus mainly on \ili{Zulu} and
%\ili{Swahili} here.\footnote{Though it is worth noting that is our anectodal understanding that
%  these patterns are relatively common across Bantu languages.} %, but the other languages surveyed all show similar patterns. 

As is common in the Bantu family, neither \ili{Zulu} nor \ili{Swahili} has overt case morphology, instead marking most grammatical relations on the \isi{verb}
itself via subject marking and object marking. This lack of case
morphology means that we cannot simply use nominal morphology to
evaluate  \citet{Harves:2012}'s generalizations. Instead, we focus on object marking
and A-bar \isi{extraction} as tests for transitive-object behavior. As  the
following patterns demonstrate,  possessees in \ili{Zulu} and \ili{Swahili} show
distinct properties from canonical  transitive objects, suggesting
that they are true counter-examples to  the generalizations in
(\ref{HKgeneralization})/(\ref{HKrevisedgeneralization}) and not
instances of covert canonical objects.

\subsection{Object markers for {\it need} and {\it have}}
Most Bantu languages can mark transitive objects on the \isi{verb} via a
morpheme  that precedes the stem and follows other inflectional
material (see \citealt{Riedel:2009,Marten:2007,Zeller:2012,Bax:2012}, a.o., for additional discussion). Abstracting
away from particular analyses of object markers, we instead take their availability to be a canonical property of transitive
objects. As \ili{Swahili} shows in (\ref{SwahiliOM}), {\it need}
uses the normal pre-stem OM to pronominalize an object, just as the transitive \isi{verb} {\it want} does, while
predicative possession requires an exceptional enclitic morpheme to
pro\isi{nominalize} an object. A pre-stem object marker (\ref{SwahiliOM}d) is ungrammatical.

%\singlespacing
\begin{exe}
\ex\label{SwahiliOM}\textbf{Swahili}\begin{xlist}

\ex \gll Gati a- li- \textbf{i}- taka.  \\
1Gati 1\textsc{s}- \textsc{pst}- 9\textsc{o}- want \\	
\glt `Gati wanted it (a house).' (remote past)

\ex \gll Gati a- li- \textbf{i}- hitaji. \hspace{1in} \\
1Gati 1\textsc{s}- \textsc{pst}- 9\textsc{o}- need \\	
\glt `Gati needed it (a house).' (remote past)

\ex \gll Gati a- li- kuwa na- \textbf{yo}.\\
1Gati 1\textsc{s}- \textsc{pst}- be with- 9\textsc{pronoun}\\
\glt `Gati had it (a house).' (remote past)
\ex[*]{\label{Swahili*OM}\gll Gati a- li- \textbf{i}- kuwa na- (yo)\\
1Gati 1\textsc{s}- \textsc{pst}- 9\textsc{o}- be with- 9\textsc{pronoun}\\}
\end{xlist}
\end{exe}
%\doublespacing

%As (\ref{\ili{Swahili}*OM}) shows, a pre-stem object marker in predicative possession is ungrammatical. 
%
%%\singlespacing
%\begin{exe}
%\ex[*]{\label{\ili{Swahili}*OM}\gll Gati a- li- \textbf{i}- kuwa na- (yo)\\
%1Gati 1\textsc{s}- \textsc{pst}- 9\textsc{o}- be with- 9\textsc{pronoun}\\}
%\end{exe}

The examples (\ref{ZuluOM})  illustrate the same pattern for \ili{Zulu}: 

\begin{exe}
\ex\label{ZuluOM}\textbf{Zulu} \begin{xlist}
\ex\gll ngi- zo- \textbf{yi}- funa. \\
1\textsc{sg}- \textsc{fut}- 9\textsc{o}- want\\
\glt `I will want it (money).'
\ex \gll ngi- zo- \textbf{yi}- dinga. \\
1\textsc{sg}- \textsc{fut}- 9\textsc{o}- need\\
\glt `I will need it (money).'

\ex \gll ngi- zo- ba na- \textbf{yo}.\\
1\textsc{sg}- \textsc{fut}- be with- 9\textsc{pronoun}\\
\glt `I will have it (money).'
\ex[*]{ \label{ZuluOM2}\gll ngi- zo- \textbf{yi}- ba na- (yo).\\
1\textsc{sg}- \textsc{fut}- 9\textsc{o}- be with (9\textsc{pronoun})\\}
\end{xlist}
\end{exe}


%\doublespacing

These contrasts show that the canonical object marking patterns that
are available for objects of transitive verbs are not available for
possessees in predicative possession for \ili{Swahili} and \ili{Zulu} (see \citealt{Antonov:2014} for similar discussion).  This pattern
suggests that the \ili{Swahili} and \ili{Zulu} counter-examples are not instances of
a transitive-like construction in disguise. 

\subsection{Object extraction for {\it need} and {\it have}}

Extraction patterns provide an additional argument for distinguishing
between possessee arguments and transitive objects. In \ili{Swahili}, for
example, the \isi{verb} {\it need} shows the same patterns for object
\isi{extraction} as transitive verbs:  an object operator can simply be
A'-moved to the left periphery. In contrast, such dislocation
in predicative possession requires a \isi{resumptive} \isi{clitic}: 

%\singlespacing
\begin{exe}
\ex \textbf{\ili{Swahili} object extraction} \begin{xlist}

\ex[]{\gll Ni- li- ona ki- tabu amba- cho Gati a- li- nunua.\\
1sg\textsc{s}- \textsc{pst}- see 7- book comp- 7{\textsc{rel}} 1Gati 1\textsc{s}- \textsc{pst}- buy \\
\glt `I saw the book that Gati bought.'}

\ex[]{\gll Ni- li- ona ki- tabu amba- cho Gati a- li- kuwa na- \textbf{cho}.\\
1sg\textsc{s}- \textsc{pst}- see 7- book comp- {7\textsc{rel}} 1Gati 1\textsc{s}- \textsc{pst}- be with- 7\textsc{pro.} \\
\glt `I saw the book that Gati had.'}

\end{xlist}
\end{exe}
%\doublespacing

The requirement of a \isi{resumptive} enclitic here is exceptional among
instances of object \isi{extraction} in \ili{Swahili}.  Notably,  it is not
exceptional for predicative possession in other Bantu languages. \ili{Zulu}
again shows the same patterns, distinguishing transitive object
\isi{extraction}, which requires object marking, from \isi{extraction} of a
possessee, which requires the enclitic: 

%\singlespacing
\begin{exe}
\ex \textbf{\ili{Zulu} object extraction}
\begin{xlist}
\ex[]{\gll y- imali- ni e- ngi- zo- \textbf{yi}- funa?\\
\textsc{cop}- \textsc{aug}.9money- what \textsc{rel}- 1\textsc{sg}- \textsc{fut}- 9\textsc{o}- want\\
\glt `How much money will I want?'}

\ex[]{\gll y- imali- ni e- ngi- zo- \textbf{yi}- dinga? \hspace{.75in} \\
\textsc{cop}- \textsc{aug}.9money- what \textsc{rel}- 1\textsc{sg}- \textsc{fut}- 9\textsc{o}- need\\
\glt `How much money will I need?'}%\footnote{http://www.mml.co.za/docs/FP\_Resources/Isi\ili{Zulu}-Mathematics-Grade-2-Workbook.pdf}}
\ex[]{\gll y- imali- ni e- ngi- zo- ba na- \textbf{yo}?\\
\textsc{cop}- \textsc{aug}.9money- what \textsc{rel}- 1\textsc{sg}- \textsc{fut}- be with- 9\textsc{pronoun}\\
\glt `How much money will I have?'}
\ex[*]{\gll y- imali- ni e- ngi- zo- yi- ba na- (yo)?\\
\textsc{cop}- \textsc{aug}.9money- what \textsc{rel}- 1\textsc{sg}- \textsc{fut}- 9\textsc{o}- be with- (9\textsc{pronoun})\\
\glt }
\end{xlist}
\end{exe}
%\doublespacing

%Again we have an instance where the possessee in predicative possession does not show the canonical properties of transitive objects, in this instance appearing with a \isi{resumptive} \isi{clitic}. 
If \ili{Swahili} and \ili{Zulu} possessive constructions were
transitive verbs disguised as B-constructions, we would expect
parallel behavior  between transitives and possessives, contrary to fact. 

In fact, a closer parallel to the \isi{extraction} properties of possessive constructions is \isi{extraction} from a prepositional
phrase, which also requires a \isi{resumptive} enclitic \isi{pronoun}: 

%\singlespacing
\protectedex{
\begin{exe}
\ex\label{Zulu-P-extraction}\textbf{\ili{Zulu} PP extraction}\begin{xlist}
\ex[]{\gll ubhuthi e- ngi- hamba na- \textbf{ye} (u)- ng- uSipho.\\
\textsc{aug}.1brother \textsc{rel}- 1\textsc{sg}- go with- 1\textsc{pronoun} (1\textsc{s})- \textsc{cop}- \textsc{aug}.1Sipho\\
\glt `The guy I'm going with is Sipho.' }

\ex[]{\gll ubhuthi e-ngi-khuluma nga- \textbf{ye} u- zo ba (ng)- umongameli.\\
\textsc{aug}.1br \textsc{rel}-1\textsc{sg}-speak \textsc{instr}- 1\textsc{pro} 1\textsc{s}- \textsc{fut}- be (\textsc{cop})- \textsc{aug}.1president\\
\glt `The guy who I'm talking about will be president.' }
\end{xlist}
\end{exe}
}

\begin{exe}
\ex\label{Swahili-P-extraction} \textbf{\ili{Swahili} PP extraction} \begin{xlist}
\ex[]{\gll mw- anafunzi ni- na- ye- enda na- \textbf{ye} ni Gati.\\
1- student 1\textsc{sg}- \textsc{pres}- 9\textsc{rel}- go with- 1\textsc{pronoun} is 1Gati\\
\glt `The student who I'm going with is Gati.' }

\ex[]{\gll m- tu ni- na- ye- zungumza na- \textbf{ye} a- ta- kuwa rais.\\
1- person 1\textsc{sg}- \textsc{pres}- 9\textsc{rel}- converse with- 1\textsc{pronoun}
1\textsc{s}- \textsc{fut}- be 9president\\\
\glt `The person who I'm talking to will be president.'}

\end{xlist}
\end{exe}



%\ex[]{\gll n- eng'we Boke a- a- sumachere na- we\\
%oc- 1who 1Boke 1\textsc{s}- \textsc{pst}- converse with- 1\\
%\glt `Who was Boke talking to/with?' \hfill [\ili{Kuria}]}
%\end{xlist}


%\doublespacing

In short, A'-\isi{extraction} in predicative possession patterns with
\isi{extraction} of obliques -- and not with \isi{extraction} of direct objects.
%What we see, then, is that the realization of enclitic forms in
%resumption in A'-\isi{extraction} is identical to the realization of \isi{clitic}
%forms in basic pronominalization. 
This pattern is consistent with an analysis of the possessive
constructions in \ili{Swahili} and \ili{Zulu} as a copula plus a prepositional
phrase, exactly what it appears to be on the surface.  This evidence
therefore supports the conclusion that \ili{Swahili} and \ili{Zulu} are true
B-languages (expressing possession via a basic copular construction),
and therefore true counter-examples to the
(\ref{HKgeneralization})/(\ref{HKrevisedgeneralization}) generalization.

\subsection{Predicative possession as a non-verbal  construction}

An additional parallel between predicative possession and other
copular constructions in Bantu is found in the distribution of the
{\it be} \isi{verb}.  The examples we have seen involve a {\it be} \isi{verb} plus
the \isi{preposition} (some version of {\it na} or {\it ne} in all the
languages considered here).  More generally, the full verbal form
appears only as needed to host overt tense morphology; in present
tense constructions, for example, we find a reduced structure with
only agreement and the \isi{preposition} in many languages, as \ili{Zulu} and \ili{Swahili} show:

%\singlespacing
\begin{exe}
\ex \begin{xlist}

\ex[]{\gll ngi- ne- mali. \hspace{2in} [\ili{Zulu}]\\
1\textsc{sg}- with.\textsc{aug}- 9money\\
\glt `I have money.'}

%\ex[]{\gll Gati n- a- na eng'ɔɔ mbe \hspace{1.5in} [\ili{Kuria}]\\
%1Gati foc- 1\textsc{s}- with 9cow\\
%\glt `Gati has a cow.'}

\ex[]{\gll ni- na nyumba.  \hspace{2.4in} [\ili{Swahili}]\\
1\textsc{sg}- with 9house\\
\glt `I have a house.'}

\end{xlist}
\end{exe}

%\doublespacing

This same pattern occurs in copular clauses, with the full copular
\isi{verb} (and inflection) only appearing  in non-present tenses, as the
examples above show in (\ref{Zulu-P-extraction}) for \ili{Zulu} and in
(\ref{Swahili-P-extraction}) for \ili{Swahili}. \citet{Buell:2013} provide a
detailed comparison of non-verbal predication in \ili{Zulu}, demonstrating
that possessive predication exhibits parallel behavior to other copular constructions, in line with what we have shown here. 

%%\singlespacing
%\begin{exe}
%\ex \label{KuriaCopula} \begin{xlist}
%\ex[]{\gll Gati n- umu- rimi  \hspace{2.5in} [\ili{Kuria}]\\
%1Gati foc- 1- farmer\\
%\glt `Gati is a farmer.'}
%\ex[]{\gll Gati n- a- a- re umu- rimi \\
%1Gati foc- 1\textsc{s}- \textsc{pst}- be 1- farmer\\
%\glt `Gati was a farmer.'}
%\end{xlist}
%\end{exe}
%%\doublespacing

%Alternately, we could go with the separate examples below: 
%%\singlespacing
%\begin{exe}
%\ex\begin{xlist}
%\ex\gll uMandla u- ng- umbulali \hfill [\ili{Zulu}]\\
%\textsc{aug}.1Mandla 1\textsc{sg}- \textsc{cop}- \textsc{aug}.1murderer\\
%\glt `Mandla is a murderer.'
%\ex\gll uma u- dubula, u- zo- ba (ng)- umbulali\\
%if 1\textsc{s}- shoot, 1\textsc{s}- \textsc{fut}- be (\textsc{cop}) \textsc{aug}.1murderer\\
%\glt `If he shoots, he will be a murderer.' 
%\end{xlist}
%
%
%\end{exe}
%%\doublespacing

\subsection{Intermediate summary}
	What we have seen in this section is that, based on evidence
        from both object marking and object \isi{extraction}, possessees in
        predicative possession constructions do not display canonical
        properties of transitive objects. Without  overt
        accusative case-marking in Bantu languages, these canonical
        object properties are the best means to examine
        whether the revised generalization in
        (\ref{HKrevisedgeneralization}) holds up in the face of the
        Bantu counter-examples.  The kind of
        exceptional copula behavior of the {\it be}-possessive in
        \ili{Finnish} does not extend to \ili{Swahili} and \ili{Zulu}, which appear to be
        truly copular-based constructions.  This conclusion was
        bolstered by the observation that the {\it be}-\isi{verb} in these
        contexts appears to pattern in normal ways for a copula, being
        null in the present tense.  These facts thus allow us to move beyond \citet{Antonov:2014}'s observation that Bantu languages form a surface counter-example to \citet{Harves:2012}'s generalization to show that the counter-examples hold even on deeper syntactic measures of transitivity and objecthood.
%While we only discuss the more detailed evidence for two languagesfor reasons of space, similar patterns occur in \ili{Swahili},\ili{Lubukusu}, and \ili{Shona}. %, yielding the same conclusions for all of these languages.  %As we illustrated in table \ref{newchart}, these languages thus fall in the typological gap that \citet{Harves:2012} argued was significant.
        
        %And as can be seen in example WHAT, then,
%        we have reproduced \citet{Harves:2012}'s chart laying out the
%        distribution of H-languages and B-languages with respect to
%        their expressions of {\it need}, with the addition of the
%        languages we considered in this paper.


%together with the previously-discussed fact that all of
%the Bantu languages reported here use transitive lexical verbs for {\it need},
%demonstrate  that both generalizations set forward by \citet{Harves:2012} do not hold
%up when the data set is extended to include Bantu
%languages.\footnote{The survey reported in \citet{Harves:2012} is
%  largely composed of Indo-European languages, with some other families represented, but
%  lacking any Niger-\isi{Congo} languages.} 

\section{The role of case} \label{RoleOfCase}
In the previous section, we concluded that Bantu languages like \ili{Zulu}
and \ili{Swahili} constitute a robust counter-example to \citet{Harves:2012}'s generalization
about {\it have} and {\it need}. As we saw in
(\ref{HKrevisedgeneralization}), however, the revised version of their
claim specifically refers to the case assignment properties of the
relevant predicates, with the idea that languages where possessees
receive \textsc{acc}\ have transitive {\it need} (that assigns \textsc{acc}\ to its
direct object). Given the Bantu counter-examples, we can draw one of
two possible conclusions.  First, we might conclude that \citet{Harves:2012}'s generalizations are empirically
  inaccurate and their resulting analysis of the decomposition of
  {\it need}  is therefore untenable.  A second alternative would be that  \citet{Harves:2012}'s revised generalization in
  (\ref{HKrevisedgeneralization}) is  on the right track, with
  the distribution of H- and B- languages relating to the the presence
  of transitive {\it need} based on the availability of
  Case-licensing.



This second alternative is not transparently correct:  the
surface forms show no evidence that objects of predicative possession and  {\it
  need} are Case-licensed identically in \ili{Zulu} and \ili{Swahili}. We noted in
the previous section that while Bantu languages show no obvious
morphological case marking on nominals, evidence from
structural diagnostics demonstrates  clear syntactic
distinctions between the objects of transitive predicates and
possessees in predicative possession. In the following subsections we nonetheless return to the issue of Case and what role its presence (or absence) might play in the Bantu possession pattern.

\subsection{Another test for {\it have} and {\it need}} \label{AugmentData}
As we saw above,  {\it need} in \ili{Swahili} and \ili{Zulu} always patterns
with transitive verbs -- and not with {\it have} (i.e. BE + P) constructions.
This raises a critical question, however: are {\it have} and {\it need} always syntactically different
in Bantu?  The short answer is {\it maybe not}: one morphosyntactic pattern in \ili{Zulu}, augment distribution, in fact suggests that both types of object are licensed in the same way.

\ili{Zulu} nouns are typically marked with an initial augment vowel that
appears before the \isi{noun class} prefix. This augment vowel can be
dropped on some indefinites\footnote{While nonveridical environments are typically necessary for augment drop, \citet{Halpert:2012} demonstrates that there are additional, independent syntactic conditions under which the process is licensed, on which we focus here.}  in certain syntactic positions -- in
particular, immediately after the \isi{verb} inside {\it v}P\
\citep{Halpert:2012}. We show in the data that follow that with
respect to augment drop, {\it have}, {\it need}, and transitive verbs
all behave in the same way in \ili{Zulu}. As the data in (\ref{augdrop1}) show, in
the relevant contexts (triggering NPI readings in these examples) an augment may be dropped on the highest DP
after a transitive \isi{verb}.  Unsurprisingly, {\it need} shows the same behavior in (\ref{augdrop2}).

\begin{exe}

%\singlespacing
\ex\label{augdrop1}\textbf{\ili{Zulu}: augment drop possible on highest DP after transitive verb}
\begin{xlist}
\ex \gll ngi- bona \textbf{u}-muntu.\\ 
1\textsc{sg}- see \textsc{aug}-1person\\
\glt `I see someone/the person.'
\ex \gll A- ngi- bon- i muntu.\\
\textsc{neg}- 1\textsc{sg}- see- \textsc{neg} 1person\\
\glt `I don't see anybody.' %/ *`I don't see the person.'


\end{xlist}

\ex\label{augdrop2}\textbf{\ili{Zulu}: augment drop possible on highest DP after {\it need}}
\begin{xlist}
\ex \gll ngi- dinga \textbf{i}-mali.\\ 
1\textsc{sg}- need \textsc{aug}-9money\\
\glt `I need money.'
\ex \gll A- ngi- ding- i mali.\\
\textsc{neg}- 1\textsc{sg}- need- \textsc{neg} 9money\\
\glt `I don't need any money.'

\end{xlist}
\end{exe}

%\doublespacing

What distinguishes this test from those in the previous section is that, as (\ref{augdrop3}) shows,  the possessee in predicative possession behaves like a transitive object, allowing the augment to be dropped
in the relevant syntactic contexts: 

%\singlespacing
\begin{exe}

\ex\label{augdrop3}\textbf{\ili{Zulu}: augment drop possible on possessee with predicative possession {\it na}}
\begin{xlist}
\ex \gll ngi-  \textbf{ne}-mali. \hfill (na+imali)\\ 
1\textsc{sg}- with.\textsc{aug}-9money\\
\glt `I have money.'
\ex \gll A- ngi- na mali.\\
\textsc{neg}- 1\textsc{sg}- with 9money\\
\glt `I don't have any money.'

\end{xlist}

\end{exe}
%\doublespacing

The behavior of the possessee in (\ref{augdrop3}) is not unique to the possessive constructure, however; rather, it seems to be a property of the {\it na} \isi{preposition} more generally that augment drop is permitted on its complement under the right structural conditions: 

%\newpage
%\singlespacing
\begin{exe}
\ex\label{P5}\textbf{\ili{Zulu}: \isi{preposition} {\it na} allows augment drop when highest element in {\it v}P}\begin{xlist}
\ex\gll  u-Mfundo u- dlala i-bhola \textbf{no}- muntu. \hfill (na+umuntu)\\
\textsc{aug}-1Mfundo 1\textsc{s}- play \textsc{aug}-5ball \textsc{na}.\textsc{aug}- 1person\\
\glt `Mfundo is playing soccer with someone/the person.'

\ex\gll u-Mfundo a- ka- dlal- i  \textbf{na}- \textbf{muntu} i-bhola.\\
\textsc{aug}-1Mfundo \textsc{neg}- 1\textsc{s}- play-  {\textsc{neg}} \textsc{na}- 1person \textsc{aug}-5ball\\
\glt `Mfundo isn't playing soccer with anyone.'

\ex\gll *u-Mfundo a- ka- dlal- i i-bhola na- muntu.\\
\textsc{aug}-1Mfundo \textsc{neg}- 1\textsc{s}- play-  {\textsc{neg}} \textsc{aug}-5ball \textsc{na}- 1person\\


\end{xlist}
\end{exe}
%\doublespacing
% \todo[inline]{jambox for (na+umuntu), and other cases in this paper?}


Crucially, the {\it na} \isi{preposition} contrasts with certain other prepositions in the language.  While  {\it na} PPs are essentially transparent with respect to the constraints on
augment drop, some prepositions do not alternate, instead always requiring the no-augment version regardless of position or interpretation, as shown 
for  {\it kwa-} and {\it ku-}  below:



%\singlespacing
\begin{exe}

\ex\label{P1}\textbf{\ili{Zulu}: prepositions {\it kwa-} and {\it ku-} prohibit augment on their complement}\begin{xlist}

%
%
%\ex\gll u-buyisele i-fowuni y-akho en-dala \textbf{kwa}-MTN Service Provider\\
%2\textsc{sg}-return.{\sjc} \textsc{aug}-9phone 9-your 9-old \textsc{kwa}-5MTN Service Provider\\
%\glt `Return your old phone to the MTN Service Provider.'\footnote{Google, accessed November 18, 2011.}



\ex\gll u-Sipho u- zo- pheka ukudla \textbf{kwa}- zingane/ *\textbf{kwe}- zingane.\\
\textsc{aug}-1Sipho 1\textsc{s}- \textsc{fut}- cook \textsc{aug}.15food \textsc{kwa}- 10child/ *\textsc{kwa}.\textsc{aug}- 10child\\
\glt `Sipho will cook food for the children.'

%\ex\gll u-Sipho u- zo- thum- ela imali \textbf{ku}- mama\\
%\textsc{aug}-1Sipho 1\textsc{s}- \textsc{fut}- send- {\textsc{appl}} \textsc{aug}.9money \textsc{ku}- 1mother\\
%\glt `Sipho will send money to mother'

\ex\gll  u-Sipho u- zo- thumela imali \textbf{ku}- bantwana.\\
\textsc{aug}-1Sipho 1\textsc{s}- \textsc{fut}- send.{\textsc{appl}} \textsc{aug}.9money \textsc{ku}- 2child\\
\glt `Sipho will send money to the children'

%
%\end{xlist}
%
%\end{exe}
%
%\begin{exe}
%
%\ex\label{P1x}\textbf{Class A: augment ungrammatical with {\it kwa-} and {\it ku-}}\begin{xlist}
%\ex\gll *u-Sipho u- zo- pheka inyama \textbf{kwe}- zingane\\
%\textsc{aug}-1Sipho 1\textsc{s}- \textsc{fut}- cook \textsc{aug}.9meat \textsc{kwa}.\textsc{aug}- 10child\\
%\glt intended: `Sipho will cook meat for the children.'\footnote{A version of this sentence with an augment would actually yield {\it kwa-bantwana}, making it indistinguishable from the \isi{preposition} in (24a).}

%\ex\gll  *u-Sipho u- zo- thumela imali \textbf{ko}- bantwana\\
%\textsc{aug}-1Sipho 1\textsc{s}- \textsc{fut}- send.{\textsc{appl}} \textsc{aug}.9money \textsc{ku}.\textsc{aug}- 2child\\
%\glt intended: `Sipho will send money to the children'
\end{xlist}
\end{exe}

%\doublespacing

\largerpage[2]
Recall that there is no \textsc{acc}-marking on transitive nominals
in \ili{Zulu} and that the complement of the \isi{preposition} in predicative
possession does not  behave like a transitive object in many ways
(triggering resumption under \isi{extraction}, different object marking
patterns). At the same time, we see here that the object of  the possessive \isi{preposition}
{\it does} share underlying similarities with the transitive objects
with respect to the distribution of augments. The apparent transparency of {\it na}\footnote{And a few other prepositions in \ili{Zulu}.} for the purposes of structurally-licensed augment-drop  is {\it not}
shared by all other prepositions in the language, which instead seem to simply replace the augment in all environments.  

To summarize, the augment drop patterns in \ili{Zulu} give us a test for possessees whose results diverge from those of the tests in the previous section, grouping the complements of {\it na} with transitive objects (and not with other prepositional complements).  This discussion becomes particularly relevant for our concerns in light of \citet{Halpert:2012}'s proposal that augment drop is only permitted in positions
where structural Case is assigned, as we discuss in the following subsection.
%\footnote{On this analysis,
%  the augment itself is an intrinsic Case licenser, along the lines of
%  Case-licensing prepositions in \ili{Zulu} and other languages. Augmented nouns therefore do not need to be licensed by a functional  projection in the \isi{clause}, while augmentless nouns do.}  
%  Interestingly, if this analysis
%is on the right track and augment drop is connected to Case (or
%something like Case-licensing), then the possessee in  predicative
%possession is in fact licensed just like to the transitive objects of
%{\it need} and other verbs, as we discuss in the following subsection.

\subsection{Case implications}

The discussion in this section concerns the role of Case Theory in the Bantu language family.  While we do not present a definitive account of Case in these languages, we show that the types of case-theoretic puzzles -- and their proposed solutions -- that emerge in the Bantu family suggest that \citet{Harves:2012}'s revised approach to {\it have} and {\it need} in (\ref{HKrevisedgeneralization}) may in fact be on the right track for these languages.


\subsubsection{Existing proposals about Bantu case}
%In this section we give a brief overview of some ongoing research on
%Case Theory in Bantu languages, not with the intention of offering a definitive solution
%for the issues at hand,  but instead to make the argument that the
%kinds of solutions that have been offered regarding Case-theoretic puzzles in other Bantu
%languages suggest  a solution to the current puzzle regarding {\it
%  have} and  {\it need}. 
  \citet{Diercks:2012}, building on a range of
research  \citep[e.g.][]{Harford:1985,Ndayiragije:1999,Alsina:2001,Baker:2003a,Baker:2008anti}, showed that a wide variety of
  constructions crosslinguistically among Bantu languages -- including raising constructions, locative inversion,
  and {\it possible}-constructions, among others -- do not behave in
  the familiar ways predicted by Case Theory, two examples of which
  are included below: the first shows a perception-\isi{verb} raising
  construction that is equivalent of the ungrammatical \ili{English} {\it
    *John seems that fell}, in which the embedded subject has raised out
  of a tensed and agreeing \isi{clause},  where it presumably should have
  been Case-licensed and  rendered inactive (known as hyper-raising). 

\begin{exe} \label{Hyper-raising}
\ex\textbf{\ili{Lubukusu} hyper-raising}
\exi{} {\gll John a- lolekhana mbo ka- a- kwa.  \\
1John 1\textsc{s}- seems that 1\textsc{s}- \textsc{pst}- fell \\
\glt `John seems like he fell/John seems to have fallen.'}
\end{exe}

The example in (\ref{PossibleConstruction}), on the other hand, shows a
\isi{noun} phrase appearing as subject of a non-finite \isi{clause} where there is
no evidence of a Case-licenser (overt or covert) to license it. 

\begin{exe} 
\ex \label{PossibleConstruction} \textbf{\ili{Swahili} overt subject of infinitive}
\exi{}{\gll I- na- wezakana (*kwa) Maiko ku- m- pig- i- a Tegani simu.  \\
9\textsc{s}- \textsc{pres}- possible (*for) 1Michael \sc{inf}- \textsc{o}- beat- \textsc{appl}- {\textsc{fv}} 1Tegan 9phone \\
\glt `It is possible for Michael to call Tegan.' }
\end{exe}

\largerpage
Diercks concluded that these patterns indicate that Bantu languages
simply lack abstract Case features, articulated in a macroparameter:
%for which most Bantu languages share the same property of {\it not} bearing Case features:

\begin{exe} 
\ex\textbf{Case Parameter:} Uninterpretable Case features are / \textbf{are not} 
  present in a language. 
\end{exe}

Such a proposal raises the question %in our current context, of course, that if
%Bantu languages simply lack case, then 
of what (if any) prediction 
\citet{Harves:2012}'s revised generalization makes about {\it have} and {\it need} in
languages without Case.  One possibility, discussed below, is that absent Case, \isi{incorporation} of {\it need} is unrestricted by the  absence of transitive HAVE.

Another approach to Case in Bantu emerged in the augment-drop discussion above.  As we saw,  
\citet{Halpert:2012} argues against parameterizing Case in Bantu,
attributing augment distribution patterns to Case-licensing.  Crucially, this Case-licensing system is distinct from
standard \textsc{nom}-\textsc{acc}\ licensing patterns: Halpert argues that augments and some
prepositions give abstract Case to the nominals they mark and that abstract case is assigned to the highest element in {\it v}P. While it is
unsurprising for prepositions to value Case features, the claim about \ili{Zulu} is that only certain prepositions do so (as illustrated in the previous section).  In addition, another surprising aspect of Halpert's proposal is that the augment, which is typically considered a DP-level prefix and not a \isi{preposition}, also licenses nominals.  Nominals without valuation in these ways are restricted to
structural Case positions, which again differ in a standard \textsc{nom}-\textsc{acc}\ language, where T
and {\it v$^o$}\ are Case-licensers. Halpert proposes that Case is
mediated by an  intermediate phrase (LP), which licenses downward to
the highest element in {\it v}P, accounting for patterns like those shown
in \sectref{AugmentData}.  The result of this analysis is that
\ili{Zulu} Case, unlike \textsc{acc}, is {\it not} connected to transitivity.

We do not attempt to resolve these differing approaches to Bantu Case here. Rather, we point out that the consistent thread  throughout all preceding work on this issue is that Bantu Case
is {\it not} business as usual.  Whether one adopts a no-Case approach or a non-\textsc{nom}/\textsc{acc}\ approach, we argue in the next section that both in fact predict a similar pattern with respect to \citet{Harves:2012}'s analysis of {\it have} and {\it need}.

\subsection{Restating the generalization}
\largerpage[2]

% whether it is that there is not Case
%in Bantu, or that Bantu Case is associated with unfamiliar positions
%and non-canonical sorts of licensing mechanisms.

We return now to the main problem that this Bantu data raises for \citet{Harves:2012}'s account: 
predicative possession shows non-transitive behavior, despite the
existence of transitive {\it need}. As discussed in the previous section, multiple proposals suggest that Case in Bantu is
divorced from transitivity -- either because there is no Case or
because nominals are licensed by a projection above {\it v}P that is  not linked to \isi{predicate} type. This consensus holds
even if we don't resolve the questions of Case-licensing in \ili{Swahili} or
\ili{Zulu} (or Bantu more broadly) here. 
  
We propose  that on either approach, the Case properties of transitive
objects and B-construction  possessees are identical: either neither has Case, or both do (say, from Halpert's LP). Either way, this Case
pattern is distinct from any traditional notion of accusative
Case but uniform across predicative possession and transitive
objects. The split that we demonstrated in section 2 between behavior of possessees and direct objects in syntactic tests for objecthood  is expected  because syntactic objecthood is divorced from structural case on either account. 
%contrasts with phrase structural diagnostics, of course,
%which showed that the possessee in predicative possession in \ili{Zulu} and
%\ili{Swahili} is clearly not the same as a transitive object. 
What we have
available to us, then, is a modification of \citet{Harves:2012}'s
revised generalization in (\ref{HKrevisedgeneralization}):



%\singlespacing
\begin{exe} 
\ex\label{NeedLicensingGeneralization} \textbf{{\it Need}-Licensing Generalization}:\\
All languages that have a transitive \isi{verb} corresponding to {\it need}
are languages in which 
%objects of verbs of possession -- I took this out because there's not really a ``\isi{verb} of possession'' in these Bantu languages...
predicative possessees are licensed in the same manner as transitive objects.
\end{exe}

%\doublespacing
Note that even if  (\ref{NeedLicensingGeneralization}) accurately
describes the current state of affairs, of course, it's not yet clear
why it should be the case. The next subsection briefly discusses some
ideas in this vein.% before we conclude in  \sectref{Conclude}. 


\subsection{Thoughts on the derivation of {\it need}}

Recall that for \citet{Harves:2012}, the role of \textsc{acc}-assignment is closely
tied to their proposed derivation of {\it need}: Transitive {\it need}
occurs when the theme of a nominal {\it need} can get \textsc{acc}\ Case.  If the nominal {\it
  need} incorporates to a transitive {\it have}, the \textsc{acc}\ from {\it have}
is available for its theme.\footnote{Under the
  assumption that incorporated nouns don't need Case, following
\citet{Baker:1988c}'s classic account: if transitive {\it have} is unattested elsewhere in the language, there's no base on which to build transitive {\it need}.}

\begin{exe} 
\ex \qtreecenterfalse  \Tree  [.VP {N + V}\\{[\node{a}{\it need}_i + HAVE]} [.NP \node{b}t_i DP ] ]
\anodecurve[b]{b}[b]{a}{.25in}
\end{exe}

We've argued, however, that transitivity has nothing to do with how
themes are licensed in Bantu. The question that arises, of course,
is: if our revised generalization regarding the relationship
between {\it have} and {\it need} holds, how does a Case-less
derivation of {\it need} fit into \citet{Harves:2012}'s story?  One possibility is that
transitive {\it need} can be built directly by incorporating the
nominal {\it need} into the (non-transitive) copular predicative
possession construction. In other words, \citet{Harves:2012}'s universal derivation for
transitive {\it need} breaks down for the Bantu languages discussed
here precisely because (accusative) Case-licensing is de-linked from
transitivity. Transitive {\it have} is unnecessary for deriving {\it
  need} because the transitivity of {\it have} or HAVE is irrelevant
for the licensing of the object of {\it need} due to the different
Case-licensing properties of these  languages. We suggest here,
therefore, that transitive verbs can be derived from non-transitive
components in this type of language if the incorporating nominal has its own theme: the theme can either be licensed by a
higher head independent of transitivity of predicates (following
Halpert 2012) or  does not need to be licensed at
all (Diercks 2012). 


\section{Conclusions} \label{Conclude}
\largerpage[2]
In this paper we have addressed recent discussion of the typological patterns surrounding the relationship between {\it have} and {\it need}.  Following \citet{Antonov:2014}, we have  shown that the typological generalizations
proposed by \citet{Harves:2012} do not hold up in the face of data
from a variety of Bantu languages. Specifically, \ili{Zulu}, \ili{Swahili} and \ili{Kuria} all are
B-languages with lexical {\it need} verbs, contrary to the proposed
generalization(s) of \citet{Harves:2012}.  We moved beyond the evidence in \citet{Antonov:2014}  to provide new syntactic tests that show that the \ili{Zulu} and \ili{Swahili} counter-examples 
are in fact true counter-examples: B-possession is non-transitive while {\it need} patterns with
other transitive verbs, ruling out the possibility that  these languages are somehow covert
H-languages. 

While \citet{Antonov:2014} conclude on the basis of similar evidence  that the \citet{Harves:2012} decompositional analysis of {\it need} is
therefore incorrect and ought to be abandoned, %do they actually say that, or just make the point that the generalization is incorrect?
we investigated a
potential alternative route. In light of recent research suggesting
deep differences between the properties of structural
Case in Bantu and those of the languages discussed in \citet{Harves:2012}'s original
survey, % which could potentially be the source of the contradiction to \citet{Harves:2012}'s generalization. Specifically, 
we proposed the revised
generalization in (\ref{NeedLicensingGeneralization}) that focuses not
on {\it have} and {\it need} both assigning \textsc{acc} case,  but instead
simply requires that {\it have} and {\it need} show the same
structural licensing properties. 

This proposal gives  us a new set of empirical predictions. \citet{Antonov:2014}
 discussed several additional languages
(Estonian, Moroccan and Algerian \ili{Arabic}; Likpe and \ili{Ewe} from the Kwa
family; and \ili{Ayacucho} \ili{Quechua}) that are surface counter-examples to
\citet{Harves:2012}'s generalizations.  If our proposal is on the right track, these languages ought to show similar licensing properties
between predicative possession and transitive {\it need}, even if they
are not transparently related to \textsc{acc} case on the surface. We see two
potential outcomes of such investigations. The first is the same
conclusion that \citet{Antonov:2014} arrive at: if there are not
underlying similarities in the licensing of objects of B-languages
with a transitive lexical {\it need}, then \citet{Harves:2012}'s generalization and our revised
generalization proposed here may simply be inaccurate.  If so,  both versions
should be abandoned, suggesting that we may not want a universal decomposition analysis of {\it need} after all. 

Alternatively, we may find that the other exceptions to \citet{Harves:2012}'s  generalization
noted by \citet{Antonov:2014} are in fact rooted in underlying
differences in structural licensing, as we have proposed for the Bantu languages discussed here. 
% between the apparently exceptional languages and those in \citet{Harves:2012}'s data set.  
If this second possibility is borne out, then we may stand to uncover a deeper universal that underlies \citet{Harves:2012}'s initial observations.  
%Such a result complicates the
%claim of whether there is  a ``universal'' at play:
%the generalizations and perhaps the decomposition of {\it need} as
%proposed by \citet{Harves:2012} are clearly {\it not} universal in the most transparent
%of senses of that word. Nonetheless, their proposals could in fact be
%rescued by investigating the structural licensing conditions in the
%relevant languages, with two results
In particular, if the predictions we discuss
here are upheld, then \citet{Harves:2012}'s generalizations (and our revision of
them) point to some deep consistencies between
languages (with  with respect to the decomposition of {\it need}) that can be obscured by  differences between languages with respect to structural
licensing patterns. It is possible that this
particular combination of traits that is problematic for \citet{Harves:2012} -- B-languages with transitive {\it need} -- could
ultimately be viewed as a diagnostic of underlying differences in structural
licensing between \citet{Harves:2012}'s languages and the `exceptional' ones. These are
of course empirical questions, meriting additional empirical
investigation, though with potentially large theoretical import.
 
 
% \section*{Abbreviations}
\section*{Acknowledgments}
 
 We would like to thank Johnes Kitololo,
    Justine Sikuku, Monwabisi Mhlophe, Mthuli Percival Buthelezi, and
    Tafadzwa Mtisi for lending their judgments to this project. Thanks
    also to Stephanie Harves,  Richard Kayne, and Rodrigo Ranero, the
    audience at ACAL 45, and two anonymous reviewers for their helpful comments and suggestions. %, and the editors/reviewers at LI squibs for their helpful comments  and critiques.


{\sloppy
\printbibliography[heading=subbibliography,notkeyword=this]
}
\end{document}
