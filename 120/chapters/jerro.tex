\documentclass[output=paper,
modfonts
]{langscibook} 
% \bibliography{localbibliography}
\ChapterDOI{10.5281/zenodo.1251708}

% \usepackage{phonetic}  
% \usepackage{qtree}
% \usepackage{textcomp}
% \usepackage{tree-dvips}
% \usepackage{stmaryrd}
% \usepackage{setspace}
% \usepackage{ upgreek }
% \usepackage{todonotes}
% \usepackage{langsci-gb4e}
 


 

\title{Linguistic complexity: A case study from Swahili } 

\author{Kyle Jerro\affiliation{University of Essex}}
 

\abstract{This paper addresses the question of linguistic complexity in Swahili, a Bantu language spoken in East and Central Africa. Literature on linguistic complexity in other languages has argued that high levels of second-language learning affect linguistic complexity over time. Swahili serves as an ideal case study for this question because it has been used as a lingua franca for several centuries. I compare the phonological and morphological systems in Swahili to five other related Bantu languages, as well as compare all six languages to the original Proto-Bantu systems. The results of the study show that there is no decrease in phonological or morphological complexity in (standard) Swahili when compared to other closely related Bantu languages, though the grammar has strongly diverged from the other related languages.\\}




%%%%%%%%%%%%%%%%%%%%%%%%%%%%%%%%%%%%%%%%%%%%%%%%%%%%%%%
\begin{document}
\maketitle
%%%%%%%%%%%%%%%%%%%%%%%%%%%%%%%%%%%%%%%%%%%%%%%%%%%%%%%

 
 
 


% \noindent \textbf{Keywords:} Language Contact, Linguistic Complexity, Bantu languages, \ili{Swahili}

\section{Introduction: the question of linguistic complexity}\label{sec:1:jerro}

It is generally assumed by linguists that all languages share the same level of complexity, with ``simpler'' areas of grammar being compensated by more complexity elsewhere. Some researchers take this as a core design feature of language (cf. work from the generative perspective, such as \citealt{Pinker1990,Pinker1994,Baker2003differences}), though this has tacitly pervaded most linguistic thought. 

Recently, however, work by various linguistic typologists has put this assumption into question, investigating several linguistic domains (see \citealt{Miestamo2008,Sampson2009} and \citealt{Givón2009} for overviews of the literature on complexity).  A core area of the research in this field is simply how to answer such a question \citep{Nichols2009, Sampson2009, MiestamoEd2008}. For example, \citet{Nichols2009} compares various features of languages, such size of phoneme inventory, number of inflectional categories on a basic \isi{verb}, number of alignments in a single language, etc. Other work situates linguistic complexity within a social context. One claim is that older languages tend to be more complex that new ones (e.g. Creoles), cf. \citet{McWhorter2008} and \citet{Trudgill2009}. Another claim is that population size relates to linguistic complexity \citep{Trudgill2004, Hay2007, Nichols2009}.

Another vein of this literature -- and the topic of this paper -- has investigated the interaction of complexity and \isi{language contact}, claiming that high amounts of second-language learning, including the use as a \isi{lingua franca}, affects linguistic complexity and increases the rate of language change (\citealt{Kusters2003phd,Kusters2003fate}; \citealt{Trudgill2009,McWhorter2008,McWhorter2011,Trudgill2011}). 
\citet{Trudgill2011} claims that that the specific effect on complexity is contingent upon the nature of second-language learning: while large amounts of second-language learning by adult speakers may result in net decomplexification, learning by children (e.g. through prolonged contact between two languages) may lead to \emph{increased} complexity. This paper tests the affects of \isi{language contact} on complexity in \ili{Swahili}, used as a \isi{lingua franca} throughout much of East and Central Africa. I compare Standard \ili{Swahili} to neighboring Bantu languages in their synchronic morphological and phonological features as well as their divergence from Proto-Bantu. 

% {\comment  flesh out the summaries of views just given}

To test this claim, I employ similar metrics of complexity to those used by Kusters and McWhorter (i.e morphology, see \sectref{sec:5:jerro}), comparing different aspects of \ili{Swahili} morphology to the grammar of five sister languages. In addition, I discuss the phonological inventories of the languages, a component absent from Kusters' and McWhorter's studies, but discussed at length by others \citep{Hay2007,Trudgill2011}. From the comparisons, I conclude that \ili{Swahili} does not exhibit any systematic decomplexification in comparison to the other languages, though it shows several grammatical differences from related languages. This situation is predicted from the framework proposed in \citet{Trudgill2011}, where long-term bilingualism (here, between \ili{Swahili} and \ili{Arabic}) may lead to the rapid change of a contact language. 
	
The remainder of this paper is organized as follows: in \sectref{sec:2:jerro}, I summarize the claims of the decomplexification hypothesis. I then outline the linguistic and sociolinguistic situations of five Bantu languages from East Africa chosen to serve as comparison cases. Sections \ref{sec:4:jerro}-\ref{sec:5:jerro} use phonological and morphological metrics, respectively, in order to compare the complexity of \ili{Swahili} to the comparison languages. Section \ref{sec:6:jerro} discusses the findings and their relation to the the decomplexification hypothesis. %Section 7 introduces a syntactic domain of complexity that has not been discussed in the literature on complexity.



\section{Contact and (de-)complexification}\label{sec:2:jerro}
%%%%%%%%%%%%%%%%%%%%%%%%%%%%%%%%%%%%%%%%%%%%%%%%%%%%%%%%%%%%%%%%%%%%%%%%%%%%%%%%%%%%%%%
In research on complexity, two opposite effects on complexity have been found, depending on the nature of the linguistic community. Languages in prolonged contact regions tend to develop high amounts of linguistic complexity (\citealt{Heine2005,Dahl2004,Givón1984}). On the other hand, situations with high numbers of sudden second-language learners result in simplification of linguistic structure. As discussed in \citet{Trudgill2011}, the crucial divide between the two groups is the critical period of \isi{language acquisition}: adult learners are not as adept as children at acquiring a (second) language. In a situation where adult speakers are acquiring a language, this ``sub-optimal acquisition" (a term from \citealt{Dahl2004}) results in the reduction of ornamental or non-obligatory elements of grammar. 

As \citet{Kusters2003fate} states, ``the more second-language learning has taken place in a speech community, the more internal dialect contact and migrations occurred, and the less prestige a language has, the more \emph{transparent} and \emph{economic} the verbal inflection will become" (275, emphasis in original). For Kusters, an inflectional system is more economic if it makes fewer category distinctions. In order to test the prediction of the decomplexification hypothesis, lingua francas that have been used by many second-language learners can be compared to sister languages or varieties that have not been used as lingua francas. 

\citet{Kusters2003fate,Kusters2003phd} provides several case studies in contact languages that have undergone decomplexification, tracing the changes from an older stage of the language to various modern sister languages. For example, one case study comes from three descendants of Old \ili{Norse}: Icelandic, \ili{Faroese}, and Standard Norwegian. He argues that the varieties that are more insular have maintained complexity that is absent in metropolitan varieties (i.e. the dialect of the capital city of the \ili{Faroese} Islands, Tórshavn). As an example,  consider the data in \tabref{tab:jerro:1}, with the \isi{verb} forms for the \isi{verb} `to awake' in Old \ili{Norse} and three descendant languages (\citealt[285, Table 5]{Kusters2003fate}). \\

\begin{table}
\caption{Verbal tense in Old Norse and descendant languages}

\label{tab:jerro:1}
\begin{tabular}{lllll}
\lsptoprule
& Old \ili{Norse} & Icelandic & \ili{Faroese} & Tórshavn\\
\midrule
 1sg & vakn-a 	& vakn-a 	& vakn-i & (-$'$) \\
 2sg & vakn-ar	& vakn-ar 	& vakn-ar	& (-$'$r)\\
 3sg & vakn-ar	& vakn-ar	& vakn-ar	& (-$'$r)\\
 1pl & vakn-um	& vökn-um	& vakn-a	&  (-$'$)\\ 
 2pl & vakn- i\_	& vakn-ið & vakn-a & (-$'$) \\
 3pl & vakn-a & vakn-a	& vakn-a & (-$'$)\\
\lspbottomrule
\end{tabular}
\end{table}
%
He argues that \ili{Faroese}, a variant that has been in prolonged contact with Danish, has reduced morphological complexity from the Old \ili{Norse}, and Tórshavn has undergone further reduction, having only stress as a indicator of tense. The only person marking is the marking of second- and third-singular, to the exclusion of all other persons and numbers. In addition, the Tórshavn dialect has completely neutralized certain inflectional categories, like past indicative and present subjunctive.

 
\citet{McWhorter2011,McWhorter2008} makes the stronger claim that second-language learning is the \emph{only} factor that drives overall simplification in a language. Namely, sweeping loss of complexity in a language is impossible without the influence of second-language learning. The argument works in the opposite direction from Kusters'; when you find an instance of decomplexification, it is predicted that this must have come from a situation of high second-language learning.
  McWhorter's metrics of complexity are similar to those of \citet{Kusters2003fate}. For example, in his 2008 paper, he compares two varieties of the Tetun language spoken in Timor. The first, Tetun Dili, is used as a \isi{lingua franca} by two-thirds of the island; the other, Tetun \ili{Terik}, is only spoken on the southern coastline. McWhorter predicts that because Tetun Dili is a \isi{lingua franca}, it has a simpler grammar than Tetun \ili{Terik}. He presents several instances where the Dili variety is more economical in the number of morphological categories it has. For example, while \ili{Terik} has three verbal affixes, Dili has two; Tetun has six numeral classifiers while Dili only has four (and those four are used optionally); Tetun has an overt marker for definiteness, while Dili uses context to indicate this; Tetun has three copulas, while Dili has only one; etc. In short, the variety that is used as a \isi{lingua franca} is systematically simpler than a sister variety without the same level of second-language use. 
  
 When two languages are in prolonged contact, and the acquirers of a \isi{second language} are mostly children, the opposite effect is found: over time, more complexity is found, often by the additive borrowing from the neighboring language. For example, \citet{Comrie2008} and \citet{Trudgill2011} cite the example of \ili{Michif}, a mixed language from contact between \ili{Cree} and \ili{French} \citep{Bakker1997}.  \ili{Michif}, from prolonged \isi{multilingualism} with \ili{French} and \ili{Cree}, developed an elaborate grammar, taking grammatical elements from both \ili{Cree} and \ili{French}, with verbal structure inherited from the former and nominal structure from the latter. The result is that \ili{Michif} employs elaborate verbal and morphological categories found in neither \ili{French} nor \ili{Cree}. 
 %concepts that are marked morphologically in related languages (such as definiteness and agentivity) are left to context. 
 
In short, work on contact and complexity has found three related effects of contact: first, \isi{language contact} increases the rate of language change; second, second-language learning by adults often leads to reduction in complexity via imperfect acquisition; and, third, prolonged contact between two languages often results in complexification as forms are taken from one and added into the other.  In this paper, I tease apart the level of complexity of standard \ili{Swahili}, comparing it to five related Bantu languages that have not had parallel situations of \isi{language contact}. 
 
\largerpage[-1]
\section{Swahili and the five comparison languages}\label{sec:3:jerro}
%%%%%%%%%%%%%%%%%%%%%%%%%%%%%%%%%%%%%%%%%%%%%%%%%%%%%%%%%%%%%%%%%%%%%%%%%%%%%%%%%%%%%%%

\ili{Swahili} serves as another ideal case study in fleshing out the claims of the decomplexification hypothesis. \ili{Swahili} is spoken as a native language along the Indian Ocean coast of \isi{Kenya} and \isi{Tanzania} and in the Zanzibari archipelago. It is also used as an official language and \isi{lingua franca} in \isi{Kenya}, \isi{Tanzania} and the Democratic Republic of the \isi{Congo} (DRC) in addition to a language of business and commerce at different points in history in Uganda, \isi{Rwanda}, and \isi{Burundi}. Because of this widespread use as a \isi{lingua franca}, nearly 140 million people use \ili{Swahili} as a \isi{second language}, while only 5 million speak it natively.  Given the overwhelming predominance of second-language speakers of the language, the decomplexification hypothesis predicts that \ili{Swahili} should be systematically less complex than related languages with little or no use by second-language speakers. 

 
  I have chosen five languages spoken in the countries where \ili{Swahili} is or has been routinely used as a \isi{lingua franca}. 
I have chosen one language from each country, and the languages are all part of the Northeastern branch of the Bantu family (with the exception of \ili{Lingala}).\footnote{A better comparison set may be languages that are more closely related to \ili{Swahili} genetically than the five chosen here. Accessibility to resources was a \isi{major} factor in linguistic choice, though the localization of these languages to East Africa is intentionally aimed at keeping to languages that are more similar to \ili{Swahili}.} The comparison languages are \ili{Gikuyu} (\isi{Kenya}, E.51), \ili{Lingala} (DRC, C.30B), \ili{Haya} (\isi{Tanzania}, JE.22), \ili{Kinyarwanda} (\isi{Rwanda}, DJ.61), and \ili{Luganda} (Uganda, JE.15).
  
	 
	
  \ili{Gikuyu} is spoken in Central \isi{Kenya} by the \ili{Gikuyu} people, numbering at approximately 7 million.  \ili{Lingala} is a language spoken by approximately 2 million people in the Republic of \isi{Congo}, the Democratic Republic of \isi{Congo}, and parts of the Central African Republic. \ili{Haya} is spoken in Northwestern \isi{Tanzania}, near the shores of Lake Victoria \citep{Byarushengo1977}. There are approximately 1 million speakers of the language. \ili{Luganda} is spoken by approximately 4 million people in Southern Uganda. Though used mostly by the Baganda people, it is also used as a \isi{second language} by approximately 1 million people in Uganda (Ethnologue 2013).
		   Although the use of \ili{Luganda} by second-language learners is not ideal as a comparison case in the current study, the situation of \ili{Luganda} is different from \ili{Swahili} in that the majority of speakers use \ili{Luganda} as a first language. \ili{Swahili} on the other hand, is used overwhelmingly as a \isi{second language}.
		 \ili{Kinyarwanda} is spoken by somewhere around 12 million people in \isi{Rwanda}, \isi{Burundi}, and parts of Uganda and DRC.
	 
\iffalse%%%%%%%%%%%%%%%%%%%
  All of these languages have been in some degree of contact with \ili{Swahili}:
	 
	  \ili{Haya}, \ili{Gikuyu}, and \ili{Lingala} are spoken in countries (\isi{Tanzania}, \isi{Kenya}, and D.R. \isi{Congo}, respectively) where bilingualism with \ili{Swahili} is very robust. 
	
	  In both \isi{Tanzania} and \isi{Kenya}, children are educated in \ili{Swahili} from first grade through the end of their high school education \citep{brock:2001}. 
	
	  \ili{Luganda} and \ili{Kinyarwanda}, on the other hand, have undergone some contact with \ili{Swahili} in their histories, especially in the early 1900s, but the use of \ili{Swahili} in these countries is becoming less and less common. 
	 
 
  It is important to note that it is not possible to rule out that any of these testbed languages has been used as a \isi{lingua franca}, or|as in the case of \ili{Luganda}|is still used by some speakers in this way. 

	 
	  \ili{Kinyarwanda} is spoken by multiple groups, and \ili{Lingala} is used across various national borders and as an official language in the Republic of \isi{Congo}. 
	
	  This suggests that both of these languages are either used currently or have been used in the past as lingua francas. 

	
	  The prediction of the  is that although many of these languages may have undergone some degree of decomplexification, \ili{Swahili} is expected to be the most simplified.
	 
		$\to$ The crucial distinction between \ili{Swahili} and all the other languages is that \ili{Swahili} is used almost|but not quite|exclusively as a \isi{second language}.
		 \fi


\section{Phonological complexity}\label{sec:4:jerro}
%%%%%%%%%%%%%%%%%%%%%%%%%%%%%%%%%%%%%%%%%%%%%%%%%%%%%%%%%%%%%%%%%%%%%%%%%%%%%%%%%%%%%%%
The first metric I use to compare the relative complexity among these languages is their phonological inventories. Phonological complexity did not figure in Kusters' and McWhorter's discussions, though several other works have used phonological inventory as a metric for calculating complexity \citep{Hay2007,Nichols2009}. The decomplexification hypothesis as outlined above predicts that \ili{Swahili} will have the smallest inventory of phonemes; over time, imperfect learning by second-language speakers would result in the reduction of phoneme contrasts not found in their first languages. Over time, this reduced vowel inventory becomes the standard inventory of the language.

\subsection{Vowel complexity}

\subsubsection{Vowel inventory}
Bantu languages generally have between five and seven vowels in their inventory, and they generally include tonal and length distinctions \citep{Hyman2003, Maddieson2003}. Proto-Bantu has been reconstructed to have seven vowels with high and low tone contrasts. \tabref{tab:jerro:vocales} indicates the number of different vowels (based on quality) in each of the languages in the test set as well as  whether each language makes a distinction between long and short vowels and between tones.  \\

\begin{table} 
\caption{Size of vowel inventories }

\label{tab:jerro:vocales}
\label{tab:jerro:2}
\begin{tabular}{lllll}
\lsptoprule
Language	& Vowels 	& Tone & Length & Source\\\midrule
Proto-Bantu				& 7						& +				& +	& \citet{Maddieson2003}\\
\ili{Swahili}					& 5						& --				& --				& \citet{Ashton1966}\\
Gikyuyu					& 7						& +				& +			& \citet{Barlow1960} \\
\ili{Lingala}					& 7						& +				& +			& \citet{Guthrie1966}\\
\ili{Haya}							& 5					& +				& 	+	& \citet{Byarushengo1977} \\
\ili{Kinyarwanda}				& 5						& +				& + 		& \citet{Myers2006}\\
\ili{Luganda}					& 5						& +				& + 	& \citet{Kirwan1951}\\
\lspbottomrule

\end{tabular}

\end{table}
 
 
\noindent Numerically, \ili{Swahili} has a simpler vowel inventory than the other languages; it has two fewer vowels than Proto-Bantu. Furthermore, \ili{Swahili} has lost the tone and length contrasts in Proto-Bantu, while the other languages have retained these features. This is the kind of inventory reduction expected by the decomplexification hypothesis.
   
   %However, various Bantu languages outside this test set have also lost tone, such as \ili{Tumbuka} and \ili{Pogolo} \citep{kissable:2003}.
 
   %I return to the discussion of vowel length and vowel hiatus in the next section. 
 
%{\comment Read Scott's paper on the typology of phonological complexity} 
 
 \subsubsection{Other kinds of vowel complexity}%%%%%%%%%%%%%%%%%%%%%%%%%%%%%%%%%%%%%%%%%%%%%%%%%%%%%%
 Although the size of vowel inventories indicates a lower level of complexity in \ili{Swahili}, another possible metric is linguistic markedness (cf. \citealt{McWhorter2008, McWhorter2011}). \ili{Swahili}, unlike its sister languages, shows three linguistically marked phonological processes that are absent in the other languages. These processes include the permission of syllabic consonants, an irregular stress system, and vowel hiatus. Unlike a numerical metric like phoneme inventory, however, phonological operations in a language are not as easily quantifiable. However, I argue here that the quantitatively fewer phonemic vowel contrasts in \ili{Swahili} are counteracted by the complexity that ensues with respect to its vowel system. 

 First, \ili{Swahili} has syllabic nasal consonants \citep{Ashton1966}. This is present on words such as \emph{mtoto} [m.toto] `child,' \emph{mtu} [m.tu] `person,' and \emph{mlango} [m.lango] `door.' Of the sister languages, only \ili{Haya} permits syllabic consonants; all maintain a minimal (C)CV syllable structure (cf. the cited grammars). Interestingly, \citet{Hyman2003} assumes this is a natural change, derived from the loss of [u] in \emph{mu-} nominal prefixes.

 A further noteworthy difference between \ili{Swahili} to the exclusion of the other languages is that \ili{Swahili} permits vowel hiatus, with juxtaposed vowels serving as nuclei of separate syllables. For example, \emph{chui} `leopard' is syllabified as [\textipa{tSu.i}], and \emph{paa} `gazelle' as [pa.a]. The other languages do not permit vowel hiatus; \ili{Kinyarwanda}, for example, deletes one of any two adjacent vowels, even between word boundaries. For example, the sentence \emph{uri umwana} `you are a child' is pronounced [\textipa{u.ru.mNa.na}], with the word-final [i] in \emph{uri} being deleted. 
	
	
 Finally, unlike the other languages of the study, \ili{Swahili} has several cases of irregular lexical stress.\footnote{Thanks to Scott Myers for suggesting this point.}  In most Bantu languages, stress falls on the penultimate syllable. In \ili{Swahili}, however, there are cases where \ili{Arabic} loanwords carry stress on the antepenultimate syllable, in words such as \emph{nusura} [\textipa{\textprimstress nu.su.ra}] `almost,' \emph{ratili} [\textipa{\textprimstress ra.ti.li}] `pound,' and \emph{thumuni} [\textipa{\textprimstress t\super hu.mu.ni}] `an eighth' \citep{Ashton1966}. Here, contact with \ili{Arabic} is the obvious influence of the complexification of the \ili{Swahili} stress system. 

 These three examples show that despite the smaller phonemic inventory, \ili{Swahili} has elements of complexity that are absent in the other languages. These features, however, are difficult to quantify, and their inclusion in metrics of complexity vary. My conclusion from the data in this section is that there is no clear reduction in complexity in the vowel system of \ili{Swahili}.  

 \subsection{Consonant inventory}%%%%%%%%%%%%%%%%%%%%%%%%%%%%%%%%%%%

  


Although the number of vowels in \ili{Swahili} is quite low, the consonant inventory is noticeably larger than the inventories of the comparison languages.\footnote{The inventories in \tabref{tab:jerro:3} come from the same sources as in \tabref{tab:jerro:2}, save for the number for Proto-Bantu, which comes from \citet{Hyman2003}.}\\


\begin{table} 
\caption{ Size of consonant inventories}

\label{tab:jerro:3}
\label{conso}
\begin{tabular}{ll} 
\lsptoprule
 Language &   Consonants\\
\midrule
 Proto-Bantu				& 11\\
\ili{Swahili}					& 30 		\\
\ili{Gikuyu}					& 14	\\
\ili{Lingala}					& 15\\
\ili{Haya}						& 19\\
\ili{Kinyarwanda}				& 22\\
  \ili{Luganda}					& 18\\
\lspbottomrule
\end{tabular}
\end{table}
%

\noindent The consonant inventory in \ili{Swahili} is striking larger than the other languages under discussion, being over two times larger than the consonant inventory of \ili{Gikuyu} and Proto-Bantu.\footnote{Nasalized consonants were not counted for any of the languages, as the descriptions of them were not satisfactorily convincing that these were indeed separate phonemes. The inclusion of these sounds in the data would not affect the trend, however, since they are also a class of sounds reported in \ili{Swahili}.} The larger inventory in \ili{Swahili} comes in part from having both voiced and voiceless stops and \isi{fricatives} for bilabial, \isi{alveolar}, and \isi{velar} places of articulation. Many languages lack a subset of these sounds, often having only the voiced or voiceless counterpart. \ili{Gikuyu}, for example, lacks the voiceless bilabial stop, the voiceless \isi{velar} \isi{fricative}, and the voiced \isi{alveolar} \isi{fricative} that are found in \ili{Swahili}. 
	
  A further difference is that \ili{Swahili} is the only language in the group with the aspirated stops and \isi{fricatives} [ p\super h t\super h \textipa{tS\super h} k\super h ] \citep{Ashton1966,Engstrand1985}. Aspiration is also found in various other Bantu languages, such as \ili{Zulu}, Swati, Makua, Doko, Chiche\^wa, and \ili{Kongo}. It has been argued that aspiration is a possible outgrowth of a consonant followed by the Proto-Bantu high vowels \citep{Hyman2003} or from an earlier voiceless pre nasalized stop \citep{Maddieson2003}. Regardless of the origin of phonemic aspiration, the presence of aspiration results in a notable increase in the phonemic inventory of \ili{Swahili}, resulting in a larger inventory than the comparison languages, as well as an innovation since Proto-Bantu.
	%=====> at any rate this is a normal thing for them to do. 

	
  Another interesting feature of the \ili{Swahili} consonant system is that all voiced stops are \isi{implosives}. \ili{Swahili} has four of these phonemes: [ \textipa{\!b \!d \!j \!g} ]. Implosive stops are not found in any of the comparison languages from East Africa, though \isi{implosive} stops are documented in the southern Bantu languages, with \citet{Maddieson2003} treating \isi{implosives} in the Bantu family as a natural development in some daughter languages.

\subsection{Discussion} 
The decomplexification hypothesis predicts that \ili{Swahili} should have a noticeably smaller phoneme inventory than the comparison languages. Although this is true with vowel inventory, the consonant inventory in \ili{Swahili} is markedly larger than any of the other comparison languages. Importantly, the \ili{Swahili} consonant system is nearly three times larger than in Proto-Bantu, suggesting considerable innovation during the evolution of \ili{Swahili}. 
  
 


  
\section{Morphological complexity}\label{sec:5:jerro}
%%%%%%%%%%%%%%%%%%%%%%%%%%%%%%%%%%%%%%%%%%%%%%%%%%%%%%%%%%%%%%%%%%%%%%%%%%%%%%%%%%%%%%%

The next domain of investigation is the morphological (dis)similarity between \ili{Swahili} and the other Bantu languages. If the decomplexification hypothesis is correct, it is expected that \ili{Swahili} will make fewer distinctions and that morphemes will be more phonologically reduced than the other languages. I investigate the domains of \isi{noun class} morphology, valency-changing morphology, and tense/aspect/mood morphology, which are all three morphological domains that are found in each of the languages. 

\subsection{Gender classes on nominals}%%%%%%%%%%%%%%%%%%%%%%%%%%%%%%%%%%%%%%%%%%%%%%%%%%%%%%%%%%%%%%%%%%%%%%%%%%%%%%%%%%%%%%%%%%%%%%%%%%%%%%%%%%%%%

 Bantu languages are well known for their rich \isi{noun class} morphology. The \isi{noun} classes for \ili{Swahili}, \ili{Haya}, \ili{Kinyarwanda}, \ili{Luganda}, and \ili{Lingala} are provided in \tabref{tab:jerro:4}, as well as the reconstructions of the Proto-Bantu inventory \citep{Meeussen1967,Schadeberg2003derivation}.\footnote{The source for \ili{Gikuyu} did not include enough detail for this comparison. The sources for the modern languages in \tabref{tab:jerro:4} are: \ili{Swahili} \citep{Ashton1966}, \ili{Haya} \citep{Byarushengo1977}, \ili{Kinyarwanda} (kinyarwanda.net), \ili{Luganda} \citep{Kirwan1951}, and \ili{Lingala} \citep{GuthrieCarrington1988}.} Given then decomplexification hypothesis, it is expected that \ili{Swahili} should be more economic in its morphological forms, either in the phonological shape of the morphemes or in the number of semantic distinctions.


\begin{table} 
\caption{Comparison of noun class morphology }

\label{tab:jerro:4}

\begin{tabular}{lcccccc} 
\lsptoprule
 Class &  \ili{Swahili} &   \ili{Haya}  &  \ili{Kinyarwanda} &  \ili{Luganda} &  \ili{Lingala} &  PB \\\midrule
1 	& m(u)-	& mu- 	& umu-	& (o)mu- 	& mo- & *mu-	\\
2	& wa-	& ba-	& aba-	& (a)ba-		& ba- & *ba-\\
3 	& m(u)-	& mu-	& umu- & (o)mu- 	& mo- & *mu- \\
4 	& mi-	& mi-	& imi- & (e)mi-		& mi- & *mi-	\\
5 	& ji-	& li-	& iri- & li-, eri-	& li- &  *\textipa{\|)i}-\\
6 	& ma-	& ma-	& ama- & (a)ma-		& ma- & *ma-\\
7  	& ki- 	& ki-	& iki- & (e)ki-		& e- & *ki-\\
8 	& vi- 	& bi-	& ibi- & (e)bi-		& bi- & *b\textipa{\|)i}-\\
9	& n-	& n-	& i(n)- & (e)n-		& N- & *n-\\
10	& n-	& n-	& i(n)- & (e)n-		& N- & *n-\\
11	& u-	& lu-	& uru- & (o)lu-		& lo- & *du-\\
12	& n-	& ka-	& aka- & (a)ka-		& bo- & *ka-\\
13	& -		& tu-	& utu- & (o)tu-	& -	& *tu-\\
14	& - 	& bu-	& ubu- & (o)bu-		& bo- & *bu-\\
15	& ku-		& ku-	& uku- & (o)ku- 	& ko- & *ku-\\
16	& pa-		& -		& aha- & wa- 	& - 	& *pa-\\
17 	& ku-	& -		& -		& ku-	& - & *ku-\\
18 	& mu-	& -		& -		& mu-	& - & *mu-\\	
19	& -	&	-		& -		& -		&  - & *p\textipa{\|)i}-\\
20 	& -	& 	-		& 	-		& (o)gu-		& - &-\\
21	& - & - & - & - & - & - \\

22	&	- & 	-		& -		& (a)ga-		& - & -\\
23	& -		& -			& 	-		& 	e-	& - & *i-\\\midrule
	& 16 	& 15 	& 16	& 21 & 14 & 21 \\\lspbottomrule


\end{tabular}
\end{table}
%
% 
% %1. inventory: similar across the board. 
% %2. swahili doesn't have pre prefix, but that's not a loss; just a fact (and perhaps compensated by the OM)
% %3. 
% 
% 
\ili{Swahili} has a comparable number of category distinctions to the other languages; although it is reduced from Proto-Bantu, only one of the other languages retains the number of category distinctions found in Proto-Bantu (i.e. \ili{Luganda}). Clearly, the prediction that \ili{Swahili} exhibit a noteworthy reduction in the number category distinctions is not borne out in this comparison. 			

As for the phonological shape of the morphemes, \ili{Swahili} lacks the pre-prefix that is found in \ili{Luganda} and \isi{Rwanda}. At a first glance, this could be argued to be an instance of phonological reduction in \ili{Swahili}. However, it has been argued in the literature that these pre-prefixes were not present in Proto-Bantu \citep{Katamba2003}, suggesting that the pre-prefix in languages that have it is an innovation. 

Support for this point is that the use of the pre-prefix varies drastically in the languages which use it. In \ili{Luganda}, a variety of features converge to predict the presence of the pre-prefix, such as whether the \isi{noun} is a dependent or main \isi{clause}, appears in the affirmative or negative, etc. \citep{Hyman1991,Hyman1993}. In \ili{Zulu}, it has been argued that the pre prefix is a case marker for nominals that lack structural case \citep{Halpert2012}.  \citet{ZerbianKrifka2008} show that features such as genericity, specificity, and definiteness are present in various languages which utilize the pre-prefix, such as \ili{Xhosa}, Bemba, and \ili{Kinande}. Crucially, it is assumed that the pre-prefix is a later innovation from Proto-Bantu, perhaps being a reanalysis of cliticized pronouns onto the main \isi{noun} \citep{Bleek1869}. 
		 
	The lack of a pre-prefix in the Proto-Bantu stems, as well as the semantic nature of pre-prefixes in the languages which have them, suggests that the reduced phonological shape of class morphology in \ili{Swahili} is not driven by phonological reduction due to second-language learning. Instead, \ili{Swahili} has retained the original shape of Proto-Bantu stems. 
	  
	 
	

  


 

 \subsection{Valency-changing morphology}%%%%%%%%%%%%%%%%%%%%%%%%%%%%%%%%%%%%%%%%%%%%%%%%%%%%%%%%%%%%%%%%%%%%%%%%%%%%%%%%%%%%%%%%%%%%%%%%%%%%%%%%%%%
 
  
Bantu languages utilize morphology to indicate valency changes to the argument structure of a \isi{verb}.  Both argument-adding (applicatives and \isi{causatives}) and argument-redu\-cing (stative, reciprocal, \isi{passive}) morphology is employed by these languages. If the decomplexification hypothesis is correct, it is expected that valency-changing morphology in \ili{Swahili} is simpler than in the comparison languages -- be it phonologically reduced or with fewer morphological category distinctions. 

\tabref{tab:jerro:5} gives the morphological forms for different valency-changing morphology in \ili{Swahili} \citep{Russell2003}, \ili{Lingala} \citep{Guthrie1966}, \ili{Kinyarwanda}\footnote{Those familiar with \citet{Kimenyi1980} will notice that the locative \isi{applicative} morpheme for \ili{Kinyarwanda} in \tabref{tab:jerro:5} differs from Kimenyi's description. \citet{Jerro2015} describes a different locative \isi{applicative} form for his speakers, who find Kimenyi's locative applicatives ungrammatical.}  \citep{Jerro2015}, \ili{Haya} \citep{Byarushengo1977}, and the reconstructed forms in Proto-Bantu \citep{Schadeberg2003historical}.\footnote{The resources for \ili{Gikuyu} and \ili{Luganda} do not explicitly discuss valency-changing morphology.}  
 

\begin{table} 
\caption{Comparison of valency-changing morphology}

\label{tab:jerro:5}

  \begin{tabular}{llllll} 
\lsptoprule
  Type &  \ili{Swahili}  & { Lingala}  & { Kinyarwanda} & { Haya} & { PB} \\\midrule
  Benefactive	& -(l)e	/-(l)i		& 	-el				& -ir/-er			& -il/-el & *-\textipa{Il}\\
  Instrumental 	& -(l)e	/-(l)i					&	-				& -ish/-esh		& -is/-es & *--\textipa{Il} \\
  Locative	& -(l)e	/-(l)i					& -					& -ir/-er					& -il/-el & *-\textipa{Il}\\
  Causative		& -ish/-esh		&	-is				& -ish/-esh		& -is/-es & *-i/-ici\\\midrule
  Stative			& -ik/-ek			& 	-an				& -ik/-ek			& -ek & *-\textipa{Ik}\\
  Reciprocal		& -an				& 	-an				& -an				& -\textipa{aNgan} & *-an \\
  Passive			& -(li)w/-(le)w	& 	-				& -w				& -w &*-\textipa{U}/-\textipa{IbU}\\
\lspbottomrule
  \end{tabular}
\end{table}

\noindent The first three types of morphology are applicatives, which add a new object to the valency of a \isi{verb}.
\iffalse 
An example of the benefactive \isi{applicative} in \ili{Kinyarwanda} is given in (\ref{buraka}).\footnote{Places where two morphemes are given indicate vowel harmony, which is triggered by the shape of the vowel in the preceding syllable.}
\begin{exe}
\ex\label{buraka}\begin{xlist}
\ex\label{som}\gll \ili{Mama} a-tek-a umwumbati.\\
				\ili{Mama} {\sc she-}cooks-{\sc asp} cassava\\
				\glt `\ili{Mama} cooks the cassava.'
\ex\label{sistema}\gll \ili{Mama} a-tek-\underline{er}-a abana umwumbati. \\
				\ili{Mama} {\sc she-}cooks{\sc-ben-asp} children cassava\\
				\glt `\ili{Mama} cooks the cassava for the children.'
\end{xlist}
\end{exe}
%
The valency of the \isi{verb} in (\ref{som}) is increased by one in the sentence in (\ref{sistema}) via the benefactive \isi{applicative} morpheme \emph{-er}. 

 The \isi{applicative} morpheme licenses the addition of the object \emph{abana} `children.' Locative and instrumental applicatives work similarly, only differing in the thematic role of the object they introduce (though see \citet{jerro:2014} for an alternative analysis of \isi{applicative} morphology).  Another argument-adding operation is the \isi{causative}, which also increases the valency of the argument structure of the \isi{verb}|often adding a novel subject.\footnote{See \citet{jerro:ma} for an argument that \isi{causative} morphemes in \ili{Kinyarwanda} actually introduce a \isi{causee} to the argument structure.} 
\fi
Reciprocals, statives, and passives all decrease the valency of a \isi{verb} by one: reciprocals link the action back to the subject, i.e. the subject does the action to him or herself; passives demote the subject to an oblique position and promote the object to subject position; and statives describe the result state of a transitive \isi{verb}. 
	 
 
 Contrary to the decomplexification hypothesis, the data in \tabref{tab:jerro:5} show that \ili{Swahili} does not have a simpler system of valency-changing morphology. From the perspective of the number of category distinctions, it has a comparable number to the other languages, and has lost no form reconstructed for Proto-Bantu.


 From the perspective of the phonological shape of the morphemes, there is no evidence that \ili{Swahili} is simpler than the other languages.   Many of the valency-changing forms in Bantu undergo vowel harmony with the nearest stem on the vowel, and \ili{Swahili} is not an exception to this; it employs vowel harmony on valency-changing morphology in the same way as its sister languages.

 \largerpage[2]
In fact, if any argument were to be made regarding the complexity of valency-changing morphemes, \ili{Swahili} is more complex in the phonological shape of its \isi{passive} morpheme, which varies by context depending on the phonological shape of the \isi{verb} to which it is applied \citep{Russell2003}.  The most productive form of the \ili{Swahili} \isi{passive} is \emph{--w}, as in \emph{fung--w--a} from \emph{funga} `fasten' and \emph{tumi--w--a} from \emph{tumia} `use.' When the \isi{verb} stem ends in [o] or [e], the form \emph{--lew} is used. If the \isi{verb} stem ends in [a] or [u], the form \emph{--liw} is used, as in \emph{za--liw--a} from \emph{zaa} `give birth' and \emph{fu--liw--a} from \emph{fua} `wash clothes'.
 \citet{Russell2003} also notes that the \isi{passive} forms \emph{--ew} and \emph{--iw} are used with verbs of \ili{Arabic} origin, such as \emph{sameh--ew--a} from \emph{samehe} `forgive' and \emph{hitaj--iw--a} from \emph{hitaji} `need.' In short, to form a \isi{passive} in \ili{Swahili}, there are complex factors that determine the phonological shape of the \isi{passive} morpheme, and these factors are not present in the comparison languages. 
 
 In \ili{Kinyarwanda} and \ili{Haya}, for example, the \isi{passive} form is \emph{--w} for all verbs, and \ili{Lingala} lacks a separate \isi{passive} morpheme altogether \citep{Guthrie1966}. This is evidence that valency-changing morphology in \ili{Swahili} is not simpler than the sister languages, and in the domain of the \isi{passive}, \ili{Swahili} is actually more complex than the other forms. 
 

\subsection{Tense, aspect, and mood}%%%%%%%%%%%%%%%%%%%%%%%%%%


  Bantu languages have rich systems of tense, aspect, and mood  (TAM). From the view of complexity, there are two ways in which a language may be simpler than the others with respect to TAM morphology. The language could make fewer distinctions in its tense, aspect, and mood categories, leaving TAM information to pragmatics. Another indication of decomplexification is if the language shows phonological reduction of the forms compared to other languages or from the protolanguage. 
 

  In Bantu languages, aspect and mood morphology generally appears as a prefix before the \isi{verb} stem, but after the agreement subject marker.  Aspect, on the other hand, appears as a suffix after the \isi{verb} stem. If a language marks subjunctive or indicative, this appears in the aspect slot. The general template for TAM on a \isi{verb} in Bantu is given in (\ref{temp}) (cf. \citealt{Meeussen1967,Nurse2003}).
  \begin{exe}
\ex\label{temp} Subject Marker -- Tense -- {\sc stem} -- Aspect/Subjunctive
\end{exe}
%
\tabref{tab:jerro:6} includes data for five different kinds of TAM that are prevalent in Bantu languages: tense, indicative/subjunctive, aspect, \isi{negation}, and idiosyncratic TAM morphology that does not fit consistently with the other categories.%  \ili{Swahili} \citep{russell:2003,ashton:1966}, \ili{Gikuyu} \citep{barlow:1960}, \ili{Luganda} \citep{pilkington:1901}, and \ili{Kinyarwanda} \citep{kimenyi:1980}. \\
\footnote{Data from \ili{Lingala} and \ili{Haya} are not included due to a paucity of description of the tense/aspect systems in those languages.} \\
% %  Prefixes are marked with a dash on the right, and suffixes are marked with a dash on the left.

\newpage 
\begin{table} 
\caption{Comparison of tense, aspect, and mood morphology}

\label{tab:jerro:6} 
\begin{tabular}{lcccc} 
\lsptoprule
 \textbf{ Type} & \textbf{ Swahili}  & \textbf{ Luganda}  & \textbf{ Gikuyu} & \textbf{ Rwanda}  \\
\midrule
  Present				& na-		& 	$\emptyset$		& $\emptyset$ 			&  $\emptyset$\\
  Present II			& a-		& 	-			 	& - 						& - \\
  Pres. Continuous	& -		& -					& ra-					& ra- \\
  Recent Past			& li-		& 	a-				& $\emptyset$			& a-	\\
  Distant Past		& -			& {\sc ms} 			& a-...-ire					& ara-\\
  Perfect				& me-		& -					& -a					& -\\
  Past Perfect		& -			& -					& -ite					& -\\
  Immediate Future 		& ta-		& naa-				& k\~u-					& za- \\
  Near Future		& -			& li-				& ka-					& - \\
  Distant Future		& -			& -					& r\~i-					& - \\
 
 \midrule
 
  Imperative			& -e		& -e				& -e    				& -e\\
  Subjunctive			& -e		& -e				& -e/-(n)i				& -e\\
  Indicative	& -a		& -	& 	-	& 	- \\
 
 \midrule
 
  Imperfective		& -	 		& -					& -ga					& -a(ga)\\
  Perfective			& -			& -					& -a					& -(y)e\\
 
 \midrule
 
  Negation 			& hu-/si- 	& si-					& ti- 					& si-\\
 \midrule
   Conditional			& nge-/ngali-	& andi-			& ng\~i-				& ni-  \\
  Habitual 			& hu-			& -				& ga-					& -\\
  Narrative			& ka-			& ne-			& -						& -  \\
  `not yet'			& ji-		& naa-				& -						& - \\
  `even if'			& japo-		& -					& -						& -	\\
  `if'				& ki-		& -					& - 						& -  \\
 `still'				& -			& kya-				& - 						& - \\
 	optative				& -			& -					& ro-					& - \\
 `also'				& -			& -					& -						& na-\\
 
 \midrule

		&  15 & 12 	& 17 & 12 	\\
\lspbottomrule
 \end{tabular}
\end{table}
 
\noindent The first section shows various tense morphemes: variants of past, present, and future. For some languages (such as \ili{Gikuyu} and \ili{Kinyarwanda}) there are various past and future tenses, depending on the temporal proximity to the speech event. 
%The past and future tenses are divided into close and distant tenses for some of the languages. In \ili{Gikuyu}, there is a three-way distinction for the closeness of future tenses. 
For languages with only one distinction for a particular tense, the form is listed in the tense closest to the present. For example \ili{Swahili} only has one past tense, which is listed in the ``Recent Past" row. The abbreviation {\sc ms} for \ili{Luganda}, indicates that a ``modified stem'' is used to indicate the distant past, formed by a lexically-determined set of stem-changing operations. In \ili{Gikuyu}, the distant past is marked by the combination of a prefix and suffix, indicated by \emph{a-...-ire}.  %The inclusion of \emph{present continuous} in this section is a bit misleading, as it encodes aspectual information. 
In \ili{Swahili}, the present \emph{na-} can also be used for present continuous.
	 
\newpage	 
The second category covers indicative, subjunctive, and imperative morphology, found consistently among all of the languages. For \ili{Swahili}, the final vowel \emph{-a} is used as a general indicative mood marker. 
	 
The third category is aspect. \ili{Kinyarwanda} and \ili{Gikuyu} both have a distinction between \isi{perfective} and imperfective, while \ili{Swahili} and \ili{Luganda} do not have morphology for these aspectual distinctions. 
	
All of the languages share cognate morphology for \isi{negation}. 

Other mood distinctions  are covered in the final section of \tabref{tab:jerro:6}. This is reserved for mood categories that are highly idiosyncratic meanings in particular languages, such as morphology for meanings such as `not yet' and `still' in \ili{Swahili} and \ili{Luganda}, respectively. Another is the ``optative'' in \ili{Gikuyu}, used for blessings and curses \citep{Barlow1960}.  The narrative morpheme is used for verbs that are in a series during a narration of events. 
	 
There is no clear indication that any of these languages has a notably simpler system of TAM morphology.  Summing the number of morphological category distinctions made in the four languages, it is clear that the inventory of distinctions is quite comparable for all the languages, and \ili{Swahili} is not noticeably less complex than any other language. It is important to note the heterogeneity among the languages' TAM morphology; few morphemes are cognate, which makes it impossible to compare the phonological reduction among the languages, meaning that the phonological reduction of forms cannot be measured for complexity. 


% \iffalse%%%%%%%%%%%%%%%%%%%%%%
%   Another factor for TAM are Compound Tense Constructions. 
%  
%   These constructions allow complex combinations of tense and aspect morphology by the use of the auxiliary verbs \emph{kuwa} `to be,' \emph{kwisha} `to finish,' \emph{kuja} `come,' and \emph{kwenda} `go.' 
% 
%   Interestingly, both the auxiliary and the main \isi{verb} \isi{agree} with the subject of the sentence. 
% 
%   An example from \ili{Swahili} is given in (\ref{ct}) from \citep{henderson:2006}.
% \begin{exe}
% \ex\label{ct}\begin{xlist}
% \ex\gll \ili{Juma} a-\emph{li}-kuwa a-\emph{me}-pika chakula.  \\
%  	\ili{Juma} 3{\sc sg-pst}-be 3{\sc sg-perf}-cook food\\
% 	\glt `\ili{Juma} had cooked food.Õ 
% \ex\gll (Mimi) ni-\emph{li}-kuwa ni-ngali ni-\emph{ki}-fanya kazi.\\
% (I) 1{\sc sg-pst}-be 1{\sc sg-}still 1{\sc sg-prog}-do work\\ 
% 	\glt `I am still working.Õ 
% \end{xlist}
% \end{exe}
% %
% 
%   In these examples, the \isi{verb} \emph{kuwa} `be' is used to host the past tense marker \emph{li}. 
% The main \isi{verb} takes a \isi{perfect} (\ref{ct}a) or \isi{progressive} (\ref{ct}b) morpheme that adds aspectual information. These are only two of the many logical combinations that are available in \ili{Swahili}. 
% 
% 
%   A similar construction is found in \ili{Kinyarwanda} (author's field notes), where the copula \emph{--ri} `to be' is used with an infinitive-marked \isi{verb} to create a present \isi{progressive}:\footnote{Note that the copula is marked with the locative morpheme \emph{--mo}. This is not obligatory, but speakers always give it as a default in elicitation. It does not require that the sentence has a location.}
% 
% 
% 
% \begin{exe}
% \ex\gll   N-di-mo kw-andika mu cyumba.\\
% 		{\sc 1sgS-cop-loc} {\sc inf}-write in room\\
% 		\glt `I am writing in the room.'
% \end{exe}    
% 
%    These systems of Complex Tense Constructions are another domain of typological complexity that await further research and discussion. 
%   
%   
% \fi
% 
\section{Discussion: complexity and language contact}\label{sec:6:jerro}
%%%%%%%%%%%%%%%%%%%%%%%%%%%%%%%%%%%%%%%%%%%%%%%%%%%%%%%%%%%%%%%%%%%%%%%%%%%%%%%%%%%%%%%

  Data comparing the phonological inventory and morphological systems among \ili{Swahili}, \ili{Gikuyu}, \ili{Kinyarwanda}, \ili{Lingala}, \ili{Haya}, and, \ili{Luganda} -- as well as a comparison with Proto-Bantu -- show that there is no instance where clear decomplexification has occurred in \ili{Swahili}. In fact, in some instances, such as in consonant inventory, \ili{Swahili} shows more complexity that the other languages. In nearly all of the grammatical properties discussed, \ili{Swahili} is highly divergent from the other languages, with notable differences in phonological inventory, such as a larger consonant inventory, a smaller vowel inventory, and irregularities with respect to stress and syllabification. Crucially, all phonological changes that have occurred have happened via natural sound changes, but at a faster rate that than the other languages, i.e. \ili{Swahili} is less similar to Proto-Bantu than the other languages. 
  
 This grammatical situation fits neatly within recent studies of the typological and sociolinguistic literature on contact: \isi{language contact} results in an increased rate of change, and prolonged contact between two languages moves towards more linguistic complexity \citep{Trudgill2011}. Prolonged contact with \ili{Arabic} via the Omanis' presence in Zanzibar since the 13\super{th} century result in a strong change in the grammar of the language in comparison to other Bantu languages; however, it never blended with \ili{Arabic} and became a pidgin or creole. \citet{Mufwene2001} and \citet{Mufwene2003} also notes the divergent behavior of \ili{Swahili} when compared to other contact languages in Africa, showing that the exogamous use of \ili{Swahili} has led to its adoption by the local population, which resulted in a relatively consistent use of \ili{Swahili}. From this perspective, \ili{Swahili}'s divergence from the other languages is attributable to the specific contact situation of prolonged bilingualism with \ili{Arabic}. Crucially, none of the comparison languages have engaged in this kind of long-term bilingualism, accounting for grammatical differences between them and \ili{Swahili}. 
 
In this paper, I have compared Standard \ili{Swahili} as described in \citet{Ashton1966} to the standard varieties of several other varieties of East African Bantu languages.  As just noted, standard coastal \ili{Swahili} has been in long-term contact with \ili{Arabic} since the 13\super{th} century, and this contact resulting in expedited change (and, at times, complexification) of several grammatical features of the standard variety. Another prediction from the literature on linguistic complexity is that simplification of grammar occurs when adult learners attempt to learn a \isi{second language}. \citet{Kusters2003phd} fleshes out this claim, comparing Standard \ili{Swahili} (the variety discussed in the present paper) to two other varieties of \ili{Swahili} that are used as lingua francas in areas where several adult speakers of the languages speak it regularly, specifically, inland Kenyan \ili{Swahili} and the \ili{Swahili} spoken in the trade town of Lubumbashi in the Katanga region of the Democratic Republic of the \isi{Congo}. Crucially, both of these varieties have less prestige than the coastal standard.

Kusters' findings fit the typological pattern predicted: these two \isi{lingua franca} languages show several reductions in category distinctions, morphophonological complexity and a reduction of inflectional information. For reasons of space, I refer the reader to Kusters' work, but the crucial point for the current discussion is that the three varieties of \ili{Swahili} are clear examples of the two kinds of second-language learning in contact areas. Standard \ili{Swahili} exemplifies the effects of long-term \isi{language contact}, with acquisition by young children: it has a radically divergent and at times more complex grammar than related non-contact Bantu languages.  Two other varieties of \ili{Swahili} that have largely been used as lingua francas by adult second-language speakers show systemic reduction in grammatical structure when compared with standard \ili{Swahili} \citep{Kusters2003phd}.




%%%%%%%%%%%%%%%%%%%%%%%%%%%%%%%%%%%%%%%%%%%%%%%%%%%%%%%%%%%%%%%%%%%%%%%%%%%%%%%%%%%%%%%%%%%%%%%%%%%%%%%%%%%%%%%%%%%%%%%%%%%%%%%%%%%%%%%%%%%%%%%%%%%%%%%%%%%%%%%%%%%%%%%%%%%%%%%%%%%%%%%%%%%%%%%%%%%%%%%%%%%%%%%%%%%%%%%%%%%%%%%%%%%%%%%%%%%%%%%%%%%%%%%%%%%%%%%%%%%%%%%%%%%%%%%%%%%%%%%%%%%%%%%%%%%%%%%%%%%%%%%%%%%%%%%%%%%%%%%%%%%%%%%%%%%%%%%%%%%%%%%%%%%%%%%%%%%%%%%%%%%%%%%%%%%%%%%%%%%%%%%%%%%%%%%%%%%%%%%%%%%%%%%%%%%%%%%%%%%%%%%%%%%%%%%%%%%%%%%%%%%%%%%%%%%%%%%%%%%%%%%%%%%%%%%%%%%%%%%%%%%%%%%%%%%%%%%%%%%%%%%%%%%%%%%% %%%%%%%%%%%%%%%%%%%%%%%%%%%%%%%%%%%%%%%%%%%%%%%%%%%%%%%%%%%%%%%%%%%%%%%%%%%%%%%%%%%%%%%%%%%%%%%%%%%%%%%%%%%%%%%%%%%%%%%%%%%%%%%%%%%%%%%%%%%%%%%%%%%%%%%%%%%%%%%%%%%%%%%%%%%%%%%%%%%%%%%%%%%%%%%%%%%%%%%%%%%%%%%%%%%%%%%%%%%%%%%%%%%%%%%%%%%%%%%%%%%%%%%%%%%%%%%%%%%%%%%%%%%%%%%%%%%%%%%%%%%%%%%%%%%%%%%%%%%%%%%%%%%%%%%%%%%%%%%%%%%%%%%%%%%%%%%%%%%%%%%%%%%%%%%%%%%%%%%%%%%%%%%%%%%%%%%%%%%%%%%%%%%%%%%%%%%%%%%%%%%%%%%%%%%%%%%%%%%%%%%%%%%%%%%%%%%%%%%%%%%%%



\subsection*{Acknowledgements}

I am indebted to Tony Woodbury, Pattie Epps, Peter Trudgill, and the anonymous reviews for their invaluable comments during the development of this paper. I would also like to thank the audience of the 45th Annual Conference on African Linguistics for their input and suggestions. All errors remain the fault of the author.

\subsection*{Abbreviations}
\begin{tabular}{llllll}
1  & first person  & {\sc asp}  &  aspect                   & {\sc sg}  &  singular\\
2  & second person & {\sc ben}  &  benefactive \isi{applicative}\\
3  &  third person & {\sc pl}   &  plural\\
\end{tabular}
{\sloppy
\printbibliography[heading=subbibliography,notkeyword=this]
}
\end{document}
