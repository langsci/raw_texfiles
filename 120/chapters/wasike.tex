\documentclass[output=paper,
modfonts
]{langscibook} 
\ChapterDOI{10.5281/zenodo.1251746}
% \bibliography{localbibliography}

 
\title{Adjectives in Lubukusu } 

\author{Aggrey Wasike \affiliation{University of Toronto } 
}

\abstract{The lexical category of adjectives is proposed to be universal, but its realization varies across languages. In languages such as English, there is a clearly distinct category of adjectives. But in other languages the category of adjectives is not entirely distinct morphologically and syntactically from nouns and verbs. In this paper I show that there is a striking resemblance between adjectives and nouns in Lubukusu. In addition, stage-level predicate meanings are expressed by use of verbs rather than adjectives. Because of these facts, it is tempting to adopt an analysis that reduces Lubukusu adjectives to either nouns or verbs. However, I argue that there is not sufficient evidence to support such an analysis. Lubukusu has true adjectives in spite of the associated nominal and verbal characteristics. A verbal characteristic such as expressing adjectival meanings by use verbs is similar to languages such Mohawk and Vaeakau-Taumako. But there are significant differences between these languages and Lubukusu with regards to this verbal characteristic. }


\begin{document}
\maketitle
 
 
% Key words: Adjectives, \ili{Lubukusu}, syntax, stage-level predicates, nominal, verbal 

\section[Introduction]{Introduction}\label{sec:wasike:1}

This paper discusses adjectives in \ili{Lubukusu}, a Bantu language spoken in Western \isi{Kenya}. This discussion is particularly useful considering the fact that there is disagreement among linguists on the status of adjectives as a universal category. For linguists such as Mark Baker and  R.\,M.\,W. Dixon, the lexical category of adjectives is universal \citep{Baker2001,Baker2003,DixonAikhenvald2004}. But for  R.\,M.\,W. Dixon in his earlier work and Chafe Wallace, the category of adjectives is not universal \citep{Dixon1982,Chafe2012}. Description and discussion of how adjectival meanings are expressed in different languages can help linguists draw a valid conclusions regarding the category of adjectives. 

The lack of agreement on the status of adjectives as a universal category among linguists is attributed to cross-linguistic variation in the expression of adjectival meanings. Some languages express adjectival meanings by use of adjectives while others express similar meanings by use of verbs and nouns. Even those languages that have a distinct class of adjectives differ from each other in terms of the properties of adjectives: in some languages, adjectives “may share at least some of their morpho-syntactic behavior with nouns, in others they may have more in common with verbs, and still in others they may be more or less independent of those other classes” \citep[1]{Chafe2012}. Languages that have a distinct category of adjectives can also differ in terms of adjective inventory: some, such as \ili{English}, have a large number of adjectives. In fact adjectives in \ili{English} constitute an open class. But other languages have a closed set of adjectives. An example of such a language is \ili{Hausa} which has approximately 12 adjectives \citep{Whaley1997}.

As has already been pointed out, some languages lack a distinct category of adjectives. Instead they express adjectival meanings by use of either verbs or nouns. Examples of languages which use verbs to express adjectival meanings include \ili{Mohawk} \citep{Baker2001}, \ili{Seneca} and other northern Iroquoian languages \citep{Chafe2012}, \ili{Manipuri} \citep{Bhat1994} and \ili{Mayali}. To illustrate how a languages use verbs to express adjectival meanings consider sentences in \REF{ex:wasike:1} and \REF{ex:wasike:2} from \ili{Manipuri} and \ili{Seneca} respectively.

\ea\label{ex:wasike:1}
\ea
\langinfo{Manipuri}{}{\citealt[190]{Bhat1994}}\\
\gll ǝy mabu u-de\\
     I him-\textsc{acc} see-\textsc{neg}(\textsc{nfut})\\
\glt ‘I did not see him.’
\ex
\gll  phi    ǝdu    ŋaŋ-de\\
     cloth  that   red-\textsc{neg}(\textsc{nfut})\\
\glt ‘That cloth is not red.’
\z
\z

\ea\label{ex:wasike:2}
\ea
\langinfo{Seneca}{}{\citealt[10]{Chafe2012}}\\
\gll Tganӧhso:t,\\
     t-ka-nӧhs-o:t-ø\\
     \textsc{cis}-\textsc{n.sg.agt} -building-upright-\textsc{sta}\\
\glt ‘The house there,’
\ex

\gll  ganӧhsasdë.\\
     ka-nöhs-astë-:\\
     \textsc{n.sg.agt} -building-big-\textsc{sta}\\
\glt ‘it was a big house.’
\z
\z

Notice that in these languages words that express adjectival meanings take inflectional morphology of verbs such as \isi{negation} in \REF{ex:wasike:1}.

An example of a language that uses nouns to express adjectival meanings is \ili{Quechua}. In this language words that express adjectival meanings take nominal inflectional morphology such as case. This is illustrated in \REF{ex:wasike:3}.

\ea\label{ex:wasike:3}
\ea
\langinfo{Peru Quechua}{}{\citealt[6]{Weber1983}}\\
\gll rumi-ta rikaa\\
     stone-\textsc{acc}  \textsc{1sg}.see\\
\glt ‘I see a/the stone.’
\ex

\gll  hatun-ta rikaa\\
     big-\textsc{acc} \textsc{1sg}-see\\
\glt ‘I see a/the big (one).’
\z
\z

Given the cross-linguistic variability illustrated above, it is understandable why linguists would fail to \isi{agree} on whether the category of adjective is universal or not. Clearly, studying how adjectival meanings are expressed in languages that have not been studied yet is important as it can improve our understanding of adjectives. 

In this paper, I show how adjectival meanings are expressed in \ili{Lubukusu}. Questions that I seek to answer include, though not limited to the following: How are adjectival meanings expressed in \ili{Lubukusu}? Is \ili{Lubukusu} \ili{English}-like, \ili{Mohawk}-like or \ili{Quechua}-like? How similar or different is \ili{Lubukusu} from other languages? What nominal and verbal features do words that express adjectival meanings have? Why shouldn’t \ili{Lubukusu} adjectives be considered nouns? 

I show that \ili{Lubukusu} has pure adjectives like \ili{English}, but it not exactly like \ili{English} in all respects. This is because there are certain adjectival meanings that are expressed by use of verbs just like in \ili{Mohawk}, \ili{Seneca}, \ili{Mayali} and \ili{Manipuri}. Thus \ili{Lubukusu} has a mixed adjectival system. I also show that that although adjectives are structurally similar to nouns, they constitute a lexical category that is distinct. Similarly, although certain adjectival meanings are expressed by use of verbs, it is cannot be true to argue that \ili{Lubukusu} lacks adjectives.

This paper is organized as follows. \sectref{sec:wasike:2} is a general description of adjectives in \ili{Lubukusu} that recognizes two \isi{major} subtypes of adjectives: basic adjectives and derived adjectives. \sectref{sec:wasike:3} describes and discusses nominal features of adjectives focusing on morphological and syntactic features. \sectref{sec:wasike:4} is a discussion of verbal features of adjectives. I show that only stage-level readings of predicates are expressed by use of verbs in \ili{Lubukusu}. \sectref{sec:wasike:5} is the conclusion.

\section{Overview of adjectives in Lubukusu}\label{sec:wasike:2}

Adjectives are nominal modifiers, and as illustrated in \REF{ex:wasike:4} they can occur in a Noun Phrase (NP) with other modifiers such as numerals, possessive pronouns, demonstratives, associative Prepositional Phrases (PP) and relative clauses.

\ea\label{ex:wasike:4}
\ili{Lubukusu}\\
\gll ba{\rmfnm}-ba-ana ba-taru ba-nge baa-bofu be lii-ria baa-kon-a       ba-no\\
     2-2{\rmfnm}-person     2-three  2-mine 2.2-big    of  5.5-respect     2.\textsc{rel}-sleep-\textsc{fv} 2-\textsc{dem}\\
\glt ‘these three big respectful children of mine who are sleeping’
\z

\footnotetext{{Orthographic B or b is [$\beta $] when it is not preceded by a nasal sound}}
\footnotetext{In the sequence of these numerals here and in the rest of the paper, the first numeral is the pre-prefix (augment); the second numeral is class prefix}

Because of the constraints of space numerals, possessive pronouns, associative PPs and relative clauses are not discussed in this paper. I have also not discussed word in the NP. Instead I have focused on adjectives only, without reference to the other \isi{noun} modifiers.

Adjectives in \ili{Lubukusu} can be divided into two broad categories: basic adjectives and derived adjectives.

\subsection{Basic adjectives}

These are adjectives that are base generated and are not derived from other lexical categories. Basic adjectives describe size, color, quantity etc. and include \textit{-bofu }‘big’,\textit{ -titi} ‘small’, \textit{-leyi} ‘tall’, \textit{-imbi} ‘short’, \textit{-balayi} ‘wide’, \textit{-mali} ‘black’, \textit{-wanga} ‘white’, \textit{-besemu} ‘red’, \textit{-kali} ‘many’, \textit{-lendafu} ‘stupid’, \textit{-kesi} ‘clever’, \textit{-silu} ‘stupid’, \textit{-kara} ‘lazy’, \textit{-miliu} ‘clean’, \mbox{\textit{-nyalu}} ‘dirty’, \textit{-siro} ‘heavy’, \textit{-angu} ‘light’, \textit{-chou} ‘big/fully grown’, \textit{-khulu} ‘old’, -\textit{bisi} ‘raw/\linebreak unripe’, \textit{-robe} ‘ripe’ etc. 

\subsection{Derived adjectives}
% \todo[inline]{make these enumerations into formatted list for better readability?}
\ili{Lubukusu} has a very productive process of deriving adjectives from verbs. The process involves suffixing the \isi{root} of the \isi{verb} with the vowel -\textit{e}\footnote{Verbs that end in -\textit{e }should not be confused with adjectives that are derived from verbs which also end in \textit{-e}. The two are distinguishable even without reference to tonal patterns: derived adjectives have an augment and class prefix but verbs have \isi{subject agreement} and verbal inflections such as tense.}. In other words the final vowel of the \isi{verb} \isi{root} is replaced with the vowel -\textit{e}. This suggests that -\textit{e} is an ‘adjectivizing’ suffix. Derived adjectives ending in -\textit{e} include\textit{-funge} ‘closed’ (from \textit{funga} ‘close’), \textit{-funule} ‘open’ (from \textit{funula}\footnote{Verbs in \ili{Lubukusu} in the imperative mood occur without any prefixes } ‘open’), -\textit{funikhe} ‘covered’ (from \textit{funikha} ‘cover’), \textit{-singe} ‘washed’ (from \textit{singa} ‘wash’), \textit{-tekhe} ‘cooked’ (from \textit{tekha} ‘cook’), \textit{-lekhule} ‘free’ (from \textit{lekhula} ‘free/let go’), \textit{-khalange} ‘fried’ (from \textit{khalanga }‘fry’), \textit{-robe} ‘ripe’ (from \textit{roba} ‘become ripe’), \textit{-simbe} ‘busy’ (from \textit{simba} ‘become busy’), \textit{-lume} ‘hard’ (from \textit{luma} ‘become hard’), \textit{-sye} ‘ground’ (from \textit{sya} ‘grind’), \textit{-ake} ‘weeded’ (from \textit{yaka }‘weed’), \textit{-male} ‘smeared’ (from \textit{mala} ‘smear’), \textit{-chichunge} ‘sifted’ (from \textit{chichunga} ‘sift’), \textit{-osye} ‘roasted’ (from \textit{yosya} ‘roast’) etc.

The derivation of adjectives from verbs with roots that end in liquids (/l/ \& /r/) and the voiceless \isi{fricative} (/x/) involve phonological processes that are different from -e final derived adjectives illustrated in the previous paragraph. To form adjectives from these verbs, the \isi{root} final liquids and /x/ are changed into \isi{fricatives} – either [f] or [s] and the final vowel is changed to [u] or [i]. Adjectives formed in this manner include: -\textit{mesi} ‘drunk’ (from \textit{mela} ‘become drunk’), -\textit{changalafu} ‘insipid’ (from \textit{changalala} ‘be dull, insipid, tasteless), \textit{-randafu} ‘brown’ (from \textit{randara} ‘become brown’, \textit{-lendafu} ‘stupid, slow’ (from \textit{lendala} ‘be stupid, be slow thinking), -\textit{angafu} ‘mature’ (from \textit{angala} ‘mature’), \textit{-nyindafu} ‘brave’ (from \textit{nyindala} ‘brave’), \textit{-labufu} ‘dirty’ (from \textit{labukha}\footnote{Orthographic kh- is [x]} ‘become dirty’), \textit{rundubafu} ‘big’ (from \textit{rundubara} ‘become big’), -\textit{khalafu} ‘sad’ (from \textit{khalala} ‘be sad’), \textit{-nefu} ‘fat’ (from \textit{nera} ‘become fat’), -\textit{balakafu} ‘dry’ (from \textit{balakala} ‘dry’), \textit{-kafu} ‘stupid’ (from \textit{kala} ‘be stupid/slow thinking’), \textit{-khandyafu} ‘proud’ (from \textit{khandyaba }‘walk proudly’) etc.

In addition to the two types of derived adjectives exemplified above, there are derived adjectives that end in the suffix -\textit{a}. These adjectives are fewer compared to the first two sub-types as they are restricted to occurring with only few nouns. Derived adjectives that end in -\textit{a} include \textit{-fumba} ‘folding’ (from \textit{fumba} ‘fold’, in \textit{endebe efumba} ‘folding chair’), \textit{-chwisya} ‘knotted’ (from \textit{chwisya} ‘knot’, in \textit{kumukoye kumuchwisya} ‘knotted rope’), \textit{-chunula} ‘unknotting’ (from \textit{chunula} ‘unknot’, in \textit{kumukoye kumuchunula} ‘unknotting rope’, \textit{-khalisya} ‘crossing/shortcut’ (from \textit{khalisya} ‘cross/shortcut’, in \textit{engila ekhalisya} ‘short cut’, \textit{-kisa} ‘hide’ (from \textit{kisa} ‘hide’, in \textit{kumwinyawe kumwikisa} ‘hide and seek game’, \textit{-kara} ‘dribble’ (from \textit{kara} ‘dribble’, in \textit{kumupira kumukara} ‘dribbling ball game (=soccer)’, \textit{-chururusya} ‘uninterrupted’ (from \textit{chururusya} ‘let go uninterrupted’ in \textit{kumupira kumuchururusya} ‘game where two people hit the soccer ball back and forth’). As already pointed out, adjectives in this subcategory are few. This may be due semantic reasons. For example only few objects can be described as folding\footnote{This contrasts with the adjective -\textit{fumbe} ‘folded’ (from \textit{fumba} ‘fold’) which is not as restricted semantically. Many things can be described as folded: \textit{engubo efumbe} ‘folded cloth’, \textit{ekaratasi efumbe} ‘folded paper’ etc. }, only few things can be described as knotting, only a handful on things can be described as dribbling etc. But it is also possible that these are noun-\isi{verb} compounds rather than adjectives. 

Up to this point we have seen that adjectives in \ili{Lubukusu} can be formed from verbs. But the reverse is also possible: verbs can be formed from adjectives through a fairly productive process of suffixing -\textit{a} to the \isi{root} of the adjective. To briefly illustrate, adjective roots such\textit{, -imbi} ‘short’, \textit{-bofu} ‘big’, -\textit{leyi} ‘tall’, and\textit{ -besemu} ‘red’ can be converted into the following verbs respectively: \textit{imbia} ‘become short’, \textit{bofua} ‘become big’,\textit{ lea} ‘become tall’ and\textit{ besema} ‘become red’. The \isi{suffixation} of -\textit{a} to -\textit{imbi }‘short’ and -\textit{bofu} ‘big’ to form a \isi{verb} is clear and cannot be contested. These two examples show that the direction of conversion is from adjective to \isi{verb}. I assume that this is also true in both \textit{lea} ‘become tall’ and \textit{besema}’ even though these two examples involve additional phonological processes beyond \isi{suffixation} of -\textit{a.} In any case the roots \textit{-imbi, -bofu, -leyi, }and\textit{ -besemu} are unequivocally adjectival, and conversion must be from adjective to \isi{verb}.  

To summarize this section, we have seen that \ili{Lubukusu} has many adjectives, some of which can be classified as basic and others as derived. It is particularly instructive that the morphological process which forms adjectives from verbs is quite productive. It is therefore reasonable to conclude that the class of adjectives in \ili{Lubukusu} is an open class. This differs from Bantu languages such as \ili{Chichewa} which has “…very few ‘pure’ adjective stems…” \citep[24]{Mchombo2004}. It also differs from \ili{Kiswahili} which lacks a productive process of forming adjectives from verbs. For example \ili{Kiswahili} lacks the \isi{noun} + verb-derived-adjective equivalent of the \ili{Lubukusu} \textit{enyama endekhe} ‘cooked meat’. In \ili{Kiswahili} to say cooked meat, one must use a \isi{relative clause} -\textit{nyama iliyopikwa }‘meat that is cooked’.

Having provided a general description of adjectives in \ili{Lubukusu}, we are now ready to tackle remaining issues that can challenge our conclusion that adjectives are indeed a separate and independent lexical category in \ili{Lubukusu}. An examination of \ili{Lubukusu} adjectives indicates that they do have what can be considered nominal features on the one hand and verbal features on the other. It is necessary to discuss these features and in the process explain how and why adjectives are neither nouns nor verbs.

\section{Nominal features of adjectives in Lubukusu}\label{sec:wasike:3}

By nominal features, I mean those features that are generally thought of as belonging to nouns. But instead of nouns being the bearers or associates of these features, it is adjectives that are. When adjectives carry many nominal features, it can be difficult to clearly and uniquely distinguish between nouns and adjectives, and it can be challenging to argue for the existence of a separate category of adjectives. In this section I examine and illustrate morphological and syntactic nominal features borne by or associated with the \ili{Lubukusu} adjectives, beginning with morphological features.

\subsection{Morphology of the adjective}

\ili{Lubukusu} adjectives must \isi{agree} with the nouns that they modify. For this reason they have a morphological structure that similar to that of nouns. Like nouns, adjectives have the structure pre-prefix (augment)-class prefix-\isi{root}. But it is the \isi{noun} that determines the form of the pre-prefix and prefix borne by the adjective. The following data illustrates the structure of nouns, adjectives and agreement patterns in \isi{noun} classes 1 to 11.

\ea\label{ex:wasike:5}
\ea
\gll o-mu-ndu o-mu-bofu\\
     1-1-person 1-1-big\\
\glt ‘big person’
\ex
\gll  ba-ba-ndu baa-bofu\\
     2-2-person 2.2-big\\
\glt ‘big people’
\z
\z


\ea\label{ex:wasike:6}
\ea
\gll ku-mu-sala ku-mu-bofu\\
     3-3-tree 3-3-big\\
\glt ‘big tree’
\ex
\gll  ki-mi-sala ki-mi-bofu\\
     4-4-tree 4-4-big\\
\glt ‘big trees’
\z
\z

\ea\label{ex:wasike:7}
\ea
\gll li-li-ino lii-bofu\\
     5-5-tooth 5.5-big\\
\glt ‘big tooth’
\ex
\gll  ka-me-eno ka-ma-bofu\\
     6-6-tooth 6-6-big\\
\glt ‘big teeth’
\z
\z

\ea\label{ex:wasike:8}
\ea
\gll si-sy-uma sii-bofu\\
     7-7-bead 7.7-big\\
\glt ‘big bead’
\ex
\gll  bi-by-uma bii-bofu\\
     8.8-bead 8.8-big\\
\glt ‘big beads’
\z
\z

\ea\label{ex:wasike:9}
\ea
\gll e-n-dika e-m-bofu\\
     9-9-bicycle 9-9-big\\
\glt ‘big bicycle’
\ex
\gll  chi-n-dika chi-m-bofu\\
     10-10-bicycle 10-10-big\\
\glt ‘big bicycles’
\z
\z

\ea\label{ex:wasike:10}
\ea
\gll lu-lu-ichi luu-bofu\\
     11-11-river 11.11-big\\
\glt ‘big river’
\ex
\gll  chi-nj-ichi chi-m-bofu\\
     10-10-river 10-10-big\\
\glt ‘big rivers’
\z
\z

As shown in (\ref{ex:wasike:5}--\ref{ex:wasike:10}), the structure of adjective is similar to that of \isi{noun} in each case both in terms of the number of morphemes and form of the morphemes. This is due to agreement requirements: the adjective must \isi{agree} with the \isi{noun} they modify. The adjective meets this requirement by copying or reduplicating the prefix form and structure of the \isi{noun} it modifies. When a \isi{noun}’s prefixes are \textit{o-mu} as in (\ref{ex:wasike:5}a), the adjective must also have \textit{o-mu}; when a \isi{noun}’s prefixes are \textit{ku-mu} as in (\ref{ex:wasike:6}a), the adjective must also have \textit{ku-mu}. The only slight variations in the \isi{noun} and adjective prefix structure can be found in (\ref{ex:wasike:5}b), (\ref{ex:wasike:7}a \& b), (\ref{ex:wasike:8}a \& b), and (\ref{ex:wasike:9}a \& b), repeated here as \REF{ex:wasike:11}, \REF{ex:wasike:12}, \REF{ex:wasike:13} and \REF{ex:wasike:14} respectively.

\ea\label{ex:wasike:11}
\gll ba-ba-ndu baa-bofu\\
     1-1-person 1-1-big\\
\glt ‘big people’
\z

\ea\label{ex:wasike:12}
\ea
\gll li-li-ino lii-bofu\\
     5-5-tooth 5.5-big\\
\glt ‘big tooth’
\ex
\gll  ka-me-eno ka-ma-bofu\\
     6-6-tooth 6-6-big\\
\glt ‘big teeth’
\z
\z


\ea\label{ex:wasike:13}
\ea
\gll si-sy-uma sii-bofu\\
     7-7-bead 7.7-big\\
\glt ‘big bead’
\ex
\gll  bi-by-uma bii-bofu\\
     8.8-bead 8.8-big\\
\glt ‘big beads’
\z
\z

\ea\label{ex:wasike:14}
\ea
\gll e-n-dika e-m-bofu\\
     9-9-bicycle 9-9-big\\
\glt ‘big bicycle’
\ex
\gll  chi-n-dika chi-m-bofu\\
     10-10-bicycle 10-10-big\\
\glt ‘big bicycles’
\z
\z

These are not counterexamples to the noun-adjective prefix similarity generalization since they can be explained phonologically. In (\ref{ex:wasike:12}b) the augment-\isi{noun} prefix turns up as \textit{ka-me} because the phonological process of vowel coalescence has taken place. In (\ref{ex:wasike:14} \& b), the nasal takes on different forms because it assimilates to the place of articulation of following stop. And finally in \REF{ex:wasike:11}, (\ref{ex:wasike:12}a), and (\ref{ex:wasike:13}a \& b), prefix haplology has taken place. Prefix haplology takes when identical prefixes such as \textit{ba-ba} (class 2), \textit{li-li} (class 5), \textit{si-si} (class 7) and \textit{bi-bi} (class 8) are followed by nominal \isi{root} or adjectival \isi{root} that begins with a consonant. In \REF{ex:wasike:11}, \textit{ba-ba} becomes \textit{baa} because the \isi{root} of the adjective begins with a consonant. This holds for \REF{ex:wasike:12} as well where \textit{li-li} becomes \textit{lii} because the adjective \isi{root} is consonant initial. Prefix haplology is not limited to adjectives alone; it takes place in nouns as well. This is illustrated in the following data.

\ea\label{ex:wasike:15}
\ea
\gll baa-soreri \textup{(from }ba-ba-soreri\textup{)}\\
     2.2-boy\\
\glt ‘boys’
\ex
\gll  lii-fumbi \textup{(from }li-li-fumbi\textup{)}\\
     5.5-cloud\\
\glt ‘cloud’
\z
\z

\ea\label{ex:wasike:16}
\ea
\gll sii-rekere \textup{(from }si-si-rekere\textup{)} \\
     7.7-village\\
\glt ‘village’
\ex
\gll  bii-rekere \textup{(from }bi-bi-rekere\textup{)}\\
     8.8-village\\
\glt ‘villages’
\z
\z

For a detailed discussion of prefix haplology and how it is accounted for in phonological theory, see \citet{Mutonyi2000}.

I end this section by reiterating the nominal features borne by \ili{Lubukusu} adjectives. \ili{Lubukusu} adjectives bear class prefixes that \isi{agree} with the \isi{noun} they modify. In addition, they undergo prefix haplology just like nouns. But is this enough to conclude that \ili{Lubukusu} adjectives are a sub-set of nouns rather than an independent lexical category? Before answering this question, let us first examine the ‘noun-ness’ of adjectives at the level of syntax.

\subsection{Syntactic function and position of Lubukusu adjectives  }

\ili{Lubukusu} adjectives seem to occupy typical \isi{noun} positions in the sentence without a modified \isi{noun}. The typical \isi{noun} positions which \ili{Lubukusu} adjectives can occupy are subject position, object position and object of \isi{preposition} position. To illustrate, consider the following sentences.

\ea\label{ex:wasike:17}
\ea
\gll O-m-bofu o-yu a-siim-a o-mw-ana wewe.\\
     1-1-blind \textsc{dem}-1 1-\textsc{prs}-love-\textsc{fv} 1-1-child his/hers\\
\glt ‘That blind person loves his/her child.’
\ex
\gll  Baa-tambi ba-a-sab-ang-a byaa-khulya.\\
     2.2-poor 2-\textsc{prs}-beg-\textsc{hab}-\textsc{fv} 8.8-food\\
\glt ‘The poor usually beg for food.’
\z
\z

\ea\label{ex:wasike:18}
\ea
\gll Ku-mu-leeyi kw-a-funiikh-e.\\
     3-3-tall 3-\textsc{pst}-break-\textsc{fv}\\
\glt ‘The tall (one) broke.’
\ex
\gll  Li-li-imbi lya-a-kw-a.\\
     5-5-short 5-\textsc{pst}-fall-\textsc{fv}\\
\glt ‘The short (one) fell.’
\z
\z

\ea\label{ex:wasike:19}
\ea
\gll Wafula a-a-yet-ang-a baa-tambi.\\
     1.Wafula 1-\textsc{prs}-help-\textsc{hab}-\textsc{fv} 2.2-poor\\
\glt ‘Wafula usually helps the poor.’
\ex
\gll  Mayi a-a-kul-il-e lii.bofu.\\
     Mother 1-\textsc{pst}-buy-\textsc{asp}-\textsc{fv} 5.5-big\\
\glt ‘Mother bought the big (one).’
\z
\z

\ea\label{ex:wasike:20}
\ea
\gll Wafula a-a-r-a sii-bofu khu-mesa.\\
     Wafula 1-\textsc{pst}-put-\textsc{fv} 7.7-big on.\textsc{prf}-table\\
\glt ‘Wafula put the big (one) on the table.’
\ex
\gll  Wafula a-a-r-a sii-tabu khu-mu-bofu.\\
     Wafula 1-\textsc{pst}-put-\textsc{fv} 7.7-book 17-17-big\\
\glt ‘Wafula put the book on the big one.’
\z
\z

Thus \ili{Lubukusu} adjectives can function as subject (17\& 18), object \REF{ex:wasike:19} and object of \isi{preposition} \REF{ex:wasike:20}. These positions – subject position, object position and object of \isi{preposition} – are \isi{noun} positions and there is no doubt nor controversy about it. We are thus confronted yet again with data and facts that underscore the striking similarity between nouns and adjectives in \ili{Lubukusu}. Does this mean that adjectives in \ili{Lubukusu} are nouns? This is the question we turn to in the next section.  

\subsection{Are Lubukusu adjectives nouns? }

In section 3.1, we saw that adjectives in \ili{Lubukusu} have a structure that this similar to that of nouns. Like nouns they have a pre-prefix and a class prefix. In addition adjectives undergo prefix haplology just like nouns. And in section 3.2, we saw that \ili{Lubukusu} adjectives take typical \isi{noun} functions of subject, object and object of \isi{preposition}. These striking similarities raise the possibility that adjectives are just a sub-type of nouns. If this is indeed the case, then it will not be justifiable to retain adjective as s separate lexical category in \ili{Lubukusu} grammar.

I argue that the nominal features of adjectives that we have seen in previous sections are not sufficient to make the lexical category of adjective in \ili{Lubukusu} irrelevant. One piece of evidence which shows that adjectives and nouns in \ili{Lubukusu} are indeed separate lexical categories comes from NPs that contain both a \isi{noun} and adjective. These NPs show unambiguously that nouns and adjectives are generated in different positions, suggesting that nouns are not adjectives and vice versa.

Nouns in \ili{Lubukusu} precede adjectives in the NP and as we have already seen in previous sections, adjectives duplicate the prefix system of the nouns that they modify. Consider \isi{word order} in the following simple Adjective-Noun structure.

\ea\label{ex:wasike:21}
\ea[]{
\gll o-mu-soreri o-mu-leyi\\
     1-1-boy 1-1-tall\\
\glt ‘the/a tall boy’}
\ex[*]{
\gll  o-mu-leyi o-mu-soreri\\
     1-1-tall 1-1-boy\\
\glt (Intended: ‘the/a tall boy’)}
\ex[]{
\gll  ku-mu-sala ku-mu-leeyi\\
     3-3-tree 3-3-tall\\
\glt ‘the/a tall tree’}
\ex[*]{
\gll  ku-mu-leyi ku-mu-sala\\
     3-3-tall 3-3-tree\\
\glt (Intended: ‘the/a tall tree’)}
\z
\z

Clearly, the \isi{noun} must precede the adjective in the NP. This is significant because it confirms that \ili{Lubukusu} adjectives and nouns are base generated in different syntactic positions. 

Notice once again that in \REF{ex:wasike:21} as in previous examples that the adjective duplicates the prefix structure of the \isi{noun}: in (\ref{ex:wasike:21}a) the adjective duplicates the \isi{noun}’s \textit{o-mu} prefix, while in (\ref{ex:wasike:21}c), the adjective copies the \isi{noun}’s \textit{ku-mu} prefix. This type of agreement is referred to as concordial agreement in the Bantu literature. It is this concordial agreement that explains the rather surprising distribution facts of \ili{Lubukusu} adjectives illustrated in (\ref{ex:wasike:17}--\ref{ex:wasike:20}) where adjectives seemed to function as subject, object and object of \isi{preposition}. Adjectives in such cases contain enough nominal features of nouns (through the prefixes) and can allow for the dropping or omission of the associated nouns without affecting grammaticality.

Thus NPs such as those in (\ref{ex:wasike:17}--\ref{ex:wasike:20}) that occur without nouns do indeed have a \isi{noun} underlyingly. This observation is supported by the fact that interpretation and comprehension of sentences (\ref{ex:wasike:17}--\ref{ex:wasike:20}) is only possible if one knows or has an idea about the nouns that the adjectives refer to. In other words, these sentences require an appropriate context: they cannot be uttered out of the blue.

The nouns in the underlying structure in (\ref{ex:wasike:17}--\ref{ex:wasike:20}) pass their nominal informationto the adjectival pre-prefix and prefix  through agreement before they are dropped.

To summarize, I have argued that that there is no compelling reason, and there is no convincing evidence to support an analysis of \ili{Lubukusu} adjectives as nouns. It is true that \ili{Lubukusu} adjectives do indeed have nominal features but this is not entirely surprising. \ili{Lubukusu} just happens to be a language (among many others perhaps) where adjectives share some features with nouns. This tendency by adjectives to share some of their morphosyntactic features with nouns has long been recognized in some world languages \citep{Chafe2012}.

The conclusion of this section, then, is that nouns and adjectives exist in \ili{Lubukusu} as separate lexical categories. 

\section{Verbal features of Lubukusu adjectives}\label{sec:wasike:4}

A sub-set of adjectives or more broadly adjectival meanings show a relationship with verbs in \ili{Lubukusu}. In particular some adjectival meanings are expressed by use of verbs rather than true adjectives. In \REF{ex:wasike:22} for example, the \ili{Lubukusu} equivalents for ‘happy’, ‘sad’ and ‘tall’ which are unambiguously adjectives in \ili{English}, are verbs as evidenced by the fact they bear \isi{subject agreement} and tense.

\ea\label{ex:wasike:22}
\ea
\gll Wafula a-a-sangal-il-e.\\
     Wafula 1-\textsc{prs}-happy-\textsc{asp}-\textsc{fv}\\
\glt ‘Wafula is happy.’
\ex
\gll  Wafula a-a-suluny-e.\\
     Wafula 1-\textsc{prs}-sad-\textsc{fv}\\
\glt ‘Wafula is sad.’
\ex
\gll  Wafula a-a-le-il-e.\\
     Wafula 1-\textsc{prs}-tall-\textsc{asp}-\textsc{fv}\\
\glt ‘Wafula has become tall.’
\z
\z

Other examples of \ili{English} adjectives whose equivalents in \ili{Lubukusu} are verbs include the following: \textit{lua} ‘be tired’, \textit{chelewa} ‘be late’, \textit{khalala} ‘be sad’, \textit{meniukha} ‘be shiny’, \textit{imbia} ‘become short’, \textit{bia} ‘become bad’ etc.

In general, the adjectival meanings that are expressed by use of verbs in \ili{Lubukusu} are stage-level. These are either ‘non-permanent’, temporary states or continuing processes or states that are yet to reach their final state. For example \textit{sangala} ‘be happy’ describes a temporary, transient state (in contrast to having a happy personality which is expressed by an adjective as will be illustrated below). Thus to say \textit{Wafula aasangalile} ‘Wafula is happy’ means Wafula is happy now, but it doesn’t mean that he will necessarily be happy later today or tomorrow. Similarly, \textit{lea} ‘be tall’ does not designate a final state. It describes a process in progress. Thus to say \textit{Wafula aaleile} ‘Wafula has grown tall’ means Wafula has grown taller from last time you saw him, and it doesn’t suggest that he is done growing. In contrast, expressing the fact that Wafula is a tall person (as his final tall state) is an individual-level \isi{predicate}. This in \ili{Lubukusu} is expressed by an adjective as will be shown below. 

Adjectival verbs such as \textit{sangala} ‘be happy’ and \textit{lea} ‘become tall’ occur only in predicative structures, and therefore they are translated in \ili{English} as predicative adjectives. Notice that the equivalent \ili{English} predicative adjectives are obligatorily preceded by BE in declarative sentences as well as in imperatives. 

As already pointed out temporary states and on-going processes adjectival meanings in \ili{Lubukusu} are expressed by use of verbs, but permanent final-state attributive adjectival meanings are expressed by use of adjectives. Adjectival meanings of this later type describe qualities of nouns that are enduring; qualities that are non-temporary. Where temporary states and on-going adjectival meanings have corresponding permanent attributive meanings, these are expressed by use of adjectives. To illustrate consider \REF{ex:wasike:23} where temporary states and their corresponding permanent states are provided.

\ea\label{ex:wasike:23}
\ea
\gll Wafula o-mu-sangafu\\
     Wafula 1-1-happy\\
\glt ‘happy Wafula’ (Individual-level)
\ex
\gll  Wafula a-a-sangal-il-e.\\
     Wafula 1-\textsc{prs}-happy-\textsc{asp}-\textsc{fv}\\
\glt ‘Wafula is happy.’ (Stage-level)
\ex
\gll  Ku-mu-sala ku-mu-leyi\\
     3-3-tree 3-3-tall\\
\glt ‘tall tree’ (Individual level)
\ex
\gll  Ku-mu-sala kw-a-le-il-e.\\
     3-3-tree \textsc{sa}-\textsc{prs}-tall-\textsc{asp}-\textsc{fv}\\
\glt ‘The tree has become tall.’ (Stage-level)
\z
\z

In (\ref{ex:wasike:23}a), there is some permanence to Wafula’s happiness state. Here Wafula has a happy predisposition; he is naturally a happy person. In (\ref{ex:wasike:23}c) \textit{kumuleyi} ‘tall’ is an attribute of \textit{kumusala} ‘tree’: the tree has the attribute tall; it is an attribute that is not expected to change any time soon. Dixon is therefore correct when he observes that “…if a language has verbs derived from adjectives, then the adjective is preferred for describing a fairly permanent property and the \isi{verb} for referring to a more transient state” \citep[32]{Dixon2004}.

Notice that adjectives that express the attributive adjectival meanings (i.e. the true adjectives) in \ili{Lubukusu} can be used predicatively. As illustrated in \REF{ex:wasike:24}, when used predicatively, they retain their adjective forms and do not become verbs.

\ea\label{ex:wasike:24}
\ea
\gll Wafula a-li o-mu-sangafu.\\
     Wafula 1-be 1-1-happy\\
\glt ‘Wafula is a happy person.’
\ex
\gll  Ku-mu-sala ku-li ku-mu-leyi.\\
     3-3-tree 3-be 3-3-tall\\
\glt ‘The tree is tall.’
\z
\z

Thus attributive adjectives can be used as predicatively just like \ili{English}. The most significant difference between \ili{English} and \ili{Lubukusu} (from the point of view of adjectives) is that \ili{Lubukusu} (but not \ili{English}) makes a distinction between the way it expresses temporary adjectival states or meanings on the one hand and permanent attributive qualities. The temporary states and on-going process adjectival meanings are expressed by verbs in \ili{Lubukusu}, but they are expressed by adjectives in \ili{English}. 

A question that arises is whether verbs that express temporary and on-going processes are adjectives at some level or not. A straight forward way of determining this is showing that adjectival verbs differ in some significant way from regular verbs. This has been shown to be true in \ili{Mohawk} and Vaeakau-Taumako. In these languages, verbs that are used to express adjectival meanings are different from regular verbs. For example \ili{Mohawk} verbal adjectives contrast with regular verbs in not taking certain aspectual markers and future tense\footnote{But see \citet{Chafe2012} who found nothing significant that distinguishes ‘adjectival verbs’ from regular verbs in \ili{Seneca}, a language that is related to Mohawk} \citep{Baker2001}. In Austronesian languages such as Vaeakau-Taumako, verbal adjectives differ from regular verbs in their ability to occur without tense-aspect-mood marking \citep{NaessHovdaugen2011}. 

It is therefore reasonable to argue that adjectival verbs are adjectives in \ili{Mohawk} and Vaeakau-Taumako at some level because they differ significantly from regular non-adjectival verbs.

In \ili{Lubukusu}, there is no compelling reason to make a similar argument. This is because adjectival verbs that describe temporary non-final states and continuing processes are not different from regular verbs in terms of tense-aspect-mood marking. To illustrate consider the tense-aspect marking on \textit{sangala} ‘be happy’ in the following data.

\ea\label{ex:wasike:25}
\ea
\gll Wafula a-la-saangal-a\\
     Wafula 1-\textsc{fut}-happy-\textsc{fv}\\
\glt Wafula will be happy today.
\ex
\gll  Wafula a-kha-saangal-e\\
     Wafula 1-\textsc{fut}-happy-\textsc{fv}\\
\glt ‘Wafula will be happy tomorrow/next week.’
\ex
\gll  Wafula a-li-saangal-a\\
     Wafula 1-\textsc{fut}-happy-\textsc{fv}\\
\glt ‘Wafula will be happy sometime in the remote future.’
\ex
\gll  Wafula a-a-saangal-a\\
     Wafula 1-\textsc{pst}-happy-\textsc{fv}\\
\glt ‘Wafula was happy a long time ago.’
\z
\z

Clearly, \textit{sangala} can occur with any tense and aspectual marker just like any regular \isi{verb}. I take this to be evidence that \ili{Lubukusu} adjectival verbs that describe temporary states and continuing processes are verbs and nothing more. They express adjectival meanings, but they are verbs in the true sense of the word.

What this means is that the \ili{Lubukusu} adjective system is different from that of \ili{Mohawk} and Vaeakau-Taumako in spite of the apparent similarities. Both languages and \ili{Lubukusu} express some adjectival meanings by use of verbs. But while all verbal adjectives in \ili{Mohawk} and Vaeakau-Taumako can be argued to be adjectives, the \ili{Lubukusu} ones are not: they are true verbs.

\section{Conclusion}\label{sec:wasike:5}
In spite of the fact that words which express adjectival meanings in \ili{Lubukusu} have nominal features on the one hand and verbal features on the other, there is enough strong evidence that support the existence of adjective as a distinct lexical category. \ili{Lubukusu} adjectives have a prefix system that is identical to that of nouns, and the adjectives seem to function as subject and object, but this doesn’t make them nouns. They remain adjectives and they acquire these features and functions by virtue of being modifiers of nouns. With regards to the adjectival meanings that are expressed by use of verbs, I showed that only stage-level \isi{predicate} readings are expressed by use of verbs in \ili{Lubukusu}. Individual-level \isi{predicate} readings are expressed by use of adjectives. I also argued that ‘adjectival verbs’ in \ili{Lubukusu} are real verbs. For this reason, \ili{Lubukusu} is different from \ili{Mohawk} and Vaeakau-Taumako where adjectival verbs have been argued to be adjectives at some level in the grammar. With regards to the existence of the lexical category of adjective, \ili{Lubukusu} is like \ili{English} (but unlike \ili{Mohawk} and Vaeakau-Taumako): it has a distinct lexical category of adjective in its grammar. But this is not to suggest that \ili{Lubukusu} and \ili{English} have identical adjective systems. There are significant differences, one of which is that stage-level \isi{predicate} readings in \ili{Lubukusu} are expressed by use verbs, but in \ili{English} it is adjectives that are used.

\section*{Acknowledgements}
I would like to sincerely thank two anonymous reviewers for their useful detailed comments and suggestions. All errors and shortcomings are my own.

\section*{Abbreviations}
Unless indicating person, numbers in glosses indicate \isi{noun class} prefixes and pre-prefixes. Abbreviations follow Leipzig glossing conventions, with the following exceptions:
\bigskip

\noindent\begin{tabular}{ll}
\textsc{asp} &  aspect\\
\textsc{fv}  & final vowel\\
\textsc{hab}  & habitual\\
\end{tabular}
\begin{tabular}{ll}
\textsc{sa}  & \isi{subject agreement}\\
\textsc{sta}  & stative\\
\\
 \end{tabular}
 
\sloppy
\printbibliography[heading=subbibliography,notkeyword=this]

\end{document}