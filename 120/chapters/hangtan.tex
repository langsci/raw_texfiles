\documentclass[output=paper,
modfonts
]{langscibook} 
% \bibliography{localbibliography}
\ChapterDOI{10.5281/zenodo.1251722}

 
\title{Language endangerment in Southwestern Burkina: A tale of two Tiefos}
\author{Abbie Hantgan-Sonko %\affiliation{SOAS}
}
\abstract{Most of the thirty or so small-population languages of southwestern Burkina Faso are still reasonably viable in spite of the spread of Jula as the dominant regional vernacular. An unusual case is Tiefo, which is really two distinct but closely related and geographically contiguous Gur languages. One, here dubbed Tiefo-N, was spoken in the villages of Noumoudara and Gnanfongo (Nyafogo). The other, Tiefo-D, was spoken in the nearby village cluster of Dramandougou. Several other ethnically Tiefo villages in the zone had already been completely Jula-ised by the mid-20th Century. Tiefo-N is moribund (a handful of ageing semi-speakers in Gnanfogo, none in Noumoudara), the villagers having gone over to Jula. By contrast, Tiefo-D is in a relatively comfortable bilingual relationship to Jula and is still spoken to some extent even by children, though everyone also speaks Jula. This paper clarifies the relationship between Tiefo-N and Tiefo-D and addresses the question why the two languages have had such different fates.}

\begin{document}
 \maketitle
 
%%please move the includegraphics inside the {figure} environment
%%\includegraphics[width=\textwidth]{}

 

\section{Tiefo}

\ili{Tiefo} (pronounced [čɛfɔ]) is an important ethnic group in southwestern Burkina Faso. There are some 20 villages that still consider themselves ethnically \ili{Tiefo}. The core is constituted by the villages of Noumoudara, Gnanfogo, and Dramandougou,\footnote{Alternative spellings are Numudara, Nyafogo, and Daramandougou or Daramandugu.} the latter two being really clusters of several distinct physical settlements. This core is located directly on (in the case of Noumoudara) or to the east of the highway from \ili{Bobo} Dioulasso to Banfora. There are other \ili{Tiefo} villages scattered around, including one to the west of \ili{Bobo} Dioulasso (on the road to Orodara) and others east and southeast of the core.\footnote{The village of Tiefora, east of Banfora on the road to Sideradougou and Gaouwa, is not far from Dramandougou, but in spite of its name it is apparently not \ili{Tiefo} ethnically.}

\ili{Tiefo} belongs to the large Gur language family, which dominates much of Burkina Faso (including the large-population \ili{Mooré} language of the Mossi ethnicity) and spreads westward into parts of \isi{Ghana}, \isi{Niger}, \isi{Togo}, Benin, and \isi{Nigeria}. \citet{Manessy1982}, who worked out the genetic sub-groupings within Gur, examined unpublished \ili{Tiefo} data from André Prost and concluded that \ili{Tiefo} constituted its own subgroup, with no especially close relatives.

The published descriptive material on \ili{Tiefo} primarily includes Kerstin Winkelmann’s invaluable monograph (in {German}) on \ili{Tiefo}-D \citealt{Winkelmann1998}). It consists of a descriptive reference grammar (emphasising phonology and morphology) and a basic lexicon. Winkelmann was part of a \ili{German}-staffed project on Gur languages and cultures that was active in the 1990’s but has now disappeared due to retirements of senior personnel and career switches by Winkelmann and others. Her fieldwork was carried out in Dramandougou, but she also did brief survey work (core lexicon and a little morphology) on \ili{Tiefo}-N.

Winkelmann commented that \ili{Tiefo}-N, even during her fieldwork period (1990--94), was at a much more advanced state of decline than \ili{Tiefo}-D. She was able to elicit a little data from two elderly men in Noumoudara and somewhat more from semi-speakers in Gnanfogo. The \ili{Tiefo}-N lexical material was included, alongside \ili{Tiefo}-D data, in her lexicon. She calculated cognate counts for the Swadesh 100-word list between Dramandougou and either Noumoudara or Gnafongo in the 75--77 percentage range, with cognates partially disguised by sound changes and grammatical differences. She stated flatly that \ili{Tiefo}-D was not understood in either of the \ili{Tiefo}-N communities.\footnote{“Die in den beiden weiteren untersuchten Dörfern gesprochenen Cɛfɔ-Dialekte weichen ganz erheblich von dem von Daramandugu ab. Weder in Nyafogo noch in Numudara ist das Daramandugu-Cɛfɔ verstehbar” \citep[5]{Winkelmann1998}.} On the other hand, there was good inter-comprehension between Noumoudara and Gnafongo. A reasonable conclusion is that \ili{Tiefo}-D and \ili{Tiefo}-N are distinct languages using normal linguistic (as opposed to political) criteria.

Given Winkelmann’s description of the dire language situation in Gnanfogo in the early 1990’s, I was rather surprised to find some speakers in \ili{Tiefo}-N in that village when I arrived in the \ili{Bobo} Dioulasso area about a decade later in 2012. In retrospect, it may be that Winkelmann slightly underestimated the state of \ili{Tiefo}-N in Gnafongo during her brief stay there, in part because of a misunderstanding of nominal plural formation. She stated that Gnafongo informants had difficulties producing such plurals, which a reader could understand as implying that the language was only imperfectly remembered by a few semi-speakers. It turns out, however, that \ili{Tiefo}-N pluralises many nouns by lengthening the final vowel, i.e.  singular …Cv1 becomes …Cv1v1. This corresponds to the productive \ili{Tiefo}-D plural with -r followed by a copy of the stem-final vowel, i.e. …Cv1 becomes …Cv1-rv1. Evidently Gnafongo \ili{Tiefo}-N lost the *r and the remaining identical vowels coalesced into a long vowel, a phonetically subtle pluralisation process that could be missed during short-term fieldwork by a linguist who was not primed to look for it.

\largerpage
Given the urgency of the language situation and the lack of substantial documentation of \ili{Tiefo}-N, I did some 5 months fieldwork with elderly Gnafongo speakers between August 2013 and the following January. Subsequently, Jeffrey Heath collected flora-fauna terminology for \ili{Tiefo}-N and local \ili{Jula} in Gnafongo.\footnote{Aminata Ouattara, a Burkina linguistics student of ethnic \ili{Tiefo} origin, was also continuing fieldwork on \ili{Tiefo}-N as of early 2015.} In order to illustrate some of the true consequences of \isi{language contact}, a greatly misunderstood phenomenon in West Africa, I show the examples of two varieties of one moribund language. I argue that our methodology is no longer data driven, and that because we have a certain set of ideals in place as to what happens when one language comes into contact with another, we are blind to the real circumstances.  Instead of mourning so-called ``\isi{language death}'' \citep{NettleRomaine2000,Price1984}, we should be celebrating the diversity of new mixed languages which are born when speakers come into contact with one another. Through an examination of different sociological, historical, and geographic paths, we see that one language has become in fact two. However, without an interdisciplinary methodology that starts from the ground up, our theoretical footing will be unsound and vice versa. In order to illustrate the differences between the presently existing \ili{Tiefo} varieties, and because there has been such little attention paid to \ili{Tiefo}-N, I present an overview and comparison of the \isi{major} grammatical features of \ili{Tiefo}-N and \ili{Tiefo}-D. The main phonological features are illustrated in \sectref{sec:hangtan:2} and the morphology in \sectref{sec:hangtan:3}. \sectref{sec:hangtan:3.4} discusses the differences in the pronominal (which in turn is related to the tense/aspect) systems of the two varieties, discussed in the following section, 3.5.

Then, \sectref{sec:hangtan:5} provides an exploration of the reasons thus far provided in the literature concerning the different fates of the \ili{Tiefo} villages. While geographical and sociolinguistic reasons have been referenced in the past, the current discussion explores the historical causes of the divergent dialects.

\section{Phonology}\label{sec:hangtan:2}

\ili{Tiefo}-N and \ili{Tiefo}-D have similar consonant inventories: stops plus \isi{palatal} affricates /p b t d tʃ dʒ k ɡ kp ɡb/, nasals /m n ɲ ŋ ŋm/, \isi{fricatives} /f s ɣ ʕ/, \isi{glottal} /ʔ/, and nonnasal sonorants /w l r j/. Note the distinction between the voiced \isi{pharyngeal} /ʕ/ (cf. \ili{Arabic}) and \isi{glottal} /ʔ/.

\begin{table}
\caption{Tiefo consonantal inventory.} 
\label{tab:hangtan:1}
\begin{tabularx}{\textwidth}{lXlXXll}
\lsptoprule
 & {\bfseries Labial} & {\bfseries Alveolar} & {\bfseries Palatal} & {\bfseries Velar} & {\bfseries Pharyngeal} & {\bfseries Glottal}\\
\midrule
{Plosive} & p b& t d&  & k ɡ&  & ʔ\\
{Nasal} & m ŋm& n& ɲ& ŋ&  & \\
{Fricative} & f& s& (ʃ)& ɣ& ʕ& \\
{Affricate} & k͡p ɡ͡b& c j&  &  &  & \\
{Approximant} & w& l& y&  &  & \\
{Trill/tap} &  & r&  &  &  & \\
\lspbottomrule
\end{tabularx}


\end{table}

Absent from the consonantal inventory of both languages are several consonants reconstructed for Proto-Gur \citep{Naden1989}: voiced \isi{implosives} /ɓ ɗ ʄ/, voiced \isi{palatal} stop /ɟ/, voiced \isi{affricate} /dʒ/, and labiodental \isi{fricative} /v/.

\ili{Tiefo}-N and \ili{Tiefo}-D likewise have similar vowel inventories, which are shared with other languages of the zone. There are seven vowel qualities, including high /i u/, low /a/, and two pairs of mid-height vowels, [+ATR] /e o/ and [-ATR] /ɛ ɔ/. The high and low vowels are ATR-neutral and may combine with either type of mid-height vowel. In \ili{Tiefo}-D \citep[20, 23]{Winkelmann1998} but not \ili{Tiefo}-N, phonemes /i u/ have optional [-ATR] phonetic variants in words with a following [-ATR] mid-height vowel. Proto-Gur is reconstructed with a ten-vowel system, including [±ATR] distinctions in high and low as well as mid-height vowels.

  
\begin{figure} 
% \includegraphics[width=\textwidth]{hangtanacal2014LSPtemplate-img1.png}
\centering
\parbox{.3\textwidth}{
\begin{vowel}
  \putcvowel{i}{1}
  \putcvowel{e}{2}
  \putcvowel{ɛ}{3}
  \putcvowel{a}{4}
  
  \putcvowel{ɔ}{6}
  \putcvowel{o}{7}
  \putcvowel{u}{8} 
\end{vowel}
}


\caption{Tiefo vocalic inventory.}
\label{fig:1}
\end{figure}

\ili{Tiefo}-N and \ili{Tiefo}-D also have the same three tone levels. High tones are marked by an acute accent [á], low tones by a grave accent [à]. Mid tones are written either without an accent \citep{Winkelmann1998} or more explicitly with a macron [ā].

In spite of the nearly identical phonemic inventories between the two languages, many actual pairs of \ili{Tiefo}-N and \ili{Tiefo}-D cognate words are disguised by phonological differences. Some examples are in \tabref{tab:hangtan:2}, which pools data from Winkelmann (KW) and myself (AH). Correspondences that occur in more than one set even in this small corpus are \ili{Tiefo}-D \isi{glottal} stop or zero for \ili{Tiefo}-N medial [ɡ], \ili{Tiefo}-D [c] for \ili{Tiefo}-N [s], and \ili{Tiefo}-D [d] for \ili{Tiefo}-N [ʒ, j].

\begin{table}
\caption{Tiefo cognates.}
\label{tab:hangtan:2}
\begin{tabularx}{\textwidth}{XXXX}
\lsptoprule
{\bfseries  {Tiefo}-D (KW)} & {\bfseries  {Tiefo}-N (KW)} & {\bfseries  {Tiefo}-N (AH)} & {\bfseries Gloss}\\
\midrule 
{ blaʔa {\textasciitilde} bla} & { báráɡà {\textasciitilde} báláɡà} & { bārāʔá} & { ‘river’}\\
{ dráⁿ} & { dáraɡá} & { dárá} & { ‘home’}\\
{ brà(ʔà)} & { bàɡàle, bàrài} & { bàɣàʔè} & { ‘hair’}\\
{ buɔⁿ} & { bɔʔɔⁿ, bɔɔⁿ} & { būɔ ⁿ} & { ‘dog’}\\
{ ceʔe} & { sereɡe} & { sérííⁿ} & { ‘skin’}\\
{ cicí} & { sisiu} & { ʃíʃíʔī} & { ‘urine’}\\
{ cùru} & { suru} & { sūsúⁿ} & { ‘millet cake’}\\
{ dè} & { ʒàɡa, yèà} & { jéjāʔā} & { ‘sun’}\\
{ dɛ} & { ʒɔ} & { ndɛ} & { ‘elder brother’}\\
\lspbottomrule
\end{tabularx}


\end{table}

\section{Morphology}\label{sec:hangtan:3}

Morphological features found in \ili{Tiefo}-N but not in \ili{Tiefo}-D are a definite prefix (\sectref{sec:hangtan:3.1}), a specific set of plural suffixes (Section3.2), and an ablaut-like system of adjective-\isi{noun} agreement (\sectref{sec:hangtan:3.3}).

\subsection{Definite prefix}\label{sec:hangtan:3.1}

The dialect of \ili{Tiefo}-N in Gnafongo has what I will call a definite prefix (but see below for qualms about this categorisation). It has three variants depending on the dominant vowel of the stem: [e-] before nouns with an [e] vowel in the stem, [o-] before nouns with a back vowel [o ɔ u], and [a-] before nouns with [a] or [ɛ] vowel in the stem. Examples are in \tabref{tab:hangtan:3}. The stem ‘moon’ irregularly has [a-] instead of expected [e-].

\begin{table}
\caption{Tiefo definite.}

\begin{tabularx}{\textwidth}{XX}
\lsptoprule
{\bfseries Noun (\ili{Tiefo}-N, Def-Sg)} & {\bfseries Gloss}\\
\midrule 
{ è-kēʔēⁿ} & { ‘spoon’}\\
{ è-j\=oēⁿ} & { ‘neck’}\\
{ ē-sāè} & { ‘ground’}\\
{ ò-ŋ\=oʕ\=o} & { ‘mosquito’}\\
{ \=o-fláɲ\=o} & { ‘baobab’}\\
{ {ò-sī}\={ɔ}{ⁿ}} & { ‘salt’}\\
{ ò-ɲū} & { ‘water’}\\
{ {à-bīt}\={ɛ}{ʔ}{}{\`{ɛ}}} & { ‘leaf’}\\
{ à-fērēé} & { ‘moon’}\\
{ {ā-k}{}{\'{ɛ}}{r}{}\={ɛ} {}\={ɛ}} & { ‘hand’}\\
{ ā-fíyāʕā} & { ‘field’}\\
\lspbottomrule
\end{tabularx}

\label{tab:hangtan:3}
\end{table}

The definite marker is generally optional in the singular but in some cases is obligatory in the plural. However, when the \isi{noun} is followed by a quantifier or by an adjective, the definite prefix is omitted. This suggests that the ``definite'' prefix functions in part to indicate that the \isi{noun} is free of modifiers.

This is more clearly the case in \ili{Tiefo}-D. \citep[132]{Winkelmann1998} describes the \ili{Tiefo}-D prefix [e-], infrequently [o-], as obligatory in citation forms. She confirms for \ili{Tiefo}-D that it vanishes in the presence of a determiner (possessor, demonstrative).

\subsection{Plural suffixes}\label{sec:hangtan:3.2}

Proto-Gur has been reconstructed as having a complex system of \isi{noun class} markers in the form of paired singular-plural combinations \citet{Naden1989}, along the lines of other Niger-\isi{Congo} families including Bantu. Many extant Gur languages still have class suffixes, and some have prefixes as well \citep{MieheEtAl2012}.

In addition to lengthening of the final vowel (mentioned above), a number of other singular/plural relationships occur in \ili{Tiefo}-N. Examples are shown in \tabref{tab:hangtan:4}.


\begin{table}
\caption{Tiefo-N plural suffixes.}
\label{tab:hangtan:4}
\begin{tabularx}{\textwidth}{lXXX}
\lsptoprule
 & {\bfseries Singular} & {\bfseries Plural} & {\bfseries Gloss}\\
\midrule
{ a.} & { nāmi} & { \=o-nāmī-j\=o} & { ‘child’}\\
& { y\=o nāmí} & { y\=o nāmí-j\=o} & { ‘fruit’}\\
& { ɲ\=o} & { \=o-ɲí-j\=o} & { ‘person’}\\
& { bī} & { bī-j\=o} & { ‘baby’}\\
& { ŋmāʕa bí} & { ŋmāʕa bí-j\=o} & { ‘star’}\\
\tablevspace
{ b.} & { c{\'{ɔ}}mī-ī} & { \={n} {}-c{\'{ɔ}}mī} & { ‘bird’}\\
& { ɲ{\'{ɔ}}mī-ī} & { \={ɛ} {{}-ɲ}{}{\'{ɔ}}{mī}} & { ‘toe’}\\
\tablevspace
{ c.} & { yē} & { yē-ʔé} & { ‘year’}\\
& { {jāá} {b}\={ɔ} {ⁿ}} & { {jāá} {b}\={ɔ} {{}-}\={ɔ}{ⁿ}} & { ‘girl’}\\
\tablevspace
{ d.} & { ɡbé-ēⁿ} & { ɡbē} & { ‘stool’}\\
\tablevspace
{ e.} & { {ʒ}{}{\'{ɔ}}{w}\={ɛ}{ⁿ}} & { \'{ɛ}{}-ʒ{\'{ɔ}}wīⁿ} & { ‘neck’}\\
\tablevspace
{ f.} & { fēreʔé} & { fērēʔē} & { ‘moon’}\\
\tablevspace
{ g.} & { {d}\={ɔ} {{}-j}\={ɛ}} & { {d}\={ɔ} {{}-r}\={ɔ}} & { ‘man’}\\
\lspbottomrule
\end{tabularx}

\end{table}

There are also some nouns that appear to have no singular-plural difference, such as [búɡúnɛ ] ‘beans (variety)’, either because of recent morphological loss or because these nouns do not lend themselves to individuation.

Winkelman reported a \ili{Tiefo}-D plural /-O/ (by which she indicates an archiphoneme representing either for [o] or [ɔ] depending on the [ATR] class of the stem), though for animates only. This corresponds to the [-j\=o] (always after i) in (4a), though often not in the same words across the \ili{Tiefo} varieties.  Some of the \ili{Tiefo}-N glosses in (4a) are inanimate (‘star’, ‘fruit’), but these are compounds including ‘child’ or ‘baby’, e.g. ‘tree-child’ = ‘fruit’. The stem ‘man’, (4g) is a rare case where \ili{Tiefo}-N has a plural [-rV] (with copied vowel quality), the productive plural in \ili{Tiefo}-D. Other \ili{Tiefo}-N singular/plural patterns (4b-f) lack known \ili{Tiefo}-D matches, and are difficult to connect to reconstructed inventories of Proto-Gur \isi{noun class} markers listed by \citet{Naden1989}.

\subsection{Adjectival harmony}\label{sec:hangtan:3.3}

In \ili{Tiefo}-N, the final vowels of certain adjectives harmonise with the vowel of the definite prefix of the modified \isi{noun}. Consider the forms for harmonising ‘black’ in examples (\ref{ex:hangtan:1}-\ref{ex:hangtan:2}) and for nonharmonising ‘big’ (\ref{ex:hangtan:2}-\ref{ex:hangtan:4}). The vowel quality of the prefixes on ‘house’ and ‘man’ match that of the final-vowel of ‘black’. This may reflect an archaic suffixal agreement pattern, creating a construction of the type *[CLASS-\isi{noun} adjective-CLASS]. Synchronically it could be described as a terminal ablaut (i.e. mutation of the final vowel into another quality). There is no similar mutation of the adjective ‘big’, which has an invariant shape in (\ref{ex:hangtan:2}-\ref{ex:hangtan:4}).

\ea\label{ex:hangtan:1}
\gll wà-          wūʕú j\=ob-á\\
     \textsc{def} hut black\\
\glt ‘the black house’
\z

% \todo{are these accents as intended?}
\ea\label{ex:hangtan:2}
\gll ò-           d\`{ɔ}\`{ɛ} jób-\=o\\
     \textsc{def} man black\\
\glt ‘the black man’
\z

\ea\label{ex:hangtan:3}
\gll à-           wūʕú s{\={ã}}ɡbānāʔà\\
     \textsc{def} hut big\\
\glt ‘the big house’
\z

\ea\label{ex:hangtan:4}
\gll ò-           d\`{ɔ}\`{ɛ} s{\={ã}}gbānāʔà\\
     \textsc{def} man          big\\
\glt ‘the big man’
\z

\subsection{Pronouns}\label{sec:hangtan:3.4}

The subject personal pronouns of \ili{Tiefo}-N are those in \tabref{tab:hangtan:5}. The singular but not plural forms vary depending on the aspect (\isi{perfective}/imperfective) of the \isi{clause} (imperfective includes \isi{progressive}). The basic \ili{Tiefo}-D forms \citep[140]{Winkelmann1998} are shown for comparison; specifically imperfective (‘present’) and negative Teifo-D combinations are omitted. \ili{Tiefo}-D distinguishes animacy in the \textsc{3sg}, and also has uses the distant demonstrative [bó] as a discourse-anaphoric \textsc{3sg} \isi{pronoun}.

\begin{table}
\caption{Tiefo pronouns.}
\label{tab:hangtan:5}
\begin{tabularx}{\textwidth}{lllQ} 
\lsptoprule
& {\bfseries  {Tiefo}-N Imperfective} & {\bfseries Perfective} & {\bfseries  {Tiefo}-D}\\
\midrule
{\textsc{1sg}} & { ɲí} & { ān} & { no}\\
{\textsc{1pl}} & { é} & { é} & { ʔejuò}\\
{\textsc{2sg}} & { mì} & { m} & { mo}\\
{\textsc{3pl}} & { nā} & { nā} & { buò}\\
{\textsc{3sg}} & { kā \=o} & { n \=o} & { {ʔ}\={ɔ}{ⁿ} {(anim),} {ʔà} {(inan),} {bó} {(anaph)} ʔò}\\
{\textsc{3pl}} & { ɲí} & { ān} & { no}\\
\lspbottomrule
\end{tabularx}
\end{table}

For \ili{Tiefo}-N, \textsc{1sg} subject is exemplified examples (\ref{ex:hangtan:5}-\ref{ex:hangtan:6}), \textsc{1pl} in (\ref{ex:hangtan:7}-\ref{ex:hangtan:8}).
\ea\label{ex:hangtan:5}
\gll ɲí wɔʕ\`{ɔ} bè kṹ\\
     \textsc{1sg}.\textsc{ipfv} \textsc{prog} come today\\
\glt ‘I am coming today.’
\z

\ea\label{ex:hangtan:6}
\gll n bàʔ jànā\\
     \textsc{1sg} come yesterday\\
\glt ‘I came yesterday.’
\z

\ea\label{ex:hangtan:7}
\gll é wɔʕ\`{ɔ} bè kṹ\\
     \textsc{1pl} \textsc{prog} come today\\
\glt ‘We are coming today.’
\z

\ea\label{ex:hangtan:8}
\gll é bàʔ jànā\\
     \textsc{1pl} come yesterday\\
\glt ‘We came yesterday.’
\z

Unlike \ili{Tiefo}-D, \ili{Tiefo}-N does not currently distinguish animacy or anaphoricity (e.g. reflexives) in the \textsc{3sg} \isi{pronoun}. This might be due to recent grammatical simplification, and the occasional use of \ili{Jula} \textsc{3sg} pronouns shows that \isi{language contact} has impacted the pronominal system.

\subsection{Verbal aspectual inflection}\label{sec:hangtan:3.5}

Verbal aspectual morphology in \ili{Tiefo}-N is more intricate than nominal or pronominal morphology. The main opposition is between imperfective and \isi{perfective} (sometimes called ‘continuous’ and ‘neutral’, respectively).

In one \isi{verb} class, the imperfective is unsuffixed while the \isi{perfective} is marked by a low- or mid-toned suffix -ra {\textasciitilde}-la \tabref{tab:hangtan:6}. It can be nasalised to -na, see ‘arrive’ (\tabref{tab:hangtan:6}, row (d)).

\begin{table}
\caption{Tiefo aspectual affixation.}
\label{tab:hangtan:6}
\begin{tabularx}{\textwidth}{lXXX}
\lsptoprule
 & {\bfseries Imperfective} & {\bfseries Perfective} & {\bfseries Gloss}\\
\midrule
{a.} & { jē} & { jé-rā} & { ‘enter’}\\
{b.} & { jè} & { jē-rà} & { ‘walk’}\\
{c.} & { bī{\'{ɛ}}{ }} & { {bī}\={ɛ} {{}-rà} } & { ‘farm’ }\\
{d.} & { d{\={ã}}} & { dā-nà} & { ‘arrive’}\\
{e.} & { dīò} & { dī\=o-là} & { ‘sell’}\\
\lspbottomrule
\end{tabularx}


\end{table}

Several other verbs show ablaut-like vocalic mutations, in some cases along with other internal changes or affixes. Two multiply attested patterns are vowel to [a] (row (a) in \tabref{tab:hangtan:7}) and [a] to [e/ɛ] (row (b) in \tabref{tab:hangtan:7}). Mutation types attested once are in (row (c) in \tabref{tab:hangtan:7}).

\begin{table}
\caption{Tiefo aspectual mutation.}
\label{tab:hangtan:7}
\begin{tabularx}{\textwidth}{lXXX}
\lsptoprule
 & {\bfseries Imperfective} & {\bfseries Perfective} & {\bfseries Gloss}\\
\midrule
{\bfseries a.} & { sè} & { sá} & { leave/go}\\
& { bè} & { bāʔ} & { come}\\
& { bē} & { b-là} & { tire}\\
& { díʔ\=\i} & { díā} & { eat}\\
& { d\=oʔò} & { dāà} & { plant}\\
& { d\=or\=oʕò} & { dárāʕā} & { buy}\\
{\bfseries b.} & { ɲānā} & { ɲéné} & { stop/stand}\\
& { náʔā} & { {n}{\'{ɛ}}{n}\={ɛ}} & { wash (clothing)}\\
& { dárāà} & { {d}\={ɛ} {r}{\`{ɛ}}{\`{ɛ}}} & { rip}\\
& { bārá} & { {bēr}\`{ẽ}} & { sweep}\\
& { jāʕà} & { {j}\={ɛ} {ɡ}{\`{ɛ}}} & { break}\\
{\bfseries c.} & { ɲ{}-à} & { ɲ{}-ū} & { drink}\\
& { bó} & { {bw}\={ɛ}} & { tie}\\
\lspbottomrule
\end{tabularx}
\end{table}

An important difference between the two \ili{Tiefo} varieties is that \ili{Tiefo}-N has a preverbal morpheme wɔ ʕɔ that marks \isi{progressive} aspect. No similar preverbal \isi{progressive} or imperfective morpheme is reported for \ili{Tiefo}-D. It is possible, however, that the \ili{Tiefo}-N form is archaic, reflecting a proto-form *bo ‘be’ \citet{Manessy1982}.

\section{Influence from Jula}\label{sec:hangtan:4}

The data in 6 consist of verbs which are suffixed with [-rV] or allomorphs [l]{\textasciitilde}[n] in an aspectual form known as ‘neutral’ or \isi{perfective}. The suffix may be a borrowing from \ili{Jula} since the \isi{perfective} suffix in \ili{Jula} is [-ra] with allomorph [-la]. An example illustrating the [-rV/lV] suffix in Gnanfongo \ili{Tiefo} is the \isi{verb} ‘hide’, borrowed directly from \ili{Jula} as [dūɡū], ‘hidden’ [dūɡū-là]. Many of the verbs in this category are probable borrowings from \ili{Jula}, even though a neutral suffix [-da/ra/ta] is attested in other Gur languages. However, according to \citet{Naden1989}, most verbal markers are treated as particles rather than affixes in other Gur languages. The most widely marked inflectional category in Gur languages is expressed through a contrast between the continuous (imperfective) and a form described as ‘neutral’. Therefore, the neutral suffix in \ili{Tiefo} is likely related to the particle found in other Gur languages, but possibly has been reanalysed in Gnanfongo \ili{Tiefo} as a \isi{perfective} suffix on \ili{Jula} borrowings.

According to lexical comparisons between \ili{Tiefo} and other Gur languages by \citet{Manessy1982}, there is but a mere 28 out of 435 correspondence, 20 percent. With sample correspondences shown in \tabref{tab:hangtan:8}, between the data gathered by the author from Gnanfongo \citegen{Manessy1982} from Dramandougou and surrounding Gur languages, we do see, however limited, some strong evidence for a a related source.

\begin{table}
\caption{Correspondences between Tiefo and Gur languages \citep[146]{Manessy1982}}

\small
\label{tab:hangtan:8}
\fittable{
\begin{tabular}{l@{~}l@{~}l@{}l@{~}l@{}l@{}lp{1cm}p{1cm}@{}l}
\lsptoprule
{\bfseries  {Tiefo} (AH)} & {\bfseries  {Tiefo} (GM)} & {\bfseries Viemo} & {\bfseries Doɣose} & {\bfseries Gan} & {\bfseries Lobi} & {\bfseries Dyan} & {\bfseries Kulango} & {\bfseries Loron} & {\bfseries Gloss}\\
\midrule
{ p{\'{ĩ}}{\'{ĩ}}} & { pini} & { pinyɔ} & { p{\'{ĩ}}{\'{ĩ}}se} &  &  &  & { p{\'{ĩ}}{\'{ĩ}}} & { piniɡu, pininyu} & { ‘excrement’}\\
{ kāʕà} & { kaʔa} & { kaasɔ} & { kaase} & { kasa} &  &  &  &  & { ‘meat’}\\
{ {s}\'{ã}{}\'{ã}} & { sãã} & { saasi} & { {}-sãã} & { {}-sãã} &  &  & { {}-sãã, {}-sãzi} & { {}-sã} & { ‘three’}\\
{ ɲēréē} & { ɲinde} &  & { ɲɛɲɛ} & { ɲeɲa} &  & { ɲeɲa} & { ɲuɡo} &  & { ‘breast’}\\
{ fērēʕé} & { fereɡi} &  & { ferɡe} & { filiki} &  &  &  &  & { ‘moon’}\\
{ nāfāʕ{\'{ɔ}}} & { donu} &  & { doni} & { doŋko} &  &  &  &  & { ‘slave’}\\
{ ɲ{\={ã}}} & { ɲã} &  &  &  &  &  & { ɲã} &  & { ‘give’}\\
{ yāá} & { ya} &  &  &  &  &  & { yɛrɛ} &  & { ‘woman’}\\
{ sāʕè} & { sari} &  &  &  & { siru} &  &  & { sáák{\`{ɔ}}} & { ‘earth’}\\
{ bēʕé} & { bẽ} &  &  &  & { bənə} &  &  & { bẽ} & { ‘wilderness’}\\
{ káʕá ɲīn} & { kaane} & { kannɔ} &  &  &  &  &  &  & { ‘tooth’}\\
{ {ɡ͡b}{\={ã}} \'{ã}} & { b{\~{ɔ}}, baa} & { baawɔ} &  &  & { bana} &  &  &  & { ‘sheep’}\\
\lspbottomrule
\end{tabular}
}
\end{table}

Manessy gives three hypotheses for how non-Gur roots are found in \ili{Tiefo}: \ili{Tiefo} should be placed within a separate branch of Gur, certain words are borrowed from an unknown Gur language, or the source of the borrowing is non-Gur, possibly Mande. If \isi{language contact} from \ili{Jula} were the dividing factor, one would expect there to be clear borrowings from \ili{Jula} into \ili{Tiefo}. If the \ili{Jula} language is an influence, it would be apparent in the lexicon.

Among plant names, we find evidence for a sustained symbiosis between \ili{Jula} and even \ili{Tiefo}-N. For example, Heath recently recorded flora-fauna terms in Gnanfogo, both in \ili{Tiefo}-N and in the local \ili{Jula}. Quite a few of these terms are phrasal, and the \ili{Tiefo}-N and local \ili{Jula} often share the phrasing. Some plant names are in \tabref{tab:hangtan:9}.

\begin{table}[t]
\caption{Tiefo plant names.}
\label{tab:hangtan:9}
\begin{tabularx}{\textwidth}{llQQ}
\lsptoprule
{\bfseries Tiefo} & {\bfseries Jula} & {\bfseries Identification} & {\bfseries Literal}\\
\midrule 
{ sòy-pûŋ} & { l{\`{ɛ}}{}-bííⁿ} & { Acanthospermum hispidum} & { ‘pig-herb’}\\
{ pô:ŋ-sà:ⁿ-wi} & { bí:ŋ-ŋwání-tígi} & { Amaranthus spinosus} & { ’herb-thorn-owner’}\\
{ bàwáⁿ-sāní} & { sàmà-ŋwánì} & { Asparagus africanus} & { ‘elephant-thorn’}\\
{ cò:-kú:ⁿ} & { sùlà-fíⁿsáⁿ} & { Cola cordifolia} & { ‘monkey-cashew.apple’}\\
{ bàwáⁿ-dùrté} & { sàmá-tìsékàà-bé} & { Combretum nigricans} & { ‘elephant can’t knock it down’}\\
{ nàf{\'{ɔ}}ɣ{\'{ɔ}}ⁿ{}-bàkó-èllè-wí} & { jààtìgì-fáɣá} & { Ficus thonningii} & { ‘host-kill’}\\
{ blákè-póróŋ} & { sándéⁿ-w{\`{ɔ}}r{\`{ɔ}}s{\'{ɔ}}} & { Heeria insignis} & { ‘rain-sickle’}\\
{ n{\`{ɔ}}ɣ{\`{ɔ}}sì-dúy} & { n{\`{ɔ}}ɣ{\`{ɔ}}sì-kúú} & { Heliotropium indicum} & { ‘chameleon-tail’}\\
{ kàⁿkóóⁿ-tòè} & { sòfàlì-túló} & { Leptadenia hastate} & { ‘donkey-ear’}\\
{ b{\v{ɛ}} yⁿ-jùsúⁿ} & { kòŋó-jèsé} & { Securidaca longepedunculata} & { ‘outback-wire’}\\
{ sóⁿ-bàⁿflà-glá-yò}  & { sò-tìgí-bàⁿflà-bɔ}  & { Senegalia macrostachya} & { ‘horseman-hat-take.off’}\\
{ sèsèré-dúy} & { bàsàⁿ-kúù} & { Stachytarpheta indica} &	{ ‘agama-tail’}\\
{ {blák}{\'{ɛ}}{{}-fl}\={ɔ}} & { sándéⁿ-sìrà-yírí} & { Sterculia setigera} & { ‘hare’s baobab fruit’}\\
{ wámbíí-ʃìnàà} & { fárátá-d{\'{ɛ}}b{\'{ɛ}}} & { Uapaca togolensis} & { ‘orphan-mat’}\\
{ sìsàɣà-dúrúŋ-tè-pô:ŋ} & { kámmélé-kóróbóo} & { clumpy grass sp.} & { ‘young.man-test-grass’}\\
\lspbottomrule
\end{tabularx}


\end{table}

These correspondences, though limited to natural species terms, are indicative of a broader pattern of calquing, the effect of which is develop a local \ili{Tiefo}-ized \ili{Jula}. Outside the core \ili{Tiefo} area, this must have the same general sociolinguistic function of marking speakers as \ili{Tiefo}, as we observe with familiar ethnically-tinged \ili{English} varieties (Yinglish, Spanglish, and the like).

While we do see some evidence of borrowing from \ili{Jula} in both dialects of \ili{Tiefo} in the \tabref{tab:hangtan:10}, according to comparisons between my data and Winkelmann’s shown in \tabref{tab:hangtan:11}, most are like the second table, with 87 out of 185 core lexical items do not bear any resemblance between the two dialects, nor to \ili{Jula} (based on my knowledge of \ili{Jula}).

\begin{table}
\caption{Potential borrowings from Jula into Tiefo.}
\label{tab:hangtan:10}
\begin{tabularx}{\textwidth}{lXXX}
\lsptoprule
{\bfseries  {Tiefo} Dramandougou} & {\bfseries  {Tiefo} Gnanfongo} & {\bfseries Jula} & {\bfseries Gloss}\\
\midrule
{ ɡūɡlīká} & { kērē kīté} & { kɔtɛ} & { ‘snail’}\\
{ blanà(-nɔ)} & { {mí}\={ɔ} {n}\={ɔ}} & { mali} & { ‘hippopotamus’}\\
{ {nākl}\={ɔ}} & { {mī}\={ɔ} {n}\={ɔ}} & { malo} & { ‘rice’}\\
{ {po-jen}\={ɔ} {,} {poka}} & { \=o-dòs\=o} & { donso} & { ‘hunter’}\\
{ ɲã} & { {s}\={ɔ} {ʕ}{}{\'{ɔ}}} & { so} & { ‘horse’}\\
{ jūw{\'{ɛ}}ʔa{\'{ɛ}}} & { ɡānāʕà} & { ɡalaji} & { ‘indigo’}\\
{ d\`{\~{{ɔ}}}} & { náfāʕ\=o} & { jɔn} & { ‘slave’}\\
{ {j}\={ɔ} {,} {j}\={ɔ} {{}-r}\={ɔ}} & { bíkā} & { jo} & { ‘fetish’}\\
{ wòrò} & { d\=oʕ\=obíy\=o} & { wòro} & { ‘kola nuts’}\\
\lspbottomrule
\end{tabularx}


\end{table}

The evidence presented from the lexicon shows that \ili{Jula} has not influenced either variety of \ili{Tiefo} to the point that one would expect if the majority language were to be blamed for the loss of the \isi{minority language}. Considering the long term contact of \ili{Jula} with \ili{Tiefo}, one would expect more of an influence on the lexicon than what is found. Further, the lexical differences between the two dialects, for the most part, cannot be attributed to influence from \ili{Jula} on either end of the dialect spectrum. The cause of the divergences within \ili{Tiefo} and within Gur must have been triggered by another source, but it remains unknown.

% \todo{check accents}
\begin{table}
\caption{Cross-dialectal lexical non-concordance not due to Jula influence.}
\label{tab:hangtan:11}
\begin{tabularx}{\textwidth}{llXX}
\lsptoprule
{\bfseries  {Tiefo} Dramandougou} & {\bfseries  {Tiefo} Gnanfongo} & {\bfseries Jula} & {\bfseries Gloss}\\
\midrule 
{ sú} & { dúrú} & { ɲinan} & { ‘mouse’}\\
{ s{\`{ɛ}}ɡ{\`{ɛ}}} & { dúwī} & { dimi} & { ‘hurt’}\\
{ s{\'{ɔ}}ʔ{\'{ɔ}}, s{\'{ɛ}}ʔ{\'{ɛ}}} & { {dūw}\`{õ}} & { cin} & { ‘sting’}\\
{ sīɡlòʔó {}-ro} & { fáʕláī} & { suruku} & { ‘hyena’}\\
{ ɡ͡bɛ bà} & { fīyāá} & { lana} & { ‘take’}\\
{ pūʔ\=o, poʔo} & { fíyāʕā} & { kunɡo} & { ‘wilderness’}\\
{ d{\`{ɛ}}, bɛ-tɔʕɔ} & { fīyáʕā} & { foro} & { ‘field’}\\
{ di{\`{ɛ}}} & { fíy{\`{ɔ}}} & { buɡu} & { ‘multiply’}\\
{ baʕa} & { fíʕī} & { tɔmɔ} & { ‘pick up’}\\
{ jūw{\'{ɛ}}ʔa{\'{ɛ}}} & {ɡānāʕà} & { ɡalaji} & { ‘indigo’}\\
{ sàk͡pè} &  kā k{\'{\~{ɔ}}} & { fali} & { ‘donkey’}\\
\lspbottomrule
\end{tabularx}

\end{table}
\newpage 
\section{Why different fates?}\label{sec:hangtan:5}

The preceding discussion demonstrates that \ili{Tiefo}-N and \ili{Tiefo}-D are two distinct, though closely related languages. Why have they suffered such different fates?

Isolation? Perhaps Dramandougou (\ili{Tiefo}-D) is more isolated than Gnanfongo and Noumoudara (\ili{Tiefo}-N). Well, it is true that Noumoudara is directly on the \ili{Bobo} Diolasso to Banfora highway, and this may have been the coup de grace factor for \ili{Tiefo}-N in that village. But Gnanfongo and Dramadougou are both located in the same lowlands area southeast of a long escarpment that cuts them off from the highway. Both are reached from the highway with some difficulty, by 4x4 or a motorcycle, either by taking a southern route that avoids the cliffs or by winding one’s way down a circuitous descent in a relatively benign part of the escarpment between Noumoudara and Gnanfongo. Government institutions (schools, clinics) are present in Gnanfongo and Dramandougou to about the same extent. They are equally ``isolated''.

Perhaps a vigorously expanding regional language had a more direct line of sight on Gnafongo than on Dramandougou due to some geographical quirk? The two candidates for “killer” languages \citep{NettleRomaine2000,Price1984} would be \ili{French} and \ili{Jula}. Indeed it was once feared that \ili{French} and \ili{English} would give the same scorched earth treatment to African languages as \ili{English} has given to the indigenous languages of Australia and North America. This has now been broadly debunked by \citet{Batibo2005} and \citet{Mufwene2009}. In West Africa, even in sophisticated and heavily Gallicized coastal megalopolises like Dakar and Abidjan, \ili{French} has developed symbiotic relationships with other languages rather than eliminating them, and new synthetic formations such as \ili{Nouchi} and Urban \ili{Wolof} are emerging. In villages far from the coast like Gnanfongo and Dramandougou, \ili{French} is a minor factor in the sociolinguistic equation. \citet[141]{Naden1989} makes the point that southwestern Burkina has historically been a “backwater” relatively unaffected by the outside world, from the late medieval Saharan trade routes to the present.

\ili{Jula} is another matter. Southwestern Burkina is a linguistic mosaic of ancient Gur languages (\ili{Tiefo}, \ili{Lobi}, \ili{Viemo}, Dogose, Turka, and others) with interspersed Mande languages like \ili{Bobo} and Zuungo that date to the Mande expansion of the late Middle Ages. The \ili{Bambara}-\ili{Jula}-Mandinke dialect group, which is also Mande genetically, has become the linguistic juggernaut throughout southern \ili{Mali} (Bamako, Segou), southwestern Burkina, and northern Cote d’Ivoire. Its spread in Burkina was spearheaded by merchants who made it into the \isi{lingua franca} in markets and then in urban concentrations. The name of the biggest city southwestern Burkina, \ili{Bobo} Dioulasso (i.e. \ili{Bobo}-\ili{Jula}-So), attests to the coexistence of \ili{Jula} with other indigenous languages. If there is a killer language in the area, it is clearly \ili{Jula}, not \ili{French}.

However, there is no obvious geographical reason why \ili{Jula} should have targeted \ili{Tiefo}-N for extinction any more than \ili{Tiefo}-D. \ili{Jula} is the dominant interethnic vernacular in the entire region, extending deeply into neighbouring Cote d’Ivoire. If Dramandougou were more isolated than Gnanfongo, \ili{Jula} might have had a more powerful foothold in the latter. But Dramandougou is no more isolated than Gnanfongo. \ili{Jula} is spoken at least as \isi{second language} by everyone in Dramandougou as well as Gnanfongo.

What about strategic self-interest as an explanation? An SIL-sponsored survey of the local situation does state that “Most \ili{Tiefo} have abandoned their language in favour of \ili{Jula} … presumably as a result of a perceived social advantage to be gained by using \ili{Jula}” \citep[5]{BertheletteBerthelette2001}. But self-interest should be just as pertinent to \ili{Tiefo}-D as to \ili{Tiefo}-N. As \citet{Showalter2008} states in his survey of the languages of Burkina Faso, only two communities in the entire country replaced their languages with \ili{Jula}, one being \ili{Tiefo}-N and \citeauthor{LüpkeStorch2013} counter such simplistic reasoning: “there is no evidence of which we are aware where the shift to another language (as opposed to maintaining it as a language in a multilingual repertoire) has yielded real socio-economic advantages” \citep[286]{LüpkeStorch2013}.

\newpage
What about differential ``prestige'' as an explanation? Aside from the elusiveness of this concept,\footnote{In the early days of American sociolinguistics, the core idea was that lower middle-class individuals sought to emulate the speech of the highest local socioeconomic class. But the data eventually forced recognition of, first, a kind of prestige in the lower echelons, and then another kind of prestige in the middle.} the fact is that \ili{Tiefo} ethnic pride is if anything stronger in the \ili{Tiefo}-N than \ili{Tiefo}-D area, and perhaps stronger there than in the other small-population ethnicities in the area between the proud, larger-population \ili{Bobo} and \ili{Lobi}. The background to this is that the \ili{Tiefo} tribe was a feared military power until the turn of the 20th Century. To this day there is a \ili{Tiefo} “chef de guerre” in Noumoudara, distinct from the regular political chief. He commands no battalions, but he does supervise a small military museum dedicated to the memory of an early chief named Amoro Ouattara. In this museum, visitors get guided tours recounting the great battles of the past and demonstrating (gently) the uses of the traditional weapons, shields, and torture equipment that are on display. It is not large, but it is more than the other small-population ethnicities in the area have.

In Africa and elsewhere, language coexistence (\isi{multilingualism}) is the norm, not the exception. There is no zero-sum fight to the death among languages. Again \citep[76]{Mufwene2009}: “Such a practice of language alternation is traditional to Africa and has sustained \isi{multilingualism}, so much so that it takes a natural disaster to force whole villages to move and find themselves in situations where they have to shift to the host population’s language.”

The cataclysmic event that accelerated the decline of \ili{Tiefo} was the military victory of the \ili{Jula} leader Samori Touré over the \ili{Tiefo}, followed by the slaughter of many \ili{Tiefo} people in 1897. This is cited as the key event in the demise of the language by \citet{Hébert1958}, \citet[31]{LeMoal1980}, and \citet[2]{Winkelmann1998}. It is likely that the \ili{Tiefo}-N villages who commanded the \ili{Tiefo} forces were the principal victims.

Dramandougou, on the periphery and not centrally involved in military activity, appears to have already reached an accommodation with the \ili{Jula}, resulting in a less confrontational relation, at the time of those hostilities. For that reason it was spared the brunt of the reprisals.

\section{Conclusion}\label{sec:hangtan:6}

Despite the fact that there are only five speakers in the village of Gnanfongo, all in their 70’s and 80’s, the dialect of \ili{Tiefo} differs from the neighbouring village, particularly in the lexicon. The differences between the two dialects of \ili{Tiefo} cannot be due to \ili{Jula} alone. In fact then, \isi{language contact}, in addition to not ``killing'' a language, may not have as much influence as we think.

Languages, differing from the metaphors we like to invoke of species, rarely simply die out without a trace, rather, they converge into and diverge from one another. Speakers do not suddenly one day wake up and decide it will be advantageous to being speaking another language. The history of many countries in Africa and the world is volatile, with environmental and political factors influencing language to a greater degree than we may account for. The example of the \ili{Tiefo} serves not only to illustrate that we are missing pieces in the history of the people, but also that we are ill equipped to gather those pieces given the framework we have been using.

Although the cause of the loss of the \ili{Tiefo} language can with a fair amount of certainly be attributed to Samori Toure and his army of invaders, beyond that, the discrepancies between the existing \ili{Tiefo} dialects which cannot be attributed to \ili{Jula} remains a mystery. In summary, \ili{Tiefo} shares some features of geographically neighbouring Gur languages but does not fit into any known branch of Gur. Further, the variety of \ili{Tiefo} that remains in the lives of the five elderly speakers in Gnanfongo differs significantly from the more robust version of the language spoken in neighbouring Dramandougou.

\section*{Acknowledgements}

This research is conducted as part of the project “Investigating the interaction of tone and syntax in the Bangime and the \ili{Dogon} languages of \ili{Mali} and Burkina Faso”, funded by BCS-1263150 (2013--16), PI Jeffrey Heath. I am grateful to Jeffrey Heath, Friederike Lüpke, Marieke Martin, and Sophie Salffner for their contributions and support throughout the writing of this paper. I would also like to thank the two anonymous referees for their helpful comments.
 
{\sloppy
\printbibliography[heading=subbibliography,notkeyword=this]
}

\end{document}