\documentclass[output=paper,modfonts]{langscibook} 
\title{Inter-party insults in political discourse in Ghana: A critical discourse analysis} 
\author{Emmanuel Amo Ofori\affiliation{University of Florida/University of Cape Coast} }
\abstract{In recent times, politics in Ghana has become the politics of personal attack, vilification, and insults. Various attempts have been made to stop this brand of politics, including one spearheaded by the Media Foundation for West Africa, which releases a weekly report to the general public aimed at shaming politicians who are involved in the politics of insults. If a country could go to the extent of shaming politicians involved in politics of insults, then it shows how the issue of intemperate language has become entrenched in Ghanaian political discourse. Thus, there is a need to conduct a thorough analysis of the realization of insults in Ghanaian political discussion. Utilizing a Critical Discourse Analysis approach, this paper analyzes the underlying ideologies in the representation of insults in pro-New Patriotic Party (NPP) and National Democratic Congress (NDC) newspapers. It further compares and contrasts the use of insults in the newspapers.}
\ChapterDOI{10.5281/zenodo.1251710}
\begin{document}
\maketitle
 
 

 
% \textbf{Keywords:} Insults, Critical Discourse Analysis, NPP, NDC, and Newspapers. 

\section{Introduction}\label{sec:ofori:1}

\isi{Ghana} is a democratic country, and this has earned the West African country enviable recognition in the world. \isi{Ghana} experienced her stable democratic dispensation in 1992 after a series of military take-overs/coup d’états from the period of independence until 1992 (4\textsuperscript{th} republic). The democratic practices in \isi{Ghana} are still in the infant stages and therefore it could be considered as an emerging democratic state. Since 1993, political discussions in \isi{Ghana} have centered on various topics, such as the economy, social policies, employment, youth development, education, national security, and health. However, in recent years, Ghanaian political discourse has become a discourse of personal attack, vilification, and insults. Many have argued that the surge of insults in Ghanaian political discussions is due to the liberalization of the media in \isi{Ghana} (\citealt{Owusu2012}; \citealt{Marfo2014}). This stems from the fact that before 1992, \isi{Ghana} did not have many radio and television stations, newspapers and online websites. Currently, there are numerous radio stations and newspapers in \isi{Ghana}, and most of the insults emanate from politicians through the various media outlets. These outlets have their own interest in promoting certain ideologies and political positions. This is seen in how the media present their audience with “a steady supply of problems and crisis, and it may be their interest to exaggerate a problem, fostering the impression that there is a crisis and not just business as usual” \citep[83]{Cameron2012}. It may seem that they are alerting the public to the surge of intemperate language; however, it is a subtle way of promoting an ideology or political position. Therefore the representation of insults from opponents is publicized or foregrounded not to alert the public about the problem of insults, but to put a political spin on it. The \textit{us} versus \textit{them} dichotomy was seen in media reportage of insults in pro-NPP and NDC newspapers. Thus, \citegen{vanDijk1998} concept of ideological square, which is expressed in terms of emphasizing the positive actions of what a media institution considers the ingroup and deemphasizing its negative actions, while, on the other hand, deemphasizing the positive actions of the outgroup, and emphasizing its negative actions, is applied in the analysis of the use of insults in Ghanaian political discourse.

In \isi{Ghana}, politicians own some of the radio stations and newspapers as the means to disseminate the ideology and philosophy of their respective political parties. This is an attempt on the part of the political parties to control the media. Different groups compete in order to control the media as an instrument of social power, or an Ideological State Apparatus (ISA) in the sense of \citet{Althusser1971}, to legitimate and naturalize their ideologies, beliefs, and values \citep{vanDijk1995}. Anyone who controls the media, to some extent, controls the minds of its listeners. This is because the media is seen as a \isi{major} source of information. Therefore, this paper analyzes the underlying ideologies in the representation of insults in pro-New Patriotic Party (NPP) and National Democratic Congress (NDC) newspapers. It further compares and contrasts the use of insults in newspapers.

\subsection{NPP and NDC}\label{sec:ofori:1.1}

\citet{Ninsin2006} observed that when the ban on political parties was lifted in May 1992, by November the same year 13 political parties had been formed and registered. However, the two dominant parties that have survived since 1992 are the National Democratic Congress (NDC) and the New Patriotic Party (NPP). 

The NPP, the current party in opposition, emerged from an old political tradition dating back to the United Gold Coast Convention (UGCC) and United Party (UP) of the Danquah-Busia tradition. They fought for independence with Dr. Kwame Nkrumah’s Conventions Peoples Party (CPP). The UGCC and the UP metamorphosed into the NPP in 1992 when the country returned to civilian rule after 11 years of military rule, in order to contest the 1992 December elections. Their political ideology is founded on capitalism, and they believe in privitalization, rule of law, and democracy. In short, they see the private sector as an engine of growth. The NPP lost both the presidential and parliamentary elections in 1992 and 1996. They won the 2000 and 2004 elections, and lost to the NDC in 2008 and 2012 in one of the most closely contested presidential elections in \isi{Ghana}’s history. The NPP lost by a margin of 0.46\% in 2008.

\newpage 
The NDC, the current government in power, is one of the newest parties in Ghanaian politics. It was formed in 1992 from the Provincial National Defence Council (PNDC) military regime with Flt. Lt. Jerry John Rawlings as its leader. The PNDC overthrew a constitutionally elected government, the People's National Party (PNP), which ruled \isi{Ghana} from 1979 to 31\textsuperscript{st} December 1981, and ruled \isi{Ghana} from 1981 to 1992. During the return to civilian rule, the PNDC metamorphosed into the NDC. The NDC won both the 1992 and 1996 elections. They then lost to the NPP in 2000 and 2004, and won the 2008 and 2012 elections. The political ideology of the NDC is founded on social democracy. 

\subsection{Data collection}\label{sec:ofori:1.2}

The data for this study were obtained from reports in pro-NPP and NDC newspapers. The newspapers are \textit{Daily Guide}, \textit{The Daily Searchlight}, \textit{The New Statesman}, and \textit{The Chronicle} (pro-NPP newspapers);\textit{The Informer, The Democrat}, \textit{The Palaver}, \textit{The Al-Hajji}, \textit{The National Democrat}, \textit{The Catalyst}, \textit{The New Voice}, \textit{Daily Post}, \textit{Daily Heritage}, \textit{Radio Gold online} (pro-NDC newspapers). In all, a total of 78 news articles were gathered from 2012 to February 2014. The articles were sampled and analyzed, using \citeauthor{Fairclough1989}’s discourse-as-text (1989, 1992a, 1995a,b, 2000, 2003) and \citegen{vanDijk1998} concept of ideological square. 

\subsection{Definition of insult}\label{sec:ofori:1.3}

There are various definitions of insult. According to Aristotle, “insult is belittlement. For an insult consists of doing or saying such things as involve shame for the victim, not for some advantage to oneself other than these have been done but for the fun of it” (Aristotle Rhetoric cited in \citealt{Yiannis1998}). Aristotle’s definition of insult focuses on shame, for the fun of it, and it is a form of belittling the target. 

In this paper, the working definition of insult adopted is a modification of \citegen[3]{Yiannis1998} social psychology definition of insults, which considers insult as “a behavior or discourse, oral or written, which is perceived, experienced, constructed and at all times intended as slighting, humiliating, or offensive. Insult can also be verbal, consisting of mocking invective, cutting remarks, negative stereotypes rudeness or straight swearing”.

I therefore define insult as “a behavior or discourse, oral or written, direct or indirect, gestural or non-gestural, which is perceived, experienced and most of the time intended as slighting, humiliating, or offensive, which has the potential of psychologically affecting not only the addressee or target but also his/her associates.” This definition should not be taken as universal because there is no universal measure of insults. The yardstick to measure insults differs from society to society and also from culture to culture.

\section{Critical Discourse Analysis (CDA)}\label{sec:ofori:2}

The revolving idea of CDA is power, and it analyzes opaque as well as transparent structural relationships of dominance, discrimination, power and control as manifested in language \citep[2]{Wodak2001}. \citet[96]{vanDijk2001} also postulates that CDA focuses on social problems, especially on the role of discourse in the production and reproduction of power abuse or domination. This means that it focuses not only on linguistic elements per se, but also complex social phenomena that have semiotic dimensions \citep{WodakMeyer2009}.  In effect, the overall aim of CDA is linking linguistic analysis to social analysis \citep[206]{WoodsKroger2000}. CDA aims at making visible and transparent the instrument of power, which is of increasing importance in the contemporary world. It is very critical of the relationship between language, discourse, speech, and social structure. As the dimensions of CDA include “the object of moral and political evaluation, analyzing them should have effects on society by empowering the powerless, giving voices to the voiceless, exposing power abuse, and mobilizing people to remedy social wrongs” \citep[25]{Blommaert2005}. These are the main concerns in analyzing insults in Ghanaian public political discourse: Who has access to the media? Who controls the media? What are the ideological standpoints of the media in \isi{Ghana}? Whose agenda are they propagating? These are some of the questions that CDA tries to uncover in contemporary societies that relate directly to the present study.

The media discourse in \isi{Ghana} has changed drastically in that before 2001 it was very difficult for media personnel to operate. This was due to the various laws governing media practices in \isi{Ghana}. Even the ones that existed were so polarized that they were divided into two distinct genres: state press and private press \citep{Hasty2005}. The state press were praise singers of the government. They published stories that projected the development, inspirational rhetoric and policies of the government. The private press was sometimes the opposite of the state press. They revealed the profligate spending, abuse of power, and social inequality attributed to the government. In analyzing the underlying ideologies in the representation of insults in pro-NPP and NDC newspapers, these developments must be taken into consideration. The two CDA approaches applied in this paper are \citeauthor{Fairclough1989}’s discourse-as-text and \citeauthor{vanDijk1995}’s ideological square.

\subsection{Fairclough’s discourse-as-text}\label{sec:ofori:2.1}

\citeauthor{Fairclough1995} situates his theory of social-discoursal approach in Halliday’s Systemic Functional Linguistics (SFL) and also draws on critical social theories, such as Foucault’s concept of order of discourse, Gramsci'S concept of hegemony, Habermas’ concept of colonization of discourses and many others. To fully understand the interconnectedness between languages, social and political thought, Fairclough proposes a three-dimensional approach to analyzing discourse. These are: discourse-as-text, discourse-as-discursive-practice and discourse-as-social practice. Since the analysis in this paper is based on discourse-as-text, I elaborate on it below (for detailed discussions on the other two dimensions see \citealt{Fairclough1989}, \citeyear{Fairclough1995}).

Discourse-as-text involves the analysis of the way propositions are structured and the way they are combined and sequenced \citep{Fairclough1995}. Here, the analyst examines the text in terms of what is present and what could have been present but is not. The text, and some aspects of it, is the result of choice, that is, the choice to describe a person, an action or a process over another; the choice to use one way of constructing a sentence over an alternative; the choice to include a particular fact or argument over another. According to \citet[57]{Fairclough1995}, choice in text “… covers traditional forms of linguistic analysis-analysis of vocabulary and semantics, the grammar of sentences and smaller units, and the sound system (phonology) and writing system. But it also includes analysis of textual organization above the sentence, including the ways in which sentences are connected (cohesion) and aspects like the organization of turn-taking in interviews or the overall structure of a newspaper article”. The application of textual analysis in CDA does not mean just focusing on the linguistic form and content; rather, it is the function that such elements play in their use in the text. Hence, the traditional forms of linguistic analysis should be analyzed in relation to their direct or indirect involvement in reproducing or resisting the systems of ideology and social power \citep{Richardson2007}.

\subsection{Van Dijk’s concept of ideological square}\label{sec:ofori:2.2}

One prominent feature of van Dijk’s socio-cognitive approach is the concept of ideological square. It is about how different social groups project themselves positively and represent others negatively. The structures of ideologies are represented along the lines of an \textit{us} verses \textit{them} dichotomy, in which members of one social group present themselves in positive terms, and others in negative terms. There is polarization of how media institutions emphasize the positive actions of ingroup members and deemphasize its negative action on one hand, and deemphasize the positive action of the outgroup while emphasizing its negative actions. The ideological square consists of four moves: (1) express/emphasize information that is ‘positive’ about \textit{us}; (2) express/emphasize information that is ‘negative’ about \textit{them}; (3) suppress/deemphasize the information that is ‘positive’ about \textit{them}; and (4) suppress/deemphasize information that is ‘negative’ about \textit{us}. Any property of discourse that expresses, establishes, confirms or emphasizes a self interested group opinion, perspective or position, especially in a broader socio-political context of social struggle, is a candidate for special attention in ideological analysis \citep{vanDijk1998}. 

\subsection{Ideology}\label{sec:ofori:2.3}

Ideology is defined as systems of ideas, beliefs, practices, and representations, which work in the interest of a social class or cultural group. \citeauthor{Gramsci1971} sees ideology as “tied to action, and ideologies are judged in terms of their social effects rather than their truth-values” \citep[76]{Fairclough1995}. Ideology has the potential to become a way of creating and maintaining unequal power relations, which is of central concern to discourse analysts who take a “particular interest in the ways in which language mediates ideology in a variety of social institutions” \citep[10]{Wodak2001}. \citet[3]{vanDijk1998} also defines ideology as political or social systems of ideas, values or prescriptions of a group that have the function of organizing or legitimating the actions of the group. The use of language reflects a person’s philosophical, cultural, religious, social, and political ideology. Hence, ideology affects the way one talks, argues, and reacts. 

\section{Textual analysis}\label{sec:ofori:3}

Textual analysis is Fairclough’s first dimension of his three dimensional framework. The linguistic tools employed under this dimension in the analysis are lexicalization and predication, presupposition and verbal process.

\subsection{Lexicalization and predication}\label{sec:ofori:3.1}

Lexicalization involves the choice or selection and the meanings of words used to refer to social actors. A typical lexical analysis looks at the denotation (the literal or primary meaning of words) and connotation (the various senses that a word invokes in addition to its literal or primary meaning). This type of analysis is important because “words convey the imprint of society and of value judgments in particular” \citep[47]{Richardson2007}. There is a strong relationship between lexicalization and ideology, as in the use of expressions such as ‘terrorist’ versus ‘freedom fighter’ for example. This suggests that language users have several choices of words to refer to the same persons, groups, social relations or issues, and in most or all cases these carry heavy semantic and ideological loads. The words in a text that communicate messages about subjects or themes in newspapers are framed ideologically. Thus “vocabulary encodes ideology, systems of beliefs about the way the world is organized” \citep[69]{Fowler1987}.

The paper also analyzes predication. \citet[27]{WodakMeyer2009} define predication strategies as terms or phrases that “appear in stereotypical, evaluative attribution of positive or negative traits and implicit or explicit predicates”. \citet[54]{ResiglWodak2001} 
 also see predicational strategies as “the very basic process and result of linguistically assigning qualities to persons, animals, objects, events, actions and social phenomena”. Predicational strategies are not used arbitrarily: there are hidden ideologies in the various forms or phases. It also reveals the ‘Us’ versus ‘Them’ dichotomy that shows positive predications for the ingroup and negative predications for the outgroup. Indeed, “predication is used to criticize, undermine and vilify certain social actors, sometimes with potential dangerous consequences” \citep[53]{Richardson2007}.

NPP newspapers reported insults with lexicalization and predication from people they considered as ingroup members targeted at the outgroup members. These included:

\ea\label{ex:ofori:1}
\textup{Specialist in lies and propaganda shouldn’t be managing our economy} - Bawumia Daily searchlight, October 2, 2012
\z

\ea \label{ex:ofori:2}
\textup{J. J. tainted with blood} - Daily Guide, December 5, 2013\z

\ea \label{ex:ofori:3}
\textup{Greedy bastards and babies with sharp teeth} - Rawlings The New Statesman, April 3, 2014
\z

\ea \label{ex:ofori:4}
\textup{Thieving Mahama caught} - Daily Searchlight, October 2, 2012
\z

The insult in \REF{ex:ofori:1} is a quote from Dr. Bawumia, a running mate to the presidential candidate of the main opposition party in \isi{Ghana}. He has a PhD in economics and believes that the NPP has the experts to manage the economy. However, he believes that those managing the economy do not have the requisite expertise to manage it; they rely only on \textit{lies} and \textit{propaganda}, meaning they are not being truthful with the true state of the economy. Looking at the background of Dr. Bawumia, it would be difficult for the public to scrutinize his comment and it may not be possible for readers to understand that this is an attempt to convince the electorate that if the NPP is voted into power they will make the economy better. The underlying idea behind the report of this insult is that the ingroup has the experts to manage the economy better than the outgroup. Therefore, the newspaper reports the insults from the ingroup member seeking to negatively portray the outgroup as people who do not have the technical know-how to manage the economy. 

Example \REF{ex:ofori:2} is a predication that projects the founder of the NDC (an outgroup member) as a killer. J. J. Rawlings has been accused several times of being responsible for the killing of the three Heads of State and three Supreme Court judges in \isi{Ghana}. Therefore, the report of this insult from the NPP newspaper is a way to remind the public that the founder of the NDC is a known killer. Thus, the underlying idea behind the report of this insult is to negatively portray the outgroup member as a killer and not worth listening to.

The predicational insult in \REF{ex:ofori:3}, \textit{greedy bastards and babies with sharp teeth} was a comment made by the former President Rawlings to his own party members vilifying him. Expanding on this insult, it can be analyzed in two ways: first, it is used to describe the activities of the NDC with regards to corruption. Secondly, the second part of the insult, \textit{babies with sharp teeth}, is a metaphor that describes the behavior of the young ones in the NDC. Rawlings made this comment at a time when most of the youth in the NDC were insulting him. So, his metaphor \textit{babies with sharp teeth} refers to these youth who have the penchant toward behavior of vilifying and insulting people. Within the Ghanaian culture, kids or babies are not supposed to engage in adult communication, let alone to insult adults. This metaphor portrays the young people within the NDC as having outgrown their wings and vilifying the adults in the party. Therefore, pro-NPP newspapers reported this insult among others to expose the confusion within the camp of the outgroup, and to show that they are not the only people saying the outgroup members are corrupt; the founder of their party and members of their party concur with them.

In \REF{ex:ofori:4}, the \textit{Daily Searchlight} paper describes the president of \isi{Ghana}, John Mahama, as a \textit{thief}, for stealing the 2012 presidential election, which the NPP contested in court and lost. Calling the president a thief was to bring to the attention of the NPP supporters that the party won the 2012 elections, and that their loss was a typical case of a stolen verdict. Therefore, the insult was aimed at satisfying the aggrieved ingroup members and portraying the outgroup as thieves.

Similarly, pro-NDC newspapers used lexicalization and predicational strategy to report insults from the ingroup members targeted at the outgroup members.

\ea \label{ex:ofori:5}
\textup{No ‘patapaa’ President Mahama warns losers in December election} - \isi{Ghana} Palaver, September 14--16, 2012\z

\ea \label{ex:ofori:6}
\textup{Arrogant \ili{Kan} Dapaah running a one man show} - \isi{Ghana} Palaver, July 20, 2012\z

\ea \label{ex:ofori:7}
\textup{Akuffo-Addo is wicked and not worth dying for} - \isi{Ghana} Palaver, June 20, 2012\z

\ea \label{ex:ofori:8}
\textup{“T$\varepsilon $ Ni” can’t govern us.} - The Al-Hajj, August 16, 2012\z

The lexical item \textit{patapaa} in example \REF{ex:ofori:5} was uttered by the current president of \isi{Ghana}, John Mahama, to advise all losers of the 2012 election, and this was reported by the \textit{\isi{Ghana} Palaver} newspaper. \textit{Patapaa} is an \ili{Akan} word, which means “a person who is violent or a violent behavior”. It was used to describe the losers in the December elections. Though the word was part of a comment made to advise all losers, knowing that \isi{Ghana}’s election is a contest between the NDC and NPP, this insult was directed at the doorsteps of the main opposition party and most importantly to the 2012 presidential candidate of the NPP, Nana Akuffo-Addo. Because he was alleged to have said prior to the 2012 elections that the elections would be “all die be die”, he was criticized for warmongering. Thus, the report of the president’s insult is to negatively present the outgroup member, the NPP presidential candidate, as a violent person who would not accept the 2012 election results and would plunge the country into chaos if he lost; by extension, it is implied he is not even qualified to be a president.

In \REF{ex:ofori:6}, the \textit{\isi{Ghana} Palaver} newspaper reports insults from NDC members of the Public Accounts Committee (PAC) of Parliament that were tasked to investigate financial malfeasance in the public service. Parliamentary regulations provide the opportunity for an MP from the opposition party to chair the (PAC) so that there is a fair investigation of government officials. The NDC members accused the chairman of the committee, Albert \ili{Kan} Dapaah, an outgroup member, of being \textit{arrogant} and \textit{running a one-man show} because he threatened to cause the arrest of an ingroup member, Alfred Agbesi Woyome. The newspaper reports this predicational insult from the NDC members of the committee to negatively portray the outgroup member as someone who does not consult them before making decisions, and has therefore taken an arrogant posture. The publication of the insult is clearly meant to discredit the outgroup member.

The \textit{\isi{Ghana} Palaver’s} predicational insult in \REF{ex:ofori:7} is targeted at Nana Akuffo-Addo, the NPP presidential candidate for 2008 and 2012 elections when he continued with his campaign at a time when the late former Vice President, Aliu Mahama, was admitted to the hospital. Akuffo-Addo was accused of not showing enough compassion and abandoning the former Vice President to die. The ingroup candidate is reported to have asked for prayers for the late vice president from Ghanaians. Therefore, the paper described Akuffo-Addo negatively as \textit{wicked} and not qualified to be president of \isi{Ghana}. The ingroup candidate is preferred over the outgroup candidate, revealing a group polarization between NPP and NDC newspapers.

The \textit{Al-Hajj} newspaper reported the insult in \REF{ex:ofori:8} from a taxi driver who claimed to have heard some Akans using the lexical item “t$\varepsilon $ ni” to insult the president. This word needs elaboration. The correct spelling of the word is “tani,” an \ili{Akan} term used to insult people from the Northern part of \isi{Ghana}. Recall that the NDC presidential candidate, John Mahama is a Northerner, and the NPP is perceived as an \ili{Akan} dominated party. So, for this term to surface on the front page of an NDC paper is a way of turning the people of the North against the NPP. The paper portrays to readers that the outgroup is presenting the ingroup presidential candidate as someone who is ethnically unfit to lead the country. This portrays the outgroup negatively for using tribal and ethnic sentiments against the ingroup candidate, while the ingroup is implicitly presented positively for not whipping up ethnic sentiments.

\subsection{Presupposition}\label{sec:ofori:3.2}

Presupposition is a “taken-for-granted, implicit claim embedded within the explicit meaning of a text or utterance” \citep[63]{Richardson2007}. \citet[214]{Wodak2007} provides a broader picture of presupposition: “presupposed content is, under ordinary circumstances, unless there is a cautious interpretive attitude on the part of the hearer, accepted without (much) critical attention (whereas the asserted content and evident implicatures are normally subject to some level of evaluation)”. The claims are not critically evaluated and are generally considered to be true regardless of whether the sentence is true. It is a useful strategy in political discourse because it makes it difficult for the audience to identify or reject views communicated in this way. That is, it persuades people to take for granted something which is actually open to debate \citep{Bayram2010}.

Pro-NPP and NDC newspapers utilized presupposition strategy using the authorial \isi{voice} and non-politicians to describe people they considered to be outgroup members. Below are examples from pro-NPP papers:

\ea \label{ex:ofori:9}
\textup{In his usual propaganda style} - The Chronicle, July 19, 2012\z

\ea \label{ex:ofori:10}
\textup{Send people who can make intellectual debate daily} - Daily Guide, December 12, 2013\z

\ea \label{ex:ofori:11}
\textup{Former President Jerry John Rawlings has suddenly found his voice} - Daily Guide, December 10, 2010\z

The \textit{Chronicle} paper used the presupposition in \REF{ex:ofori:9} to report an insult hurled on one of their reporters by Deputy Minister for Information, James Agyenim Boateng when he asked a question on whether the Constitutional Review Committee (CRC) was considering putting a \isi{clause} in the constitution about health status of presidents and presidential candidates. Recall that it was during this time that the late president Mills was alleged to have been taken ill. The paper therefore presented members of the outgroup as people fond of using propaganda. Employing the lexical item \textit{usual} by \textit{The Chronicle} paper indicates that the outgroup member is noted for his habitual propaganda style. 

In \REF{ex:ofori:10}, the presupposition was reported from a source that presented the outgroup negatively. Fiifi Banson, a broadcast journalist with \textit{Peace FM}, a radio station in Accra, is reported by the \textit{Daily Guide} paper to have uttered that presupposition. He is presented as ``an award winning Ghanaian broadcaster'' to portray the genuineness of the presupposition. The meaning behind the presupposition is that the crop of panelists sent by the NDC to \textit{Peace FM} are not astute enough to present the agenda of the party. 

The former President Rawlings is also presented in the \textit{Daily Guide} paper in \REF{ex:ofori:11} as suddenly finding his \isi{voice} when he insulted former president Kufuor as an “autocratic thief”. The use of \textit{sudden} presupposes that Rawlings has been quiet for sometime and was now speaking. His sudden \isi{voice}, however, was directed at an ingroup member not the outgroup, which the paper believes is noted for corruption. Therefore, the use of this presupposition is to negatively present the former president, an outgroup member, as directing his attention towards the wrong person. In sum, pro-NPP newspapers presented the outgroup members negatively using presupposition from both the authorial \isi{voice} (depicting the ideology of the papers) and other sources.

Pro-NDC newspapers also employed presupposition strategy to present the outgroup negatively. They used presuppositions from the authorial \isi{voice} and members in the ingroup. Some of the presuppositions included:

\ea \label{ex:ofori:12}
\textup{Loose-talking ‘Genocide’ MP on the loose again} - The Catalyst, September 7, 2012\z

\ea \label{ex:ofori:13}
\textup{NPP turned \isi{Ghana} into a cocaine country} - Felix Kwakye-Ofosu country radiogold.com, June 7, 2013\z

\ea \label{ex:ofori:14}
\textup{True NPP old evil Dwarfs at work} - \isi{Ghana} Palaver, July 20, 2012\z

\textit{The Catalyst} reported this presupposition in \REF{ex:ofori:12}, using the authorial \isi{voice}, to represent Kennedy Agyapong, the outgroup MP, negatively for insulting the President, John Mahama, the entire membership of NDC and the police service. The use of the lexical item \textit{again} in the report presupposes that the MP’s loose talking is not the first time. That is, the MP is known for his habitual loose talk. He is even described as a \textit{genocide MP}, a term which negatively portrays the MP. Thus, the presupposition and the description are used to negatively present the outgroup member. 

Radio Gold reported the presupposition in \REF{ex:ofori:13} from a deputy minster for information, Felix Ofosu Kwakye who described as unfortunate attempts by the NPP to link the arrest of the Managing Director of SOHIN Security in the United States for drug trafficking to the Mahama administration. The word \textit{turned} presupposes that before the NPP came to power there was nothing like cocaine in \isi{Ghana}; therefore, the NPP are responsible for turning \isi{Ghana} into a cocaine country. The ingroup’s administration is implicitly presented positively for not being responsible for the cocaine business in the country. 

The \textit{\isi{Ghana} Palaver} paper employed the presupposition in \REF{ex:ofori:14} to insult the outgroup. The former President Rawlings was the first person to use the description \textit{old evil} \textit{dwarfs} to refer to some members of the NDC. For an NDC newspaper to use this same description with the adjective \textit{true} to refer to the outgroup is interesting. The adjective \textit{true} is used to qualify the \isi{noun} phrase \textit{NPP old evil dwarfs}. \citet{Richardson2007} indicates that the use of nouns and adjectives to modify \isi{noun} phrases trigger presuppositions he calls nominal presuppositions. So, \textit{true NPP old evil dwarfs} is a nominal presupposition meant to present the outgroup negatively. \textit{True} presupposes that the NPP are indeed the real or actual \textit{old evil dwarfs},\textit{} not the NDC. Therefore, the nominal presupposition was used to present the outgroup negatively. In sum, pro-NDC newspapers employed presuppositions, using authorial \isi{voice} and the ingroup to present the outgroup negatively.

\subsection{Verbal process}\label{sec:ofori:3.3}
\largerpage

Verbal process is “any kind of symbolic exchange of meaning” as well as predicates of communication \citep{Halliday1985}. This means that they represent the action of talking, saying and communicating. Journalists use verbal processes to introduce the speech of people they are reporting on, and this can reveal the feelings and the attitudes of the journalists towards the people they consider important whose words or actions they report. Such reportage can be used to marginalize others and focus readers’ attention towards the direction of the reporter. Thus, “choosing certain verbal process rather than others, the producer of a text is able to foreground certain meanings in discourse while others are suppressed” \citep[34]{Chen2005}.

\citet{Chen2005}, following \citeauthor{Halliday1985}’s analysis of verbal process, proposed three sub-cat\-e\-go\-ries of verbal process. The first is negative verbal process, which demonstrates a certain negative feeling on the part of the writer towards the person whose words the verbal process is used to introduce. Examples of such verbs are ‘insisted’, ‘denied’, ‘claimed’, ‘admitted’, ‘complained’. The second is positive verbal process, which is used to promote in a reader the feeling that the person whose words are being reported is wise, authoritative, benign or in some other sense positive. Examples are ‘pointed out’, ‘announced’, ‘explained’, ‘declared’, ‘indicated’, and ‘urged’. The last is neutral verbal process, where the writer’s choice of \isi{verb} does not indicate an endorsement or disparagement of what the person being reported is saying. Examples include ‘said’, ‘told’, ‘described’, ‘asked’, ‘commented’.

Pro-NPP and NDC newspapers employed different verbal processes to report insults from people they considered ingroup positively and those they considered outgroup negatively. The analysis of the data showed that pro-NPP newspapers employed negative and neutral verbs to report insults directed at the outgroup. However, NDC newspapers adopted positive, negative and neutral verbs to report insults directed at the outgroup.

\tabref{tab:1} sums up all the verbal processes employed by both NPP and NDC Newspapers.

\begin{table}
\begin{tabularx}{\textwidth}{XXXX}
 \lsptoprule
newspapers  &    positive    &  negative   &   neutral\\
\midrule
NPP         &          & lashed out     & describe   \\
 &  & 			Jabbed  &    challenge\\
 &  &             blasted   &   say\\
 &  &             accused           & \\
 \tablevspace
NDC  &       explain  &     warn       &  describe\\
 &       Confirm   &   blast   &     say\\
 &  &             lashed-out\\
 &  &             condemned\\  
 \lspbottomrule
\end{tabularx}
\caption{NPP and NDC verbal process.}
\label{tab:1}
\end{table} 

In sum, positive verbal processes were not very common; they only appeared in the reports of pro-NDC newspapers, showing that positive ingroup representation was more common in pro-NDC newspapers compared to NPP newspapers. However, negative other-representation manifested in both NPP and NDC newspapers.

\section{Discussion}\label{sec:ofori:4}

In this section, I expatiate on the textual analysis to investigate the broader sociopolitical and sociocultural context. Drawing on van Dijk’s (1995) ideological square, I discuss the various ideological structures utilized by both pro-NPP and NDC newspapers on the textual analysis to represent the ingroup positively and the outgroup negatively (group polarization), paying particular attention to the sociopolitical context that necessitated this polarization. The two ideological structures employed by pro-NPP and NDC newspapers are negative lexicalization and predications and detailed description. These revealed group polarization between the two dominant political parties in \isi{Ghana}.

In the first place, the lexical forms that are used to describe the political opponents show \textit{Us} verses \textit{Them} dichotomy. Pro-NPP newspapers for example employed the lexical item “propaganda” from Dr. Bawumia to refer to the managers of the Ghanaian economy. According to the United States Institute for Propaganda Analysis (USIPA) (2001), the word propaganda “is an expression of opinion or action by an individuals or groups deliberately designed to influence opinions or actions of individuals with reference to predetermined ends”. They further state that the main idea of a propagandist is to put something across, either good or bad. The USIPA provides a list of seven propaganda devises. These are: (1) name calling, (2) glittering generalities, (3) transfer, (4) testimonial, (5) plain folks, (6) card stacking, and (7) band wagon (for detailed discussion of these devises, see USPIA 2001). The pro-NPP newspapers, therefore, published this lexical item for ideological purposes, that is, to present the outgroup as: (1) not giving the true state of the economy and (2) bad handlers of the economy. However, they portrayed to the reader that the NPP would be transparent and have the needed expertise to manage the economy better than the outgroup. This reveals group polarization, in that, the ingroup is presented positively as transparent and good managers of the economy while the out-group is presented negatively as liars and bad handlers of the economy.

Pro-NDC newspapers also employed lexical items “patapaa” and “tani”. The current president of \isi{Ghana}, John Mahama prior to the 2012 general elections, used the word “patapaa”, which means “a violent person or violent behavior or a thug”. The meaning can be extended to “someone who uses force to claim what does not belong to him/her”. This word was uttered to advise all losers of 2012 elections. However, this was an insult directed at the presidential candidate of the NPP, describing him as violent, adopting a “patapaa” stance to win the elections. NDC newspapers reported this negative lexicalization from the president to present the presidential candidate of the outgroup as violent while maintaining that the ingroup candidate is peaceful and not violent.

NDC newspapers employed the term “tani”, an \ili{Akan} word that is normally uttered to insult those from the Northern part of \isi{Ghana}. There is no consensus as to the meaning of this word. While \citet{Agyekum2010taboo} argues that it is a derogatory term for people who move in pairs (\textit{ntafoɔ}{}-twins and \textit{tani}{}-one of the twins), reference is made to immigrants from the Northern part of \isi{Ghana}; others have the understanding that it is an insult which makes reference to people from the northern part of \isi{Ghana} as “dirty people”. The common idea out of the two meanings is that it is not a good term to use for Northerners. Interestingly, this term surfaced on the front page of an NDC newspaper “\textit{tani can’t govern us}”. Take note of the fact that the 2012 presidential candidate of the NDC and the current President is a Northerner. Also, it is important to recall that the NPP has long been perceived as an \ili{Akan} or \ili{Ashanti} dominated party, and historically Northerners were considered as servants of the Akans. Putting all these facts in context, the pro-NDC newspaper pitches an old ideological battle between the ingroup and the outgroup. The outgroup is presented negatively as insulting and marginalizing people from the North, consequently raising ethnic tension between people from the North and the outgroup, that is, turning the people from the North against the NPP. According to \citet{vanDijk2001}, ethnic prejudice and ideologies are not innate; rather, they are acquired and learned through communication, that is, through text and talk. 

Secondly, with regard to detailed description, pro-NPP newspapers employed detailed positive descriptions to describe ingroup members as well as non-politicians insulting the outgroup. For example, the \textit{Daily Searchlight} paper described Mr. Mohammed Ameen Adams, who accused the deputy energy minster, Alhaji Inusah Fuseini of being economical with truth concerning the energy crisis in the country as “an energy expert or economist”. Similarly, the \textit{Daily Guide} paper described Fiifi Banson, a broadcast journalist, as “an award winning Ghanaian broadcaster” when he criticized the outgroup for not sending people who can make an intellectual debates. The paper, however, referred to former President Rawlings as “founder of the ruling NDC” when he insulted the party he founded. Negative descriptions were also used to refer to outgroup members for insulting the ingroup. For example, the former President Rawlings was described by the \textit{Daily Guide} paper as “see no evil” for calling former President Kufour an “autocratic thief”.

NDC newspapers, on the other hand, employed honorifics and official titles to describe ingroup members who insulted the outgroup. For example, the \textit{Voice} newspaper described Hamza Abugri, the Bantama constituency organizer of the NDC, as “honorable” for insulting the chairman of the NPP as “ignorant”. This is very interesting because “honorable” is a title given to Ministers of state, MPs, Metropolitan, Municipal and District Chief executives (MMDCE), and assembly members but not leaders of political parties. Members of the outgroup who insulted the outgroup were given more detailed descriptions such as “leading member of NPP” (\textit{Daily Post}), “senior member of NPP” (Radio Gold), and “a stalwart of the opposition” (\textit{Informer}), to report such insults as credible. Negative descriptions were also used to refer to outgroup members’ insults targeted at the in-group. For example, “genocide MP” and “loose talking MP” (\textit{The Catalyst}).

\Citet{vanDijk1995} aptly states that one of the structures used to present the ingroup positively is detailed description. This is supported by \citet{Blommaert2005}, who points out that members of the ingroup employ indexical meaning such as terms of politeness to elevate them to a particular social status. It is however important to note that, in this paper, it has been revealed that positive descriptions were used to describe the outgroup members who insulted their own party to portray to readers that the source of the insult or information is credible and authentic. Though the opposing group members are presented positively, it gives an impression to readers that there is confusion at the camp of the outgroup, which eventually presents them negatively.

In sum, the two important ideological structures used were negative lexicalization and predication as well as detailed description. A detailed analysis of these structures revealed group polarization between the pro-NPP and NDC newspapers.

\section{Conclusion}\label{sec:ofori:5}

Utilizing \citeauthor{Fairclough1995}’s textual analysis and \citeauthor{vanDijk2001}’s concept of ideological square, this paper has revealed the Us/Them representation of insults between the NPP and NDC newspapers. That is to say, the actions of the ingroup were presented positively while those of the outgroup were presented negatively. The ideological differences and political spin in the representation of insults showed a clear group polarization between NPP and NDC newspapers. The various lexicalizations and predications employed by both NPP and NDC newspapers revealed the Us/Them dichotomy between these two dominant political parties in \isi{Ghana}. Using the authorial \isi{voice} and other sources, the presuppositions also showed the ideological differences between the papers. Finally, verbal processes employed by NPP and NDC papers clearly manifested the positive ingroup representation in NDC papers as compared to NPP paper. Negative other representation was a common trait in both papers.
 
 {\sloppy
\printbibliography[heading=subbibliography,notkeyword=this]
}
\end{document}