\documentclass[output=paper,
modfonts
]{langscibook} 
% \bibliography{localbibliography}
\ChapterDOI{10.5281/zenodo.1251744}

 
\title{Structural transfer in third language acquisition: The case of Lingala-French speakers acquiring English}
\author{Philothé Mwamba Kabasele \affiliation{University of Illinois at Urbana-Champaign, University of Calgary, Institut Superieur Pedagogique de la Gombe (RDC), University of KwaZulu-Natal }}
\shorttitlerunninghead{Structural transfer in third language acquisition}





\abstract{This paper tests the claims of Cumulative Enhancement Model, the ‘\textsc{l}\oldstylenums{2} status factor’, and the Typological Primacy Model in investigating how \textsc{l}\oldstylenums{1} Lingala, \textsc{l}\oldstylenums{2} French speakers express in English an event which took place and was completed in the past. The linguistic phenomena understudy informs us that English uses the simple past in a past-completed event while French and Lingala use the ‘passé composé’ and the remote or recent past, respectively. The study circumscribes the tense similarities and differences between the three languages. 

The paper strives to answer the questions on which previously acquired language between the \textsc{l}\oldstylenums{1}, \textsc{l}\oldstylenums{2}, or both \textsc{l}\oldstylenums{1} \& \textsc{l}\oldstylenums{2} overrides in \textsc{l}\oldstylenums{3} syntactic transfer. The paper aims to determine whether the \textsc{l}\oldstylenums{2} is the privileged source of syntactic transfer even when the \textsc{l}\oldstylenums{1} offers syntactic similarities with the \textsc{l}\oldstylenums{3}. Finally, the study purports to determine whether subjects are more accurate when communicating in explicit mode than in implicit mode. That is, the study further aims to investigate whether subjects make less transfer errors in a task that promotes reliance on explicit knowledge than they do in task that promotes reliance on implicit knowledge. 

The findings of the study show that subjects used the simple past tense in the context of a past-completed event. The use of the simple past tense in the context of a past-completed event might be attributed to transfer from the \textsc{l}\oldstylenums{1} or might be considered as a consequence of positive learning. 

The results further show that subjects have transferred more explicit knowledge than implicit. And the results have ruled out the \textsc{l}\oldstylenums{2}-status factor claim that the \textsc{l}\oldstylenums{2} is the privileged source of transfer in \textsc{l}\oldstylenums{3} acquisition.
}


% Keywords: Structural transfer, TPM, CEM, The L2-status factor, \textsc{l}\oldstylenums{3} acquisition

\begin{document}
\maketitle
\section{Rationale}\label{sec:kabasele:1}

This paper tests the claims of the Cumulative Enhancement Model (\textsc{cem}) by  \citet{FlynnEtAl2004}, the ‘\textsc{l}\oldstylenums{2} status factor’ by \citet{BardelFalk2007}, and the Typological Primacy Model (\textsc{tpm}) by \citet{Rothman2010,Rothman2011} in investigating how \textsc{l}\oldstylenums{1} \ili{Lingala} \textsc{l}\oldstylenums{2} \ili{French} speakers express in \ili{English} an event which took place and was completed in the past. 

The work aims to test the claims of those three models of \textsc{l}\oldstylenums{3} acquisition in terms of source of transfer and determine the factor, which takes precedence in determining the source of transfer when there is the potential for competition between multiple factors. The Cumulative Enhancement Model claims that previously acquired linguistic knowledge from both \textsc{l}\oldstylenums{1} and \textsc{l}\oldstylenums{2} positively impact the acquisition of any subsequent language. The ‘\textsc{l}\oldstylenums{2} status factor’ privileges and restricts the source of transfer from only the \textsc{l}\oldstylenums{2} while the Typological Primacy Model constrains transfer to the language that is perceived to be (psycho)-typologically closer to the \textsc{l}\oldstylenums{3}. 

The paper studies the population of twenty-five \ili{Lingala} speakers who also speak \ili{French} as \textsc{l}\oldstylenums{2} and who are learning \ili{English} as \textsc{l}\oldstylenums{3}, three languages of which two are Indo-European and one is a Bantu language. This is the first study, which combines those three languages in the context of third \isi{language acquisition}.

The linguistic phenomena under study inform us that \ili{English} uses the simple past to talk about an event which took place in the past and was completed in the past while \ili{French} and \ili{Lingala} use the ‘passé composé’ and the past (remote or recent past), respectively. For the sake of this study, the simple past (historical past) in \ili{French} is not considered as a potential factor that can trigger transfer because as \citet[26]{Rowlett2007} argues that changes in the spoken language in \ili{French} have taken place in the use of the passé composé’ in which the \isi{perfect} is used instead of the past historic when talking about past completed event. Furthermore, with reference to the economy of cognitive design and linguistic architecture \citep{FlynnEtAl2004,Rothman2010} and in relation to the biological theory of \isi{language acquisition} \citep{Chomsky2007}, it is observed and documented that speakers of a language prefer the most economical linguistic option and this preference is hardwired into human cognition or better in the grammar of the language \citep[271]{Rothman2010}. It is postulated in this paper that the \textsc{l}\oldstylenums{2}-speaking \ili{French} subjects would resort to the ‘passé composé’ to talk about a past completed event rather than the historical past (simple past) because the former is the option that is available to them and the parser would strongly and straightforwardly prefer the option which offers easier access. 

Both \ili{French} and \ili{English} present some syntactic similarities in terms of form while they differ in terms of function. Their similarity is observed between the form of the ‘present \isi{perfect} tense’ and the form of the ‘passé compose’ which are structured as ‘\textsc{aux have/avoir + past participle}’ in both languages. The differences are observed in their function; the ‘passé composé’ is used in \ili{French} to talk about an event which took place and was completed in the past while the ‘present \isi{perfect} tense in \ili{English} expresses an event that started in the past but has some implication in the present. Whereas, \ili{Lingala} and \ili{English} show some syntactic similarities in terms of both form and function because the simple past in both languages are used to talk about a past completed event and in terms of form both languages use inflectional morphemes to morpho-syntactically mark the past tense. The study circumscribes the similarities and differences, in tense, between the three languages. 

The paper strives to answer the questions on which previously acquired language between the \textsc{l}\oldstylenums{1}, \textsc{l}\oldstylenums{2}, or both \textsc{l}\oldstylenums{1} \& \textsc{l}\oldstylenums{2} overrides in \textsc{l}\oldstylenums{3} syntactic transfer. The paper aims to determine whether the \textsc{l}\oldstylenums{2} is the privileged source of syntactic transfer even when the \textsc{l}\oldstylenums{1} offers syntactic similarities with the \textsc{l}\oldstylenums{3}. Finally, the study purports to determine whether subjects are more accurate when communicating in explicit mode than in implicit mode. That is, the study further aims to investigate whether subjects make less transfer errors in a task that promotes reliance on explicit knowledge than they do in task that promotes reliance on implicit knowledge. The predictions of the study permit to test the three models, specifically to test the descriptive and explanatory adequacies of the \textsc{cem}, ‘\textsc{l}\oldstylenums{2} status factor’, and \textsc{tpm}. 

Apart from the rationale and the conclusion, \sectref{sec:kabasele:2} is on the review of literature. \sectref{sec:kabasele:3} discusses the linguistic phenomenon that motivates the study. \sectref{sec:kabasele:4} provides the predictions and research questions of the study. \sectref{sec:kabasele:5} presents the design and methodology of the study. \sectref{sec:kabasele:6} is on the results while \sectref{sec:kabasele:7} presents the discussion on the findings of the paper.

\section{Literature review}\label{sec:kabasele:2}
\subsection{Transfer in L3 acquisition}

Transfer in \textsc{l}\oldstylenums{3} varies depending on the domain and two languages are identified as source of transfer; the \textsc{l}\oldstylenums{1} or the \textsc{l}\oldstylenums{2}. When a learner assumes that his \textsc{l}\oldstylenums{1} is closely related to the \textsc{tl}, there is a high likelihood for the \textsc{l}\oldstylenums{1} to trigger more transfer than when he assumes the opposite. Clearly, it is more probable to have less transfer when a learner assumes great linguistic distance between, say, his \textsc{l}\oldstylenums{1} and the \textsc{tl} in \textsc{l}\oldstylenums{3} acquisition. The speaker’s perception of language similarity, which is psychological and does not necessarily reflect the actual linguistic distance between the languages, may trigger or constrain transfer in the acquisition of \textsc{l}\oldstylenums{3}. \citet{Ringbom2003} has restricted the importance of perceived typological distance in the transfer of lexis. When \textsc{l}\oldstylenums{2} and \textsc{l}\oldstylenums{3} offer a considerable number of common cognates, the speaker perceived both languages as similar and this psychotypological effect favors transfer.

\subsection{Syntactic models in L3 acquisition}

The three syntactic models in \textsc{l}\oldstylenums{3} acquisition \isi{agree} upon the influence of, at least, one previously acquired language. They, however, depart from one another by the way they formulate their predictions.

\subsubsection{Cumulative enhancement model (CEM)}

The \textsc{cem} \citep{FlynnEtAl2004} claims that language learners rely on both their \textsc{l}\oldstylenums{1} and \textsc{l}\oldstylenums{2} cumulated linguistic knowledge when acquiring an additional language. This claim identifies \isi{language acquisition} in a multilingual context as a cumulative process. The multilingual learner’s reliance on the previously acquired linguistic knowledge is restricted to only transfer which has a noticeable rewarding impact in the learning process of the subsequent language. The previously acquired languages can positively contribute to the acquisition of a \textsc{l}\oldstylenums{3}. The insistence of \textsc{cem} on the sole beneficial effects of previous linguistic knowledge in the acquisition of an additional language implies a denial of negative transfer from previously acquired languages. 

% \todo{is this a citation by  Flynn or by Rothman?}
\citet{FlynnEtAl2004} ascertained that, “Language acquisition has a scaffolding effect” \citep[110]{Rothman2010}. This means any previously acquired linguistic knowledge’s role is twofold. It can either enhance the acquisition of any additional language or remain neutral. The impact of both \textsc{l}\oldstylenums{1} and \textsc{l}\oldstylenums{2} in the process of the acquisition of an additional language is relevant. The \textsc{l}\oldstylenums{2} contribution only supersedes that of \textsc{l}\oldstylenums{1} when, say, structure wise, the syntactic features which are in play are not available in the \textsc{l}\oldstylenums{1} linguistic system.

\subsubsection{The L2 status factor}

The ‘\textsc{l}\oldstylenums{2} status factor’ \citep{BardelFalk2007} privileges the \textsc{l}\oldstylenums{2}; it argues that the \textsc{l}\oldstylenums{2} is the only linguistic system, which imposes its features onto the subsequent language. \citet{BardelFalk2007} claim that the acquisition of the \textsc{l}\oldstylenums{3} is qualitatively different from those of the previously acquired languages because the linguistic knowledge of \textsc{l}\oldstylenums{2} plays a substantial role in facilitating the process (see also \citealt{Hufeisen1998,CenozJessner2000,Cenoz2001,Cenoz2003}).      

The claim, that the \textsc{l}\oldstylenums{2} is the strongest source of transfer in \textsc{l}\oldstylenums{3} acquisition stems from the findings of the studies by  \citet{BardelFalk2007} and \citet{FalkBardel2011}. The findings of this study were congruent with the claim that \textsc{l}\oldstylenums{2} is the strongest source of initial transfer in \textsc{l}\oldstylenums{3}. In their recent paper, \citet{FalkBardel2011}, they studied the placement of object pronouns and their findings confirmed the privileged role of \textsc{l}\oldstylenums{2} in acquiring an \textsc{l}\oldstylenums{3}.

\subsubsection{Typological primacy model (TPM)}

\citet[233]{Rothman2011} stipulates, “Initial State transfer for \isi{multilingualism} occurs selectively, depending on the comparative perceived typology of the language pairings involved or psychotypological proximity.” The model argues that typological proximity or psychotypology constrains transfer to the \textsc{l}\oldstylenums{3}. The prevailing role of typological similarity and its role as a crucial variable in the acquisition of an \textsc{l}\oldstylenums{3} are significant. 

 Transfer does not always happen in a facilitative fashion. \textsc{tpm} predicts that in a pair of previously acquired languages only the one, which offers typological proximity with the target language, serves as the source of transfer. \textsc{tpm} constrains transfer from two perspectives: the actual typological proximity or the perceived typological proximity, which is also called psychotypological proximity existing between the three grammars \citep[19]{GarcíaMayoRothman2012}.

\citet[19]{GarcíaMayoRothman2012} argue that, “At the initial state upon a limited amount of exposure to the target \textsc{l}\oldstylenums{3}, the \textsc{tpm} proposes that the internal parser assesses relative typological proximity and selects which system should be transferred.” The \textsc{tpm} is selective and conditionally non-facilitative. The parser selects the closest system to the \textsc{tl}. Any syntactic feature such as \isi{word order}, tense similarity, or any other syntactic similarity depending on the case that is observed at the syntactic level may determine the selectivity of one of the competing previously acquired languages.

The criticism that is formulated against \textsc{tpm} addresses its apparent incapacity to predict the source of transfer when the languages at hand do not present any clear typological proximity. The \textsc{tpm} suggests that transfer can be non-facilitative when psychotypology conditions the transfer by matching and misanalysing the underlying syntax of \textsc{l}\oldstylenums{1} or \textsc{l}\oldstylenums{2}.  Should it be noted that \textsc{tpm} is not clear on the interpretation of ‘typology’? The term is unclear and it lends itself to ambiguous interpretation. One can interpret it as referring to a specific linguistic structure, which is otherwise called for the sake of this paper ‘local typology’; or one can also interpret it as referring to the whole language, which is called ‘global typology’. For the sake of clarity in this study, I refer to typology as a specific linguistic structure or local typology.

\section{The linguistic phenomenon}\label{sec:kabasele:3}

This section discusses and contrasts the use of the past tense, simple present, and the present \isi{perfect} tense in the three aforementioned languages. The simple past exists in \ili{French}, \ili{English}, and \ili{Lingala} while the form, “Aux (have/avoir) + past participle” exists only; form wise, in both \ili{French} and \ili{English}. 

The present \isi{perfect} tense in \ili{English} is made up of the auxiliary “have” plus the past participle. This tense is similar in form to ‘passé composé’ in \ili{French}, which is also made up of the auxiliary “avoir” (have) plus the past participle. The present \isi{perfect} and the ‘passé composé’ tenses present the same formal paradigm but differ in terms of their function. 

The present \isi{perfect} tense is always used in \ili{English} to talk about a past until now event while the ‘passé composé’ in \ili{French} is often used to express a past-completed event. Syntactic change has taken place in \ili{French} in which the ‘passé composé’ is used instead of the past historic to express a past-completed event. At this point, I can claim that the \ili{English} present \isi{perfect} tense is similar to \ili{French} ‘passé composé’ with respect to form but it does not exist in \ili{Lingala}. Therefore, different tenses are used to express the same idea but in a different language. For instance, in \ili{English}, the present \isi{perfect} tense is used with expressions like ‘the first time’, ‘the second time’, since, for, etc., while in \ili{French} and \ili{Lingala} the present and the immediate past are respectively used.

The simple past tense is used in \ili{English} to talk about events which took place in the past and when the time period is completed. In \ili{French} and \ili{Lingala} the ‘passé composé’ and the past are respectively used. In \ili{Lingala}, an appropriate past tense form needs to be selected depending on whether the event was completed in the recent past or in the distant past. Example (1) illustrates the case.

\ea
{\ili{English} simple past}\\
	Joe bought a car last year.\\
\z


\protectedex{
\ea
{\ili{French} passé composé}\\
\gll Joe a acheté une voiture l’ année passé.\\
     Joe has bought a car the year past\\
\glt ‘Joe bought a car last year.’
\z
}

\ea
{\ili{Lingala} recent past}\\
\gll Joe a- somb -aki mutuka mbula eleki.\\
     Joe 1pssva buy rec.pst car year past\\
\glt ‘Joe bought a car last year.’
\z

Tables \ref{tab:kabasele:1} and \ref{tab:kabasele:2} summarize tenses in these three languages. 

\begin{table}
\caption{Past event that was completed in the past.}
\begin{tabularx}{\textwidth}{llQQ} 
\lsptoprule
& {\bfseries Tense} & {\bfseries Example} & {\bfseries Gloss}\\
\midrule
{ English} & { Simple past} & { \textit{Andy went to Paris last month.}} & \\
\tablevspace
{ French} & { Passé composé} & { \textit{Andy est parti à Paris le mois passé.}} & { ‘Andy went to Paris last month.’}\\
\tablevspace
{ Lingala} & { Remote past} & { \textit{Andy akendáká na Paris bambula eleka.}} & { ‘Andy went to Paris years ago.’}\\
& { Recent past} & { \textit{Andy akendaki na Paris sanza eleki.}} & { ‘Andy went to Paris last month.’}\\
\lspbottomrule
\end{tabularx}
\label{tab:kabasele:1}
\end{table}

\begin{table}
\caption{Past event with connection in the present.}
\begin{tabularx}{\textwidth}{lQQQ}
\lsptoprule
 & {\bfseries Tense} & {\bfseries Example} & {\bfseries Gloss}\\
\midrule
{ English} & { Present perfect} & { \textit{Nathan has lived in Urbana since 2011.}} & \\
\tablevspace
{ French} & { Indicatif present (Simple present)} & { \textit{Nathan vit à Urbana depuis 2011.}} & { ‘Nathan has lived in Urbana since 2011.’}\\
\tablevspace
{ Lingala} & { Past (Immediate past)} & { \textit{Nathan afandi na Urbana banda 2011.}} & { ‘Nathan has lived in Urbana since 2011.’}\\
\lspbottomrule
\end{tabularx}
\label{tab:kabasele:2}
\end{table}

In this paper, my attention is first focused on past-completed event whereby the simple past is used in \ili{English} while the ‘Passé composé’ and remote/recent past are used in \ili{French} and \ili{Lingala} respectively. The linguistic phenomenon which is the focus of the predictions in this study informs us that both the ‘present \isi{perfect}’ in \ili{English} and the ‘passé composé’ in \ili{French} present form similarities in terms of their syntactic structure which is “the auxiliary have + the past participle” but they diverge in terms of their function. The ‘passé composé’ is used to express a past-completed event while the ‘present \isi{perfect} tense’ is used to express a past until now event. The ‘passé simple’ in \ili{French} will not transfer because it is marked and is hardly used in spoken communication; it is a tense that is used by highly educated people in literary discourse.

Second, my interest is oriented to past until now event in which the present \isi{perfect} tense is used in \ili{English} while for the same event, \ili{French} uses the simple present tense but \ili{Lingala} uses the (immediate) past.

\section{Predictions and research questions}\label{sec:kabasele:4}

The predictions of this paper are organized as such: Based on the \textsc{tpm} which claims that only the language with syntactic proximity with the \textsc{tl} serves as the source of transfer, the study posits that if subjects are tapping their linguistic knowledge from the \textsc{l}\oldstylenums{1} to talk about a past completed event in \ili{English} they will use the simple past tense. This tense choice will be triggered by the local syntactic similarity in terms of form between the simple past in \ili{English} and the remote/ recent past in \ili{Lingala} in the context of a past-completed event. 

Transfer occurs because subjects make an interlingual identification; they perceive and judge that the form of the syntactic structure of the remote/recent past in \ili{Lingala} is similar to the form of the syntactic structure of the simple past tense in \ili{English}. It is also the form of the syntactic structure of the simple past in \ili{English}, which has invited the perception of the similarity between the forms of the sentences in both languages. Transfer is triggered by the psychotypological constraint, which enables subjects to perceive similarity between the two tenses. This similarity is observed at the level of form of the tenses. It is hence clear that the syntactic structure of a previously acquired language is susceptible to transfer as \citet[174]{Jarvis2010} puts it, “only if it is perceived to have a similar counterpart in the recipient language.” The perception of the similarity is not only observed on the surface level but subjects’ perception of the similarity at the psychological level plays also a role for transfer to occur.  

With reference to the “\textsc{l}\oldstylenums{2}-status factor” model which claims that the \textsc{l}\oldstylenums{2} is the strongest source of transfer in \textsc{l}\oldstylenums{3} acquisition and that the \textsc{l}\oldstylenums{2} blocks any syntactic transfer from the \textsc{l}\oldstylenums{1} syntactic system, the study posits that if subjects are tapping their linguistic knowledge from the \textsc{l}\oldstylenums{2} to talk about a past completed event in \ili{English} they will use the present \isi{perfect} tense. As discussed earlier, the ‘passé simple’ in \ili{French} will not transfer because it is marked and is hardly used in spoken communication. It is a tense that is used by highly educated people in literary discourse and it has been replaced by the ‘passé composé’.

Based on the \textsc{cem} which claims that learners rely on their cumulated linguistic knowledge from both \textsc{l}\oldstylenums{1} and \textsc{l}\oldstylenums{2} as source of transfer and that transfer is only positive or null; the study posits that if subjects are tapping their linguistic knowledge from both \textsc{l}\oldstylenums{1} and \textsc{l}\oldstylenums{2} to talk about a past completed event in \ili{English} they will use the simple past tense.

In light of the decisive factors, closeness between the \textsc{l}\oldstylenums{1} \ili{Lingala} and \textsc{l}\oldstylenums{3} \ili{English} (in form) but difference in ‘form’ between the \textsc{l}\oldstylenums{2} (passé compose) and \textsc{l}\oldstylenums{3} (simple past) and the aforementioned predictions, the work seeks to answer the following questions: Which previously acquired language between the \textsc{l}\oldstylenums{1}, \textsc{l}\oldstylenums{2}, or both \textsc{l}\oldstylenums{1} \& \textsc{l}\oldstylenums{2} takes precedence in \textsc{l}\oldstylenums{3} syntactic transfer? Is the \textsc{l}\oldstylenums{2} the privileged source of syntactic transfer even when the \textsc{l}\oldstylenums{1} offers some syntactic similarities with the \textsc{l}\oldstylenums{3}? Answers to these concerns will shed light on my study.


\section{Design and methodology}\label{sec:kabasele:5}
\largerpage
\subsection{Subjects}

Twenty-five adult Congolese immigrants who live in the USA participated in the study. The average age when they started to be exposed to \ili{English} is 15 years old and their average length of residence in the USA is 3 years. Most of them acquired \ili{French} through instructional exposure at school and their average length of exposure through formal instruction in \ili{French} is 4 years. They also formally learned \ili{English} as a school subject.

All the subjects grew up in Kinshasa and attended school in the same setting. They are all native speakers of \ili{Lingala} who also speak \ili{French} as an official language. The latter is used as an official language and as the language of instruction from elementary school upward. \ili{French} was also learned as a school subject whereby emphasis was made on the grammar of \ili{French}. \ili{English} was exclusively learned as a school subject. Subjects started taking \ili{English} from ninth grade of high school up to twelfth grade. However, \ili{English} was heavily taught structurally. Little attention was paid to other language basic skills. Therefore, students completed the high school program with very poor speaking, reading, writing, and listening skills. They all have at least a high school state diploma from the Democratic Republic of the \isi{Congo}. Subjects are, however, exposed to \ili{English} in the USA on a daily basis at work place, stores, and public places. They tend to interact in \ili{Lingala} whenever they meet with other Congolese fellows. 

Subjects with advanced level of proficiency in \ili{English} could hold a long conversation of approximately ten minutes in \ili{English}. They exhibited oral fluency but with a few grammatical errors. They could ask clarification questions and could answer questions on social life topics with certain ease. Intermediate proficiency level subjects could ask questions with hesitation and were able to hold an intelligible interaction despite some observed limitation they displayed in vocabulary. They made random language errors, which sometimes could lead to communication breakdown. Beginner subjects had limited \ili{English} proficiency level. They had significant amount of difficulty in the four basic language skills. They were hesitant in their speech and their grammar was poor.  Subjects were administered a cloze test to determine their proficiency level in \ili{English}.

The control group was made up of five American native speakers of \ili{English}. All the five subjects grew up in the USA and have been exposed to \ili{English} since birth. They all have at least a high school degree and had taken at least a foreign language at school.

\subsection{Task and procedure}

I administered the interview and the written elicitation task to the subjects to collect the data of the study. The interview was always administered prior to the written elicitation task an each subject took both tasks on the same day. Subjects were interviewed in \ili{English} and the interview, which was audio recorded elicited the data through oral mode of interaction and under time pressure. I used the smartphone Alcatel one touch to record the interviews. Whereas, the written elicitation task aimed to elicit data through the written mode of communication and allowed enough time to subjects to express themselves. Obviously, the interview elicited the data through the implicit mode while the written elicitation task did so through the explicit mode. All the questions were used only once in each task. They were never repeated in another task. All the interviews were recorded and then transcribed. 

The interview was related to past-completed event. It aimed to elicit \isi{verb} tense forms in the simple past (questions 1 and 3). The future (question 4) was used as a distractor in the study. The questions aimed to trigger a specific \isi{verb} tense in the speech production of each subject. The simple past category had two questions while the future category had only one question. The question related to the future was a distractor. For the sake of this study, after analysis of the questions, only questions 1 and 3 were reported. Data related to question 2 would be incorporated in the larger project, which is related to this study. Question 4 was not reported because it was a distractor. The following are the interview questions: Tell me about something that you remember from your life in \isi{Congo}? Tell me about your two big accomplishments in the last six months? Tell me about your first arrival in the USA? Tell me about something that you would like to do in six months? All the four interview questions were asked in the same order prior to handing out the written elicitation task to the subjects. 

Later on in the analysis of the interview, three coders determined the obligatory contexts in which the simple past tense had to be used. I was the first coder. Then, two other coders who were native speakers of \ili{English} contributed with their expertise. The native speakers were teachers of \ili{English} who were trained as teachers of \ili{English} to speakers of other languages or linguists. The coding was first done separately. And then, all the three coders came together to discuss some minor differences, which were observed. 

However, the written elicitation task had 24 questions. The task was organized into a category of six items. The targeted category was the simple past tense and the present \isi{perfect} tense; the future, the simple present and the present \isi{progressive} were distractors. In this study, only the category of items that are related to the use of the simple past tense are reported and data related to the use of the present \isi{perfect} tense will be reported in the other parts of the whole projects. All the instructions were given in \ili{English}.

\subsection{Proficiency test}
\largerpage
The proficiency of the subjects was determined through the administration of the cloze test. The cloze test was an adaptation from the American Kernel Lessons that was drawn from the Advanced Students’ Book by \citet{O’NeillEtAl1981}. The cloze test provided blanks with three options of which subjects had to choose one in order to fill in the blank space with the option s/he deemed as the correct answer. The results of the test divided the subjects into three proficiency groups: beginning, intermediate, and advanced levels with the scores varying between 18 to 24, 25 to 29, and 30 to 37 respectively. Beside the cloze test, subjects were also administered the language learning history in order to elicit their language learning background, their personal data, and the family linguistic history.

\section{Results}\label{sec:kabasele:6}

Discussing the results of the paper, the first task was the interview while the second was the written elicitation and the results are presented in tables, which quantify the former with respect to the category of items. The columns present the required context in which a given tense was expected to be used (this is the target tense), the prediction(s) to the category of items, i.e., the various tenses which were predicted, and finally the unexpected answers which were called “Other verbal forms”. 

The inferential statistics was conducted to compare the control group’s use of the simple past and present \isi{perfect} tense with the 3 proficiency groups that is, beginner, intermediate, and advanced groups in the context of past-completed event. Its goal was to determine whether the control group’s use of the simple past and present \isi{perfect} tense in the aforementioned contexts was significantly different from that of beginner, intermediate, and advanced groups respectively. Moreover, it also aimed to help draw sound decisions therefore on whether the use of the simple past and the present \isi{perfect} tense by the 3 proficiency groups could be attributed to transfer or not.  

Although the goal of this study was to examine the kinds of forms that subjects used in different circumscribed contexts rather than merely focusing on comparing the different groups in the study, I hope that inferential statistics will also contribute in inducing sound decisions on the interpretation of the results.

\begin{table}
\caption{Response types to interview eliciting the context of past completed event (Task 1).}
\begin{tabular}{lrrrrrrrrr} 
\lsptoprule
& \multicolumn{2}{c}{Simple past}  & \multicolumn{2}{p{1.5cm}}{\centering Present perfect}  & \multicolumn{2}{p{1.5cm}}{\centering Simple present} & \multicolumn{2}{p{1.8cm}}{\centering  Other verbal forms}\\\cmidrule(lr){2-3}\cmidrule(lr){4-5}\cmidrule(lr){6-7}\cmidrule(lr){8-9}
& N & \% & N & \% & N & \% & N & \% \\\midrule
Beginner & 66& 64.7& 0& 0& 36& 35.2& 0      & 0\\
Intermediate & 140& 79.0& 0& 0& 35& 19.7& 2 & 1.1\\
Advanced & 162& 92.5& 0& 0& 13& 7.4& 0      & 0\\
Control & 41& 95.3& 2& 4.6& 0& 0& 0         & 0\\
\lspbottomrule
\end{tabular}
%%please move \begin{table} just above \begin{tabular
\label{tab:kabasele:3}
\end{table}

\begin{figure}
\begin{tikzpicture}
 \begin{axis}[
 compat=1.14,
  ybar stacked,
  nodes near coords,
  bar width=25pt,
    width=9cm,
  xtick=data,
  symbolic x coords={Beginner, Intermediate, Advanved, Control Group},
%   enlarge y limits={abs=10,upper},
  legend style={at={(0.5,-0.1)},anchor=north},
    cycle list={black},
  legend columns=-1,
  ymin=0,
 ]
 \addplot+[ybar, pattern color=black!40, pattern=crosshatch] 		plot coordinates {(Beginner,64.7) (Intermediate,79)   (Advanved,92.5) (Control Group,95.3)};
 \addplot+[ybar, pattern color=black!60, pattern=dots] 			plot coordinates {(Beginner,0)    (Intermediate,0)    (Advanved,0)    (Control Group,4.6)};
 \addplot+[ybar, pattern color=black!40, pattern=north west lines]	plot coordinates {(Beginner,35.2) (Intermediate,19.7) (Advanved,7.4)  (Control Group,0)};
 \addplot+[ybar, pattern color=black!40, pattern=north east lines] 	plot coordinates {(Beginner,0)    (Intermediate,1.1)  (Advanved,0)    (Control Group,0)};
 \legend{Simple Past, Present Perfect, Simple Present, Other verbal forms}
 \end{axis}
\end{tikzpicture}
\caption{Interview eliciting the context of past completed event.}
\label{fig:kabasele:1}
\end{figure}

A one-way \textsc{anova} was conducted to compare the control group with the 3 proficiency groups with respect to the use of the simple past tense in the context of past-completed event whereby the use of the simple past tense expressed in percentage was the dependent variable and the groups the independent variables. The \textsc{anova} reveals that there were no significant differences between the control group and the 3 proficiency groups [F (3, 29) = 2.36, p=.094]. A word of caution should be mentioned that given the small sample size in this study, I suspect that the small sample size might have impacted the statistical power to reach the significant difference between the control group and the 3 proficiency groups.

\begin{table}
\caption{Response types to the written elicitation task on past completed event (Task 2).}
\begin{tabular}{lrrrrrrrrr} 
\lsptoprule
& \multicolumn{2}{c}{Simple past}  & \multicolumn{2}{p{1.5cm}}{\centering Present perfect}  & \multicolumn{2}{p{1.5cm}}{\centering Simple present} & \multicolumn{2}{p{1.8cm}}{\centering  Other verbal forms}\\\cmidrule(lr){2-3}\cmidrule(lr){4-5}\cmidrule(lr){6-7}\cmidrule(lr){8-9}
& N & \% & N & \% & N & \% & N & \% \\\midrule
Beginner & 40& 74& 0& 0& 2& 3.7& 12      & 22.2\\
Intermediate & 49& 90.7& 4& 7.4& 0& 0& 1 & 1.8\\
Advanced & 41& 97.6& 1& 2.3& 0& 0& 0     & 0\\
Control & 30& 100& 0& 0& 0& 0& 0         & 0\\
\lspbottomrule
\end{tabular}
\label{tab:kabasele:4}
\end{table}

\begin{figure}
\caption{Written elicitation task on past completed event.}
\begin{tikzpicture}
 \begin{axis}[
 compat=1.14,
  ybar stacked,
  nodes near coords,
  bar width=25pt,
    width=9cm,
  xtick=data,
  symbolic x coords={Beginner, Intermediate, Advanved, Control Group},
%   enlarge y limits={abs=10,upper},
  legend style={at={(0.5,-0.1)},anchor=north},
    cycle list={black},
  legend columns=-1,
  ymin=0,
 ]
 \addplot+[ybar, pattern color=black!40, pattern=crosshatch] 		plot coordinates {(Beginner,74) (Intermediate,90.7)   (Advanved,97.6) (Control Group,100)};
 \addplot+[ybar, pattern color=black!60, pattern=dots] 			plot coordinates {(Beginner,0)    (Intermediate,7.4)    (Advanved,2.3)    (Control Group,0)};
 \addplot+[ybar, pattern color=black!40, pattern=north west lines] 	plot coordinates {(Beginner,3.7) (Intermediate,0) (Advanved,0)  (Control Group,0)};
 \addplot+[ybar, pattern color=black!40, pattern=north east lines]	plot coordinates {(Beginner,22.2)    (Intermediate,1.8)  (Advanved,0)    (Control Group,0)};
 \legend{Simple Past, Present Perfect, Simple Present, Other verbal forms}
 \end{axis}
\end{tikzpicture}
\label{fig:2}
\end{figure}

A one-way \textsc{anova} was conducted to compare the control group with the 3 proficiency groups with respect to the use of the simple past tense in the context of past-completed event whereby the use of the simple past tense expressed in percentage was the dependent variable and the groups the independent variables. The \textsc{anova} reveals that there were no significant differences between the control group and the 3 proficiency groups [F (3, 29) =2.17, p=.11]. As mentioned earlier, the small sample size in this study might have impacted the statistical power to reach the significant difference between the control group and the 3 proficiency groups.

\section{Discussion}\label{sec:kabasele:7}

Both the \textsc{tpm} and \textsc{cem} predicted that subjects will use, to talk about past completed event, the simple past tense as a result of respectively transfer from the \textsc{l}\oldstylenums{1} which shows syntactic proximity with the \textsc{tl} and transfer from both the \textsc{l}\oldstylenums{1} and \textsc{l}\oldstylenums{2} as a result of cumulated knowledge. The “\textsc{l}\oldstylenums{2}- status factor” predicted that subjects will use the present \isi{perfect} tense in the aforementioned context. 

In the present study, I investigated structural transfer in third \isi{language acquisition}. In the interview and written elicitation tasks dealing with past-completed event the results showed that subjects used the simple past tense in the context of past-completed event. The results are contrasted with the predictions of the study in \tabref{tab:kabasele:5}.

\begin{table}
\small
\begin{tabularx}{\textwidth}{QQQQl}
\lsptoprule
{\bfseries Context} & {\bfseries \textsc{l}\oldstylenums{1} Transfer} & {\bfseries \textsc{l}\oldstylenums{2} Transfer} & {\bfseries \textsc{l}\oldstylenums{1} \& \textsc{l}\oldstylenums{2} Transfer} & {\bfseries English}\\
\midrule
{ Past completed event} & { Simple past (\textsc{tpm})} & { Present \isi{perfect} (\textsc{l}\oldstylenums{2} status factor)} & { Simple past (\textsc{cem})} & { Simple past}\\
\lspbottomrule
\end{tabularx}
\caption{Predictions and results of the study.}
\label{tab:kabasele:5}
\end{table}

Referring to the context of past-completed event, the results raise the question of knowing whether the use of the simple past tense by the subjects in the context was due to transfer from the previously acquired languages or whether it was the result of successful acquisition of the tense in the \textsc{l}\oldstylenums{3}. The results of the inferential statistics, which I take with reserve, in relation to the use of the simple past tense in the context of past completed event in both tasks, that is, the interview and the written elicitation task fairly shows that there were no significant differences between the control and the 3 proficiency groups. In both the interview and the written elicitation task, the \textsc{anova} showed respectively that there was no significant differences between the control group and the 3 proficiency groups [F (3, 29) = 2.36, p= .094] and [F (3, 29) = 2.17, p= .11]. However, because of the small sample size of the study, the inferential statistics is not taken into consideration because I suspect that the small sample size of the study might have affected the statistical power to reach the significant difference between the control group and the 3 proficiency groups, yet numerically the difference between those groups are obvious.

Referring to the descriptive statistics, specifically to the numerical results as they are depicted on table 5 and 6, there seem to be obvious differences between the control group and the 3 proficiency groups. It is likely that subjects are tapping their linguistic knowledge from the \textsc{l}\oldstylenums{1} to express in the \textsc{tl} an event, which took place in the past and was completed in the past. However, the possibility of interpreting the results as a consequence of learning from the input is still open because if the use of the simple past tense was solely attributed to transfer effects, we could expect to have more transfer with beginners than with advanced proficiency groups.

Considering the transfer option, the results suggest that when an \textsc{l}\oldstylenums{1} offers some syntactic similarities with the \textsc{tl}, its (\textsc{l}\oldstylenums{1}) syntactic system becomes transparent and thus accessible to the learners. This finding challenges the claims of the \textsc{l}\oldstylenums{2} status factor, which postulate that the \textsc{l}\oldstylenums{2} blocks, the access to the syntactic system of the \textsc{l}\oldstylenums{1}. I assume that the \textsc{l}\oldstylenums{2} blocks access to the \textsc{l}\oldstylenums{1} syntactic system only when the latter does not display any similarities with the syntactic system of the \textsc{tl}.

Should it be mentioned that it is not clear whether transfer from the \textsc{l}\oldstylenums{1} was due to the effects of previously cumulated linguistic knowledge or just a matter of syntactic proximity which was observed between the two linguistic systems. With reference to the numerical results on the aforementioned tables, I suspect that \textsc{l}\oldstylenums{1} transfer into the \textsc{tl} in this study was triggered by the syntactic proximity. The great number of simple past tense use by advanced learners in the context of past-completed event shows that there was positive transfer or positive learning as I will discuss it later in this section. However, the high use of the simple present tense by beginners at the rate of 35.2\% and by intermediate learners at the rate of 19.7\% implies that those learners are using the simple past tense but they just fail to inflect the \isi{verb} with the appropriate past tense inflectional morpheme. The proficiency factor boosts and ameliorates the access to the syntactic system of the \textsc{l}\oldstylenums{1}.

\newpage 
The other reading of the results attributes subjects’ performance to learning. It is likely that the use of the simple past tense in the past-completed context may be due to learning. It might further be interpreted that subjects successfully learned the use of the simple past tense in past completed event context and that the occurrence of simple present tense use in this context might be just attributed to failure to append the simple past tense inflectional morpheme to the \isi{verb} stem since subjects have not mastered the morphology inherent to the simple past tense yet. 

Furthermore, contrasting their performance in interview versus written elicitation task with reference to subjects’ use of the simple present tense in the context of past completed event, it is observed that in the interview whereby subjects had to resort to their implicit knowledge due to time pressure they made more omission errors than in the written elicitation task which required explicit knowledge. The rate of omission errors was decreasing and correlated with the level of proficiency: beginners 35.2\%, intermediate 19.7\%, and advanced 7.4\%. Whereas, in the written elicitation task, beginners’ rate of omission errors was relatively low, i.e., 3.7\% while intermediate and advanced subjects did not make any omission error at all. The type of knowledge one resorted to can account for this difference. In the interview, subjects did not have enough time to think and readjust their speech as they were being interviewed while in the case of written elicitation task, subjects had more time to prepare their answers and to observe that there was an inflectional morpheme missing and they could self-correct their mistakes by appending the omitted simple past tense inflectional morpheme to the \isi{verb} form. 

\figref{fig:kabasele:1} depicts, in a stairs-like manner, how the use of the simple past tense correlates with the level of proficiency. Inversely, it also depicts how the occurrence of the simple past tense inflectional morpheme omission errors correlates with the same level of proficiency.   This reinforces the option that subjects are at a learning stage whereby they have learned that the simple past tense should be used in the context of past-completed event but they are still struggling with inflecting the \isi{verb} with the appropriate morphological marker, which will express and mark the simple past tense.  

The use of the simple present tense in this context could be justified as the result of error of inflectional morpheme omission. This could imply that subjects made positive transfer but just failed to appropriately inflect the verbs in the past tense. Subjects need more time to reinforce the learning of function/use of the simple past tense which seems to be acquired but mostly to digest and control the appropriate morphological form to append in order to fully acquire the tense.

With reference to the prediction related to past-completed event, the use of the simple past tense by the subjects is the result of positive transfer. The use of the simple present tense in this context is considered as the result of error of the simple past tense inflectional morpheme omission.

\largerpage
In light of the research questions which sought to determine the language that takes precedence as source of syntactic transfer in \textsc{l}\oldstylenums{3} acquisition, the research question which aimed to determine whether the \textsc{l}\oldstylenums{2} syntactic system blocks the syntactic transfer even when the \textsc{l}\oldstylenums{1} offers some syntactic similarities with the \textsc{l}\oldstylenums{3}, the interpretation of the results could be twofold. With reference to the inferential statistics, the latter did not have enough statistical power to determine the difference between the control group and the three proficiency groups. The statistical power was affected and weakened by the small sample size of the subjects.  However, because of the small sample size in the study, which might have affected the statistical power, one can consider looking at the descriptive statistics, particularly the numerical results as they are presented on table 5 and 6. Numerically, it is obvious that there was transfer. Responding to the question ‘Which previously acquired language between the \textsc{l}\oldstylenums{1}, \textsc{l}\oldstylenums{2}, or both \textsc{l}\oldstylenums{1} \& \textsc{l}\oldstylenums{2} takes precedence in \textsc{l}\oldstylenums{3} syntactic transfer’ the answer would be that transfer came from the \textsc{l}\oldstylenums{1}. 

Answering the second question which aimed to determine whether the \textsc{l}\oldstylenums{2} was the privileged source of syntactic transfer even when the \textsc{l}\oldstylenums{1} offers some syntactic similarities with the \textsc{l}\oldstylenums{3}, the answer would be no. The \textsc{l}\oldstylenums{2} does not serve as the privileged source of transfer when the \textsc{l}\oldstylenums{1} offers syntactic similarity with the \textsc{l}\oldstylenums{3}.

Finally, attempting to answer the question of knowing whether subjects have more and easy access to their implicit knowledge than the explicit knowledge and therefore transfer more explicit knowledge than the implicit one when tapping linguistic knowledge from a previously acquired linguistic system, the results have shown that subjects are more accurate when given the opportunity to use their explicit knowledge. This finding corroborates with those of previous studies whereby it was attested that subjects were more accurate when in explicit mode than in implicit one \citep{Schmidt2001,Schmidt1995,Leow1998,Robinson1997}. It should, however, be noted that the erroneous use of the simple present tense in the context of past completed event was mostly observed in the context of implicit task. This shows and might imply that learners are linguistically unsecured when in implicit mode and thus they become inaccurate when they rely upon implicit knowledge in their use of the target language.

\section{Conclusion}\label{sec:kabasele:8}

The findings of the study attribute the use of the simple past tense in the context of past-completed event to positive transfer. However, the possibility of attributing the results to positive learning is also to consider since inferential statistics did not reach any significance differences. The findings of inferential statistics were discarded because they were affected by the small sample size and thus could not determine significant difference between the control and the 3 proficiency groups. 

The use of the simple present tense, in the context of past-completed event, which cannot be accounted for by transfer in this context is the result of omission of the simple past tense inflectional morpheme. This failure by the subjects to append the simple past inflectional morpheme to the \isi{verb} to express the simple past tense shows that subjects have not fully acquired the morphology inherent to the simple past tense and this was mostly observed in implicit task.

The study has further shown that subjects were more accurate in using their explicit than implicit knowledge. They also made more positive transfer from explicit knowledge than from implicit one. The use of the simple past tense in the past completed event context could possibly suggest that the learners only know one way to discuss events that happened in the past using the simple past.

I envisage replicating this study with a representative number of subjects in order to avoid any negative implication on the statistical power. I will integrate the comprehension aspect of language transfer to have a full understanding of both production and comprehension. I further project to present a hierarchical matrix of potential factors which can trigger transfer and rank them in pairs, triplet or in quadruplet depending on the factors which will be controlled.
 
 
{\sloppy
\printbibliography[heading=subbibliography,notkeyword=this]
}
\end{document}