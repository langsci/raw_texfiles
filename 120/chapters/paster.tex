\documentclass[output=paper,modfonts]{langscibook} 
\title{Gender instability in Maay} 
\author{Mary Paster\affiliation{Pomona College} }
\abstract{This paper discusses variation in the gender of nouns in Maay, a language of Somalia. Languages of the Eastern Omo-Tana subgroup of East Cushitic (including Maay, Somali, Rendille, and Tunni) have gender systems wherein every noun is masculine or feminine. Masculine nouns take k-initial variants of suffixes including the definite marker, demonstratives, and possessive markers; these suffixes are \textit{t}-initial with feminine nouns. As is now well known, gender in these languages is sensitive to plurality in various ways: in some languages, gender ‘polarity’ reverses the gender of nouns in the plural; in others, feminine nouns change to masculine when their plurals are formed with certain suffixes but not others. In Maay, plurals are all masculine regardless of how they are formed, but the gender of many singular nouns is inconsistent across individuals. The masculine plural pattern makes the gender of singular nouns unrecoverable from their plurals, so nouns that are frequently plural are susceptible to gender instability. If there is uncertainty about the gender of some nouns, speakers may be inclined to guess masculine, thereby producing more feminine to masculine changes than the reverse, due to the prevalence of masculine nouns in the Maay lexicon.}
\ChapterDOI{10.5281/zenodo.1251732}
\begin{document}
\maketitle 
\section{Introduction}\label{sec:paster:1}
% \todo{almost all examples in this paper should be tables.}
The \ili{Maay} language is known to exhibit significant inter-speaker variation in its phonology and morphology \citep{Paster2013}. This paper describes variability in the gender assignment of \ili{Maay} nouns and considers explanations for why gender is unstable for certain nouns in this language. I will argue that gender instability is connected to, and facilitated by, a regular pattern in the language where gender is neutralized to masculine in plural nouns.

The structure of the paper is as follows. First, in \sectref{sec:paster:2} I give some background on the \ili{Maay} language and its classification. In \sectref{sec:paster:3}, I explain the gender neutralization pattern in \ili{Maay} plurals and discuss similar phenomena in related languages. \sectref{sec:paster:4} describes the problem of gender instability in \ili{Maay}. In \sectref{sec:paster:5}, I propose an explanation of gender instability that attributes the emergence of gender instability in part to gender neutralization in plurals; I also consider and reject a number of alternative explanations. \sectref{sec:paster:6} concludes the paper. 

\section{Background on Maay}\label{sec:paster:2}

\ili{Maay} (also known as Af-\ili{Maay} or MayMay; see \citealt{Paster2007,ComfortPaster2009,Paster2010}) is a Cushitic language spoken in \isi{Somalia} that is related to, but not mutually intelligible with, \ili{Somali}. It is classified as an East Cushitic language, for which a tree is given in \REF{ex:paster:1}.

 
% \todo{this tree causes a pgf bug}
\ea%1
\label{ex:paster:1} 
East Cushitic (modified from \citealt[3]{Saeed1999})
{\small
\begin{forest}
[East Cushitic
 [Saho-\ili{Afar}]  [Macro-\ili{Oromo}]  [\ili{Omo}-Tana, s sep=20mm
  [Western [Dasenach]  [Arbore] [Elmolo]] [Northern]  [Eastern,name=Eastern, delay={where content={}{shape=coordinate}{}}
     [\ili{Rendille},name=Rendille] [ [Boni,name=Boni] [ [\ili{Somali},name=Somali] [\ili{Tunni},name=Tunni] [\ili{Maay},name=Maay] ] ] [Bayso,name=Bayso]
  ]    
 ]   [Sidamo]  [Burji]  [Dullay]  [Yaaku]
] 
\node[draw, fit=(Eastern) (Rendille) (Boni) (Somali) (Tunni) (Maay) (Bayso)] {};
\end{forest}
}
\z

In this paper, I will be focusing on the Eastern \ili{Omo}-Tana (EOT) subgroup of East Cushitic, marked in the tree above.
% \todo{Due to software limitations, we had to change the graph a little. sorry!}

\section{Gender in EOT languages}\label{sec:paster:3}

Cushitic studies often refer to the existence of three genders (masculine, feminine, and plural; cf.  \citealt{CorbettHayward1987}). In \ili{Maay}, this is essentially how third person \isi{subject agreement} works for verbs, as seen in four different tenses in \REF{ex:paster:2} (data and tense/aspect category names are from \citealt{Paster2007}).

\eabox[-.9\baselineskip]{\label{ex:paster:2}
\begin{xlist}
\parbox{.3\textwidth}{
\ex   \textbf{Simple Past }    \\
\gll roor-i      \\       
run-\textsc{3sgm.past}  \\         
\glt ‘he ran’ 

\medskip
\gll  roor-ti     \\   
run-\textsc{3sgf.past}      \\     
\glt ‘she ran’ 

\medskip
\gll roor-eena     \\    
run-\textsc{3pl.past}    \\
\glt ‘they ran’ 
 }
\parbox{.3\textwidth}{
\ex
\textbf{Simple Present B}\\
\gll  ɗeer-ya \\
     be.tall-\textsc{3sgm.stative}\\
\glt    ‘he is tall’

\medskip
\gll  ɗeer-ta   \\
  be.tall-\textsc{3sgf.stative}\\
\glt   ‘she is tall’ 

\medskip
\gll ɗeer-yena  \\
  be.tall-\textsc{3pl.stative}\\
\glt  ‘they are tall’
}
\end{xlist}
}

\begin{exe}
\sn%no numbering here
\parbox[t]{.8\textwidth}{
      \vspace{-.9\baselineskip} 
\begin{xlist}
\setcounter{xnumii}{2}
\parbox{.3\textwidth}{
\ex    \textbf{Immediate Future\vphantom{j}}  \\
\gll kooy-e   \\     
  come-\textsc{3sgm.future}         \\
 \glt ‘he will come’  

 \medskip
\gll kooy-ase    \\
  come-\textsc{3sgf.future}         \\
 \glt ‘she will come’   

\medskip
\gll kooy-ayeena \\
come-\textsc{3pl.future}            \\
 \glt ‘they will come’  
}
\parbox{.4\textwidth}{
 \ex
  \textbf{Present Progressive}\\ 
 \gll aam-oy-e \\
 eat-\textsc{pres.prog-3sgm.present}\\
 \glt ‘he is eating’

 \medskip
\gll  aam-oy-te \\
  eat-\textsc{pres.prog-3sgf.present}\\
 \glt ‘she is eating’

 \medskip
\gll  aam-oy-eena  \\
   eat-\textsc{pres.prog-3pl.present}\\
\glt  ‘they are eating’
}
\end{xlist}
}
\end{exe}


For purposes of \isi{noun} morphology, however, there are only two genders in \ili{Maay}: masculine and feminine. Masculine nouns (\ref{ex:paster:3}a) take \textit{k-}initial suffixes for definites, demonstratives, and most possessive markers, while these suffixes are \textit{t-}initial\textit{} with feminine nouns (\ref{ex:paster:3}b). 

\eabox[-2.5\baselineskip]{\label{ex:paster:3}
\begin{tabbing}
a.~ \= XXXXXXXX \= XXXXXXXXX \= b.~  \= XXXXXXXX \= XXXXXXXXXX \kill \\

a. \> geet-ki   \>  ‘the tree’ \>    b.  \> bilaan-ti   \>  ‘the woman’ \\
\>geet-kaŋ   \>  ‘this tree’ \>    \>  bilaan-taŋ   \>  ‘this woman’ \\
\>geet-kas   \>  ‘that tree’ \>   \>      bilaan-tas   \>  ‘that woman’ \\  
\>geet-kew   \>  ‘which tree’ \>  \>       bilaan-tew   \>  ‘which woman’ \\
\>geet-key   \>  ‘my tree’ \>   \>      bilaan-tey   \>  ‘my woman’ \\
\>geet-ka     \>  ‘your tree’ \>   \>      bilaan-ta   \>  ‘your woman’ \\
\>geet-’ye   \>  ‘his/her tree’ \>  \>       bilaan-tis   \>  ‘his woman’ \\
\>	  \>                  \>   \> bilaan-tie \>  ‘her woman’\\
\>geet-kaynu   \>  ‘our tree’ \>   \>      bilaan-tayno   \>  ‘our woman’ \\
\>geet-kiŋ     \>  ‘your pl. tree’ \>  \>       bilaan-tiŋ   \>  ‘your pl. woman’ \\
\>geet-’yo   \>  ‘their tree’ \>  \>       bilaan-tio   \>  ‘their woman’ \\
\end{tabbing}
}

For the purposes of this paper, I will focus on this type of gender agreement. 

As has been documented elsewhere, gender in EOT languages is sensitive to plurality. This is broadly referred to as “gender polarity”, but it sometimes manifests as neutralization rather than polarity \textit{per se}. This varies from language to language; below I summarize the situation in a number of different EOT languages.

In Standard \ili{Somali} (SS), according to \citet[54--55]{Saeed1999}, most plural nouns reverse their gender. There are multiple different plural suffixes, and several of them trigger gender polarity, as can be seen in (\ref{ex:paster:4}a). However, plurals of masculine nouns formed by reduplication, as well as ‘a subgroup of masculine suffixing nouns’ retain their masculine gender in the plural, as seen in (\ref{ex:paster:4}b).

\eabox[-2.5\baselineskip]{\label{ex:paster:4}
\begin{tabbing}
a.~ \= XXXXXXXX \= XXXXXXXXX \= b.~  \= XXXXXXXX \= XXXXXXXXXX \kill \\
a. \> abtí (m)     \>  ‘maternal uncle’ \>     \>    abti-yó (f)   \>  ‘maternal uncles’ \\
\>  túke (m)    \>  ‘crow’ \>     \>        tuka-yáal (f)   \>  ‘crows’ \\
\>  káb (f)     \>  ‘shoe’ \>    \>         kab-ó (m)   \>  ‘shoes’ \\
\>  galáb (f)   \>  ‘afternoon’ \>    \>       galb-ó (m)   \>  ‘afternoons’ \\
b.\>  wán (m)   \>  ‘ram’ \>      \>       wan-án (m)   \>  ‘rams’ \\
\>    béer (m)   \>  ‘liver’ \>    \>         beer-ár (m)   \>  ‘livers’ \\
\>    dhéri (m)   \>  ‘clay pot’ \>    \>       dhery-ó (m)   \>  ‘clay pots’ \\
\>    wáran (m)   \>  ‘spear’ \>    \>         warm-ó (m)   \>  ‘spears’ \\
\end{tabbing}
} 

Thus, in SS, in the plural all feminine nouns become masculine, and some masculine nouns become feminine but some stay masculine.

Central \ili{Somali} exhibits a different pattern, where plurals formed with the suffix \textit{-o} exhibit polarity (\ref{ex:paster:5}a), while plurals formed with \textit{-(i)yaal} are masculine regardless of their gender in the singular (\ref{ex:paster:5}b) \citep[11--12]{Saeed1982}.


\eabox[-2.5\baselineskip]{\label{ex:paster:5}
\begin{tabbing}
a.~ \= XXXXXXXXXXX \= XXXXXXXXX \= b.~  \= XXXXXXXXXXXXXXX \= XXXXXXXXXX \kill \\
a. \>    fileer-taas (f)     \>  ‘that arrow’ \>    \>    fileer-o-gaas (m)   \>  ‘those arrows’ \\
\>    laan-taas (f)     \>  ‘that branch’ \>  \>      laam-o-gaas (m)   \>  ‘those branches’ \\
\>    shiid-kaas (m)     \>  ‘that stone’ \>  \>      shiid-o-daas (f)     \>  ‘those stones’ \\
\>    eleeŋ-kaas (m)     \>  ‘that ram’ \>   \>     eleem-o-daas (f)   \>  ‘those rams’ \\
  b.  \>   jeer-taas (f)     \>  ‘that hippo’ \>  \>      jeer-iyaal-kaas (m)   \>  ‘those hippos’ \\
\>    shimbir-taas (f)     \>  ‘that bird’ \>  \>      shimbir-iyaal-kaas (m)   \>  ‘those birds’ \\
\>    ba’iid-kaas (m)     \>  ‘that oryx’    ba’ \>  \>  iid-iyaal-kaas (m)   \>  ‘those oryxes’ \\
\>    weer-kaas (m)     \>  ‘that jackal’ \>  \>      weer-iyaal-kaas (m)   \>  ‘those jackals’ \\
\end{tabbing}
}

\citegen{Lecarme2002} discussion of an unidentified \ili{Somali} dialect (which appears to be distinct from both SS and CS) includes the observation that each of several different pluralization strategies tends to result in plural forms with a particular gender. For example, all plurals in \textit{-o} (whose singulars are mostly feminine, but some masculine nouns also occur in this group) (\ref{ex:paster:6}a) are masculine \citep[118]{Lecarme2002}. Plurals in \textit{-oyin} (whose singulars are always feminine) are always masculine (\ref{ex:paster:6}b) \citep[119]{Lecarme2002}. Plurals in \-\textit{-yaal} (\ref{ex:paster:6}c) (whose singulars can be masculine or feminine) are “masculine or feminine, depending on regional variation, and thus either polaric or not” \citep[119]{Lecarme2002} (though note that the only plural forms provided are feminine).


\eabox[-2.5\baselineskip]{\label{ex:paster:6}
\begin{tabbing}
a.~ \= XXXXXXXXXXXX \= XXXXXXXXXX \=  \= XXXXXXXXXXXXXXX \= XXXXXXXXXX \kill \\
a. \>  fár (-ta) (f)     \>  ‘finger’ \>    \>     far-ó (-á-ha) (m)   \>  ‘fingers’ \\
\>náag (-ta) (f)     \>  ‘women’ \>   \>    naag-ó (-á-ha) (m)   \>  ‘women’ \\
\>maálin (-ka) (m)   \>  ‘day’ \>    \>     maalm-ó (-á-ha) (m)   \>  ‘days’ \\
\>wáran (-ka) (m)     \>  ‘spear’ \>   \>      warm-ó (-á-ha) (m)   \>  ‘spears’ \\
\end{tabbing}
}
 
\eaboxctd{
\vspace*{-2.5\baselineskip}
\begin{tabbing}
a.~ \= XXXXXXXXXXXX \= XXXXXXXXXX \=  \= XXXXXXXXXXXXXXX \= XXXXXXXXXX \kill \\
b. \>  hóoyo (-áda) (f)     \>  ‘mother’ \>  \>     hooyo-óyin (-ka) (m)   \>  ‘mothers’ \\
\>eeddó (-áda) (f)     \>  ‘paternal aunt’ \>   \>    eeddo-óyin (-ka) (m)   \>  ‘paternal aunts’ \\
\>magaaló (-áda) (f)   \>  ‘town’ \>   \>      magaalo-óyin (-ka) (m)   \>  ‘towns’ \\
\>xeró (-áda) (f)     \>  ‘enclosure’ \>   \>    xero-óyin (-ka) (m)   \>  ‘enclosures’ \\
c.\>   maroodí (-ga) (m)   \>  ‘elephant’ \>    \>   maroodi-yáal (-sha)   \>  ‘elephants’ \\
\>waraábe (-áha) (m)   \>  ‘hyena’ \>   \>      waraaba-yáal (-sha) (f)   \>  ‘hyenas’ \\
\>áabbe (-áha) (m)   \>  ‘father’ \>    \>     aaba-yáal (-sha) (f)   \>  ‘fathers’ \\
\>jáalé (-áha) (m)     \>  ‘comrade’ \>   \>    jaala-yáal (-sha) (f)   \>  ‘comrades’ \\
\end{tabbing}
}

Lecarme’s analysis locates gender features in the various plural suffixes, explaining the connection between the gender changes and the use of each of the suffixes.

In \ili{Maay}, plural nouns are masculine, regardless of whether the plural is formed with the suffix -\textit{o} (\ref{ex:paster:7}a), the suffix -\textit{yal} (\ref{ex:paster:7}b), or with both (\ref{ex:paster:7}c) (\citealt{Paster2007,ComfortPaster2009,Paster2010}) (note that /k/ and /t/ lenite to [ɣ] and [ð] intervocalically).


\eabox[-2.5\baselineskip]{\label{ex:paster:7}
\begin{tabbing}
a.~ \= XXXXXXXXXX \= XXXXXXXXX \=    \= XXXXXXXXXXXXXX \= XXXXXXXXXX \kill \\
a.  \>   d͡ʒeer-tey (f)     \>  ‘my hippo’ \>   \>      d͡ʒeer-o-ɣey (m)     \>  ‘my hippos’ \\
\>gewer-tey (f)     \>  ‘my daughter’ \>    \>     gewer-o-ɣey (m)   \>  ‘my daughters’ \\
\>walaal-key (m)      \>  ‘my brother’ \>   \>      walaal-o-ɣey (m)   \>  ‘my brothers/sisters’ \\
\>ad͡ʒir-key (m)     \>  ‘my thigh’ \>   \>      ad͡ʒir-o-ɣey (m)     \>  ‘my thighs’ \\
b.  \>   mindi-ðey (f)     \>  ‘my knife’  \>  \>      mindi-yal-key (m)   \>  ‘my knives’ \\
\>gaʔan-tey (f)     \>  ‘my hand’ \>   \>      gaʔan-yal-key (m)   \>  ‘my hands’ \\
\>bakeeri-ɣey (m)     \>  ‘my cup’ \>  \>       bakeeri-yal-key (m)   \>  ‘my cups’ \\
\>miis-key (m)     \>  ‘my table’ \>   \>      miis-yal-key (m)   \>  ‘my tables’ \\
c.  \>    d͡ʒeer-tey (f)     \>  ‘my hippo’ \>   \>      d͡ʒeer-o-yal-key  (m)   \>  ‘my hippos’ \\  
\>gaʔan-tey (f)     \>  ‘my hand’ \>   \>      gaʔam-o-yal-key (m)   \>  ‘my hands’ \\
\>ad͡ʒir-key (m)     \>  ‘my thigh’ \>   \>      ad͡ʒir-o-yal-key  (m)   \>  ‘my thighs’ \\  
\>miis-key (m)     \>  ‘my table’ \>   \>      miis-o-yal-key (m)   \>  ‘my tables’ \\
\end{tabbing}
}

In \ili{Tunni}, the “gender-opposition is neutralized in Plural nouns, all of which are Masculine” (\citealt{Tosco1997}: 43; no illustrative examples are provided). And finally, \ili{Rendille} exhibits a complex pattern where some nouns appear to switch gender in the plural, while others take a separate set of plural-agreeing suffixes beginning with /h/ rather than feminine /t/ or masculine /k/ (see \citealt{Oomen1981} for much more detailed discussion).

Summing up, it can be observed that some EOT languages have true polarity and others have only masculine plurals. None of the languages (as far as I am aware) have only feminine plurals.\footnote{A reviewer points out that K’abeena has feminine plurals. According to \citet[143]{Mous2008}, plurals “trigger masculine agreement in the demonstratives, [plural] agreement in the definite markers, but feminine agreement on the \isi{verb} in external, clausal agreement.” Since I am focusing on agreement with determiners, demonstratives, and possessors, for the present purposes I would not consider K’abeena to have feminine plurals.} It seems likely that Proto-EOT did exhibit polarity, since polarity exists in Cushitic languages outside of the EOT group as well, e.g., in \ili{Oromo} \citep{Andrzejewski1960} and \ili{Burunge} \citep{Wolff2014}. 

This background on gender phenomena in EOT will be relevant to the discussion of gender instability in \sectref{sec:paster:4}, since I will argue that it is the masculine-plural pattern in \ili{Maay} that provides the mechanism for the emergence of instability.

\section{Gender instability in Maay}\label{sec:paster:4}

\citet{Wolff2014} observes that “[a] notable historical feature [of Afro-Asiatic languages] is ‘gender stability’, meaning that words for common things tend to share the same gender across the languages of the Afro-Asiatic \isi{phylum}, no matter whether or not the particular words are cognate across the specific languages in question” (the implication being that Afro-Asiatic exhibits a greater degree of gender stability than other families do). But as I will show below, there are exceptions to the ‘gender stability’ generalization within the \ili{Maay} language itself. 

Considering a sample of 55 common lexical items in \ili{Maay}, each elicited from up to 6 speakers, I found 34 of them to be consistently masculine, shown in \REF{ex:paster:8} (note that animals are deliberately excluded, for reasons to be explained below).


          
\eabox[-.7\baselineskip]{\label{ex:paster:8}
Consistently masculine nouns\footnote{In these and examples to follow, where there are inter-speaker pronunciation differences, transcriptions are of speaker HJ or BM; see below for details on these speakers.}
\vspace*{-.5\baselineskip}
\begin{tabbing}
a.~ \= XXXXXXXX \= XXXXXXXXX \= b.~  \= XXXXXXXX \= XXXXXXXXXX \kill \\
\>moos-ki   \>  ‘the banana’ \>  \>     suŋ-ki     \>  ‘the belt’ \\
\>doŋ-ki     \>  ‘the boat’ \>  \>     buug-i     \>  ‘the book’{\rmfnm} \\
\>baaka-ɣi   \>  ‘the box’ \>  \>     kawaʃ-ki   \>  ‘the cabbage’ \\
\>hɛɛl-ki     \>  ‘the cardamom’ \>  \>     kuraas-ki   \>  ‘the chair’ \\
\>belet-ki     \>  ‘the city’ \>  \>     nard͡ʒiŋ-ki   \>  ‘the coconut’ \\
\>hawuug-i   \>  ‘the corn’ \>   \>    bakeri-ɣi    \>  ‘the cup’ \\
\>ilbap-ki     \>  ‘the door’ \>  \>     dɛp-ki      \>  ‘the fire’ \\
\>ɛɛs-ki      \>  ‘the grass’ \>  \>     maða-ɣi    \>  ‘the head’ \\
\>miniŋ-ki    \>  ‘the house’ \>  \>     fur-ki     \>  ‘the key’ \\
\>nal-ki     \>  ‘the light’ \>  \>     beer-ki     \>  ‘the liver’ \\
\>af-ki     \>  ‘the mouth’ \>  \>     basal-ki     \>  ‘the onion’ \\
\>los-ki      \>  ‘the peanut’ \>  \>     galaŋ-ki     \>  ‘the pen’ \\
\>biiŋ-ki      \>  ‘the pin’ \>  \>     barit-ki     \>  ‘the rice’ \\ 
\>wo$\beta $i-ɣi    \>  ‘the river’ \>  \>     d͡ʒit-ki     \>  ‘the road’ \\ 
\>haðag-i     \>  ‘the rope’ \>  \>     kasab-ki   \>  ‘the sugarcane’ \\
\>miis-ki     \>  ‘the table’ \> \>      nyaanya-ɣi   \>  ‘the tomato’ \\  
\>gɛɛt-ki     \>  ‘the tree’ \>   \>    hidig-i     \>  ‘the star’ \\
\end{tabbing}\vspace*{-.5\baselineskip}
}
\footnotetext{Masculine nouns ending in \textit{k} appear to have the suffix -\textit{i} rather than \textit{-ki} because the /k/ of the stem and /k/ of the suffix reduce to a single /k/ and then optionally undergo intervocalic voicing. This stop does not undergo intervocalic lenition to [ɣ] because lenition applies before degemination. \citet{Paster2007} provides a deeper discussion of \ili{Maay} phonology.}

\clearpage 
Six nouns in the dataset were consistently feminine, shown in \REF{ex:paster:9}.


 
\eabox[-.5\baselineskip]{\label{ex:paster:9}
Consistently feminine nouns 
\begin{tabbing}
a.~ \= XXXXXXXX \= XXXXXXXXX \=   \= XXXXXXXX \= XXXXXXXXXX \kill \\
\>saɁad-i     \>  ‘the clock’{\rmfnm} \> \> ɗɛk-ti     \>  ‘the ear’ \\
\>far-ti     \>  ‘the finger’ \>     \>  saan-ti     \>  ‘the footprint’ \\
\>luk-ti     \>  ‘the leg’ \>   \>    suk-ti     \>  ‘the market’ \\  
\end{tabbing}
}
\footnotetext{Feminine nouns ending in \textit{t} appear to have the suffix \textit{-i} due to degemination as with masculine nouns ending in \textit{k}.}    

Interestingly, 15 nouns in the dataset showed inconsistent gender across speakers, as shown in \REF{ex:paster:10}.


\eabox[-.5\baselineskip]{\label{ex:paster:10}
Unstable nouns
\begin{tabbing}
a.~ \= XXXXXXXXXXXXXXX \= XXXXXXX \= b.~  \= XXXXXXXXXXXXX \= XXXXXXXXXX \kill \\
 \> ukun-ti {\textasciitilde} ukuŋ-ki   \>  ‘the egg’ \>    \>      farketi-ði {\textasciitilde} farketi-ɣi    \>  ‘the fork’ \\
 \> il-i {\textasciitilde} il-ki     \>  ‘the eye’{\rmfnm} \>  \>        sun-ti {\textasciitilde} suŋ-ki     \>  ‘the poison’ \\  
 \> beer-ti {\textasciitilde} beer-ki     \>  ‘the garden’ \>   \>       dariʃa-ði {\textasciitilde} dariʃa-ɣi    \>  ‘the window’ \\  
 \> gaɁan-ti {\textasciitilde} gaɁan-ki   \>  ‘the hand’ \>    \>      siin-ti {\textasciitilde} siiŋ-ki     \>  ‘the hip’ \\  
 \> mindi-ði {\textasciitilde} mindi-ɣi    \>  ‘the knife’ \>   \>       hambal-i {\textasciitilde} hambal-ki   \>  ‘the leaf’ \\  
 \> embe-ði {\textasciitilde} embe-ɣi    \>  ‘the mango’ \>    \>      kaal-i {\textasciitilde} kaal-ki     \>  ‘the spoon’ \\
 \> istaraʃa-ði {\textasciitilde} istaraʃa-ɣi   \>  ‘the napkin’ \>    \>      baloon-ti {\textasciitilde} baloon-ki   \>  ‘the ball’ \\
 \> irbid-i {\textasciitilde} irbit-ki  \>  ‘the needle’    
\end{tabbing}
}
\footnotetext{Feminine nouns ending in \textit{l} appear to have the suffix \textit{-i} due to a regular phonological rule that deletes /t/ after /l/.}
In the following section, I will propose an analysis of the unstable nouns in \REF{ex:paster:10}, attempting to explain how their gender came to be ambiguous across speakers.  I will suggest that the pattern of gender neutralization to masculine in \ili{Maay} plurals creates ambiguity in the gender of singular nouns (particularly when the plural form is more familiar), leading speakers to sporadically reassign the gender of some nouns.

\section{Analysis}\label{sec:paster:5}

To explain the unstable nouns, we could look at properties of both the speakers and the nouns themselves. Before presenting my proposed analysis that attributes the instability to gender neutralization in the plural, I consider a number of other factors that could potentially be relevant to gender instability, showing that none turns out to provide an explanation for the observed phenomenon.

To start, we might ask whether age, gender, region, or language use could explain the divide among the speakers. \tabref{tab:paster:11} provides some demographic and language use information for each of the \ili{Maay} speakers consulted for this project.


\begin{table}
\caption{Demographic and language use information}
\label{tab:paster:11}
 \begin{tabularx}{\textwidth}{lllQQ}
\lsptoprule
         Speaker  & Gender  & Age  & Origin &   Languages\\
\midrule
OM  &  M  &  33   & Kowan (near Jamaame)  &   \ili{Zigua}, \ili{Swahili}, \ili{English}\\
HJ  &  M  &  30s  & Jamma; lived in \isi{Kenya} &  \ili{Zigua}, \ili{English}, \ili{Somali}\\
JA  &  M  &  62   & Jilib                 &  Some \ili{English}, some \ili{Somali}\\
HM  &  M  &  42   & Jamaame; grew up in Mogadishu &  \ili{Somali}, \ili{Zigua}, \ili{English}, \ili{Swahili}\\
BM  &  F  &  48   & Jamaame               &  \ili{Zigua}, \ili{Somali}, \ili{English}, some \ili{Swahili}\\
LJ  &  M  &  52   & Jamaame               &  \ili{Zigua}, \ili{Somali}, \ili{English}, \ili{Swahili},  \ili{Italian}, some \ili{Spanish}\\
AM  &  M  &  27   & Kismaayo; lived in \isi{Kenya}  &  \ili{Zigua}, \ili{Somali}, \ili{English}, \ili{Swahili}, Turkana, Giryama\\
HA  &  F  &  30s  & Jamaame; grew up in \isi{Kenya} &  \ili{Zigua}, \ili{Swahili}, \ili{English}\\
MA  &  F  &  23   & Jamaame               &  \ili{Zigua}, \ili{Swahili}, \ili{Somali}, some \ili{English}\\
KJ  &  F  &  50s  & Jamaame; Kismaayo; lived in \isi{Kenya} &   \ili{Zigua}, \ili{Somali}, \ili{Swahili}, some \ili{English}          \\
\lspbottomrule
\end{tabularx}
\end{table}

In \tabref{ex:paster:12}, I give the gender of each \isi{noun} according to the available data elicited from each speaker.


\begin{table}
\caption{Noun gender by speaker}
\label{ex:paster:12}
\begin{tabularx}{.55\textwidth}{Xcccccc} 
\lsptoprule
& \textbf{LJ} & \textbf{HJ} & \textbf{HA} & \textbf{AM} & \textbf{OM} & \textbf{BM}\\
\midrule
ball   &  &  & M & F & F & \\
egg    &  &  & M & F &  & \\
eye    &  & M & M & F & F & \\
fork   &  & F & M &  &  & \\
garden &  &  & M & F &  & F\\
hand   & M &  & F & F & F & \\
hip    &  &  & M &  &  & F\\
knife  & M & F & M &  & F & F\\
leaf   &  & M & F & F & F & \\
mango  & M & M & M &  &  & F\\
napkin & M &  & M &  & F & \\
needle &  & F & M & F &  & \\
poison &  & F & M &  &  & M\\
spoon  &  & M & M & F & F & F\\
window & M &  &  &  & F & \\
\lspbottomrule
\end{tabularx}
\end{table}

Considering the \isi{noun} gender data by speaker, a few generalizations emerge. First, the available data from LJ indicate that he has assigned masculine gender to all of the nouns in question. Second, HA also has many of these nouns as masculine, and HJ has some masculine. All available instances of these 15 words from AM and OM were feminine.

In an attempt to align these observations with the speakers’ demographic characteristics, it can be observed that LJ, HA, and HJ, who had more masculine forms of these nouns than other speakers did, are from the same area. However, OM and BM are also from the same region and had almost all of these nouns as feminine, so an explanation in terms of a geographically defined dialect feature is unlikely. The age of the speakers also does not seem to be a relevant factor, since the group of three speakers who produced the most masculine forms includes one of the oldest speakers (LJ) and two of the youngest (HA and HJ). Gender seems irrelevant, since LJ and HJ are male while HA is female. And finally it can be observed that the two speakers who produced all feminine forms, OM and AM, are younger men, but HJ is also a younger man and produced several masculine forms. Thus, properties of the speakers themselves do not seem to provide any insight into the behavior of the unstable nouns.

It is possible that language experience and/or language attitudes play a role in determining which gender each speaker will assign to a given \isi{noun}, but here again no clear explanation emerges. In language attitude surveys, HA and HJ, both of whom produced more masculine nouns in this set than other speakers, stated that they identify more with \ili{Zigua} (a Bantu language spoken by many members of the community) than \ili{Maay} as their mother tongue and that they speak \ili{Zigua} at home. However, it is not clear what influence \ili{Zigua} might have on the gender of their \ili{Maay} nouns apart from general interference with regular \ili{Maay} usage, since \ili{Zigua} has \isi{noun} classes rather than binary gender. No other details about any of the speakers’ experiences with languages other than \ili{Maay} appear to correlate with the data.

Having considered the possibility that the gender of the unstable nouns is a dialect feature relating to the geographic origin of, or other facts about, the speakers and finding none, we might consider whether there are properties of the nouns themselves that can shed light on why they are unstable across speakers. 

One obvious potential factor could be the phonological form of the nouns, but a consideration of the nouns in (\ref{ex:paster:8}-\ref{ex:paster:10}) does not reveal any good candidates for a phonological property that might unify any set of nouns. The masculine nouns \REF{ex:paster:8} and unstable nouns \REF{ex:paster:10} end in either a consonant or a vowel, and they can have anywhere from one to three syllables (one unstable \isi{noun} has four syllables). The feminine nouns in \REF{ex:paster:9} all end in a consonant, and only one has more than one syllable, but there are stable masculine nouns and unstable nouns that also have these properties, so the shape of the six feminine nouns in \REF{ex:paster:9} does not reflect a phonological natural class.

A second property of the nouns that could be considered is whether they are native or borrowed, and if borrowed, the source of the borrowing. But here again, there is no clear pattern. \tabref{tab:paster:13} gives the sources for four of the nouns in the unstable class that can be identified as borrowings.


\begin{table}
\caption{Sources of borrowed unstable nouns}
\label{tab:paster:13}
 \begin{tabular}{ll}
 \lsptoprule
\textbf{Noun} &     \textbf{Source}\\
\midrule
ball  &  \ili{Italian} or \ili{English}\\
fork  &  \ili{Italian}\\
napkin   & \ili{Italian}\\
poison  &  \ili{Arabic}\\
\lspbottomrule
\end{tabular}\end{table}

While it is true that two or three of the unstable nouns are \ili{Italian} borrowings, one of them is from \ili{Arabic}, and the remaining 11 unstable nouns are apparently native. Therefore once again no solid generalization can be made. The list stable masculine nouns in \REF{ex:paster:8} also includes both native words and borrowings from \ili{English}, \ili{Italian}, and \ili{Arabic}; the stable feminine class in \REF{ex:paster:9} includes both native words and \ili{Arabic} borrowings.

A third possibility is that the semantics of the unstable nouns might explain their behavior. At first this does not seem to be a likely source for an explanation, since nouns of many different semantic categories have unstable gender, including body parts, utensils, foods, and miscellaneous others. Recall that animals were deliberately omitted from this study. The reason is that if an animal \isi{noun} exhibits gender variability, this could be attributed to a functional use of gender corresponding to the animal’s sex. In fact, several animal nouns do show gender variability, as seen in \REF{ex:paster:14}.

\ea%14 
\label{ex:paster:14}
\parbox{5cm}{yahas-ti {\textasciitilde} yahas-ki}    ‘the crocodile’ \\ 
\parbox{5cm}{mayoonda-ði {\textasciitilde} mayoonda-ɣi}  ‘the (monkey sp.)’\\
\parbox{5cm}{d͡ʒeer-ti {\textasciitilde} d͡ʒeer-ki}   ‘the hippo’
\z

When these forms were produced, it is possible that the speaker had an animal of a particular sex in mind, even if this was not necessarily indicated in the \ili{English} translation, (e.g., even if the speaker did not specify ‘a \textit{male} crocodile’ when giving the form \textit{yahas-ki}).

Given this, the set of unstable nouns does not initially seem to form a semantic natural class. However, a closer look reveals a possible unifying property for several of the unstable nouns. It is instructive to compare the unstable nouns in \ili{Maay} with the so-called ‘double gender’ nouns in \ili{Dutch}, where Semplicini argues that “nouns whose referents are characterized by a high degree of individuation tend to trigger common agreement, while nouns with less individuated referents are more likely to trigger neuter agreement” (2012: 176). The notion of the “degree of individuation” may be relevant in \ili{Maay} as well, since most of the unstable nouns in \REF{ex:paster:10} are at least somewhat likely to occur frequently in the plural, and furthermore, some (especially paired body parts) probably occur much more often in the plural form than the singular. This observation about the unstable nouns is noteworthy because it links gender instability to the masculine-plural pattern in \ili{Maay}: because the masculine-plural pattern makes the gender of singular nouns unrecoverable from their plurals, nouns that are frequently plural may be susceptible to gender instability. 

Supposing that the gender of a \isi{noun}’s singular form is determined by analogy using the plural when the gender of the singular \isi{noun} is unknown, speakers will arrive at different singular forms depending on which words they choose to form the analogy. For example, when a speaker tries to determine the singular form of ‘hip’ from the plural form \textit{siim-o-ɣi} ‘the hips’, if he/she analogizes to \textit{sum-o-ɣi} ‘the belts’ as in (\ref{ex:paster:15}a), then the singular ‘hip’ will end up masculine. However, if the speaker instead analogizes to \textit{saam-o-ɣi} ‘the footprints’ as in (\ref{ex:paster:15}b), then the singular ‘hip’ will end up feminine.



\eabox[-.5\baselineskip]{\label{ex:paster:15}
Two possible analogies for recovering the gender of a \isi{noun} from its plural
\begin{tabbing}
a.~ \= XXXXXXXX \= XXXXXXXXX \= b.~  \= XXXXXXXX \= XXXXXXXXXX \kill \\
a. \>    sum-o-ɣi  \>  ‘the belts’ \>   \>   (m.) : suŋ-ki  \>  ‘the belt’ (m.) ::  \\
    \>    siim-o-ɣi  \>  ‘the hips’ \>  \>    (m.) : siiŋ-ki  \>  ‘the hip’  (m.)\\
b.   \>  saam-o-ɣi  \>  ‘the footprints’ \>  \>    (m.) : saan-ti  \>  ‘the footprint’ (f.) :: \\ 
    \>     siim-o-ɣi  \>  ‘the hips’ \>  \>    (m.) : siin-ti  \>  ‘the hip’  (f.)\\
\end{tabbing}
}

The analogy in (\ref{ex:paster:15}a) may overapply relative to (\ref{ex:paster:15}b) due to a preponderance of masculine nouns in the lexicon, since if a speaker selects a phonological neighbor to form the analogy, more often than not the neighbor will happen to be masculine. This would cause feminine nouns to shift to masculine more often than the reverse, although shifts could occur in either direction.\footnote{Note that my analysis does not hinge on the proposal that analogy is to phonological neighbors in particular. The analogy could instead be to semantic relatives, for example, and the same result would be predicted to emerge.}

\section{Conclusion}\label{sec:paster:6}

A number of predictions follow from the hypothesis presented above. Four predictions are enumerated in \REF{ex:paster:16}; I will discuss each in more detail below, showing that the available evidence is consistent with the predictions.

\ea
\label{ex:paster:16}
\ea  More unstable nouns should have feminine cognates in related languages than masculine cognates
\ex  More speakers will have the unstable nouns as feminine than masculine
\ex  Nouns that are frequently plural are most likely to change gender in languages with the masculine-plural pattern
\ex  Languages with the masculine-plural pattern are more likely to have gender instability
\z
\z

The first prediction, that unstable nouns should have feminine cognates in related languages, follows from the observation made earlier that, given the scenario I have outlined for how singular forms are recovered from their gender-ambigious plural counterparts, the overall masculine skewing of the lexicon will cause more feminine nouns to shift to masculine than the reverse. Thus, the instability will more often affect historically feminine nouns. A look at cognates in EOT does appear to uphold this prediction, although the sample size is small. For example, in Central \ili{Somali}, all of the cognates of the unstable nouns given by \citet{Saeed1987} are feminine (‘egg’, ‘eye’, ‘hand’, and ‘leaf’). None of the other nouns in the unstable class have Central \ili{Somali} cognates provided (‘knife’ is also feminine in Central \ili{Somali}, but it is not a cognate with the \ili{Maay} form). Similarly, in \ili{Tunni} \citep{Tosco1997}, ‘egg’, ‘eye’, ‘hand’, ‘knife’, ‘leaf’ are all feminine (no other cognates are provided). Given the skewing of stable \ili{Maay} nouns towards masculine, it is striking that all of the cognates that were found in Central \ili{Somali} and \ili{Tunni} for the unstable nouns are feminine. Thus, the prediction appears to be accurate, though a consideration of cognates in other EOT languages is warranted.\footnote{Note that masculine nouns can also become unstable through the same mechanism; they are just posited to be less likely to do so than feminine nouns. Note also that a \isi{noun} that is not more commonly attested in the plural form than in the singular can still become unstable if the speaker does not know its gender; in that case, rather than analogizing from a plural form, the speaker might just guess at the \isi{noun}’s gender. Thus, although I am suggesting that a feminine \isi{noun} that is frequently used in the plural and rarely in the singular is the most likely type of \isi{noun} to become unstable, other types of nouns may also become unstable.}

The second prediction, that more speakers will have the unstable nouns as feminine than masculine, also relates to the idea that the analogical recovery of singular gender will tend to shift previously feminine forms into the masculine category. If this is indeed the mechanism producing the instability, we expect to see a recurring pattern where most speakers have a particular \isi{noun} as feminine but one or more speakers innovates a masculine form. In that case we expect to see a recurring pattern where most speakers continue to treat a given \isi{noun} as feminine, while a smaller number of speakers treat it as masculine. Again the data do appear to uphold this prediction, though again the sample is small. For all but two of the unstable nouns (‘mango’ and ‘poison’), at least as many speakers have the \isi{noun} as feminine as masculine. And in several cases the feminines outnumber the masculines by a ratio of at least three to one. ‘Mango’ and ‘poison’, while they do not uphold the predicted trend themselves, are still not problematic since the proposed mechanism does allow masculine nouns to shift to feminine. A finding that these two nouns have masculine cognates in EOT would add further weight to the conclusion that this second prediction is upheld in the data.

The third prediction was that nouns that are frequently plural are most likely to change gender in languages with the masculine-plural pattern. This is a cross-linguistic prediction that could not be tested within the scope of this study. Within the EOT group, \ili{Tunni} is the other language that has a uniform masculine-plural pattern, so one might expect to find a similar pattern of \isi{noun} gender instability in that language, whereas languages with true gender polarity (like \ili{Rendille}) would not be as likely to have gender instability since the gender of the singular form is unambiguously recoverable (though reversed) from the plural. I am not aware of the existence of multi-speaker gender-marked \isi{noun} datasets for these or other relevant languages that would allow us to test this prediction cross-linguistically at present, but as I have argued here, it does seem to be true for \ili{Maay}.

The final prediction was that languages with the masculine-plural pattern are more likely to have gender instability than other languages are. As with the previous prediction, this is a cross-linguistic prediction that has yet to be tested, though I have argued that \ili{Maay} is an example of a language upholding this prediction. It is possible that the lack of the \ili{Maay}-type masculine-plural pattern elsewhere in Cushitic enables us to reconcile the gender instability in \ili{Maay} with Wolff’s assertion, cited at the beginning of this paper, that Afro-Asiatic in general exhibits gender stability, but this remains an empirical question to be tested by comparing \ili{Maay} and other languages that have the masculine-plural pattern with those that do not.

As a whole, then, the explanation I have proposed for the gender instability in \ili{Maay} nouns does find support within the language and tentatively within the EOT group. Its applicability outside of EOT and Cushitic in general remains to be tested.

A final observation is that a number of languages have genderless plurals, and this analysis of \ili{Maay} does predict that such languages should be susceptible to gender instability.\footnote{Thanks to an anonymous reviewer for raising this issue.} It is possible that further research will reveal that gender polarity has indeed developed in such languages; it is also conceivable that there are additional factors that have facilitated its development in \ili{Maay} that are not present in other languages. For example, because the speakers are refugees and live in a community where not everyone speaks \ili{Maay} and where several other languages are used, the language as a whole could be considered somewhat unstable (which is also consistent with the high degree of inter-speaker variability discussed in \citealt{Paster2013}). Perhaps the gender neutralization in the plural has combined with a generally unstable language situation to produce the phenomenon we observe in \ili{Maay}. Further research is needed to determine how widespread the phenomenon is and what determines when and where it emerges.

\section*{Acknowledgements}
Many thanks to the \ili{Somali} Bantu Community of San Diego, to my fall 2012 Field Methods class at Pomona College, and to Rodrigo Ranero and Rebekah Cramerus for participating in this research. Thanks also to the audience at ACAL 45 and to the anonymous reviewers of this paper for helpful comments. This research was funded in part by an Arnold L. and Lois S. Graves Award and by a Wig Teaching Innovation Grant from Pomona College.
 
\printbibliography[heading=subbibliography,notkeyword=this]

\end{document}