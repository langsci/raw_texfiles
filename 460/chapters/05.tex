\chapter{The linguistic resources of a ninth-century German \textit{scriptus} creator}\label{sec:chap:5}

\section{Introduction}
As I discussed in the previous chapter, certain changes associated with literization and ausbau are universal. This point follows from the fact that the communicative contexts of the languages of immediacy and distance are cross-linguistic, and the pressures they exert on spontaneously produced, intimate language on the one side of the pole and on planned, public language on the other side will shape all languages in similar ways. Thus, varieties shaped by immediacy, for example, will exhibit less integration and lexical density than those shaped by distance; they will be dialogic, more expressive, and less objective (see \sectref{sec:3:3.2}). Similarly, when speakers begin literizing their exclusively oral vernacular, they must elaborate its lexicon and grammar so that the new \textit{scriptus} can meet the communicative demands of a dislocated distance. None of the existing varieties of the vernacular, which have evolved in response to, and to accommodate, conceptual orality and literacy within \textit{an exclusively phonic medium}, will have all of the linguistic tools required by this new written communicative context. So, literizers must augment the emerging \textit{scriptus}’s lexical and grammatical coherence by developing the explicit means for expressing these relationships.

However, the early medieval German corpus, as I indicated in \chapref{sec:chap:2}, exhibits considerable linguistic variability from one text to the next. So, if the communicative pressures associated with the production of language in different contexts are universal, how can scholars explain linguistic variation? In the conclusion of \chapref{sec:chap:4}, I pointed toward an answer by noting that literization and ausbau are conscious processes that individual speakers undertake. Each literizer has their own set of multilectal and multilingual linguistic resources on which they may draw to create a \textit{scriptus}, and it makes sense that they would do so in unique ways. One can further expect that these choices will be guided by the writing project itself. That is, what are the literizer’s literary aims for their \textit{scriptus}? For example, the \textit{scriptus} created to familiarize the gospel story to a suspicious Saxon audience will differ from the one created to raise the status of a people and their language.

If structural differences are simply the result of the idiosyncratic decisions of individuals, this narrative still would not provide the scholar with the methodological means for analyzing variation. Early medieval German literizers are largely anonymous after all. Yet, analyzing the early German texts within the context of Carolingian culture can elucidate the ausbau choices that these linguistic innovators make. For example, literature on Carolingian history and documentary culture tells us about the education vernacular writers would have received in Latin. We also know which Latin grammars were particularly popular during this time. Training in a Latinate literacy and classical linguistic thought influenced the ways in which early literizers conceptualized their own vernacular and devised ways of adapting it so it could meet the new demands of a written language of distance. Additionally, we have the texts themselves, from which one may extrapolate the set of goals that motivated their creation. This analysis has direct bearing on the decisions the literizer made regarding the shape their \textit{scriptus} and its structures should take. Of particular interest in this chapter is whether the project itself inclines the author to lean more heavily on some of the structural patterns of their elaborated orality, that is, their spoken varieties of distance, or on the descriptive and/or prescriptive norms of Latin.

At first blush, this methodology seems to resemble \citegen[591--592]{KochOesterreicher1994} ideas of a \textit{fremdiniitierten} (‘extrinsic’) ausbau in that literizers ostensibly choose between extrinsic and intrinsic structures when creating a \textit{scriptus}. So, on the one hand, they may lean heavily on Latin structures and norms, or, on the other hand, they may eschew foreign influence in favor of a more authentically German ausbau. Koch and Oesterreicher themselves do not couch this discussion in the terminology of the deficit approach, but this framing of an extrinsic ausbau, which presumably stands in opposition to an intrinsic ausbau, should illustrate how easily the parameter can be co-opted into a deficit mind frame. The idea of extrinsic ausbau works best as a way of identifying languages whose literizations occurred through contact with another literate language. Literization as acculturation is much more frequently attested, the authors note. With respect to the history of European languages, for instance, Ancient Greek is one example of an intrinsic literization and ausbau, which later informed the literization and ausbau of Latin. It was then a Latin-Greek bilingualism that influenced the development of many of Europe’s literized languages (see, for example, \citealt[77]{Höder2010}). A better term would perhaps be “contact\hyp induced literization and ausbau.”

For the study of individual early German literizations, however, a more nuanced approach is required in which one considers how vernacular authors engaged all of their linguistic resources in order to create a new \textit{scriptus}. These resources include their literacy in Latin and Latinate education. That is, learning Latin provides early Carolingians with a conceptual framework for understanding what a written language is and how it can be constructed. This Latinate education relied on grammatical treatises that engaged with the discipline of \textit{grammatica}, which included descriptive and prescriptive treatments of Latin. Indeed, no clear distinction between description and prescription was maintained in classical linguistic thought. The resources of an early German \textit{scriptus} creator also include a whole constellation of oral vernacular varieties ranging from the more immediacy- to more distance-shaped, the latter of which connect to the oral tradition and the communicative constraints that shape their production. One might expect to find in the data a creative combining of, say, Latinate structures and norms, on the one hand, and vernacular forms, on the other hand. For example, a writer may choose to avoid double negatives in their \textit{scriptus} because classical rhetoric prescribes against it. Alternately, they could create a vernacular structure based on a Latin one. \citet[159--160]{Höder2010} argues that this is the case for Written Old Swedish, which gains a new set of monosemantic, polymorphemic subordinators as the result of a contact-based ausbau. According to Höder’s account, writers created grammatical calques out of Latin compound conjunctions like, \textit{sicut} (\textit{sic} + \textit{ut}, ‘just as’) and \textit{antequam} (\textit{ante} + \textit{quam}, ‘before, until’), yielding \textit{sva sum} and \textit{fyr än}, respectively. In my mind, it makes more sense to focus on the idea of the \textit{scriptus} as a unique creative engagement of different linguistic resources, rather than to couch the question in terms of extrinsic and intrinsic ausbau. Whatever the result of this process is, whether the author draws heavily on Latinate norms or not, the \textit{scriptus} is part of the history of a written German.

In this chapter, then, I propose that understanding the structural variability attested across early German \textit{scripti} from a sociolinguistic perspective necessitates understanding both the nature of the different linguistic resources a vernacular literizer would draw on to create their \textit{scriptus} and the individual literizer’s particular orientation toward those resources. Because my ultimate goal is to connect structural differences across early German texts to the different ways that a vernacular writer could build a \textit{scriptus}, it is important to establish reasons beyond the structures themselves why a writer might rely more or less on one type of linguistic resource than the other. Otherwise, the argument becomes circular quickly. In order to demonstrate how this analysis can work, I analyze the two originally composed ninth-century gospel harmonies in terms of their creators’ dispositions toward their projects. In the case of the \textit{Evangelienbuch}, the work contains direct and indirect evidence of Otfrid’s reliance on the Latinate tradition of literacy and the conceptual framework for approaching vernacular literization that his education in \textit{grammatica} provided him (\sectref{sec:5.2}). In contrast, the \textit{Hêliand} poem provides indirect evidence of its creator’s desire to build a \textit{scriptus} that would be more culturally familiar to German-speaking listeners. Thus, they drew more explicitly on their oral vernacular resources and the tradition of elaborated orality (\sectref{sec:5.3}).\footnote{\textrm{\citet{Somers2021b} is the starting point for this argument. What I present here, though it draws on many of that article’s examples, amounts to an updated and more specific theoretical and sociocultural understanding of what the author refers to as the poet’s orientation toward a “Latinate literacy” and the oral tradition. See the article’s page 34 for a summary of its author’s argument.} } As part of this section (\sectref{sec:5.3.1}), I also address~-- and present evidence to counter~-- \citegen{Haferland2010} argument that the \textit{Hêliand} is a transcribed, orally composed poem. If what Haferland claims is true, the \textit{Hêliand} text would not constitute a \textit{scriptus} at all, nor would it be the result of language ausbau, as I defined it in \chapref{sec:chap:4}. I begin this chapter, however, with a discussion of the Latinate tradition of literacy, in which I describe in greater detail the type of literacy education that was on offer in early medieval Francia (\sectref{sec:5.1}). This section should provide the reader with a better sense of how this discourse shaped the ways in which people thought about language generally and their own specifically.

\section{Training German-speaking clergy in the Latinate tradition of literacy}\label{sec:5.1}

Before engaging in the analysis of the two gospel harmonies and the extent to which their authors oriented themselves toward or away from the “Latinate tradition of literacy,” I would like to explain more fully what this term refers to and how it connects to another term the reader has already encountered in this book, \textit{grammatica}. This short section, then, introduces the reader to the clerical education that was available in the ninth century, when Otfrid and the \textit{Hêliand} poet were active, as well as in the early medieval period more generally. What I ultimately intend to show is that Latin instruction in Carolingian Europe exposed its students to a long classical tradition of scholarly engagement with language. This training, thus, gave vernacular literizers the vocabulary and conceptual framework to approach their own spoken language, that is, a metalanguage. It is perhaps not obvious to those who have grown up in literate communities how difficult it would be to conceptualize an exclusively oral vernacular, which exists only as sound, as a visual phenomenon without any sort of model of a written language to bootstrap, as it were, onto. As \citet[250]{Law1997} aptly phrases it, conceptualizing language is a reflexive endeavor in that “we use language to formulate our thoughts about language.” The study of Latin and \textit{gramamtica} provided a new means for German-speakers to formulate their thoughts about their vernacular, including thoughts on how it worked and what was good (and bad) about it. I return to this point in \chapref{sec:chap:7}, where I elaborate the argument that classical metalanguage shaped how Carolingians thought about their vernacular and, thus, how they literized it. As modern linguists whose own literate subjectivity is difficult to overcome, I propose that we must also consider how our metalanguage has determined the way we have analyzed historical varieties.

Research in different fields, including Carolingian documentary culture and the neuropsychology of literacy education can provide insight into what sort of education an aspiring monk would have received in Charlemagne’s empire, how effective that education would have been, and the types of cognitive changes it might have effected. As I discussed in \sectref{sec:3:3.1.1}, it was imperial policy that clergy be trained in Latin and \textit{grammatica}, along with other skills that made them better able to perform the liturgy and contribute to monastic life, for example, singing and computation \citep[19]{Brown1994}. Learning Latin was complicated by a number of factors.\footnote{\textrm{This discussion of factors that made the acquisition of Latin literacy challenging for Carolingian clergy mirrors to a certain extent that of \citet[88]{Law1994}.} } First, no one spoke Latin as a first language. Even those students who spoke Romance were not speakers of Latin, particularly the revived classical Latin that Charlemagne was interested in reestablishing, but rather of divergent “rustic” Latin varieties.

There was the added challenge of teaching students who had never encountered written forms of their own vernacular a different language that they would experience primarily in writing.\footnote{\textrm{See \citet{Barrau2011} for evidence that medieval monks generally could not, and if they could, did not,} \textrm{\textit{speak}} \textrm{Latin.} } Research on the difficulties associated with teaching illiterate adults a new literate language lays out the neuropsychological changes that occur as the result of literacy education. \citet[496]{Kotik-friedgut2014}, in its review of this research, notes that learning how to read changes the way people perceive written words, setting up “an association between sounds and graphic symbols-letters, synthesizing rows of these symbols into meaningful words and synthesizing groups of words into sentences that describe things and events." In their study, which presents a method for more effectively teaching illiterate Ethiopian adults Hebrew, \citet[498]{Kotik-friedgut2014} emphasized the development of students’ phonological awareness and visual perception so that they were better able to distinguish phonemes and connect sound to letters. They also relied on the learners’ first language to provide clarification and instruction. Education in monastic settings could begin when a student is quite young, so the modern circumstance of adult illiterates is not necessarily parallel to the medieval one. However, the research is relevant in what it reveals about how literacy education affects the brain and how it precisely hones those cognitive processes that make the acquisition of a new written language possible. The medieval student learning Latin at a cathedral church school, regardless of age, is faced with similar challenge as the Ethiopian immigrant learning Hebrew in Israel. How successful either is at acquiring literacy in their language depends on a number of factors, some within (e.g., degree of effort, motivation) and beyond (e.g., level of aptitude, degree of neuroplasticity) the individual’s control.

  The final challenge associated with teaching illiterate German-speaking learners how to read and write Latin was the availability of appropriate teaching materials. Carolingians inherited from the classical world a tradition of \textit{grammatica}, expressed in numerous treatises on the Latin language. \textit{Grammatica} refers to a whole discourse dealing with the descriptive and prescriptive norms of the Latin language. Already beginning in the fourth century BCE., Plato, Aristotle, and the Stoics began discussing and writing about language as an object of study. This work undergirds the Roman rhetorician Quintilian’s codification of the two tracks of grammatical study in his twelve-volume, \textit{Institutio} \textit{Oratoria} \citep[6]{Ciccolella2008}. \textit{Grammatica} entailed the study of the correct usage of written and spoken language as well as the “reading and expounding (\textit{enarratio}) of poets and other authors” \citep[2]{Matthews1994}. This definition is the basis for those grammatical treatises, particularly those written by Donatus and Priscian, that would prove especially popular among Carolingian thinkers. Classical grammarians sought to describe the sounds, letters, and parts of speech of Latin. They also wanted to comprehensively catalog the “faults and virtues of discourse” \citep[2]{Matthews1994}. These treatises on \textit{grammatica} were written for proficient speakers, rather than learners, of Latin (\citealt[88--89]{Law1994}, \citealt[8--9]{Matthews1994}). However, these were the texts that were available to Carolingians, and they made ample use of them, adding their own commentaries to sections that caught their interest. They used the simpler presentations of the basic parts of speech, like that found in Donatus’s fourth-century \textit{Ars Minor}, for language instruction (\citealt[503]{AurouxEtAl2000}), or Donatus’s more comprehensive \textit{Ars Maior}, which included brief treatments of different kinds of faults and figures of speech. Only the most advanced learners of Latin, people like Alcuin, engaged with more sophisticated treatments of Latin grammar, particularly Priscian’s multi-volume \textit{Institutiones Grammaticae} (\citealt[136--137]{Law1997}, \citealt[507]{AurouxEtAl2000}). \citet[145]{Luhtala1993} argues that Priscian’s more “advanced inquiry into linguistic issues” significantly shaped grammatical thought during the ninth century. \citet[299]{Barrau2011} argues that the term \textit{grammatica} underwent semantic change in medieval contexts because medieval intellectuals used the textual fruits of a classical engagement with Greek and Latin as foreign language instruction manuals.

There are several important takeaways from this discussion. First, learners of Latin in the early medieval, German-speaking setting faced an uphill battle. The task itself was a challenge, and students did not have access to anything close to optimal learning materials. This characterization is consistent with that of \citet[296]{Barrau2011}, which concludes that learning Latin was a long, difficult process and not every medieval monk was successful at it. This view also casts the educational regimen of the time in a new light. In brief, it proceeded as follows. Primary teaching would include learning to “memoriz[e] most of the Psalter both orally and in written form,” a task that monks could accomplish without knowing any “basics of Latin grammar” \citep[139]{Vineis1994}. Vineis’s description is interesting in that it calls to mind a child who does not yet know how to read, that is, has not formed “letter-sound connections to bond the spellings, pronunciations, and meanings of specific words in memory” \citep[5]{Ehri2014},\footnote{This is the definition of a process called “orthographic mapping,” which research on reading acquisition has shown is necessary to become a proficient reader. See \citet{Seidenberg2018} for a discussion of the topic directed at a general audience.} but has memorized their favorite book by noting the relationships between the visual and oral form of the word wholesale. It was only the middle and higher levels of monastic education that would expose students to the \textit{trivium}, which included explicit instruction in \textit{grammatica} and rhetoric~-- through exposure to some of the treatises mentioned just above~-- as well as dialectic (\citealt[139]{Vineis1994}, \citealt[37]{Brown1994}). Students would not reach a more advanced level of education, which included interpreting the Bible and exegesis, until they had achieved an “optimum knowledge of Latin” and could easily read the Latin-language texts to which their earlier education would have exposed them. \citegen{Barrau2011} study concludes that it is “doubtful” that all students who began this course of study would attain the literacy associated with these middle and higher levels of education or would be able to “read the Bible or the Fathers” (page 302).\largerpage

Those students who did reach an advanced state of Latin literacy, on the other hand, would have access to a treasure trove of literature on the subject of language. Indeed, their education would have relied on precisely this literature. These same texts would be immensely helpful for any vernacular literization project and likely shape any resultant \textit{scriptus} to some extent. It seems likely to me that anyone who undertook a project of German literization in the ninth century and whose texts are still extant, which is to say that they were preserved and copied within the system of Carolingian documentary culture, did so within the context, and with the support, of a monastic center and after receiving the sort of literacy education I just described.  Furthermore, I would also argue that these literizers were high achievers, that is, those who attained a high level of literacy, which suggests that they had a more than passing familiarity with the Latinate tradition of literacy, including \textit{grammatica}. Fortunately, in the case of Otfrid von Weissenburg and the \textit{Hêliand} poet, one need not assume. Their projects simply would have been impossible in the absence of this education. The fact that both poems are independently rendered gospel harmonies indicates that their authors could read and interpret the gospels, Tatian’s gospel harmony, and the extensive Latin-language Biblical commentaries written by doctrinal authorities. I discuss additional evidence of the poets’ facility with advanced Latin-language writing and \textit{grammatica} in the sections below.

  Another topic that I address as it becomes relevant to the \textit{scripti} under examination is the more specific content of these treatises on \textit{grammatica} that influenced the conceptualization and execution of German literization. This choice reflects my argument that different authors undertaking their own writing project will draw on this tradition in their own unique way and to an extent that is commensurate with the author’s goals for their project. As I discussed in this chapter’s introduction, the main axis I adopt for understanding structural variation is the degree to which \textit{scriptus}{}-creators orient themselves toward the Latinate tradition of literacy or away from it, drawing instead on the linguistic structures of their elaborated orality. My discussions of descriptive and prescriptive Latin norms detailed in the grammatical treatises focus on those that influenced German \textit{scripti}.

One aspect of these texts’ presentations of language is worth mentioning now, however, lest one think that my parameter for understanding \textit{scripti} variation is strictly one of a literacy-orality binary. In fact, the situation is more complex. Latin grammarians themselves still conceived of linguistic structures in oral terms and integrated sound structures into their conceptions of syntax. For example, in the seventeenth volume of his \textit{Institutiones Grammaticae}, Priscian sets out to describe the syntax of Latin, beginning with sounds, which combine into syllables; he then turns to syllables, which combine into words, and, finally, words combining into sentences \citep[27--31]{Priscian2010}.\footnote{Citations from Priscian are from Schönberger’s 2010 German-Latin bilingual edition of the grammarian’s volume on syntax: \citet{Priscian2010}. All translations into English are mine, unless indicated otherwise. The word that Priscian uses for “sentence” is \textit{oratio} (see page 30), a word that could mean ‘speech’ in both definite and indefinite senses, as well as \textit{Satz}, as Schönberger translates it here. The German word \textit{Satz} corresponds to English ‘sentence’ or ‘clause.’ Priscian’s use of the word refers to the notion of completeness; that is, an \textit{oratio} is a unit of speech that expresses a complete sense in an appropriate way. I return to this point in \chapref{sec:chap:7}.} Similarly, Donatus includes among the basic units of grammar sounds, syllables, and metrical units, i.e., feet \citep[15--25]{Donatus2009}. He also defines different sentential units in terms of speech, in addition to referencing meaning. So, the semantic definition of ‘sentence’ is clear in his use of the word \textit{sententia}, which also means ‘thought’ and ‘meaning’ (page 36--37). However, Donatus identifies breaks within the sentence, i.e., the complete thought, as pauses, what Schönberger translates as \textit{Atempause}. Donatus writes, “[a]n intermediate pause is available, when almost as much of the sentence is still to be said as has already been said, but it is necessary to take a breath” (page 38--39).\footnote{\textrm{Citations from Donatus’s} \textrm{\textit{Ars Maior}} \textrm{are all from Schönberger’s  German-Latin bilingual edition of the work \citep{Donatus2009}. Any translations into English are mine, unless indicated otherwise.} } Thus, the orality of language, how it is spoken, informs how the classical grammarians thought about linguistic structures, including the clause and the sentence. These ideas become particularly important in \chapref{sec:chap:6}, where I discuss clauses and clause complexes.

\section{Vernacular literization within a Latinate framework: Otfrid’s \textit{Ad Liutbertum}}\label{sec:5.2}

The main goals of this section are as follows. First and foremost, I intend to demonstrate that Otfrid von Weissenburg approached the creation of his \textit{scriptus} within the conceptual framework he gained through his education in Latin and Latinate literacy. The main evidence supporting this argument is Otfrid’s own words, not just his epistolary preface, the \textit{Ad Liutbertum}, which constitutes a rare metalinguistic commentary on the task of vernacular literization from the period, but also how he justifies the project in the work’s introductory chapter. From this writing, one may extrapolate what the monk’s goals were for his project and its \textit{scriptus}: namely, to develop a proper or correct written form of Frankish. To that end, Otfrid created a set of prescriptive norms based on the notion of metrical, or rhythmic, discipline. However, Otfrid also indicates his desire to create good Frankish that still adhered to what he perceived were the language’s descriptive norms. In this way, the \textit{Evangelienbuch} provides insight into how Otfrid reconciled what he knew his contemporaries would view as the inherent incompatibility of these two goals, given the ambivalent attitudes Carolingians generally harbored toward their vernacular (see \sectref{sec:3:3.1.2}). Its \textit{scriptus} is the result of a creative interaction between the descriptive norms of the vernacular and a Latinate notion of linguistic prescription.

\subsection{How do you solve a problem like Frankish? }\label{sec:5.2.1}
\begin{sloppypar}
In the \textit{Ad Liutbertum}, Otfrid identifies two goals for his vernacular writing project, both of which orient him away from his own vernacular oral tradition. The first goal is didactic, as expressed in the following passage, which is also highlighted in \citet[37]{Somers2021b}.
\end{sloppypar}

\begin{quote}
[ . . . ] I have written down a selection from the Gospels, composed by me in Frankish [ . . . ] so that he who shudders at the difficulty of a foreign language [ . . . ] may here in his own language become familiar with the most holy words and, understanding in his own language the Law of God, may, therefore, guard well against straying from it by even a little through his own erroneous thinking.  (\textit{Ad Liutbertum}, \citealt[875--876]{Magoun1943}\footnote{\textrm{All English translations of Otfrid’s} \textrm{\textit{Ad Liutbertum}} \textrm{are from \citet{Magoun1943}, unless indicated otherwise.} })
\end{quote}

\noindent Otfrid alludes to a feeling of urgency surrounding the project in that it was prompted by numerous requests from other Christian Franks in his circle, who felt under some degree of threat from their sonic environment.

\begin{quote}
When formerly the noise of worldly futilities smote on the ears of certain men exceedingly well-tried in God's service and the offensive song of laymen disturbed their holy way of life, I was asked by certain monastic brethren worthy of consideration and especially moved by the words of a certain reverend lady, Judith by name, who urged me very often that I should compose for them in German (i.e., Frankish) a selection of the Gospels, so that a little of the text of this poem might neutralize the trivial merriment of worldly voices and that, engrossed in the sweet charm of the Gospels in their own language, they might be able to avoid the noise of futile things. (\textit{Ad Liutbertum}, \citealt[873]{Magoun1943})
\end{quote}

\noindent It is notable how Otfrid especially invokes his oral world in this passage, placing in opposition to one another the “sweet” sound of a Frankish-language retelling of the gospels, on the one hand, and the “offensive song of laymen,” on the other hand. His \textit{Evangelienbuch} should help drown out “the trivial merriment of worldly voices” making it easier for pious Franks to remain focused on what mattered: the salvation of their soul through the word of God. In seeing his project as an opposing force that will counteract the “noise” of the non-Christian oral tradition, it makes sense to hypothesize that Otfrid chose to avoid language that listeners might have associated too strongly with its stories and songs. Details like the fact that Otfrid opts for an innovative Latin-inspired form of verse over the traditional alliterative verse supports this hypothesis \citep[35--38]{Somers2021b}.

Otfrid’s second more ambitious goal for the project, I argue, will also lead the author to eschew linguistic markers of the oral tradition. This goal has to do with the prestige of the Frankish empire vis-à-vis other great civilizations throughout history. Consider the following passage, also highlighted in \citet[37]{Somers2021b}.

\begin{quote}
Indeed, the Franks do not, as do many other peoples, commit the stories of their predecessors to written record nor do they adorn in literary style the deeds or the life of these out of appreciation for their distinction. But if on rare occasions it does happen, by preference they set forth in the language of other peoples, that is, of the Romans or the Greeks. [ . . . ] A remarkable thing it is, however, that great men, constant in good judgment, distinguished for careful attention, supported by nimbleness of wit, broad in wisdom, famed for sanctity, should carry over all these virtues into the glory of a foreign language and not have the habit of composition in their native language. (\textit{Ad Liutbertum}, translation from \citealt[886--887]{Magoun1943})
\end{quote}

\noindent Otfrid more explicitly compares Frankish culture to the classical cultures of the Romans and the Greeks in the introductory book to the work, which is entitled, ‘Why the author composed this work in the vernacular’ (\textit{cur scriptor hunc librum theotisce dictaverit}). Otfrid begins this chapter with the following observation \REF{ex:5:1}.

\ea%1
    \label{ex:5:1}
\gll Was líuto      filu    in  flíze,          in  managemo   ágaleize sie   thaz  in  scríp  gicleiptin       thaz   sie     iro      námon breittin\\
was people.\textsc{gen.pl}  many   in   eagerness  in   much     striving they   that  in  writing  secure.\textsc{pret.subj}  that   they  their  names glorify.\textsc{pret.subj}\\

\gll Sie   thés     in       io    gilícho  flizzun       gúallicho in  búachon    man   giméinti            thio   iro    chúanheiti\\
they   this.\textsc{gen}  them.\textsc{refl}  always  likewise  strive.\textsc{pres.subj}  eagerly in   books     one  proclaim.\textsc{pret.subj}  \textsc{det}   their  bold.deeds\\

\gll Tharána      dátun  sie    ouh  thaz   dúam          óugdun      iro  wísduam óugdun       iro    cléini           in  thes    tíhtonnes       reini\\
through.this   did   they    also   \textsc{det}   glorious.deed  demonstrated   their wisdom demonstrated  their  feeling.for.art  in   \textsc{det}    composing.\textsc{gen.sg}  purity\\

\glt ‘Many peoples strove eagerly and strenuously so that they might fix that in writing, with which/so that they may glorify their names. Likewise, for this they always strove eagerly (that) one may proclaim in books their bold deeds. Through this, they also performed a glorious deed (and) demonstrated their wisdom (and) demonstrated their feeling for art in the purity of their composing.’ (I 1, 1--6)
    \z

\noindent Who these groups of people are becomes clear in lines 13--16, where Otfrid states how successfully the Greeks and Romans cultivated the written word and, thus, solidified their own legacies. So why, Otfrid asks, have the Franks not developed their own traditions of writing in the vernacular: “Why should the Franks be the only ones who refrain from singing God’s praises in Frankish?”\footnote{\textrm{Wánana sculun Fránkon éinon thaz biwánkon ni sie in frénkisgon bigínnen sie gotes lób singen (lines 33--34)}} They are equally capable of such an achievement, argues the poet. The implied conclusion to this line of argumentation is that, if the Franks wish to assume their rightful position in history as the leaders of a great empire, they must secure their own legacy, not just in writing, but in written Frankish.

Thus, the \textit{Evangelienbuch} constitutes the monk’s proof of concept: Otfrid argues in his preface and introductory book that Franks should write in Frankish and then presents a poem that comprises many thousands of lines of written Frankish.\footnote{\textrm{Otfrid thereby avoids the irony of Martin Opitz’s obituary on the Latin language delivered in the fall of 1617, in Latin \citep[142]{Krebs2011}.} } \citet[31; 37--38]{Somers2021b} argues that Otfrid was not simply interested in writing a work of significant length in Frankish, but doing so in good Frankish, an ambition that necessitated the monk create a \textit{scriptus} that could qualify as such. That Otfrid believed that Frankish would not be up to the task without concerted shaping and amendment is clear in how he characterizes his vernacular. In the \textit{Ad Liutbertum}, for example, he calls Frankish “rude […] unpolished and unruly and unused to being restrained by the regulating curb of the art of grammar” (see \citegen[880]{Magoun1943} translation of lines 58--59). Later in the preface, he continues: “This (Frankish) language is, indeed, regarded as rustic because it has at no time been polished up by the natives either by writing or by any grammatical art” (\citealt[886]{Magoun1943}). In the introductory chapter, Otfrid strikes a more hopeful tone \REF{ex:5:2}.

\ea%2
    \label{ex:5:2}
\gll Níst   si   so  gisúngan       mit    régulu    bithuúngan\\
  \textsc{neg}{}-ist  it  so  sung    with  regulation  curbed\\

\gll si  hábet  thoh  thia  ríhti        in  scóneru    slíhti\\
it   has     yet    the  rectitude   in  beautiful  simplicity\\

\glt ‘It has not been sung in such a way as to be shaped by regularity; yet it has the rectitude of beautiful simplicity.’ (I 1, 35--36)
  \z

\noindent This statement constitutes a politic reframing of his earlier characterization of Frankish as “rustic” and “rude” and furthermore is more consistent with the simple fact that Otfrid opted to write an entire poem in this language. Clearly, Otfrid thought Frankish had the potential to be something more than a spoken vernacular.

\subsection{A metrical prescription }\label{sec:5.2.2}
\begin{sloppypar}
In \citet[44--46]{Somers2021b}, I conclude that the prescriptive ideal that Otfrid chooses is a metrical one, highlighting Otfrid’s introductory discussion of how, in his estimation, the Greeks and Romans effected “purity and refinement in their writing” through “strict adherence to a meter” (I 1:21--28; from \citealt[45]{Somers2021b}). Similarly, Otfrid creates for his poem a new, Latin-inspired metrical scheme featuring a pattern of alternating stresses and dips, designed to regulate his unruly vernacular. \citet[46]{Somers2021b} references the following passage in support of her argument, drawn again from the work’s introductory chapter (lines 41--48).
\end{sloppypar}

\begin{quote}
Let God’s law be sweet unto you, \textit{then feet, tempo, and rules, will also shape it [Frankish]}; indeed, those are the words of God himself. If you have in mind to adhere to the meter, create prestige in your language and create beautiful verse, strive to fulfill God’s will all the time; then the servants of the Lord will write in Frankish \textit{in accordance with the rules}; \textit{let your feet proceed} in the sweetness of God’s law; \textit{don’t let time escape you}: then beautiful verse is created forthwith. [emphasis in the original]
\end{quote}

\noindent Thus, in the absence of any established norms for the writing of Frankish, Otfrid opts for a metrical prescription. \citet[46]{Somers2021b} notes the double meaning of “feet” and “time,” words that also reference poetic structures: in order to create beautiful, more elevated Frankish, the poet must keep time and regulate their feet, as it were, in order to maintain Otfrid’s self-fashioned alternating stress-dip pattern. Otfrid also applies end-rhyme to the lines of his poem, its first known use in German. This poetic device and the metrical scheme can effect distortions in the prosodic patterns of the vernacular \citep[44--45]{Somers2021b}. That is, the work’s poetic scheme moves the listener or reader away from the familiar prosodic patterns of Frankish, a strategy that reflects the monk’s goal of elevating Frankish beyond its existing oral varieties. Thus, any Frank who heard the \textit{Evangelienbuch} read aloud would have been aware of how different its language \textit{sounded} from any of their own spoken varieties or immediacy or distance.

I see this argument at the center of \citet{Somers2021b} as important for the current work in that it points to two additional conclusions regarding vernacular literization and, especially, the development of an exclusively or mostly oral vernacular’s capacity to function in fully dislocated, written contexts. First, it indicates that Otfrid consciously engaged in the literization of Frankish. That is, he understood his project as the creation of a \textit{scriptus} that would be appropriate to written distance contexts, in which the vernacular had never before had to function and, indeed, could not have functioned well without ausbau. Though his strategy for creating good, written Frankish might seem odd to literate speakers of German and English, whose prescriptive standards are not based on regularizing meters, Otfrid’s choice is not so strange considering, for instance, how his metrical prescription had the potential to influence, or distort, vernacular prosody, thereby immediately catching a listener’s attention with its crafted language. In \chapref{sec:chap:6}, I also discuss the important functional role that poetic language plays in exclusively oral communities, noting that planned, public, formal varieties of an oral vernacular are crucially poetic. Poetic language was “special” language,\footnote{\textrm{I paraphrase here \citegen{Bakker1997} term “special speech,” which he applies to Homeric poetry (page 17).} } and so Otfrid’s choice to write in verse makes sense. The second conclusion that the ideas in \citet[42]{Somers2021b} point to is that Otfrid conceived of vernacular literization in terms of the descriptive and prescriptive norms of Latin. The author discusses, for example, how Otfrid built his unique poetic scheme on Latin models. Drawing on \citet[208--209]{Bostock1976}, she notes that Otfrid’s likeliest sources for his poetic scheme are the Ambrosian hymns, which feature alternating stresses and dips, and the Latin hymns of Hrabanus Maurus, which exhibited end-rhyme. Thus, Otfrid conceives of his prescription for Frankish in terms of Latin and, in implementing it, moves the vernacular away from its more familiar, colloquial prosodic patterns.

Further examination of the \textit{Ad Liutbertum} reveals additional evidence that the monk understood his task of literizing Frankish in terms of Latin and his Latinate education. For example, he writes of his difficulties spelling Frankish words in ways that were consistent with their pronunciation but did not run afoul of prescriptive and sometimes descriptive classical norms (lines 58--70). For example, he struggled to find the appropriate letter for certain reduced vowel sounds in Frankish and also found no alternatives to using \textit{k} and \textit{z} letters, despite Latin grammarians having declared them “superfluous” \citep[880--881]{Magoun1943}. That is, Frankish phonemes did not map perfectly onto Latin letters, and Otfrid found this fact problematic enough to warrant an explicit mention in the preface. Otfrid struggled similarly with grammatical conflicts between Frankish and Latin. For example, he acknowledges that Latin prescribes against double negatives because they ostensibly equal a positive. In Frankish, however, they “almost invariably negate,” and, so, he includes them because he took “pains to write as customary speech has showed itself to be” (see Magoun’s translation of lines 93--96, \citealt[885]{Magoun1943}). In the next sentence, Otfrid laments the fact that he could not avoid committing “barbarisms” and “solecisms,” two types of errors identified in classical rhetoric, by using Frankish genders and numbers that were inconsistent with those of Latin (lines 96--101). That this fact pains Otfrid points again to his desire to create and write in good Frankish, which, in his opinion, should transgress classical prescriptions as little as possible.

  These lines also demonstrate Otfrid’s awareness of contemporaries’ ambivalent feelings toward his vernacular, including, perhaps, his own. On the one hand, he routinely opts for structures that are consistent with “customary speech.” On the other hand, he feels the need to defend these decisions because they conflict with Latin patterns and prescriptions. His defensiveness becomes even clearer in lines 101--105, where he writes:

\begin{quote}
I might put down from this book examples in German of all these faults noted above, except that I would avoid the derision of readers; for when the unpolished words of a rustic language are sown in the smooth ground of Latin, they give occasion for loud laughter to readers. (\citealt[886]{Magoun1943})
\end{quote}

\noindent Thus, all of the following statements are simultaneously true for Otfrid. First, he is concerned about the rusticity of the vernacular and would like to offer correctives to its unacceptable state. Next, Frankish’s rusticity appears to stem in large part from the fact that it is not Latin, something that might strike the modern reader as more of a statement of fact than a defect on Frankish’s part. So, it might seem strange that the man who wants to elevate Frankish might also see its “Frankishness” or, perhaps more accurately, its “non-Latinness” as a problem. Yet, when faced with Frankish structures that directly conflict with Latin descriptive and prescriptive norms, Otfrid chooses the former because they reflected common linguistic usage.

The apparent contradiction in these statements disappears when one treats them as a commentary on the relative literization of Frankish and Latin, the latter of which constitutes a thoroughly literized ausbau language and the former falling well short of that ideal. These are the terms within which Otfrid saw his literization process: his job was to take the rustic Frankish vernacular and polish it up, as one might a rough stone into a jewel. He approaches this task in a similar way to how Latin speakers drew on the writings of Greek grammarians when literizing their own language. Consider as an example the late Roman grammarian, Priscian, who wrote the multi-volume \textit{Institutiones Grammaticae}. This work becomes one of the most influential grammars in early medieval Carolingian Europe, along with Donatus’s \textit{Ars Maior} and \textit{Ars Minor} \citep[504]{AurouxEtAl2000}. Priscian’s presentation of Latin grammar relies entirely on the parts of speech that are systematically defined in Greek grammars, particularly the works of the renowned grammarian, Apollonius Dyscolus.\footnote{\textrm{As discussed by Schönberg in his comments on the Priscian text and his translation of it \citep[487]{Priscian2010}.} } Thus, Priscian includes a section on articles in his own grammar of Latin, along with the commentary that, while Greek has articles, Latin does not \citep[99]{Priscian2010}. That is, Priscian’s description of Latin occurs within the framework of the categories identified by grammarians like Dyscolus for Greek. In line with this approach to grammatical discourse, Otfrid describes Frankish in relation to Latin and notes when forms do not align. That the former deviates so significantly from the latter only emphasizes Frankish’s lack of literization. However, in searching for solutions to these misalignments, Otfrid does not abandon his own grammatical intuitions and observations on how Frankish speakers around him used the vernacular.\largerpage

Ultimately, what Otfrid describes in his preface and introduction is his own conscious engagement with literization and ausbau. The monk’s poem is the product of a delicate balancing of several factors: his linguistic intuitions as a Frankish speaker, his divergent attitudes toward Frankish and Latin, his Latinate education, which represented the state of grammatical thought in Carolingian Europe, and, finally, his own ambitions for the project. This constellation of interconnected features also exposes the limitations of \citegen{KochOesterreicher1994} conception of an extrinsic versus an intrinsic ausbau, which does little to elucidate this complex situation and instead encourages more reductive analyses. Instead, one must approach the creation of each \textit{scriptus} as an occasion for the literizer to marshal their various linguistic resources in the service of a particular writing project, for which the literizer has their own particular goals. In the case of Otfrid, his framework for understanding Frankish’s unliterized state and conceiving of strategies for addressing the problem is Latinate. In that sense, it might be appropriate to refer to his \textit{scriptus} as resulting from extrinsic ausbau, as Koch and Oesterreicher define it. However, Otfrid also wanted to create a \textit{scriptus} that was distinctly Frankish, even though Latin was the framework for that assessment, and it was accompanied by the judgment that Frankish deviations from Latin were proof of its rusticity. These are all relevant factors for understanding the way he builds his \textit{scriptus}.

\section{Vernacular literization with reference to the oral tradition: The \textit{Hêliand}}\label{sec:5.3}

The \textit{Hêliand} provides an instructive counterpoint to the \textit{Evangelienbuch} in that the \textit{Hêliand} poet aligns their work with the vernacular oral tradition, while Otfrid intentionally moves his poem away from it. \citegen{KochOesterreicher1994} notion of extrinsic versus intrinsic ausbau might incline one to see the \textit{Hêliand}’s \textit{scriptus} as resulting entirely from the former. However, it is important to recognize that both gospel harmonies were produced in the same Carolingian cultural environment in which literacy was a Latinate phenomenon first and foremost. That is, one should not discount the possibility that the \textit{Hêliand} poet had certain descriptive and prescriptive norms of Latin in mind when constructing their \textit{scriptus}. Furthermore, a Latin influence does not constitute evidence that the \textit{Hêliand} is a non-autochthonous or inauthentic text, whose data scholars must view with a suspicious eye. It simply means that the \textit{Hêliand} poet, like any other aspiring vernacular writer in ninth-century German-speaking Francia, would have been educated in Latin and, thus, had access to this literary tradition when undertaking vernacular literization. It also means that a metalanguage stemming from classical linguistic thought likely shaped how the poet conceptualized literization and ausbau.

In the section that follows, I argue that the \textit{Hêliand} poet engaged their linguistic resources differently from how Otfrid did. The latter poet’s interest is in creating good, written Frankish, and so he orients himself away from the oral tradition. The former poet, in contrast, leans into the oral tradition by Germanicizing certain aspects of the story and composing the poem in traditional alliterative verse. Yet the Latinate tradition of literacy, as \citet{Somers2021b} calls it, informs the work generally. For example, the \textit{Hêliand} poet draws on Latin-language sources and the Latinate tradition of biblical commentaries when crafting their gospel narrative. I begin this section with a discussion of these Latinate influences and then discuss how the \textit{Hêliand} poet draws on vernacular oral tradition to create a gospel harmony that is stylistically distinct from the \textit{Evangelienbuch}.

Though the \textit{Hêliand} was originally composed in the vernacular, the sources that informed its composition were Latin. \citet[178--179]{Bostock1976} indicates that the poet or their clerical advisor (more on this possibility below) made use of a Latin version of Tatian’s gospel harmony, rather than drawing independently on the Scriptures. This idea is echoed in \citet[168--169]{Haferland2010}, who concludes that the poet likely had a Latin manuscript of the Diatessaron in front of them while composing because the section divisions of the \textit{Hêliand} match up with those of Tatian’s work. The poet omits some sections and rearranges others. Some receive further elaboration. \citet[174]{Bostock1976}, for example, notes how the poet expands the eleven biblical verses describing the wedding feast at Cana to more than eighty lines depicting a “grand carousal in the no doubt contemporary Saxon style.” Other expansions, however, are in line with Latin-language commentaries, including Rhabanus’s Commentary on St. Matthew, the Commentaries of Bede on St. Luke, and, to a lesser extent, Alcuin’s Commentaries on St. John \citep[178--179]{Bostock1976}. \citet[169]{Haferland2010} explains that one can identify clear instances of correspondence in which the \textit{Hêliand} and the commentaries share stylistic features, like sentence boundaries, though the poet’s own “method and style” remain apparent. In sum, the composition of this vernacular poem, though original to the poet, was built on Latin renderings of the gospels and the Latin-language exegetical tradition that was so central to Carolingian literacy and learning. The Germanicization of the poem’s form and narrative details, thus, occurs within a Latinate framework. It is a conscious cultural transfer of the story of Jesus from its canonical setting and framing into a vernacular context.

Turning now to the Germanicized narrative details, the \textit{Hêliand} poet shifts the setting of the gospel away from its original geographical and cultural setting of the Biblical Middle East to one that would feel more familiar to German-speaking listeners. Well-known examples\footnote{\textrm{These examples are drawn from \citet[169--174]{Bostock1976} and are also cited in \citet[35]{Somers2021b}.} } include  the shepherds of the Bible becoming horse herds in the \textit{Hêliand} (line 388). Similarly, the Star of Bethlehem leads the Magi through forests, rather than deserts (line 603). The argument that the \textit{Hêliand} poet is “play[ing] to Germanic tastes,” as Somers puts it (page 35), is strengthened by the poet’s depiction of Jesus. Whereas the gospels emphasize that Jesus was meek and self-effacing (see, for example, Matthew 11:29), the \textit{Hêliand} highlights his courage, wisdom and might. For example, the Son of God is referred to “the Lord of the peoples” [\textit{thiodo drohtin} (e.g., 2950); \textit{managoro drohtin} (e.g., 1999)]; “the guardian of the Land” [\textit{landes uuard} (e.g., 626)]; “the mighty protector of many” [\textit{mahtig mundboro} (e.g., 1544)]; “a wise king, renowned and powerful [and] of the finest lineage,” [\textit{ein uuiscuning, mari endi mahti [ . . . ] thes bezton giburdies} (582–84)]; and  “courageous and powerful” [\textit{bald endi strang} (e.g., 999)] \citep[35; 47]{Somers2021b}. In sum, the \textit{Hêliand} poet reshaped the gospel narrative significantly, though, it should also be noted, never in ways that ran afoul of what was considered orthodoxy at the time. This Germanicization of narrative details is why scholars have seen the \textit{Hêliand} as a literary expression of Carolingian expansion into Saxony, a campaign that involved military action as well as (sometimes forced) religious conversion \citep[35]{Somers2021b}. This view turns the \textit{Hêliand} into a political text that would have been designed to encourage the transformation of recalcitrant Saxons into obedient Christian subjects of the Frankish Empire.

The Germanicization of the poem extends to its form, which is that of alliterative verse. \citet[170--174]{Haferland2010} argues that the work provides considerable evidence of the poet~-- or a team of people that included a poet and a clerical advisor~-- having been well versed in what he calls “the sociolect of the singers” (page 173). He notes, for example, how the poem draws from a broad inventory of poetic formulas shared with alliterative poetry in other Germanic traditions, including Old English and Old Norse. Drawing on the language of the oral tradition and its prosodic patterns represents a harmonization of form and content, both of which would render the gospel story more familiar to German-speaking listeners. This tack contrasts significantly with Otfrid, whose \textit{Evangelienbuch} does nothing to accommodate Frankish tastes and, indeed, uses a metrical pattern that effects prosodic distinctiveness \citep{Somers2021b}.

Unfortunately, nothing like Otfrid’s metalinguistic commentary on the why and how of literization exists for the \textit{Hêliand}. The goals of the text must be extrapolated from the poem itself and how the poet approached their subject matter. Consideration of the two short texts written in Carolingian Latin that many scholars have taken to be prefaces to the \textit{Hêliand} (see \citealt[181--182]{Bostock1976}), the prose \textit{Prefatio in librum antiquam Saxonica lingua conscriptum} and the poetic \textit{Versus de poeta et interprete huius codicis}, do not elucidate much beyond these extrapolations. In the prose text, for example, the unknown author states that “Ludiuuicus piissimus Augustus,” usually understood to refer to the Emperor Louis the Pious, “ordered a certain man of the Saxon peoples, who had among themselves poets of no small worth, that he work to poetically translate the Old and New Testaments into the German language, such that the sacred reading of the divine precepts might be open to not only the literate but also the illiterate” (translation from \citealt{Carlton2019}).\footnote{Praecepit [Ludouuicus] ... cuidam viro de gente Saxonum, qui apud suos non ignobilis vates habebatur, ut vetus ac novum Testamentum in Germanicum linguam poetice transferre studeret, quatenus non solum literatis, verum etiam illiteratis sacra divinorum praeceptorum lectio panderetur \citep[3]{Sievers1935}.} This description supports the conclusion that the text’s Germanicized details and use of alliterative verse already point to, namely, that this gospel harmony was intended as an aid to Christianization efforts in Saxony. \citet[183]{Bostock1976} cautions that the link between these texts and the \textit{Hêliand} is by no means conclusive, noting that while some of the details mentioned in the prefaces match the German-language text, others do not. He concludes that the prefaces “merely show that there existed in the ninth century a tradition or belief that, at the command of the Emperor Louis, a Saxon had produced poems dealing with the Old and New Testaments.”

\section{Was the \textit{Hêliand} poet illiterate \citep{Haferland2010}?}\label{sec:5.3.1}\largerpage

The question of whether one poet composed the \textit{Hêliand} or it was the product of a collective effort is irrelevant to the argument that the poem is oriented toward the vernacular tradition of elaborated orality. That is, regardless of whether the poet was one person or several people, the pervasive influence of the oral tradition is indisputable. The question of authorship, however, is relevant to the idea that the work’s \textit{scriptus} is the product of ausbau in the first place. Indeed, \citet{Haferland2010} presents a more serious challenge to my argument that the creation of early medieval \textit{scripti} required innovative linguistic ausbau to enhance the lexical and grammatical coherence beyond what existing vernacular linguistic intuitions provided for. This work argues that the \textit{Hêliand} was composed orally by an illiterate poet with the aid of a clerical advisor. If this view is correct, then the poem would be a \textit{Verschriftung}, rather than a \textit{Verschriftlichung}, that is, a transcription of a spoken variety~-- in this case, likely one of distance, as it would represent a song from the oral tradition~-- and not a conscious literization resulting from ausbau. Additionally, the poem would constitute a more limited “pragmatic ausbau” of German, a term that refers to the gradual recognition of the fact that texts can be produced and received entirely within the graphic realm, with no reference made to conceptual orality or the oral culture (\citealt[590]{KochOesterreicher1994}). In classical and medieval cultures, as \citet[15--16]{Green1994} notes, the written word remained connected to their still vibrant oral cultures. Texts were treated as “written storage” (\textit{schriftliche Speicherung}) and were secondary to, and produced in support of, a spoken utterance.\footnote{\citegen[15--16]{Green1994} description of how medieval correspondence worked is a useful example of this phenomenon (drawing from \citealt[340]{Köhn1986}). First, the letter-sender dictated their missive aloud to a scribe. Whoever delivered the letter to the addressee would also have conveyed an oral communication, which, in fact, could be more important than the contents of the letter itself. The delivered letter was then read aloud to the recipient.} If the \textit{Hêliand} had been orally composed and then transcribed, the written word in this case would be mere storage, a state-of-affairs that is inconsistent with ausbau in that the \textit{scriptus} would be linked to a conceptual orality and, thus, constrained by the cognitive demands of composing orally and spontaneously. Ultimately, I am unconvinced by Haferland’s characterization of the \textit{Hêliand} as an orally composed and transcribed work. I devote the remainder of this section to describing Haferland’s arguments and laying out a case against them.

\citegen{Haferland2010} argument that the poet must have been illiterate rests on one central assumption: that it is impossible that the two “distinct talents” required to produce the \textit{Hêliand} resided in one person and that these competencies must have been spread out across a team that minimally included one ecclesiastical advisor and one illiterate singer (pages 201, 203). The clerical advisor, on the one hand, was trained in the Latinate tradition of literacy and could marshal all the biblical sources required for the project. They provided the material to the illiterate poet, who, on the other hand, had “total command of the art and lexicon of Germanic singers.” Then the singer “refashioned” the material provided from memory (page 177) and scribes transcribed what they heard (page 186). Haferland imagines the actions of the illiterate poet in the model of the legend of Caedmon from Bede’s \textit{Historia Ecclesiastica}. The poet would listen to one discrete section of the gospels told to him by the clerical advisor, ruminate over it, perhaps in the evenings “to find the proper wording,” and perform that part of the poem the next morning spontaneously and from memory (pages 201--202).

Imagining the poem as the work of one poet points to one of only two possibilities, both of which cannot be true, according to \citet{Haferland2010}. One possibility is that it was written by a member of the clergy well-trained in a Latinate literacy but only capable of “feigning” orality because they devoted themselves to the church rather than the oral tradition. Alternatively, a later convert, who underwent the long training to become a singer and only afterwards joined a monastery, wrote the poem. Neither possibility can be true, says Haferland. In the case of the latter, the singer-turned-clergy could not have been able to read Latin fluently (page 203) and, thus, would not have been able to access all of the sources required to compose the poem. In the case of the former, the orality of the poem is too “constitutive” to be feigned (page 176). Additionally, the clergy-turned-poet would not have made the errors of interpretation that can be found in the parts of the text (pages 192--200). It is these “involuntary reflexes of orality” (page 204) and errors that lead Haferland to conclude that an illiterate poet must have composed the poem orally with the help of a clerical advisor and that a writer or team of writers transcribed the work.

In order to address Haferland’s claims, I begin by noting that the author rests his arguments on two false dilemmas. The first is that a person living in ninth-century Francia could not be educated in Latin while also having some degree of command over more planned, public oral varieties that reside on the right end of the “language of immediacy–language of distance” continuum (see \chapref{sec:chap:3} for discussion of Koch and Oesterreicher’s model of linguistic output). The \textit{Hêliand} poet’s (or poets’) world is, as I discussed in \chapref{sec:chap:3}, still overwhelmingly oral. \citegen{Coseriu1974} principle of historicity reminds the scholar that people in the ninth century were multilectal, just as people are today, and that they, like us, produced immediacy and distance varieties of their spoken language. This is not to say that everyone could spontaneously produce long, epic songs without special training. But it seems to me that Haferland treats the sociolect of the singers as something wholly apart from ninth-century vernaculars, as opposed to one special variety embedded at the extreme end of a whole continuum of spoken varieties of distance that were accessible to many speakers, not simply those who underwent specific training as a singer. I see no good reason to accept Haferland’s proposition that a ninth-century writer could not have simultaneously accessed a Latinate literacy and a distance-shaped spoken competency when creating a \textit{scriptus}.

The second false dilemma is that the poem’s “constitutive orality” necessarily implies that the poem must have been composed orally. This dilemma connects to the previous one that postulates that a single person could not have had command of elaborated oral varieties of distance and be literate in Latin. Therefore, only a trained singer, who must have been illiterate, could have produced poems that featured an authentic and constitutive orality of the kind supposedly exhibited in the \textit{Hêliand}. \citegen{Haferland2010} evidence of the poem’s constitutive orality varies in degree of persuasiveness. None of it unequivocally points to the work having been orally composed~-- rather than featuring a \textit{scriptus} constructed with oral varieties of distance in mind~-- while some examples point to the opposite conclusion, that is, that the work is a literate one. For example, the author discusses the poet’s use of repetition, which speaks to the work’s oral orientation, but also explains how passages of multiple lines are repeated \textit{verbatim} (pages 177--180). \citet[57--65]{Ong2012}, discussing the work of Milman Parry and Albert Lord, notes that verbatim memorization is generally not a strategy employed in the spontaneous composition of epic songs and instead points to the existence of a static text. Haferland’s conclusion that a literate approach to the composition of a text is incompatible with repetitions of this sort (pages 179--180) does not discount the possibility that the \textit{Hêliand} poet(s) composed the work in writing but drew on the linguistic properties of orality in the construction of their \textit{scriptus}.

Other examples are even less convincing, such as when \citet[184--185]{Haferland2010} claims that the poet’s descriptions of the act of writing come from an “outsider’s perspective.” The author bases this conclusion on the fact that the poet describes the act of writing in ways that emphasize the physicality of the act, that it involves producing letters (\textit{bi bôcstabon} ‘with letters,’ 230) in a book (\textit{an buok} ‘in book,’ 14) with fingers and hands (\textit{mid handon} ‘with hands,’ 7; \textit{fingron} ‘with fingers,’ 32). Why such descriptions are any more likely to have come from an illiterate person than a literate one is unclear. In fact, medieval scribes were known to comment on the fact that writing was hard work that could tax the body (\citealt[41, 43]{Gullick1995}). Other arguments, like the idea that a clerical poet would never have made some of the interpretational errors in the spiritual meaning of the text (see \citealt[192--200]{Haferland2010}), are more convenient than convincing.

With these assumptions, Haferland paints himself into a corner, where he has to create an elaborate story of the \textit{Hêliand}’s composition that involves a “clerical advisor who selected and arranged the material, with the poet being responsible only for the composition and the blunders,” a division of labor that \citet[178]{Bostock1976} characterizes as “highly improbable.” One should also reflect on the practicalities of undertaking such an endeavor in ninth-century Francia. Consider all of the tools modern linguists have at their disposal when creating a transcription of spoken language. For example, they can capture an oral performance with recording devices that allow them to play it back, stop and restart the stream of sound, and, thus, transcribe what they hear incrementally and, to some degree, faithfully. How do scribes in the medieval setting accomplish anything resembling this process? Does the spontaneously composing poet simply speak very slowly? Do they repeat themselves? Can the rhythm of spontaneous oral composition be adjusted like that? \citegen{Ready2019} extensive examination of the process of “textualizing” an oral performance through dictation and transcription provides a wealth of comparative evidence from oral traditions around the world that is fatal to \citegen{Haferland2010} imagining of the \textit{Hêliand}’s composition as an orally composed, unliterized transcription.\footnote{See in particular \citegen{Ready2019} chapter 3, “Textualization.” I discussed Ready’s use of the term “textualization” in \chapref{sec:chap:3} and how it relates to my use of the term “literization.”} The “dictation model,” \citet{Ready2019} argues, can only be considered plausible with the presence of a collector \textit{cum} collaborator, who shapes the text considerably and contributes to the oral performance’s textualization (pages 158--173). \citet{Ready2019}, for example, demonstrates that the structure and language of a dictated poem are fundamentally different from a performed poem (pages 168--169). Additionally, collectors and scribes alike intervene heavily in the performance of the song and its textualized manifestation (pages 169--173). Note here that I use the term textualization to refer strictly to the act of transferring an exclusively oral discourse tradition into the graphic medium. This process yields examples of “textualized orality.” The \textit{Hildebrandslied} is one of the few texts in the history of German that we may characterize as textualized orality. In order to textualize orality, the writer must still literize the oral variety associated with that discourse tradition, that is, deal with questions of ausbau.

Other practicalities come to mind. For example, modern linguists rely on established orthographic conventions designed to represent sound on a page: the International Phonetic Alphabet, if one requires a closer transcription, or an alphabet, when such phonetic specificity is unnecessary. In \chapref{sec:chap:3}, I explained how there were no writing conventions established for the vernacular in the early medieval period. Those who decided to write in the vernacular had to establish their own systems for representing on the page what had only existed as sound. This fact means that a would-be transcriber of the \textit{Hêliand} first would have to develop an alphabet, which is to say, a system of sound-letter correspondence, for their illiterate poet’s vernacular, and then figure out how to spell the words that the poet says and decide on questions like where the word boundaries should go.

One might also ask whether ninth-century scribes could write quickly enough to make any type of transcription possible. Michael Gullick for his chapter, “How fast did scribes write?,” consulted historical sources as well as the modern calligrapher Donald Jackson before concluding that a European scribe in the Romanesque period, so from the year 1,000 to 1,200, could write roughly twenty-five lines an hour, possibly managing as much as 200 lines per day (page 50).\footnote{\textrm{I discovered \citegen{Gullick1995} chapter through the blog post, “Clever sluggards? How fast did medieval scribes work?,” written by Deborah Thorpe on October 28, 2015 (\citealt{Thorpe2015}) and accessible at:} \url{https://thescribeunbound.wordpress.com/2015/10/28/clever-sluggards-how-fast-did-medieval-scribes-work/}. } From the perspective of the scribe, the poet would have to dictate very slowly indeed for the scribe to be able to keep up. Alternatively, the scribe could try to write as quickly as they can, sacrificing fidelity to what they hear and requiring lots of repair after the fact, which again undermines the possibility that the poem is a transcribed oral performance. The \textit{Hêliand} poem is about 6,000 lines long,\footnote{\textrm{\citet[168]{Bostock1976} notes that 5,983 lines of the poem have been preserved. The conclusion is missing, but its length is “unlikely to be very great.”} } so one may assume that under ideal circumstances it could take around 30 days of working eight-hour days to write the whole poem out in the normal scribal fashion, which is to say, copying from an exemplar.\footnote{\textrm{Writing in the medieval environment was not always conducive to speed, \citet[43]{Gullick1995} reminds us. Conditions in} \textrm{\textit{scriptoria}} \textrm{could be cold, as many scribes complain in colophons and marginalia. Cold fingers do not write as quickly as warm ones do.} } It is difficult to surmise how long it would take to perform and transcribe such a long work. For one, the poet would need a prodigious cognitive stamina to listen to the clerical advisor describe to him the material, which would include stories and doctrinal interpretations that, according to Haferland’s narrative, the illiterate poet would otherwise have no access to, to organize that material without the aid of writing and using the lexicon and practices of their singer training, and then immediately perform that material in the mode of a slower dictation that, \citet[114--117]{Ready2019} notes, singers often find challenging. The poet, as well as the rest of the team, would presumably have to repeat this process day after day. One particular question emerges after considering all these practicalities. \textit{Why} would those responsible for the work choose to compose the poem in this arduous way?\footnote{\textrm{It is not as if there existed a long tradition of oral compositions of the gospel narrative in Frankish Europe that people might have wanted to capture in writing, as was the case in Ancient Greece and its tradition of Homeric epic poetry.} }

  This discussion, I hope, is instructive in elucidating some of the practical challenges that would arise in producing a transcription of spoken language in early medieval German-speaking Francia. These challenges point to the conclusion that transcription without at least some degree of literization, that is, an editing and/or shaping of the poem as a \textit{written} text, simply would have been impossible. From the German-language tradition, we have only two works of textualized orality, the \textit{Merseburg Charms} and the \textit{Hildebrandslied}; these are the closest extant representations of a German oral tradition. Whoever was responsible for their textualization and literization certainly drew on their oral varieties of distance to create and structure their \textit{scriptus}, similar to how the \textit{Hêliand} poet surely did. However, one might also expect notable differences in the resultant \textit{scripti} for the following reasons. First and foremost, unlike the \textit{Hildebrandslied}, the \textit{Hêliand}’s story does not stem from the oral tradition of the community that created the text. It is an imported story, and it is the poet’s particular literization of it that would have made the story seem more familiar and accessible to listeners in communities in East Francia and Saxony. Let us also consider more closely what it would have meant for a story to be embedded in and grown from the Germanic-language oral tradition. Unlike the \textit{Hêliand}, the \textit{Hildebrandslied}’s story, characters, themes, structure, and lexicon would have already been well known to the members of the community. These elements are in and of the community’s cultural and linguistic cosmology. Whoever wrote the \textit{Hildebrandslied} did not have to engage with the literization process as intensively as the \textit{Hêliand} poet, who had to transfer all of these same elements to a new milieu. One might hypothesize, therefore, that the author of the \textit{Hildebrandslied} engaged in less ausbau than that of the \textit{Hêliand}.

\section{Conclusion}\label{sec:5.4}

An important goal of this chapter was to describe the basic parameters of a Carolingian education, which focused on fostering in its students a Latinate literacy, though likely with mixed results. The ultimate aim of this educational program was to teach people, clergy in particular, how to read and write Latin so that they could understand the Bible. For this task, they relied on late Roman grammatical treatises that detailed the descriptive and prescriptive norms of Latin. The Romans inherited this tradition of \textit{grammatica} from the Greeks, and, just as the Romans conceived of the literization of their oral vernacular in terms set by classical linguistic thought, so too did the Carolingians. That is, in this chapter I argued that the Latin language and its discourse on language (i.e., \textit{grammatica}) provided the conceptual framework within which the Carolingians understood the literization of their own language.

The \textit{Evangelienbuch} is one of the best early sources of evidence that speaks to this point. Otfrid comments directly on the literization process and details the Frankish language’s many ills. Chief among these problems is the fact that Frankish is not Latin or, rather, \textit{like} Latin: it is not a written language and, thus, has not been regulated in any way. It is wild, unruly, and unworthy of the great civilization that speaks it. Otfrid’s prescription for the language is metrical, and so he introduces a poetic scheme that, in his view, disciplines it while still allowing him to build a \textit{scriptus} that does not directly contradict his own grammatical intuitions. Given Otfrid’s desire to turn Frankish into a proper written language and the fact that he had a particularly Latinate, or classical, model of a written language in mind when composing the work, I propose the following. When Otfrid made choices regarding ausbau, that is, how to effect the requisite degree of grammatical cohesion by augmenting Frankish’s lexical density and integration, he turned to the grammatical treatises that formed the backbone of his literacy education. In light of this orientation, Otfrid was inclined to avoid the types of linguistic features that characterized the spoken distance varieties comprising the oral tradition that surrounded him.

Thus, in this chapter I demonstrated how to analyze an early medieval text with respect to its author’s orientation toward their oral vernaculars of distance (the language of the oral tradition), on the one hand, and their education in a Latinate literacy, on the other hand. My contention is that this analysis constitutes a method for approaching linguistic variation across early German texts. That is, language ausbau processes that shape oral vernaculars into a graphic form that is functional in the dislocated visual space of the page promote new degrees of lexical and grammatical coherence (\chapref{sec:chap:4}), but the resulting \textit{scripti} will still be distinct because their creators draw on their linguistic resources in ways that reflect their particular project and goals. Because ausbau is a conscious process of linguistic change, how a literizer like Otfrid thought about his vernacular and language in general is relevant to the \textit{scriptus} he creates. Though his multilectal linguistic intuitions surely inform his vernacular writing, so too do the literate model provided by Latin, the metalanguage he learned through his education in \textit{grammatica}, and the prescriptive solutions he devised to address Frankish’s rusticity based on this linguistic training. A literization approach, then, considers all these factors in an analysis of a \textit{scriptus}, rather than just focusing on delineating the structures that comprise those intuitions.

The \textit{Hêliand} poet (or poets), in contrast, covered the same narrative ground as Otfrid did in the \textit{Evangelienbuch} but drew more directly on the oral tradition for its formulation. They transferred the gospel story from its original milieu to that of northern Europe and used what would have been the widely familiar patterns of alliterative verse. Similarly, one should expect that the \textit{Hêliand}’s \textit{scriptus}, generally, and more specifically its ausbau structures will reflect, to some degree, the lexically dense and integrated structures of oral varieties of distance, a description and account of which is the focus of \chapref{sec:chap:6}. Where Otfrid eschewed structures that he would have associated with elaborated orality, the \textit{Hêliand} poet embraced these same patterns. However, as I argued in \chapref{sec:chap:4}, the act of creating a \textit{scriptus} would still push its creator toward augmenting, or making some attempt to augment, the grammatical coherence of their vernacular. This conclusion follows from the fact that writing has the potential to dislocate language from the language producer and the time, place, and culture of its production. Thus, the written medium impels writers toward grammatical and lexical explicitness. In \chapref{sec:chap:6}, I highlight the ambiguities that arise when the structures in a \textit{scriptus} are more directly inspired by oral varieties of distance and their different systems of syntactic organization and when literizers engage less in ausbau, specifically.

What I would like to work toward is a more multifaceted, sociolinguistic understanding of these early German \textit{scripti} than the one that focuses only on reconstructing the structures of an underlying competence. Consider the following. First, there is an early author who consults their intuitions on the various generalizations, patterns, and forms that are associated with the multiple (spoken) varieties of which they have command. Some of these varieties are shaped by the communicative demands and constraints of distance and, thus, feature a degree of lexical density and integration that all distance contexts, be they oral or written, require. I elaborate this point in \chapref{sec:chap:6}. None of these varieties can function as a written language without some level of literization, even if that simply involves mapping vernacular sounds onto graphemes. Most writing projects require a good deal more, however, including the ausbau of lexicon and grammar to increase the vernacular’s semantic and grammatical coherence. I have demonstrated that the early literizer of German also had some knowledge of Latin, as well as education in the classical tradition of \textit{grammatica}. This training can also be counted among the linguistic resources to which an author had access.

This literization process does not and cannot happen naturally or \textit{only} by virtue of people tapping into an underlying competence.\footnote{\textrm{A corollary to this statement is the fact that people do not learn how to read and write unless they consciously undertake this task and, not to mention, devote many years to it.} } Relatedly, a systematic written language cannot exist until people create it, and creating this written language requires conscious decision-making from its creators. For example, which generalizations, patterns, and forms will the literizer transfer from their different oral varieties into their \textit{scriptus}? To what extent will they consult or eschew Latinate descriptive and prescriptive norms during its crafting when solving the unavoidable puzzle that is ausbau? How successful will they be in implementing these ideas consistently? Answers to these questions can elucidate linguistic variation not just within, but also across early medieval texts. In contrast, a singular focus on the reconstruction of underlying structures means that we are telling only one~-- small, I believe~-- part of the story. In not considering literization’s role in the early history of the German language, I maintain that we have ignored one of the most revolutionary forces, if not \textit{the} most revolutionary force, that can effect linguistic change; namely, the literization of an oral vernacular.

In the chapters that follow, especially \chapref{sec:chap:7}, I propose that there is another important aspect to conceptualizing the choices that early German literizers made when crafting a \textit{scriptus}. Namely, they also had to develop a sense of written well-formedness for this new variety of the vernacular. In other words, necessarily accompanying the task of selection, i.e., ‘I will do it this way, not that way,’ are considerations of suitability, appropriateness, and aesthetics, i.e., ‘I \textit{should} do it this way, not that way.’ One structure might offer greater coherence or clarity in the graphic medium than the other or might evoke elaborated orality and, thus, be more or less desirable, depending on the nature of the project. Another way of referring to the development of well-formedness might be to speak of an evolving sense of “literary style,” though the former has been used in the linguistic literature in reference to grammatical constraints, while the latter is not seen as a matter of grammar. In \chapref{sec:chap:6}, I focus particularly on the syntactic characteristics of elaborated orality so that we may better understand the oral varieties of distance and, thereby, the full spectrum of linguistic resources available to an early German literizer. This discussion will also be a necessary corrective to our modern literate subjectivity in that we will try to approach unliterized, exclusively oral vernaculars on their own terms. That is, the metalanguage from the classical linguistic tradition introduces to speakers of oral vernaculars the \textit{ideas of structures}. If ausbau is conscious linguistic innovation undertaken to address real communicative deficiencies arising from the functional gap between planned oral varieties of distance and written language, to what extent do the classical conceptualizations and categories discussed in the popular grammars shape or effect proffered solutions to the ausbau puzzle?

I conclude this chapter with an acknowledgment of what it is missing, namely discussion of the third ninth-century gospel harmony, the German translation of Tatian’s \textit{Evangelienharmonie}. My interest in exploring the early ausbau of German and the role of these first \textit{scripti} in the creation of early \textit{scripti} and a written German means that, in my mind at least, this text is a less significant development. This is not to say that it cannot reveal interesting aspects of early ausbau, particularly with respect to how or to what extent German speakers creatively shaped their vernacular to accommodate a largely literal translation of a Latin Christian text. The German Tatian translation remains important, if for no other reason than it is one of the few German-language texts of any notable length from the early medieval period. However, it has less to tell us about the process of \textit{scriptus} creation and ausbau than the other two ninth-century gospel harmonies. The authors of these latter two texts developed a written medium to tell the story of Jesus from their own vernacular point of view; the \textit{scripti} they created reflected their individual orientation toward the subject matter. In other words, their poems are the product of a unique, vernacular, and literary engagement with the source material. In contrast, the Tatian project is not the product of a similar literary or vernacular engagement with the source material. Its creators did not develop their own point-of-view; they simply adopt that of Tatian. In this way, the translation is actually a more accurate representation of the state of vernacular literacy in early medieval Carolingian Europe in that resources were funneled into revitalizing and promoting Latin literacy, rather than into creating a vernacular written medium. The Tatian text with its transparent line-by-line~-- and largely word-for-word~-- translation of its source would have been an excellent study aid for German speakers eager to understand Biblical Latin better and was probably created for that reason. The same cannot be said for Otfrid’s work, the \textit{Hêliand}, or for that matter, the Isidor translation. The Tatian’s \textit{scriptus} is a by-product of the translational process, rather than the intentional vernacular literization that the other two gospel harmonies represent.

