% ATTENTION: Diacritics on the following phonetic characters might have been lost during conversion: {'ɑ'}
\chapter{Introduction}\label{sec:chap:1}

Imagine you live in East Francia in the early ninth century and have decided to write a continuous text in your German vernacular. You are probably, though not necessarily, a man and member of the clergy.\footnote{See \citet[124--138]{Garver2009} and \citet[223--227]{McKitterick1989} for evidence that women, both religious and lay, were literate. It is for this reason that, throughout this book, I refer to the anonymous people engaged in literization with the gender-neutral \textit{they}. Using the supposedly, but not actually, gender\hyp neutral \textit{he} only reinforces the erroneous conclusion that all medieval women were illiterate and unengaged in Carolingian literary and documentary culture. Regarding the assumption that all literary activity was religious, see the entirety of McKitterick’s chapter~6, which explores the literacy of the Carolingian laity, concluding on page 270 that it was a “literate laity.”} You can undoubtedly read and write Latin. No one else writes in German, though. What would be the point? It is Latin, the language of the Church, that matters. With already more than seven centuries of development as a literate language, Latin should suit your needs perfectly. It is tailor-made to relate and discuss the most important topic of all, the Word of God made manifest in the Latin Bible. That is why the Carolingians have invested so many resources into its revitalization and promotion. That is also why your library is full of copies of Latin-language texts. What little written German can be found there is confined to the glossing of Latin words. The vernacular is none of the things Latin is. It is an embodied phenomenon~-- that is, there is no \textit{corpus} separate from the people who speak it~-- and exists only as ephemeral sound. It is secular, rather than spiritual. It is the language of paganism and the epic songs from the Franks’ barbarian past. This vernacular writing project that you have undertaken is not wholly consistent with the sociocultural values of the empire and the community in which you live. To be frank, taking it on makes you a bit of an oddball.

It will also be a significant challenge to wrangle your phonic vernacular onto parchment in a graphic form. \textit{Literizing} your oral vernacular, which itself consists of multiple spoken varieties, is not simply a matter of transcription. Transcription would be difficult enough, as German has no orthographic conventions~-- and early medieval Germans, no access to recording devices. But you must do more than match sound to grapheme and push your vernacular beyond its existing spoken competencies so that it can function in the fully dislocated context that writing alone can create. Because your colleagues across the empire are thoroughly engaged in a Latinate literacy~-- writing in and improving their Latin, making multiple copies of Latin-language texts~-- you also cannot rely on a community of vernacular literizers to support your work. With no German tradition of literacy on which to draw, you have no sense of how German might or should be written. So, you must decide for yourself how to engage all your linguistic resources, your multilectal\footnote{“Multilectal” and “multilectalism” are terms that I have adopted from \citet[42--44]{Höder2010}, which points out that crosslinguistically communities tend to be multilingual and are certainly multilectal. That is, there is intralingual variation; no one speaks just one variety of a language.} vernacular and the Latin in which you were educated, to create a new written variety, or a \textit{scriptus} (plural, \textit{scripti}).

My purpose in beginning with this thought experiment, which asks you, the reader, to imagine yourself as an early medieval German-speaker engaged in literization is to draw attention to your subjectivity as a literate person. From your vantage point, living in a society in which the written word plays a significant or even all-encompassing role, it can be difficult to conceptualize the many ways in which an oral vernacular must change to become functional in the graphic medium. If you are a linguist or literary scholar, this act of imagination is even more difficult in that the very terminology of our fields is rooted in literacy and stems particularly from classical discourses on language retained in the writings of the Greeks and Romans. Egbert Bakker, for example, describes this struggle in the introductory chapter to his \citeyear{Bakker1997} book, \textit{Poetry in Speech: Orality and Homeric Discourse}. 

\begin{quote}
Working within the speech perspective implied by my methodology and forcing myself to read Homer as the transcoding of one medium into another, a flow of speech through time that has become a transcript, \textit{I began to realize just how much of the vocabulary and the notional apparatus used for our study of language and style is overtly or covertly literate, pertaining to our writing culture, and thus perhaps more indicative of the perspective of the philologist than of speech studied in the form of a text}.\\
\hbox{}\hfill(\citealt[3]{Bakker1997}, emphasis added)\hbox{}
\end{quote}

\noindent Setting aside, for the time being, the validity of the approach he describes in the non-italicized portion of the quote, Bakker identifies the challenge of approaching Homer’s text specifically as an early literization of Greek. As such, one’s methodology must accommodate the fact that the first written testaments of a language reflect emerging written grammars whose structures might be organized in ways more characteristic of spoken varieties than written ones. The danger of not recognizing one’s subjectivity as a literate person in command of a modern written variety, which itself is the product of many centuries of literization, can lead to an inability to see the \textit{scriptus} on its own terms. 

Two main questions animate this book. First, what did the process of creating an early medieval \textit{scriptus} entail when German was still an almost exclusively oral medium of communication and only just beginning its long journey toward literization? Second, how does the process of \textit{scriptus}{}-creation shape early linguistic structures? Specifically, how does it interact with the syntactic variation that is attested across the \textit{scripti} from this first period of literization? One of this book’s central tasks is to define literization and \textit{scripti}, particularly as they relate to the history of the German language. A good starting point, however, is to define literization in two ways. First, in the synchronic sense, it is the act of a speaker transforming their multilectal spoken vernacular into an ad hoc written variety or \textit{scriptus}. It also refers to the diachronic process of a community of speakers creating a written language and writing culture for their vernacular; that is, the development of increasingly functional \textit{scripti}, in turn, opens up new domains to the written word. My conception of a \textit{scriptus} builds on the one found in \citet[596--597]{KochOesterreicher1994}, which itself draws on \citegen{Gossen1967} description of early literizations in French medieval charter texts. Koch and Oesterreicher’s article discusses the “early writing traditions” (\textit{Schreibtraditionen}) of a mostly oral vernacular, which, they note, may never expand beyond their original spheres of use or connect to later written languages (\textit{Schriftsprachen}).\footnote{{\citet{KochOesterreicher1994} and \citet{Gossen1967} use the feminine participle,} {\textit{scripta}} {(plural,} {\textit{scriptae}}{), rather than the masculine} {\textit{scriptus-scripti}}{, presumably inflected for the feminine noun,} {\textit{traditio}}{, ‘tradition.’ I adopt instead the masculine form to move the term away from being more explicitly diachronic.}} This definition highlights a \textit{scriptus}’s possible diachronic isolation. My notion of the term is similar, though with an additional emphasis placed on the \textit{scriptus} as a material artifact resulting from incipient literization, as I hope to make clear in the pages to follow.

Returning to the questions I posed at the beginning of the last paragraph, there are several propositions embedded in them that are worth stating outright. First, I propose that when people write in a vernacular \textit{for the first time}, they cannot draw on any principles of a vernacular written grammar because one does not yet exist. While the first literizers can~-- and surely do~-- consult linguistic intuitions that guide their production of a multilectal spoken vernacular, these intuitions alone will be insufficient to create a variety that is functional in the graphic medium. This conclusion follows logically from the fact that the intuitions connected to exclusively spoken varieties are determined by and perfectly suited to the phonic medium. Before the literization process begins, none of these varieties, whose production and reception had always been facilitated by interlocutors being in the same place at the same time, is equipped with the linguistic means of establishing the unprecedented degree of grammatical and lexical coherence required by the written word. Consider that the new domain of the page could disconnect completely the vernacular from the one who produces it and anyone who might read it. 

\begin{sloppypar}
  Thus, literization begins with the construction of individual, idiosyncratic \textit{scripti} for which the literizer must consciously innovate if they are to create a written form of their vernacular that has the grammatical and lexical tools required to function in these new dislocated spaces. This innovation does not occur in a vacuum, but rather is guided by all the linguistic resources to which the new vernacular writer has access: in the case of German, these resources include multiple varieties of spoken German, as well as Latin, the main written language with which a literizer would have been familiar.\footnote{{Knowledge of Greek was fairly limited in Carolingian Europe (\citealt[7]{Persig2020}, \citealt{Dickey2016}:} {\textit{passim}}{), though its grammar was accessible through the Late Roman grammars that were popular at the time. These grammars drew explicitly on the Greek tradition of} {\textit{grammatikós}}{. German-speakers would have encountered Romance speakers from the western parts of Francia. They spoke what was often called a “rustic” Latin (more on this in \chapref{sec:chap:3}). Their written language was simply Latin, though the ever-growing gulf between how people wrote Latin and how they spoke it was a recognized problem and prompted Charlemagne to implement reforms to restore a prescriptive, classical Latin.}} This argument indicates that my account of the early German \textit{scripti} is more compatible with usage-based approaches to language than it is with structural approaches, in particular, generative syntax. Usage-based approaches assume that the ways that people use language is determinative of the structure and organization of that language. In contrast, generative approaches maintain a separation of language use, that is, performance, and language structure, that is, competence.\footnote{{More recently in generative theory the terms i-language and e-language have replaced competence and performance, respectively. The former, to quote an anonymous reviewer, is “internal, intrinsic, and individual,” the latter, “external.” Though the terms are new, the divide between a mental grammar, i.e. competence or i-language, and its actual manifestations, i.e. performance or e-language, remains. I retain the older terminology because they are more accessible than the new terms to non-generativists.}}
\end{sloppypar}

\begin{sloppypar}
My intent is not to argue against generative and other structuralist approaches to linguistic inquiry. Rather it is to propose that, in concerning themselves primarily with the identification and elucidation of mental grammars, they are poorly equipped for a comprehensive study of early \textit{scripti}, whose creation demanded that literizers must also innovate beyond their existing linguistic intuitions. I, furthermore, propose that these approaches have led to a type of linguistic presentism, which has hampered our understanding of incipient literizations on their own terms. That is, in the absence of native speaker judgments, which a framework like generative syntax tells us are the only direct evidence of a mental grammar, modern scholars have too often projected their own intimate knowledge of highly literized written grammars, for example, Modern English and German, backwards onto the first literizations of a mostly oral vernacular. This fallacy has led to scholars creating, as it were, structures in historical data that may or may not have actually been a part of  any of the literizers’ mental grammars. 
\end{sloppypar}

My two main research questions contain a second proposition, namely that the linguistic production of the oral and written varieties in early medieval, German-speaking Europe relied on different processes. Setting aside for now any implications this statement has beyond the context of incipient literization, I argue that, for the \textit{scriptus} creator, linguistic production in the graphic medium is fundamentally different from that in the phonic medium. 

In the phonic medium of an exclusively oral vernacular, I propose that linguistic production is shaped by the convergence of human consciousness and language. Humans modulate that stream to accommodate the communicative requirements of “immediacy” and “distance” contexts. The contexts of immediacy require spontaneous, intimate, and dialogic language, while distance contexts require planned, public, monologic language.\footnote{The terms “language of immediacy” and “language of distance” are from Peter Koch and Wulf Oesterreicher’s seminal \citeyear{KochOesterreicher1985} article, \textit{Sprache der Nähe – Sprache der Distanz: Mündlichkeit und Schriftlichkeit im Spannungsfeld von Sprachtheorie und Sprachgeschichte}.} Varieties on both ends of this spectrum must rely on the same cognitive faculties for their production. Thus, the challenge of processing the spoken varieties of immediacy lies in the speaker verbalizing their focus of consciousness in real-time with little to no planning. Fortunately, and not coincidentally, the language of immediacy is influenced, even determined, by the naturally restless movement of a fickle human consciousness. Potential misunderstandings between interlocutors can be smoothed over through extralinguistic cues and a more participatory engagement from all involved. Distance varieties, in contrast, must work against the roving human consciousness, and speakers must find ways to slow down their~-- and interlocutors’~-- shifts in focus. For example, an effective public address requires language that is more cogent and organized, and does not jump discursively from one topic to another. These types of linguistic production are also less participatory and the onus of effective communication falls more heavily on the speaker. There is an additional challenge with the production of oral distance varieties in that speakers must plan their language with no access to writing. This fact requires speakers to engage those memory systems that they rely on to produce spontaneous language, but to a much enhanced degree.

Creating a new written variety, a \textit{scriptus}, must be a different cognitive process from the ones I just described. That is, the first \textit{scripti} of a vernacular are not subject to the same communicative pressures that characterize how speakers process its oral varieties of immediacy and distance. Furthermore, as I argued earlier in this introduction, the structural generalizations that underlie these vernacular productions, which one might collectively refer to as a mental grammar, will lack the linguistic tools the early literizer needs to effect the grammatical and lexical coherence required by the new graphic medium. My conception of \textit{scriptus} creation gives primacy to the active human choice that the process requires over the unconscious expressions of the mental grammars that also find their way into early \textit{scripti}. For example, the literizer must decide which of their many linguistic resources they intend to draw on to create a \textit{scriptus}, from their multilectal oral vernacular to Latin. The writer must also work out how they will improve the functionality of an exclusively oral vernacular as they transform it into a graphic mode of communication. They must improve German’s grammatical and lexical coherence so that it gains the ability to signify the relationships between constituents and ideas in a more unequivocal fashion. These relationships can remain more implicit when interlocutors are in the same place at the same time, as must be the case in communities that speak an exclusively oral vernacular (in the absence of recording technology). One should expect that grammatical and lexical patterns from the literizer’s oral varieties will find their way into a new \textit{scriptus}. However, the process of literization itself will augment these patterns with new systems of grammatical and lexical coherence that would and could not have been present in the literizer’s existing multilectal vernacular grammars. 

This dynamic process that I just described in which a \textit{scriptus}{}-creator decides, first, how to make their \textit{scriptus} more grammatically and semantically coherent and, second, how to engage their various linguistic resources interacts with two other important changes: the development of concepts of both well-formedness and a literary style. My notion of well-formedness refers specifically to coherence. It involves the literizer recognizing the gap that exists between their oral vernacular varieties, which have always been functional in their exclusively phonic medium, and a written variety, which functions best when syntactic and semantic relationships are explicitly marked. This recognition prompts the literizer to create explicit and/or more graphically visible means of marking such relationships. Steffen \citegen[139--160]{Höder2010} analysis contains examples of how medieval language planners created a more coherent \textit{scriptus} for Written Old Swedish by drawing on Latin models. More specifically, they added to its subjunction inventory a series of monosemantic, polymorphemic subordinators. While earlier varieties would often link clauses with polysemous, monomorphemic particles, like \textit{än}, these new, constructed subordinators were more specific. \textit{Än} could be used in a whole host of contexts, from comparative to conditional to adversative. In contrast, the newly introduced \textit{for þy}, which was modeled on Medieval Latin’s \textit{pro eo}, was used to mean one thing: ‘because.’ Thus, well-formedness is what results from literizers recognizing an orally shaped variety’s lack of coherence in the written medium and implementing linguistic strategies to augment it. A \textit{scriptus} may be more or less well-formed depending on the extent to which its creator addressed an oral variety’s inherent shortcomings in the graphic medium. 

Alongside the cultivation of well-formedness in a new \textit{scriptus}, literizers must also develop a conception of literary style. This idea refers specifically to how a literizer chooses to engage their numerous linguistic resources in order to effect greater coherence in their innovative \textit{scriptus} and create a text that is suited to their goals for the project. For example, the literizer might want to textualize a story that has circulated in the oral tradition for generations, let us say, a tragic song about a father who must face his long-lost son in battle. This goal would logically lead the literizer to create a \textit{scriptus} that is evocative of the oral tradition. As a result, they attempt to augment the coherence of their \textit{scriptus} in ways that are consistent with, or at least reminiscent of, the language of the oral tradition, that is, the planned and public oral varieties of distance used in their speech community. Another literizer, however, might want to create a vernacular \textit{scriptus} that carries the same prestige as the classical languages and is appropriate for a retelling of the Gospels in its original Middle Eastern milieu and a discussion of their theological significance. In this case, they would intentionally eschew the same linguistic characteristics that our previous literizer embraced. They must also create a \textit{scriptus} that has the means, that is, the vocabulary and syntactic structures, to engage in a type of discourse for which Carolingians have always used Latin in the past. Both literizers create \textit{scripti} that align with and are appropriate to their goals. This process yields what one might reasonably call two different vernacular writing, which is to say, literary styles. 

Conceptualizing the choices that a literizer makes in terms of well-formedness, on the one hand, which is about creating a coherent \textit{scriptus}, and literary style, on the other hand, which has to do with resource engagement, points to another hypothesis about the process of creating \textit{scripti}. That is, there exists a relationship between these two types of decisions, and a literizer might be more or less focused on establishing well-formedness depending on the type of literary style they aim to create. Let us return to the two imagined \textit{scriptus}{}-creators from the previous paragraph: the first is textualizing a song from the oral tradition, and the second is creating a vernacular \textit{scriptus} as medium for the treatment of theological topics that would otherwise be discussed exclusively in Latin. A logical proposal is that the latter \textit{scriptus}{}-creator becomes more aware of the functional gap between exclusively oral and written varieties and, thus, takes greater care to create structures and lexical items that augment their \textit{scriptus}’s coherence than the former literizer. Consider the fact that the Latin of early medieval Europe had already undergone about eight centuries of development as a literate language. Furthermore, this development included a robust discourse on issues surrounding well-formedness. In aiming to create a German \textit{scriptus} that can function as well as Latin in the graphic medium, the shortfalls of the vernacular, still an oral phenomenon, would come into clearer relief for the literizer. Drawing more extensively on the varieties of the oral tradition, as the former literizer does, invites fewer direct comparisons to Latin. Furthermore, the vernacular is already well suited to telling a story that stems from its own oral tradition; for example, it does not require the introduction of as many new words and phrases that refer to topics that were once unknown to the community. It will lack the requisite coherence for optimal functionality in the graphic medium, but the literizer might be less inclined to fill that gap and happier to leave syntactic and semantic relationships implicit. This decision can result in early medieval \textit{scripti} that are structurally ambiguous, particularly to modern readers, but perhaps also to contemporary readers who were not well versed in the oral traditions that inspired the \textit{scriptus} in the first place. 

It is important to emphasize that I see the process that I just described as something distinct from that of norm formation. That is, this book is not about early medieval attempts at developing a prescriptive German. A more structurally oriented linguist might characterize it in this way because they assume a strict division between competence and performance and believe that identifying structures that truly reflect competence is the most important goal of diachronic linguistic analysis. These linguists see themselves as being interested in description rather than prescription. In other words, they care much less about the extent to which an early medieval writer wonders how they \textit{should} write something than they do about how that writing expresses only their linguistic intuitions, i.e., competence. As a result, my discussion of topics like literary style may give the impression that I want to describe the origins of a prescriptive German. This is not the case. Rather, I argue that \textit{scripti} result from literizers’ necessary and individual engagement with questions of well-formedness and literary style, which requires two things from the literizer: the conscious recognition of the functional shortfalls in their oral varieties and the creation of stylistically appropriate ways of addressing them. These are not questions of prescription. That is, I see the literization process itself as having transformed early German in ways that are difficult to elucidate if one assumes a strict competence-performance binary and adopts the epistemological orientation that competence has primacy, while the literizers’ stylistic or prescriptive considerations are confounding factors stemming from performance. 

This discussion brings me to my third major proposal; that is, in order to understand structural variation across the first German \textit{scripti}, we must first investigate the process of individual \textit{scriptus}{}-creation and the range of choices that was available to the early medieval literizer. Understanding this range of choices necessitates that Germanic linguists consider topics that have generally not been viewed as relevant to early German syntax. These topics connect to the wider social, cultural, and political contexts of Carolingian Europe. For example, while previous studies have dealt extensively with medieval Latin’s relationship with German, scholars have focused primarily on whether or to which extent Latin syntax confounds a native German competence. In the approach I propose here, in contrast, a literizer’s knowledge of Latin is a logical linguistic resource for \textit{scriptus}{}-creation. An important question to ask then is whether or to what extent the literizer drew on their Latinate literacy, which included training in the classical discourse surrounding Latin and its use, termed \textit{grammatica}, in order to create their \textit{scriptus}. As an already highly literized language, Latin would have a multitude of structures that augment well-formedness in the written medium, as well as stylistic discussions of how to write appropriately, which is to say, well. Research on Carolingian documentary culture and the history of linguistics describes the Carolingian preoccupation with the classical tradition of \textit{grammatica}. It also tells us which classical grammars were the most influential. Chief among these were works by the Latin grammarians Donatus (mid-fourth century) and Priscian (sixth century), both of whom approached the study of language in ways that were heavily influenced by the writings of Aristotle. These texts created a framework within which the Carolingians thought about their own vernacular in a systematic way. This intellectual environment of classical literacy is where each individual act of literization occurred, and we should take it as a given that Latin influenced every vernacular \textit{scriptus} that was created. 

Because diachronic generativists and other structurally oriented linguists have focused on isolating a supposedly native German competence, they might be reluctant to acknowledge the widespread influence that Latin and \textit{grammatica} must have had on \textit{every} early German literization. The usual method in the diachronic syntactic literature has been to isolate any data that might have been affected or even effected by so-called confounding factors, for example, a Latin source text or, in the case of poetic texts, a metrical scheme. Such data, according to this view, obscure the true nature of the underlying competence and are often excluded from analysis or perhaps treated as evidence of some older, competing grammar. Within this methodology, the possibility that \textit{all} the early German data are ostensibly influenced by the literizers’ Latinate literacy would imply that it might not be possible to control for its confounding effects. Scholars’ narrow focus on competence means that we have been treating the Latin influence on early German as a problem, rather than as a relevant and even enlightening sociolinguistic factor that indelibly shaped its literization. I propose, therefore, that Germanists treat the Carolingians’ engagement with Latin and \textit{grammatica} as a window into the first literizers’ thoughts on language. Moreover, this metalinguistic thinking shaped to some extent how they literized their vernacular. To ignore this sociolinguistic context in favor of, what I maintain is, a myopic interest in structure and competence means that scholars are not simply hampering their own efforts by shrinking their datasets unnecessarily. They are also missing out on the one story that these data can reliably tell: namely, that of the beginning of a literary German. 

A final note on terminology that has theoretical and empirical implications: the reader has perhaps noticed that I have not used the usual terms, “Old High German” or “Old Saxon” to describe early medieval German.  These names reflect a linguistic difference; that is, Old High German texts show evidence of the Second Consonant Shift (or “High German Consonant Shift”), while Old Saxon texts do not. Separating the early German \textit{scripti} into two different categories based on the effects of a sound change, however, obscures the fact that they were all products of a Carolingian documentary culture and the empire’s monastic network. This is why I refer to both as one linguistic phenomenon, that of “early medieval German” or simply “early German.” My definition of the term encompasses the vernacular \textit{scripti} created during this period, but also the many spoken varieties of the vernacular. As scholars of language and literature, German’s few early written testaments are precious to us. However, we must remember that together they constitute the smallest fraction of the linguistic activity that was carried out in German during this period. This means that the vast proportion of this linguistic activity is lost to us: the \textit{scripti}, in that they cannot be transcriptions of a spoken vernacular, are able to provide only indirect evidence of what early medieval German as a broad spoken phenomenon was like.\largerpage

There is good reason to abandon the terms Old High German and Old Saxon. Their use, I propose, has encouraged scholars to treat these first attestations of German as equivalent to the language that occupies the opposite pole of the diachronic continuum: New High German. New High German is a broad linguistic phenomenon, like early medieval German, in the sense that it comprises multiple varieties of language. However, the varieties to which speakers of early medieval German, on the one hand, and speakers of modern German, on the other hand, have access are radically different. First, one prominent variety of modern German is Modern Standard German,\footnote{Modern German is a pluricentric language with more than one standard, including Austrian Standard German and Swiss Standard German. I refer only to Modern Standard German, the standard language of Germany.} while early medieval German has nothing remotely equivalent to this variety. Modern Standard German is a highly literized language, cultivated by generations of people who worked to create a codified written language. In modern Germany, the written standard is central to culture and society; this statement is especially true the further north one is. The superimposing of a written variety on top of the regional varieties has caused dialect change and even loss. The most widespread change has been the emergence of regiolects, regionally flavored oralizations of the standard variety.\footnote{{I have drawn on Roland \citegen{Kehrein2020} contribution to the} {\textit{Handbook of the Changing World Language Map}}{, in which he discusses Germany, “Vertical Language Change in Germany: Dialects, Regiolects, and Standard German." I also relied on Christine Evans’ depiction of regiolects in her \citeyear{Evans2023} dissertation, \textit{Variation, change, and the left periphery: Dislocation phenomena in contemporary northern German varieties} (see page 9).}} \citet{Kehrein2020} characterizes the regiolects as the “predominant varieties of everyday language in Germany today.”

This state-of-affairs leads to a second main difference between modern and early medieval German: while it is reasonable to speak of a Modern German grammar, it is a mistake to do the same for early medieval German. The influence that Modern Standard German has had on written and spoken varieties means that it is theoretically and empirically more viable to assume a uniform syntactic basis for all linguistic varieties and conceptualize structural variation as derivations of that basic pattern. One example of a pattern that structuralists have hypothesized is canonical is the verbal frame; when the surface pattern deviates from the main one, it is called a  “dislocation” or “extraposition” \REF{ex:01:1}.\footnote{These examples are (adapted) from \citet[45]{Evans2023}.}

\ea%1
    \label{ex:01:1}
  \ea          Ich \ul{habe} den Hut in der Stadt gekauft.\\
  ‘I bought the hat in the city.’
  \ex  \textbf{Mein Hut}, den \ul{habe} ich in der Stadt gekauft.\\
  ‘The hat, I bought it in the city.’
  \ex  Ich habe den Hut gekauft \textbf{in der Stadt}. \\
  ‘I bought the hat in the city.’
\z
\z

\ea
\ea \gll ich  habe  den Hut     in der Stadt   gekauft\\
  I  \textsc{aux}    the hat.\textsc{acc}     in the city  bought\\

\ex  \gll meinen Hut  den       habe   ich   in der Stadt  gekauft\\
  the hat.\textsc{acc}   \textsc{dem.pro}   \textsc{aux}    I   in the city   bought\\

\ex   \gll ich   habe  den Hut   gekauft  in der Stadt\\
I  \textsc{aux}    the hat.\textsc{acc}  bought  in the city\\

\z
\z


\noindent The example in (\ref{ex:01:1}a) exhibits the assumed canonical structure of a German main clause: the finite auxiliary \textit{habe} ‘have’ is in clause-second position, while the non-finite verb \textit{gekauft} ‘bought’ occurs after the sentential constituents, \textit{den Hut} ‘the hat’ and \textit{in der Stadt} ‘in the city’ in clause-final position. Together, the finite verb and non-finite verb demarcate the clause’s verbal frame. In both (\ref{ex:01:1}b) and (c), however, a sentential constituent occurs outside of the clause’s main verbal frame. 

In identifying these syntactic patterns as \textit{dis}locations or \textit{extra}positions, we convey through our terminology more than just a neutral theoretical assumption that the verbal frame is the underlying structure and the orderings in (\ref{ex:01:1}b) and (c) are derivations of this pattern. Consider that the surface ordering in (\ref{ex:01:1}a) has been codified as belonging to the standard, while dislocations to the left or right of the verbal frame, (\ref{ex:01:1}b) and (c), respectively, have not. The association of only one of these orderings with the standard has led more prescriptively minded scholars to characterize the non-standard ones as “anomalous” or “deviations from standard language norms” (\citealt{Dewald2012}: 25; see also \citealt{Altmann1981}: 33--34). Relatedly, it has led to diachronic change by exerting a normative pressure on spoken German varieties, which is to say, on individual mental grammars. For example, \citet[207]{Evans2023} finds that while rates of left dislocation are higher in Low German dialects than in the more normative regiolects, there has been a marked decline in the frequency of left dislocations in Low German speaking regions over the last half century as speakers have abandoned their dialects in favor of regiolects. In other words, the growing influence of a standard has reduced or eliminated structural variation in some speakers’ mental grammars. This diachronic change, in turn, bolsters the credibility of the initial assumption that an underlying structure, whose linguistic manifestations vary according to performance factors, exists in the first place. Crucially, it is the association of one surface ordering with the norm and its other orderings with variations of, or indeed deviations from, that norm that can be seen as setting the stage for an analysis of the former as part of underlying grammar and the latter as the product of performance-driven derivations. 

This is to say that I wonder to what extent literization and its later phases of standardization within the context of modern nation states, which have the political and sociocultural prominence to impose language norms on broad swaths of their populace and, thereby, effect language change, have made structuralist approaches to linguistic inquiry seem more viable and empirically supported than they would otherwise be. In this respect, diachronic generative syntax could be seen as a theoretical reification of the normalizing tendencies of nationalism, analogous to the ways in which nationalism projects its modern construction of a national identity backwards onto history.\footnote{This practice reflects one of the three perplexing paradoxes of nationalism described in Benedict \citeauthor{Anderson2006}’s influential \textit{Imagined communities: Reflections on the origin and spread of nationalism} (\citeyear{Anderson2006}): while nations are objectively modern, in the subjective view of nationalists they are ancient (page 5).} In the nationalist discourses of the eighteenth and nineteenth centuries, for example, intellectuals promoted a national spirit that they believed bound all Germans together, even in the absence of a German nation. This national spirit was defined in linguistic, cultural, historical, and, increasingly throughout the nineteenth century, racial terms. As Christopher \citeauthor{Krebs2011} demonstrates in his \citeyear{Krebs2011} book, \textit{A most dangerous book: Tacitus’s} Germania \textit{from the Roman Empire to the Third Reich}, nationalistically minded scholars mined the past~-- with Tacitus’s ethnographic treatise on the northern barbarians of Late Antiquity becoming the \textit{Urtext}~-- for evidence of a distinctly German antiquity.

Jacob Grimm, one of the founding figures of Germanic linguistics and German Studies as an academic discipline, saw his own work as part of this project of defining the German nation \citep[188--189]{Krebs2011}. Along with his brother Wilhelm, he collected folktales and legends in his \textit{Kinder- und Hausmärchen}, first published in 1812 (\cite{Grimm1812}). In line with writers like Johann Gottfried Herder and Johann Gottlieb Fichte, Grimm believed such expressions of so-called “low culture” represented a “people’s spirit.” The German language was the peoples’ “full breath,” while its stories were manifestations of this spirit \citep[189]{Krebs2011}. In this way, Grimm was building on the foundations laid in the seventeenth century by German scholars who began the search for~-- and the construction of~-- a more perfect, ancient, and original (\textit{ursprünglicheres}) German to serve as the standard language for the German people (\citealt{Langer2000,Langer2002}). In my mind, Grimm’s philological work on Germanic takes on a new significance in light of his larger project of national identity construction. In reconstructing the sounds of Proto-Germanic and the German language’s prehistory, Grimm was uncovering what he saw as a distinct and, indeed, discrete German language. Because he believed that a uniquely German spirit and language had always existed, it was a given that one could also identify \textit{a} prehistoric German phonological system. It is not difficult to see how these methods of reconstruction, rooted in the assumption that there is \textit{a} German structure to reconstruct in the first place, can be extended into other areas of grammar, including syntax. 

It is also the case that Grimm described sound changes in teleological terms that related directly to the emerging modern standard. For example, he referred to early medieval Upper German dialects that were affected by all of the Second Consonant Changes as “strict Old High German” (\textit{strengalthochdeutsch}) \citep[13]{Braune2018}. This contrasted with so-called normal Old High German (\textit{normalalthochdeutsch}), said to be best represented in the ninth-century East Franconian translation of Tatian’s \textit{Evangelienharmonie}. The linking of the Tatian text to a future normative variety of language is due to the fact that its consonantal system is closest to the modern Central German dialects on which the eventual standard was based. It strikes me as unlikely that Grimm actually intended to identify the Tatian text as a synchronic normative variety of early medieval German. Moreover, even Wilhelm \citegen{Braune1886} first edition of what would become the standard handbook of Old High German grammar, the \textit{Althochdeutsche Grammatik}, cautions against using terms like “strict Old High German.”\footnote{See page 3 of the grammar’s introduction. Note, however, there is no justification provided for the recommendation. Braune only states that the term was used often in earlier works.} Later editions of the \textit{Althochdeutsche Grammatik}, for example the latest 2018 edition, update Braune’s original introduction to make clearer the fact that the “Old High German dialects” should not be confused with a national language (“Als “deutsch” im Sinne einer Nationalsprache sind die althochdeutschen Dialekte nicht zu bezeichnen,” page 3).

Despite these acknowledgments, the grammar maintains the traditional practice of arranging its word index according to the East Franconian forms of words. The choice is a logical one in that readers who use the reference work are invariably familiar with Modern Standard German. It is also a necessary decision in that an index of this sort requires lemmas to serve as individual entries. Consider, however, the possibility that the practice itself implies that the early form that most closely reflects the modern standard one is a kind of norm. That is, within the context of this index, the East Franconian is indeed \textit{acting} as a norm and will implicitly convey to the modern user, especially the one who did not read the reference work from cover to cover, that a norm existed, as does a diachronic continuity from German’s earliest to most recent attestations. This example also illustrates what I see as an important truth that largely goes unrecognized in the literature on the history of the German language. Namely, that the literization process itself encourages linguists to establish norms~-- even for early varieties that are known not to have had any~-- because our metalanguage requires them. Indexes and dictionaries, for instance, are inventions of literacy, created through literization processes. They require normalization, better yet standardization. They push us not just into thinking about language primarily in structural terms, but into creating those structures in historical varieties \textit{where they did not actually exist}. 

Ultimately, my concern is that modern diachronic linguists have not fully recognized the extent to which their emphasis on structure reflects two influences. The first is an epistemological orientation that was indelibly shaped by the nationalistic ideals of our disciplinary forebears, as well as their drive to establish an authentically German norm. I propose that this influence has unduly shaped linguists’ assumptions about what the goal of diachronic linguistic inquiry should be, which data are important, and why. The second influence is the extent to which our metalanguage, which itself has grown out of literacy, encourages us to seek structure in historical varieties that predate the literization processes that establish those structures in the first place. The arguments I offer here are distinct from the claim that structuralist minded linguists have ignored synchronic and diachronic variation in German. Such a statement would be demonstrably false. As I noted just above, it is widely acknowledged that, despite what the parallel phrasing implies, Old High German is different from New High German. Similarly, there are diachronic generativists who routinely work with corpora of data, which inevitably evince tremendous variation in forms. It is the effects of the literization process and our subjectivity as literate scholars that have not been acknowledged. It is my hope that this book brings these possibilities to the attention of diachronic linguists. 

It is for these reasons that in this book I use the terms “early medieval German” or “early German” to describe what is more traditionally referred to as Old High German and Old Saxon. The first two terms simply refer to a particular time period during which many varieties of German were in use \textit{and} from which we have our first written attestations. When I do use the terms Old High German or Old Saxon it will be as reference to more traditional treatments of early medieval German. This terminological shift, I propose, highlights the fact that the first attestations of German are crucially different from Modern German with respect to literization. Embedded in this proposal is another parallel one: namely, that the degree to which a language has been literized should be foremost in the diachronic linguist’s mind, not just because literization has a transformative effect on language but because our subjectivity as highly literate people presents us with a perpetual challenge when investigating data from the earliest periods of a language’s literization. In other words, I propose that we cultivate an awareness of literization in our practice of diachronic linguists. For me, at least, thinking of early medieval German, rather than Old High German and Old Saxon, as my object of study helped in this endeavor. I was better able to move beyond old patterns of thinking that had in my previous work sent me off primarily in search of the grammatical intuitions (i.e., the mental grammars) of speakers who lived over 1,200 years ago. It was, moreover, easier for me to maintain a scholarly focus on the \textit{scripti} themselves as the material artifacts of a multilectal linguistic community in a Carolingian sociocultural context.\footnote{{My engagement with the question of how epistemological orientations can be conveyed through their terminology and affect the scholar’s work is analogous to, and inspired by, the ongoing discussion in early English studies about the field’s widespread use of “Anglo\hyp Saxon” to refer to the people, language, literature, etc. of pre-Conquest England. The term itself came into prominence as part of the eighteenth- and nineteenth-century discourse on racial Anglo-Saxonism, which “portrayed the white English race as a transhistorically superior one” (\citealt[135]{Rambaran-OlmWade2022}). The English medievalists of the past were interested in establishing a national history that supported contemporary colonialist and imperialist agendas by constructing the supposed medieval origins of an imagined Anglo-Saxon people, language, and culture (see \citealt{Rambaran-OlmWade2022}:} {\textit{passim}}{). Present-day medievalists who adopt such terminology risk perpetuating an inaccurate narrative of the early English and worse, Rambaran-Olm and Wade note, the racism and white supremacy that underlay these ideas. In his book,} {\textit{Barbarian Tides: the Migration Age and the Later Roman Empire}}{, Walter \citet{Goffart2006} presents a similar argument against the uncritical use of Germanic in German Studies. This term, like Anglo-Saxon, is anachronistic and gives the impression of a unified ethnicity and culture where nothing of the sort existed. Moreover, it was created to serve nationalist agendas and has inspired scholars to search for~-- or more accurately, construct~-- evidence of their glorious Germanic prehistory (see Goffart’s introduction).}}

I close this introduction with a précis of this book’s six chapters, excluding the \chapref{sec:chap:8} conclusion. I recommend that the reader read them in order and not skip around, for each chapter elaborates the book’s arguments in cumulative fashion. In \chapref{sec:chap:2}, I describe the ways in which linguists have approached the study of early German syntax, a methodology I call the deficit approach. The deficit approach grows out of the long-standing structuralist assumptions that diachronic linguists’ main object of study ought to be the native speaker intuitions~-- or structures~-- that underlie historical attestations. There is another way to characterize this assumption: namely, that native speaker intuitions are the most significant factor that determines the shapes \textit{scripti} might take. Though structurally minded linguists may well acknowledge the multilectalism of early medieval German speakers, their methodologies for analyzing their innovative \textit{scripti} do not operationalize this fact. That is, generative and other structural accounts of early medieval German texts have focused on identifying the confounding influences, factors like a poetic meter or Latin borrowings, that effected attested variation. Because the deficit approach is often an implicit methodology, that is, scholars adopt its assumptions without either realizing or discussing this fact, I describe what this approach is and where it is evident in the literature on early German syntax. I argue that the deficit approach does not incorporate the multilectalism of medieval speakers into its analytical model. Moreover, I propose that it is based on traditional and, indeed, outdated notions of what prose is, how grammatical the linguist can assume this genre of writing to be, and what its relationship is with poetry.

\chapref{sec:chap:3}’s main purpose is twofold. First, I present evidence that supports my characterization of the early medieval German \textit{scripti} as isolated and idiosyncratic acts of literization. For this argument, I draw on literature from outside the field of historical German syntax and discuss the history of orality and literacy in early medieval German-speaking Europe and Carolingian documentary culture. This discussion will convey to the reader a better sense of the sociocultural context of the Carolingian empire and what it meant to write in the vernacular in this environment. The second goal of \chapref{sec:chap:3} is to consider what it means to literize German in this context from a theoretical point of view. Specifically, I argue that developing a \textit{scriptus} for a vernacular that is still mostly or exclusively oral requires linguistic innovation on the part of the \textit{scriptus} creator. By innovation I mean that the vernacular writer must create linguistic structures that were not~-- and indeed, could not have been~-- present in any of its spoken varieties or vernacular mental grammars. Though the first literizers of a language can and certainly do draw on their linguistic intuitions, none of these intuitions had ever before contended with, or been shaped by literacy as a conceptual category, what \citet{KochOesterreicher1985} call “language of distance” (\textit{Sprache der Distanz}). In other words, the early literizer cannot create a functional \textit{scriptus} by simply transcribing one of their spoken varieties into the graphic medium, even if transcription were technologically possible.\footnote{I discuss the practicalities of transcription in an early medieval context in \chapref{sec:chap:5}.} The new context of literacy demands more than a medial transfer of the vernacular from the phonic to the graphic. That is, literization effects lexical and grammatical change.

In \chapref{sec:chap:4}, I elaborate on the nuts and bolts of \textit{scriptus}{}-creation. Drawing on the concept of language “ausbau” from the works of Heinz Kloss, I discuss how a literizer must adapt the vernacular’s lexicon and syntax so that it can be functional in the graphic medium, which has the potential to dislocate a linguistic production utterly from the language producer and the moment of production. This possibility requires literizers to augment the grammatical and semantic coherence of the vernacular, a process that I also refer to generally as closing the functional gap between the oral varieties of distance and the more extreme distance of the written medium. These processes are universal in the sense that all exclusively oral vernaculars require ausbau in order to become functional written languages. There is, however, plenty of room for variation in how literizers solve the problem of ausbau.

\chapref{sec:chap:5} aims to provide a basis for understanding why early medieval German \textit{scripti} vary structurally and lexically. In it I demonstrate how scholars might piece together evidence elucidating the ausbau choices a particular literizer made with their \textit{scriptus}. Early medieval authors are largely anonymous. Thus, I outline a set of guardrails, as it were, that is relevant to the sociocultural environment of East Francia and implies certain ausbau possibilities. One guardrail is provided by the oral varieties of distance that make up the oral tradition or “elaborated orality.” The other guardrail is the Latinate tradition of literacy, which formed the basis of a Carolingian education. Literizers, I argue, would have been educated in the descriptive and prescriptive norms of Latin, encompassed in the Latin-language term \textit{grammatica}. Latin and the linguistic varieties of the vernacular, particularly the oral tradition, thus constitute the two main linguistic resources on which a \textit{scriptus}{}-creator could draw. With the two ninth-century Gospel harmonies, Otfrid von Weissenburg’s \textit{Evangelienbuch} and the \textit{Hêliand} as my sample texts, I demonstrate how one may establish authors’ different orientations toward these two resources, an analysis that can support further investigation of the \textit{scriptus} itself. In this chapter, I also return to the question of whether any early medieval \textit{scriptus} can be a transcription of spoken language and provide evidence that the practicalities of this task render such an eventuality inconceivable.

In \chapref{sec:chap:6}, I begin to lay out the implications that \chapref{sec:chap:5}’s analysis has on the shape a \textit{scriptus} may take. Specifically, if a literizer was oriented toward their oral varieties of distance, that is, the language of the oral tradition, this chapter explores how this influence will be reflected in the syntactic structure of the \textit{scriptus}. In order to investigate how these varieties might affect syntax, we require a method for identifying the structural characteristics of linguistic varieties that no longer exist. Scholars generally approach the challenge of reconstructing German’s prehistory armed with the comparative method. This methodology is particularly useful for identifying the phonological and morphological characteristics of a prehistoric German. However, because the method requires working backward from historical \textit{scripti}, that is, written varieties created through literization and some degree of ausbau, I argue that it cannot elucidate the structures of an \textit{oral} variety of distance. Instead, I propose that we may better understand German’s prehistoric syntax, which was necessarily a \textit{spoken} syntax, if we consider the central challenge that processing distance varieties in the phonic medium poses. Namely, if spoken language is ephemeral, and the focus of human consciousness moves restlessly from one topic or idea to the next, how do people process a language of distance that requires coherent, progressively organized, and planned discourse?

The answer to this question is a collection of strategies I refer to as “layered elaboration,” the use of which, I maintain, ameliorates this challenge. They involve the elaboration of constituents or ideas through overlapping reverbalizations that layer more detail on top of the original idea. Reverbalizations are linked in discourse through the use of semantic and/or structural parallelism, as well as through deictic connectors. These strategies leave their structural imprint on \textit{scripti} whose creators drew particularly on the linguistic resources of their oral vernacular in order to evoke the (language of the) oral tradition. In fact, the Greeks, Aristotle in particular, already identified this mode of expression and called it unperiodic syntax. In his treatment, he does not recognize its particular characteristics as psycholinguistically determined, as my concept of layered elaboration intends to do. Instead, he associates his unperiodic syntax with a certain literary \textit{style}, that of the ancient poets. This style, Aristotle argues, is inappropriate for the prose style of composition he is developing. 

The purpose behind exploring how classical discourse on language associated layered elaboration with a particular literary style becomes clearer in \chapref{sec:chap:7}. It is in this chapter where I outline two important conceptual changes that must accompany literization: the cultivation of notions of well-formedness and literary style. The idea of well-formedness connects to my earlier argument that the early German \textit{scripti} cannot be transcriptions. That is, the literizer cannot simply consult their linguistic intuitions because, as I discuss in Chapters~\ref{sec:chap:3} and~\ref{sec:chap:4}, their spoken vernacular intuitions have never before had to contend with the new written context of distance. Thus, \textit{scriptus} creation requires that the literizer develop for their vernacular more explicit means of indicating the syntactic and semantic relationships between constituents, a process that I characterized in \chapref{sec:chap:4} as ausbau. Engaging with the project of ausbau, however, requires that the literizer become aware of the demands of conceptual literacy or, to phrase it another way, the functional gap between even their most coherent, distance-shaped oral varieties and a well-functioning written variety. Thus, ausbau and a conceptualization of well-formedness are complementary processes.

The development of a notion of literary style, on the other hand, refers to how a literizer leverages their linguistic resources~-- their multilectal spoken vernacular, as well as the Latinate tradition of literacy~-- in order to execute ausbau. I submit that they will not engage in this process in the manner of the desultory diner, who samples randomly from the dishes on offer at an all-you-can-eat buffet. Rather, the literizer will make selections based on what they find to be appropriate in the context of their project, as discussed in \chapref{sec:chap:5}. Appropriateness in the \textit{scriptus} will be determined by the associations that the literizer has formed between linguistic production and the contexts that invariably shape it. For example, early literizers would connect the structures of layered elaboration to the communicative contexts of distance within an oral community; the former is, thus, appropriate to the latter. This association becomes a conception of literary style when the literizer creates a \textit{scriptus} that is suited to the telling of, say, a story that stems from their oral community. There is a larger argument that underlies this idea: namely, that even the earliest literizations of a language are indelibly shaped by considerations of appropriateness or literary style. In structuralist approaches, generative syntax in particular, literary style is a matter of performance. Thus, diachronic generativists’ interest in questions of appropriateness and style will be limited to how one controls for their potentially competence-obscuring effects on mental grammars. I argue, in contrast, that questions of well-formedness, appropriateness~-- or literary style~-- are so integral to the literization process itself and the \textit{scripti} it produces, that it makes little sense to try to disentangle these constitutive factors from those provided by speaker intuitions.

The two conceptual developments of well-formedness and style are related in that the decision to rely more on oral varieties of distance for the creation of a \textit{scriptus} will suggest different, and, I argue, less desirable and more ambiguous ausbau outcomes. Creating a \textit{scriptus} in the style of the oral tradition also places the literizer in opposition with principles of composition that emerge in the classical discourse on \textit{grammatica}. Aristotle, for example, passes a negative judgment on an overly “poetic” style. We may recognize the style of the ancient poets as layered elaboration and cognitively advantageous in the phonic medium, but for Aristotle it amounted to poor style for composition in which clarity of expression is paramount. One might not think that classical notions of well-formedness and literary style are relevant to the creation of an early German \textit{scriptus}. I maintain that they are, however, in that aspiring German writers first encountered literacy in their Latinate education, which was mediated entirely by the Bible and by various treatises that were part of the classical tradition of \textit{grammatica}. Through implicit and explicit means, these texts conveyed to German literizers Greek and Roman constructions of well-formedness and literary style, both the results of already advanced processes of literization and ausbau. The implicit means refer to the highly literized Latin language in which grammatical treatises and, more importantly the Bible, were composed. The grammatical treatises with their descriptions of well-formed sentences constitute the explicit means by which German literizers were exposed to what classical writers thought were appropriate solutions to the challenges of lexical and grammatical ausbau in different contexts. For example, I discuss in this chapter how Otfrid von Weissenburg, the author of the \textit{Evangelienbuch}, finds inspiration for his \textit{scriptus} in these classical notions of well-formedness. This chapter presents this book’s final and, perhaps, most controversial proposal, which is that modern linguists have mistaken early literizers’ adoption of classical constructs of ausbau and associated notions of well-formedness for an inherent grammatical structure. More generally, I propose that the structural similarities that linguists have identified across European languages might also reflect the widespread influence that classical literacy exerted on their literization. 

