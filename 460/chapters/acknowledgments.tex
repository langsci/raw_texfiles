\addchap{\lsAcknowledgementTitle} 


This book’s inciting idea grew out of a course that I developed during the first year of my appointment at the University of Wisconsin–Madison, academic year 2018--2019. I shamelessly called it “Barbarian Language and Culture,” in the hopes of attracting more students with a title that alluded to epic tales of battles and heroes. Preparation for the course introduced me to literature on the topic of orality and the oral tradition. I started wondering why I had never considered orality in a more systematic way before. In that my research had focused on German’s first attestations, which were produced when the language was still an almost exclusively oral phenomenon, I felt remiss in having neglected this relevant context for so long. Thus, a slow process of reconceptualizing what early German was began. It has been a long, challenging, and intellectually stimulating journey. The COVID-19 pandemic made it more difficult and, at times, isolating than it needed to be. But there were many shining lights in my life that illuminated my path and, thus, made this book possible.

I begin my thanking my colleagues in the German, Nordic, and Slavic+ Department, especially, Jolanda Vanderwal Taylor, Mark Louden, and Thomas DuBois for their support in navigating the not-insignificant bureaucratic aspects of these initial years of my appointment at the UW. I also thank my mentors, Kirsten Wolf and Monica Macaulay, who routinely checked in with me and gave me excellent (work and life) advice.

With respect to the development and execution of the book, I am grateful to Hannah Eldridge and Sabine Gross, who organized and invited me to present at the Wisconsin Workshop in the fall of 2019. This event and the feedback offered by its participants helped me turn my inciting idea into a more concrete analysis. Thanks belong also to the graduate students of the German section in GNS+, especially those who were in my Orality and Literacy course in fall 2022. This course was another incubator for my book project in that its students brought incredible energy and creativity to the class and their own projects. What a fun experience that was! I am also grateful to the anonymous reviewers for Open Germanic Linguistics, who engaged with my work rigorously, with an open mind, and on a tight timeline. Thank you so much for your extensive feedback, which did so much to help me clarify my arguments. This book is better because of you.

This penultimate set of thanks belongs to people who supported me and the work during (what ended up being) trying times. First, I thank my friend and mentor of more than two decades, Rob Howell. It’s difficult to put into words the significance of Rob’s influence on me as a scholar and a person. With respect to this particular project, Rob spent hours letting me talk through the project, listening carefully, asking questions to help my ideas become more coherent. Rob, I am grateful for your friendship, support, and delightful company. I also thank my friends, Hieyoon Kim, Wendee Gardner, Christine Evans, and Melissa Sheedy. Hieyoon and I wrote together every morning (via FaceTime) during the COVID lockdowns of 2020, and she has continued to be a staunch advocate for my mental health, even when I ruthlessly attack it with negative self-talk. Wendee, in addition to reading and commenting extensively on my introduction, has also encouraged me to bring more balance into my life, to trust my intuitions, and to celebrate my achievements. Christine Evans and I were in parallel situations in 2023. We were both working at a near frantic pace, she to complete her dissertation, I, a draft of this manuscript. I found solidarity in that, as well as comfort in her unstinting kindness and support. Speaking of unstinting kindness and support, thank you, Melissa Sheedy, for all the many, many ways you have encouraged me and this project over the last several years. You inspire me to be a better colleague and friend, though, I have to admit, you are a tough act to follow.

Finally, I thank my parents, Bob and Barbara Somers and my sister, Nina Williamson, who have been my persistent cheerleaders. I also thank, in the most heartfelt way a person can, my husband, Matt Hermann. It is not an overstatement to say that I would not be where I am right now and doing the things I’m doing without you. You supported me in countless ways and I am so grateful and happy to have you as my partner. Thank you for everything.
