\chapter{Creative literizations in the absence of a vernacular writing tradition}\label{sec:chap:3}

\section{Introduction}
In \chapref{sec:chap:2}, I identified what I call the deficit approach to the early German corpus and connected this methodology to the structuralist orientation of our disciplinary forebears. Since its inception, historical linguistics as a discipline has sought to identify the structures of an authentic German grammar or “competence,” to use a more modern term. The deficit approach has led diachronic structuralists to winnow down an already small corpus to something even smaller, say, one text whose data they believe are most reflective of this competence. This competence is imagined to be a neutral or ordinary German of the sort produced in everyday speech. In contrast, data from other texts seen as negatively impacted by confounding factors like a poetic meter are largely ignored or analyzed as indicative of a peripheral, e.g., poetic, grammar. Far more consequential is the fact that diachronic linguists see only the delineation of an early German competence as the goal of analysis. This long-standing orientation has made it challenging for scholars to move beyond the mental grammar as their analytical object and treat the first German texts as \textit{scripti}, that is, as the first material artifacts of a long process of vernacular literization.

  In this book, I offer an alternate narrative, that is, the literization approach. Namely, the German speakers who wrote in the vernacular in early medieval East Francia, mostly members of the clergy, were multilectal and multilingual. They each have multiple linguistic varieties that they could engage in order to create their \textit{scriptus}. Among these varieties is a multilectal spoken German, but also Latin, the only language in which these first literizers were literate. What they lacked was any sort of vernacular writing tradition on which to draw. They needed to create one first. This fact means that each literization resulted from a writer drawing on their linguistic resources in ways that were commensurate with their own linguistic and educational background and the goals they had for their particular writing project. Thus, one should expect to see variety across the different literizations.

This alternate narrative implies an alternative approach to the study of early German. It allows scholars to break free from the assumption that identifying an Old High German grammar, as many previous studies claim to do, is the only worthwhile scholarly pursuit. Instead, we might see the first textual evidence of German as a constellation of idiosyncratic \textit{scripti}, each the product of the literizer engaging their linguistic resources in a unique way. Each \textit{scriptus}, however, will also reflect a creative engagement with the vernacular. That is, the writer must shape their vernacular into a written form, which, I argue, places entirely new demands on it. In other words, none of the varieties of the early German oral vernacular had ever before had to contend with the new dislocated communicative domain of the written word, which had the potential to disconnect entirely the linguistic production from its creator and receiver. Thus, literizers must also stretch their vernacular beyond its existing conventions, or their linguistic intuitions, as generativists would say, and creatively adapt the vernacular so that it can be functional in this new communicative context.

In this chapter, I justify my characterization of early medieval German as an almost exclusively oral vernacular, and its first literizations as acts that demand linguistic creativity from the \textit{scriptus} creator. In \sectref{sec:3:3.1}, I first demonstrate the absence of a \textit{vernacular} writing tradition in Carolingian Europe and then, detail the sociocultural barriers that inhibited its development. These discussions underpin the conclusion that German’s first attestations were isolated, idiosyncratic, and produced in a culture that was, at best, indifferent, at worst, hostile toward its own vernacular. I tackle these topics in some depth and draw on literature from fields outside of historical syntax, primarily the history of medieval orality and literacy and Carolingian documentary culture. The sources I discuss in this section do not often find their way into the bibliographies of published works on Germanic and early German syntax. This makes a certain amount of sense given diachronic linguists’ enduring interest in identifying an early German grammar or competence and the attendant belief that sociocultural factors are relevant only to performance. But I maintain that this sociocultural context is crucial to our understanding of the early \textit{scripti} and so, I dedicate this sub-section to discussing them.

In \sectref{sec:3:3.2}, I elaborate my argument that these first literizations required linguistic creativity from its literizers. That is, these \textit{scripti} are not and could not have been simple transcriptions of some variety of the literizer’s oral vernacular. I rely here on \citegen{KochOesterreicher1985} framework of conceptual orality and literacy to demonstrate that the literization of an oral vernacular involves not simply a medial transfer of a phonic linguistic phenomenon into the graphic medium. It also requires a whole conceptual reworking of these phonic varieties, as literizers work out how to make the phonic vernacular functional in the dislocated context of the page.

\section{Innovative \textit{scripti} in German-speaking Francia}\label{sec:3:3.1}

Vernacular culture was overwhelmingly oral in ninth-century German-speaking Francia and would remain so for centuries to come. The Carolingians had established a documentary culture, but literacy was Latinate with only minimal encroachment on the vernacular. Comparatively fewer resources were devoted to developing and maintaining vernacular \textit{scripti}. I argue in \sectref{sec:3:3.1.1} that none of the \textit{scripti} that emerged in the eighth and ninth centuries were adopted to any significant extent outside of the limited context in which they were created. Furthermore, these \textit{scripti} did not grow out of an interest in literizing the vernacular but were created in the service of making Latin texts more accessible to German-speakers. That is, the preponderance of vernacular writing during the period was a by-product of the Carolingian project of Latin education and reform, not the result of a concerted effort to literize the vernacular. In \sectref{sec:3:3.1.2}, I argue that there were significant barriers to the vernacular’s literization across the empire. Those few texts that \textit{do} constitute an author’s conscious engagement with vernacular literization, like the \textit{Hêliand} and \textit{Evangelienbuch}, were written in an indifferent, or even hostile, cultural environment that offered no significant vernacular written tradition that could guide their linguistic output. The \textit{Hêliand} poet and Otfrid had to work out literization for themselves. That is, their poems are written in two idiosyncratic and innovative vernacular \textit{scripti}.

\subsection{Literacy in Carolingian Europe}\label{sec:3:3.1.1}

For this brief introduction to the history of literacy in Carolingian Europe, I draw extensively on \citegen{Green1994} monumental \textit{Medieval Listening and Reading}, especially its Chapter 2 (and more specifically, pages 35--39). Literacy arrives through contact with Mediterranean cultures. Of particular interest are the interactions between the Germanic-speaking tribes and the Roman Empire during late Antiquity that results in the northern barbarians learning to speak, and in some cases, write Latin. \citet[Chapter 7]{Goffart2006} details the ways in which barbarians, including ones who spoke varieties of Germanic, not only joined the Roman army, but rose to high positions in its (post-Diocletian) military aristocracy and married Romans. Though these men were valued primarily for their prowess on the battlefield and generally did not have any background in the liberal arts, the ability to read and write Latin would have been, as \citet[36--37]{Green1994} puts it, “indispensable.” Additionally, many and diverse barbarian tribes during this period had extensive contact with Latin speakers in and around the frontier region of Europe that Romans called \textit{Germania}, sometimes as enemies, sometimes as \textit{foederati}, that is, as allies of Rome. The many words that Germanic-speakers borrowed from Latin before the occurrence of the Second Consonant Shift, i.e., \textit{Pfirsich} ‘peach’ and \textit{Pflaume} ‘plum,’ is evidence of this early linguistic contact.

The next significant development that promoted literacy among Germanic-speaking peoples was their conversion to Christianity. This process began already with the Goths, who became Arian Christians starting in the mid-fourth century. It is, however, not until after the locus of power in Western Europe shifted away from the Italian peninsula that Germanic-speakers began to convert to orthodox Catholicism. Clovis I (ca. 486--511), who initiated Frankish territorial expansion in earnest, famously converted to Catholicism in 496, thereby linking the salvation of Germanic-speakers to the Latin language. Rome was, after all, still the center of the Catholic Church, and the language of this church was Latin. Most importantly, the dominant versions of the Bible at the time~-- the Vulgate mainly, but the \textit{Vetus Latina} remained important~-- were in Latin. As a religion of the book, it was the Bible that provided the map to heaven, a fact that ensured a continued interest in the Latin language. Christianity later introduced a writing culture to German-speaking regions. In the seventh and eighth centuries monasteries largely following the Rule of Saint Benedict were founded. Reading and writing practices were embedded in this monastic culture; they were entirely Latinate (see \citealt[39]{Green1994}).

The educational reforms of the so-called Carolingian Renaissance instituted under Charlemagne’s reign in the late eighth and early ninth centuries provide evidence of a continued preoccupation with the Latin language in German\hyp speaking Europe. As \citet[17--19]{Brown1994} explains, Charlemagne’s \textit{Admonitio generalis}, legislation that was promulgated in 789, makes clear the importance of literacy and learning in the empire. In Charlemagne’s estimation, the emperor’s most important duty was to save souls. To this end, he required a clergy that was capable of carrying out this task. These clergy would need access to reliable copies of important Christian texts and the ability to read and understand them. The \textit{Admonitio generalis}, thus, directed that schools be established in monasteries and cathedral churches in order to train clergy, as well as \textit{scriptoria}, which would produce the necessary texts.

Charlemagne was also concerned with the state of written Latin, which over the previous centuries had undergone considerable change through interference from spoken Romance \citep[162--163]{Zeller2021}. By the year 700, the state of the written language was “completely chaotic” (\citealt[31]{Norberg1968}, quoted in \citealt[166]{Zeller2021}). This linguistic situation was seen as antithetical to the advancement of an orthodox Christianity, which required “uniformity of faith” \citep[20]{Brown1994}. Uniformity of faith required uniformity of observance, which itself depended, not just on widespread access to texts, but those texts being identical and accurate. Thus, in his \textit{Epistola de litteris colendis} (‘Letter on the cultivation of learning’) to Abbot Baugulf of Fulda, composed sometime in the last decades of the eighth century, the emperor emphasized that correct Latin was crucial to Christian learning \citep[20--21]{Brown1994}. Through it one could “unlock the mysteries of the scriptures” contained in the Latin Bible \citep[20]{Brown1994}, as well as guard against doctrinal error and heresy. To the Carolingians, establishing standards of correct Latin was a matter of eternal salvation or damnation. The stakes could not have been higher.

So, it makes sense that under the Carolingians there was a renewed interest in the classical tradition of \textit{grammatica}, or ‘the art of grammar,’ which went beyond learning to read and write. It primarily entailed the study of literature, understood as “the science of the things said by poets, historians and orators; its principal functions [were]: to read, to write, to understand and to prove” (\citealt[37]{Brown1994}, from Marius \citegen{Victorinus1967} \textit{Ars Grammatica}, pages 65--66). The art of grammar provided students with the tools to uncover the many layers of meaning in the Bible as part of Christian exegesis; it also made more accessible to them the Latin-language commentaries of the early church fathers \citep[37]{Brown1994}.

This developing literary tradition and engagement with a written language, however, is an entirely Latinate phenomenon. That is, there is no good evidence of a widespread, concerted effort to literize the vernacular in the early medieval period. For example, imperial directives along the lines of the \textit{Admonitio generalis}, calling for the clergy to cultivate a \textit{German} literacy, are not extant. Charlemagne’s biographer, Einhard, could be seen as contradicting this statement. He reports in the \textit{Vita Karoli} \textit{Magni} that the emperor “began forming a grammar of the ancestral language” (\textit{inchoavit et grammaticam patrii sermonis}, 29, \citealt{Einhard1880}). This passage gives the modern reader the impression that, not only did German already have a written form, but speakers had even begun applying the same sort of linguistic analysis to this new variety that they applied to the Latin language~-- that they were able to identify and discuss regularities in structure and agree on prescriptive norms. \citet[516--520]{Matzel1970}, however, warns against such an anachronistic interpretation. Drawing on \citet{Grundmann2019} [1958], Matzel argues on page 519 that “grammar" cannot be understood as anything resembling a modern descriptive or prescriptive grammar of a language. As \citet[49]{Green1994} observes in his own discussion of Matzel’s work, a project of this nature would have been beyond the capabilities of vernacular writers of this early period. German, in fact, had no literary form, and it is this deficit that Charlemagne is addressing; it was “the emperor’s concern [to make] the vernacular subject to the same rules of written language as Latin, [by] making the vernacular \textit{capable of being written}”\footnote{The early medieval conception of linguistic rules should not be confused with a modern descriptive one. That is, rules were seen as applicable only to the written language.} (\citealt[49]{Green1994}, emphasis added).

Whatever Charlemagne’s level of concern for the state of vernacular actually was, the Carolingian literary apparatus remained primarily engaged with Latin and did not undertake any coherent project to literize German. This focus on Latin, and inattention toward German, is reflected in two facts. First, monastic time and energy were directed toward the production, copying and preservation of Latin, not vernacular, texts. Second, the comparatively small number of vernacular texts that are part of the early medieval corpus were generally, though not exclusively, produced in the service of understanding Latin better, rather than developing a vernacular \textit{scriptus}. The one notable exception to this statement, the eighth-century translation of Isidor’s \textit{De fide} \textit{catholica contra Iudaeos}, has hardly any influence on texts produced outside of the immediate ecclesiastical network in which it was produced.\footnote{See also \sectref{sec:2.2.2} for a discussion of the Isidor text.}

In order to elaborate on the first point, I turn again to \citet[49]{Green1994}, who notes that while original vernacular poetry may be of great import to Germanists, these texts survive in relatively few manuscripts or fragments, whereas the Latin \textit{Waltharius} is preserved in twelve manuscripts, Einhard’s \textit{Vita Karoli} in eighty, and the \textit{Historia Langobardorum} of Paulus Diaconus in about 200. Even writers in Charlemagne’s own orbit at court “composed hundreds of lines of Latin \textit{carmina}, but not one of them composed a line in the vernacular.” Green also cites \citet{Kuhn1959}, who likens the relationship between vernacular and Latin texts to the image of isolated islands~-- experimental German-language texts~-- floating in a sea of Latin-language literature. In short, there is an absence of evidence supporting the conclusion that there was any noticeable response to Charlemagne’s reported wishes to see German literized. As \citet[34, note 2]{Somers2021b} argues, these facts raise two possibilities: that these vernacular texts never existed in the first place, or that they existed at some point, but no one cared about them enough to preserve them. Either possibility points to the literization of German not mattering much to those Carolingians who were engaged in the empire’s documentary and literary culture.

This conclusion is also reflected in the fact that the preponderance of texts that Carolingians produced were glosses and translations, and that these text types were created to foster Latin literacy among the clergy, not to cultivate a literary German \citep[49--50]{Green1994}. For example, there are “weighty volumes” (page 49) of early medieval glosses that were compiled for use in monastic schools, where the vernacular was simply a means to teach Latin. The interlinear translations in texts like the eighth-century \textit{Benediktinerregel} and the \textit{Murbacher Hymnen} do not, and indeed are not meant to, constitute “consecutive text[s] in the vernacular” (page 49), but instead help the reader understand the construction of the Latin original. \citet[98]{Masser1989} calls such texts “bilingual” texts in that they exist alongside, and are subordinate to, a Latin text. Unlike autonomous writing, they are tied to their original Latin text, deriving their function from that text and continually referring back to it.

\citet[49--50]{Green1994} places the translation of Tatian’s \textit{Evangelienharmonie} into this category as well, noting how it is often a close translation of the Latin and that markers of textual organization, like the use of capital letters to divide sections, are present in the Latin text, but not its German equivalent. To this argument, one could add the fact that the German is copied alongside the Latin, and that the translators maintained the syntactic transparency of an interlinear gloss by having each line of the German translation correspond to a line of the Latin, a fact that is lost in the standard \citet{Sievers1961} edition but is made clear in \citegen{Masser1994} edition.

\ea%1
\label{ex:3:1}
\parbox[t]{\linewidth}{%
      \vspace{-.7\baselineskip}%
      \begin{tabular}{@{}ll@{}}
      Et cogitabat intra se dicens  &  Inti thahta innan imo sus quędenti\\
      quid faciam quod non habeo    & uuaz tuon thaz \textbf{ih} ni haben\\
      quo congregem                 & uuara \textbf{ih} gisamano         \\
      \textbf{fructus meos}         &  \textbf{mine uuahsmon}            \\
      \end{tabular}
}\medskip\\
He thought to himself, ‘What shall I do? I have no place to store my crops.’ (Luke 12, 17. King James Version)\\
\gll  Inti  thahta   innan  imo   sus     quędenti uuaz  tuon     thaz    ih   ni   haben uuara   ih   gisamano  mine  uuahsmon\\
and  thought  to     himself  thus   saying what  do.\textsc{1.sg.}    that    I  \textsc{neg}   have where  I   store.\textsc{pres.subj} my     crops\\
\glt ‘And thought to himself saying thus, what will I do in that I do not have a place where I can store my crops’ (\citealt{Masser1994}, 170, 3--6)
\z

\noindent The syntax of the translation is largely guided by the syntax of the original. As \citet[21]{DittmerDittmer1998} detail, the translators maintained the order of sentential constituents set by the Latin, though would re-order within sentential constituents. See, for example, \textit{fructus meos}, and \textit{mine uuahsmon} in line 4 in \REF{ex:3:1}. The translation of the constituent “my crops” is confined to the same line as the Latin equivalent~-- this, despite the fact that placing it before the verb \textit{gisamano} would have more clearly marked the clause as subordinate~-- though the translators did reorder the possessive determiner and its head noun. The same study also notes how readjustments in any ordering of constituents generally involved two constituents swapping places, and that this swapping was likely to involve at least one function word (page 21--22 and 23).

\ea%2
    \label{ex:3:2}(Tatian 156, 2, \citealt{Sievers1961})\medskip\\
   \parbox{.55\linewidth}{
      quid \textbf{fecerim} \ul{uobis}\\
      \gll quid  fecerim           uobis\\
           what   do.1.\textsc{sg.perf.act.subj}    you.\textsc{ab.pl}\\}
\hfill\parbox{.4\linewidth}
  {uuaz ih \ul{iu} \textbf{teta}\\
\gll uuaz   ih   iu         teta\\
what  I  you.\textsc{dat.pl.}   did\\}
\glt ‘what I did to you all’
\z

\noindent In other words, the translators were loath to rearrange lexical constituents with other lexical constituents. In \REF{ex:3:2}, \textit{fecerim uobis} is switched around, yielding \textit{iu teta} in the German translation.

These facts are consistent with Green’s characterization of the Tatian as a bilingual text and indicate that its literization of German is incidental to the project of making the Vulgate version of the \textit{Evangelienharmonie} more comprehensible to German speakers. The translator’s\footnote{A reminder that, though scholars are not sure how many translators were responsible for the German translation of the Tatian text, I refer to one translator for simplicity’s sake.} main goal was to clarify the Latin constructions, not create a systematic vernacular \textit{scriptus}. To reiterate, my argument is not that the Tatian data have no relevance to the study of early German syntax. In fact, \citet{DittmerDittmer1998} make a compelling case that the translators made systematic, if generally limited, changes to the Latin syntax. It also seems eminently plausible that the deviations from the source text were intended to render the translation more legible to German-speaking readers, thereby clarifying what the Latin original actually \textit{means}, not just what each individual word \textit{says}. In this way, the deviating structures, or \textit{Differenzbelege}, as they are called in the literature (see \chapref{sec:chap:2}), could well be evidence of some early German structural tendency to have possessive determiners occur before their noun phrase. However, this argument is complicated by the fact that the Tatian translation constitutes neither an independent German \textit{scriptus}, nor a simple medial transfer of some variety of the spoken vernacular into writing.\footnote{I elaborate on the impossibility of a simple medial transfer of a spoken vernacular into writing in \sectref{sec:3:3.2} and \chapref{sec:chap:4}.} This point is important because it means that any syntactic systematicity a modern researcher identifies could well be attributable to the act of translation itself. That is, the translator’s focus was on how best to translate Tatian’s gospel harmony so as to elucidate for the German-speaking reader the original’s structure and meaning. Their goal was not to create an optimally authentic-seeming or grammatical German \textit{scriptus}. If it had been, they would not have hewed so closely to the source text in both the text’s translation and presentation.

Scholars have largely not recognized this reality and have treated any structural systematicity found in the translation as evidence of an authentic German, to recall Fleischer’s terminology (\chapref{sec:chap:2}), which is to say, some undefined variety of ninth-century spoken German. An example of this type of analysis is \citegen[chapter~6, 303--306]{Axel2007} treatment of \textit{pro}{}-drop. Returning to the glossed excerpt in \REF{ex:3:1}, and repeated in \REF{ex:3:3}, note another difference between the Latin original and its translation: Sometimes the translators rendered Latin \textit{pro}{}-drop, that is, the dropping of subject pronouns, with overt subjects in the German translation.

\ea%3
\label{ex:3:3}\begin{tabular}[t]{@{}ll@{}}
quid faciam quod non habeo & uuaz tuon {\longrule} (a) thaz \ul{\textbf{ih}}  (b) ni haben   \\
quo congregem              & uuara \ul{\textbf{ih}} (c) gisamano                                \\
fructus meos               & mine uuahsmon                                                 \\
\end{tabular}
\z

\noindent The translators added the pronoun \textit{ih} ‘I’ twice, in (b) and (c), but not in (a). Studies like \citet{Eggenberger1961} and \citet{Hopper1975} advocate for the “loan syntax” hypothesis, which proposes that the translators transferred Latin’s \textit{pro}{}-drop into German. \citet[306--308]{Axel2007}, however, points out that there is a pattern to the distribution of overt and null pronouns. Namely, the translators were much more likely to drop the subject pronouns of main clauses than they were those of subordinate clauses. The excerpt in \REF{ex:3:1} demonstrates this asymmetrical distribution. Note that the clause with \textit{pro}{}-drop, (a), is a main clause, while (b) and (c) occur in subordinate clauses. \citet[303--306]{Axel2007} takes this systematicity to mean that \textit{pro}{}-drop must have been part of an authentic Old High German grammar. If it were not a true feature of its grammar, how does one explain its non-random, seemingly syntactically determined, distribution?\footnote{{\citet{Axel2007} offers a structural explanation for the phenomenon: that} \textrm{\textit{pro}}\textrm{{}-drop is licensed through the fronting of the finite verb from its generated clause-final position to the second position in the clause.}}

If one keeps in mind what the early German Tatian is, that is, an incidental literization of the German vernacular, rather than an intentional \textit{scriptus}, alternative explanations open up. For example, the insertion of the subject pronoun perhaps might serve a disambiguating function by more clearly demarcating the text’s subordinate clauses.

\eabox{%4
    \label{ex:3:4}
\begin{tabular}{@{}ll@{}}
quid faciam quod non habeo &    uuaz tuon thaz ni haben       \\
quo congregem              & uuara gisamano                   \\
fructus meos               & mine uuahsmon  (lines 4--6)       \\
\end{tabular}\medskip\\
  \parbox{3.5cm}{subordinate clause:}  uuara \textbf{ih} gisamano mine uuahsmon\\
  \parbox{3.5cm}{main clause:}       uuara gisamano \textbf{ih} mine uuahsmon    \medskip\\
  \parbox{3.5cm}{subordinate clauses:}  uuaz \textbf{ih} tuon; thaz \textbf{ih} ni haben\\
  \parbox{3.5cm}{main clauses:}    uuaz tuon \textbf{ih}; thaz ni haben \textbf{ih}\\
}

\begin{sloppypar}
\noindent Much of the clausal ambiguity stems from the project’s translational constraints. For example, the ordering of the finite verb \textit{gisamano} and the object \textit{mine uuahsmon} is fixed by the Latin. Thus, the translators are locked into having a Verb-Object sequence. Adding the subject pronoun, however, allows the translators to carve out a clearer subordinate clause, or a main clause, if that is what they had wanted, simply by adding the first person singular subject pronoun before the verb. Similar clausal ambiguity arises in the German translation of line 4, which can be mitigated through the addition of subject pronouns. Both clauses, \textit{uuaz tuon} and \textit{thaz ni haben} could be main or subordinate clauses. Furthermore, \textit{thaz}, (that) can be translated in a number of different ways, including as a complementizer, a demonstrative or relative pronoun, or as a causative conjunction meaning ‘in that’ or ‘because.’ This explanation, I maintain, is more consistent with the realities of this text’s creation, specifically that the translators prioritized the Latin, and translated it in such a way so as to maintain the original’s syntactic transparency. In the case of \REF{ex:3:4}, then, one should first assume that the translators are elucidating for the German-speaking reader the distinction between Latin main and subordinate clauses. In that the translator’s literization of the vernacular is incidental rather than the immediate goal of the project, unlike contemporary writers who composed directly in the vernacular, \citegen{Axel2007} analysis of \textit{pro}{}-drop seems more far-fetched to me.
\end{sloppypar}

In contrast to the Tatian translation, that of Isidor’s \textit{De fide catholica contra Iudaeos} stands out as an early \textit{scriptus} that, though it is still “geared to the needs of the Latin primary text” \citep[50]{Green1994}, represents a more concerted and intentional engagement with the project of vernacular literization. For one, the translation is much freer. Though one cannot know for sure how idiomatic that makes the translation (see \chapref{sec:chap:2}), one fact is certain: that an independent vernacular translation is more of a challenge for the L2 Latin learner than, say, the Tatian translation with its (mostly) transparently one-to-one correspondence between Latin and German constituents. Another indication of the Isidor translator’s interest in the literization of their vernacular as its own end is the fact that they took steps to create and apply a “regulated language” in their text. \citegen{Matzel1970} extensive study traces the existence of a consciously implemented and differentiated orthography in the Isidor (495--496, \textit{passim}). That the Isidor translator went to the trouble of developing this \textit{scriptus} and applying it throughout their translation is indicative of an interest in the vernacular that is not evident in the Tatian.

Matzel imagines that the Isidor translation represents a direct response to Charlemagne’s call for the development of a \textit{grammaticam patrii sermonis}. He speculates that the Isidor translator was one brilliant and highly educated scholar, who was able to not only meet the linguistic challenges inherent in the project but also understand the difficult Latin and sophisticated theological arguments of the original and find ways to express these complicated ideas in  German without the benefit of an already existing vernacular \textit{scriptus}. He thus deduces that the translator must have been close to, or part of, the emperor’s intellectual circle, for such a person could not have come from an isolated provincial monastery (page 522). Regardless of who the translator was or what their motives were, the end result of their toils was a new written variety of German that was \textit{schrift- und damit literaturfähig}, (i.e., that could be written and, thus, was a suitable variety for literature), that could be used to write texts \textit{höchster Inhalte} (‘of the most rarefied content,’ page 495--496, but also pages 519, 521--522).\largerpage

The \textit{scriptus} that the Isidor translator created for his project, however, remains an isolated phenomenon and does not become the basis for \textit{scriptus} development across the German-speaking monastic network. Matzel notes how there are hardly any traces of the \textit{scriptus} outside of the Isidor family of texts, known as the \textit{Isidor-Sippe}, and argues that its regulated language was both too specific to the translator’s own Rhine Franconian dialect and too difficult for the average person to learn (pages 530--532). Otfrid von Weissenburg, who was not an average monk, seems not to have been aware of the Isidor translator’s efforts, for he laments in the \textit{Ad Liutbertum}, one of the \textit{Evangelienbuch}’s several prefaces, how the Franks have no tradition of vernacular literary culture. He states that they, unlike other great cultures, do not “commit the stories of their predecessors to written record” \citep[886]{Magoun1943}. He continues to note that they write in a “foreign language” and, so, Frankish is “unused to being restrained by the regulating curb of the art of grammar” and that it has never been “polished up by the natives” through writing \citep[880, 886]{Magoun1943}. In section VII, Otfrid complains about how challenging it was to wrestle his spoken vernacular onto the page. Its words were difficult to spell, its grammatical patterns unlike Latin \citep[880]{Magoun1943}. In short, Otfrid had to literize Frankish from scratch. He had no vernacular \textit{scriptus} on which to build, and it was as if the eighth-century \textit{scriptus} from the dialectally close \textit{Isidor-Sippe} did not exist.

\subsection{Inhibited vernacular literization in the Carolingian Empire}\label{sec:3:3.1.2}

In this section I expand on the argument that the sociocultural environment of the day was not well disposed toward the vernacular, and, so, it makes sense that few people decided to dedicate resources toward its literization. That is, the Carolingians felt little need or incentive to divert extensive resources from their Latinate documentary culture to the cultivation of the vernacular. This decision came at a cost in that the literary apparatus remained focused on a written language that did not easily connect to how any of the empire’s subjects spoke. This choice stands in marked contrast to early England, which saw both the revitalization of Latinate literacy and the establishment of a more far-reaching education in English \citep[1--2]{Marsden2004}. Why did the Frankish empire not do the same? Why is there only scant indication of imperial interest in vernacular \textit{scripti}? Einhard indicates Charlemagne spared a thought or two for the vernacular: as discussed in \sectref{sec:3:3.1.1}., the emperor is reported to have asked for a \textit{scriptus} of, one assumes, his own vernacular. Einhard also states that Charlemagne requested that the “barbarous and ancient songs” of the Carolingian ancestors be committed to parchment. But where is the corresponding legislation? Where are the letters that would indicate some official, systematic program of vernacular literization that rises to the level of anything remotely resembling the Carolingians’ engagement with Latin? Put simply, evidence of this sort does not exist.

I would like to first explore the consequences of the Carolingian focus on Latin. Drawing on \citet[12, 25]{vonSee1985}, \citet[42--43]{Green1994} describes the chaotic state of written Latin during the Merovingian period. Romance vernaculars had begun to emerge from the Latin of Late Antiquity, which, combined with a decline in education and a rise in illiteracy, led to confusion over how one should write Latin. The decision, then, to restore written Latin to its older, classical state, when it was the language of intellectual pursuits, as well as legal and commercial transactions, meant that the empire’s \textit{lingua franca} moved further away from all of its subjects’ vernaculars, including the Romance ones. This fact is recognized by church directives that preachers should translate (\textit{transferre}) their Latin-language sermons into \textit{rusticam Romanam linguam} and \textit{linguam Theotiscam}, ‘rustic Latin’ and ‘the German language’ \citep[43]{Green1994}. Similarly, the \textit{Strasbourg Oaths} were translated into German and a language that more closely resembled Romance vernaculars, what we might simply refer to now as Old French. These developments limit the potential for the literization of vernaculars: the reforms of the Carolingian renaissance re-established Latin as a language that was entirely fit for purpose, be those needs pragmatic, theological or intellectual, and left little incentive for anyone to develop vernaculars in a similar way \citep[42]{Green1994}. It is true that, as Green notes, these imperial language policies created a domain for vernacular literization in that certain important texts would have to be translated for illiterate subjects. However, this is a restricted literization that exists only to render Latin texts more accessible. It is a support to Latinate literacy and not an end in itself.

Why did Charlemagne re-establish a normative, classical Latin as the language of the empire when it was apparent that doing so made that language less functional as a means of communicating with his subjects? First and foremost, the policy makes religious sense. Latin was the language of the Vulgate, as I discussed in \sectref{sec:3:3.1.1}, and the Carolingians engaged in exegesis, the process whereby they would uncover the multiple layers of meaning of Bible verses through close analysis. This project necessitated a more than rudimentary understanding of Latin and \textit{grammatica}, though it would also require that clergy were able to convey important Christian lessons to the empire’s subjects in language that they could understand. It seems to me that it would have also made political sense to re-establish classical Latin, rather than reform the written language to reflect the language change that had occurred. Charlemagne was crowned the Holy Roman Emperor by the pope in 800, a symbolic act that reflected the mutually beneficial relationship that existed between the Frankish Empire and Rome, as well as the fact that the Franks had begun to see themselves as the true inheritors of the Roman Empire and its church. Furthermore, the Carolingians themselves drew legitimacy from the church, which backed their ultimately successful efforts to supplant the Merovingians as the ruling family \citep[383]{Wickham2009}. All of these factors contributed to the promotion of Latin at the expense of the vernacular.

One final circumstance that I believe contributed to the decision to devote time and effort to Latin, is the linking of Latin to the Carolingian program of moral reform. In fact, the flowering of education associated with the so-called, Carolingian renaissance was part of a larger, ambitious project of correcting (from Latin \textit{correctio}) how both lay and ecclesiastical subjects thought and acted \citep[382--383]{Wickham2009}. The prescriptions, stemming from all manner of ecclesiastical texts, describe in Latin how one should live and worship. The program, thus, not only ensured continued engagement with Latin, it linked moral behavior to the Latin language in the minds of those who were directly involved in literary culture and the production of texts, which is to say, the clergy. The vernacular, on the other hand, was associated with the profane and confined to secular oral spaces and secular topics.

One famous example of the inherent incompatibility of vernacular culture and Christian piety comes from Alcuin, the Northumbrian clergyman who would exert tremendous influence on the direction of Christian learning in Charlemagne’s empire. In his letter to Higbald, the bishop of Lindisfarne, he argues against the introduction of secular songs into the monastic setting, stating:

\begin{quote}
{\dots} [L]et the words of God be read in the gatherings of priests. There it is fitting for a reader, not a harp-player, to be heard; the teachings of the Fathers [of the Church], not the songs of the pagans. What has Ingeld to do with Christ?\footnote{See \citet[181--183]{Dümmler1895}.}

verba Dei legantur in sacerdotali convivio. Ibi decet lectorem audiri, non citharistam; sermones patrum, non carmina gentilium. Quid Hinieldus cum Christo?
\end{quote}

\noindent This attitude toward vernacular culture had an effect on how the church viewed the vernacular language. \citet[141--142]{Edwards1994} notes how the clergy began to associate Latin with God and the Bible, which was the only means through which one’s immortal soul could be saved. In contrast, German spoken varieties were the language of pagan, secular culture, to which clergy would frequently attach epithets, like \textit{obscenus}, \textit{inutilis}, \textit{barbara}, \textit{rustica}, \textit{indisciplinabilis} (‘obscene, harmful/useless, barbaric, rustic, undisciplined’). One monk, who heard secular songs from his cell, declared that the vernacular reeked of “the stench of dung and the sweat of the warrior” (\citealt{Edwards1994}: 141, citing \citealt{Haubrichs1988}: 42).

These negative attitudes were surely sharpened by the fact that oral pagan culture was still ubiquitous. Though Frankish society had become more Christian by the eighth and ninth centuries, conversion, in addition to correction, was an integral part of the Carolingian agenda. Otfrid illustrates how clergy viewed vernacular culture as a persistent threat to a Christian way of life. In his \textit{Ad Liutbertum} he explains that one reason why he wrote the \textit{Evangelienbuch} was to counteract the “noise of (worldly) futilities” and “the offensive song of laymen” (\textit{laicorum cantus … obscenus}), which had become a burden for people of God \citep[873]{Magoun1943}. He expresses the hope that “a little of [his] poem (\textit{huius cantus lectionis}) might neutralize the trivial merriment of worldly voices (\textit{secularium vocum}) and (that), engrossed in the sweet charm of the Gospels in (their) own language, [people of God] might be able to avoid the noise of futile things” \citep[873--874]{Magoun1943}. Vernacular, or pagan, culture threatened to overtake the sonic environment of the Franks, unless people like Otfrid could mitigate its deleterious influence with the sound of the Gospel in Frankish.

Even a writer like Otfrid von Weissenburg, who ultimately arrived at the conclusion that composing extensively in the vernacular was worth the effort, expresses ambivalence vis-à-vis his own spoken Frankish. He makes clear in the \textit{Ad Liutbertum} that it is not simply vernacular culture that he finds problematic. He writes in section VII that Frankish is “rude,” “unpolished,” and “unruly,” “unused to being restrained by the regulating curb of the art of grammar” (\textit{Hutus enim linguae barbaries, ut (just as) est inculta et indisciplinabilis atque insueta capi regulari freno grammaticae artis}, translation from \citealt[880]{Magoun1943}). Though painting Frankish in a unflattering light, this passage evinces an optimism about the vernacular that Haubrichs’s disgruntled monk certainly did not feel. Otfrid believes Frankish to be in a barbaric state, but the problem lies not in its supposed pagan associations, but rather in the fact that no one has bothered to tame it. This task of turning Frankish into a proper written language is one that Otfrid takes on himself. However, he also feels the need to justify the decision extensively. In addition to the Latin-language \textit{Ad Liutbertum}, in which he explains why he wrote the \textit{Evangelienbuch}, Otfrid devotes the work’s first chapter, “Why the author composed this work in the vernacular” (\textit{Cur scripto hunc librum theotisce dictaverit}), to the topic. At 126 lines, it is one of the longer chapters of the work and indicates that Otfrid was aware of the fact that the cultural environment that would receive his gospel harmony was at best, indifferent to, at worst hostile toward, the vernacular.

  This section’s discussion of Carolingian attitudes toward vernacular culture and language supports the characterization of early German \textit{scripti} as isolated and idiosyncratic occurrences. That is, people who were part of literate culture were engaged with Latin, the language of the church and of their canonical Bible. The vernacular, on the other hand, was the language of paganism, which the Carolingians were eager to stamp out. It was also the language of the still ubiquitous oral tradition, the sound of which threatened to drown out more pious utterances. It is in this sociocultural context that the vernacular \textit{scripti} of the early medieval period emerged. Few indications exist that many people were particularly invested in their existence. No good evidence indicates that there was ever a concerted or centralized effort to literize German in the first place. When vernacular writers, like Otfrid von Weissenburg, embarked upon their German-language writing projects, they were on their own and faced with the task of literization, of turning the auditory into the visual, without any written tradition on which to draw.

\subsection{A new way to approach four old texts}\label{sec:3:3.1.3}

I close this section by highlighting how my treatment of the four most significant early German texts~-- the Tatian and Isidor translations and the originally composed \textit{Evangelienbuch} and \textit{Hêliand}~-- differs from the one informed by the structuralist deficit approach. Of the four, I have identified the Tatian translation as the least likely to contain a coherent \textit{scriptus} because the translators were demonstrably and primarily engaged with the task of elucidating the structure of the original Latin text. This characterization is consistent with the fact that most vernacular literizations in Carolingian Europe were part of the imperial policy of Latin reform and education. Its significance is that any systematicity the modern linguist discovers in the translation could be attributable to the translation process itself, that is, is the result of negotiating the vernacular around the constraints of the Latin syntax. Accepting this argument does not mean the Tatian data are useless, corrupted, inauthentic, or ungrammatical. It does mean, however, that one must reject the \textit{Differenzprinzip} as a methodology, which, as I explained in \chapref{sec:chap:2}, involves analyzing just the tokens that deviate from the original under the assumption that these data must be authentic German. Rather, all of the Tatian data stem from an incidental German \textit{scriptus} that is dependent upon the Latin original and geared toward rendering it more comprehensible to German-speaking readers. They do not result from an explicit attempt at German literization with a new authentic seeming \textit{scriptus} as the intended goal.

In contrast, the Isidor translation features a more independently and intentionally constructed \textit{scriptus} than the Tatian. This conclusion is supported by the fact that the translation itself is much freer and, as \citet{Matzel1970} demonstrates, the text features a regulated, differentiated, and complex orthography. This argument indicates that its translator consciously engaged with the project of vernacular literization and managed to create a fairly consistent \textit{scriptus}, at least in terms of its graphemic representations of sound. Ultimately, though, the Isidor translation is still dependent on a Latin text.

The two most significant original works from the period, the \textit{Hêliand} and Otfrid’s \textit{Evangelienbuch}, do not present these same complications for the researcher. They feature independent \textit{scripti} in that there is no vernacular writing tradition that could have informed their writers’ composition process. This statement does not imply that they did not draw on different linguistic influences, including and especially, their training in Latin. Latin \textit{was} the language of literacy for German-speakers in the medieval period, so it seems highly implausible to me that either Otfrid or the \textit{Hêliand} poet would or could have eschewed this tradition entirely. More likely is that they embraced it, along with appropriate vernacular linguistic resources in an effort to solve the puzzle of literization.

\section{Ad hoc \textit{scripti} and linguistic creativity}\label{sec:3:3.2}

  Ninth-century German-speakers who go against the grain and write in German, like Otfrid and the \textit{Hêliand} poet, embark upon the daunting task of shaping their exclusively oral variety into a written \textit{scriptus}. The fact that they undertake these projects in the complete absence of a vernacular writing tradition means that each writer (or team of writers) must work out for themselves what shape their \textit{scriptus} should take. This is not to say that their vernacular linguistic intuitions do not inform this process. They certainly do. However, analyzing the early German \textit{scripti} cannot be a simple matter of identifying some underlying grammatical system that fed into them. That is, the principle of historicity reminds us that the early literizers were multilectal and multilingual, just as today’s speakers are. Different syntactic patterns will characterize different varieties of the exclusively oral varieties of early German, as they do modern varieties. A literizer must decide which variety or varieties they will draw on.

A logical choice would be those (exclusively oral) planned and public varieties that are shaped by a communicative context of distance, as writing too will require a greater level of coherence than is present in the contexts of proximity or immediacy that determine spontaneous and intimate varieties. However, even the most planned and public distance varieties of the literizer’s oral vernacular will not be able to function effectively in the wholly dislocated context that only the technology of writing can effect. Thus, literizers like Otfrid and the \textit{Hêliand} poet must innovate linguistically in order to create their written \textit{scriptus}. Their linguistic intuitions are a necessary, but not sufficient ingredient for literization. That each \textit{scriptus} creator operated in the absence of any tradition of vernacular literacy also suggests that these first \textit{scripti} will vary structurally and lexically.

In this section, I discuss the basis for understanding the development of a vernacular \textit{scriptus}, fundamentally, as a linguistically creative act. I draw on the work of Koch and Oesterreicher because their descriptions of a conceptual orality and literacy are useful for explaining what the process of creating the first literizations of a German vernacular necessarily entailed. In short, though ninth-century oral vernaculars were perfectly suited to the communicative contexts in which they had always been used, they required further development before they could become a vehicle for literary expression in the new graphic medium. Aspiring vernacular writers, thus, shaped their vernacular into a new written form that met the needs of their particular project.

\subsection{Conceptual orality and literacy}\label{sec:3:3.2.1}

\citegen{KochOesterreicher1985} seminal article provides an excellent starting point for an investigation of the development of early medieval \textit{scripti}. It elucidates how turning ninth-century spoken German into a written German \textit{scriptus} was not mere mechanical transfer of language from one medium to another. Rather the early literization of an exclusively oral vernacular depends on the creation of literacy as a conceptual category that is qualitatively distinct from orality, a process that \citet{KochOesterreicher1994} call \textit{Verschriftlichung}. \textit{Verschriftlichung} is often translated as ‘textualization.’ The term textualization, however, has associations that present problems for the current analysis. In the field of Homeric studies, for example, scholars have argued that textualization and textuality can reasonably be applied to works of oral art. \citet[109]{Nagy1996} uses these concepts as means of explaining changes in the tradition of Homeric poetry \textit{before} its transfer into the graphic medium, maintaining that “there can be textuality~-- or better, textualization~-- without written text.” \citet{Ready2019}\footnote{\citegen{Ready2019} chapters 1 and 2, which together make up Part One, deal extensively with these arguments.} similarly argues for the usefulness and applicability of these terms to oral art. His argument is literary: the world constructed in Homeric poetry seems to presume the existence of stories and song beyond the moment of their performance; that is, they exist in this literary world as text, and poets can adopt strategies of “entextualization” that create these “oral texts” within the story being told (pages 15--16). In modern text linguistics, \citet[191]{Oesterreicher1997} notes, scholars have also applied the term “text” in broad ways: they have referred to both spoken and written discourses as texts. As will become clear in the pages that follow, I aim to follow Oesterreicher’s lead and maintain a strict division between the medial categories of phonic and graphic, on the one hand, and the conceptual categories of orality and literacy, on the other hand.

Thus, to avoid terminological confusion, I adopt the term literization~-- not textualization~-- for the diachronic process of making an oral vernacular a written variety. From what I have been able find in scholarly literature across disciplines, “literization” has not been much used. That it has not found much favor might be due to the word’s awkwardness. One of the few scholars who has put the term to good use, Sheldon Pollock, characterized it as a “rebarbative” translation of \textit{Verschriftlichung}. Still, Pollock has used the concept of literization as a way of describing the development of literate and literary languages of India. He has defined it as the process “whereby a language (or what thereby becomes \textit{a language}) acquires written form” \citep[283--284]{Pollock2007}. Literization has been used to refer to, more or less, the same process in a few sociolinguistic and anthropological studies of pidgins and creoles. For instance, Jeff \citegen{Siegel1981} article on written Tok Pisin adopts a similar definition of literization to Pollock’s: “[f]or lack of a better word, I will use “literization” to refer to such development, both planned and unplanned, of a previously unwritten language into a written one” (page 20). Laura \citegen[11--13]{Hills2001} Ph.D. thesis on Mauritian Creole argues that it is more fruitful to talk of literization of the vernacular, rather than the more commonly used phrase the “vernacularization of literacy.” That is, a focus on simply learning how to read and write in Mauritian Creole skips over the necessary processes of literizing that vernacular in the first place. Hills’s own conception of literization is narrower than Pollock’s (and mine) in that she states it involves only the establishing of written norms and a written oeuvre. Pollock’s more expansive definition, in contrast, encompasses the development of literacy as a conceptual category. So, it is his basic description of literization that I elaborate in the rest of this book.

Returning to \citet{KochOesterreicher1985}, its central insight lies in its distinction between the phonic and graphic media, on the one hand, and \textit{Mündlichkeit}, ‘orality,’ and \textit{Schriftlichkeit}, ‘literacy,’ as conceptual categories, on the other. The phonic and graphic codes constitute a binary: Language is either spoken or written. In contrast, all instances of linguistic production, be they spoken or written, can reflect some degree of “spokenness” or “writtenness.” For example, an academic presentation written out in advance and read aloud will have literate qualities, while the hastily dashed-off instant message to a close friend will exhibit oral qualities that characterize spoken language. In order to disentangle the conceptual from the literal, the authors imagine a continuum of linguistic production: language at the left pole is shaped by the communicative context of closeness or proximity and is called \textit{Sprache der Nähe}, ‘language of immediacy,’ while language at the right pole is shaped by a communicative context of distance and is called \textit{Sprache der Distanz}, ‘language of distance.’ The contexts of immediacy and distance constrain and shape their linguistic utterances in many ways (\citealt[23]{KochOesterreicher1985}). For example, paradigmatic language of immediacy is dialogic, subjective, expressive, and spontaneous; it occurs in intimate and familiar contexts. Language at this extreme end of the continuum is (probably) spoken, excluding sign languages from the discussion. Paradigmatic language of distance, in contrast, is monologic, objective, detached and planned; it occurs in formal and unfamiliar or public contexts. Language at this extreme end of the continuum is often written, assuming the culture or individual in question is literate.

Linguists have long acknowledged the many ways in which context shapes an individual’s language. A person’s utterances, for example, vary considerably when they are speaking spontaneously and in intimate contexts compared to the way that same person speaks in formal or public contexts with the benefit of time to plan. The differences reach into every aspect of language, from syntax to pronunciation. Consider lexical variation, for example, and the ways in which our word choices are distinct when we are speaking in immediacy contexts versus writing something formal for an unknown readership. It is not simply a matter of register; it is also a question of specificity. When a speaker shares the physical space of their interlocutor, it is easier to arrive at a shared meaning than when a language producer and receiver are disconnected from one another in time and space. For example, the formal writer must take care that what they write accurately reflects what they mean. They must choose their words carefully and precisely; it is in fact the only thing they can do to facilitate arriving at a shared meaning with a wholly unknown reader.

\citegen{KochOesterreicher1985} disentangling of medial and conceptual orality and literacy, respectively, allows for a more systematic way of assessing different types of linguistic output. The article’s figures 2 and 3 on pages 18 and 23, respectively, reproduced here as Figures \ref{fig:3:1} and \ref{fig:3:2}, are useful in their schematic representation of how the conditions and constraints of immediacy and distance link to different strategies and, thus, different characteristics of linguistic production.\footnote{I relied on Peter Koch and Wulf Oesterreicher’s translation of these figures in their \textcite{KochOesterreicher2012}.}

\begin{figure}
\caption{Disentangling medium and conceptual orality-literacy}
\label{fig:3:1}
\includegraphics[height=.4\textheight]{figures/wulffoesterreicher0.pdf}
\end{figure}


\begin{figure}
\caption{The communicative conditions and corresponding strategies of orality and literacy}
\label{fig:3:2}
\includegraphics[width=\textwidth]{figures/wulffoesterreicher.pdf}
\end{figure}


Of particular interest to the current study is the authors’ identification of certain syntactic characteristics that may be associated with the language of distance versus the language of immediacy. \figref{fig:3:2} indicates that the language of immediacy is less information dense, compact, integrated, complex, elaborated, planned, while the language of distance is more so. The conditions of the language of immediacy involve an immediate, intimate, often spoken, dialogic, and face-to-face context. These conditions may also be seen as constraints in that there is no time to plan one’s speech, and, as a result, one produces utterances that are structurally marked in accordance with this constraint. Immediacy utterances are not only tailored to and, indeed, a product of the constraints under which a speaker must produce them; they are also suited to the same context in which they are received by a physically present interlocutor. The same logic applies to the language of distance: it is shaped by and suited to the context of distance.

I would now like to make the terms in \figref{fig:3:2}, that is, information density, compactness, integration, and elaboration, more explicit by discussing a couple of examples.

\ea%5
    \label{ex:3:5}
\ea\label{ex:3:5a}
\gll Der Hund  da    unter dem Tisch  der  ist  müde\\
    \textsc{det} dog     there   under the table  \textsc{det}  is   tired\\
\glt     ‘That dog there under the table, that (one) is tired’

\ex\label{ex:3:5b}
\gll Der   unter dem Tisch  liegende   Hund  ist  müde\\
  the   under the table   lying    dog   is  tired\\
\glt ‘The dog under the table is tired’
    \z
\z

\noindent German speakers will immediately recognize the utterance in \REF{ex:3:5a} as an example of language of immediacy: it features the left dislocated deictic noun phrase \textit{der Hund da}, which one could translate as something like, ‘that dog there’ or the more colloquial ‘that there dog.’ This phrasing is characteristic of colloquial German. The use of the demonstrative pronoun \textit{der} as the resumptive third person subject pronoun is also characteristic of spoken German. The utterance in \REF{ex:3:5b}, in contrast, does not convey the same sense of shared space between speaker and interlocutor. Furthermore, its extended participial, \textit{der unter dem Tisch liegende Hund}, is a characteristic of distance language. A construction like the one in \REF{ex:3:5b} is virtually unattested in spoken German varieties and can only be found in its more formal, generally written varieties.

Consider, then, how the two clauses in \REF{ex:3:5} differ with respect to integration. The example sentence in \REF{ex:3:5b} is more integrated than \REF{ex:3:5a} in that the subject \textit{der Hund}, ‘the dog,’ along with its extended participial modifier and embedded predicate, \textit{liegende} ‘lying,’ function as the clause’s subject. However, in \REF{ex:3:5a} the location modification is not integrated into the main clause’s subject but is instead featured in the topic of a topic-comment construction. Information density, another verbalization strategy listed in \figref{fig:3:2}, correlates with syntactic integration. That is, the more syntactically integrated a clause or sequence of clauses is, the more information that clause or sequence of clauses is likely to contain. Note how the more syntactically integrated clause in \REF{ex:3:5b} conveys the same information as (a) but accomplishes this with one fewer word. Its nominalized predicate, \textit{liegende} ‘lying,’ furthermore elaborates on the position of the dog~-- she is lying under the table, information that is missing in \REF{ex:3:5a}. Admittedly, this spatial modification would also be unnecessary in the context of shared physical space in which a spoken language of immediacy is generally produced.

In light of this foregoing analysis, one might also conclude that the clause in \REF{ex:3:5b} is more complex than the one in \REF{ex:3:5a}, as \citegen{KochOesterreicher1985} figure indicates. I do not think, however, that complexity is a useful parameter by which to measure linguistic output. First, complexity, as a concept, brings nothing to the analysis that integration does not and instead introduces a qualitative judgment into a discussion where it does not belong. As \citet[14]{Schleppegrell2004} explains in her discussion of the features of what she calls the “language of schooling,” that is, the type of written standard language that literate cultures teach and hold in high esteem, complexity is generally equated with more subordination, hypotaxis, or the hierarchical embedding of one clause in another.\footnote{\citegen{Schleppegrell2004} work is aimed at researchers and students of language in education. She makes explicit the variety of English that students are expected to produce at school. This variety is vastly different from the “interactional language” that students use outside of school.} These features of subordination, hypotaxis, and embedding are all measurable types of integration in fact. Complexity, in contrast, is a syntactically vacuous term. When it is associated with the syntactic features of written standard language, rather than, say, the language of a free-wheeling conversation between a group of old friends, it reflects a literate bias. The latter, in fact, evinces its own complexity, as any discourse analysis would reveal.

I also question the usefulness of compactness and elaboration as variables to assess different syntactic structures. Similar to complexity, compactness is another qualitative parameter that can only be demonstrated empirically by referring to the measurable, syntactic characteristics of integration. That is, one might conclude that more hypotactically arranged clauses are more compact, paratactically linked clauses less so. Compactness could refer to the fact that speakers or writers who plan language of distance are able to economize with their linguistic production by avoiding, say, lexical items that producers of spontaneous language use to buy time or shape discourse (modal particles, discourse adverbials), or by eliminating redundant language. Still, the assumption that planned distance language is \textit{necessarily} compact seems to reflect a modern style preference for how written language should be. Namely, that it ought to be concise. Elaboration, on the other hand, will be useful as a general modifier for different types of distance varieties, for example, “elaborated orality,” as I discuss just below, rather than as a syntactic feature whose presence in the data confirms their identity as data shaped by the communicative context of distance.

In sum,  \citegen{KochOesterreicher1985} conceptual categories of “language of immediacy” and “language of distance” are analytically useful in that they correlate with different syntactic features, particularly those connected to the degree of integration and lexical density. I explore this topic further in \chapref{sec:chap:4}. In keeping literal and conceptual orality\slash literacy distinct, the authors also provide a framework for imagining the linguistic production of, what \citet[11]{Ong2012} [1982] calls, “primary oral cultures,” that is, cultures that are totally unfamiliar with writing. In such cultures, the medium is naturally always oral. However, contexts of immediacy and distance are still relevant and, thus, affect the shape of linguistic structures, just as they do in cultures that are literate. In fact, one can expect that distance languages in exclusively oral and literate cultures are similar in some ways because they both, to some degree, are influenced by the communicative conditions of distance and so, would make use of similar verbalization strategies. \citet{Chafe1981}, for example, demonstrates that the ritual, or planned, variety of Seneca, an Iroquoian language the author characterizes as strictly oral, is structured similarly to written distance varieties: For example, ritual Seneca features more syntactic embedding, like other written languages and unlike colloquial Seneca.

Yet, \citet[30]{KochOesterreicher1985} caution that a primary oral culture’s varieties of distance, which they collectively refer to as “elaborated orality,” must be treated as distinct from written distance varieties because their speakers have no recourse to writing. This circumstance places an additional cognitive constraint on their production, that of memorability. \citet[31--36]{Ong2012} [1982]\footnote{\textrm{Walter Ong’s book} \textrm{\textit{Orality and literacy: The technologizing of the word}} \textrm{was first published in 1982. I used the 30}\textrm{\textsuperscript{th}} \textrm{anniversary edition.}} elucidates the effect that the memorability constraint can have on the language itself. In cultures with no writing, words are evanescent in that they are only ever sound and have no visual presence at all. In order to ensure a tradition of knowledge and cultural memory, people must devise ways to make their thoughts and language memorable. Because one knows only what one can remember, Ong remarks, “experience is intellectualized mnemonically” (page 36). Another way of saying this is that the planned distance language of an oral culture must take a particular form. The following quote is from \citet[34]{Ong2012} [1982].

\begin{quote}{}
[T]hought must come into being in heavily rhythmic, balanced patterns, in repetitions or antitheses, in alliterations and assonances, in epithetic and other formular expressions, in standard thematic settings (the assembly, the meal, the duel, the hero’s helper, and so on), in proverbs which are constantly heard by everyone so that they come to mind readily and which themselves are patterned for retention and ready recall, or in mnemonic form.
\end{quote}

\noindent \citet[33]{Somers2021b} points out that modern readers are likely to recognize these mnemonic devices as features of poetic language, concluding that “the language of the oral tradition was crucially poetic, its form the instrument of its survival.”\footnote{This observation could yield a companion clarification to \chapref{sec:chap:2}’s examination of the definition of “prose.” Just as modern literates tend to see prose as a neutral and natural type of language, rather than a constructed written distance variety, they might view poetry as the unnatural and constructed variety, wholly unlike the way people actually speak. This literate orientation might lead them to see the early poetry of newly literate cultures in those same terms \citep[368--370]{Somers2021a}.} A closer examination of Ong’s features reveals how Ong’s mnemonic devices alleviate the cognitive burden of planning and producing distance varieties in exclusively oral cultures. Some features constrain the prosodic and phonological form of language, as is the case with the use of rhythm, balanced patterns, alliterations, and assonances. Other forms, in contrast, constrain content, as is the case with the prominence of formulaic language (like epithets), repetitions, antitheses, and standard themes or clichés. In that this type of language exists in and of the culture that produces it, interlocutors should also find it easier to process. These same patterns in form and content shape their own engagement with the world around them.

Memorability is one of the central distinguishing features of oral and written distance varieties, though it is not the only one. As soon as speakers become aware and take advantage of writing’s ability to release thinking and language planning from mnemonic constraints, written distance varieties can, and as I argue in the next section, \textit{must}, adopt different grammatical forms that prioritize the visual over more sound-based patterns of construction. As writing makes in-roads into culture, speakers also become aware of the potential for the complete disconnection that writing can effect between language producer, the language they produce, and the person who receives the language. This next section, \sectref{sec:3:3.2.2}, explores how extreme, which is to say, written, contexts of distance force the early literizers of an exclusively oral vernacular to creatively shape their vernacular in new ways that are better suited to this new visual environment.

\subsection{\textit{Verschriftlichung} (`literization') and the creation of early \textit{scripti}}\label{sec:3:3.2.2}

The goal of this section is to demonstrate that the creation of early \textit{scripti} must involve linguistic creativity. My argument in support of this idea is based on the premise that all vernacular varieties in an exclusively or mostly oral society, which would include immediacy and distance languages, are perfectly suited to the environments in which they develop and are used. However, they are not yet suited to the new distance contexts and projects that writing itself makes possible. Thus, each early literizer must innovate linguistically in order to create a written language that can function in these new contexts and for these new projects; \textit{a simple medial transfer of an oral vernacular into a written one will not suffice}. This point connects directly to \chapref{sec:chap:2}’s argument against the deficit approach to early medieval German syntax. There I pointed out the problems associated with diachronic linguists’ assumption that early German autochthonous prose was the best place to find authentic German structures that are characteristic of an early German grammar. Beyond the fact that this approach is built on a number of fallacies, in this section I argue that early German \textit{scripti} could never have resulted from a simple medial transposition of an oral variety into a graphic form. Rather, each \textit{scriptus} required the literizer to build on existing spoken competencies and create brand new structures to accommodate the similarly new, more extreme communicative context of distance that writing alone effects. Referring back to the deficit approach discussed in \chapref{sec:chap:2}, the structures that literizers necessarily created in the eighth and ninth centuries are the beginning of a written, literary German, regardless of the degree to which so-called non-autochthonous factors shaped them or whether they may or may not have been attested in some spoken variety of the vernacular.

  I begin, then, by explaining my argument’s initial premise, namely, that exclusively oral, unliterized vernaculars are perfectly functional in exclusively oral societies but do not remain so once literization begins. Consider \citegen[23]{KochOesterreicher1985} list of the communicative constraints associated with extreme contexts of immediacy and distance, adapted from their figure 3.

\begin{figure}
\caption{Communicative constraints/requirements}\label{fig:3:3}
\begin{tikzpicture}
\matrix [matrix of nodes, nodes in empty cells, 
         column sep=4mm,
         column 1/.style={anchor=base west}, column 2/.style={anchor=base west}] 
         (matrix)
  {
     1. Dialogue                          & 1. Monologue \\
     2. Familiarity with the interlocutor & 2. Unfamiliarity with the interlocutor \\
     3. Face-to-face interaction          & 3. Spatiotemporal separation \\
     4. Fluid topic development           & 4. Rigid topic setting \\
     5. Intimate, not public              & 5. Public \\
     6. Spontaneous                       & 6. Planned, self-conscious \\
     7. Involvement                       & 7. Detachment \\
     8. Situation-dependent               & 8. Situation-independent \\
     9. Expressive, affective             & 9. Objective \\
   };
   \node [below=\baselineskip of matrix.south west, anchor=west] (immediacy)
     {Language of Immediacy};
   \node [below=\baselineskip of matrix.south east, anchor=east] (distance)
     {Language of Distance};
   \draw[{Triangle[]}-{Triangle[]}] (immediacy.north west) -- (distance.north east);
\end{tikzpicture}
\end{figure}

The first important point regarding this list is that it establishes characteristics for languages of extreme distance and immediacy that could apply to literate cultures only. That is, several communicative constraints associated with distance contexts are \textit{effected} by language literization and, so, cannot characterize exclusively oral vernaculars. The second noteworthy observation is that exclusively oral vernaculars, be they shaped by immediacy or distance, are contextually bound to their moment of production in ways that written languages need not be.

In order to demonstrate this point, consider certain specific constraints of extreme distance identified in \figref{fig:3:3}. For example, spatiotemporal separation (number 3) cannot characterize the distance varieties of an oral vernacular in the same way that it does any given written variety. While people can read texts produced by writers who are long dead and lived somewhere they themselves have never been, the words of an exclusively oral variety can only ever be \textit{heard} and, in the absence of any recording technology, have a reach that is restricted to those within earshot. This reality forces a reimagining of other characteristics for exclusively oral vernaculars, for example number 2’s “unfamiliarity with the interlocutor.” While written languages of distance may conceivably be read by anyone in space and time, exclusively oral distance languages can only be received by people who are physically present during the speech act. The speaker may not know their interlocutors well or at all, but all inhabit the same spaciotemporal location. Thus, the speaker \textit{perceives} their interlocutors, and vice-versa, in ways that writers often cannot; neither speaker nor interlocutor is wholly unfamiliar or unknown to the other.

The contrast pairings of numbers 4, 7, and 9 must similarly be reimagined for exclusively oral contexts. So, the fact that interlocutors will always be physically present for productions of oral vernaculars places natural limits on how detached or objective the speaker will be (numbers 7 and 9). Consider, for example, the narrative song, one possible genre of an oral distance variety. Parry and \citegen{Lord2000} field research on Serbo-Croatian poets, detailed in \textit{The Singer of Tales}, describes the extent to which the audience influences how the poet performs, which is to say, composes, his song. They note how the singer must engage his “dramatic ability and his narrative skill” in order to maintain the audience’s attention. The poet also radically adjusts the length of the song based on his perceptions of his listeners. If the audience is restless or unreceptive, he will shorten the song accordingly; if they are interested and attentive, he will “lengthen the song savoring each descriptive passage” (page 89--90, online version). That is, there is no objective detachment between speaker and listeners. Rather, the poet continually assesses his audience and shapes his song based on what he perceives. In skipping certain narrative beats or expanding on others, the poet also demonstrates a more “fluid topic development” than could ever be found in a written text (number 4). Overall, the poet’s linguistic output is situation-dependent (number 8). While more varieties of an exclusively oral vernacular that are more ritualistic than epic poetry may be less subject to the whims of an audience, they are still spoken in the presence of interlocutors and, thus, are contextually bound to the moment of their production in ways that the written language is not.

The shared context of speaker and interlocutor is linguistically significant in that it allows both to draw on visual and auditory cues while producing and receiving utterances. The spoken utterance represented in \REF{ex:3:6}, which is from \citegen[60]{MillerWeinert1998} study on spontaneous spoken syntax, illustrates how important these context cues are.

\ea%6
    \label{ex:3:6}
         no if we can get Louise/ I mean her mother and father/ Louise’s parents would give us/ they’ve got a big car and keep the mini for the week// but Louise isnae too keen on the idea so …
    \z

\noindent This spoken utterance contains fragmented and incomplete syntax. Its clauses are not always explicitly linked to one another through some grammatical means and its information is not as logically arranged as it could be. The written version of this utterance that \citet{MillerWeinert1998} offers fixes these apparent deficiencies \REF{ex:3:7}.

\ea%7
    \label{ex:3:7}
         No. Louise’s parents have got a big car. If we can get them to give us the big car and if they would take the Mini for the week [we could all travel by car together]. But Louise is not too keen on the idea, so [we will not be traveling in the big car].
    \z

\noindent In contrast to the original spoken utterance, the written version has “complete and coherent syntax” (\citealt[60]{MillerWeinert1998}). Its information is well\hyp organized, and conjunctions make the relationship between clauses grammatically explicit. The written version also features complete sentences, or clause complexes, through the addition of the clauses in square brackets.

  The difference between the spoken utterance in \REF{ex:3:6} and its written version in \REF{ex:3:7} is significant for two reasons. First, though the reader might judge the example in \REF{ex:3:6} to be grammatically vague and, thus, difficult to parse, \citet[60]{MillerWeinert1998} note that none of the participants in the actual conversation found the utterance difficult to understand or otherwise problematic. This was also true for the field worker, who was listening in, and the students who were later asked to listen to the recording and identify any syntactic problems they heard. These facts indicate not only that the utterance was functional in the context in which it was uttered, but that the speaker’s prosodic cues were sufficient in conveying the connections between clauses that might otherwise have been signaled through grammatical means. That is, people who were listeners and denied any gestural cues still thought the utterance sounded fine. It was only when those same students saw a transcript of the conversation that they identified any syntactic problems in \REF{ex:3:6}. In other words, when stripped of the information that can only be conveyed through being present during the speech act, either because one is physically there or transported there through technological means, the same utterance is rendered dysfunctional. Phrased yet another way, performing a simple transfer of the spoken utterance into a written medium~-- that is not a close transcription~-- detached it from the communicative context in which such an utterance was functional and to which it was suited.

This discussion elucidates the early medieval Carolingian context in which incipient vernacular literization occurred. The exclusively oral varieties of German would now have to exist in the dislocated visual space of the page; that is, they would have to function in the communicative context of extreme distance that literization itself creates. A simple transposition or transcription of prehistoric oral vernaculars, which are always and necessarily bound to a particular place in time, would not suffice. A quick note: I intentionally use the term “dislocated,” rather than “decontextualized,” despite the fact that a number of studies describe so-called literate texts, with their written languages of distance, decontextualized language (\citealt{GumperzEtAl1984,MichaelsCazden1986,MichaelsCollins1984,Olson1977,Olson1980}, \citealt{ScollonScollon1981,Snow1983,TorranceOlson1984}; citations from \citealt{Schleppegrell2004}). I agree with \citet[6--7]{Schleppegrell2004}, however, who argues that formal languages of distance, the likes of which are taught to children in school, are bound to particular contexts just as surely as immediacy utterances are. This view is also consistent with \citegen{KochOesterreicher1985} framework, which identifies the contexts for the whole range of linguistic production. There is in fact no such thing as decontextualized linguistic production.

Because simply transposing exclusively oral vernaculars creates a dysfunctional written language on the page, the first writers of German were obliged to bridge the functional gap, as it were, and innovate with their vernacular so that it could meet the novel demands of operating as a visual, dislocated language. I refer to this conscious building up of an oral vernacular’s ability to function in a graphic medium as “ausbau” and detail the particular types of changes one should associate with it in \chapref{sec:chap:4}.

\section{Conclusion}\label{sec:3:3.3}

I begin this chapter conclusion with a brief overview of the book’s argument up to this point. So far, I have argued against the deficit approach, which reflects a traditional and structuralist orientation toward diachronic linguistics and is based on several misapprehensions about the nature of the early German corpus. I have advocated viewing each early German text as an individual, idiosyncratic literization or \textit{scriptus}, rather than as belonging to an inherently problematic corpus riddled with inauthentic data and subject to the influence of confounding factors. I elaborated on my characterization of early German \textit{scripti} as isolated artifacts by drawing on substantial secondary literature, which concludes that there was no concerted, collective effort to literize the vernacular in the Carolingian Empire. Primary literature, chiefly Otfrid von Weissenburg’s \textit{Evangelienbuch}, supports this conclusion. Literizers like Otfrid, therefore, had no vernacular writing tradition or linguistic norms to ameliorate the difficulty of their task. They had to work out literization on their own and in a cultural environment that did not necessarily view their decision to write in the vernacular with a friendly eye.

I have also argued that, what linguists have been calling autochthonous prose is, in fact, often not particularly autochthonous and is never actual prose. Furthermore, they have assumed that this hardly-attested genre of early medieval text is the most likely of the four genre categories to contain authentic German data representative of an everyday language. Diachronic structuralists, including generativists, have maintained that it is the linguist’s job to delineate the core competence or structure that underlies performance or data. They have erroneously associated this early medieval competence with some colloquial spoken variety and have, thereby denied the multilectalism and multilingualism of early German speakers as an analytically consequential fact. I have advocated that we instead approach the study of each early German \textit{scriptus} as a material artifact that resulted from the unique way in which the literizer engaged their many linguistic resources, including their multilectal oral vernacular and their Latinate literacy.

In this chapter, I also began to describe the multilectalism of early German speakers in more concrete terms and discussed how literization itself necessitates linguistic innovation and, thus, language change. By drawing on \citet{KochOesterreicher1985}, I aimed to replace this construct of everyday German that has so preoccupied diachronic linguistics with a more realistic view of early medieval German. That is, early medieval Germans had access to a whole range of spoken varieties that were indelibly shaped by the communicative contexts in which they were produced. Contexts of extreme immediacy and distance mark the two poles, and varieties anywhere on the continuum that connects the poles will be structurally and lexically distinct. However, none of the literizers’ spoken competencies will yield a \textit{scriptus} that can be entirely functional in the new written context that literization itself introduces. That is, writing disrupts the equilibrium that is maintained through the modulation of linguistic output in accordance with communicative context, whereby the former is shaped by and perfectly suited to the latter. Before literization interlocutors were always in the same place at the same time; once literization begins, writers have to contend with the possibility that their linguistic output can be disconnected from them and the time and place in which they wrote it. Thus, their written German requires a degree of grammatical and lexical explicitness that was unprecedented and also unnecessary when the vernacular was an entirely phonic phenomenon. \chapref{sec:chap:4} elaborates on this point in its discussion of the “ausbau” process. Finally, because early literization unfolds in isolated pockets, not as part of a concerted program, scholars should expect that the resultant \textit{scripti} will also vary considerably.

