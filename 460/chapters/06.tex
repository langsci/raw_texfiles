\chapter{German’s prehistory as elaborated orality}\label{sec:chap:6}

\section{Introduction}
In \chapref{sec:chap:5}'s conclusion, I argued that scholars require a better understanding of the varieties, or linguistic resources, as I called them, on which a literizer necessarily relied for a \textit{scriptus}’s creation if we hope to understand German’s earliest \textit{scripti} and the linguistic variation they exhibit. Thus, the task for this chapter is to elucidate how elaborated orality is structured. My approach here is different from the usual ways in which linguists and philologists have examined the German language’s prehistory, which has been conceptualized by way of the comparative method and linguistic reconstruction. This approach entails looking backwards from the standpoint of the language’s literized stages to project its oral past. Doing so makes sense in that there can be no direct evidence of German’s prehistory. Only the literate traces remain. I propose, however, that such methods invite anachronism in that they do not take into consideration literization as an impetus for radical linguistic change. The historical traces of a language are definitionally literate, that is, result from processes of literization like ausbau, while prehistoric varieties were definitionally oral and thoroughly shaped by the phonic medium. If we only apprehend an exclusively oral past through its literate fossils without considering the fundamental structural differences between orally organized and literized language, we risk imposing our own literacy-shaped views onto unliterized language.

The goal of this chapter, then, is to elucidate German’s prehistory without referring to its literate history, that is, to try to understand it on its own terms. Prehistoric German speakers, like everyone else, were multilectal; crucially, all their linguistic production was oral. This fact eliminates one axis of variation that applies to literized languages: the medial binary between phonic and graphic. But the spectrum of communicative contexts, ranging from immediacy at one pole to distance at the other, shaped all linguistic output, just as it does today. Thus, I aim to account for prehistoric structures by centering the phonic and considering how speakers constructed sound based language in its varying, but always exclusively oral contexts. My starting point, then, is the following assumption: speakers with no notion of written languages will apprehend their vernacular through primarily phonic means. Similarly, organizational strategies on which speakers must rely for linguistic production in contexts of communicative distance will also be sound based. For this discussion, I draw from Wallace \citegen{Chafe1994} monograph, \textit{Discourse, consciousness, and time}, which conceptualizes the structure of spoken language as the convergence of linguistic expression and the “flow of consciousness.” The challenge of elaborated orality, then, entails managing and slowing this rapidly flowing stream of language and thought to allow for the planning of language, the developing of ideas, and the telling of coherent narratives. The strategies that people adopt for meeting this challenge are universal in that they are determined by human cognition. Thus, elaborated orality across communities will have structural similarities.

This chapter is divided into two main sections. The first of these (\sectref{sec:6.1}) elucidates the intonation unit, or IU, as the main organizational unit of spoken language, be that a spoken language of immediacy or distance. Building on \citet{Chafe1994}, I define the IU as the linguistic expression of a speaker’s focus of consciousness. The IU is bounded by prosody and cognition, both of which are biological constraints. With respect to prosody, speakers produce language in spurts constrained by the need to draw breath; with respect to cognition, human brains have a restless and limited focus of consciousness. Thus, the IU is where the two converge. I propose, then, that the IU is the locus of a speaker’s modulation of their linguistic production to fit the contexts of immediacy or distance. In this chapter, my primary focus is on the strategies speakers use to address the challenge of processing oral varieties of distance, something I collectively refer to as “layered elaboration.” I illustrate two of these strategies, “reverbalizations” and “nominalizations,” with data taken from the unliterized North American language Seneca and explain how they build the lexically denser and more integrated utterances required in distance contexts. In this way, elaborated orality works to similar ends to those of ausbau, which also effect greater integration and density. However, layered elaboration is always subject to and shaped by the cognitive constraints of its exclusively phonic medium. The phonic medium itself ensures that its linguistic output will never evince the unattainable and, not coincidentally, unnecessary degree of lexical and grammatical coherence effected through ausbau and required in the graphic medium. It also requires speakers to undergird the layered elaboration they build with their utterances with mnemonic devices that make linguistic production easier for speaker and listener to process.

In this chapter’s second main section (\sectref{sec:6.2}), I trace the structural fossils of layered elaboration in the longer of the two extant examples of textualized orality produced by the Carolingians, the \textit{Hildebrandslied}. Here I argue that the basic poetic unit of traditional verse stemming from elaborated orality is a formalized verse-IU (intonation unit). In the case of Germanic alliterative poetry, I propose that the verse-IU is the main locus for the building of its mnemonically grounded, lexically denser, and more integrated language of distance. I also note in this section that I am far from the first to describe this variety of language that I call layered elaboration. It has, in fact, been the object of scholarly interest for more than two millennia, though these discussions have generally not recognized layered elaboration’s cognitive basis, Egbert \citegen{Bakker1997} monograph, \textit{Poetry in speech: Orality and Homeric discourse}, being one notable exception. It has, however, long been associated with different types of ancient poetry. For example, I discuss in \sectref{sec:6.2.2}, how Aristotle described the \textit{lexis} of the ancient poets, that is, their formal speech and artistic prose, in ways similar to the mode of expression I call layered elaboration. Aristotle saw layered elaboration~-- composing as the ancient poets did~-- as resulting only from the composer’s stylistic choice. It was one which Aristotle discouraged: he deemed its structures poor style for unmetered \textit{logos} (loosely translated as ‘language,’ but see footnote~\ref{fn:2:10} in \sectref{sec:2.2.1}) in that they did not foster the clarity of expression this type of composition, in his opinion, required.

This discussion, I argue, suggests two conclusions. The first is that the \textit{scriptus} that is created by an early literizer who wanted to evoke oral composition by employing layered elaboration to effect lexical density and integration will also exhibit less ausbau. That is, a literizer like the \textit{Hêliand} poet employed conceptually oral strategies for dealing with communicative distance, while the graphic medium of the \textit{scriptus} actually requires a conceptually literate solution to address the oral vernacular’s lack of lexical and grammatical coherence. The resultant \textit{scriptus}, I propose, will be less coherent and more ambiguous in its grammatical relations than the \textit{scriptus} created by a literizer like Otfrid, who was inspired by their Latinate literacy, a tradition which itself was rooted in Greek grammatical treatises. This is because a more orally organized \textit{scriptus} for which the literizer drew extensively on their oral tradition will rely on various types of parallelism, i.e., the simple juxtaposition of similar types of constituents with no syntactic linking between them. The process of ausbau, however, replaces those implied connections with explicit ones. I argue that the classical discourse on \textit{grammatica} recognized the difference between these two modes of expression: the older, conceptually oral juxtaposition of verse-IUs, on the one hand, and the newer, conceptually literate ausbau structures created through literization. In turning away from the oral tradition, Otfrid draws more directly on the intellectual and metalinguistic foundations of classical linguistics.

\section{The structure of elaborated orality }\label{sec:6.1}

Identifying and accounting for early German’s exclusively oral distance varieties might seem like an impossible task in that there is no extant tradition of elaborated orality in the German language. However, I propose that one is able to draw conclusions about the structure of a long-gone elaborated orality by considering how the dual communicative pressures of distance and memorability shape linguistic production in an exclusively phonic medium. Instrumental in this analysis is the work of Wallace Chafe, whose writing on the relationship between the production of spoken language and the “flow of consciousness” highlights the central functional challenge of elaborated orality. Namely, the ephemeral nature of the spoken word combines with the significant cognitive limitations on how many thoughts can stay active in human consciousness and for how long; the oral tradition, therefore, is built around creating cognitive and linguistic strategies for making the impermanent less so.

Consider again \citegen[23]{KochOesterreicher1985} schematic representation of the interrelatedness of communicative conditions and constraints, on the one hand, and verbalization strategies, on the other hand. I first discussed these ideas in \sectref{sec:3:3.2} when introducing conceptual orality and literacy; returning to the figure now in an adapted form is useful for understanding the challenge of language planning in mostly or exclusively oral communities and the linguistic strategies humans developed for meeting them.

\begin{table}
\caption{The continuum of linguistic production, in mostly or exclusively oral cultures}
\small
\label{tab:6:6.1}
 \fittable{
\begin{tabular}{lll}
\lsptoprule
&                 \textbf{Language of immediacy} &   \textbf{Language of distance}\\
\midrule
1. Communicative               &    Spontaneous                          &    Planned                                           \\
 conditions/                               &            Dialogue                              & Monologue                                            \\
  constraints                              &           Familiar interlocutor                 & \textbf{Less familiar} interlocutor                  \\
                                &         \textbf{Physically present interlocutor}& \textbf{Physically present interlocutor}             \\
                                &         Free topic development                  &   More topic fixation                                \\
                                &         Intimate context                        &    Public context                                    \\
                                &         More context-dependent                  &    Less context-dependent                            \\
                                &       More expressive, emotional                &    More objective                                    \\
                                &       \textbf{Need not be memorable}            &      \textbf{Must be memorable}                      \\
\tablevspace
\mbox{2. Verbalization}      &     \textbf{Less mnemonic (poetic)}   &  \textbf{More mnemonic (poetic)}                               \\
 strategies                               &     Processuality                     &   Reification                                                  \\
                                &   Preliminarity                       &   Conclusiveness                                               \\
                                &     \multicolumn{2}{c}{\textbf{Less}  \hfill    \textbf{Information dense}  \hfill  \textbf{More}  }                                     \\
                                &     \multicolumn{2}{c}{\textbf{Integration}}\\
                                &     \multicolumn{2}{c}{\textbf{\st{Compact, complex, elaborate}}}\\
\lspbottomrule
\end{tabular}
}
\end{table}

\tabref{tab:6:6.1} is a version of \citegen[18; 23]{KochOesterreicher1985} figures 2 and 3. My adaptations are bolded to reflect my arguments in \chapref{sec:chap:3}.  For instance, I argued in \sectref{sec:3:3.2} that the type of complete physical dislocation of speech act and person that can be achieved in the graphic medium is impossible in the phonic one. The latter is always embodied; the former has the potential to become disconnected entirely from the person who produced it and find its way into the hands of a complete stranger long after the author is dead. Also  in \sectref{sec:3:3.2}, I discussed \citet{Ong2012} [1982], in which the author identifies common mnemonic strategies that make possible the survival of the songs, stories, etc. of an oral tradition. I pointed out that these mnemonic strategies strike modern speakers of German and English as poetic language and concluded that the language of elaborated orality was “crucially poetic.” In this same section, I also discussed the verbalization strategies identified in \citet{KochOesterreicher1985} that connect to syntactic characteristics: information density, integration, compactness, complexity, and elaboration. There I concluded that, of the five listed, only information density and integration would be useful as structural markers distinguishing the language of immediacy from that of distance.\footnote{{This is not to say that there no other features that similarly relate to syntax and correlate with whether the utterance is produced in a context of immediacy or distance. In this book, however, I focus on these two and leave it to additional research to identify others.} } The challenge of an elaborated orality, stated in the terms of the table above, is how the speaker can create the more information-dense and integrated language required by the distance context in ways that are mnemonically supported.

\subsection{The intonation unit and its cognitive basis}\label{sec:6.1.1}

I draw now on the work of Wallace Chafe, that is, \citet{Chafe1980, Chafe1981, Chafe1987, Chafe1994}, to elucidate how speakers of exclusively or mostly oral vernaculars can build lexical density and integrated structures into their varieties of distance. Understanding this process requires a discussion of Chafe’s ideas on the constitutive role played by consciousness and prosody in the production of spoken language. Namely, he identifies the “idea unit” \parencite{Chafe1980, Chafe1981}, later changing the name to the “intonation unit” (\citeyear{Chafe1994}), as a defining unit shaping all spoken language. In arguing for the significance of the IU, Chafe seems to advocate for a different view of spoken language to the one offered in works like \citet{MillerWeinert1998}, in which the authors posit that a syntactic unit, that is, the clause, underlies all spoken production. They do this despite the often fragmented syntax of spontaneously spoken language, which does not always feature coherently constructed clauses. \citet{Chafe1994} is also reluctant to deny the clause a constitutive role in shaping spoken utterances. On pages 65--66, he indicates that in his data of spoken language IUs and clauses were isomorphic roughly 60\% of the time. He concludes that speakers try to verbalize “a focus of consciousness in the format of a clause” but are not always able to manage it because of constraints on how information is distributed across the IU, an example of which I discuss below. I return to the topic of clauses in \chapref{sec:chap:7}. For now, however, I assume that spoken language is made up of IUs and that the IU, as a cognitively defined unit that is crucial to the production and reception of the stream of spoken language, is the locus of mnemonically grounded strategies for building the more integrated and lexically dense language of a planned orality.

Wallace Chafe’s ``IU'' began its life as the “idea unit” and eventually became the “intonation unit.” This evolution in terminology reflects the fact that these segments of speech are defined with respect to both cognition and prosody. In his \citeyear{Chafe1981} article, Chafe argues that spontaneous spoken language is “produced in a series of spurts” that he called idea units (page 136).

\ea%1
    \label{ex:6:1}
It was … it .. had .. evidently … been under snow, \hfill (IU 1)\\
and just recently melted off,                      \hfill (IU 2)\\
and the mosquitoes were … incredible.              \hfill (IU 3) \\
… So we also left.                                 \hfill (IU 4)
  \z

\noindent Both the name of the unit and its associated example indicate that Chafe does not define the unit in primarily syntactic terms. As \REF{ex:6:1} demonstrates, IUs do not always comprise complete clauses or sentences.\footnote{{These IUs are reminiscent of \citegen[58--59]{MillerWeinert1998} “blocks of syntax,” (discussed in \sectref{sec:4.2.2}), a term that reflects how fragmented spontaneously spoken utterances can be. I return to \citet{MillerWeinert1998} in \chapref{sec:chap:7}.} } Ultimately, however, Chafe settles on a more prosodically driven definition of the IU (“intonation unit”), a decision that, I believe, reflects the fact that its borders are more easily identifiable through prosodic, as opposed to cognitive, means. \citet[15--17]{Simpson2016}, in her review of existing scholarship on IUs, defines them as “segments of speech uttered with a coherent intonational contour.” The other advantage to a prosodically defined unit is that it fits more neatly into the theoretical landscape on prosody. As \citet[17]{Simpson2016} explains, a number of different theories on the flow of speech have already identified units that are similar to the IU, including the “intonational phrase” of Prosodic Phonology (\citealt{NesporVogel2007} [1986], \citealt{Selkirk1984}) and the “intonation group” (\citealt{Cruttenden1997} [1986]). The main difference between the IU and other units, like the intonational phrase, lies in the flexibility of the former, as opposed to the latter’s more rigidly, prosodically defined boundaries \citep[17]{Simpson2016}.

The flexibility in delimiting the intonation unit, i.e., the IU, is important, as \citegen[58; 60]{Chafe1994} discussion of how to identify IUs in data illustrates. He identifies several prosodic markers of the boundaries between IUs: pauses that precede and follow them; a pattern of acceleration-deceleration; an overall decline in pitch level; a falling pitch contour at the end of the IU; and vocal fry at the end of the IU. It is, however, not always possible to unambiguously demarcate all IUs in acoustic data, Chafe argues. He uses the analogy of breaking eggs into a frying pan, after which it is difficult to tell where one egg ends and another begins (page 58). The unit’s inherent fuzziness is the result of the IU also being a cognitively relevant unit. In order to understand this point, one must also consider Chafe’s ideas on activation states within the flow of consciousness.

Chafe understands consciousness as analogous to vision. Both consciousness and vision are similar with respect to focus. A person can focus only on limited information at any given time. There is a “small area of maximum acuity,” a peripheral area of focus, which provides the “context for the current focus but also suggest[s] opportunities for next moves,” and everything else that lies beyond but could “at some time be brought into focal or peripheral consciousness” or vision (quotes are from \citealt[53]{Chafe1994} but see also \citealt[12--13]{Chafe1980}). \citet[11]{Chafe1980} emphasizes the limited capacity of consciousness (and vision); compared to the enormous amount of information available, only a tiny proportion can be active at any given moment. \citet[53]{Chafe1994} refers to information in the peripheral consciousness as semi-active, while the remainder of a human’s informational field is inactive.

Chafe and other scholars have connected humans’ narrow focus of consciousness to constraints on short-term memory. \citet[873]{Croft1995}, for instance, notes that there are correlations between what studies have found to be the average length of IUs and the proposed size of short-term memory, both of which seem to hover between four and six words (\citealt[282]{Altenberg1990}, \citealt[256]{Crystal1969}, \citealt[14]{Chafe1980}). Croft’s interest is in mapping common grammatical collocations, or “constructions,” onto IUs; his study indicates that these constructions and the IUs themselves are expressions of the limitations short-term memory places on processing and that they evolved in this manner based on these cognitive abilities and constraints.\largerpage[-1]

Croft’s constructions have a cognitive basis; they fit into the model of a general cognitive process that scholars have called “chunking.” Chunking involves the fusing together of sequential experiences that occur with repetition (\citealt{Bybee2010}:34; but see also \citealt{Miller1956}, \citealt{Newell1990}, \citealt{Haiman1994}, \citealt{Ellis1996}, and \citealt{Bybee2002a}). \citet[34]{Bybee2010} cites \citet[7]{Newell1990}:

\begin{quote}
A chunk is a unit of memory organization, formed by bringing together a set of already formed chunks in memory and welding them together into a larger unit. Chunking implies the ability to build up such structures recursively, thus leading to a hierarchical organization of memory. Chunking appears to be a ubiquitous feature of human memory.
\end{quote}

\noindent \citet{Bybee2010} argues that this same process is responsible for morphosyntax and its hierarchical organization. Collocation of smaller chunks that occur frequently can form increasingly larger chunks. Chunking, \citet[34]{Bybee2010} explains, is a process that is relevant to both production and perception; speakers rely on chunking to produce fluent language and listeners, to process it. She continues: “[t]he longer the string that can be assessed together, the more fluent the execution and the more easily comprehension will occur.” \citegen{Simpson2016} three empirical studies support Bybee’s argument. The results of her studies also collectively indicate that IUs~-- more so, in fact, than the syntactic unit of the clause~-- act as the domain in which chunking processes occur. Listeners break up the continuous stream of spoken language into chunks that correspond to IUs and whose size is set by limitations on short-term memory. These chunks, shaped by the convergence of intonational contours and cognition, are easier for listeners to process. In sum, they are the most efficient vehicle for a speaker to express linguistically what is active in their consciousness and through their utterance transmit a (hopefully) “reasonable facsimile” of that information into the active consciousness of the interlocutor \citep[63]{Chafe1994}.

The question I would like to explore now is how the IU as the linguistic manifestation of consciousness shapes linguistic output. Understanding this relationship is important to understanding the challenges of language planning in the absence of writing and how people use those same cognitive capabilities they rely on for fluent conversation to plan and produce their oral varieties of distance. In order to engage with this question, I consider Chafe’s “one new idea constraint,” approaching it first from the perspective of how it makes producing elaborated orality more difficult. \citegen[108--109]{Chafe1994} “one new idea” constraint states that in spontaneously spoken language an IU supports no more than one new idea. This constraint can be understood from the perspective of speakers and listeners: invoking more than one new idea per IU is cognitively too challenging for the speaker to spontaneously produce and for the listener to process. Herein lies the artificiality of many of the sentences that turn up in syntactic analyses and textbooks for language learners. Consider the following.

\ea%2
    \label{ex:6:2}

         The brave woman reads a scary book.
    \z

\noindent This sentence is grammatical; however, it is not an utterance that would likely ever be produced in natural conversation. Namely, it contains two noun phrases with attributive adjectives, \textit{the brave woman} and \textit{the scary book}, that is, four content words in total with the possibility that each of those content words expresses completely new information within their discourse context (pages 117--118). It is difficult to imagine naturally occurring circumstances in which three of these four ideas are already accessible or given information when the clause is uttered. Instead, speakers would spread all new information out across IUs, perhaps expressing the referents’ properties as predicate adjectives with copula verbs, e.g., \textit{That book was scary}.

One may see the difficulties in producing and processing a sentence like the one in \REF{ex:6:2} as a result of undue activation costs. Activation costs build on the idea that the focus of consciousness is limited, a state of affairs that applies to speaker and interlocutor alike. In this way, focus is a limited cognitive resource \citep[71--81]{Chafe1994}. When a speaker wants to introduce new information into a conversation, that is, information that was until that point inactive within discourse, there is a cognitive cost for both speaker and interlocutor. The speaker could, of course, ignore the needs of their interlocutor and carry on with their flow of discourse never having established that the interlocutor is following along. In this case, communication would be less effective.\footnote{{Note that ignoring your presumed interlocutor is easier when writing than speaking face-to-face. In the latter case, you receive visual and verbal feedback cues from your interlocutor, perhaps a blank stare, ostentatious yawning, or requests for further explanations. In the former case, readers are generally not positioned over your shoulder declaiming how the footnote that you just wrote makes no sense at all.} } There are no activation costs associated with given information; that information was already active at that moment in discourse. Information can also be accessible, or semi-active, in which case some cognitive coin must be spent to fully reactivate that information, but not as much as would have been necessary had the information been fully inactive, or new. There are prosodic and structural correlates to whether a piece of information is given (active), accessible (semi-active), and new (inactive) within a discourse. Speakers give new ideas more prominence than given ideas. So, new ideas within discourse are more likely to be full noun phrases and/or strongly accented. Given ideas, in contrast, will likely be pronouns and/or weakly accented.\footnote{{Accessible information is usually expressed in similar ways to new information, with accented full noun phrases \citep[75]{Chafe1994}. The special characteristics of accessible information are not pertinent to the discussion at hand.} }

Thus, the linguistic expression of the flow of human consciousness mirrors human consciousness in that it is restless \citep[66--67]{Chafe1994}. Speakers move swiftly from one small, cognitively constrained area of focus to another, a process wherein a speaker, for instance, raises a topic that was inactive, makes it the focus of discourse, and then moves on to some other piece of information, thereby letting the first topic fall into a semi-active or perhaps an inactive state. No single idea does, or indeed can, remain the focus of attention for long, though \citet[66--67]{Chafe1994} also notes that different types of substantive IUs~-- IUs that express lexical information, as opposed to regulatory IUs that control interaction and the flow of discourse (see pages 63--64)~-- will persist in the active state longer than others. For instance, the idea of an event or state verbalized in an IU is relatively transient in the active consciousness and will constantly be replaced by ideas of new events or states. \citegen{Chafe1994} example from page 66 illustrates this point, as well as the way a speaker moves from one idea of an event or state to another.

\ea%3
    \label{ex:6:3}

\ea(A)  …  Cause I had a … a thíck pátch of bárley there,\footnote{{I adopted the symbols Chafe uses: 〈\,´\,〉 indicates a primary accent; 〈\,{\textasciigrave}\,〉, a secondary accent; 〈\,..\,〉, a brief pause; 〈\,…\,〉, a typical pause. The use of 〈\,=\,〉 in \REF{ex:6:3} indicates a lengthening of the preceding vowel or consonant.}}     \hfill (state)
\ex(B)  …   mhm,                      \hfill (regulatory)
\ex(A)  ..     about the sìze of the .. kìtchen and líving room,     \hfill (state)
\ex(A) …   and I went òver ít,                \hfill (event)
\ex(A)  ..     and then,                    \hfill (regulatory)
\ex(A)  …   when I got dóne,                  \hfill (event)
\ex(A) I had a little bit léft,                  \hfill (state)
\ex(A) ..     so I tùrned aróund,                 \hfill (event)
\ex(A) and I wènt and spràyed ìt twíce.            \hfill (event)
\ex(A)  ..     and ìt’s just as yèllow as … can bé          \hfill (state)
\z
\z

\noindent In contrast, ideas that express referents, or the “participants in events or states” persist longer in human consciousness (pages 67--68). In Chafe’s example in \REF{ex:6:3}, the referent ‘I,’ which is to say, the idea of the speaker, is included in the state idea in (\ref{ex:6:3}a), and the referent verbalized as ‘thick patch of barley’ in the same IU remain active participants in the expressions of later events and states, e.g., (\ref{ex:6:3}d), (\ref{ex:6:3}i), and (\ref{ex:6:3}j).

\subsection{Processing elaborated orality within the domain of the IU}\label{sec:6.1.2}

The discussion in \sectref{sec:6.1.1} is directly relevant to the challenge of producing and receiving oral distance vernaculars. To wit, the constraints associated with \textit{spoken} language on the one hand, and those associated with \textit{distance} language on the other hand, are contradictory. The limitations of human short-term memory mean that its spoken, linguistic expressions, which is to say, IUs, must be economical in how information is distributed across them. Parceling out information that is new in discourse across multiple IUs, for instance, ensures ease of production and reception. However, the communicative conditions and constraints on distance language push linguistic expression in the opposite direction and demand from the speaker and listener linguistic feats that appear to lie beyond human capability. If human consciousness is restless and has a small area of focus, which itself is defined by a biologically constrained short-term memory, how can speakers produce and listeners process distance-shaped spoken language, i.e., an elaborated orality that is capable of relating complex, culturally-central narratives, and thereby convey important information gleaned from generations of lived experience?\footnote{{It is important to recognize that elaborated orality is not simply a challenge for the trained poet, bard, or storyteller. In fact, it presents a cognitive challenge for the people who} {\textit{listen}} {to distance varieties, which in mostly or exclusively oral cultures is everyone. \citet[134--135]{Ready2019} serves as a reminder that audiences are active participants in the performance of a work of oral art. For example, performers want to “please” their audience (Ready cites here \citealt[167]{Diop1995} and \citealt[126--127]{Jensen2011}. They also vary their performances according to their audience (\citealt[27]{Kaivola-Bregenhøj1996}, \citealt[132]{Jensen2011}, \citealt[66]{Okpewho2014}). See also \sectref{sec:3:3.2.2} for my discussion of \citegen{Lord2000} observations on the influence audiences have on poets’ performances. These listeners are active participants, not passive receivers; thus, the language of elaborated orality is and must be something that audience members can and do process.} } For this type of linguistic expression, people must be able to linger on topics and the usually transient ideas of referents and, most particularly, events and states; they must be able to create lexically dense and integrated language.

  The question is: How do speakers and listeners meet the challenge of elaborated orality? Its answer is complex, but a good place to start is to emphasize the fact that, in order to produce and process planned oral varieties, people in exclusively oral cultures must exploit the same cognitive and linguistic capabilities they rely on for other more immediacy-shaped spoken varieties. There is no writing to ameliorate the cognitive burden of planning language. Furthermore, the strategies that a writer relies on to create lexically dense, integrated writing in, say, an academic manuscript are ill-suited to and generally unattainable in any varieties produced in the phonic medium. In this section, I discuss two linguistic processes, nominalization and reverbalization, that effect more lexically dense and integrated utterances in elaborated orality and allow speakers to slow down the flow of consciousness, as it were, by keeping ideas of referents, events, and states active longer than is the case in spontaneous conversation. These tools give speakers more time to develop a topic (‘topic fixation’) while easing the processing burden for listeners who, as a result of this slowing down of discourse, no longer have to follow the speaker’s restless movement from one topic to the next. This more elaborated language, however, also strains speakers’ and listeners’ short-term memories in that it involves utterances~-- IUs, in particular~-- that could be too lexically complex to process in real time. For this reason, people must rely on what I will refer to as mnemonically-supported, or even a mnemonically-driven, chunking to produce and process oral varieties of distance. As a reminder, chunking is the recursive construction of a hierarchically organized inventory of routinized collocations. Elaborated orality results from this same cognitive process but with a particular reliance on, or attention paid to, the building of chunks whose memorability is reinforced through mnemonic means, not just through frequency.

I begin, then, with a discussion of nominalization and reverbalization, the two processes identified in \citet[68]{Chafe1994} that can ameliorate a language’s transience by allowing ideas of events and states to persist in active consciousness and across more than one IU. Chafe’s example on pages 67--68 illustrates both strategies.

\ea%4
    \label{ex:6:4}
\ea(A)  … Have the ánimals,
\ex(A) … ever attacked anyone ín a car?
\ex(B)  … Well I
\ex(B)  well Í hèard of an élephant,
\ex(B)  .. that sát dówn on a {\textasciigrave}VẂ one time.
\ex(B)  … There’s a gìr
\ex(B)  .. Did you éver hear thát?
\ex(C)  … No,
\ex(B)  … Some élephants and these
\ex(B)  … they
\ex(B)  … there
\ex(B)  these gáls were in a Vólkswagen,
\ex(B)… and uh,
\ex(B)  … they uh kept hónkin’ the hórn,
\ex(B)  … hóoting the hóoter,
\ex(B) … and uh,
\ex(B)  … and the .. élephant was in frónt of em,
\ex(B)  so = he júst procèeded to sit dówn on the {\textasciigrave}VẂ.
\ex(B)  … But théy .. had .. mánaged to get óut first.
\z
\z

\noindent First, the event of the participants honking the horn is expressed in (\ref{ex:6:4}n) and then reverbalized in (\ref{ex:6:4}o), with a similarly alliterating verb phrase no less. This strategy allows the speaker to dwell on the event longer. Speakers can also keep the idea of an event or state in their and, thus, in the interlocutor’s, active consciousness through nominalization, which is evident in (\ref{ex:6:4}g), in which the speaker turns the events expressed in (\ref{ex:6:4}c--f) into the referent \textit{that}. The nominalization of already activated ideas of events or states lets speakers and listeners conceptualize them “as if they had a temporal persistence. Once an event or state has been given this derived status as a referent, it may then, like other referents, participate in and persist through a series of other events or states” \citep[68]{Chafe1994}.

\begin{sloppypar}
Nominalization in Modern English can be achieved through other means, namely, derivational morphology and complementation. Both processes can effect greater lexical density and integration either within the clause or the IU. The following example of derivational morphology comes from \citet[137]{Chafe1981}.
\end{sloppypar}

\ea%5
    \label{ex:6:5}

          One \ul{tendency} is the \ul{preference} of speakers for \ul{referring} to entities by \ul{using} words of an intermediate degree of abstractness.
    \z

\noindent The first significant feature of the sentence in \REF{ex:6:5} is that it conveys all information via nouns rather than finite predicates. In fact, it only has one lexically light copula verb and no agentive subject. One particularly notes the presence of derived, non-finite verb forms, \textit{referring} and \textit{using}, both of which convey the idea of a predicate but are nominal in function. Additionally, the nouns \textit{tendency} and \textit{preference} are derived from verbs. Integrated into this single sentence is the idea of multiple predicates, \textit{to tend}, \textit{to prefer}, \textit{to refer}, \textit{to use}, yet the sentence itself consists entirely of content-rich nominal forms and a content-poor verbal form. Unfurling all of these embedded predicates into individual clauses would leave us with a sequence looking something like the one in \REF{ex:6:6}.

\ea%6
    \label{ex:6:6}
Speakers use words of an intermediate degree of abstractness.\\
Speakers refer to entities in this way.\\
Speakers prefer this.\\
We observed this tendency.
    \z

\noindent This sort of nominalized integration makes sentences like the one in \REF{ex:6:6} difficult to produce and process because so many ideas of different predicates and their assumed agents are subsumed within its single syntactic unit. This is why such sentences are more characteristic of formal written language and would not be produced in immediacy contexts. Another means of nominalization in English is through the use of the complementizer \textit{that}, as example \REF{ex:6:7} illustrates.

\ea%7
    \label{ex:6:7}
         We noticed \textbf{that} you weren’t at the party.
    \z

\noindent \textit{That} signals that the clause that follows is embedded in the preceding clause. In a more literal sense, the two clauses can be read as: ‘We noticed something, that thing being that you weren’t at the party.’ In this respect, \textit{that} is a nominalization of, and stand-in for, an entire clause.

\subsection{Elaborated orality in an unliterized vernacular: The case of Seneca}\label{sec:6.1.3}

  I now turn to how nominalization and reverbalization unfold in a language that, first, is not English and, second, is unliterized and, thus, has distance varieties that can still be characterized as elaborated orality. I have chosen samples of elaborated orality from \citegen[185--186]{Chafe2014} grammar of the Seneca language. The data come from a transcription of a story told by Solon Jones of the Cattaraugus Reservation of the Seneca Nation of Indians on May 7, 1957 that describes the origins of Seneca False Face masks. Seneca is a Northern Iroquoian language indigenous to New York State. It is spoken by people who refer to themselves as Onödowá’ga:’, or ‘those of the big hill’ \citep[1]{Chafe2014}. At the turn of the twenty-first century, no more than a few dozen people could speak Seneca fluently. Ethnologue classifies Seneca as a “dying" language, which means that the only fluent users of the language are beyond child-bearing years, making it too late to “restore intergenerational transmission through the home” \citep{EberhardEtAl2020}.\footnote{{Efforts are underway to increase everyday use of Seneca among the people living in Seneca territories in western New York State. See} \url{https://senecalanguage.com/seneca-language-departments-programs/} {for information on their Seneca language and cultural programs.} } Seneca’s moribund state, as is the case with many other Native North American languages throughout the United States and Canada, can, in no small part, be attributed to children having been removed from their homes and sent to government- and missionary-run boarding schools, where they were forced into cultural assimilation and punished for using their native languages.\footnote{{See} \url{https://americanindian.si.edu/nk360/code-talkers/boarding-schools/} {for more information on the experience of living in these Indian boarding schools.} }

In that Seneca is, and was in 1957, an unliterized oral vernacular with no codified standard variety, it is a good source for data on the linguistic structure of varieties of elaborated orality. The other advantage to using data from Seneca is the fact that Wallace Chafe published extensively on the language. He recorded, transcribed, and glossed the examples that I discuss in this section, and my analysis of these data was supported by the freely available grammar he also compiled.\footnote{\citegen{Chafe2014} grammar is available for free at \url{https://senecalanguage.com/grammar-seneca-language-wallace-chafe/}, where one can also access transcriptions of three spoken “texts.”} As a linguist who was interested in the analysis of the grammar of discourse and worked extensively with the concept of the IU, Chafe’s transcriptions are sensitive to the boundaries between IUs. Because I also adopt for this study the IU as the basic unit of spoken linguistic expression, it made particular sense to consider structures from Seneca.

  One final point of justification is in order, that is, an explanation as to why I do not begin this discussion with an early German text such as the \textit{Hildebrandslied}, as a culturally closer testament of the elaborated orality of Germanic-speaking Europe. First, it bears repeating that the \textit{Hildebrandslied}, like every other text that existed first as an embodied tradition of oral art, is not and cannot be a transcript of elaborated orality. As I argued in \chapref{sec:chap:5}, there is no way to capture faithfully, i.e., through dictation and transcription, the performance of a song, poem, or story. The text that this process creates is fundamentally different from the oral art to which it connects. Rather, the \textit{Hildebrandslied}, like Homeric poetry and \textit{Beowulf}, and unlike the \textit{Hêliand}, which never existed as embodied oral art, is an instance of textualized orality. On the one hand, one might expect that whoever wrote the \textit{Hildebrandslied} down tapped into the more distance-shaped oral varieties of their vernacular to produce their story. Thus, it would not be surprising to find traces of, what one could call, a more orally organized syntax in its \textit{scriptus}. On the other hand, the attested song is the product of some degree of literization, which means that the author would have also engaged with questions of ausbau as a way of making the \textit{scriptus} or text more grammatically and semantically coherent and, therefore, more functional in the dislocated context of writing (see \chapref{sec:chap:4}). Thus, I begin with a discussion of Seneca, for which data on elaborated orality, and not simply textualized orality, exists. In this unrelated and unliterized language, speakers producing distance utterances use reverbalization and nominalization as two distance strategies for building the integrated language that the context requires. I examine the \textit{Hildebrandslied} from the Germanic-language vernacular tradition in \sectref{sec:6.2}.

\subsubsection{Reverbalizations and nominalizations}\label{sec:6.1.3.1}

  Now to two samples of elaborated orality in Seneca. The goal here is to observe the division of the stream of spoken language into prosodically and cognitively defined IUs and the building of more lexically dense and integrated linguistic expressions within this domain. While reading these data, keep in mind the strategies identified in \citet{Chafe1994} that allow speakers to maintain transient ideas, particularly those of events and states, active in consciousness. Speakers can achieve this through reverbalization and nominalization. What also becomes apparent in these data, however, is how speakers use similar strategies to elaborate or, to use Chafe’s term, “amplify,” ideas of referents, events, and states. I begin with the story’s opening lines (see \citealt[185--186]{Chafe2014}). Note that each line corresponds to an IU.

\ea%8
    \label{ex:6:8}
\ea
\gllll Da:h  o:nëh  ëgátšonyá:no:’,\\
      ~     ~      ë-k-athrory-a-hnö-:’\\
     ~      ~       \textsc{fut}{}-1.\textsc{sg}.\textsc{agt}{}-tell.about-\textsc{lk-dist-pun}\\
    so       now     {I will tell about things}\\

\textit{So now I will tell about things,}

\ex
\gllll heh     nijáwësdáhgöh,\\
      ~      ni-t-yaw-ë-st-a-hk-öh\\
      ~   \textsc{part-cis-n.sg.pat}{}-happen-\textsc{caus-lk-inst-sta}\\
      ~ {how it happened}\\

\textit{how it happened,}

\ex
\gllll shagojowéhgo:wa:h,\\
       shako-atyowe-h-kowa:h\\
       \textsc{m.sg.agt}/3.\textsc{pat}-defend-\textsc{hab-aug}\\
       the~great~defender\\

\textit{the false face,}

\ex
\gllll \textbf{ne’hoh}    në:gë:h  odadö:ni:h,\\
       ~               ~       o-atat-öni-:h\\
     ~                 ~        \textsc{n.sg.pat-refl}{}-make-\textsc{sta}\\
    that/there  this      {it has made itself}\\

\textit{that/there it came into being,}

\ex
\gllll \textbf{hë:öweh}   yeyá’dade’                       \textbf{neh}    ö:gweh.\\
          ~             ye-ya’t-a-te-’  ~ ~\\
          ~ \textsc{f.sg.agt}{}-body-\textsc{lk}{}-be.present-\textsc{sta} ~ ~ \\
  where        {they are there}                     namely  people\\

\textit{among the people.}
    \z
\z


\noindent First, consider the semantic relationship between the events expressed in the IUs contained in (\ref{ex:6:8}b) and (\ref{ex:6:8}d). (\ref{ex:6:8}b) refers more generally to the (idea of the) event of how the first false face came into existence, while (\ref{ex:6:8}d) is a more specific restatement and, thus, an elaboration of that same idea. This relationship is also expressed grammatically in the distributive (not plural) suffix *-\textit{hnö}-: in (\ref{ex:6:8}a), which is glossed as the indefinite object ‘things,’ but, in fact, merely implies a patient. The IU that follows in (\ref{ex:6:8}b) elaborates this patient, particularly through the initial particle \textit{heh}. \citet[143]{Chafe2014}, whose grammar identifies \textit{heh} as a general adverbial subordinator, glosses this token as the manner subordinator ‘how.’ Further elaboration of the idea of what happens, that is, how the first false face comes to be, follows in the remaining IUs. Thus, these first lines of the story demonstrate how reverbalization not only is a reiteration of a previously expressed idea, it also layers additional information on top of that initial idea and uses particular grammatical means~-- which in Seneca may involve the interaction between verbal affixes and particles~-- to support this layering.

There are other examples of this interaction between verbal affix and particles in this excerpt, for instance the particle \textit{ne’hoh} in (\ref{ex:6:8}d) \citet[119; 122--123]{Chafe2014} explains that \textit{ne’hoh} is a commonly used deictic. It refers to a whole topic rather than a particular referent and can function as a discourse particle that points to either something that was already said or some location mentioned elsewhere in discourse. Hence, Chafe’s proposed translations of ‘that’ or ‘there,’ respectively. There are two possible readings for \textit{ne’hoh} in this context. First, it could point back to the series of ideas expressed in (\ref{ex:6:8}a--c); this seems to be Chafe’s interpretation in that he translates the particle as ‘that’ in his glossing on page 185. In this case, the particle would make (\ref{ex:6:8}d)’s status as a reverbalization of the idea in (\ref{ex:6:8}b) more explicit through the deictic. I see another reading of \textit{ne’hoh}, however, namely as a deictic reference to the location of the false face’s emergence described in (\ref{ex:6:8}e). According to this reading, \textit{ne’hoh} would be a locative particle that is explicitly elaborated in the subsequent IU and links to its initial adverbial, \textit{hë:öweh}. The reverbalization of \textit{ne’hoh}, ‘there,’ in (\ref{ex:6:8}e) emphasizes the significance of where the false face emerges, which is to say, among people. This event is what leads to the hunter encountering the first false face, becoming dangerously ill, and overcoming his sickness by having false face masks constructed and certain rituals performed. The construction \textit{neh ö:gweh} also indicates that the speaker sought to emphasize the false face’s emergence among people, in particular, and could be evidence that supports the second reading. The pronominal prefix \textit{ye} is ambiguous in that it can refer to either a feminine singular, ‘she’ or ‘her,' or a non-specific/unidentified people \citep[32]{Chafe2014}. The occurrence of \textit{ö:gweh} reverbalizes and elaborates the idea of the referent first conveyed by the pronominal prefix and is emphasized through the particle \textit{neh}.\footnote{Chafe’s use of commas in the transcription is intentional and the fact that there is no comma separating \textit{neh} and \textit{ö:gweh} in (\ref{ex:6:8}e) should indicate that the speaker thought of the pronominal and nominal references to the (idea of the) referent ‘people’ at the same time, rather than including the elaboration of the full NP as an afterthought. In other words, Chafe saw this as a planned construction.} It is possible that the recording of the story reveals whether the storyteller, Solon Jones, intended one reading over the other, and perhaps Chafe’s translation reflects this analysis. Regardless, I think it is also notable that the deictic particle itself~-- and in its context in (\ref{ex:6:8}d)~-- is ambiguous. I propose that \textit{ne’hoh}, in the moment it was uttered, was indeed synchronically ambiguous from both the standpoint of production and perception. The idea of synchronic structural ambiguity is one to which I return in this chapter and the next.

\subsubsection{Layered elaboration}\label{sec:6.1.3.2}

The storyteller’s mode of expression, as illustrated in \REF{ex:6:8}, is what I propose to call “layered elaboration.” New ideas are elaborated through semantically similar or related reverbalizations. These semantic relationships between IUs can be marked grammatically through mutually reinforcing deictic elements, like demonstrative pronouns, adverbs, or, in the case of Seneca, verbal suffixes and particles. In \REF{ex:6:9} I rephrase the content of \REF{ex:6:8} to demonstrate this point.

\ea%9
    \label{ex:6:9}
I will tell you \{things\}\textsuperscript{1}                                         \\
\{About this one thing/how\}\textsuperscript{1} that/it happened                      \\
\{The false face\}\textsuperscript{2}                                                 \\
\{This thing\}\textsuperscript{2} came into being \{there\}\textsuperscript{3}        \\
\{Namely\}\textsuperscript{3}, among the people (where it will do damage)             \\
    \z

\noindent Three things to note in \REF{ex:6:9}: first, my paraphrasing reflects the conclusion that \textit{ne’hoh} is a cataphoric reference to the locative IU of (\ref{ex:6:8}e). Second, I added the parenthetical, ‘where it will do damage,’ to indicate how this emphasis of the false face’s appearance among people sets up the narrative that begins in the next line: a hunter walking through the woods encounters the false face and suffers ill effects. Finally, the subscripts indicate where the narrator links related ideas through deictic markers.

My term of “layered elaboration” is similar to \citegen[152]{Kirk1976} “cumulation.” For Kirk, cumulation refers to the defining style of Homeric verse in which “each new piece of information, as the story proceeds, can be envisaged as being heaped upon its predecessor.” \citet[50]{Bakker1997} provides a good example of cumulation from the \textit{Iliad}.

\begin{figure}
\caption{Cumulation in the \textit{Iliad}}    \label{exfig:6:10}

\includegraphics[width=\textwidth]{figures/Somersinpress-img003.png}
\end{figure}


\noindent Note that Bakker’s presentation of the excerpt in \figref{exfig:6:10} is not the traditional one in which hexameters occupy their own lines. Instead, each line contains a single prosodically defined unit, which one might call IUs, and the boundaries between them align with the metrical breaks at the end of the hexameter line or the middle caesura. This presentation builds on Bakker’s argument that the verses evinced in instances of textualized orality correspond to an “intonational reality” (page 50).

Let us begin by noting some of the similarities in the mode of expression in the Homeric verses and the Seneca data. For example, the excerpt in \figref{exfig:6:10} contains reverbalizations that elaborate the ideas of referents and events that were just introduced: in \figref{exfig:6:10}a--b Antilokhos fights a helmeted man in battle; the idea of this referent, the helmeted man, is explicitly reverbalized in \figref{exfig:6:10}d and identified as Thalusias' son, Ekhepolos.\footnote{{The masculine singluar accusative adjectival appositive in \figref{exfig:6:10}c describes Ekhepolos.} } Similarly the idea of the event in \figref{exfig:6:10}e, which is Antilokhos fatally striking Ekhepolos in the head during battle, is elaborated in a series of overlapping IUs.

\ea%11
\label{ex:6:11}
\begin{tabularx}{\linewidth}[t]{l>{\hangindent=1em}Q}
Antilokhos stikes Ekhepolos                & The event in plain terms\\
On the crest of his helmet                 & General location of strike: forehead to crown of head\\
Drives (some weapon) into his forehead     & Specific location of strike and strike intensity: so hard it is implanted into forehead\\
(So far that) it pierces through the skull & Elaboration of intensity of the strike\\
The weapon is a bronze spearpoint          & Specifics on the weapon used for the strike
\end{tabularx}
\z

\noindent Though Seneca and Homeric Greek are two languages separated by a vast geographic and temporal gulf, they both introduce and expand on ideas of referents, events, and ideas, in similar ways. I see this as a reflection of the universal constraints on the processing of distance oral varieties, which are evident in the structure of elaborated and, to some extent, textualized orality.

With respect to terminology, I prefer “layered elaboration” to Kirk’s “cumulation” because the former links more clearly this mode of expression to the distance varieties of oral vernaculars, collectively called “elaborated orality.” With respect to the concept itself, my notion of layered elaboration is different from Kirk’s cumulation in that he sees this mode of communication as a matter of style, associating it with the particular type of poetry found in the Homeric epics. In contrast, my concept of layered elaboration refers to how people meet the communicative demands of distance within the phonic medium, a challenge that fundamentally shapes the structure of exclusively oral vernaculars. In that these oral varieties of distance feed into the creation of a vernacular’s first \textit{scripti}, the structures associated with them will come to be associated with particular literary or poetic styles. But the structures themselves, I maintain, have a cognitive or psycholinguistic basis.

It is important to recognize how written distance modes of expression differ from this mode of layered elaboration. If one were to edit the Seneca lines in \REF{ex:6:8} and the Greek lines in \figref{exfig:6:10} with the stylistic norms of written Modern English in mind, one would likely see the layered elaboration as unnecessary redundancy and eliminate it.

\ea%12
\label{ex:6:12}
  \ea I will now tell you the story of how the false face came into being among the people.
  \ex Antilokhos struck Ekhepolos right at the crest of his horse-haired helmet, driving his bronze spearpoint so far into his forehead that it pierced through his skull.
\z
\z

\noindent In \REF{ex:6:12}, I edited \REF{ex:6:8} and \figref{exfig:6:10} to minimize redundancies. The resulting sentences have a smoother quality than the versions that build up to the same idea through overlapping IUs and instead have a more fragmented, stuttering quality. My edited versions are also more concise. For example, (\ref{ex:6:12}a) contains only one main clause and one subordinate clause, while still conveying the same basic information. The layered counterparts, though wordier, are more functional as series of spoken utterances, however. They feature precisely those redundancies that good modern prose style eschews, but they are constructed in ways that reflect the processing realities of oral varieties of distance.

I turn now to another excerpt from later in the story of the origins of the false face; this one illustrates nominalization, another strategy of distance orality that turns more transient ideas of events and states into referents that remain active in consciousness longer. Thus, nominalization has a similar effect to reverbalization in that it can slow down the linguistic expression of the flow of consciousness, affording the speaker time to develop a topic and construct a cogent narrative, while also easing the processing burden for the listener. Speakers of Seneca effect nominalization through different means than those observed for Modern English. One device in particular is the particle \textit{neh}, which I first mentioned in the discussion of \REF{ex:6:8} just above. \citet[138--139]{Chafe1981} explains that, while \textit{neh} has no good English translation, it can act like a definite article by nominalizing a subsequent constituent and integrating it into a larger structure. In this way, it functions similarly to the English devices I have already discussed. Chafe also notes that, in his dataset, utterances from what he calls ritual Seneca had almost twice as many \textit{neh} particles as conversational Seneca. In other words, Seneca speakers producing language in contexts of distance, language that is definitionally planned and public, relied on this nominalization device more than speakers conversing in immediacy contexts did. I draw again on \citet[189--190]{Chafe2014}

\ea%13
\label{ex:6:13}
\ea
\gllll Da:h   o:nëh  wá:dihšo:ni’            gagöhsa’,        \\
       ~     ~      wa’-hati-hsröni-’           ka-köhs-a’  \\
       ~     ~ \textsc{fac-m.pl.agt}{}-make-\textsc{pun}     \textsc{n.sg.agt}{}-face-\textsc{nsf}\\
so     then   {they made it}             face\\

\textit{So they made a face,}

\ex
\gllll o’tadiyëöda:ak                 në:gë:h\\
o’-t-hati-yerötar-a-hk-ø               ~\\
\textsc{fac-dup-m.pl.agt}{}-resemble-\textsc{lk-inst-pun} ~\\
{they made it resemble it}             this\\

\textit{they made it resemble this}

\ex
\gllll wá:tšo:wi’             heh    nigáya’dó’dë:h.\\
      wa’-ha-athrori-’        ~       ni-ka-ya’t-o’të-:h\\
\textsc{fac-m.sg.agt}{}-tell.about-\textsc{pun}   ~    \textsc{part-n.sg.agt}{}-body-be.so-\textsc{sta}\\
{he told about it}           {there}  {the way it was}\\

\textit{the shape he told about.}

\ex
\gllll Da:h  tgaye:i’            në:gë:h  wa:diyë’gwahso:nye:t,\\
       ~     t-ka-yeri-’         ~   wa’-hati-yë’kw-a-hsöry-e:ht-ø\\
       ~ \textsc{cis-n.sg.agt}{}-be.right-\textsc{sta}   ~    \textsc{fac-m.pl.agt}{}-tobacco-\textsc{lk}{}-savor-\textsc{caus-pun}\\
{so}     indeed             this   {they burned tobacco for it}\\

\textit{So they burned tobacco for it,}
\ex
\gllll waënödöišök\\
wa’-hën-at-öhisyöhk-0\\
\textsc{fac-m.sg.agt}{}-\textsc{mid}{}-persist-\textsc{pun}\\
{they prayed}\\

\textit{they prayed}

\ex
\gll \textbf{neh}      \textbf{ne’hoh} i:gë:h    sgë:nö’   \\    
       namely          that                who   well-being\\ 
 \gllll  hö:saya:wëh,\\
 h-öö-sa-yaw-ëh-0\\
 \textsc{trans-hyp-rep-n.sg.pat}{}-happen-\textsc{pun}\\
 {it would happen}\\

\textit{that he would get well again,}
\ex
\gll në:gë:h  heh    niyó’dë:h      \\   
this     how   {it is a certain way}\\
\gllll dagáiwadiyö:dë’.\\
  t-a-ka-rihw-a-tiyöt-ë-’\\
\textsc{cis-fac-n.sg.agt}{}-topic-\textsc{lk}{}-stretch-\textsc{ben}{}-\textsc{pun}\\
{it caused him distress}\\

\textit{from what was wrong with him.}
    \z
\z

\noindent The IU in (\ref{ex:6:13}f) begins with the particle \textit{neh}, which operates similarly to \textit{that}{}-complementation in English in that it integrates the idea of the potential event expressed in (\ref{ex:6:13}f), that is, that the man who fell ill after encountering the first false face might become well again, with the idea of the event of praying expressed in (\ref{ex:6:13}e) The particle \textit{ne’hoh}, which has a demonstrative function and means something like ‘that’ or ‘there,’ reverbalizes ideas of events and states elsewhere in discourse \citep[122--123]{Chafe2014}. Thus, one should understand it as a reverbalization of the pronominal neuter, singular, patient prefix -\textit{yaw}- that occurs later in the same IU and not confuse it with the modern English complementizer.

In addition to its example of integration through nominalization with \textit{neh}, the excerpt in \REF{ex:6:13} also contains multiple instances of amplification and/or reverbalization that use deictic elements to link IUs together and effect a more integrated string of utterances. For instance, the implied neuter singular patient ‘it’ in (\ref{ex:6:13}a) is amplified through the noun \textit{gagöhsa’}.\footnote{{Neuter singular patients (or agents) are only overtly marked when they are not also combined with a human agent (or patient) \citep[30--31]{Chafe2014}. Because there is a human agent conveyed through the pronominal prefix *-\textit{hati}-, the patient ‘it’ is implied. Note also that the neuter singular agent prefix *-\textit{ka}- in \textit{gagöhsa}’ does not indicate its theta-role, but rather indicates that its noun root does not refer to a possessed entity, along the lines of ‘my face’ (see \citealt{Chafe2014}: 89).}} The IU in (\ref{ex:6:13}b) reverbalizes the original event idea of making a false face mask, while layering onto this event the new information that they made it so that it would look like the first false face that the hunter encountered in the woods. First, the implied neuter singular patient ‘it’ in (\ref{ex:6:13}b) is amplified through the demonstrative \textit{në:gë:h}. This demonstrative pronoun then sets up the IU in (\ref{ex:6:13}c) which elaborates ‘this’ in the preceding IU to further explain how they made the mask to resemble the face that the hunter described to them.

Here is a summary of the key conclusions from the examples of elaborated orality in Seneca. First, the stream of elaborated orality, just like that of spontaneously spoken language, is organized into IUs (or intonation units), the boundaries between which are prosodically marked. Next, the data show how speakers elaborate the ideas of referents, events, and states through the reverbalization of these ideas in additional IUs. These reverbalizations layer new information on top of the idea that was first introduced, a process that I call “layered elaboration.” Its effect is also to keep the ideas of referents, events, and states active as the focus of consciousness longer, thereby affording the speaker more time to develop these ideas, while also making the narrative easier for the listener to follow. The Seneca examples demonstrate one way to create a more integrated oral language of distance. The speaker relies on nominalization and reverbalization to link IUs with one another, as well as deictic elements, for example, pronouns, or pronominal affixes, and particles that function as demonstrative pronouns and adverbials. These data constitute a window into how speakers organize their stream of spoken language when producing an elaborated orality that meets its cognitive demands of distance.

\section{Traces of elaborated orality in an early Germanic-language \textit{scriptus}}\label{sec:6.2}

  Having examined elaborated orality from an unrelated linguistic tradition, the task now is to assess whether there are any indications that elaborated orality in the Germanic-language tradition might have been organized similarly. As I have already mentioned, there is no possibility of examining any Germanic- or German-language elaborated orality directly as there are no extant traditions. I also argued in \sectref{sec:3:3.2.2} that textualized orality cannot be treated as transcriptions of elaborated orality. This is to say that there are no direct testaments to a German-language oral tradition. The two works of textualized orality, the \textit{Merseburg Charms} and the \textit{Hildebrandslied}, are the best one can do in establishing anything close to a starting point of a literary German. I focus attention in this chapter on the \textit{Hildebrandslied} with the intent of demonstrating that the strategies for processing lexically dense and more syntactically integrated oral vernaculars are relevant to the construction of a \textit{scriptus} for a work of textualized orality and, thus, are key to our understanding of the \textit{scriptus}’s structures. This discussion further elucidates the structural ambiguity that such \textit{scripti} evince. It also marks the beginning of an argument that I develop further in \chapref{sec:chap:7}, where I propose that scholars have too often tried to disambiguate structures that were, in fact, synchronically ambiguous. In other words, we have examined the earliest \textit{scripti} of a language looking for the systematic, grammatically cohesive structures that were effected through~-- and only through~-- centuries of literization and ausbau.

I begin this section by returning to the intonation unit, that is, the IU, of \sectref{sec:6.1} and assessing how this prosodically and cognitively defined unit shapes the structure of the Germanic-language oral tradition. I also trace the elaboration of the ideas of referents, events, and states through nominalizations and, particularly, reverbalizations. That is, the layered elaboration that I identified in Seneca is also present in the \textit{Hildebrandslied}. We will see, however, that the deixis that linked IUs together in the Seneca examples is less evident in the \textit{Hildebrandslied}.

\subsection{The verse-IU of alliterative poetry }\label{sec:6.2.1}
\begin{sloppypar}
My analysis begins with the IU, which I argue is formalized as one of the main metrical units of alliterative poetry, that is, the verse. Alliterative poetry is broadly attested across most of the earliest Germanic languages, including medieval German, English, and Scandinavian, as well as the earlier runic inscriptions; and in textualized oral works like the \textit{Hildebrandslied} and \textit{Beowulf}. As I argued in Sections~\ref{sec:3:3.2.2} and~\ref{sec:5.3.1}, such literized works are not transcriptions of the oral art that inspired them. Yet similarities in the basic pattern of alliterative verse across multiple Germanic languages indicates that elaborated orality, or the planned, public distance language of exclusively or mostly oral, Germanic-speaking communities was organized and processed along similar lines. In order to appreciate the implications of this point, it is important to remember the argument I made earlier in \sectref{sec:3:3.2.1}, that elaborated orality is crucially poetic. The devices that modern-day literate speakers characterize as poetic and associate with a belletristic mode of expression are mnemonic and, thus, served the functional purpose of making the language of elaborated orality more memorable. Similarly, I argue that alliterative poetry’s presentation of textualized orality in verses reflects what was a functionally motivated organization of the linguistic expression of consciousness into mnemonically reinforced IUs.
\end{sloppypar}

First, let us consider how a functionally motivated organization of elaborated orality into formalized IUs, or verse-IUs, differs from the basic units that modern literate people are likely to associate with poetry. This short poem from William Carlos Williams, \textit{This is Just To Say}, illustrates the modern tendency to view verses as visually and perhaps aesthetically defined units.

\ea%14
    \label{ex:6:14}
1~  \parbox[t]{3cm}{I have eaten\\
                    the plums\\
                    that were in\\
                    the icebox
                    }\medskip

2~  \parbox[t]{3cm}{and which\\
                  you were probably\\
                  saving\\
                  for breakfast
                  }\medskip

3~ \parbox[t]{3cm}{Forgive me\\
                they were delicious\\
                so sweet\\
                and so cold
                }\medskip
\z

\noindent Stylistically, the poem is recognizable as poetry primarily through how it is visually presented on the page, that is, in the lines and stanzas that one associates with poetic expression rather than the continuous text of prose writing. The language itself does not have many of the stereotypical hallmarks of poetry; for example, the lines do not rhyme, and neither the word choice nor the syntax is particularly marked or poetic-sounding. The organization of the language into lines and stanzas also does not provide consistent cues for reciting the poem. Note how the line breaks do not always correspond to prosodic breaks. For example, pausing between the third and fourth lines of stanza 1 and separating the preposition ‘in’ from its complement ‘the icebox’ would yield a disjointed-sounding recitation. Yet, pausing between the lines comprising stanza 3 would yield a natural reading, especially before the two appositive modifiers of the plums, ‘so sweet’ and ‘so cold,’ which comprise perhaps the most poetic-sounding lines of the otherwise syntactically straightforward poem. In other words, the poet arranged his language into poetic units according to aesthetic, stylistic or, perhaps, thematic principles, and these units are most consistently represented in the poem’s visual form, rather than in performance or recitation. The poem in \REF{ex:6:14} is just one example of modern poetry. However, it represents how the possibility of poetry existing as written text obscures what I argue is the original functional motivation of poetic units like the verse as the formalized IU of elaborated orality.

In order to examine this functional motivation behind verses, let us now turn to the alliterative verse that characterizes the elaborated orality of the Germanic languages. Here is a brief overview of how it works. Though alliterative verse is defined by its pattern of alliterating stressed syllables, called “staves,” it is the structure of the individual verses that constrains this variety of language in ways that are perceptually salient and consistent with human short-term memory constraints. The three categories of alliterating line are exemplified in \tabref{tab:6:6.2}.

\begin{table}
\caption{Categories of alliterating lines (\textit{Hildebrandslied}, lines 42–44). \textbf{a} = alliterating lift, i.e., stave;
x = non-alliterating lift;
: = pause between verses.}
\label{tab:6:6.2}
\fittable{\begin{tabular}{lll}
\lsptoprule
Category & Alliterative pattern & Examples of verses in lines\\
\midrule
1 & \textbf{a}   x   :   \textbf{a}   x & dat \SomersStackA{\textbf{s}}ágetun \SomersStackX*{m}í   :   \SomersStackA{\textbf{s}}éol\SomersStackX*{í}dante  \\
\tablevspace
2 & \textbf{a}   \textbf{a}   :   \textbf{a}   x & \SomersStackA{\textbf{w}}éstar ubar \SomersStackA{\textbf{w}}éntilseo   :   dat inan \SomersStackA{\textbf{w}}íc fur\SomersStackX{n}ám\\
\tablevspace
3 & x   \textbf{a}   :   \textbf{a}   x & \SomersStackX*{t}ót ist \SomersStackA{\textbf{H}}íltibrant   :   \SomersStackA{\textbf{H}}éribrantes \SomersStackX*{s}úno\\
\lspbottomrule
\end{tabular}}
\end{table}

\noindent First some basics: each verse has two lifts, that is, stressed syllables, yielding four lifts per line. Each line has two or three alliterating lifts, also known as “staves.” That is, not every stressed syllable, or lift, participates in the alliterative structure. Lifts can be separated by unstressed syllables, also called “dips.” I add them now to the schematic representation of the sample lines in \REF{ex:6:15}. This time I have bolded both alliterating and non-alliterative lifts.

\ea%15
    \label{ex:6:15}
\SomersStackD{d}at  \SomersStackA{\textbf{s}}ág\SomersStackD{e}t\SomersStackD{u}n  \SomersStackX{\textbf{m}}í  :  \SomersStackA{\textbf{s}}éo\textbf{l}\SomersStackX{í}d\SomersStackD{a}nt\SomersStackD{e}\medskip\\
\SomersStackA{\textbf{w}}ést\SomersStackD{a}r  \SomersStackD{u}b\SomersStackD{a}r  \SomersStackA{\textbf{w}}ént\SomersStackD{i}ls\SomersStackD{e}o  :  \SomersStackD{d}at  \SomersStackD{i}n\SomersStackD{a}n    \SomersStackA{\textbf{w}}íc  \SomersStackD{f}ur\SomersStackX{\textbf{n}á}m\medskip\\
\SomersStackX{\textbf{t}}ót  \SomersStackD{i}st  \SomersStackA{\textbf{H}}íl\SomersStackD{ti}b\SomersStackD{r}ant  :  \SomersStackA{\textbf{H}}é\SomersStackD{ri}\SomersStackD{br}ant\SomersStackD{e}s  \SomersStackX{\textbf{s}}ú\SomersStackD{no}
    \z

\noindent There is no limit to the number of dips that can occur before, between, and after lifts, which theoretically could lead to verses and lines of indeterminate length. However, there are functional constraints on a verse’s length set by the overwhelming preference for lifts to be content-conveying morphemes and dips to be grammatical morphemes. Consider again our sample lines, this time glossed in \REF{ex:6:16}.


\ea%16
\strutshortanchors{}
    \label{ex:6:16}
\gll \SomersStackD{d}at  \SomersStackA{\textbf{s}}ág\SomersStackD{e}t\SomersStackD{u}n  \SomersStackX{m}í  :  \SomersStackA{\textbf{s}}éol\SomersStackX{í}d\SomersStackD{a}nt\SomersStackD{e}\\
that  said    to.me ~ sea-travelers\\

\gll \SomersStackA{\textbf{w}}ést\SomersStackD{a}r    \SomersStackD{u}b\SomersStackD{a}r  \SomersStackA{\textbf{w}}ént\SomersStackD{i}ls\SomersStackD{e}o  :  \SomersStackD{d}at  \SomersStackD{i}n\SomersStackD{an}    \SomersStackA{\textbf{w}}íc    \SomersStackD{f}ur\SomersStackX{ná}m\\
westward   over  Wendelsea ~   that  him    battle  took\\

\gll tót    ist  \textbf{H}íltibrant  :  \textbf{H}éribrantes    súno\\
     \SomersStackX{d}ead  \SomersStackX{is}  \SomersStackA{H}ilt\SomersStackD{i}b\SomersStackD{r}ant   ~ \SomersStackA{o}f.\SomersStackD{H}e\SomersStackD{r}ibr\SomersStackD{an}t    \SomersStackX{s}on\\
\vskip.5\baselineskip
\textit{Seafarers told me that, westwards over the Wendel sea, that battle took him, Hildebrand, the son of Heribrand, is dead}
    \z

\noindent \citet[55--56]{Suzuki2004} notes how lexical words or morphemes are more likely to be staves or lifts within the alliterative structure, with substantives being the most likely. This category includes nouns and adjectives, as well as substantive and adjectival verbs, for example, \textit{We have \textbf{lost} the battle} and \textit{The battle is \textbf{lost}}. Function words or morphemes, for example, pronouns, copula verbs, and inflection, are the least likely to be prominent within the alliterative structure and, thus, be dips.

These poetic structures associated with alliterative verse break up the stream of sound in prosodically salient ways. That is, stressed lexical morphemes that are (generally) nominal in nature demarcate the verses, while the alliterative pattern demarcates a line of two verses. These structures also effectively limit how long a verse, and thus a line, can be in that lifts are associated with lexical morphemes and dips with functional morphemes. Though dips can be as numerous as the poet wants, the restriction of lexical morphemes, in particular, to two per verse ensures that they will never stretch on indefinitely. The clear prosodic marking of unit boundaries is different from the fuzzier boundaries that exist between the IUs of immediacy-shaped spoken varieties. As I noted earlier in this chapter, even with linguistic training, people are sometimes only able to delimit the broader contours of IUs in spontaneous spoken language rather than its exact shape. It makes a certain degree of sense, however, that planned orality would make use of IUs whose boundaries have been formalized into clearly perceptible patterns.

In this way, I propose that there is a functional, psycholinguistic basis to the verses, or verse-IUs, of this alliterative form of elaborated orality in Germanic. Verses are formalized IUs. As the more fuzzily delineated IU is the basic unit for spontaneous immediacy varieties, so the more clearly demarcated verse is the basic unit of linguistic expression for oral distance varieties. Also like the IU of immediacy varieties, the processing of distance-shaped verse-IUs is constrained by the same limits of human short-term memory. It stands to reason that people (must) rely on the same cognitive abilities, whether they are processing immediacy or distance varieties. The cognitive process on which I have focused so far in this chapter as relevant to the construction of spoken language is chunking. In fact, I argue that the same chunking process underlies the processing of spontaneous and elaborated orality. Chunking, let us recall, is when frequently occurring sequential units coalesce into increasingly complex units. These chunks are stored together in what \citet[7]{Bybee2010} calls rich memory, which includes the “details of experience with language, including phonetic detail for words and phrases, contexts of use, meanings and inferences associated with utterances.” The creation of chunks that can be accessed wholesale makes possible the fluent discourse of spontaneous spoken language. It also allows for the processing of more lexically dense and integrated language that the communicative pressures of distance require, though speakers must build up the required inventory of chunks, that is, conventionalized constructions that exhibit these linguistic features. According to this view, the processing of both immediacy and distance varieties is dependent on memory. However, the more distance-shaped the utterances, the more mnemonic those utterances must be so that people can still process them.

It is worth recalling at this point \citegen[34]{Ong2012} description of the language of elaborated orality, which I first discussed in \sectref{sec:3:3.2.1}. I repeat the quote I presented in that section here.

\begin{quote}{}
[T]hought must come into being in heavily rhythmic, balanced patterns, in repetitions or antitheses, in alliterations and assonances, in epithetic and other formular expressions, in standard thematic settings (the assembly, the meal, the duel, the hero’s helper, and so on), in proverbs which are constantly heard by everyone so that they come to mind readily and which themselves are patterned for retention and ready recall, or in mnemonic form.
\end{quote}

\noindent These mnemonic (or poetic) devices illustrate how chunking can be expressed linguistically in an exclusively or mostly oral community. The most obvious point of intersection between chunking and Ong’s list of mnemonic devices is the formulaic expression. Here Ong is no doubt making an explicit reference to the oral-formulaic theory, first articulated by Milman Parry and further elaborated by Albert Lord. \citet[272]{Parry1971} [1930] describes the formula as “a group of words which is regularly employed under the same metrical conditions to express a given essential idea.” While a discussion of the details of the oral-formulaic approach to oral poetry lies beyond the scope of this chapter, we may note that this basic definition of the formula bears a resemblance to the chunks described in usage-based approaches to language processing. Formulas, like chunks, are collocations that have become conventionalized through repetitive use. Unlike the chunk, however, the formula will make better use of various mnemonic devices, like alliteration, in order to solidify them in the memory. Parry’s definition indicates that formulas are ready-made for slotting into the poetic scheme of the song. Translated into the more functionally oriented terms of this discussion, we might expect a mnemonically reinforced, conventionalized collocation to comprise a verse-IU and feature two stressed lifts or even two staves, that is, alliterating lifts. Consider, for example, these two formulas from early English: \textit{twelfe under tunglum} ‘twelve beneath the stars’ (see \textit{Andreas} 2a) and \textit{sweart under swegle} ‘dark beneath the heavens’ (see \textit{Genesis} 1414a). Both feature two staves and could easily be dropped into a line to fill out one of its verses. \citet[297--298]{Riedinger1985} explains that these commonly attested formulas contribute to the “stylistic tone” of the verse, while adding little in terms of semantic content. They are filler verses whose function is prosodic and stylistic.

\subsection{Reverbalizations, nominalizations, and layered elaboration}\label{sec:6.2.2}

Let us now turn to an excerpt from the alliterative \textit{Hildebrandslied}, as one of the few examples of textualized orality in early German. It is also the only example of epic poetry in early German, though in terms of style, it is unlike other Germanic-language poems, like \textit{Beowulf}. \citet[52]{Bostock1976} describes the \textit{Hildebrandslied}’s language as “terse, simple, and direct,” lacking the epic variation of the early English poem and its other “elaborations of style.” Still, one finds similarities to the Seneca data in how the \textit{Hildebrandslied} poet employs reverbalization and nominalization, thereby layering verse-IUs on top of one another in the service of creating more lexically dense, integrated utterances. My focus is on the poem’s first six lines, first presented with their alliterative structure intact, then glossed, then translated \REF{ex:6:17}.\footnote{{Note that the scribe alternates the spelling of the two protagonists' names, sometimes rendering them as \textit{hiltibraht} and \textit{heribraht}, sometimes as \textit{hiltibrant} and \textit{hadubrant}, respectively. This is one of many consequences of the scribe's ill-advised attempts to translate the poem from a High German into a Low one. My spellings reflect what is found in the manuscript.}}

\TabPositions{0pt,2em}
\ea%17
    \label{ex:6:17}
1 \tab Ik gihorta ðat seggen\\
2 \tab ðat sih urhettun \qquad ænon muotin\\
3 \tab hiltibraht enti haðubrant \qquad untar heriun tuem\\
4 \tab sunufatarungo \qquad iro saro rihtun\\
5 \tab garutun sê iro guðhamun \qquad gurtun sih iro suert ana\\
6 \tab helidos ubar hringa \qquad do sie to dero hiltiu ritun\\

1a\tab \gll Ik  gihorta  ðat  seggen\\
            I  heard   that  tell\\

2a\tab \gll ðat     sih       urhettun\\
           that   \textsc{refl.pro}   challengers\\

2b\tab \gll ænon   muotin\\
           one     met\\

3a\tab \gll hiltibraht  enti    haðubrant\\
            Hiltibrant   and    Hadubrant\\

3b\tab \gll untar     heriun  tuem\\
            between   two     armies\\

4a\tab \gll sunufatarungo\\
            son.and.father\\

4b\tab \gll iro    saro   rihtun\\
            their   armor  prepared\\

5a\tab \gll garutun  sê    iro     guðhamun\\
            readied  they    their   {fighting clothes}\\

5b\tab \gll gurtun  sih        iro     suert  ana \\
            belted   \textsc{refl.pro}  their  swords   on\\

6a\tab \gll helidos   ubar  hringa\\
            heroes  over   rings\\

6b\tab \gll do     sie    to  dero   hiltiu  ritun\\
            \textsc{part}  they   to  the    battle  rode\\

‘I have heard tell how two challengers met alone, Hildebrant and Hadubrant, between two armies, son and father, prepared their armor, they readied their fighting clothes, belted on their swords, heroes over chainmail, then/when they rode into battle.’
    \z

\noindent The poem begins with a common formula used to express the general idea of the oral origins of the story that is about to be related. \citet[56]{Reichl2010} notes how other examples of epic poetry in Germanic, i.e., \textit{Beowulf} \REF{ex:6:18} and the \textit{Nibelungenlied} \REF{ex:6:19}, begin with an idea similar to the \textit{Hildebrandslied}’s ‘Ik gehorta ðat seggen’ (line 1).

\ea%18
    \label{ex:6:18}
Hwæt \textbf{wē} Gār-Dena \qquad in gēar-dagum\\
þēod-cyninga \qquad þrym \textbf{gefrūnon}\\
hū ðā æþelingas \qquad ellen fremedon.\\\medskip

So. The Spear-Danes in days gone by\\
and the kings who ruled them had courage and greatness.\\
We have heard of those princes’ heroic campaigns.\footnote{{This translation is from \citet{Heaney2000}. See \citet{Walkden2013} for an alternate reading of \textit{hwæt} as an underspecified interrogative: “How much we have heard of the might of the nation-kings in the ancient times of the Spear-Danes.”}}
  \z

\ea%19
    \label{ex:6:19}
\textbf{Uns ist in alten mæren} \qquad \textbf{wunders vil geseit}\\
von helden lobebæren \qquad von grôzer arebeit\\
von fröuden hôchgezîten \qquad von weinen und von klagen\\
von ku\"{} ener recken strîten \qquad muget ir nu wunder hoeren sagen\\\medskip

We have been told in old legends many wondrous things (translation from Reichl page 57)
    \z

\begin{sloppypar}
\noindent Contained within that formulaic first line is the demonstrative pronoun \textit{ðat}, which is a cataphoric and deictic nominalization of the entire poem that follows but, particularly, the next nine lines that set the stage with the (idea) of the event of two challengers meeting for single combat. In this way, the event of Hildebrand and Hadubrand’s meeting has been integrated into, or embedded in, the poem’s introductory statement. 
\end{sloppypar}

The next several lines contain examples of reverbalization as layered elaboration. The idea of the referent \textit{urhettun}, ‘challengers,’ is reverbalized twice more, first in 3a’s verse-IU in which their names are provided. Information on their relationship to each other, possibly already hinted at with their alliterating names, Hildebrand and Hadubrand,\footnote{{See \citegen[41]{Jeep1996} comment on alliteration in Germanic naming conventions. See also \citet[24]{Woolf1937}.} } is made explicit in 4a with the compound noun \textit{sunufatarungo}. There are two important points to note about these reverbalizations, one connected to the formulaic \textit{sunufatarungo} ‘father and son,’\footnote{{Many details of the \textit{sunufatarungo} compound remain mysterious. For example, \citet[65--66]{Schürr2013} notes that the -\textit{ung} suffix has no parallels. Within its context, many scholars have been inclined to do as \citet[418]{Lachmann1876} [1833] did and assume that the nominative plural \textit{sunufatarungôs} was the intended form and the attested one, an error (see \citealt{Bostock1976}:44). If one takes the -\textit{o} suffix at face value, as \citet[149]{Haubrichs1988} does, then one must make a genitive plural form fit semantically and stylistically within the context, perhaps along the lines of: ‘that two challengers met one-on-one, Hildebrand and Hadubrand, between\slash among two armies of the son and the father.’ \citet[66]{Schürr2013} points out that this reading makes little semantic sense in that it positions Hildebrand and Hadubrand as kings in their own right, when the poem highlights the latter’s loyalty to his lord, Dietrich, (lines 18--22; 25--26) and Hadubrand’s having a good master, who treats him well (lines 46--48). This reading also undermines the tragedy and dramatic tension of the poem’s narrative, which pits the loyalty between kinsmen against loyalty toward one’s lord. These difficulties lead \citet[28]{Schützeichel1981} to conclude that the poet intended a dative singular form, meaning something like: \textit{in einer Sohn und Vater betreffenden Sache} (‘a father-and-son-related situation’). This solution is also unsatisfactory; it requires that we still assume an inflectional error, but for a reading that makes less sense than the simple nominative plural understanding of the word.} } the other to the way in which the verse-IUs in lines 3--4 (and beyond) link to one another. Beginning with \textit{sunufatarungo}, \citet{Schürr2013} notes how similar coordinative compounds expressing kinship relationships are attested in other early Germanic-language texts \REF{ex:6:20}.

\ea%20
    \label{ex:6:20}
\ea The \textit{Hêliand}, lines 1173--1176\\
furðor quâmun \qquad thô fundun sie [thar] ênna frôdan man\\
sittean bi them sêuua \qquad endi is suni tuuêne,\\
Iacobus endi Iohannes \qquad uuârun im iunga man.\\
Sâtun im thâ \textbf{gesunfader} \qquad an ênumu sande uppen,\\\medskip

They found there an old man\\
sitting by sea and both of his sons\\
Jacob and John; they were still young men\\

There sat \textbf{son-and-father} up on the sand\footnote{{The translation of these lines and the ones from} {\textit{Beowulf}} {in \REF{ex:6:20} is based on the one provided in \citet[67]{Schürr2013}. Both examples stem from \citet[145--146]{Haubrichs1988}.} }

\ex \textit{Widsith}, ll. 45--46\\
Hroþwulf ond Hroðgar \qquad heoldon lengest\\
sibbe ætsomne \qquad \textbf{suhtorfaedran}\\
siþþan hy forwræcon wicinga cynn\\
ond Ingeldes ord forbigdan,\\
forheowan æt Heorote Heaðobeardna þrym.\\
Hroþwulf and Hroðgar held the longest\\\medskip

peace together, \textbf{uncle and nephew}\\
since they repulsed the Viking-kin\\
and Ingeld to the spear-point made bow,\\
hewn at Heorot Heaðobard's army.

\ex \textit{Beowulf}, 1163b-4

þær þa godan twegen \\
sæton \textbf{suhtergefæderan} \qquad þa gyt wæs hiera sib ætgædere,\\\medskip

There the good two (both)\\
sat, \textbf{nephew-and-uncle}, their family still held together then
    \z
    \z

\begin{sloppypar}
\noindent \citet{Schürr2013} raises the possibility that these formulaic father-and-son (or uncle-and-nephew) coordinative compounds might connect to the \textit{sagentypisches Thema} (‘a theme typical of epic poetry,’ \citeyear[68]{Schürr2013}) of kinsmen separated through conflict. In (\ref{ex:6:20}a) ‘Iacobus’ and ‘Iohannes’ will soon leave their old father on his own (\textit{iro aldan fadar ênan forlêtan}). (\ref{ex:6:20}b) and (c) are both references to Hroþwulf and Hroðgar; Hroþwulf, Hroðgar’s nephew, would ultimately usurp the throne from Hroðgar’s children. Finally, Hildebrand and Hadubrand are about to come to blows on the battlefield, the latter not realizing that he means to fight his own father. Thus, as a compound noun this construction is lexically dense; it also connects this story and this particular father-son relationship thematically to the wider constellation of stories that existed as part of the oral tradition, and with which poet and listeners would have been familiar. In this respect, the coordinative compound is also allusion rich. With respect to its prosodic qualities, it can conveniently bear two stressed lifts and comprise a whole verse on its own, if need be, as it does in the \textit{Hildebrandslied} and \textit{Widsith}.
\end{sloppypar}

Interestingly, Seneca has a similar coordinative compound, \textit{yadátawak}, which \citet[124]{Chafe2014} explains has the form of a stative verb and can be translated literally as ‘they are father and son to each other’ \REF{ex:6:21}.

\ea%21
    \label{ex:6:21}
\gll Né:ne:’  wa:ya:jö’s                      neh,   \textbf{yadátawak}\\
    those   {they (masculine dual) visited}   namely  {a father and son}\\
\glt `They visited, a father and his son.'
    \z

\noindent Seneca’s \textit{yadátawak} is but one such kinship term among many; see \citegen{Chafe2014} chapter 9 for a description of this complex system. The point to be made here is naturally limited by the fact that we are comparing two isolated examples from two unrelated languages. The tokens in \REF{ex:6:20}, however, do contain interesting structural parallels in that the copulative compounds are used to elaborate an earlier mentioned referent. The Seneca example shows amplification of the pronominal affix through \textit{neh}, as we saw in earlier examples in this chapter, while the German \textit{sunufatarungo} provides the second elaboration of \textit{urhettun}.

Returning now to the \textit{Hildebrandslied}, let us focus on how the reverbalizations of \textit{urhettun} are layered into discourse. I repeat here the poem’s first six lines for the reader’s convenience, as \REF{ex:6:22}.

\ea%22
    \label{ex:6:22}
1a \tab \gll Ik  gihorta  ðat  seggen\\
             I  heard   that  tell\\

2a\tab \gll ðat     sih       urhettun\\
            that   \textsc{refl.pro}   challengers\\

2b\tab \gll ænon   muotin\\
            one     met\\

3a\tab \gll hiltibraht  enti    haðubrant\\
            Hiltibrant   and    Hadubrant\\

3b\tab \gll untar     heriun  tuem\\
            between   two     armies\\

4a\tab \gll sunufatarungo\\
            son.and.father\\

4b\tab \gll iro    saro   rihtun\\
            their   armor  prepared\\

5a\tab \gll garutun  sê    iro     guðhamun\\
            readied  they    their   {fighting clothes}\\

5b\tab \gll gurtun  sih        iro     suert  ana\\
            belted   \textsc{refl.pro}  their  swords   on\\

6a\tab \gll helidos   ubar  hringa\\
           heroes  over   rings\\

6b\tab \gll do     sie    to  dero   hiltiu  ritun\\
            \textsc{part}  they   to  the    battle  rode\\
\z

\noindent One should note that \textit{hiltibraht enti haðubrant} in (\ref{ex:6:22}-3a) and \textit{sunufatarungo} in (\ref{ex:6:22}-4a) are not explicitly integrated into a clausal structure but instead are in a loose, asyndetic coordination with the verse-IUs surrounding them, i.e., they are appositives. The first appositive in 3a, in fact, occurs between the clause expressed in line 2, \textit{ðat sih urhettun ænon muotin} and the locative prepositional phrase modifier, \textit{untar heriun tuem}. The positioning of the appositive verse-IU in 3a effects for the modern reader, who is used to the smooth prose of a literized ausbau language, a fragmented disjointedness. It also primes the reader or listener to interpret (\ref{ex:6:22}-4a)’s coordinative compound as another appositive elaboration of \textit{urhettun}.

\subsection{Parallelism, deixis, and structural ambiguity}\label{sec:6.2.3}
\begin{sloppypar}
The loose appositive linking of verse-IUs in \REF{ex:6:22}, the reader may have noted, yields a structural ambiguity, which becomes particularly evident when this orally organized construction is used in this graphic medium. Though the idea of the two referents, Hildebrand and Hadubrand, father and son, is active throughout these lines, it is unclear whether \textit{sunufatarungo} is intended to be the grammatical subject of the object + verb collocation in the verse-IU of (\ref{ex:6:22}-4b). Modern readers of the poem, especially those who are literate in languages whose clauses require expressed subjects in order to be considered “complete,” might be inclined to assume a grammatical relationship between (\ref{ex:6:22}-4a) and (4b); otherwise the clause in (\ref{ex:6:22}-4b) would be “missing” a subject. However, the poem contains examples of clauses that have no subject pronouns.
\end{sloppypar}

\ea%23
    \label{ex:6:23}
\ea
\gll du  bist  alter  hun\\
  you  are  old    hun\\

\gll ummet       spaher\\
immeasurably  clever\\

\gll  \textbf{spenis}  mih  mit     dinem  wuotrun\\
goad  me  with  your  words\\

\gll wili    mih  dinu speru     werpan\\
want   me  your spear.\textsc{instr}  throw\\

\gll  \textbf{pist}  also    gialtet  man\\
are  thus    old     man\\

\glt ‘You are an old Hun, immeasurably clever; you goad me with your words, but want to spear me with you spear; so you are an old man’ (39a--41a)

\ex
\gll do     lettun   se     ærist\\
  then   let     they   first\\

\gll asckim       scritan\\
ashen.spears     glide\\

\gll scarpen    scurim\\
sharp    showers\\

\gll dat   in   dem   sciltim   stont\\
that  in   the    shields   stuck\\

\gll do    stoptun   tosamane\\
they   came   together\\

\gll staimbort     chludun\\
bright boards   split\\
\glt  ‘And they first let ashen spears fly, sharp showers, such that they stuck in their shields; they moved toward each other, splitting each other’s bright shields’ (63a--65b)
    \z
\z

\noindent The subject pronouns that one would expect to find in a language that requires overtly expressed grammatical subjects are absent in \REF{ex:6:23}. If early German is like modern German in this respect, the poet should have repeated the second person singular pronoun \textit{du} in one or all of the three clausal IUs that follow its first mention in the first line. Similarly, the poet should have repeated \textit{se}, as in \textit{do stoptun (se) tosamane}. Given that there is evidence in the poem that its poet was not beholden to the rules of Modern Standard German with respect to pronominal syntax, one cannot be certain that the poet intended for \textit{sunufatarungo} to be the subject of the b-verse IU.\footnote{{This footnote is an acknowledgment that much has been written on the topic of} {\textit{pro}}{{}-drop in early German; see, for example, \citet{Eggenberger1961}, \citet{Hopper1975}, \citet{Axel2007}, \citet{Axel-Tober2012}, and \citet{Somers2018}. I do not weigh in on any of the broader claims made in such studies, such as whether the apparently missing subject pronouns in the} {\textit{Hildebrandslied}}{’s lines (4b), 40--42, and 65a constitute cases of referential} {\textit{pro}}{{}-drop, topic drop, or evidence of the apparent unreliability of poetry or the imposition of supposedly foreign Latin syntactic patterns (see, though \chapref{sec:chap:2} for my discussion and refutation of the deficit approach to early German syntax).}}

The structural ambiguity continues in lines 5 through 6a. Here the narrator elaborates how Hildebrand and Hadubrand prepare their gear (\textit{iro saro rihtun},): they ready their armor and strap on their swords~--  in that order presumably~-- though this logical sequence is reflected only in the linear presentation of the two verses. For example, there are no text-organizational words, such as \textit{first} …; \textit{then} … and no overt paratactic linking through some kind of coordinating conjunction. One might interpret the following verse IU, \textit{helidos ubar hringa}, yet another elaborated reverbalization of the idea of the referent \textit{urhettun} in (2a), as the subject of the verse-IU in (5b), which itself has no subject pronoun. Also possible, however, is to treat the idea contained in the verse-IU in (6a) as yet another appositive, occupying yet another a-verse. It elaborates the original idea of the challengers to emphasize not their relationship to each other as the earlier verse-IUs do, but that they are warriors who are now ready for battle after having prepared their armor and weapons. It is not necessarily integrated into a clausal structure as a grammatical subject, though a reader today might feel inclined to integrate the noun phrase into a clause in order to create a syntactic unit that is complete by the standards of modern English or German.

This particular mode of expression, i.e., layered elaboration, was recognized already by the Greeks who characterized it as an ancient poetic “style” \citep[36--39]{Bakker1997}. From Aristotle’s \textit{On Rhetoric} (the second 2007 edition of George Kennedy’s translation, pages 214--218):

\begin{quote}
The strung-on style is the ancient one; for in the past all used it, but now not many do. I call that \textit{strung-on} which has no end in itself unless the thing being said has been completed. It is unpleasant because it is unlimited; for all wish to foresee the end. Thus, as they complete the course [runners] pant and are exhausted; for they do not tire before the goal is in sight.

This, then, is the strung-on style of composition, but the turned-down style is that in periods. I call a \textit{period} an expression having a beginning and an end in itself and a magnitude early taken in at a glance. Such a style is pleasant and easily understood, pleasant because opposed to the unlimited and because the hearer always thinks he has hold of something, in that it is always limited by itself, whereas to have nothing to foresee or attain is unpleasant.
\end{quote}

\noindent Aristotle compares the periodic style, \textit{léxis katestramménē}, or, as Kennedy translates it, the “turned-down way of composition,” to the unperiodic style, \textit{léxis eiroménē}, or “strung-on way of composition.” In the periodic style of composition, ideas are presented in what is characterized as balanced ways and brought to a natural end; in the unperiodic style, they are added to one another and continuous until, to paraphrase \citet[90]{Fowler1982}, the composer simply runs out of subject matter. \citet[252]{Parry1971} connected Aristotle’s comments on unperiodic style to the “cumulative” or “adding style” of Homeric poetry’s verse and sentence structure.

This description matches the structure of the \textit{Hildebrandslied} that I have outlined thus far, namely in how its multiple elaborating reverbalizations~-- of the ideas of the referent \textit{urhettun} and the event of Hildebrand and Hadubrand preparing themselves for battle~-- are “strung onto” one another with little in the way of explicit grammatical signaling of the relationships between these constituents. Aristotle’s definition of periodic and unperiodic syntax is problematically vague and emphasizes the two styles’ different affective impact. Yet, as modern readers we have some insight, I believe, into what Aristotle means with these descriptions. For example, in the introduction to the \textit{Hildebrandslied}, there are no clear clause and sentence boundaries, nor the clearly indicated grammatical relationships between clausal constituents that we expect in, say, modern English prose. Consider how someone practiced in prose writing might edit the poem’s first lines. Here is my attempt:

\begin{quote}
I once heard a story of two challengers named Hildebrand and Hadubrand, who met for single combat between two armies. Though father and son, they prepared their gear for battle, first donning their armor, then strapping on their swords. Afterwards they rode into battle as two chain-mailed heroes.
\end{quote}

\noindent The boundedness of the clauses in this re-imagining of the lines, as well as the clearly indicated relationships between them stands in contrast to the original version, in which clauses seem to continue or overlap with and run into one another. My version also eliminates the structural ambiguities by providing delineated clauses and sentences, though the fact that multiple modern prose versions could be carved from the original speaks to the original lines’ structural ambiguity. I resolved these ambiguities simply by creating connections between constituents and clauses that were not evident or, one might say, \textit{did not exist} before I created them. Anticipating my \chapref{sec:chap:7} discussion of the development of well-formedness as a component of ausbau, I note now how Aristotle attaches stylistic judgments to both structures: the periodic style is pleasing, while the unperiodic style is not. In this way, he is engaging in the ausbau of Greek by trying to establish guidelines for what constitutes well- and ill-formed composition or good and bad style.

\citet[40--42]{Bakker1997} observes that scholars like Antoine Meillet and Pierre Chantraine describe the same unperiodic or strung-on style of composition, but from a syntactic perspective (see, for example, \citealt[358--359]{Meillet1937}, \citealt[598--599]{MeilletVendryes1963}, \citealt[12]{Chantraine1953}). They characterize Homeric syntax as appositional, a structure that ancient Greek inherited from Proto-Indo-European. \citet[40]{Bakker1997} characterizes this argument as follows: “phrases or even single words in Homer tend to have considerable syntactic autonomy, being loosely attached to each other by appositional relationships and having the semantic autonomy of independent sentences.” This is but another way to capture the same phenomenon that Aristotle described. Referring again to the introduction of the \textit{Hildebrandslied}, one sees that its reverbalizations of the referent, \textit{urhettun} are not integrated into, for example, predicate nominative clauses, as in “…~challengers, \textit{who were called} Hildebrand and Hadubrant.” Similarly, no conjunctions link the appositional layered elaborations of \textit{iro saro rihtun} (`they readied their armor') in (\ref{ex:6:23}-5a) and (5b) to surrounding clauses to create a sentence.

This concept of appositional syntax does not offer much in the way of explanatory potential, however. It defines the connection between constituents as the absence of any syntactic connection and offers no positive account of these structures. The hint of a diachronic explanation that these accounts do offer, along the lines of “appositional syntax is inherited and, therefore, old,” invites the facile conclusions reached by the philologists I discussed in \sectref{sec:4.2.2}. Recall how late nineteenth- and early twentieth-century scholars like Henry Sweet and Ernst Windisch characterized the syntax of early Indo-European texts as primitive and unlike the apparently sophisticated constructions found in contemporary German- and English-language texts.

Generative attempts to provide formal accounts of appositions are also focused on the absence of structure and are reluctant to take such data at face value. In his overview of the generative literature on appositional syntax, \citet[1--6]{Griffiths2015} notes how researchers have assumed that an underlying and obligatorily unpronounced structural difference must underlie a supposed functional difference between reformulative and attributive appositions.\footnote{{I refer the reader to \citeapo{Griffiths2015} introduction if they want to learn more about the different underlying structures that scholars have proposed for the two types of appositions. Griffiths recognizes a similar ambiguity in function to the one I discuss in this section but maintains that it is acceptable to assume formal differences that do not connect to actual differences in function.}}

\ea%24
    \label{ex:6:24}

      \ea Dietrich’s righthand man, Hildebrand, recognizes his son on the battlefield.
\ex Hadubrand, a legendary hothead, refuses the extended olive branch.
\ex Hildebrand, the tragic, surrenders his will to fate.
    \z
    \z

\noindent (\ref{ex:6:24}a) is a reformulative apposition in that it provides “additional and often more informative names” for their “anchors,” the term for the constituent to which the apposition refers. Attributive appositions like in (\ref{ex:6:24}b), in contrast, predicate properties of the anchor \citep[1]{Griffiths2015}. The assumption that these two functions are indicative of two different underlying structures runs into problems. To begin with, the function of an apposition is often ambiguous; this is the case with the (imagined) epithet “the tragic” in (\ref{ex:6:24}c), which is both a more informative name for Hildebrand and predicates the property of “being tragic” to him. The fact that descriptive epithets like this one are by their very nature reformulative \textit{and} attributive undermines the premise that distinct appositional functions should link to distinct, unobservable structures. Similarly, the processing of the collocation would depend on the point-of-view of the processor. If the person is part of the community in which the referent, Hildebrand, has been frequently associated with the attribute “tragic,” then the apposition is more reformulative. If the person is unfamiliar with this narrative tradition, the apposition is more attributive. One can imagine the case of a poet saying “Hildebrand, the tragic” as an automatic reformulative collocation, which is heard by an interlocutor as an attributive collocation. The synchronic structural ambiguity in the collocation is only a problem if one insists on the existence of unobserved formal structures. In other words, it is a problem only for the modern researcher who is reluctant to accept the structurally ambiguous data for what they are: structurally ambiguous due to the absence of grammatically explicit linkages, which could only be created through syntactic ausbau.

More insight into these constructions can be gained if, instead of focusing on what is ostensibly missing, one concentrates on what is actually there. I propose that the concept of parallelism, as it has been approached by folklorists and scholars writing on orality, provides a way to understand the nature of the relationship between constituents that have no overt grammatical linkage. Parallelism also has the advantage of being consistent with my account of elaborated orality and the relationship between its structures and the communicative challenge of processing distance varieties only in the phonic medium. Scholars have identified parallelism in poetry, particularly oral poetry, a term that can encompass oral art and textualized orality. \citet[206]{FrogTarkka2017} define parallelism broadly as referring to “a perceivable quality of sameness in two or more commensurate units of expression so that those units refer to one another as members of a parallel group.” One can encounter parallelisms on every linguistic level; that is, sameness can be expressed and perceived, in words, syntactic structure, meanings, sounds, and prosody, including rhythm (\citealt{FrogTarkka2017}, \citealt{Jakobson1966}). For example, alliteration is a phonic parallelism in that it involves the “recurrent returning” (to paraphrase \citealt[399]{Jakobson1966}) of certain sounds, their sameness made perceptible by the repeated sounds’ immediacy to one another. In the Germanic languages, as I discussed in \sectref{sec:6.2.1}, the repetition of onset consonants or vowels, which all alliterate with one another, is constrained by the domain of a line, that is, two verse-IUs.

Investigators of parallelism have long distinguished between semantic and grammatical, also known as syntactic, parallelism. These categories, along with the term “parallelism” itself, originate from the late eighteenth-century work in biblical poetry undertaken by Robert Lowth. Let us begin with a description of semantic parallelism, which involves two or more lexical items that refer to the same idea, e.g., referent, event, or state. Using the terminology that I established earlier in this chapter, semantic parallelisms are reverbalizations of the same idea. Illustrating the fact that parallelism of this sort is attested across the Germanic languages, \citet[207; 212]{FrogTarkka2017} provide the following examples from old Germanic varieties, the first from Old Norse, the second from Old English.

\ea%25
    \label{ex:6:25}
  Semantic parallelism from Germanic\\

\ea
{Þá gengo \{regin ǫll\}\textsuperscript{1} á rǫcstóla,}\\
{\{ginnheilog goð\}\textsuperscript{1}, oc um þat gættuz,}\\
{hvárt scyldo æsir afráð gialda}        ~\\
{eða scyldo goðin ǫll gildi eiga}      ~\\

\gll      Þá     gengo  regin         ǫll       á  rǫcstóla,\\
then  went   powers.\textsc{nom.pl}   all.\textsc{nom.pl} to  {the seats of judgment}\\

\gll ginnheilog       goð,       oc   um  þat  gættuz\\
most.holy.\textsc{nom.pl}  gods.\textsc{nom.pl}   and  on   that  deliberated\\

\gll hvárt   scyldo  æsir     afráð       gialda\\
whether  should   the.Aesir  a.heavy.tribute  pay\\

\gll eða  scyldo  goðin         ǫll   gildi    eiga.\\
or   should   the.gods.\textsc{nom.pl}   all  offerings  receive\footnotemark{}\\
\footnotetext{I glossed the Old Norse example with the aid of the morpheme-by-morpheme break-down found at this link: \url{https://lrc.la.utexas.edu/eieol/norol/90}.}

\glt ‘Then all the powers went to the seats of judgment, the most holy gods, and on that deliberated, whether the Aesir should pay a heavy tribute, or (whether) all the gods should receive offerings.’ (Vǫluspá 23, 1--4)

\ex
þa \{nædran\}\textsuperscript{1} sceop  \qquad \{nergend user\}\textsuperscript{2}\\
\{frea ælmihtig\}\textsuperscript{2} \qquad \{fagum wyrme\}\textsuperscript{1}\\

\gll þa     nædran         sceop  nergend  user\\
then  the.viper.dat.sg  made  savior   our\\

\gll frea    ælmihtig  fagum  wyrme\\
Lord  Almighty  colorful   worm.dat.sg\\

\glt ‘Then our savior, the Lord Almighty, made the colorful worm.’ (Genesis 903--904)
\z
    \z

\noindent \citet[207]{FrogTarkka2017} allude to the fact that there is a syntactic dimension to semantic parallelism. “[P]arallelism is built into the syntax of how language is used.” In their examples, one sees how parallel reverbalizations of an already expressed idea occupy their own verse-IU, constituting its own prosodically delineated building block. The poet constructs their language and elaborates their ideas by adding more blocks. \citet[207]{FrogTarkka2017} note how parallel reverbalizations are dislocated from their clauses and identify this as a hallmark of Germanic-language poetic varieties. In (\ref{ex:6:25}a), for example, the noun phrase \textit{ginnheilog goð} (‘the most holy gods’) occurs outside of what is considered the clausal framework of \textit{gengo á rǫcstóla}, (‘went to the seats of judgment’). It is, however, integrated into the alliterative pattern. That is, there is the coherence of parallel sounds, if not the grammatical coherence modern readers might expect. The same structures are present in (\ref{ex:6:25}b) in that the chiastically arranged parallel reverbalizations \textit{frea ælmihtig} and \textit{fagum wyrme}, while cut out from their clausal frameworks, are integrated with each other through alliteration. This pattern is also present in the introduction of the \textit{Hildebrandslied} \REF{ex:6:26}.

\ea%26
    \label{ex:6:26}
1 \tab Ik gihorta ðat seggen\\
2 \tab ðat sih \textbf{urhettun}  \qquad  ænon muotin\\
3 \tab \textbf{hiltibraht enti haðubrant}   \qquad       untar heriun tuem\\
4 \tab \textbf{sunufatarungo}  \qquad   iro saro rihtun\\
5 \tab garutun sê iro guðhamun  \qquad     gurtun sih iro suert ana\\
6 \tab \textbf{helidos ubar hringa} \qquad   do sie to dero hiltiu ritun\medskip

1a \tab \gll Ik  gihorta  ðat  seggen\\
              I  heard   that  tell\\

2a \tab \gll ðat     sih       urhettun\\
            that   \textsc{refl.pro}   challengers\\

2b\tab \gll ænon   muotin\\
            one     met\\

3a\tab \gll hiltibraht  enti    haðubrant\\
            Hiltibrant   and    Hadubrant\\

3b\tab \gll untar     heriun  tuem\\
            between   two     armies\\

4a\tab \gll sunufatarungo\\
            son.and.father\\

4b\tab \gll iro    saro   rihtun\\
            their   armor  prepared\\

5a\tab \gll garutun  sê    iro     guðhamun\\
            readied  they    their   {fighting clothes}\\

5b\tab \gll gurtun  sih        iro     suert  ana\\
            belted   \textsc{refl.pro}  their  swords   on\\

6a\tab \gll helidos   ubar  hringa\\
            heroes  over   rings\\

6b\tab \gll do     sie    to  dero   hiltiu  ritun\\
            \textsc{part}  they   to  the    battle  rode\\
  \z

\noindent The three parallel reverbalizations of \textit{urhettun} (‘challengers’) each comprise their own verse-UI building block, outside of the structure of the clauses they ostensibly belong to but integrated into the parallel sound patterns of alliteration.

Naturally syntactic parallelism will have a clear syntactic dimension; it often co-occurs with semantic parallelism but need not involve any reverbalizations of ideas. Here is an example from \citet[448]{Frog2017}; it is a kalevalaic description of a fiery eagle that Lemminkäinen, the hero, passes on his way to the otherworld.

\ea%27
\label{ex:6:27}Syntactic parallelism, from kalevalaic epic poetry (SKVR I2 742.119--125, from \citealt[448]{Frog2017})\medskip\\
\begin{tabular}{@{}ll@{}}
1  Jo tuli \{tulini\}\textsuperscript{1} koski                         &      Already came \{a fiery\}\textsuperscript{1} rapids                                           \\
2  Kosell’ on\textsuperscript{2} \{tulini\}\textsuperscript{1} korko   &    On\textsuperscript{2} the rapids is\textsuperscript{2} \{a fiery\}\textsuperscript{1} shoal    \\
3  Koroll’ on\textsuperscript{2} \{tulini\}\textsuperscript{1} koivu   &    On\textsuperscript{2} the shoal is\textsuperscript{2} \{a fiery\}\textsuperscript{1} birch     \\
4  Koivuss’ on\textsuperscript{2} \{tulini\}\textsuperscript{1} kokko  &   In the birch is\textsuperscript{2} \{a fiery\}\textsuperscript{1} eagle                         \\
5  Sep’ on hampahieh hivove    & That one is its teeth grinding         \\
6  Kynsiähä kitkuttauve        & Its claws scraping                     \\
7  Peän varalla Lemminkäisen   &  Ready for the head of Lemminkäinen    \\
\end{tabular}
\z

\noindent The language in \REF{ex:6:27} is structured in a “chain,” or is “terraced,”  \citet[212]{FrogTarkka2017} explain in their article.\footnote{{According to \citet[212]{FrogTarkka2017}, these descriptions stem from \citet[I: 79]{Krohn1918}, \citet[120--122]{Steinitz1934}, and \citet[63--69]{Austerlitz1958}.} } The last word of verse begins the next verse, a rhetorical figure the Greeks named \textit{anadiplosis}. Though no line is a reverbalization of an idea of a referent, event or state that was previously mentioned, the linked structures collectively elaborate a larger event, and each new line is directly linked to a previous one through syntactic “sameness.” The Merseburger charms are a similar example from the German language tradition \REF{ex:6:28}.

\ea%28
    \label{ex:6:28}
         Syntactic parallelism in the history of German\\
1  Phol ende uuodan  \qquad   uuorun zi holza\\
2  du uuart demo balderes uolon  \qquad   sin uuoz birenkit\\
3  thu biguol en sinthgunt  \qquad   sunna era suister\\
4  thu biguol en friia  \qquad   uolla era suister\\
5  thu biguol en uuodan  \qquad   so he uuola conda\\
6  sose benrenki sose bluotrenki  \qquad   sose lidirenki\\
7  ben zi bena  \qquad   bluot si bluoda\\
8  lid zi geliden  \qquad   sose gelimida sin\medskip

\gll Phol   ende  uuodan       uuorun  zi   holza\\
Phol  and     Wotan    went  to  the.forest\\

\gll du     uuart   demo   balderes   uolon     sin uuoz   birenkit\\
then  became  \textsc{det}    Baldur.\textsc{gen}  horse  his foot    sprained\\

\gll thu     biguol   en    sinthgunt     sunna   era suister\\
then  sang   to.it   Sinthgunt  Sunna  her sister\\

\gll thu     biguol   en    friia      uolla   era suister\\
then  sang  to.it   Freya  Volla  her sister\\

\gll thu     biguol   en    uuodan       so   he   uuola   conda\\
then  sang  to.it   Wotan    as  he  well    {was able}\\

\gll sose   benrenki   sose   bluotrenki,       sose   lidirenki\\
be.it    bone.sprain  be.it    blood.sprain    be.it    limb.sprain\\

\gll ben     zi   bena       bluot   si   bluoda\\
bone  to  bone  blood  to  blood\\

\gll lid     zi   geliden  sose   gelimida     sin\\
limb  to  limb  as.if    {stuck together}  be\\

\glt ‘Phol and Wodan went into the forest,

Then Balder’s horse sprained its foot.\\
Then Sinthgunt, the sister of Sunna, charmed it.\\
Then Frija, the sister of Volla, charmed it.\\
Then Wodan charmed it, as he was well able to do:\\
Be it sprain of the bone, be it sprain of the blood, be it sprain of the limb:\\
Bone to bone, blood to blood,\\
limb to limb, as if they were stuck together!’\\
    \z

\noindent Note in particular the parallel syntax of lines 3 through 5, which repeat the collocation, adverbial + verb + pronoun + noun phrase, exactly. As in the Finnish example, though none of these verse-IUs are linked through syntactic sameness, they do collectively elaborate a larger scene. The example in \REF{ex:6:29} demonstrates how semantic and syntactic parallelism can directly reinforce each other; it stems from the \textit{Hanvueng}, a ritual text from southern China. In this excerpt, Covueng tells his half-brother, Hanveung, that their father is ill.

\ea%29
\label{ex:6:29}Syntactic and semantic parallelism, from west-central Guangxi, southern China\\
\{Boh raeuz\}\textsuperscript{1} \{gwn \{raemx\}\textsuperscript{2} lwt\}\textsuperscript{3} \\
\{Boh raeuz\}\textsuperscript{1} \{swd \{raemx\}\textsuperscript{2} rong\}\textsuperscript{3} \\
\{Boh raeuz\}\textsuperscript{1} fuz mbouj hwnj \\
\{Our father\}\textsuperscript{1} \{drinks \{water\}\textsuperscript{2} from a small bamboo cup\}\textsuperscript{3}   \\
\{Our father\}\textsuperscript{1} \{drinks \{water\}\textsuperscript{2} through a rolled-up leaf\}\textsuperscript{3}\\
\{Our father\}\textsuperscript{1}, even if supported, cannot stand up.
(H 666--668, from \citealt{Holm2017}:388)
\z

\noindent As \citet[388]{Holm2017} explains, all three lines share a subject, which is repeated in each line with no anaphoric reference. The first two lines have exactly the same syntactic structure. With respect to content, \citet[212]{FrogTarkka2017} notes that these syntactically parallel verses are semantically similar without expressing precisely the same thing.

Parallelism, I propose, is one relationship that can weld constituents together in oral varieties, particularly those of distance for which humans have to meet the challenge of processing planned, lexically dense, integrated language. This “element-to-element sameness,” as \citet[208]{FrogTarkka2017} calls it, can be expressed on any perceptible linguistic level, that is, patterns of sameness in sound, prosody, meaning, and grammatical or syntactic constructions. For example, it is a semantic and perhaps also structural sameness that connects an apposition to its anchor, not an underlying, unpronounced syntactic structure. Apposition and anchor are only unlinked or loosely linked by the standards of the modern ausbau languages in which the researchers investigating this phenomenon are literate. I propose also that parallelism as a means of indicating which constituents in discourse belong together fits well with the account that I offered earlier in the chapter of the challenge of processing oral varieties of distance. That is, people must manage the rapid, fickle flow of human consciousness while constrained by their limited short-term memories. I argued that the reverbalization of ideas of referents, events, and states is one important strategy for meeting this challenge. The reverbalization itself can manifest as a type of semantic parallelism, but other types of parallelism can elucidate further the connection between the multiple verbalizations of an idea. People engaging in the ausbau of their vernacular, which involves enhancing a written \textit{scriptus}’s grammatical and lexical coherence, look for more grammatically explicit means of expressing those relationships that are implied through parallelism in oral distance varieties. If the literizer, however, draws more on their oral vernacular resources, the \textit{scriptus} they create will likely have more parallelisms.

In treating parallelism as grounded in the cognitive challenge of processing distance varieties in the phonic medium, I diverge from the classic structuralist approach offered in \citegen{Jakobson1966} oft-cited article on parallelism. Jakobson maintains throughout that parallelism is inherently artificial, approvingly quoting \citet[84]{Hopkins1959}, who wrote:

\begin{quote}
    The artificial part of poetry, perhaps we shall be right to say all artifice, reduces itself to the principle of parallelism. The structure of poetry is that of continuous parallelism, ranging from the technical so-called Parallelisms of Hebrew poetry and the antiphons of Church music to the intricacy of Greek or Italian or English verse.
\end{quote}

\noindent According to this view, parallelism is “the cardinal poetic artifice” (page 401) particularly as it is attested across many poetic grammars from many different cultural and linguistic traditions. One might see parallelism’s ubiquity as evidence of the phenomenon’s connection to humans’ psycholinguistic capabilities. Yet, \citet[423]{Jakobson1966} argues that parallelism is not ubiquitous \textit{enough} to result from “a mental automatism” or “mnemotechnical processes upon which the oral performer is forced to rely." He supports his conclusion with the following points. First, he claims that there are whole folk traditions “totally unfamiliar with pervasive parallelism.” Second, he notes that other folkloric systems have different “poetic genres,” which are distinguished through parallelism’s presence or absence. Finally, he mentions written poetry from China that is thousands of years old and has strict parallelistic rules; these rules, however, may be relaxed somewhat in native folklore. Jakobson does not discuss specific examples or identify the traditions that illustrate his points. It is, therefore, impossible to attempt to refute his claims.

It seems to me that Jacobson’s view of parallelism is too narrow in that he does not see it as one possible and, I argue, cognitively grounded means to link together reverbalized ideas of referents, events, or states. Even if one stipulated that he is correct and that there are traditions of oral art that do not have the sort of formalized parallelism that most interests him, this possibility does not invalidate parallelism as a mnemonically useful tool. In fact, the examples of Seneca examined earlier in the chapter did not have parallelism. But they did have reverbalizations, and these were linked together primarily through deixis, rather than structural similarity and the use of appositions. \citet[206]{FrogTarkka2017} offers the following insightful comment on the interplay between parallelism and deixis.

\begin{quote}
Unlike deictic words such as \textit{it}, \textit{this} or \textit{that}, which refer to a preceding stretch of text, parallelism has a formal aspect that allows it to become perceivable without such explicit terms: a parallel member of a group is recognized in part through a formal equivalence to the preceding member as a unit of utterance whether it is a verse line, hemistich, or stanza, or a clause or phrase in a form of discourse that lacks recurrent meter. The deixis or indexicality of parallel members creates formal relations between signs and qualifies as a type of syntax (\citealt[22]{Morris1971}, \citealt[387--400]{DuBois2014}). Recognizing and interpreting those relations relies on perception.
\end{quote}

\noindent Thus, the introduction of the \textit{Hildebrandslied} and the excerpts from Seneca’s elaborated orality illustrate two strategies for knitting together reverbalizations. In the case of the former, the three reverbalizations of the referent \textit{urhettun} evince a semantic sameness and, while occupying their own verse-IUs, are integrated into the sound structure through parallel sounds, i.e., alliteration. A parallelism in the syntactic configuration of the two lines reverbalizing \textit{iro saro rihtun} (‘the prepared their gear’), \textit{garutun sê iro guðhamun} and \textit{gurtun sih iro suert ana} (‘they prepared their armor’ and ‘they belted on their swords,’ respectively), welds these two verses together as elaborations of the preceding event of two warriors preparing themselves for battle. Their structural similarity, not to mention the alliterative structure that stretches across these verses, effects a perceptible link between them. Constituents can also relate to one another more explicitly through the use of deixis, as was the case for the Seneca examples examined in \sectref{sec:6.1}, where the speaker connected reverbalizations of ideas expressed in IUs with deictic pronominal affixes and particles. Thus, different oral distance varieties may rely on one or perhaps both of these two systems of coindexing ideas to create more coherent utterances. For example, it makes some sense that ritual language in particular would make use of syntactic parallelism, as we see in the Merseburger charms, while narratives might feature less structural repetition of this sort and more deixis.

\citegen[206]{FrogTarkka2017} quote makes one final, important point about the use of deixis versus that of parallelism. Namely, in that parallelism indicates the concatenation of constituents without any additional elements to mark these connections, the use of deixis is the more \textit{explicit} strategy for elucidating these same relationships. That is, the latter strategy features grammatical morphemes that indicate relationships that are implicit in parallelisms. Think back now on \chapref{sec:chap:4}, where I argued that syntactic ausbau is one of the main tasks of literization. This human-initiated and -directed process involves the development of more of these explicit means of indicating the grammatical and semantic connections between constituents. It makes sense, then, that it is precisely these functional, deictic markers that would make excellent subordinating conjunctions and relative pronouns in a literizing written variety. Indeed, the etymological link between subordinating conjunctions and deictics and anaphorics in Indo-European languages (see \citealt{Clackson2007}: 172--173) is consistent with the narrative that literizers expanded and elaborated the functioning of an original set of pointing morphemes to suit the needs of a written distance variety.

According to the view I have developed thus far, humans have always connected ideas and constituents together in discourse. This argument is in line with those expressed in, for example, \citet[308--310]{HarrisCampbell1995}, which points out that early written and “unwritten” languages, as they call them, also have subordination and embedding. In accordance with the foregoing chapters, I would edit this statement to include and emphasize the role that literization and ausbau play in changing the ways in which people connect ideas and constituents in discourse: the communicative requirements of distance yield integrated and lexically dense oral varieties. These varieties have linguistic strategies for connecting ideas and expressing logical hierarchies. Among these are parallelism, which implicitly indicates the relationships through semantic or structural sameness, and deixis, which more explicitly co-indexes ideas or constituents. When people begin to write in their vernacular, they must engage in ausbau, which, among other tasks, involves augmenting the vernacular’s grammatical coherence and the explicit means of expressing these relationships. It is the dislocation of the vernacular to the graphic medium that prompts people to make these changes.

\begin{sloppypar}
Recall the disagreement between Harris and Campbell, on the one hand, and more traditional philologists like Henry Sweet, on the other hand (see \sectref{sec:4.2.2}). The latter scholar makes a developmental argument: young and primitive languages are more paratactic and become more hypotactic as they mature and become more sophisticated. The former scholars argue that all languages are hypotactic, and that scholars like Sweet overemphasize the paratactic nature of early written and “unwritten” languages. \citet{HarrisCampbell1995} is correct to point out that all languages, including unliterized ones, have integrated structures. Sweet’s surmise that languages become more hypotactic over time is correct in the sense that highly literized languages express more and more specific grammatical relationships between constituents and do so more explicitly. They rely less on the implicit means of signaling these relationships, like parallelism, which, if viewed strictly from the perspective of syntax, seems to entail only a loose paratactic linking of constituents. Both parties, I argue, miss out on the actual crux of the matter, which is that it is literization itself, and ausbau in particular, that is the impetus for this shift.
\end{sloppypar}

\section{Conclusion}\label{sec:6.3}

The main goal of this chapter was to describe German’s pre-history from a different perspective than that of the comparative method. This goal is based on my contention that one cannot learn about the structure of distance varieties in the phonic medium if one works backwards from written varieties. Prehistoric linguistic varieties are by definition oral and have not undergone any of the fundamental changes that literization sets into motion. Thus, if we want to understand the structural implications of a literizer drawing on their oral varieties of distance in order to create their \textit{scriptus}, we need a way to investigate elaborated orality. This is a difficult proposition as we have no \textit{direct} access whatever to spoken medieval German in that all ninth-century German speakers are long dead. I proposed the following method to deal with this problem.

First, I began by outlining the universal challenges of processing oral varieties of distance. These include slowing down the fickle, fast-moving convergence of consciousness and linguistic expression. The basic unit of language is the intonation unit (IU), which can be defined according to several parameters. The two main parameters are cognitive and prosodic. Regarding the former, natural limits on short-term memory constrain the IU’s length. Regarding the latter, intonational characteristics, like pauses for breaths and changes in pitch, signal the IU’s boundaries. To some extent, IUs are syntactically defined in that they tend to be isomorphic with constituents, though notably not necessarily with complete clauses. The challenge, then, lies in how humans manage to process, i.e., produce and receive, oral varieties that also meet the communicative demands of distance contexts. For example, how do they slow down their restless consciousness to focus on and develop ideas (of referents, events, or states)? How do they process the greater density and integration required in exclusively oral distance contexts?

The nature of the challenge inherent in processing varieties of distance in the phonic medium suggests the strategies for ameliorating it. If the stream of cognition and language is too fast-paced to develop an idea or tell a story coherently, find a way to slow down the flow of discourse and keep ideas active in consciousness longer. The reverbalization and nominalization of ideas of referents, events, and states are two ways to accomplish this goal. Use reverbalizations, for example, to gradually layer more detail on top of the originally stated idea. A nominalization of an event or state, perhaps with some deictic particle linking the nominalization to the original idea, is another way to add more information. I referred to these strategies collectively as “layered elaboration.”

But what about our biologically limited short-term memories? How can we process lexically dense, integrated language? In this chapter, I presented a two-part answer. First, I argued that people rely on the same psycholinguistic abilities for both spontaneous spoken language and oral varieties of distance: chunking. Fluent communication is possible because people build and access an inventory of conventionalized collocations stored in rich memory. The difference between oral varieties of immediacy and those of distance is that with the latter, speakers must rely more on mnemotechnical processes to solidify lexically dense chunks in the memory. For example, one can remember a sequence of words in an IU chunk more easily if there is some manner of repetition, like each word beginning with the same sound, i.e., alliteration. Similarly, one IU chunk can be connected to another more readily if they both evince the same rhythmic pattern, i.e., meter, or mirror each other structurally.

We, modern readers, associate these mnemonic devices with poetry. Thus, it can prove insightful to consider even the most basic elements of poetry, like the verse, from the perspective of how they might have afforded a psycholinguistic advantage to speakers of oral vernaculars. In this chapter I argued that the verse of alliterative poetry, or the verse-IU, is a formalization of the basic unit of spoken language. The verse-IU, unlike the IU of more spontaneously produced language, is clearly demarcated by intonational cues, particularly the pause between verses for taking a breath. It constrains the content of the verse-IU by limiting the number of stressed lifts to two, which effectively limits the number of content-carrying substantives that can occur in the verse. Such features of poetry, once functionally motivated in the production of elaborated orality, leave their traces behind in the early \textit{scripti}. There they no longer serve the same purpose of easing language processing in the phonic medium but instead come to define a literary style of writing, adopted by the \textit{scriptus}{}-creator because that is what they deemed would be appropriate for their literization project.

Though the focus of this chapter was not ausbau specifically, its analyses are still relevant to the process in that they can elucidate the extent to which a literizer both engages with and succeeds in creating a more grammatically and semantically coherent \textit{scriptus}. I argued in this chapter that oral varieties of distance have their own more implicit means for indicating the types of relationships between constituents and ideas. For example, parallelism places two elements in implicit juxtaposition and suggests a link between them. Adding deictic connectors can make these connections more explicit. Such methods, though they are perfectly functional in the phonic medium, are less so in the graphic medium and effect structural ambiguity when they are not augmented through ausbau. Thus, an early German literizer who draws on their oral varieties of distance develops a literary style that has a propensity for structural ambiguity, unless the \textit{scriptus}{}-creator takes care to augment cohesion in a way that these varieties do not require. It takes some imagination to recognize the implications of producing language in a dislocated written variety, including the fact that your text might be read by people who are not from your immediate temporal and sociocultural context. Such an unfamiliar audience, which includes the modern reader, would not be practiced in processing your particular, localized oral varieties of distance, which rely heavily on a shared, community-wide inventory of conventionalized language and concepts. The so-called structural ambiguities that stand out to the modern reader as syntactic puzzles they must resolve escape the notice of, and are perfectly legible to, the ninth-century local. Thus, the literizer whose emerging literary style evokes the oral tradition might see less success in the ausbau of their \textit{scriptus} than a literizer who intentionally moves away from orality and aims to develop in their \textit{scriptus} a more innovative literary style. To rephrase this idea in the terminology of the next chapter, the latter literizer will engage more with the task of establishing a concept of well-formedness for a written vernacular than the former.

