\chapter[Historical linguist seeks good data]
        {Historical linguist seeks good data: The deficit approach to early German syntax}\label{sec:chap:2}

\section{Introduction}
\citet[vii]{Schrodt2004} begins his reference grammar on Old High German syntax with the following summary:
\begin{quote}
    Die Darstellung der althochdeutschen Syntax ist ein Wagnis: Die Verschiedenartigkeit der Sprachregionen, die Abhängigkeit vom Lateini\-schen, die große, mehrere Jahrhunderte umfassende Zeitspanne, der mögliche Einfluss von Reim und Metrum erschweren sehr oft eine aus\-reichende Begründung für die Beschreibung syntaktischer Kategorien. Dazu kommt eine problematische Überlieferungslage. Es ist wohl kein Zufall, dass eine solche Darstellung bisher fehlte.

    The presentation of Old High German syntax is risky. The variable nature of the dialect regions, the dependence on Latin, the large time span stretching across several centuries, the possible influence of rhyme and meter make it difficult to provide sufficient justification for the description of syntactic categories. To these challenges, one may add problematic transmission histories. Indeed, it is no coincidence that such a depiction has been lacking until now (my translation).
\end{quote}

\noindent Thus, the main challenge in Schrodt’s eyes stems from the variability exhibited by the syntactic data. Texts originate from different places across the German-speaking continuum and different centuries. They are manipulated and distorted by extragrammatical influences like a Latin source text or a poetic meter. These factors make it difficult to describe early German\footnote{In this book’s introduction I explained why I discuss “early German,” rather than Old High German or Old Saxon.} syntax.

For example, consider the finite verb, which is attested in many different surface positions in main \REF{ex:2:1} and subordinate clauses \REF{ex:2:2}.

\ea%1
\label{ex:2:1}Main clauses:
\ea Verb-first (Isidor, IV. \textit{De Trinitatis Significantia}, Sentence 8, \citealt{Eggers1964})\\
\gll \textbf{Quhad}  got    see    miin chnecht  ih  inan   infahu  chiminni     mir\\
said     God    behold  my  child     I   him   receive    beloved.one     me.\textsc{dat}\\
\glt ‘God said, behold my child, I will receive him; my beloved one’

\ex Verb-second (Tatian, Chapter 13, sentence 20, \citealt{Sievers1961})\\
\gll thô   \textbf{fragetun}  sie     inan      uuaz   nu    bist   thu     Helias\\
then   asked     they    him      what   now  are   you     Elias?\\
\glt ‘And they asked him, what now are you Elias’

\ex Verb-third (Otfrid L 75, \citealt{Erdmann1973})\\
\gll Állo zíti    thio the  sín   Kríst    \textbf{lóko}          mo       thaz múat sin\\
In.all hours   that he  be  Christ    gladden.\textsc{pres.subj}  him.\textsc{dat}    the heart his\\
\glt ‘In all the hours that he be alive, may Christ gladden unto him his heart.’

\ex Verb-later (Masser 100, 14--16; Tatian, from Luke 7, 22)\\
\gll Her   thó      antlingenti    \textbf{quad}  ín       gét   inti  saget   Iohanne    thaz  ír   gisahút   inti    gihórtut\\
he  then  answering  said   them.\textsc{dat}  go  and  tell    John.\textsc{dat}  what        you.\textsc{pl}  saw    and    heard\\
\glt ‘So he answering said to them,~Go back and report to John what you have seen and heard.’ 

\ex Verb-final (Hêliand, 52, 4372b-73, \citealt{Sievers1935})\\
\gll that  odar   al  brinnandi  fiur       ia    land  ia   liudi     logna farteride\\
the   other  all   burning     fire.\textsc{nom}  both   land   and   people  flame\textsc{{}.nom} destroyed\\   
\glt ‘The burning fire, flame, destroyed everything,\footnote{{\emph{That odar al} can be translated into German as ‘alles Andere.’}}  both land and people’
\z
\z

\ea%2
    \label{ex:2:2}Subordinate clauses:
\ea Verb-first (Hêliand, 2, 131a, \citealt{Sievers1935})\\
\gll Het      that   ic   thi   thoh  sagdi      that    it    scoldi  gisid uuesan  heuancuninges    het        that    git      it    heldin   uuel \textbf{tuhin}{}   thurh  treuua\\
commanded   that   I  you   yet     tell.\textsc{pret.subj}   that    he   should   companion become   Heaven’s.king.\textsc{gen}  commanded   that    you.\textsc{du}    him  hold  well raise.\textsc{pret.subj}       in     faith\\
\glt ‘He commanded that I should say to you also that he shall be a companion of Heaven's King, commanded that you two should hold him well, raise him in faith’

\ex Verb-second (Hêliand, 2, 95b-96a, \citealt{Sievers1935})\\
\gll Tho  uuard   thiu   tid     cuman  that    thar  gitald   habdun uuisa  man  mid   uuordun  that    \textbf{scolda}  thana   uuih   godes Zacharias  bisehan\\
then   became   the    time  come    that   there   told    had wise     men      with  words          that    should  the    temple   God.\textsc{gen} Zacharias   watch\\
\glt ‘Then/when the time was come, which wise men there had foretold with words, that Zacharias should watch the temple of God’ 

\ex Verb-third (Luke 12, 17, ‘He thought to himself, ‘What shall I do? I have no place to store my crops,’ Tatian, Masser 170, 3--6)\\
\gll Inti  thahta   innan   imo   sus    quędenti  uuaz   tuon thaz  ih   ni   haben  uuara   ih  \textbf{gisamano}   mine  uuahsmon\\
and   thought   to     himself   thus    saying     what  do.\textsc{1.pres.ind} that   I  \textsc{neg}  have   where  I   store     my     crops\\
\glt ‘And (he) thought to himself, saying then, what will I do in that I do not have anyplace to store my crops’ 

\ex Verb-later (Hêliand, 2, 95b-96a, \citealt{Sievers1935})\\
\gll Tho  uuard  thiu    tid    cuman     that   thar   gitald  \textbf{habdun} uuisa  man  mid    uuordun  that    scolda    thana  uuih  godes Zacharias  bisehan\\
then  became   the    time   come  that   there  told   had wise    men    with  words    that    should    the    temple  God.\textsc{gen} Zacharias   watch\\
\glt ‘Then/when the time was come, which wise men there had foretold with words, that Zacharias should watch the temple of God’ 

\ex Verb-final (Otfrid, L 1--2, \citealt{Erdmann1973})\\
\gll Lúdowig  ther   snéllo  thes   wísduames   fóllo er  óstarrichi      ríhtit   ál  so   Fránkono    kúning   \textbf{scal}\\
Ludwig   the   brave  \textsc{det}   wisdom    full he   eastern.kingdom   rules  all  as   Franks.\textsc{gen.pl}  king   ought\\
\glt ‘Ludwig, the brave, full of wisdom, he rules all the eastern kingdom as a Frankish king ought’
    \z
\z

\noindent Descriptions of Modern Standard German often assume three basic positions for the finite verb with the possibility of a pragmatically motivated rearranging of surface constituents. So, most clauses in standard and standard-influenced varieties of German have clause-first, -second, or    {}-final verbs. There also exists in these varieties a clear main-subordinate clause asymmetry, whereby subordinate clauses correlate with verb-final syntax and main clauses with verb-first or -second syntax. These patterns in the data underpin the hypothesis that the German verb phrase is underlyingly a head-final template and that finite verbs may be fronted to a surface first or second position to form main clauses. Verbs in early German, as the examples in \REF{ex:2:1} and \REF{ex:2:2} demonstrate, surface in many positions within the clause and do not exhibit the same tidy distributional asymmetry in main and subordinate clauses as modern German does.

Considering data like these, how does one describe the basic syntactic category of the verb phrase? There are tantalizing hints that early German syntax was a lot like modern German syntax. So, the clauses in (\ref{ex:2:1}a) and (\ref{ex:2:2}e) exhibit the verb-second and verb-final patterns in main and subordinate clauses, respectively. The clauses in (\ref{ex:2:2}b) and (\ref{ex:2:2}c) are reminiscent of constructions in more colloquial registers, in which a finite verb can sometimes be fronted in subordinate clause, as in (\ref{ex:2:2}b), or a nominal constituent is extraposed and occurs after an underlyingly clause final verb, as in (\ref{ex:2:2}c). Such similarities might convince a historical linguist to assume that early German has an underlyingly verb-final clause like modern German, though doing so requires the linguist to also account for the surface orderings in clauses like (\ref{ex:2:1}c), (\ref{ex:2:1}d), (\ref{ex:2:1}e), and (\ref{ex:2:2}a).

In other words, data variation for Schrodt is a sort of static that can easily obscure the fundamental structure of basic syntactic categories, such as the verb phrase. \citet{Fleischer2006} adopts a similar view when he begins his article on the methodology of early German syntactic research with this same passage from \citet{Schrodt2004}. He elaborates the problem sketched out in \citet{Schrodt2004} as being one of an unfortunate corpus that is rife with compromised data, an orientation that I call the deficit approach to early German. Unlike other old Germanic corpora, like that of early English, which contains a good selection of both prose and poetic, translational and original texts, the early German corpus is almost entirely made up of the most problematic of these genres: original poetry and translational prose. Like Schrodt, Fleischer assumes these texts contain many inauthentic (\textit{unecht}) syntactic constructions, a situation that cannot be mitigated by asking a native speaker. The article, thus, outlines a method for filtering out such transgressive structures, a process that should leave behind only genuine syntactic data. Returning then to the clauses in \REF{ex:2:1} and \REF{ex:2:2}, Fleischer’s method offers a means for deciding which tokens represent a real early German competence, and which are simply static that obscure the true nature of that system.

The deficit approach frames inquiry into early German syntax as the search for the data that represent an authentic German competence, using methods that are meant to cut through all the variation and identify the grammatical structure or mental grammar, to use generative terminology, from which attested structures are derived. Structuralist assumptions about linguistic inquiry underlie this approach. That is, one adopts this deficit mentality primarily if it is taken as a given that the object of analysis should be underlying structure. Surface variation can certainly result from the usual derivational procedures of generative syntax, for example, but not \textit{all} attested variation need be integrated into accounts of an early German competence. These approaches also assume that some surface patterns are created by confounding factors, for example, the influence of a non-native grammar, like Latin, or a competence-altering metrical pattern. In order to delineate the real structure beneath the surface, the linguist must be aware how these factors can affect competence and control for their effects.

I argue that the deficit approach encourages scholars to view surface variation with suspicion and even associate it with linguistic distortion. Relatedly, it discourages researchers from considering the possibility that variable structures might link to variable sociolinguistic conditions. In this way, the deficit view of the early German corpus promotes a monolectal understanding of this stage of the language. As I explained in my introduction, this argument is different from the claim that scholars with a more structuralist focus ignore or are somehow unaware of empirical variation. In fact, many generative accounts of historical Germanic varieties published over the last two decades have moved well beyond basing sweeping conclusions on a limited set of cherry-picked examples and instead draw on extensive corpora for their analyses. The work of scholars like Katrin Axel-Tober, Christopher Sapp, and George Walkden immediately spring to mind. My argument regarding the monolectal leanings of structuralist approaches refers to the researcher’s orientation toward that variable data. It seems to me that structuralists gather as much data as possible in the hopes that the patterns of an underlying competence will become evident. With respect to data that are recalcitrant to the emerging patterns and their possible derivations, the methodology incentivizes their characterization as competence distorting static, i.e., as the product of confounding factors.

A more multilectal orientation toward that same variability, in contrast, would bear in mind that early Germans are in command of multiple spoken varieties, whose structures vary in accordance with the communicative context in which they are produced. The language’s first literizers, thus, had a multilectal German on which to draw in order to create their \textit{scripti}. Variation in the \textit{scripti} themselves would then result from the literizer engaging their multiple and variable linguistic resources in different ways. The researcher gains insight into the structural variation of early German texts by cultivating a better understanding of early medieval \textit{multilectal} German and how varying structures connect to different communicative and sociocultural conditions. \citet{Coseriu1974} neatly encapsulates these ideas in its concept of “historicity,” which emphasizes how all languages are synchronically and diachronically variable and that people on the individual and societal level are multilectal. The literization methodology is my attempt to take historicity seriously: as an analytically relevant fact that must serve as the basis for any attempt to explain early German syntax.

The purpose of this chapter, then, is to elaborate the deficit approach and show how it has shaped research into early German syntax. I demonstrate how this approach is informed by, but also reinforces, a monolectal view of German’s earliest attested stage. I argue that the deficit approach and its associated monolectalism is supported by a traditional understanding of the corpus as comprising two main genres, that is, poetry and prose, rather than as a literizations of an exclusively spoken vernacular. Many modern syntactic studies have adopted this traditional view without discussion. They have also associated early non-poetic texts with a prose style of writing, which they connect to the notion of authenticity, which I argue is grammaticality in disguise. In other words, it has been the assumption that these early “prose” texts constitute better reflections of an underlying early German competence and, so, the researcher may use its structures as an alternative to the grammaticality judgments of long dead native speakers. In contrast, the researcher ought to treat data from poetic texts with greater suspicion because of the assumption that poetic structures confound competence in ways for which it is difficult to control. Therefore, so the argument goes, the German of poetic texts must be further removed from a natural, grammatical spoken German than is the case for the period’s prose texts. These methods make an already small corpus smaller, in that one is discouraged from working with significant texts because they are poetic. I also argue that they push scholars toward a monolectal understanding of early German by, first, maintaining that the ultimate goal of diachronic syntactic analysis should be identifying a German competence and, second, equating this competence with one variety of early German, an imagined “ordinary” spoken German.

This chapter progresses as follows: in \sectref{sec:2.1}, I discuss the deficit approach to early German syntax by focusing on \citet{Fleischer2006}, which contains, as far as I am aware, the most thorough description of this method. Many studies on early German syntax rely, often implicitly, on assumptions that \citet{Fleischer2006} makes explicit. I focus particularly on how it associates prose texts with grammaticality, on the one hand, and poetic texts with ungrammaticality, on the other hand, and demonstrate how the author uses such conclusions as a substitute for native speaker judgments. In \sectref{sec:2.2}, I examine existing studies that are influenced by the deficit approach to early German. These works mostly analyze translational prose and claim to describe an early German grammar. \sectref{sec:2.3} concludes the chapter.

\section{The deficit approach to early German syntax: \citet{Fleischer2006}}\label{sec:2.1}

\subsection{The four genres of early medieval German}\label{sec:2.1.1}\largerpage

\citegen[27--31]{Fleischer2006} methodology begins with the identification of two binaries within the early German corpus: poetic versus prose texts, on the one hand, and translational versus autochthonous texts, on the other. Poetry and prose, according to the author, are the two main genres of early German, a characterization that is widely accepted. For example, the standard reader for Old High German, \citegen{Braune1994} \textit{Althochdeutsches Lesebuch}, divides all texts into a prose or a poetry section. The next important descriptive feature is whether a text is “translated” or “translational” (\textit{übersetzt}), on the one hand, or “autochthonous” or “indigenous” (\textit{autochthon}), on the other. The positioning of the two binaries in this way already implies that a text’s status as poetic is more analytically consequential, especially with respect to its syntactic structures, than its status as translated. \citet[30]{Fleischer2006} combines the two binaries to form the four logical genre possibilities for early German texts, each of which is associated with different advantages and challenges. It will become clear in this discussion of these categories below that Fleischer uses the terms “authentic” and “authenticity” as substitutes for “grammatical” and “grammaticality.”

\begin{table}
\caption{The genres of early German (adapted from \citegen[30]{Fleischer2006})}
\label{tab:2:2.1}
\begin{tabularx}{\textwidth}{Qp{4.5cm}Q}
\lsptoprule
~ & Autochthonous texts & Translational texts\\
\midrule
Poetic, i.e., metrically bound, texts & Otfrid, \textit{Hildebrandslied}, etc. & \textit{Psalm 138}?\newline \textit{Sigihart's Prayers}?\\
\tablevspace
Prose, i.e., metrically unbound, texts & \textit{Die Altdeutschen Gespräche}, \textit{Markbeschreibungen} & Tatian,\newline Isidor\\
\lspbottomrule
\end{tabularx}
\end{table}

\noindent Note that Fleischer does not include the glosses in his typology in that they convey little to no information on \textit{German} syntax.

The first noteworthy category in \tabref{tab:2:2.1} is autochthonous poetry. One need not worry, so the story goes, about Latin interference with texts like the \textit{Hildebrandslied} because they are not translations. Another advantage to working with this genre of text is that there are two substantial, early, which is to say, ninth-century works of original poetry: the \textit{Hêliand} and Otfrid’s \textit{Evangelienbuch}. However, \citet[35--37, 42]{Fleischer2006} argues that poets also routinely compromised “authenticity” for the sake of metrical structures. Thus, texts like the \textit{Evangelienbuch} are not ideal sources of good data reflective of an early German competence.

An example of poetic adjustment that moves the surface structure further away from competence comes to us from Otfrid: that is, the inflection of the present participle in periphrastics that are formed with \textit{wesan} (‘to be’), illustrated in \REF{ex:2:3}.

\ea\label{ex:2:3}%3
(Otfrid, I 4, 5--7)\medskip\\
\textbf{Wárun} siu béthịu \qquad     góte filu drúdịu \\
joh íogiwar sínaz \qquad     gibot \textbf{fúllentaz}, \\
Wízzod sínan  \qquad    ío \textbf{wírkendan}\medskip\\
\gll Wárun  siu    béthịu  góte                filu    drúdịu  joh  íogiwar     sínaz      gibot fúllentaz        wízzod  sínan      ío    wírkendan\\
     were  they    both  God.\textsc{dat}    very  intimate and  everywhere  his    commandment fulfilling{}.\textsc{acc.part}  law     his    always   carrying.out.\textsc{acc.partc}\\
\glt `They were both very intimate with God and fulfilling his commandment everywhere, his law always carrying out.'
\z

\noindent In line 5, the adjective \textit{drudiu} agrees with the subject \textit{siu}, as one might expect. However, in lines 6 and 7 the adjectival present participles, \textit{fullentaz} and \textit{wirkendan}, agree with their objects, \textit{gibot} and \textit{wizzod}, respectively. \citet{Fleischer2006} relies here on \citet[216]{Erdmann1874}, who characterizes the agreement of participles with the object as a rarely occurring phenomenon in Otfrid, though with no direct statements on how many object-agreeing tokens occur across the work, compared to subject-agreeing ones, to back up this claim. Based on this information, \citet[37]{Fleischer2006} agrees with \citet[421]{Nemitz1962}, who concludes that these forms are effected entirely by the rhyme, with no native speaker intuition behind them. These tokens, then, constitute a “phantom structure” that had “no status in the Old High German grammar.” These conclusions imply that such forms are, in fact, ungrammatical in that native speakers would not produce them because they are not part of an early German competence. \citet[36]{Fleischer2006} extrapolates a broader conclusion from these data that has serious implications for the study of early German: Otfrid was willing to break the “grammatical rules of Old High German” for the sake of “poetic freedom.” This argument places the \textit{Evangelienbuch}, the corpus’s longest original work, under suspicion of containing structures that are ungrammatical in that they do not align with any native speaker judgments. Perhaps researchers should avoid the work entirely in favor of prose texts. In the latter type of text, at least, one need not also contend with poetic structures as a confounding influence on an early German competence.

In the case of the second category, translational prose, poetic interference is not a problem, but the possible influence of the Latin source on German syntactic structures is. The translation of Tatian’s \textit{Evangelienharmonie} into ninth-century East Franconian contains many examples of what one assumes is transferred Latin syntax. This assumption is based on the fact that the German is, for the most part, a word-for-word translation of the source text \REF{ex:2:4}.

\ea%4
\label{ex:2:4}(from John 15, 15, Tatian, Masser, 284, 16--18)\medskip\\
\begin{tabular}{@{}ll@{}}
Iam non dico vos servos                  &    Ih ni quidu íu iu scalca                        \\
quia servus nescit quid \textbf{faciat}  &    uuanta scalc ni uueiz uuaz \textbf{duot}  \\
dominus \textbf{eius}                    &    \textbf{sin} hérro              \\
\end{tabular}\medskip\\
\gll Ih  ni    quidu  íu       iu       scalca     uuanta   scalc    ni    uueiz uuaz  \textbf{duot}  sin  hérro\\
I   \textsc{neg}   call   you.\textsc{dat}   no.longer   servants    because   servant  \textsc{neg}  knows what  does  his  lord \\
\glt ‘I no longer call you servants because a servant does not know what his master does’
\z

\noindent As becomes evident in \citegen{Masser1994} edition, the German is generally a line-by-line translation of the Latin, with only a minimal rearrangement of constituents, for example, the swapping of the order of a noun and its possessive adjective, as seen in line 18. The translator\footnote{{There is no unanimity in the scholarship on whether one or multiple people translated Tatian’s text into early German. For simplicity’s sake, I refer to one translator. With respect to this chapter’s arguments, it does not matter how many people worked on the project.} } was more reluctant to rearrange constituents across line boundaries, as seen in lines 17--18 in that \textit{duot} has not been moved to the end of clause in line 18. \citet[18--20]{DittmerDittmer1998} argue that it is through small adaptions within these textual constraints, as seen in line 18, that the translator was able to shape the syntax into more “German-like directions.” Still, many clauses look distinctly un-German, as it were, because of the project’s translation principles, as seen in line 17. Dittmer \& Dittmer would caution against accepting its early placement of the finite verb at face value, as one cannot know if it reflects a native or a foreign syntactic pattern.

\citegen{Fleischer2006} next predicted genre is the poetic translation, which is more a logical possibility stemming from the premise of the two binaries than a real early German text type. On page 30, the author identifies two potential examples of poetic translation, but it is doubtful that either text, the \textit{Psalm 138} or \textit{Sigihard’s Prayers}, qualifies. Both texts are closer to a loose rendering or paraphrase of a Latin text, structured in Otfridian verse. \citet[218--220]{Bostock1976} notes that the early tenth-century \textit{Psalm 138} is a “very free” paraphrase of the Latin Vulgate, in which the writer interprets the psalm according to contemporary thinking, even rearranging the order of ideas (page 219). Similarly, \textit{Sigihard’s Prayers} are German renderings of formal prayers that marked a task brought to completion, rather than the translation of one fixed text. The fact that their authors used a style of verse created by a German for the German vernacular further speaks to the independence of these works from their sources. If, despite these circumstances, one did characterize these works as poetic translations, one would also have to concede that they are too short to be of much consequence: \textit{Psalm 138} is only 38 lines long, while \textit{Sigihard’s Prayers} comprise only four lines, two per short prayer. In my mind, there is insufficient evidence to conclude that the poetic translation was a real genre of early German.

The final predicted genre is autochthonous prose, which is also poorly attested in the early German corpus. \citet[28]{Fleischer2006} characterizes this text type as the most ideal for syntactic analysis. These texts are bound neither by poetic constraints nor a Latin source and, so, the author maintains that they will exhibit the most authentic German of the four genres. Fleischer equates authenticity with closeness to “the spoken language,” and while all text types will exhibit reflexes of this spoken language, autochthonous prose texts are the mostly likely to contain these reflexes. The article identifies two example texts from this genre: the \textit{Markbeschreibungen}, which comprise two charters, and the \textit{Altdeutsche Gespräche} (‘the Old German Conversations’). To this list, one may add the prose section of the \textit{Wessobrunner Gebet}, but nothing much else.\footnote{{The early German corpus is notably different from the early English corpus, in which vernacular prose is more robustly attested.} } Of these works, it is the \textit{Gespräche}, with their colloquial, everyday language, that stand out in the author’s mind as representing a particularly authentic German (page 28--29).

\subsection{Judging grammaticality in the absence of native speakers: The dative absolute}\label{sec:2.1.2}

Within the deficit approach, it is this basic understanding of the four genres, their advantages, and challenges with respect to providing sound data, that should inform which text a researcher decides to work with and how they work with it. Practically speaking, \citet{Fleischer2006} advocates a structure-by-structure, comparative approach whereby the researcher decides whether a token is authentically German, or whether a translational or poetic imperative created it. Assessing the relative authenticity of a text vis-à-vis other texts in the corpus is a central consideration. I argue in this section that Fleischer’s search for authenticity is a search for grammaticality in disguise. In other words, the author’s comparative structure-by-structure method is meant to distinguish data generated by a real German competence, from those structures and patterns that have been affected or, indeed, effected by confounding influences. If one thinks about this approach in terms of linguistic intuitions, the former data would have been judged grammatical by a native speaker, the latter, as less than grammatical. This method, then, is operates as a substitute for the native speaker grammaticality judgments.\footnote{{This argument was inspired by \citet[370--373]{Somers2021a}.} }

Fleischer’s analysis of the dative absolute serves as a good example of how his method should work. In \REF{ex:2:5}, I reproduce Fleischer’s examples, to which I have added glosses.

\ea%5
    \label{ex:2:5}(Tatian 90,5 \citealt{Sievers1961}, \citealt{Masser1994}: 144, 13--14)\medskip\\
    \begin{tabular}{@{}ll@{}}
    Tunc \textbf{conuocata turba}  &  cum discipulis suis dixit eis\\
    tho \textbf{gihalatero menigi} &   mit sinen iungoron quad in     \\
    \end{tabular}\medskip\\

\gll Tunc  conuocat-a      turba  cum    discipulis  suis    dixit  eis\\
  then   assembled-\textsc{fem.ab.sg}  crowd  with  disciples    his    said  them\\

\gll   tho    gihalat-ero       menigi  mit    sinen  iungoron  quad   in\\
  then  gathered-\textsc{fem.dat.sg}  crowd  with  his    disciples    said    them.\textsc{dat}\\

\glt ‘Then having gathered the crowd with his disciples, (he) spoke to them’
\z

\noindent One might begin with the assumption that the dative absolute was not an authentic early German structure as it is not present in modern German. Further cause for suspicion arises from the fact that it is attested mostly in translational prose as a German rendering of the Latin ablative absolute. Perhaps, then, German’s dative absolute is just a superficial transference of a Latin construction. Though such data imply the construction’s inauthenticity with respect to German, they do not settle the matter entirely. \citet[32--33]{Fleischer2006} argues, for example, that texts like the “Latin-dependent” Tatian can still provide valuable data if focus remains on tokens that deviate from the Latin, referred to in German as \textit{Differenzbelege} (‘tokens exhibiting difference’). Underlying this approach is the so-called \textit{Differenzprinzip} (‘difference principle’). Discussed as early as \citet{Ruhfus1897}, it maintains that deviating tokens \textit{must} be reflections of genuine German. Thus, any instances of the dative absolute that do not correspond to a Latin ablative absolute could be treated as authentic early German. There are, however, no such \textit{Differenzbelege}: citing \citet[179--181]{Lippert1974}, Fleischer states that the translator rendered practically every Latin ablative absolute with a German dative absolute. This one-for-one correspondence points to the latter’s total dependence on the former, which also implies its inauthenticity as an early German structure.

But what if the dative absolute occurs in the Isidor translation? Scholars largely agree that the Isidor is the superior translation of the two because it is largely independent from its Latin source. Furthermore, \citet[349]{Matzel1970} notes that the translator did not schematically translate all Latin ablative absolutes into German as dative absolutes. The \textit{Differenzprinzip} would suggest that these independent renderings of the dative absolute must be based on native speaker intuitions (\textit{Sprachgefühl}). Matzel himself viewed the construction as one that would have been rated as “acceptable” (\textit{tragbar}) by native speakers at the time. \citet[160]{Lippert1974}, however, argues that this conclusion is unwarranted because the Isidor translator confines the German dative absolute to “certain formulaic contexts” when the translator presumably wanted simply to transfer a Latin-inspired style into the vernacular. Thus, the dative absolute was a non-autochthonous structure (\textit{unautochthone Struktur}) and never a “living idiom” of medieval German, as \citet[56]{Gelderen1991} puts it when writing about a similar state-of-affairs in Notker.

What if the dative absolute were attested in an autochthonous text like Otfrid’s \textit{Evangelienbuch}, though? Given its lack of a Latin source text, would its presence there indicate that it is a living idiom or autochthonous structure of medieval German? And, indeed, Otfrid’s poem does contain a few examples of the dative absolute, but only a few, exactly two, according to \citet[V 25, 7 and IV 13, 52--53, page 259]{Erdmann1874}. I present one of these examples in \REF{ex:2:6}. Both Erdmann and Lippert conclude that the form is too rarely attested to be treated as authentic. Furthermore, the tokens surface in phrases that correspond to common Latin phrases \citep[43]{Fleischer2006}.

\ea%6
    \label{ex:2:6}(Otfrid V 25, 7)\\
    Bin \textbf{gote hélphante}  thero árabeito zi énte\\
    \gll Bin  got-e       hélphant-e      thero    árabeito        zi    énte\\
    am   God-\textsc{dat}  helping-\textsc{dat.sg.abs}   the.\textsc{gen.pl}  work.\textsc{gen.pl}     to     end\\
    \glt ‘I am at the end of my toils with God’s help’
\z

\noindent For this reason, Fleischer notes, historical German linguists have not seen the dative absolute as reflective of an authentic German grammatical structure. Rather it is a mere mechanical transfer of Latin syntax with no native speaker feeling behind it. The author cites \citet[259]{Erdmann1874}, who argues that such constructions “did not develop in original German conversation” (\textit{in orginaler deutscher Rede nicht entwickelt hat}).

\begin{sloppypar}
Indeed, the fact that Otfrid included such constructions in the first place, places the whole text’s linguistic authenticity in doubt. On pages 43--44, Fleischer adopts \citeapo[8]{Haubrichs2004} characterization of Otfrid, whose attempt to compose in his Frankish tongue never breaks free from the Latin model because the poet did not want to or could not manage it (\textit{Otfrids “Versuch, in fränkischer Zunge zu dichten, sich nie vom lateinischen Vorbild lösen kann noch will},” my translation). So, while one should expect a prose translational text to have mechanically transferred structures like the dative absolute, their presence in the \textit{Evangelienbuch} is treated as a problem. In the former type of text, the reason for these borrowed structures is clear: namely, the pressure of rendering a Latin text in German. Moreover, the scholar may control for these structures through the \textit{Differenzprinzip}. In Otfrid’s text, however, where one cannot apply the \textit{Differenzprizip}, even the odd occurrence of the dative absolute seems to indicate that the text is an unreliable source of authentic data.
\end{sloppypar}

Considering these arguments, it might make sense to avoid the \textit{Evangelienbuch} entirely. Does the poem contain any authentic German at all? How can one tell? \citegen[44]{Fleischer2006} discussion of multiple negation provides an answer. I present again here his examples to which I have added glosses.

\ea%7
    \label{ex:2:7}
    \ea (\textit{Altdeutsche Gespräche} 74)\\
\gll          ne    haben   ne    trophen       (non abeo quid)\\
  \textsc{neg}    have  \textsc{neg}    nothing.at.all\\
\glt   ‘I don’t have anything at all’
\ex (Tatian, \citealt{Masser1994}:155,11)\\
\gll  inti    n-ioman    imo    ni-gab\\
  and    \textsc{neg}-one     him    \textsc{neg}-gave\\
\glt  ‘And no one gave him anything’ 
\ex (Otfrid, III 4, 23)\\
\gll Drúhtin  quad  er  gúato     n-ist    n-íaman    thero fríunto\\
lord    said   he  the.good    \textsc{neg}-is  \textsc{neg}-one.\textsc{nom}  the friends.\textsc{gen.pl}\\
\glt ‘Lord, he said, the good, there are no friends (i.e., I have no friends) \textit{…who could help toss me into the water}. 
    \z
  \z

\noindent In contrast to the dative absolute, the author concludes that multiple negation is an authentically German syntactic construction because it is attested in autochthonous and translated prose texts. So, for example, the Tatian translator sometimes rendered Latin single negation as double negation in German. Most importantly, however, the apparently colloquial prose of the \textit{Gespräche} exhibits multiple negation as well, which implies that speakers produced multiple negation in their everyday language.

It is useful at this point to make explicit the difficulties involved in establishing grammaticality for historical varieties and how this focus on grammaticality pushes historical linguists in directions that might well be inconsistent with their views on synchronic linguistic variation. The most glaring issue is that it is impossible to confirm or refute the supposed grammaticality, or authenticity, to use Fleischer’s word, of a structure because early German speakers are long dead. In place of native speaker grammaticality judgments, Fleischer’s discussion of the dative absolute reveals that he has established two conditions for authenticity in early German, both of which would and could never be acceptable to scholars who study modern languages. The first condition for authenticity is a structure’s presence in spoken, preferably colloquial, German, as evidenced by Fleischer’s characterization of the \textit{Gespräche} as an early German text of particular interest and its structures as uniquely authentic. The second condition is a structure’s “Germanness.” That is, authentic German structures are autochthonous or “original,” to use Erdmann’s term. In contrast, a structure that is borrowed from Latin into German and is used in limited or formulaic ways is assumed to have no native speaker intuitions behind it. It is, therefore, not authentically German.

It is also worth reflecting on Fleischer’s use of the terms “authentic” and “authenticity,” in place of “grammatical” and “grammaticality.” Explicitly generative analyses use similar euphemisms, such as “extragrammatical” to refer to structures that are not \textit{really} part of an Old High German grammar. For example, \citet[70, 77--78]{Axel2007} argues that exceptions to the asymmetric distribution of the finite verb in main and subordinate clauses in early German are attributable to extragrammatical, i.e., confounding factors.\footnote{{See \citet[360--364]{Somers2021a} for an earlier and somewhat different version of this argument. My focus then was still on structure, and so, there is much in that article that I would approach differently today. However, I maintained there, as I do here, that the generativist “extragrammatical” is seen as a way around the limitations of “grammatical” with respect to historical varieties.}}  Referring back to the early German clauses I presented at the beginning of the chapter, the assumption among historical linguists is that early German, like Modern German, generated verb-final clauses (\ref{ex:2:2}e) from which the verb-second main clause ordering (\ref{ex:2:1}b) was derived. Clauses like that in \REF{ex:2:8} occur regularly in the poetic \textit{Evangelienbuch} and undermine this assumption. However, Otfrid’s tokens have largely been characterized in the literature as “conditioned by rhyme and, therefore, not meaningful pieces of evidence” (\textit{reimbedingt und damit keine aussagekräftigen Belege}, to cite \citegen[204]{Schrodt2004} \textit{Althochdeutsche Grammatik, II}).

\ea%8
    \label{ex:2:8}(Otfrid, L 27)\\
\gll           Er  uns  ginádon  sinen  \textbf{ríat}       thaz     súlichan   kúning  uns    \textbf{gihíalt}\\
he  us   mercy  his     meted     that     such    king  us.\textsc{dat}  preserved\\
\glt ‘He meted out to us his mercy in that he preserved such a king for us.’
    \z

\noindent Hence, in Axel’s account of Old High German clause structure, the complementary distribution of finite verb and complementizer was part of an early German competence. In contrast, asyndetic verb-late clauses, that is, verb-late or verb-final clauses with no complementizer, were produced only to satisfy the exigencies of a poetic scheme. In that generative syntax assumes multiple mental grammars for each speaker, it is possible that a diachronic generativist like Axel would include the asyndetic verb-late clause in an early German \textit{poetic} grammar. However, the structure is outside of (\textit{extra}{}-) the Old High German grammar that Axel intends to describe, which is the patterns of a \textit{prose} syntax. This conclusion is reminiscent of the discussion of object-inflected present participle periphrastics (\sectref{sec:2.1.1}) and the dative absolute (\sectref{sec:2.1.2}); that is, asyndetic verb-late clauses are a poetically created “phantom structure” \citep[421]{Nemitz1962}, not a “living idiom” of early German (\citealt{Gelderen1991}: 56).

This terminological avoidance of the terms “grammatical” and “grammaticality” on the part of diachronic structuralists points to two conclusions. First, it reflects a discomfort with using the term grammatical, which is established through the grammaticality judgments of native speakers, for historical languages. Second, the use of “authentic” instead of “grammatical” creates a new standard for attested historical structures. While the latter term applies to the many synchronic mental grammars that generativists imagine each language user has, diachronic structuralists of all stripes only have eyes for an “authentic” historical grammar. This grammar, as Fleischer’s explicit methodological treatment demonstrates, is that of an early colloquial, everyday spoken German.

One would be hard pressed to find linguists today who would make the argument that colloquial spoken Modern German is particularly grammatical. There are many grammatical structures that are absent in many Germans’ spoken varieties but have been maintained in the written standard, e.g., the simple past in Upper German and the genitive case across the dialect continuum. The extended participial (e.g., \textit{Ich sehe den \textit{unter dem Tisch liegenden Hund}}, ‘I see the \textbf{under the table lying dog’}) is an example of a structure that is all but absent from spoken German and exists only as a written construction. These examples demonstrate that structures can be part of a written or a standard grammar while also being absent from people’s everyday spoken language. This conclusion will strike no one as particularly insightful or controversial. Generativists, for example, would simply include the extended participial in a written German grammar and exclude it from Germans’ spoken grammars. Yet, when analyzing a historical text, the search is no longer for grammatical data with the goal of piecing together the grammars of the multilectal German speaker who wrote it. Rather, it has become about identifying the “authentic” data that reflect one of the multiple spoken varieties we should assume a text’s writer produced in their everyday life. This is why I characterize structuralist methodologies like Fleischer’s as monolectal. While they might acknowledge variation and/or multilectalism, their methods do not promote a multilectal understanding of historical varieties. Quite the opposite, in fact.

This brings me to the second problematic aspect of Fleischer’s conditions for authenticity, which is that the structure must be autochthonous or original. As I discussed in this book’s introduction, our disciplinary forebears, e.g., the grammarians of the seventeenth and eighteenth centuries and the philologists of the nineteenth century, were interested in identifying the ancient German of their imagined ancestors and cultivating “good” German based on what they found. A nationalistic belief that there was a uniquely German language to uncover in the first place underlay this project and, I have proposed, underpins structuralist linguistic inquiry. This monolectal narrative of the history of German is transmitted to modern students of historical linguistics through one of the first representations of language evolution that they encounter: the tree model. This model invites comparison (not coincidentally) to family trees and phylogeny in the biological sciences, where shared origins indicate genetic and racial relatedness. In doing so, it simultaneously obscures the fact of historicity, that is, that all speakers and communities, then as now, are multilectal and, more often than not, multilingual. This means that language contact is a fact of human existence.

It seems possible to me that modern structuralist methods for identifying an authentic early German grammar represent an unconscious adopting of the assumptions of early nationalists, who believed that an ancient German language existed and that it was their job to uncover it. They laid the groundwork on which the discipline of historical linguistics has been built. Thus, I see a connection between these early views and the fact that many diachronic linguists today unquestioningly believe that their object of study must be a \textit{real} early German grammar and that their methods must help them reconstruct this system by identifying where confounding factors have moved grammatical structures away from this authentic system. To reiterate, I do not think that Fleischer and other more structurally minded modern linguists have nationalistic aims or use the term “authentic German” in the same way that Erdmann or Grimm used the phrase. Fleischer’s “authentic German,” as I have argued above, is something of an oblique reference to grammaticality, while philologists like Grimm meant the term literally. These same linguists, I believe, would balk at the suggestion that, say, foreign borrowings into German do not and have not become part of a German grammar. However, modern linguists’ use of terms like “autochthonous” or “original” places analytical constraints on their arguments nonetheless and has made it less likely that they will approach German’s first \textit{scripti} from the literization perspective of this book, or any sociocultural approach, for that matter.

Returning to the specifics of \citet{Fleischer2006}, we see that a hierarchy of early German genres takes shape over the course of his discussion, one in which genres are distinguished with respect to how likely they are to yield authentic German and, thus, their suitability for syntactic analysis.

\ea%9
    \label{ex:2:9}

          autochthonous prose > translational prose > autochthonous poetry\footnote{{I make no mention of translational poetry, which receives practically no attention in \citet{Fleischer2006}.} }

  \textit{Die Altdeutschen Gespräche} > Tatian > Isidor > Otfrid
    \z

\noindent Fleischer himself never explicitly states as much, but his analyses make clear that he believes prose texts to be generally superior sources of syntactic data to poetic ones: he sees prose as a neutral form of language, lacking as it does a poetic structure. Autochthonous prose, in particular, is the most desirable genre of all, with its independence from Latin. Also evident is that the author believes \textit{translational} prose to be superior to \textit{autochthonous} poetry: though both genres are subject to distortions, it is only in translational texts where one can easily control for any confounding effects on data patterns. Fleischer’s enthusiasm for the \textit{Differenzprinzip} and its ability to identify authentic data underlies, I believe, his more indifferent attitude toward the other significant translational prose text, the Isidor, which lacks the neat one-to-one correspondence between the source and its translation. Scholars often point to the independence of the Isidor translation as indicative of the text’s quality and suitability for syntactic investigation (see \citealt[3]{Robinson1997}). \citet[33--34]{Fleischer2006}, however, emphasizes how the Latin source still exerts pressure on German structures, though its influence is harder to isolate than is true for the Tatian. This treatment of Isidor is reminiscent of \citet{Gering1876}, who concludes that the Tatian is the more significant text, in part, because the Isidor translation is of such high quality. It is impossible to know, Gering argues, whether deviations are due to style of grammar. The language of the Isidor is the product of one exceptionally erudite individual, a masterful and artificial creation. The Tatian text, on the other hand, is a better reflection of authentic, everyday German.

\begin{sloppypar}
In this way, Fleischer’s methodology amounts to a scale that ranks each genre and text based on how authentic the data are and, thus, how suitable they will be for syntactic analysis. Because there are no living native speakers of ninth-century Frankish to elucidate the shape of their competence by providing grammaticality judgments, linguists are unable to establish grammaticality for their historical varieties. The texts that these speakers left behind is mere performance and only ever an imperfect reflection of competence. Given the extent to which the data in this particular corpus have been characterized as problematic, a method for sorting through the good structures and the bad seems essential. In place of grammaticality, modern diachronic linguists have authenticity, and \citet{Fleischer2006} encapsulates perfectly the ways in which we have approached the corpus, which is to say, through the deficit approach. It compares genres with respect to their suitability for syntactic analysis and their likelihood to contain competence-reflecting, authentic data. Authenticity is established only in the relative sense: one genre is more or less likely than another to evince authentic data. Autochthonous prose texts, like the \textit{Altdeutschen Gespräche}, are free from the confounding influence of a Latin source or poetic meter. Fleischer’s method treats data from these texts as prima facie the most authentic. Autochthonous poetry, like the \textit{Evangelienbuch}, is prima facie inauthentic. Structures from such texts, like multiple negation or the dative absolute, are established as authentic only through comparison with prose texts, both autochthonous and translational. Translational prose texts occupy a middle ground in that they are prima facie authentic when they deviate from the source text, that is, are \textit{Differenzbelege}. One can isolate these deviations most easily in Tatian’s \textit{Evangelienharmonie} and so that text is a better source of authentic data than the Isidor.
\end{sloppypar}

\section{The search for the most authentic early medieval German}\label{sec:2.2}

In \sectref{sec:2.1}, I elucidated how \citet{Fleischer2006} connects genre to authenticity as a substitute for historical grammaticality, as well as the argument’s structuralist underpinnings. In this section, I demonstrate how these assumed correlations appear in the scholarship and outline the disadvantages to the deficit approach, which is based on the notion that the corpus is rife with inauthentic structures and patterns resulting from the influence of confounding factors. The most widely accepted and problematic assumptions are, first, that early German prose is linguistically more authentic, or indeed grammatical, than poetry and, second, that autochthonous prose is the most linguistically authentic or grammatical. These notions have become axiomatic in the literature. For example, \citegen{Axel2007} study of Old High German syntax excludes clauses from poetic texts, like Otfrid’s \textit{Evangelienbuch}, from its dataset~-- they are brought into the analysis only on a structure-by-structure or comparison basis~-- but does not discuss the reasons for doing so (see, in particular, chapter 1).\footnote{{\citegen{Axel2007} section 1.2 seeks to introduce the reader to the Old High German corpus but discusses only prose texts. Its one mention of poetic texts comes on page 3, when the author notes that “autochthonous prose texts” are so rare that some studies focus on autochthonous poetry, e.g., the “lyrical”} {\textit{Evangelienbuch}} {and} {\textit{Hildebrandslied}}{. Though there is no explicit statement to the effect of “prose texts are better than poetry,” this belief clearly underlies the study’s chosen focus. The same applies to section 1.6, which introduces the study’s corpus of prose texts with no discussion of why poetic texts were excluded.} } The problem, as I see it, is not that authors choose to analyze data from translational prose or ignore poetry. One of this book’s main arguments is that all early German data are legitimate evidence of the language’s early literization, that is, the process whereby a multilectal oral vernacular acquires a written form. Rather, at issue here is the belief that an analysis of \textit{only} prose data is tantamount to an analysis of authentic, or grammatical, Old High German syntax. Note again the assumption of monolectalism that is inherent in this approach: these studies search for evidence of, and look to account for, one grammatical system, in particular a colloquial, everyday early German. Whereas analyzing variable data with historicity in mind would encourage the researcher to treat an early German \textit{scriptus} as a material artifact into which the writer poured their multiple linguistic resources.

This section is divided into two parts. In \sectref{sec:2.2.1}, I assess prose as a genre and its various definitions. I argue that classical attempts to distinguish prose from poetry have shaped the traditional and colloquial definitions of prose, in particular. Earlier philologists and today’s linguists who study Germanic syntax have adopted these classical definitions uncritically, a practice that has yielded incoherent and misleading references to the non-poetic texts of the early German(ic) corpus. Ultimately, it has allowed scholars to believe that the early prose texts are a written expression of an authentic and even colloquial spoken language. I counter this line of thinking with a more specific definition of prose as one linguistic register that writers begin to develop in the early medieval period. In \sectref{sec:2.2.2}, I demonstrate how overly broad definitions of prose and the association of prose with (one) grammatical German system have colored recent studies of early German syntax.

\subsection{What is prose?}\label{sec:2.2.1}

I return here to \citet{Fleischer2006}, in which conclusions regarding the relationship between the genre of an early German text and its presumed authenticity~-- indeed, even grammaticality~-- are made the most explicit. The studies I discuss in \sectref{sec:2.2.2} are built on these same ideas, whether their authors acknowledge these assumptions or not. I begin, then, with the two binaries that shape \citegen{Fleischer2006} four genre distinctions: autochthonous–translational and prose–poetry. For the reader’s convenience, I reproduce here Table~\ref{tab:2:2.1} as \tabref{tab:2:2.2}, to which I have added additional relevant texts.

\vfill
\begin{table}[H]
\caption{Fleischer's genre classification for early German}
\label{tab:2:2.2}
\begin{tabularx}{\textwidth}{>{\raggedright}p{\widthof{Prose, i.e., metrically}} >{\raggedright}p{\widthof{Translational texts}}Q}
\lsptoprule
~ & Translational texts & Autochthonous texts\\
\midrule
Poetic texts & \textit{Psalm 138}? \textit{Sigihart's Prayer}? & Otfrid, \textit{Hêliand}, \textit{Hildebrandslied}, etc.\\
Prose, i.e., metrically unbound texts & Tatian, Isidor & \textit{Die Altdeutschen Gespräche}, \textit{Markbeschreibungen} (\textit{Wessobrunner Gebet}, the final short paragraph)\\
\lspbottomrule
\end{tabularx}
\end{table}
\vfill\pagebreak

\noindent In the case of both binaries, Fleischer defines one side only in relation, and in opposition, to its counterpart. So, some early German texts are translations of a Latin source and, thus, are “translational” (or “translated,” \textit{übersetzt}), while non-translational texts are described as “autochthonous.” Similarly, prose texts are defined by their lack of poetic structures. That is, they are “metrically unbound” texts, or not poetic.

\begin{sloppypar}
These binaries, however, are not particularly good at creating categories that accurately describe early German texts, their similarities, and dissimilarities. Consider, for example, two texts that land in opposite genres: \textit{Psalm 138} and the \textit{Markbeschreibungen}. According to Fleischer’s definitions, the charters of the \textit{Markbeschreibungen} are described as autochthonous because they do not have one specific source text from which they stem. In contrast, \textit{Psalm 138} is not autochthonous, or translated, simply because it is based on the Vulgate’s \textit{Psalm 138}. Identifying one text as autochthonous and the other as not, however, obscures how similar these texts are with respect to the degree of cultural transfer that was involved in their production.
\end{sloppypar}

\citet{Bostock1976} characterizes the German \textit{Psalm 138} as a free translation of the Vulgate text that is so free, it is more of a paraphrase. The German-speaking writer adapted and rearranged the original text, its language and ideas, in order to fashion a vernacular text that was appropriate to their contemporary ecclesiastical setting \citep[219--222]{Bostock1976}. Furthermore, the author composed in Otfridian verse, which represents another way in which the writer molded the original into a distinctly Carolingian text. The context of Carolingian Europe is also relevant to the \textit{Markbeschreibungen}, which comprises two charters, one that records the land in Hamelburg, Bavaria given to the monastery at Fulda on January 7, 777 (the \textit{Hamelburger Markbeschreibung}), the other delineates the boundaries of Würzburg on October 14, 779 (\textit{Würzburger Markbeschreibung}). Their language is formulaic and repetitive, as is the case with all charters; it also contains a mixture of Latin and German. This bilingualism indicates how charters were important bureaucratic vehicles for an unbroken tradition of Roman law that maintained a tight grip on administrative systems across Europe for centuries to come.

Both \textit{Psalm 138} and the \textit{Markbeschreibungen}, in fact, are products of Carolingian culture and society, while also being influenced by classical traditions of religion and administration. It is misleading to label the latter texts as “autochthonous,” simply because they do not have one specific source text, but rather a whole legal tradition as a source, while calling the former text “translated.” Also note the sense of “nativeness” that is conveyed with the term “autochthonous” and how easily one might transfer that notion to language. As autochthonous texts, the \textit{Markbeschreibungen} gain an air of authenticity and must surely exhibit more authentic German than the translated, non-autochthonous \textit{Psalm 138}, which also happens to be poetry and, thus, must exhibit less authentic German because of the influence of confounding factors.\footnote{{Simple fixes, like replacing the term “autochthonous” with “original,” run into similar problems.} }

The prose-poetry binary is equally misleading as a means of describing early German texts, and its problems begin with the fact that “prose” is largely defined only in relation to poetry. \citet[28]{Fleischer2006} defines prose as comprising metrically unbound texts; that is, prose includes anything that is not poetic. Defining prose in terms of what it is not has a long tradition that reaches back to Classical Greece. Poetry’s status as the terminological fixed point around which prose is defined stems from the fact that, in evolutionary terms, “poetry precedes prose, at least as a self-conscious or artistic mode of verbal performance and literary composition” \citep[303]{Graff2005}. For example, \citet[303--304]{Graff2005} notes in his article, \textit{Prose versus Poetry in Early Greek Theories of Style}, that significant evidence exists of a robust tradition of oral and written verse going back to Greece’s archaic period. In contrast, formal prose was a comparatively late invention of the classical period, during which it also became a major topic of scholarly inquiry. Classical rhetoricians like Aristotle were interested in the art of speech and, therefore, attempted to identify the different types of prose and establish prescriptions for making prose, which is to say “unmetered” oratory and writing, effective. \citet{Graff2005} demonstrates in his article that the classical rhetoricians were unwilling or unable to define prose in positive terms in that their descriptions of and prescriptions for appropriate prose always made crucial reference to poetry. To quote \citet[305]{Graff2005}, who cites \citet{Nimis1994}:

\begin{quote}
This negative and basically formal conception of prose, though pervasive in antiquity and widespread even today, has been a source of confusion. It both assumes and asserts a distinction that proves to be illusory. As Steve Nimis [1994] remarks, the definition of prose as non-verse “makes prose a very unstable category … If prose is the “other” of verse, then what prose is depends on what “verse” is, and this is not a stable category either.
\end{quote}

\noindent This same categorial incoherence is evident in the work of the nineteenth\hyp century philologists, such as the Grimm brothers. In their \textit{Deutsches Wörterbuch} (\citeyear{Grimm1889}) they define prose as “die ungebundene schreibart und rede \textbf{im gegensatz zu} verse und poesie” (‘unbound writing and speaking as opposed to verse and poetry,’ page 2170) and has been carried over into modern scholarship, as \citet{Fleischer2006} demonstrates.

Old-school philologists, Germanists, and lexicographers alike have uncritically adopted another idea from the classical rhetoricians, namely, the notion that prose, unlike poetry, represents a natural, everyday mode of linguistic expression. Consider the continuation of the Grimm brothers’ definition of prose, which describes it as \textit{eine ledige oder ungezwungene red, die nit in reim gestelt oder gezwungen} (‘an unattached and unforced speech, which is not placed or forced into a meter’). Prose is free, unbound speaking and writing. In contrast, poetry “forces” and “binds” language. Though this definition does not mention qualities like “grammaticality” or “authenticity,” it implies that prose and poetry have opposite relationships to both concepts, with “free” prose likely to be more authentic or grammatical than “forced” poetry. More contemporary, colloquial definitions of prose follow the traditional lead of defining prose in opposition to poetry, while strengthening the apparent link between it and naturalness or grammaticality. Consider Merriam-Webster’s definition for \textit{prose}:\footnote{\url{https://unabridged.merriam-webster.com/unabridged/prose}\label{fn:2:10}} 

\begin{itemize}
\item  the ordinary language people use in speaking or writing; language intended primarily to give information, relate events, or communicate ideas or opinions
\item  a literary medium distinguished from poetry by its greater irregularity and variety of rhythm, its closer correspondence to the patterns of everyday speech, and its more detailed and factual definition of idea, object, or situation~--  compare verse
\end{itemize}

\noindent Note how this definition connects prose to everyday, “ordinary language,” which implies the extraordinariness and markedness of poetry. Wikipedia, often people’s first online stop for the basics on any given topic, links prose and grammaticality even more explicitly: “Prose is a form of written or spoken language that typically exhibits a natural flow of speech and grammatical structure”.\footnote{\url{https://en.wikipedia.org/wiki/Prose}} Poetry, it is implied, is unnatural and follows a different set of rules than so-called ordinary language.

This linking of prose to authentic, everyday, and unmarked language also has classical origins, namely, Aristotle’s \textit{On Rhetoric}, in which the philosopher and rhetorician, like modern Germanists, wrestles with the same incoherent categories of poetry and prose and, therefore, runs into similar analytical problems. The main focus of \textit{On Rhetoric}’s book 3 is to establish what constitutes good style in unmetered \textit{logos},\footnote{{Aristotle defines} {\textit{logos}} {in various ways in} {\textit{On Rhetoric}}{, including as ‘word, sentence, argument, reason, speech, tale, and esteem’ \citep[317]{Kennedy2007}. In this context, the translation ‘speech’ seems the most apt, though it likely carries other connotations, like ‘reason’ and ‘argument’ with it. Kennedy also notes that Aristotle had “no technical term for prose” and that the word he used literally meant “in bare words” (\citealt[page 198, fn. 17]{Kennedy2007}).} } which encompassed prose oratory and writing. Here, Aristotle positions himself against other orators whose prose style was, in his mind, too poetic and, therefore inappropriate (\citealt[308]{Graff2005}; \textit{On Rhetoric} 3.2.1). From this same paragraph of book 3, chapter 2:

\begin{quote}{}
[L]et the virtue of style [\textit{lexeōs aretē}] be defined as “to be clear” [\textit{saphē}] (speech is a kind of sign, so if it does not make clear it will not perform its function)~-- and neither flat nor above the dignity of the subject, but appropriate [\textit{prepon}].\footnote{{Translations of} {\textit{On Rhetoric}} {are from George Kennedy’s 2007 second edition of the work \citep{Kennedy2007}.} }
\end{quote}

\noindent Clear language, according to Aristotle, results from orators using common words and metaphors to convey their intended meaning, as well as adhering to the “rules for speaking idiomatic, grammatically sound Greek,” detailed in book 3’s fifth chapter, “To Hellēnizein, or \textit{Grammatical Correctness}” \citep[314--315]{Graff2005}. Based on this discussion, \citet[314]{Graff2005} notes how Aristotle arrives at his idea of what “naked or unmarked” Greek was, “a sort of stylistic zero-degree.” This description of a “flat” Greek that is free from the excessive ornamentation of poetic language and, thus, grammatical resembles that similarly neutral early German that Germanic syntacticians have tried to isolate in their data. However, Aristotle’s rules for idiomatic Greek are not descriptive rules at all, but rather prescriptive rules designed to promote clarity in expression.\footnote{{See \chapref{sec:chap:7}’s discussion of well-formedness for a more thorough treatment of Aristotle’s rules of idiomatic Greek.} } In fact, what Aristotle advocates for is a particular \textit{style} of prose that paradoxically takes inspiration from “regular” and poetic speech alike. On the one hand, the author should consciously construct their language in ways that resemble everyday speech and seem natural or authentic. On the one hand, they should also draw linguistically on the right types of poetry, tragedy, mainly, in order to defamiliarize the speech for the interlocutor and lend it a degree of distinctiveness (\citealt[314--315, 330--331, 333]{Graff2005}).\footnote{I return to the topic of (literary) style in \chapref{sec:chap:7}’s discussion of well-formedness.}\largerpage[2]

It seems to me that classical, traditional, and colloquial definitions that associate prose with ordinary or natural speaking and writing underlie \citegen{Fleischer2006} methodology, something that becomes particularly evident in the value the study places on the \textit{Altdeutschen Gespräche}. As autochthonous prose, Fleischer’s hierarchy predicts that this text contains the most authentic data in that it is supposedly free from the confounding influences of Latin sources and poetic structures. Yet \citet[28--29]{Fleischer2006} sees the text as especially ideal for syntactic analysis for its unparalleled authenticity in that it exhibits a “prose” that is close to an early \textit{spoken} German. Consider the examples from the \textit{Gespräche}\footnote{{The} {\textit{Altdeutschen Gespräche}} {refer to two similar, short texts, one stemming from the Kassel Glosses, the other from a manuscript originally housed in the Bibliothèque Nationale in Paris (now Vatican manuscript Reg. 566) \citep[101]{Bostock1976}. Together they comprise a list of words, phrases, and short sentences; the} {\textit{Paris Conversations}}{, the longer of the two texts, has just a little over 100 sentences.}} in \REF{ex:2:10}.\largerpage[2]

\ea%10
    \label{ex:2:10}
\ea
\gll Gueliche   lande  cumen  ger\\
 which  land   come  you.\textsc{pl}\\
\glt ‘Where do you come from?’ (20)

\ex
\gll Guer    is  tin     erro\\
 where  is   your  lord\\
\glt ‘Where is your lord?’ (31)

\ex
\gll Nu   guez\\
 \textsc{neg}   know\\
\glt ‘I don’t know’ (32)

\ex
\gll Ubele     canet    minen teruæ\\
 bad     sevant    indeed\\
\glt ‘Bad servant, indeed!’ (36)

\ex
\gll Min  erro    guillo   tin   esprachen\\
 my   lord    wants  you   speak.\textsc{inf}\\
\glt ‘My lord wants to speak with you’ (43)

\ex
\gll Ne  haben   ne trophen\\
 \textsc{neg}   have  \textsc{neg} nothing\\
\glt ‘I don’t have nothing’ (74)

\ex
\gll Gimer    min   ros\\
 Bring.me   my   horse\\
\glt ‘Bring me my horse!’ (51)

\ex
\gll Coorestu  narra\\
 hear.you   fool\\
\glt ‘Do you hear me, fool?’ (65)

\ex
\gll vndes     ars  in  tino   naso\\
 dog.\textsc{gen.sg}   ass  in   your  nose\\
\glt ‘A dog’s ass in your nose!’ (42)
    \z
\z

\noindent These short sentences and phrases exhibit an “everyday language” (\textit{Alltagssprache}) that is not attested elsewhere in the corpus. For example, the language is emotive, at times vulgar, and indicative of the subjective involvement of the speaker. The utterances themselves are dialogic in that they assume the presence of an interlocutor. It is this “closeness to everyday life” (\textit{Nähe zum täglichen Leben}) that Fleischer identifies as the reason why the \textit{Gespräche}’s data are so authentic that the presence or absence of a syntactic structure among its lines is significant for the analysis of other, far more substantial, early German texts. Recall, for example, Fleischer’s arguments regarding multiple negation and how the fact that it was present in the \textit{Gespräche} confirmed their authenticity in a way that their presence in the poetic \textit{Evangelienbuch} could not.

This belief in the \textit{Gespräche}’s superior authenticity vis-à-vis data from other texts primarily because it exhibits everyday language demonstrates how pervasive these colloquial and traditional definitions of prose that associate the genre with ordinary, natural speaking and writing have been. Applying similar conclusions and assumptions to directly observable stages of the language, like modern German, reveals how this approach to early German denies the existence and legitimacy of different linguistic registers or varieties in the early medieval period. One would be hard pressed, for example, to argue that the genitive case, the simple past and extended participials are inauthentically German because they are completely absent from the everyday language of many German speakers. Modern German, like medieval German, is a multilectal phenomenon. Speakers have command over multiple different linguistic varieties and the language they produce in a colloquial context varies, likely considerably, from how they speak or write in a different context.

In sum, I argue that any claims that the \textit{Gespräche} exhibit a particularly authentic early German cannot simply rest on the observation that its language represents colloquial language: everyday language is no more or less authentic than any other register of language. In fact, there is good reason to question the \textit{Gespräche}’s linguistic authenticity. Consider, for example, that the manuscript shows Romance spellings for German sounds, such as <gu> for /w/, the omission of initial /h/ and its inclusion in places where it does not belong. These details indicate that an adult L2 learner of German initially wrote the text. The scribe was likely a French speaker, who spelled the words as they would have pronounced them, a circumstance that makes locating the text within a specific region impossible. \citet[102]{Bostock1976} explains how some features indicate the text is a French speaker’s approximation of Middle Franconian, while others point to West Franconian. Given the extent to which the scribe’s first language influenced spelling, it is not unreasonable to wonder whether other aspects of the text were affected.

I return now to the question that animates this section: what is prose? I have argued that prose has been too vaguely defined in scholarly discourse on early German syntax. It has been linked to spoken and written language that is imagined to be ordinary, natural, and grammatical, and this line of thinking has discouraged approaching historical languages as multilectal phenomena. This vague conception of prose underlies the conclusion that the \textit{Altdeutschen Gespräche}, with its reflexes of ordinary spoken language, constitute the most authentic text in the corpus. It promotes autochthonous prose as the epitome of authenticity, which, given this text type’s rareness, confirms for diachronic linguists how flawed the early German corpus is. And, finally, it frames the search for authentic early medieval German as a search for a colloquial, spoken competence, which is precisely the variety to which historical linguists have \textit{no reliable access}. By this I mean two things: first, that there are no living speakers of early medieval German to render grammaticality judgments for linguists and, second, that isolating data that are supposedly reflective of this colloquial, spoken competence involves constructing a relative scale of the supposed authenticity of early German’s four genres that is based on a faulty understanding of prose and, I might add, poetry.\footnote{{I engage more with the question of what poetry is in \chapref{sec:chap:6}, though I will not address it to the extent that I did prose in this chapter.} } Now, this does not mean we cannot gather data from across the corpus and identify basic patterns, for example, in the placement of finite verbs vis-à-vis other constituents. That is, they tend to occur in earlier position in stand-alone clauses and later position in clauses that might be connected to other clauses. But, as generativists will remind us, performance, i.e., data gathered from texts, is \textit{not} competence, and these data show lots of variability that makes reconstructing the underlying competence that supposedly produced them difficult. The reconstruction of the underlying competence is what sends diachronic linguists back to the deficit approach to find the authentic, or grammatical, data. And the deficit approach, as I have argued in this chapter, is a house of cards.

Given the fact that spoken and written language can differ so significantly from each other, it makes sense to adopt a more specific definition of prose that reflects this fact. The definition in the Oxford Dictionary of Literary Terms (4\textsuperscript{th}~edition) is a step in the right direction:

\begin{quote}{}
[Prose is] [t]he form of written language that is not organized according to the formal patterns of verse; although it will have some sort of rhythm and some devices of repetition and balance, these are not governed by a regularly sustained formal arrangement, the significant unit being the sentence rather than the line. Some uses of the term include spoken language as well, but it is usually more helpful to maintain a distinction at least between written prose and everyday speech, if not formal oratory. Prose has as its minimum requirement some degree of continuous coherence beyond that of a mere list. \citep{Baldick2015}
\end{quote}

\noindent Baldick’s definition links prose to the written language, while still acknowledging that formal spoken language can be like prose. It also considers how prose writing is organized in ways that distinguish it from other types of linguistic production. So, for example, the sentence is a basic unit of prose, and its language typically has some degree of “continuous coherence.” According to this definition, the \textit{Altdeutschen Gespräche} is not a prose text, while the \textit{Markbeschreibungen}, as well as the Isidor and Tatian translations are.

In linking prose to formal oratory, Baldick’s definition alludes to prose as a diachronic linguistic phenomenon. Garrett \citegen{Stewart2018} treatment of prose emphasizes the notion that prose is an invention that has developed over time.\footnote{{Garrett Stewart’s discussion of prose in its entirety approaches the topic within the discipline of literary criticism and, thus, falls outside of the scope of this book. It can be accessed here:} \url{https://oxfordre.com/literature/view/10.1093/acrefore/9780190201098.001.0001/acrefore-9780190201098-e-1084}. }

\begin{quote}
Prose is a fabrication, not a linguistic axiom. It has a complex history well before its intricate literary genealogy. Made, not given, prose comes down to modern use with the form, formally determined, of a world-historical invention. […] Born of empiricism and print culture, \textit{prose is neither neutered poetry nor transcribed speech}. Only its immediate ancestry is oratorical. [emphasis added]
\end{quote}

\begin{sloppypar}
\noindent \citet{Stewart2018} explicitly contradicts traditional and colloquial definitions of prose, stating that prose exists as its own historical phenomenon, independent of poetry and speech. Later in his discussion, he offers an explanation on how prose has come to seem so natural to today’s speakers, linking it to the emergence of new styles in prose writing: authors started to abandon earlier “embellished declamatory models” in favor, first, of “epistemological lucidity, later of verisimilitude in narrative fiction.” Still, this language of narrative fiction, one must remember, is generally unlike the language of actual spoken language, with its fragmented syntax and reliance on the pragmatic context of the immediate dialogic setting to create meaning.\footnote{{I explore this point in greater detail in Chapters~\ref{sec:chap:3} and \ref{sec:chap:4}.}} Though, of course, some writers intentionally try to represent more faithfully naturally occurring dialogues or a protagonist’s interior monologue. The point is, however, that writers \textit{created} prose as part of the long process of literization and continue to consciously develop it. The ways in which they shape their prose reflect their historical context, the purpose of the text itself, and their own educational background and/or identification with a particular intellectual or cultural movement.
\end{sloppypar}

Drawing on the definitions of Stewart and Baldick, I define prose as follows: it is one linguistic variety that is generally written; it is a human innovation developed in specific sociocultural contexts to meet the needs of the writer, rather than a neutral representation or transcription of spoken language or denuded poetry. It does not simply exist, and it is not a naturally occurring linguistic phenomenon. Rather, people must create it. This definition of prose fits more comfortably within a multilectal view of early medieval German.

\subsection{Which early medieval text has the best data and why isn’t it Otfrid?}\label{sec:2.2.2}

What follows is not an exhaustive discussion of every study that reflects \citegen{Fleischer2006} ideas and views variable data in the corpus as a problem requiring remediation, relies on traditional and colloquial definitions of prose, and approaches the language as a monolectal phenomenon. Most studies on early German syntax fall into this category, even if they do not directly identify their assumptions, as I argued at the beginning of the section is the case with \citet{Axel2007}. Instead, I discuss a selection of studies that are reflective of Fleischer’s methodology and have settled on different texts as the best source of authentic data. Each adopts different strategies for sorting through the structural variation in order to delineate an early German competence. In this way, these works all offer structural analyses of early medieval German and approach their chosen text as a means of reconstructing an underlying grammar, rather than as an artifact, a \textit{scriptus}, created by a person with access to multiple linguistic resources.

I first examine \citet{PetrovaSolf2009} and \citet{Robinson1997} in some detail. Both studies are emblematic of the argument that one prose text can constitute the best source of early German syntax, though the former chooses the Tatian translation, while the latter chooses Isidor. Another unstated assumption becomes apparent in Robinson’s work: that authenticity or grammaticality correlates with structural regularity. This association between authenticity and regularity is also evident in \citet{Lötscher2009}, who analyzes Otfrid’s \textit{Evangelienbuch} and argues that variable data result from the presence of two syntactic systems. That is, grammatical word order rules, or a core grammar, produced the regular data that mirror modern-like syntactic patterns, which are also evident in texts like the Isidor, while data that do not reflect modern patterns are pragmatically arranged according to archaic and poetic information structural principles.

I begin with \citet{PetrovaSolf2009}, which directly references \citet{Fleischer2006} and its deficit, structuralist approach to the early German corpus. Thus, this work also adopts the attendant assumption that an underlying competence can find expression in supposedly neutral or natural prose writing, which would have had a close relationship to ordinary speaking. The article characterizes the lack of autochthonous prose as indicative of a problem in the “reliability” of the data that researchers must counter by asking “whether the word order patterns and syntactic constructions […] are representative for the system of the language under investigation” (pages 122--123). Because that perfect text containing “authentic prose [that is] representative for the system of the dialects spoken at the period of time” (page 123) does not exist, the authors propose that “true Germanic syntax” can by isolated in the \textit{Differenzbelege} from the Tatian,\footnote{{This discussion does not contend with the Isidor translation, though the authors state that they want a text of “considerable size” (page 123). The implication is that the Isidor was not long enough for the authors’ purposes. The study’s analytical focus on the} {\textit{Differenzbelege}} {also suggests that they rejected the German translation of the Isidor as a source text because it is a more independent translation than the Tatian. This decision is also consistent with \citegen{Fleischer2006} implied hierarchy of texts (see \sectref{sec:2.1.2}).} } that is, those data in which the German deviates sufficiently from the Latin source. Specifically, they argue that one can investigate the role of information structure in early German simply by looking at deviating data from this one particular translational prose text (page 154). This argument indicates that, though the authors recognize the existence of a multilectal early medieval German, they still assume that one competence underlies this grammar. Analytically, their approach is monolectal in that they do not recognize the possibility that an individual’s multilectalism (and multilingualism) plays a role in the \textit{scriptus} they create. Furthermore, the authors assume that they can reconstruct this one competence if they consider only its genuine reflexes in the data. Data from other texts can contribute to this understanding, the authors note on page 126, but it is a unitary grammar that \citet{PetrovaSolf2009} aims to depict.\largerpage

\citet{Robinson1997} has similar goals to \citet{PetrovaSolf2009}, though its author concludes that it is the Isidor that is the “best early source of Old High German prose” (page 2). It also becomes clear that Robinson’s ideas about prose and its connection to an early German grammar and ordinary speaking resemble Petrova \& Solf’s. Consider how the author claims that the prose of the Isidor was “quite near” to “the common spoken language” (page 4) because the translator wanted their work “read out loud as a polemical piece in a contemporary theological debate, aimed at persuasion more than instruction” (see \textcites[29]{Nordmeyer1957}[27; 33]{Nordmeyer1958} as the sources of Robinson's argument). Presumably, Robinson means to argue that the translator wanted to make the language as accessible as possible so that listeners would be persuaded rather than confounded and, so he composed in a “common,” which is to say, colloquial, perhaps neutral language that resembled ordinary speaking. Recalling my earlier observation that this conflation of prose with ordinary speaking and writing obscures the fact of a multilectal early medieval German, it is telling that Robinson does not comment on which register or variety of an eighth-century spoken vernacular would have been appropriate for the translation of the rhetorically and theologically sophisticated Isidor treatise.\largerpage

A more specific definition of prose, however, as a human innovation and a socioculturally determined variety of written language, might incline one toward the arguments in \citegen{Matzel1970} detailed study instead. Matzel summarizes the Isidor project on page 495, emphasizing the difficulty of the undertaking. The translator, first, had to understand a theologically challenging text, written in a complex and advanced Latin. Then they needed to find a way to transfer the original treatise into German such that “the meaning of its sources was retained, and the needs of the vernacular met” ([…] \textit{die lateinischen Texte} […] \textit{so in die deutsche Sprache umzusetzen, dass der Sinn ihrer Vorlagen und die Belange der Muttersprache gewahrt bleiben}). In addition to these formidable challenges, Matzel notes, the translator created “regulated language” (\textit{geregelte Sprache}), which most particularly expressed a “differentiated orthography” (\textit{differenzierten Orthographie}) and consistently implemented this system throughout the text’s composition. The language of the Isidor, thus, is not a neutral representation of some spoken variety. Rather it is a carefully constructed work of prose~-- perhaps the first in the history of German~-- and the translator’s attempt to rise to the challenge laid down by the emperor himself to turn spoken vernaculars into a written language that was \textit{schrift- und damit literaturfähig} (‘the language has a written form and, thus, one can compose literature in that language’).\footnote{{The reference here is to the famous line from Einhard’s biography of Charlemagne: \textit{inchohavit et grammaticam patrii sermonis}. I discuss this line and its significance in \sectref{sec:3:3.1.1}.}} This description is reminiscent of \citegen[1--3]{Gering1876} assessment that the Isidor text and its language were “a product of erudition, an artistic creation, for which the author’s mother tongue provided no model” (\textit{ein produkt der gelehrsamkeit, eine künstlerische schöpfung, die in der muttersprache kein vorbild hatte}).

\citet{Robinson1997} does not accept Gering’s conclusions, and the reasons behind this rejection are twofold. First, Robinson misapprehends prose as a simple graphic transfer of seemingly authentic, ordinary language into writing, rather than as a necessarily human construction that, as I argue in Chapters~\ref{sec:chap:3} and~\ref{sec:chap:4}, requires linguistic innovation. Second, Robinson seems to believe that a neutral, grammatical syntax will exhibit regular, not overly variable, patterns, especially if one controls for confounding factors and selects a text that is more likely to exhibit so-called authentic German. This linking of grammaticality and regularity becomes evident when the author discusses why he decided not to analyze Otfrid’s \textit{Evangelienbuch}: its poetic structure is a “significant (and negative) factor” affecting the ability to gather reliable data (pages 2--3). It is telling how Robinson summarizes the confounding effects of meter and rhyme: he targets \textit{only} variability in Otfrid’s clause structure as the negative consequence of poetic influence. Robinson cites \citegen[485]{Wunder1965} study on the subordinate clause in the \textit{Evangelienbuch} and characterizes the following quote as the latter author “essentially throw[ing] up his hands”: “These examples illustrate how difficult, indeed impossible, it is to formulate a clear description of the placement of the finite verb in subordinate clauses in Otfrid’s work” (\textit{Die Beispiele machen deutlich, wie schwierig, meines Erachtens unmöglich, es ist, ausgehend von Otfrid, genauere Aussagen über die Stellung des finiten Verbs im Nebensatz zu machen}). As \citet[370--373]{Somers2021a} points out, in implying that Otfrid’s syntactic data is of dubious worth because of its structural variability, Robinson asserts a correlation between authentic, grammatical early German and a certain degree of syntactic regularity.

This idea that ordinary early spoken and written German must have been syntactically regular is made more explicit when \citet[3]{Robinson1997} argues against \citegen[3]{Gering1876} characterization of the Isidor translation as an “artistic creation” (\textit{künstlerische schöpfung}) with no existing model in the vernacular. Robinson counters: “[i]f this were correct, of course, my book would be nothing but a description of this “künstlerische schöpfung,” and the regularities I find would be nothing but the creation of one brilliant translator.” This statement combined with the discussion of the disadvantages of the Otfrid text demonstrates Robinson’s underlying belief that the structural regularity of the Isidor translation is evidence of its particular authenticity. The Isidor’s data are so genuine, in fact, that the author concludes that, as the “earliest large and dependable sample we have of the syntax of the German language […] [w]e should not simply merge it with all other Old High German texts to serve as the starting point for a history of German syntax. In many ways, it should \textit{be} the starting point, certainly for word-order issues” (page 5; emphasis in the original).

In this passage, one also finds a clear distillation of Robinson’s structuralist orientation toward the early medieval German corpus. That is, the significance of its texts lies solely in their ability to reveal an underlying competence. Ironically, in arguing against Gering’s description of the Isidor, Robinson comes close to a different way of approaching its German translation without realizing it. Gering’s characterization is reminiscent of the literization approach in that he identifies the Isidor \textit{scriptus} as a human construct. While Gering sees this creation primarily as artistic, I see it as the product of necessary linguistic innovation. This innovation is rooted in the linguistic resources to which the literizer had access. They do not literize in random fashion or without consulting their own linguistic intuitions, though stylistic, i.e., aesthetic or artistic, concerns surely play a role as well. Robinson thinks that it is a trivial matter to describe the artifact, the \textit{scriptus}, and that significant work in diachronic linguistics can only be about identifying a competence. I submit that describing a \textit{scriptus} and identifying how different linguistic varieties shaped its creation as a singular artifact is a worthwhile endeavor that offers diachronic linguists a new and exciting way forward.

To be clear: I am not arguing that natural language or spoken language is devoid of structural regularities.\footnote{{Though compare the syntax of the structures produced in a spontaneous conversation between two friends to a work email that one of those friends sends to their boss and see which one is more syntactically regular.} } Instead I argue that the presence of regular patterns does not confirm the authenticity of the Isidor’s language as a reflection of an early German competence, as Robinson implies. Nor does it speak against its being the result of the conscious literization of the spoken vernacular undertaken by one writer, who had the educational and linguistic resources to try such a thing. Indeed, the care taken to develop a regularized orthography, as \citet{Matzel1970} lays out, speaks in favor of the opposite conclusion, that the text is exactly as Gering described it: a masterful, creative achievement. Accepting Gering’s assessment does not diminish the significance of the Isidor or mean that \citet{Robinson1997} is not an important study. It simply changes why they are important. Instead of being the starting point of German syntax, the Isidor translation marks the beginning of a history of German \textit{prose}, as a specific linguistic, socioculturally determined development.

I turn now to \citet{Lötscher2009}, a work that seems to fly in the face of the studies discussed so far in that its focus is the poetic \textit{Evangelienbuch}. Recall that \citet{Fleischer2006} concluded that Otfrid’s text is the least likely among those in the corpus to yield authentic data because the poet was willing to break the “grammatical rules” of early German and produce “mistakes” for the sake of the end-rhyme. \citegen{Robinson1997} discussion of Otfrid is consistent with this argument; he identifies the \textit{Evangelienbuch}’s apparent variability in constituent ordering as the main effect of such poetic distortions. \citet{Lötscher2009}, as a result, must develop some alternate understanding of what constitutes authentic or grammatical data to justify an analysis of Otfrid in the first place. If these poetic data are so unreliable, why not simply ignore them?

\citegen[283]{Lötscher2009} solution is to hypothesize that all data are authentic, but some of them are evidence of a core “grammatical” system that produces modern-looking clauses, like the ones Robinson found in the Isidor text, while the data that do not align with this core grammar were ordered pragmatically, or according to “information-structural conditions.” This approach builds on the work of \citet{Lenerz1984, Lenerz1985}, which argues that variability in the placement of clausal constituents is due to speakers having access to an inherited and an innovative grammar. In the former, speakers produce a new, unique to Germanic, head-final complementizer phrase. This phrase forms the basis of Modern German’s clausal syntax. Speakers also produced an inherited COMP-less verb-final clause (see, for example, \citealt[117--119]{Lenerz1985}). According to \citet[182]{Lötscher2009}, the \textit{Evangelienbuch} exhibits clauses that fit the modern pattern and can, therefore, be explained as products of the “core grammar of his language.” Yet these “grammatical” clauses exist alongside other clauses that follow an archaic, pragmatically driven pattern. These informational structural principles are “special rules, be it of an older tradition, be it of poetical language as a subsystem of the overall system.” Note that the assumption that early German poetic language must have also been archaic (i.e., that poetic syntax is archaic syntax) makes this whole argument seem perfectly logical: naturally, Otfrid would reach for poetic, and thus, archaic language in the composition of a poem, and, as a result, his more variable syntax, which often defies the modern clausal pattern, is neatly explained.

The information structure approach to early German syntax is more tolerant of variable data than the approaches described in \citet{Fleischer2006} and \citet{Robinson1997} in that it encourages scholars to look for patterns in data that they might otherwise write off as inauthentic or ungrammatical, which is to say, not representative of a real early German competence.\footnote{{This is the approach taken, for example, in the many articles that comprise the volume,} {\textit{Information structure and language change: New approaches to word order variation in Germanic}}{ \citep{Hinterholzl2009}.} } However, this approach does not actually rehabilitate the \textit{Evangelienbuch} as a text of analytical consequence. If you adopt \citegen{Lötscher2009} methods and conclusions, Otfrid’s work can only confirm the regular syntactic patterns that are attested more broadly in the translational Isidor and Tatian translations. Any data that are inconsistent with such patterns are treated as essentially different, produced by a special, separate, and more peripheral system. The decision to examine the \textit{Evangelienbuch} within this framework would only make sense if one had a particular interest in so-called poetic grammars or wanted to take advantage of Otfrid’s supposed window into a prehistoric syntactic system. For \citet{Lötscher2009} the variable data do not prompt a more fundamental reexamination of how scholars might approach Otfrid’s text in the first place. Instead of insisting there is only one grammar diachronic linguists ought to be interested in identifying, there are now two: the “core” grammar, that yields an ordinary, authentic language similar to what the prose texts exhibit, and a “poetic” one that produces special or archaic language.

\section{Conclusion}\label{sec:2.3}

In this chapter I laid out a case against the deficit approach to the study of early German syntax, which maintains that the corpus comprises mostly problematic texts (\citealt{Fleischer2006, Schrodt2004}). The problem stems from two sources. First, the structuralist orientation and goals of our disciplinary forebears have given modern diachronic linguists the idea that the only object worth pursuing in our research is the competence that underlies performance. Thus, Germanic linguists have sought the data, i.e., performance, that in their view faithfully reflects an early German grammar. The preoccupation with structure leads to the second problem, which is that early German texts do not represent the sorts of genres that diachronic linguists assume will contain the most authentic data, i.e., the data most reflective of an early German competence. Of the four logically possible genres, autochthonous prose, autochthonous poetry, translation prose, translational poetry, it is the first genre that is the most likely to yield reliable data, so says the deficit approach. This is because the confounding influences of a Latin source text and a poetic meter, which compromise the authenticity of data gathered from the second and third text genres in the above list, have no effect on autochthonous prose. Unfortunately, the corpus is made up almost entirely of autochthonous poetry and translational prose, with only a few short texts providing reliable data on the structure of a spoken, everyday medieval German, without the complication of having to control for the effects of confounding factors. Thus, diachronic structuralists need the deficit approach to help them navigate this tricky empirical landscape.

This story is axiomatic in the literature and, so, it was my goal in this chapter to identify the assumptions that underlie it, explain where I thought they were based on misapprehensions, and propose that there is another way to approach the study of early medieval German. For example, colloquial and traditional definitions of prose as representing the same neutral, underlying grammar as “ordinary speaking and writing,” have convinced scholars that these translational prose texts contain data that are more authentic or grammatical than the era’s major poetic texts. Thus, many investigations of early German syntax gather data exclusively from the prose translations (for example, \citealt{Axel2007, Robinson1997, PetrovaSolf2009, Schlachter2012}). Approaching the corpus in a monolectal way has allowed scholars to claim that, in analyzing what they assume is the best source of authentic data, be it the Tatian or Isidor translation (never Otfrid), they are delineating an authentic early German competence or, as these studies generally call it, an Old High German grammar. Even studies that do consider data from poetic texts, like \citet{Lötscher2009} and \citet{Lenerz1985}, do not adjust their fundamental ideas about where authentic data can be found. Instead, they characterize the poetic data as peripheral to a core German grammar, to use Lötscher’s turn of phrase, and propose a different set of rules to account for this other grammar.

What none of these studies has done is treat each text like an individual linguistic artifact that results from an early medieval speaker literizing their vernacular for the first time. This literizer does not speak just one variety of German, say, this imagined ordinary variety or an everyday, colloquial language. Rather they are multilectal, in that they speak many varieties of German ranging from colloquial to formal, and to some extent multilingual, in that they have been educated in Latin and the classical discourse of \textit{grammatica}.\footnote{{It is of course possible that a Carolingian literizer spoke other languages as well, say, the so-called rustic Latin of West Francia, which had diverged considerably from written Latin. There is also the possibility, if the literizer was one of the missionary monks from England, that the literization of early English influenced early German’s literization. One still returns to Latin as the main model of a written language in that English’s literization was also influenced by Latin and a Latinate literacy.} } In order to produce a \textit{scriptus}, they must engage these multiple linguistic resources. How they choose to do so reflects the literizer’s sociocultural context and the nature of their writing project. The resultant \textit{scriptus}, as I argue in the next chapter, necessarily involves linguistic innovation so that the theretofore exclusively oral vernacular can function in the new dislocated context that writing alone can create.

