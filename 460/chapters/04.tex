\chapter{How to create an early German \textit{scriptus}}\label{sec:chap:4}

\section{Introduction}
In \chapref{sec:chap:3}, I explained what I have been referring to as the “functional gap,” that exists between the many varieties of an exclusively oral vernacular and the lexical and grammatical explicitness that writing requires. That is, the written language must be able to function in places that are completely dislocated from the language producer and the moment of linguistic production. Thus, the initial literizations of a vernacular require that the writer innovate linguistically. They must shape their spoken language into a written form, thereby creating a \textit{scriptus} that is a functional graphic representation of linguistic expression. Recall from \sectref{sec:3:3.1} that a \textit{scriptus} is an early and \textit{ad hoc} vernacular literization that one writer or a team of writers creates for the purposes of their individual project. These \textit{scripti} have limited or no influence on one another in that they are produced before the establishing of any writing tradition. In this chapter, I elaborate on the process of creating an early \textit{scriptus} by drawing on the concept of “ausbau” (from German \textit{Ausbau}, meaning ‘construction’), which originates from the works of Heinz Kloss (\citealt{Kloss1967,Kloss1978}). Kloss’s analyses focus on the creation of modern ausbau languages, and, so, his work is less directly applicable to medieval \textit{scripti}. Fortunately, \citet{KochOesterreicher1994} develops the concept of ausbau further so that it may also apply to earlier \textit{scripti}. The discussion that follows elaborates on the specifics of creating a \textit{scriptus} and strengthens \chapref{sec:chap:3}’s argument that vernacular literization necessarily constitutes more than a simple medial transfer of some spoken variety into a written form.

This chapter contributes to this book’s overarching argument that researchers in historical syntax, particularly those focused on the earliest attestations of a language, should consider how the act of writing an exclusively oral language down for the first time itself interacts with the shape that the \textit{scriptus} takes. With this statement, I mean that the people who give their spoken language a written shape in the absence of a vernacular literary tradition must decide how the oral vernacular’s graphic form will meet the new demands of a written distance language (\textit{Sprache der Distanz}). This chapter’s discussion, then, describes literization and ausbau as universal processes, rather than ones that are specific to the case of German.

The literature on ausbau, particularly \citet{Kloss1967, Kloss1978} and \citet{KochOesterreicher1994}, provides a clearer view of what the structural, textual, and lexical requirements of a dislocated language of distance are and how unliterized languages fall short of these requirements. With respect to eighth- and ninth-century German \textit{scripti}, these early, individual literizations exhibit only as much coherence and consistency as the person or people who constructed the \textit{scriptus} were able, or cared, to effect. In order to illustrate this point, consider again the case of the Isidor text. As \citet[521--523]{Matzel1970} explains, even its erudite translator, who consciously worked toward orthographic uniformity and consistency, was not able to achieve the sort of regularity that characterizes modern written languages. This point applies just as readily to ausbau: each literizer who is working out how to create a functional \textit{scriptus} based on their multilectal and multilingual linguistic resources will not simply arrive at different solutions than the literizer engaging in their own literization project a couple hundred miles and years away. There will also be variation within each \textit{scriptus} because achieving consistency is challenging, especially so in an early medieval context. This will be particularly the case in the longer, originally composed texts of the \textit{Hêliand} and Otfrid’s \textit{Evangelienbuch}, each of which comprises thousands of lines of poetry.

  The structure of this chapter is as follows. I begin by discussing the concept of “language ausbau” as it relates to the literization of exclusively oral vernaculars in \sectref{sec:4.1}. Of interest here is the work of Kloss, both his 1967 article and 1978 book. Though Kloss’s model requires significant adaption before it can be useful for the analysis of the earliest attestations of a language, I devote this whole section to his ideas on ausbau for two reasons. First, the concept of ausbau has made little inroads into scholarship on early German and Germanic syntax, and I hope this discussion convinces historical linguists that they ought to consider it.\footnote{\textrm{One must exercise caution when engaging with the work of Heinz Kloss. \citet{Wilhelm2002} describes how Kloss wrote propaganda masquerading as research for the Nazi regime during the 1930s and '40s. In 1944, Kloss also wrote} \textrm{\textit{Statistik, Presse und Organisationen des Judentums in den Vereinigten Staaten und Kanada~}}\textrm{(‘Statistics, press, and organizations of Jewry in the United States and Canada,' \citealt{Kloss1944}), which is a detailed census of Jewish populations and organizations in North American cities. I was able to find only digital copies of the odd page; it seems to comprise mostly lists comparing the number of Jews to the overall population, from the largest cities and their many thousands of Jews to the smallest towns with Jewish populations only in the dozens (}\url{https://www.cbc.ca/news/canada/ottawa/hitler-book-library-and-archives-canada-1.4989961}{)}\textrm{. The work was commissioned by the Nazi regime and was part of Hitler’s personal library. Such detailed statistics, regardless of how neutrally presented they may be, are not} \textrm{\textit{actually}} \textrm{neutral when understood in their historical context. That is, Kloss wrote it in and for a country whose government meticulously planned and carried out the murder of up to 6 million Jews (}\url{https://encyclopedia.ushmm.org/content/en/article/final-solution-in-depth}\textrm{). With respect to the current work, I considered whether Kloss’s basic concept of the ausbau language reflects Kloss’s role as a Nazi researcher cum propagandist in ways that delegitimize the concept itself. I do not think it does. Additionally, there appears to be no link between Kloss’s concept of ausbau language and the statistical work he carried out for the Nazis. An argument certainly can be made that Kloss’s early work compromises his later work to the point that one should simply avoid him. I chose, instead, to refer to his writings on ausbau languages, while also making clear the possibility that his linguistic writings are compromised by bias and anti-Semitism.}} Second, recognizing ausbau as central to the long and arduous language literization process provides more ammunition against the deficit approach to historical German syntax (see \chapref{sec:chap:2}) by clarifying what neutral, natural prose really is: namely, the end product of a long process of literization and ausbau. In \sectref{sec:4.2}, I discuss the goals of language ausbau as processes in which writers attempt to enhance the semantic and grammatical coherence of a conceptually oral vernacular. Here I draw on \citet{KochOesterreicher1994}, which expands on Kloss’s work in ways that make it more directly applicable to the creation of early \textit{scripti}. I also further develop their mostly sketched out descriptions of ausbau’s potential areas of focus.

One final introductory comment is in order: I am indebted to \citegen{Höder2010} book on the ausbau of Old Swedish, which introduced me to the concept of ausbau and the work of Kloss. Though I do not engage extensively with Höder’s analysis in this chapter, the way he approached his topic of an early Scandinavian \textit{scriptus} informed how I am approaching mine.

\section{Literization and language ausbau: Kloss (1967) and (1978)}\label{sec:4.1}\largerpage[1.75]

\citet[37]{Kloss1978} argues that turning an exclusively oral vernacular into a written language comprises three kinds of tasks. The first of these tasks is alphabetization, which Kloss conceptualizes as a simple medial transfer of the phonic into a graphic medium. Because no normalized orthography exists at this stage, early \textit{scripti} will exhibit variable spellings (\textit{Rechtschreibungsweisen}) (page 46). After alphabetization, writers regularize the language’s orthography, grammatical structures, and lexicon; they then creatively shape the language so that it can become a “standardized tool of literary expression” in a modern society. This molding, or remolding, of a language is what Kloss calls “ausbau” \citep[29]{Kloss1967}. It can and, one assumes, often does, unfold simultaneously with regularization.

Ausbau, which is the real locus of linguistic innovation in \citegen[37--38]{Kloss1978} model, involves the development of new stylistic devices (\textit{Stilmittel}), on the one hand, and new domains for the use of distance varieties (\textit{Anwendungsbereiche}), on the other.\footnote{\textrm{Kloss’s work pre-dates that of \citet{KochOesterreicher1985} and their descriptions of languages of immediacy and distance (}\textrm{\textit{Sprachen der Nähe und der Distanz}}\textrm{). Kloss himself did not see the new linguistic domains effected by ausbau in these terms. The current presentation is the result of my using \citet{KochOesterreicher1985} to elucidate Kloss’s ideas.}} Developing new stylistic devices generally entails the creation of a more differentiated lexicon and syntax, though sometimes it involves their simplification. The new written domains created through ausbau fall into four categories: literary genres (\textit{Belletristik} or \textit{belles-lettres}), including poems, plays, epics and narrative prose; expository prose (\textit{Sachprosa}, \textit{nicht-dichterische Prosa}); and so-called \textit{Schlüsseltexte}, which Kloss defines as translations of original texts that contain ideologically significant content, such as, vernacular translations of the Bible in Western Europe or of Karl Marx’s \textit{Das Kapital} in Cuba and Angola. The fourth and final distance variety domain created through ausbau is the “speech-text” (\textit{Zusprachetext}), which are texts conceived in/with the aid of writing but delivered orally. Included in this category are the news, delivered on television or the radio, sermons delivered from the pulpit, and speeches delivered from the podium (pages 38--39).

A language is a fully realized ausbau language once it has “conquered” each of these four domains, though Kloss does not give every domain equal weight in the ausbau process \tabref{tab:4:4.1}.

%%please move \begin{table} just above \begin{tabular
\begin{table}
\caption{Kloss's quantitative representation of ausbau}
\label{tab:4:4.1}
\begin{tabular}{lr@{~}l}
  \lsptoprule
  “key-texts” (\textit{Schlüsseltexte})   &   100  & \\
  literary texts (\textit{Belletristik})  &    200 & \\
  “speech-texts” (\textit{Zusprachetexte})&    300 & \\
  expository prose (\textit{Sachprosa})   &   400  & \\
  \midrule
                  \hfill $\to$       &         1,000 & points\\
        \lspbottomrule
\end{tabular}
\end{table}

\noindent In \tabref{tab:4:4.1}, \citet[39]{Kloss1978} offers a quantitative representation of how much the development of each new domain contributes to the language’s overall ausbau. According to this view, the development of key-texts contributes the least to ausbau: these translations are “thematically restricted” by the contents of the original text and, thus, require only minimal ausbau of the language’s lexicon, according to Kloss. I can find no explicit discussion of why Kloss weights the other domains as he does. What he does provide is a discussion of the difference between expository prose and literary texts. Namely, literary texts are beholden to important aesthetic objectives; how a writer constructs their literary writing is just as important as what they write, that is, the content. In contrast, the purpose of expository prose is to convey meaning with no attention paid to its artfulness; the content of the writing alone is what matters. Expository prose runs the gamut from the basic “everyday prose” (\textit{Alltags-\slash Jedermannsprosa}) that is taught in schools and that citizens of developed countries are expected to have some command of, to the prose of academic research (\textit{Forscherprosa}) \citep[41--45]{Kloss1978}.

As I see it, Kloss defines expository prose as a sociolinguistically specific phenomenon that only emerges after an industrialized society has engaged in extensive language ausbau. This type of prose is intended to be broadly accessible. Thus, it requires the existence of a regularized or, indeed, standardized, written language that large swaths of people can use routinely and in clearly defined, widely recognized linguistic domains (Kloss’s \textit{Anwendungsbereiche}). Recalling \citegen{KochOesterreicher1985} framework, I would place Kloss’s expository prose at the far right, “language of distance” pole of the continuum: its intended accessibility means that the contexts in which it will be produced or received cannot easily be anticipated. It must be equipped with a grammar and lexicon that enable the language’s total dislocation by allowing for clarity in any possible context. Thus, there is a logic to Kloss’s conclusion, represented in \tabref{tab:4:4.1}, that the development of expository prose constitutes a strong marker of that language’s ausbau.

A comparatively weak marker of a language’s ausbau, according to Kloss, is the existence of literary texts. This weighting makes sense from a historical perspective, at least with respect to the history of German. Consider how writers, like Otfrid and the \textit{Hêliand} poet, composed original literary works in the vernacular long before the development of widely accessible expository prose and the regularized grammar and lexicon upon which this sort of writing relies. Note that early translations, like that of Tatian’s \textit{Evangelienharmonie}, do not belong in Kloss’s category of expository prose at all. Rather, they are key-texts (see \tabref{tab:4:4.1}) and developmentally far removed from the accessible, neutral- and natural-seeming style of prose that writers of German may produce today (see my arguments in \sectref{sec:2.2.1}).

One particular aspect of the way in which Kloss imagines the literization of an oral vernacular deserves closer examination. Namely, the alphabetization phase. I see two problems with Kloss’s conceptualization of this first stage of literization: first, that Kloss characterizes it simply as alphabetization and, second, that by calling it a \textit{Vorphase} (‘preliminary stage') of literization, he indicates that it occurs separate from, or as a prelude to, the supposedly real literization processes of ausbau and regularization. Supporting this interpretation is \tabref{tab:4:4.1} which assigns no ausbau points to the development of early \textit{scripti}, indicating that this phase is not part of ausbau at all. In this respect, Kloss’s conceptualization of a language’s first attestations is consistent with the deficit-approach treatment of the earliest German texts as simple medial transpositions of existing spoken varieties into a written form requiring no additional change. The variability across early texts, according to Kloss, stems only from literizers making different spelling choices in the absence of orthographic norms. Compared to the literization process that I have been outlining in this book so far, Kloss sees the creation of an oral vernacular’s first \textit{scripti} in a severely restricted way. Recall that I have defined literization as entailing both the development of literacy as a conceptual category and the overt changes that literizers make to the language. In imagining the creation of early \textit{scripti} as an alphabetization that involves no structural changes or innovative ausbau, Kloss implies that exclusively oral vernaculars require no adaption to work in the graphic medium. Early literizers give it a new visual form, but otherwise are simply reproducing an existing vernacular grammar and lexicon in a new medium. According to this view, historical linguists should theoretically have direct access to this early spoken grammar, as long as they are able to sift through the orthographic idiosyncrasies across texts and eliminate data influenced by extragrammatical factors, like meter, rhyme, or Latin.

Kloss’s characterization of early literizations as resulting from engagement only with spelling, that is, a medial transfer, and not with other linguistic structures becomes more apparent in his treatment of the translation of key-texts. He does not think their production contributes much to the language’s overall ausbau (see \tabref{tab:4:4.1}) because the translated text requires only the development of new lexical items as determined by the original text \citep[39]{Kloss1978}. So, taking the early German translation of Tatian’s \textit{Evangelienharmonie} as an example, the primary challenge for translators would have been creating German words for Biblical concepts. What Kloss does not consider is the structural gulf that must have existed between Latin, a language with a written tradition over a millennium long, and German, an almost exclusively oral vernacular in its first stages of literization. By the early medieval period, Latin was a well-developed written language that could meet all of the demands of distance that existed at the time. Literizers of German, in this case the translator of the Tatian text, had to create German structures that mirrored the Latin ones. This had to involve grammatical, as well as lexical innovation.

Overall, though, Kloss’s work is still necessary and important in that it emphasizes the human agency involved in the construction of written languages. People must deliberately undertake literization; they must innovate and plan in order to create an ausbau language, for example. Modern ausbau languages, like German and English, do not evolve on their own \citep[38]{Kloss1967}. Kloss’s dissection of expository prose (\textit{Sachprosa}) is also a reminder that a neutral prose style of writing, with its focus on conveying a message clearly, is the product of considerable ausbau. One should not expect to find such texts among the newly created and \textit{ad hoc} early medieval vernacular \textit{scripti}. Nor do such texts resemble unliterized oral vernaculars or provide evidence of especially natural or authentic~-- or grammatical~-- language.

There are two main problems I see in Kloss’s model of literization, The first is that it does not incorporate the possibility that the early \textit{scripti} are also products of ausbau. This blind spot in Kloss’s model could stem from the author’s focus on modern language developments and literizations. Given these research interests, it follows that Kloss conceives of ausbau as the more targeted language planning that happens in conjunction with modernization, industrialization, and nationalization, all of which create new and numerous domains in which the written language must also be able to function. This problem relates to the second: Kloss fails to consider the conceptual changes that must be part of the early literization process. In fact, pushing oral vernaculars into dislocated written contexts in which those varieties are going to be, to one extent or another, dysfunctional, becomes the source of a growing awareness of literacy as a conceptual category that is distinct from orality. As a result, Kloss’s work never examines or appreciates the roles that literization generally and, as I discuss in \sectref{sec:4.2}, ausbau specifically play. Both are in fact necessary for filling in the functional gap between the varieties of an exclusively oral vernacular and a written variety, which requires a new level of semantic and grammatical coherence.

\section{Language ausbau and early \textit{scripti} }\label{sec:4.2}

\citet[589]{KochOesterreicher1994} offers a broader definition of ausbau than Kloss in that this work develops the concept of ausbau to include modern and medieval literizations alike. That is, it defines language ausbau as the long, difficult process of molding a written language so that it can meet all of the communicative demands of distance. Thus, ausbau as a process is also relevant to early literizations where the functional gap between what were once exclusively oral vernaculars and the new distance contexts created by writing is particularly acute. It applies equally well to later ausbau that allows a language to function in any new context of distance~-- or immediacy, for that matter~-- that result from industrialization, modernization, or any other historical development. \citet[589]{KochOesterreicher1994} also distinguish two types of ausbau: intensive and extensive. Intensive ausbau is specifically about how writers adapt the language structurally and lexically, in order to meet the demands of literization, while extensive ausbau describes the process of extending a \textit{scriptus} into all existing and newly emerging contexts. It is the former process of intensive ausbau that is particularly relevant to the current project and its focus on early German \textit{scripti}.

  In the section that follows, I examine \citegen[590--591]{KochOesterreicher1994} descriptions of intensive ausbau, which they divide into three categories. First, they identify the “textual and pragmatic” dimension; next, the syntactic dimension; and, finally, the lexical and semantic aspects of intensive ausbau. For reasons that become clearer below, I do not maintain all of the distinctions that the authors themselves establish. In particular, I conclude that one aspect of their textual ausbau, namely, the augmenting of the grammatical coherence of a language, relates better to their separately discussed category of syntactic ausbau. It is this dimension of ausbau that is the most relevant to the current study and so, it receives the most attention here. Furthermore, I rearrange and reconceptualize Koch and Oesterreicher’s categories in a fashion that, I believe, better captures the commonalities that different aspects of ausbau have. The two main analytical categories that I explore in some depth are the lexical ausbau of an exclusively oral vernacular (\sectref{sec:4.2.1}) and their syntactic ausbau (\sectref{sec:4.2.2}). Both of these aspects of ausbau are primarily concerned with augmenting the systems of coherence~-- semantic and grammatical, respectively~-- that characterize distance languages.

\subsection{Lexical ausbau and the cultivation of semantic coherence}\label{sec:4.2.1}

  Semantic coherence has to do with how a speaker or writer organizes their linguistic production so that it makes sense to the interlocutor or reader. The ausbau of this aspect of an exclusively oral vernacular involves turning discourse organization into text organization. In order to elaborate on this point, consider how speakers in immediacy contexts organize their utterances.

\ea%1
    \label{ex:4:1}
\ea Modal particles\\
\gll    Das      hab’    ich   \textbf{doch}   schon  probiert\\
    That  have   I   \textsc{partl}  already  tried\\
\glt     ‘But I’ve already tried that’

\ex Discourse adverbials\\
      \ea \textbf{Anyway}, I left the party without saying “hi.” (signals the end of a discourse topic)
      \ex \textbf{Right}, let’s see what we can do. (signals the beginning of a new one)
      \z
    \z
\z

\noindent (\ref{ex:4:1}a) shows that speakers use modal particles, for example, to indicate their subjective orientation toward a topic. This organizational strategy corresponds to the fact that linguistic utterances produced in immediacy contexts reflect the speaker’s subjectivity rather than a distanced objectivity. In this example, note how the modal particle, \textit{doch}, expresses the wider context in which such a statement must have been uttered and the speaker’s personal orientation toward that context. That is, the speaker reacts here to some real or perceived disagreement, or perhaps some misguided assumption about what they have, and have not ‘already tried' (\textit{schon probiert}). Speakers also use discourse adverbials to organize their utterances and indicate, for example, when they intend to begin or end a conversation, or when they want to continue discussing a topic or start a new one (\ref{ex:4:1}b). Additional “organizational signals” (\textit{Gliederungssignale}) rely on metalinguistic and prosodic cues, for example, turn-taking signals, hesitation phenomena, and repairs.

However, these conceptually oral means of organizing linguistic production are not functional in, or specific enough for, contexts of extreme distance. Turn-taking signals, for example, do not transfer into a graphic medium and are only functional for organizing dialogue, while extreme distance language is monologic. Modal particles, so useful for expressing the subjectivity of the speaker in immediacy contexts, become superfluous when they are extracted from discourse and used in more objective distance language contexts. Discourse adverbials, on the other hand, can still be functional in more distance-shaped written language, but the writer must opt for adverbs expressing more specific meanings over vaguer, immediacy-shaped adverbs, like ‘so’ and ‘anyway.’ Thus, in order to enhance the semantic coherence of linguistic production in dislocated, “distance language” contexts, writers must develop an inventory of lexical items that can bind together the semantic content of linguistic production concretely, hierarchically, and logically. Such lexical items would need to make explicit how writers are organizing their language within a text, as well as express a wide range of logical relationships between propositions. For example, sets of words, like ‘first, second, … finally,' make clear the writer’s decisions about the order in which they decided to present their information. Other examples Koch and Oesterreicher mention include ‘on the one hand, … on the other hand,' which juxtaposes two ideas, and \textit{zwar}, \textit{aber} (‘admittedly, yet’), which express concessive or contrastive semantics, respectively. A written ausbau language will want numerous and varied lexical items that can organize text along these lines.

I would like to digress from this section’s main analysis briefly to make explicit certain assumptions that are inherent to \citegen{KochOesterreicher1994} presentation but remain unacknowledged in their work. Namely, the authors turn an observation about the synchronic differences between conceptually oral, immediacy utterances, on the one hand, and conceptually written, distance utterances, on the other hand, into a hypothesis about how exactly the structures of exclusively oral vernaculars are deficient in distance contexts and how exactly ausbau tries to address these deficiencies. So, if a structure is characteristic of modern spoken language, as modal particles and discourse adverbs are, then it also must be characteristic of the varieties of exclusively oral vernaculars. Along these same lines, if modern written varieties require systems of \textit{text} organization because the \textit{discourse} organization of immediacy language is dysfunctional in distance contexts, then one could expect intensive ausbau to create systems of organization based around text and not discourse.

To a certain extent, I think Koch and Oesterreicher’s assumption can be useful in that it provides a concrete framework for conceptualizing the ausbau of oral vernaculars into written \textit{scripti}. In accepting it, one does run the risk of establishing a false equivalence though. That is, it would be wrong to conclude that the immediacy (spoken) varieties of languages like Modern English or German are structurally the same as oral vernaculars. Recall that exclusively oral vernaculars have both distance and immediacy varieties (see \sectref{sec:3:3.2}). The latter set of varieties can be expected to be similar to the spontaneously spoken varieties of ausbau languages, like English and German. Exclusively oral vernaculars of distance, in contrast, will be distinct from the spoken immediacy varieties of ausbau languages. Though both types of language are spoken, the former must function in ways that are similar to written languages of distance~-- it is the planned, public form of an oral vernacular~-- without the benefit of writing itself. That is, it must also be memorable language. Yet, one must also remember that oral and written varieties of distance are themselves distinct from one another. As I argued in \sectref{sec:3:3.2}, there is a “functional gap” between the most distance-shaped variety of an exclusively oral vernacular and distance-shaped written language, which allows for a complete disconnection between the language producer, the text they produce, and the person who ultimately reads it. In sum, comparing systems of linguistic organization in modern distance varieties to those of modern immediacy varieties can still be a good place to look for indications of how language ausbau turns exclusively oral vernaculars into \textit{scripti}. But one must also be clear on the important differences that exist between exclusively oral varieties of immediacy and distance and the varieties that exist in a literate culture.

Returning to the topic of lexical and semantic aspects of language, \citet[591]{KochOesterreicher1994} provide additional details on what intensive ausbau entails. Producing language in a context that is at the far pole of immediacy language presents the speaker with certain advantages and challenges. With respect to the former, the speaker can make use of the physical space they share with the interlocutor in order to make meaning. Language of immediacy also occurs between intimates; thus, speaker and interlocutor know each other and share a set of experiences, which facilitates communication. The challenge of producing language in immediacy contexts arises from the fact that it is spontaneous, and one does not have~-- or does not take~-- the time to plan an utterance. Fortunately, in such informal contexts, one can get away with vague referents, such as, “stuff” or “thing,” what Koch and Oesterreicher call “passe-partout”, or “master key,” words. Other strategies, like redundancy, or the repetition of a word or phrase, alleviates the cognitive burden for both speaker and listener: the speaker need not worry about varying lexical items simply for variety’s sake, and the listener benefits from hearing a word or phrase more than once.

A new language of distance that can function in written contexts, in contrast, requires a vastly expanded, differentiated, and more precise vocabulary. Without one, people would not be able to capture in dislocated language “the totality of their social reality and the full range of all bodies of knowledge connected to their world” \citep[591]{KochOesterreicher1994}. Thus, lexical ausbau and its attendant semantic coherence has the main goal of elaborating the lexicon. Writers engaged in the project of ausbau must increase the vernacular’s vocabulary, establish consistency in nomenclature and distinguish synonyms. They can create new words through derivational morphology and by borrowing words, morphemes, or concepts from other languages with which they are in contact. In developing a \textit{scriptus}, its creators are obliged to expand the lexicon in ways that correspond to their particular project. In the case of early medieval German, for example, writers needed German words for the many items and concepts that did not originally exist in vernacular culture but were part of a Latin Christianity (\tabref{extab:4:2}).

\begin{table}
  \caption{Latin loans into early German}
    \label{extab:4:2}
\begin{tabularx}{\textwidth}{llQl}
\lsptoprule
                 & Latin &      Ninth-century German    & Modern English\\
\midrule
\multicolumn{3}{l}{Loan translations}\\
                 & \textit{trinitas} &     \textit{drînissa} ‘three-ness’     & ‘trinity’   \\
                 & \textit{propheta}  &    \textit{forasago} ‘before-sayer’ &   ‘prophet’ \\
\tablevspace
\multicolumn{3}{l}{Loan creation}\\
                 & \textit{incensum}     &  \textit{wîhrouh}      ‘holy smoke’    &  ‘incense’  \\
                 & \textit{apostulus}    &  \textit{zwelifboto}  ‘twelve-messenger’ & ‘apostles’\\
                 & \textit{misericordia} &  \textit{irbarmherzida}, \textit{miltida}, \textit{ginada}, \textit{eregrehte}, \textit{armherziu}, \textit{irbarmherzi}, \textit{irbarmida}  &  ‘loving kindness'\\
\lspbottomrule
\end{tabularx}
\end{table}

The examples in \tabref{extab:4:2} include both concrete and abstract terminology. Note the many translations early German literizers created for the multifaceted concept of \textit{misericordia}. Koch and Oesterreicher note that lexical ausbau is particularly concerned with the creation of abstract words, e.g. more \textit{Sachverhaltsabstrakta}, ‘abstract nouns describing states, circumstances’ and \textit{konsequentere Begriffshierarchien}, ‘more consistent hierarchies in terminology.’ This description brings to mind works like Walter \citegen{Ong2012} [1982] \textit{Orality and Literacy}, which maintains that literacy makes possible a level of abstract thought that is wholly inaccessible to speakers of exclusively oral cultures. According to this view, the ausbau of an abstract lexicon would be both the result, and a reflection, of the literizers having unlocked their full cognitive potential for more conceptual, abstract thinking. This notion has rightly been taken to task for its technological determinism (see \citealt{Best2020}).

\subsection{Syntactic ausbau and the cultivation of grammatical coherence}\label{sec:4.2.2}

Grammatical coherence, the second type of Koch and Oesterreicher’s textual and pragmatic intensive ausbau, refers to the ability to use grammatical means to express the many relationships that can exist between constituents. The ausbau of a variety’s grammatical coherence, then, involves creating, what \citet[590]{KochOesterreicher1994} call, a \textit{planungsintensive Textphorik}, that is, planning-intensive systems of coreferentiality between the linguistic elements within a text. An example of this sort of ausbau is the development of rules of agreement (\textit{Kongruenzregeln}). The authors do not elaborate on this point and, so, it may seem like, in identifying agreement as a product of ausbau, they are claiming that systems like subject-verb agreement are not present in exclusively oral vernaculars, and that they only exist as a result of conscious development. I do not think that this is Koch and Oesterreicher’s argument, as I hope to illustrate by way of a few examples.

First, consider left dislocation, a structure that is characteristic of modern spoken German but not Modern Standard German.\footnote{{See \citealt{Evans2023} Chapter 2, for a recent overview of the literature on left dislocation.}} Notably, certain types of left dislocated phrases need not be integrated into the predicate of the clause.

\ea%3
    \label{ex:4:3}(from \citealt[240]{MillerWeinert1998})\\
\gll ja  und \textbf{dies-en}   \textbf{Flusslauf}   de-m     folgen wir jetzt\\
yes  and this-\textsc{acc}  river.course   this.one-\textsc{dat.det}   follow we now\\
\glt ‘yes and this river course, we will follow it now’ \\

\z 

\noindent Consider how the left dislocated noun phrase, \textit{diesen Flusslauf}, is in the accusative case, while \textit{dem}, is in the dative case. In other words, the demonstrative pronoun is integrated into the clause as the object of the verb, \textit{folgen}, while \textit{diesen Flusslauf}, the noun phrase to which \textit{dem} refers, is not. In this way, the utterance in \REF{ex:4:3} is less grammatically coherent than one in which all of the constituents are morphosyntactically integrated into the clause either through agreement rules, in the case of verbs, or, to use the terminology of generative syntax, government relations, in the case of sentential constituents.

\citegen[590]{KochOesterreicher1994} example of clause linkage is another good one to consider more closely because it concerns explicit concatenation across clauses, rather than simply within clauses. While languages of immediacy will often contain paratactically or asyndetically concatenated clauses, the relationship between which is signaled with the help of, say, intonation or the addition of a particle (\citealt[23]{MillerWeinert1998}), written distance language will more visibly and explicitly link clauses through the use of a broader inventory of conjunctions that express differentiated and more precise relationships between constituents.

\ea%4
    \label{ex:4:4}(From \citet[103]{MillerWeinert1998})\\
\begin{itemize}
\item[A1]    it’s the same chap that takes us hillwalking on Sundays \textbf{and} we had one about a fortnight ago at Comrie \textbf{and} the weather was really bad \textbf{and} we were in a snow blizzard \textbf{and} we didn’t know how we were going to get out \textbf{and} we were petrified
\item[M10]    can you ever em did you ever feel when you were on that thing at any time that you were really going to get lost there
\item[A10]   yeah quite often when we were on the top the top of the hill there was just about a whiteout [-] you couldn’t see where the farm was [-] you were lucky if you saw a foot in front of you and I was convinced we were still going in the wrong direction
\end{itemize}
    \z

\noindent In \REF{ex:4:4}, the clause in A1 demonstrates how spontaneous spoken language will often rely on coordinating conjunctions, in this instance \textit{and}. Here the speaker does not shape their sequence of clauses into distinct clause complexes that would organize the narrative beyond a simple linearization. In A10, the speaker produces a sequence of clauses with no formal linkage at all, indicated by the [-]. Another example of how spontaneously spoken language tolerates the absence of an explicit formal linking of two clauses can be seen in the tendency for speakers to produce direct speech, rather than integrate the direct speech as a complement clause into the main clause.

\begin{table}
\caption{Complement clause embedding (from \citealt[54]{MillerWeinert1998})}
\label{extab:4:5}
\fittable{\begin{tabular}{lll}
\lsptoprule
& Direct speech            &  Complement clause\\
\midrule
a. & So I asked what are you doing                                    &  I asked what he was doing                          \\
b. & I said we’ll help.                                               &   I said that we would help                         \\
c. & Then she explained: the baby was      &   Then she explained (that) the baby was \\
   & ill and she had to stay home &  ill and (that) she had to stay home \\
\lspbottomrule
    \end{tabular} }
\end{table}

\noindent The direct speech constructions in \tabref{extab:4:5} are more characteristic of spontaneously spoken language. In written distance language, in contrast, the clause containing the direct speech is likely to be recast as a subordinate clause with an initial complementizer. In \tabref{extab:4:5}(a) and (b) there is also a readjustment of tense and mood, which constitutes yet another layer of grammatical encoding of the hierarchical relationship between main and subordinate clause.

\citet[58--71]{MillerWeinert1998} demonstrates that spontaneously spoken syntax can be more fragmented than in \REF{ex:4:4} and in the direct speech utterances of \tabref{extab:4:5}, so much so in fact that the line between \textit{identifying} underlying syntactic units, like the clause and the clause complex, and \textit{creating} them ex post facto becomes blurred. They discuss the work of \citet[20--34]{Sornicola1981}, who shows how spontaneously spoken Neapolitan Italian contains utterances with no evident syntagmatic relations, as in \REF{ex:4:6}.

\ea%6
    \label{ex:4:6}
    programmi che (pause) per i bambini (pause) [...] a l'indomani (pause) vedono (pause) guardono (pause) per la scuola \\
\gll        programmi   che  per   i   bambini     a l’indomani \\
  programs   that  for  the   children                     for tomorrow\\

\gll vedono  guardono   per la scuola\\
they-see     they-watch  for the school\\
\z

\noindent By filling in background information and drawing on general knowledge, Sornicola was able to produce a semantic interpretation of the utterance with coherent clauses that are logically ordered in a clause complex.

\ea%7
    \label{ex:4:7}
\gll   programmi  che   [i bambini]    vedono   [perché  sono        loro   utili]\\
      programs   that   [the children]  see     [because  they.are   to.them   useful]\\

\gll per  la scuola   il giorno  dopo\\
for   the school   the day   after\\
\z

\noindent For a full explanation of the inferences and reorderings that were required to turn the fragmented syntax of \REF{ex:4:6}, in which it is difficult to pick out individual clauses, much less a clause complex, into the utterance in \REF{ex:4:7}, see Miller and Weinert’s pages 58--59. The main point here is that the fragmented spontaneous utterance, taken on its own terms, is made up of “blocks of syntax,” to use Miller and Weinert’s term, with no evident clauses or clause complexes. As Sornicola observes on pages 22--23, the utterance evinces an “extreme fragmentary nature.” Each constituent seems to stand on its own with pauses demarcating the boundary between them. The more coherently organized syntactic units of \REF{ex:4:7} can only be created, rather than found, if one ignores the constituent order and interpolations of the original utterance and fills in, as it were, all of the semantic and syntactic blanks. I develop this point of the creation of ideal structures in spoken clauses and clause complexes in \chapref{sec:chap:7}.

To summarize what I have discussed thus far, spontaneously spoken, or immediacy language is more fragmented than distance language in the sense that it features less explicit concatenation between constituents, both within, and across, clauses. As I explained in \sectref{sec:4.2.1}, Koch and Oesterreicher assume that the spoken varieties of existing ausbau languages, like Modern English and German, have characteristics in common with the immediacy \textit{and} distance varieties of exclusively oral vernaculars. However, it is important to remember that, though the planned, formal varieties of an exclusively oral vernacular will feature more integrated and lexically dense structures than its immediacy varieties, it will never feature the same degree of grammatical coherence and explicit concatenation that written language can and will. I base this argument on two observations. The first observation, I have already discussed in this section, as well as \sectref{sec:3:3.2}. That is, literization itself effects a new context of extreme distance that is entirely dislocated from the moment of language production and the person producing it. The first literizers of an exclusively oral vernacular must, therefore, creatively shape their language so that it can function in ways that were theretofore unnecessary and impossible. I discuss the second observation in \chapref{sec:chap:6}, where I note that exclusively oral varieties are all subject to memory constraints. I argue that such constraints ensure that exclusively oral distance varieties will effect more integrated and lexically dense utterances through memory-supporting means, while syntactic ausbau, as a written elaboration of language, is unfettered by any mnemonic concerns.

\citet[591]{KochOesterreicher1994} identifies several of these strategies of syntactic ausbau that enhance the grammatical coherence of a \textit{scriptus}, including the following items.

\begin{enumerate}
\item  The differentiation of prepositions and hypotactic conjunctions
\item  The regularization in the use of tense and mood
\item  An increase in the use of subordination and hypotaxis
\item  The development of nominalizations
\end{enumerate}

\noindent This list requires clarification and amendment. First, I propose that regularization processes, like that of the use of tense and mood, not be counted as part of syntactic ausbau. Both \citet{Kloss1978} and \citet{KochOesterreicher1994} define ausbau as the creative shaping of language that results from writers pushing their exclusively oral vernaculars into new, written distance contexts. Regularization of form and structures is certainly a concern of literization. I believe it also contributes to a written variety’s grammatical coherence. However, I do not think it is an ausbau process in that it involves a systematic selection and deselection of linguistic forms, rather than a creative shaping of language so that it is functional in the new distance contexts of writing.

The remaining strategies listed above, I argue, are appropriately characterized as part of syntactic ausbau, but must also be clarified and amended. Beginning with the differentiation of prepositions and hypotactic conjunctions, I would like to broaden the wording of this strategy in line with \citegen[139--140]{Höder2010} discussion of the syntactic ausbau of Old Swedish. Höder’s argument is similar to \citegen{KochOesterreicher1994} in that he emphasizes how spoken and written language organize information differently. The former relies more on implicit connections between utterances that are coherent with respect to the discourse in which they occur. The latter, on the other hand, requires the explicit linking of constituents and clauses that elucidate the semantic and logical relations between these elements. As a result, Höder concludes that syntactic ausbau must involve the creation of an inventory of more, and more differentiated, subordinators that can enhance the grammatical cohesion of a written text. One example of this sort of development is the creation of polymorphemic, monosemous subordinators in Old Swedish, such as \textit{så att} ‘so that’ with consecutive semantics, and \textit{eftersom} ‘because,’ which takes on a causal meaning.

In contrast, more orally structured varieties will have a smaller inventory of subordinators that includes more monomorphemic, polysemous subordinators. The connector \textit{that} or \textit{thaz} in the early German \textit{scripti} is a good example of a polysemous subordinator. In the \textit{Hêliand}, for instance, it is associated with a number of different meanings \REF{ex:4:8}.

\ea%8
    \label{ex:4:8}
\ea
\gll he  gisah   thar    aftar thiu  enna   engil  godes\\
  he   saw    there   after this   an     angel   God.\textsc{gen}\\

\gll an  them   uuihe  innan  the     sprac   im       mid    is   uuordun  to\\
in   the     temple   inside  \textsc{partl}   spoke   him.\textsc{dat}   with   his  words          too\\

\gll het       \textbf{that}   frod    gumo  forht   ni     uuari,\\
commanded  that   venerable   man   afeared   \textsc{neg}   become.\textsc{pret.subj}\\

\glt ‘He saw there then an angel of God  inside the temple, who spoke to him with these words too; he commanded that the venerable man should not be fearful’ (2, 113a-15b)

\ex
\gll thin    thionost  is  im       an thanke  \textbf{that}    thu  sulica   githaht  haues\\
  your  service   is   him.\textsc{dat}  a favor     because   you  such  faith   have\\

\gll an  is   enes  craft\\
in  his  one     power\\
\glt ‘Your service is a favor to Him, because (in that) you have such faith in His one power (2, 118b-119a)

\ex
\gll Tho    uuard   thar    gisamnod  filu     thar    te  Hierusalem\\
  then  were  there   gathered   many   there   in   Jerusalem\\

\gll Iudeono     liudio       uuerodes     te   them  uuiha\\
the.Jews.\textsc{gen.pl}  people.\textsc{gen.pl}  crowd.\textsc{gen.sg}  to  the    temple\\

\gll thar     sie     uualdand  god    suuido  theolico   thiggean  scoldun\\
there/where  they    ruling     God    very   humbly   beg     should\\

\gll herron   is  huldi  \textbf{that}    sie    heuancuning   ledes     aleti\\
master   his   grace  so.that   them  Heaven’s.King  evil.\textsc{gen.sg}  deliver.\textsc{pret.subj}\\

\glt ‘Then were gathered there in Jerusalem, many of the Jewish people of the crowd at the temple, there/where they should very humbly beg the Ruler God, the Master, for his grace, so that Heaven's King might deliver them from evil’ (2, 96b--101a)

\ex
\gll Bethiu    ne   andradad  gi   iu       thero    manno    nid\\
therefore  \textsc{neg}   fear.\textsc{imp}    you   you.\textsc{refl}  the.\textsc{gen.pl}   men.\textsc{gen.pl}  hate\\

\gll ne    forhteat   iro      fiundskepi\\
\textsc{neg}   fear.\textsc{imp}    their  enmity\\

\gll thoh   sie     hebbean      uuuas ferahes   geuuald\\
though  they    have    your life.\textsc{gen.sg}  power\\

\gll \textbf{that}    sie     mugin  thene  lichamon  libu     beneotan\\
that    they    may    your  body    life.\textsc{instr}  rob\\
\glt ‘Therefore do not fear the men who hate you, nor fear their enmity; though they have power over your living; (with the effect) that they may rob your body of life’ (22, 1903b-05b)
    \z
\z

\noindent In (\ref{ex:4:8}a), \textit{that} functions as a complementizer that marks its clause as an object of the main clause’s predicate, \textit{het}, ‘commanded.’ In (b), \textit{that} has causal semantics. In the last two examples in \REF{ex:4:8}, \textit{that} occurs at the beginning of adverbial clauses that modify the predicate of the main clause. In (c), \textit{that} is a final conjunction, indicating the purpose of an action, while in d. it indicates the result of an action, that is, is a consecutive conjunction.

The conjunction \textit{that} in the \textit{Hêliand} raises another point that is relevant to this discussion. Namely, if a language relies more on monomorphemic, polysemous subordinators, the modern reader may well associate them with more than one syntactic category, not simply multiple meanings. The examples in \REF{ex:4:8} already show how \textit{that} functions in ways that modern linguists would say are syntactically distinct, that is, as a complementizer and as a adverbial subordinator. The former embeds one clause in another; the other merely modifies. Yet, one morpheme connects to both functions. Ausbau languages, like Modern English, have added a morpheme to distinguish the adverbial uses of \textit{that} from the complementizer: ‘so that,’ as the final conjunction, ‘such that,’ as the consecutive conjunction, and ‘in that,’ as the causal conjunction. Contributing to the multi-categorial, polysemous nature of \textit{that} in early German is the fact that the morpheme can also function as a relative pronoun (\ref{ex:4:9}a), demonstrative pronoun (\ref{ex:4:9}b), and determiner (\ref{ex:4:9}c). Consider examples of these usages from the \textit{Hêliand}.

\ea%9
    \label{ex:4:9}
\ea
\gll so  uuard  ok   \textbf{that}  \textbf{fiur}   kuman  het   fan      himile\\
  so  was    also   \textsc{det}     fire    come  hot  from  heaven\\

\gll \textbf{that}   thea   hohon   burgi     umbi  Sodomo    land\\
that   the    high  mountains  around  Sodom    land\\

\gll suart  logna   bifeng     grim   endi    gradag\\
black   flames  encircled  grim  and    greedy\\

\glt ‘So was the fire also come hot from heaven that encircled the high mountains around Sodom-land with black flames grim and greedy’ (52, 4366b-69a)

\ex
\gll Ne   uuilleat     feho     uuinnan   erlos     an      unreht\\
\textsc{neg}   wish.\textsc{imp}  property    gain     earls.\textsc{voc}  through   injustice\\

\gll ac   uuirkead   up         te gode   man     aftar      medu\\
but  work.\textsc{imp}  in.the.direction  to God  man.\textsc{nom}  according.to   reward\\

\gll \textbf{that}  is   mera  thing  than  man     hir     an  erdu  odac\\
that   is  greater   thing  than  man.\textsc{nom}  here    on  earth  rich.\textsc{nom.sg.m}\\

\gll libbea       uueroldscattes  geuuono\\
live.\textsc{pres.subj}  treasure.\textsc{gen.sg}  accustomed.to\\

\glt ‘Do not wish to gain property, earls, through injustice, but work toward God, man according to reward; that is a greater thing than when man here on earth may live, rich (ones), accustomed to treasure’ (19, 1637b-41a)

\ex
\gll Iohannes  tho     gimahalde   endi    tegegnes   sprac   them   bodun\\
  John      then  answered  and    in.reply   spoke  \textsc{det}     messengers.\textsc{dat.pl}\\

\gll baldlico  ni   bium   ic   quad  he   \textbf{that}   barn   godes     uuar   uualdand Krist\\
boldly   \textsc{neg}   am    I  said   he  \textsc{det}   child  God.\textsc{gen}   true  ruling    Christ\\

\glt ‘John answered and spoke boldy in reply to the messengers. “I am not", he said, “the Child of God the True Ruler Christ”’ (11, 914a-16a)
\z
\z

\noindent Speakers of Modern German are not particularly troubled by the fact that the sequence [dɑs] is associated with multiple meanings and syntactic functions. Homophony of this sort is generally functional in oral contexts. However, when multifunctional, polysemous morphemes are transferred into a written variety, the modern researcher might have difficulty attaching them to one meaning and one function. That is, they can seem ambiguous.

There are two conclusions that may be drawn from these examples. First, the differentiation of [dɑs] and other polysemous morphemes like it only becomes necessary in the new distance contexts that literization itself creates. This new dislocated context of the page requires a similarly new specificity in the matching of lexical items to grammatical function. One-size-fits-all style subordinators become less functional in the written language. The other conclusion both foreshadows \chapref{sec:chap:7}’s discussion and points back to the point I made in this book’s introduction. Recall that in \chapref{sec:chap:1}, I argued that the literization process itself pushes people into the development of linguistic norms. The example I used there had to do with the indices compiled by the early philologists who published the first grammars of and readers for historical German. In order to create a functional index, they had to settle on single lemmas that would represent all of the variable instantiations of a lexical phenomenon. The decision to choose lemmas that reflected the modern standard was certainly a practical one, rather than reflective of some mistaken belief that there was a standard form of early German. However, I also argued that two unavoidable consequences of the compiling of these important reference works were the creation of both an actual norm and the illusion of one. The actual norm is the word list contained in an index or a glossary, which provides a necessary lexical starting point for the grammar or the reader. The illusion of a norm emerges when modern learners of these historical varieties assume that the lemmas of the index represent a linguistic norm for an early German language, just as they would for Modern German. Given that the early German lemmas represent a similar regional variety to the one that became the basis of the standard, this possibility is far from outlandish.

I propose that a similar conflation can occur between the methods and practices that are necessarily part of our literate engagement with these historical varieties and the varieties themselves. In order to clarify the meaning behind this statement, consider the comprehensive glossaries that different philologists compiled for significant early German texts. Two of the best examples are Paul \citegen{Piper1884} glossary for Otfrid’s \textit{Evangelienbuch} and Edward \citegen{Sehrt1925}, for the \textit{Hêliand}.\footnote{{That both men aimed for completeness in their glossaries can be gleaned from the titles they gave them. Piper called his an }\textrm{\textit{Ausführliches}}\textrm{ (`detailed')} \textrm{\textit{Glossar}}\textrm{, while Sehrt called his a }\textrm{\textit{vollständiges Wörterbuch zum Hêliand}}\textrm{ (`a complete dictionary of the} \textrm{\textit{Hêliand}}\textrm{').}} These works are stunning achievements in that their compilers attached each individually occurring token in the text to a full grammatical identification and modern German translation. For some lemmas, this exercise was relatively straightforward. For example, the Modern German word \textit{der Kelch}, ‘chalice, cup, or goblet,’ occurs once in Otfrid, once in the \textit{Hêliand}. The glossaries provide the citation for the form, its basic grammatical information, that is, that it is a masculine (non-\textit{n}{}-stem) noun, and that the word in its context is a singular accusative noun.

\begin{sloppypar}
Other lemmas, in contrast, are significantly more complex. For example, Sehrt’s entry for the lemma \textit{sô} (pages 481--488) not only lists citations for all occurrences, it also categorizes each token with respect to Sehrt’s own assessment of how that token functioned in its context. Sehrt identifies seven larger grammatical categories for \textit{sô}, within which there are even smaller sub-categories of different functions and meanings. \textit{Sô} can be an adverbial or a relative particle; it can have a causal meaning like ‘because,’ a temporal meaning, like ‘when,’ or a concessive meaning, like ‘although.’ It can introduce a contrary to fact conditional sentence or a correlative sentence. This process of assessment is like the one I presented in \REF{ex:4:8} and \REF{ex:4:9} just above, where I noted how the sequence [dɑs] performs a number of different grammatical function. Sehrt’s entries for \textit{that} (and Piper’s for \textit{thaz}) are similarly elaborate in the number of categories and sub-categories the compiler had to add to their glossary in order to describe each individual attestation of that lemma.
\end{sloppypar}

Such detailed entries are helpful to the modern reader, who may not intuit that one lemma can exhibit this multiplicity of meanings and functions. If this modern reader is trained in historical linguistics, they may also conclude that the published glossary is a description of a historical mental grammar and that lemmas like \textit{sô} and \textit{that} contain a whole host of variable sub-categories that later become associated with their own distinct lemmas.  The notion of ausbau, however, raises the possibility that it is literization itself that creates these different categories in the first place. In other words, it is literization that effects the written context that, in turn, requires a new level of grammatical and lexical explicitness and the means of achieving it. To look for the same inventory of explicitly defined syntactic categories in the first attestations of an oral vernacular, for which ausbau has only just begun, could lead to anachronism. Indeed, one could see Piper's and Sehrt’s respective glossaries as inherently anachronistic. I do not, however. I see them as important, not just for their comprehensive descriptions of each lemma. They are also important in that they illustrate another point I made in my introduction that is relevant to literization and ausbau; namely, that our metalanguage~-- the way we think and write about language~-- is shaped, even effected, by literization.

I now turn to \citegen[591]{KochOesterreicher1994} third aspect of syntactic ausbau, which aims to enhance grammatical coherence through increasing the use of subordination and hypotaxis. This strategy links to the one discussed just above in that the project of increasing hypotaxis is facilitated by a concomitant differentiation and expansion of the inventory of subordinators. This strategy also connects to this section’s introductory observations on the more fragmented nature of spoken and more conceptually oral language. The creation of an ausbau language, then, requires writers, not just to build more grammatically coherent clauses, but also more grammatically coherent links between these clauses that clarify and make visible the hierarchical relations between them.

This argument is consistent with accounts of syntactic change advanced in the Indo-Europeanist tradition of the last one hundred and fifty years (see \citealt[25--27]{HarrisCampbell1995}). While not all Indo-Europeanists have agreed on whether Proto-Indo-European had complex or merely simple sentences, they all accept that it had less integrated syntax and more parataxis and asyndetic linking than its attested daughter languages. For example, \citet[8]{Brugmann1925} argued that simple sentences came first. Complex sentences developed not long thereafter and first comprised sequences of sentences that were more closely related to each other. Coordinating and subordinating relations developed eventually, then explicit marking of these relations. \citet[59--60]{Sweet1900} also describes an earlier state of language in which utterances are less grammatically coherent but become more so over time.

\begin{quote}
In primitive language permanent attribute-words [later adjectives] were naturally put in juxtaposition with the substance-words [later nouns] they qualified. Many languages then found it natural and convenient to bring out more clearly the connection between head-word and adjunct-word by repeating the form-words or inflections of the former before or after the latter as well, the result being grammatical concord. Thus, in \textit{I bought these books at Mr. Smith’s, the bookseller’s}, the repetition of the genitive ending serves to show more clearly that \textit{bookseller} is an adjunct to~-- stands in apposition to~-- \textit{Mr. Smith’s}.
\end{quote}

\noindent These accounts mirror to some extent the ausbau narrative I have presented in this chapter: exclusively oral vernaculars feature blocks of syntax that are loosely connected to one another. Syntactic ausbau augments the grammatical coherence of linguistic production through the identification and marking of a wide range of syntactic relations that may exist between constituents. The main point of distinction lies in the source of such changes. \citet{Sweet1900} and \citet[205]{Windisch1869}, for example, invoke the notion of evolutionary advancement by declaring that the overt marking of syntactic relations is a sign of sophistication, its absence, a sign of primitiveness. My argument, in contrast, is that it is the literization process and its attendant ausbau that effect these developments, not because writing is inherently more sophisticated, but because writing demands an explicitness that is neither required for spoken language, nor achievable for exclusively oral vernaculars whose production is cognitively constrained in ways that that of the written language is not.

\citet[282--284; 308--310]{HarrisCampbell1995} is critical of the traditional view that parataxis is more common than hypotaxis in earlier stages of a language’s development and, thus, could be seen as a challenge to the ausbau narrative of \citet{KochOesterreicher1994}. Harris and Campbell offer two critiques that are relevant to the current study’s arguments. First, they state that it is wrong to associate sophistication with hypotaxis and primitiveness with parataxis. Indeed, these are qualitative judgments that reflect a western European bias against exclusively oral cultures, which tend to be non-western and non-White. The literization narrative, as I stated just above, provides a straightforward explanation for why syntactically integrated language increases once people start writing in their vernacular.

More potentially damaging to the literization narrative is the authors’ implication on page 308 that it is a “common belief,” rather than an accepted fact that writing has more hypotaxis than spoken language does. Harris and Campbell themselves present no argument or data contradicting this conclusion and indeed cite works like \citet{Chafe1982} that demonstrate that, for example, written language contains more finite subordinate clauses than spoken language. To this work one could add \citet{MillerWeinert1998}, a monograph full of evidence that the integrated hypotaxis of written language will not be heard in spoken language, especially in its most spontaneous or informal varieties. \citet{HarrisCampbell1995} notes that exclusively oral vernaculars do contain other types of hypotactic integration, like non-finite clauses. This observation is undoubtedly true and contradicts only the most extreme narratives in which exclusively oral languages are meant to have no integrated utterances at all. These narratives do not recognize the fact that the contexts of distance and immediacy are just as relevant to exclusively oral languages as they are to written ausbau languages (see \chapref{sec:chap:3}) and that the distance varieties of exclusively oral vernaculars will feature more integrated and lexically dense utterances than its immediacy varieties. Integration, however, is achieved by different means than may be employed in an ausbau language because, first of all, language producers always share the physical space of their interlocutors (see \sectref{sec:3:3.2}) and, second, they are subject to the cognitive constraint of memorability, a point that I discuss in detail in \chapref{sec:chap:6}.

The final strategy that \citet[591]{KochOesterreicher1994} identifies as contributing to a vernacular’s syntactic ausbau is the development of nominalizations. This aspect of ausbau is more about creating integrated language than it is about creating grammatically coherent language. Intense nominalization is, in fact, a hallmark of written distance language, especially academic and technical varieties. Consider this sentence from \citet[137]{Chafe1981}: “One tendency is the preference of speakers for referring to entities by using words of an intermediate degree of abstractness.” Note how the sentence contains only one finite verb, the copular \textit{to be}, while all other verbs are in nominalized, non-finite forms, that is, “referring” and “using.” Also note the presence of derived nouns, such as “abstractness.” I return to this topic in earnest in \chapref{sec:chap:6}, where I argue that the speakers of exclusively oral vernaculars can make better use of nominalization as an integration strategy than, say, clausal hypotaxis because the former strategy can draw on mnemonic devices more easily.

In sum, I have identified three main aspects of syntactic ausbau that are relevant to the current study: 1. the creation of an inventory of more, and more-differentiated, subordinators; 2. an increase in the use of subordination and hypotaxis; and 3. the development of nominalizations. These changes have two main goals. The first is to create more integrated, which is to say, lexically dense, language; the second is to make linguistic production more grammatically coherent. With respect to the first goal, I clarified that some strategies of syntactic ausbau are not wholly absent from the planned, oral varieties of exclusively oral vernaculars, which are also subject to some of the same constraints of distance. So, for example, nominalizations are employed in elaborated orality as a means of effecting lexical density. To reiterate, my argument is not that ausbau creates hypotaxis or that hypotaxis develops out of parataxis. Rather, integration is present in spoken and written forms of language, but the organization of clauses into explicitly marked and highly elaborated sequences of (finite) clauses develops primarily from literization and ausbau. I also argued in this section that exclusively oral means of organizing language for distance contexts are different from written means because, among other reasons, memorability constrains the former. With respect to the second goal of syntactic ausbau, which is to enhance the grammatical coherence of linguistic production, I discussed studies that show how fragmented spoken utterances can be in that they can feature blocks of syntax that may not be fully integrated into a clause or into clause complexes. Thus, syntactic ausbau also involves the development of more explicit means of concatenation between constituents in order to create more grammatically coherent clauses and clause complexes.

In this section, I established a definition of language ausbau, a term that originates from the work of Kloss and is extended to the first stages of literization by \citet{KochOesterreicher1994}. As a diachronic process, ausbau involves the innovative shaping and reshaping of a linguistic variety so that it can meet all of the communicative demands of distance that a society may make. Language ausbau begins when people first write in, what was until that point, their exclusively oral vernacular. The corpus of early medieval German, then, is the textual evidence of this first stage of German’s ausbau. Each text is one synchronic snapshot of the beginning of this long diachronic process. \citegen{KochOesterreicher1994} elaboration of the lexical and syntactic aspects of Kloss’s ausbau helps the researcher conceptualize the types of adjustments a writer must make to their vernacular so that it can become a functional written language.

\section{Conclusion}\label{sec:4.3}

In this chapter, I began to make more concrete what an innovative literization looks like by discussing the universal characteristics of language ausbau. Ausbau comprises the structural modifications that writers undertake as part of that language’s literization. In particular, it refers to those changes that enhance the lexical and grammatical means by which writers can produce the more explicit and coherent language required in the graphic medium. Lexical ausbau (\sectref{sec:4.2.1}) is concerned with enhancing the vernacular’s semantic coherence through the creation of a larger, more differentiated, and precise vocabulary. It involves developing, for example, a more abstract vocabulary that can describe a whole range of states and circumstances and more consistent and coherent hierarchies in terminology that are capable of, among other things, organizing language within the visual space of a text. Syntactic ausbau (\sectref{sec:4.2.2}) looks to increase the grammatical coherence of the vernacular by creating explicit systems of coreferentiality between the various elements in a string of language. Examples of such changes might include the stricter implementation of agreement rules, like subject-verb agreement or integrating all sentential constituents within a clause as part of either the subject or the predicate. Elements can also be integrated more coherently across clausal boundaries through the development of a larger and more differentiated inventory of conjunctions. The idea of “transparent linking” connects both lexical and syntactic ausbau processes. That is, both types of ausbau entail the conscious development of explicit, visible systems of semantic and grammatical concatenation.

Two specific types of change that \citet{KochOesterreicher1994} identify as language ausbau are worthy of special mention in this conclusion: the increase in the use of subordination and hypotaxis and the development of nominalizations. Both require a more subtle analysis in that one should not associate these processes \textit{only} with ausbau, specifically, and literization, generally. Oral vernaculars also make use of hypotactic integration (\citealt[282--284; 308--310]{HarrisCampbell1995}) and nominalizations \citep{Chafe1981}, namely in their planned, public varieties of distance.\footnote{{I discuss \citegen{Chafe1981} data and arguments in detail in \chapref{sec:chap:6}.}} In other words, one should expect that written and oral distance varieties will share structural features because they are subject to a similar set of communicative constraints and conditions, as \citegen{KochOesterreicher1985} framework explains. However, the conditions that shape written languages of distance are not \textit{identical} to the ones that shape the elaborated orality of a mostly or exclusively oral culture, the form of which must support its memorability and people’s ability to plan important and/or public language in advance with no writing to support its performance. The production and reception of written language, in contrast, have no such constraints and can tolerate a level of elaboration that would render an oral vernacular dysfunctional. Combine these facts with the argument that a functional written language requires new levels of explicit semantic and grammatical concatenation, and it becomes clear that the elaborated structures in written ausbau languages, on the one hand, and varieties of distance of mostly/exclusively oral vernaculars, on the other hand, must be structurally distinct.

This chapter, then, constitutes an important step in imagining a new methodology for the analysis of the first German \textit{scripti}. Its discussion of the universal properties of ausbau could be applied to other early literizations, though it is a task for others to decide whether or the extent to which an alternative methodology is useful for understanding other language histories. As I indicated in the previous paragraph and will make clearer in subsequent chapters, it is also important to consider the multilectal contexts in which people undertake these literizations. Understanding these environments and how they interact with language is, I submit, one of the central challenges of working toward an accurate understanding of early literization. This statement must be especially true for literate people living in the hyper-literate world of twenty-first century academia, an environment that is far removed from ninth-century Carolingian Europe. As \citet{KochOesterreicher1994} indicate, one may use spoken language patterns to imagine which aspects of the lexicon and of syntax literizers must elaborate to bridge the functional gap between the oral vernacular as a multilectal phenomenon and a written language of distance. However, one must also bear in mind that planned varieties of an oral vernacular will also be marked by the more integrated and lexically dense structures generally associated with literization and ausbau. In \chapref{sec:chap:6}, I demonstrate how speakers accomplish this within exclusively oral environments.

  In this narrative, several sources of variation across the different early German \textit{scripti} emerge. Though literization and ausbau, in particular, aim to establish certain universal structural characteristics, for example, the augmentation of a vernacular’s means of creating integrated and lexically dense language, they are still conscious processes undertaken by individuals, each of whom has a whole host of linguistic resources they may draw on when constructing their \textit{scriptus}. These linguistic resources include their own multilectal vernacular. The early medieval Carolingian literizer, furthermore, creates an \textit{ad hoc} \textit{scriptus} for the purposes of their own writing project. Their immediate set of concerns encourages the literizer to engage their linguistic resources in one particular way that will likely vary from the ways in which other literizers, working in different monasteries on different projects, will engage theirs. This relationship between project and \textit{scriptus} provides yet another important nexus for understanding the early German \textit{scripti}. This topic is the main focus of \chapref{sec:chap:5}.

This depiction of early medieval German as a set of variable and consciously constructed \textit{scripti} runs counter to the way that diachronic linguists have studied German’s earliest attested forms. Reflecting a traditional, structuralist orientation toward historical linguistics, their focus has been on describing and accounting for the authentic grammar that underlay attested structures. This authentic German grammar is assumed to be the so-called natural prose that characterized early Germans’ everyday spoken language. It is also, as one of this book’s anonymous reviewers phrased it, the “heart of the matter,” that is, understood to be the self-evident target of diachronic linguistic analysis. In \chapref{sec:chap:2} and \chapref{sec:chap:3}, my goal was to point out the problems with this approach and to propose that scholars might instead view the \textit{scripti} more holistically and as a sociocultural and linguistic artifact. For example, we might consider the individual circumstances of a \textit{scriptus}’s creation. In order to create a reasonably well functioning \textit{scriptus}, the literizer cannot simply transpose their multilectal oral vernacular into the graphic medium. In other words, though existing spoken competencies, which are shaped by and perfectly suited to their exclusively oral contexts, certainly feed into the first \textit{scripti}, they will always fall short to some extent of the demands of the written word, which is a new context that can disconnect language production from producer and reader. Thus, the literizer must also innovate linguistically to create new explicit systems of lexical and grammatical coherence, a process I call ausbau. Understanding more about how orally organized language is less coherent, how conceptually literate language is more so, and how ausbau can bridge the gap from the former to the latter is essential to understanding literization processes in general. This relationship provides an essential context for analyzing historical varieties, especially those created before nationally directed standardization movements take hold. It also constitutes a means of analyzing variation within and across \textit{scripti}. Because literization and ausbau are conscious processes undertaken for a particular goal, that is, to produce a particular type of text, different literizers will activate their multilectal and multilingual linguistic resources differently and devise varying solutions to solve the problem of the functional gap.

