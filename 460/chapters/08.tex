\chapter{Moving forward}\label{sec:chap:8}

In this book I have presented a narrative of the earliest attestations of German that is significantly different from how we scholars have treated them so far. I argued that our overwhelming interest in uncovering or reconstructing the structure of an early German competence has caused us to neglect several significant factors that shaped the extant \textit{scripti}. First, we have not considered the possibility that the social and cultural context of Carolingian Europe was sociolinguistically relevant and influenced the German transmitted to us on parchment. Second, we have ignored entirely the literization process as a human innovation that fundamentally transforms language.

My argument is simple: we should no longer analyze early medieval German from the structuralist perspective alone. We must move beyond the narrow and, one should add, nationalistically minded concerns of our disciplinary forebears and treat each \textit{scriptus} as a precious sociolinguistic artifact and a testament to perhaps the most important developmental moment in the history of the German language, the beginning of its literization. Instead of just focusing on reconstructing the competence of early German speakers, a task that is complicated not just by the fact that they are long dead, but by their multilectalism, let us work instead on reconstructing the socioculturally guided choices a literizer made when creating their \textit{scriptus}.  This work would crucially involve considering how a literizer shaped the varieties of their exclusively oral vernacular into a \textit{scriptus}. In order to do so, the literizer must innovate linguistically, especially if they are motivated to create a \textit{scriptus} that does not evoke the oral tradition and is better suited to and functional in the new dislocated context of the graphic medium. If we see \textit{scriptus}{}-creation as a conscious act, we can try to reconstruct how a literizer might have been influenced by the non-German writing they would have encountered as serious consumers of the Latin Bible and the Latinate tradition of literacy, which allowed them to access God’s teachings. How, for example, does classical linguistic thought influence German’s early literization? Which problems of ausbau does the metalanguage of \textit{grammatica} help our intrepid literizers solve?

There are so many new avenues of research that open up to us as researchers if we approach the study of early medieval German in this way. True, we perhaps leave aside the goal of a comprehensive accounting of an early German grammatical system. But as \citet[87]{Diewald2007} explains~-- and as I argued in \chapref{sec:chap:7}~-- overspecifying a historical grammar in this way invites “anachronistic distortions,” whereby current grammatical systems that are well-known to the researcher and superficially similar are “projected back onto the historical data.” And there is so much to gain, including all of the early German medieval texts, even the ones that we have largely ignored for being too problematic (see \chapref{sec:chap:2} on the deficit approach). For example, the entirety of the German translation of Tatian’s \textit{Evangelienharmonie} becomes significant, not just its \textit{Differenzbelege}, that is, the passages in which the German translation deviates from the Latin source.

Questions that we might ask of this text that are different from the usual ones (i.e., How genuinely German is the translation’s syntax? Did the spoken competence of the translator include \textit{pro}{}-drop?) might include the following. What exactly was the translator’s goal for the translated text? How did they manage the usual challenge that translators face of producing translations that convey both what the original says and what it means? What is the literary style of Tatian’s Latin, and how was this captured in German, in which literary style and well-formedness norms are emergent? Does the resultant German itself yield a Latin-inspired biblical, literary style that affects later pre-Luther translations of the Bible?

The Latin-language texts from the period also become relevant in that they can illuminate the ausbau choices of literizers. It is astonishing that Otfrid’s Latin language preface to his \textit{Evangelienbuch}, the \textit{Ad Liutbertum}, is hardly ever discussed in the literature, despite its being a rare metalinguistic window into an early medieval German literizer’s views and composition process. That it has largely been ignored makes sense for a couple of reasons. First, if one is only interested in reconstructing an underlying German competence through its imperfect performance as the text of the \textit{Evangelienbuch}, a Latin-language preface is irrelevant. Relatedly, in looking for those unconscious expressions of genuine competence, the metalinguistic comments of the author himself are irrelevant, unless they indicate that he was adhering to some prescriptive norm that would have obscured those genuine patterns. In this case, deficit-approach structuralists have yet another reason not to work with Otfrid’s monumental, original composition. A final, and not inconsequential, reason is that we are trained Germanists, not Latinists and, so, Latin texts might be less accessible to us. The reader may remember that I worked with a translation of the \textit{Ad Liutbertum}; I was fortunate enough to find one that hit the right balance between conveying what the original text meant and what it said.

If, however, one adopts the literization approach and finds ways of overcoming the required, but more challenging interdisciplinary view of what is relevant to the creation of early German \textit{scripti}, texts like the \textit{Ad Liutbertum} become important textual data. The Latin itself could even be relevant to how Otfrid shaped his German \textit{scriptus}: \citet[872]{Magoun1943} notes that Otfrid’s Latin composition is not “always easy and, generally speaking, exhibits an inflated rhetorical style.” This characterization mirrors exactly how I would describe Otfrid’s German literary style. I have been working with Otfrid’s text for decades and teaching it for almost as long, and I still find his overly complicated style of writing challenging. I wonder to what extent Otfrid’s ideas of good Latin composition affected how he constructed his German \textit{scriptus}, especially in light of the fact that he wanted to raise Frankish’s prestige across the empire and beyond.

Another advantage, then, of adopting the literization approach is that we can examine poetic texts without worrying about the performance issues that supposedly obscure a genuine early German competence. It is particularly in the examination of \textit{scripti} whose style was designed to evoke the layered elaboration of the oral tradition where we could conceivably have a window into a prehistoric past. However, what interests me the most about these \textit{scripti} is how they connect to the elaborated oral traditions of other non-writing communities and how these data collectively connect to the psycholinguistics of processing linguistic production in the phonic medium. It is worth focusing on the \textit{Hêliand} briefly, which, like the \textit{Hildebrandslied}, represents a self-conscious adopting of this particular literary style. However, unlike the \textit{Hildebrandslied}, its subject matter is imported, and part of the literizer’s job in creating their \textit{scriptus} was to familiarize the audience to the gospel story and its lessons. While Otfrid’s project of creating good Frankish pushed him to build a \textit{scriptus} that used ausbau constructions to build denser, more integrated language, the \textit{Hêliand} poet was more constrained by his chosen literary style in ways that perhaps made his job more difficult. That is, the poet had to discuss these same narratives and theological topics, whose expression was rooted in and grew within this Latinate linguistic tradition, but he had to do so mimicking the basic building blocks of elaborated orality. While the verse-IU building blocks on display in the \textit{Hildebrandslied} represented colloquial, mnemonic, and community-embedded chunks of language, the poet would have to create the verse-IUs of the \textit{Hêliand} and link them together in ways that would hopefully prove not to effect too much ambiguity for the reader. Speaking from the point-of-view of a long-time reader of the \textit{Hêliand}, I would say they did a fair job.

  It may seem to the more structurally minded reader that I am asking them to abandon the linguistic analysis of early German entirely. Indeed, many of the possible research questions I have posed in this conclusion will not be answered through, say, the generative analyses that have dominated in the last several decades. There are alternate ways of conceptualizing linguistic productions, however, ones that might be better suited to the early literizations of oral vernaculars, in particular usage-based approaches, as represented in the works of scholars like Joan Bybee. I can also see constructionist approaches, as represented in works like \citet{Goldberg1995} and \citet{Goldberg2006}, as a methodology that could aid in the nuts-and-bolts analysis of an early German \textit{scriptus}. They provide a method for analyzing linguistic patterns as constructions, that is form and meaning pairings that are created or learned and, in this way, could be more consistent with the situation of early medieval German. It remains to be seen whether such usage-based methodologies can add much in the way of explanatory power to a literization analysis.

I end this book with a final word on structuralism, which received a lot of criticism in the foregoing pages. In fact, one of the anonymous reviewers of this manuscript remarked that my “rhetoric against structure verges on being anti-intellectual/post-modernist.” In maintaining that the literization approach, which I see as a sociolinguistic methodology, can more fully capture the linguistic phenomenon of the early German \textit{scripti}, I do not deny that structure matters. That is, the linguistic intuitions of the speakers who created the \textit{scripti} fed directly into their linguistic creations. Rather, I maintain that first, structure cannot tell the whole story of their creation and attested patterns, and second, we have focused on historical structures alone for too long, so much so, in fact, that many diachronic linguists believe that structure is the most important, or, indeed the\textit{ only}, factor they need consider when investigating historical \textit{scripti}. I do not think that it constitutes anti-intellectualism to argue against this view. In my future research, I will look forward to connecting these early German texts, which have fascinated me for so long, to the broader cultural, social, political, and linguistic world in which they were created, as well as the long tradition of literary German that lies at the heart of German studies.
