\chapter{Historical linguist seeks well-formed sentences in early vernacular \textit{scripti}}\label{sec:chap:7}

\section{Introduction}
I closed \chapref{sec:chap:6} with the argument that there is a relationship between the answers to these two questions: First, which linguistic resources does a literizer lean on more heavily to create their \textit{scriptus}? Second, how extensively do they engage with ausbau and, thus, look to enhance the grammatical and semantic coherence of their writing by making the logical and hierarchical relationships between constituents and ideas explicit? I proposed that the more a literizer was interested in evoking the oral tradition, the more they relied on orally organized ways of building lexically denser and more integrated language into their \textit{scriptus}, strategies I collectively called layered elaboration. Similarly, a literizer like the \textit{Hêliand} poet would engage less with the process of ausbau, which I have defined as the construction of more grammatically specific and visually explicit systems of coreferentiality between the linguistic elements of a text. In contrast, a literizer like Otfrid, who eschews the planned distance varieties of their elaborated orality and engages more intentionally in vernacular ausbau, will try to establish this type of systematicity. I see these two seminal ninth-century texts as marking the beginning of literary German; the two poets’ orientation toward and away from the oral tradition, respectively, demarcates the two broad categories of literary style that characterize early composition in German.

Literary style is not a topic with which modern historical linguists often contend in that it is considered a matter of linguistic performance and not competence, and these scholars interest themselves primarily in delineating the contours of the latter. I argue, however, that recognizing emerging literary styles in the early \textit{scripti} is important to understanding the attested syntactic differences between them. This proposal follows logically from the relationship I see between a literizer’s orientation vis-à-vis the oral tradition and the degree to which they engage with vernacular ausbau. Because a literizer like the \textit{Hêliand} poet is less concerned with ausbau, their resultant \textit{scriptus} will contain more structurally ambiguous sequences (see \sectref{sec:6.2.3}). That is, they maintain more conceptually oral ways of modulating linguistic production according to the communicative contexts of distance in the phonic medium. In contrast, literizers like Otfrid want to create a new written variety that will not remind listeners or readers of the oral tradition. So, they look to create a new variety to suit the context of communicative distance. In that Otfrid was specifically interested in creating a written form of the vernacular, one that reflected the prestige of the empire in which he was a subject, it would have been logical for him to engage more intentionally with vernacular ausbau and, to this end, draw on his education in Latin and its long literate tradition.

In this chapter, I consider a conceptual development that, I argue, necessarily accompanies the literizer’s engagement with vernacular ausbau; namely, the cultivation of a sense of well-formedness. That is, in engaging with this literization process, a \textit{scriptus}{}-creator must also develop an awareness of literacy as a conceptual category. This change happens on an individual and societal level, though, as I argued in \chapref{sec:chap:3}, German \textit{scripti} emerged mostly in isolation, and so there was no concerted, society-wide engagement with the question of how literizers must shape their oral vernacular varieties to be more functional in the written medium. An analysis of German’s first attestations, thus, mainly involves individual linguistic change. I see the development of a notion of well-formedness as another variable that is dependent on a literizer’s stylistic choices. That is, an early German literizer who turns away from the varieties of elaborated orality and, consequently, toward ausbau and the Latinate tradition of literacy, will directly compare their multilectal vernacular to Latin, which has already undergone significant ausbau. As a result, the disparity in German and Latin’s suitability for the dislocated written word comes into more immediate and sharper relief.

Direct comparison may also lead to the literizer drawing on the linguistic norms that are featured and discussed in the classical grammatical treatises that were popular in the Carolingian period. In the parlance of modern linguistics, these norms are both descriptive and prescriptive, though classical linguistic thought did not distinguish one from the other, as will also become clear in this chapter. Based on my \chapref{sec:chap:5} arguments (see especially \sectref{sec:5.2}), one could characterize a linguistic innovator like Otfrid as concerned with descriptive and prescriptive norms: he states in his preface that he was at pains to create a \textit{scriptus} that was simultaneously idiomatic Frankish and a proper, written language. Pursuing both goals presented Otfrid with a conundrum in that Frankish was a rustic, barbarous language, whose main defect was how unlike Latin it was. On the one hand, it would make sense for Otfrid to use the Latinate norms he had encountered as part of his education and continuing engagement with Latin literacy as a model for the ausbau of his German \textit{scriptus}. Yet would not a Latin-inspired ausbau be incompatible with the monk’s stated desire to create a \textit{scriptus} that is also idiomatic?

Recalling my arguments of Chapters~\ref{sec:chap:3} and~\ref{sec:chap:4}, the answer to this question is no. That is, the ausbau of an exclusively oral vernacular demands from the literizer linguistic innovation. Before the moment of literization, none of its varieties~-- either of immediacy or distance~-- had ever needed to function in the graphic medium, which can dislocate linguistic output completely from the moment of its production. Thus, literization itself demands from the vernacular speaker a level of explicitly marked, linguistic cohesion that had never before been achievable, functional, or required in the phonic medium. Phrased another way, an exclusively oral Frankish did not yet have idiomatic ways of being fully functional in the graphic medium. Individual literizers during this early medieval period devised their own solutions to the problem of ausbau in ways that reflected their chosen literary style. Over time, as more people engage with vernacular literization and ausbau, writing conventions emerge at a societal level, and ausbau constructions become increasingly idiomatic, especially for the most literate people. To be clear: my argument is \textit{not} that the immediacy and distance varieties of exclusively oral vernaculars manifest no grammatical relationships, e.g., agreement rules, or surface order patterns. Nor is it that literizers like Otfrid find no inspiration in their idiomatic vernacular when creating \textit{scripti}. Human language before and after literization and ausbau features conventionalized patterns, and literizers can certainly look for ausbau solutions that reflect the patterns of exclusively oral vernaculars as much as possible.\footnote{{However, as I discussed in \chapref{sec:chap:6}, if a literizer draws more on their oral vernaculars of distance, they are engaging less with the project of ausbau and producing} {\textit{scripti}} {that are less functional and more grammatically and lexically ambiguous in the graphic medium. I argued that the} {\textit{Hêliand}} {was an example of this possibility.} } Rather my argument is that it is literization alone that effects the need for ausbau, which yields a theretofore unrequired degree of grammatical and lexical systematicity, specificity and explicitness, as well as the means for developing these features.

This chapter is divided into three main sections. In \sectref{sec:7.1}, I discuss the related developments of notions of well-formedness and literary style. I argue that well-formedness in the early German \textit{scripti}, indeed in any written language, is always modulated in accordance with the desired literary style. Indeed, there is no such thing as decontextualized linguistic output, that is, a neutral variety that is not indelibly shaped by a communicative context. This view is not inconsistent with structuralist approaches to historical syntactic analysis. For instance, diachronic generative linguists maintain that performance is never a perfect reflection of competence. Where these linguists and I differ is in their belief that the most important goal of diachronic analysis is the reconstruction of a historical underlying competence. In contrast, I propose that this goal is too narrow to account for the broad linguistic phenomenon that is the early German \textit{scripti}. A focus on competence alone encourages the investigators to minimize or reject entirely the possibility that literization itself causes significant language change. It has also sent scholars on the misguided search for historical varieties they believe most accurately represent a neutral or contextless early German (see \chapref{sec:chap:2}). In this section I explain how the belief in an unmarked, neutral variety of language stems from classical discourses on language. In particular, Aristotle describes “flat Greek,” which he characterizes as a neutral and so-called grammatical form of Greek but is actually a long cultivated and particular literary style that emphasizes clarity of expression. This sort of classical metalanguage provides literizers like Otfrid a road map for their own literization project. More specifically, I hypothesize that the categories and linguistic descriptions in these treatises can directly shape the ausbau structures of a German \textit{scriptus} in that they give early literizers a conceptual framework for the vernacular innovation that ausbau demands.

In \sectref{sec:7.2}, I elaborate on how the metalanguage of classical linguistics can influence the early German \textit{scripti} of literizers like Otfrid, who consciously turned away from the layered elaboration of their community’s oral varieties of distance. As I discussed in \chapref{sec:chap:6}, interlocutors must process lexically denser, more integrated structures in contexts of communicative distance using the same building blocks as they do for language processing in immediacy contexts. That is, all phonic varieties are comprised of the prosodically defined intonation unit (IU) and are subject to the same cognitive constraints. I also concluded in \chapref{sec:chap:6} that the IU was the main locus for the modulation of linguistic output in accordance with changing communicative contexts. A broader point undergirds these arguments, which is that non-literate speakers of exclusively oral vernaculars will conceptualize language in exclusively phonic terms.

In this section, I discuss how, in turning to the Latinate tradition of literacy as a model for ausbau, an early German literizer encounters a new conceptualization of language, one that references sound but also structure. So, while popular Late Roman grammars identify linguistic categories that foreshadow modern syntactic categories, e.g., nouns, pronouns, verbs, and conjunctions, they begin their grammatical description of Latin with the smallest linguistic units, which they define with respect to sound. These grammars also identify larger linguistic units akin to the clause and the sentence, though they do not define them in terms of grammatical relations. Rather, these definitions reference primarily prosody and meaning, with some allusions to structure. \textit{Grammatica}’s engagement with these units, I argue, introduces to literizers like Otfrid the idea of what \citet{KochOesterreicher1994} call \textit{Satzförmigkeit}, a word that is difficult to translate into English, but what I take to mean a sequence of words that is, or indeed should be, constitutive of a complete discourse unit, like a clause or a sentence. This concept stands in contrast to the conceptually oral system of discourse organization that I call layered elaboration, with its overlapping parallel IUs that build dense and integrated language without clearly signaled boundaries between larger discourse units. I conclude that a \textit{grammatica}{}-inspired concept of \textit{Satzförmigkeit} influences how Otfrid decides to construct his larger clause-like discourse units.

As the discipline of linguistics moves toward structuralism, scholars eventually conceptualize the discourse units of language as clauses and clause complexes. They, furthermore, define these units in purely structural ways. So, a structurally oriented linguist would maintain that they are simply identifying the underlying features that characterize all human language; for example, all languages organize linguistic production in clauses which minimally comprise, say a subject and a predicate. My final argument of this chapter (\sectref{sec:7.3}) is that, though one can surely identify in the early German data subjects, predicates, and, thus, clauses, it does not necessarily follow that these linguistic elements must constitute the basic organizational structures around which a literizer organized their historical \textit{scriptus}. They are certainly the ones around which modern linguists have tended to form their own conceptualizations of language. But I wonder to what extent the conclusion that all human languages organize linguistic output according to clauses, which entail certain constituent types and clear beginnings and ends, is supported by structuralists’ preference for examining modern ausbau languages for which there is a thoroughly developed and societal notion of \textit{Satzförmigkeit} and well-formedness generally.

In contrast, a literizer like Otfrid developed notions of well-formedness and \textit{Satzförmigkeit} in a vacuum and for what was theretofore an exclusively oral German. He was accustomed to a vernacular whose organization was based on prosodic units and relied on layered elaboration. In that the \textit{scriptus} results from a creative shaping of Frankish in the graphic medium, one should imagine that Otfrid could have organized the larger units of his written production in different ways. Recall again that ausbau necessitates linguistic innovation; vernacular intuitions are restricted to the phonic medium and, though conventionalized native speaker patterns certainly feed into \textit{scripti}, they will offer no direct solutions to the problem of ausbau itself. We should not simply assume that Otfrid’s emergent sense of well-formedness, which effects his \textit{scriptus}’s consciously constructed syntax, was identical to our modern ideas of well-formedness and \textit{Satzförmigkeit}, in particular. For this conclusion to be true, there should be some evidence that, say, the metalanguage to which Otfrid was exposed in the popular grammars of his day contained descriptions of \textit{Satzförmigkeit} that mirror modern definitions of the same. This evidence does not exist. I, furthermore, argue that if scholars look for the structures reflecting modern notions of well-formedness~-- which themselves are created in a centuries-long process of ausbau~-- in a historical variety~-- especially one for which the literizer adopts the style of oral vernaculars of distance~-- they invite anachronism and presentism into their analyses.

\section{The development of well-formedness and literary style }\label{sec:7.1}

I begin this section with \citet[590--591]{KochOesterreicher1994}, the work that originated my definition of well-formedness. The authors characterize syntactic well-formedness as resulting from a process of selection and deselection: literizers consciously cultivate the grammatical coherence of their \textit{scriptus}, molding utterances so that they become \textit{satzförmig} or ‘constitutive of a complete discourse unit, for example, a clause or sentence.’ This definition of well-formedness as \textit{Satzförmigkeit}\footnote{{When a written variety evinces} {\textit{Satzförmigkeit}}{, it has the quality of being constitutive of a complete clause or sentence. I could think of no good way to capture this term in English, whose derivational morphology simply is not up to the task.} } has interesting theoretical implications. If well-formedness is about conformance to an innate competence, as generative approaches assume, then complete clauses and clause complexes should be provided to the literizer by their mental grammar and should not be constructions that they must first consciously cultivate. Furthermore, ill-formed clauses and clause complexes, according to the generative view, should result only from performance errors. Koch and Oesterreicher’s presentation of well-formedness implies, however, that the ill-formed constructions in a \textit{scriptus} arise instead through a less consistent or effective implementation of syntactic ausbau. I do not interpret Koch and Oesterreicher’s linking of well-formedness to syntactic ausbau as indicating that German-speakers at this time never produced complete, well-formed clauses and clause complexes by the standards of a modern literized language like German or English. Rather, I understand their proposal to indicate that well-formedness, rather than referring to a speaker’s innate competence, is an actively constructed concept, guided by the principles of syntactic ausbau, as outlined in \chapref{sec:chap:4}, and considerations of literary style, as I argue below.

\citet[590--591]{KochOesterreicher1994} gives several examples of ill-formed constructions, whose presence in a \textit{scriptus} would undermine its grammatical coherence. Two of their examples, are figures from classical rhetoric: anacolutha and aposiopesis. The first figure refers to unexpected discontinuities in a grammatical sequence: “I saw the sun for the first time in~-- how long?” The second is the deliberate breaking off of an utterance: “If I get my hands on you~…!” In the example of anacolutha, the discontinuity yields an incomplete clause. The utterance with aposiopesis, in contrast, features an incomplete clause complex; there is only a subordinate clause, while the expected main clause is left to the interlocutor’s imagination. The modern reader may, as did the Greeks who gave these two rhetorical phenomena their names, associate them with spoken language. Indeed, a third example of ill-formedness from Koch and Oesterreicher further suggests that ill-formedness in early \textit{scripti} can result from transferring certain construction types from the phonic medium, where they are functional because speakers and interlocutors share the physical space, into the graphic medium, where they are not. Holophrastic expressions convey a complex idea in one or a few words: “Help!” or “Again, please.” Such expressions are highly dependent on the contexts in which they are spoken and require elaboration should they be presented outside of that context.

If literization were merely a matter of ausbau, then \textit{scriptus}{}-creators would deselect all such disjointed or telegraphic constructions and not include them in their writing. The classical discourse on certain rhetorical figures reminds us that literization is more nuanced than that. Yes, literizers must augment the grammatical coherence of their oral vernacular before it can be more functional in the graphic medium. However, the ausbau process will also be influenced by the creation of appropriate writing, that is, literary styles. Consider the late-Latin grammarian \citegen{Quintilian1922} treatment of aposiopesis in which he explains how it has a place in formal, legal oratory. He writes in book 9 (chapter 4, section 54)\footnote{{Quintilian’s text is accessible online here:} \url{http://www.perseus.tufts.edu/hopper/text?doc=Perseus\%3Atext\%3A2007.01.0066\%3Abook\%3D9\%3Achapter\%3D4\%3Asection\%3D54}. } of his influential \textit{Institutio Oratoria} that aposiopesis can be useful for conveying passion or anger, as Koch and Oesterreicher’s example, “If I get my hands on you~…!” demonstrates. Similarly, the figure “may serve to give an impression of anxiety or scruple.” Quintilian draws his example from the speech (the \textit{Pro Milone}) Cicero wrote in defense of his friend, Milo, who was on trial for the murder of the politician Clodius.\largerpage[-2]

\begin{quote}
Would he have dared to mention this law of which Clodius boasts he was the author, while Milo was alive, I will not say was consul? For as regards all of us~-- I do not dare to complete the sentence.
\end{quote}

\noindent Quintilian explains later in the same book 9 (chapter 4, section 59)\footnote{{This section is accessible online here:} \url{http://www.perseus.tufts.edu/hopper/text?doc=Perseus\%3Atext\%3A2007.01.0066\%3Abook\%3D9\%3Achapter\%3D4\%3Asection\%3D59}. } why incorporating aposiopesis into formal oratory can be effective.

\begin{quote}
There are other devices also [that is, in addition to aposiopesis] which are agreeable in themselves and serve not a little to commend our case both by the introduction of variety and by their intrinsic naturalness, since by giving our speech an appearance of simplicity and spontaneity they make the judges more ready to accept our statements without suspicion.
\end{quote}

\noindent This passage indicates, first of all, an awareness of  orality and literacy as conceptual categories. Quintilian identifies this device as characteristic of spoken language or, phrased in the terms of the current study, as a conceptually oral construction.

Prescriptions around the use of such oral features were more subtle, however, than simply declaring them off-limits for written compositions, as we can see when Quintilian notes that they can be useful for feigning the language of immediacy. There is a particular irony to these suggestions: speakers would craft formal oratory in advance with the aid of writing. Though planned in advance, the language should \textit{seem} unstudied, affective, and spontaneous, which, Quintilian argues, lends it a feeling of sincerity and authenticity. Thus, one incorporates conceptually oral structures of immediacy into one’s distance language. \citet[200, 205]{Oesterreicher1997} describes this phenomenon as one of his eight types of orality in text: “mimesis of immediacy or simulated orality.” He notes that it seems like an oral phenomenon “on the surface” but that the communicative conditions that are responsible for its production are not those of immediacy. “Such imitations of casual speech function as \textit{literary devices}” (emphasis added), Oesterreicher explains. They can only ever be “a mimesis of the language of immediacy [and] can never match authentic immediacy,” he adds.

Quintilian, Peter Koch, and Wulf Oesterreicher are all pointing to the same insight but from different perspectives: the development of different literary styles is a concomitant development to the establishing of well-formedness through ausbau. To begin, \citet{KochOesterreicher1994} notes that the disjointedness of the language of immediacy is one of the primary targets of ausbau-related change. Quintilian recognizes that a writer can consciously adopt constructions that feature this sort of disjointedness and are associated with conceptual orality. Finally, \citet{Oesterreicher1997} reminds us that this feigned orality is not the same thing as language that is actually shaped by the communicative context of immediacy. It is instead a literary style.

Another literary style that emerges out of similar discussions of well-formed, appropriate language and its supposed relationship with the spoken language is what \citet[204--205]{Oesterreicher1997} refers to as a “plain” rhetorical style of writing. I first discussed this type of composition in \sectref{sec:2.2.1}, where I explained that the idea of a neutral, grammatical prose as representative of some natural, spoken language stems from the classical tradition of linguistic thought and cannot actually describe any actual variety. I return now to \textit{On Rhetoric} briefly to point out that what Aristotle actually describes in his section on grammatical Greek (book~3, chapter 5; see \citealt{Kennedy2007}: 206--209) is the development of well-formedness through ausbau, a process that he also relates to literary style. \citet[206]{Kennedy2007} notes that while his discussion of the rules of grammar in this chapter is meant to be about the grammatical rules of “speaking Greek,” they are in fact about “clarity.” Indeed, all five principles that Aristotle mentions\footnote{{One presumes there are many more than five. Kennedy (\citealt[207, fn. 58]{Aristotle2007}) characterizes this chapter as possibly the least satisfactory in the} {\textit{Rhetoric}}{. As in previous chapters, all of my direct quotes from the} {\textit{Rhetoric}} {are from Kennedy’s translation.} } are concerned with enhancing the grammatical and semantic coherence of a composition and correspond to ausbau-related changes. For example, Aristotle calls for the correct use of “connective particles” (book 3, 5:2), such as ‘on the one hand’ and ‘on the other hand.’ This example connects to my discussion of lexical ausbau to effect semantic coherence in \sectref{sec:4.2.1}. Writers aiming for clarity in distance compositions must think in terms of text organization, not discourse organization, and develop a vocabulary for this purpose. His other example of the correct use of connective particles has, in fact, nothing much to do with particles.

\begin{quote}
But I, when he spoke to me (for there came Cleon both begging and demanding), went, taking them along.
\end{quote}

\noindent Aristotle prescribes that “correlatives should occur while the first expression is still in the mind and not be widely separated.” Otherwise, the “result is unclear.” It is, however, difficult to match Aristotle’s prescription to the example whose unclear syntax stems simply from the clause complex’s disjointedness. Integrating disjointed constituents into a clause or series of clauses through grammatical systems of coreferentiality is the particular task of syntactic ausbau, as I discussed in \sectref{sec:4.2.2}.

We may categorize the remaining four principles similarly into prescriptions surrounding lexical or syntactic ausbau. Aristotle’s insistence that one use specific names for things rather than circumlocutions falls into the category of developing an expanded, differentiated, and precise vocabulary (book 3, 5:3; see {\sectref{sec:4.2.1}}).\footnote{{Unfortunately, Aristotle provides no examples of this principle, and so I cannot comment more extensively on it.} } The other three principles are concerned with establishing greater clarity in the concatenation of various constituents, which is the goal of syntactic ausbau. In book 3, 5:4, Aristotle warns against using amphibolies, which are syntactically ambiguous constructions, unless it is one’s intent to obscure the meaning. An old Marx brothers’ joke is an example of an amphibole: “I shot an elephant in my pajamas,” the humor arising from the uncertainty about who is wearing the pajamas, the shooter or the elephant. The final two principles (book~3, 5:5--6) set forth prescriptions for proper inflection of participles: “Having come and having spoken, she departed” and “Having come, they beat me.” Aristotle is reminding readers here to observe the classifications laid out by the philosopher Protagoras (fifth century BCE) that participles inflect according to the gender and number of the pronoun. Aristotle closes this chapter with some general prescriptions regarding clarity, stating that “[w]hat is written should generally be easy to read and easy to speak~-- which is the same thing.” If one has an easy time punctuating one’s writing, that is a indication that one has achieved clarity. In contrast, writing that is hard to punctuate is less desirable because “it is unclear what goes with what, whether with what follows or with what precedes” (book~3, 5:6). I interpret these statements also as relating to ausbau: establishing more explicitly marked connections between constituents within and across clauses makes clear syntactic boundaries easier to delineate.\largerpage

In sum, though Aristotle characterizes his discussion of “grammatical Greek” as establishing some principles of a spoken grammar, what he, in fact, describes is a series of well-formedness norms that promotes grammatical and lexical coherence. There is another unintended parallel between Aristotle’s discussion of well-formedness, in his terminology “grammatically correct Greek,” and my assertion that an ausbau-driven well-formedness occurs in relation to the development of literary style. Aristotle argues in the earlier chapters of this book 3 (chapters 1 and 2) that grammatically correct Greek represents, as \citet[314]{Graff2005} puts it, a “stylistic zero-degree” whose “flatness” makes it stylistically inappropriate for anything but ordinary, everyday speech. Good prose style, in contrast, should be more dignified and special. One can achieve this result only by consciously constructing one’s prose artfully. Crucially, however, the writer must take steps to conceal this art. In other words, good prose should seem like everyday speech without actually resembling it \citep[314--317]{Graff2005}.

\begin{sloppypar}
Aristotle conflates a number of phenomena with one another in a way that foreshadows future conflations. On the one hand, we have the construct of “flat” Greek, which is meant to represent a neutral, contextless, spoken language. On the other hand, there is the notion of well-formedness, which involves selecting some constructions as well-formed, others as not, and attaching the descriptor of “grammatical” to the former set.  This linking of an imagined neutral, context-independent language with an ordinary, everyday speech, along with well\hyp formedness and grammaticality is evident in discourses on language long after Aristotle. This same constellation of ideas surfaced in the Middle Ages, for example, as part of scholarly discourse on the literary career of Virgil, which was envisaged as a wheel, the \textit{rota Vergilii} \citep[138]{Laird2010}. This diagrammatic representation of Virgil’s works categorized his different literary styles, including the \textit{stilus humilis/stilus planus}, in other words a “plain style rhetoric” \citep[204--205]{Oesterreicher1997}. Authors writing in this style, Oesterreicher explains, employed a “markedly simple language in opposition to what may be called linguistic mannerism or rhetorical bombast” (page 204). \citet[204--205]{Oesterreicher1997} continues:
\end{sloppypar}

\begin{quote}
It is characteristic of this style to draw on a number of features that have been conventionalized in a certain literary tradition in order to create the impression of naturalness, spontaneity, simplicity, and ease. This style is therefore aesthetically motivated and is not intended fundamentally as an imitation of the language of immediacy [unlike Oesterreicher’s category “mimesis of immediacy”]. In Renaissance Europe this stylistic advice was condensed to the imperative formula “Write like you speak!”
\end{quote}

\noindent The dictum, “write like you speak,” will strike Germanists as familiar. It has been identified as a motivating principle behind Martin Luther’s translation of the Bible into a German that was more comprehensible than that of the earlier Mentel Bible. It also motivated his multiple revisions of his original translation, in which we can see greater structural clarity and regularity emerge. It seems to me, in any case, that there is good evidence of this long-standing belief in the existence of a style of prose writing that both mirrors the spoken language and is representative of a neutral grammar (see also \chapref{sec:chap:2}). I propose instead that this plain rhetorical style is the product of ausbau that writers over time have shaped into a particular direction because they wanted to create a clear literary style. The result is a style of prose that works best for an uncomplicated transfer of information from the mind of the writer to all possible readers in any spatiotemporal context. Writing like one speaks, in contrast, is impossible unless one means to record and transcribe speech. As a recommendation on literary style, however, the dictum makes more sense (Avoid rhetorical excesses. Keep clauses and sentences short and unambiguous). This style is but one of many that can guide the writing of prose; it is no more or less like “the spoken language” than any other literary style.

\section{\textit{Satzförmigkeit}}\label{sec:7.2}

In \sectref{sec:7.1}, I argued that the early literizers of an exclusively oral vernacular must develop a concept of well-formedness in tandem with literary style. By this I mean that what writers decide is well-formed in a particular \textit{scriptus} is not only a matter of creating grammatically and semantically coherent written language, i.e., ausbau. It is also a matter of creating a \textit{scriptus} that is appropriate to the writer’s goals. As I proposed in \chapref{sec:chap:5}, a literizer’s orientation toward their two main linguistic resources~-- the elaborated orality of community discourse traditions and the Latinate tradition of literacy~-- correlates with the degree to which they engage with ausbau. For example, the \textit{Hêliand} poet, who consciously adopts the style of elaborated orality (see \chapref{sec:chap:6}), will rely more on the organizational strategies of layered elaboration, like parallelism, and less on establishing new systems of grammatical concatenation. Otfrid, in contrast, who wants to move the vernacular away from the oral tradition and toward great literature, would focus more on creating a grammatically coherent \textit{scriptus}.

In this section, I discuss \citegen{KochOesterreicher1994} assumption that well-formed, which is to say, grammatically and semantically coherent \textit{scripti} will be \textit{satzförmig}, that is constitutive of a complete clause or sentence.\footnote{{The derived noun} {\textit{Satzförmigkeit}} {refers to the quality of being constitutive of a complete clause or sentence.} } This is the logical proposition that Koch and Oesterreicher imply with their presentation of well-formedness as tied to not only improving the functionality of an oral vernacular in written distance contexts, i.e., ausbau, but also involving the cultivation of \textit{Satzförmigkeit}. If we re-phrase this understanding of well-formedness in the terms of the preceding chapters, we arrive at the following arguments: through engaging in ausbau and the creation of well-formedness, both of which are modulated by a concomitant development of a sense of literary style, the literizer shapes their oral vernacular into a \textit{scriptus}. The multiple varieties of their oral vernacular are based around the intonation unit (IU); in contexts of communicative distance, speakers create more lexically dense, integrated utterances by layering IUs onto one another and linking them implicitly through parallelism and more explicitly through deictic particles. According to Koch and Oesterreicher, however, shaping one’s vernacular into a \textit{scriptus} also involves creating larger discourse units akin to complete clauses and sentences. If I have convinced you, the reader, that the basic unit of immediacy and distance oral vernaculars was the IU, then Koch and Oesterreicher’s introduction of \textit{Satzförmigkeit} as a goal of ausbau means that literizers may abandon the prosodically defined IU in favor of more syntactically defined units. On its face, this possibility makes some sense in the context of the literizing language’s trajectory from existing only as sound, where prosody matters, to gaining a graphic presence, where prosody matters less. It would also be a logical consequence of a literizer drawing on the Latinate tradition of literacy, whose texts feature literary styles that diverge from the layered elaboration of the oral tradition and whose grammars point to means other than prosodic for the larger organization of linguistic output.

Clauses and sentences can feel inevitable not just for generativists, who might assume that both are part of an innate human grammar, but for speakers of the literized modern Germanic languages where one finds clauses aplenty and is conditioned to look for them in their textual antecedents. Leaving aside the question of whether the development of clauses and sentences is a universally inevitable consequence of literization and ausbau, the history of a written German indicates that, at the very least, clauses and clause complexes provided an effective domain for the literization of its oral vernacular. With this statement I point to the following proposal: over the course of the development of a written German, literizers coalesced around the clause and clause complex as the primary domains in order to create functional written languages that could meet all of the requirements of a fully dislocated communicative distance. To reiterate a point I made in this chapter’s introduction, I fully acknowledge that modern linguists have no trouble finding collocations that will fit their conceptualization of clauses or sentences in historical data. However, the ease with which they can find such constructions might well say more about their own literacy and internalized norms of a written well-formedness than it does about the \textit{scriptus} itself and the literizer’s process of creating it. I elaborate this argument in the sections to follow.

In \sectref{sec:7.1}, I discussed some significant figures identified in the classical discourse on rhetoric. Koch and Oesterreicher, I noted, called these constructions ill-formed with respect to ausbau because they were not \textit{satzförmig}. The Greeks and Romans would have characterized those same constructions as ill-formed only if used in compositions where it would have been considered poor style to have conceptually oral constructions. In this section, I discuss some examples of constructions from modern languages that are more ill-formed than the examples of aposiopesis and anacolutha discussed in \citet{KochOesterreicher1994}. Those constructions, though not evincing complete clauses or sentences, still had a level of coherence that makes them functional in certain literary styles of composition, especially the two styles that invoke orality in text \citep{Oesterreicher1997}, the mimesis of spoken language and the plain prose style. I venture that no one would consider the fragmented and disjointed utterances of the next section appropriate for writing in any literary style. In these examples one must become a more active editor of the attested data in order to create sequences of words that fit modern conceptualizations of clauses and clause complexes or even coherent constituents. My hope here is to highlight how scholarly treatments of modern spoken data parallel our treatment of historical data in some important ways. Chief among these is how structurally oriented linguists are reluctant to analyze spoken varieties of immediacy and historical \textit{scripti} that reflect the prosodically driven organization of exclusively oral varieties on their own terms. Instead, they have approached both types of data with the assumption that a structure or competence underlies all attestations, including the well-formed written ones that were consciously shaped across generations of literization and ausbau. I propose that it is these latter ausbau structures that structuralists have tended to adopt as the underlying structure.

\subsection{Speakers produce discourse units, while literizers create clauses }\label{sec:7.2.1}

I have taken the first two ill-formed constructions from  \citet[60]{MillerWeinert1998}. These examples demonstrate that spoken utterances, especially those that are spontaneously produced, can show what may seem like surprising levels of disjointedness and incoherence when transcribed into the graphic medium.\footnote{{It is important to remember, as I argued in Chapters~\ref{sec:chap:3} and~\ref{sec:chap:4}, that linguistic production in contexts of immediacy is both shaped by and suited to its context. It is when we transpose such language into the graphic medium that their disjointedness and incoherence become particularly evident.} } In this first example, note the absence of complete clauses and clause complexes.

\ea%1
    \label{ex:7:1}



no if we can get Louise/ I mean her mother and father/ Louise’s parents would give us/they’ve got a big car and keep the mini for the week// but Louise isnae too keen on the idea so …
    \z

\noindent In \REF{ex:7:1}, clauses are incomplete and are not integrated into clause complexes. For example, the non-finite complement that would complete the clause, \textit{if we can get Louise …}, remains unexpressed, while the clause, \textit{Louise’s parents would give us}, has no object complement. In fact, the utterance contains not one single, complete main clause-subordinate clause complex. The second example, from \citegen{Sornicola1981} study of spoken Neapolitan, is even less coherent than the one in~\REF{ex:7:1}.

\ea%2
    \label{ex:7:2}
    programmi che (pause) per i bambini (pause) [...] a l'indomani (pause) vedono (pause) guardono (pause) per la scuola \\
\gll         programmi   che per   i   bambini   a l’indomana \\
  programs   that  for  the   children      for tomorrow\\

\gll vedono  guardono   per la scuola\\
they.see     they.watch  for the school\\
    \z

\noindent As I discussed in \sectref{sec:4.2.2}, this spoken utterance is notable for its discontinuities, which undermine the establishing of any syntagmatic relations. It does not contain any grammatically and syntagmatically coherent constituents, clauses, or clause complexes. Instead of “coherently organized pieces of syntax,” one finds a “juxtaposition of information blocks” (\citealt[60]{MillerWeinert1998}).

One final example of how spoken language can be disjointed and lack coherent constituents is from the Santa Barbara Corpus of Spoken American English. \REF{ex:7:3} is a transcription of a lecture delivered in a Chicano Studies class at the University of California, Riverside. The utterances are organized into IUs, indicated through ellipses.

\ea%3
    \label{ex:7:3}



… Well,

\textbf{… if we remember} .. \textbf{our Chicano history,}

\textbf{… my point,}

as we get into the whole question of Chicano political participation,

\%is that,

\textbf{(H) … it's difficult,}

.. \textbf{for us as Chicanos,}

.. \textbf{as we get into it,}

\textbf{… to understand,}

\textbf{… why is it,}

.. including ourselves,

.. because a lot of Chicanos don’t even understand why we don't vote,

.. why don't we participate,

.. and they come up with all kinds of (H) somewhat superficial analysis,

… that,

.. m=uch of our political behavior,

… is a product of what … was covered in this class,

.. under history,

(H) and culture,

and what did I say.

… Two.

… major currents,

… of history,

… two .. major currents of political culture,

… right?
    \z

\noindent This more extended spoken utterance was not spontaneously produced in conversation, like the data from \citet{MillerWeinert1998}. Instead, though it seems to be delivered extemporaneously, the language to some extent was probably prepared beforehand. Furthermore, this speech was delivered publicly as a university lecture on an academic topic. This context falls on the “distance” end of the continuum, the opposite pole to more private conversations among intimates or friends. Finally, though the lecturer is not identified, university lectures are usually given by people with at least a college degree and, more likely, those with advanced degrees. This supposition indicates that the speaker already had a long engagement with literacy, including the norms of Standard American English. Yet, note the discontinuities in clauses and clause complexes. For example, “my point” is a constituent that remains unintegrated into any clausal structure. The subordinate clause “if we remember our Chicano history” is never paired with a main clause to create a complete clause complex. Finally, the clause complex “it’s difficult for us as Chicanos, as we get into it, to understand, why is it~…” contains a discontinuity in how the expected non-finite clause “why it is (that Chicanos don’t vote)” is phrased as question (“Why is it that …?”) that is never brought to satisfactory completion.

As I discussed in \chapref{sec:chap:4}, the examples of spoken language in \REF{ex:7:1}, \REF{ex:7:2}, and \REF{ex:7:3} are shaped by, suited to, and functional in the contexts in which they were produced. This statement holds despite the fact that they fall short of written norms of well-formedness. It is when we replicate spoken utterances in the dislocated space of the written page, a domain that requires a lexical and grammatical explicitness that is unnecessary and unachievable in the spoken domain, that the differences between spoken and written varieties become glaringly apparent. These examples also serve to remind us that even highly literate people like university lecturers can and do produce constructions when speaking that fall short of the well-formedness norms created through ausbau and internalized over a lifetime of engagement with literacy. In these cases, they are not communicating sloppily or badly. Rather they are simply communicating in the phonic medium and are, therefore, subject to its particular pressures and constraints. I would even venture to argue that the students listening to the lecture on Chicano history excerpted in \REF{ex:7:3} understood more than if the lecturer had written a text ahead of time and read it out loud. This would particularly be true if, when writing the text, the lecturer ignored the fact that the text would be delivered orally. That is, the spoken delivery of a written text shaped by the context of distance will be less functional, as a room full of sleeping or confused students would undoubtedly confirm.

Of course, people who are educated and practiced in a written ausbau language have little problem turning the disjointed and incoherent blocks of syntax in these same examples into language that meets standards of well-formedness and is the most functional in written distance contexts \REF{ex:7:4}.

\ea%4
    \label{ex:7:4}
\ea No. Louise’s parents have got a big car. If we can get them to give us the big car and if they would take the Mini for the week [we could all travel by car together]. But Louise is not too keen on the idea, so [we will not be traveling in the big car].
\ex
\gll programmi  che   i bambini     vedono   perché  sono   loro   utili\\
programs   that  the children  see     because  they.are   to.them   useful\\

\gll per  la scuola   il giorno  dopo\\
for   the school   the day   after\\

\ex  Bearing in mind Chicano history and the question of Chicano political participation, it can be difficult to understand why Chicanos don’t vote. It is even difficult for us Chicanos to understand why we don’t participate politically. Superficial analysis indicates that our political behavior is the product of two major currents: history and political culture.
    \z
    \z

\noindent In turning the spoken utterances into grammatically coherent constituents in \REF{ex:7:4}, I propose that I acted like the middle school composition teacher I mentioned earlier in the chapter, who adjusts, reorganizes, and augments a student’s more orally influenced writing so that it conforms to the well-formedness standards of the written language. Where before there were fragmented, juxtaposed blocks of syntax, now there are logically and hierarchically arranged, coherently integrated constituents, clauses, and clause complexes. In examples (\ref{ex:7:1}--\ref{ex:7:3}), I can easily find subjects and predicates, and I can put them together to form clauses and clause complexes. My argument is simply that I will gain a more complete understanding of a speaker's linguistic output if I consider how they mapped their cognition onto prosodic units, as I discuss in \chapref{sec:chap:6}, rather than focus solely on the structure of the underlying competence.

Because of the notable differences between spoken utterances, especially those produced in immediacy contexts, and the well-formed written language of distance contexts,  \citet[26--27]{MillerWeinert1998} argue that the former should not be understood as a failed attempt at the latter. They cite the following passage from \citet[108]{Heath1985}.

\begin{quote}
  There has been a recurrent tendency in much syntactic research to distinguish between an underlying, rather crystalline “grammar,” which then interacts in real speech with a distinct outer “psycholinguistic” component, the latter being especially concerned with short-term memory limitations, linear ordering of major clause constituents, resolution of surface ambiguities, etc.
\end{quote}

\noindent Heath’s quote describes what linguists ought not to do with spoken language data. That is, do not assume that the well-formed clauses and clause complexes of the written variety are competence, and the syntactically more fragmented and disjointed utterances, the imperfect performance of this competence. If one does, one runs the risk of not accounting for the actual spoken data. The idea of the crystalline, underlying grammar is similarly problematic from the perspective of literacy education. Returning yet again to middle school composition class, consider a student there whose writing reflects linguistic production that is shaped by contexts of immediacy in that it, for example, features ambiguous constructions, sentence fragments, and run-on sentences. It takes years of literacy education to learn how to write good, clear prose (with clearly delineated and complete clause complexes). Most learners, in fact, do not reach that level of mastery. What are the structures that comprise their competence? Should they include the well-formed clauses or clause complexes that the language user never realizes in any medium?

These arguments apply particularly to the early \textit{scripti} of a language, for which there will not yet be any established norms of written well-formedness. Literization itself involves both the cultivation of ideas of well-formedness and the concomitant process of ausbau through which writers, first individually, then on a societal level, creatively shape their vernacular in order to effect a well-formed, distance-appropriate, and -functional written variety. Unlike the lecturer on Chicano history in \REF{ex:7:3}, the ninth-century German vernacular literizer has not benefited from a lifetime of education in and engagement with an established written standard language. The linguistic intuitions they have access to are shaped by and suited to the phonic medium. Ausbau structures designed to bridge the functional gap between the phonic and graphic medium are just being worked out, literizer by literizer and for the first time. It is, for example, likely that the ausbau structures in an early \textit{scriptus} are not as coherent and consistently executed as the those that eventually find their way into a standard language, which emerges centuries later and is the product of a collective, society-wide effort. Linguists must examine the historical data with these contexts in mind and treat the early \textit{scripti} of a language as artifacts of this whole context and not just of an underlying crystalline, to use Heath’s word, grammar. Furthermore, I argue that if we consider the historical data without the full context of their creation in mind, we run the risk of mistaking our own literacy-based notions of well-formedness for competence. Thus, in looking for that competence in the data, we are, in fact, anachronistically shaping them to fit this competence, not unlike the middle school composition teacher who corrects a student’s writing according to their own internalized sense of how written language ought to be.

\subsection{Are clauses and sentences universal syntactic categories? }\label{sec:7.2.2}

The arguments I have offered in this chapter so far indicate how I intend to answer the question posed in this section’s title: I propose that clauses and sentences (or clause complexes) are creations of literization and ausbau. Orally organized language produced entirely in the phonic medium has similar units, but their production is based around the convergence of cognition and sound. Their organizational logic differs from that of the clause, for example, which sets structural requirements for its construction. By this statement I specifically mean that clauses and sentences are written ideals that people establish through literization and, specifically ausbau; they are not inherent parts of a human competence. These clausal ideals can and do become cognitively real templates for some people’s linguistic production; this eventuality holds for those who attain some level of literacy and will especially apply to their production of written language. It makes the least amount of sense to assume the existence of Heath’s crystalline grammar based around clauses and sentences for the first \textit{scripti} of an exclusively oral vernacular, for which literizers are only beginning to work out what could constitute written well-formedness. My claim that sentences are not part of a universal competence might strike the reader as less controversial than claiming the same thing for the clause. In this section, I argue that the case for the clause’s universality rests on similar shaky ground to that for the sentence’s. The structure of an individual clause is also dependent on how one assesses the extent to which it is integrated into other clauses in its orbit. Hence, how one analyzes the structure of sentences affects one’s analysis of the structure of the clause. If one category is universal, it stands to reason that the other one would also have to be universal.

In this book, I have used “sentence” and “clause complex” interchangeably. In the literature, however, it seems that linguists will use one or other depending on whether they see groups of clauses as discourse units, in which case “sentence” seems to be the preferred term, or as syntactic units, in which case “clause complex” is used. For example, \citet[82--89]{Halliday1989} attempts a syntactic definition of clause sequences that can apply to the spoken language, a unit that he calls the “clause complex.” The term “sentence,” on the other hand, connotes considerations of rhetorical choice and style (which are generally not considered elements of grammar).  \citet[32--33]{MillerWeinert1998} differentiates the two as follows:

\begin{quote}
Clauses occur singly and in complexes, and clause and clause complexes are indispensable concepts for the study of both spoken and written syntax. Sentences in written language developed from the desire to mark clause-complexes; the initial capital letter of the first word in a clause-complex and the full stop following the final word signal which clauses the writer wants the reader to construe as interconnected. Of course, clauses are also interconnected in spoken language […]; the difference is that interconnectedness is not signaled by adjacency nor even by the relevant clauses occurring in the same turn (of conversation) or under the same intonation contour (in narrative).
\end{quote}

\noindent Thus, Miller and Weinert see sentences as features of the written language: they result from writers visually marking the boundaries between clause complexes, and these orthographic decisions are a matter of rhetorical style. Clause complexes, they argue, are present in the spoken language, but as larger discourse units, the boundaries of which are only loosely signaled.\footnote{{The language of a speaker who is reading from a prepared text may have well-formed, complete sentences. That is, the more planned the spoken language and the more practiced the speaker in writing, the more likely it is that one hears well-formed sentences.}}

In support of this argument, they present examples of sentences from different cultures and historical periods, noting that how writers construct them and where they place sentence boundaries is neither static nor uniform. Rather these characteristics vary from one text genre to another and from one writer to another (\citealt{MillerWeinert1998}: 41). An example of prose writing from the seventeenth-century Pepys’s Diary \REF{ex:7:5} illustrates this point.

\ea%5
    \label{ex:7:5}

          to Whitehall to the Duke, who met us in his closett; and there he did desire of us to know what hath been the common practice about makeing of forrayne ships to strike sail to us: which they did all do as much as they could, but I could say nothing to it, which I was sorry for; so endeed, I was forced to study a lie; and so after we were gone from the Duke, I told Mr Conventry that I had heard Mr Selden often say that he could prove that in Henry the 7ths time he did give commission to his captains to make the King of Denmark’s ships to strike him in the Baltique.
    \z

\noindent Latham's \citep[34]{Pepys1978} editing of the text brings the excerpt more in line with modern orthographic conventions. For example, the editor replaced Pepys’s dashes, which he used to mark boundaries within sentences with semicolons. While Pepys’s writing indicates an awareness of phrases and clauses, as well as larger discourse units, more invasive editing would be required to carve out the prototypical sentences of modern English with its main finite clauses and coordinated and/or subordinated final clauses.

The development of Latin prose provides additional illustrations of how writers in different cultures and historical periods shape their sentences differently. For example, \citet[119]{Palmer1954} describes how early Latin texts show a “naïve juxtaposition of simple sentences.” Over time, however, writers created a more integrated prose style comprising carefully constructed clause complexes of main and subordinate clauses, which is to say, sentences. Where these sentence breaks occur also changes over time and differs from where, for example, a writer of modern English prose would place them. For example,  \citet[43]{MillerWeinert1998} note that Latin relative clauses did not need to be orthographically marked as belonging to the sentence of some matrix clause. Consider the lines in \REF{ex:7:6}.

\ea%6
    \label{ex:7:6}
\ea
\gll ergo  telis       undique       obruitur\\
then   with.weapons   {from everywhere}   {it is showered}\\

\ex
\gll confossoque         eo  in vehiculum   Porus  imponitur\\
{having been transfixed}  it  in carriage     Porus  {is placed}\\

\ex
\gll \textbf{Quem}   rex      ut     vidit   … miseratione   commotus   … inquit\\
whom   {the king}   when  {he saw}   … by.pity     moved     … {he said}\\

\ex
\gll Quae   amentia    te    coegit   … belli     fortunam  experiri\\
What  madness    you  moved   … of.war   {the fortune}  to.try\\

\ex
\gll  cum    Taxilis    esset   … tibi     exemplum\\
when  of.Taxilis  was   … to.you   {the example}\\
\glt ‘Then it [the elephant] was showered with weapons from all sides. When it was dead, Porus was placed in a carriage. When the king saw him … moved by pity … he said: ‘What madness made you try the fortunes of war when you had the example of Taxiles? (Quintus Curtius Rufus)
    \z
\z

\noindent \textit{Quem} in (\ref{ex:7:6}c) is a relative pronoun, but its clause is orthographically marked as belonging to a separate sentence. This sort of construction is a regular feature of classical Latin, according to Miller and Weinert, and was even considered good style. As an indefinite deictic, they further note, \textit{quem} could point to referents within and across clause and sentence boundaries.

\citegen[44--45]{MillerWeinert1998} presentation of the temporal and cultural variability in sentence construction is consistent with the ideas of syntactic ausbau I laid out in \chapref{sec:chap:4} and the development of a written well-formedness. The authors emphasize that the sentences in different writing traditions result from “conscious effort, unlike changes in spontaneous spoken language.” Similarly, syntactic ausbau as I have defined it is the human-directed elaboration of once exclusively oral varieties so that they may function in the new, fully dislocated contexts that writing itself creates. Sentences result from language ausbau, whose aim is to create more grammatically and semantically coherent language, and well-formedness, which involves assessing potential ausbau structures in terms of both coherence and appropriateness. In the case of a language’s first \textit{scripti}, there are no vernacular norms surrounding how smaller linguistic units might be integrated into larger ones in the written domain. Each writer must work out for themselves how to structure their sentences. As the project of language literization progresses, writers develop a series of norms surrounding sentence construction. These norms can and do change over time. These conventions of a literized ausbau language are not innate. They must first be created; then they must be accepted on a broader, community- or society-wide level; finally, they must be taught.

The argument that writers create systems of clausal organization through conscious language ausbau connects to a broader scholarly discourse concerned with whether linguistic typologies can be taken for granted as crosslinguistic formal categories and which categories should be regarded as universal.\footnote{{\citet{Newmeyer2007}, who argues for the necessity of syntactic categories for linguistic typology, provides a succinct overview of the discourse.} } This narrative indicates that sentences are not a universal or innate formal category. If they were, they should be evident in all varieties of linguistic production. But as works like \citet{MillerWeinert1998} demonstrate, they are largely absent in spontaneously spoken syntax. Though the researcher can certainly identify or find sentences in the data, they are actually creating sentences based on their own ideas of well-formedness, which themselves are informed by literacy. \citet[180]{Mithun2005} expounds this view and also alludes to the consequences of assuming the presence of a sentence in data where it does not actually exist.

\begin{quote}
Our view of the sentence as a distinctive, static, even innate category may be overly simplistic, perhaps colored by the norms of European literacy. Most of us are all too familiar with the run-on sentences typical of early student papers; in many cases, the structure of the sentence is something that must be taught. If our syntactic analyses are based uniquely on written renditions of single sentences constructed or elicited in isolation through translation, we may miss some of the subtleties of the syntactic structures we are trying to understand, as well as the forces that create them.
\end{quote}

\noindent In the next section, I illustrate Mithun’s point using data from the \textit{Hêliand}, and demonstrate how scholars have typically approached such early data with patterns of a literate well-formedness in mind.

But what of the clause? As I have already indicated, linguists do not find it as easy to let go of the clause as a universal feature of a human competence. Consider \citet{MillerWeinert1998} as an example. The authors present convincing arguments in favor of the sentence’s status as a culturally and diachronically variable discourse unit. Yet, they are constrained by their insistence of one competence or “language system” that underlies all linguistic production, from the most spontaneously spoken utterances to the most planned written varieties of a language. “Language behavior” is the variable production in the spoken and written media, and the vast differences between competence and performance are a matter of degree of complexity,\footnote{{Though see \sectref{sec:3:3.2.1}, where I argue that “complexity” is an analytically vacuous concept that does not reference anything that cannot be more accurately and neutrally captured by the term “integration.”} } rather than of fundamentally different organizational principles (pages 28--31). So, it makes sense that the authors settle on the clause as the basic unit of grammar, rather than the sentence. Namely, it is difficult to assume the sentence as a universal syntactic category, when it is largely absent in the spoken data, especially its more spontaneous varieties. Clauses, however, are easier to find~-- or pick out, I would say~-- in written and spoken data. Furthermore, they argue, it is the “locus of the densest dependency and distributional properties, although a few dependency relations cross clause boundaries, and although, in written language, a few dependency relations cross sentence boundaries” (\citealt[46, 50]{MillerWeinert1998}).

Miller and Weinert’s decision to treat the clause as a universal category of syntax, however, is inconsistent with the spoken language data they themselves present or their own characterization of these data. For example, they acknowledge that spoken language, especially its more spontaneous varieties, often exhibits neither neatly integrated clause complexes, nor, in fact, well composed clauses, i.e., that meet the minimal standard for a clause. Instead, spoken language, especially when produced in contexts of immediacy, features “blocks of syntax with little or no syntactic linkage” (page 28). \citegen[20--34]{Sornicola1981} data from spoken Neapolitan, which \citet{MillerWeinert1998} reference to illustrate how fragmented spoken syntax can be, is a good example of this more disjointed syntax. Sornicola’s conclusion is that many spontaneous spoken utterances are not “coherently organized pieces of syntax,” as Miller and Weinert put it, replete with fully integrated constituents and logically arranged subjects and predicates. Rather they are juxtaposed “information blocks” (page 60). Is assuming that the clause is the basic unit of an underlying competence for these orally organized utterances any more warranted than assuming neatly integrated sentences as part of the same competence? Might we be \textit{creating} a coherently and structurally organized underlying competence, rather than \textit{uncovering} one. And, in focusing on structure, are we perhaps neglecting to account for linguistic production itself? Miller and Weinert seem to indicate that creating a clause by, to paraphrase the authors on page 30, picking out verbs and complements is justifiable. For the authors, the clause that the analyzer pieces together out of disjointed phrases is what the speaker intended but, for presumably performance-related reasons, failed to produce. Similarly, this cobbled together clause would be what the interlocutor puts together. I prefer to view the production and processing of orally organized varieties of immediacy as shaped or determined by, not to mention suited to, their communicative context rather than hindered by it.\largerpage[2]

This contradiction in Miller and Weinert’s presentation of spoken language leads us to another related one: their acknowledgment that fragmented “blocks of syntax with little or no syntactic linkage” are functional in their contexts (\citealt[60--61]{MillerWeinert1998})\footnote{{See \sectref{sec:3:3.2.1} for a more extended discussion of Miller and Weinert’s argument.} } but also “requir[e] from the listener a larger than usual exercise of inference based on contextual and world knowledge” (\citealt[28]{MillerWeinert1998}). In other words, they imply that such spoken utterances are unmarked but sometimes perhaps also marked and more difficult for listeners to process, presumably because they are constructing more coherent clause-like structures out of them. I think these inconsistencies result from Miller and Weinert’s discomfort with the theoretical implications of taking the more fragmented, spontaneously spoken data at face value. They are prepared to dispense with the sentence as a universal syntactic category, but not the clause. In that it is less disruptive to many modern theories of syntax to assume that spoken and written language are fundamentally the same with the same underlying competence, their reluctance is understandable. It might have perhaps struck the reader that the intonation unit introduced in my \chapref{sec:chap:6} aligns with Miller and Weinert’s description of “blocks of syntax.” I see them as the same basic linguistic unit, simply described from different perspectives. The prosodic description is definitional in that spoken language will be organized primarily along prosodic lines. But each IU naturally contains syntactic material, say, a constituent, and speakers link IUs with one another either implicitly or explicitly, as I discussed in \chapref{sec:chap:6}.

These ideas are consequential for how we approach historical syntax, especially data that stem from the first attestations of a language. When one examines historical German data, for example, one will have little difficulty finding clauses that appear to conform to modern parameters. For modern German those parameters can be clearly mapped out with the topological field model (see, for example, \citealt[53--54]{Wöllstein-LeistenEtAl1997}).

\begin{figure}
% % \includegraphics[width=\textwidth]{figures/Somersinpress-img004.png}
\begin{tikzpicture}
	\matrix (matrix) [font={\strut},matrix of nodes] 
	  {
	    \textbf{prefield} &
	    left sentence bracket & 
	    \textbf{inner field} &
	    right sentence bracket &
	    \textbf{postfield}\\
	  };
	\draw [thick] (matrix-1-2.south) |- ++(0pt,-1ex) -| (matrix-1-4.south) node [near start, below] {sentence bracket};
\end{tikzpicture}
\caption{The Topological Field Model of the modern German clause}
\label{fig:7:7.1}
\end{figure}

\noindent These two samples from the \textit{Hêliand} and the \textit{Evangelienbuch}, respectively, contain some data that are consistent with the field model’s template for modern German, other data that are not.\largerpage

\ea%7
    \label{ex:7:7}
\ea
\gll Tho    habda   eft    is   uuord  garu  mahtig    barn  godes\\
    then  had    {in turn}   his   answer  ready   powerful  child   God.\textsc{gen}\\

\gll    endi    uuid  is  moder   sprac\\
and      with   his   mother   spoke\\

\glt ‘Then (he) had his word ready in turn, the mighty child of God, and spoke with his mother’ (\textit{Hêliand}, 2023b-24)

\ex
\gll Spráh  tho     zi   iru   súazo      ther   ira   sún   zeizo\\
  spoken   then  to  her   sweetly  the      her  son   precious\\

\gll sconen     wórton        {ubar ál}        so   sun   zi   múater   sca\\
beautiful  words.\textsc{dat.pl}  {in every respect}  as   son  to  mother   should\\
\glt ‘The precious son then spoke sweetly to her with beautiful words, in every respect, as a son should (speak) to his mother’ (Otfrid, II 8 15--16)
    \z
\z

\begin{table}
\caption{\REF{ex:7:7} in the Topological Field Model }
\label{tab:7.1}
\small
\begin{tabularx}{\textwidth}{lllQQQQ}
\lsptoprule
 &  & prefield & l. bracket & inner field & r. bracket & postfield\\
 \midrule
a. &  & tho & habda & [eft] [is uuord] [garu] &  & [mahtig barn godes]\\
\tablevspace
b. & endi &  &  & [uuid is moder] & sprac & \\
\tablevspace
c. &  &  & Spráh & [tho] [zi iru] [súazo] [ther ira sún zeizo] [sconen wórton] [ubar ál] &  & \\
\tablevspace
d. &  &  & so & [sun] [zu múater] & scal & \\
\lspbottomrule
\end{tabularx}
\end{table}

\noindent Before discussing how well these clauses conform to the field model’s template, I acknowledge that one could quibble with certain decisions that I made. For example, my placement of \textit{mahtig barn godes} (\tabref{tab:7.1}a) in the postfield might be controversial. If one wants to assume modern German patterns in early German data, then this subject noun phrase, as the predicate’s external argument, should have been the first constituent of the inner field, as \textit{sun} is in \tabref{tab:7.1}d. Of course, one could assume the scrambling of arguments in the inner field as a way to explain \textit{mahtig barn godes} appearing as the last constituent of the inner field, rather than placing it outside of the bounds of the sentence bracket, that is, in the extraclausal postfield. The conjunction \textit{endi} (\tabref{tab:7.1}b) occurs in the leftmost extraclausal position, sometimes called the pre-prefield. This is the structural slot that in generative analyses would be the landing site of any left dislocated constituents. My assigning of \textit{sprac} to the right bracket rather than the left, reflects the fact that \textit{uuid is moder}, the verb’s complement occurs to the finite verb’s left and not to the right, where one would expect it in modern German.

Here are my takeaways from this example of mapping historical data onto modern templates. First, one can fit the clauses into the template, a result that is not too surprising because many of the basic elements that people today associate with clauses are present, namely, subjects and predicates. However, the process of mapping the template onto the data involves some degree of fitting round pegs into square holes. The inner field in \tabref{tab:7.1}c, for example, contains six separate constituents, which is more than the three slots that the topological field model usually allocates to it. The placement of \textit{mahtig barn godes} requires the analyzer to also work creatively with the template, as does the entirety of \tabref{tab:7.1}b, whose finite verb \textit{sprach}’seems to appear in the right bracket, though the left bracket is unfilled. The only example in \tabref{tab:7.1} in which the data easily fit the template is (d).

I would also like to highlight the extent to which one must appeal to positions that are not considered part of the clause proper when working with historical data, that is, the “pre-prefield” and the “postfield.” For example, \citegen[24]{Sapp2011} study of verb cluster phenomena in the history of German demonstrates that extraposition occurs about 21\% of the time in his dataset of subordinate clauses drawn from medieval German prose texts. The author notes on the same page that extraposition rates, that is, the frequency at which constituents occur outside the clause’s boundaries in the postfield, was high enough that \citet[19]{Lehmann1971} argued that the basic template of the German clause was different from how it is represented in the topological field model.\footnote{{\citegen{Sapp2011} conclusion that medieval German was a subject-object-verb (SOV) language is in line most other studies of historical German syntax, like \citet{Axel2007}, \citet{Axel-Tober2012}, \citet{Lenerz1984}, \citet{Lenerz1985}, among many others. \citet{Lehmann1971}, however, for an SVO clause structure.}} In structuralist oriented approaches, like the topological field model and generative frameworks like X-bar theory, constituents that cannot be incorporated within the clausal template in \figref{fig:7:7.1} are characterized as “extraclausal.” They are assumed to be generated within the clause and then are moved beyond its boundaries. These statements apply also to spoken language data, especially those produced in immediacy, rather than distance, contexts, where people with even excellent command of standard language norms are liable to produce non-standard forms. Recall the example of left dislocation from \sectref{sec:4.2.2}, reproduced here as \REF{ex:7:8}.\largerpage[2]

\ea%8
\label{ex:7:8}(from \citealt[240]{MillerWeinert1998})\\
\gll ja und \textbf{dies-en} \textbf{Flusslauf} dem folgen wir jetzt\\
yes and this-\textsc{acc} river.course \textsc{dat.det} follow we now\\
\glt ‘yes and this river course, we will follow it now’
\z

\begin{table}[H]
\small
\caption{\REF{ex:7:8} in the Topological Field Model}
\begin{tabular}{llllll}
\lsptoprule
pre-prefield & prefield & l. bracket & inner field & r. bracket & postfield\\
\midrule
{}[ja] [und]  & dem & folgen & [wir] [jetzt] &  & \\
{}[diesen Flusslauf]\\
\lspbottomrule
\end{tabular}
\end{table}

\noindent The integrity of the template is maintained only by scholars assuming the movement of constituents~-- \textit{ja}, \textit{und} and \textit{diesen Flusslauf}~-- from their supposedly real, underlying position in the clause to some other position that is defined only by \textit{not} being part of the clause. Countering that the pre-prefield and the postfield could be considered clausal positions in their own right strikes me as a somewhat disingenuous argument. These structural slots are generally occupied only by constituents that are not generated there but have been displaced from their underlying position for pragmatic reasons. That is, these extraclausal positions are treated as the exceptions that prove the rule of the basic clausal structure, rather than as evidence that speaks against what has been assumed to be the underlying structure in the first place.

In sum, my argument is not that early German data contain no surface order patterns or structures that resemble well-formed clauses by the standards of modern linguists, who look for elements such as subjects and predicates. Rather it is that our finding the clauses in the data by mapping them onto modern templates is not the same as those templates being the actual structure that underlies and can explain all linguistic production in the \textit{scripti}. Considering all the various sociocultural factors I have discussed in this book so far, as well as the innovative literization process itself, it does not make sense to focus all our scholarly attention on identifying an early German competence. In the paragraphs that follow, I consider how the state of syntactic thought at the time the first German \textit{scripti} were produced could have influenced the ausbau choices a literizer made. Their job was to shape their orally organized vernacular, which was structured around IUs, into structural units that were more suitable for a graphic medium. So, how might the popular late Latin grammars of the day have affected a German literizer’s reconceptualization of these discourse units, especially if that literizer decided to forge a new literary path that was distinct from their oral varieties of distance, as was the case for Otfrid von Weissenburg? Of particular interest is Donatus’ \textit{Ars Maior} and Priscian’s \textit{Institutiones Grammaticae}. Not only were the works themselves widely read and used among those educated in Latin and interested in \textit{grammatica}, they were also an influence on important Carolingian thinkers, like Alcuin, as well as countless other anonymous scholars who wrote commentaries on Donatus and Priscian \citep[145]{Luhtala1993}. The questions with which we might approach these sources is the following: How had classical writers conceptualized the clause? Is there any indication that they thought of clauses in any way like the way modern syntacticians have come to conceptualize early German clauses? If the answer is yes, then the possibility that Otfrid shaped his \textit{scriptus} into structural patterns that match modern conceptualizations of the clause becomes less remote. This finding would be interesting in the direct connection it would establish between classical linguistic thought and the conceptualization of a written German.

Of the two works, Priscian’s \textit{Institutiones Grammaticae} stands out as the more exceptional in that its seventeenth volume is entirely dedicated to syntax, something no other work to which the Carolingians had access purported to do \citep[146]{Luhtala1993}. This does not mean that Donatus did not comment on the larger units of language that generally fall under the purview of syntax. However, as we will see just below, Donatus’ approach to syntax is much different from how modern scholars approach the subject. Even Priscian’s syntax-focused writings differ significantly from modern syntax and offer no definition of a clause or of the relationships between clausal constituents that we would recognize today as syntactic definitions. None of this is surprising; Schönberger notes in the article appended to his 2010 German translation of Priscian’s volume, that classical syntactic thought was not so advanced to have, for example, analyzed the functional roles of the different parts of speech, that is, grammatical relations (page 497). He also offers a brief comment that I interpret as a more extreme version of the point of view for which I advocate here: “Axiomatic positings (\textit{axiomatische Setzungen}), to which we [i.e., language users] are oblivious, underlie today’s grammars of the old Indo-European languages” (page 497). Of course, Schönberger’s statement could fit a generative viewpoint, which assumes that a speaker's grammar is unconscious. His use of the term “axiomatic positings," however, implies to me that Schönberger questions the nature of the relationship between the grammars “posited" by scholars and the way multilectal speakers of historical varieties used their language or, indeed, fashioned their \textit{scripti}. This relationship, which historical linguists have assumed is causal, i.e., the grammar that we posit \textit{underlies} all varieties of a historical vernacular, warrants further examination. Specifically, we should consider the possibility that the linguistic categories posited as universal and constitutive of language reflect to some extent the metalinguistic discourse of whichever scholarly tradition formulates them. 

Donatus’ treatment of syntax in the \textit{Ars Maior} focuses on Latin’s parts of speech, which according to \citet[497]{Schönberger2010}, is characteristic of classical syntactic thought generally. Thus, his aim was to provide comprehensive descriptions of the different parts of speech, i.e., nouns, verbs, pronouns, conjunctions, etc. Comprehensiveness in this regard entailed determining all possible \textit{Akzidentien}, or properties associated with each category. The purpose of these investigations was didactic and prescriptive. The goal was not to engage with these concepts from a theoretical point of view \citep[497]{Schönberger2010}. The details of Donatus’ presentation of grammar indicate that people still conceptualized language in phonic terms, that is, as an \textit{essentially} oral phenomenon. For example, he begins his treatment of grammar with the smallest units of language, sound and defines these sounds in prosodic rather than phonetic terms; he lists syllables, metrical feet, syllable weight, and syllable stress as properties of sound (see pages 15; 21--37).\footnote{All citations of Donatus’ \textit{Ars Maior} come from Schönberger's German translation \citep{Schönberger2009}. I am responsible for the English translations.}

\begin{sloppypar}
Donatus moves on to discuss larger units of language, which he also approaches from the perspective of sound. He names the three larger units of language: the cola, the commata, and the sentence (1.6., page 39 in Schönberger), but defines them only in terms of the three types of “breaks” or “interruptions” that occur between them: \textit{die Pause}, \textit{der Einschnitt}, and \textit{die Atempause} (page 37). The largest units of speech, i.e., sentences or periods, are divided by \textit{Pausen} ‘pauses’ (1.6.1.1). Donatus offers no comment on how to define sentences, but we know that in general the Greeks and Romans defined them in terms of sound and meaning \citep[292]{Harrison2007}. Recalling my discussion in \sectref{sec:6.2.3}, for example, Aristotle offers a definition of periods in chapter 9 of his third book in \textit{On Rhetoric}, though it is unsatisfyingly vague by modern standards. According to Aristotle, a period contains a complete thought and has a clear beginning and end. He also, however, indicates a rhythmic definition.
\end{sloppypar}

\begin{quote}
I call a \textit{period} an expression having a beginning and an end in itself and a magnitude early taken in at a glance. Such a style is pleasant and easily understood, pleasant because opposed to the unlimited and because the hearer always thinks he has hold of something, in that it is always limited by itself, whereas to have nothing to foresee or attain is unpleasant. And it is easily understood because easily retained in the mind. This is because utterance in periods has number, which is the most easily retained thing. Thus, all people remember verse better than prose; for it has number by which it is measured. But a period should also be complete in thought and not cut off, as it is in iambic lines.
\end{quote}

\noindent As Kennedy \citep[214]{Kennedy2007} notes in his comments on this passage, the period is a rhythmical unit in that it “has magnitude, is limited, and has number.” Its rhythmic nature prompts Aristotle to compare the period with a line of verse. The period also contains a “complete thought,” which indicates a semantic definition, though admittedly not a precise one. The \textit{Atempause}, literally ‘breath-pause’ (1.6.1.3, \citealt[39]{Schönberger2010}) is the “intermediate” break that occurs when “about half of the sentence is left, but one needs to take a breath.” Finally, the \textit{Einschnitt} ‘break’ (1.6.1.2, \citealt[39]{Schönberger2010}) occurs when not much sentence is left, yet the speaker or reader still finds it necessary to pause their speech. Donatus finally links these intermediate breaks to the cola and commata, noting that the cola whose boundaries are marked by the \textit{Atempause} is the larger unit within the period, while the commata, which is demarcated by the \textit{Einschnitt}, is the smaller unit.

\begin{sloppypar}
\citet[292]{Harrison2007} indicates that Donatus’ treatment of these larger linguistic units largely represents how the classical grammarians and rhetoricians thought about clausal syntax. This is to say that their approach does not much resemble modern approaches to the same, which is perhaps why works like \citet{Luhtala1993} state that the Carolingians were largely bereft of any classical syntactic treatises, with the notable exception of Priscian. Donatus’ discussion of clausal syntax is not a syntactic approach by our standards, but it does indicate how people approached the topic of larger linguistic units: they saw them as a prosodic rather than structural phenomenon. Phrased another way, classical grammarians \textit{did} write about syntax. They simply approached the topic in a way that reflected a more conceptually oral and less literate orientation toward language. Such an orientation makes sense if one considers the possibility that these discussions are part of the ausbau of, in Donatus’ case, Latin and that Latin was also still, importantly, an oral phenomenon in the communities in which it was used. Just as Donatus’ definitions of the smaller units of language were prosodic so too are his definitions of the larger units, and he highlights their intonational, not structural, boundaries. So, the fact that constituents like clauses and phrases were not self-evident to classical grammarians, does not mean that they did not discuss syntax. It simply means that they approached syntax in terms that made more sense to them than they do to modern scholars.
\end{sloppypar}

Priscian’s volume on syntax is a step closer to, but still falls well short of, a more modern understanding of syntax. He approaches the grammar of Latin in the traditional way. Like Donatus, for example, Priscian identifies the smallest units of language as sounds and syllables (see pages 27--29 of Schönberger’s translation, \citealt{Priscian2010}). He also discusses the different parts of speech, as other classical treatments do. Priscian’s terminology for the larger units of language are vague: he refers to the sentence as \textit{oratio}, which can be translated as anything from ‘speech,’ ‘discourse,’ or ‘language’ to ‘clause’ or ‘sentence.’ Just as Donatus’ \textit{period}, \textit{cola}, and \textit{commata} cannot be reliably linked to any modern notions of “clause” or “sentence,” Priscian’s \textit{orationis} is a general referent pointing to larger units of language. However, Priscian also expresses interest in determining \textit{how} the parts of speech link together to form a complete “sentence,” which he asserts is a necessary discourse for the “composition of all authors” (\textit{expositionem}, \citealt[27]{Schönberger2010}). How do parts of speech connect in a sentence then? To answer this question, the modern syntactician would surely talk about grammatical relations. For example, nouns take on different functions in a clause: e.g., subject, object, and indirect object. Grammatical relations is how modern syntacticians understand cohesion within utterances. The more cohesive the utterance, the more people are processing all uttered constituents in terms of the role they play within a particular discourse. Priscian, however, finds cohesion in utterances through philosophical concepts, in particular Aristotelian categories \citep[146]{Luhtala1993}.

  Priscian’s analysis leads him to an understanding of the clause/sentence~-- it is here where the German word \textit{Satz}, which refers to both, is helpful~-- as minimally comprising nouns and verbs. He interprets the noun, in particular, as having primacy among the parts of speech, followed by the verb (pages 67--69 in the Schönberger translation). As \citet[146--147]{Luhtala1993} states, the noun’s importance stems from the fact that substance must precede action; thus, nouns must exist before verbs, and the verb is an \textit{Akzident} or property of the substance, i.e., the noun. These ideas precede and foreshadow those of grammatical relations, subject-verb agreement, and clauses minimally comprising subjects and predicates, and one can see how these more modern concepts might emerge from Priscian’s treatment. \citet{Luhthala1993}, in fact, argues that Carolingian commenters on Priscian and Donatus arrive at the concept of subjects and predicates in the later tenth century. However, when early German’s first literizations start to appear, the state of grammatical thought in Carolingian Europe would not have included any notion of the functional, grammatical relations that existed between different parts of speech (see Schönberger's commentary in \citet[495]{Schönberger2010}.

Now imagine an early German literizer like Otfrid von Weissenburg consulting the \textit{Ars Maior}, the seventeenth volume of Priscian’s \textit{Institutiones Grammaticae}, or any of the commentaries of these works for guidance on creating a \textit{scriptus}. To remind the reader, I see Otfrid’s task as one of how to effect syntactic ausbau or, in other words, how to create a more coherent variety of his spoken vernacular so that it can be more functional in the written medium. As I argued in \chapref{sec:chap:5}, Otfrid would like to move his \textit{scriptus} away from the trappings of the oral tradition and into the direction of becoming a great literary language, like Latin and Greek. This intention suggests that Otfrid was more inclined to pay attention to the linguistic prescriptions laid down by classical grammarians when reconceptualizing his vernacular for the graphic medium. We could hypothesize, then, that Otfrid would have aimed to avoid the unperiodic syntax of the oral tradition with its layered elaboration, which, as I demonstrated in \chapref{sec:chap:6}, does not yield clearly defined \textit{Sätze} ‘clauses', or ‘sentences.’ The grammars offer little on how to create coherence in language, however, and say nothing of the sort of grammatical coherence that a modern writer expects in a written language. What Otfrid had access to instead was ideas of prosodically and semantically defined discourse units along the lines of Donatus: \textit{Sätze} comprise complete thoughts and are prosodically demarcated. Along the lines of Priscian, he would have learned that \textit{Sätze} need nouns and verbs, first and foremost. Missing is any discussion of these parts of speech (or others) as syntactic categories, much less as sophisticated of a description of the \textit{Satz} as one finds in, say, the topological field model. Also missing is any discussion of grammatical relations that would have helped Otfrid to understand words as syntactic constituents and, thus, the ways in which they relate to one another in an utterance. With little understanding of the syntax of a clause as we understand it in modern terms, it would be entirely unreasonable to assume that any German literizer is conceptualizing the clause complex or sentence as a combination of clauses. If part of the challenge of early literization processes lies in reconceptualizing an orally organized language as structural units that function better in the graphic medium, then the guidance of the Latin grammars would not have directed writers anywhere close to the well-formed, grammatically coherent structures that we have been supposedly finding in the data for the past several decades.

In order to illustrate how classical prescriptions surrounding the creation of \textit{Sätze} are evident in Otfrid’s \textit{scriptus}, I examine a short excerpt from the poet’s retelling of the wedding feast in Cana, where Jesus performs his first miracle. Here is the corresponding Bible verse that relates this story. They comprise only two lines, John 2: 3--4.\footnote{{This translation is the American Standard Version of the Bible. You can access it here:} \url{https://www.biblegateway.com/passage/?search=john+2 & version=ASV}. }

\begin{quote}
\begin{tabularx}{\linewidth}{@{}lQ@{}}
3 & And when the wine failed, the mother of Jesus saith unto him, they have no wine.\\
4 & And Jesus saith unto her, Woman, what have I to do with thee? Mine hour is not yet come.
\end{tabularx}
\end{quote}

\noindent Tatian’s retelling of these lines in his Gospel Harmony hews closely to John’s version. In \REF{ex:7:9}, I gloss both the Latin and the German so that readers can compare the two.

\eabox{%9
\label{ex:7:9}(81, 18--23 in \citegen{Masser1994} edition of the Tatian)\\
\begin{tabular}{@{}ll@{}}
\& deficiente vino.’          &  thó ziganganemo themo uúine            \\
dicit mater Ihesu ad eum.,    &   quad thes heilantes muoter zi imo.    \\
vinum non habent.,            & sie nihabent uúin                       \\
\& dicit ei Ihesus.,          &  thó quad iru ther heilant.             \\
Quid tibi \& mihi est, mulier?&      uuaz ist thih thes inti mih, uúib. \\
nondum venit hora mea.,       &  noh nú ni quam mín zít.                \\
\end{tabular}\medskip\\

\gll \& deficiente         vino\\
and   deficient.\textsc{ab.sg.partc}  wine.\textsc{ab.sg}\\

\gll dicit    mater  Ihesu   ad eum\\
says   mother   Jesus.\textsc{gen}     to him\\

\gll vinum  non    habent\\
wine  \textsc{neg}    have.\textsc{3.pl.pres}\\

\gll \& dicit    ei        Ihesus\\
and  says   this.\textsc{nom}    Jesus.\textsc{nom}\\

\gll Quid  tibi     \& mihi   est  mulier\\
what   to.you   and  to.me   is   woman\\

\gll nondum   venit      hora  mea\\
not.yet   comes.\textsc{perf.ind}  hour my     \\

\gll thó    ziganganemo           themo     uúine\\
then  {having exhausted}.\textsc{part.dat.sg.m}   the.\textsc{dat.sg}   wine.\textsc{dat.sg}\\

\gll quad  thes   heilantes   muoter   zi  imo\\
said   the.\textsc{gen}   savior.\textsc{gen}   mother   to  him\\

\gll sie     nihabent   uúin\\
they   \textsc{neg}{}.have   wine.\textsc{acc}\\

\gll thó    quad   iru     ther      heilant\\
then  said   her.\textsc{dat}   the.\textsc{nom}   savior.\textsc{nom}\\

\gll uuaz  ist  thih     thes     inti  mih     uúib\\
what  is   you.\textsc{acc}   this.\textsc{gen}    and   me.\textsc{acc}  woman\\

\gll noh    nú     ni     quam    mín   zít\\
yet    now   \textsc{neg}    came.\textsc{pret}  my     time/hour\\

\glt ‘Then having exhausted the wine, the savior’s mother said to him, they do not have wine. Then the savior said to her, what is this to you and me, woman? My time has not yet come.’
}

\noindent Compare now Tatian’s version to Otfrid’s. Note how the latter author reconceptualizes the passage from his own particular point of view. He expands the raw material in a way that emphasized what he thinks is important, which includes how he thinks the reader should interpret the lines.

\begin{table}
\caption{Otfrid, II 8, 11--24}
\label{tab:7.2}
\begin{tabularx}{\textwidth}{lQQ}
\lsptoprule
 & a-verse &  {b-verse}\\
 \midrule
11 & Thó zigiang thes lídes & {joh brást in thar thes wínes}\\
12 & María thaz bihúgita & {joh Kríste si iz giságeta}\\
13 & Ih scal thir ságen, min kínd & {then híon filu hébig thing}\\
14 & theih míthon ouh nu wésta & {thes wínes ist in brésta}\\
15 & Spráh tho zi iru súazo & {ther ira sún zeizo}\\
16 & sconen wórton ubar ál & {so sun zi múater scal}\\
17 & Wib, ih zéllu thir ein & {waz drífit sulih zi úns zuein}\\
18 & ni quam min zít noh so frám &  {theih óuge}   weih fon thír nam\\
19 & Sar so tház irscínit & {waz mih fon thír rinit}\\
20 & so ist thir állan then dag & {thaz hérza filu ríuag}\\
21 & Thaz thu zi mír nu quáti & {inti eina klága es dati}\\
22 & mit gótkundlichen ráchon & {scal man súlih machon}\\
23 & Thiu muater hórta thaz tho thár & {si wéssa thoh in álawar}\\
24 & thaz íru thiu sin gúati & {nirzígi thes siu báti}\\
\lspbottomrule
\end{tabularx}
\end{table}

\ea%10
    \label{ex:7:10}
Glossed version of II 8, 11--24\\

\gll Thó     zigiang   {thes lídes}     joh  brást    in       thar thes wínes\\
Then  {ran out}   {the wine.\textsc{gen}}   and  {was lacking}   them.\textsc{dat}   there the wine.\textsc{gen}\\

\gll María  thaz  bihúgita  joh Kríste      si     iz    giságeta   \\ 
Mary   that  noticed   and Christ.\textsc{dat}  she  it   said \\

\gll Ih   scal    thir   ságen  min  kínd then    híon filu  hébig        thing      theih   míthon   ouh  nu    wésta\\
I   must  you.\textsc{dat}     say     my  child the.\textsc{acc}  very   much uncomfortable.\textsc{acc}   situation.\textsc{acc}   that.I   just    also  now   discovered\\

\gll thes  wínes     ist     in       brésta  \\
the wine.\textsc{gen.sg}  is   them.\textsc{dat.pl}  lacking  \\

\gll Spráh tho   zi iru súazo    ther  ira  sún  zeizo\\
spoke then  to her sweetly  \textsc{det}  her  son  precious\\

\gll sconen wórton     {ubar ál}       so sun   zi múater   scal\\
beautiful words.\textsc{dat.pl}  {in every respect}  as son  to mother  ought\\

\gll Wib    ih   zéllu thir           ein      waz      drífit     sulih     zi úns  zuein\\
      Woman  I  say    you.\textsc{dat}   {one thing}   what  {to do with}  {such a thing}  to us   two\\

\gll Ni  quam  min zít  noh  so frám   theih   óuge         weih   fon thír    nam\\
\textsc{neg}   came   my hour   yet  so soon  that.I  show.\textsc{subj}   what.I  from you   took\\

\gll Sar so    tház  irscínit       waz    mih  fon    thír    rinit    so   ist   thir állan then dag-\textsc{acc}  thaz hérza  filu ríuag \\
{As soon} as   that  {is made apparent}  what  me  from   you.\textsc{dat}  touches   so  is  you.\textsc{dat} all the day.\textsc{acc}  the heart     very troubled \\

\gll     Thaz  thu  zi mír  nu   quáti     inti eina klága     es     dati \\
  That   you  to me   now   said   and a complaint.\textsc{acc}  it.\textsc{gen}  made  \\

\gll mit gótkundlichen ráchon  scal    man  súlih  machon\\
through godly intervention  must  one   such  do\\

\gll Thiu muater   hórta   thaz  tho  thár      si  wéssa   thoh   in   álawar\\
The mother  heard  that  then  there  she   knew  however   with  certainty\\

\gll thaz íru  thiu sin gúati    nirzígi         thes    siu  báti\\
that  her   \textsc{det} his goodness   \textsc{neg}.deny.\textsc{pret.subj}  \textsc{det.gen}  she   asked.\textsc{pret.subj}\\
    \z

\ea%11
    \label{ex:7:11}
Translation of II 8, 11--24

Then the fruit-wine ran out and they were lacking wine. Mary noticed this, and she mentioned it to Christ: I have to tell you, my child, about the very uncomfortable situation that I just now also discovered. They have run out of wine. Her precious son then spoke sweetly to her with beautiful words, in every respect, as a son should (speak) to his mother: woman, I say to you one thing. What does such a thing have to do with the both of us? My hour has not yet come so soon that I may demonstrate what I have received from you. As soon as that is made apparent, what touches me from you (i.e., what I have from you), so your heart will be troubled for all days. That which you now say to me and you make a complaint about it, one must do such a thing through Godly intervention. The mother heard this, but she knew with certainty that his goodness would not deny her that for which she asked.
    \z

\noindent Before turning to structure, consider the passage from the perspective of content: Otfrid elaborates the original material, extending its length considerably. While the \textit{Hêliand} poet elaborates material through layered elaboration, as I discussed in \chapref{sec:chap:6}, Otfrid’s version is made longer through additions that I interpret as the poet attempting to contextualize Jesus’ seemingly brusque reply to his mother. That is, the response, “what does this situation have to do with me, woman?” seems rude. Otfrid, therefore, is at pains to make clear that Jesus was responding in an entirely appropriate manner: he is a “precious son,” who answers with “beautiful words” that are warranted in “every respect.” He also softens Jesus’ words by having him offer Mary a few words of explanation: the revelation of my power will lead to your sadness. That is, Jesus performing his first miracle and manifesting his divinity for all to see sets in motion the prophesied series of events that leads to his crucifixion.

Let us move on now to how Otfrid structured his larger discourse units, which I will simply call \textit{Sätze} (‘sentences/clauses,’ singular \textit{Satz}) as an acknowledgment of the fact that one should not assume the presence of neatly delineated clauses and clause complexes in an early \textit{scriptus} like that of the \textit{Evangelienbuch}. One aspect of his \textit{scriptus} that is consistent with classical prescriptions encouraging periodic syntax and Otfrid’s desire to move away from the oral tradition is the boundedness of his \textit{Sätze}. That is, he generally reinforces the boundaries of a \textit{Satz} by containing it within the prosodic unit of the verse-IU. Note how in \tabref{tab:7.2} most verses contain a \textit{Satz}, or in modern linguistic terms, a clause. I have reproduced \tabref{tab:7.2} in a new table, \tabref{tab:7.3}, but with alternating shading of the cells to indicate where one \textit{Satz} begins and another ends. For example, 11a contains one \textit{Satz} and 11b a new \textit{Satz}, while another \textit{Satz} begins in 15a and ends with 16a.

\begin{table}
\caption{Otfrid, II 8, 11--24, with shading}
\label{tab:7.3}


\begin{tabularx}{\textwidth}{lllQ}
\lsptoprule
 & a-verse &   b-verse\\
 \midrule
11 & Thó zigiang thes lídes                 & \multicolumn{2}{l}{\shadecell joh brást in thar thes wínes}\\
12 & María thaz bihúgita                    & \multicolumn{2}{l}{\shadecell joh Kríste si iz giságeta}\\
13 & Ih scal thir ságen, min kínd           & \multicolumn{2}{l}{then híon filu hébig thing}\\
14 & \shadecell theih míthon ouh nu wésta   & \multicolumn{2}{l}{thes wínes ist in brésta}\\
15 & \shadecell Spráh tho zi iru súazo      & \multicolumn{2}{l}{\shadecell  ther ira sún zeizo}\\
16 & \shadecell sconen wórton ubar ál       & \multicolumn{2}{l}{so sun zi múater scal}\\
17 & \shadecell Wib, ih zéllu thir ein      & \multicolumn{2}{l}{waz drífit sulih zi úns zuein}\\
18 & \shadecell ni quam min zít noh so frám & theih óuge   & \shadecell  weih fon thír nam\\
19 & Sar so tház irscínit                   & \multicolumn{2}{l}{\shadecell waz mih fon thír rinit}\\
20 & so ist thir állan then dag             & \multicolumn{2}{l}{thaz hérza filu ríuag}\\
21 & \shadecell Thaz thu zi mír nu quáti    & \multicolumn{2}{l}{inti eina klága es dati}\\
22 & \shadecell mit gótkundlichen ráchon    & \multicolumn{2}{l}{\shadecell scal man súlih machon}\\
23 & Thiu muater hórta thaz tho thár        & \multicolumn{2}{l}{\shadecell si wéssa thoh in álawar}\\
24 & thaz íru thiu sin gúati                &  nirzígi    &  \shadecell  thes siu báti\\
\lspbottomrule
\end{tabularx}
\end{table}

In this short excerpt, there are sixteen verse-IUs that contain a \textit{Satz} and feature a finite verb (11a, 11b, 12a, 12b, 14a, 14b, 16b, 17a, 17b, 18a, 19a, 19b, 21a, 21b, 23a, and 23b). There are three \textit{Sätze} that occupy two verse-IUs; in these cases, Otfrid again uses the prosodic boundaries within the poetic structure, i.e., the line, to reinforce the structural boundary of the \textit{Satz}. Two of the remaining \textit{Sätze} occur in the same verse-IU, 18b, where the poetic structure still serves as something of a structural boundary. The \textit{Satz} in 24a stretches across a prosodic boundary, however, though the line is filled out by a second \textit{Satz} whose ending corresponds with the end of the verse-IU.

It is particularly lines 15a to 16a where we see a relationship between prosodic and structural unit that is more consistent with the orally influenced \textit{scriptus} of the \textit{Hêliand}. Here Otfrid does not exercise the same discipline he shows elsewhere by confining \textit{Sätze} to prosodic units and instead builds one \textit{Satz} that stretches across multiple verse-IUs. The resultant structure is more reminiscent of the \textit{Hêliand}, especially with its full subject noun phrase occurring in another verse-IU and the reverbalization of \textit{sprach} (‘spoke') with \textit{sconen worten} (‘with beautiful words') and the extra modifying \textit{ubar al} (‘regarding this'). The \textit{Satz}’s effect is more disjointed and fragmented than the \textit{Sätze} that surround it. I see Otfrid’s approach to building \textit{Sätze} within the prosodic confines of the verse-IU as the logical extension of \citegen{Somers2021b} argument, which I discussed in \chapref{sec:chap:5}. That is, Otfrid looks to prescribe good Frankish through the imposition of a regular metrical pattern. In other words, he uses his verses and lines to effect boundedness in his \textit{scriptus}’ structures and create a more “periodic” syntax. This inconsistency in Otfrid’s commitment to creating \textit{Sätze}, that is, the fact that there are \textit{Sätze} that are not built within the confines of his metrical units, illustrates another principle of literization and ausbau that I discussed in \chapref{sec:chap:6}, but also the current chapter. Namely, consistency in early German written structures depends on the accomplishments of fallible humans, who were engaged in a difficult task~-- creating a \textit{scriptus} for an oral vernacular~-- in the early medieval context of Carolingian Europe.

To reiterate a point I made earlier in this chapter, Otfrid’s data also exhibit collocations that contain familiar-looking subjects and predicates. I do not intend to deny their presence. I also do not deny the presence of surface order patterns or that these patterns probably characterized early German spoken varieties. Instead, I reject the conclusion that those structural patterns are the only things that are worth investigating or that they can account for the full linguistic phenomenon of the early German \textit{scripti}. My argument is, in fact, that we cannot understand the German language’s earliest history, unless we treat its attestations as sociocultural artifacts effected by the transformative processes of literization and ausbau, in particular.

\section{If you look for well-formed clauses and sentences, you will create them}\label{sec:7.3}

Let us now follow up on the discussion of the introduction to the \textit{Hildebrandslied} that we began in \chapref{sec:chap:6}. There I demonstrated how this poem, one of the few instances of textualized orality in the early German corpus, evinces the layered elaboration that I argued was characteristic of oral varieties of distance. For example, it features verse-IUs that reverbalize and elaborate on the idea of the two referents, Hildebrand and Hadubrand. The poet also relies more on parallelism to implicitly link IUs together, rather than explicit connectors.

\ea%12
    \label{ex:7:12}
1  \tab Ik gihorta ðat seggen\\
2  \tab ðat sih urhettun    \qquad    ænon muotin\\
3  \tab hiltibraht enti haðubrant   \qquad     untar heriun tuem\\
4  \tab sunufatarungo  \qquad      iro saro rihtun\\
5  \tab garutun sê iro guðhamun   \qquad     gurtun sih iro suert ana\\
6  \tab helidos ubar hringa   \qquad     do sie to dero hiltiu ritun\\
7  \tab hiltibraht gimahalta heribrantes sunu    \qquad    her uuas heroro man\\
8  \tab ferahes frotoro  \qquad      her fragen gistuont\\
9  \tab fohem uuortum   \qquad     hwer sin fater wari\\
10 \tab fireo in folche\\\medskip

1a \tab \gll Ik  gihorta  ðat  seggen\\
        I  heard   that  tell\\

2a \tab \gll ðat     sih       urhettun\\
        that   \textsc{refl.pro}   challengers\\

2b \tab \gll ænon   muotin\\
        one     met\\

3a \tab \gll hiltibraht  enti    haðubrant\\
        Hiltibrant   and    Hadubrant\\

3b \tab \gll untar     heriun  tuem\\
        between   two     armies\\

4a \tab \gll sunufatarungo\\
        son.and.father\\

4b \tab \gll iro    saro   rihtun\\
        their   armor  prepared\\

5a \tab \gll garutun  sê    iro     guðhamun\\
        readied  they    their   {fighting clothes}\\

5b \tab \gll gurtun  sih        iro     suert  ana\\
        belted   \textsc{refl.pro}  their  swords   on\\

6a \tab \gll helidos   ubar  hringa\\
        heroes  over   rings\\

6b \tab \gll do     sie    to  dero   hiltiu  ritun\\
        \textsc{part}  they   to  the    battle  rode\\

7a \tab \gll hiltibraht   gimahalta   heribrantes     sunu\\
        Hildibrant  said     Heribrant.\textsc{gen}  son\\

7b \tab \gll her  uuas   heroro       man\\
        he   was   the more senior   man\\

8a \tab \gll ferahes     frotoro\\
        life.\textsc{gen}    {the wiser one}\\

8b \tab \gll her  fragen   gistuont\\
        he   began   to ask\\

9a \tab \gll fohem    uuortum\\
        few.\textsc{dat.pl}  words.\textsc{dat.pl}\\

9b \tab \gll hwer  sin  fater  wari\\
        who   his  father   was.\textsc{pret.subj}\\

10a \tab \gll fireo       in     folche\\
         people.\textsc{gen.pl}  among   the host.\textsc{dat.sg}\\

\glt ‘I have heard tell how two challengers met alone, Hildebrant and Hadubrand, between two armies, son and father, prepared their armor, they readied their fighting clothes, belted on their swords, heroes over chainmail, then/when they rode into battle. Hildebrand said, Heribrand’s son, he was the more senior man, the wiser one of life, he began to ask with few words, who his father was of the people among the host.’
    \z

\begin{sloppypar}
\noindent One can certainly carve out from these lines clauses and clause complexes, though this exercise obscures the evidence of an orally organized system of layered elaboration and instead highlights what the researcher themselves made from the data. So, one recruits \textit{sunufatarungo} (4a) into the following verb phrase, \textit{iro rihtun} (4b) as the grammatical subject, because clauses are not complete without subjects. This analysis, however, undermines the poet’s layering of elaborated, reverbalized noun phrases of \textit{urhettun} that occupy a-verses: (3a), (4a), and (6a). If one also pulls \textit{helidos ubar hringa} (6a) into the preceding verb phrase, \textit{gurtun sih iro suert ana} (5b), the poet’s parallelism between a series of noun phrases referring to Hildebrand and Hadubrand practically disappears. If one has a modern German well-formedness in mind, which requires that complete clauses have subjects, one might even think that these clauses must reflect the poet’s underlying competence; that is, these clauses were what the poet intended, and we modern researchers have uncovered this intent.
\end{sloppypar}

According to this structurally focused view, the \textit{scriptus} seems to result from the literizer mapping the clauses and sentences created by their grammar onto poetic units. It becomes more difficult to consider other possibilities like the following. Are the verse-IUs themselves the building blocks of this extended narrative? Might they be organized along the lines of an orally shaped system of layered elaboration? And finally, does this system yield clause-like and sentence-like utterances that literizers eventually turn into actual clauses and sentences once they reconceptualize orally organized utterances as well-formed written utterances, structured around the principle of \textit{Satzförmigkeit}? One can pick out clause complexes in the excerpt as well, though they, like the clauses I just described, manifest primarily through the application of modern well-formedness norms. Of interest here is the asymmetrical distribution of finite verbs in main and subordinate clauses that scholars have observed for modern literized languages like German and Dutch. On the basis of surface similarities between these languages and older historical varieties, scholars conclude that similar grammars underlie both. For example, \citet{Kemenade1987} applies Government and Binding Theory to the analysis of Old English clause structure and proposes that its clauses are surface realizations of a Subject-Object-Verb template. In main clauses, the absence of a lexical subordinator in the complementizer position prompts a fronting of the finite verb from its original location at the end of the clause to COMP’s clause-second slot. In subordinate clauses, the finite verb remains in its underlying clause-final position because the COMP slot is already filled. Thus, the finite verb and lexical subordinator are in complementary distribution to one another. \citet[95]{Erickson1997} analyzes Old Saxon clause structure along similar lines, pointing out that it too has clauses that seem to exhibit the same distributional pattern as modern German and Dutch. On this basis, Erickson, like van Kemenade, concludes that similar grammars underlie all of these languages (page 104).

If we take our cue from these analyses, a particular interpretation of the IU in (6b), \textit{do sie to dero hiltiu ritun}, becomes apparent. The IU begins with a deictic particle \textit{dô}, which links the IU to the events described in surrounding IUs. However, the initial deictic, especially when combined with the IU-final finite verb, makes the sequence look very much like a clause, particularly a subordinate clause. If (6b) contains a subordinate clause, then the researcher is obliged to find a main clause on which the subordinate clause is dependent. Yet, finding its main clause mate is not straightforward. Part of the problem is that none of the many definitions scholars have assigned to \textit{dô} match this token all that well. Look the word up in \citegen{Schützeichel1974} \textit{Althochdeutsches Wörterbuch}, ‘Old High German dictionary,’ and one finds adverb and subordinator translations.

\begin{table}
\caption{The many translations of \textit{dô}}
\label{tab:7.4}
\begin{tabular}{l@{~}l l@{~}l}
\lsptoprule
\multicolumn{2}{l}{Adverb} & \multicolumn{2}{l}{Subordinator}\\
\midrule
damals   &  ‘then, at the time’   & als &  ‘as, when, while’\\
dann     &  ‘then’                & während &  ‘while, whereas, when’\\
hingegen &  ‘however’             & nachdem &  ‘after, whereas’\\
daher    &  ‘thus, so, therefore’ & weil &  ‘because, since’\\
darauf   &  ‘after that’ & dadurch dass &  ‘as a result of’\\
doch     &  ‘still, after all’ & obgleich &  ‘although’\\
da       &  ‘there, here’ & da &  ‘given that, because, since’\\
\lspbottomrule
\end{tabular}
\end{table}


\noindent Thus, any translation of the clause beginning with \textit{dô} would have to take the ambiguity in its status: ‘Son and father, (they) prepared their armor (and) readied their fighting clothes, they belted on their swords, heroes over rings, when (while? as? whereas? because? although? given that? after?) they rode into battle.' Selecting from the buffet of subordinate conjunction translations on offer in \tabref{tab:7.4}, one realizes that none of the temporal translations of \textit{dô} are satisfying: they imply an order of operations that makes little sense in the context. Surely, one arms oneself before riding into battle, not while or after. The causal translations are better, e.g., ‘because,’ but not as good as the temporal adverbial translation: son and father, they prepared their armor … then rode into battle, thereupon Hildebrand speaks. Alternatively, one might instead connect (6b) to the following verse-IU: ‘after they rode to battle, Hildebrand, the son of Heribrand, spoke.' That might be the best option if one insists on analyzing (6b) as a subordinate clause. In any case, note how different analyses of clause complexes affect the analysis of individual clauses.

\citet[141--144]{Höder2010} indicates another way of looking at connectors like \textit{dô}, however. The author describes morphemes in historical varieties of Swedish that are reminiscent of these early German connectors. One of his examples is the polysemous subordinator \textit{än}, which could be a conditional conjunction (‘if’), an additive conjunction (‘and [on the other hand]),’ or an adversative conjunction (‘but,' ‘whereas’), among other things (page 143). Höder argues on page 144 that monomorphemic, polysemous subordinators like \textit{än} were probably “semantically neutral” and that their function lay in simply marking two clauses as related, while the particular type or meaning of the connection would be disambiguated through context. Similar arguments have been made, Höder notes, for the oldest attestations of Old Swedish \citep{Kotcheva2002} and for the oldest Germanic languages generally \citep{Braunmüller1995}. The latter work identifies Gothic \textit{jah} and Old High German \textit{ouh} as examples. Höder distinguishes between “vagueness” and “ambiguity,” arguing that such subordinators are more semantically vague than they are ambiguous (page 144). I interpret this statement as referring to how speakers process polysemous subordinators. To state that the morphemes are vague indicates that, though they are semantically neutral, language users do not interpret them as ambiguous and can glean a specific meaning from context.

I propose that Höder’s argument regarding the difference between vagueness and ambiguity leads to an important point about the validity of disambiguation analyses. To begin with, Höder’s claim that speakers could disambiguate semantically vague conjunctions strikes me as appropriate. For example, research on spontaneously spoken language supports this claim in that, though spoken language of immediacy features much more vagueness than, say, a written language of distance, language users do not necessarily have any issues processing its more disjointed and fragmented expressions (see \sectref{sec:3:3.2.2}). What strikes me as inappropriate and having the potential to yield anachronistic results is using disambiguation analyses as the means of supposedly uncovering underlying competence. To phrase this a different way, speakers are able to process without issue semantically or functionally vague constructions in their context. So, while orally organized linguistic production is vague, those who process it in the phonic medium do not find it to be ambiguous. Why then should a researcher assume that their own disambiguation analysis represents a synchronic reality for the interlocutors who had no difficulties with the vagueness in the first place? The vagueness, instead, is a problem for the researcher who insists that uncovering the underlying structure must be the goal of their investigation. It also become a problem for the literizer, who wants to move their \textit{scriptus} away from the organizational systems of elaborated orality. This person will have to identify where the oral vagueness yields graphic ambiguity and engage in ausbau in order to make the semantic and grammatical relationships in linguistic production more explicit and specific.

In a sense, the glossaries for historical varieties like Schützeichel’s for Old High German can give the modern scholar, especially one with a structuralist bent, the impression that the vague forms from early \textit{scripti} have been disambiguated to the extent that we now know the different entries that existed in the imagined Old High German speaker’s mental lexicon. Consider the Low German equivalent of \textit{dô}, \textit{thô}: it is comprehensively glossed in \citegen{Sehrt1925} glossary for the \textit{Hêliand} and features multiple sub-entries indicating that it is alternatively a lexical adverb, discourse adverb, temporal relative subordinator, or correlative particle. Each function is also associated with many meanings. For example, adverbial \textit{thô} can be the lexical ‘then’ or the more semantically empty discourse particle along the lines of modern German’s \textit{nun} or \textit{also} (‘so,’ ‘and so’). Each entry, furthermore, contains a list of every occurrence of \textit{thô} that the analyzer deemed to be an example of that usage. This glossary presentation gives the impression that each use of \textit{thô} in the \textit{Hêliand} is associated with a discrete syntactic category, i.e., adverb or subordinating conjunction, and with a particular meaning. It also implies that, though an individual morpheme might seem ambiguous to you, the modern reader~-- perhaps this is why you consulted the glossary in the first place~-- someone has already sorted out what the poet intended in that instance.

Let us bring the discussion back to \textit{dô} in (6b)’s \textit{do sie to dero hiltiu ritun}. As I indicated just above, it is possible to build a clause complex around this one clause that fits around a modern written well-formedness. This analysis requires, however, that you disambiguate the \textit{dô}{}-clause by selecting from the possible translations that fit best and adjust surrounding clauses accordingly. Höder’s discussion of \textit{än}, however, raises another possibility with his idea of the semantically neutral morpheme whose function is to mark two clauses as related. Applying this notion to \textit{dô}, which could also function as an adverb, I conclude that it is a polysemous deictic morpheme whose function was to either link ideas presented in discourse in space and time, i.e., as an adverbial, or connect IUs to each other in discourse. Listeners might have resolved the morpheme’s vagueness in one of the ways I described above, namely interpreted the \textit{dô} as an adverbial or a clausal connector. The poet might have intended one or the other~-- or perhaps something the modern reader cannot imagine.

What I would like to advance here is not that one analysis is definitively right and the other wrong. In line with my arguments against the deficit approach in \chapref{sec:chap:2}, I maintain instead that it is impossible for a modern scholar to do anything more than speculate as to whether the clause in (6b) was a main or subordinate clause in the poet’s or the listeners’ minds. In way, the question is moot because the \textit{scriptus} represents an orally organized linguistic production. Similarly, there is no principled way to show that the poet intended \textit{sunufatarungo} and \textit{helidos ubar hringa} to be subjects of their adjacent verb phrases and not parallelisms that elaborate the poem’s protagonists, while only loosely connecting them to adjacent verb phrases. Instead, my proposal is that we, in such cases, leave aside these unanswerable and irrelevant questions about underlying structure. This argument suggests then that scholars spend less energy engaging in disambiguation analyses when investigating the early literizations of a vernacular and take better care not to project their modern notions of well-formedness onto historical data, where they cannot be applicable.

It is also the case that much of the literature published on historical Germanic syntax has approached the data as a disambiguation project. \citet{SomersDubenion-Smith2014} is an example of this trend in that the authors begin their analysis with the assumption of the complementarity of finite verb and complementizer and then identify all clauses in the \textit{Hêliand}{}-based dataset that supposedly confirm its existence as a cognitively real pattern. \citegen{Linde2009} analysis of Old Saxon is similar in that the author begins with the assumption of a modern-like complementarity, while also seeking explanations for the clauses that remain resistant to such accounts. This study argues that information structural principles, that is, pragmatic factors, influenced the syntax of the deviating clauses and is consistent with other “two grammar” approaches to early Germanic syntax like \citet{Lenerz1985}.\footnote{{See \sectref{sec:2.2.2} for my discussion of the “two grammar” approach.}} Such analyses, which include \citet{Schlachter2012}, \citet{Lötscher2009}, \citet{PetrovaSolf2009}, among others, are also an iteration of the deficit approach, which attempts to separate the regular, authentic clauses that the underlying grammar created from the “inauthentic” clauses. The difference is that the two-grammar approach offers some account of deviating clauses, either by referencing an inherited or poetic syntactic system, while studies that are influenced by the deficit approach often do not.\footnote{{Many studies assume that poetic and inherited grammars are one and the same based on the supposition that poetic language is archaic language. See, for example, \citet[315]{Lötscher2009}.} }

Other studies, instead of offering disambiguating analyses, simply take for granted that their data contain clause complexes comprising clearly delineated main and subordinate clauses. Works like \citet{Robinson1997}, and \citet{Axel2007}, and \citet{FischerEtAl2000} assume at the outset that Old High German syntax (Old English syntax in the case of \citealt{FischerEtAl2000}) has main and subordinate clauses, whose boundaries are so clearly delineated that no discussion of how data were sifted into one category or the other is necessary. Works like this also do not discuss whether the disambiguation of data is appropriate for an early medieval \textit{scriptus} or recognize that disambiguation can be the means of an anachronistic projection of modern well-formedness norms onto historical data. \citet{Lötscher2009}, for example, begins his study on verb placement in Otfrid’s \textit{Evangelienbuch} with the assumption that sentences comprising clear main and subordinate clauses are present in the monk’s \textit{scriptus}. The author’s finding, then, that “Otfrid has the same differentiation between main clauses and subordinate clauses as later stages of German” (page 281) comes as no surprise. In the absence of a discussion of how he disambiguated main clause from subordinate, one is left to wonder if the author analyzed the data with modern norms of clausal well-formedness in mind. Walkden’s treatment of Old Saxon, in contrast, is more cautious (see \citealt[15--16]{Walkden2014} and \citealt[565]{Walkden2016}). The author acknowledges that clause type in the \textit{Hêliand} is often ambiguous and that disambiguation analyses run the risk of circularity and anachronism. However, he does not delve further into the question of how to resolve these problems, adopts the disambiguated clauses and sentences as they expressed in the punctuation of the \citet{Taeger1985} edition, and finally notes that these orthographic decisions may be wrong.

\begin{sloppypar}
In this way, scholars who disambiguate historical data, especially those from German’s earliest attestations, have run the unrecognized risk of anachronistically shaping these data in the image of their modern well-formedness norms. Their often tacit insistence that identifying the underlying structures of a speaker’s competence should be the primary goal of linguistic investigation is in part to blame for this state of affairs. Our neglecting of the literization process as a factor in language change has, I believe, prevented our approaching the early \textit{scripti} as linguistic artifacts shaped by sociocultural factors, rather than simply as data to be mined for traces of competence.
\end{sloppypar}

I conclude this chapter by briefly noting one contrastive study of early German and English syntax that stands out for not assuming the existence of delineated main and subordinate clauses: \citet{Cichosz2010}, which takes seriously \citegen{Mitchell1985} comments on what he refers to as clausal ambiguity in Old English syntax.\footnote{{Mitchell’s focus is Old English. However, the sources of ambiguity he highlights for Old English are identical to the sources of ambiguity across early Germanic} {\textit{scripti}} {and so, I have extrapolated his comments to early Germanic in general. \citet{Cichosz2010} also cites Mitchell’s work.} } \citet[769--773]{Mitchell1985} encourages scholars to admit what Walkden perhaps suspects, that is, that there are, in fact, no “infallible criteria” that can distinguish main and subordinate clauses in these early texts. Features, like word order, verb mood, presence or absence of a possible subordinator, can be suggestive but never conclusive because, Mitchell argues, subjecting early Germanic clauses to disambiguation is itself anachronistic. It assumes that early Germanic \textit{scripti} already had sentences as they are traditionally defined and that clauses introduced by vague morphemes, like \textit{thô} and \textit{thâr} in \tabref{tab:7.4}, should be disambiguated into one of the two clausal types, subordinate or main, that make up these sentences. Mitchell notes, however, that “ambiguity is frequently of importance only to the classifier” (page 773).

Taking her cue from \citet{Mitchell1985}, \citet{Cichosz2010} resolves to leave be the “irresolvable” ambiguities in her data. She argues that subordination was not “fully developed” in Old English and Old High German and so, none of the clauses that seem like subordinate clauses were actually subordinate clauses in the modern sense. They were “semi-subordinated” and “neither separated [to] nor connected [with]” some other clause (page 138). Cichosz supposes, also on page 138, that literization correlated with the development of an inventory of formal clause connectors, stating that this change was influenced by “Latin with its highly developed literary style.” In contrast to the current study, however, Cichosz does not indicate a causal link between literization and explicitly marked clause complexes, a consequence of that study’s different objectives. That is, \citegen{Cichosz2010} main interest is in establishing word order patterns across Germanic, with the desired end result being a description of a prehistoric Germanic grammar. This focus on identifying a Germanic competence is similar to the goals of the disambiguation studies I just discussed. However, it seems to me a step in the right direction that Cichosz did not resolve in her database what we both maintain were synchronically vague constructions. Where Cischosz and I part ways is in the significance of the vagueness. While she seems to make an evolutionary argument in stating that early medieval clauses had not yet developed fully, I reference literization and ausbau as the processes that effect more coherent grammatical constructions and ameliorate the vagueness that was synchronically present in all varieties of an exclusively oral vernacular.

