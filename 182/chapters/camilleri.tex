\documentclass[output=paper]{LSP/langsci} 
\ChapterDOI{10.5281/zenodo.1181795}
\author{Maris Camilleri\affiliation{University of Essex}} 
\title{On raising and copy raising in Maltese} 
%\epigram{Change epigram in chapters/01.tex or remove it there }
\abstract{This paper seeks to describe and account for the (morpho)syntactic behaviour of lexically determined raising predicates and constructions, and will be considering a list of properties that characterise these. Different raising-to-{\sc subj} constructions available in Maltese are discussed, and eventually formalised within the Lexical Functional Grammar framework. We will argue that raising constructions in Maltese can be divided into two analyses: raising that involves a structure-shared dependency, and raising that involves an anaphoric binding dependency between the matrix {\sc subj} and any embedded grammatical function, subject to the identified constraints that will be discussed. We illustrate how in Maltese, raising structures are of the former type, while copy raising is of the latter.}
\maketitle

\begin{document}

\section{Introduction} 

To date, there has not been any descriptive account of the different properties and behaviours that characterise \isi{raising} constructions in Maltese, except initial discussions of various behaviours in \Citet{CES:LFG14} and an account of the \isi{raising} behaviours of various aspectual auxiliaries in \Citet{Camilleri16}, as well as a mention of these structures in \Citet{fabri93phd}. The main aim of this study is to discuss alternations of the sort in (\ref{raisingmaltese}), where (\ref{noraisingexpl}) involves a default {\sc 3sgm} matrix form, while (\ref{realraisingmt}) involves the \isi{raising} of the {\sc 3pl} embedded \isi{subject} ({\sc subj}), and where in the latter structure, an overt DP/{\sc subj} in the embedded \isi{clause} is not possible, hence the \isi{ungrammaticality} of (\ref{nosubjpossible}).\footnote{Unless specified, the data should be understood as being provided by the author, a native \isi{speaker}.} 

\ea \label{raisingmaltese}
\ea \label{noraisingexpl}
\gll J-i-dher (li) (it-tfal) sejr-in tajjeb (it-tfal){\footnotemark}\\
3{\sc m}-{\sc frm.vwl}-appear.{\sc sg} {\sc comp} {\sc def-}children go.{\sc act.ptcp-pl} good.{\sc sgm} {\sc def-}children\\
\footnotetext{The segmentation followed in this study is based on the account in \Citet{Camilleriphd14}.}
\glt `It seems that the children are doing well'
\ex \label{realraisingmt}
\gll It-tfal j-i-dhr-u (li) sejr-in tajjeb\\
{\sc def-}children 3-{\sc frm.vwl}-appear.{\sc impv-pl} {\sc comp} go.{\sc act.ptcp-pl} good.{\sc sgm}\\
\glt `The children seem to be doing well'
\ex \label{nosubjpossible} *J-i-dhr-u$_{i}$ li t-tfal/huma$_{i}$ sejr-in tajjeb
\z
\z  

We here start our discussion with an example from the \Citet[244]{culicover2009natural} textbook in order to better understand what we are to understand when we say that a \isi{verb} is a \isi{raising predicate}. In English, given the contrast in (\ref{EngRaising}), the fact that `something can be a \isi{subject} of \emph{appear to} VP whenever it can be a \isi{subject} of a \emph{that}-complement containing VP' suggests that \emph{appear} is a \isi{raising predicate}. When \isi{raising} is not present, as in (\ref{EngRaisinga}), what we have is the formation of what is referred to as an \emph{It}-Extraposition structure. While the sentences in (\ref{EngRaising}) are syntactically distinct, the semantic composition is the same. This follows from the fact that since \emph{appear} is a \isi{raising predicate} and only selects for a \isi{clausal argument}, the non-thematic \isi{external argument} function is filled in by the semantically \isi{vacuous pronoun} \emph{it}, which in turn has no effect whatsoever on the semantic interpretation of the construction.

\ea \label{EngRaising}
\ea \label{EngRaisinga}It appears that I have forgotten to do my work\\
\ex \label{EngRaisingb}
I appear to have forgotten to do my work\\
\z
\z

The predicates that are able to license \isi{raising} structures are idiosyncratic, and one has to specifically determine these on the basis of a number of \isi{syntactic} properties that may well be language internal. However, crosslinguistically one finds that similar and corresponding lexical items keep displaying the same behaviour \Citep{stiebels2007towards}. In this study we aim to provide an overview of the \isi{raising} predicates available, whilst identifying which \isi{syntactic} properties are associated with \isi{raising} predicates and structures in Maltese. Reference to the term \emph{raising} with respect to the set of lexical items and constructions we will be discussing here comes from the transformational rule used in \Citet{Rosenbaum:Complement} to account for {\sc subj}-to-{\sc subj} \isi{raising construction} alternations, such as the one illustrated in (\ref{EngRaisingb}). \Citet{Postal:Raising}, on the other hand, generalised over this rule to account for all sort of \isi{raising} constructions, including {\sc subj}-to-{\sc obj} (ECM) constructions. Another term provided in the literature for verbs which display \isi{raising} behaviours and involve a one-place predication that is a \isi{clausal argument}, is that of `aspectualisers' in \Citet[8]{newmeyer1975english}.\footnote{Here we choose not to use this term, as ``aspectualisers" elsewhere in the literature refer to a set of predicates, auxiliaries, light verbs and particles which provide information with respect to {\sc phasal aspect} (\Citealt{binnick1991time}; \Citealt{michaelis1998aspectual}); and \Citealt{vanhovephd} and \citealt{Camilleri16} for specific reference to phasal verbs or aspectualisers in Maltese.} While we choose to refer to the predicates under discussion as `\isi{raising}' predicates, we won't be employing any transformational sort of analysis. Rather, we will formalise our account within the Lexical Functional Grammar ({\sc lfg}) framework, where all constructions are assumed to be base-generated, and the relationship between the semantically equivalent but syntactically distinct sentences in the pairs in (\ref{raisingmaltese}) and (\ref{EngRaising}) in Maltese and English, respectively, boils down to the presence or absence of functional binding/structure-sharing via a functional equation that defines the equivalence between the {\sc subj} in the main \isi{clause} and the embedded \isi{clause}. Rather than movement, relations and dependencies in {\sc lfg} are understood `in terms of relations between functions' and not structural positions \Citep[400]{Bresnan:ContComp}. In (\ref{realraisingmt}) and (\ref{EngRaisingb}), there thus holds an interpretive/referential dependency between the {\sc subj} in the matrix and the unexpressed \isi{external argument} of the \isi{predicate} in the embedded \isi{clause}. This relation is referred to as \emph{control}. As we will discuss, \isi{raising} constructions in Maltese differ  as to whether they involve functional control or \isi{anaphoric control}. The former involves structure-sharing between the {\sc subj} \isi{grammatical} functions ({\sc gf}s) across both clauses, while \isi{anaphoric control} involves binding, i.e.~a co-referential dependency. %While \emph{likely} is a \isi{raising predicate}, \emph{plan} is a two-place \isi{predicate} whose argument-structure involves the subcategorisation of a {\sc subj gf} to which a theta-role is assigned. For this reason, while \emph{It} Extraposition is available as a counterpart to (\ref{raisingmix}), this is not possible in the case of (\ref{controlmix}), since the expletive \emph{it} can be assigned no theta-role. In this study we will thus not be considering equi/control predicates such as \emph{pprova} `try', which as illustrated in (\ref{controlmt}) does not display a parallel contrast to (\ref{raisingmaltese}).

%\ea \label{controlmt}
%\ea
%\gll *J-i-pprova m-mur\\
%3-{\sc epent.vwl}-try.{\sc sgm} 1-go.{\sc impv.sg}\\
%\glt `*It tries that I go' 
%\ex
%\gll N-i-pprova m-mur\\
%1-{\sc epent.vwl}-try.{\sc impv.sg} 1-go.{\sc impv.sg}\\
%\glt `I try to go'
%\z
%\z  

%Throughout this study we will be highlighting additional properties which can't be possibly available to control/equi predicates, given the fact that they subcategorise for a thematically-licensed {\sc subj}. Seeking to describe and account for the behaviour of lexically-determined control (as opposed to the structurally-determined control displayed in the case of control/equi predicates such as \emph{pprova} `try') highlighting the array of predicates available, as well as the issues that arise with respect to considerations that have to do with clausal finiteness. We will essentially argue that Maltese \isi{raising} constructions are of two types. Raising and hyperraising constructions involve functional control, while copy-\isi{raising} involves \isi{anaphoric control}. 

The paper proceeds as follows: In \S2 we provide a very brief overview of the framework of {\sc lfg} and how \isi{raising} is dealt with. In \S3 we delve further into the details of the basic properties of \isi{raising} constructions in Maltese, and the predicates involved. We provide evidence as to why it is believed that they should be analysed as \isi{raising} predicates. \S4 discusses copy \isi{raising} and how it involves a distinct mechanism, when compared with non-copy raised structures. \S5 then concludes the paper.

\section{Raising in {\sc lfg}}

\subsection{LFG: The theory}

{\sc lfg} employs a \isi{parallel architecture}/correspondence \Citep{KaplanBresnan:LFG} and models a theory of language analysis. Such an architecture allows for distinct co-present projections that relate to one another via functional correspondences modelling different representations of linguistic analysis, each having their own rules and constraints. {\sc lfg} is primarily a lexicalist theory that relies heavily on lexical entries and the information present in them. Lexicalist approaches are thus based on an underlying assumption that it is not syntax which should deal with a number of structures and relations. Rather, these are best left to the morphological domain and the lexicon, including the argument-structure. The argument-structure essentially represents predicate-argument relations. The arguments and their thematic roles are then mapped onto \isi{grammatical} functions ({\sc gf}s). What concerns us most, for the purpose of this study, is where in the model, \isi{syntactic} analyses take place.

{\sc lfg} employs two representational levels where \isi{syntactic} analyses can be done, based on an important principle whereby \isi{syntactic} functions are analysed independently of any sort of configurational structure (\Citealt{Bre00}; \Citealt{Dalrymple01}; \Citealt{Falk}; \Citealt{bresnan2015lexical}). This split between function and constituency translates into the constituent-structure (\emph{c-}structure) and the functional-structure (\emph{f-}structure). The \emph{c-}structure has to do with the external properties related with syntax, which allow and account for the variation that exists across languages. It takes into account \isi{word order} considerations, constituency, \isi{syntactic} categories, dominance and precedence. Through the use of \isi{phrase structure} rules that build up \isi{syntactic} trees, the surface linear order configurationality (or the lack of it), is represented. While X-Bar syntax \Citep{Chomsky:RON} is used for configurational or semi-configurational languages, flatter \emph{c-}structures that do not need to be restricted to binary branched tree structures are also available. The other level of \isi{syntactic representation}, i.e. the \emph{f-}structure, is concerned with internal \isi{syntactic} properties, which are believed to be more universal in nature. The \emph{f-}structure thus represents the relevant {\sc gf}s, i.e. {\sc subj}, {\sc obj} etc., as well as other syntactically relevant features involved in any \isi{syntactic} construction.
%The core {\sc gf}s are: {\sc subj}, {\sc obj}, {\sc obj}$\theta$.\footnote{This {\sc gf} is also known as the `restricted {\sc obj}' or {\sc obj}2 in the literature. Essentially the $\theta$ represents the fact that this secondary {\sc obj} is restricted to a special set of thematic roles.} The non-core functions on the other hand are: {\sc obl}, {\sc adj}, {\sc comp}, {\sc xcomp}, with {\sc adj}s and {\sc udf}s functioning as non-arguments. {\sc udf}s (unbounded discourse functions) is the umbrella term which \citet{Asudeh:PhD, Asudeh12} uses to categorise the discourse functions: {\sc top}ic and {\sc foc}us.

Every level of linguistic representation in the \isi{parallel architecture} that constitutes the {\sc lfg} model makes use of a distinct language. The \emph{f-}structure, which is our main concern here, makes use of hierarchical attribute value matrices ({\sc avm}s). The information necessary for the \emph{f-}structure comes from the \isi{lexical entry} as well as information coming from the annotation on \emph{c-}structure nodes. The functional head of an \emph{f-}structure is a {\sc pred} feature, which takes a list of semantic/thematic arguments represented through their enclosure in angle brackets. These are then mapped onto {\sc gf}s on the basis of a default hierarchy of mappings (\Citealt{Kibort04}; \Citealt{Kib:07}) or through lexical specifications, if necessary. %Apart from the {\sc pred} and other {\sc gf}s, the \emph{f-}structure is also made up of a list of features and their values. Such values could be either atomic, i.e. a symbol (e.g. {\sc num sg}); a semantic form, as is always the case with respect to the value of the {\sc pred}; sets, or feature-structures in themselves. The \emph{f-}structure then feeds into the semantic-structure (\emph{s-}structure), which provides the necessary interpretations. The fact that it is not the \emph{c-}structure that does this explains why there is `no motivation for phonologically null heads' in {\sc lfg} \Citep[p. 497]{Fal84}. The different feature-value pairs function as defining equations. For example, ($\uparrow$ {\sc subj num}) = {\sc sg} functionally designates that the {\sc num} value internal to the {\sc subj}'s \emph{f-}structure is {\sc sg}. The requirement of the presence of a particular feature-value pair in a given \emph{f-}structure necessitates a constraining equation. Unlike defining equations, these do not create a feature or provide its value. Rather, constraining equations require the presence of the feature-value pair. Thus, a constraining equation such as: ($\uparrow$ {\sc subj num}) = \textsubscript{c}{\sc sg} ensures that the value {\sc sg} for the {\sc num} feature within the {\sc subj}'s \emph{f-}structure, must be present. There are also negative constraining equations. In order to constrain the value of {\sc tense} from being {\sc present}, this can be stated as follows: $\neg$($\uparrow$ {\sc tense}) = {\sc pres} or ($\uparrow$ {\sc tense}) $\neq$ {\sc pres}. %Conditional equations are also present: `useful in testing for the presence of a feature as a condition for functional specifications`: [ ... $\land$ ...] $\Rightarrow$ ... $\equiv$ $\neg$A $\lor$ B.
%Functional-uncertainty equations are heavily relied upon for a number of local and long distance dependency constructions, which in turn functionally- or anaphorically-identify different \isi{grammatical} relations, allowing a given {\sc gf} or {\sc udf} to be associated with two distinct \isi{syntactic} functions in the \emph{f-}structure.
While these two levels of \isi{syntactic representation} feed information into one another, agreement, binding, complementation, local dependencies including \isi{raising} and control, long distance dependencies and other such constructions, are all done at the \emph{f-}structure level, on the basis of a reference to the different relations and dependencies that are present across and amongst the {\sc gf}s. 
%While independent structures in their own right, the \emph{c-} and \emph{f-}structures are related to one another through the $\phi$ function. The relation between both these structures can be non-isomorphic, and for this reason, the \emph{f-}structure is not to be conceived as a composite of the different nodes and categories internal to the \emph{c-}structure. It is in fact possible for \emph{f-}structure fragments to not be associated with any piece of \emph{c-}structure, as in the case of incorporated {\sc pro} instances, for example. In these cases, it is solely the morphology on the \isi{verb} form, for example, that signals the {\sc subj gf}, rather than any overt piece of structure.\footnote{The same follows in the case of the presence of $\epsilon$ in \isi{phrase structure} rules, where the lack of a \emph{c-}structure correspondence may nevertheless still imply a feature at the \emph{f-}structure: `$\epsilon$ corresponds to an empty string and represents the absence of a \isi{phrase structure} constituent. Importantly [when this is present], the rule does not license the presence of an empty category or node in the \emph{c-}structure tree. It simply constitutes an instruction to introduce some functional constraints in the absence of some overt word or \isi{phrase}' (Dalrymple, 2001, pp. 175-176).} At times it is possible that a piece of \emph{c-}structure expresses some sort of {\sc gf} which could have otherwise been solely signalled through morphological marking on the \isi{verb}. When such a structurally-expressed argument fills and satisfies one of the \emph{f-}structure's required {\sc gf}s, the relevant marking on the \isi{verb} merely comes to function as agreement, and this \isi{inflectional} material loses its \isi{pronominal} value in such contexts. It is also possible for one \emph{c-}structure node to be related with a complex \emph{f-}structure, just as different nodes or split inflection and distributed exponence across different nodes could also contribute to a single \emph{f-}structure, e.g. a feature value. 

%In the same way that the two distinct levels of \isi{syntactic} representations make use of their own language, they also adhere to different conditions that constraint their wellformedness. 
%The \emph{c-}structure is governed by X-Bar principles, i.e. principles of endocentricity, at least when applied to configurational languages. Such \isi{phrase structure} rule constraints, which determine the annotations on the \emph{c-}structure nodes, are the following: 
%\begin{enumerate}
%\item{Complement of lexical category (open) = a {\sc gf} (argument) or a co-head;}
%\item{Specifier of a lexical category is a dependent;}
%\item{Complement of a functional category (closed) = a co-head;}
%\item{Specifier of functional category = discourse function \hfill{\Citep[pp. 118--119]{Bre00}};}
%\item{Non-projecting nodes must be adjoined to another head, which can be either a functional or a lexical category, and can be an argument of a co-head \hfill{\Citep{Toivonen:NonProj}}}
%\end{enumerate}
%\emph{C-}structure nodes in {\sc lfg} are optional, following the principle of Economy of Expression, especially when they contribute nothing to the semantic composition. This principle states that:\\\\
%Economy of Expression: `All \isi{syntactic} \isi{phrase} nodes are optional and are not used unless required by independent principles (completeness, coherence, semantic expressivity)' (Bresnan, 2001, p. 188).\footnote{Lexocentricity allows for flatter structures, typically making use of the S category, which is exocentric, and hence can head any category. It is however not constrained by any \isi{phrase structure} rules, and can head multiple categories at the same time. In languages which employ this sort of \emph{c-}structure, functional relations are not determined by \isi{word order}, but rather through \isi{grammatical} relations identified via head or dependent marking.} %\emph{C-}structures in {\sc lfg} are \isi{subject} to the: Extended head theory, which allows different categories at the \emph{c-}structure to share their role as functional heads internal to the \emph{f-}structure, i.e. bearing co-head roles. This is the case when a functional terminal category takes a lexical head as its external head \Citep{Bresnan:LFG97}. 
%The \emph{c-}structure in {\sc lfg} also applies the principle of Lexical Integrity, where one morphologically-complete word per node is assumed.\footnote{Lexical sharing \Citep{wescoat2005english, buttcatalan} is nevertheless allowed when \isi{phonological} words happen to be composed out of words sitting at terminal (X\textdegree) nodes.} The principle is stated as follows:\\\\
%Lexical Integrity: `Morphologically complete words are leaves of the \emph{c-}structure tree and each leaf corresponds to one and only one \emph{c-}structure node' (Bresnan, 2001, p. 92).\\\\
%Note that this doesn't mean that the syntax is not able to see inside the word-form. Rather, the morphological form of a word is able to feed relevant information internal to the \emph{f-}structure, in turn allowing both the syntax and the morphology to contribute the same sort of information, which is possible via the unification of the function-values involved.

For what concerns us in this study, the relevant constraints include those related with the \emph{f-}structure, which is constrained by the Uniqueness, Completeness and Coherence conditions. Uniqueness requires that there be no duplication in the \emph{f-}structure, such that every attribute/feature is itself unique and takes its own unique value. In the case of unbounded discourse functions ({\sc udf}s) such as {\sc top}ic and {\sc foc}us and adjuncts ({\sc adj}s), set values for these do not violate Uniqueness. As a result, many of these could be co-present. The Completeness condition requires that the {\sc pred}'s argument-structure requirements be satisfied within the \emph{f-}structure, while Coherence checks that every {\sc gf} present in the \emph{f-}structure is one that is selected by the {\sc pred}. {\sc udf}s as well as other `\isi{syntactic} functions requiring that they be integrated appropriately into the \emph{f-}structure' \Citep[63]{Bre00}, partake in the Extended Coherence Condition \Citep[746]{BM87}, which states that: `Focus and Topic must be linked to the semantic \isi{predicate} argument structure of the sentence in which they occur, either by functionally or anaphorically binding an argument.'

\subsection{The theory of raising in {LFG}}

%Here no reference to the \emph{c-}structure. Token-identity, be it via structure-sharing as in \isi{raising}, or anaphoric-binding; {\sc gf}s divided into core - non-core; \isi{subcategorisation frame}, which {\sc gf}s also represent another mapping with thematic-roles in the argument-structure.
A constraint imposed on \isi{raising} constructions in {\sc lfg} is that the `raised' {\sc gf} be a \emph{term}/core-argument, and should thus be an embedded {\sc subj}, {\sc obj}, or {\sc obj}{\texttheta} (\Citealt[419]{Bresnan:ContComp}; \Citealt[10]{Dalrymple01}) and that `lexically controlled local dependencies [...] involve simultaneous instantiations of two \isi{grammatical} functions to a single \emph{f-}structure value' \Citep[6]{asudeh2012copy}. This is thus a `functional predication relation' \Citep[270]{Bre00}, and can be defined as a relation that `involves a dual assignment of \isi{grammatical} relations: a single NP functions as an argument of both the subordinate \isi{clause} and the \isi{matrix clause}, and bears a \isi{grammatical} relation in both clauses' \Citep[107]{Kroeger:LFG}. This view of control thus entails a symmetrical relation between the {\sc gf}s involved, and is referred to as functional-identity, token-identity or structure-sharing. Unlike unbounded distance dependency constructions, where one finds dependencies involving {\sc udf}s occupying multiple instantiations, in the case of \isi{raising} (and control), there is a limitation to the `sentence node', and the dependency is hence bounded/local. Having said this, however, it is possible to also have `multiple structure-sharing, resulting from [...] further embedding' (\Citealt[491]{Asudeh05}; \Citealt[22]{ashT06}; \Citealt[18]{Alsina:08}), as long as the clauses proceed locally. See (\ref{Cascade}) for an illustration of chained \isi{raising} cascades in Maltese.

\ea \label{Cascade}
\gll La\textcrh q-u dehr-u qis-hom donn-hom \textcrh a j-i-bde-w j-e-r{\.g}g\textcrh-u j-morr-u{\footnotemark}\\
reach.{\sc pfv.3-pl} appear.{\sc pfv.3-pl} as.though-{\sc 3pl.acc} as.though-{\sc 3pl.acc} {\sc prosp} 3-{\sc frm.vwl}-start.{\sc impv-pl} 3-{\sc frm.vwl}-repeat.{\sc impv-pl} 3-go.{\sc impv-pl}\\
\glt `They did happen to have appeared as though they will start going again.'
\z

\footnotetext{An anonymous reviewer questions the acceptability of this construction: `The co-occurrence of \emph{qis-hom} and \emph{donn-hom} next to each other is unacceptable since one of them is redundant.' I assure the reader that this sentence is pretty acceptable for the author, with the presence of {\bf{both}} the predicates \emph{qis-} and \emph{donn-}, although of course this chained cascade is not obligatory and indeed only one of them may be present. Furthermore, neither of them, for that matter, need be present, given that they simply reinforce the same interpretation which \emph{deher} `seem' itself renders in the overall structure. Data from the \isi{MLRS} further support this claim (as in (a)), including data involving the reversed order of these same predicates.

\renewcommand{\exfont}{\tiny\footnotesize}
\renewcommand{\transfont}{\tiny\footnotesize}
\ea
\langinfo{}{MLRS}{v3.0}\\
\gll qis-u donn-u in$\langle$t$\rangle$esa koll-u\\
as.though-{\sc 3sgm.acc} as.though-{\sc 3sgm.acc} forget.{\sc pass.pfv.3sgm} all-{\sc 3sgm.gen}\\ 
\glt `it's as though all has been forgotten'
\z
\renewcommand{\exfont}{\normalsize\itshape}
\renewcommand{\exfont}{\footnotesize\itshape}

Additionally I point out that redundancy at the \isi{syntactic} level, which is what we have here, should not entail, or be equated to unacceptability, as is being implied by the reviewer. Redundancy can in fact be observed in several aspects of a language's grammar.}

%Bresnan (1982/2001)/Falk (2001)/Kroeger (2004)/Asudeh (2005)/Sells (2006) allow for both \isi{raising} and equi verbs to be provided with an identity relation, while a coindexation (although as we will see below this is not quite the correct description for the use in the constructions mentioned by these scholars, as we will see below) is reserved solely for arbitrary control and control that involves split reference. Under this account, \isi{raising} and equi verbs are in fact analysed in the same way, except for a small difference in the representation of the controller, i.e. that unexpressed or expressed element which comes to be referentially dependency with the obligatory unexpressed controlled element.

In \isi{raising} constructions of the type in (\ref{realraisingmt}) and (\ref{EngRaisingb}) the \isi{complement clause} is mapped onto an {\sc xcomp gf}. A {\sc gf} of this type, as opposed to the {\sc comp gf} is an open complement, and licenses structure-sharing between the relevant matrix and embedded {\sc gf}s to take place. The {\sc{xcomp}} embodies distinct \emph{c-}structure constituents that function predicatively, such that {\sc{xcomp}} $\equiv$ \{NP $|$ VP $|$ AP $|$ PP\} (and CP under Falk's (2001) view based on his account of \emph{to}). The {\sc xcomp} \isi{clausal argument} is thus the only {\sc gf} which these \isi{raising} verbs subcategorise for. The {\sc subj}'s `appearance outside the brackets' \Citep[12]{ZaenenKaplan02} represents the fact that the \isi{external argument} is not selected by the \isi{predicate}, i.e. $\langle${\sc xcomp}$\rangle${\sc subj}. The brackets are what would otherwise `enclose the semantically selected arguments of the lexical form' \Citep[283]{Bre00}. This formal distinction, i.e. between {\sc gf}s within, or external to the brackets, functions as a means with which to represent whether the matrix imposes restrictions on such {\sc{gf}}s or not.

In the absence of \isi{raising}, the semantically \isi{vacuous} position of the \isi{external argument} is filled by dummy/expletive pronouns, since these lack a semantic {\sc pred} value \Citep[283]{Bre00}. The availability of such pronouns is itself lexically specified \Citep[123]{Kroeger:LFG}. When \isi{raising} is not available, and hence no structure-sharing is involved, the \isi{lexical entry} is: $\langle${\sc comp}$\rangle${\sc subj}. This distinction at the \isi{lexical entry} level is summarised as follows from \Citet[404]{Bresnan:ContComp}: `Unlike {\sc xcomp}s, closed {\sc comp}s may undergo \emph{It} Extraposition ...' in English. The \isi{raising}/non-\isi{raising} ambiguity of English \emph{seem} is in \Citet[14]{asudeh2012copy} reduced to the following constraint in the \isi{lexical entry}: ($\uparrow$ {{\sc subj expletive}}) = \textsubscript{c}{\sc{it}} $\wedge$  $\neg$ ($\uparrow$ {{\sc xcomp}}) $|$ ($\uparrow$ {{\sc subj}}) = ($\uparrow$ {\sc{xcomp subj}}). This constraint states that we either have a constraining equation that requires the presence of an expletive \emph{it} when the \isi{complement clause}'s function is not an {\sc xcomp}; or, in the absence of the expletive as the matrix {\sc subj}, equality between matrix {\sc subj} and {\sc xcomp subj} applies.\footnote{\Citet[137]{Falk} approaches this ambiguity by positing a `Functional Control Rule' which states that: `If ($\uparrow${\sc xcomp}) is present in a lexical form, add the equation: ($\uparrow${\sc subj}$|${\sc obj}) = ($\uparrow${\sc xcomp subj}). When this rule is not present, we get the non-thematic argument filled by an expletive (p. 138).} With this brief introduction to the classic {\sc lfg} treatment of \isi{raising}, we can now proceed to characterise in more detail, \isi{raising} in Maltese.

\section{Raising in Maltese}

In this section we first highlight the main \isi{raising} predicates in the language, and then provide morphosyntactic behaviours that serve as evidence sustaining our claim that these predicates are \isi{raising} predicates.

\subsection{Raising predicates}

The primary \isi{raising predicate} in Maltese is \emph{deher} `appear, seem'. The data in (\ref{arrayofpred}), exemplified through the behaviour associated with \emph{deher}, illustrates the array of phrasal categories that can function as a complement of \emph{deher}: CP/VP (\ref{cp/vp}); NP (\ref{np}); AP (\ref{ap}); PP (\ref{pp}).

\ea \label{arrayofpred}
\ea \label{cp/vp}
\gll T-i-dher (li) miexj-a 'l quddiem\\
3{\sc f}-{\sc frm.vwl}-appear.{\sc impv.sg} {\sc comp} walk.{\sc act.ptcp-sgf} {\sc all} front\\
\glt `She/It seems to be moving forward'
\newpage 
\ex \label{np}
\gll Marija t-i-dher tifla bilg\textcrh aqal{\footnotemark}\\
Mary 3{\sc f}-{\sc frm.vwl}-appear.{\sc impv.sg} girl with.{\sc def.}wisdom\\
\glt `Mary seems to be a good girl'
\ex \label{ap}
\gll T-i-dher tajb-a\\
3{\sc f}-{\sc frm.vwl}-appear.{\sc impv.sg} good-{\sc sgf}\\
\glt `She/It seems good'
\ex \label{pp}
\gll T-i-dher bil-bajda m-dawwr-a\\
3{\sc f}-{\sc frm.vwl}-appear.{\sc impv.sg} with.{\sc def-}egg.{\sc sgf} {\sc pass.ptcp}-turn-{\sc sgf}\\
\glt Lit: She seems with the egg turned\\
\glt `She seems to be grumpy (today)'
\z
\z 
\footnotetext{It should be mentioned that if we had the construction in (i) instead, \emph{hija} in this context would not be functioning as the {\sc subj}, but rather as the copula. This data should therefore not be confused with what has been said with respect to the \isi{ungrammaticality} of (\ref{nosubjpossible}). Furthermore, it is clear from such a context that the {\sc xcomp gf} which maps onto a CP embeds a sentential complement (S) headed by the \isi{pronominal} copula. 


\renewcommand{\exfont}{\tiny\footnotesize}
\renewcommand{\transfont}{\tiny\footnotesize}
\ea
\gll Marija t-i-dher li hija tifla bilg\textcrh aqal\\
Mary 3{\sc f}-{\sc frm.vwl}-appear.{\sc impv.sg} {\sc comp} {\sc cop.3sgf} girl with.{\sc def.}wisdom\\ 
\glt `Mary seems that she is a good girl' 
\z


\renewcommand{\exfont}{\tiny\normalsize}
\renewcommand{\transfont}{\tiny\normalsize}
}

In this paper we will not delve into issues that have to do with finite \isi{raising}, i.e. hyperraising, which \Citet{landau2011predication} refers to as `non-ordinary \isi{raising}'. 
%While 
Maltese does employ finite morphological forms even in the embedded \isi{clause}, apart from the \isi{predicate} types just considered, which are also available in the embedded \isi{clause} (as one may have already noticed in e.g. (\ref{Cascade})). However, one should make it clear that as discussed in \citet{Sells06}, there need not be an isomorphic relationship between morphological and \isi{syntactic} finiteness. Clear, unambiguous instances of finite embedded clauses are (\ref{findeher}), where the presence of \emph{kont} in (\ref{perfdeher}) provides a {\sc tense} feature with value {\sc past}. In (\ref{modaldeher}), we then have an epistemic modal value realised syntactically. We take both these instances to suggest that the embedded complement in Maltese can map onto an IP, which is itself indicative of a finite \isi{clause}. We will here say nothing more about such construction types and how they may be the same or different from non-finite \isi{raising} structures. For more discussion on hyperraising in Maltese, refer to \citet{CamilleriTA}.

\newpage 
\ea \label{findeher}
\ea \label{perfdeher}
\gll N-i-dher (li) kon-t mor-t tajjeb, id-darba l-o\textcrh r-a\\
1-{\sc frm.vwl}-appear.{\sc impv.sg} {\sc comp} be.{\sc pfv-1sg} go.{\sc pfv-1sg} good.{\sc sgm} {\sc def-}once.{\sc sgf} {\sc def-}other-{\sc sgf}\\
\glt Lit: `I seem that I had done well last time'\\
\glt `I seem to have done well, the last time' 
\ex \label{modaldeher}
\gll Dehr-et (kien) kel-l-ha mnejn semg\textcrh-et ming\textcrh and-hom, dakinhar\\
appear.{\sc pfv.3sgf} be.{\sc pfv.3sgm} be.{\sc pfv.3sgm}-have-{\sc 3sgf.gen} from.where hear.{\sc pfv-3sgf} from.at-{\sc 3pl.gen} {\sc dem.sgm.def-}day\\
\glt `She seemed she had perhaps heard from them that day' 
\z
\z  


On the basis of the overview of the analysis of {\sc subj}-to-{\sc subj} \isi{raising} in {\sc lfg}, presented briefly in the previous section, we provide the \isi{lexical entry} associated with \emph{deher} `seem' that allows for the expletive and \isi{raising} alternation. Following \Citet{Berman:German} we account for the default {\sc 3sgm} agreement in the matrix as being itself indicative of a {\sc pred}less {\sc subj} analysis. Although never discussed previously, an expletive \isi{pronoun}, namely \emph{huwa}, which is equivalent in form to the long version of the {\sc 3sgm} \isi{subject} \isi{pronoun}, alternating 
%which alternates 
with the short form \emph{hu}, could be said to exist in Maltese. In (\ref{huwasubjfunction}) it is not as controversial to assume that the \isi{pronoun} \emph{huwa} is functioning as a semantically \isi{vacuous pronoun} filling in a non-thematic {\sc subj} position. In the data in (\ref{huwanonsubjfunction}), on the other hand, we find that \emph{huwa} must have another function, and could well be some sort of \isi{clause} force that provides an exclamative interpretation and sarcastic tone to this sort of construction.\footnote{Parallel structures are mentioned in passing in \Citet[195]{BAA:97}. For want of a better translation, I gloss \emph{huwa} in such constructions as: he.{\sc expl} so that it is not confused with the syncretic form \emph{huwa} when being used referentially as the \isi{pronoun} meaning `he'.}%\footnote{This data parallels data from \ili{Egyptian} as follows, where Jelinek (1988: 95) makes reference to the overt expletive \emph{huwwa}, even when this is clearly not fulfilling a {\sc subj} function. 

%\ea
%\ea 
%\gll Huwwa int bi-ti-{\v{s}}rab ahwa?\\
%he you {\sc bi-2-}drink.{\sc sg} coffee\\
%\glt `Is it that you drink coffee?'
%\ex
%\gll Huwwa i\textcrh na ni-ru{\textcrh} dilwa\textglotstop t?\\
%he we 1-go.{\sc impv.sg} now?\\
%\glt `Is it that we are leaving now?' \hfill{\ili{Egyptian} \ili{Arabic}, Jelinek 1988: 95}
%\z
%\z} 

%We could argue that, at least if we are on the right track, and that \emph{huwa} is indeed some kind of expletive \isi{pronoun} in the language, then when it appears in {\sc subj} position, it fills in the {\sc pred} value of the {\sc subj} {\sc gf}.
 
\ea \label{huwasubjfunction}
\gll Huwa j-i-dher li/kemm sejr-a tajjeb!\\
he.{\sc expl} 3.{\sc m}-{\sc frm.vwl}-appear.{\sc impv.sg} {\sc comp} go.{\sc act.ptcp-sgf} good.{\sc sgm}\\
\glt `It shows that she is doing well!'
\glt `It shows how good she's going!'
\z 

\ea \label{huwanonsubjfunction}
\ea
\gll Huwa t-i-dher kemm sejr-a tajjeb!\\
he.{\sc expl} 3{\sc f}-{\sc frm.vwl}-appear.{\sc sg} {\sc comp} go.{\sc act.ptcp-sgf} good.{\sc sgm}\\
\glt `It is clearly showing how well she is going (sarcastic)'
\ex
\gll Huwa intom t-i-dhr-u li nisa, t-af-x!\\
he.{\sc expl} you.{\sc pl} 2-{\sc frm.vwl}-appear.{\sc impv-pl} {\sc comp} women 2-know.{\sc impv.sg-neg}\\
\glt Lit: `It is clearly showing that you are women, don't you know'
\z
\z 

The conflated \isi{lexical entry} for \emph{deher} is the following:\\
\\
\lexentry{deher: I/V}{
(\up$\mu$ {\sc pred vform}) = Perfective\\
(\up$\mu$ {\sc pred vform pol}) = {\sc pos}\\
\{(\up {\sc pred}) = $\langle${\sc xcomp}$\rangle${\sc subj}\\
((\up {\sc xcomp compform}) = {\sc li})\\
(\up {\sc subj}) = (\up {\sc xcomp subj})\\
(\up {\sc subj person}) = 3\\
(\up {\sc subj num}) = {\sc sg}\\
(\up {\sc subj gend}) = {\sc m}\\
$|$\\
(\up {\sc pred}) = $\langle${\sc comp}$\rangle${\sc subj}\\
((\up {\sc comp compform}) = {\sc li})\\
\{$\neg$(\up {\sc subj pred})\\
(\up {\sc subj person}) = 3\\
(\up {\sc subj num}) = {\sc sg}\\
(\up {\sc subj gend}) = {\sc m}\\
$|$\\
$\neg$(\up {\sc subj pred})\\
(\up {\sc subj}) = {\sc pro}\\
(\down {\sc prontype}) = {\sc expletive}\\
(\down {\sc form}) = \emph{huwa}\}\} }
\\
\\
(\ref{ssrf-str}) represents the {\sc subj}-to-{\sc subj} \isi{raising construction} in (\ref{realraisingmt}), repeated in (\ref{repeat}) below.

\ea \label{repeat}
\gll It-tfal j-i-dhr-u (li) sejr-in tajjeb\\
{\sc def-}children 3-{\sc frm.vwl}-appear.{\sc impv-pl} {\sc comp} go.{\sc act.ptcp-pl} good\\
\glt `The children seem to be doing well'
\z 

\ea
\label{ssrf-str}
\begin{avm}
\[ {\sc pred} & `\emph{jidhru} $\langle${\sc xcomp}$\rangle${\sc subj}'\\
   {\sc subj} & {\[{\sc pred} & `\emph{tfal}' \\
                    {\sc pers} & 3\\
                    {\sc num} & {\sc pl}\\
                    {\sc def} & +\] [1]}\\
   {\sc xcomp} & \[{\sc pred} & `\emph{sejrin}$\langle${\sc subj}$\rangle$'\\
                      {\sc subj} & {\[\phantom{{\sc pred}}&\phantom{`FOOT'} \] [1] }\\
                      {\sc adj} & \{\[ {\sc pred} & `\emph{tajjeb}'\]\} \] \]

\end{avm}
\z

The index [1] in the \emph{f-}structure in (\ref{ssrf-str}) represents the functional identity between the {\sc subj} in the matrix and the {\sc subj} in the embedded \isi{clause}. This dependency is therefore not achieved via movement, but rather via structure-sharing, i.e.~where one and the same \isi{syntactic} item takes on two distinct functions, which in this context are the matrix {\sc subj} and the embedded {\sc subj}. The Uniqueness constraint then ensures that the identical material only be overt in one position. Since (\ref{ssrf-str}) accounts for the forward \isi{raising} present in (\ref{repeat}), we observe how the expressed argument is in the matrix. This then controls the relation/dependency with the {\sc subj} in the embedded \isi{clause}.

%One should mention that w
While \emph{deher} and other \isi{raising} predicates idiosyncratically display an alternation with the \isi{expletive construction}, this is not necessarily the case for all \isi{raising} predicates in the language. Similarly, it is neither the case that all \isi{raising} predicates available in the language necessarily display the array of \emph{c}-structure complement types listed in (\ref{arrayofpred}). Furthermore, the availability of a \isi{complementiser} introducing the \isi{clausal complement} is itself a lexical restriction imposed by the clause-taking \isi{raising predicate}.\footnote{See \Citet[288-292]{Camilleri16} for additional discussion, including a reference to \isi{complementiser} forms other than \emph{li}, including \emph{billi} and \emph{biex}.} \emph{Beda} `begin', which has in \citet{Camilleri16} been shown to function as a \isi{raising predicate}, along with other aspectualisers in the language, such as \emph{qabad} lit. `catch, start', \emph{re{\.g}a'} `repeat' and \emph{qag\textcrh ad} `fit, endure', does not allow its embedded \isi{clause} to be introduced by a \isi{complementiser}. Syndetic marking is thus not allowed, as the \isi{ungrammaticality} of (\ref{nolibeda}) illustrates. Nevertheless, changes in the canonical  constituent order, such as the preposing of the {\sc adj}unct in (\ref{obliglibeda}), results in the obligatory presence of the \isi{complementiser}.%\footnote{Camilleri (2016) provides data involving \isi{aspectualiser} \isi{raising} constructions where it is not just \emph{li} that can introduce the \isi{clause}, but also the complementisers \emph{biex} and \emph{billi}.} 

\ea
\ea \label{nolibeda}
\gll *Bdie-t li t-mur\\
start.{\sc pfv-3sgf} {\sc comp} 3{\sc f}-go.{\sc impv.sg}\\
\glt Intended: `She started to go'
\ex \label{obliglibeda}
\gll Bdie-t li, kuldarba *(li) n-e-rsaq lej-ha, t-i-tlaq t-i-{\.g}ri\\
start.{\sc pfv-3sgf} {\sc comp} every.time {\sc comp} 1-{\sc frm.vwl}-get.close.{\sc impv.sg} towards-{\sc 3sgf.gen} 3{\sc f}-{\sc frm.vwl}-leave.{\sc impv.sg} 3{\sc f}-{\sc frm.vwl}-run.{\sc impv.sg}\\

\glt `She started that, every time I draw close to her, she runs away' \hfill{\Citep[242]{Camilleri16}}
\z
\z

A parallel behaviour with respect to the obligatory or optional presence of the \isi{complementiser} also follows for \emph{deher}. Although \emph{li} is optional, as in (\ref{cp/vp}), this becomes obligatory in contexts where there is a {\sc focus} discourse function in the embedded \isi{clause}, as in (\ref{focdeher}), or when there is a right-dislocation of the (matrix) {\sc subj}, as in (\ref{rightdislocdeher}), for example, where here we observe how as a consequence, the \isi{complement clause} itself becomes right-dislocated.

\ea
\ea \label{focdeher}
\gll T-i-dher *(li) {\sc hi} marr-et tajjeb\\
3{\sc f}-{\sc frm.vwl}-seem.{\sc impv.sg} {\sc comp} she go.{\sc pfv-3sgf} good.{\sc sgm}\\
\glt `She seems that as for her, she did well'
\ex \label{rightdislocdeher}
\gll T-i-dher, Marija, *(li) marr-et tajjeb\\
3{\sc f}-{\sc frm.vwl}-seem.{\sc impv.sg} Mary {\sc comp} go.{\sc pfv-3sgf} good.{\sc sgm}\\
\glt `As for Mary, she seems that she did well'
\z
\z

Apart from \emph{deher} and \isi{aspectualiser} predicates, other \isi{raising} predicates in Maltese include the pseudo-verbs \emph{qis-} and \emph{donn-}, as discussed in \citet{CES:LFG14}.\footnote{We do not here engage in a discussion on pseudo-verbs and what they are. For more information the reader can refer to \Citet{comrie82maltese}; \Citet{Fabri:87}; \Citet{vanhovephd}; \Citet{Peterson:09}; \Citet{Camilleri16}.} These two forms easily co-occur with, or substitute \emph{deher} except in two identified contexts, as we will see. Other pseudo-verbs such as \emph{g\textcrh and-} `have', \emph{g\textcrh odd-} `almost', \emph{g\textcrh ad-} `still; just' and \emph{il-} `have', 
%have in \cite{Camilleri16} also been shown to display behaviours attributed to \isi{raising} predicates.
were also shown shown to display behaviours attributed to \isi{raising} predicates in \cite{Camilleri16}.\footnote{Evidence includes agreement facts; subcategorisation-frame requirements; and other independent evidence that has to do with evidence that favours a matrix \isi{verb} -- \isi{complement clause} analysis, as opposed to a complex \isi{predicate} analysis, or an analysis where the pseudo-verbs simply come to render a feature value in the \emph{f-}structure. Under this analysis,
%under which analysis, 
the lexical \isi{predicate} does not function as a complement, but as the \isi{clause}'s head.} The set of \isi{raising} predicates which have not been described previously,
%concentrated upon, 
are happenstance verbs. These include \emph{inzerta} (\ref{inzerta}), \emph{se\textcrh el} (\ref{sehel}), \emph{la\textcrh aq} (\ref{lahaq}) and \emph{\textcrh abat} (\ref{habattjb}).\footnote{Note that \emph{\textcrh abat} in Maltese also functions as an {\sc inceptive} \isi{aspectualiser}. See \Citet{Camilleri16} for more detail.} The paradigm in (\ref{inzerta}) is made up of data from the \isi{MLRS} Corpus \Citep{gatt2013digital}. In (\ref{inzertaa}) we have a structure which can be interpreted as an \emph{It}-Extraposition, (although one can argue that it is structurally ambiguous), while the constructions in (\ref{inzertali}) and (\ref{inzertanoli}) are {\sc subj}-to-{\sc subj} \isi{raising} structures ({\sc ssr}). 

%Apart from this sub-set of pseudo-verbs, aspectualisers and happenstance verbs also display \isi{raising}. We will here not spend time justifying the {\sc subj} nature of the attached \isi{pronominal} {\sc acc}/{\sc gen} forms on the pseudo-verbal hosts and the \isi{raising} behaviour displayed. For more detail see Camilleri (2014) and Camilleri (2016). For a general discussion on Maltese pseudo-verbs see Peterson (2009).


%We find that the presence of a \isi{complementiser} is available in Maltese, although the properties of the use of \emph{li} with \emph{deher} need not be the same with other verbs.

%\ea
%\ea
%\gll J-i-dher (li) (Marija) marr-et tajjeb (Marija)\\
%3-{\sc frm.vwl}-seem.{\sc impv.sgm} {\sc comp} Mary go.{\sc pfv-3sgf} good Mary\\
%\glt `It seems that Mary did well'
%\ex
%\gll T-i-dher (li) marr-et tajjeb\\
%3-{\sc frm.vwl}-appear.{\sc impv.sgf} {\sc comp} go.{\sc pfv-3sgf} well\\
%\glt `She seems that she did well'
%\z
%\z
 
%Other predicates available include: 

%\emph{{\.G}ara} 'happen': a defective \isi{verb} that only takes 3rd PERS agreement morphology in both the perfect and imperfect sub-paradigms. We will here not be concerned with when there is 3PL/3SGF agreement available, as when this is the case, this is only due to the SUBJ-\isi{verb} agreement in intransitive non-complement taking uses of the \isi{verb}, as in (\ref{gara1})-(\ref{gara2}).

%\ea
%\ea \label{gara1}
%\gll {\.G}ra-w \textcrh wejje{\.g} tal-g\textcrh a{\.g}eb\\
%happened.{\sc.imp.3-pl} things of.{\sc def}-wonder\\
%\glt Magnificent/wonderous things happened
%\ex \label{gara2}
%\gll {\.G}ra-t-l-i bi{\.c}{\.c}a t-i-n-kiteb\\
%happened.{\sc pv-3sgf} thing.{\sc sgf} 3-{\sc epent.vwl}-{\sc pass}-written.{\sc imp.sgf}\\
%Lit: A piece that may be written happened on-me\\
%\glt Something (that affected me badly) happened
%\z
%\z

%The use of 3SGM morphology, on the other hand, at least when this is not used as an agreement marker with a SUBJ that requires default 3SGM agreement, such as (\ref{gara4}), entails the presence of a complement that is obligatory introduced by \emph{li}, and a non-thematic SUBJ (\ref{gara3}).

%\enumsentence{\label{gara4}
%\gll
%X'{\.g}ara-l-ha?\\
%what.happened.{\sc pv.3sgm-dat-sgf}\\
%What happened on-her?}

%We are however concerned with \emph{{\.g}ara} in such contexts, where it comes to display an Expletive construction.

%\ea \label{gara3}
%\gll Mela {\.g}ara li dak i{\.z}-{\.z}mien Marija kien-et g\textcrh ad g\textcrh and-ha biss 8 snin \\
%then happened.{\sc pv.3sgm} {\sc comp} {\sc dem.sgm} {\sc def-}time Mary be.{\sc pv-3sgf} still at-{\sc 3sgf} only 8 year.{\sc pl}\\
%\glt So then it happened that at that time Mary was still just 8 years old ...
%\z

%\emph{Se\textcrh el} (and its variants \emph{\textcrh eles}/\emph{\textcrh asel} (Maas 2009)) is a complement taking \isi{verb} that can take either default 3SGM agreement, or agreement with the SUBJ of the embedded \isi{clause}. The \ili{Romance} \isi{verb} \emph{inzerta} does not obligatorily embed a \isi{complement clause}, (as shown in (\ref{inzerta})). As an embedding \isi{verb} there is then only one interpretation that this \isi{verb} takes i.e. that of 'happen/occur'.   

%\ea
%\ea
%\gll Hekk se\textcrh el {\.g}ara\\
%like.this happened.{\sc pv.3sgm} happened.{\sc 3sgm}\\
%\glt This is what happened to (have) happened
%\ex
%\gll J-i-s\textcrh el (li) ma n-kun-x g\textcrh araf-t-ek jekk ma n-sellim-l-ek-x\\
%3-{\sc frm.vwl}-happen.{\sc imp.sgm} {\sc comp} {\sc neg} 1-be.{\sc imp.sg-neg} recognised.{\sc pv-1sg-2sg.acc} if {\sc neg} 1-greet.{\sc imp.sg-dat-2sg-neg}\\
%\glt It happens that I don't recognise you if I do not greet you
%\ex
%\gll Se\textcrh el (li) sajr-et tajjeb \textcrh afna dakinhar\\
%happened.{\sc pv.3sgm} {\sc comp} cooked.{\sc pv-3sgf} good handful {\sc dem.def}.day\\
%\glt It happened that she cooked very well that day
%\z
%\z

%The \isi{MLRS} examples below displays an example with \emph{inzerta} taking a \isi{complement clause} introduced by the \isi{complementiser}. The presence of \emph{li} is not obligatory, here, however. 

\ea \label{inzerta}
\ea \label{inzertaa}
\gll Inzerta li j-a-qa' ta\textcrh t il-...\\
happen/occur.{\sc pfv}-{\sc 3sgm} {\sc comp} 3{\sc m}-{\sc frm.vwl}-fall.{\sc impv.sg} under {\sc def}\\
\glt `It happens that he falls under ... (He is managed by)' \hfill{(\isi{MLRS} v2.0)}
\ex \label{inzertali}
\gll Inzertaj-t li \textcrh dim-t \textcrh afna fuq dak il-pro{\.g}ett u ...\\
happen/occur.{\sc pfv}-{\sc 1sg} {\sc comp} work.{\sc pfv}-{\sc 1sg} a.lot on {\sc dem.sgm} {\sc def}-project.{\sc sgm} {\sc conj}\\
\glt `I happened such that I have worked a lot on that project' \hfill{(\isi{MLRS} v2.0)}
\ex \label{inzertanoli}
\gll Inzertaj-t n-af xi w\textcrh ud minn-hom\\
happen/occur.{\sc pfv}-{\sc 1sg} 1-know.{\sc impv.sg} some some from-{\sc 3pl.gen}\\
\glt `I happened to know a few of them' \hfill{(\isi{MLRS} v2.0)}
\z
\z

\ea \label{sehel}
\gll Ma s\textcrh el-t-x / ma se\textcrh il-x {\.{g}}ej-t mag\textcrh-na dakinhar, int.\\
{\sc neg} happen.{\sc pfv-2sg-neg} / {\sc neg} happen.{\sc pfv.3sgm-neg} come.{\sc pfv-2sg} with-{\sc 1pl.gen} {\sc dem.def.sgm}.day you\\
\glt `You didn't happen to have come with us that day' \hfill{{\sc ssr}}
\glt `It happened that you weren't with us that day' \hfill{(\emph{It}-Extraposition)}
\z  

\ea \label{lahaq}
\gll La\textcrh q-et g\textcrh aml-et lumi i-kbar, is-si{\.g}ra\\
achieve.{\sc pfv-3sgf} do.{\sc pfv-3sgf} lemons {\sc compar}-big, {\sc def}-tree.{\sc sgf}\\
\glt Lit: `She achieve she did bigger lemons, the tree'\\
\glt `It happened that (at some point earlier in the past), the tree produced bigger lemons'
\z

%J\emph{\textcrh abat}, whose literal meaning is 'crash', takes a number of different interpretations, as shown in the next chp.\footnote{It is interesting to have found a rather colloquial use of \emph{\textcrh abat} on the \isi{MLRS} corpus, where \emph{\textcrh abat}, in its imperfect morphology means: 'happen to be', interpreted as 'is located', in this specific context.}

%\ea 
%\gll J-a-\textcrh bat e{\.z}att fejn il-knisja g\textcrh al min ma j-af-x.\\
%3-{\sc frm.vwl}-crashes.{\sc imp.sgm} exactly near {\sc def-}church for who {\sc neg} 3-know.{\sc imp.sgm-neg}\\
%\glt Lit: It crashes exactly near the church, for who does not know (where the place is)\\
%\glt It is located exactly near the church, for those (of you) who don't know (where the place is).
%\z


%\eenumsentence{\label{habat}
%\item{
%\gll \textcrh abat wasal-t tard dakinhar\\
%crash.{\sc pv.3sgm} arrived.{\sc pv-1sg} later {\sc dem.def}.day\\
%It happened I arrived late that day}
%\item{
%\gll \textcrh abat li kien-u {\.g}a marr-u qabil-na\\
%crash.{\sc pv.3sgm} {\sc comp} be.{\sc pv.3-pl} already went.{\sc pv.3-pl} before-{\sc 1pl.acc}\\
%It happened that they had already gone before}
%\item{
%\gll Jekk j-a-\textcrh bat t-kun festa, kollox i-kun mag\textcrh luq\\
%if 3-{\sc frm.vwl}-crashes.{\sc imp.sgm} 3-be.{\sc imp.sgf} feast.{\sc sgf} all 3-be.{\sc imp.sgm} closed.{\sc pass.prt.sgm}\\
%If it happens to be a feast, everything will be closed}
%\item{\label{habatdisloc}
%\gll Qed j-a-\textcrh bat li kuldarba li n-a-qbad tal-linja (qed) n-asal tard\\
%{\sc prog} 3-{\sc frm.vwl}-crashes.{\sc imp.sgm} {\sc comp} every.once {\sc comp} 1-{\sc frm.vwl}-catch.{\sc sg} of.{\sc def-}line 1-arrive.{\sc sg} late\\
%It is happening that everytime I catch the bus, I arrive/I am arriving late}}

%One of the most frequent collocational uses of the defaultive perfect form of the \isi{verb} is (\ref{habattjb}) below, retrieved from the \isi{MLRS} corpus.
 
\protectedex{
\ea \label{habattjb}
\ea
\gll Kien \textcrh abat tajjeb li l-parlament Malt-i beda j-i-ddiskuti ...\\
be.{\sc pfv.3sgm} crash.{\sc pfv.3sgm} good.{\sc sgm} {\sc comp} {\sc def-}parliament.{\sc sgm} Maltese-{\sc sgm} start.{\sc pfv.3sgm} 3{\sc m}-{\sc epent.vwl}-discuss.{\sc impv.sg} ...\\
\glt `It happened well that the Maltese parliament started to discuss/started discussing ...' \hfill{(\isi{MLRS} v2.0)}
\ex
\gll g\textcrh ax \textcrh bat-t qbad-t lilu\\
because happen.{\sc pfv-1sg} catch.{\sc pfv-1sg} him\\
\glt `because I happened to have caught him'
\z
\z
}

%The following instance from the corpus is interestingly ambiguous such that the presence of the \isi{complementiser} could be either due to the presence of a default 3SGM vs. a real 3SGM instance on the copula, (although this is not necessary, as in the sentences in (\ref{habat}), or else it could be that \emph{hu} is not the copula, but rather the SUBJ, or even a FOC UDF of some sort. The presence of a \isi{complementiser} in such instances, i.e. when the embedded \isi{clause} involves some dislocation of sorts, is also present in (\ref{habatdisloc}) above.  

%\ea
%\gll Wie\textcrh ed mill-persun-i kkon{\.c}ernat-i \textcrh abat li hu canvasser kbir tieg\textcrh-u\\
%one.{\sc sgm} from.{\sc def}-person-{\sc pl} concerned.{\sc pl} happen.{\sc pv.3sgm} {\sc comp} he canvasser big.{\sc sgm} of-{\sc 3sgm.acc}\\
%\glt One of the persons concerned happens that he is a big canvasser of his \hfill{MLRS}
%\z

\subsection{Evidence in favour of a raising analysis}

%Primarily note that evidence that we definitely have the {\sc subj} present in some way or another in the embedded \isi{clause} comes from reflexivisation behaviour, whereby, irrespective of whether our account is to say that there is a gap or a \isi{pronoun} present in that position, the {\sc subj} is still in some way also present in the embedded \isi{clause} as well, and not simply present in the matrix.

Raising tests vary. Primarily, one needs to establish that a dependency exists between the matrix and the embedded \isi{clause}. In instances of (forward) {\sc ssr}, one needs to establish that the {\sc subj} is indeed present within the embedded \isi{clause}, for this to then also function as the {\sc subj} of the \isi{matrix clause}, which is where it is overtly expressed or pronominally incorporated. Additionally, one also needs to establish that the matrix {\sc subj} position is indeed non-thematic.

%To start with 
Establishing that the {\sc subj} of the embedded \isi{clause} is still salient in the overall dependency, and that it in fact exists even though it may not be pronounced, 
%is to 
would verify the expectation that if an embedded {\sc subj} is indeed available, then this should be able to reflexively bind a local \isi{direct object}. This is the case in (\ref{refldeher}).

\ea \label{refldeher}
\gll A\textcrh na n-i-dhr-u {\bf{n}}-\textcrh obb-{\bf{u}}$_{i}$ {\bf{lilna}} {\bf{nfus-na}}$_{i}$\\
we 1-{\sc frm.vwl}-appear.{\sc impv-pl} 1-love.{\sc impv-pl} us self.{\sc pl-1pl.gen}\\
\glt `We seem to love ourselves'
\z

%\ea
%\gll T-i-dher b\textcrh allikieku qal-u xi \textcrh a{\.g}a \textcrh abba l-fatt li (int)$_{i}$ t-\textcrh obb \textcrh afna lilek nnifs-ek$_{i}$\\
%2-{\sc frm.vwl}-appear.{\sc impv.sg} like.{\sc irrealis.comp} say.{\sc pfv.3-pl} some thing because {\sc def-}fact {\sc comp} you 2-love.{\sc impv.sg} handful you.{\sc sg.acc} self-{\sc 2sg.gen}\\
%\glt You seem as though they said something about the fact that you love yourself a lot
%\z  

Another argument in support of the fact that the {\sc subj} is also available in the embedded \isi{clause} comes from the behaviour of floating quantifiers: %, such that 
The quantifier \emph{kollha} `all.{\sc pl}' can appear in the matrix or the embedded \isi{clause}, as illustrated through (\ref{dehergf}). %and knowledge that the \emph{kollha} is an {\sc adj} part of the {\sc subj} of the embedded \isi{clause}, and not simply part of a matrix post verbal {\sc subj}, comes from its presence after the \isi{complementiser}, when this is present.
 
\ea \label{dehergf}
\gll (Kollha)     j-i-dhr-u                            li        (kollha)      marr-u            (kollha) flimkien\\
     all.{\sc pl} 3-{\sc frm.vwl-}appear.{\sc impv-pl} {\sc comp} all.{\sc pl} go.{\sc pfv.3-pl}   all.{\sc pl}       together\\
\glt `All appear to have gone together'
\z

%Note that the arguments in favour of a forward \isi{raising analysis}, whereby the {\sc subj} is a non-thematic \isi{external argument} in the matrix do not stop short to the agreement contrast between:

%\ea
%\ea
%\gll J-i-dher marr-et tajjeb\\
%3-{\sc frm.vwl}-appear.{\sc impv.sgm} go.{\sc pfv-3sgf} well\\
%\glt `It seems she did well'
%\ex
%\gll T-i-dher marr-et tajjeb\\
%3-{\sc frm.vwl}-appear.{\sc impv.sgf} go.{\sc pfv-3sgf} well\\
%\glt `She seems to have done well'
%\z
%\z

%Retaining our reference to {\sc 3sgm} default agreement here, one should mention that \isi{raising} verbs in Maltese display an interesting split between ones that allow for expletives as opposed to ones that don't, and which, as we will see, will have an influence on the \isi{subcategorisation frame} of these verbs. 

%\begin{table}[h] 
%\begin{tabular}{llll}
%Oblig AGR & AGR/default & defult \\\hline
%non-\emph{sar} `become' aspectualisers & \emph{deher} `appear' & \emph{{\.g}ara} `happen' \\
%\emph{g\textcrh ad-} `still' + VP & \emph{g\textcrh ad-} `still' + VP & \emph{fadal} `remain' \\
%\emph{il-} `long time' + VP & \emph{il-} `long time' + CP & \emph{baqa'} `remain'\\
%\emph{g\textcrh and-} & \emph{g\textcrh and-} + \emph{jkun} &\\
%& \emph{inzerta} `happen, occur' &\\
%& \emph{se\textcrh el} `happen, occur' &\\
%& \emph{\textcrh abat} `happen, occur' &\\
%& \emph{la\textcrh aq} `reach, happen' &\\
%& \emph{sar} `become' &\\
%& \emph{g\textcrh odd-} `almost' & \\
%& \emph{seta'} `be able' &\\
%\hline

%\end{tabular}
%\caption{A summary of the morphosyntactic agreement behaviours displayed by the different raising verbal predicates in Maltese}
%\label{aspectualisers2}
%\end{table}

%This is the main split in the behaviour of \isi{raising} predicates in Maltese. Nonetheless, faced with an impersonal \isi{predicate} at the bottom of the chain, one is able to observe that it is possible to raise the agreement morphology and not the actual {\sc subj} itself. Establishing that the impersonal predicates involve non-canonical {\sc subj}s, is the fact that these {\sc subj}s can indeed be raised to the matrix {\sc subj} position.

A piece of evidence that suggests that the {\sc subj} in the matrix \emph{is} non-thematic, as expected of the \isi{external argument} of a \isi{raising predicate}, is the fact that it is possible for the {\sc subj} to be {\sc pred}less as a consequence of the \isi{raising} of the {\sc 3sgm} impersonal morphology of the embedded impersonal \isi{verb}. %In i
Instances such as (\ref{ambig}) are 
%it is 
in principle ambiguous as to whether 
%it is 
this sort of \isi{raising} %that 
is involved, given that the \isi{raising} predicates \emph{donn-} `as though', \emph{se\textcrh el} `happen' and \emph{deher} `appear' all allow for an alternation with the \emph{It}-Extraposition construction.

\ea \label{ambig}
\ea
\gll Hawn donn-u/donn-ok qieg\textcrh d-a j-fettil-l-ek g\textcrh a{\.c}-{\.c}ikkulata\\
{\sc exist} as.though-{\sc 3sgm.acc}/as.though-{\sc 2sg.acc} {\sc prog-sgf} 3{\sc m}-decide.abruptly.{\sc impv.sg-dat-2sg} for.{\sc def-}chocolate\\
\glt `Here it seems/you seem to be craving for chocolate all of a sudden'
\ex
\gll Jekk se\textcrh el/s\textcrh il-t irnexxie-l-ek, g\textcrh ala ma \textcrh ad-t-x i{\.c}-{\.c}ans?\\
if happen.{\sc pfv.3sgm}/happen.{\sc pfv-2sg} manage.{\sc pfv.3sgm-dat-2sg} why {\sc neg} take.{\sc pfv-2sg-neg} {\sc def-}chance\\
\glt `If it/you happened to have managed, why didn't you take the chance?'
\ex
\gll J-i-dher/t-i-dher g\textcrh and-ek/kel-l-ek b{\.z}onn ftit mistrie\textcrh\\
3{\sc m}-{\sc frm.vwl}-appear.{\sc impv.sg}/2-{\sc frm.vwl}-appear.{\sc impv.sg} have-{\sc 2sg.gen}/be.{\sc pfv.3sgm}-have-{\sc 2sg.gen} need a.little rest\\
\glt `You seem to need/have needed some rest'
\z
\z 

On the other hand, if we consider what takes place in the case of \isi{aspectualiser} predicates such as {\sc repetitive}-expressing \emph{re{\.g}a'} and {\sc inceptive}-expressing \emph{qabad} lit. `catch' and \emph{beda} `start', 
%we find that as illustrated through 
the \isi{ungrammaticality} of the sentences in (\ref{nodefaultasp}), %these 
shows that they are not able to display an alternation with an \emph{It-}Extraposition, i.e. they do not take an alternative non-raised structure involving a default {\sc 3sgm} form. 

\ea \label{nodefaultasp}
\ea
\gll *J-e-r{\.g}a' n-a-g\textcrh mel xi \textcrh a{\.g}a\\
3{\sc m}-{\sc frm.vwl}-repeat.{\sc impv.sg} 1-{\sc frm.vwl}-do.{\sc impv.sg} some thing\\
\glt Intended: `I do something again'
\ex
\gll *J-a-qbad/j-i-bda n-a-g\textcrh mel xi \textcrh a{\.g}a\\
3{\sc m}-{\sc frm.vwl}-catch.{\sc impv.sg}/3{\sc m}-{\sc frm.vwl}-start.{\sc impv.sg} 1-{\sc frm.vwl}-do.{\sc impv.sg} some thing\\
\glt Intended: `I start to do something'
\z
\z
 
Due to the inability of \isi{aspectualiser} predicates
%' inability 
to alternate with the Expletive construction, the availability of the data in (\ref{3sgmchainraising}), consisting of sentences involving a number of stacked aspectualisers, clearly suggests that what is 
%we have 
taking place is the chained \isi{raising} of the default non-referential {\sc 3sgm} morphology of the impersonal \isi{verb} at the bottom of the dependency. We take this to imply that aspectualisers also allow for {\sc pred}less non-thematic {\sc subj}s, at least in specific constrained contexts such as this, i.e. ones involving impersonal verb-forms in the embedded \isi{clause} (and predicates with non-canonically indexed {\sc subj}s more broadly).

\ea \label{3sgmchainraising}
\ea
\gll Qorob/qrob-t biex j-e-r{\.g}a' j-a-qbad j-i-bda j-kol-l-i mara t-g\textcrh in-ni fid-dar\\
draw.close.{\sc pfv.3sgm}/draw.close.{\sc pfv-1sg} in.order.to 3{\sc m}-{\sc frm.vwl}-repeat.{\sc impv.sg} 3{\sc m}-{\sc frm.vwl}-catch.{\sc impv.sg} 3{\sc m}-{\sc frm.vwl}-start.{\sc impv.sg} 3{\sc m}-be.{\sc impv.sg}-have-{\sc 1sg.gen} woman 3{\sc f}-help.{\sc impv.sg-1sg.acc} in.{\sc def-}house\\
\glt Lit: `He was close/I was close in order to he repeats he starts he be to-me woman she helps me in the house'\\
\glt `I am close to once again start having a woman helping me in the house' \hfill{\citep[294]{Camilleri16}}
\ex
\gll Rama/\textcrh asel qis-u \textcrh abat \textcrh a j-i-ftil-l-i g\textcrh al bi{\.c}{\.c}a {\.c}ikkulata\\
arm.{\sc pfv.3sgm}/wash.{\sc pfv.3sgm} as.though-{\sc 3sgm.acc} crash.{\sc pfv.3sgm} {\sc prosp} 3{\sc m}-{\sc epent.vwl}-decide.abruptly.{\sc impv.sg-dat-1sg} for piece chocolate\\
\glt Lit: `He started as though he was on the verge of long.for.all.of.a.sudden for piece of chocolate'\\
\glt `I started as though I was on the verge of craving for a piece of chocolate' \hfill{\citep[294]{Camilleri16}}
\z
\z

%In this way, therefore, although they themselves do not allow for a default {\sc 3sgm}, and hence the impossibility of a *$\langle${\sc comp}$\rangle${\sc subj} subcategorisation, they nevertheless allow for {\sc 3sgm} default morphology \isi{raising} via their usual $\langle${\sc xcomp}$\rangle${\sc subj} \isi{subcategorisation frame}.
Additional evidence in support of the non-thematic status of the matrix {\sc subj} comes from the free availability of \isi{idiom} chunks in this position.\footnote{Differing behaviours will be discussed in \S4 with respect to the data in (\ref{noidiomsmtcr}).} %\footnote{More research needs to be done to better account for whether the presence/absence of the \isi{complementiser}, when this is allowed by the matrix \isi{raising predicate}, has any effect on the \isi{idiomatic}/literal interpretation of the \isi{idiom} chunk. In the contrast in (i) below it seems to me that in the presence of the \isi{complementiser} \emph{li} in (i b), the literal reading starts winning over the \isi{idiomatic} one.

%\ea 
%\ea 
%\gll Il-\textcrh abel j-i-dher/donn-u skappa (minn ta\textcrh t id-ej-na)\\
%{\sc def-}rope.{\sc sgm} 3-{\sc frm.vwl}-appear.{\sc impv.sgm}/as.though-{\sc 3sgm.acc} escape.{\sc pfv.3sgm} from under hand-{\sc pl-1pl.gen}\\
%\glt Lit: The rope seems to have escaped from under our hands\\
%\glt `The opportunity seems to have been lost'
%\ex \label{withli}
%\gll Il-\textcrh abel j-i-dher/donn-u li skappa\\
%{\sc def-}rope.{\sc sgm} 3-{\sc frm.vwl}-appear.{\sc impv.sgm}/as.though-{\sc 3sgm.acc} {\sc comp} escape.{\sc pfv.3sgm}\\
%\glt Lit: The rope seems to have escaped\\
%\glt `??The opportunity seems to have been lost'
%\z
%\z
%}

\protectedex{
\ea
\ea
\gll Daqqa t'id t-i-s\textcrh el t-a-g\textcrh mel il-{\.g}id, kultant\\
hit.{\sc sgf} of.hand 3{\sc f}-{\sc frm.vwl}-happen.{\sc impv.sg} 3{\sc f}-{\sc frm.vwl}-do.{\sc impv.sg} {\sc def-}benefit sometimes\\ 
\glt Lit: `A hit of hand happens it does the benefit sometimes'\\
\glt `Providing help or advice does well every now and then'
\ex
\gll Na\textcrh qa ta' \textcrh mar dehr-et (li) qatt m'hi se t-i-tla' s-sema, biss, {\.z}ied-u j-i-sfida-w, u rnexxie-l-hom.\\
bray.{\sc sgf} of donkey appear.{\sc pfv-3sgf} {\sc comp} never {\sc neg}.{\sc cop.3sgf} {\sc prosp} 3{\sc f}-{\sc frm.vwl}-go.up.{\sc impv.sg} {\sc def-}sky, however add.{\sc pfv.3-pl} 3-{\sc epent.vwl}-defy.{\sc impv-pl} {\sc conj} manage.{\sc pfv.3sgm-dat-3pl}\\
\glt Lit: `A bray of a donkey appeared that never it is going to reach the sky/heaven, but/however they increased they defy, and they managed'\\
\glt `The cry of the poor or someone insignificant appeared that it was not going to reach far, however, they increased in their defiance, and they managed (to get what they wanted)'
\ex 
\gll Ri{\textcrh} ta' sieg\textcrh a deher g\textcrh odd-u naddaf qieg\textcrh a\\
wind.{\sc sgm} of hour appear.{\sc pfv.3sgm} almost-{\sc 3sgm.acc} clean.{\sc cause.pfv.3sgm} trashing.floor\\
\glt Lit: `Wind of an hour appeared almost cleaned the place where wheat is scattered'\\
\glt `An instant/moment can and may seem to result in more important things'
\z
\z 
}

As discussed in the literature (e.g. \Citealt{davies2008grammar}), if the matrix \isi{predicate} is a \isi{raising} one, semantic equivalence is expected, irrespective of whether the \isi{predicate} in the (deepest) embedded \isi{clause} is active or passive. Observe this behaviour through the constructions below. 

\protectedex{
\ea
\ea
\gll Beda/baqa' j-i-{\.g}bor l-iltiema\\
start.{\sc pfv.3sgm}/remain.{\sc pfv.3sgm} 3{\sc m}-{\sc frm.vwl}-collect.{\sc impv.sg} {\sc def-}orphan.{\sc pl}\\
\glt `He started/continued gathering the orphans' \hfill{(Active)}
\ex
\gll Bde-w/baqg\textcrh-u j-i-n-{\.g}abr-u l-iltiema\\
start.{\sc pfv.3-pl}/remain.{\sc pfv.3-pl} 3-{\sc epent.vwl}-{\sc pass}-gather.{\sc impv-pl} {\sc def-}orphan.{\sc pl}\\
\glt `The orphans started/continued to be gathered' \hfill{Passive: (\citealt[20]{Alotaibi2013})}
\z
\z
}

\ea
\ea
\gll T-i-dher donn-ha/donn-u ta-t xi flus g\textcrh all-karit{\'a}\\
3{\sc f}-{\sc frm.vwl}-appear.{\sc impv.sg} as.though-{\sc 3sgf.acc}/as.though-{\sc 3sgm.acc} give.{\sc pfv-3sgf} some money for.{\sc def-}charity\\
\glt `She seems as though she gave some money to charity' \hfill{(Active)}
\ex
\gll J-i-dhr-u donn-hom/donn-u n-g\textcrh ata-w xi flus g\textcrh all-karit{\'a}\\
3-{\sc frm.vwl}-appear.{\sc impv-pl} as.though-{\sc 3pl.acc}/as.though-{\sc 3sgm.acc} {\sc pass}-give.{\sc pfv.3-pl} some money for.{\sc def-}charity\\
\glt `Some money for charity seem to have been given' \hfill{(Passive)}
\z
\z

Passivisation data also provides  yet another context where \isi{idiom} chunks can come to function as the matrix {\sc subj}, once passivisation promotes the \isi{idiom} from {\sc obj} to {\sc subj} position. 

\ea \label{pfvraising} 
\ea
\gll T-i-dher/donn-ha da\textcrh \textcrh l-et fellus f'mo\textcrh\textcrh-ha\\
3{\sc f}-{\sc frm.vwl}-appear.{\sc impv.sg}/as.though-{\sc 3sgf.acc} enter.{\sc cause.pfv-3sgf} chick.{\sc sgm} in.brain-{\sc 3sgf.gen}\\
\glt Lit: `She seems/She's as though she caused to enter a chick in her brain'\\
\glt `She seems to have fixed an idea/doubt in her mind' 

\newpage 
\ex
\gll Fellus kbir j-i-dher d-da\textcrh\textcrh al f'mo\textcrh\textcrh-ha\\
chick.{\sc sgm} big.{\sc sgm} 3{\sc m}-{\sc frm.vwl}-appear.{\sc impv.sg} {\sc pass}-enter.{\sc cause.pfv.3sgm} in.mind-{\sc 3sgf.gen}\\
\glt Lit: `A big chick appears to have been entered in her mind'\\
\glt `A fixed idea seems to have got to her mind'
\z
\z

\ea
\ea
\gll Hawn donn-u qatg\textcrh a-l-hom i{\.z}-{\.z}ej{\.z}a\\
here as.though-{\sc 3sgm.acc} cut.{\sc pfv.3sgm-dat-3pl} {\sc def-}breast.{\sc sgf}\\
\glt Lit: `Here it seems he cut on-them the breast'\\
\glt `It seems that their illegal source has been cut'
\ex
\gll Hawn i{\.z}-{\.z}ej{\.z}a donn-ha n-qatg\textcrh-et-i-l-hom\\
here {\sc def-}breast.{\sc sgf} as.though-{\sc 3sgf.acc} {\sc pass}-cut.{\sc pfv-3sgf-epent.vwl-dat-3pl}\\
\glt Lit: `Here the breast seems it has been cut on-them'\\
\glt `The illegal source has been cut'
\z
\z

Further evidence in support of the claim that the constructions under discussion involve \isi{raising} predicates comes from scoping effects and the availability of both a narrow and wide reading of a quantified {\sc subj}. A narrow reading would not have been available for a control/equi \isi{predicate}, since the {\sc subj} of such predicates does not originate in the embedded \isi{clause}, but is in fact a thematic argument of the matrix itself.

\ea
\gll \textcrh add ma j-i-dher/qis-u j-o-qg\textcrh od hemm\\
no.one {\sc neg} 3{\sc m}-{\sc frm.vwl}-appear.{\sc impv.sg}/as.though-{\sc 3sgm.acc} 3{\sc m}-{\sc frm.vwl}-live.{\sc impv.sg} there\\
\glt `It seems to be the case that no one lives there'\\ \glt \hfill{(\emph{seem} scopes over \emph{no one}: Narrow Scope)}\\
\glt `There is no one such that he/she seems to live there'\\
\glt \hfill{(\emph{no one} scopes over \emph{seem}: Wide Scope)}
\z 

Having established a number of properties that provide evidence for \isi{raising} constructions, there remains another, which essentially deals with meteorological {\sc subj}s. The availability of such {\sc subj}s (as in (\ref{metsubj})) uncontroversially implies a non-thematic {\sc subj} status.

\ea \label{metsubj}
\ea \label{nolexpredsubj}
\gll Ix-xita donn-ha ma t-rid-x t-e-hda\\
{\sc def-}rain.{\sc sgf} as.though-{\sc 3sgf.acc} {\sc neg} 3{\sc f}-want.{\sc impv.sg-neg} 3{\sc f}-{\sc frm.vwl}-relent.{\sc impv.sg}\\
\glt `The rain appears as though it does not want to relent' 
\ex \label{nolightsubj}
\gll Il-kes\textcrh a t-i-dher qieg\textcrh d-a {\.z}-{\.z}id\\
{\sc def-}cold.{\sc sgf} 3{\sc f}-{\sc frm.vwl}-seem.{\sc impv.sg} {\sc prog-sgf} 3{\sc f}-increase.{\sc impv.sg}\\
\glt `The cold seems to be increasing'
\z
\z

Such constructions appear to be the usual forward \isi{raising} constructions we have been considering up till now, i.e. \isi{raising} constructions where the expressed {\sc subj}, be it overt or an incorporated \isi{pronoun}, is in the matrix. It however seems to us that backward \isi{raising} also exists in Maltese, as argued in \citet[292]{Camilleri16}, following data such as that in (\ref{backward??}) below. 

\ea \label{backward??}
\ea \label{lexpredsubj}
\gll Baqg\textcrh-et nie{\.z}l-a \textcrh afna xita\\
remain.{\sc pfv-3sgf} down.{\sc act.ptcp-sgf} a.lot rain.{\sc sgf}\\
\glt Lit: `She remained downing the rain'\\
\glt `It kept raining' 
\ex \label{lightsubj}
\gll Bdie-t t-a-g\textcrh mel xebg\textcrh a s\textcrh ana\\
begin.{\sc pfv-3sgf} 2-{\sc frm.vwl}-do.{\sc impv.sg} \isi{smacking} heat.{\sc sgf}\\
\glt Lit: `She started she does \isi{smacking} heat'\\
\glt `It started being very hot' \hfill{\citep[19]{Alotaibi2013}}
\z
\z

In both instances in (\ref{backward??}), the phrases \emph{\textcrh afna xita} and \emph{xebg\textcrh a s\textcrh ana}, which are the respective {\sc subj}s shared between the matrix \isi{aspectualiser} and the lexical \isi{predicate}, are not able to neutrally occur in front of the \isi{aspectualiser} in the matrix, and can thus only ever surface in the embedded \isi{clause}. 
%In passing we here 
We suggest in passing that this data may display instances of backward \isi{raising} structures \citep{PotsdamPolin2012}, where %there is 
only `covert' \isi{raising} to the matrix is
%, that's 
involved. Linearly, on the other hand, the {\sc subj} is retained as an overt DP in the embedded \isi{clause}. If our hypothesising of a backward \isi{raising analysis} is on the right track, then it would account for why we are not able to get neutrally ordered pre-verbal {\sc subj}s in (\ref{backward??}), but yet we still get the agreement matching on the \isi{aspectualiser} in the matrix. The agreement available comes about as a result of the structure-sharing of the {\sc subj} in the embedded \isi{clause} with that in the matrix. 


%In general: {\sc neg} placement on either {\sc pred}:

%\ea
%\ea
%\gll Ma j-i-dhir-x/t-i-dhir-x marr-et tajjeb\\
%{\sc neg} 3-{\sc frm.vwl}-appear.{\sc impv.sgm-neg}/3-{\sc frm.vwl}-appear.{\sc impv.sgf-neg} go.{\sc pfv-3sgf} good\\
%\glt It doesn't seem she did well/She doesn't seem to have done well
%\ex
%\gll J-i-dher/T-i-dher ma marr-it-x tajjeb\\
%3-{\sc frm.vwl}-appear.{\sc impv.sgm}/3-{\sc frm.vwl}-appear.{\sc impv.sgf} {\sc neg} go.{\sc pfv-3sgf-neg} good\\
%\glt It seems she didn't do well/She seems to not have done well
%\z
%\z

%Apart from the availability of a $\langle${\sc comp}$\rangle${\sc subj} and the $\langle${\sc xcomp}$\rangle${\sc subj} \isi{subcategorisation frame} alternations, there is evidence for a thematic {\sc subj} associated with \emph{deher}. From data such as (\ref{deherthematic}), we observe non-default agreement on \emph{deher}, but where no \isi{raising} is present. A {\sc 3pl pro} is what we have to assume is filling in the thematic {\sc subj gf}, satisfying the \isi{subcategorisation frame}: $\langle${\sc subj}, {\sc comp}$\rangle$. There is however an additional requirement associated with this availability. The {\sc subj} must be a Perceptual Source (Psource), i.e. where one can deduce the truth of the utterance from something which the {\sc subj} is showing, i.e. something about the {\sc subj}'s aspect of their demeanour etc. The Psource should not be understood as a thematic role/argument: `Psource is not a thematic role assigned to a semantic argument' (Asudeh and Toivonen, 2012, pp. 136, 142). Rather, it is a semantic entailment conditioning the participant, in this case that which takes the \isi{syntactic} function of a {\sc subj}, which can either come from some aspect of the individual, or from something in the context (i.e. event Psource) that is telling us more about the {\sc subj} with respect to what is reported by the \isi{verb}. This structure-sharing is not possible, potentially, under an account that: 'dummy arguments cannot be topics' (Dalrymple/King 2000: 13).\footnote{The importance of (\ref{deherthematic}) is that it completely excludes any possible TOP status. We cannot have a TOP analysis, when there is no TOP involved, and only AGR features on the \isi{verb}. Of course, it is in fact possible to have TOPs, and which TOPs are also endowed with a Psource role.

%\ea
%\ea
%\gll Dat-tfal, j-i-dhr-u li l-missier vjolenti\\
%{\sc dem.pl.def-}children 3-{\sc frm.vwl}-appear.{\sc imp-pl} {\sc comp} {\sc def-}father violent\\
%\glt These children, they seem that the father is violent (i.e. it is evident from their bruises, burns etc.)
%\ex
%\gll Dal-karozzi, j-i-dhr-u li l-magna spi{\.c}{\.c}ut-a\\
%{\sc dem.pl.def-}children 3-{\sc frm.vwl}-appear.{\sc imp-pl} {\sc comp} {\sc def-}machine.{\sc sgf} ended.{\sc pass.prt-sgf}\\
%These cars, they seem that their machine is a total loss (by the big crash they endured and by the oil on the floor)
%\z
%\z
%} 

%\ea \label{deherthematic}
%\gll It-tfal, j-i-dhr-u li Marija \textcrh ad-et gost warakollox!\\
%3-{\sc frm.vwl}-seem.{\sc impv-pl} {\sc comp} Mary took.{\sc pfv-3sgf} fun after.all\\
%\glt They seem that Mary had fun afterall
%\z

%One should here add that it is possible to also have the additional intercalation of perceptual report verbs such as the pseudo-verbs \emph{qis-} and \emph{donn-}, as in (\ref{dehercascadenonthem}).

%\ea \label{dehercascadenonthem}
%\gll J-i-dhr-u qis-hom Marija \textcrh ad-et gost warakollox!\\
%3-{\sc frm.vwl}-appear.{\sc impv-pl} as.though-{\sc 3pl.acc} Mary take.{\sc pfv-3sgf} fun after.all\\
%\glt They seem as though Mary had fun after all! \hfill{Camilleri et al., 2014, p. 198}
%\z

%However, as identified in Camilleri et al. 2014, in contexts such as (\ref{dehercascadenonthem}), it is not possible to substitute \emph{deher} with the pseudo-verbs \emph{qis-} and \emph{donn-}, which otherwise can substitute \emph{deher} in almost all contexts (see below). Rather, only an optional co-occurrence is possible. This identified behaviour thus implied that \emph{qis-}/\emph{donn-} never takes a thematic {\sc subj}. Camilleri et al. (2014, p. 198) verbalise this as follows:\\
%\\
%`A question then arises as to which \isi{predicate} in the cascade takes a thematic \isi{subject}. A natural assumption is that it is the lowest \isi{predicate} in the {\sc xcomp} cascade that takes a thematic \isi{subject} and a \isi{complement clause}, while the higher verbs and pseudo-verbs are \isi{raising} predicates. However this leads us to expect that the pseudo-verbs can occur (alone) in structures such as [(\ref{deherthematic})], and this is not the case [as illustrated through the ungrammatical (\ref{qisnodeherbad})]. We conclude therefore that \emph{deher} alone has an additional subcategorisation with a thematic {\sc subject}'. 

%\ea \label{qisnodeherbad}
%\gll *It-tfal qis-hom Marija \textcrh ad-et gost\\
%as.though-{\sc 3pl.acc} {\sc comp} Mary took.{\sc pfv-3sgf} fun after.all\\
%\glt Intended: They're as though Mary had fun afterall \hfill{Camilleri, 2016, p. 182}
%\z

%The \emph{f-}structure associated with (\ref{dehercascadenonthem}), taken from Camilleri et al. (2014, p. 198) is provided below.

%\ea \label{f-structureofcascade}
%\begin{avm}
%\[ {\sc pred} & `jidhru $\langle${\sc subj}, {\sc xcomp}$\rangle$'\\
   %{\sc subj} & \[{\sc pred} & `{\sc pro}'\\
                    %{\sc pers} & 3\\
                    %{\sc num} & {\sc pl}\] [1] \\   
   %{\sc xcomp} & \[{\sc pred} & `qishom $\langle${\sc comp}$\rangle${\sc subj}'\\
                      %{\sc subj} & {\[\phantom{PRED}&\phantom{`FOOTBALL'} \] [1] }\\
                      %{\sc comp} & \[ {\sc pred} & `\textcrh adet $\langle${\sc subj}, {\sc obj}$\rangle$'\\
                                          %{\sc subj} & \[{\sc pred} & `Marija'\]\\
                                          %{\sc obj} & \[{\sc pred} & `gost'\\
                                                             %{\sc def} & {\sc -}\\
                                          %{\sc adj} & \{\[{\sc pred} & `warakollox'\]\}\] \] \] \]

%\end{avm}
%\z

%Although not made explicit in Camilleri et al. (2014), but expanded upon further in Camilleri (2016) (see therein for more detail), we here assume that the {\sc 3pl} agreement on the pseudo-\isi{verb} in (\ref{dehercascadenonthem}) is derived from a parasitic morphological relation (in parallel to the type one finds in \ili{Germanic} languages (\Citet{Wiklund, 2001}, \Citet{Sells, 2004}, \Citet{Wurmbrand, 2010}) between the \emph{deher} `seem' \isi{predicate} and the pseudo-\isi{verb}, given that \emph{qis-} in (\ref{dehercascadenonthem}) could have in fact taken a {\sc 3sgm} form, without affecting in any way the sentence's interpretation, as in (\ref{deherqis3sgm}) below.

%\ea \label{deherqis3sgm}
%\gll J-i-dhr-u qis-u Marija \textcrh ad-et gost warakollox!\\
%3-{\sc frm.vwl}-appear.{\sc impv-pl} as.though-{\sc 3sgm.acc} Mary take.{\sc pfv-3sgf} fun after.all\\
%\glt They seem as though Mary had fun after all! \hfill{Camilleri, 2016, p. 186}
%\z



%In the generative literature one finds accounts such as those in \Citet{Alexiadou \& Anagnostopoulou 1999}, for example, where Subjunctive clauses out of which \isi{raising} is possible are analysed as involving a defective embedded T, which lacks the inheritance of a {\sc tense} feature from the C that governs it. This account based on temporal defectiveness, fails, however, since as pointed out by \Citet{Zeller 2006}, Subjunctive clauses can still have an independent interpretation. Maltese also displays this sort of behaviour, i.e. where the embedded \isi{clause} may display a distinct temporal anchoring than that of the matrix (\ref{temporaladjs}). Having said this, however, one should mention that compatibility with temporal adjuncts need not necessarily imply the \isi{syntactic} presence of a specific feature. Rather, the data below merely suggests that there exists a specific temporal anchoring in the semantic interpretation of the embedded \isi{clause} that can be distinct from that in the matrix.

%\ea \label{temporaladjs}
%\ea
%\gll Mill-evidenza li g\textcrh and-na f'id-ej-na illum, Marija t-i-dher (li) kien-et telq-et {\.g}img\textcrh a il-u\\
%from.{\sc def-}evidence {\sc comp} at-{\sc 1pl.gen} in.hand-{\sc pl-1pl.acc} today Mary 3-{\sc frm.vwl}-appear.{\sc impv.sgf} {\sc comp} be.{\sc pfv-3sgf} leave.{\sc pfv-3sgf} week to-{\sc 3sgm.acc}\\ 
%\glt `From the evidence we have in our hands today, Mary seems to have left a week ago' 
%\ex
%\gll Milli j-i-dher, Marija donn-ha g\textcrh and-ha t-mur il-{\.g}img\textcrh a d-die\textcrh l-a\\
%from.{\sc comp} 3-{\sc frm.vwl}-appear.{\sc impv.sgm} Mary as.though-{\sc 3sgf.acc} at-{\sc 3sgf.gen} 3-go.{\sc impv.sgf} {\sc def-}week.{\sc sgf} {\sc def-}enter.{\sc act.ptcp-sgf}\\
%\glt `From what it looks like, Mary seems that she must go next week'
%\ex 
%\gll Ilbiera{\textcrh} Marija dehr-et (li) be\textcrh sieb-ha t-i-{\.g}i mag\textcrh-na g\textcrh ada\\
%yesterday Mary appear.{\sc pfv-3sgf} {\sc comp} in.though-{\sc 3sgf.acc} 3-{\sc frm.vwl}-come.{\sc impv.sgf} with-{\sc 1pl.acc} tomorrow\\
%\glt `Mary yesterday appeared to be considering coming with us tomorrow' 
%\z
%\z  

%Zeller's (2006) account on the matter is based on Carlson's (1992) argumentation, where he considers subjunctives to be in between indicatives and infinitives, which in turn make her propose that such clauses should be assigned a [+ N] feature. (Also see Lehmann (  )). Critiques of this account have included the fact that it is not the case that subjunctive clauses outside of \ili{Bantu} have noun-like behaviours. Another analysis of hyperraising present in the literature includes a small \isi{clause} analysis (Deprez 1992), where it is assumed that a \isi{raising predicate} takes a small \isi{clause} as its argument. 
%In this study, we will be exploring an account which finds its origins in \Citet{joseph1976raising} and \Citet{perlmutter1979syntactic}. In their discussions of hyperraising constructions in \ili{Greek}, they claim that such constructions should be analysed as an instance of copy \isi{raising}. More recently, \Citet{Ademola10} provides a crosslinguistic study of hyperraising, and he analyses these as involving a \isi{pronominal} copy that is left in-situ in the embedded {\sc subj} position once this is `raised' to the matrix position, hence copy \isi{raising}. Ademola-Adeoye's (2010) analysis is thus based on the premise that \isi{raising} out of a finite complement \emph{should} entail the presence of a \isi{pronoun} in the embedded {\sc subj} position, which {\sc subj} must be a covert \emph{pro}, and which in turn provides an explanation for Ura's generalisation and the crosslinguistic constraint that hyperraising is only present in `pro-drop' languages. It is the restriction to a \emph{covert} `pro' that makes hyperraising different from copy \isi{raising} proper, which otherwise involves overt \isi{pronominal} forms, at least in Ademola-Adeoye's (2010) analysis. %His view of hyperraising and the theory of resumption collapses the differences that pertain to A vs. {\={A}} distinctions, such that resumption is no longer viewed as a phonomenon that simply deals with constructions that have to do with {\={A}} syntax. This collapse in fact follows analyses such as those in McCloskey (????) and Sells (????). 


%Finite \isi{raising} has not been discussed in {\sc lfg}, except for a passing reference in \Citet[p. 4]{Arka:control} in his discussion on Indonesian, where he claims that the language allows for \isi{raising} from finite clauses, for which he provides an \emph{f-}control analysis, i.e. one in parallel with the analysis for \isi{raising} from non-finite clauses. Finite control has however been discussed in \Citet{Asudeh05} with respect to Serbo-Croatian. The analysis of finite control assumes that a {\sc pred} `pro' fills in the {\sc subj} position in the embedded \isi{clause}. This is in contrast to the structure-sharing analysis that is otherwise assumed in exhaustive and partial non-finite control structures. For Asudeh (2005), finite control is to control as copy \isi{raising} is to \isi{raising} (p. 509). Before delving in a discussion on copy-\isi{raising}, where we do in fact find a \isi{pronoun} in the embedded \isi{clause}, and which could be in a long-distance, as expected from \isi{anaphoric} chaining in general, as opposed to the restriction to local chaining in \isi{raising} constructions, we here mention that \isi{raising} in Maltese has in Camilleri et al. (2014) been analysed as involving functional-control, irrespective of the nature of the verb-form in the embedded \isi{clause}. 

%\ea 
%\ea \label{perfdeher}
%\gll (Marija) t-i-dher (li) kien-et marr-et tajjeb\\
%Mary 3-{\sc frm.vwl}-appear.{\sc impv.sgf} {\sc comp} be.{\sc pfv-3sgf} go.{\sc pfv-3sgf} good\\
%\glt Lit: `Mary seems that she had done'\\
%\glt `Mary seems to have done well'
%\ex
%\gll (Marija) t-i-dher li hi, kien-et marr-et tajjeb\\ 
%Mary 3-{\sc frm.vwl}-appear.{\sc impv.sgf} {\sc comp} she be.{\sc pfv-3sgf} go.{\sc pfv-3sgf} good\\
%\glt Lit: `Mary seems that she had done'\\
%\glt `Mary seems to have done well'
%\ex
%\gll *T-i-dher$_{i}$ li Marija$_{i}$ kien-et marr-et tajjeb\\
%3-{\sc frm.vwl}-appear.{\sc impv.sgf} {\sc comp} Mary be.{\sc pfv-3sgf} go.{\sc pfv-3sgf} good\\
%\glt `*She$_{i}$ seems that Mary$_{i}$ had done well'
%\z
%\z   



%\ea
%\gll Marija t-i-dher (li) kel-l-ha t-i-{\.g}i l-{\.g}img\textcrh a l-o\textcrh r-a\\
%Mary 3-{\sc frm.vwl}-appear.{\sc impv.sgf} {\sc comp} be.{\sc pfv.3sgm-dat-3sgf} 3-{\sc frm.vwl}-come.{\sc impv.sgf} {\sc def-}week.{\sc sgf} {\sc def-}other-{\sc sgf}\\
%\glt `Mary seems that she had to come last week'  
%\z 


%If the \emph{c-}structure based claim in Ademola-Adeoye (2010) that a covert `pro' is available in the embedded \isi{clause} is carried forward into an \emph{f-}structure based \isi{syntactic} account, then it is not possible to claim that we have functional-control or structure-sharing involved. This is because if the {\sc pred} values of the matrix and embedded {\sc subj}s are distinct, then a violation of the Uniqueness Principle, which as discussed in \S2.1 requires {\sc pred} values to be distinct, will result. Ademola-Adeoye's (2010) analysis of hyperraising as copy \isi{raising} involving a null pro in {\sc subj} position translates in the presence of a {\sc pred} {\sc pro} filling in the {\sc comp subj} in {\sc lfg} terms, such that the relation between the matrix and embedded {\sc subj}s is one of \isi{anaphoric control} due to the binding between the {\sc gf}s. Anaphoric control is the analysis associated with copy \isi{raising} in Asudeh (2005); \Citet{AshT:12}.\\
%\\
%The question we are faced with here, particularly in the absence of discussions on hyperraising in {\sc lfg}, is whether we really want to say that hyperraising in Maltese is a copy \isi{raising} structure but where the {\sc subj} copy is covert. We will illustrate further in \S5 below how Maltese \emph{does} have copy \isi{raising} constructions but hyperraising cannot be treated in the same way. We thus provide evidence in support of our claim that a copy \isi{raising analysis} cannot be correct for hyperraising constructions in Maltese, even if at the \emph{c-}structure \isi{syntactic representation}, a {\sc pro} satisfying the {\sc subj gf} of the embedded \isi{clause} is indeed available, by virtue of the morphosyntactic agreement present, which is available through the agreement on the element heading the IP, be it a temporal or modal auxiliary. We will argue that the {\sc pro} that is available to the \emph{c-}structure syntax is however `inactive' with respect to the \emph{f-}structure, i.e. where for \emph{f-}structure purposes, this equals to a gap. `Inactive' pronouns are thus in contrast with `syntactically active' \isi{pronoun} types, which \emph{do} fill {\sc pred} {\sc pro} functions in the \emph{f-}structure. Under such an analytic proposal, for \emph{f-}structure purposes, hyperraising, at least from the evidence present in Maltese, must be analysed as a local dependency that involves functional-control, and not \isi{anaphoric binding}.\\ 
%\\
%Following our account of {\sc cr} in the \S?? below, the evidence we provide in support of our proposal, are threefold. In Maltese, as opposed to the evidence one finds from the \ili{Bantu} data set discussed in Ademola-Adeoye, distinct readings are present between hyperraised (i.e. local {\sc subj}-to-{\sc subj}) vs. copy raised (i.e. involving an overt `pro' in local non-{\sc subj} contexts, or `pro' in any other non-local context) constructions. Secondly, \isi{idiom} chunks lose their \isi{idiomatic interpretation} in copy \isi{raising} contexts, but do not do so in hyperraising contexts, as opposed to data in \ili{Bantu}. Thirdly, the contrast between long-distance copy \isi{raising} behaviour vs. strict local {\sc subj} behaviour in hyperraising. Additionally, since saying that there is a gap is not the same as saying there is a `null' {\sc pro}, we discuss mainstream tests that can help us establish whether these pronouns are truly behaving like gaps, or whether they are really pronouns at the \emph{f-}structure level that liaison with the \emph{s-}structure. \S6 discusses {\sc cr} constructions, after which we can then be in a position to better anchor our evidence in favour of the hyperraising analysis being prepared here. 
%it is possible to have an overt `pro', and in such context the \isi{raising} relation must be one involving \isi{anaphoric control}; d. behaviours of the `pro'. This latter fact comes from the proposal pushed in previous accounts that resumption need not be a phenomenon related with {\={A}}-constructions alone, but can be related with A-movement structures as well. In fact, for an {\sc lfg} account, there is in fact no difference whatsoever in the mechanism employed, when anaphoric-binding is involved. In this way I am will delve into some detail with respect to differences between null \isi{pronoun} and gaps, which are two distinct matters, but which Ademola-Adeoye does not discuss in his account. We here aim to collapse the differences between A- and {\={A}}-stuctures by illustrating how at the \emph{f-}structure level, \isi{raising}/copy \isi{raising} and unbounded distance dependencies, such as relative clauses, which involve resumption, by the behaviour obtained at the \emph{f-}structure, where the most crucial constraint is that the highest {\sc subj} (Borer's Highest {\sc subj} restriction)/local {\sc subj} involves a gap, and hence structure-sharing/functional control/binding. 
 
\section{Copy raising}

We consider another type of \isi{raising} structure in Maltese: copy \isi{raising}. Copy \isi{raising} ({\sc cr}) involves `a construction in which some constituent appears in a non-thematic position with its thematic position occupied by a \isi{pronominal} copy' \citep{PotRun}. In English, unlike what is the case `in infinitival {\sc ssr}, in {\sc cr}, the \isi{predicate} takes a tensed \isi{clause} complement introduced by one of the particles \emph{like}, \emph{as if}, or \emph{as though}' \Citep[433]{PotRun} which, following \Citet{Maling:Adjectives} and \Citet{Heycock}, are prepositions. The same view is upheld in \Citet{asudeh2012copy} and \Citet{landau2011predication}. Such prepositions are then assumed in \Citet{fujii07} to take a \isi{complement clause}, given that these complements display the same conditions as the \emph{that}-trace effect (p. 301). A CP complement analysis is motivated, and is in turn taken to imply an account where `copy \isi{raising} involves overt \isi{raising} out of a finite CP' (p. 302). In \Citet{asudeh2012copy} the \emph{in-situ} copy pronouns are  analysed just as other resumptive pronouns. As stated in \Citet[325]{asudeh2012copy}, the difference between resumptives in copy \isi{raising} constructions vs. those in \isi{unbounded discourse dependency} structures is that the relation between the matrix non-thematic {\sc subj} and the embedded copy \isi{pronoun} is `lexically-controlled', as opposed to what is the case in unbounded discourse dependencies. Illustrations of {\sc cr} constructions in English are provided in (\ref{engcr}), with the copy \isi{pronoun} represented in bold.

\ea \label{engcr}
\ea There seems like {\bf{there}} are problems \hfill{\Citep[454]{PotRun}}%\footnote{According to Asudeh \& Toivonen, this specific example involves `chained \isi{raising}'/`double \isi{raising}'.}
\ex Tom seemed to me as if {\bf{he}} had won \hfill{\Citep[332]{asudeh2012copy}}
\ex Tom seemed like Bill hurt {\bf{him}} again \hfill{\Citep[346]{asudeh2012copy}}
\z
\z

Apart from the presence of a \isi{pronoun} in the embedded \isi{clause}, and the finiteness of the \isi{clause} (at least in English), another property that distinguishes \isi{raising} or \emph{it}-expletive constructions from copy \isi{raising} ones is that the {\sc subj} in the latter must be obligatorily interpreted as a perceptual source ({\sc psource}): `a copy \isi{raising} \isi{subject} is interpreted as the {\sc psource} -- the source of perception -- and ascribing the role of {\sc psource} to the \isi{subject} is infelicitous if the individual in question is not perceivable as the source of the report' \Citep[334]{asudeh2012copy}. \emph{It}-expletive and non-{\sc cr} constructions allow for both an Individual or Event Psource reading. In (\ref{EventPsource}) below, where we have a usual non-copy raised construction, the available Event Psource reading is made obvious by the {\sc adj} complement involved. %As Landau (2011, pp. 787-788) would have it, `some perceptual event is entailed, owing to the lexical semantics of perceptual source verbs', and in fact, verbs such as `appear/seem' and `as.though' pseudo-verbal predicates may be considered as the `most bleached perceptual source verbs'. 

 \ea \label{EventPsource}
 \gll Donn-hom/-u / qis-hom/-u qed j-a-qra-w ktieb tajjeb xi kwiet hawn!\\
 as.though-{\sc 3pl.acc/-3sgm.acc} / as.though-{\sc 3pl.acc/-3sgm.acc} {\sc prog} 3-{\sc frm.vwl}-read.{\sc impv-pl} book.{\sc sgm} good.{\sc sgm} what silence {\sc exist}\\
 \glt `It's as though they are reading a good book, how quiet it is!/They're as though they're reading a good book' \hfill{\citep[181]{Camilleri16}}  
 \z
 
%This is in marked contrast with examples of putative SSR, such as (40). In these cases, the PSOURCE can be the individual or any other aspect of the eventuality. Thus (49) might be felicitously uttered after entering a room and discovering that she was not present in the room, corresponding to an epistemic reading (concluding from the evidence). Similarly, a scenario for (50) might be one in which the ?she? in question habitually puts on slippers when returning to the house, again as a conclusion from the evidence (the absence of the slippers).
%(49) T-i-dher g?a telq-et 3-FRM.VWL-seem.IPFV.SGF already leave.PFV-3SGF
%She seems to have left already (e.g. the room is empty).
%(50) T-i-dher g?ie-t mill-mixi 3-FRM.VWL-seem.IPFV.SGF come.PFV-3SGF from.DEF-walking
%She seems to be back from walking (e.g. her slippers have gone). Camilleri et al, p. 193

A {\sc cr} structure with an obligatory {\sc psource} rendering of the {\sc subj} is (\ref{compobjrp}). `This is infelicitous if inferred from a pile of files on the desk, but fully appropriate if she is present and looking panicky and stressed. That is, this sentence is only appropriate then if `she' is the direct source of perception' \Citep[193]{CES:LFG14}.

\ea \label{compobjrp}
\gll {\bf{T}}-i-dher {\.g}{\`a} ta-w-{\bf{ha}} xebg\textcrh a xog\textcrh ol x't-a-g\textcrh mel\\
3{\sc f}-{\sc frm.vwl}-appear.{\sc impv.sg} already give.{\sc pfv.3-pl-3sgf.acc} \isi{smacking} work what.3{\sc f}-{\sc frm.vwl}-do.{\sc impv.sg}\\
\glt `She seems as though they already gave her a lot of work to do' \hfill{\citep[{\sc comp obj};][192]{CES:LFG14}}
\z

While Maltese copy \isi{raising} constructions can simply involve the `seem; appear; as though' \isi{predicate}(s) discussed so far, it is also possible to have a structure that is closer to {\sc cr} constructions in English, in the presence of the (optional) \isi{preposition} \emph{b\textcrh al} `like' or the preposition-headed \isi{complementiser} \emph{b\textcrh allikieku} `as though', built out of the \isi{preposition} \emph{b\textcrh al} `like', the usual \isi{complementiser} \emph{li} and the \isi{counterfactual complementiser} \emph{kieku}, as illustrated in (\ref{bhalnocr}).\footnote{One could argue that \emph{li kieku} is the full form of the \isi{counterfactual complementiser}. This \isi{complementiser} without the P head is not able to occur in {\sc cr} constructions, as the \isi{ungrammaticality} of (i) below, suggests.

\ea 
\gll *It-tifla qis-ha (li) kieku ma ta-t-x kas\\
{\sc def-}girl as.though-{\sc 3sgf.acc} {\sc comp} {\sc comp} {\sc neg} give.{\sc pfv-3sgf-neg} notice\\
\glt Intended: `The girl's as though she didn't bother'
\z

} %The pair in (\ref{Hebrewcrpair}) from \ili{Hebrew} illustrates how the same \isi{preposition} + \isi{complementiser} introduces both {\sc cr} and non-{\sc cr} constructions. On the other hand, as illustrated through the \isi{ungrammaticality} of (\ref{kanna}) for \ili{Arabic}, the \isi{complementiser} \emph{ka\textglotstop anna} is not possible in non-{\sc cr} contexts.    

\ea \label{bhalnocr}
\gll Qis-{\bf{ha}} (b\textcrh al(likieku)) ta-w-{\bf{ha}} xebg\textcrh a\\
as.though-{\sc 3sgf.acc} as.if give.{\sc pfv.3-pl-3sgf.acc} \isi{smacking}\\
\glt `She's as though they gave her a \isi{smacking}'
\z

%\ea \label{Hebrewcrpair}
%\ea \label{Hebrewcrpaira}
%\gll (ze) nireh ke-ilu {\v{s}}e \textsubdot{h}aim samea\textsubdot{h}\\
%it appears as-if {\sc comp} Haim (is.happy)\\
%\glt It appears as if Haim is happy \hfill{No {\sc cr}}
%\ex
%\gll \textsubdot{h}aim nireh ke-ilu {\v{s}}e hu samea\textsubdot{h}\\
%Haim appears as-if {\sc comp} he happy\\
%\glt Haim appears as if he is happy \hfill{{\sc cr} - \ili{Hebrew}: Lappin (1984, p. 247)}
%\z
%\z

%\ea \label{kanna}
%\ea
%\gll badat-i l-bint-u ka\textglotstop anna-h{\={a}} katab-at-i r-ris{\={a}}lat-a\\
%seem.{\sc pfv-3sgf-indic} {\sc def-}girl-{\sc nom} as-if-{\sc 3sgf.acc} write.{\sc pfv-3sgf-indic} {\sc def-}letter-{\sc acc}\\
%\glt The girl seemed as if she wrote the letter \hfill{{\sc cr} MSA: Salih (1985, p. 138)}
%\ex
%\gll *t-abd{\={u}} \textrevglotstop alay-ha wa ka\textglotstop anna al-awl{\={a}}d-a y-akrah-{\={u}}na John\\
%3-seem.{\sc impv.sgm.indic} on-{\sc 3sgf.gen} {\sc conj} as-if {\sc def-}children-{\sc acc} 3-hate.{\sc impv-pl.indic} John\\
%\glt Intended: She seems as though the children hate John \hfill{No {\sc cr} - MSA: Camilleri et al. (2014, p. 186)}
%\z
%\z

In the presence of this fused \isi{grammatical form}, which provides an evidential-like interpretation, it becomes possible to even drop the \isi{raising predicate} itself, as in (\ref{bhallikieku}), for example.%Of course, if this is dropped, we would be left with the pseudo-\isi{verb} or \emph{deher} construction once again. Additionally, when the \isi{complementiser} is dropped it becomes more likely that the pseudo-\isi{verb} or \emph{deher} starts showing up again as in (\ref{psCR}).  

\ea \label{bhallikieku}
\ea
\gll It-tifla b\textcrh al(likieku) ma ta-t-x kas\\
{\sc def}-girl as.though {\sc neg} give.{\sc pfv-3sgf-neg} notice\\
\glt `The girl's as though she did not bother'
\ex
\gll {\bf{It-tifla}} b\textcrh al ta-w-{\bf{ha}} xebg\textcrh a\\
{\sc def-}girl like give.{\sc pfv.3-pl-3sgf.acc} \isi{smacking}\\
\glt `The girl's as though they gave her a \isi{smacking}'
\z
\z


{\sc cr} in Maltese comes in two flavours. It is not necessarily the case that it should always include an embedded \isi{clause} that maps onto a {\sc comp gf}, which is otherwise what we have when anaphoric-binding is involved. If a P like \emph{b\textcrh al} or its fusion with the \isi{counterfactual complementiser} \emph{(li)kieku} is present, then we can argue that this is functioning as the {\sc pred} of the complement that mediates between the matrix \isi{raising predicate} and the clausal {\sc comp gf} argument which the P then subcategorises for. In such an instance we would then have an analysis where \emph{deher}\textsubscript{b\textcrh al} is associated with the \isi{lexical entry}: $\langle${\sc xcomp}$\rangle${\sc subj}, but where the {\sc subj} is in an anaphorically-bound relation, which in this case would be: ($\uparrow${\sc subj})\textsigma = (($\uparrow${\sc xcomp comp gf}){\textsigma} Antecedent). Independent proof that suggests that \emph{b\textcrh al} can function as a {\sc pred} that in turn subcategorises for an embedded \isi{clause} comes both from examples such as (\ref{bhallikieku}) as well as from data such as (\ref{itbhal}), where \isi{raising} is not even involved. 

\ea \label{itbhal}
\gll J-i-dher b\textcrh al(likieku) marr-u we\textcrh id-hom\\
3{\sc m}-{\sc frm.vwl}-seem.{\sc impv.sg} like go.{\sc pfv.3-pl} alone-{\sc 3pl.gen}\\
\glt `It seems they went on their own' \hfill{(No \isi{raising})}
\z 

Additional evidence in favour of our account that \emph{b\textcrh al} does indeed function as a {\sc pred} comes from the availability of verbless constructions such as the one in (\ref{verblea}). The difference between (\ref{verblea}) and (\ref{bhallikieku}) simply boils down to the fact that \emph{b\textcrh al} displays a distinct \isi{subcategorisation frame} in each: An {\sc obj} argument in (\ref{verblea}) and a \isi{complement clause} in (\ref{bhallikieku}). (See \Citet{DL:COMP} for a discussion of such sorts of alternations in English).

\ea \label{verblea}
\gll It-tifla b\textcrh al-ek\\
{\sc def-}girl like-{\sc 2sg.gen}\\
\glt `The girl is like you'
\z

The data in (\ref{allgfs}) illustrate a number of {\sc cr} constructions with copies in different {\sc gf}s within the structure.\footnote{Note that it is not possible to have a {\sc subj} copy in the highest embedded {\sc subj}. See \Citet{CES:LFG14} for more detail.}

\ea \label{allgfs}
\ea \label{tidhrubhallikieku}
\gll {\bf{T}}-i-dhr-{\bf{u}} b\textcrh allikieku xi \textcrh add qal-i-{\bf{l-kom}} biex t-i-tilq-u\\ 
2-{\sc frm.vwl}-appear.{\sc impv-pl} like.that some no.one say.{\sc pfv.3sgm-epent.vwl-dat-2pl} in.what 2-{\sc frm.vwl}-leave.{\sc impv-pl}\\
\glt `You appear as if someone told you to leave' \\\citep[Embedded {\sc comp obj};{\textsubscript{\texttheta}}][192]{CES:LFG14}
\ex \label{cooccurboth}
\gll Dehr-{\bf{et}} qis-{\bf{ha}} donn-{\bf{ha}} g\textcrh ajt-u mag\textcrh-{\bf{ha}}\\
seem.{\sc pfv-3sgf} as.though-{\sc 3sgf.acc} as.though-{\sc 3sgf.acc} shout.{\sc pfv.3-pl} with-{\sc 3sgf.gen}\\
\glt `She seemed as though they shouted at her' 
\glt \hfill{\citep[Chained \isi{raising} + Embedded {\sc comp obl obj};][193]{CES:LFG14}}
\ex
\gll {\bf{Marija}} qis-ha b\textcrh al t-i-dher li {\.z}ew{\.g}-{\bf{ha}} re{\.g}a' lura d-dar, x'inhi fer\textcrh an-a\\
Mary as.though-{\sc sgf.acc} as 3{\sc f}-{\sc frm.vwl}-seem.{\sc impv.sg} {\sc comp} husband-{\sc 3sgf.gen} return.{\sc pfv.sgm} back {\sc def-}house, what.{\sc cop.3sgf} happy-{\sc sgf}\\
\glt `Mary's as though her husband returned back to the house, how happy she is'  \\(Embedded {\sc comp subj poss})   
\z
\z

(\ref{sehelcr}) illustrates a {\sc cr} construction with the presence of the happenstance \isi{predicate} \emph{se\textcrh el} `happen'.  

\ea \label{sehelcr}
\gll Kollha se\textcrh l-{\bf{u}} qabad-{\bf{hom}} in-ng\textcrh as\\
all.{\sc pl} happen.{\sc pfv.3-pl} catch.{\sc pfv.3sgm-3pl.acc} {\sc def-}sleepiness.{\sc sgm}\\
\glt `All happened to be overcome by sleepiness' \hfill{\citep[24]{Alotaibi2013}}
\z

The \emph{f-}structure in (\ref{f-str}) is the one associated with (\ref{tidhrubhallikieku}), and illustrates an instance of a mediated {\sc cr} structure, along with an \isi{anaphoric dependency} between the matrix {\sc subj} and the {\sc xcomp comp obj}\textsubscript{\texttheta} that is accounted for at the semantic-structure. 

\ea \label{f-str}
\begin{avm}
\[ {\sc pred} & `\emph{tidhru} $\langle${\sc xcomp}$\rangle${\sc subj}'\\
   {\sc subj} & {\[{\sc pred} & `{\sc pro}' \\
                    {\sc pers} & 2\\
                    {\sc num} & {\sc pl}\]} [1] \\   
   {\sc xcomp} & \[{\sc pred} & `\emph{b\textcrh al} $\langle${\sc subj}, {\sc comp}$\rangle$'\\
                      {\sc subj} & {\[\phantom{{\sc pred}}&\phantom{`FOOTBALL'} \] [1]}\\
                      {\sc comp} & \[{\sc comp form} & `\emph{likieku}'\\
                                            {\sc pred} & `\emph{qal} $\langle${\sc subj}, {\sc obj}\textsubscript{\texttheta}, {\sc xcomp}$\rangle$'\\
                                           {\sc subj} & \[{\sc pred} & `\emph{\textcrh add}'\\
                                                                {\sc spec} & \[{\sc pred} & `\emph{xi}'\]\]\\
                                           {\sc obj}\textsubscript{\texttheta} & {\[{\sc pred} & `{\sc pro}' \\
                                                                           {\sc pers} & 2\\
                                                                           {\sc num} & {\sc pl}\\
                                                                           {\sc case} & {\sc dat}\]} [2]\\                                                                                                      
                                           {\sc xcomp} & \[{\sc comp form} & `\emph{biex}'\\
                                                                   {\sc pred} & `\emph{titilqu} $\langle${\sc subj}$\rangle$'\\
                                                                   {\sc subj} & {\[\phantom{{\sc pred}} &\phantom{`foot'}\][2]}\]\]\]\]

\end{avm}
\z

%What is important to mention is that copy \isi{raising} when no mediation via \emph{b\textcrh al} is present, as in (\ref{compobjrp}) and (\ref{cooccurboth}), for example, then binding a non-thematic {\sc subj} inside a thematic {\sc gf} requires the revision of the Extended Coherence Constraint, as in Camilleri et al. (2014), such that the {\sc subj}, just like {\sc udf}s must be bound and properly integrated, even if itself non-thematic. 

%The \emph{f-}structure of (\ref{tidhrubhallikieku}) (also applicable in a similar way to (\ref{husband})), involving mediation with \emph{b\textcrh al} is provided below.

%\ea
%\begin{avm}
%\[ {\sc pred} & `tidhru $\langle${\sc xcomp}$\rangle${\sc subj}'\\
   %{\sc subj} & {\[{\sc pred} & `{\sc pro}' \\
                    %{\sc pers} & 3\\
                    %{\sc num} & {\sc pl}\]$\sigma$} [1] \\   
   %{\sc xcomp} & \[{\sc pred} & `b\textcrh al $\langle${\sc comp}$\rangle${\sc subj}'\\
                      %{\sc subj} & {\[\phantom{{\sc pred}}&\phantom{`FOOTBALL'} \] [1] }\\
                      %{\sc comp} & \[ {\sc pred} & `qal $\langle${\sc subj}, {\sc obj\texttheta}, {\sc xcomp}$\rangle$'\\
                                           %{\sc comp form} & `likieku'\\
                                           %{\sc subj} & {\[{\sc spec} & `xi'\\
                                                                %{\sc pred} & `\textcrh add'\\
                                                                 %{\sc pers} & 3\\
                                                                 %{\sc num} & {\sc sg}\\
                                                                 %{\sc gend} & {\sc m} \]}\\
                                           %{\sc obj\texttheta} & {\[{\sc pred} & `{\sc pro}'\\
                                                                              %{\sc pers} & 2\\
                                                                              %{\sc num} & {\sc pl}\\
                                                                              %{\sc case} & {\sc dat}\]$\sigma$} [2]\\
                                           %{\sc comp} & \[{\sc pred} & `titilqu $\langle${\sc subj}$\rangle$'\\
                                                                 %{\sc comp form} & `biex'\\
                                                                 %{\sc subj} & {\[\phantom{{\sc pred}}&\phantom{`FOOTBALL'} \] [2] } \] \]\]\]

%\end{avm}
%\z

%For the data in (\ref{allgfs}), we formalise the \isi{anaphoric dependency} involved, representing the array of {\sc gf}s that can bind the matrix {\sc subj}, as in (\ref{dependencycr}). (\ref{dependencycr}) also states that the copy can in fact be in {\sc subj} position, as long as this is not the {\sc subj} of the highest {\sc comp gf}, hence the {\sc comp}\textsuperscript{+} annotation. This requirement comes from data in (\ref{subjembeddedcr}), for example, where Camilleri et al. (2014, p. 196) mention how this `would be appropriate in a scenario in which the addressee has been to an interview for a child-minding post, and some aspect of his/her demeanour indicates that the prospective employers (`they') have seen that the addressee can deal well with children'. %So a Psource vs. Contextual event reading really comes out from the basis on which one is making an evaluation.

%\ea \label{dependencycr} ($\uparrow${\sc subj})$\sigma$ = (($\uparrow${\sc comp} $\neg${\sc subj} $|$ $\uparrow${\sc comp}\textsuperscript{+} \{{\sc subj} $|$ {\sc obj} $|$ {\sc obj}\texttheta $|$ {\sc poss} $|$ {\sc obl obj}\})$\sigma$ {\sc antecedent}) \hfill{Camilleri et al. (2014, p. 196)}
%\z

%\ea \label{subjembeddedcr}
%\gll {\bf{T}}-i-dher li {\.{g}}a j-af-u li {\bf{t}}-af {\bf{t}}-mur mat-tfal\\
%2-{\sc frm.vwl}-appear.{\sc sg} {\sc comp} already 3-know.{\sc impv-pl} {\sc comp} 2-know.{\sc impv.sg} 2-go.{\sc impv.sg} with.{\sc def-}children\\
%\glt `You seem (from some positive and upbeat aspect of your demeanour) as though they already know that you know how to deal with children' \hfill{Camilleri et al. (2014, p. 196)}
%\z

%\ea
%\gll Toni$_{i}$ qis-u b\textcrh al Marija qal-et li (hu)$_{i}$ we{\.g}{\.g}a' lil Rita\\
%Tony as.though-{\sc 3sgm.acc} like Mary say.{\sc pfv-3sgf} {\sc comp} he hurt.{\sc pfv.3sgm} {\sc acc} Rita\\
%\glt Tony seems as though Mary said that he hit Rita \hfill{Camilleri, 2016 p. 179}
%\z

Constraints on the path of the \isi{anaphoric dependency} in {\sc cr} constructions are present. As identified in \citet[179]{Camilleri16}, the availability of optionally up to three `seem/as.though/as.if' predicates simultaneously, as in (\ref{donnqisdehermix}), allow us to clearly demonstrate their existence.
%the existence of such. 
The \isi{ungrammaticality} of (\ref{donnqisdehermix}) illustrates that it is not possible to have the matrix {\sc subj} being anaphorically bound with the {\sc comp xcomp (xcomp)} non-{\sc subj} {\sc gf} when a local or optionally chained {\sc subj}-to-{\sc subj} \isi{raising} is nested within. %{\sc comp subj} = {\sc comp xcomp subj} and {\sc comp xcomp subj} = {\sc comp xcomp xcomp subj} (i.e. when chained {\sc subj} \isi{raising} is present).

\protectedex{
\ea \label{donnqisdehermix}
\gll *Dehr-{\bf{et}} donn-hom (qis-hom) qed j-kellm-u-{\bf{ha}} \textcrh a{\.z}in\\
appear.{\sc pfv-3sgf} as.though-{\sc 3pl.acc} as.though-{\sc 3pl.acc} {\sc prog} 3-talk.{\sc impv-pl-3sgf.acc} bad.{\sc sgm}\\
\glt Intended: `She seemed as though they talked badly to her' 
\glt \hfill{\citep[179]{Camilleri16}} 
\z
}

%Additionally, from the \isi{ungrammaticality} of (\ref{donnhajidher}), it also implies that it is not just any long distance {\sc subj} that can be a copy. A gap is obligatory if the {\sc comp} {\sc pred} itself takes a non-thematic {\sc subj}. 

%\ea \label{donnhajidher}
%\gll *Donn-{\bf{ha}} j-i-dher {\bf{hi}} kien-et bdie-t\\
%as.though-{\sc 3sgf.acc} 3-{\sc frm.vwl}-appear.{\sc impv.sgm} she be.{\sc pfv-3sgf} start.{\sc pfv-3sgf}\\
%\glt Intended: `She's as though it seems like she had started'
%\z

{\sc cr} is not only available with \emph{deher} and happenstance verbs. It is also present with \isi{aspectualiser} predicates. The restriction identified in \citet{Camilleri16} with respect to such constructions is that for the {\sc subj} of aspectualisers to display \isi{anaphoric binding}, the {\sc pred} value of the highest embedded \isi{clause} must be either the pseudo-\isi{verb} \emph{qis-} or \emph{donn-}. The path for the \isi{anaphoric dependency} associated with \isi{aspectualiser} predicates as opposed to the `seem/appear/as.though' and `happenstance' predicates obligatorily involves a {\sc comp}$|${\sc xcomp}\textsuperscript{+} path, and where the {\sc pred} of the highest {\sc comp}$|${\sc xcomp} must be \emph{qis-} or \emph{donn-}, and cannot be substituted by \emph{deher}. Alternatively, the {\sc cr} structure can be mediated through \emph{b\textcrh al}. These facts can be compared and contrasted through the data in (\ref{mtaspcr}). %In turn this provides us with another \isi{syntactic} context, apart from when the {\sc subj} is a thematic Psource, where these pseudo-verbs are not simply in free variation with\emph{deher}. This is an interesting contrast with what occurs in \ili{Arabic} vernaculars (e.g. \ili{Egyptian}, as in (\ref{arabicasp})), where the path involved can be {\sc comp}* for these aspectualisers, as opposed to the {\sc comp}\textsuperscript{+} restriction in Maltese (see the \isi{ungrammaticality} of (\ref{nodehercra})), with the first {\sc comp pred} being additionally constrained to either of the pseudo-verbs \emph{qis-/donn-}, as in (\ref{mtaspcr}), and a substitution with the main \isi{raising predicate} \emph{deher} `appear' is not possible, as the \isi{ungrammaticality} of (\ref{nodehercra})-(\ref{nodehercrb}) illustrates. Note that we take this to imply that in such contexts, these aspectualisers subcategorise for a {\sc comp} {\sc gf} (p. 295). 

%\ea \label{arabicasp}
%\gll mona bada\textglotstop-et yi-day\textglotstop-{\={u}}-ha el-wel{\={a}}d\\
%Mona start.{\sc pfv-3sgf} 3-annoy.{\sc impv}-{\sc pl-3sgf.acc} {\sc def-}boys\\
%\glt Mona started to be annoyed by the boys \hfill{\ili{Egyptian}: Alotaibi et al. (2013, p. 22)}
%\z 

\ea \label{mtaspcr}
\ea \label{nodehercrb}
\gll *Bde-w j-i-dher qabad-hom in-ng\textcrh as\\
start.{\sc pfv.3-pl} 3{\sc m}-{\sc frm.vwl}-appear.{\sc impv.sg} catch.{\sc pfv.3sgm-3pl.acc} {\sc def-}sleepiness.{\sc sgm}\\
\glt Intended: `They started seeming as though sleepiness came on-them'
\ex
\gll It-tfal {\bf{bde-w}} qis-u/-hom / donn-u/-hom dejjem qed {\bf{j-a-sl-u}} tard\\
{\sc def-}children start.{\sc pfv.3-pl} as.though-{\sc 3sgm.acc}/-{\sc 3pl.acc} / as.though-{\sc 3sgm.acc}/-{\sc 3pl.acc} always {\sc prog} 3-{\sc frm.vwl}-arrive.{\sc impv-pl} late\\
\glt `The children started as though they are arriving always late' \hfill{({\sc subj})}
\ex \label{nodehercra}
\gll Re{\.g}g\textcrh-{\bf{et}} b\textcrh al qabad-{\bf{ha}} u{\.g}ig{\textcrh} fl-istonku\\
repeat.{\sc pfv-3sgf} as.though catch.{\sc pfv.3sgm-3sgf.acc} pain.{\sc sgm} in.{\sc def}-stomach\\
\glt `She again started feeling pain in her stomach' \hfill{({\sc obj})}
%\ex
%\gll Marija {\bf{bdie-t}} qis-u/donn-u j-be{\.z}{\.z}ag\textcrh-{\bf{ha}} il-film\\
%Mary start.{\sc pfv-3sgf} as.though-{\sc 3sgm.acc} 3-{\sc cause}.frighten.{\sc impv.sgm-3sgf.acc} {\sc def-}film.{\sc sgm}\\
%\glt Mary started as though the film was frightening her \hfill{{\sc obj}}
%\ex
%\gll It-tfal {\bf{re{\.g}g\textcrh-u}} qis-u qed t-g\textcrh ajjat mag\textcrh-{\bf{hom}}\\
%{\sc def-}children repeat.{\sc pfv.3-pl} as.though-{\sc 3sgm.acc} {\sc prog} 3-shout.{\sc impv.sgf} with-{\sc 3pl.acc}\\
%\glt Once again they are as though she is shouting with them \hfill{{\sc obl obj}}
%\ex
%\gll \textcrh ut-i {\bf{qabd-u}} donn-u \textcrh a j-a-\textcrh bat j-i-tlag\textcrh-l-hom xebg\textcrh a deni\\
%sibling.{\sc pl-1sg.gen} catch.{\sc pfv.3-pl} as.though-{\sc 3sgm.acc} {\sc prosp} 3-{\sc frm.vwl}-crash.{\sc impv.sgm} 3-{\sc frm.vwl}-go.up.{\sc impv.sgm-dat-3pl} \isi{smacking} fever.{\sc sgm}\\
%\glt They started as though fever was on the verge of rising up on-them \hfill{{\sc comp xcomp xcomp obj\texttheta} - Camilleri 2016, p. 296}  
\z
\z
 
Another property associated with {\sc cr} constructions, at least in English, is that \isi{idiom} chunks as matrix {\sc subj}s are not possible, as the \isi{ungrammaticality} of (\ref{noidiomsengcr}) illustrates, unlike normal \isi{raising} constructions (Lappin, 1984, p. 241). 

\ea \label{noidiomsengcr}
\ea *Much headway appears as if {\bf{it}} had been made on the project
\ex *Advantage seems as if {\bf{it}} has been taken of John
\z
\z

Parallel facts are also present in Maltese, except that instead of being ungrammatical, the \isi{idiomatic} reading of an \isi{idiom} chunk is entirely lost in {\sc cr} constructions, giving way to a literal reading only, as illustrated in the data in (\ref{noidiomsmtcr}), since the matrix {\sc subj} must itself be a {\sc psource}, in such constructions. %This provides additional support to the claim in Sichel (2012, p. 4) that the presence of {\sc rp}s blocks an \isi{idiomatic} reading. 

\ea \label{noidiomsmtcr}
\ea 
\gll I{\.z}-{\.z}ej{\.z}a donn-ha qatg\textcrh-u-{\bf{hie}}-l-hom\\
{\sc def-}breast.{\sc sgf} as.though-{\sc 3sgf.acc} cut.{\sc pfv.3-pl-3sgf.acc-dat-3pl}\\
\glt `The breast seems as though they cut-it on-them' 
\glt \hfill{(Literal interpretation)}\\
\glt *`The illegal source appears as though they cut-it on-them' 
\glt \hfill{(*Idiomatic interpretation)}
\ex
\gll Il-fellus j-i-dher da\textcrh\textcrh l-u-{\bf{hu}}-l-ha f'mo\textcrh\textcrh-ha\\
{\sc def-}chick.{\sc sgm} 3{\sc m}-{\sc frm.vwl}-appear.{\sc impv.sg} enter.{\sc cause.pfv.3-pl-3sgm.acc-dat-3sgf} in.mind-{\sc 3sgf.gen}\\
\glt `The chick seems like they put it inside her mind' 
\glt \hfill{(Literal interpretation)}\\
\glt *`The doubt seems like they put it inside her head' 
\glt \hfill{(*Idiomatic interpretation)}
%\ex
%\gll Ir-riedni j-i-dher \textcrh ad-it-{\bf{u}}\\
%{\sc def-}rein.{\sc sgm} 3-{\sc frm.vwl}-appear.{\sc impv.sgm} take.{\sc pfv-3sgf-3sgm.acc}\\
%\glt The rein seems she took-it \hfill{Literal interpretation}\\
%\glt She seems to have taken control \hfill{*Idiomatic interprettion}
%\ex
%\gll L-arja qis-ha g\textcrh amil-{\bf{ha}} ma' s\textcrh ab-u\\
%{\sc def}-air.{\sc sgf} as.though-{\sc 3sgf.acc} do.{\sc pfv.3sgm-3sgf.acc} with friend.{\sc pl-3sgm.gen}\\
%\glt The air seems like he did it with his friends \hfill{Literal interpretation}\\
%\glt He seems to have acted arrogantly with this friends \hfill{*Idiomatic interpretation}
\ex 
\gll Qalb-hom qis-ha qatg\textcrh-u-{\bf{ha}}\\
heart.{\sc sgf}-{\sc 3pl.gen} as.though-{\sc 3sgf.acc} cut.{\sc pfv.3-pl-3sgf.acc}\\
\glt `They seem to have cut their heart' \hfill{(Literal interpretation)}\\
\glt *`They seem to have lost hope' \hfill{(*Idiomatic interpretation)}
\z
\z

%Having provided these Maltese data facts, it here becomes relevant to mention that these differ from the \ili{Igbo} and \ili{Yoruba} facts, which languages also display finite \isi{raising} and copy \isi{raising} structures, and which Ademola-Adeoye (2010) concentrates upon, in support of his claim that finite \isi{raising} should be analysed as {\sc cr}, except that {\sc subj}-to-{\sc subj} \isi{raising} out of finite clauses includes a null {\sc pro}. As illustrated through the data in (\ref{igbo}) and (\ref{yoruba}), in these languages, \isi{idiom} chunks in the matrix {\sc subj}, allow for {\bf{both}} a literal and \isi{idiomatic} reading. 

%\ea \label{igbo}
%\langinfo{Igbo} {Bantu} {\Citet[p. 84]{Ademola10}}\\
%\gll Ihe {\'e}-kp{\`u}-ru na {\'n}gw{\`o} di k{\`a} {\bf{\textsubdot{o}}}  gh{\'a}-sh{\'i}a la\\
%thing {\sc pref}-cover {\sc loc} palm seems {\sc comp} it come-off {\sc perf}\\
%\glt The cover on the palm tree seems that it has come off \hfill{Literal interpretation}\\
%\glt The secret seems to have been revealed \hfill{Idiomatic interpretation}
%\z

%\ea \label{yoruba}
%\langinfo{Yoruba} {Bantu} {\Citet[p. 86]{Ademola10}}\\
%\gll Om{\'i} jo p{\'e} {\bf{\'{o}}} p{\`o} ju ok{\`a} lo\\
%water seems that it is more yam flour {\sc superl}\\
%\glt The water seems to be more than the yam flour \hfill{Literal interpretation}\\
%\glt A person seems to be living beyond his means \hfill{Idiomatic interpretation}
%\z

%Contrasting the Maltese facts in (\ref{noidiomsmtcr}) with the \ili{Igbo} (\ref{igbo}) and \ili{Yoruba} (\ref{yoruba}) facts, is one way with which we here argue that it cannot be assumed that hyperraising in Maltese should be simply analysed as copy \isi{raising} involving a null {\sc pro} in the embedded {\sc subj} position, even if we still want to maintain the position that indeed, at the \emph{c-}structure representation, a {\sc pro} fills in the {\sc subj} position of the embedded \isi{clause}, by virtue of the presence of an IP. At the \emph{f-}structure level, the \isi{pronoun} is not present, and a gap is involved. This is in contrast to the \isi{anaphoric binding} relation proposed in Ademola-Adeoye (2010). A similar contrast between the Maltese and the \ili{Igbo} and \ili{Yoruba} facts, is that {\sc subj}-to-{\sc subj} \isi{raising} as well as hyperraising allows for both individual and event Psource readings in Maltese, in contrast to the restriction to an individual Psource in the case of copy \isi{raising} structures. If we are to claim that a \isi{pronoun} fills in the embedded {\sc subj gf} in hyperraising structures, then one is not to account for the distinct readings available. This difference in reading is not present in the \ili{Igbo} and \ili{Yoruba} data. Further evidence we can use in favour of our gap proposal as opposed to a covert `pro', although not shown here, for reasons of space, is the parallel behaviour observed with respect to relative \isi{clause} structures, for example, where in Island contexts, an obligatory overt \isi{pronoun} has to surface. With this collection of facts we conclude that Maltese hyperraising must be analysed differently from copy \isi{raising}. 

With this we conclude our discussion on {\sc cr} in Maltese, and how it is distinct from {\sc ssr}.

\section{Conclusion}

In this paper we concentrated on raising-to-{\sc subj} structures in Maltese highlighting the (morpho)\isi{syntactic} properties and the constraints that characterise \isi{raising} and copy \isi{raising} in the language. Working within the {\sc lfg} framework, we analysed {\sc ssr} differently from copy \isi{raising} at the \emph{f-}structure level. Broadly speaking, the former always involves functional control, while the latter will always have to resort to \isi{anaphoric binding}, at some level, even if the matrix \isi{raising predicate} can associate its \isi{clausal complement} with an {\sc xcomp gf}, and not a {\sc comp gf}, in {\sc cr} contexts, especially as a result of our discussion of what \emph{b\textcrh al(likieku)} imparts to the structure.

In this overview of raising-to-{\sc subj} in Maltese we have considered various \isi{raising predicate} types available in the language, whilst highlighting how their behaviour is not necessarily homogeneous, and the different predicates themselves impose distinct (morpho)\isi{syntactic} constraints. While we have left questions unanswered, such as whether Maltese does indeed have backward {\sc subj} \isi{raising} structures, or whether raising-to-non-{\sc subj} constructions exist, our aim in this paper was to provide a first approximation and advance our knowledge on the broad behaviour of \isi{raising} in Maltese.

%Discussing resumption in the realm of non-{\={A}} involving phenomena is by no means new.  
  
%Evidence in favour of the syntactically inalienable. 

%From the data below we observe how {\sc subj}-\isi{raising} is indeed local, and it is not possible to have a resumptive \isi{pronoun} in {\sc subj} position, when this is indeed not analysed as an instance of long distance copy \isi{raising}. 

%\ea
%\ea
%\gll *Donn-ha bdie-t hi t-a-\textcrh seb\\
%as.though-{\sc 3sgf.acc} start.{\sc pfv-3sgf} she 3-{\sc frm.vwl}-think.{\sc impv.sgf}\\
%\glt Intended: She's as though she started to think
%\ex
%\gll *Donn-ha \textcrh la\textcrh q-et hi kien-et bdie-t\\
%as.though-{\sc 3sgf.acc} manage.{\sc pfv-3sgf} she be.{\sc pfv-3sgf} start.{\sc impv-3sgf}\\
%\glt Intended: She's as though she had managed to have started
%\ex
%\gll *Donn-ha j-i-dher hi kien-et bdie-t\\
%as.though-{\sc 3sgf.acc} 3-{\sc frm.vwl}-appear.{\sc impv.sgm} she be.{\sc pfv-3sgf} start.{\sc pfv-3sgf}\\
%\glt Intended: She's as though it seems like she had started
%\z
%\z

%A long-distance finite copy \isi{raising} allows for an optional resumptive as below, implying how (c) above cannot be analysed as {\sc cr}. And the behaviour of this below is indeed Island behaviour of the RP, such that when not available, one cannot say that there is a null rp, because this can indeed obligatorily surface. This goes to show that it is not the case that finite \isi{raising} should involve a null pro. 

%\section{Hyperraising is not copy raising}

%Additional evidence that finite \isi{raising} involves a gap in {\sc subj} position, is because in the appropriate contexts, resumptive \isi{pronoun} may be obligatory, or may optionally alternate with a gap. Finite (and local) \isi{raising} is however only associated with a gap. 

%{\sc rp}s. So, in the same way that we observe optionally surfacing {\sc rp}s, their absence and their obligatory non-overt rendering cannot force us to assume that they are present. 
%This {\sc gap} analysis in the \emph{f-}structure parallels the {\sc hsr} otherwise employed in Maltese, and the fact that in both {\sc ldd} as well as in Island contexts, we get the surfacing of the resumptive \isi{pronoun} as in other contexts. Associating an analysis with a gap in {\sc subj} position also allows us to account for the retention of an idiom-chunk in finite \isi{raising} structures vs. the behaviour observed in real copy \isi{raising} structures, and the eventuality vs. individual Psource readings. If we already have a \isi{pronoun} fulfilling the {\sc subj gf} then one is not able to differentiate between strict non-surfacing of the \isi{pronoun} in {\sc subj} position, as opposed to the optionality one finds. If there is truly a null pro there, the optional surfacing of an already null pro is redundant, as opposed to claiming that in all contexts gap, except when a rp in Islands and other effects. The issue is that this rp in both Islands and non-Islands can still involve reconstruction (Camilleri \& Sadler, 2011). We will not hang a lot on the reconstruction data here, however, but will stick to parallel gap-rp observations we had in {\={A}}-dependencies, such as in relative clauses, when Islands and {\sc ldd}s are involved, as well as the interpretational differences observed. 

%\ea
%\gll T-i-dher li qal-u li (int) kien kel-l-ek t-mur dakinhar\\
%2-{\sc frm.vwl}-appear.{\sc impv.sg} {\sc comp} say.{\sc pfv.3-pl} {\sc comp} you be.{\sc pfv.3sgm} be.{\sc pfv.3sgm-dat-2sg} 2-go.{\sc impv.sg} {\sc dem.sgm.def.}day\\
%\glt You seem that they said that you had to go that day
%\ex
%\gll Raj-t lit-tifla li intom t-a-\textcrh sb-u li t-a-f-u ir-ra{\.g}el li donn-ha (hi) kien-et t-\textcrh obb-u ferm\\
%see.{\sc pfv-1sg} {\sc acc.def-}girl {\sc comp} you.{\sc pl} 2-{\sc frm.vwl}-think.{\sc impv-pl} {\sc comp} 2-{\sc frm.vwl}-know.{\sc impv-pl} {\sc def-}man {\sc comp} as.though-{\sc 3sgf.acc} she be.{\sc pfv-3sgf} 3-love.{\sc impv.sgf-3sgm.acc} firm\\
%\glt I saw the girl who you thought that you know the man that she seems like she used to love him a lot \hfill{Complex NP Island}
%\ex
%\gll It-tifla b\textcrh al donn-ha hi u Marija kien-u qeg\textcrh d-in j-a-\textcrh sb-u f'xi \textcrh a{\.g}a\\
%{\sc def-}girl like as.though-{\sc 3sgf.acc} she {\sc conj} Mary be.{\sc pfv.3-pl} {\sc prog-pl} 3-{\sc frm.vwl}-think.{\sc impv-pl} in.some thing\\
%\glt The girl's as though she and Mary were thinking of something \hfill{Coordinate Island Constraint}
%\ex
%\gll Liem huma t-tfal li ma t-a-f-x jekk dehr-u-x humiex kien-u qed j-i-b{\.z}g\textcrh-u jew le?\\
%which {\sc cop.3pl} {\sc def-}children {\sc comp} {\sc neg} 2-{\sc frm.vwl}-know.{\sc impv.sg-neg} if appear.{\sc pfv.3-pl-neg} they.{\sc neg} be.{\sc pfv.3-pl} {\sc prog} 3-{\sc frm.vwl}-be.afraid.{\sc impv-pl} or no\\
%\glt Which are the children that you don't know whether they appeared they.not were afraid or no\\
%\glt Which are the children who you don't know whether they seemed as though they were afraid or not? \hfill{Wh-Island} 

%That can alternate with the following:

%\ex
%\gll Liem huma t-tfal li ma t-a-f-x jekk dehr-u-x kien-u-x qed j-i-b{\.z}g\textcrh-u jew le?\\
%which {\sc cop.3pl} {\sc def-}children {\sc comp} {\sc neg} 2-{\sc frm.vwl}-know.{\sc impv.sg-neg} if appear.{\sc pfv.3-pl-neg} be.{\sc pfv.3-pl-neg} {\sc prog} 3-{\sc frm.vwl}-be.afraid.{\sc impv-pl} or no\\
%\glt Which are the children that you don't know whether they appeared they.not were afraid or no\\
%\glt Which are the children who you don't know whether they seemed as though they were afraid or not? \hfill{Wh-Island}   
%\z
%\z 

%Additionally, {\sc hsr} gap restriction is also broken with the presence of epithets which can be inserted, which in turn go to show that when the \isi{pronoun} is actually available, then this must be truly an active one. Other evidence that the \isi{pronoun}, when it surfaces, is an active one, comes from the fact that \isi{idiomatic} readings are not maintained, but the real \isi{idiomatic} ones are.

%\ea
%\gll Donn-ha li ja kelb-a l'hi kien-et {\.g}a marr-et u ma qal-et 'l \textcrh add\\
%as.though-{\sc 3sgf.acc} {\sc comp} {\sc voc} dog-{\sc sgf} {\sc comp} she be.{\sc pfv-3sgf} already go.{\sc pfv-3sgf} {\sc conj} {\sc neg} say.{\sc pfv-3sgf} {\sc acc} no.one\\
%\glt She as though the-bitch-she-is she had already been there/gone and didn't tell anyone
%\z

%Reconstruction effects in Maltese do not seem to help much in establishing the nature of the gap/rp, as in principle syntactically active {\sc rp}s, such as the one one finds when weak island constraints are violated, should block reconstruction, but in fact Maltese doesn't. In parallel, reconstruction is not blocked in non-Island contexts, and a de re reading is still present, i.e. making them syntactically active, notwithstanding.

%\ea
%\gll [L-istudent-a tieg\textcrh-u$_{i}$]$_{j}$ t-i-dher b\textcrh al donn-ha \textcrh add ma j-i-xtieq j-g\textcrh id lill-g\textcrh alliem$_{i}$ li (hi)$_{j}$ kien-et serq-et il-lapes\\
%{\sc def-}student-{\sc sgf} of-{\sc 3sgm.acc} 3-{\sc frm.vwl}-appear.{\sc impv.sgf} like as.though-{\sc 3sgf.acc} no.one {\sc neg} 3-{\sc frm.vwl}-wish.{\sc impv.sgm} 3-say.{\sc impv.sgm} {\sc acc.def-}teacher.{\sc sgm} {\sc comp} she be.{\sc pfv-3sgf} steal.{\sc pfv-3sgf} {\sc def-}pencil\\
%\glt His student seems like as though no wishes to tell the teacher that she had stolen the pencil (and so she will have to do it herself
%\z


 %... Therefore structure-sharing in the f-structure.  

\section*{Acknowledgements}

The research work disclosed in this work is partially funded by the {\sc reach high} Scholars Programme -- Post Doctoral Grants. The grant is part-financed by the EU, Operational Programme II -- Cohesion Policy 2014 - 2020 ``Investigating in human capital to create more opportunities and promote the well being of society" -- ESF.

\sloppy
\printbibliography[heading=subbibliography,notkeyword=this]

\end{document}
