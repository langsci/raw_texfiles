\documentclass[output=paper]{langsci/langscibook} 
\ChapterDOI{10.5281/zenodo.1181787}

\author{Gilbert Puech}
\title{Loss of emphatic and guttural consonants: From medieval to contemporary Maltese}
\shorttitlerunninghead{Loss of emphatic and guttural consonants}

%\ChapterDOI{} %will be filled in at production

\abstract{Medieval Maltese inherited a set of three contrastive ‘emphatic’ obstruents from Arabic: \textit{ṭ}, \textit{ḍ}, \textit{ṣ}, completed by sonorant \textit{ṛ}. It also inherited a set of ‘gutturals’: plosive \textit{q}, fricatives \textit{χ} and \textit{ħ}, sonorants \textit{ɣ} and \textit{ʕ}, and laryngeal \textit{h}. In late medieval Maltese, the contrast between emphatic and plain consonants was lost, while stem vowels took over relevant lexical contrasts. In the eighteenth century, Maltese grammarians took note of ongoing changes in gutturals: weakness of \textit{h}, loss of \textit{χ} merged with \textit{ħ}, and of \textit{ɣ} merged with \textit{ʕ}. In the nineteenth century, the set of distinctive gutturals was reduced to three consonants in most dialects: voiceless stop \textit{q}, or its modern reflex \textit{ʔ}, voiceless fricative \textit{ħ}, and sonorant \textit{ʕ}. The latter triggered complex processes of vowel diphthongization and pharyngealization. In modern Maltese, \textit{ʕ} and vowel pharyngealization were lost. In contemporary Maltese, the allophonic realization [h], without pharyngeal constriction, gains ground over [ħ]. In Element Theory (ET), consonants share melodic elements \{I\}, \{U\} and \{A\} with vowels. Element\{A\}, which characterized the whole set of medieval emphatic and guttural consonants, is only involved in contemporary Maltese for /ʔ/ and /h/, corresponding to orthographic \textit{q} and \textit{ħ} respectively. I also propose a version of ET in which the element \{C\} characterizes surfacing consonants; the position is left empty if the consonant is lost. Empty positions are part of the phonological word structure and contribute to determining syllabic structure and stress assignment.}
\maketitle

\begin{document}    
% \textbf{Keywords:} Maltese \isi{phonology}, Emphatic consonants, Guttural consonants, Intrusive vowels and \isi{diphthongization}, Vowel \isi{pharyngealization}, Element Theory, Empty positions, Phonologically-conditioned allomorphy.

\section{Introduction}

The earliest attestation of written Maltese is a poem which came down to us through a copy unexpectedly found among notarial documents dating back from 1585, but composed in the mid-fifteenth century. The text, in \ili{Latin} script, has been established by the poem's discoverers in a seminal publication: \textit{Peter Caxa\-ro's Cantilena, a Poem in Medieval Maltese} \citep{Wettinger-Fsadni1968}. Philological variants have been proposed by these authors in 1983. \citet{Cohen-Vanhove1991} undertook a linguistic analysis of the Cantilena and suggested alternative philological variants. 

Furthermore, in his book on \textit{The Jews of \isi{Malta} in the Late Middle Ages}, \citet{Wettinger1985} published notarial documents written in \ili{Hebrew} script. According to the author, these texts deserve to be called "Judaeo-Maltese". They attest the use of three \ili{Hebrew} letters for \isi{emphatic} consonants not only in \ili{Arabic} words but also in words of \ili{Romance} origin:

\begin{table}
 \caption{Hebrew letters for emphatic consonants}
 \label{extab:puech:1}
\begin{tabular}{ccc lll}
\lsptoprule
\ili{Hebrew} &   \ili{Arabic} & Transcription & Examples & Modern Maltese & Gloss\\
  \midrule  
  \hebrewscript{ט} & \arabscript{ط} & ṭ &qunṭinṭ  &\textit{kuntent} & satisfied\\
          
&&&nṭr&  \textit{nutar} & notary\\
&&&ju\={g}ṭi & \textit{jagħti} & he gives\\
\tablevspace
\hebrewscript{צ}&
\arabscript{ص} 
& ṣ & nṣf & \textit{nofs} & half\\

&&&ṣḥh & \textit{saħħa} & strength\\
\tablevspace
\hebrewscript{צ֗ }&
  \arabscript{ض}
& ḍ  &ajḍa & = ukoll & also
\\ 
&&&ḫḍrh & \textit{ħadra} & green\\
\lspbottomrule
\end{tabular}
\end{table}

Even before such \textit{prima facie} evidence was published, \citet{Cowan1966}, among others, had postulated the \isi{emphatic} consonants mentioned in \tabref{extab:puech:1}, and \textit{ṛ}, for medieval Maltese by internal reconstruction. After the sixteenth century no Maltese spelling system used special symbols to represent \isi{emphatic} consonants.\footnote{Notice, however, that \citet{Saada1986} transcribes consonants coarticulated with back vowels as \isi{emphatic} in her study of Maltese in Tunisia. This choice of transcription may have been influenced by \ili{Tunisian} \ili{Arabic}; cf. \citet{Ghazeli1977}.}

Maltese also inherited from \ili{Arabic} a set of consonants produced with primary constriction in the posterior region of the vocal tract. For \citet[179]{Hayward1989}:


\begin{quote}
One class of sounds which has been given recognition in traditional descriptions of \ili{Semitic} languages is that of ‘gutturals’ or ‘laryngeals’. This class includes the laryngeals proper (IPA [h], [ʔ]), the pharyngeals (IPA [ħ], [ʕ]) and, though somewhat less frequently, the uvulars (IPA [q], [χ], [ʁ]), though the exact composition of the class will vary from language to language. It is typically associated with low vowels and/or \isi{phonological} processes involving \isi{vowel} lowering. We wish to argue that ‘\isi{guttural}’ needs recognition as a \isi{natural class} in generative \isi{phonology} as well. 
\end{quote}

According to \citet[191]{McCarthy1994} “Standard \ili{Arabic} and most modern \ili{Arabic} dialects have retained the full set of gutturals usually reconstructed for Proto-\ili{Semitic}: laryngeals \textit{ʔ} and \textit{h}; pharyngeals \textit{ħ} and \textit{ʕ}; and uvulars \textit{χ} and \textit{ʁ}”. 


This applies to pre-modern Maltese. However, it should be carefully noted that the modern Maltese \isi{glottal} stop is the reflex of the voiceless \isi{uvular} stop \textit{q}, not the reflex of \ili{Arabic} ‘hamza’.

By the end of the Middle Ages, \isi{emphatic} consonants had been subtracted from the \isi{sound pattern} with compensatory \isi{phonologization} of back stem vowels; cf. \citet[237]{Comrie1991}. In (pre)modern times, \citet{AgiusdeSoldanis} and \citet{Vassalli1796} took note of ongoing changes in gutturals: persistent weakness of \textit{h}, loss of \textit{χ} merged with \textit{ħ}, and of \textit{ɣ} merged with \textit{ʕ}. In the nineteenth century, complex processes of \isi{diphthongization} and \isi{pharyngealization} triggered by the pharyngeal \isi{sonorant} on adjacent vowels are attested. During the twentieth century, \textit{ʕ} was lost in almost all dialects, and \isi{vowel pharyngealization} ceased being discriminant, except residually. As already observed, the \isi{uvular} stop \textit{q} has been progressively replaced by \isi{laryngeal} \textit{ʔ} in mainstream Maltese, a change which also took place in many modern \ili{Arabic} dialects.

After this introduction, I review different approaches to the \isi{phonological} representation of \isi{emphatic} and \isi{guttural} consonants in medieval Maltese. Then I analyze data in pre-modern Maltese, modern Maltese, and contemporary Maltese (sections 3 to 5). \sectref{sec:puech:6} is devoted to what kinds of abstractness should be allowed in \isi{phonology}. Sections 7 and 8 are devoted to the representation of sounds involving \isi{orthographic} \textit{h} or \textit{għ}. \sectref{sec:puech:9} introduces the table of contemporary consonants in Element Theory, to be compared to that given in section 2 for medieval Maltese. I conclude on the metamorphosis of ‘gutturals’ during the last millenium. Diachronic steps are recapitulated in the appendix.


\section{Phonological features for ``back" consonants}
\subsection{SPE features, Feature Geometry, and Elements}

In his synchronic analysis of modern Maltese, \citet{Brame1972} divides consonants into major classes with two SPE binary features: [±\isi{consonant}] and [±\isi{sonorant}]. Consonants and vowels share features [±low] and [±back]. There is an interaction between \isi{guttural} consonants, which are [+low] and [+back], and vowels through a rule of ‘Guttural Assimilation’: “the \isi{vowel} \textit{i} assimilates to \textit{ħ} and \textit{ʔ} in lowness and backness”:  

\ea
Guttural Assimilation: 
\phonrule{i}{a}{{\longrule}\featurebox{+cons\\+low\\+back}}
 \citep[33]{Brame1972}

cf. \citet[171]{Hume1994} for an alternative formulation of this rule.
\z

\citet[185]{Hayward1989} argued against the use of [+low, +back] features in the representation of gutturals:

\begin{quote}
 The class of \isi{guttural} sounds cannot be equated with the class of [+low] segments, however. As has often been pointed out, the specification [+low] is simply not appropriate for the laryngeals [h] and [ʔ] because the definition of the feature refers to the position of the body of the tongue, and this organ is not involved in any primary way in \isi{laryngeal} articulations. Furthermore, even if the laryngeals were allowed to be [+low] ‘by convention’, there are cases, as we have seen, where uvulars need to be included in the class, and these have been classified as [-low]. \citet[cf.][305]{ChomskyHalle1968}.

Invocation of [+back] is even less useful, for this would not only leave out the laryngeals (for exactly the same reasons as those just considered) but would bring in the velars, which, unless modified in some way [\textellipsis], do not, as far as we are aware, pattern with gutturals phonologically. 
\end{quote}

For the authors, who support their analysis by adducing data from several \ili{Semitic} and \ili{Cushitic} languages, “crucial to the definition of ‘\isi{guttural}’ is a satisfactory distinctive characterization of the laryngeals” (p. 186):

\begin{quote}
It seems to us that any attempt at providing a comprehensive solution to the problems raised by the various sorts of behaviours exhibited by [h] and [ʔ] cross-linguistically will in all likelihood be made within the framework of Feature Geometry, in which hierarchical relations between features and classes of features are given explicit recognition (cf., for example, \citealt{Clements1985}; \citealt{Sagey1986}). The events involved in producing [h] and [ʔ] would be assigned to a separate ‘\isi{laryngeal node}’. In languages where the laryngeals behaved as ‘\isi{guttural} consonants’, it would be necessary to give overt recognition to the relationship existing between the \isi{laryngeal node} features and a particular ‘zone of constriction’, namely the \isi{guttural} zone. This relation would, of course, obtain in virtue of the location of the larynx within this zone.
\end{quote}

In independently conducted research, \citet{McCarthy1991,McCarthy1994} recognized the feature [pharyngeal] and bound the representation of emphatics and gutturals in these terms (1994: 219):

\begin{quote}
The phonetic evidence establishes important points of similarity between the gutturals and the emphatics. Broadly, the gutturals and the emphatics share constriction in the pharynx, and narrowly, the \isi{uvular} gutturals share with \textit{q} and the coronal emphatics a constriction in the oropharynx produced by \isi{raising} and retracting the tongue body. We expect to find two principal types of \isi{phonological patterning} corresponding to these phonetic resemblances: a class of primary and secondary [pharyngeal] sounds, including gutturals, \underline{q}, and emphatics; and a class of sounds with [pharyngeal] constriction produced by the [dorsal] articulator, including \isi{uvular} gutturals, \underline{q}, and emphatics.
\end{quote}

After a detailed discussion, McCarthy concludes that in \ili{Arabic} “the laryngeals are classified as [pharyngeal] and so belong to the \isi{guttural} class” (p. 224). Altogether, medieval Maltese data support McCarthy's analysis on the \isi{phonological patterning} of emphatics and gutturals, including \isi{uvular} \textit{q} and \isi{laryngeal} \textit{h}.


In his dissertation \textit{Towards a Comparative Typology of Emphatics} \citet{Bellem2007} adopted Element Theory. In   \citet{HarrisLindsey1995} the theory includes the resonance ‘elements’ listed in \tabref{tab:puech:harrislindseyelements}.

\begin{table}
\small
 \caption{Resonance elements in \citet{HarrisLindsey1995}}
 \label{tab:puech:harrislindseyelements} 
 \begin{tabularx}{\textwidth}{lllc}
  \lsptoprule
\bfseries Element & \bfseries Salient acoustic property  & \bfseries Articulatory target C  & \bfseries Articulatory target V\\
   \midrule
{\db}A & F\textsubscript{1}{\textasciitilde}F\textsubscript{2}: convergence & pharyngeality & {a}{\db}\\
{\db}I & F\textsubscript{1}{\textasciitilde}F\textsubscript{2}: wide divergence &  palatality &  {i}{\db}\\
{\db}U & F\textsubscript{1}{\textasciitilde}F\textsubscript{2}: downwards shift &  (velar-)labiality &{u}{\db}\\
(@&  none (acoustic baseline) &  velarity &  {ə})\\
\lspbottomrule
 \end{tabularx}
\end{table}

\citet[131]{Bellem2007} argues that pharyngeals are \{A\}-headed, while coronals in languages with a salient contrast ‘front–back’ are characterized by the presence of \{I\}. It follows that the element \{A\} is involved as primary melodic feature for gutturals, and secondary for \isi{emphatic} coronals. I retain this analysis, rather than that proposed by \citet{Backley2011}, where the element \{A\} may also characterize plain coronals. The formal implications of headedness in elements are analyzed in \citet{Breit2013}.

\subsection{Medieval Maltese consonants in Element Theory}\label{sec:puech:2.2}

%In \citet{Puech2016} I proposed an architecture which, however, I have revised for this study. 
I propose an architecture in which the elements \{C\} and \{V\} play the role of the elements \{ʔ\}, \{H\}, and \{L\} in previous models; cf. \citet{HarrisLindsey1995}; \citet{Bellem2007}; \citet{Backley2011}; and \citet{Puech2016}.
A segment in a string is represented as a column organized in two sets of elements. The structural elements \{C\} and \{V\} refer to the \textit{manner of articulation}, including \isi{laryngeal} voice; melodic elements refer to the \textit{place of articulation} through profiles of resonance. The melodic elements are \{I\}, \{U\}, and \{A\}. Headedness (underlined element) expresses the dominance of an element's main property. In the absence of front rounded vowels, \{I\} and \{U\} may not combine; thus, they are hosted on the same line. In the presence of mid-vowels, \{I\} or \{U\} may combine with \{A\}: they are hosted on two separate lines.

%Consonants are divided into two major categories: obstruents (stops, fricatives and affricates) and sonorants. The vertex (top cell in a column) of a \isi{consonant} is element \{C\}. Stops are characterized by this element alone; in fricatives, element \{C\} is merged with element \{V\}, occupying the cell below the vertex. In \citet[24]{JakobsonEtAl1952} affricates are considered as “strident stops” and in \citet{Clements1999} as “noncontoured stops”. As observed by \citet[note 176]{Bellem2007} “the status of pulmonic affricates is also not entirely clear”. I propose to represent them as strong fricatives, with headed \{C\} merged with dependent \{V\}.

%Obstruents are characterized by turbulent release, while airflow is non-turbulent in sonorants. In fricatives and affricates, element \{C\} is merged with \isi{headless} \{V), while sonorants are characterized by the headed element \{V\}.

%Obstruents and sonorants either are underspecified on a third line, or have \{V\} or \{C\} as specifier. \{V\} expresses voice in obstruents. If an \isi{obstruent} has no \isi{voiced} counterpart nor \isi{voiced} \isi{allophone}, it is marked with \{C\} on the third line: this applies in Maltese to voiceless gutturals \textit{q}, χ, and \textit{ħ}. For sonorants element \{C\} on the third line features the absence of oral airflow in nasals; lateral /l/ is unspecified, while the rhotic (plain or \isi{emphatic}) is specified for \{V\}.

Consonants are divided into two major categories: obstruents and sonorants. The former includes stops and affricates, spirants and fricatives; the latter includes liquids, nasals and glides. In \citet[24]{JakobsonEtAl1952}, affricates are considered as “strident stops” and in \citet{Clements1999} as “noncontoured stops”. As observed by \citet[note 176]{Bellem2007}, “the status of pulmonic affricates is also not entirely clear”. I propose to represent them as strong stops (headed \{C\}). Similarly, fricatives may be ‘weak’, like approximants, or ‘strong’, like sibilants. They will be represented with headed or \isi{headless} \{C\} merged with \isi{headless} \{V\}. Sonorants are represented with headed \{V\} dominated by \{C\}, which corresponds to segments produced with ‘spontaneous voice’ in \citet{ChomskyHalle1968}.

Obstruents and sonorants either are underspecified on a third line, or have \{V\} or \{C\} as specifier. \{V\} expresses voice in obstruents. If an \isi{obstruent} has no \isi{voiced} counterpart nor a \isi{voiced} \isi{allophone}, it is marked with \{C\} on the third line: this applies in Maltese to the voiceless gutturals \textit{q}, \textit{χ}, and \textit{ħ}. For sonorants, the element \{C\} on the third line features the absence of oral airflow in nasals; lateral /l/ is unspecified, while the rhotic (plain or \isi{emphatic}) is specified for \{V\}.

%\eabox{
% \label{ex:puech:2}
% \begin{tabular}{cccc}
%          Stops  & Fricatives    & Affricates  & Sonorants\\
%  C&   C&   \underline{C}&   \underline{C}\\
%   & V&  V & V  \\
%  (V or C)&   (V or C)  & (V or C)&   (C or V)\\   
% \end{tabular}
% }

\eabox{
 \label{ex:puech:2}
 \begin{tabular}{ccccccccc}
          \multicolumn{3}{c}{\bf Stops}  & \multicolumn{3}{c}{\bf Fricatives} & \multicolumn{3}{c}{Sonorants}\\
          \textit{weak} & / & \textit{strong}			& \textit{weak} & / & \textit{strong}		& \textit{weak} & / & \textit{strong} \\
          C		& & \underline{C} 	& C & & \underline{C} 	& \multicolumn{3}{c}{C} \\
          		& &					& V & & V 				& \multicolumn{3}{c}{\underline{V{}}} \\
          \multicolumn{3}{c}{(C or V)} & \multicolumn{3}{c}{(C or V)} & (V) & & C \\   
 \end{tabular}

  }

Studies in \ili{Arabic} dialectology suggest that the \isi{affricate} /ʤ/ may also be realized as either /ʒ, ɡ/ or /j/, depending on the geographical region of dialects; cf. \citet{Kaye1972}. Maltese retained the post-alveolar \isi{affricate} \isi{pronunciation}. Contrary to other ‘coronal’ consonants, however, Maltese /ʤ/ is not a ‘sun letter’ to which the \isi{definite article} \textit{l} assimilates; cf. \citet[18]{Sutcliffe1936}, \citet[25]{Comrie1980}. This suggests that in Medieval Maltese the \isi{phoneme} was still ‘felt’ as a \isi{voiced} (post)palatal \isi{obstruent}. On the other hand, prefixed \textit{t} in verbal forms assimilates to /ʤ/, as it does to other coronal obstruents; cf. \citet[chapter V]{Sutcliffe1936}; concerning regressive rounding \isi{vowel} harmony, /ʤ/ behaves as other coronal obstruents; cf. \citet{Sutcliffe1936}, \citet[387]{Puech1978}.

In \ili{Arabic} dental (weak) fricatives /θ/ and /ð/ are phonemic; cf. \citet[2-3]{AlKhairy2005}. However, I did not include the phonetic symbol \textit{ð} in Table 1 to interpret the transcription of modern \textit{deheb} ‘gold’ from \ili{Arabic} \textit{ðahab} as ``veheb'' by \citet[20, word 42]{Megiser1606}. Other words point to \textit{θ} (voiceless interdental \isi{fricative}). In any case, dental fricatives correspond to the plosives \textit{t} or \textit{d} in (pre)modern Maltese; cf. \citet[127]{Aquilina1961}, \citet[220]{Cowan1964}, \citet[82, note 75]{Cassola1987}, \citet[241]{Comrie1991}, \citet[17]{Kontzi1994}, \citet[243-44]{Brincat2011}. 

Regarding \isi{laryngeal} \textit{h}, I follow \citet[92]{Laufer1991} whose observations for \ili{Hebrew} and \ili{Arabic} “show that in the production of every [h] there is a narrowing of the glottis. The frication in [h] looks as in any other fricatives, except for the place of articulation”. 
%I interpret it as a breathy-\isi{voiced} \isi{glottal} approximant, /ɦ/, which neither triggers nor prevents \isi{voicing harmony} in an \isi{obstruent} cluster.
I interpret it as a \isi{glottal} approximant realized as [h] or [ɦ], which neither triggers nor prevents \isi{voicing harmony} in an \isi{obstruent} cluster

For resonance, labials are characterized by \{U\} and coronals by \{I\}, as in \citet{Bellem2007}. I follow \citet[75]{Backley2011} in considering that fricatives \textit{s} and \textit{z} are characterized by the \isi{headless} \isi{melodic element} \{I\}, while post-alveolar \textit{š} is characterized by headed \{I\}. The \isi{glide} \textit{y} and the tense \isi{vowel} \textit{i} as well are \{I\}-headed, while lax \textit{ɩ} is characterized by \isi{headless} \{I\}.

Emphatics and \isi{guttural} obstruents form a \isi{natural class} defined by the presence of element \{A\}. From his experimental work on \ili{Hebrew} and \ili{Arabic}, \citet[198]{Laufer1988} concludes “that \isi{emphatic} and pharyngeal sounds share, qualitatively, the same \isi{pharyngeal constriction}. However, the \isi{pharyngeal constriction} is the primary one for pharyngeal and a secondary one for emphatics”. 
%I assume that this analysis applies to medieval Maltese as well. Pharyngeal consonants \textit{ħ} and \textit{ʕ} are characterized by headed \{A\}, peripheral gutturals by nonheaded \{A\}, and \isi{emphatic} coronals by dependent \{A\}:
Uvular obstruents \textit{q} and \textit{χ} are characterized by the \isi{headless} element \{A\}, 
pharyngeal consonants \textit{ħ} and \textit{ʕ} by headed \{A\}; 
in \isi{emphatic} coronals the element \{I\} is combined with \{A\} (Tables \ref{tab:puech:1} and \ref{tab:puech:2}).

% \begin{table}
% \begin{tabularx}{.8\textwidth}{X @{~}c@{~}c@{~}c@{~}c@{~}c@{~}c c@{~}c@{~}c@{~}c@{~}c c@{~}c@{~}c@{~}c@{~}c@{~}c}
% \lsptoprule
% % {\textbf{Se}}
% & \textbf{b} & \textbf{f} & \textbf{t} & \textbf{d} & \textbf{ṭ} & \textbf{ḍ} & \textbf{ṣ} & \textbf{s} & \textbf{z} & \textbf{š} & \textbf{ʤ} & \textbf{k} & \textbf{g} & \textbf{q} & \textbf{χ} & \textbf{ħ} & \textbf{ɦ}\\
% \midrule 
% 	    & C & C & C & C & C & C & C & C & C & C & \underline{C} & C & C & C & C & C & C\\
% {Structure} &  & V &  &  &  &  & V & V & V & V & V &  &  &  & V & V & V\\
% 	    & V &  &  & V &  & V &  &  & V &  & V &  & V & C & C & C & \\
% \midrule
% \multirow{2}{*}{Melody
%         } & U & U & I & I & I & I & I & I & I & I & I &  &  &  &  &  & \\
% 	  &  &  &  &  & A & A & A &  &  &  &  &  &  & A & A & A & A\\
% \lspbottomrule
% \end{tabularx}
% \caption{Obstruents in (post)medieval Maltese %(Se: segment, St: Structure, Me: Melody)
% }
% \label{tab:puech:1}
% \end{table}

\begin{table}[t]
\begin{tabularx}{.9\textwidth}{X @{~}c@{~}c@{~}c@{~}c@{~}c@{~}c c@{~}c@{~}c@{~}c@{~}c c@{~}c@{~}c@{~}c@{~}c@{~}c}
\lsptoprule
{\em Segment}	& {\bf	b	} & {\bf	f	} & {\bf	t	} & {\bf	d	} & {\bf	ṭ	} & {\bf	ḍ	} & {\bf	ṣ	} & {\bf	s	} & {\bf	z	} & {\bf	š	} & {\bf	ʤ	} & {\bf	k	} & {\bf	g	} & {\bf	q	} & {\bf	χ	} & {\bf	ħ	} & {\bf	h} \\
\midrule
 \multirow{3}{*}{\em Structure} & C & \underline{C}	&	C	&	C	&	C	&	C	&	\underline{C}	&	\underline{C}	&	\underline{C}	& \underline{C}	&	\underline{C}	&	C	&	C	&	C	&	C	&	\underline{C}	&	C \\
		&		&	V	&		&		&		&		&	V	&	V	&	V	&	V	&		&		&		&		&	V	&	V	&	V \\
 		&	V	&		&		&	V	&		&	V	&		&		&	V	&		&	V	&		&	V	&	C	&	C	&	C	&	\\
\midrule        
\multirow{2}{*}{\em Melody}		&	U	&	U	&	I	&	I	&	I	&	I	&	I	&	I	&	I	&	\underline{I}	&	\underline{I}	&		&		&		&		&		&	\\
		&		&		&		&		&	A	&	A	&	A	&		&		&		&		&		&		&	A	&	A	&	\underline{A}	&	\\
\lspbottomrule
\end{tabularx}
\caption{Obstruents in (post)medieval Maltese}
\label{tab:puech:1}
\end{table}

% \begin{table}
% \begin{tabularx}{.8\textwidth}{Xccccccccc}
% \lsptoprule
% % {\textbf{Se}} 
% & \textbf{m} & \textbf{n} & \textbf{l} & \textbf{r} & \textbf{ṛ} & \textbf{y} & \textbf{w} & \textbf{ɣ} & \textbf{ʕ}\\
% \midrule
%           & C & C & C & C & C & C & C & C & C\\
% % \hhline{-~~~~~~~~~}
% Structure & V & V & V & V & V & V & V & V & V\\
%           & C & C &  & V & V &  &  &  & \\
% \midrule           
% Melody& U & I & I & I & I & I & U &  & \\
% &  &  &  &  & A &  &  & A & A\\
% \lspbottomrule
% \end{tabularx}
% \caption{Sonorants in (post)medieval Maltese}
% \label{tab:2}
% \end{table}

\begin{table}[t]
\begin{tabularx}{.9\textwidth}{Xccccccccc}
\lsptoprule
% {\textbf{Se}} 
& \textbf{m} & \textbf{n} & \textbf{l} & \textbf{r} & \textbf{ṛ} & \textbf{y} & \textbf{w} & \textbf{ɣ} & \textbf{ʕ}\\
\midrule
\multirow{3}{*}{\textit{Structure}} & C & C & C & C & C & C & C & C & C\\
% \hhline{-~~~~~~~~~}
 & \underline{V} & \underline{V} & \underline{V} & \underline{V} & \underline{V} & \underline{V} & \underline{V} & \underline{V} & \underline{V}\\
 & C & C &  & V & V &  &  &  & \\
\midrule           
\multirow{2}{*}{\textit{Melody}} & U & I & I & I & I & \underline{I} & \underline{U} &  & \\
&  &  &  &  & A &  &  & A & \underline{A}\\
\lspbottomrule
\end{tabularx}
\caption{Sonorants in (post)medieval Maltese}
\label{tab:puech:2}
\end{table}

   
\begin{table}[p] 
 \caption{Arabic roots, medieval and modern forms}
 \label{extab:puech:3}
\begin{tabularx}{.9\textwidth}{Qlll}
\lsptoprule
          \ili{Arabic} root & Medieval Maltese  & Modern Maltese &  Gloss\\
\midrule
  ${\surd}$ f ṣ d & faṣad &\textit{fasad} & to bleed            \\
  ${\surd}$ χ b ṭ & χabaṭ &\textit{ħabat} & to bump             \\
  ${\surd}$ ħ ṣ d & ħaṣad &\textit{ħasad} & to reap             \\
  ${\surd}$ χ ṭ f & χaṭaf &\textit{ħataf} & to snatch           \\
  ${\surd}$ q b ḍ & qabaḍ &\textit{qabad} & to catch            \\
  ${\surd}$ m š ṭ & mašaṭ &\textit{maxat} & to comb             \\
  ${\surd}$ n ṣ b & naṣab &\textit{nasab} & to set a net        \\
  ${\surd}$ q r ṣ & qaraṣ &\textit{qaras} & to pinch            \\
  ${\surd}$ q ṭ r & qaṭar &\textit{qatar} & to fall by drops    \\
  ${\surd}$ r b ṭ & rabaṭ &\textit{rabat} & to tie              \\
  ${\surd}$ ṭ l b & ṭalab &\textit{talab} & to request          \\
  \lspbottomrule
     \end{tabularx}
\end{table}
\clearpage

\subsection{Loss of emphatic consonants and compensatory effects}

In (pre)modern Maltese, forms whose \ili{Arabic} etymon had an \isi{emphatic consonant} are characterized by stem vocalism \textit{a}. Other stems have vocalism \textit{i} by default, or \textit{u} for some of them. In \tabref{extab:puech:3}, \ili{Arabic} roots are given after \citeauthor{Aquilina1987}'s dictionary \citep{Aquilina1987,Aquilina1990}. Medieval forms are reconstructed; modern forms are \isi{orthographic}.


As is well known, an \isi{emphatic consonant} prevented ‘imaala’, i.e. fronting and \isi{raising} of /ā/ to lax and diphthongized /ɪ\textsuperscript{ə}/ \citep{Cowan1966,Borg1976}, \citep[271]{Borg1997}. Even more interesting is the split between two \textit{ū}, represented as \textit{ǔ} and \textit{û} by \citet[XVIII]{Vassalli1796} and \citet[11]{Vassalli1827}. The author describes the former as the “contraction of \textit{o}, and of \textit{u}”, while the latter is the “contraction of \textit{e}, and of u”. In past participles, the stem-infixed \isi{vowel} is \textit{ǔ} for ‘back’ (formerly \isi{emphatic}) stems, while it is \textit{û} for ‘front’ stems. The two vowels are merged in Standard Maltese but remain distinct in \ili{Gozitan} Maltese pausal forms \citep{Borg1977}:

\eabox{
 \label{extab:puech:4}
\begin{tabular}{l@{\,} l@{\,}l@{\,}l @{\,}l@{\,} lll}
  &\ili{Arabic} root & Pf-3-\textsc{m.sg}  &Gloss & PP-\textsc{m.sg}:  &Vassalli & \ili{Gozitan}. & Standard M.\\
 & ${\surd}$ f ṣ d & fasad & `to bleed' && mifsǔd&  mifsoud̥  & mifsūd̥ \\
 & ${\surd}$ f s d & fised & `to spoil' && mifsûd  &mifseud̥  & mifsūd̥ \\
\end{tabular}
}

  
Modern \ili{Gozitan} diphthongized realizations [oĭ] vs. [eĭ] of \textit{ī} in pausal context are also attested by \citet[11]{Vassalli1827} and in \citet[vol. IV, 97]{Bonelli1897}:\footnote{Bonelli's footnote: “In \isi{emphatic} position, especially at the end of a sentence, the items \textit{bylli}, \textit{dīn} or similar, will be pronounced in the country \textit{byllei, dein} etc.; \textit{bylli ma \.gejtš? byllei?} why you did not come, why?”. [The term ‘\isi{emphatic}' refers here to \isi{phrase} focus, not to \isi{consonant} properties].}

\eabox{
  \label{ex:puech:5}%(5)
\begin{tabular}{l llll}
a. & ${\surd}$ ṭ l b & taboip (l. 19)  &\textit{tabīb} & ‘doctor’\\
b. & ${\surd}$qss    & qasseis (l. 23) & \textit{qassīs}&   ‘priest’\\    
\end{tabular}
}

\subsection{Conclusion}

In medieval Maltese, the whole stem domain was ‘back’ in presence of an \isi{etymological} \isi{emphatic consonant}, otherwise it was ‘front’ (except in forms with stem vocalism \textit{u}). We can reconstruct two steps: 

\begin{itemize}
\item  Stem backness is anchored on a radical \isi{emphatic consonant}, and extended to the whole stem: the element \{A\} is shared by the \isi{emphatic consonant} and stem vowels.
\item  Stem backness is anchored on vowels; \isi{emphatic} consonants are merged with their plain counterpart: ‘back’ stem vowels are characterized by a headed element \{A\}, while \{I\} is assigned by default to ‘front’ stem vowels.
\end{itemize}


\section{Gutturals in pre-modern Maltese} 
Canon Agius de Soldanis (1712-1770), born in Rabat (\isi{Gozo}), and Mikiel Anton Vassalli (1761-1829), born in Żebbu\.g (\isi{Malta}), were two erudite Maltese scholars. In the eighteenth century, the prevailing opinion was that Maltese ancestors were \ili{Punic}, \ili{Hebrew}, Syriac, or even \ili{Etruscan}; cf. \citet[chapter 7]{Brincat2011}. Thus, de Soldanis called the book he published in \citeyear{AgiusdeSoldanis}: \textit{Della Lingua Punica, presentamente usata da Maltesi}. In the introduction to his dictionary, \citet{Vassalli1796} suggests that Maltese is a legacy from several \ili{Semitic} languages: \ili{Punic}, Phoenician, \ili{Hebrew}, Chaldean, Samaritan, Syriac, and \ili{Arabic}. Moreover, he connects these substrata to Maltese \isi{dialectal} variations. In subsequent work, however, \citet{Vassalli1827} agreed that Maltese is, in fact, an offshoot from \ili{Arabic}.

\subsection{Description of gutturals by A. de Soldanis 1750}

In \textit{Alfabeto Punico-Maltese}, \citet{AgiusdeSoldanis} lists 22 symbols. The following excerpts (p. 72-74) have been translated into English; modern \isi{orthographic} forms are in italics.
%have been added in square brackets:
%\todo{better use italics, angle brackets, or quotations marks}

\ea
\parbox{1.5cm}{k  [k]}   Grave, acute as \ili{Greek} \textit{k}, and more forced than \textit{q}, e.g. \textit{Kaws} ‘bow’; \textit{Kera} ‘house rent’; \textit{qaws, kera}.
\\\medskip
\parbox{1.5cm}{gk [g]}   shall be pronounced instead of \ili{Hebrew} \textit{Ghimel}, and \ili{Greek} \textit{Gamma} $\gamma $,\footnote{``Gimel'' is the third letter of consonantal alphabets in some \ili{Semitic} languages. Its sound value in Phoenician is the \isi{voiced} plosive [g]. The \ili{Greek} letter ``gamma''  is derived from it.} especially if it comes before a \isi{vowel} as a \isi{consonant}, e.g. \textit{Gkrieżem} ‘throats’; \textit{grieżem}. 
\\\medskip
\parbox{1.5cm}{q  [q]}   Thin, acute, is pronounced in the summit of the throat, e.g. \textit{Qolla} ‘jar’; \textit{qolla}. 
\\\medskip
\parbox{1.5cm}{hh [ħ]}   Is pronounced with strong aspiration, e.g\textit{. Hhait} ‘wall’; \textit{Hharbiʃc} ‘to scratch’. If there is a dot on one of the \textit{h}s, then the aspiration should be more open, while always born from the throat with a light or a strong push from the chest, e.g. \textit{\.{H}hamar}, donkey {\textasciitilde} stupid; \textit{ħmar}. 
\\\medskip
\parbox{1.5cm}{ch [χ]}   Is pronounced grave, hoarse in the summit of the throat, with a bit more force than preceding [hh], e.g. \textit{Chait} ‘thread’; \textit{ħajt}. 
\\\medskip
\parbox{1.5cm}{h  [h]}   Nicely aspirated, e.g. \textit{Hem} ‘there’; with a dot on top, it should be pronounced with more breathing, but gently, e.g. \textit{\.{H}em} ‘trouble’; \textit{hemm, hemm}. 
\\\medskip
\parbox{1.5cm}{gh [ʕ/ɣ]} The most difficult letter, which is grave, and is pronounced in the middle of the throat, among modern Arabs and among \ili{Punic}-Maltese, e.g. \textit{Ghain} ‘eye’. If on top of the \textit{g} a dot has been noted, the \isi{pronunciation} shall be deeper, and if more than one dot, the aspiration is growing, e.g. \textit{\.Ghar} ‘grotto’, \textit{\"{G}}\textit{har} ‘shame’, ¨\textit{\"{G}}\textit{har} ‘envious of’; \textit{għar, għar, għer}.  
\z
 
 

The author distinguishes different realizations of `għ' (Aain) 
%\todo{/gh/??} (Aain) 
by diacritic dots (\tabref{extab:puech:6}).


\begin{table}
\caption{Realisations of `għ'}
    \label{extab:puech:6}
\fittable{    
\begin{tabular}{llllll}
    \lsptoprule
A. de Soldanis  & Gloss            &  \ili{Arabic} root  & Vassalli (IPA) &  Modern spelling\\
\midrule
   \.Ghar       & cave             & ${\surd}$ ɣ w r & [ɣo:r] & għar\\
   \"{G}har     & shame            & ${\surd}$ ʕ j r & [ʕa:r] & għar\\
  ¨\"{G}har     & he got  jealous  &${\surd}$ ɣ j r &  [ɣa:r] & għer\\
  \lspbottomrule
\end{tabular}
}
%\todo[inline]{IPA has no macrons. Use length \isi{sign} instead?}
\end{table}

Concerning \textit{h}, Agius de Soldanis uses a diacritic dot to distinguish \textit{Ḣem} ‘noise’ from \textit{Hem} ‘there’, which may indicate that initial \textit{h} was better preserved in nouns or verbs than in cliticized adverbs. In modern Maltese, intervocalic \textit{h} is dropped: \textit{deheb}, [dēb]) ‘gold’, except in dialects where \textit{h} is realized as \textit{ħ}: [deħeb].


There is no doubt that Agius de Soldanis was aware of \isi{dialectal} differences between different varieties of Maltese pronunciations. The distribution of \isi{velar} \textit{k} and \isi{uvular} \textit{q} in his work differs from modern mainstream Maltese. The author records the words in his \textit{Dizionario} (1750) listed in \tabref{extab:puech:7}.

The alternation \textit{q} {\textasciitilde} \textit{k}, well spread at this time, is still attested in Great Harbour (\isi{Malta}) and Rabat (\isi{Gozo}). In my fieldwork in the 1980s, I recorded the forms in \tabref{extab:puech:8} in Rabat (near the hospital) and Xewkija (close to Rabat); cf. \citet{Puech1994}.





Vassalli's \textit{Lexicon} (\citeyear{Vassalli1796}) is preceded by a \textit{Preliminary Discourse to the Maltese Nation}, which provides us with reliable \isi{dialectal} descriptions of gutturals. The following excerpts have been translated from \ili{Italian} into English:\footnote{Special thanks to Michelangelo Falco, who assisted me in translating the original text. I am the only one to be held responsible for any error of translation or interpretation.} 


\puechlengths{1.5cm}{1cm}{9cm}{1cm}{1cm}{1cm}
\ea 
\puechtriple{LIV}{\textbf{h} [h]}{  To the symbol H, I have assigned an aspirated sound and called it He, such as Havn \textit{here}, hynn \textit{there}, hi \textit{she}, ybleh \textit{silly}.}

\puechtriple{LXXIV}{{~~}}{     Among the new symbols added, a majority was necessary to describe GUTTURAL sounds. }

\puechtriple{LXXV }{ {\ጸ}  [ħ] }{ To the first \isi{guttural sound}, called Hha, I assigned a symbol similar to an A compressed in this way {\ጸ}. This sound is found in many Oriental languages, and it is very aspirated, profound and dry, like the \ili{Arabic} \arabscript{ح}.}
\puechtriple{LXXVI}{\textsf{\ⵀ} [χ] }{The second \isi{guttural sound}, which resembles an O with a perpendicular line down the middle, indicates a hoarse and almost hampered \isi{pronunciation}. The appropriate sound is \ili{Arabic} {\arabscript{خ}}:
like  {{\ⵀ}}ǐa  
\textit{my brother}.}

\puechtriple{LXXVII }{ \textbf{${\cap}$}  [ʕ]  }{ The third \isi{guttural sound}, called Aajn, is represented by this symbol \textbf{${\cap}$}, which I took from Phoenician, as it is found in the inscriptions, and modified it to better fit with the other letters. It describes a very \isi{guttural} and slightly husky sound, common among the Oriental languages. Since it is often unpronounced at the end of a word, I marked this instance as \textbf{${\cap}$}\textbf{\={} } to make it distinct; and, therefore, its presence is maintained in order to preserve the root of the word. }

\puechtriple{LXXVIII}{  \textbf{\ᴟ}   [ɣ]}{ I wanted to describe the fourth \isi{guttural sound}, which denotes a big, huskier and more \isi{guttural sound}, with two Aajn united in this way \textbf{${\cap}{\cap}$}, but in order to avoid confusion with the \ili{Latin} letter m I depicted it as {\ᴟ}. }

\puechtriple{LXXIX  }{\textbf{¢}   [q] }{ There is another sound in our language common among Oriental languages, which is considered by some a \isi{guttural sound}, and by others a palatal, that is formed in the roof of the mouth, like a K. Nevertheless, it differs for its sharpness of \isi{pronunciation}, half palatal and half \isi{guttural}, and produces a certain epiglottal sound, which is very difficult to describe. For this reason, I have included it among the \isi{guttural} sounds. It is not a low-pitched sound, instead it is harsh and very high-pitched. The symbol that represents it, \textbf{¢}, is Phoenician as well, but I gave it a better shape more fitting with the present font. }
\z

\begin{table}
 \caption{Distribution of velar \textit{k} and uvular \textit{q} listed in \citet{AgiusdeSoldanis}}
    \label{extab:puech:7}
\begin{tabular}{lllll}
\lsptoprule
& A. de Soldanis & page & Modern orthography & Gloss\\
          \midrule
  k & kadìm & 148  &\textit{qadim}&  `old'\\
    &kasma  &149  &\textit{qasma} & `break'\\
    &kaui  &149  &\textit{qawwi}  & `strong'  \\
\tablevspace
  q & qbir & 167 & \textit{kbir} & `big'  \\
    &Qemmùna & 167&  \textit{Kemmuna} & `Comino'  \\
    &qelp  &168 & \textit{kelb}  & `dog'\\
    &qlàmàr & 168&  \textit{klamar}&   `calamary'\\
    &qtieb  &170&  \textit{ktieb} & `book'\\
    &qul  &170 & \textit{kiel} & `he ate'\\
\lspbottomrule
\end{tabular}
\end{table}


\begin{table}
\caption{\textit{q} {\textasciitilde} \textit{k} alternation, after \citet{Puech1994}}
    \label{extab:puech:8}
    \begin{tabular}{lllll}
    \lsptoprule
          Orthography & Gloss & Standard & Rabat & Xewkija\\
          \midrule
 \textit{qalb}   & `heart'   & [ʔalb̥] &   [qɑlb̥]  &  [kɑlb̥]\\
  \textit{kelb}  & `dog'  &  [kɛlb̥]  &  [kælb̥]  &  [kælb̥]  \\
    \lspbottomrule
    \end{tabular}
\end{table}


\subsection{Minimal pairs}

In \textit{Grammatica della lingua Maltese}, \citet[14-15]{Vassalli1827} gives lists of minimal pairs. Examples below have been transcribed in IPA. Some words are obsolete in modern Maltese (MM):
\largerpage[2.5]


\puechlengths{1.2cm}{2.6cm}{1.5cm}{1.5cm}{2.8cm}{2cm}
\ea
 \label{ex:puech:9}
\begin{xlist}
 \exbox{
\puechsextuple{\textit{k} }{ Gloss  }{  MM  }{  \textit{q}  }{  Gloss  }{  MM}
\puechsextuple{karkar   }{  to drag along   }{ \textit{karkar}  }{ qarqar  }{  to rumble }{   \textit{qarqar}}
\puechsextuple{kī\textsuperscript{ə}s  }{  drinking glass  }{  \textit{kies} }{  qī\textsuperscript{ə}s  }{  to measure  }{  \textit{qies}}
\puechsextuple{klūbi  }{  ravenous  }{   \textit{klubi} }{  qlūbi  }{  courageous  }{   \textit{qlubi}}
\puechsextuple{krīb  }{  groaning }{   \textit{krib}  }{ qrīb  }{  nearness  }{   \textit{qrib}}
\puechsextuple{ʕakar  }{  viscous  }{  \textit{għakar}  }{  ʕaqar  }{  to ulcerate }{   \textit{għaqar}}
\puechsextuple{joktor  }{  it abounds  }{  \textit{joktor}  }{ joqtor  }{  it leaks  }{   \textit{joqtor}}
\puechsextuple{ħarrī\textsuperscript{ə}k  }{  who prosecutes  }{  \textit{ħarriek}  }{  ħarrī\textsuperscript{ə}q   }{ who ignites  }{   \textit{ħarrieq}}  
}
\exbox{
\puechsextuple{\textit{ħ}  }{~}{~}{    χ    }{~}{~}
\puechsextuple{ħajjar }{ to allure }{ \textit{ħajjar} }{ χajjar }{ to let choose }{ \textit{ħajjar}}
\puechsextuple{ħallæ }{ breaker, pile }{ \textit{ħalla}}{  χallæ }{ to leave }{ \textit{ħalla}}
\puechsextuple{ħajt }{ wall }{ \textit{ħajt}}{  χajt}{  thread }{ \textit{ħajt}}
\puechsextuple{ħall }{ to untie }{ \textit{ħall}}{  χall }{ vinegar}{  \textit{ħall}}
\puechsextuple{ħarat }{ to plough}{  \textit{ħarat} }{ χarat}{  to strip off leaves }{ \textit{ħarat}}
\puechsextuple{ħarqa}{  burn }{ \textit{ħarqa} }{ χarqa }{ a strip of clothes }{ \textit{ħarqa}}
\puechsextuple{ħazen }{ to show respect}{  \textit{ħażen}}{  χazen}{  to store }{ \textit{ħażen}}
\puechsextuple{ħɩlæ }{ to become sweet  }{ \textit{ħila} }{ χɩlæ }{ to waste}{  \textit{ħela}}
\puechsextuple{baħħar }{ to sail}{  \textit{baħħar} }{ baχχar }{ to perfume }{ \textit{baħħar}}
}
\exbox{
\puechsextuple{\textit{ʕ}}{~}{~}{  \textit{ɣ} }{~}{~}
\puechsextuple{ʕabbæ  }{to load      }{\textit{għabba}  }{ɣabbæ }{ to deceive     }{ \textit{għabba}}
\puechsextuple{ʕalaq  }{bloodsucker  }{\textit{għalaq}  }{ɣalaq }{ to close      }{ \textit{għalaq}}
\puechsextuple{ʕâli   }{high         }{\textit{għali}   }{ɣâli  }{ expensive     }{ \textit{għali} }
\puechsextuple{ʕâr    }{shame        }{\textit{għar}    }{ɣâr   }{ cave          }{ \textit{għar} }
\puechsextuple{ʕazel  }{to choose    }{\textit{għażel}  }{ɣazel }{ to spin (wool)}{  \textit{għażel}  }
\puechsextuple{ʕɩraq  }{to sweat     }{\textit{għereq}  }{ɣɩraq }{ to sink       }{  \textit{għereq}  }
}
\end{xlist}
\z


\subsection{Dialectal variation in pre-modern Maltese}

Vassalli knew perfectly well that many speakers do not respect what is the ‘correct’ \isi{pronunciation} of gutturals for him. In his introduction to the \textit{Lexicon} (\citeyear{Vassalli1796}) he comments on speech habits in different areas in the following terms:

\begin{itemize}
 \item[XVII]  If we want to explore the subtleties of this language (Maltese), and, so to say, carry out a fine-grained analysis, exploring its dialects, we would also find that they are like the related oriental languages, each with a special and varied inclination to one of these languages. Our language is usually divided into five dialects by the population, using these dialects we jokingly make ourselves incomprehensible to each other. They are named as follows in Maltese %\todo{font correct?}
= Lsŷn tal blŷd , 
lsŷn tal {\ᴟ}awdeɰ,
lsŷn tar-r{\ጸ}ajjël t' ysfel , 
lsŷn tar-r\ጸ{}ajjël ta fǔq , 
lsŷn tar-r{\ጸ}ajjël tan-nofs  
= \textit{Dialect of the city}, 
\textit{\isi{dialect} of the Gozo}, 
\textit{\isi{dialect} of the low villages}, 
\textit{\isi{dialect} of the high villages}, 
\textit{and \isi{dialect} of the middle villages}. 
Each \isi{dialect} has its own subdialect of a certain place, and they make it possible to identify which area you come from, since they have appreciable differences. Mainly they are distinguished by \isi{pronunciation}, that is by the sounds:  consonants, or vowels, or both.

\item[XVIII]  With the \textit{\isi{dialect} of the towns}, which I call the \textit{\isi{dialect} of the harbour}, since it is spoken in the towns by the main harbour, we intend to refer to the language of the new capital and its suburb, of the town called l'Isola - since it is a peninsula inside the harbour -, of Bermula, of Borgo-Santangelo, and of the castles around. In the \isi{dialect} of these places which can be considered as one big town, subdialects can be distinguished: as a matter of fact, the citizens of Isola differ considerably in their speech from the inhabitants of Bermula, and they differ from the people of Borgo-Santangelo, and they all differ from the people of Valletta …

\item[XIX]  The defect of this language can be recognized mainly through the lack of the sounds {\ⵀ} {\ᴟ} e \textbf{¢} [respectively: χ ɣ q], which are pronounced by the speakers of this \isi{dialect} as {\ጸ} ${\cap}$ e K [respectively: ħ ʕ k], without any real distinction: therefore, they are often confused in the discourse and one word is taken for another. A major part of the speakers naturally lacks these sounds because they did not acquire them in their childhood. Many have these sounds though, but they either abstain from using them, believing to speak in a trendier way, or they use them in the wrong way.

\item[XX]  The \isi{dialect} of \isi{Gozo} Island is little different from those of the countryside of \isi{Malta} as to the \isi{pronunciation} … very ancient \ili{Arabic} expressions are used there, especially by the peasants, whose speech is Arabized a lot.

\item[XXI]  Now we come to the dialects of the countryside of \isi{Malta}. The one which is spoken (fyr-r{\ጸ}ajjël ta fǔq) in the high lands, that is in the West, is the purest \isi{dialect} of \isi{Malta}; the ancient capital, called \textit{li Mdìna} with its suburbs where a few barbarisms are more widespread than elsewhere is excluded. I cannot hear any defects in the \isi{guttural} sounds …

\item[XXII]  Similarly, in the oriental villages of \isi{Malta} called (r-r{\ጸ}ajjël t’ysfel) \textit{low villages}, there is a good \isi{dialect}, undamaged in the \isi{guttural} sounds …

\item[XXIII]  Finally, despite sharing the mistakes of the neighboring areas the best Maltese \isi{pronunciation} can be found in the middle villages. In this area, the \isi{guttural} sounds are preserved in their entirety, as can easily be observed by those who have some knowledge of Oriental languages. The very aspirate sound of the root H at the end of the word is pronounced as it is, like Ybleh\footnote{\textit{ibleh} `foolish'; cf. \textit{belleh}} \textit{silly}, Ykreh\footnote{\textit{ikreh}: `ugly'; cf. \textit{kerah}} \textit{ugly}, Nebbyh\footnote{\textit{nebbieħ} `that makes one aware of s.th.'; cf. \textit{nebbaħ} or \textit{nebbeh}} \textit{who wakes up}, which differs from Nebby{\ጸ}\footnote{\textit{nebbieħ} `barker' (`animal that barks'); cf. \textit{nebaħ}} \textit{who barks,} though throughout the domain badly pronounced {\ጸ} …
\end{itemize}

I give Vassalli's examples of ‘ideal \isi{pronunciation}’ in IPA in \tabref{extab:puech:10}.

\begin{table}
 \caption{Examples of Vassalli's `ideal pronunciation'}
    \label{extab:puech:10}
    \fittable{
\begin{tabular}{llll}
\lsptoprule
          Harbour  &Vassalli's norm  &Modern orthography  &Gloss \\
          \midrule
mʊχrɩɛt &  mɑħrɩ\textsuperscript{ə}t   & \textit{moħriet} & ‘plough’\\
χlɩmt &χolma  ħlɩmt ħolma & \textit{ħlomt ħolma} & ‘I dreamt about’\\
’nχossni ɣɩrkān & ’nħossni ʕɩrqān & \textit{inħossni} \textit{għarqan} & ‘I feel sweaty’\\
qaʕqa  & kaʕka  &\textit{kagħka} & ‘ring-cake’\\
jɩtqaʕweʃ & jɩtkaʕweʃ & \textit{jitkagħwe\.g} & ‘he moves (spasm)’\\
jħoqq & jħokk & \textit{iħokk} & ‘he rubs’\\
buqaʕwār & bukaʕwār & \textit{bukagħwar} & ‘black beetle’\\
\lspbottomrule
\end{tabular}
}
\end{table}


\subsection{Allophonic variation in gutturals}

According to Vassalli's observations and idealized norm, radical \textit{h} is maintained in uncorrupted dialects in all positions. However, if \textit{h} stands for the 3rd masculine object \isi{suffix}, it may be realized as [ħ]; cf. \citet[§24]{Vassalli1827}:

\begin{quote}
The He, H, h merely denotes the aspirated and soft sound; such as \textit{il- kerha, u il-belha harbet mal ybleh} ‘the ugly and the silly [female] fled with the silly [male]’; \textit{Bh\v{ı}ma mhejjma} ‘spoiled animal’. The same sound is kept at the end of words when it is radical, e.g. \textit{\.gieh} ‘honor’, \textit{mwe\.g\.geh} ‘honored’; \textit{blyieh}, or \textit{tbelleh} ‘he grew foolish’\textit{; ikreh} ‘ugly’, or derived: \textit{kerreh}, \textit{tkerreh}. However, if word final h is an affixed \isi{pronoun}, then it will be pronounced \textit{ħ} …
\end{quote}

Vassalli's examples in square brackets have been transposed into IPA:



\puechlengths{2cm}{4.5cm}{2.5cm}{1.5cm}{1cm}{1cm}

\ea%(11a) 
\begin{xlist}
\exbox{
\puechquadruple{[χallūħ]}{ħalla\textsc{-impr} 2\textsc{pl}+Obj 3\textsc{m.sg}}{ `leave him!'}{  \textit{ħalluh}}\\[-.9em]
\puechquadruple{[χallīħ]}{  ħalla\textsc{-impr} 2sg+Obj 3\textsc{m.sg}}{  `leave him!'}{  \textit{ħallih}}
}
\exbox{%(11b) 
\puechquadruple{[fīħ]}{fi prep+Obj 3\textsc{m.sg}}{`in it'}{  \textit{fih}}
}
\end{xlist}
\z

In final position, the 3rd person feminine singular and 3rd person \isi{plural} are respectively /ha/ and /hom/, with variations in \isi{vowel} quality which are irrelevant for the representation of /h/. When the stem ends in a \isi{guttural} \isi{consonant}, /h/ assimilates the place of articulation of the \isi{stem consonant} (cf. §25):

\ea%(11c)
\begin{xlist}
  \exbox{ \puechquadruple{[selaχχa]}{  selaχ-\textsc{pf}-3\textsc{sg} \textsc{pf}+Obj-3\textsc{f.sg}}{  `he skinned it'}{  \textit{selaħha}}}
  \exbox{ \puechquadruple{[fetaħħa]}{  fetaħ-\textsc{pf}-3\textsc{sg} \textsc{pf}+Obj-3\textsc{f.sg}}{  `he opened it'}{  \textit{fetaħha}}}  
\end{xlist}	
\z

To sum up, /h/ has four allophones: [h], [ħ], [χ] and zero. The 3rd person masculine singular object \isi{suffix} has three allomorphs: /h/, /hū/ or /ū/, whose distribution depends on their position in the word.

Sonorant /ɣ/ is realized as a voiceless \isi{uvular} \isi{fricative} [χ] when it is in word final position or followed by a voiceless \isi{consonant} (cf. §28):

\ea%12
    \label{ex:puech:12}
    \begin{xlist}
     \exbox{
      \puechquadruple{[aχsel] }{ ɣasel\textsc{-impr}-2sg}{`wash!' }{ \textit{aħsel}}
      }
      \exbox{
      \puechquadruple{[ferraχχem]}{\small ferraɣ-\textsc{pf}-3\textsc{sg}+Obj-3\textsc{pl} }{`wash them!'}{  \textit{ferragħhom}}
      }
    \end{xlist}
\z

Pharyngeal /ʕ/ (cf. §17) has three allophones: [ʕ], [ħ] if followed by \isi{suffix} -\textit{h}, and zero in \isi{word-final position}:


\ea%13
    \label{ex:puech:13}
    \begin{xlist}
     \exbox{
      \puechquadruple{ [samʕet]}{  sema’-\textsc{pf}-3\textsc{f.sg}}{`she heard'}{  \textit{semgħet}}
      }
      \exbox{
      \puechquadruple{  [samaħħem]}{  sema’-\textsc{pf}-3\textsc{m.sg}+Obj-3pl}{`he heard them'}{  \textit{semagħhom}}
      }
      \exbox{
      \puechquadruple{  [sama] }{ sema’--\textsc{pf}-3\textsc{m.sg}}{`he heard'}{\textit{sema’}}
      }
    \end{xlist}
\z

\newpage 
\subsection{Conclusion}

In eighteenth century Maltese, the \isi{sound pattern} has a maximal set of six \isi{guttural} consonants: 
\textit{q}, 
\textit{χ}, 
\textit{ɣ}, 
\textit{ħ}, 
\textit{ʕ}, and 
\textit{h}. 
However, some dialects have \isi{velar} 
\textit{k} rather than 
\textit{q}; 
\textit{ɣ} or zero for 
\textit{ʕ}, or 
\textit{ʕ} for 
\textit{ɣ}; χ for 
\textit{ħ}, or 
\textit{ħ} for χ; 
\textit{ħ} or χ or zero for \textit{h}. 

Dialectal variation and allophonic changes undergone by \textit{ɣ} and \textit{ʕ} in different contexts, and the assimilation of place of articulation by \textit{h} preceded by a \isi{guttural}, contributed to the loss of identity for these sounds. Such variation induced predictable changes, which, indeed, became established in the nineteenth century.

\section{Gutturals in modern Maltese}
Different sources contributed to the documentation on Modern Maltese in the twentieth century. First, urban and rural dialects have been documented by Bonelli (1897-1900) and \citet{Stumme1904}. Altogether, their descriptions are convergent, even if their perception of \isi{guttural} sounds is somewhat different. \citet{Saada1986} published ethnotexts recorded in the 1960s by residents in Tunisia from Maltese families. Her transcription of \isi{guttural} sounds is almost like Bonelli's. \citet{Vanhove1991} described “the survival of [ʕ] in a Maltese \isi{idiolect} at Mtaħleb in \isi{Malta}”. \citet{Schabert1976} described conservative idiolects in which [ʕ] appears to be an onglide of pharyngealized vowels. Altogether, I call ‘modern’, as opposed to ‘contemporary’, varieties which still include a pharyngeal \isi{sonorant} and/or pharyngealized vowels. Thus, ‘modern’ Maltese includes conservative \ili{Gozitan} dialects which have kept [ɣ] but not [ʕ]; cf. \citet[texts 8 to 10 from Għarb]{Puech1994}. See also \citet{AquilinaIsserlin1981}.

\subsection{Bonelli: Archivio Glottologico Italiano}
\citet{Bonelli1897} published Maltese \isi{idiomatic} expressions, jingles and two traditional narratives recorded during a two-month stay in urban and rural areas of \isi{Malta} and \isi{Gozo}. He completed his study on “the Maltese \isi{dialect}” in 1898 and 1900. His set of \isi{guttural} sounds includes \textit{q}, \textit{ʕ}, \textit{ħ}, \textit{h}. The postvelar stop \textit{q} is general and does not alternate with its mutated form \textit{ʔ}. This reflects his informants' \isi{pronunciation} from Valetta and Rabat (\isi{Gozo}). The pharyngeal \isi{sonorant} \textit{ʕ} is the reflex of both \textit{ʕ} and \textit{ɣ}. The pharyngeal \isi{fricative} \textit{ħ} is the reflex of both \textit{ħ} and χ. From Bonelli's transcriptions, it is not clear whether \textit{h} should be granted full \isi{phonemic status}.


Whether Bonelli's \textit{h} should be granted \isi{phonemic status} or not, it is present in instances where it is usual in the spelling system:

 

\puechlengths{1cm}{3cm}{4.5cm}{3cm}{1.5cm}{1.5cm}
\ea%14
    \label{ex:puech:14}
in final position (3 \textsc{m.sg} \isi{direct object} after a long or diphthongized \isi{vowel}):\\
\begin{xlist}
\exbox{
\puechquadruple{~p. 88}{ dufrejh}{`his nails'}{  \textit{difrejh}}
\puechquadruple{p. 98}{ḥudowh (\isi{Gozo}) }{`they took him'}{ \textit{ħaduh} }
\puechquadruple{~}{saqs\textsuperscript{i}ēh }{`he asked him' }{ \textit{saqsieh}}
}
\exbox[-.55\baselineskip]{in internal stem position (alternating with stem final \textit{ħ}):\\
\puechquadruple{p. 89}{kerha}{   `ugly-\textsc{f.}' }{ \textit{kerha}}
\puechquadruple{cf. }{koroħ}{  `ugly-pl.' }{ \textit{koroh}}
}
\exbox[-.55\baselineskip]{in \isi{intervocalic position} (\isi{direct object} initial h):\\
\puechquadruple{~}{ bdī\textsuperscript{e}t yssaqsī\textsuperscript{e}ha}{   `she began to ask her' }{ \textit{bdiet issaqsieha}}
}
\exbox[-.55\baselineskip]{in personal pronouns:\\
\puechquadruple{p. 89}{u hū ma …  }{`and he did not …'}{  \textit{u hu ma …}}
\puechquadruple{~}{   u hī'a qaltlu}{`and she told him'}{  \textit{u hija qaltlu}}
}
\exbox[-.55\baselineskip]{in adverbs:\\
\puechquadruple{p. 97}{\small beq‘eu sejrīn hekk }{\small `they had continued that way'}{  \textit{baqgħu sejrin hekk}}
}
\end{xlist}
\z

Notice that \citet[78]{Stumme1904} takes note of Bonelli's retention of \textit{h} but never uses it in his own phonetic transcriptions.

%AG: Put reversed comma symbol in parentheses
Bonelli transcribes the pharyngeal \isi{sonorant} by the reversed comma (‘) symbol. It is present in radical positions where it is expected:

\largerpage
% \puechlengths{1cm}{3cm}{3cm}{3cm}{1cm}{1cm}
\ea%(15a) 
\begin{xlist}
\exbox[-.55\baselineskip]{
 In first \isi{radical position}:\\
\puechquadruple{p. 88}{‘adda }{`he passed'}{\textit{għadda}}
\puechquadruple{~}{š-‘andek?}{`what do you have?'}{\textit{x'għandek?}}
\puechquadruple{~}{na‘mlu}{`we do'}{\textit{nagħmlu}}
\puechquadruple{p. 89}{‘aijat}{`he shouted'}{\textit{għajjat}}
}
\exbox[-.55\baselineskip]{
In second \isi{radical position}:\\
\puechquadruple{p. 88}{ qa‘at}{`he stayed'}{  \textit{qagħad}}
\puechquadruple{p. 89}{we‘da}{`a vow'}{  \textit{wegħda}}
}
\exbox[-.55\baselineskip]{
In third \isi{radical position}:\\
\puechquadruple{p. 88}{ma sat‘ouš}{`they could not'}{\textit{ma setgħux}}
\puechquadruple{p. 89}{ sem‘ou}{`they heard'}{ \textit{semgħu}}
}
\end{xlist}
\z

In Bonelli's contributions, no \isi{vowel} is transcribed as pharyngealized.

\subsection{Stumme: Maltesische Studien}

\citet{Stumme1904} faithfully reports the \isi{dialectal} variation between (post)\isi{velar} \textit{q}, maintained in urban areas, and the \isi{glottal} realization \textit{ʔ} in countryside dialects. He claims that the sound \textit{h} is “totally lacking” (p. 78). Moreover, none of his informants made a distinction between pharyngeal \textit{ḥ} (IPA \textit{ħ}] and \isi{velar} \textit{ḫ} (IPA [χ]); nor between \ili{Arabic} {\arabscript{ع}} (IPA [ʕ], transcribed as \textit{{\З}{}}) and \textit{\arabscript{غ}} (IPA [ɣ]). On the other hand, Stumme carefully analyzes \isi{vowel pharyngealization} in relevant contexts (p. 79).

\subsubsection{Dialectal variants of \textit{q}}

Post-\isi{velar} stop \textit{q} contrasts with (post)palatal \textit{k} appears in texts from Valetta:

\ea%(16a)  
\textit{qalb}  `heart'  vs.  \textit{kelb}  `dog'
\z


Glottal stop \textit{ʔ} contrasts with (post)palatal \textit{k} in texts from countryside towns:

\ea%(16b)
\textit{ʔalb}    vs.  \textit{kelb}  
\z


Only one \textit{k} in texts from Victoria (\isi{Gozo}), the contrast being supported by the \isi{vowel} quality: 

\ea%(16c) 
\gllll ~    {kalb (\textit{qalb})}  vs.  {ke̜lb (\textit{kelb})}\\
     cf.  kabdu                  for  \textit{qabdu}  {‘they caught’}\\
    ~     fok                    ~     \textit{fuq}    ‘upon’\\
   ~        fkar                 ~       \textit{fqar}  ‘poor-pl.’\\
\z

In Maltese, the change from \textit{q} to \textit{ʔ} has spread from peripheral towns and villages to Valetta (\textit{il-Belt}) and its suburbs. It has been generalized in the twentieth century. However, in my own fieldwork in the 1980s, I still heard postvelar \textit{q} in the Great Harbour area, and \textit{k} instead of \textit{q} or \textit{ʔ} in Xewkija, a village close to Victoria (\isi{Gozo}).


It should also be noticed that in Standard Maltese some speakers use \textit{k} for \textit{q} (realized as a \isi{glottal} stop) for some words; cf. \citet[27]{Borg2011}.

\newpage 
\subsubsection{Reflexes of \textit{h}}

“The sound \textit{h} is totally lacking in my texts” \citet[78]{Stumme1904}. Its reflexes are:


\puechlengths{1.5cm}{3cm}{4cm}{2cm}{1cm}{1cm}
\ea%(17a)  
\ea no direct correspondence (\isi{virtual consonant} for \isi{stress assignment}):
\puechquadruple{p. 7}{  joqtólom }{`he's killing them' }{ \textit{joqtolhom}}
\puechquadruple{~}{fuq-râsom}{`on their heads'}{  \textit{fuq rashom}}
\puechquadruple{p. 9}{  î}{`she'}{  \textit{hi}}
\ex a \isi{glottal} stop:
\puechquadruple{  p. 5}{  tara’ómš }{`she does not see them'  }{\textit{tarahomx}  }
\ex \textit{għajn} in \isi{radical position}:
\puechquadruple{p. 19}{  ke̜r{\З}{a}}{`ugly-\textsc{f.}' }{ \textit{kerha}  }
\ex a long or diphthongized \isi{vowel}: 
\puechquadruple{  p. 53 }{ dê̜p/dé̜ĕp}{`gold'}{  \textit{deheb}}
\puechquadruple{variants:}{  de̜’ep/de̜{\З}{e̜p}}{}{}
\ex a \isi{glide}:
\puechquadruple{p. 27}{  raptûwom}{`they tied them'}{  \textit{rabtuhom}}
  \puechquadruple{p. 47 }{ idé\u{\i}ja}{`her two hands'}{  \textit{idejha}}
 \puechquadruple{p. 5 }{ ḥallîjom  }{`he left them'}{  \textit{ħalliehom}}
\ex pharyngeal \textit{ḥ}:
\puechquadruple{  p. 9 }{ íkraḥ }{`(the) ugliest'}{ \textit{ikrah}}
  \puechquadruple{p. 7}{  talbûŏḥ}{`they asked him for'}{  \textit{talbuh}   }
\puechquadruple{p. 6}{  taḥḥom}{`their'}{  \textit{tagħhom}}
\z
\z

Moreover, Stumme notes that English \textit{h} is pronounced \textit{ḥ}, e.g. [ḥarri] for ‘Harry’. 

\subsubsection{Effect of pharyngeal sonorant \textit{{\З}{}} on contiguous vowels}

\citet[75]{Stumme1904} describes the sound transcribed by the glyph \textit{{\З}{}} (IPA [ʕ]) as “strongest throat pressure sound (arab \textit{\arabscript{ع}} )”. If \textit{{\З}{}} immediately precedes radical or suffixal \textit{ī} or \textit{ū}, an ‘intrusive’ \isi{vowel} is inserted; cf. \citet{Hall2006}. The intrusive nucleus and the high long \isi{vowel} form a \isi{diphthong}. In other terms, the first element of the \isi{diphthong} does not stand for the vocalization of \isi{sonorant} /ʕ/ but for the \isi{phonologization} of the vocalic transition between the pharyngeal \isi{sonorant} and \textit{ī} or \textit{ū} (examples from Stumme's first text: \textit{Bočča}, \isi{dialect} of Valetta):


\puechlengths{2cm}{3.5cm}{4cm}{3cm}{1cm}{1cm}
\eabox{
    \label{ex:puech:18} 
\puechquadruple{~}{  Stumme   }{ Gloss }{ Modern orth.} 
\puechquadruple{t{\З}{e}\u{\i}t }{ʕīd-\textsc{impf}.3\textsc{f.sg}  }{`she says'}{  \textit{tgħid}}
\puechquadruple{tî{\З}{e}\u{\i} }{ tīʕ-1\textsc{sg}       }{`my'}{  \textit{tiegħi}}
\puechquadruple{{\З}{o}ŭda      }{ʕūd-\isi{noun}.\textsc{f.sg}   }{ `(a piece of) wood'}{  \textit{għuda}}
\puechquadruple{tî{\З}{o}ŭ      }{tīʕ-3\textsc{m.sg}       }{`his'}{\textit{tiegħu}}
\puechquadruple{jisím{\З}{o}ŭ   }{sema’-\textsc{impf}-3\textsc{pl} }{`they hear'}{  \textit{jisimgħu}}
}


\textit{{\З}{}} is obligatorily adjacent to a \isi{vowel}; thus, the stem-initial \isi{vowel} is not syncopated in (\ref{ex:puech:19}c):


\ea
\label{ex:puech:19}
\begin{xlist}
\exbox{
\puechquadruple{čá{\З}{aq}}{noun-collective}{`pebbles'}{\textit{\.cagħak}}
}
\exbox{
\puechquadruple{{\З}{ámel}}{ʕamel-\textsc{pf}-3\textsc{m.sg}}{`he made'}{\textit{għamel}}
\puechquadruple{{\З}{ámlu}}{ʕamel-\textsc{pf}-3\textsc{pl}}{`they made'}{\textit{għamlu}}
\puechquadruple{{\З}{ámlet}}{ʕamel-\textsc{pf}-3\textsc{f.sg}}{`she made'}{\textit{għamlet}}
\puechquadruple{{š{\З}ámel}}{what-ʕamel-\textsc{pf}-3 \textsc{m.sg}}{`What did he make?'}{\textit{x’għamel?}}
}
\exbox{
\puechquadruple{{\З}{amílt}}{ʕamel-\textsc{pf}-1\textsc{sg}}{`I made'}{\textit{għamilt}}
\puechquadruple{cf. contra}{kitib- \textsc{pf}-1\textsc{sg}}{`I wrote'}{\textit{ktibt}}
}
\end{xlist}
\z

Adjacent to \textit{{\З}{}}, a stem or suffixal mid-\isi{vowel} is more open:

\eabox{%20
    \label{ex:puech:20}    
\puechquadruple{bo̜{\З}{o̜t} }{ noun}{  `far' }{ \textit{bogħod}}
\puechquadruple{sém{\З}{e̜t} }{ sema’- \textsc{pf}-3\textsc{f.sg} }{ `she heard'} { \textit{semgħet}}
}

\subsubsection{Pharyngealized vowels}

\citet[79]{Stumme1904} describes some vowels as ‘{\З}{ain-retaining’ (}\textit{{\З}{ain-haltig}}). These vowels, which are noted with a subscribed tilde, keep strong \isi{guttural} pressure (\textit{starke Kehlpressung}) during their whole length. They stand for \textit{{\З}{}} merged with a low or mid \isi{vowel} (represented by IPA ɔ and ɛ below):


\eabox{%(21a) 
\puechquadruple{šâ̰mel}{š + ʕámel-\textsc{pf}-3\textsc{m.sg}}{`what did he make?'}{\textit{x'għamel}} 
\puechquadruple{jâ̰mel}{ʕamel- \textsc{impf}-3\textsc{m.sg}}{`he makes'}{\textit{jagħmel}} 
\puechquadruple{nâ̰mlu}{ʕamel-\textsc{impf} 1\textsc{pl}}{`we make'}{\textit{nagħmlu}} 
}

\eabox{%(21b) 
\puechquadruple{milbɔ̰t}{adverbial locution}{`from far'}{\textit{mill-bogħod}} 
\puechquadruple{šɔ̰l}{work-\isi{noun} \textsc{m.sg}}{`work'}{\textit{xogħol}} 
}


A word-final stem \isi{vowel} may be pharyngealized, but never a suffixal \isi{vowel}:

\ea%(21c) 
sɛba̰  ‘seven’ \textit{sebgħa}; cf. contra sém{\З}ɛt (not *sémɛ̰t) \textit{semgħet} ‘she heard’
\z

\subsubsection{Comparison with Tunisian Arabic}

A few years before his fieldwork in \isi{Malta}, \citet{Stumme1896} had published a grammar of \ili{Tunisian} \ili{Arabic}. Comparing Stumme's transcriptions for \ili{Tunisian} \ili{Arabic} and Maltese is enlightening (\tabref{extab:puech:22}). 


\begin{table}
\caption{Comparison between Tunisian Arabic and Maltese}
\label{extab:puech:22}
\begin{tabular}{lll}
 \lsptoprule
  \ili{Tunisian} \ili{Arabic} (1896: 9) &  Maltese 1904 & Gloss\\
\midrule
   smáʕ  &   séma \textit{sema’} & hear-\textsc{pf}.3\textsc{m.sg}\\
  smáʕt  &  smá\u{\i}t  \textit{smajt} & hear-\textsc{pf}.1/2sg\\
  sémʕat &   sémʕet   \textit{semgħet}  & hear-\textsc{pf}.3\textsc{f.sg}\\
  sémʕu  &  sémʕoŭ   \textit{semgħu}  & hear-\textsc{pf}.3pl\\
    \lspbottomrule
\end{tabular}
\end{table}


\subsubsection{Conclusion}

\citet[60]{Brame1972} claims that in modern Maltese a rule of “\isi{absolute neutralization}” changes the ‘abstract’ \isi{sonorant} \textit{ʕ} into \isi{vowel} \textit{a} (cf. below 6.2). Stumme's transcriptions for Maltese, by contrast with \ili{Tunisian} \ili{Arabic}, prove that \textit{ʕ} followed by long \textit{ī} or \textit{ū} triggered the \isi{diphthongization} of the \isi{vowel}. Thus, the path of change has not been the vocalization of the \isi{guttural} \isi{sonorant} (ʕ $\rightarrow$ a) but its deletion in twentieth century Maltese in all contexts (residual idiolectal attestations):


\puechlengths{3.2cm}{1.2cm}{6.2cm}{1cm}{1cm}{1cm}
\ea%(23a) 
\ea
\smallpuechtriple{‘hear-\textsc{pf}.3pl’}{sémʕū}{(underlying long final  {vowel})}
\smallpuechtriple{\textit{ū}- {diphthongization}}{sémʕoŭ}{~}
\smallpuechtriple{ʕ-deletion}{sémoŭ}{(deletion of \textit{ʕ} and  {diphthong}  {phonologization})\hspace*{-2cm}}
\ex  
\smallpuechtriple{‘make-\textsc{impf}.1\textsc{sg}'}{naʕmel}{~}
\smallpuechtriple{ʕ-deletion}{nā̰mel}{(compensatory length and  {pharyngealization})\hspace*{-2cm}}
\z
\z

\newpage 
\subsection{Pharyngealization in the twentieth century}

The reference book published by \citet{Aquilina1959} (\textit{The Structure of Maltese}) was first written before the World War as a Ph.D. thesis submitted at SOAS (School of Oriental and African Studies). The author postulates three sets of vowels (\tabref{tab:puech:3setsvowels}).

\begin{table}
\caption{Three sets of vowels}
\label{tab:puech:3setsvowels}
\begin{tabular}{llll}
\lsptoprule
Short  &  Long (unpharyngealized) & Pharyngealized\\
\midrule
a  & aː &  \textsuperscript{ʕ}a  {\textasciitilde}  a\textsuperscript{ʕ} {\textasciitilde}  a\textsuperscript{ʕ}a: &  a̰ \\
  e  & eː  & \textsuperscript{ʕ}e  {\textasciitilde}  e\textsuperscript{ʕ} {\textasciitilde}  e\textsuperscript{ʕ}e:  & ḛ \\
  i  & iː  & \textsuperscript{ʕ}i  {\textasciitilde}  i\textsuperscript{ʕ} {\textasciitilde}  i\textsuperscript{ʕ}i:  & ḭ  (\isi{dialectal})\\
  o  & oː  & \textsuperscript{ʕ}o  {\textasciitilde}  o\textsuperscript{ʕ} {\textasciitilde}  o\textsuperscript{ʕ}o: &  o̰ \\
  u  & uː  & \textsuperscript{ʕ}u  {\textasciitilde}  u\textsuperscript{ʕ} {\textasciitilde}  u\textsuperscript{ʕ}u:  & ṵ  (\isi{dialectal})\\  
  \lspbottomrule
\end{tabular}
\end{table}


Aquilina, however, adds this important comment:

\begin{modquote}
The above pharyngealized vowels are classified as special vowels to distinguish them from the unpharyngealized ones. Such differentiation is necessary to maintain the phonetic and historical individuality of the two sets; but it must be borne in mind that \isi{pharyngealization} is so weakened that although it is dialectally perceptible in some of our villages and towns, it is hardly perceptible in others. 
\end{modquote}

Based on his fieldwork in the 1970s, \citet{Schabert1976} analyzed two conservative varieties of Maltese, one from St Julians (in the periphery of Valetta) and the other from the coastal village of Marsaxlokk. Like \citet{Aquilina1959}, \citet[16]{Schabert1976}  postulates three sets of vowels as in \tabref{tab:puech:schabert}: 

\begin{modquote}
Pharyngealization is realized in the following way: the pharyngealized \isi{vowel} is phonetically longer than its non-pharyngealized counterpart (even in \isi{unstressed position}), and during the whole length of the \isi{vowel} or during a portion of its length the pharynx is slightly constricted. 
\end{modquote}

\begin{table}[!ht]
\caption{Vowels according to \citet{Schabert1976}}
\label{tab:puech:schabert}
\begin{tabular}{rl cc ll}
\lsptoprule
 \multicolumn{2}{c}{short vowels}&
 \multicolumn{2}{c}{pharyngealized vowels}&
 \multicolumn{2}{c}{long vowels}\\
 \midrule
  i & u &   &   & ī & ū\\
    &   &   &   & ɪ\textsuperscript{ə} & \\
    & o &   & ọ &   & \=o\\
  æ & a & æ̣ & ạ & \={æ} & ā\\
  \lspbottomrule
\end{tabular}
\end{table}

According to \citet[18]{Schabert1976}, it sounds as if a faint \textit{ʕ} slips into part of the \isi{vowel}. In words which start by a pharyngealized \isi{vowel} there is no prosthetic \isi{glottal} stop, but when pharynx constriction occurs in the first part an initial sound like [ʕaː] may be heard. He gives the following examples:

\puechlengths{2cm}{3cm}{3cm}{3cm}{1cm}{1cm}
%\todo{tuples only done up to here}
\ea% (24) 
\ea \mbox{Stressed position (glissando towards a centralized non-syllabic vocoid)\hspace*{-2mm}}
\puechquadruple{ /fæ̣ m/ }{ [fɛ̰ ˑḛ̆ m] \textasciitilde [fɛ̰ ːm] }{ \textit{fehem} }{ `he understood' }
\puechquadruple{ /ʃọ l/ }{ [ʃɔ̰ ˑŏ̰ l] \textasciitilde [ʃɔ̰ ːl] }{ \textit{xogħol} }{ `work' }
\puechquadruple{ /ʔạ d / }{ [ʔa̰ ˑă̰ t] \textasciitilde [ʔá̰ ːt] }{ \textit{qagħad} }{ `he stayed' }

\ex Unstressed position
\puechquadruple{ /æ̣ l\'{ī}ʔi/ }{ [ɛ̰ ˑlɪˑĕʔi] }{ \textit{għelieqi} }{ `fields' }
\puechquadruple{ /yoʔọ d/ }{ [jóʔɔ̰ ˑt] }{ \textit{joqgħod} }{ `he stays' }
\puechquadruple{ /nạ mlūh/ }{ [na̰ ˑmlúˑə̯ ħ] }{ \textit{nagħmluh} }{ `we do it-\textsc{m.sg}' }
\z
\z

Schabert's description is of great interest although not, in my opinion, representative of the present \isi{dialectal} situation in the urban area in which St Julians is integrated; neither is the \isi{idiolect} he recorded from Marsaxlokk representative for this village;
%neither is representative from Marsaxlokk the \isi{idiolect} he recorded in this village; 
cf. \citet[text 43 and 44-50]{Puech1994} and \citet[235-253]{AzzopardiAlexander2011}. Finally, we shall mention the very careful description of an \isi{idiolect} spoken in Mtaħleb (\isi{Malta}) by \citet{Vanhove1991}. This \isi{idiolect} attests the survival of \textit{ʕ} and illustrates the complex relationship between \isi{vowel length} and \isi{pharyngealization} in the context of phonemic \textit{ʕ}. See also \citet{Vanhove1994} for a phonetic and \isi{phonological} analysis of the \isi{dialect} spoken in M\.garr (\isi{Malta}).

\section{Gutturals in contemporary Maltese}

In phonemic terms, contemporary Maltese includes two \isi{laryngeal} obstruents: /ʔ/ and /h/; the latter may be expressed by a pharyngeal or postvelar voiceless \isi{allophone}: [ħ, χ]; cf. \citet[259]{Borg1997}. Vowels are no longer pharyngealized. We will distinguish two stages in contemporary Maltese. In more conservative idiolects, formerly pharyngealized vowels keep some degree of length even in \isi{unstressed position}. In more innovative idiolects \isi{vowel length} is maintained in stressed position only:

\ea%(25a) 
\puechquadruple{ \'āmel }{ \'āmel }{ \textit{għamel} } { `he did'}
\puechquadruple{ āmɪlt }{ amɪlt }{ \textit{għamilt} }{ `I/you-\textsc{sg.} did' }
\z

\ea%(25b) 
\puechquadruple{ \'ēmes }{ \'ēmes }{ \textit{hemeż} }{ `he fastened' }
\puechquadruple{ ēmɪst }{ emɪst }{ \textit{hemiżt} }{ `I/you-\textsc{sg} fastened' }
\z

According to \citet[36-38]{Hume2009}, length is similar for an underlying long \isi{vowel} and a short \isi{vowel} adjacent to \textit{għ} in stressed position:

\puechlengths{1.5cm}{2.5cm}{4.5cm}{3cm}{1cm}{1cm}
\ea%26
    \label{ex:puech:26}
\puechquadruple{t\'āma}{\textit{tagħma}}{għama-imperf.3\textsc{f.sg}}{`she grows blind'}
\puechquadruple{t\'āma}{\textit{tama}}{{noun} \textsc{f.sg}}{`hope'}
\puechquadruple{taʃʃa}{\textit{taxxa}}{{noun} \textsc{f.sg}}{`tax'}
\puechquadruple{t\'āʃʃaʔ}{\textit{tgħaxxaq}}{għaxxaq-imperf.3\textsc{f.sg}}{`she makes happy'}
\z

According to my own observations and phonetic observations in Hume et al., a short \isi{vowel} adjacent to \textit{għ} is not lengthened in \isi{unstressed position}:

\ea%(27a)  
\puechquadruple{ nɔrbɔt }{ \textit{norbot} }{ rabat-imperf.1\textsc{sg} }{ `I tie' }
\puechquadruple{ nɔbɔt }{ \textit{nobgħod} }{ bagħad-\textsc{impf}.1\textsc{sg} }{ `I hate' }
\z

\ea%(27b)  
\puechquadruple{ tálap }{ \textit{talab} }{ talab-perf.3.\textsc{m.sg} }{ `he asked' }
\puechquadruple{ nɪlap }{ \textit{nilgħab} }{ nilgħab-\textsc{impf}.1\textsc{sg} }{ `I play' }
\z

There are, however, other idiolects in which a short \isi{vowel} adjacent to \textit{għ} is lengthened in this position. Thus, \citet[60]{Camilleri2014} gives /nilaːb/ `I play'. According to \citet{Vanhove1991}, the long \isi{vowel} attracts word stress, which yields [nil\'āp]. Notice also the following variation:


\eabox{%28
\label{ex:puech:28}
\begin{tabular}{llllll}
\citet{Hume2009}& \citet{Camilleri2014}\\{}
[l\'āpt]  & [lápt] & \textit{lgħabt} & ‘I played’ \\{}
[l\'āptu] & [l\'āptu] &  \textit{lgħabtu} &  ‘you played’ 
\end{tabular}                              
%\puechsextuple{ \citet{Hume2009}: } { [l\'āpt] }{ \citet{Camilleri2014}: }{ [lápt]  }{ \textit{lgħabt} }{ ‘I played’ }
%\puechsextuple{						      ~}{ [l\'āptu] }{                               ~}{ [l\'āptu] }{ \textit{lgħabtu}  }{  ‘you played’ }
}

There are two distinct stem pattern classes for ‘h-medial’ verbs (\citealt[66]{Camilleri2014}; \citealt{MLRS}):

  
\puechlengths{1cm}{2.5cm}{1.3cm}{1.3cm}{2.5cm}{1.5cm}
\ea%(29a) 
\puechsextuple{~}{class 1}{~}{class 2}{~}{~} 
\vspace*{-3mm}
\begin{xlist}
\exbox{
\puechsextuple{[f\'ēm]}{  ‘he understood’}{  \textit{fehem} }{   [d\'ēr]  }{‘he appeared’  }{ \textit{deher}}
}
\vspace*{-3mm}
\exbox{
\puechsextuple{[fɪmt] }{ 1/2 \textsc{sg} }{ \textit{fhimt} }{   [dért] }{ 1/2 \textsc{sg} }{ \textit{dhert}}
}
\end{xlist}
\z

\newpage 

Class 2 perfect conjugation is like that of ‘għ-medial’ verbs, e.g. \textit{xehed} ‘to give evidence’ and \textit{xegħel} ‘to switch on’. Plural imperfect forms, however, are kept distinct; cf. \citet[123]{Camilleri2014}.


\citet[42]{Hume2009} evoke the question of a third degree of \textit{phonemic} length.
%(authors' emphasis). 
In any case, the actual length of vowels in different contexts depends on several factors, with variation in individual or \isi{dialectal} speech habits. Main factors are:

\begin{enumerate}
\item underlying representation;
\item position in open or closed \isi{syllable};
\item word stress position;
\item intonation pattern.
\end{enumerate}


\section{On the abstractness of phonology: Maltese \textit{ʕ} and \textit{h}}\label{sec:puech:6}

\citet{Brame1972} called his main contribution on Maltese: \textit{On the Abstractness of Phonology: Maltese ʕ.} As shown by this title, the focus was on the pharyngeal \isi{sonorant}. Prior to analyzing Brame's arguments, I will refer to David Cohen's comparison between Jewish Tunis \ili{Arabic} and Maltese with respect to \textit{ʕ} and \textit{h}. Finally, I will briefly expose the theoretical background on which my proposal for the representation of \textit{ʕ} and \textit{h} is based.

\subsection{Cohen's virtual phoneme}
The loss of the pharyngeal \isi{sonorant} \textit{ʕ} and the approximant \textit{h} broke up the unity of morphological paradigms in forms which had either \isi{phoneme} as a root \isi{consonant}. \citet{Cohen1966,Cohen1970} compared the Maltese case with that of Tunis, where the loss of phonemic \textit{h} in the Jewish community was compensated by different strategies which maintained the morphophonemic unity of the \ili{Arabic} \isi{dialect} spoken in the city by different communities. For Maltese, \citet[131]{Cohen1970} postulated a virtual segment occupying the position of \textit{ʕ} but never realized as such:

\begin{quote}
The non-articulation of a sound corresponding to graphic signs \textit{għ} and \textit{h} is characteristic of a part of the population. This part is in constant contact with elements of other groups for whom these signs present diverse realizations from mere intervocalic hiatus to the articulation of pharyngeal and \isi{laryngeal} consonants in some positions. The existence of a \isi{phoneme} in positions marked in spelling by \textit{għ} and \textit{h} is felt by all people, at least in a great number of forms. Apparently, the situation is comparable with Jewish Tunis where the non-articulation of \isi{phoneme} \textit{h} which still exists in the surrounding Muslim \isi{dialect} maintains the awareness of a sort of \isi{virtual phoneme}, pure phonic quantity with no defined form, realized in different ways depending on contexts. [my translation].
\end{quote}

In her tribute to Cohen's eminent contribution, \citet[6]{Vanhove2016} concludes that

\begin{quote}
the complementary distribution of the various allophones Cohen proposes is not exhaustive because of the limited documentation he had access to, but is accurate for the Maltese language of the first half of the twentieth century.
\end{quote}


\subsection{Brame's abstract ʕ}
Even if the author concedes that “abstract segments and \isi{absolute neutralization} must be countenanced in linguistic theory”, \citet[60]{Brame1972} concludes that the preservation of morphological regularities induces the inclusion of \textit{ʕ} in the underlying \isi{sound system} for new generations of language learners:

\begin{quote}
Instead, great pains were taken to demonstrate that the evidence for underlying ʕ is in the phonetic data. That is, the child coming to the \isi{language-learning situation} is capable of inducing ʕ on the basis of Maltese \isi{phonetics} alone. It would be absurd to ascribe historical knowledge to the language learner. The fact that abstract ʕ must be postulated in the \isi{phonological} component of Maltese is of consequence for the ultimate formulation of a natural principle of the type some have recently been interested in developing. First, any such condition will have to allow for \isi{phonological} segments that never show up on the phonetic surface. Second, the condition will have to allow for rules of \isi{absolute neutralization}, since apparently one of the rules needed to account for the \isi{phonetics} of Maltese is of this type. This rule is stated as:
\\
  ʕ → a
\\
Among others, this rule will account for the phonetic reflex of ʕ observed in the derivations listed [above].
\end{quote}

Some data quoted by Brame in support of his analysis are well attested in diachrony but not usual any more. Let us take the example of \textit{imgħadt} [imʕatt] ‘I chewed’. Prosthetic \textit{i} was motivated by the initial stem cluster [mʕ]; cf. \citet[46]{Brame1972}, \citet[14]{Comrie1986}. In contemporary Maltese, the form is simply pronounced [m\'ātt]. Conservative realizations with prosthetic \textit{i} are, however, preserved in some expressions, e.g. \textit{bl-imgħarfa} ‘with a spoon’. \citet[262]{Borg1997} is certainly correct in his overall conclusion:

\begin{quote}
the generative interpretation of abstract \textit{għajn} may ultimately prove more faithful to historical fact than to the synchrony of M. At all events, the notion that M. speakers perceive an underlying ‘pharyngeal segment’ \textit{għajn} finds little support in written usage, since the correct assignment of this digraph in written M remains a notorious source of error even among highly literate speakers.
\end{quote}

Brame did not investigate \textit{h} reflexes. Logically, the same arguments put forward for \textit{ʕ} lead to including \textit{h} into the underlying \isi{sound system} as well. 

\subsection{Emptiness in CV Phonology}
\subsubsection{CV-only Phonology}
Within the framework of Government Phonology, \citet{Lowenstamm1996} made the radical claim that “\isi{syllable structure} universally, i.e. regardless of whether the language is \isi{templatic} or not, reduces to CV.” In this model, a Maltese form like \textit{kitbu} ‘they wrote’ is analyzed as a sequence of three light open syllables, the second of which has an \isi{empty nucleus}: ‘k i t · b u’ (the \isi{empty nucleus} position is here represented by a ‘median point’). Words obligatorily start with a C position, and end in a full or \isi{empty nucleus}, to comply with the CV-only principle. 

In other versions, words can start with a V position, or end in a C position; cf. \citet[111]{Polgardy2012}. Independently of Government Phonology, strict C/V alternation may be viewed as an effect of applying the Obligatory Contour Principle (OCP) to \isi{syllabic structure}. This principle stipulates that, at a given level of structure, adjacent identical elements are prohibited \citep{McCarthy1986,Odden1986}. In this approach, a sequence of strictly adjacent consonants, like ‘C C’, or strictly adjacent nuclei, like ‘V V’ are prohibited. On the other hand, sequences like ‘C · C’, where the \isi{medial position} stands for an \isi{empty nucleus}, and ‘V · V’, where the \isi{medial position} stands for an empty \isi{consonant}, are allowed. Under the OCP application, two adjacent empty positions are also prohibited: a sequence like *‘C · · V’ is ill-formed.

\subsubsection{Segmental length}
In Lowenstamm's model, a \isi{geminate} \isi{consonant} is represented by two C's straddling an \isi{empty nucleus} (represented by the zero symbol); conversely, a long \isi{vowel} is represented by two V's straddling an empty C-position:

\ea%(30a)
\begin{forest}
[{\textit{demmi} ‘my blood’}, baseline, for children={no edge}, s sep=0.8cm
	[C, for children={no edge} [d ]] 
	  [V, for children={no edge} [e ]] 
	  [C, for children={no edge}, name=C1 [ ~]] 
	  [ø,for children={no edge} [m̅, name=M ]] 
	  [C, for children={no edge}, name=C2 [ ~ ]] 
	  [V, for children={no edge} [i ]]
]
\draw (M.north) -- (C1.south);
\draw (M.north) -- (C2.south);
\end{forest}
%\gll C  V  C  ø  C  V   \\
 %     d  e ~  m̅   ~  i     \\
 \z

\ea%(30b)
%\textit{dāri} ‘my house’
\begin{forest}
[{\textit{dāri} ‘my house’}, baseline, for children={no edge}, s sep=0.8cm
	[C, for children={no edge} [d ]] 
	  [V, for children={no edge}, name=V1 [~ ]] 
	  [ø, for children={no edge} [ ā, name=A]] 
	  [V,for children={no edge}, name=V2 [~]] 
	  [C, for children={no edge} [ r ]] 
	  [V, for children={no edge} [i ]]
]
\draw (A.north) -- (V1.south);
\draw (A.north) -- (V2.south);
\end{forest}
\z
%\gll C  V  ø  V  C  V\\
  %   d  ~  ā  ~  r  i\\
%\z
%\todo{draw tikz}

%Alternatively, a long segment may be interpreted as element \{C\} or element \{V\} occupying a position and controlling a contiguous empty unit of \isi{phonological} space, cf. \citet{Russo2016}. In this view, \{C\} is merged with ø (empty space) for a \isi{geminate}, and \{V\} with ø for a long \isi{vowel}:

Alternatively, a long segment may be interpreted as an element \{C\} or \{V\} occupying a position and controlling a contiguous empty unit of \isi{phonological} space; cf. \citet{Russo2016}. In this view, an \isi{empty position} precedes the position occupied by the element \{C\} for geminates; it follows the position occupied by the element \{V\} for a long \isi{vowel}:

\eabox{
% (31a) 
\begin{tabular}{lllll}
 C & V & ø & V & V \\
 \hline
 \multicolumn{1}{|c|}{d} & \multicolumn{1}{|c|}{e} & \multicolumn{2}{|c|}{m̅}   & \multicolumn{1}{|c|}{i}\\
 \hline
\end{tabular}
}

\eabox{
% (31b)
\begin{tabular}{lllll}
C & V & ø & C & V\\
 \hline
\multicolumn{1}{|c|}{d} & \multicolumn{2}{|c|}{ā} & \multicolumn{1}{|c|}{r}   & 
 \multicolumn{1}{|c|}{i}\\
 \hline
\end{tabular}
}


The melody of monophthongal long segments occupies the double space of the vertex.

\subsubsection{Maltese diphthongs in Element Theory}
Maltese has two types of diphthongs: The ‘bogus’ diphthongs in (31) are in fact a \isi{vowel} contiguous to the \isi{glide} \textit{y} or \textit{w}; the type of \isi{diphthong} in (34) results from melodic \textit{fission} due to an ‘intrusive’ nucleus between the long \isi{vowel} and a pharyngeal \isi{sonorant}; on ‘intrusive’ vowels, cf. \citet{Hall2006}. The \isi{diphthong} is an \isi{allophonic realization} of the long \isi{vowel} in the presence of /ʕ/ and becomes phonologized when the \isi{consonant} is lost. When /ū/ or /ī/ are followed by \textit{q} or \textit{ħ} the long \isi{vowel} is also perceived as altered and slightly diphthongized; cf. \citet[38, 54]{Aquilina1959} and \citet[304, 305]{BorgAzzopardi1997}. In this context, however, there is no loss of the \isi{consonant} and, thus, no \isi{phonologization} of the \isi{diphthong}.

\paragraph{a- Sequence  nucleus+glide}
\ea
The vowels \textit{e}, \textit{o}, and \textit{a} merge with glides \textit{y} or \textit{w} to yield a bogus \isi{diphthong}:

\ea%(32a) 
\parbox{2cm}{s  a  y  f}\parbox{1.5cm}{\textit{sajf}}\parbox{2cm}{‘summer’}\\
\parbox{2cm}{b  ɛ  y  t}\parbox{1.5cm}{\textit{bejt}}\parbox{2cm}{‘roof’}
%\parbox{2cm}{b  ɛ  y  t}  \parbox{1.5cm}{\textit{bejt}} \parbox{1.5cm}{  ‘roof’}
\ex% (32b) 
\parbox{2cm}{b  ɛ  w  s}\parbox{1.5cm}{\textit{bews}}\parbox{2cm}{  ‘kisses’}\\
\parbox{2cm}{d  a  w  l}\parbox{1.5cm}{\textit{dawl}}\parbox{2cm}{‘light’}
\z
\z

The syllabic pattern of these forms is CVCC, like in:


\ea%33
    \label{ex:puech:33}
\parbox{2cm}{ħ  ɔ  b  z}\parbox{1.5cm}{\textit{ħobż}}\parbox{2cm}{‘bread’}
\z

\paragraph{b-  \textit{Complex nucleus melodic fission}}
When \textit{ʕ} immediately precedes a long tense \isi{vowel} \textit{ī} or \textit{ū}, the long \isi{vowel} diphthongizes. In (39a) stem initial \textit{ʕ} is preceded by an empty space to avoid an OCP clash with the prefix \textit{t}, and is immediately followed by an underlyingly \isi{long nucleus} (step 1); the melody of the long \isi{vowel} splits: The element \{A\} is copied from the pharyngeal \isi{glide} to occupy the first \isi{space unit}, while the element \{I\} occupies the second \isi{space unit} (step 2); in (39b) the \isi{diphthong} is phonologized.

Melodic fission is illustrated by Stumme's examples where \isi{sonorant} \textit{ʕ} triggers the \isi{diphthongization} of the adjacent \isi{vowel}, i.e. in ‘t{\З}{e}\u{\i}d’ \textit{tgħid}-\textsc{impf}.3 \textsc{f.sg} ‘she says’ and ‘sam{\З}{ou’} \textit{semgħu}-\textsc{pf}.3\textsc{pl} ‘they heard’:


\ea%(34a)  
\ea
Step 1 (etymon)\\


\begin{tabularx}{\textwidth}{|p{4mm}|p{4mm}|p{4mm}|p{4mm}|p{4mm}| c |p{4mm}|p{4mm}|p{4mm}|p{4mm}|p{4mm}|p{4mm}|} 
\multicolumn{1}{c}{t} & \multicolumn{1}{c}{} & \multicolumn{1}{c}{ ʕ} & \multicolumn{1}{c}{ ī} & \multicolumn{1}{c}{ d} & \multicolumn{1}{c}{} & \multicolumn{1}{c}{ s} & \multicolumn{1}{c}{ a} & \multicolumn{1}{c}{ m} & \multicolumn{1}{c}{} & \multicolumn{1}{c}{ ʕ} & \multicolumn{1}{c}{ ū}\\
\hhline{-----~------}
 C &  & C & V & C &  & C & V & C &  & C & V\\
&  & \underline{V} &  &  &  & V &  & \underline{V} &  & \underline{V} & \\
&  & V &  & V &  &  &  & C &  & V & \\
\hhline{-----~------}
 I &  &  & I & I &  & I &  & U &  &  & U\\
&  & \underline{A} &  &  &  &  & A &  &  & \underline{A} & \\
\hhline{-----~------} 
\end{tabularx}

\ex
Step 2: \isi{diphthongization}



\begin{tabularx}{\textwidth}{|p{4mm}|p{4mm}|p{4mm}|p{4mm}|p{4mm}|p{4mm}| c |p{4mm}|p{4mm}|p{4mm}|p{4mm}|p{4mm}|p{4mm}|p{4mm}|} 
\multicolumn{1}{c}{t} & \multicolumn{1}{c}{} & \multicolumn{1}{c}{ ʕ} & \multicolumn{2}{c}{ ei̯ }& \multicolumn{1}{c}{ d} & \multicolumn{1}{c}{} & \multicolumn{1}{c}{ s} & \multicolumn{1}{c}{ a} & \multicolumn{1}{c}{ m} & \multicolumn{1}{c}{} & \multicolumn{1}{c}{ ʕ} & \multicolumn{2}{c}{ ou̯ }\\
 
\hhline{------~-------}
 C &  & C & \multicolumn{2}{c|}{ V} & C &  & C & V & C &  & C & \multicolumn{2}{c|}{ V}\\
&  & \underline{V} & \multicolumn{2}{|c|}{} &  &  & V &  & \underline{V} &  & \underline{V} & \multicolumn{2}{c|}{}\\
&  & V & \multicolumn{2}{c|}{} &  &  &  &  & C &  & V & \multicolumn{2}{c|}{}\\
\hhline{------~-------}
 I &  &  &  & I & I &  & I &  & U &  &  &  & U\\
&  & \underline{A} & A &  &  &  &  & A &  &  & \underline{A} & A & \\
\hhline{------~-------} 
\end{tabularx}
\z
\z

\section{Phonological interpretation of orthographic \textit{h}}

In this contribution, I will overlook the residual role of \textit{h} in stems. Some of the relevant paradigms and alternations are commented in \citet[67]{Camilleri2014}. Suffice it to say that, in modern Maltese, stem-\textit{h} is most often assimilated to \textit{għ}, or to \textit{ħ}. My focus will be on the representation of \textit{h} in suffixes.

\subsection{\textit{h}-initial suffixes}

Orthographic \textit{h} behaves as a \isi{consonant} in object pronouns:

\ea\label{ex:puech:35}%(35a) 
  \ea\textit{ha} (or -\textit{hie}):  3\textsc{f.sg} object  (35b)  -\textit{na} (or -\textit{nie}):  1\textsc{pl} object
  \ex\textit{hom}:    3\textsc{pl} object     -kom:  2\textsc{pl} object
  \z
\z

In (\ref{ex:puech:35}a) \textit{h} patterns with suffixes starting by a \isi{consonant} in (\ref{ex:puech:35}b) with respect to stem structure and \isi{stress assignment} but has no surface realization:

\ea\label{ex:puech:36}
  \ea%(36a) 
      \textit{kitibha} ‘he recruited her’  [kitíba]\\
       \textit{kitibna} ‘he recruited us’  [kitíbna]
  \ex%(36b) 
      \textit{kitibhom}  ‘he recruited them’  [kitíbom]\\
      \textit{kitibkom}  ‘he recruited you-all’  [kitíbkom]
  \z
\z

If the object \isi{pronoun} is V-initial, the second stem \isi{vowel} in an open \isi{syllable} is deleted:

\ea\label{ex:puech:37}
\ea%(37a)
kitbek  ‘he recruited you-sg’  [kítbek]
\ex%(37b) 
kitbu  ‘he recruited him’  [kítbu]
\z
\z

In \REF{ex:puech:37} the second stem nucleus is syncopated in \isi{intervocalic position}, while a \isi{consonant} position blocks stem nucleus syncope in \REF{ex:puech:36}. An \isi{empty position} (represented by a median dot) blocks syncope as well in (\ref{ex:puech:38}b), since the second stem nucleus is not in \isi{intervocalic position}:

\ea\label{ex:puech:38}%(38a)
\ea
\gll ~ k  i  t  i  b  ·  n  a          \\
     cf. k  i  t  i  b  ·  k  o  m  \\
\ex
\gll k  i  t  i  b  ·  a\\
      k  i  t  i  b  ·  o  m\\
\z      
\z

The underlying representation of pronouns with \textit{h} in (\ref{ex:puech:35}a) above is:

\eabox{\label{ex:puech:39}
\begin{tabular}{|lllllllll|}
\multicolumn{1}{c}{\textit{h}} &  \textit{a} & or &  \textit{h} & \textit{ie} \textellipsis  &  &  \textit{h}  & \textit{o} & \multicolumn{1}{c}{\textit{m}} \\
 \hline
	 ·         &  V           &      &     ·          &  \={V}                          &  &  ·              &  V           & C\\
               &               &      &                 &                                   &   &                &               & \underline{V}\\
               &               &      &                &                                    &   &                &               & C\\
               &               &      &                & I                                  &   &                & U           &  U \\
               & A            &      &                &                                    &    &              &   A           & \\
               \hline
\end{tabular}
}

Notice that in \textit{kitibna} the \isi{suffix} \isi{consonant} is separated from the last \isi{stem consonant} by an \isi{empty position}, since a sequence ‘CC’ would be an OCP violation. Similarly, the representation of \textit{kitibha} as */k i t i b · · a/, with two adjacent empty positions, would be an OCP violation. Orthographic \textit{h} occupies a single \isi{space unit} left empty, i.e. not occupied by either the element \{C\} or \{V\}. 

% \subsubsection{Intervocalic \textit{h}: hiatus resolution}

In \isi{intervocalic position}, the empty \isi{space unit} is occupied by a palatal or labial \isi{glide} in agreement with the preceding \isi{vowel}. According to \citet[275]{Borg1997}, the underlying long \isi{vowel} is shortened in case of \isi{glide insertion}:


\puechlengths{3cm}{2.8cm}{2cm}{3.5cm}{1.5cm}{1.5cm}

\ea%(40a) 
\begin{xlist}
 \exbox{
\puechquadruple{k  ·  s  ī  ·  a }{  [k s í y a] }{ \textit{ksieha} }{ ‘he covered it-\textsc{f.sg}’}
}
\exbox{
\puechquadruple{y  i  š  ·  t  ·  r  ū  ·  o  m }{ [yištrúwom] }{ \textit{jixtruhom} }{ ‘they buy them’}
}
\end{xlist}
\z

Yet, Borg quotes \isi{dialectal} forms in which the long \isi{vowel} and the inserted \isi{glide} yield a \isi{geminate} \isi{glide}; cf. \citet[87]{Puech1994}:


\ea%41
    \label{ex:puech:41}
          \puechquadruple{t  ·  r  ī  d  ū  ·  i  m }{ [tridúwwim] }{ \textit{triduhim} }{ ‘you-\textsc{pl} want them’ }
    \z


If both vowels are [nonhigh] they are fused \citep[276]{Borg1997}:

\ea%(42a)
\begin{xlist}
\exbox{
\puechquadruple{š  ·  t  ·  r  ā  ·  a }{ [štr\'ā] }{ \textit{xtraha} }{ ‘he bought it-\textsc{f.sg}’ }
\puechquadruple{ ~}{\hspace*{-.7mm}[štrá] }{ ~}{ ~}
}
\exbox{
\puechquadruple{š  ·  t  ·  r  ā  ·  o  m }{ [štrá.om] }{ \textit{xtrahom} }{ ‘he bought them’}
\puechquadruple{ ~}{\hspace*{-.7mm}[štr\'{\=o}m] }{ ~}{ ~}
}
\end{xlist}
\z

In the mirror sequence ‘V · \={V}\'{} ’, due to word-stress shift, \isi{glide insertion} is optional:

\ea%(42c)
\puechquadruple{š  ·  t  ·  r  ā  ·  ī  l  i }{ [štra.\'{ī}li] }{ \textit{xtrahieli} }{ ‘he bought it for me’ }
\puechquadruple{ ~}{ [štray\'{ī}li] }{ ~}{ ~}
\z

\subsection{Word-final \textit{h}}
When the 3\textsc{m.sg} object \isi{suffix} is immediately preceded by a \isi{vowel} and is word-final (or only followed by enclitic negation \textit{š}), its \isi{allomorphic} realization is \textit{ħ}:

\ea%43
    \label{ex:puech:43}
\smallpuechtriple{ \textit{kitbuh} }{ ‘they wrote it-\textsc{m}’ }{ [k i t b \'{ū} ħ] }
\smallpuechtriple{ \textit{ktibnieh} }{ ‘we wrote it-\textsc{m}’ }{ [k t i b n \'{ī} ħ]}
    \z

This applies to stems with \isi{etymological} \textit{h} in \isi{word-final position}: 

\ea%44
    \label{ex:puech:44}
          \puechquadruple{\textit{kerah} }{ ‘ugly-\textsc{m.sg}’ }{ [ k é r a ħ] }{ cf. \textit{kerha} ‘ugly-\textsc{f.sg}’ }
    \z


\subsection{Conclusion}
Orthographic \textit{h} stands for a \isi{virtual consonant}, i.e. occupies a C-position whose vertex is empty. Its underlying presence is revealed by its effect on the \isi{syllabic structure} and \isi{stress assignment}, or by a \isi{glide} preventing hiatus. In \isi{word-final position} (disregarding the negative enclitic) it is represented by an allomorph \textit{ħ}, an instance of phonologically conditioned allomorphy.

\section{Phonological interpretation of the digraph \textit{għ}}

The sonorants \textit{ʕ} and \textit{ɣ}, represented in modern spelling by the digraph \textit{għ}, are \isi{etymological} in many roots as first, second, or third radical. My focus will be on diachronic changes in stems whose one radical was \textit{ʕ} or \textit{ɣ}. I identify four stages, which synchronically correspond to overlapping lengths. %in synchrony to overlapping lects.

\subsection{\textit{Għ} adjacent to a (mid)low vowel}
In premodern Maltese, \textit{ʕ} and \textit{ɣ} correspond to two distinct sonorants. In most dialects, however, they have been merged. The examples below are drawn from the \isi{verb} \textit{għamel} ‘to make’ for two forms in the perfect (3rd \textsc{m.sg} and 1/2 sg). Concerning the quality \textit{i} or \textit{e} of the second stem \isi{vowel}, suffice it to say that it depends on contexts and dialects; cf. \textit{għamel} ‘he made’ vs. \textit{għamilt} ‘I made’:

\ea
\textbf{Lect A} (lost in contemporary Maltese)

%  (45a)
%  (45b)  
\begin{tabular}{|lllll p{1cm} lllll l l|}
\multicolumn{1}{c}{ʕ} & á & m & e & l &&  ʕ & a&  m & í & l & & \multicolumn{1}{c}{t}\\
\hline
C & V & C & V & C &   & C & V & C & V & C & · & C\\
\underline{V} && \underline{V} && \underline{V} && \underline{V} && \underline{V} && \underline{V} & &\\
  &   & C &   &   &   &   &   & C &   &   &  & \\
 &    & U & I & I &   &   &   & U & I & I &  & I\\
\underline{A}
 & A  &   & A &   &   & \underline{A} & A &   &   &   &  & \\ 
 \hline
\end{tabular}

\z

%\todo{redo remaining graphs}
The loss of articulation of the \isi{sonorant} is compensated by \isi{vowel pharyngealization} and length. The initial C-position is represented by an \isi{empty vertex} associated to a headed element \{A\}. This position is merged with the adjacent (mid)low nucleus. Pharyngealization results from the embedding of the pharyngeal \isi{sonorant} in the \isi{vowel}: it is marked by the element \{C\} in dependent position under the vertex \{V\}. Length is kept in the pharyngealized \isi{vowel} even if it is \isi{unstressed}:

\ea %(46a) 
\textbf{Lect B} (conservative lect; cf. \citealt{Aquilina1959}, \citealt{Schabert1976})
\begin{tabular}{|lllll p{1cm} lllllll|}
\multicolumn{1}{c}{}& {\'ā̰ } & m & e & l & & & ā̰ &  m & í & l & & \multicolumn{1}{c}{t}\\
\hline
 	·	&	\={V}	&	C	&	V	&	C	&&	·	&	\={V}	&	C	&	V	&	C	&	·	&	C \\
		&	C	&\underline{V}	&		&\underline{V}	&&		&	C	&\underline{V}	&		&\underline{V}	&		&	\\
		&		&	C	&		&		&&		&		&	C	&		&		&		&	\\
		&		&	U	&	I	&	I	&&		&		&	U	&	I	&	I	&		&	I \\
\underline{A}	&	A	&		&	A	&		&&\underline{A}	&	A	&		&		&		&		&	\\
\hline	
\end{tabular}
\z

 
Pharyngealization has been lost but the \isi{vowel} retains length, whether it is stressed or not. The paradigm of stems with initial \textit{għ} is similar to that of stems with initial \textit{h}, like \textit{hemeż} ‘to pin’; cf. \citet[19]{Camilleri2014}. The initial underlying position is represented by an \isi{empty vertex} associated to a \isi{headless} \isi{melodic element} \{A\}. In being merged with the \isi{empty position}, this nucleus remains ‘long’ in all positions: 

  
\ea%(47a) 
\textbf{Lect C}   \\
\begin{tabular}{|llll p{1cm} llllll|}
\multicolumn{1}{c}{ā́}&m	&	e	&	l	&&	ā	&	m	&	í	&	l	&		&\multicolumn{1}{c}{t} \\
\hline
\={V}	&	C	&	V	&	C	&&	V	&	C	&	V	&	C	&	·	&	C\\
C	&\underline{V}	&		&\underline{V}	&&	C	&	\underline{V}	&		&	\underline{V}	&		&	\\
	&	C	&		&	C	&&		&	C	&		&		&		&	\\
	&	U	&	I	&	I	&&		&	U	&	I	&	I	&		&	I\\
A	&		&	A	&		&&	A	&		&		&		&		&	\\
\hline
\end{tabular}
\z


The initial nucleus is long if it is stressed, and short if stress migrates rightward. Thus, the underlying representation is restructured. It starts with an underlying \isi{long nucleus} which behaves as an ordinary \isi{long nucleus} with respect to length:

\ea%(48a)
\textbf{Lect D}       \\
\begin{tabular}{|llll p{1cm} llllll|}
\multicolumn{1}{c}{ā́}&m	&	e	&	l	&&	a	&	m	&	í	&	l	&		&\multicolumn{1}{c}{t}\\
\hline
\={V}	&	C	&	V	&	C	&&	V	&	C	&	V	&	C	&	·	&	C\\
	&\underline{V}	&		&	V	&&		&\underline{V}	&		&\underline{V}	&		&	\\
	&	C	&		&		&&		&	C	&		&		&		&	\\
	&	U	&	I	&	I	&&		&	U	&	I	&	I	&		&	I\\
A	&		&	A	&		&&	A	&		&		&		&		&	\\
\hline
\end{tabular}
\z

In the nineteenth century, the conservative lects A and B overlapped; cf. \citet{Bonelli1897,Bonelli1898,Bonelli1900}, \citet{Stumme1904}. Lects B and C are well documented in the twentieth century; cf. \citet{Aquilina1959}, \citet{Schabert1976}, \citet{Vanhove1991}. Conservative lect C and innovative lect D are representative of contemporary Maltese.

\subsection{\textit{Għ} adjacent to an underlying high vowel}
When initial \textit{għ} precedes the vowels \textit{ī} or \textit{ū}, a (pharyngealized) \isi{diphthong} occurred in most dialects, including Standard Maltese: 


\puechlengths{1.7cm}{1.7cm}{3cm}{4.4cm}{1.5cm}{1.5cm}

\ea%(49a) 
\begin{xlist}
\exbox{%
\puechquadruple{[ái̯d]}{\textit{għid}}{‘(religious) feast’}{\textit{għ} stands for \isi{etymological}  \textit{ʕ}} 
\hspace*{-.5mm}\puechquadruple{[zái̯r]}{\textit{żgħir}}{‘small’}{\hfill \textit{ɣ}}
}
\exbox{%
\puechquadruple{[áu̯d]}{\textit{għud}}{‘wooden hook’}{\textit{għ} stands for \isi{etymological}    \textit{ʕ}}
\hspace*{-.5mm}\puechquadruple{[áu̯l]}{\textit{għul}}{‘ogre’}{\hfill\textit{ɣ}}
}
\end{xlist}
\z


  
The \isi{diphthong} is kept in internal position:

 
\puechlengths{1.7cm}{1.7cm}{5cm}{4.4cm}{1.5cm}{1.5cm}
\ea%(50a) 
\begin{xlist}
\exbox[-.55\baselineskip]{
\smallpuechtriple{[zái̯ra] }{ \textit{żgħira} }{ ‘small-\textsc{f.sg}’}
}
\vspace*{-5mm}
\exbox[-.55\baselineskip]{
\smallpuechtriple{[áu̯da] }{ \textit{għuda} }{ ‘(piece of) wood-\textsc{f.sg}’}
}
\end{xlist}
\z

The quality of the nucleus in the \isi{diphthong} is variable: \textit{a}\u{\i}{\textasciitilde}\textit{e}\u{\i}{\textasciitilde}\textit{o}\u{\i} for \textit{għī}, \textit{e}ŭ{\textasciitilde}\textit{a}ŭ{\textasciitilde}\linebreak\textit{o}ŭ for \textit{għū}; cf. 
\citet[54]{Aquilina1959}, 
\citet[73]{Borg1978}, 
\citet[270]{Borg1997}, 
\citet[299]{BorgAzzopardi1997}. Variations are observed between rural dialects, with several intervening factors, but also within Standard Maltese. Notice also the absence of \isi{diphthongization} triggered by \textit{għ} in the \isi{dialectal} area of the Grand Harbour. All realizations below are attested for \textit{żgħir} ‘small-\textsc{m.sg}’:


\ea%51
    \label{ex:puech:51}
\begin{xlist}    
\exbox[-.55\baselineskip]{
[zɣai̯r], [zoi̯r] (in \isi{Gozo})
} 
\exbox[-.55\baselineskip]{
 [za̰i̯r], [zḛi̯r], [zei̯r], [zīr] (in the Grand Harbour)
 }
\end{xlist}
 \z


\subsection{Digraph \textit{għ} followed by \textit{h}-suffix}

Concerning \textit{għ} as third radical, if the stem is followed by an object \isi{suffix} starting with \textit{-h}, the sequence ‘għ-h’ is realized as [ħħ]:

\ea%(54a) 
\begin{xlist}
 \exbox[-.55\baselineskip]{
\smallpuechtriple{\textit{bela’} }{ [béla] }{ ‘he threw’ }
}
\vspace*{-5mm}
% (54b) 
\exbox[-.55\baselineskip]{
\smallpuechtriple{\textit{belagħha} }{ [beláħħa] }{ ‘he threw it’}
}
\end{xlist}
\z

It should also be noticed that there are innovations in inflections, even more so in the language of young people. For example, \citet[99]{Fabri2011}:


\puechlengths{1cm}{2cm}{4cm}{3.5cm}{1.5cm}{1.5cm}

\ea%(55a)
\label{ex:puech:55}
\begin{xlist}
\exbox[-.55\baselineskip]{ 
\smallpuechtriple{\textit{raha} }{ [r\'{ā} ] }{ ‘he saw her’ }
}
\vspace*{-5mm}
\exbox[-.55\baselineskip]{
% (55b) 
\smallpuechtriple{ ~}{ [raħħa] }{ on the model of (\ref{ex:puech:55}b) }
}
\end{xlist}
\z


\subsection{Conclusion}
In diachrony, the main reflexes corresponding to \isi{etymological} \isi{guttural} sonorants are: \isi{vowel diphthongization} and/or \isi{pharyngealization}, lengthening, and \isi{allomorphic} \textit{ħ}. In the synchrony of contemporary Maltese, I claim, using the same terms as Brame, that on the basis of Maltese \isi{phonetics} and morpho-phonemic patterns, the child coming to the \isi{language-learning situation} is capable of inducing underlying stems that include an \isi{empty vertex} associated to the \isi{melodic element} \{A\}:  

\begin{itemize}
 \item 
 in position of first radical; cf.  \textit{għamel}:  \\
\begin{tabular}{|lllll|}
\hline
  · &  V &  C &  V &  C \\
  A & A &      &     & \\
\hline
\end{tabular}  

\item  in position of second radical; cf.  \textit{lagħab}:  \\
\begin{tabular}{|lllll|}
\hline
  C & V & · & V & C \\
      & A & A & A &     \\
\hline
\end{tabular}  

\item in position of third radical; cf.  \textit{sema’}:  \\
\begin{tabular}{|lllll|}
\hline
  C  & V & C & V &  ·\\
       &    &     & A & A \\
\hline
\end{tabular}       
\end{itemize}


Applicable \isi{phonological} processes are fusion, \isi{diphthongization} and deletion. Other cases are accounted for by (phonologically-conditioned) suppletive allomorphy. 

\section{Representation of modern Maltese consonants}\label{sec:puech:9}

Tables \ref{tab:puech:3} and \ref{tab:puech:4} below give the inventory of contemporary Maltese consonants. Compared to the inventory in Tables 1 and 2, there is on the one hand the loss of \isi{emphatic} and \isi{guttural} consonants, and, on the other hand, the introduction of new phonemes, due to massive borrowings from \ili{Sicilian}, \ili{Italian} and, nowadays, from English; cf. \citet{Mifsud1995loanverbs}. Concerning \textit{ħ},  \citet[301]{BorgAzzopardi1997} state: 

\begin{quote}
Orthographic \textit{ħ} always corresponds to /h/; \isi{orthographic} \textit{għ} and \textit{ħ} correspond to /h/ in word final position or when they occur together (\isi{orthographic} \textit{għ} + \textit{ħ}). /h/ is articulated as a convexed (central) post-palatal, \isi{velar}, \isi{glottal} or pharyngeal \isi{voiceless fricative}. Its place of articulation varies according to the vocalic context that follows it. However, partially (but often fully) \isi{voiced} when it precedes \isi{voiced} obstruents but does not occur in opposition to a \isi{voiced} \isi{velar} or post-\isi{velar} \isi{fricative}. 
\end{quote}



\begin{table}
\caption{Obstruents in Modern Maltese}
\label{tab:puech:3}
\begin{tabularx}{\textwidth}{X @{~}c@{~}c c@{~}c c@{~}c c@{~}c c@{~}c c@{~}c c@{~}c  c@{~}c c@{~}c}
\lsptoprule
{\em Segment} & \textbf{p} & \textbf{b} & \textbf{f} & \textbf{v} & \textbf{t} & \textbf{d} & \textbf{s} & \textbf{z} & \textbf{ʦ} & \textbf{ʣ} & \textbf{š} & \textbf{ž} & \textbf{ʧ} & \textbf{ʤ} & \textbf{k} & \textbf{g} & \textbf{h} & \textbf{ʔ}\\
\midrule
\multirow{3}{*}{\em Structure} & C & C & \underline{C} & \underline{C} & C & C & \underline{C} & \underline{C} & \underline{C} & \underline{C} & \underline{C} & \underline{C} & \underline{C} & \underline{C} & C & C & \underline{C} & C\\
& & & V & V & & & V & V & & & V & V & & & & & V & \\
& & V & & V & & V & & V & C & V & & V & C & V & & V & C & C \\
\midrule
\multirow{2}{*}{Melody} 
& U & U & U & U & I & I & I & I & I & I & \underline{I} & \underline{I} & \underline{I} & \underline{I} &  &  &  & \\
&  &  &  &  &  &  &  &  &  &  &  &  &  &  &  &  & A & A\\
\lspbottomrule
\end{tabularx}
\end{table}

\begin{table}
\caption{Sonorants}
\label{tab:puech:4}
\begin{tabularx}{\textwidth}{XZ{1cm}Z{1cm}Z{1cm}Z{1cm}Z{1cm}Z{1cm}}
\lsptoprule
  & \textbf{m} & \textbf{n} & \textbf{l} & \textbf{r} & \textbf{y} & \textbf{w}\\
  \midrule
& C & C & C & C & C & C\\
Structure & \underline{V} &  \underline{V} &  \underline{V} &  \underline{V} &  \underline{V} &  \underline{V}\\
& C & C &  &  &  & \\
\midrule 
\multirow{2}{*}{Melody} 
& U & I & I & I & \underline{I} & \underline{U}\\
  &  &  &  & A &  & \\
 
\lspbottomrule
\end{tabularx}
 \end{table}

\newpage  
I repeat for convenience the representation of major categories given in \sectref{sec:puech:2.2} in   \tabref{tab:puech:repeatedcategories}. 




 \begin{table}
\caption{Representation of major categories (repeated)}
\label{tab:puech:repeatedcategories}
% \begin{tabularx}{\textwidth}{XCCCC}
%    & Stops & Fricatives & Affricates & Sonorants \\
%    \midrule
% 	      &  C   & C          &  \underline{C} &  C \\
% Structure	      &		& V       &  V		   &  \underline{V}   \\
% 	      &  (V)  	& C or V & (V)		   &  (C or V) \\
% \midrule
% \end{tabularx}
 \begin{tabularx}{\textwidth}{QcccccccccQ}
 \lsptoprule
      & \multicolumn{3}{c}{\bf Stops}  & \multicolumn{3}{c}{\bf Fricatives} & \multicolumn{3}{c}{Sonorants} &\\
      \midrule 
      &  \textit{weak} & / & \textit{strong}			& \textit{weak} & / & \textit{strong}		& \textit{weak} & / & \textit{strong} &\\
      &  C		& & \underline{C} 	& C & & \underline{C} 	& \multicolumn{3}{c}{C} &\\
      &  		& &					& V & & V 				& \multicolumn{3}{c}{\underline{V{}}} &\\
      &  \multicolumn{3}{c}{(C or V)} & \multicolumn{3}{c}{(C or V)} & (V) & & C &\\   
 \end{tabularx}

 \medskip

\begin{tabularx}{\textwidth}{Xccccc}	
  & {\bf Labials} & {\bf (Denti)alveolar} & {\bf Post-alveolars} & {\bf Palato-velar} &  {\bf Guttural} \\
\midrule
{\em Melody}	&	U  &	 I	    & \underline{I} & none 	& A \\
\lspbottomrule
\end{tabularx}			  
\end{table}
  
\section{Conclusion}
The loss of \isi{emphatic} consonants in postmedieval Maltese transferred the burden of maintaining lexical contrasts to stem vowels only. Four centuries later, the loss of \isi{guttural} consonants broke up regular morphophonemic alternations, inducing opacity in the \isi{sound pattern}. In other words, there has been a trade-off between ‘less’ on the \isi{phonological} side and ‘more’ on the morpho-\isi{phonological} side. Until the (post)medieval stage the \isi{natural class} of \isi{emphatic} and \isi{guttural} consonants was characterized by the element \{A\}. In pre-modern Maltese, in which \textit{h} was already on its way out and \textit{q} had not yet been replaced by a \isi{glottal} stop, the narrow ‘\isi{guttural}’ class was characterized by \{A\}. This class extended from (post)velars to pharyngeal consonants. In modern Maltese, only voiceless pharyngeal \textit{ħ} and \isi{glottal} stop \textit{ʔ} retain the element \{A\} in their representation. A further step in the \isi{sound pattern} shift tends to favor more urban \isi{laryngeal} \textit{h} over more rural pharyngeal \textit{ħ} as the main \isi{allophone} for the ‘\isi{guttural}’ \isi{fricative}. 

Using the CV framework, I argued that the \isi{etymological} \isi{laryngeal} approximant \textit{h} must be analyzed as an empty C-position: its phonotactic behavior is that of a \isi{consonant} with respect to \isi{syllabic structure} and stress, but it has no autonomous realization. In some contexts, however, it is directly represented by a \isi{glide} in \isi{intervocalic position}, or by the voiceless \isi{guttural} \isi{fricative} in \isi{word-final position}. An \isi{empty vertex} hosting the \isi{melodic element} \{A\} is the direct reflex of former \isi{guttural} sonorants, indirectly expressed by pharyngealizing and lengthening effects on adjacent vowels. Once \isi{pharyngealization} has been lost in contemporary Maltese, it is no longer justified to maintain two different underlying representations corresponding to \isi{orthographic} \textit{h} and \textit{għ}. The \isi{sound pattern} only requires that underlying representations may include a content-empty unit of \isi{phonological} space. Surface forms are generated by regular \isi{phonological} processes, or phonologically-conditioned allomorphy. In Brame's terms, children “coming to the \isi{language-learning situation}” are endowed by UG with the capacity of inducing the role of empty consonants in the \isi{sound pattern} they acquire. Yet, some regularities are of an \isi{allomorphic} rather than a \isi{phonological} nature. Thus, canonical paradigm acquisition is also necessary in the learning process. 

I would like to end this contribution by extending the \isi{phonological} predictions made by \citet[99]{Fabri2011}:

\begin{quote}
Bringing in the acquisition perspective once again, another observation is relevant in this context. A diary of language development of my own son, Noah, shows clearly that he often omitted the \isi{glottal} stop for quite a long time during his acquisition phase. Moreover, even when he learnt how to write at school, he would systematically omit the letter ‘q’, which represents the \isi{glottal}, thus implying that he was not even aware of its occurrence. It is, therefore, not implausible to speculate that one way in which Maltese could change is the occurrence of the \isi{glottal} stop, a change that also affects its \isi{phonemic status} within the \isi{phonological} system.
\end{quote}

If Noah's children induce a \isi{sound pattern} without any ‘\isi{guttural}’ \isi{consonant} characterized by the \isi{melodic element} \{A\}, the millenary cycle of transferring \isi{guttural} load from consonants to vowels will have been completed. 


\section*{Abbreviations}
% \todo{please conform to Leipzig Glossing Rules}
\begin{tabularx}{.45\textwidth}{lQ}
\textsc{pf} & Perfect\\
\textsc{\textsc{impf}} & Imperfect\\
\textsc{imper} & Imperative\\
\textsc{pp}&  Past participle    \\ 
1, 2, 3&   1st, 2nd, 3rd person\\
\end{tabularx}
\begin{tabularx}{.45\textwidth}{lQ}
\textsc{m}&Masculine\\
\textsc{f} & Feminine\\   
\textsc{sg}& Singular\\
\textsc{pl} & Plural\\
Obj&   (suffixed) Object \isi{pronoun}\\
\end{tabularx}

\clearpage 
\section*{Appendix: Diachronic changes in Maltese gutturals}
% \newcommand{\tikzmark}[2]{\tikz[overlay,remember picture] \node (#1) {#2};}
% %
% \newcommand{\connect}[2]{%
%   \tikz[overlay,remember picture]{%
%     \draw[-,thick] (#1) -- (#2) node   {};  %
%   }
% } 
 


\begin{table}[h!]
 \caption{Diachronic changes in Maltese gutturals}  
\begin{tabularx}{\textwidth}{lccccQ}
\lsptoprule
\multicolumn{5}{l}{\textbf{Pre-modern Maltese}} \\
i  &  ḵ/q\textsuperscript{1} & χ ħ                &  h{\textasciitilde}⌀ & ɣ\textsuperscript{2} ʕ & \citet{AgiusdeSoldanis}\\

\tablevspace
% [Warning: Draw object ignored][Warning: Draw object ignored]
ii &  q                       & χ{\textasciitilde}ħ&\tikzmark{hii}{h} 
						      & ɣ{\textasciitilde}ʕ & \citet{Vassalli1796,Vassalli1827}\\

\tablevspace
\multicolumn{3}{l}{\textbf{Modern Maltese}} & 
				  \tikzmark{hmod}{h{\textasciitilde}⌀}  
						  &\tikzmark{glottmod}{ʕ} 
							    &   \citet{Bonelli1897,Bonelli1900}\\
iii & q/ʔ\textsuperscript{3} &  ħ  & \tikzmark{apoiii}{’{\textasciitilde}⌀}  & ʕ & \citet{Stumme1904,Saada1986}\textsuperscript{4}\\
\tablevspace
iv & ʔ                       & ħ   &                      &\textit{ɣ}{\textbar}A̰{\textbar}\textsuperscript{5}  &    \citealt{Aquilina1959}\\
% % [Warning: Draw object ignored]
   &                         &     &                      & \textsuperscript{ʕ}{\textbar}A{\textbar}{\textasciitilde}{\textbar}\=A̰{\textbar}\textsuperscript{6}  
							      & \citet{Schabert1976}\\
\tablevspace
v & ʔ                        &  ħ &                      &    & \citet{Vanhove1991}\\
  &                          &    &                      &\textsuperscript{ʕ}{\textbar}\=A{\textbar}{\textasciitilde}{\textbar}\=A{\textbar} 
							      &\citet{Cohen1966,Cohen1970}\\
\tablevspace
\multicolumn{3}{l}{\textbf{Contemporary Maltese}} & & & \citet{AquilinaIsserlin1981}\textsuperscript{7}  \\ 
vi& ʔ                       &  ħ &                      &{\textbar}\=A/A{\textbar}\textsuperscript{8}
							      & \citet{Puech1994}\\
\tablevspace
vii&                        &h\textsuperscript{9}&      &      & \citet{Borg1997,BorgAzzopardi1997,Hume2009}\\
   & ʔ                      &  h &  \tikzmark{glidevii}{\isi{glide}{\textasciitilde}⌀\textsuperscript{10}} 
						       & {\textbar}\=A/A{\textbar} 
						              &\citet{Camilleri2014}\\
\lspbottomrule	    
\end{tabularx} 
\connect{hii}{hmod}
\connect{hii}{glottmod}
\connect{apoiii}{glidevii}
\end{table}
            
\subsubsection*{Notes to Table} 

\noindent
\textsuperscript{1}  Uvular \textit{q} alternates with retracted \textit{ḵ } before a [back] \isi{vowel}. Velar \textit{k} in this environment is still attested in Xewkija (\isi{Gozo}); e.g. [ḵalp] for \textit{qalb} ‘moon’ vs. [kælp] for \textit{kelb} ‘dog’.

\noindent
\textsuperscript{2}  Postvelar \textit{ɣ} is still attested for older speakers in some \ili{Gozitan} villages (Għarb, Qala); e.g. [ɣɑlɑʔ, ɣlɒʔt] for \textit{għalaq}, \textit{għalaqt}, ‘he / I closed’.

\noindent
\textsuperscript{3} Postvelar \textit{q} is still attested, at least for older speakers, in the Grand Harbour (\isi{Malta}) and \isi{Gozo} (Rabat). Phonemic \textit{ʔ} is distinct to the optional \isi{glottal} prosthetic onset in V-initial words..

\noindent
\textsuperscript{4}  Maltese spoken in Tunisia by \ili{French} citizens from Maltese families until the 1950s.

\noindent
\textsuperscript{5} The author postulates a set of mid- or low pharyngealized vowels (p. 18). These vowels are preceded or followed by the symbol \textit{ɣ}, corresponding to \isi{orthographic} \textit{għ} or \textit{h}. The Symbol {\textbar}A{\textbar} refers to mid or low vowels whose \isi{phonological} expression includes element \{A\}. 

\noindent
\textsuperscript{6} \citet[19-20]{Schabert1976} contrasts the mid/low long vowels [ā \={æ} ɔ] with the allophones [ā̰ ~~\={æ}̰  ɔ̰ ] of pharyngealized vowels; the two dialects described by Schabert are conservative.  

\noindent
\textsuperscript{7} \citet{AquilinaIsserlin1981} “cannot rule out that \isi{pharyngealization} may actually occur still in \ili{Gozitan} dialecrs” (p. 137), but “encountered no clear instances of ‘pharyngealized’ vowels near a no longer pronounced /ʕ/” (p. 114).

\noindent
\textsuperscript{8} At this stage, no \isi{pharyngealization}, but variable length due to several factors.

\noindent
\textsuperscript{9} Allophones of /h/ are [h, ħ, χ], depending on contexts and speech habits. In urban Maltese, the pharyngeal articulation is less valued than the \isi{laryngeal} one.

\noindent
\textsuperscript{10} In general, the reflex of \isi{etymological} \textit{h} is zero; e.g. [nifmu] \textit{nifhmu} ‘we understand’. Maltese avoids hiatus by inserting a homorganic \isi{glide} or by fusing two [nonhigh] vowels. In some dialects proto-\textit{h} has been assimilated to \textit{ħ}; cf. \textit{fehem}: [fēm] vs. [feħem].




 
\sloppy
\printbibliography[heading=subbibliography,notkeyword=this] 
\end{document}