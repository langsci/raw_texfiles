\documentclass[output=paper]{langsci/langscibook} 
\ChapterDOI{10.5281/zenodo.1181797}
\author{Sarah Grech\lastand
Alexandra Vella\affiliation{University of Malta}}
\title{Rhythm in Maltese English}
%\shorttitlerunninghead{}
%\ChapterDOI{} %will be filled in at production
\abstract{There is evidence to suggest that rhythm may be a key element in the identification of Maltese English, MaltE. A number of characteristics at different levels of structure have been noted in research on this variety. These include a number of phonetic and/or phonological features, some of which may combine to trigger the perception of a pronunciation which is identifiably MaltE. Amongst these features, examining aspects of duration and/or timing has been shown to be a worthwhile starting point in understanding the nature of the rhythm of MaltE. Such elements include, but may not be limited to, the preference for full over reduced vowels, the tendency to production of post-vocalic `r', and gemination of consonants \citep{Calleja1987, Vella1995, Debrincat1999, Grech2015}. It has been pointed out in research to date \citep{Arvaniti2009, Arvaniti2012, NokesHay2012}, that while durational characteristics cannot be assumed to be entirely responsible for different rhythm patterns, they remain pivotal, together with features including pitch, or intensity, in the perception of patterns of prominence which collectively could be referred to as rhythm. Following previous research by \citet{Grech2015} and \citet{GrechVella2015}, there are indications that a Pairwise Variability Index \citep{GrabeLow2002} can capture aspects of vowel duration and timing which can, in turn, translate into some measure of lesser or greater degrees of identifiability of this variety of English. This paper therefore reports on a study carried out using a normalised Pairwise Variability Index, nPVI, to measure local patterns of variability in vowel duration, as an indicator of rhythm patterns in 6 MaltE speakers. These speakers were rated in an earlier study \citep{Grech2015} as representing different degrees of identifiability as MaltE speakers on a continuum of variation. The extent of identifiability of these speakers is correlated to the nPVI results obtained in an attempt at addressing the matter of the extent to which rhythm characteristics may trigger listener perceptions of this variety.
} 
\maketitle

\begin{document}


% Malt(ese)E(nglish), Normalised Pairwise Variability Index (nPVI), Rhythm, Vowel Duration


\section{Introduction: Describing a new variety of English}
Native speakers of \ili{Maltese English} (\ili{MaltE}) frequently report recognising another \ili{MaltE} \isi{speaker} within a few seconds of speech, even if that speech is decontextualised, such as in an online video clip, or at an airport. The speed and certainty with which such instances of recognition are reported hints at predictable and systematically realised characteristics and features of speech at various levels of linguistic structure, but possibly most noticeably, at the phonetic and/or the \isi{phonological} levels. A recent study, \citet{Grech2015}, taps into this intuitive recognition in an attempt at beginning to determine more precisely which phonetic/\isi{phonological} features may be likely to trigger such perceptions in the first place. 

The presence of characteristics and features serving to distinguish this variety from other varieties of English would hardly be considered unusual, given that some form of English, alongside other languages, has been widely used throughout the Maltese islands since the British established a colony there in the early 1800s. However, there has been – and to a large extent there still is – hotly debated discussion surrounding the kind of English that is actually developing, with the ‘complaint tradition’ \citep{MilroyMilroy2012} about failing standards, and broadly termed ‘bad’ English frequently being very much at the heart of such debates. Traditionally dismissive attitudes towards the variety of English used in \isi{Malta} have perhaps until more recently, stymied focused research on variation in \ili{MaltE} and any of the socially meaningful patterns some of its features and characteristics might present. 

The English language first became relevant in \isi{Malta} in the context of some 200 years of colonial rule, making it the latest in a range of languages adopted alongside \ili{Maltese} as the island sought to tap into the Mediterranean trade routes and socio-political dynamics \citep{Brincat2011}.  Increasingly rooted in \isi{Maltese society}, English has become established as part of the \isi{bilingual} reality of the islands’ inhabitants, and as such it can be shaped and moulded to suit different contexts and social situations. It has therefore become increasingly important to be able to recognise the emerging \ili{MaltE} not simply in relation to an established ‘other’, such as Southern Standard \ili{British English}, SSBE, closely associated with school models of English, but more pertinently, in relation to the potentially socially meaningful range of variation within the variety itself. With the island’s geo-political position making it a feasible location for economic migrants, and an inevitable staging-post for refugees fleeing war, poverty and climate change in Africa and the Middle East, \ili{MaltE} also sometimes takes on the role of a lingua franca, as new communities seek access to employment, healthcare and schooling. \citet{Thusatetal2009}, \citet{Vella2012}, and \citet{CamilleriGrima2013} all refer to use of English as evident across different strata of \isi{Maltese society}, and this, together with Bonnici’s (2010) in-depth \isi{sociolinguistic} study of communities where \ili{MaltE} is the primary means of communication, suggests that this variety is on the cusp of an endonormative stage of development, which \citet{Schneider2003} refers to as ‘nativisation’. 

A study in \citet{Grech2015} sought to circumvent the more strongly held attitudinal stances towards \ili{MaltE} by drawing on introspective perception judgments instead. An experimental study with 28 native \ili{MaltE} listeners judging ten speakers, was designed in such a way as to bypass more overtly held attitudinal positions towards \ili{MaltE}, and to focus instead on its structure. In each case, the 28 listeners were presented with ten 12-15 second clips involving ten different \ili{MaltE} speakers. While the speakers were all Maltese, one of the speakers had also lived in England for a few years, and was therefore expected both to have acquired some new features or to have modified some of the expected \ili{MaltE} features, and to be identified by native \ili{MaltE} listeners as a little different from the rest of the cohort. The remaining speakers were all Maltese, having grown up, been schooled and then established themselves in \isi{Malta}. Nevertheless, they also displayed different degrees of linguistic variation, due to a number of social and linguistic factors widely recognised as having an impact on language usage in \isi{Malta}, such as type of schooling, social background, or peer group identity \citep{Vella2012,CamilleriGrima2013}. The recorded clips prepared were extracted from longer conversations and tasks designed to generate a similar range and type of lexis and use of language across speakers. All the clips contained phonetic/\isi{phonological} features which an earlier study \citep{Grech2015} had identified as relevant to the identification of \ili{MaltE}. 

It is important to recognise that \ili{MaltE} is not a homogenous entity, but in fact also presents variation within the variety, what Mori (forthcoming) refers to as a ``continuum of continua''. Findings in \citet{Grech2015} echo aspects of earlier research on the variety of English used in \isi{Malta} to suggest that variation can be found at all levels of linguistic analysis. However, it is also suggested (see also \citealt{Vella1995, Bonnici2010} that while variation at the phonetic/\isi{phonological} levels is likely to cut across different social groups and linguistic backgrounds, variation at other levels is likely to be more contained within a particular subgroup of the variety. In particular, \citet{Bonnici2010} found this to be the case with respect to the question of rhoticity. She suggests that earlier generations may have aimed at a less rhotic variety in a semi-conscious effort to emulate perceived standards of correctness in relation to SSBE, possibly also an impression which might have been transmitted through schooling. Conversely, younger generations may be adopting a more rhotic \isi{accent} in an effort to distance themselves precisely from too close an association with this variety.  \figref{fig:key:grech1} below describes a possible schema for some of the characteristics which have featured most prominently in research on \ili{MaltE}. Those features related to \isi{phonetics}/\isi{phonology} have so far been reported to be the ones most likely to be present to some extent across all varieties of \ili{MaltE} \citep{Vella1995}; by contrast features in other domains, such as pragmatic features, for example, may be drawn on in more specific or restricted contexts. 

\begin{figure}
\caption{\label{fig:key:grech1}Two dimensions of variation in MaltE} 
%\includegraphics[width=.8\textwidth]{figures/a8GrechVella-img1.png} 

\fittable{
\begin{tabular}{L{4cm}|L{4cm}L{4cm}L{4.5cm}}
\lsptoprule
 & 
 \multicolumn{1}{c}{Syntax/Morphology}  &
 \multicolumn{1}{c}{{Semantics/Lexicon}} & 
  \multicolumn{1}{c}{Pragmatics}
 \\
 \midrule
  Different features can be present \textbf{to varying degrees} including \textbf{not at all}
& 
 Pronoun copying
 \newline[1em]
 Sentence final \textit{but}
 \newline[1em]
 Topicalisation/fronting
 \newline[1em]
 Variant use of modals
 \newline[1em]
 Variant question formation
 \newline[1em]
 …   & 
 \textit{even I} (‘me too’)
 \newline[1em]
 \textit{pocket} (‘pencil case’)
 \newline[1em] 
 \textit{slipper} (‘running shoes’)
 \newline[1em]
 \textit{stay}+\textit{ing} (continuity)
 \newline[1em]
 periphrastic \textit{of} for possession
 \newline[1em]
 …   &  
  Complimentation
 \newline[1em]
 Discourse markers \textit{mela}, \textit{ta}
 \newline[1em]
 Phatic communication
 \newline[1em]
 Politeness strategies
 \newline[1em]
 Register (formal, careful vs. casual speech)
 \newline[1em]
 … 
\\
\midrule\midrule 
 Usually present \textbf{to some degree} even in the absence of other features &
  \multicolumn{3}{L{13.5cm}}{Segmental features such as neutralization or variant \isi{pronunciation} of /θ/-/ð/ contrast, absence of dark ‘l’, \isi{pronunciation} of /ŋ/ in \textit{ing} as [ŋg] 
 \newline[1em]
 Features such as \isi{vowel} quality and duration, rhoticity, \isi{consonant} gemination
 \newline[1em]
 Reflex of the above on rhythmic characteristics
 \newline[1em]
 Idiosyncratic stress patterns
 \newline[1em]
 Idiosyncratic intonation
 }
 \\
 \midrule
& \multicolumn{3}{c}{ {Phonology/Phonetics}}\\
\lspbottomrule
\end{tabular}
}
 
% \todo[inline]{redo this as a proper table}
\end{figure}

We can therefore consider the notion of \ili{MaltE} as one which operates more on a continuum of variation, particularly in the case of phonetic and \isi{phonological} features which may serve to identify the \isi{speaker} to a greater or lesser extent as a \isi{speaker} of \ili{MaltE} as opposed to as a \isi{speaker} of some other variety of English. This view takes its cue from the notion of a ‘cline’ proposed in earlier \isi{sociolinguistic} accounts of world varieties of English \citep[57]{Kachru1992}, where the different functional uses of English in a given community may generate variation within that particular variety of English. For \ili{MaltE}, \citet[4]{Borg1980} also makes reference to the presence of such intra-variety variation in the English used in \isi{Malta} when he talks of ‘gradation’ of usage across different social strata (but again, see also Mori, forthcoming).

In this respect, one of the richest levels of linguistics to yield evidence of variation which both distinguishes \ili{MaltE} from other varieties, and also distinguishes individual \ili{MaltE} speakers from each other, involves the phonetic/\isi{phonological}. The rest of this chapter reports on a study investigating durational characteristics in \ili{MaltE}, using the so-called Pairwise Variability Index \citep{GrabeLow2002}, as a means of measuring variation in the rhythm of \ili{MaltE}. \sectref{sec:key:grech2} presents the background to the study, beginning with an overview of rhythm and its measurement, and continuing with a brief investigation of other structural features evident in \ili{MaltE} which are likely to influence the overall perception of rhythm. The methodology and design of the experimental study are presented in \sectref{sec:key:grech3}, while \sectref{sec:key:grech4} describes the findings and preliminary indications for further study to be carried out as we move in the direction of a more comprehensive description of \ili{MaltE}.

\section{Rhythm and durational characteristics}
\label{sec:key:grech2}
\subsection{Rhythm and its measurement}
\label{sec:key:grech2.1}
When attempting to identify characteristic features of the speech of a newly emerging language variety such as \ili{MaltE}, an approach accounting for both the localised, physical events of speech as well as their “symbolic value” \citep[348]{Ladd2011} is crucial to a more holistic understanding of the variety. Thus, the actual \isi{phonetic realisation} of phonemic categories, and the abstract phonemic categories themselves both require investigation. The study of variation in rhythm presents itself as an ideal domain for combining a phonetic analysis with a \isi{phonological} one. The combined approach advocated here and highlighted in \citet{Ladd2011} assumes an understanding of the relationship between \isi{phonetics} and pho\-nol\-o\-gy as being two related facets of the same broad area of study. Rhythm may be one of those domains where it is useful to keep in mind this constant interplay of \isi{phonetics} and \isi{phonology}. 

The study of rhythm has seen a good deal of progress especially concerning the relationship between the occurrence of specific and measurable linguistic elements in context on the one hand, and the more abstract global characterisation that such linguistic elements might come to symbolise in listener perception on the other. Studies in the area of linguistic rhythm have investigated the connection between the \isi{phonetic realisation} of duration and timing, for example, alongside the broader \isi{phonological} classification of languages into “stress-timed” or “syllable-timed” languages, as originally proposed by Abercrombie in 1967.

Rhythm has been described recently by \citet{NokesHay2012} as “the patterning of prominent elements in spoken language, as perceived by the listener” (2012: 1). Besides providing a succinct description of the essence of rhythm in language, this definition also focuses on the notion that understanding rhythm is as much about understanding listeners' perceptions of the patterns of prominent and non-prominent elements, as it is about these elements themselves. 

Traditionally, definitions of different rhythm patterns across languages are credited to \citet{Pike1945} and \citet[96]{Abercrombie1967} who first presented the notion that languages could be typologically distinguished on the basis of their rhythm patterns. Since then, this view has gone full circle from being gradually debunked, to being more recently partly restored in modified form. The original views expressed by Pike and by Abercrombie resulted in the division of languages into “syllable-timed” or “stress-timed” according to whether all syllables, stressed or \isi{unstressed}, are produced with more or less even timing (syllable-timed) or whether timing is organised primarily around stressed syllables, with any intervening syllables being modified through reduction or weakening as compensation (stress-timed). 
\citet[97]{Abercrombie1967} also described rhythm in terms which suggest an observable activity complete with corresponding physiological correlates as “Speech rhythm is essentially a muscular rhythm”. Although this suggests that rhythm is essentially something that a \isi{speaker} produces, Abercrombie also goes on to give a surprisingly prescient suggestion that the notion of rhythm might be better typified if viewed in terms of a combined understanding between the \isi{speaker} and listener “empathetically” in tune with one another, where, if the \isi{speaker}/listener pair does not share the same mother tongue, “the sounds will not be recognized as accurate clues to the movements that produce them” \citep[97]{Abercrombie1967}. 

This hint of a linguistic element not being exclusively governed by a \isi{speaker}'s output is echoed years later by Roach, who observes that “the distinction between stress-timed and syllable-timed languages may rest entirely on perceptual skills acquired through training” \citep[73]{Roach1982}. The underlying belief, up to the 1990s, remained that perhaps rhythm was best studied within the domain of perception. Nokes and Hay in fact quote \citet{Beckman1992} who refers to the attempts to capture rhythm patterns as “one of the most persistent metaphors in the history of our struggle to understand speech rhythms” (2012: 3). The word ‘metaphor’ might give an indication as to why some linguists have preferred to treat rhythm as a perceptual phenomenon, rather than as an objectively measurable one in temporal terms. \citet{CouperKuhlen1986}, for example, takes this route while noting that “it is a natural human tendency to impose structure on perceptual stimuli” (1986: 52).

Nevertheless, Roach also hints at another route to understanding rhythm better when he suggests that “there is no language which is totally syllable-timed or totally stress-timed” (1982: 79). This latter perspective involving a continuum, rather than mutually exclusive categorisation, also encouraged subsequent research into the domain of phonetic, as well as \isi{phonological}, interpretations of rhythm, where discrete events such as pitch change, or the durational features of different segments, for example, could be measured and correlated with the perceptions of rhythm being more or less syllable- or stress-timed.

The assumption is then that identifying how prominent elements are ordered in speech \citep{Nesporetal2011} will yield information about the rhythm as it is perceived. This at last, allows at once for both a broader, and also a more refined understanding of rhythm. Rhythm is accounted for at its most generic as patterned sequences of prominent and non-prominent elements, with prominence here not necessarily being defined any further. Alternatively, we can try to identify some or all of those elements considered to generate a perception of prominence, and isolate them to study their behaviour further. \citet{NokesHay2012} did just that in their real-time study of the duration of segments in New Zealand English. As the authors describe it, New Zealand English is understood to be more syllable-timed than other varieties of English, and further, this current observation is seen as a shift from earlier rhythm patterns, observed to have been much more stress-timed. 

A series of studies now widely regarded as pivotal in trying to capture the acoustic correlates of rhythm manifested in durational characteristics are reported on in \citet{GrabeLow2002} and \citet{Lowetal2000}. The analyses in these papers are based on a formula developed to calculate the \isi{durational variability} of successive pairs of \isi{phonological} units. In these studies, in order to account for differing speech rates across individual speakers, a version of the Pairwise Variability Index (PVI) referred to as the normalised Pairwise Variability Index (nPVI) was used when measurements of vocalic and intervocalic intervals were carried out. nPVI analyses of a number of languages including both those identified as syllable- or stress-timed and those hitherto unclassified were carried out resulting in the pegging of these languages to different points on the continuum of stress- and syllable-timed languages. The emphasis here is on durational features, in response to the notion that the perception of rhythm can be correlated to a series of measurable events. In this case, the measurable events are successive pairs of intervals either vocalic – and therefore syllabic – or intervocalic, which, while not syllabic, may still affect perceptions of duration. If successive pairs of \isi{vowel duration} measurements vary considerably, then the resulting index will be higher than if \isi{vowel duration} is more uniform. A language variety like SSBE, for example, with its notable tendency to having weak or reduced vowels in \isi{unstressed} positions, contrasted with full vowels, long vowels or diphthongs in stressed syllables, could be expected to have a high nPVI index of variability. Conversely, a language such as \ili{Maltese}, which is normally said to be a language which does not tend to weaken or reduce vowels in \isi{unstressed} positions \citep{BorgAzzopardi-Alexander1997, Azzopardi1981} might have a lower nPVI index, also indicating that the variability in duration across successive vowels is not as high as it might be in SSBE. 

Other contemporary studies measuring different aspects of duration and
timing have produced similar results. 
\citet[265]{Ramusetal1999} measured \isi{vowel} and \isi{consonant} intervals, based
on the premise that “the measurements suggest that intuitive rhythm
types reflect specific \isi{phonological} properties, which in turn are
signaled by the acoustic/phonetic properties of
speech”. \citet{Dellwo2006} presented a method called VarcoΔC
to account for between-language fluctuations in speech tempo, due, in
part, to the different \isi{syllable structure} and phonotactic patterns
typical across languages. The measures and acoustic correlates
introduced by \citet{Ramusetal1999} or by \citet{Dellwo2006} aimed to
capture ways in which durational features might have a bearing on the
perception of rhythm patterns. The formula for a Pairwise Variability
Index, normalised to account for differences in speech rate across
speakers, the nPVI described above and adopted in 
\citet{GrabeLow2002}, and \citet{Lowetal2000}, also gave the
added dimension of capturing localised variability between pairs of
vocalic or intervocalic intervals. Durational characteristics of
segments are often considered a strong indicator of some form of
prominence, and the ordering of such prominent elements in relation to
non-prominent ones may lead to a perception of different rhythm as \citet[4]{NokesHay2012} note: “Other factors held equal, a longer
\isi{vowel length} will give rise to a percept of \isi{syllable} stress, and thus
rhythmic prominence, in English”.

\subsection{Durational features in Maltese English}
The relevance of taking note of durational factors as a ‘marked’ characteristic of \ili{MaltE} has often been foregrounded in the literature, as well as anecdotally, and here we return to the idea that essentially, our mental image of what rhythm captures, can be described as the ordering of prominent and non-prominent elements in the flow of a person's speech. In the case of \ili{MaltE}, the issue of the duration of segments may be seen as one type of realisation of prominence, though clearly not the only one. But certainly, it can be considered a good angle from which to begin examining the concept of rhythm in this variety of English. It is of course quite likely that prominence is variously realised by a range of elements and that these together combine to create certain effects in speech. In other words, the study of rhythm in a given variety may well only begin to come together once different phonetic/\isi{phonological} features have been analysed, and then eventually examined in relation to each other. 

Although research on \ili{MaltE} to date has not often focused overtly on rhythm, there are repeated, even if only oblique references to features which have durational characteristics embedded in them. Descriptions relating to the \isi{phonemic inventory} of \ili{MaltE} are relevant to this research (for example, \citep{Vella1995, Debrincat1999, Bonnici2010}. \citet[74]{Vella1995} concludes that: “The M[alt]E vowels differ from their R[eceived ]P[ronunciation] equivalents in terms of their quality since they tend to approximate to the quality of corresponding vowels in the Maltese system.”. \citet{Azzopardi1981} presents a comprehensive description of the \isi{vowel} inventory of \ili{Maltese}. Amongst other conclusions, she notes patterns of \isi{vowel duration} that may have a bearing on similar patterns in \ili{MaltE}. Although the issue of possible transfer of \ili{Maltese} as L1 onto \ili{MaltE} is not considered further here it is still worth bearing in mind Azzopardi's conclusion that in \ili{Maltese}, “Vowels in \isi{unstressed} syllables are as long and sometimes longer than vowels in stressed syllables” \citep[120]{Azzopardi1981}. 

Particular attention is given to \isi{schwa}, both in its own right as a \isi{vowel} not readily found in \ili{MaltE}, but also, with regards to its pivotal role in the rhythm patterns of SSBE and other major and widely codified varieties (see e.g. \citealt{Deterding2001}). \citet{Giegerich1992} suggests that the \isi{vowel} \isi{schwa} does not constitute part of the \isi{phonemic inventory} of English (variety unspecified), as it is not in contrast with any other \isi{vowel}, but rather, is a popular option for reduction in weak-stressed syllables. \citet[102]{Roach2009} also comments that “ə is not a \isi{phoneme} of English, but is an \isi{allophone} of several different \isi{vowel} phonemes when those phonemes occur in an \isi{unstressed} \isi{syllable}”. Schwa is also not part of the \isi{phonemic inventory} of \ili{Maltese} \citep{Azzopardi1981,BorgAzzopardi-Alexander1997}. \citet[90]{Calleja1987} notes that her \ili{MaltE} speakers “make minimal use of \isi{vowel reduction} and of weak forms”.

Not enough research has as yet been carried out on \isi{natural speech} data in \ili{Maltese} for it to be possible to assert that \isi{schwa} is never present in the language. This is in fact even more so for \ili{MaltE}. However, given its potentially questionable status as a \isi{phoneme} both in English as an idealised or prototypical unspecified variety, and more definitely, in \ili{Maltese}, it may be expected that spoken \ili{MaltE} is likely to show a preference for full vowels and less evidence of \isi{schwa}. As \citet[75]{Vella1995} notes: “The fact that /ə/ is rarely realised in M[alt]E) can therefore be hypothesized to be an important factor in the different rhythmic quality of M[alt]E as compared to that of R[eceived]P[ronunciation]”. \citet[70]{Debrincat1999} further describes how 48.5\% of her samples of \ili{MaltE} speech did not contain evidence of \isi{schwa}, which she took as “a clear indication of the fact that [the relative infrequency of] /ə/ is probably a contributing factor to the \isi{accent} of M[alt]E speakers”. 

There is a healthy body of previous research both on \ili{MaltE} and on other varieties of English that encourages a closer look at aspects of the durational characteristics of \ili{MaltE} which may combine to generate a perception of variation in the rhythmic characteristics of this variety. \sectref{sec:key:grech3} below describes the study carried out. Data from six speakers of \ili{MaltE} were analysed. An earlier perception study \citep{Grech2015} had served to locate the six speakers on a continuum ranging from highly identifiable as Maltese people speaking in English, through to not at all identifiably Maltese. 

\section{Methodology}
\label{sec:key:grech3}
\subsection{Speaker data}
Both \citet{Vella1995}, and \citet{Bonnici2010} point towards a distribution of phonetic variation as a function of specific registers or contexts and this could only be adequately analysed in more \isi{natural speech}. At the same time, a durational analysis of vowels across different speakers using the formula described in \sectref{sec:key:grech2.1} above requires directly comparable data. It was considered useful, therefore, to record speakers performing a series of tasks ranging from reading scripted text aloud (these data were labelled as {\textquotedbl}TextAloud{\textquotedbl} in the study), to speaking more spontaneously. Only the data from the scripted text is considered for the nPVI analysis here given the requirement of speech involving directly comparable data which would allow comparison of the realisation of aspects of duration by different \ili{MaltE} speakers. Variability in the reduction or non-reduction of full vowels to \isi{schwa} nevertheless also draws on and is informed by the analysis of the data involving samples of more \isi{spontaneous speech}. It has been noted that the context and register of \isi{natural speech} in \ili{MaltE} may well trigger slightly different speech styles, which may in turn affect aspects of duration and rhythm \cite{Vella1995}. Thus while the study of both inter-and intra-\isi{speaker} variability in \isi{vowel} durations is necessarily restricted to directly comparable scripted texts, the study of \isi{schwa} adds another dimension to the question of \isi{vowel duration} in \ili{MaltE} across different registers. The directly comparable scripted text (TextAloud) gave participants the opportunity to do a careful reading, and may also have triggered an echo of drilled \isi{pronunciation} practice from earlier schooldays. On the other hand, the more \isi{spontaneous speech} data elicited as participants were focused on a range of tasks was expected to yield more naturalistic – and therefore, presumably, less carefully monitored – speech. 

Six speakers, three male and three female, were recorded in settings familiar to them, using a Tascam DR-100DKII 24bit palm-held digital recorder. The speakers were identified as Maltese, having been brought up and schooled in \isi{Malta}, and were aged between 38 and 65 years old. One of the speakers, Sp6, had the same background and linguistic profile as the others, but had also lived in England for 4 years. It was expected that she would present some features more closely associated with the SSBE variety, having been directly exposed to this while in England, but it was considered important to include her contribution, in order to evaluate listener responses, as well as corresponding nPVI indices. In particular, greater variability across \isi{vowel} durations was expected for this \isi{speaker}. 

\subsection{Data collection and analysis}
The same theme, \isi{subject} matter, and therefore lexis, were retained across all speaking tasks, and centred around an Information Gap type of activity commonly used in communicative language teaching classes. Information Gap speaking tasks are typically devised in order to simulate the need to communicate, but at the same time, they also serve to distract participants (or learners, in a class) from worrying about being observed. The HCRC Map Task \citep{Andersonetal1991} is one such activity which was devised specifically for this purpose. The tasks tend to be engaging so that participants become more focused on successfully managing and completing the task at hand, rather than worrying about the fact that they are being recorded (or observed in a class). 

The key Information Gap activity around which all other tasks were centred here took the shape of the familiar childhood game ‘Spot the Difference’, with the information gap generated by a task where two speakers worked as a pair. Each \isi{speaker} was given a different version of a picture and instructed to identify six differences between the two pictures. The other related tasks involved using the same lexis provided by the activity to describe each picture in full, to frame in sentences, and finally, to read out loud in a descriptive story format. The latter task was coded as ‘TextAloud' in the analysis, and was used to carry out an nPVI analysis. All the other data were coded according to their task format as ‘Difference’ for the Spot the Difference activity, ‘Sentences’, in which speakers were recorded saying sentences using the same target words generated in the Spot the Difference activity, and finally ‘Description’, where speakers were asked to simply describe the picture in front of them. Across the text types, all vowels including instances of ‘\isi{schwa}’ where this could be expected in a weak stressed position were measured and analysed.  

The nPVI analysis was based on the formula established originally in \citet{GrabeLow2002}. The present study also incorporated \citet{NokesHay2012}'s modification to measure individual segments rather than vocalic or intervocalic intervals. In the current study, \isi{vowel duration} was used to capture the aspect of timing in rhythm. Therefore the nPVI formula was applied to measure the duration of each \isi{vowel}, together with the difference in duration between each successive \isi{vowel} pair. The final index of \isi{durational variability} across all vowels was then calculated from an average of all the differences between the successive \isi{vowel} durations in each \isi{speaker}'s TextAloud data. A high index indicates more variability across pairs of vowels, while a low index indicates less variability. TextAloud transcriptions for the six speakers were extracted, tabulated in Excel and sorted into \isi{vowel} segments as shown below in~\tabref{tab:key:grech1}. The table illustrates an example of the itemisation of each word recorded, as in this case, Speaker 2 read the scripted text out loud. \tabref{tab:key:grech1} shows the \isi{vowel} segment of each word (or segments if the word is multisyllabic, as in \textit{cartoon}) together with its duration measured in milliseconds. The final column presents a normalised PVI, computed as the absolute value of the difference in duration between each pair of vowels, divided by the mean duration of each pair. 

%\begin{table}
%\caption{\label{tab:key:grech1}Sample, extract from Sp(eaker) 2 vowel segment analysis using nPVI}
%\includegraphics[width=.75\textwidth]{figures/a8GrechVella-Table1.png}
%\todo[inline]{redo this as a proper table }
%\end{table}

\begin{table}
\caption{\label{tab:key:grech1} Sample, extract from Sp(eaker) 2 vowel segment analysis using nPVI}
%\begin{tabularx}{\textwidth}{XllSS}
\begin{tabularx}{\textwidth}{llcSS}
\lsptoprule
%\textbf{Speaker/\newline[1em] Location} & \textbf{Word} & \textbf{Segment} & \textbf{Segment Duration (ms)} & \textbf{nPVI\newline[1em] (normalised)}\\
\textbf{Speaker/Location} & \textbf{Word} & \textbf{Segment} & \textbf{Segment\newline Duration\newline (ms)} & \textbf{nPVI\newline (normalised)}\\
\midrule 
Sp2\_TextAloudpvi\_textgrid & This & i & 59 & \\
Sp2\_TextAloudpvi\_textgrid &  is  & i & 45 & 0.27 \\
Sp2\_TextAloudpvi\_textgrid &  a   & a & 58 & 0.25 \\
Sp2\_TextAloudpvi\_textgrid & cartoon & a & 49 & 0.17\\
Sp2\_TextAloudpvi\_textgrid & cartoon & oo  & 157 & 1.05\\
Sp2\_TextAloudpvi\_textgrid & of & 0 & 55 & 0.96 \\ 
\lspbottomrule
\end{tabularx}
\end{table}


The final index (shown in~\tabref{tab:key:grech2}) is then calculated as the average of all the differences measured for each \isi{speaker}, resulting finally, in an index for each of the six speakers. This entire calculation is referred to as nPVI. Note here that Grabe and Low's vocalic intervals are replaced by individual \isi{vowel} segments. In the original \citet{GrabeLow2002} study, a vocalic interval is measured from the onset of the first \isi{vowel} to the offset of the last one, thus in \textit{the arched handlebars}, /ɪ/ or /ə/ in \textit{the} together with the following /ɑ:/ in \textit{arched} would be measured as one interval together. Since we are interested in \isi{vowel} durations as a possible indicator of rhythm, we have followed \citet{NokesHay2012}, in measuring vowels as segments, rather than as vocalic intervals. The results therefore describe the durations of vowels in the six different \ili{MaltE} speakers, whilst also giving an indication of any variability in \isi{vowel length} that may or may not be immediately evident. 


\section{Results: Variability in vowel segments in Maltese English}
\label{sec:key:grech4}
The results of the nPVI analysis measuring variation in the duration
of successive \isi{vowel} segments are given in~\tabref{tab:key:grech2}. The
results indicate a high degree of variability in \isi{vowel duration}
patterns in Sp6, expressed as the highest index, while Sp1, Sp2 and
Sp3 have a comparatively much lower index, indicating much less
variability in duration across successive pairs of vowels.


\begin{table}
\caption{\label{tab:key:grech2}Normalised Pairwise Variability Index (nPVI) for 6 MaltE speakers ranked in order of increasing nPVI value}
\begin{tabularx}{.6\textwidth}{Xr}
\lsptoprule
 \textbf{Speaker} & \textbf{nPVI} \\
 \midrule 
 Sp3 – male & 49.5\\
 Sp1 – male & 55.1\\
 Sp2 – female & 56.8\\
 Sp4 – male & 57.9\\
 Sp5 – female & 69.7\\
 Sp6 – female & 81.1\\
\lspbottomrule
\end{tabularx}
\end{table} 


The index range across the six speakers is particularly remarkable
considering they can all, to different extents, be considered to be
speakers of the same variety of English (although see comment on Sp6,
below). The resulting indices give a clear picture of the extent to
which \isi{vowel duration} patterns vary across the six speakers. There is a
particularly large difference between Sp3 with an index of 49.5
compared with Sp6, with an index of 81.1. For comparison,
\citet{NokesHay2012} obtained roughly the same range of index, from
51.5 to 82.5 \citep[11]{NokesHay2012}, with the higher indices
corresponding to earlier recordings, and the lower indices
corresponding to more recent recordings over 120 years, during which
time, New Zealand English was coming to be perceived as more
syllable-timed\footnote{\citet{GrabeLow2002} also obtained similar
  ranges of indices, this time in a synchronic study of normalised PVI
  of vocalic intervals in 18 different languages. The languages
  examined included English, \ili{German} and Dutch, perceived as
  stress-timed languages, as well as \ili{Spanish}, considered
  syllable-timed, and Polish, considered rhythmically mixed \citep{GrabeLow2002}.}. Although it is to be noted that nPVI results across
different participant cohorts producing different texts cannot be
directly compared, the pattern of results is still nevertheless
informative. This present study, together with the first comprehensive
study in \citet{GrabeLow2002}, followed later by Nokes and Hay's
(2012) reinterpretation all yield a picture of a clear continuum of
variation in the realisation of \isi{vowel} durations. In all cases, the
higher the index, the closer the association with the traditional
perception of “stress-timed” rhythm. Conversely, a lower index is
associated with a perception of “syllable-timed” rhythm. On
\citet{GrabeLow2002}'s scale, for example, \ili{Spanish}, an example of a
  purportedly syllable-timed language, obtained an index of 29.7,
  compared with a much higher index of 57.2 for English, an example of
  a stress-timed language.  In the present study, variation in the
  extent to which \isi{vowel} durations differ within speakers is evident in
  the six speakers chosen as examples of different points on the
  continuum of variation in \ili{MaltE} (see \figref{fig:key:grech1}). In
  \figref{fig:key:grech2}, Speakers 1 to 6 have been ordered according
  to the perception ratings they received when judged in the listening
  task by the 28 native \ili{MaltE} speaker-listeners in the earlier study
  \citep{Grech2015}. Accordingly, Sp1 was perceived as highly identifiably
  Maltese by 89\% of native \ili{MaltE} listeners while Sp6 was perceived as
  identifiably Maltese by only 4\% of the participants, and thus was
  considered the least identifiable amongst the \ili{MaltE} speakers
  studied. Notably, Sp6 is the \isi{speaker} marked as the potential
  outlier, having lived for some time in England, and for whom
  features of \isi{vowel duration} were expected to pattern differently as
  compared to those of the rest of the participant cohort. Sp2 and Sp3
  were also highly identifiable as Maltese, while Sp4 and Sp5 were
  judged to be moderately identifiable.

\begin{figure}
\caption{\label{fig:key:grech2}Vowel duration patterns and identifiability judgments for 6 MaltE speakers}
%\includegraphics[width=.8\textwidth]{figures/a8GrechVella-fig2.png}

\begin{tikzpicture}
  \begin{axis}[width  = \textwidth,
	height = .3\textheight,
        major x tick style = transparent,
	axis lines*=left, 
        ybar,
        bar width=17pt,
        nodes near coords, 
	xtick=data,
	x tick label style={},  
	ymin=0,	
        ylabel = {\%},
        symbolic x coords={Sp1,Sp2,Sp3,Sp4,Sp5,Sp6},
        legend style={at={(0.01,0.01)},anchor=south west} 
    ]
	\addplot[ybar,lsRichGreen!80!black,fill=lsRichGreen] plot coordinates {
	    (Sp1,55.1) 
	      (Sp2,56.8) 
	      (Sp3,49.5) 
	      (Sp4,57.9) 
	      (Sp5,69.7) 
	      (Sp6,81.1)	      
	}; 
	\addplot[ybar,lsDarkBlue!80!black,fill=lsDarkBlue] plot coordinates {
	    (Sp1,89) 
	      (Sp2,82) 
	      (Sp3,71) 
	      (Sp4,57) 
	      (Sp5,54) 
	      (Sp6,4)	      
	}; 
        \legend{PVI,identifiability Ratings}
    \end{axis} 
  \end{tikzpicture} 
\end{figure}

Confirming the visible correspondence evident in \figref{fig:key:grech2}, Pearson's correlation indicates a significant negative correlation -0.883, (p value = 0.02) for identifiability and nPVI. Those speakers rated as highly identifiable have a correspondingly relatively low \isi{variability index}. Sp6, rated as not identifiably Maltese, had the highest \isi{variability index}, whilst the two moderately identifiable \ili{MaltE} speakers also presented a relatively low \isi{variability index}, though not as low as that for the most identifiable speakers. 

Further investigation of the \isi{vowel} durational patterns of each of the six \ili{MaltE} speakers’ extent of the use of the \isi{schwa} \isi{vowel} yields a correspondingly predictable pattern. \figref{fig:key:grech3} presents the proportion of full vowels preferred over \isi{schwa}, across all instances where \isi{schwa} was possible, for each \isi{speaker}.

\begin{figure}
  \caption{\label{fig:key:grech3}Percentage (\%) of full vowels in words where schwa could be expected in 6 MaltE speakers}
  %\includegraphics[width=.8\textwidth]{figures/a8GrechVella-fig3.png}
  \barplot{}{\%}{Sp1, Sp2, Sp3, Sp4, Sp5, Sp6}{  
	      (Sp1,63) 
	      (Sp2,42) 
	      (Sp3,62) 
	      (Sp4,33) 
	      (Sp5,21) 
	      (Sp6,8)
	      }
\end{figure}


As the figure illustrates, the most highly identifiable \ili{MaltE} speakers show a strong preference for using full vowels where \isi{schwa} could have been used. Conversely, Sp6, rated the least identifiably \ili{MaltE} \isi{speaker}, had very few instances of full vowels, showing, instead, a preference for \isi{schwa}. Pearson's correlation indicated a significant correlation between highly identifiable \ili{MaltE} and a preference for full vowels over weakened ones. Analysis returned a positive correlation 0.857157  (p value = 0.03) for highly identifiable \ili{MaltE} and preference for full vowels. These results provide further support to the idea that the \isi{variability index} yielded by the nPVI analysis, which is itself designed to test variability in \isi{vowel duration} patterns, may be a useful way to approach the matter of trying to identify features and characteristics more likely to trigger the perception of a \ili{MaltE} \isi{accent} in a \isi{speaker}. 


Further analysis of the preference for full vowels over \isi{schwa} across different speech styles (spontaneous and more \isi{natural speech} \textit{vs}. scripted and more careful speech) also yields a potential indication of endonormative variation in \ili{MaltE} (see more on this below). \figref{fig:key:grech4} illustrates the proportion of vowels realised as full vowels rather than as \isi{schwa} in the scripted TextAloud, compared with those in \isi{spontaneous speech}, by \isi{speaker}.


\begin{figure} %4
  \caption{\label{fig:key:grech4}Percentage (\%) of full vowels in words where schwa could be expected in two different speech styles}
  %\includegraphics[width=.8\textwidth]{figures/a8GrechVella-fig4.png}
  
\begin{tikzpicture}
  \begin{axis}[width  = \textwidth,
	height = .3\textheight,
        ybar,        
	axis lines*=left, 
        nodes near coords, 
	xtick=data,
	x tick label style={},  
	ymin=0,	
        ylabel = {\%},
        symbolic x coords={Sp1,Sp2,Sp3,Sp4,Sp5,Sp6},
        legend style={at={(0.01,0.01)},anchor=south west} 
    ]
	\addplot[ybar,lsRichGreen!80!black,fill=lsRichGreen] plot coordinates {
	    (Sp1,34) 
	      (Sp2,19) 
	      (Sp3,32) 
	      (Sp4,23) 
	      (Sp5,14) 
	      (Sp6,7)	      
	}; 
	\addplot[ybar,lsDarkBlue!80!black,fill=lsDarkBlue] plot coordinates {
	    (Sp1,29) 
	      (Sp2,24) 
	      (Sp3,30) 
	      (Sp4,10) 
	      (Sp5,6) 
	      (Sp6,2)	      
	}; 
        \legend{Spontaneous, Text Aloud}
    \end{axis} 
  \end{tikzpicture}
\end{figure}

The data shown in this figure confirm that the first three speakers, also rated most identifiably \ili{MaltE}, have a preference for full vowels over \isi{schwa}, although the proportion of full vowels is sometimes higher in the \isi{spontaneous speech} styles. The consistent distinction between the greater preference for full vowels over reduced ones in \isi{spontaneous speech} could be seen as an indicator of trends of change in the variety of \ili{MaltE}. While this needs further investigation, it is reasonable to suggest that scripted text triggers learnt patterns typical of those encouraged in a school environment, where undoubtedly standardised versions of SSBE may have been the ones modelled, or at least, aspired to. Conversely, \isi{spontaneous speech} might be seen to capture speech patterns which undergo less self-monitoring, and therefore, potentially, are a more robust indicator of how this \isi{dialect} is likely to change over time. 

Interestingly, the same pattern is also observed in the remaining speakers, who are all rated as less identifiably \ili{MaltE}. Again, the least identifiably \ili{MaltE} \isi{speaker}, Sp6 shows a clear preference for \isi{schwa} over full vowels, while the moderately identifiable \ili{MaltE} speakers, Sp4 and Sp5, show moderate preference for full vowels, but much less so than Sp1, Sp2 and Sp3. However, all 3 less identifiably \ili{MaltE} speakers still show a greater preference for full vowels over reduced ones in spontaneous, compared with scripted speech. This is interpreted here as an indicator of \ili{MaltE} starting to shape its own norms, rather than looking to other more established dialects for doing this. 

\section{Conclusion}
There is considerably less variability in the duration of successive vowels as measured by the nPVI amongst speakers more readily identified as being Maltese based on their \ili{MaltE} \isi{accent}. A corresponding pattern of slightly greater variability in the duration of successive vowels, again as measured by the nPVI, is seen in those speakers still identified as being Maltese, but who are considered more moderately typical of a Maltese person speaking in English. Conversely, Sp6, the \isi{speaker} expected to have some features of SSBE, having lived in the UK for some time, and only considered minimally identifiable by 4\% of the 28 native \ili{MaltE} speaker-listeners, showed a marked preference for \isi{vowel reduction} and \isi{vowel} weakening and consequently a higher nPVI reflecting the highly variable nature of durations in the successive vowels for this \isi{speaker}. 

The combined effect of more or less variability in the duration of successive \isi{vowel} segments over longer stretches of speech may in turn lead to a perception of different rhythm patterns. This may be especially noticeable at the extreme ends of the index range, where one \isi{speaker} presents a high index of variability and another \isi{speaker} presents a much lower one. However it is also noticeable that the 3 most identifiably Maltese speakers cluster within the lower end of the index, while the moderately identifiable speakers display higher indices, but still not approaching the highest index obtained by the \isi{speaker} who is least identifiable as a \ili{MaltE} \isi{speaker}. On the one hand, therefore, the nPVI can be interpreted in relation to how the indices cluster around 3 main points, ranging from little variability to high variability. On the other hand, the nPVI may also serve to refine the broad categories to capture more subtle distinctions between one \isi{speaker} and the other, including among those who might be described as using a more-or-less “syllable-timed” as compared to a “stress-timed” rhythm. Therefore within these broad categorisations, it can be suggested that the nPVI could be used as a means to identify further variation. This interplay between broad categorisation and within-category variation may be a useful feature to capture in the exploration of emergent varieties of languages. 

A key observation which emerges from these results is that they can be seen to provide evidence of variation within the variety, suggesting a shift towards endonormative stabilisation. Native listeners can establish when somebody is or is not using \ili{MaltE}, but they can also distinguish variation within \ili{MaltE}. The high degree of negative correlation between different listener ratings for \ili{MaltE} identifiability, and indices of variability in the duration of successive vowels suggests that this feature is a strong indicator of \ili{MaltE} as a distinct variety, as well as of variation within \ili{MaltE}. Results show that a low index representing less variability in \isi{vowel duration} as measured by the nPVI correlates with a highly identifiable \ili{MaltE} \isi{speaker}, a midway index correlates with a moderately identifiable \ili{MaltE} \isi{speaker}, while a high index indicating a strong degree of variability is linked with a \isi{speaker} not readily identifiable as \ili{MaltE}. Predictably, the \isi{schwa} feature across these same speakers also yielded evidence of variation to echo the nPVI findings, in that the highly identifiable \ili{MaltE} speakers (Sp1, Sp2, Sp3) made significantly less use of \isi{schwa} across all speech styles, while the least identifiable \ili{MaltE} had more widespread use of \isi{schwa}. Further indications that variability in the use of \isi{schwa} and \isi{vowel duration} more generally may also be a function of different speech styles also emerge from the analysis. It is worth noting that this is not a case of categorical presence or absence. Rather, there is evidence of both intra-\isi{speaker} variation, as well as inter-\isi{speaker} variation. All speakers exhibited a degree of variability across \isi{vowel} durations, and all speakers also presented some instances of \isi{vowel reduction}, including use of a \isi{schwa} at times. 

This paper therefore presents evidence of a fair degree of variation within \ili{MaltE} with respect to \isi{vowel duration}, which in turn has a bearing on the perception of rhythm. Variation in \isi{vowel duration}, both in itself (preferred use of full vowels rather than \isi{schwa}), and in so far as variation in successive \isi{vowel} durations contributes to differences in rhythm, can also be seen to be a trigger in the perception of \ili{MaltE}. 

The findings from this study set the stage for further work on variation in \ili{MaltE} at the phonetic/\isi{phonological} levels, particularly in relation to those elements which may affect the duration of both vowels and consonants at the local level, and consequently rhythm more globally. Among the characteristics and features already under preliminary investigation in \citet{Grech2015}, rhoticity is noteworthy, also because greater use of a postvocalic `r' may trigger compensatory shortening in the preceding \isi{vowel}, while an absence of this feature may also in part account for differences in \isi{vowel} durations as compared to contexts where an `r' would not be expected. The features discussed here, and others where durational properties can be captured and analysed at the phonetic level, may combine to generate a perception of variability in rhythm in \ili{MaltE} at the \isi{phonological} level. This dual focus of analysis at both the phonetic and the \isi{phonological} levels of certain features may therefore be a useful approach to developing a more refined understanding of variation in this emerging variety of English.
 
 
  

\sloppy
\printbibliography[heading=subbibliography,notkeyword=this] 
\end{document}
