\documentclass[output=paper]{langsci/langscibook} 
\ChapterDOI{10.5281/zenodo.1181791}
\author{Christopher Lucas \affiliation{SOAS, University of London}\lastand
Michael Spagnol\affiliation{University of Malta}
}
\title{Conditions on /t/-insertion in Maltese numeral phrases: {A} reassessment}
%\shorttitlerunninghead{}
%\ChapterDOI{} %will be filled in at production
\abstract{There has, for a considerable period, been disagreement and confusion as to the conditions governing the appearance of the /t/ morpheme that sometimes intervenes between the numerals 2–10 and a following plural noun in Maltese, as in \textit{ħames skejjel} / \textit{ħamest\textsuperscript{} }\textit{iskejjel} ‘five schools’ (e.g. \citealt{Aquilina1965}: 118; \citealt{Borg1974}; \citealt{Cremona1938}: 204–205). In recent work (\citealt{LucasSpagnol2016}) we reported on a native-speaker production experiment designed to improve our understanding of this issue. The results of that experiment suggested that the key factor determining /t/-insertion was onset of the plural noun: CV-initial plurals virtually never permit /t/-insertion, whereas CC-initial and V-initial plurals at least sometimes do. Number of syllables also appeared to be a relevant factor, in that, e.g., monosyllabic CC-initial plurals were found to strongly favour /t/-insertion, disyllabic CC-initials less so, and polysyllabic CC-initials not at all.

The present work builds on this earlier research, arguing that a more accurate and more general statement of the conditions on /t/-insertion is one that makes reference primarily to morphological pattern, rather than to onset and number of syllables. This conclusion stems from a new production experiment focusing specifically on /t/-insertion with CC-initial disyllabic plurals. The experiment tested combinations of numerals with a number of both "sound" (suffixing, non-stem-altering) plurals and "broken" (non-suffixing, stem-altering) plurals. The latter fell into one of three patterns: CCVVCV(C), CCVjjVC and CCVCVC. The basic prediction was that the broken plurals would, in general, be much more favourable to /t/-insertion than the sound plurals. This prediction was borne out (broken plural mean insertion rate: 32\%; sound plural mean insertion rate: 5\%). Additionally, we predicted that broken plurals of the CCVCVC pattern, such as \textit{gwerer} ‘wars’, in which two consonants occupy the initial root-consonant slot in the basic, highly /t/-resistant, CVCVC pattern (cf. \citealt{Mifsud1994}), would be less favourable to /t/-insertion than the other CC-initial broken plural patterns tested. This too was borne out (mean insertion rates: CCVCVC 23\%; CCVjjVC 37\%; CCVVCVC 55\%). Taken together, these two findings show that morphological pattern should be taken as the key determinant of /t/-insertion, with onset and number of syllables contributing only secondarily. 
}
\maketitle
\begin{document}
% \todo{abstract very long}

\section{Introduction}
\subsection{Overview}

In a recent article (\citealt{LucasSpagnol2016}), the present authors made a first attempt at a definitive statement of the conditions governing so-called /t/-insertion in Maltese \isi{numeral} phrases. We provided experimental evidence that the incidence of /t/-insertion correlates strongly with \isi{phonological} properties of the nominal head of a \isi{numeral} \isi{phrase}. The present article shows, on the basis of new experimental data, that, notwithstanding our earlier findings, /t/-insertion is better seen as a morphologically-governed phenomenon, and that the apparent role of \isi{phonology} is at least partly epiphenomenal.

\subsection{What is /t/-insertion?}

Maltese cardinal numerals from ‘two’ to ‘ten’ have two main forms: a \textstyleLangSciCategory{dependent} form, used when the \isi{numeral} modifies a following \isi{plural noun}, and an \textstyleLangSciCategory{independent} form for non-modifier uses. While there is only one version of the independent form, the dependent form comes in two versions: with or without /t/.\footnote{Maltese orthography treats this /t/ as a \isi{suffix} on the \isi{numeral}, but phonologically it behaves as a prefix on the following \isi{plural noun} (see \citealt{LucasSpagnol2016} for details). We follow Maltese orthography here, but will refer to this \isi{morpheme} simply as /t/, not as a \isi{suffix}.}  This is illustrated in \tabref{tab:lucas:1} and the example in ‎\REF{ex:lucas:1}, in which it can also be seen that /t/-insertion before a \isi{plural noun} beginning with a \isi{consonant cluster} triggers insertion of a prothetic /i/.

\ea\label{ex:lucas:1}
\ea
{ \textit{ħames      skejjel}}\\
\ex 
{ \textit{ħamest      iskejjel}}\\
 
\glt {   ‘five        schools’}
\z
\z

 As can be seen from ‎\REF{ex:lucas:1}, \textit{skejjel} ‘schools’ is an example of a Maltese \isi{plural} for which /t/-insertion with a preceding \isi{numeral} is optional, at least for some speakers. As we will see, there are dozens of Maltese plurals with this property, though the previous literature on /t/-insertion sometimes gives the impression that this is a non-optional process (e.g. \citealt{Cremona1938}: 204–205; \citealt{Sutcliffe1936}: 188–189). In fact, this literature (e.g. \citealt{Borg1974}, \citealt{Cremona1938}; \citealt{Fabri1994}) is characterized by a remarkable lack of consensus on the details of what triggers /t/-insertion. To take the most striking example: \citet[118]{Aquilina1965} suggests that /t/-insertion is licit with any \isi{plural} with a vocalic onset, while \citet[294]{Borg1974} claims that plurals with vocalic onsets are the precise context in which /t/-insertion does not occur. Despite this lack of consensus, there is nevertheless general agreement that the most important factors governing /t/-insertion are the onset and number of syllables of the \isi{plural noun} (for further details on the previous literature on this topic, see \citealt{LucasSpagnol2016}).

\begin{table}
\begin{tabularx}{\textwidth}{QQQQ} 
\lsptoprule
& \textbf{Independent form} & \textbf{Bare dependent form} & \textbf{Dependent form with /t/}\\
\midrule 
%‘two’ & tnejn & żew\.g / \.giex & żew\.gt / \.gixt\\
%‘three’ & tlieta & tliet & tlitt / tlett\\
%‘four’ & erbgħa & erba’ & erbat\\
%‘five’ & ħamsa & ħames & ħamest\\
%‘six’ & sitta & sitt & sitt\\
%‘seven’ & sebgħa & seba’ & sebat\\
%‘eight’ & tmienja & tmien & tmint\\
%‘nine’ & disgħa & disa’ & disat\\
%‘ten’ & għaxra & għaxar & għaxart\\
‘two’ & \textit{tnejn} & \textit{żew\.g / \.giex} & \textit{żew\.gt / \.gixt}\\
‘three’ & \textit{tlieta} & \textit{tliet} & \textit{tlitt / tlett}\\
‘four’ & \textit{erbgħa} & \textit{erba’} & \textit{erbat}\\
‘five’ & \textit{ħamsa} & \textit{ħames} & \textit{ħamest}\\
‘six’ & \textit{sitta} & \textit{sitt} & \textit{sitt}\\
‘seven’ & \textit{sebgħa} & \textit{seba’} & \textit{sebat}\\
‘eight’ & \textit{tmienja} & \textit{tmien} & \textit{tmint}\\
‘nine’ & \textit{disgħa} & \textit{disa’ }& \textit{disat}\\
‘ten’ & \textit{għaxra} & \textit{għaxar} & \textit{għaxart}\\
\lspbottomrule
\end{tabularx}
\caption{Independent, bare dependent and dependent /t/-form cardinal numerals 2–10 in Maltese.
}
\label{tab:lucas:1}
\end{table}

\subsection{Previous experiment}\label{sec:prevexp}
Our aim in \citet{LucasSpagnol2016} was to put the description of this construction on a firmer footing by testing these two factors experimentally with multiple native speakers. All earlier work on this topic had depended on authors’ personal intuitions or informal observations. Accordingly, we recruited 35 native speakers of Maltese for a production experiment. In this experiment the test items were pairings of a \isi{numeral} between ‘two’ and ‘ten’ (presented as a figure) and one of 56 singular nouns whose plurals fell into eight different categories: mono-, di- and \isi{polysyllabic} (3+) CC-initial words; mono-, di-, and \isi{polysyllabic} CV-initial words; and di- and \isi{polysyllabic} V-initial words (there being no \isi{monosyllabic} V-initial plurals in Maltese). The subjects’ task was then to produce what they saw, as the \isi{phrase} would naturally occur in context, i.e.~with the \isi{numeral} in the dependent form, the \isi{noun} in the \isi{plural}, and with /t/-insertion if considered appropriate.\footnote{The task was demonstrated by means of examples, not explained conceptually. Subjects were asked after the test (which included equal numbers of test and filler items) whether they had any idea what it was investigating. None realized that /t/-insertion was the topic of investigation.} For example, if our target were as in example \REF{ex:lucas:1} above, the test item would appear as in \REF{ex:lucas:2}, \textit{skola} being the singular ‘school’.

\ea\label{ex:lucas:2}
\itshape 5   skola\\
\z


The results of this experiment showed that, among our 56 test items, there was indeed a very strong main effect of both onset and number of syllables in the incidence of /t/-insertion.\footnote{The experiment also tested a third factor: choice of specific \isi{numeral} between 2 and 10. We found no main effect of this factor, though there was an interaction of all three factors. See \textsection\ref{sec:design} of the present article and \citet{LucasSpagnol2016} for further details.} This can be clearly seen from \figref{fig:lucas:1}.


\begin{figure}
% \todo[inline]{figure missing}
\caption{/t/-insertion rates (\%) by onset and number of syllables (adapted from \citealt{LucasSpagnol2016})}
\label{fig:lucas:1}


\begin{tikzpicture}
  \begin{axis}[width  = .8\textwidth,
	height = .3\textheight,
        major x tick style = transparent,
	axis lines*=left, 
        ybar,
        bar width=12pt,
        nodes near coords, 
	xtick=data,
	x tick label style={},  
	ymin=0,	
        ylabel = {\%},
        symbolic x coords={CC~onsets,CV~onsets,V~onsets},
        legend style={at={(1,1)},anchor=north east} 
    ]
	\addplot[ybar,lsRichGreen!80!black,fill=lsRichGreen] plot coordinates {
	      (CC~onsets,91) 
	      (CV~onsets,1) 
	      (V~onsets,0)  	      
	}; 
	\addplot[ybar,lsDarkBlue!80!black,fill=lsDarkBlue] plot coordinates {
	      (CC~onsets,53) 
	      (CV~onsets,0) 
	      (V~onsets,32)       
	}; 
	\addplot[ybar,lsLightBlue!80!black,fill=lsLightBlue] plot coordinates {
	      (CC~onsets,1) 
	      (CV~onsets,0) 
	      (V~onsets,11)       
	}; 
        \legend{Monosyllabic,Disyllabic,Polysyllabic}
    \end{axis} 
  \end{tikzpicture} 
% CC onsets 91 53 2
% CV onsets 1 0 0 
% V onsets 0 32 11
\end{figure}



 


In this dataset, /t/-insertion is essentially absent with plurals with CV onsets, no matter the number of syllables. With CC onsets the picture is entirely different, and number of syllables appears to be crucial: \isi{monosyllabic} CC-initial plurals tested trigger /t/-insertion approximately 90\% of the time, \isi{disyllabic} CC-initial plurals approximately 50\% of the time, and \isi{polysyllabic} CC-initial plurals essentially never. With V initials, matters are much less clear cut: neither disyllabics nor polysyllabics particularly favour /t/-insertion, but polysyllabics do so noticeably less than disyllabics, though without disallowing it altogether.

As we noted in the earlier article, however, there is likely more to these results than meets the eye. In the previous literature on this topic, \citet[297]{Borg1974} and \citet[91]{Ambros1998} both suggest that the distinction between \textsc{sound} and \textsc{broken} plurals plays an important role in the occurrence of /t/-insertion. This distinction, which Maltese inherits from \ili{Arabic}, concerns the morphological means by which \isi{plural} number is indicated on nouns and adjectives. Sound plurals are those in which \isi{plural} is indicated by suffixation of a \isi{plural} \isi{morpheme} to the singular form, with little or no alteration to the stem, as in \textit{kelma} ‘word’, \textit{kelm-iet} ‘word-\textsc{pl’}. Broken plurals, by contrast, are those in which \isi{plural} is indicated by means of an abstract \textsc{pattern} \isi{morpheme} – a vocalic melody that combines with the root consonants of the word in question, as in \textit{kelb} ‘dog’, \textit{klieb} ‘dog.\textsc{pl}’. \citet[297]{Borg1974} claims that “sound plurals do not take /t/”. 

It was not possible to include the sound vs. broken \isi{plural} distinction as another factor in the experiment reported on in \citet{LucasSpagnol2016}, as this would have necessitated an impractically large number of test items. We did, however, ensure that both sound and broken plurals were represented as test items in all the conditions where this was possible, so as to gain some preliminary insights as to the relevance of this factor. For example, a /t/-\isi{insertion rate} across all 35 subjects of 20\% with the sound \isi{plural} \textit{ajruplani} ‘aeroplanes’ (sg. \textit{ajruplan}), and 26\% with the sound \isi{plural} \textit{idejn} ‘hands’ (sg. \textit{id}), suggests that Borg’s claim cannot be completely correct, at least as far as vowel-initial plurals are concerned. Regarding CV-initial plurals, we have already seen that none of these were favourable to /t/-insertion, and this was true for both sound and broken plurals. The most interesting case was that of the CC-initial plurals. Recall that, with these, the incidence of /t/-insertion varied sharply according to the number of syllables, with /t/-insertion rates of around 90\% with monosyllabics, around 50\% with disyllabics, and essentially zero with polysyllabics. A crucial point to realise here, however, is that, among CC-initial plurals in Maltese, there are no \isi{polysyllabic} broken plurals and no \isi{monosyllabic} sound plurals. So the results for these two conditions cannot help us determine the relative importance to /t/-insertion of number of syllables and the sound vs. broken \isi{plural} distinction. With \isi{disyllabic} CC-initials, on the other hand, both sound and broken plurals are amply represented. \tabref{tab:lucas:2} shows the CC-initial \isi{disyllabic} plurals tested in \citet{LucasSpagnol2016} along with the frequency with which our test subjects inserted /t/ with these items.

\begin{table}
\begin{tabularx}{\textwidth}{XQr} 
\lsptoprule
& \textbf{Test items} & \textbf{/t/-insertion frequency (\%)}\\
\midrule 
\textbf{Broken plurals}
& \textit{ljieli}    ‘nights’ & 80\\
% \hhline{-~~}
& \textit{\.granet}    ‘days’ & 77\\
& \textit{bramel}    ‘buckets’ & 74\\
& \textit{kmamar}  ‘rooms’ & 71\\
& \textit{skejjel}    ‘schools’ & 56\\
\tablevspace
\textbf{Sound plurals}  
& \textit{platti}    ‘plates’ & 7\\
& \textit{stampi}    ‘pictures’ & 6\\ 
\lspbottomrule
\end{tabularx}
\caption{
/t/-insertion rates for CC-initial {disyllabic} plurals (adapted from \citealt{LucasSpagnol2016}).
}

\label{tab:lucas:2}
\end{table}


We see that /t/-insertion rates for the two sound plurals are close to zero, while for the broken plurals they are much higher. It therefore seems that Borg’s (1974: 297) claim (that /t/-insertion is incompatible with sound plurals), while not absolutely correct, is certainly on the right track. This and related issues were investigated in detail in the experiment reported on in the following sections.

\largerpage[-1]
\section{New experiment}
\subsection{Rationale and predictions}\label{sec:lucas:‎2.1}

The purpose of the follow-up experiment that we report on here was twofold: a) to investigate the relevance of the sound vs. broken \isi{plural} distinction to /t/-insertion rates; and b) more generally, to tease apart the relative importance to /t/-insertion of \isi{phonological} and morphological factors. We approached these problems by focusing in detail on just one of the seven conditions investigated in the first experiment: CC-initial \isi{disyllabic} plurals (both broken and sound). Restricting our investigations to these was the natural choice for a variety of reasons. Since we have already established that onset and number of syllables correlate strongly with patterns of /t/-insertion, it makes sense to hold these constant while examining whether other factors play a role. We have already seen that CV-initial plurals – both broken and sound – are very hostile to /t/-insertion no matter the number of syllables, so these are not useful for further investigation (but see the discussion below of the morphologically related \textit{gwerer}{}-type broken plurals), while the lack of \isi{monosyllabic} CC-initial sound plurals and \isi{polysyllabic} CC-initial broken plurals rules these out too. It would certainly be interesting to investigate further what combination of factors regulates /t/-insertion with V-initials, but since there is a relative paucity of V-initial broken plurals, we leave these aside for present purposes to focus on the \isi{disyllabic} CC-initials, of which there is an abundance, both broken and sound. 

Focusing on \isi{disyllabic} CC-initial plurals allows us to test a number of hypotheses. First, and most straightforwardly, that broken plurals of this type are more favourable to /t/-insertion than phonologically equivalent sound plurals. 

Second, it may be that not all broken \isi{plural} patterns are favourable to /t/-insertion. There are three Maltese broken \isi{plural} patterns that are CC-initial \isi{disyllabic}: CCVVCV(C), with a long medial \isi{vowel}, as in \textit{bramel} ‘buckets’ (sg. \textit{barmil}) and \textit{qsari} ‘flower pots’ (sg. \textit{qasrija}); CCVjjVC, with a medial \isi{geminate} \isi{glide}, as in \textit{knejjes} ‘churches’ (sg. \textit{knisja}); and, most significantly, CCVCVC, with two short vowels, as in \textit{gwerer} ‘wars’ (sg. \textit{gwerra}). This final pattern is significant because it features an \isi{initial consonant} cluster in a morphophonological slot that in other \ili{Arabic} and \ili{Semitic} varieties, and most of the Maltese lexicon, would usually host just a single root \isi{consonant}. The pattern CCVCVC is therefore one sub-type of the more general pattern (C)CVCVC, of which a more prototypical example than \textit{gwerer} is \textit{bozoz} ‘bulbs’ (sg. \textit{bozza}), with three straightforward candidates for the three root consonants that this pattern typically requires. As can be seen from \tabref{tab:lucas:2}, we tested four plurals of the CCVVCV(C) type in the first experiment, and recorded /t/-insertion rates from 71\% to 80\% with these, and one of the CCVjjVC type, for which the /t/-\isi{insertion rate} was 56\%. We did not test any of the CCVCVC (\textit{gwerer}) type. Among our CV-initial \isi{disyllabic} test items, however, we tested one \isi{plural} of the CVCVC type: \textit{bozoz}. In keeping with all the other CV-initial plurals in the first experiment, there were no instances of /t/-insertion with \textit{bozoz}. What this means is that CC-initial plurals of this pattern – like \textit{gwerer} – are therefore an ideal testing ground for the idea, universal in the previous literature, that all CC-initial broken plurals are favourable to /t/-insertion. Our hypothesis is that, in reality, onset, like number of syllables, is only of secondary importance to /t/-insertion: we predict that plurals of the \textit{gwerer}{}-type, despite being CC-initial, will, like their CV-initial counterparts of the \textit{bozoz}{}-type, be unfavourable to /t/-insertion, due to their membership of the same basic (C)CVCVC pattern.

\newpage 
Third, we saw from the results of the first experiment that /t/-insertion with CC-initial \isi{disyllabic} broken plurals is by no means obligatory (see \tabref{tab:lucas:2}). This raises the question of whether any factors in addition to the specific broken \isi{plural} pattern influence the frequency of /t/-insertion where this is optional. It seems plausible that various factors do indeed play a role. For example, we hypothesized in \citet{LucasSpagnol2016} that the string frequency in corpora \citep{Krug1998} of specific \isi{numeral}–\isi{noun} combinations would positively correlate with frequency of /t/-insertion for nouns where this is optional. This is something we are investigating in presently ongoing work and will not take further here. Another factor that could plausibly play a role here is the precise phonetic composition of the onset. As shown in ‎\REF{ex:lucas:1}, /t/-insertion also triggers insertion of a prothetic /i/ with non-V-initial plurals. Prothetic /i/-insertion elsewhere in Maltese grammar is sensitive to the precise composition of \isi{initial consonant} clusters, so it would not be a surprise to find that this also has an effect on rates of /t/-insertion where this is optional. We had no grounds to formulate a specific hypothesis in relation to this point, but we made sure to test plurals with as wide a range of CC onsets as possible (see §\ref{sec:design} for more details), so that we could discover any structured variation that does exist in this domain.

From another perspective, it also seems likely, given the findings of half a century of variationist sociolinguistics (cf. \citealt{Chambers2003}), that, where /t/-insertion is optional, this linguistic variation will have been co-opted to index one or more social variables. In the experiment reported in \citet{LucasSpagnol2016}, we found no effect of gender or age, but recall that in that experiment we found little or no optionality of /t/-insertion for the majority of conditions (see \figref{fig:lucas:1} above), meaning the scope for \isi{sociolinguistic} variation was limited. What optionality we did observe in that experiment was concentrated, as noted above, in the CC-initial \isi{disyllabic} condition that is the focus of the present investigation. It makes sense, therefore, to revisit the possibility of \isi{sociolinguistic} variation here. We cannot know at present whether variable /t/-insertion is stable or represents a change in progress. If it is the latter, we would expect to find that /t/-insertion behaviour varies according to the speakers’ age, as well as their gender, since it is a well-established finding of variationist sociolinguistics (e.g. \citealt{Labov2001}) that females tend to be more linguistically innovative than males. As a practical matter, it proved impossible in the period available for this research to recruit adequate numbers of subjects representing age groups higher than that of undergraduate university students. We ensured, however, that males and females were equally represented among our subjects, so that any gender-based differentiation in /t/-insertion could also be readily discovered.

Finally, while our principal hypothesis is that sound plurals will be relatively unfavourable to /t/-insertion, research into exemplar-based linguistic processing (e.g. \citealt{RumelhartMcClelland1986}; \citealt{Bod1998}; \citealt{Eddington2009}) leads us to hypothesize that sound plurals will not be totally incompatible with /t/-insertion (cf. also the non-zero results from the previous experiment for the sound plurals \textit{platti} ‘plates’ and \textit{stampi} ‘pictures’ shown in \tabref{tab:lucas:2}), and, moreover, that /t/-insertion with sound plurals will be sensitive to their \isi{phonological} similarity to the broken-\isi{plural} patterns that favour /t/-insertion. We therefore selected plurals to test that varied according to two parameters. First, \isi{suffix} type: we ensured that all three of the most frequently occurring sound \isi{plural} suffixes for \isi{disyllabic} plurals – -\textit{iet}, \textit{{}-i} and \textit{{}-s} – were represented among the test items. Note that the items taking the -\textit{iet} \isi{plural} \isi{suffix}, such as \textit{brimbiet} ‘spiders’ (sg. \textit{brimba}), have final stress, whereas items taking the other two suffixes, such as \textit{gruppi} ‘groups’ (sg. \textit{grupp}) and \textit{stejpils} ‘staples’ (sg. \textit{stejpil}), have initial stress. All CC-initial \isi{disyllabic} broken plurals have initial stress, so we predict that sound plurals with the \textit{{}-iet} \isi{suffix} will be less favourable to /t/-insertion than those with the suffixes triggering initial stress. Second, we chose sound-\isi{plural} items that varied according to the nature of the first \isi{syllable}, specifically whether or not it resembled that of the two broken \isi{plural} patterns we predicted would favour /t/-insertion: CCVVCVC and CCVjjVC. Thus we had items with a long \isi{vowel} in the first \isi{syllable}, e.g. \textit{stili} ‘styles’ (sg. \textit{stil}), or with a medial \isi{geminate}, e.g. \textit{vja\.g\.gi} ‘journeys’ (sg. \textit{vja\.g\.g}), and others without these properties, e.g. \textit{spaners} ‘spanners’ (sg. \textit{spaner}). Since a number of the items that fall into the latter category have a light first \isi{syllable} (i.e. a short \isi{vowel} and no \isi{coda}), for convenience we henceforth refer to the whole category as ``light first \isi{syllable}", as opposed to ``heavy first \isi{syllable}" for items of the \textit{stili/vja\.g\.gi} category.\footnote{Note, however, that a number of items in the ``light" category do, in fact, have a \isi{coda} (e.g. \textit{flipflops}). The label is thus for brevity and convenience only. The key distinguishing feature for this factor is resemblance to the broken \isi{plural} patterns CCVVCVC and CCVjjVC.} These research questions and hypotheses are summarized in \tabref{tab:lucas:3}. The following section gives details on the design of the experiment.

\begin{table}
\begin{tabularx}{\textwidth}{XXX}
\lsptoprule
\textbf{Factor} & \textbf{Predicted pattern} & \textbf{Type of factor}\\
\midrule
\textit{General} &  & \\
Plural type & Broken > sound & Morphological\\
\tablevspace
\textit{Broken plurals} &  & \\
Broken \isi{plural} pattern & CCVCVC > CCVVCV(C), CCVjjVC & Morphological\\
Phonetic properties of onset & Exploratory & Phonological\\
Gender & Exploratory & Social \\
\tablevspace
\textit{Sound plurals} &  & \\
Suffix type/stress position & Non-stress-attracting suffixes > -\textit{iet} & Morphophonological\\
Weight of first \isi{syllable} & Heavy > light & Phonological\\
\lspbottomrule
\end{tabularx}
\caption{
Summary of research question and hypotheses
}
\label{tab:lucas:3}
\end{table}


\subsection{Experiment design}\label{sec:design}

The basic design of the experiment followed that of the first experiment, reported on in §\ref{sec:prevexp}. Subjects were recruited from the University of \isi{Malta}, their ages ranging from 18 to 22. The experiment was split into a broken-\isi{plural} section and a sound-\isi{plural} section. 20 subjects (10 male, 10 female) took the sound-\isi{plural} part. The same 20, and 10 more (5 male, 5 female) took the broken-\isi{plural} part. In the broken-\isi{plural} part there were 70 test items and an equal number of fillers, and in the sound-\isi{plural} part there were 49 test items and an equal number of fillers.\footnote{The decision was made to split the experiment into a sound-\isi{plural} and a separate broken-\isi{plural} section, rather than combining them into a single dataset, because we were confident, based on the previous literature and informal observation, that the difference in /t/-insertion rates between the two \isi{plural} types would be totally apparent, removing the necessity to analyse the broken/sound distinction as an additional fixed effect within a single dataset. This confidence was borne out by the results reported in §\ref{sec:finding1}. Splitting the experiment in this way had two advantages. First, it meant that in each of the two sections the test items could be coded differently, and different hypotheses could be tested. Second, it meant that we could collect more data without having to ensure equal numbers of subjects for both sections. Limited time to carry out this research imposed constraints on the preparation and use of test materials. The broken-\isi{plural} test materials were ready first and were used to collect data from ten subjects immediately. At the next opportunity for data collection, the sound-\isi{plural} materials were ready, and so both sets of materials were then used to collect data from 20 further subjects. These 20 took the broken-\isi{plural} test first, then the sound-\isi{plural} part after a short break, so that all 30 subjects took the broken-\isi{plural} test under identical conditions. It is possible that having subjects take the sound-\isi{plural} test after the broken-\isi{plural} one resulted in some sort of learning effect, but note that all subjects were asked after completing both tests what they thought the topic of the investigation was, and none ascertained its true purpose.} 

As in the first experiment, test items consisted of a pairing of a \isi{numeral} between ‘two’ and ‘ten’, presented as a figure, and the singular form of the \isi{plural noun} we were targeting, the task being to realise the \isi{phrase} as it would be in context: the \isi{noun} in the \isi{plural} and the \isi{numeral} in the dependent form, and /t/ optionally inserted between the two. Refer to \tabref{tab:lucas:1} for all forms of the Maltese numerals ‘two’ to ‘ten’. In the experiment, the \isi{numeral} was in fact just one of the following seven: 2, 4, 5, 7, 8, 9, 10. \textit{Tliet} ‘three’ (/t/-form \textit{tlitt/tlett}) and \textit{sitt} ‘six’ (/t/-form \textit{sitt}) were not included because, in the case of the former, the /t/-form and non-/t/-form are too similar phonetically to reliably tell apart phonetically every time, and, in the case of the latter, the two forms are identical. 

Since our aim in this second experiment was primarily to investigate the effect of \isi{phonological} and morphological properties of the \isi{plural noun} itself, holding other factors constant as far as possible, we considered pairing the different nouns with the same \isi{numeral} every time: \textit{ħames(t)} ‘five’, for example. We decided against this, however, for two reasons. First, as noted in footnote 3, the results of the first experiment showed there was no main effect of \isi{numeral} choice on rates of /t/-insertion. So including various numerals in the test stimuli or always just the same one ought not to have a meaningful effect on the variable we are investigating. Second, we suspected that always having the same single \isi{numeral} in the stimuli would make it too easy for test subjects to correctly guess precisely what the experiment was designed to investigate – something we successfully avoided (cf. footnote 5). As such, we used the seven numerals specified above and each was used an equal number of times: with ten nouns in the broken-\isi{plural} part (70 test items ÷ 7) and seven nouns in the sound-\isi{plural} part (49 test items ÷ 7).

As in the first experiment, fillers, which alternated regularly with test items, consisted of pairings of a \isi{numeral} between ‘eleven’ and ‘nineteen’ and a \isi{noun}, the nouns varying widely according to onset and number of syllables. Note that the \isi{noun} following a \isi{numeral} from the 11–100 set is always in the singular, and thus never triggers /t/-insertion. 

The first six stimuli (including fillers) that subjects encountered in the broken-\isi{plural} and sound-\isi{plural} parts of the experiment are illustrated in \REF{ex:lucas:3} and ‎\REF{ex:lucas:4}, respectively. Stimuli were presented in a PDF file on a laptop, with one stimulus per page, a page filling the screen. Subjects had to produce the appropriate form in response to the onscreen stimulus and scroll down to the next page of the PDF having done so. Their responses were given orally and the audio recorded. Responses were categorized independently by both authors, according to whether each one featured /t/-insertion or not. If the presence of /t/-insertion in an individual response was unclear to one or both authors it was excluded from analysis. In total, 15\% of responses in the broken-\isi{plural} data and 13\% in the sound-\isi{plural} data were excluded for this reason, or because subjects gave responses featuring non-target plurals.
%there were 15\% missing cases in the broken and 13\% in the sound \isi{plural} data. 

\protectedex{
\ea\label{ex:lucas:3}
{Broken-\isi{plural} test items and fillers}\\
\ea
{   12      qasba} \\
\ex
{            2        qasrija        (target: \textit{żew\.g(t i)qsari})}\\
\ex
{              13      bandiera}\\
\ex
{            5        raħal     (target: \textit{ħames(t i)rħula})}\\
\ex
{              15      għalqa}\\
\ex
{              10      xkora     (target: \textit{għaxar(t i)xkejjer})}\\
\z
\z
}
\ea\label{ex:lucas:4}
{Sound-\isi{plural} test items and fillers}\\
\ea\label{ex:lucas:}
{   19      bniedem} \\
\ex
{            4        kwadru      (target: \textit{erba(t i)kwadri})}\\
\ex
{              16      ħabsi}\\
\ex
{            5        brama   (target: \textit{ħames(t i)bramiet})}\\
\ex
{              11      gallarija}\\
\ex
{              4        slogan   (target: \textit{erba(t i)slogans})}\\
\z
\z

The nouns to be tested were selected as follows (a full list can be found in \tabref{tab:lucas:7} and \tabref{tab:lucas:10} in \sectref{sec:lucas:3}). With the broken plurals first of all, the three patterns CCVVCV(C), CCVjjVC, and CCVCVC had to be represented. It should be noted here that plurals from the first of these patterns are far more numerous than plurals from the other two (and CCVjjVC is much more frequent than CCVCVC). Since the plurals selected had to be fairly frequent and familiar to our young, mostly town-dwelling subjects, and we also wanted a reasonable balance of different onset types, we chose not to have equal numbers of test nouns from each of the three patterns. Instead, there were ten plurals of the CCVCVC pattern, 15 of the CCVjjVC pattern, and 45 of the CCVVCV(C) pattern. 

\begin{table}
\begin{tabularx}{\textwidth}{lrll}
\lsptoprule
\bfseries Onset types & \bfseries No. of Items & \bfseries Example & \\
\midrule 
{\textsc{stop-stop}} & 6 & \textit{qtates} & ‘cats’\\
% \tablevspace
{\textsc{stop-\isi{fricative}/affricate}} & 6 & \textit{gżejjer} & ‘islands’\\
% \tablevspace
{\textsc{stop-sonorant}} & 9 & \textit{drabi} & ‘times’\\
% \tablevspace
{\textsc{affricate-stop}} & 2 & \textit{\.gkieket} & ‘jackets’\\
% \tablevspace
{\textsc{affricate-sonorant}} & 6 & \textit{\.cwievet} & ‘keys’\\
% \tablevspace
{\textsc{fricative-stop}} & 8 & \textit{stilel} & ‘stars’\\
% \tablevspace
{\textsc{fricative-\isi{fricative}/affricate}} & 5 & \textit{ħxejjex} & ‘vegetables’\\
% \tablevspace
{\textsc{fricative-sonorant}} & 9 & \textit{flieles} & ‘chicks’\\
% \tablevspace
{\textsc{sonorant-stop}} & 5 & \textit{mkatar} & ‘handkerchiefs’\\
% \tablevspace
{\textsc{sonorant-\isi{fricative}/affricate}} & 6 & \textit{r\.gejjen} & ‘queens’\\
% \tablevspace
{\textsc{sonorant-sonorant}} & 4 & \textit{mrietel} & ‘hammers’\\
% \tablevspace
{/sk/-}\textmd{\textsc{sonorant}} & 4 & \textit{skrapan} & ‘shoemakers’\\
\lspbottomrule
\end{tabularx}
\caption{
The broken {plural} test items by onset type
}
\label{tab:lucas:4}
\end{table}

\largerpage
Regarding onset, the 70 broken-\isi{plural} test items selected fell into the 12 categories listed in \tabref{tab:lucas:4}.\footnote{Given the presence of the /sk/–\textsc{sonorant}{}-initial items, the abbreviation ``CC" in this article should be understood as standing for ``\isi{consonant cluster}" in general, rather than for a cluster of exactly two consonants. That said, two-\isi{consonant onset} clusters are much more numerous than three-\isi{consonant onset} clusters, both in our test items and in Maltese generally.} This categorization also entails less fine-grained categorizations of course, such as a three-way division into stop-initial (including affricate-initial; 29 tokens), fricative-initial (26 tokens), and sonorant-initial (15 tokens), or a binary division into sonorant-initial (15 tokens) and others (55 tokens).

\largerpage
Turning to the sound-\isi{plural} test items, we extracted all 225 CC-initial \isi{disyllabic} sound plurals recorded in Aquilina’s (1987–1990) dictionary. Most of these were archaic and/or infrequent and had to be discarded. Among the remainder, the most commonly represented \isi{plural} \isi{suffix} was \textit{{}-i}, as in \textit{sferi} ‘spheres’ (sg. \textit{sfera}), followed by \textit{{}-iet}, as in \textit{brimbiet} ‘spiders’ (sg. \textit{brimba}), then \textit{{}-s}, as in \textit{slogans} ‘slogans’ (sg. \textit{slogan}). We selected all tokens of the latter two \isi{plural} types that we judged sufficiently frequent and familiar (11 tokens of each type) and 27 of the more familiar tokens of \textit{{}-i} plurals, choosing items with a range of onset types and initial-\isi{syllable} weights. Of this total of 49 sound-\isi{plural} tokens, 19 had a light initial \isi{syllable}, and the remaining 30 had a heavy initial \isi{syllable}.

\subsection{Statistical analysis}
The two experimental data sets (broken and sound plurals) were analyzed separately using hierarchical (mixed-effects) logistic regression models with random intercepts for test item and \isi{subject}. The models were fit in R \citep{R2016} using the function ``glmer" in the ``lme4" package \citep{BatesEtAl2015}. As summarized in \tabref{tab:lucas:3}, the objective of this study was to investigate one primary factor (the distinction between sound and broken plurals), and five additional factors (3 for broken, 2 for sound plurals). The latter factors were added as fixed effects to the models for the two data sets. The presentation of results in the following sections proceeds as follows. First, the statistical significance of the individual factors is reported in a table listing the results of Wald Chi-square tests calculated with the function Anova in the ``car" package (\citealt{FoxWeisberg2011}). These tests reveal whether – based on the data at hand – a predictor may be considered as contributing information about the distribution of /t/ insertion in Maltese plurals. Next, a graphical summary of the internal structure of the factors is given using effect displays. These provide estimates about /t/-insertion rates in the different conditions (e.g. male vs. female subjects, heavy vs. light initial \isi{syllable}). All graphs include 95\% confidence intervals and show estimates while holding other factors at their means. The effect displays were constructed with the packages ``effects" \citep{Fox2016}, ``lattice" \citep{Sarkar2008}, and ``latticeExtra" (\citealt{SarkarAndrews2016}). The figures plotted in the effect displays are also provided in tabular form. Finally, technical details about the model coefficients are given in the appendix.

\largerpage
\section{Results}\label{sec:lucas:3}
\begin{figure}[b]
\includegraphics[height=.3\textheight]{figures/a5LucasSpagnol-img1.png}
\caption{
Effect displays showing estimated /t/-insertion rates in broken plurals. In each panel, other variables are held at their means (error bars: 95\% confidence intervals)
} 
\label{fig:lucas:2}
\end{figure}
\subsection{Basic finding}\label{sec:finding1}
The headline result is that the key hypothesis – that /t/-insertion rates are sensitive to the sound vs. broken \isi{plural} distinction – is strongly confirmed. The mean /t/-\isi{insertion rate} for all broken plurals tested was 32\%, while for all sound plurals tested it was 5\%. However, the results are far from uniform in either set of test items, and particularly with the broken plurals, as can be seen in \figref{fig:lucas:2}. The reasons for this heterogeneity are explored in the following sections. The full list of items tested, together with the per item results, can be found in \tabref{tab:lucas:7} and \tabref{tab:lucas:10}. 


\begin{table}
\small
\begin{tabularx}{\textwidth}{lrr X lrr}
\lsptoprule
\textbf{Test items} & \textbf{Meaning} & \multicolumn{1}{c}{\textbf{\% /t/}} &  & \textbf{Test items} & \textbf{Meaning} & \textbf{\% /t/}\\
\midrule 
qsari & flower pots & 87\% &  & skieken & knives & 46\%\\
slaleb & crosses & 87\% &  & skrata\.c & cartridges & 45\%\\
ħnieżer & pigs & 87\% &  & rwiefen & gales & 44\%\\
qtates & cats & 86\% &  & sfafar & whistles & 44\%\\
flieles & chicks & 85\% &  & \.crieket & rings & 43\%\\
\.gkieket & jackets & 79\% &  & bziezen & bread rolls & 43\%\\
kpiepel & hats & 76\% &  & msielet & earrings & 41\%\\
żwiemel & horses & 73\% &  & \textbf{\textit{ħxejjex}} & vegetables & \textbf{\textit{41\%}}\\
qniepen & bells & 73\% &  & \textbf{\textit{ħrejjef}} & fables & \textbf{\textit{39\%}}\\
dbielet & skirts & 72\% &  & \.cpiepet & bracelets & 37\%\\
drabi & times & 67\% &  & \textbf{\textsc{MĦADED}} & pillow & \textbf{\textsc{37\%}}\\
blalen & balls & 67\% &  & \textbf{\textit{xmajjar}} & rivers & \textbf{\textit{36\%}}\\
\.cmieni & chimneys & 67\% &  & \textbf{\textit{skejjel}} & schools & \textbf{\textit{33\%}}\\
kwiekeb & stars & 65\% &  & ktieli & kettles & 33\%\\
sħaħar & wizards & 64\% &  & bdiewa & farmers & 32\%\\
\textbf{\textit{mwejjed}} & tables & \textbf{\textit{64\%}} &  & mkatar & handkerchiefs & 31\%\\
bżieżaq & balloons & 63\% &  & \.craret & pcs of cloth & 30\%\\
ħbula & ropes & 62\% &  & \textbf{\textit{nbejjed}} & wines & \textbf{\textit{30\%}}\\
\textbf{\textit{skrejjen}} & propellers & \textbf{\textit{62\%}} &  & zlazi & sauces & 30\%\\
xfafar & blades & 62\% &  & \textbf{\textit{xkejjer}} & sacks & \textbf{\textit{29\%}}\\
skrapan & shoemakers & 59\% &  & \textbf{\textsc{GWERER}} & wars & \textbf{\textsc{28\%}}\\
\textbf{\textit{knejjes}} & churches & \textbf{\textit{57\%}} &  & \textbf{\textit{stejjer}} & stories & \textbf{\textit{27\%}}\\
\.cwievet & keys & 56\% &  & rkiekel & bobbins & 26\%\\
rdieden & sp. wheels & 56\% &  & \textbf{\textit{gżejjer}} & islands & \textbf{\textit{23\%}}\\
\.grieden & mice & 54\% &  & msiemer & nails & 20\%\\
ħwienet & shops & 53\% &  & \textbf{\textsc{SKWERER}} & set-squares & \textbf{\textsc{19\%}}\\
rziezet & farms & 53\% &  & \textbf{\textsc{PLAKEK}} & plugs & \textbf{\textsc{17\%}}\\
\textbf{\textsc{XKAFEF}} & shelves & \textbf{\textsc{53\%}} &  & \textbf{\textsc{VLE\.GE\.G}} & arrows & \textbf{\textsc{15\%}}\\
mrietel & hammers & 53\% &  & \textbf{\textit{r\.gejjen}} & queens & \textbf{\textit{15\%}}\\
qżieqeż & piglets & 52\% &  & \textbf{\textsc{STILEL}} & stars & \textbf{\textsc{13\%}}\\
\textbf{\textit{ktajjen}} & chains & \textbf{\textit{50\%}} &  & twieqi & windows & 13\%\\
rħula & villages & 48\% &  & \textbf{\textsc{SPONOŻ}} & sponges & \textbf{\textsc{13\%}}\\
mqaret & date pastries & 48\% &  & \textbf{\textsc{PJAZEZ}} & squares & \textbf{\textsc{10\%}}\\
\textbf{\textit{rwejjaħ}} & smells & \textbf{\textit{46\%}} &  & \textbf{\textit{ħsejjes}} & sounds & \textbf{\textit{3\%}}\\
kxaxen & drawers & 46\% &  & \textbf{\textsc{FLOTOT}} & fleets & \textbf{\textsc{0\%}}\\
\lspbottomrule
\end{tabularx}
\caption{
CC-initial  {disyllabic} broken plurals – test items and /t/-insertion rates. Key: CCVVC(C): plain typeface; CCVjjVC: \textbf{\textit{bold italics}};
CCVCVC: \textbf{BOLD CAPS}.
}\label{tab:lucas:7}
\end{table}


\begin{table}
\begin{tabularx}{\textwidth}{llr X llr}
\lsptoprule
\textbf{Test items} & \textbf{Meaning} & \multicolumn{1}{c}{\textbf{\% /t/}} &  & \textbf{Test items} & \textbf{Meaning} & \textbf{\% /t/}\\
\midrule 
\textbf{\textit{sferi}} & spheres & \textbf{\textit{31\%}} &  & bramiet & jellyfish & 0\%\\
\textbf{FJAMMI} & flames & \textbf{22\%} &  & \textbf{TNALJI} & tongs & \textbf{0\%}\\
\textbf{\textit{pjagi}} & plagues & \textbf{\textit{21\%}} &  & stensils & stencils & 0\%\\
\textbf{\textit{friżers}} & freezers & \textbf{\textit{20\%}} &  & ħjariet & cucumbers & 0\%\\
\textbf{\textit{bdoti}} & pilots & \textbf{\textit{17\%}} &  & qronfliet & carnations & 0\%\\
\textbf{\textit{xkupi}} & brooms & \textbf{\textit{14\%}} &  & briksiet & bricks & 0\%\\
\textbf{\textit{skedi}} & cards & \textbf{\textit{12\%}} &  & \textbf{PLATTI} & plates & \textbf{0\%}\\
\textbf{GRUPPI} & groups & \textbf{11\%} &  & drillers & drills & 0\%\\
\textbf{\textit{stili}} & styles & \textbf{\textit{11\%}} &  & \textbf{TRA\.C\.CI} & traces & \textbf{0\%}\\
\textbf{\textit{travi}} & beams & \textbf{\textit{11\%}} &  & \textbf{\textit{gradi}} & grades & \textbf{\textit{0\%}}\\
\textbf{ŻBALJI} & mistakes & \textbf{10\%} &  & skandli & scandals & 0\%\\
pruniet & plums & 9\% &  & brimbiet & spiders & 0\%\\
\textbf{DVALJI} & tab. cloths & \textbf{7\%} &  & blackboards & blackboards & 0\%\\
\textbf{drogi} & drugs & \textbf{7\%} &  & \textbf{\textit{cruises}} & cruises & \textbf{\textit{0\%}}\\
\textbf{kwoti} & shares & \textbf{6\%} &  & \textbf{\textit{stejpils}} & staples & \textbf{\textit{0\%}}\\
\textbf{GRAMMI} & grams & \textbf{6\%} &  & \textbf{ĦNEJJIET} & arches & \textbf{0\%}\\{}
ħ\.gi\.giet & panes & 6\% &  & spagiet & pcs of string & 0\%\\
\textbf{SKOSSI} & bumps & \textbf{5\%} &  & \.gbejniet & small cheeses & 0\%\\
\textbf{\textit{skużi}} & excuses & \textbf{\textit{5\%}} &  & stampi & pictures & 0\%\\
flipflops & flipflops & 5\% &  & \textbf{VJA\.G\.GI} & journeys & \textbf{0\%}\\
\textbf{FROTTIET} & fruits & \textbf{5\%} &  & brackets & brackets & 0\%\\
\textbf{\textit{slogans}} & slogans & \textbf{\textit{5\%}} &  & brushes & brushes & 0\%\\
spaners & spanners & 5\% &  & pjanti & plants & 0\%\\
\textbf{DRAMMI} & dramas & \textbf{5\%} &  & statwi & statues & 0\%\\
\textbf{\textit{kwadri}} & paintings & \textbf{\textit{0\%}} &  &  &  & \\
\lspbottomrule
\end{tabularx}
\caption{
CC-initial  {disyllabic} sound plurals – test items and /t/-insertion rates. Key: light first  {syllable}: plain typeface; 
heavy (long  {vowel}): \textbf{\textit{bold italics}}; heavy (medial  {geminate}): \textbf{BOLD CAPS}}
\label{tab:lucas:10}
\end{table}

\subsection{Broken plurals}
\tabref{tab:lucas:5} shows the statistical significance of the three factors investigated in relation to broken plurals: pattern (i.e. CCVVCV(C), as in \textit{bramel}, CCVjjVC, as in \textit{knejjes} and CCVCVC, as in \textit{gwerer}); phonetic properties of the onset; and gender of the test \isi{subject}. We can see that pattern is a statistically significant factor, as we predicted it would be, while the factors of onset and gender of \isi{subject} (about which we had no specific hypotheses) turn out not to be statistically significant.\footnote{\tabref{tab:lucas:5} shows that the factor of onset does not contribute useful information to an understanding of how /t/-insertion is distributed among the broken plurals tested, when onset is coded as a binary distinction between initial \isi{sonorant} and initial \isi{obstruent} consonants. Onset is similarly not useful with a more fine-grained coding, as in \tabref{tab:lucas:4} and the accompanying discussion.}

\begin{table}
\begin{tabularx}{\textwidth}{XXrrrl}
\lsptoprule
\bfseries Factor & \bfseries Levels & \bfseries Wald Chi-square & \bfseries \textit{df} & \bfseries \textit{p} & \\
\midrule
Broken \isi{plural} pattern & CCVCVC, CCVVCV(C), CCVjjVC & 32.67 & 2 & <0.0001 & ***\\
\tablevspace
Phonetic properties of onset & \isi{obstruent}, \isi{sonorant} & 2.11 & 1 & 0.15 & \\
\tablevspace
Gender & female, male & 0.83 & 1 & 0.36 & \\
\lspbottomrule
\end{tabularx}
\caption{
Contribution of the factors to /t/-insertion in broken plurals: Wald Chi-square tests
}
\label{tab:lucas:5}
\end{table}


The internal structure of these three factors is shown in \figref{fig:lucas:2} and \tabref{tab:lucas:6}. Note that the prediction that the \textit{gwerer}{}-type CCVCVC pattern is less favourable to /t/-insertion than the other two is confirmed: there was a statistically significant difference both between the CCVjjVC pattern and the CCVCVC pattern (z = 2.56, p = 0.01), and between the CCVVCV(C) pattern and the CCVCVC pattern (z = 5.39, p < 0.0001).\footnote{The difference between the CCVVCV(C) and CCVjjVC patterns was also found to be statistically significant (z = -2.39, p = 0.02).} 



\begin{table}
\begin{tabularx}{\textwidth}{XrXrr} 
\lsptoprule
&  &  & \multicolumn{2}{c}{\bf 95\% confidence interval}\\
\bfseries Factor & \bfseries Estimated /t/-\isi{insertion rate} &  & \bfseries Lower limit & \bfseries Upper limit\\
%\cmidrule{1-2}\cmidrule{4-5}
\midrule
\textit{Pattern} &  &  &  & \\
%\midrule
  CCVCVC & 10\% &  & 4\% & 23\%\\
  CCVjjVC & 29\% &  & 14\% & 50\%\\
  CCVVCV(C) & 55\% &  & 37\% & 72\%\\
  \tablevspace
\textit{Gender} &  &  &  & \\
%\midrule
  Female & 34\% &  & 16\% & 57\%\\
  Male & 48\% &  & 26\% & 71\%\\
  \tablevspace
\textit{Onset} &  &  &  & \\
%\midrule
  Obstruent & 44\% &  & 27\% & 61\%\\
  Sonorant & 31\% &  & 16\% & 53\%\\
\lspbottomrule
\end{tabularx}
\caption{
Estimated /t/-insertion rates in broken plurals
}
\label{tab:lucas:6}
\end{table}



\largerpage[-1]
\subsection{Sound plurals}
As noted in §\ref{sec:finding1} and illustrated in \tabref{tab:lucas:10}, although rates of /t/-insertion were generally much lower with sound plurals than with broken plurals, rates were not consistent across all the sound \isi{plural} items tested. While 25 of the 49 items tested triggered no /t/-insertion at all, the other 24 did at least 5\% of the time, and the item with the highest rate of /t/-insertion was \textit{sferi} ‘spheres’ (sg. \textit{sfera}), at 31\%. We suspected before collecting the data that any such differences would be due to morphophonological properties of the plurals, specifically: 1) the nature of the first \isi{syllable} (i.e. heavy, due to a long \isi{vowel} or a word-medial \isi{geminate}, and thus phonologically similar to the /t/-favouring broken-\isi{plural} patterns, or otherwise light), and 2) the position of stress (i.e. on the initial \isi{syllable}, in the case of items with the \isi{plural} suffixes \textit{{}-i} and \textit{{}-s}, or on the final \isi{syllable}, in the case of items with the \isi{plural} \isi{suffix} \textit{{}-iet}). As can be seen from \tabref{tab:lucas:8}, only initial \isi{syllable} weight turned out to be a significant predictor of /t/-insertion rates. Specifically, while /t/-insertion was low or very low across the board with the sound \isi{plural} items tested in this experiment, it was significantly less low with those items with a heavy initial \isi{syllable}. This is illustrated in \figref{fig:lucas:3} and \tabref{tab:lucas:9}.

\begin{table}
\begin{tabularx}{\textwidth}{p{2.3cm}Qr rrc}
\lsptoprule
\bfseries Factor & \bfseries Levels & \bfseries Wald Chi-square & \bfseries \textit{df} & \bfseries \textit{p} & \\
\midrule
Weight of first \isi{syllable} & heavy, light &7.59 & 1 & 0.006 & **\\
\tablevspace
Suffix type / stress position & {}-\textit{iet}, \mbox{non-stress-attracting suffixes} &0.00 & 1 & 0.95 & \\
\lspbottomrule
\end{tabularx}
\caption{
Contribution of the factors to /t/-insertion in sound plurals: Wald Chi-square tests
}
\label{tab:lucas:8}
\end{table}


\begin{figure}
\includegraphics[height=.3\textheight]{figures/a5LucasSpagnol-img2.png}
\caption{
Effect displays showing estimated /t/ insertion rates in sound plurals. In each panel, other variables are held at their means (error bars: 95\% confidence intervals)
}
\label{fig:lucas:3}
\end{figure}


\clearpage 
\begin{table}
\begin{tabularx}{\textwidth}{lr X rr} 
\lsptoprule
&  &  & \multicolumn{2}{c}{ 95\% confidence interval}\\
\bfseries Factor & \bfseries Estimated /t/-\isi{insertion rate} &  & \bfseries Lower limit & \bfseries Upper limit\\
%\cmidrule{1-2}\cmidrule{4-5}
\midrule
\textit{Weight} &  &  &  & \\
Heavy &0.6\% &  &0.1\% &5.7\%\\
Light &0.1\% &  &0.0\% &1.1\%\\
\tablevspace
\textit{Stress} &  &  &  & \\
Final (-\textit{iet}) &0.3\% &  &0.0\% &3.8\%\\
Initial &0.3\% &  &0.0\% &2.9\%\\
\lspbottomrule
\end{tabularx}
\caption{
Estimated /t/ insertion rates in sound plurals 
}
\label{tab:lucas:9}
\end{table}
\subsection{Summary of results}

In this new experiment on /t/-insertion behaviour with CC-initial \isi{disyllabic} plurals we found, in line with our predictions, that broken plurals trigger /t/-inser\-tion much more often than sound plurals, even when onset and number of syllables – the two factors generally cited in the previous literature as most significant to /t/-insertion – are held constant. 

Also in line with our predictions was the finding that broken plurals of the \textit{gwerer}{}-type CCVCVC pattern trigger /t/-insertion significantly less often than the other two CC-initial \isi{disyllabic} broken \isi{plural} patterns. 

We suspected that the gender of our test subjects and the precise nature of the onset might have an influence on rates of /t/-insertion. This proved not to be the case, however.

Finally, we suspected that /t/-insertion behaviour with sound plurals would be influenced by the weight of the initial \isi{syllable}, and the position of stress, in the \isi{plural} items tested. Stress position was found not to be a significant factor, but we found that items with a heavy initial \isi{syllable} were associated with significantly higher rates of /t/-insertion than those with a light initial \isi{syllable}. This should not, however, obscure the fact that overall the rate of /t/-insertion with the sound \isi{plural} items tested was close to zero.

\section{Discussion}

We have seen that the two main predictions were borne out: when we hold onset and number of syllables constant (at CC and two, respectively), broken plurals are much more favourable to /t/-insertion than sound plurals, and broken plurals of the \textit{gwerer}{}-type CCVCVC pattern are much less favourable than plurals of other patterns. This is clear confirmation that /t/-insertion cannot be understood purely in \isi{phonological} terms, such as the onset type and number of syllables of \isi{plural} nouns: their morphological profile is at least as important.

As explained in \sectref{sec:lucas:‎2.1}, the genesis of the hypothesis that \textit{gwerer}{}-type CCVCVC plurals would be unfavourable to /t/-insertion, despite beginning with a \isi{consonant cluster}, was the insight that they belong to the same basic (C)CVCVC pattern as plurals such as \textit{bozoz} ‘bulbs’ (sg. \textit{bozza}), which, like all CV-initial plurals, seems to be particularly hostile to /t/-insertion. If onset type had been more important, we would have expected \textit{gwerer}{}-type plurals to pattern with other CC-initial broken plurals and be favourable to /t/-insertion. In the event, it was the morphological identity of these items – their membership of the /t/-resistant (C)CVCVC pattern – that proved decisive, not their onset. 

We must be careful not to take this line of argument too far, however. Our results show that the kinds of \isi{phonological} factors considered in the literature to date clearly cannot do all the work of explaining what governs /t/-insertion. But our results also indicate that \isi{phonology} has a role to play. This is perhaps easiest to see by considering the sound-\isi{plural} items with the highest rates of /t/-insertion. These show that \citeauthor{Borg1974}’s (1974: 297) blunt claim that “sound plurals do not take /t/” is too sweeping: they certainly do not favour /t/-insertion, but they do not rule it out altogether. Consider in particular items such as \textit{pjagi} ‘plagues’ (sg. \textit{pjaga}), with a /t/-\isi{insertion rate} of 21\%, and \textit{travi} ‘beams’ (sg. \textit{travu}) with a rate of 11\%. These have an identical \isi{phonological} profile to broken \isi{plural} items we tested, such as \textit{qsari} ‘flower pots’ (sg. \textit{qasrija}), with a /t/-\isi{insertion rate} of 87\%, and \textit{drabi} ‘times’ (sg. \textit{darba}), with a rate of 67\%. More generally, note the following parallels between the broken and sound plurals that we tested. The two broken \isi{plural} patterns in our data that were more favourable to /t/ were CCVVCVC, as in \textit{bramel} ‘buckets’ (sg. \textit{barmil}), with a long \isi{vowel} in the first \isi{syllable}, and CCVjjVC, as in \textit{knejjes} ‘churches’ (sg. \textit{knisja}), with a medial \isi{geminate}; and it was also the sound plurals with either a long \isi{vowel} in the initial \isi{syllable} or a medial \isi{geminate} that triggered /t/-insertion significantly more frequently than the others. It is unlikely that this is a coincidence. Rather it seems that \isi{phonology} is playing a secondary role here: an item’s morphological identity as a sound \isi{plural} ensures it will be basically hostile to /t/-insertion, but this hostility can be lessened to a limited extent, just in case its \isi{phonology} closely resembles that of an appropriate broken \isi{plural} pattern.

\newpage 
A similar dynamic seems to hold with the \textit{gwerer}{}-type CCVCVC broken plurals. One might have expected these to be totally incompatible with /t/-insertion, rather than permitting it with an average frequency of 23\%. After all, these have the same basic pattern as CV-initial plurals such as \textit{bozoz}, which seem to totally exclude the possibility of /t/-insertion.\footnote{A degree of caution is required here. In the first experiment, \textit{bozoz}, which never triggered /t/-insertion, was the only \isi{plural} of this type that we tested. Since a) it has never been suggested in the previous literature that /t/-insertion is possible with CV-initial items in general, and b) we found in our first experiment that CV-initial items, including other broken plurals such as \textit{kotba} ‘books’ (sg. \textit{ktieb}), were uniformly hostile to /t/-insertion, it is reasonable to extrapolate that this generalises to all broken plurals of the CVCVC pattern. But we do not, at present, have the data to prove that this is the case.}  Instead it seems that, as with the sound plurals such as \textit{sferi}, the morphological pressure on plurals such as \textit{gwerer} to resist /t/-insertion is mitigated somewhat by their \isi{phonological} similarity (having a CC onset) to the broken-\isi{plural} patterns which actively favour /t/-insertion.  

It is noteworthy, finally, that we found no significant effect of gender on speakers’ /t/-insertion behaviour. With linguistic variation of this kind, where, in a sufficiently well-defined context (e.g. with broken plurals of the CCVVCV(C) type) there seems to be total optionality from a linguistic point of view, it is natural to expect that inter-\isi{speaker} variation might be invested with social meaning. But this is especially likely to be the case with variation that is the result of changes in progress, and it could be that the optionality of /t/-insertion, at least in the restricted domain of \isi{disyllabic} CC-initial plurals investigated here, is in fact a system that has been stable for several generations or more. Future studies could investigate this issue by repeating the kind of experiment described here, with test subjects stratified by age, gender, and perhaps other \isi{sociolinguistic} variables.

\section{Conclusion}

This article has provided evidence that, contrary to previous work on the topic, it is morphological, not \isi{phonological} properties of the \isi{plural noun} that should be seen as the prime determinants of whether /t/-insertion is triggered in the presence of an accompanying \isi{numeral}. Specifically, at least as far as CC-initial plurals are concerned, /t/-insertion is favoured only by particular broken-\isi{plural} patterns. On the other hand, we have seen that \isi{phonology} does seem to play a secondary role. While sound plurals are, on the whole, very hostile to /t/-insertion, there are some whose \isi{phonology} happens to closely resemble that of the /t/-favouring broken-\isi{plural} patterns, and it seems to be this which causes them to trigger /t/-insertion, if only rarely.

This is by no means all there is to be said on this topic. Aside from the \isi{sociolinguistic} dimension suggested above, there are several aspects of the grammar of /t/-insertion that remain unclear, for example the conditions determining /t/-insertion with vowel-initial plurals, and whether the frequency of a \isi{plural noun} (or a \isi{numeral}–\isi{noun} string) has an effect on its /t/-insertion behaviour. It is to be hoped that puzzles such as these can be solved in future work.

  
\section*{Acknowledgements}
The authors would like to sincerely thank Lukas Sönning for his generous assistance and input into the analysis and presentation of this research, which was made possible by means of a Humboldt research fellowship for postdoctoral researchers. Any shortcomings of this work are entirely the responsibility of the authors.

\section*{Appendix}
\begin{table}
\begin{tabularx}{\textwidth}{X rrrr}
\lsptoprule
\textbf{Fixed effects} & \bfseries $\beta $ & \textbf{SE} & \textbf{\textit{z}} & \textbf{\textit{p}}\\
\midrule
Intercept (CCVCVC, Obstruent, Female) &$-$2.42 & 0.63 &$-$3.87 & 0.0001\\
Pattern: CCVjjVC &1.32 & 0.52 &2.56 & 0.01\\
Pattern: CCVVCVC &2.42 & 0.45 &5.39 & <0.0001\\
Onset: Sonorant &$-$0.52 & 0.36 &$-$1.45 & 0.15\\
Gender: Male &0.61 & 0.67 &0.91 & 0.36\\
\tablevspace
\midrule
\textbf{Random effects} & \textbf{Variance} & \textbf{SD} &  & \\
\midrule
Item & 1.22 & 1.10 &  & \\
Subject & 3.24 & 1.80 &  & \\
\lspbottomrule
\end{tabularx}
\caption{Parameter estimates for the broken-plural model}
\end{table}

\clearpage 
\begin{table}
\begin{tabularx}{\textwidth}{X rrrr}
\lsptoprule
\textbf{Fixed effects} & \bfseries $\beta $ & \textbf{SE} & \textbf{\textit{z}} & \textbf{\textit{p}}\\
\midrule 
Intercept (Heavy, Final) &$-$5.17 & 1.43 &$-$3.61 & 0.0003\\
Weight: Light &$-$2.08 & 0.75 &2.76 & 0.006\\
Stress: Final &0.05 & 0.90 &0.06 & 0.95\\
\tablevspace
\midrule
\textbf{Random effects} & \textbf{Variance} & \textbf{SD} &  & \\
\midrule
Item & 1.21 & 1.10 &  & \\
Subject & 7.32 & 2.71 &  & \\
\lspbottomrule
\end{tabularx}
\caption{Table A2. Parameter estimates for the sound plural model}
\end{table}


  
 
 
 
\sloppy
\printbibliography[heading=subbibliography,notkeyword=this] 
\end{document}