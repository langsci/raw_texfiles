\chapter{Verbal Derivation}\label{VerbalDerivation}
\is{derivation|(}
\section{Introduction}
Verbal derivation in Nyakyusa is accomplished by means of several devices. The most frequent is the use of derivational suffixes. These are also known as verbal extensions in Bantu studies \citep{SchadebergT2003a} and are common in the postulated Niger-Congo phylum \citep{HymanL2007}. The canonical verbal extension has the shape\mbox{-VC}, with the exception of the passive and causative extensions (\mbox{-V} and \mbox{-VCV}). While some verbal extensions, e.g. the applicative,\is{applicative} are highly productive, others like the \isi{tentive} can be segmented on the basis of their shape and meaning, but are not used productively to derive new verbs. Between these two extremes lie other extensions whose productivity is harder to determine as there seems to be a close interaction with verbal semantics (see \citealt{FleischA2000}). In these cases productivity is described on a tentative basis.

Verbal derivation by means of derivational suffixes is cyclic: once a derived verbal base has acquired a special or idiosyncratic meaning, this functions as the point of departure for subsequent derivations. 
\begin{exe}
\ex
\begin{tabbing}
 > \= > \= > \=\kill
\textit{fuma}\\
\lq come from'\\
 > \textit{fum\textbf{uk}a} \\
\>\lq be(come) known, famous' (separative)\\
\> > \textit{fum\textbf{usy}a}\\
\> \>\lq announce' (separative + causative)\\
\>\> > \textit{fum\textbf{usigw}a}\\
\>\>\> \lq be announced' (separative + causative + passive)
\end{tabbing}
\end{exe}

Verbal derivations cover a set of functions that can be subsumed under valency changing operations, semantic alternations, aktionsart\is{aktionsart} (see \sectref{Aktionsart}) and alternations in Aristotelian aspect\is{aspect!Aristotelian} (see \sectref{AristotelianAspect}). Some devices express a combination of these functions.

This chapter begins with an account of morphophonological processes affecting verbal extensions (\sectref{MorphophonologyOfVerbalExtension}), followed by a section on the form and function of each verbal extension (\sectref{Causative1}--\ref{extensive}). The treatment of the less productive extensions includes observations on lexical co-occurrences that are derived forms from the same root (commutations). Thereafter the combinations of verbal extensions are dealt with in more detail, including their respective order and some cases of specialized combinations (\sectref{CombinationOfVerbalExtensions}). Following the verbal extensions, denominal verb bases are dealt with (\sectref{Denominal}). Lastly, there is a note on partially reduplicated\is{reduplication} verbs (\sectref{PartialReduplication}).
\section{Verbal extensions}
\subsection{Morphophonology of verbal extensions}
\label{MorphophonologyOfVerbalExtension}
Verbal extensions in Nyakyusa are subject to two progressively operating processes affecting the degree of vowel opening: vowel height harmony and high vowel raising. 
\subsubsection{Vowel height harmony}\label{VowelHarmony} 
\is{vowels!vowel height harmony|(}
The verbal extensions with an underlying front vowel /ɪ/ surface with mid vowel /e/ following a syllable containing mid vowels /e, o/. 
The examples in (\ref{exVHHfront}) illustrate this for the applicative extension (only one possible reading given for each example).
\begin{exe}
\ex \label{exVHHfront}
\begin{tabbing}
\textit{bʊʊkɪla}x\=`measure with'x\= < \textit{bʊʊka}x\=\kill%Unsinn, fuer Tabulatoren
\textit{fikɪla}\>`arrive at'\> < \textit{fika}\>`arrive'\\
\textit{pɪmɪla}\>`measure with'\> < \textit{pɪma}\>`weigh, measure'\\
\textit{temela}\>`cut with'\> < \textit{tema}\>`cut' \\
\textit{jabɪla}\>`give off to'\> < \textit{jaba}\>\lq divide; distribute'\\
\textit{bopela}\>`run to'\> < \textit{bopa}\>`run'\\
\textit{bʊʊkɪla}\>`go to'\> < \textit{bʊʊka}\>`go (to)'\\
\textit{guulɪla}\>`wait for'\> < \textit{guula}\>`wait'
\end{tabbing}
\end{exe}

A similar rule applies to extensions beginning with the back vowel /ʊ/. The separative extensions -\textit{ʊl} and -\textit{ʊk} occur as -\textit{ol} and -\textit{ok} following a syllable with /o/ (but not /e/). They surface as -\textit{ul}/-\textit{uk} following a syllable with the high back vowel /u/. That is, front and back vowels are treated asymmetrically. Previous discussions of vowel harmony in Nyakyusa (e.g. \citealt{MwangokaNVoorhoeveJ1960b}; \citealt{LabroussiC1998}; \citealt{HymanL1999}) did not notice the raising of /ʊ/ to /u/. This rule of high vowel rising seems to be the expression of a more general constraint against stem-internal /*uCʊ/. These rules are illustrated in (\ref{exVHHback}) for the separative transitive extension. 
\begin{exe}
\ex \label{exVHHback}
\begin{tabbing}
\textit{niembʊla}x\=`disentangle'x\= < \textit{niemba}x\=\kill%Unsinn fuer Tabulatoren
\textit{kingʊla}\>`uncover'\> < \textit{kinga}\>`cover' \\
\textit{pɪndʊla}\>`convert'\> < \textit{pɪnda}\>`bend; wrap'\\
\textit{niembʊla}\>`disentangle'\> < \textit{niemba}\>`wrap up'\\
\textit{matʊla}\>`demolish'\> < \textit{mata}\>\lq plug, stop up'\\
\textit{bonola}\>`pay off'\> < \textit{bona}\>`see'\\
\textit{tʊngʊla}\>`pick'\> < \textit{tʊnga}\>`hang; string together'\\
\textit{fumbula}\>`solve'\> < \textit{fumba}\>`enclose in mouth or hands'
\end{tabbing}
\end{exe}

In principle, the rules of vowel harmony apply to all verbal extensions in question. (\ref{ruleVHH}) gives a formalized account. Extensions containing /i/ or /a/, such as the passive or the reciprocal, do not change their vowel quality. 

\begin{exe}
\ex \label{ruleVHH}
\phonc{ɪ}{e}{\oneof{
\mbox{e C \phold} \\
\mbox{o C \phold}}}
\\
\phonl{ʊ}{o}{o C} \\
\phonl{ʊ}{u}{u C}
\end{exe}

\label{VHHMonosyllabicVerbs} The shape of the verbal extensions subject to vowel height harmony on monosyllabic roots is not predictable in a straightforward fashion, at least in synchronic terms. (\ref{MonosyllabicAPPL}) lists the applicative forms of monosyllabic verbs together with their \ili{Proto-Bantu} forms.\footnote{Defective \textit{lɪ} and \textit{tɪ} do not take derivational suffixes.} 
\begin{exe} 
\ex
\begin{xlist} \label{MonosyllabicAPPL}
\ex 
\begin{tabbing}
\textit{mwa}x\=(*mò)x\=`be plenty (esp. fish)'x\=\kill%semantisch unsinnige Zeile, aber fuer Tabulatoren
\textit{pa}\>(*pá)\>`give'\> > \textit{peela}\\
\textit{ja}\>(*gɪ̀)\>`be(come)'\> > \textit{jɪɪla}
\end{tabbing}
\ex \begin{tabbing}
\textit{mwa}x\=(*mò)x\=`be plenty (esp. fish)'x\=\kill%semantisch unsinnige Zeile, aber fuer Tabulatoren
\textit{fwa}\>(*kú)\>`die'\> > \textit{fwɪla}\\
\textit{gwa}\>(*gʊ̀)\>`fall'\> > \textit{gwɪla}\\
\textit{kwa}\>(*kó)\>`pay dowry'\> > \textit{kwela} \\
\textit{lwa}\>(*dʊ̀)\>`fight'\> > \textit{lwɪla}\\
\textit{mwa}\>(*mò)\>`shave'\> > \textit{mwela}\\
\textit{nwa}\>(*ɲó)\>`drink'\> > \textit{nwela}\\
\textit{swa}\>(*tu?)\>`spit; forgive'\> > \textit{swela}\\
\textit{twa}\>(*tó?)\>`be plenty (esp. fish)'\> > \textit{twɪla}
\end{tabbing}
\ex
\begin{tabbing}
\textit{mwa}x\=(*mò)x\=`be plenty (esp. fish)'x\=\kill%semantisch unsinnige Zeile, aber fuer Tabulatoren
\textit{kya}\>(*ké)\>`dawn; cease to rain'\> > \textit{kyela}\\
\textit{lya}\>(*lɪ́)\>`eat'\> > \textit{lɪɪla}\\
\textit{nia}\>(*nè)\>`defecate'\> > \textit{niela}\\
\textit{pya}\>(*pɪ́)\>`be(come) burnt'\> > \textit{pɪɪla}\\
\textit{sya}\>(*cè)\>`grind'\> > \textit{syela}
\end{tabbing}
\end{xlist}
\end{exe}

As can be gathered, \textit{pa} is treated as underlying \textit{pa-a} for derivational purposes. With all other monosyllabic roots, the surface glide is retained, unless the sequence is /*ɪ-ɪ/. The vowel quality of the derivational suffix as such is conditioned by the historic root vowel and shows the same alternations as with longer verbs. The only exceptions are \textit{twa} `be plenty (esp. of fish)', whose origin in \textit{*tó} `bite' is merely tentative, and \textit{swa} \lq spit; forgive' < *\textit{tú} \lq spit', where *\textit{fwa} would be expected according to the rules of diachronic phonology.
\is{vowels!vowel height harmony|)}
\subsubsection{High vowel raising}\label{HighVowelRaising}
\is{vowels!high vowel raising|(}
Two further and related processes affect the quality of vowels in verbal extensions. What both have in common is the raising of the second degree vowels /ɪ, ʊ/ to the first degree /i, u/. First, the underlying vowels of verbal extensions are raised to first degree vowels when following the palatal nasal,\footnote{This seems to be the expression of a more general constraint against stem-internal /*nyɪ, nyʊ/, cf. also nominal stems like \textit{unyu} \lq salt' < PB\textit{*jɪ́nyʊ̀}.} as illustrated in (\ref{exVowelRaisingAfterPalatalNasal}) with the \isi{applicative} (-\textit{ɪl}), \isi{neuter} (-\textit{ɪk}), \isi{causative}\textsubscript{2} (-\textit{ɪsi}) and the \isi{separative} (-\textit{ʊl}/-\textit{ʊk}) extensions.

\begin{exe}
\ex \label{exVowelRaisingAfterPalatalNasal}
\begin{tabbing}
\textit{tuuny\textbf{u}ka}x\=(\degree tuuny-ʊk-a)x\=`disencourage'x\= < \textit{tuunya}x\=\kill
\textit{many\textbf{i}la}\>(\degree many-ɪl-a)\>`learn'\> < \textit{manya} \lq know'\\
\textit{many\textbf{i}ka}\>(\degree many-ɪk-a)\>`be known'\\
\textit{many\textbf{i}sya}\>(\degree many-ɪsi-a)\>`teach'\\
\textit{kany\textbf{i}sya}\>(\degree kany-ɪsi-a)\>`fill up, stuff'\>< \textit{kanya}\>`tread on'\\
\textit{tuuny\textbf{i}la}\>(\degree tuuny-ɪl-a)\>`throw at'\> < \textit{tuunya}\>`throw'\\
\textit{tuuny\textbf{u}ka}\>(\degree tuuny-ʊk-a)\>`fall (from)'\\
\textit{kiny\textbf{u}la}\>(\degree kiny-ʊl-a)\>`disencourage'\> < \textit{kinya} \>`hit'
\end{tabbing}
\end{exe}

This rule does not apply when the vowel in question is affected by vowel height harmony\is{vowels!vowel height harmony} (\ref{exVowelPalatalRaisingBlockedByVHH}). Also, only the directly adjacent vowel is subject to high vowel raising (\ref{exVowelRaisingPalatalNoSpreading}).

\begin{exe}
\ex \label{exVowelPalatalRaisingBlockedByVHH} 
\begin{tabbing}
\textit{toony\textbf{e}sya}x\=`break, wreck; harvest corn'x\=< \textit{keenya}x\=\kill%Unsinnszeile fuer Tabulatoren
\textit{toony\textbf{e}sya}\>`cause to fall or drip'\> < \textit{toonya}\>`drip, ooze'\\ 
\textit{keeny\textbf{e}sya}\>`insult, shout at'\> < \textit{keenya}\>`insult'\\
\textit{kony\textbf{o}la}\>`break, wreck; harvest corn'\> < *\textit{kóny}\>`fold, bend, twist'\\
\textit{kony\textbf{o}ka}\>`break (intr.)'
\end{tabbing}
\ex\label{exVowelRaisingPalatalNoSpreading}
\begin{tabbing}
\textit{keeny\textbf{e}sya}x\=`break, wreck; harvest corn'x\=< \textit{keenya}x\=\kill%Unsinnszeile fuer Tabulatoren
\textit{many\textbf{i}l\textbf{ɪ}la}\>`know (much)'\> < \textit{manya}\>`know'\\
\textit{piny\textbf{i}l\textbf{ɪ}la}\>`tie up, bind repeatedly'\> < \textit{pinya}\>`bind; detain; fix'\\
\textit{keny\textbf{u}l\textbf{ɪ}la}\>`add too much; oversalt'
\end{tabbing}
\end{exe}

The underlying second degree vowels /ɪ, ʊ/ of verbal extensions are also realized as first degree /i, u/ when they follow a sequence of low vowel /a/ plus the coronal or bilabial nasals /n/ or /m/. Examples are given in (\ref{exHighVowelRaisingAfterReciprocalPositional}) for the extensions in question when following the \isi{reciprocal} and \isi{positional} extensions, while (\ref{exHighVowelRaisingAfteranmRootfinal}) illustrates this for root-final sequences.\footnote{Again, this seems to be the expression of a more general phonotactic constraint: Regardless of syntactic class, no stem containing /an, am/ followed by a second degree vowel is attested in the data.}

\begin{exe}
\ex \label{exHighVowelRaisingAfterReciprocalPositional}
\begin{tabbing}
\textit{swigan\textbf{i}ka}x\=(\degree swig-an-ɪk-a)x\=\lq turn upside down' \= < \textit{batama}x\=`be silent'\kill%unsinnige Zeile, fuer Tabs
\textit{koman\textbf{i}la}\>(\degree kom-an-ɪl-a)\>`fight for'\> < \textit{koma}\>`hit'\\
\textit{swigan\textbf{i}ka}\>(\degree swig-an-ɪk-a)\>`wonder (much)' \> < \textit{swiga}\>`wonder' \\
\textit{sulam\textbf{i}ka}\>(\degree sulam-ɪk-a)\>`turn upside down'\> < \textit{sulama}\>`bend, droop'\\
\textit{batam\textbf{i}sya}\>(\degree batam-ɪsi-a)\>`silence, caress'\> < \textit{batama}\>`be silent'
\end{tabbing}
\ex
\label{exHighVowelRaisingAfteranmRootfinal}
\begin{tabbing}
\textit{swigan\textbf{i}ka}x\=(\degree swig-an-ɪk-a)x\=`turn upside down' \= < \textit{batama}x\=`be silent'\kill%unsinnige Zeile, fuer Tabs
\textit{gan\textbf{i}la}\>(\degree gan-ɪl-a)\>`love + \textsc{appl}'\> < \textit{gana} \>`love'\\
\textit{gan\textbf{i}sya}\>(\degree gan-ɪsi-a)\>`cause to love'
\\\textit{kaan\textbf{i}ka}\>(\degree kaan-ɪk-a)\>`dispute'\> < \textit{kaana}\> `refuse'\\
\textit{kaan\textbf{i}la}\>(\degree kaan-ɪl-a)\>`refuse'\\
\textit{kaan\textbf{i}sya}\>(\degree kaan-ɪsi-a)\>`forbid'\\
\textit{kam\textbf{i}la}\>(\degree kam-ɪl-a)\>`milk + \textsc{appl}'\> < \textit{kama}\>`milk, squeeze'
\\ \textit{kam\textbf{u}la}\>(\degree kam-ʊl-a)\>`squeeze out'
\\ \textit{lam\textbf{u}la}\>(\degree lam-ʊl-a)\>`stop fight; judge'\> *\textit{dàm-ʊd}\>`settle dispute'
\\ \textit{saam\textbf{i}la}\>(\degree saam-ɪl-a)\>`move + \textsc{appl}'\> < \textit{saama}\>`move, migrate'
\\ \textit{saam\textbf{i}sya}\>(\degree saam-ɪsi-a)\>`transfer; displace'
\end{tabbing}
\end{exe}

Again, only the directly adjacent vowel undergoes raising:
\begin{exe}
\ex \label{exNoSpreadingAfterRaisinganmn}
\begin{tabbing}
\textit{saam\textbf{i}k\textbf{ɪ}sya}x\=(\degree fwan-ɪkɪsi-a)x\=`migrate + \textsc{intns}'x\= < \textit{sanuka}x\=`move, migrate'\kill%unsinnszeile, fuer tabulatoren
\textit{fwan\textbf{i}k\textbf{ɪ}sya}\>(\degree fwan-ɪkɪsi-a)\>`compare'\> < \textit{fwana}\>\lq resemble'
\\\textit{kaan\textbf{i}l\textbf{ɪ}la}\>(\degree kaan-ɪlɪl-a)\>`refuse +\textsc{intns}'\> < \textit{kaana}\>`refuse'
\\\textit{saam\textbf{i}k\textbf{ɪ}sya}\>(\degree saam-ɪkɪsi-a)\>`transfer; exile to'\> < \textit{saama}\>`move, migrate'
\\\textit{saam\textbf{i}l\textbf{ɪ}la}\>(\degree saam-ɪlɪl-a)\>`migrate + \textsc{intns}'
\\\textit{san\textbf{u}k\textbf{ɪ}la}\>(\degree sanuk-ɪl-a)\>`turn to'\> < \textit{sanuka}\>\lq alter'\\
\textit{am\textbf{u}l\textbf{ɪ}sya}\>(\degree amul-ɪsi-a)\>`make answer'\>< \textit{amula}\>`answer'
\end{tabbing}
\end{exe}

The examples in (\ref{exNoRaisingAC}) show that other /aC/ sequences do not induce raising of the second degree vowels.\footnote{A few combinations are not attested in the data: /aŋɪ, afɪ, afu/. The lack of the first is due to the scarcity of the velar nasal, while the lack of the other two stems from the fact that the bilabial fricative /f/ has its main diachronic source in sequences of \ili{Proto-Bantu} plosives followed by a first degree vowel.} The examples in (\ref{exNoRaisingVN}) show that other /VN/ sequences likewise do not induce high vowel raising (but see above on the effects of the palatal nasal, and below on the sequence /mu/).
\clearpage

\begin{exe}
\ex\label{exNoRaisingAC}
\begin{tabbing}
\textit{nangɪsya}x\=`be(come) unravelled'xxx\=\textit{nangɪsya}x\=\kill
/ap, at, ak/ \>\> /amb, and, aŋg/ \\
\textit{paapɪla}\>`give birth + \textsc{appl}' \> \textit{bambɪka}\>`arrange in line'\\
\textit{tapʊka}\>`separate'\>\textit{sambʊka}\>`rebel'\\
\textit{ʊbatɪla}\>`embrace'\>\textit{andɪsya}\>`establish; repeat'\\
\textit{latʊla}\>`rip' \> \textit{andʊla}\>`change; convert'\\
\textit{pakɪla}\>`load + \textsc{appl}' \> \textit{nangɪsya}\>`show'\\
\textit{sakʊka}\>`reappear' \> \textit{pangʊla}\>`dismantle'\\
/aβ̞, al, aɟ, aɰ/\>\> /as, ah/\\
\textit{laabɪla}\>`get up early; be early' \> \textit{lasɪla}\>`stab + \textsc{appl}'\\
\textit{abʊla}\>`release; open' \> \textit{pasʊka}\>`burst'\\
\textit{malɪka}\>`(be) finish(ed) (intr.)' \> \textit{hahɪla}\>\lq propose + \textsc{appl}'\\
\textit{saalʊka}\>`be(come) unravelled' \> /aŋ/\\
\textit{baajɪka}\>`kick + \textsc{appl}' \> \textit{kang'ʊla}\>`remove stopper'\\
\textit{tajʊka}\>`break up (intr.)' \\
\textit{bagɪla}\>\lq be able; suit'\\
\textit{pagʊka}\>`fall apart'
\end{tabbing}
%I have tried to solve the above with multicols, but had no success
\ex
\label{exNoRaisingVN}
\begin{tabbing}
\textit{nangɪsya}x\=\kill
/V$\neq$a\{n, m\}/\\
\textit{pɪmɪla}\>\lq measure + \textsc{appl}'\\
\textit{inɪsya}\>\lq dirty (tr.)'\\
\textit{inʊla}\>\lq lift'\\
\textit{timɪla}\>\lq rain + \textsc{appl}'\\
\textit{ʊmɪla}\>\lq dry + \textsc{appl}'
\end{tabbing}
\end{exe}

A formalized account of the rules of high vowel raising is given in (\ref{ruleVowelRaisingAfterPalatalNasal}, \ref{exRuleHigVowelRaisingAfteranm}).
\begin{exe}
\ex \label{ruleVowelRaisingAfterPalatalNasal}
\phonl{
	\oneof{
	\mbox{ɪ} \\
	\mbox{ʊ}}
	}
	{\oneof{
	\mbox{i} \\
	\mbox{u}}
	}
	{ny}
\ex \label{exRuleHigVowelRaisingAfteranm}
\phonl{
	\oneof{
	\mbox{ɪ} \\
	\mbox{ʊ}}
	}
	{\oneof{
	\mbox{i} \\
	\mbox{u}}}
{a\{n, m\}}
\end{exe}

Lastly, in a few stems the sequence /mu/ is found in contexts where it cannot be accounted for by the above rules (\ref{exMuStem}). This seems to be the expression of a general phonotactic constraint against /mʊ/.\footnote{This kind of constraint against certain CV sequences may be more widespread in Bantu. \citet{BennettWGLeeSJ2015} describe in detail how the sequence /li/ is strongly dispreferred in \ili{Tsonga} S53.} While the latter sequence may be the outcome of vowel coalescence between the vowel of a prefix and a vowel-initial stem (\sectref{HiatusSolution})\is{vowels!hiatus solution}, no stem or affix containing it is attested.\footnote{Note that this constraint alone cannot explain the raising of the front vowel /ɪ/ after \mbox{/a\{m, n\}/}, nor can it explain the raising of /ʊ/ after /an/, as /mɪ, nɪ, nʊ/ are licensed stem-internal syllables.}

\begin{exe}
\ex \label{exMuStem}
\begin{tabbing}
\textit{nyenyemusya}x\=`speak much'x\=\kill
\textit{lendemuka}\>`crack'
\\\textit{nyenyemusya}\>`excite'
\\\textit{telemuka}\>`slip'\> < *\textit{tèdɪmʊk} 
\\\textit{syelemuka}\>`slip'\> < *\textit{tɪ̀edɪmʊk}
\\\textit{tyemula}\>`sneeze'\> < *\textit{tɪ́emʊd}
\\\textit{tyesemula}\>`sneeze'
\end{tabbing}
\end{exe}

\is{vowels!high vowel raising|)}
\subsection{Causative 1} \label{Causative1} 
\is{causative|(}
The verbal suffix -\textit{i} serves to derive causative verbs. Before turning to a closer examination of its function, some formal aspects require discussion.

Synchronically speaking, the vowel of this extension is not directly observable; instead it surfaces as the glide /y/. It is interpreted as -\textit{i} because of the morphophonological changes it induces, which in diachronical terms go back to a first degree front vowel. Lingual plosives and approximants preceding this causative suffix are spirantized\is{spirantization} to /s/, while their labial counterparts change to /f/. This rule, which constitutes a typical case of Bantu \isi{spirantization} (see \citealt{BoestonK2008}), is given (\ref{RuleSpirantizationCaus1}) and illustrated in (\ref{exSpirantizationCaus1}).
\begin{exe}  
\ex Spirantization\is{spirantization} triggered by causative -\textit{i}\label{RuleSpirantizationCaus1}
\begin{xlist}
\ex \phonr{\{ t, l, j, k, g \}}{s}{i}
\ex \phonr{\{ p, b \}}{f}{i}
\end{xlist}
\ex\label{exSpirantizationCaus1} 
\begin{tabular}[t]{llll}
\textit{bosya}&`cause to rot'& < \textit{bola}&`rot'\\
\textit{sesya}&`make laugh'& < \textit{seka}&`laugh'\\
\textit{osya}&\lq bathe (tr.); baptize'& < \textit{oga}&`bathe (intr.)'\\
\textit{pyʊfya}&`warm, heat up'& < \textit{pyʊpa}&`get warm'\\
\textit{sofya}&`loose; mislead'& < \textit{soba}&`be lost; be wrong'
\end{tabular}
\end{exe}

\largerpage %long distance for 'moo'
When prenasalized plosives are spirantized, the preceding vowel becomes long (\ref{exCausative1NC}).\footnote{This can be analyzed either as retention of the compensatory lengthening triggered by the NC cluster or as subsequent deletion of the word-internal non-syllabic nasal plus compensatory lengthening, cf. the first person singular object prefix\is{object marker} (\sectref{SubjectConcordsParticipants}).} The \isi{causative} -\textit{i} followed by passive\is{passive!productive} -\textit{igw} surfaces with a short vowel;\is{vowels!length} see p.\nobreakspace\pageref{exCausativePassiveShortVowel} in \sectref{Passive}.

\begin{exe}
\ex\label{exCausative1NC}
\begin{tabbing}
\textit{joosya}x\=\lq make pass, pass through; allow'x\= < \textit{kɪnda}x\=\kill %nonsense line for tabs
\textit{kɪɪsya} \>\lq make pass, pass through; allow' \> < \textit{kɪnda} \> \lq pass'\\
\textit{joosya} \>`elope with girl; lose' \> < \textit{jonga} \> `run away'
\end{tabbing}
\end{exe}

When serving as a typical causative, this extension increases the valency of the verb by one. It introduces an agent that causes the act of the underlying verb and demotes the original subject to an object. The following examples illustrate this.
\begin{exe}
\ex
\begin{tabular}[t]{llll}
\textit{elʊsya} & \lq clean, rinse' & < \textit{elʊka} & \lq become white, clean'\\
\textit{fulasya} & \lq hurt (tr.)' & < \textit{fulala} & \lq (be)come hurt'\\
\textit{isʊsya} & \lq fill' & < \textit{isʊla} & \lq (be)come full'\\
\textit{sumusya} & \lq make stand up, get up' & < \textit{sumuka} & \lq get up, depart'\\
\textit{kusya} & \lq blow away' & < \textit{kula} & \lq blow (intr.), drift'
\end{tabular}
\end{exe}

Causative -\textit{i} has developed idiosyncratic readings with a number of verbs, but it is no longer productive in the present-day language. These issues are dealt with in more detail in \sectref{TwoCausatives}.
\subsection{Causative 2}\label{Causative2}
The extension -\textit{ɪsi} (allomorphs -\textit{esi}, -\textit{isi}, see \sectref{MorphophonologyOfVerbalExtension},) serves to derive causative verbs. \mbox{-\textit{ɪsi}} is the only causative extension used with monosyllabic verbs and verbs ending in the palatal nasal (\ref{exCaus2MonoPalatal}). It is also the only productive causative in Nyakyusa; see \sectref{TwoCausatives} for discussion.

\begin{exe}
\ex\begin{xlist}
\ex \label{exCaus2MonoPalatal}Causatives of monosyllabic verbs:
\begin{tabbing}
\textit{kuunya}x\=`bind; detain; fix'x\= > \textit{kunuyisya}x\=\kill%Unsinnszeile für Tabulatoren 
\textit{gwa}\>`fall'\> > \textit{gwɪsya}\>`overturn, throw down'\\
\textit{lwa}\>`fight'\> > \textit{lwɪsya}\>`cause to fight'\\
\textit{nwa}\>`drink'\> > \textit{nwesya}\>`make drink, water'\\
\textit{lya}\>`eat'\> > \textit{lɪɪsya}\>`feed'\\
\textit{sya}\>`grind'\> > \textit{syesya}\>`make grind'%needs checking
%evtl weiteres bsp
\end{tabbing}
\ex Causatives of bases ending in -\textit{ny}:
\begin{tabbing}
\textit{kuunya}x\=`bind; detain; fix'x\= > \textit{kuunyisya}x\=\kill%Unsinnszeile für Tabulatoren 
\textit{manya}\>`know'\> > \textit{manyisya}\>`teach'\\
\textit{pinya}\>`bind; detain; fix`\> > \textit{pinyisya}\>`make bind'\\
\textit{kuunya}\>`push, bump'\> > \textit{kuunyisya}\>`make push, bump'
\end{tabbing}
\end{xlist}
\end{exe}

The suffix -\textit{ɪsi} may be analysed as consisting of two morphemes -\textit{ɪs}-\textit{i}. In combination with the reciprocal/associative\is{reciprocal} it often surfaces as -\textit{ɪs}-\textit{an}-\textit{i}; see also \sectref{ReciprocalAndCausative}. Also note that any causative followed by the passive \mbox{-\textit{igw}}\is{passive!productive} surfaces with a short vowel;\is{vowels!length} see p.\nobreakspace\pageref{exCausativePassiveShortVowel} in \sectref{Passive}.
\begin{exe}
\ex
\begin{tabular}[t]{llll}
\textit{lwɪsania}&`make fight each other'& < \textit{lwa}&`fight'\\
\textit{sobesania}&`loose each other'& < \textit{soba} & \lq be lost'
\end{tabular}
\end{exe}

Causative\textsubscript{2} -\textit{ɪsi} increases the valency of the verb by one, introducing an agent-causer and demoting the original subject to an object. See (\ref{exShortCausativeLexicalized}--\ref{exCausative2Productive}) in \sectref{TwoCausatives} for numerous examples. Causative\textsubscript{2} can further be used to add an intensive, evaluative meaning without changing the verb's argument structure (\ref{exCaus2Intensifying1}, \ref{exCaus2Intensifying2}). Such an intensifying use of the causative has also been reported for neighbouring \ili{Ndali} \citep[73f]{BotneR2003} and other Bantu languages such as \ili{Chewa} N20 \citep[78f]{IntensiveChichewa}, \ili{Bemba} M42 \citep[83--92]{vanSambeekJ1955} and \ili{Kalanga} S16 \citep[397]{MathangwaneJ2001}. For a typological perspective see \citet{KittilaeS2009}.
\begin{exe}
\ex \label{exCaus2Intensifying1}\gll i-kʊ-mmw-amul-ɪsy-a\\
1-\textsc{prs}-1-answer-\textsc{caus}-\textsc{fv}\\
\glt `1. S/he makes him/her answer.'\\
`2. S/he answered him/her snottily.' [ET]
\ex \label{exCaus2Intensifying2}\gll i-kʊ-ba-hah-ɪsy-a a-ba-kiikʊlʊ\\
1-\textsc{prs}-2-persuade-\textsc{caus}-\textsc{fv} \textsc{aug}-2-woman\\
\glt `He goes around proposing to women.' [ET]
\end{exe}

\subsection{The relationship between the two causatives}
\label{TwoCausatives}
As was seen in the preceding sections, Nyakyusa has two causative morphemes: -\textit{i} and -\textit{ɪsi}. Their distribution is partly conditioned by phonology. Only -\textit{ɪsi} applies with monosyllabic verbs and following /ɲ/. These phonological contexts aside, in the variety described by \citet{SchumannK1899} and \citet{EndemannC1914}, causative\textsubscript{1} -\textit{i} figures as the more productive morpheme of the two. For the present-day language, however, \citet{LabroussiC1999} observes that causative\textsubscript{2} -\textit{ɪsi} has widely replaced causative\textsubscript{1} -\textit{i}. Labroussi's observations are corroborated in the data. First, in a number of cases, causative\textsubscript{1} -\textit{i} is lexicalized with idiosyncratic meanings, whereas causative\textsubscript{2} -\textit{ɪsi} yields a more transparent meaning and syntax. (\ref{exShortCausativeLexicalized}) illustrates a few of these.

\begin{exe}
\ex \label{exShortCausativeLexicalized}
\begin{tabbing}
\textit{taama}x\=\lq recover; escape'x\= > \textit{taamisya}x\=\kill %nonsense line for tabs
\textit{kʊba} \> \lq beat; ring; play' \> > \textit{kʊfya} \> \lq cause trouble' not: \lq make beat'\\
\> \> > \textit{kʊbɪsya} \> \lq make beat, ring, play'\\
\textit{oga} \> \lq bathe (intr.)' \> > \textit{osya} \> \lq bathe (tr.); baptize'\\
\> \> > \textit{ogesya} \> \lq bathe (tr.)' not: \lq baptize'\\
\textit{pona} \> \lq recover; escape' \> > \textit{ponia} \> \lq greet, visit' not: \lq cure'\\
\> \> > \textit{ponesya} \> \lq cure, rescue'\\
\textit{syala} \> \lq remain' \> > \textit{syasya} \> \lq leave over' not: \lq make remain'\\
\> \> > \textit{syalɪsya} \> \lq make remain'\\
\textit{taama} \> \lq moo' \> > \textit{taamya} \> \lq trouble; persecute'\\
\>\>\>not: \lq make moo'\\
\> \> > \textit{taamisya} \> \lq make moo'
\end{tabbing}
\end{exe}

In other cases, both causatives are attested without any apparent difference in meaning. With most of these, there is a preference for the long causative\textsubscript{2}. However, with a few verbs, the short causative\textsubscript{1} is strongly preferred or is the only acceptable form (\ref{exPreferenceCaus1}). The data at hand suggest that this kind of lexicalization is particularly the case with verbs featuring the separative intransitive (\sectref{Separative}) and the extensive (\sectref{extensive}).

\begin{exe}
\ex \label{exPreferenceCaus1}\begin{tabbing}
\textit{hoboka}x\=\lq be(come) happy'x\= > \textit{hobokesya}x\=\kill %nonsene line for tabs
\textit{fulala}\> \lq be(come) hurt'\> > \textit{fulasya} \> \lq hurt (tr.)'\\
\textit{hoboka}\> \lq be(come) happy'\> > \textit{hobosya} \> \lq amuse\\
\> \> \hphantom{> }*\textit{hobokesya}\\
\textit{kalala}\> `be(come) angry' \> > \textit{kalasya} \> \lq enrage'\\
\> \> \hphantom{> }*\textit{kalalɪsya}\\
\textit{lɪla}\>i.a. \lq cry' \> > \textit{lɪsya} \> \lq make cry' (preferred)\\
\> \> > \textit{lɪlɪsya} \> \lq make cry'\\
\textit{pyʊpa}\> i.a. \lq get warm' \> > \textit{pyʊfya} \> \lq heat, warm up'\\
\> \> \hphantom{> }*\textit{pyʊpɪsya}
\end{tabbing} 
\end{exe}

Furthermore, causative\textsubscript{2} -\textit{ɪsi} is the only one which is applied productively (\ref{exCausative2Productive}). Causative\textsubscript{1} \mbox{-\textit{i}} was rejected with most roots, including a number of those listed by \citet{MeinhofC1966}[1910], \citet{SchumannK1899}, and \citet{EndemannC1914}.\footnote{\textit{hobokesya} is acceptable nevertheless as the applicativized causative. \citet{FelbergK1996} lists \textit{kalalɪsya} for the variety of the lake-shore plains,\is{dialects} so topological differences might also come into play. Some of these forms exist as causatives of other verbs: \textit{baasya} < \textit{baala} \lq increase, thrive', \textit{pɪɪsya} < \textit{pya} \lq be(come) burnt'. \citet{MwangokaNVoorhoeveJ1960c} list \textit{keesya} < \textit{keeta}, which was rejected by the speakers consulted for this study.} 

\begin{exe}
\ex \label{exCausative2Productive}
\begin{tabbing}
\textit{baaja}x\= \lq be(come) satisfied\lq x\= > \textit{keetesya}x \= \lq make cultivate'x\= \kill %Unsinnszeile für Tabulatoren
\textit{baaja} \> \lq kick' \> > \textit{baajɪsya} \> \lq make kick' \> not: *\textit{baasya}\\
\textit{bona} \> \lq see' \> > \textit{bonesya} \> \lq show' \> not *\textit{bonia}\\%ist auch eins aus alten Quellen
\textit{funja} \> \lq harvest' \> > \textit{funjɪsya} \> \lq make harvest' \> not: *\textit{fuusya}\\%dito
\textit{ikʊta} \> \lq be(come) satisfied' \> > \textit{ikʊtɪsya} \> \lq satisfy' \> not: *\textit{ikʊsya}\\
\textit{jaata} \> \lq walk' \> > \textit{jaatɪsya} \> \lq make walk' \> not: *\textit{jaasya}\\
\textit{jenga} \> \lq build' \> > \textit{jengesya} \> \lq make build' \> not: *\textit{jeesya}\\
\textit{keeta} \> \lq look, watch' \> > \textit{keetesya} \> \lq make look' \> not: *\textit{keesya}\\%ist eins von denen von endemann,schumann
\textit{lɪma} \> \lq cultivate' \> > \textit{lɪmɪya} \> \lq make cultivate' \> not: *\textit{lɪmya}\\
\textit{pɪɪja} \> \lq cook' \> > \textit{pɪɪjɪsya} \> \lq make cook' \> not: *\textit{pɪɪsya}
\end{tabbing}
\end{exe}

Lastly, causative\textsubscript{2} -\textit{ɪsi} is found with the same function as causative\textsubscript{1} -\textit{i} in the derivation of pluractionals\is{pluractional} (see \sectref{Pluractional}). It is also subject to the same templatic requirements as -\textit{i} in the formation of applicativized\is{applicative} causatives (\sectref{ApplicativizedCausatives}) and perfective stems (\sectref{Imbrication}),\is{imbrication} both of which can be diachronically traced back to \isi{spirantization} triggered by causative\textsubscript{1} \textit{-i}.

\citet{LabroussiC1999} regards the loss of productivity of causative\textsubscript{1} -\textit{i} as an epiphenomenon of the more general decline of \isi{spirantization} in Nyakyusa, which is also observed in the agent noun suffix -\textit{i}. The latter is lexicalized with spirantizing forms,\is{spirantization} but does not productively cause consonant mutation. What Labroussi does not consider is the fossilization of passive\is{passive!fossilized} *-\textit{ʊ} (\sectref{FossilizedPassive}), which has been replaced by the reflex of PB *-\textit{ibʊ} (\sectref{Passive}).\is{passive!productive} The alternation between these two was historically also triggered by phonological context \citep[78]{SchadebergT2003a}. Thus, apart from a general decline of spirantization,\is{spirantization} present-day Nyakyusa also shows a general preference for suffixes of the shape -VCV over -V.
\is{causative|)}
\subsection{Reciprocal/Associative}
\label{Reciprocal}\is{reciprocal|(}
The reciprocal/associative extension has the shape -\textit{an}. With some roots, this extension surfaces with a long vowel,\is{vowels!length} which is maintained in complex derivations (\sectref{CombinationOfVerbalExtensions}). The long allomorph is induced predominantly, but not exhaustively, by roots\is{root} of the shape (C)VNC.

\begin{exe}
	\ex \label{exReciprocalLong}
	Long reciprocals -\textit{aan}:
	\begin{xlist}
		\ex Following (C)VNC:
		\begin{tabbing}
			\textit{jʊʊgaanika}x\=\lq kill each other with blade'x\= < *\textit{bʊnga}x\=\kill%Unsinnszeile fuer Tabulatoren
			\textit{bʊngaana}\>`gather, be assembled'\> < *\textit{bʊ́nga} `gather up'\\
			\textit{lɪngaania}\>\lq explain; narrate'\> < \textit{lɪnga}\>`peep through'\\
			\textit{lʊngaana}\>`join together (intr.)'\>< \textit{lʊnga}\>`join; add spice'\\
			\textit{ongaana}\>`be together, be mixed'\> < \textit{onga}\>`suck'\\
			\textit{sangaana}\>\lq kill each other with blade'\> < \textit{sanga}\>\lq slaughter'\\
			\textit{tengaana}\>\lq settle; live in peace'\> < \textit{tenga}\>`make bed'
		\end{tabbing}
		\ex Following CV(V)C:
		\begin{tabbing}
			\textit{jʊʊgaanika}x\=\lq kill each other with blade'x\= < *\textit{bʊnga}x\=\kill%Unsinnszeile fuer Tabulatoren
			\textit{jʊʊgaania}\>`shake, aggitate'\\
			\textit{jʊʊgaanika}\>`tremble'\\
			\textit{kolaana}\>`grasp, hold e.o.; accuse e.o.'\> < \textit{kola}\>\lq grasp, hold' \\
			\textit{kolaania}\>`multitask'\\
			\textit{lekaana}\>`leave e.o.; release e.o.'\> < \textit{leka}\>\lq release; let'\\
			\textit{manyaana}\>`be(come) acquainted'\> < \textit{manya}\>`know'
		\end{tabbing}
	\end{xlist}
\end{exe}


However, not all (C)VNC roots\is{root} have the long allomorph (\ref{exReciprocalShortNC}). Further, one minimal pair -\textit{an} vs. -\textit{aan} is attested (\ref{exMPReciprocalsLongShort}).\is{vowels!length}
\pagebreak
\begin{exe}
	\ex \label{exReciprocalShortNC} Short reciprocal -\textit{an} following (C)VNC:
	\begin{tabbing}
		\textit{komaana}x\=`follow e.o.; depend upon e.o.'x\= < \textit{manya}x\=\kill%unsinnszeile fuer tabulatoren
		\textit{andana}\>`be early'\> < \textit{anda}\>`start'\\
		\textit{kongana}\>`follow e.o.; depend upon e.o.'\> < \textit{konga}\>\lq follow'\\
		\textit{ningana}\>`give e.o.; be opposite e.o.'\> < \textit{ninga}\>`give'\\
		\textit{pɪngana}\>`debate, disagree'\> < \textit{pɪnga}\>`obstruct'
	\end{tabbing}
	\ex \label{exMPReciprocalsLongShort}
	\begin{tabbing}
		\textit{komaana}x\=`follow e.o.; depend upon e.o.'x\= < \textit{manya}x\=\kill%unsinnszeile fuer tabulatoren
		\textit{komana}\>`fight'\> < \textit{koma}\>`hit' \\
		\textit{komaana}\>`hold a meeting'
	\end{tabbing}
\end{exe}

Having broached the formal issues of the reciprocal/associative, its function can now be examined. The reciprocal/associativ is commonly used with transitive verbs and yields a reciprocal action. The verb's valency in this case decreases by one.

\begin{exe}
	\ex
	\begin{tabbing}
		\textit{tʊʊlana}x\=\lq divide amongst each other'x\= < \textit{tʊʊla}x\=\kill%unsinnszeile fuer tabulatoren
		\textit{sekana} \> \lq make fun of each other' \> < \textit{seka} \> \lq laugh (at)'\\
		\textit{tʊʊlana} \> \lq help each other' \> < \textit{tʊʊla} \> \lq help'\\
		\textit{titana}\> \lq pinch each other' \> < \textit{tita} \> \lq pinch'
	\end{tabbing} 
\end{exe}

Another reading of the reciprocal/associative is that of a joint action. Accordingly, valency remains unchanged:

\begin{exe}
	\ex\begin{tabbing}
		\textit{tʊʊlana}x\=\lq divide amongst each other'x\= < \textit{tʊʊla}x\=\kill%unsinnszeile fuer tabulatoren
		\textit{jabana}\>\lq divide amongst each other' \> < \textit{jaba} \> \lq divide; distribute'\\
		\textit{gonana}\>\lq sleep together, copulate' \> < \textit{gona} \> \lq rest, sleep'
	\end{tabbing}
\end{exe}

A closer look at examples (\ref{exReciprocalLong}, \ref{exReciprocalShortNC}) above shows that the reciprocal/associative often expresses a further range of related meanings in the area of middle voice \citep{KemmerS1993}. A preliminary classification, including some of the above examples, is given in the following:
\begin{exe}
	\ex\label{exReciprocalMiddle}\begin{xlist}
		\ex Verbs of being (dis-)connected:
		\begin{tabbing}
			\textit{bulungana}x\=\lq be(come) quiet; settle'x\= < \textit{bulunga}x\=\kill%Unsinnszeile für Tabulatoren
			\textit{pangʊkana}\>`break apart'\> < \textit{pangʊka}\>`collapse'\\
			\textit{ongaana}\>`be together, be mixed'\> < \textit{onga}\>`suck'
		\end{tabbing}
		\ex Collective eventualities:
		\begin{tabbing}
			\textit{bulungana}x\=\lq be(come) quiet; settle'x\= < \textit{bulunga}x\=\kill%Unsinnszeile für Tabulatoren
			\textit{bʊngaana}\>`gather, be assembled'\> < *\textit{bʊ́nga}\>`gather up'\\
			\textit{palamana}\>`be close to each other'\\
			\textit{papatana}\>`be squeezed together'\> < *\textit{pát}\>`hold'
		\end{tabbing}
		\ex Chaining relation:
		\begin{tabbing}
			\textit{bulungana}x\=\lq be(come) quiet; settle'x\= < \textit{bulunga}x\=\kill%Unsinnszeile für Tabulatoren
			\textit{kongana}\>\lq follow each other'\>< \textit{konga} \>\lq follow'
		\end{tabbing}
	
	\clearpage
	
		\ex Intransitive, resultative:\footnotemark
		\begin{tabbing}
			\textit{bulungana}x\=\lq be(come) quiet; settle'x\= < \textit{bulunga}x\=\kill%Unsinnszeile für Tabulatoren
			\textit{andana}\>`be early'\> < \textit{anda}\>`start'\\
			\textit{bulungana}\>`be(come) round'\> < \textit{bulunga}\>`roll up; knead'\\
			%\textit{pɪndana}\>`wrinkle'\> < \textit{pɪnda}\>`bend; wrap'\\%check vowel length and meaning
			\textit{tengaana}\>\lq be(come) quiet; settle'\> < \textit{tenga}\>`make bed'
		\end{tabbing}
	\end{xlist}
\end{exe}
\protect\footnotetext{\citet{BotneR2008}, observing that a number of verbs with -(\textit{a})\textit{an} in \ili{Ndali} are aspectually inchoative and have a resultative meaning, stipulates a homophonous ``resultative'' extension.}

The majority of these readings include a plurality of participants. Thus the grammatical subject is often expressed as plural (\ref{exReciprocalPluralSubjekt1}).
Depending on the specific verb and on context, other strategies are also encountered. Two conjoined noun phrases may form a plural subject (\ref{exReciprocalPluralConjoinedSubject}). Lastly, conjoined subjects may be expressed discontinuously. In this case, the corresponding plural may be cross-referenced on the verb (\ref{exReciprocalNaSubject1}).
Alternatively, the first noun phrase or a participant in the context (\ref{exReciprocalNaSubject2}) may be cross-referenced. This latter strategy is attested much less frequently in the data.

\begin{exe}
	\ex\label{exReciprocalPluralSubjekt1}
	\gll ɪ-fi-nyamaana ɪ-fi \textbf{fy}-\textbf{a}-\textbf{many}-\textbf{eene} fiijo n=ɪ-m-bombo sy-abo sy-osa \textbf{fy}-\textbf{a}-\textbf{tʊʊl}-\textbf{an}-\textbf{aga}\\
	\textsc{aug}-8-animal \textsc{aug}-\textsc{prox.8} 8-\textsc{pst}-know-\textsc{recp.pfv} \textsc{intens} \textsc{com}=\textsc{aug}-10-work 10-\textsc{poss.pl} 10-all 8-\textsc{pst}-help-\textsc{recp}-\textsc{ipfv}\\
	
	\glt \lq These animals were close friends and helped each other with everything.' [Hare and Chameleon]
	
	
	\ex\label{exReciprocalPluralConjoinedSubject}
	\gll kalʊlʊ \textbf{n}=\textbf{ʊ}-\textbf{lʊ}-\textbf{bʊbi} \textbf{ba}-\textbf{lɪnkʊ}-\textbf{job}-\textbf{an}-\textbf{a} kʊ-kʊ-mwanya kʊ-m-piki\\
	hare(1) \textsc{com}=\textsc{aug}-11-spider 2-\textsc{narr}-speak-\textsc{recp}-\textsc{fv} 17-17-high 17-3-tree\\
	\glt \lq Hare and Spider talked high in the tree.' [Hare and Spider]
	\ex\label{exReciprocalNaSubject1} \gll paapo \textbf{ba}-\textbf{al}-\textbf{iitɪk}-\textbf{eene} \textbf{na} kalʊlʊ ʊ-kʊ-bop-a a-ma-eli ma-haano\\
	because 2-\textsc{pst}-agree-\textsc{recp.pfv} \textsc{com} hare(1) \textsc{aug}-15-run-\textsc{fv} \textsc{aug}-6-mile(<EN) 6-five\\
	\glt \lq Because they (Tugutu and Hare) had agreed to run five miles.' [Hare and Tugutu]
	\ex\label{exReciprocalNaSubject2}
	\gll jʊ-la \textbf{i}-\textbf{kw}-\textbf{itɪk}-\textbf{an}-\textbf{a} \textbf{na}=nuuswe\\
	1-\textsc{dist} 1-\textsc{prs}-agree-\textsc{recp}-\textsc{fv} \textsc{com}=\textsc{com.1pl}\\
	\glt \lq That one agrees with us.' [ET]
\end{exe}
\is{reciprocal|)}
\subsection{Applicative}
\is{applicative|(}The applicative extension, also called \textit{dative} \citep{SchadebergT2003a}, has the underlying shape \mbox{-\textit{ɪl}}. Allomorphs are \mbox{-\textit{el}}, \mbox{-\textit{il}}; see \sectref{MorphophonologyOfVerbalExtension}. In its most productive use, the applicative increases the valency of the verb by one. The semantic roles of the additional argument can be grossly classified as being beneficiary (\ref{exApplicativeBeneficient1}), location or direction/goal (\ref{exApplicativeDirection}), instrument (\ref{exApplicativeInstrument1}), manner (\ref{exApplicativeManner1}) or reason (\ref{exApplicativeReason1}).\is{semantic roles}

\begin{exe}
\ex\label{exApplicativeBeneficient1}Beneficiary:\\
\gll bo g-ʊʊl-ile ʊ-lond-e ʊ-mu-ndʊ ʊ-gw-a \textbf{kʊ}-\textbf{kʊ}-\textbf{jeng}-\textbf{el}-\textbf{a}\\
as \textsc{2sg}-buy-\textsc{pfv} \textsc{2sg}-search-\textsc{subj} \textsc{aug}-1-person \textsc{aug}-1-\textsc{assoc} 15-\textsc{2sg}-build-\textsc{appl}-\textsc{fv}\\
\glt \lq When you have bought one [place to build], you should look for a person to build for you.' [How to build modern houses]

\ex\label{exApplicativeDirection}Direction:\\
\gll ba-lɪnkw-igʊl-a ʊ-tʊ-supa tʊ-la \textbf{n}=\textbf{ʊ}-\textbf{kʊ}-\textbf{si}-\textbf{sop}-\textbf{el}-\textbf{a} ɪ-n-gambɪlɪ\\
2-\textsc{narr}-open-\textsc{fv} \textsc{aug}-13-bottle 13-\textsc{dist} \textsc{com}=\textsc{aug}-15-10-throw-\textsc{appl}-\textsc{fv} \textsc{aug}-10-monkey\\
\glt \lq They opened those little bottles and threw them at the monkeys.' [Thieving monkeys]

\ex \label{exApplicativeInstrument1} Instrument:\\
\gll ʊ-n-nyambala i-kʊ-lond-igw-a ʊ-kʊ-j-a n=ii-kʊmbʊlʊ ly-ake ɪ-ly-a \textbf{kʊ}-\textbf{lɪm}-\textbf{ɪl}-\textbf{a}. ɪ-n-gwego ɪ-j-aa \textbf{kʊ}-\textbf{las}-\textbf{ɪl}-\textbf{a}. ɪɪ-sengo ɪ-j-aa \textbf{kʊ}-\textbf{seng}-\textbf{el}-\textbf{a} \ldots\\
\textsc{aug}-1-man 1-\textsc{prs}-want-\textsc{pass}-\textsc{fv} \textsc{aug}-15-be(come)-\textsc{fv} \textsc{com}=5-hoe 5-\textsc{poss.sg} \textsc{aug}-5-\textsc{assoc} 15-farm-\textsc{appl}-\textsc{fv} \textsc{aug}-9-spear \textsc{aug}-9-\textsc{assoc} 15-stab-\textsc{appl}-\textsc{fv} \textsc{aug}-sickle(9) \textsc{aug}-9-\textsc{assoc} 15-chop-\textsc{appl}-\textsc{fv}\\
\glt \lq A man is required to have his hoe for farming with. A spear for stabbing. A sickle for clearing with \ldots' [Types of tools in the home]

\ex \label{exApplicativeManner1} Manner:\footnotemark\\
\gll ʊ-swe tʊ-ka-pɪliike a-ka-jʊni a-ka mu-no \textbf{ki}-\textbf{kw}-\textbf{ɪmb}-\textbf{ɪl}-\textbf{a}\\
\textsc{aug}-\textsc{1pl} \textsc{1pl}-12-hear.\textsc{pfv} \textsc{aug}-12-bird \textsc{aug}-\textsc{prox.12} 18-\textsc{dem} 12-\textsc{prs}-sing-\textsc{appl}-\textsc{fv}\\
\glt \lq We have heard how the little bird is singing.' [Man and his in-law]

\footnotetext{The expression of manner is a common extension of \isi{locative} class 18.}

\ex \label{exApplicativeReason1} Reason:\\
\gll Pakyɪndɪ a-alɪ-n-kalal-\textbf{iile} fiijo ʊ-n-kasi kʊ-m-bombo ɪ-si a-a-si-bomb-ile ɪ-li-sikʊ lɪ-la\\
P. 1-\textsc{pst}-1-be(come)\_angry-\textsc{appl.pfv} \textsc{intens} \textsc{aug}-1-wife 17-10-work \textsc{aug}-\textsc{prox.10} 1-\textsc{pst}-10-do-\textsc{pfv} \textsc{aug}-5-day 5-\textsc{dist}\\
\glt \lq Pakyindi got very angry with his wife for what she had done that day.' [Sokoni and Pakyindi]
\end{exe}


An applicative is sometimes found with an indefinite \isi{locative} or directional meaning \lq somewhere; someplace', as in (\ref{exApplicativesLocationalDirectionalBroadText1}--\ref{exApplicativesLocationalDirectionalBroadText3}).


\begin{exe}
\ex \label{exApplicativesLocationalDirectionalBroadText1}
\gll kʊ-m-malɪɪkɪsyo ɪɪ-sofu jɪ-lɪnkw-igal-a ʊ-lʊ-komaano. po ɪ-fi-nyamaana fy-osa \textbf{fi}-\textbf{lɪnkʊ}-\textbf{bal}-\textbf{an}-\textbf{il}-\textbf{a}\\
17-3-end \textsc{aug}-elephant(9) 9-\textsc{narr}-close-\textsc{fv} \textsc{aug}-11-meeting then \textsc{aug}-8-animal 8-all 8-\textsc{narr}-shine-\textsc{recp}-\textsc{appl}-\textsc{fv}\\
\glt \lq Finally Elephant closed the meeting. All the animals dispersed.' [Hare and Chameleon]
\ex \label{exApplicativesLocationalDirectionalBroadText2}
\gll \textbf{jɪ}-\textbf{lɪnkw}-\textbf{ag}-\textbf{an}-\textbf{il}-\textbf{a} n=ɪ-m-bwa ɪ-jɪ jɪ-l-iile ɪɪ-nyama\\
9-\textsc{narr}-find-\textsc{recp}-\textsc{appl}-\textsc{fv} \textsc{com}=\textsc{aug}-9-dog \textsc{aug}-\textsc{prox.9} 9-eat-\textsc{pfv} \textsc{aug}-meat(9)\\
\glt \lq He [somewhere on his way] met the dog that had eaten the meat.' [Dogs laughed at each other]
\ex \label{exApplicativesLocationalDirectionalBroadText3}
\gll kɪ-laabo ɪ-kɪ-ngɪ Sokoni a-lɪnkʊ-bʊʊk-a kangɪ \textbf{kʊ}-\textbf{kʊ}-\textbf{kʊng}-\textbf{ɪl}-\textbf{a} ɪɪ-ng'ombe\\
7-tomorrow \textsc{aug}-7-other S. 1-\textsc{narr}-go-\textsc{fv} again 17-15-tie-\textsc{appl}-\textsc{fv} \textsc{aug}-cow(10)\\
\glt \lq The next day Sokoni went again to tie the cows [someplace].' [Sokoni and Pakyindi]
\end{exe}

In many verbs the applicative has become lexicalized with a divergent meaning, with or without an increase in valency:
\begin{exe}
\ex \begin{tabular}[t]{llll}
\textit{angalɪla} & \lq tease, mock' & < \textit{angala} & i.a. \lq be well, converse'\\
\textit{ɪmɪla} & \lq preside' & < \textit{ɪma} & \lq stand, stop'\\
\textit{lagɪla} & \lq enforce; dictate; rule' & < \textit{laga} & \lq bid farewell'\\
\textit{manyila} & \lq learn' & < \textit{manya} & \lq know'\\
\textit{sookela} & \lq appear at; happen' & < \textit{sooka} & \lq leave'\\
\end{tabular}
\end{exe}
\is{applicative|)}

\subsection{Productive Passive}\label{Passive}
\is{passive!productive|(}
The productive passive extension has the shape -\textit{igw}. While the category of passive rather belongs to the inflectional than the derivational domain, it is discussed at this place because it fills the pre-final slot of the verb template, just as other verbal extensions.
\begin{exe}
\ex\begin{tabular}[t]{>{\itshape}llll} 
bomba &`do; work'& > \textit{bombigwa}&`be done; be worked'
\\ega & `take; marry'& > \textit{egigwa}&`be taken; be married'
\\senga &`chop' & > \textit{sengigwa}&`be chopped'
\end{tabular}
\end{exe}

When the passive follows one of the \isi{causative} extensions, the resultant vowel remains short:\is{vowels!length}
\begin{exe}
\ex \label{exCausativePassiveShortVowel}
\begin{tabular}[t]{lll} 
\textit{fumus\textbf{i}gwa} & (\degree fum-ʊk-i-igw-a) & `be announced'\\
\textit{sof\textbf{i}gwa} & (\degree sob-i-igw-a) & `be mislead'\\
\textit{taam\textbf{i}gwa} & (\degree taam-i-igw-a) & `be troubled'\\
\textit{manyis\textbf{i}gwa}& (\degree many-ɪsi-igw-a) & `be taught sthg.'
\end{tabular}
\end{exe}

The passive of monosyllabic verbs is formed by inserting -\textit{ɪl}/-\textit{el} between the \isi{root} and the passive extension. See \sectref{MorphophonologyOfVerbalExtension} on vowel alternations.
\begin{exe}
\ex
\begin{tabular}[t]{lll} 
\textit{pa}&`give'& > \textit{peeligwa}\\
\textit{mwa}&`shave'& > \textit{mweligwa}\\
\textit{nwa}&`drink'& > \textit{nweligwa}\\
\textit{lya}&`eat'& > \textit{lɪɪligwa}\\
\textit{sya}&`grind'& > \textit{syeligwa}
\end{tabular}
\end{exe}

For \textit{pa} \lq give', a variant \textit{peegwa} exists, while for \textit{lya} \lq eat' there is a variant form \textit{lɪɪgwa} (see \sectref{ImbricationDisyllabic} for the perfective stems\is{stem} of these verbs).\footnote{According to \citet[212f]{BergerP1938}, \textit{lɪɪgwa} is semantically restricted to human beings being eaten by beasts of prey. This could not be confirmed by the present data and is contradicted by \citeauthor{BergerP1933}'s (\citeyear{BergerP1933}) own text collection, where on page 122 it is found in reference to beans being eaten by birds.} The speakers consulted preferred the regular forms \textit{peeligwa} and \textit{lɪɪligwa}, which are also the only ones found in the textual data.

The original or underlying subject of the passive can be introduced by a proclitic form of the comitative \textit{na}:\footnote{In the variety described by \citet[35]{SchumannK1899} and \citet[88]{EndemannC1914}, the agent/force of the passive is introduced by the locative class 17 \textit{kʊ}-. In elicitation this was rejected as agent marking and only accepted with a locational reading.}
\begin{exe}
\ex \gll jo a-a-pon-ile kʊsita kʊ-seng-\textbf{igw}-a \textbf{n}=\textbf{ʊ}-\textbf{mw}-\textbf{ene} \textbf{ka}-\textbf{aja} ʊ-jʊ a-a-kol-ile ɪɪ-sengo\\
\textsc{ref.1} 1-\textsc{pst}-save-\textsc{pfv} without 15-chop-\textsc{pass}-\textsc{fv} \textsc{com}=\textsc{aug}-1-owner 12-homestead \textsc{aug}-\textsc{prox.1} 1-\textsc{pst}-hold-\textsc{pfv} \textsc{aug}-sickle(9)\\
\glt `He was saved without being cut by the owner of the house, who held a sickle.' [Wage of the thieves]
\ex \gll ɪ-kɪ-siba ɪ-kyo kɪ-sisya kangɪ kɪ-syʊngʊʊtɪl-i\textbf{igw}e \textbf{n}=\textbf{ɪ}-\textbf{fy}-\textbf{amba} na a-m-ɪɪsi ga-ake ma-sisya\\
\textsc{aug}-7-pond \textsc{aug}-\textsc{ref-8} 8-frightening again 8-surround-\textsc{pass.pfv} \textsc{com}=\textsc{aug}-8-mountain \textsc{com} \textsc{aug}-6-water 6-\textsc{poss.sg} 6-frightening\\
\glt `That pond is frightening. It is surrounded by mountains and its water is frightening.' [Selfishness kills]
\end{exe}

With \isi{locative} subjects, passives of intransitive verbs can be formed. This yields an impersonal meaning:
\begin{exe}
\ex \gll paa-sokoni pi-kʊ-jweg-igw-a\\
16-market(9)(<SWA) 16-\textsc{prs}-shout-\textsc{pass}-\textsc{fv}\\
\glt `At the market, there is shouting.' [ET]
\ex \gll kʊ-ka-aja kʊ-my-ɪnʊ kʊ-kʊ-hobok-igw-a fiijo\\
17-12-homestead 17-4-\textsc{poss.1pl} 17-\textsc{prs}-be(come)\_happy-\textsc{pass}-\textsc{fv} \textsc{intens}\\
\glt `At our home, they rejoice much.' [ET]
\ex \gll n-nyumba mu-la mu-kʊ-mog-igw-a\\
18-house(9) 18-\textsc{dist} 18-\textsc{prs}-dance-\textsc{pass}-\textsc{fv}\\
\glt `In that house, there is dancing.' [ET]
\end{exe}

For a discussion of the syntax of the passive see \citet{LusekeloA2012}. Nyakyusa has symmetric passives (see e.g. \citealt{BresnanJMoshiL1990}): Both objects of three-argument-verb can be promoted to subject in passivization. The following examples illustrate this with objects occupying various semantic roles.\is{semantic roles}
\begin{exe}
\ex
\begin{xlist}
\ex Active voice:\\
\gll i-kʊ-ba-pɪɪj-ɪl-a ɪ-fi-ndʊ a-ba-heesya\\
1-\textsc{prs}-2-cook-\textsc{appl}-\textsc{fv} \textsc{aug}-8-food \textsc{aug}-2-foreigner\\
\glt \lq S/he cooks food for the guests.'
\ex Passive voice, theme as subject:\\
\gll ɪ-fi-ndʊ fi-kʊ-pɪɪj-ɪl-igw-a a-ba-heesya\\
\textsc{aug}-8-food 8-\textsc{prs}-cook-\textsc{appl}-\textsc{pass}-\textsc{fv} \textsc{aug}-2-foreigner\\
\glt \lq The food is cooked for the guests.'
\ex Passive voice, beneficiary as subject:\\
\gll a-ba-heesya bi-kʊ-pɪɪj-ɪl-igw-a ɪ-fi-ndʊ\\
\textsc{aug}-2-foreigner 2-\textsc{prs}-cook-\textsc{appl}-\textsc{pass}-\textsc{fv} \textsc{aug}-8-food\\
\glt \lq The guests are cooked food for.'\footnotemark
\end{xlist}
\ex
\begin{xlist}
\ex Active voice:\\
\gll i-kʊ-kolog-el-a ʊ-n-tingo ɪ-fi-ndʊ\\
1-\textsc{prs}-stir-\textsc{appl}-\textsc{fv} \textsc{aug}-3-wooden\_spoon \textsc{aug}-8-food\\
\glt \lq S/he stirs (the) food with a/the spoon.'
\ex Passive voice, theme as subject:\\
\gll ɪ-fi-ndʊ fi-kʊ-kolog-el-igw-a ʊ-n-tingo\\
\textsc{aug}-8-food 8-\textsc{prs}-stir-\textsc{appl}-\textsc{pass}-\textsc{fv} \textsc{aug}-3-wooden\_spoon\\
\glt \lq (The) food is stirred with a/the spoon.'
\ex Passive voice, instrument as subject:\\
\gll ʊ-n-tingo gʊ-kʊ-kolog-el-igw-a ɪ-fi-ndʊ\\
\textsc{aug}-3-wooden\_spoon 3-\textsc{prs}-stir-\textsc{appl}-\textsc{pass}-\textsc{fv} \textsc{aug}-8-food\\
\glt \lq A/the spoon is used to stir (the) food.'
\end{xlist}
\ex
\begin{xlist}
\ex Active voice:\\
\gll ʊ-mw-ana i-kʊ-sop-el-a ɪ-m-bwa a-ma-bwe\\
\textsc{aug}-1-child 1-\textsc{prs}-throw-\textsc{appl}-\textsc{fv} \textsc{aug}-9-dog \textsc{aug}-6-stone\\
\glt \lq A/the child throws stones at a/the dog.'

\ex Passive voice, theme as subject:\\
\gll a-ma-bwe gi-kʊ-sop-el-igw-a ɪ-m-bwa\\
\textsc{aug}-6-stone 6-\textsc{prs}-throw-\textsc{appl}-\textsc{pass}-\textsc{fv} \textsc{aug}-9-dog\\
\glt \lq Stones are thrown at a/the dog.'

\clearpage %otherwise first line of the following example would stand alone

\ex Passive voice, goal as subject:\\
\gll ɪ-m-bwa jɪ-kʊ-sop-el-igw-a a-ma-bwe\\
\textsc{aug}-9-dog 9-\textsc{prs}-throw-\textsc{appl}-\textsc{pass}-\textsc{fv} \textsc{aug}-6-stone\\
\glt \lq A/the dog is thrown stones at.' [all examples elicited]
\end{xlist}
\end{exe}
\protect\footnotetext{This example created some amusement, as it also allows for a reading of the subject being the instrument, hence cannibalism.}
\is{passive!productive|)}

\subsection{Fossilized Passive} \label{FossilizedPassive}\is{passive!fossilized|(}
For Proto-Bantu,\il{Proto-Bantu} the passive extension has been reconstructed with two allomorphs \mbox{*-\textit{ʊ}/}\mbox{-*\textit{ibʊ}}, the latter of which came to be used only with vowel-final bases \citep[78]{SchadebergT2003a}. While the reflex of the longer passive extension is productive in Nyakyusa (\sectref{Passive}), the short allomorph is only found in a relatively small number of lexicalized cases.\footnote{\citet[36]{SchumannK1899} observes for a number of stems that ``[s]ome verbs are only formally passives, but also concerning formation of the perfect deviate from the regular passive form (deponentia)'' (translated from the original German, BP). See \citet{GoodJ2007} for a discussion of deponency in Bantu.} These have in most cases undergone a semantic shift, as shown in the following examples.

\begin{exe}
\ex \label{exLexicalizedPassivesSemanticShift}
\begin{tabbing}
\textit{pondwa}x\=`be burdened, weak'x\=;< \textit{ponda}x\=\kill%unsinnszeile für Tabulatoren
\textit{gogwa}\>`dream; have vision'\> < \textit{goga}\>`kill; destroy'\\
\textit{komwa}\>`sob'\> < \textit{koma}\> \lq hit'\\
\textit{logwa}\>`copulate'\> < \textit{loga}\> \lq bewitch'\\
\textit{milwa}\>`drown'\> < \textit{mila}\> \lq swallow; devour'\\
\textit{pondwa}\>`fail to; miss'\> < \textit{ponda}\> \lq forge'\\
\textit{tolwa}\>`be burdened, weak'\> < \textit{tola}\> \lq defeat'
\end{tabbing}
\end{exe}

Some of these fossilized passives are transitive:
\begin{exe}
\ex
\begin{tabbing}
\textit{pondwa}x\=`be burdened, weak'x\=;< \textit{ponda}x\=\kill%unsinnszeile für Tabulatoren
\textit{ibwa}\>`forget (tr.)'\> < \textit{iba}\> \lq steal'\\
\textit{syʊkwa}\>`miss sadly (tr.)'\> < \textit{syʊka}\> \lq be resurrected'
\end{tabbing}
\end{exe}

For some of these short passives, no underived \isi{base} is available, but commutations can be found in most cases (\ref{exLexicalizedPassiveCommutation}). With other verbs that appear from their shape to be fossilized passives, neither is attested and it may be questionable whether diachronically these verbs constitute passives at all (\ref{exQuestionableLexicalizedPassives}). However, all of these verbs except \textit{ilaamwa} pattern together with fossilized passives in the formation of their perfective stems; see \sectref{ImbricationDisyllabic}.
\clearpage

\begin{exe}
\ex \label{exLexicalizedPassiveCommutation}
\begin{tabbing}
\textit{tumukɪlwa}x\=`be full beyond capacity'x\=cf. \textit{fukula}x\=\kill%Unsinnszeile für Tabs
\textit{fukwa}\>`be full beyond capacity'\> cf. \textit{fukula}\>`dig up'\\
\textit{sulwa}\>`dare to, presume'\>cf. \textit{sulula}\>`pour'\\
\textit{keelwa}\>`be plentiful'\>no *\textit{keela} or commutations\\%auch im PB nichts passendes
\textit{kunwa}\>`be in great need'\>no *\textit{kuna} or commutations
\end{tabbing}
\ex \label{exQuestionableLexicalizedPassives}
\begin{tabbing}
\textit{tumukɪlwa}x\=`be full beyond capacity'x\=cf. \textit{fukula}x\=\kill%Unsinnszeile für Tabs
\textit{ilaamwa}\>`disregard, doubt'\>no *\textit{(i)laama} attested\\
\textit{nyonywa}\>`long for, desire'\>no *\textit{nyonya} attested\\
\textit{miimwa}\>`crave, envy'\> < *\textit{mìim} `sprinkle'?
\end{tabbing}
\end{exe}

The fossilized passive extension often co-occurs with the \isi{applicative} (\ref{exFossilizedPassiveAppl}). It is further apparent in a number of deverbal nouns (\ref{exFossilizedPassiveNouns}).
\begin{exe}
\ex\label{exFossilizedPassiveAppl}
\begin{tabbing}
\textit{tumukɪlwa}x\=`be(come) short on; late for'x\=cf. \textit{ʊlʊlaka}x\=\kill%Unsinnszeile für Tabs
\textit{agɪlwa}\>`diminish (by); lack'\> < \textit{aga}\>`diminish (tr.)'\\
\textit{kʊbɪlwa}\>`suffer'\> < \textit{kʊba}\>`beat; ring'\\
\textit{lakɪlwa}\>`choke (on)'\>cf. \textit{ʊlʊlaka}\>`great thirst'\\
\textit{tumukɪlwa}\>`be(come) short on; late for'\>cf. \textit{tumula}\>`cut; come to decision'\\
\textit{ʊmɪlwa}\>`be thirsty (for)'\> < \textit{ʊma}\>`dry; wither'
\end{tabbing}
\ex \label{exFossilizedPassiveNouns} \begin{tabbing}
\textit{tumukɪlwa}x\=`be(come) short on; late for'x\=cf. \textit{ʊlʊlaka}x\=\kill%Unsinnszeile für Tabs
\textit{ʊmfwalwa}\>\lq clothing, dress'\> < \textit{fwala} \> \lq dress, wear'\\
\textit{iikolwa}\>\lq shell' \> < \textit{kola} \> \lq grasp, hold'\\
\textit{ɪkɪsʊʊjwa}\>\lq porridge from maize milk'\> < \textit{sʊʊja} \> \lq filter, strain'\\
\textit{ʊntʊmwa}\>\lq slave, servant'\> < \textit{tʊma}\> \lq send'
\end{tabbing}
\end{exe}

When asked in elicitation to form true passives of fossilized passives, the language assistants replaced the fossilized extension (or what is treated as such) with the productive long passive extension:
\begin{exe}
\ex
\begin{tabbing}
\textit{ilaamwa}x\=`disregard, doubt'x\=> \textit{ilaamigwa}x\=\kill
\textit{ibwa}\>`forget'\> > \textit{ibigwa}\>`be forgotten'\\
\textit{milwa}\>`drown'\> > \textit{miligwa}\>`be drowned at' (\textsc{loc} subject)
\\\textit{miimwa}\>`crave, envy'\> > \textit{miimigwa}\>`be craved for'\\
\textit{ilaamwa}\>`disregard, doubt'\> > \textit{ilaamigwa}\>`be disregarded, doubted'
\end{tabbing}
\end{exe}

However, this was felt to be an artificial device, and with some roots\is{root} it creates ambiguity (\ref{exAmbiguousPassiveReplacement}). Topicalization of the patient/theme through fronting (\ref{exTopicalizationOfMediopassive}) was suggested as more natural for such verb constructions.
\begin{exe}
\ex \label{exAmbiguousPassiveReplacement}\begin{xlist}
\ex \textit{ɪngamu jangʊ jɪ-kw-ib-igw-a nagwe}\\
1. \lq My name is being forgotten by him/her.'\\
2. \lq My name is being stolen by him/her.' [ET]
\ex \textit{ɪnjosi jɪ-kʊ-gog-igw-a nagwe}\\
1. \lq A dream is dreamt by him/her.'\\
2. \lq A dream is killed by him/her.' [ET]
\end{xlist}
\ex \label{exTopicalizationOfMediopassive}
\begin{xlist}
\ex \gll ɪ-n-gamu j-angʊ i-kʊ-j-iibw-a\\
\textsc{aug}-9-name 9-\textsc{poss.1sg} 1-\textsc{prs}-9-forget-\textsc{fv}\\
\glt `(As for) my name, s/he forgets it.' [ET]
\ex \gll ɪ-n-josi i-kʊ-jɪ-gogw-a\\
\textsc{aug}-9-dream 1-\textsc{prs}-9-dream-\textsc{fv}\\
\glt `(As for) the dream, s/he dreams it.' [ET]
\end{xlist}
\end{exe}
\is{passive!fossilized|)}

\subsection{Neuter}\label{Neuter}
\is{neuter|(}
The neuter (also commonly called \textit{stative}) extension has the underlying shape -\textit{ɪk} (allomorphs -\textit{ek}, -\textit{ik}; see \sectref{MorphophonologyOfVerbalExtension}). The neuter promotes the object of a transitive verb to the subject of a now intransitive verb, which is construed as potentially or factually experiencing a certain state. (\ref{exNeuterListe}) gives some examples. Further research is needed to determine the productivity and the semantic and syntactic\is{syntax} constraints on the use of the neuter.

\begin{exe}
\ex \label{exNeuterListe}
\begin{tabular}[t]{llll}
\textit{malɪka}&`finish (intr.), be finished'& < \textit{mala}&`finish (tr.)'\\
\textit{manyika}&`be known; be famous'& < \textit{manya}&`know'\\
\textit{nweka}&`be drinkable'& < \textit{nwa}&`drink'\\
\textit{oneka}&`gush; be spilled, scattered'& < \textit{ona}&`spill'\\
\textit{onangɪka}&\lq be(come) spoiled; perish'& < \textit{onanga}& \lq destroy'\\
\textit{ongeleka}&`increase (intr.)'& < \textit{ongela}& \lq increase (tr.)'\\
\textit{swɪlɪka}&`be(come) tame'& < \textit{swɪla}&`feed; rear; tame'
\end{tabular}
\end{exe}

At least the following two verbs derived with the neuter extension have idiosyncratic meanings:
\begin{exe}
\ex
\begin{tabbing}
\textit{boneka}x\=`happen; also: appear, be seen'x\= < \textit{bona}x\=\kill %Zeile für Tabulatoren
\textit{boneka}\>`happen; also: appear, be seen'\> < \textit{bona}\>`see'\\
\textit{silɪka}\>`faint'\> < \textit{sila}\>`protest by refusal (tr.)'
\end{tabbing}
\end{exe}

Unlike with the passive,\is{passive!productive} the original agent or force of a neuter verb cannot be expressed:
\begin{exe}
\ex[*]{\gll ʊ-n-nyambala i-kʊ-bon-ek-a \textup{(}na=\textup{)}a-ba-ndʊ\\
\textsc{aug}-1-man 1-\textsc{prs}-see-\textsc{neut}-\textsc{fv} (\textsc{com}=)\textsc{aug}-2-person\\
\glt (intended: `The man is seen by people.')}
\ex[*]{\gll ɪ-n-jʊni si-la si-kʊ-peeny-ek-a \textup{(}na=\textup{)}a-ba-kiikʊlʊ\\
\textsc{aug}-10-bird 10-\textsc{dist} 10-\textsc{prs}-remove\_feathers-\textsc{neut}-\textsc{fv} (\textsc{com}=)\textsc{aug}-2-woman\\
\glt (intended: `Those birds have their feathers plucked by women.')}
\ex[*]{\gll ɪɪ-nyumba ɪ-jɪ jɪ-kʊ-jeng-ek-a \textup{(}na=\textup{)}a-ba-fundi\\
\textsc{aug}-house(9) \textsc{aug}-\textsc{prox.9} 9-\textsc{prs}-build-\textsc{neut}-\textsc{fv} (\textsc{com}=)\textsc{aug}-2-worker(<SWA)\\
\glt (intended: `This house is being built by workers.')}
\end{exe}

With one verb the combination of neuter and passive\is{passive!productive} is attested: \textit{malɪkigwa} \lq run out' < \textit{mala} \lq finish (tr.)'.%das muss ergaenzt werden
\is{neuter|)}
\subsection{Intensive}
\is{intensive|(}
The intensive extension has the underlying shape -\textit{ɪlɪl} (and allomorphs, see \sectref{MorphophonologyOfVerbalExtension}). It denotes repetition, greater intensity and/or continuity. As the meaning of this extension is very much dependent on the semantics of the underlying verb as well as on the context, it is best illustrated with some examples from texts.

\begin{exe}
\ex \gll a-lɪnkw-and-a ʊ-kʊ-pɪɪj-a, kangɪ a-a-lʊng-\textbf{ɪliile} kanunu fiijo\\
1-\textsc{narr}-start-\textsc{fv} \textsc{aug}-15-cook-\textsc{fv} again 1-\textsc{pst}-add\_spice-\textsc{ints.pfv} well \textsc{intens}\\
\glt `She started to cook [it] and spiced [it] very well.' [Thieving woman]
\ex \gll ba-kʊ-tuufiifye fiijo ʊ-gwe ʊkʊtɪ ʊ-lɪ n-nunu, looli fi-fy-ɪma fy-ako fi-kɪnd-\textbf{ɪliile} ʊ-bʊ-nywamu\\
2-\textsc{2sg}-praise.\textsc{pfv} \textsc{intens} \textsc{aug}-\textsc{2sg} \textsc{comp} \textsc{2sg}-\textsc{cop} 1-good but 8-8-thigh 8-\textsc{poss.2sg} 8-pass-\textsc{ints}.\textsc{pfv} \textsc{aug}-14-big\\
\glt `They have praised you a lot, that you are a good person, but your thighs are too big [lit. have intensively surpassed size].' [Hare and Hippo] %beispiel zwar schonmal verwendet, aber mit mehr drumherum, das muss halt gehen
\ex \gll bo muu$\sim$mo iisib-\textbf{ɪliile} ʊ-n-kasi gw-a Pakyɪndɪ, a-lɪnkʊ-bʊʊk-a kʊ-kw-aganil-a n=ʊ-n-nyambala ʊ-jʊ-ngɪ kʊ-n̩-gʊnda gw-a ma-jabʊ g-a n̩-dʊme Pakyɪndɪ\\
as \textsc{redupl}$\sim$\textsc{ref.18} 1.be(come)\_accustomed-\textsc{ints.pfv} \textsc{aug}-1-wife 1-\textsc{assoc} P. 1-\textsc{narr}-go-\textsc{fv} 17-15-meet-\textsc{fv} \textsc{com}=\textsc{aug}-1-man \textsc{aug}-1-other 17-3-field 3-\textsc{assoc} 6-cassava 6-\textsc{assoc} 1-husband P.\\
\glt `Just as she was very accustomed to do, Pakyindi's wife went to meet with another man in Pakyindi's cassava field.' [Sokoni and Pakyindi] 
\ex \gll ʊ-ne kʊ-my-angʊ kʊ-no n-gʊ-fum-a, n-dɪ malafyale. mu-n-geet-\textbf{elel}-e=po panandɪ!\\
\textsc{aug}-\textsc{1sg} 17-4-\textsc{poss.1sg} 17-\textsc{prox} \textsc{1sg}-\textsc{prs}-come\_from-\textsc{fv} \textsc{1sg}-\textsc{cop} chief(1) \textsc{2pl}-\textsc{1sg}-look-\textsc{ints}-\textsc{subj}=\textsc{part} a\_little\\ 
\glt `At my home where I come from I am a king. You should look at me a little!' [Hare and Hippo]
\end{exe}

With a number of verbs, the intensive gives an idiosyncratic reading:
\begin{exe}
\ex\label{exIntensiveIdeosyncratic}
\begin{tabular}[t]{@{}llll}
\textit{ambɪlɪla}&`receive; entertain guests'& < \textit{amba}&`hold out to receive'\\ 
\textit{bombelela}&`weed'& < \textit{bomba}&`work; do'\\  
\textit{endelela}&\lq continue'& < \textit{enda} & \lq walk, travel'\\%endelela auch 'hurry up'?
\textit{keetelela}&`take care of; watch'& < \textit{keeta}&`watch'\\
\textit{kʊbɪlɪla}&\lq flap, fan' & < \textit{kʊba}& \lq beat; ring'\\
\textit{paatɪlɪla}&\lq prune' & < \textit{paata}&\lq way of harvesting'\\
\textit{tabɪlɪla}&\lq stammer' & < \textit{taba}&\lq extend; crawl'
% genug bsp
\end{tabular}
\end{exe}

A handful of verbs feature -\textit{ɪɪl}/-\textit{eel} (\ref{exIntensiveEela}). Where the underived root is available, a comparison of meaning suggests that these verbs feature lexicalized intensives where the first /l/ has dropped out.
\begin{exe}
\ex\label{exIntensiveEela}\begin{tabular}[t]{llll} 
\textit{bʊʊkɪɪlwa}&\lq drown and be carried by water' & < \textit{bʊʊka}&\lq go'\\
\textit{boteela}&`be calm; be settled'& < \textit{bota}&`be calm'\\
\textit{eleela}&`float'&&\\
\textit{embeela}&`wander; prostitute'&&\\
\textit{obeela}&`rumble, scorn'&&\\
\textit{ogeela}&`swim'& < \textit{oga}&`bathe (intr.)'\\
\textit{tendeela}&`peep'&&\\
\end{tabular}
\end{exe}
\is{intensive|)}
\subsection{Separative}\label{Separative}\is{separative|(}
There are two separative extensions in Nyakyusa, one yielding transitive verbs (-\textit{ʊl}) and one yielding intransitive ones (-\textit{ʊk}). See \sectref{MorphophonologyOfVerbalExtension} for morphophonological processes affecting the vowel quality of these extensions. \citet[78]{SchadebergT2003a} characterizes the abstract semantic core of the separative as ``movement out of some original position''. Other common labels in Bantu studies include \textit{reversive} and \textit{inversive}. Although the separative extensions are essentially unproductive, derived verbs are frequent in the lexicon. The following list gives some examples:

\newpage 
\begin{exe}
\ex\label{exSeparativeBoth}
\begin{tabbing}
\textit{kuupula}x\=`break; harvest'\=-- \=\textit{bulutuka}x\=`be(come) uncovered'\=< \textit{pɪnda}x\=\kill%Unsinnszeile für Tabulatoren
\textit{baalʊla}\>`widen (tr.)'\>--\>\textit{baalʊka}\> \lq bloss, expand'\> < \textit{baala}\>`thrive'\\
\textit{bulutula}\>`shell corn'\>--\>\textit{bulutuka}\>`collapse'\\
\textit{fyogola}\>`sprain'\>--\>\textit{fyogoka}\>`be(come) sprained'\\
\textit{hobola}\>`free, relax'\>--\>\textit{hoboka}\> \lq be(come) happy'\\
\textit{kingʊla}\>`uncover'\>--\>\textit{kingʊka}\>`be(come) uncovered'\>< \textit{kinga}\>`cover'\\
\textit{lumbula}\>`expand'\>--\>\textit{lumbuka}\>`stretch, expand'\\  
\textit{pagʊla}\>`force open'\>--\>\textit{pagʊka}\> \lq be(come) dislocated'\\
\textit{pɪndʊla}\>`convert'\>--\>\textit{pɪndʊka}\>\lq repent'\>< \textit{pɪnda}\>\lq bend'\\
\textit{konyola}\>`break; harvest'\>--\>\textit{konyoka}\>`be(come) broken'\\
\textit{tonola}\>\lq dab'\>--\>\textit{tonoka}\>`bounce'\\ %unklare root, also nix angeben
\textit{kuupula}\>`uproot'\>--\>\textit{kuupuka}\>`be uprooted'
\end{tabbing}
\end{exe} %fertig!

As can be observed in (\ref{exSeparativeBoth}), there is frequent commutation between the two separative extensions. This is a strong tendency rather than an absolute rule. Other attested commutations are with the \isi{impositive} (\sectref{Impositive}) and \isi{positional} (\sectref{Positional}) extensions.
The separative extensions are also found in denominal\is{denominal verbs} derivations. If the underlying stem ends in a first degree vowel, this is elided and the separative extension surfaces with /u/ (i.e., the height feature is maintained).
\begin{exe}
\ex
\begin{tabbing}
\textit{mangʊka}x\=`treat medically, heal'x\=< \textit{ʊn̩ganga}x\=\kill%Unsinnszeile für Tabulatoren
\textit{biibuka}\>`be(come) bad, ugly'\> < \textit{biibi}\>`bad, ugly'
\\\textit{gangʊla}\>`treat medically, heal'\> < \textit{ʊn̩ganga}\>`doctor; healer'
\\\textit{niinuka}\>`decrease (intr.)'\> < \textit{niini}\>`little'
\\\textit{pimbʊka}\>`be(come) short'\> <	\textit{pimba}\>`short'
\end{tabbing}
\end{exe}

In a few verbs, two separative extensions are found. With the exception of \textit{sʊngʊlʊla} (\ref{exseparativesungulula}), the separative intransitive follows the separative transitive (\ref{exseparativeTwiceTransitive}).
\begin{exe}
\ex \label{exseparativesungulula}
\begin{tabbing}
\textit{kunguluka}x\=`descend; appear (spirit)'x\=cf. \textit{sʊngʊla}x\=\kill%Unsinnszeile für Tabulatoren
\textit{sʊngʊlʊla}\>`dissolve'\>cf. \textit{sʊngʊla}\>`choose; sift'
\end{tabbing}
\ex \label{exseparativeTwiceTransitive}
\begin{tabbing}
\textit{kunguluka}x\=`descend; appear (spirit)'x\=cf. \textit{sʊngʊla}x\=\kill%Unsinnszeile für Tabulatoren
\textit{bululuka}\>`scatter (intr.)'\>cf. \textit{bulula}\>`scatter (tr.)'\\
%bʊmbʊlʊka?
\textit{bunguluka}\>`toss and turn'\> < *\textit{búng}\>`wrap up'\\
\textit{kololoka}\>`slacken'\> < \textit{kola}\> \lq grasp, hold'\\
\textit{kunguluka}\>`foretell, prophecy'\> < \textit{kunga}\>`pour out'\\
\textit{sololoka}\>`descend; appear (spirit)'\> < \textit{solola} \>`prophecy'\\
\textit{sululuka}\>`drip'\>cf. \textit{sulula}\>`pour'
\end{tabbing}
\end{exe}
\is{separative|)}
\subsection{Tentive}\is{tentive|(}
The tentive extension has the shape -\textit{at}. It is not productive and relatively few verbs are found that contain this suffix. \citet[77]{SchadebergT2003a} notes that the semantic core for this Bantu extension can be stated as ``actively making firm contact''. This characterization holds for Nyakyusa in most cases, see the list in (\ref{exTentive}). No noticeable pattern of commutation is attested.
 
\begin{exe}
\ex \label{exTentive}
\begin{tabbing}
\textit{kambatʊla}x\=\kill 
\textit{finyatɪla}\>`put close to'\\
\textit{fumbata}\>`enclose in hands or mouth'\\
\textit{isunyata}\>`brood; ponder'\\
\textit{kambatʊla}\>`grip'\\
\textit{lalata}\>`become paralytic' \\
\textit{pagata}\>`put in lap, pull to chest; hold child'
\end{tabbing}
\end{exe}
\is{tentive|)}
\subsection{Positional} \label{Positional}\is{positional|(}
The non-productive positional extension has the shape -\textit{am}. A common semantic element of assuming a physical posture or position can be observed \citep[75]{SchadebergT2003a}. All Nyakyusa verbs derived by means of the positional extension are inchoative.\is{inchoative verbs} The positional appears not to be productive in Nyakyusa. The following list gives some representative examples.
\begin{exe}
\ex
\begin{tabbing}
\textit{kupama}x\=\kill
\textit{alama}\>`settle at bottom; duck'\\
\textit{asama}\>`gape'\\
\textit{batama}\>`be(come) quiet, silent'\\
\textit{fugama}\>`kneel'\\
%\textit{galama}\>`lie on back'\\
\textit{kupama}\>`lie on stomach'\\
\textit{egama}\>`lean'
\end{tabbing}
\end{exe}

Some positional verbs have their transitive counterpart formed with the combination of the positional and \isi{impositive} extensions (\ref{exPositiona+Impositive}), while others replace the positional with the \isi{impositive} (\ref{exImpositiveNotPositional}). In a few cases, both devices are attested without any apparent difference in meaning (\ref{exImpositivePositionalBothDevices}). Further commutations include the \isi{separative} (\sectref{Separative}) (\ref{exPositionalSeparative}).
\begin{exe}
\ex \label{exPositiona+Impositive}
\begin{tabbing}
\textit{gundamika}x\=`turn sidewards (intr.)'xx\=> sulamikax\=\kill%Unsinnszeile für Tabs
Positional plus impositive:\\
\textit{bɪlamika}\>`bend (tr.)'\>< \textit{bɪlama}\>`bend (intr.)'
\\\textit{gundamika}\>`bend over/down'\>< \textit{gundama}\>`stoop, incline'
\\\textit{telamika}\>`lower'\>< \textit{telama}\>`be(come) low'
\end{tabbing}

\ex\label{exImpositiveNotPositional}
\begin{tabbing}
\textit{gundamika}x\=`turn sidewards (intr.)'xx\=> sulamikax\=\kill%Unsinnszeile für Tabs
Commutation between positional and impositive:\\
\textit{batama}\>`be(come) quiet, silent'\>cf. \textit{batɪka}\>`soothe'\\
%\textit{bɪlɪka}\>`unbekannt'\>cf. \textit{bɪlama}\>`bend (intr.)'\\
\textit{pɪngama}\>`turn sidewards (intr.)'\>cf. \textit{pɪngɪka}\>`set across'
\end{tabbing}
\newpage 
\ex\label{exImpositivePositionalBothDevices}
\begin{tabbing}
\textit{gundamika}x\=`open mouth in surprise'xx\=> sulamikax\=\kill%Unsinnszeile für Tabs
Commutation as well as additive derivation:\\
\textit{egama}\>`lean (intr.)'\>cf. \textit{egeka}\>`lean (tr.)\\
\>\> > \textit{egamika}\>`lean (tr.)'\\
\textit{sulama}\>`bend'\>cf. \textit{sulɪka}\>`turn upside down'\\ 
\>\> > \textit{sulamika}\>`turn upside down'
\end{tabbing}
\ex \label{exPositionalSeparative}
\begin{tabbing}
\textit{gundamika}x\=`open mouth in surprise'xx\=> sulamikax\=\kill%Unsinnszeile für Tabs
Commutation between positional and separative:\\
\textit{alama}\>`settle at bottom'\>cf. \textit{alʊla}\> \lq remove from top'\\
%\>\>cf. \textit{alʊka}\>`stand up'\\
\textit{sulama}\>`bend'\>cf. \textit{suluka}\>`descend'\\
%\>\>cf. \textit{sulul}\>`pour'\\
\textit{gasama}\>`open mouth in surprise'\>cf. \textit{gasʊka} \>`be astonished'
%\>\>cf. \textit{gasʊla} \>`amaze'
\end{tabbing}
\end{exe}
\is{positional|)}

\subsection{Impositive}\label{Impositive}
\is{impositive|(}
The impositive has the underlying shape -\textit{ɪk}. Allomorphs are -\textit{ek}, -\textit{ik} (see \sectref{MorphophonologyOfVerbalExtension}; also note the exceptions below). It is thus homophonous to the \isi{neuter} extension (\sectref{Neuter}). The core meaning of the impositive may be paraphrased as ``to put (sth.) into some position'' \citep[73]{SchadebergT2003a}. The following list provides some examples.

\begin{exe}
\ex\label{exImpositive}
\begin{tabular}[t]{llll} 
\textit{bambɪka}&`arrange in line'& < \textit{bamba}& \lq stand in line'\\
\textit{bɪɪka}&`put; store; calve'& < *\textit{bá}&`dwell; be; become'\\
\textit{jubɪka}&`dip, soak'\\
\textit{jumbɪka}&`praise'& < \textit{jumba}& `swell (river)'\\
\textit{olobeka}&`soak (tr.)'& < \textit{oloba}& `get wet'\\
\textit{fubɪka}&`soak (tr.)'&& %< \textit{fuba}\> 'i.a. make dirty; spoil'?
\\\textit{tegeka}&'set a trap for'& < \textit{tega}&`trap; catch'
\end{tabular}
\end{exe}

The impositive extension can be considered the transitive counterpart to the \isi{positional} extension, with which a number of commutations are attested (see \sectref{Positional}). Other commutations include the \isi{separative} (\ref{exImpositiveCommutationSeparative}). In one verb, the impositive is found as the transitive counterpart to the extensive;\is{extensive} see \sectref{extensive}.
\begin{exe}
\ex \label{exImpositiveCommutationSeparative}
\begin{tabbing}
\textit{baatɪka}x\=\lq turn upside down'x\=cf. \textit{bambʊla}x\=\kill %nonsense line for tabs
\textit{anika}\>`set out to dry' \> cf. \textit{anula} \>`remove from drying'\\
\textit{baatɪka} \> `arrange; fix' \> cf. \textit{baatʊla}\>`offload, unload'\\
\textit{bambɪka}\>`arrange' \> cf. \textit{bambʊla} \> `peel'\\
\textit{fyɪka}\>`insert' \> cf. \textit{fyʊla} \> `remove'\\
\textit{lɪmbɪka}\>`accumulate' \> cf. \textit{lɪmbʊla} \> `serve out; gather honey'\\
\textit{lʊndɪka}\>`pile up' \> cf. \textit{lʊndʊla} \> `take cows out; divide'\\
\textit{sulɪka}\>`turn upside down'\>cf. \textit{sulula}\>`pour'
\end{tabbing}
\end{exe}%das ganze zu impositive alles kurz in odt, damit ordnung!

In a few cases, what appears to be the impositive extension is found with a first degree vowel /i/ that cannot be accounted for by any regularity:
\begin{exe}
\ex
\begin{tabular}[t]{llll}
\textit{ɪm\textbf{i}ka}&\lq erect; bring to halt; respect'& < \textit{ɪma}&`stand (up), stop'\\
&& cf. \textit{ɪm\textbf{ɪ}la} & `preside'\\
\textit{ʊm\textbf{i}ka}& `dry (tr.)' & < \textit{ʊma} & `dry; wither'\\
\textit{bin\textbf{i}ka} & `spoil, ruin, destroy' & < \textit{bina} &`fall sick'?\\
&&< \textit{bini}&`malicious'?\\
&&cf. \textit{bin\textbf{ɪ}sya}&`make sick'\\
\textit{kit\textbf{i}ka}&`set up; stick into ground'
\end{tabular}
\end{exe}
\is{impositive|)}
\subsection{Extensive} \label{extensive}\is{extensive|(}
The extensive extension has the shape -\textit{al}. It is unproductive in Nyakyusa. There is no overarching semantic element for verbs derived with this extension. As observed by \citet[77]{SchadebergT2003a}, there is a certain tendency for it to occur with verbs denoting two semantic fields: being in a spread-out position (\ref{exExtensiveSpreadOutPosition}) and debilitation or illness (\ref{exExtensiveDebilitation}). Some verbs with miscellaneous meanings are given in (\ref{exExtensiveMisc}). No reoccurring pattern of commutation is attested in the data.
\begin{exe}
\ex \label{exExtensiveSpreadOutPosition}
\begin{tabbing}
\textit{niongala}x\=\kill
\textit{bagala}\>`carry load on shoulder'\\
\textit{tʊʊgala}\>`get seated, sit; live, inhabit; stay'\\
\textit{twala}\>`carry load, bring'
\end{tabbing}
\ex\label{exExtensiveDebilitation}
\begin{tabbing}
\textit{niongala}x\=\kill
\textit{fulala}\>`be(come) hurt, injured'\\
\textit{kalala}\>`be(come) angry, annoyed'\\
\textit{katala}\>`be(come) tired, exhausted'\\
\textit{kangala}\>`be(come) old and worn'\\
\textit{lemala}\>`be(come) crippled, disabled'
\end{tabbing}
\ex\label{exExtensiveMisc}
\begin{tabbing}
\textit{niongala}x\=\kill
\textit{angala}\>`be well, feel fine; converse, talk, be in good company'\\
\textit{langala}\>`glisten, glitter, shine'\\
\textit{niongala}\>`be(come) bent, crooked, twisted'\\
\textit{syala}\>`remain'
\end{tabbing}
\end{exe}

In two verbs, what seems to be a reduplicated\is{reduplication} extensive suffix was found. One of these has the combination of extensive plus impositive (\sectref{Impositive}) as its transitive counterpart:
\begin{exe}
\ex
\begin{tabbing}
\textit{lambalala}x\=`lie down, sleep'xx\=xxcf. \textit{lambalɪka}x\=\kill
\textit{lambalala}\>`lie down, sleep'\>cf. \textit{lambalɪka}\>`make lie down, put to bed'\\
\textit{tambalala}\>`lie flat'
\end{tabbing}
\end{exe}
\is{extensive|)}
\section{Combinations of verbal extensions}\label{CombinationOfVerbalExtensions}
Often more than one verbal extension appears on a single verb base.\is{base} In the following sub-sections, some generalizations over the respective order of morphemes will be given (\sectref{OrderOfExtensions}), followed by a discussion of the derivation of pluractionals\is{pluractional} by means of combining the reciprocal/associative\is{reciprocal} and the \isi{causative} (\sectref{Pluractional}) and a discussion of the shape of applicativized\is{applicative} causatives (\sectref{ApplicativizedCausatives}).

\subsection{Morpheme order}\label{OrderOfExtensions}
When several extensions appear in a verbal base, their respective ordering is subject to several restrictions. The unproductive extensions appear closest to the root and follow the ordering illustrated in \figref{FigureUnproductiveExtensionsOrder}.

\begin{figure}	
\centering
\begin{tabular}{cc}
\midrule 
Positional\is{positional} & Impositive\is{impositive} \\ 
Extensive\is{extensive} & Separative\is{separative} \\ 
& Tentive\\
\midrule
\end{tabular} 
\caption{Order of unproductive verbal extensions}
\label{FigureUnproductiveExtensionsOrder}
\end{figure}

The following examples illustrate the attested combinations of unproductive extensions:\footnote{The verb \textit{pangalatʊla} `destroy by taking part after part out' (cf. \textit{pangʊla} `dismantle') has the sequence -\textit{al}-\textit{at}-\textit{ʊl}, which resembles the combination of \isi{extensive} plus \isi{tentive} plus separative.\is{separative} It is unclear if this is a chance resemblance or a case of three unproductive extensions.}
\begin{exe}
\ex
\begin{xlist}
\ex Positional\is{positional} and impositive:\is{impositive}\\
\textit{gundamika} `bend over/down (tr.)'
\ex Extensive\is{extensive} and impositive:\is{impositive}\\
\textit{lambalɪka} `put to bed'
\ex Extensive\is{extensive} and separative:\is{separative}\\
\textit{nyagalʊka} `get well (health)'
\end{xlist}
\end{exe}

\largerpage
The more productive extensions follow the unproductive ones. The passive (including the fossilized passive)\is{passive!productive}\is{passive!fossilized} always occupies the last position.
\begin{exe}
\ex \begin{xlist}
\ex Fossilized passive:\is{passive!fossilized}
\begin{tabbing}
\textit{bombeleligwa}x\=(\degree saam-ɪsi-ɪl-igʊ-a)x\=\kill
\textit{kʊbɪlwa}\>(\degree kʊb-ɪl-ʊ-a)\>`suffer'\\
\textit{ʊmɪlwa}\>(\degree ʊm-ɪl-ʊ-a)\>`be thirsty (for)'\\
\textit{agɪlwa}\>(\degree ag-ɪl-ʊ-a)\>`diminish (by); lack'\\
\textit{tumukɪlwa}\>(\degree tum-ʊk-ɪl-ʊ-a)\>`be(come) short on; late for'
\end{tabbing}
 \ex Productive passive:\is{passive!productive}
\begin{tabbing}
	\textit{bombeleligwa}x\=(\degree saam-ɪsi-ɪl-igʊ-a)x\=\kill
	\textit{bɪɪkɪligwa}\>(\degree bɪɪk-ɪl-igʊ-a)\>`be put for'\\
	\textit{meleligwa}\>(\degree mel-ɪl-igʊ-a)\>`owe'\\
	\textit{bombeleligwa}\>(\degree bomb-ɪlɪl-igʊ-a)\>`be weeded'\\
	\textit{manyisigwa}\>(\degree many-ɪsi-igʊ-a)\>`be taught'\\
	\textit{fumusigwa}\>(\degree fum-uk-i-igʊ-a)\>`be announced'\\
	\textit{ɪmikigwa}\>(\degree ɪm-ɪk-igʊ-a)\>`be respected'\\
	\textit{saamikɪsigwa}\>(\degree saam-ɪsi-ɪl-igʊ-a)\>\lq be made exile to'
\end{tabbing}
\end{xlist}
\end{exe}

A \isi{causative} -\textit{i}, either the short causative\textsubscript{1} \mbox{-\textit{i}} or the last segment of a split-up long causative\textsubscript{2} \mbox{-\textit{ɪs}-\textit{i}}, normally occupies the last position within the base,\is{base} unless it is followed by the passive.\is{passive!productive}\footnote{See \sectref{Causative1} for the process of \isi{spirantization} induced by the \isi{causative}\textsubscript{1} -\textit{i}, and \sectref{ApplicativizedCausatives} for the formation of applicativized causatives.\is{applicative}}
\begin{exe}
\ex\begin{xlist}
\begin{tabbing}
\textit{ʊlɪkɪsania}x\=(\degree pɪlɪk-ɪs<an>i-a)x\=\kill%Unsinnszeile für Tabulatoren
\textit{fulasania}\>(\degree fulal<an>i-a)\>`hurt each other'\\
\textit{ʊlɪkɪsania}\>(\degree ʊl-ɪkɪs<an>i-a)\>\lq sell each other sthg.'
\end{tabbing}
\end{xlist}
\end{exe}

Apart from these generalizations, the ordering of the productive extensions in Nyakyusa requires a dedicated study of its own, given the high number of logical possibilities and the question of how morpheme order, meaning and syntax relate to each other. As \citet{HymanL2002} points out, in many Bantu languages the relative order of certain verbal extensions follows a default pattern, which can have both a compositional reading (morpheme order reflecting semantic scope) and a non-compositional one, while the opposite order exclusively receives the compositional reading. Further, \citet{LusekeloA2013} indicates that the relative position of the applicative in Nyakyusa may be linked to the semantic role of the argument it licenses.

Lastly, a few cases of doubled verbal extensions are attested in the data. In most cases it is unclear what the semantic and syntactic functions of these are. Doubling of a verbal extension may serve the purpose of fulfilling the requirement for both a default morpheme order and a compositional order at the same time \citep{HymanL2002}, 
\begin{exe}
\ex \label{exExtensionDoubling}\begin{xlist}
\ex Two applicatives:\is{applicative}
\begin{tabbing}
\textit{tiimɪsyanisya}x\=`fight with each other for'x\= < \textit{tiima}x\=\kill
\textit{l\textbf{ɪɪl}an\textbf{il}a}\>`eat together with sb.' \> < \textit{lya}\>`eat'\\
\textit{lw\textbf{ɪl}an\textbf{il}a}\>`fight with each other for'\> < \textit{lwa}\>`fight'
\end{tabbing}
\ex Two reciprocals/associatives:\is{reciprocal}
\begin{tabbing}
\textit{tiimɪsyanisya}x\=`fight with each other for'x\= < \textit{tiima}x\=\kill
\textit{sop\textbf{an}il\textbf{an}a}\>`throw to each other'\> < \textit{sopa}\>`throw' %check semantics
\end{tabbing}
\ex Two causatives:\is{causative}
\begin{tabbing}
\textit{tiimɪsyanisya}x\=`fight with each other for'x\= < \textit{tiima}x\=\kill
\textit{tiim\textbf{ɪsy}an\textbf{isy}a}\>\lq make each other herd'\footnotemark \> < \textit{tiima}\>\lq herd'
\end{tabbing}
\end{xlist}
\end{exe}
\protect\footnotetext{This example was elicited on the basis of \citet{LusekeloA2012}.}
\label{ReciprocalAndCausative}
\subsection{Complex derivations: pluractional}\label{Pluractional}
\is{pluractional|(}
The combination of the \isi{reciprocal} and \isi{causative} extensions often gives a pluractional reading. The range of possible meanings includes re-iteration, intensification or the involvement of multiple subjects or objects (also cf. \citealt[79]{SchumannK1899}). This combination is used on transitive bases\is{base} and verbs of \isi{motion} (\ref{exPluractionalTransitive}, \ref{exPluractionalMotion}), the only attested exception being the intransitive \textit{sulumania} `afflict, be sorry'. With verbs denoting `to return', this combination gives a cyclic reading. When the short causative -\textit{i} is used, sometimes \isi{spirantization} takes place. This seems not to be predictable and is probably a function of time depth and lexicalization. Concerning the length\is{vowels!length} of the vocalic segment in the reciprocal, see \sectref{Reciprocal}.


\citet[86]{BotneR2008} and \citet{GrayMS} observe a similar pluractional function of \mbox{\textit{an}-\textit{y}-} / \mbox{\textit{an}-\textit{i}-} in neighbouring \ili{Ndali} and \ili{Kisi} G67, respectively, and \citet[557]{KisseberthC2003} for \ili{Makhuwa} P30 gives \textit{ú}-\textit{hókól}-\textit{an}-\textit{yáán}-\textit{ih}-a `to go and come back the same day'. All these suggest that the combination of the causative and the reciprocal yielding pluractionality might have a wider distribution in Bantu. 

\largerpage
The intransitive counterpart to the pluractional has the shape -\textit{anik} and can be analysed as consisting of the \isi{reciprocal} -\textit{an} and \isi{neuter} -\textit{ɪk} extensions. It is used on transitive as well as intransitive bases (\ref{exPluractionalIntransitive}).
\begin{exe}
\ex
\begin{xlist}
\ex Pluractionals derived from transitive verbs: \label{exPluractionalTransitive}
\begin{tabbing}
\textit{gomoka}x\=`bind; detain; fix'\=> \textit{nyambanika}x\= \kill%Unsinnszeile für Tabulatoren
\textit{buuta}\>`cut; slaughter'\> > \textit{buutania}\>`cut into pieces'\\
\>\> > \textit{buutanika}\>`break into pieces (intr.)'\\
\textit{joba}\>`speak to/about'\> > \textit{jobesania}\>`dispute about'\\
\>\> > \textit{jobanika}\>`speak much'\\ 
\textit{lʊnga}\>`add spice;\> > \textit{lʊngaania}\>`join, connect (tr.)'\\
\>put together'\> > \textit{lʊngɪsaania}\>`join, connect (tr.)'\\
\>\> > \textit{lʊngaanika}\>`be confused'\\
\textit{menya}\>`break; chop'\> > \textit{menyania}\>`chop into pieces'\\
\>\> > \textit{menyanika}\>`be chopped into pieces'\\ 
\textit{nyamba}\>`throw'\> > \textit{nyambania}\>`scatter'\\ \>\> > \textit{nyambanika}\>`disperse, be scattered'\\
\textit{pinya}\>`bind; detain; fix'\> > \textit{pinyania}\>`splice; tie together'\\
\>\> > \textit{pinyanika}\>`be spliced; tied manifold'\\ 
%\textit{suka}\>\lq wake sb. up'\> > \textit{sukaania}\>`shake (tr.)'\\
%\>\> > \textit{sukaanika}\>`shake (intr.)' %6 examples should suffice, because it is 2*3 for the others; also: page breaks; might be included again to adjust page breaks, etc
\end{tabbing}

\ex Pluractionals derived from verbs of motion:\is{motion} \label{exPluractionalMotion}
\begin{tabbing}
\textit{gomoka}x\=`bind; detain; fix'\=> \textit{nyambanika}x\= \kill%Unsinnszeile für Tabulatoren
\textit{buja}\>`return (to)'\> > \textit{bujɪsania}\>`go \& return (same day)'\\
\>\> > \textit{busania}\>`go \& return (same day)'\\
\textit{gomoka}\>`return; reprove'\> > \textit{gomosania}\>`go \& return (same day)'\\
\textit{kɪnda}\>\lq pass'\> > \textit{kɪɪsania}\>`pass by'
\end{tabbing}
\ex Pluractionals derived from intransitive verbs: \label{exPluractionalIntransitive}
\begin{tabbing}
	\textit{gomoka}x\=`cry; sound; mourn'x\=> \textit{nyambanika}x\= \kill%Unsinnszeile für Tabulatoren
	\textit{jeeta}\>\lq turn pale'\> > \textit{jeetanika} \>`faint'\\
	\textit{lɪla}\>`cry; sound; mourn'\> > \textit{lɪlanika}\>`complain'\\
	\textit{tʊʊja}\>`pant; breathe out'\> > \textit{tʊʊjanika}\>`pant heavily'
\end{tabbing}
\end{xlist}
\end{exe}
\is{pluractional|)}
\subsection{Applicativized causatives} \label{ApplicativizedCausatives}
\is{applicative|(}\is{causative|(}Applicativized causatives have a special form -(\textit{ɪ})\textit{kɪsi} / -(\textit{ɪ})\textit{kɪfi}. The alternations in vowel height as described in \sectref{MorphophonologyOfVerbalExtension} apply. When they are derived from a causativized \isi{base} subject to \isi{spirantization} (see §\ref{Causative1}) -\textit{kɪsi} / -\textit{kɪfi} is suffixed to the underlying non-causativized base,\is{base} with /k/ replacing the base-final consonant. The fricative is /f/ if the replaced consonant is a labial, and is /s/ elsewhere. In other words, it is the fricative that causative \isi{spirantization} would produce.

\begin{exe}
\ex \label{exApplCausSpirantizing}
\begin{tabbing}
\textit{fula\textbf{l}a} \=`get warm' \= > \textit{fula\textbf{sy}a} \=`return (tr.)' \= > \textit{fula\textbf{kIsy}a} \=\kill %Zeile nur fuer Tabs mit jeweils laengstem Element deshalb
%\textit{jonga} \lq run away, escape' > \textit{joosya} \lq elope; lose' > \textit{jookesya} \lq make escape to/for'
\textit{bu\textbf{j}a}\>`return'\> > \textit{bu\textbf{sy}a}\>`return (tr.)'\> > \textit{bu\textbf{kɪsy}a}\>`return (tr.) + \textsc{appl}'
\\\textit{fula\textbf{l}a}\>`be hurt'\> > \textit{fula\textbf{sy}a}\>`hurt'\> > \textit{fula\textbf{kɪsy}a}\> \lq hurt + \textsc{appl}'
\\\textit{o\textbf{g}a}\>`bathe'\> > \textit{o\textbf{sy}a}\>`baptise'\> > \textit{o\textbf{kesy}a}\>`baptise + \textsc{appl}'
\\\textit{pyʊ\textbf{p}a}\>`get warm'\> > \textit{pyʊ\textbf{fy}a}\>`warm up'\> > \textit{pyʊ\textbf{kɪfy}a}\>`warm up + \textsc{appl}'
\end{tabbing}
\end{exe}

If the causativized base is derived by causative\textsubscript{1} -\textit{i} following non-spirantizing\is{spirantization} consonants or if it is derived by causative\textsubscript{2} -\textit{ɪsi}, a suffix -\textit{ɪkɪsi} is attached to the non-causativized base:% (\ref{exApplCausNasal}, \ref{exApplCaus2}).

\begin{exe}
\ex
\begin{tabbing}
\textit{saama} \=`migrate' \= > \textit{saam-y-a} \=`transfer' \= > \textit{saam-ikɪsy-a} \=\kill %Zeile nur fuer Tabs mit jeweils laengstem Element deshalb
\textit{saama}\>`migrate'\> > \textit{saam-y-a}\>`transfer'\> > \textit{saam-ikɪsy-a}\>`transfer + \textsc{appl}'
\\\textit{ʊla}\>`buy'\> > \textit{ʊl-ɪsi-a}\>`sell'\> > \textit{ʊl-ɪkɪsy-a}\>`sell to/for/at'
\end{tabbing}
\end{exe}

This uncommon phonological realization of applicativized causatives has been noticed from the first treatments of Nyakyusa on. \citet{MeinhofC1966} as well as \citet{SchumannK1899} mention this and \citet{EndemannK1900} presents an attempt at a purely phonological explanation. Unfortunately, Endemann does not discuss cases of the suffixing of \mbox{-\textit{ɪkɪsi}} after nasals, which cannot be accounted for by his approach. In his later grammar sketch, he further gives a rather curious explanation in which he tries to link this phenomenon to distal/itive \textit{ka}- \citep[51]{EndemannC1914}.\is{itive}

The examples given by \citet[76]{BotneR2008} suggest a comparable formation in \ili{Ndali}. \citet[63]{WolffR1905} describes a similar phenomenon for neighbouring Kinga,\il{Kinga} where applicativized causativizes take the shape -\textit{ihitsa}, although without replacing the final consonant.

Alhough non-transparent from a synchronic point of view, the morphophonology of applicativized causatives finds a diachronic explanation in a sequence of analogy formations, as \citet{HymanL2003b} plausibly illustrates (\citealt[266f]{BergerP1938} develops a parallel interpretation). In this scenario, the point of departure would have been a stage in which infixing of the applicative, together with a cyclic application of \isi{spirantization}, took place (thus e.g. \textit{sook-a} > \textit{soos-i-a} > \textit{soos-el-i-a} > \textit{*soos-es-i-a}), followed by despirantization\is{spirantization} of the root-final consonant (> \textit{sook-es-i-a}). See \ili{Nyamwezi} F22 \citep[20--22]{SchadebergTMagangaC1992} for a comparable case. In the next stage, despirantization\is{spirantization} to /k/ was generalized; note that \isi{spirantization} leads to a merger of the six non-labials affected. In the case of non-labials, one possible interpretation of the sequence /kɪs/ would be that the final consonant spirantized\is{spirantization} in the first place was being post-posed. This re-analysis was then extended to cases of labials, yielding -\textit{kɪfi}. Once established, this pattern of applicatived causativizes surfacing as /kɪ\{s, f\}i/ was extended to non-spirantizing\is{spirantization} consonants (introducing what Hyman labels an ``extra k'') and thus fully generalized.

The main source for Hyman's interpretation is the chronolect described by \citet{SchumannK1899}, in which causative\textsubscript{1} -\textit{i} is the most productive of the two derivations. The present data show that Nyakyusa has gone one step further, extending this pattern to causative\textsubscript{2} -\textit{ɪsi} and thus generalizing the requirement that any applicativized causative must surface with the sequence /kɪ\{s, f\}/ (and respective vowel alternations).\footnote{\citet{LusekeloA2012} also discusses the shape of applicativized causatives in Nyakyusa. Though he rejects \citeauthor{HymanL2003b}'s (\citeyear{HymanL2003b}) analysis, at no point throughout his work does he give either an alternative reconstruction for the diachronic origin of these forms, or a motivated explanation for their spread to non-spirantizing bases and forms containing the long causative\textsubscript{2} -\textit{ɪsi}.}
\is{applicative|)}\is{causative|)}
\section{Denominal verbs}\label{Denominal}
\is{denominal verbs|(}
A number of verbal stems are derived from nominals (including adjectives) by means of suffixation. This seems not to be a particularly productive process. Three monosegmental suffixes are found in denominal verbs: while -\textit{p} yields intransitive verbs (\ref{exDenominalP}), -\textit{l} seems to yield only transitives (\ref{exDenominalL}). -\textit{k} (\ref{exDenominalK}) is not associated with any specific valency. Further, the separative extensions are employed in noun/adjective to verb derivation; see \sectref{Separative}.
\begin{exe}
	\ex\label{exDenominalP}
	\begin{tabbing}
		\textit{tʊngʊlʊpa}x\=`be(come) fermented, sour'x\=< \textit{ʊbʊtʊngʊlʊ}x\=\kill%Unsinnszeile für Tabulatoren
		\textit{kalɪpa}\>`be(come) fermented, sour'\> < \textit{kalɪ}\>`spicy; strict; sour'
		\\\textit{kiikʊlʊpa}\>`grow up to puberty (girl)'\> <	\textit{ʊnkiikʊlʊ}\>`woman'
		\\\textit{kʊʊlʊpa}\>`become old and worn'\> < \textit{kʊʊlʊ}\>`old'
		\\\textit{tʊngʊlʊpa}\>`lie'\> < \textit{ʊbʊtʊngʊlʊ}\>`lie'
		\\\textit{tungupa}\>`lie'\> < \textit{ʊlʊtungu}\> \lq testicle'\footnotemark
	\end{tabbing}
	\footnotetext{While `testicle' > `lie' might at first seem an implausible metaphor, note that much of Nyakyusa profanity is based on body parts; see \citet[146]{MeyerT1989}.}
	\ex\label{exDenominalK}
	\begin{tabbing}
		\textit{tʊngʊlʊpa}x\=`be(come) fermented, sour'x\=< \textit{ʊbʊtʊngʊlʊ}x\=\kill%Unsinnszeile für Tabulatoren
		\textit{bulika}\>`hit with the fist'\> < \textit{ɪkɪbuli}\>`fist'
		\\\textit{mulika}\>`glow'\> < \textit{ɪɪmuli}\>`light, brightness'
		\\\textit{tolika}\>`drip'\> < \textit{iitoli}\>`drop'
		\\\textit{pafuka}\>`be greedy'\> < \textit{pafu}\>`greedy'
		\\\textit{tiitʊka}\>`be(come) dark, black'\> < \textit{tiitʊ}\>`black'
	\end{tabbing}
	\ex\label{exDenominalL}
	\begin{tabbing}
		\textit{tʊngʊlʊpa}x\=`be(come) fermented, sour'x\=< \textit{ʊbʊtʊngʊlʊ}x\=\=\kill%Unsinnszeile für Tabulatoren
		\textit{heelula}\>`abuse'\> < \textit{iiheelu}\>`abusive language'\\
		\textit{tusula}\>`shoot'\> < \textit{ɪndusu}\>`gun, rifle'
	\end{tabbing}
\end{exe}
\is{denominal verbs|)}


\section{Partial reduplication}\label{PartialReduplication}\is{reduplication|(}
A number of verbs begin with two identical sequences of consonant and vowel. For many of these, the path of derivation is hard to track down. Possible sources are partial reduplication of a verbal base,\is{base} reduplicated nouns or ideophones.\is{ideophones} See also \citet[79]{SchadebergT2003a} for a pan-Bantu perspective and \citet[262]{SeidelF2008} for a similar observation in \ili{Yeyi} R41. For some of these verbs, a semantic element of repetition, oscillation or intensification can be observed. Examples are given in (\ref{exPartielleOhneSuffixe}), ordered by the attested patterns of reduplication.


\begin{exe}
\ex \begin{xlist} \label{exPartielleOhneSuffixe}
\ex Shape C\textsubscript{1}(G\textsubscript{1})V\textsubscript{1}.C\textsubscript{1}(G\textsubscript{1})V\textsubscript{1}C\textsubscript{2}:
\begin{tabbing}
\textit{mwemweka}x\=`shake, shiver; be afraid; care for'x\=cf. mwekax\=\kill%unsinnszeile fuer tabs
\textit{fufula}\>`endure, tolerate'\> < \textit{fula} \lq hurt'?\\
\textit{lelema}\>`have shaky voice'\\
\textit{tetema}\>`shake, shiver; be afraid; care for'\\
\textit{bwabwata}\>`blab, talk nonsense'\\
\textit{bwabwaja}\>`blab, talk nonsense'\\
\textit{mwemweka}\>`glitter'\>cf. \textit{mweka} \lq glow'
\end{tabbing}

\clearpage

\ex Shape C\textsubscript{1}V\textsubscript{1}.C\textsubscript{1}V\textsubscript{1}V\textsubscript{1}C\textsubscript{2}:
\begin{tabbing}
xxxxxxxxxxx\=xxxxxxxxxxxxxxxxxxxx\=xxxxxxxxxx\=\kill
\textit{boboota}\>`grunt'\\
\textit{hohoola}\>`jeer, laugh at'\\
\textit{ng'ong'oola}\>`grimace at'\\
\textit{ng'ung'uuta}\>`whine'\\
\textit{nyinyiila}\>\lq squint'\\
\textit{sisiila}\>`close (eyes)'\\	%<	\textit{sila} \lq refuse'?
\textit{sosoola}\>`point finger at'
\end{tabbing}
\end{xlist}
\end{exe}
The list in (\ref{exPartieleWithSuffixes}) illustrates verbs containing reduplication of the initial syllable as well as further suffixes.
\begin{exe}
\ex \begin{xlist} \label{exPartieleWithSuffixes}
\ex Shape C\textsubscript{1}(G\textsubscript{1})V\textsubscript{1}.C\textsubscript{1}(G\textsubscript{1})V\textsubscript{1}C\textsubscript{2}:
\begin{tabbing}
\textit{i-ng'weng'wesya}x\=x\lq `overturn; annulx\=< \textit{sanusya}x\=\kill%unsinnszeile für tabs
\textit{kakajʊla}\>`break by chewing'\> < \textit{kajʊla}\>`force open'\\
\textit{myamyasya}\> \lq smoothen out'\> < \textit{myasya}\>`smear, spread'\\
\textit{nyenyemusya}\>`excite'\\
\textit{i-ng'weng'wesya}\>`grumble'\\
\textit{papatana}\>`be squeezed'\> < *\textit{pát}\>`hold'\\
\textit{popotoka}\>`bend, twist'\> < \textit{pota}\>`steer; twist'\\
\textit{popotola}\>`strain; strangle'\\
\textit{sasanusya}\>`overturn; annul'\> <
\textit{sanusya}\>\lq altern'\\
\textit{sosomela}\>`protude'
\end{tabbing}
\ex Shape C\textsubscript{1}V\textsubscript{1}.NC\textsubscript{1}V\textsubscript{1}C\textsubscript{2}:
\begin{tabbing}
\textit{i-ng'weng'wesya}x\=x\lq `overturn; annul'x\=< \textit{sanusya}x\=\kill%unsinnszeile für tabs
\textit{jenjelʊka}\>`dawn'\\
\textit{junjumala}\>`crouch'
\end{tabbing}
\ex Shapes C\textsubscript{1}V\textsubscript{1}.C\textsubscript{1}V\textsubscript{1}V\textsubscript{1}C\textsubscript{2} / C\textsubscript{1}V\textsubscript{1}.C\textsubscript{1}V\textsubscript{1}NC\textsubscript{2}:
\begin{tabbing}
\textit{i-ng'weng'wesya}x\=x `overturn; annul'x\=< \textit{sanusya}x\=\kill%unsinnszeile für tabs
%papaalɪkɪsya?
\textit{bobonjala}\>`be(come) flat'\\
\textit{babanjala}\>`be(come) flat'\\
\textit{luluutila}\>\lq ululate'\\
\textit{nyonyoofya}\>`attract desire'\> < \textit{nyoofya}\>`allure'\\
\textit{pʊpʊʊtɪka}\>`stagger'
\end{tabbing}
\end{xlist}
\end{exe}
\is{reduplication|)}
\is{derivation|)}