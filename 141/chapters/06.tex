\chapter{Tense and aspect constructions 1: present and past tense}\label{TenseAspectConstructions}
\largerpage
\section{Introduction}
In this and the following two chapters, constructions expressing \isi{tense} and grammatical aspect\is{aspect!grammatical} will be described. This chapter contains a general overview of tense and aspect in Nyakyusa, followed by a discussion of negation\is{negative} (\sectref{Negations}) and an investigation into two recurring aspectual suffixes, which show considerable morphophonemic alternation (\sectref{AlternationsIPFVaga}, \ref{Imbrication}). Its main body consists of a description of present and past tense constructions. This description is divided into constructions consisting of just the inflected verb (\sectref{SimpleConstructions}) and \isi{auxiliary} or compound constructions (\sectref{ComplexConstructions}). What is for convenience termed \lq present tense'\is{tense!present} throughout this study can be understood as having non-past reference. This is discussed in \sectref{PRSasNonPST}. Note that the dedicated narrative markers,\is{narrative markers} though they have past tense\is{tense!past} reference, will be dealt with separately in Chapter \ref{NarrativeMarkers}.

\section{Overview of tense and aspect in Nyakyusa}
A key element of the Nyakyusa TMA system in the present\is{tense!present} (non-past) and past tense\is{tense!past} is the opposition between imperfective\is{aspect!imperfective} and perfective aspect.\is{aspect!perfective} The use of grammatical aspect\is{aspect!grammatical} is closely linked to the lexical opposition between inchoative\is{inchoative verbs} and non-inchoative verbs (see Chapter \ref{AspectualClassification}). A central element here is the completion of the Nucleus phase\is{phase!Nucleus phase} of an eventuality, which is discussed in \sectref{PerfectivityCompletion}.  As each of the major present and past tense\is{tense!present}\is{tense!past} constructions is also marked for aspect, neither \isi{tense} nor grammatical aspect\is{aspect!grammatical} can be considered primary. Instead, when linguistically construing  a given state-of-affairs, a speaker of Nyakyusa has to decide on both the temporal dimension\is{tense} (\sectref{Tense}) as well as the aspectual vantage point (\sectref{Aspect})\is{aspect!grammatical} in relation to the verb's inherent aspectual potential\is{aspect!Aristotelian} (\sectref{AristotelianAspect}).

Concerning constructions with future time reference, which build on the simple present,\is{simple present} as well as the subjunctive\is{mood!subjunctive} and desiderative moods,\is{mood!desiderative} Nyakyusa rather has an opposition between aspectually neutral and imperfective aspect,\is{aspect!imperfective} where the latter frequently also adds modal\is{modality} nuances.

\section{Negation in Nyakyusa}\label{Negations}
\is{negative|(}In \sectref{SimpleConstructions}, \ref{ComplexConstructions}, \ref{Subjunctive}, the description of each affirmative construction will be followed by its negative counterpart. As has been pointed out by \citet{ContiniMoravaE1989}, among others, the semantic relationship between affirmative and negative forms is not a straightforward one. Nevertheless, when language assistants were asked for the negative equivalent of a given form, they responded readily and unanimously (see also \citealt[196]{NurseD2008}). Negation in Nyakyusa shows some uncommon characteristics, however, which deserve a short discussion.

Nyakyusa has three negative prefixes: \textit{ti}-, \textit{ka}- and \textit{nga}-. All three stand in the  post-initial slot. Their distribution is delimited along two major lines: mood and temporal reference. The \textit{nga}- negation is limited to the negative subjunctive (\sectref{NegativeSubjunctive}). Of the remaining two negative prefixes, \textit{ka}- is used in constructions that make reference to a point of time prior to the moment of speech,\footnote{The moment of speech is to be understood here as the default reference point.} while \textit{ti}- occurs solely in the present and in futurates.\footnote{This contrasts with Ndali, which has \textit{ta}- for non-pasts as well as pasts (\citealt{BotneR2008}; \citealt{SwillaI1998}). The \ili{Ngonde} varieties described by \citet{KishindoP1999} and \citet{LabroussiC1998} seem to exhibit a certain variability and stand between Nyakyusa and \ili{Ndali} with regards to the negative markers.} The apparent exception is the negative copula, which for the non-generic\is{aspect!generic} present is \textit{ka-j-a} (\sectref{Copulae}). However, this can clearly be attributed to its origin in the negation of \lq become', which, depending on the context, is still a possible reading. What has just been outlined is uncommon in two ways. First, in the majority of those Bantu languages having more than one negative marker, these occur in different positions in the verbal word. Second, in those Bantu languages with three negative markers, the typical distribution is main clause vs. subjunctive vs. relative clause \citep[184--191]{NurseD2008}.

A common typological criterion concerning verbal negation is symmetry. As \citet[556]{MiestamoM2007} defines it, ``symmetric negative constructions do not differ from non-negatives in any other way than by the presence of the negative marker(s)''.\footnote{More precisely, Miestamo in the quoted section is concerned with \lq standard negation', that is, the negation of declarative verbal main clauses, excluding, for instance, existential or copula clauses or non-declarative ones such as the imperative.} While negation in Nyakyusa is mostly symmetric, as can be seen from the sample constructions in \tabref{tableSymmetricNegation}, there are several cases of asymmetric negation; see \tabref{tableAsymmetricNegation}. There are also cases of syncretism, where more than one affirmative paradigm shares a common negative counterpart; see \tabref{tableSyncretismNegation}. \citet{MiestamoM2005} calls the latter ``paradigmatic A/Cat asymmetry''.


\begin{table}[H]
\begin{center}
\begin{tabular}{ccc}
\lsptoprule 
& \footnotesize{Affirmative} & \footnotesize{Negative}\\ 
\midrule 
Simple present & \textit{kʊ}-\textsc{vb}-\textit{a} & \textit{ti}-\textit{kʊ}-\textsc{vb}-\textit{a}\\ 
Past perfective & \textit{a(lɪ)}-\textsc{vb}-\textit{ile} & \textit{ka}-\textit{a(lɪ)}-\textsc{vb}-\textit{ile}\\ 
Past imperfective & \textit{a}-\textsc{vb}-\textit{aga} & \textit{ka}-\textit{a}-\textsc{vb}-\textit{aga}\\ 
\lspbottomrule
\end{tabular} 
\caption{Cases of symmetric negation}
\label{tableSymmetricNegation}
\end{center}
\end{table}

\begin{table}[H]
\begin{center}
\begin{tabular}{ccc}
\lsptoprule 
& \footnotesize{Affirmative} & \footnotesize{Negative}\\ 
\midrule 
Present perfective & \textit{ø}-\textsc{vb}-\textit{ile} & \textit{ka}-\textsc{vb}-\textit{a}\\ 
Subjunctive & \textit{ø}-\textsc{vb}-\textit{e}(\textit{ge}) & \textit{nga}-\-\textsc{vb}-\textit{a}(\textit{ga})\\
Narrative tense & \textit{lɪnkʊ}-\textsc{vb}-\textit{a} & \textit{lɪnkʊ}-\textit{sit}-\textit{a} + infinitive\\
\lspbottomrule
\end{tabular} 
\caption{Cases of asymmetric negation}
\label{tableAsymmetricNegation} 
\end{center}
\end{table}

\begin{table}[H]
\begin{center}
\begin{tabular}{cc}
\lsptoprule 
\footnotesize{Affirmative} & \footnotesize{Negative}\\ 
\midrule 
Simple present & \multirow{2}{*}{Negative present}\\
Present progressive & \\ \hline
Past imperfective & \multirow{2}{*}{Negative past imperfective}\\
Past progressive & \\ \hline
Subjunctive & \multirow{2}{*}{Negative subjunctive}\\
Distal/itive subjunctive& \\
\lspbottomrule
\end{tabular}
\caption{Syncretism in negation}
\label{tableSyncretismNegation}  
\end{center}
\end{table}
\is{negative|)}

\newpage 
\section{Morphophonology of common TMA suffixes}
\subsection{Alternations of imperfective -\textit{aga}}\label{AlternationsIPFVaga} 
\is{aspect!imperfective|(}The imperfective suffix surfaces as -\textit{aga} in those paradigms characterized by the default final vowel -\textit{a}, and as -\textit{ege} in the affirmative subjunctive.\is{mood!subjunctive} The defective verb \textit{tɪ} (\sectref{defectiveti}) occurs as \textit{tɪgɪ}.
\begin{exe}
\ex
\begin{tabular}[t]{ll}
\textit{twajobaga}&`we were speaking'
\\\textit{tʊjobege}&\lq we should be speaking'
\\\textit{twatɪgɪ}&`we were saying'
\end{tabular}
\end{exe}

When one of the clitics\is{enclitic} (see \sectref{PostfinalClitics}) \textit{=po}, \textit{=mo} or \textit{=ko} (independent of their specific function) or \textit{=kʊ} \lq where' follows, the velar segment is prenasalized (\ref{exEncliticsAgaPrenasalized}). With the \isi{enclitic} form of \textit{ki} `what', no prenasalization takes place (\ref{exEncliticsAgaKiNotPrenasalized}).
\begin{exe}
\ex\label{exEncliticsAgaPrenasalized}
\begin{tabbing}
\textit{gwabʊʊk\textbf{anga}kʊ}x\=(\degree ba-a-sook-aga=mo)x\=\kill
\textit{baaswɪl\textbf{anga}po}\>(\degree ba-a-swɪl-aga=po)\>`they were raising there (class 16)'\\
\textit{biigal\textbf{anga}ko}\>(\degree ba-a-igala-aga=ko)\>`they were closing there (class 17)'\\
\textit{baasook\textbf{anga}mo}\>(\degree ba-a-sook-aga=mo)\>`they were going outside (class 18)'\\
\textit{ʊswɪl\textbf{enge}po}\>(\degree ʊ-swɪl-ege=po)\>`you should raise there (class 16)'\\
\textit{gwigal\textbf{enge}ko}\>(\degree ʊ-igal-ege=ko)\>`you should close there (class 17)'\\
\textit{ʊsook\textbf{enge}mo}\>(\degree ʊ-sook-ege=mo)\>`you should go outside (class 18)'\\
\textit{gwabʊʊk\textbf{anga}kʊ}\>(\degree ʊ-a-bʊʊk-aga=kʊ)\>`Where were you going?'\\
\textit{mbʊʊk\textbf{enge}kʊ}\>(\degree n-bʊʊk-ege=kʊ)\>`Where should I go?'
\end{tabbing}
\ex\label{exEncliticsAgaKiNotPrenasalized}\begin{tabbing}
\textit{gwabʊʊk\textbf{anga}kʊ}x\=(\degree ba-a-sook-aga=mo)x\=\kill
\textit{gwabomb\textbf{aga}ki}\>(\degree ʊ-a-bom-aga=ki)\>`What were you doing?'
\end{tabbing}
\end{exe}
\is{aspect!imperfective|)}
\subsection{Perfective -\textit{ile} and its variants}\label{Imbrication} 
\is{imbrication|(}\is{aspect!perfective|(}
Perfective stems in Nyakyusa are subject to complex allomorphic variation. With the widespread Bantu suffix -\textit{ile} as the underlying form, surface forms are diverse and in many cases show characteristics of fusional morphology. Before going into detail with the numerous shapes perfective stems take and the phonological and morphological factors triggering the choice of these, it should be stated that perfective stem formation can be understood as variation on three re-occurring themes. The first and most straightforward is suffixation of -\textit{ile}:
\begin{exe}
\ex
\begin{tabular}[t]{@{}>{\itshape}llll}
\textit{nwa}&`drink'& > \textit{nwile}
\\\textit{gana}&`love'& > \textit{ganile}
\\\textit{nyunyuuta}&`whine'& > \textit{nyunyuutile}
\end{tabular}
\end{exe} 
The second theme is called \textit{imbrication}, a term coined by \citet{BastinY1983}. In its prototypical form, imbrication consists of infixing -\textit{i}- before the last base consonant and suffixing -\textit{e}. The rules of word-internal hiatus solution (\sectref{HiatusSolution}) apply.
\begin{exe}
\ex
\begin{tabular}[t]{@{}>{\itshape}llll}
\textit{lwasya}&`nurse the sick, care for'& > \textit{lwa<\textbf{i}>sy-\textbf{e}}& > \textit{lwesye}
\\\textit{bukuka}&`flare'& > \textit{buku<\textbf{i}>k-\textbf{e}} & > \textit{bukwike}
\\\textit{ambɪlɪla}&`receive; entertain guests'& > \textit{ambɪlɪ<\textbf{i}>l-\textbf{e}}& > \textit{ambɪliile}
\end{tabular}
\end{exe} 
A third theme will be referred to as \textit{copying}. In its basic form, copying consists of a surface alternation /-CG/ $\rightarrow$ /-CiiCGe/, where CG stands for the base-final consonant plus a following glide. Although at first this looks like a case of reduplication,\is{reduplication} the discussion below will show that these forms can best be explained by assuming \isi{templatic} requirements of certain types of stems.
\begin{exe}
\ex
\begin{tabular}[t]{@{}>{\itshape}lll}
\textit{busya}&`return (tr.)'& > \textit{busiisye}
\\\textit{pʊfya}&`whistle'& > \textit{pʊfiifye}
\\\textit{ibwa}&`forget'& > \textit{ibiibwe}
\end{tabular}
\end{exe}

The following in-depth description of perfective stem formation will be structured according to the syllable count of the verbal base, as this is a major conditioning factor and each syllable count can be associated with a default process. Some of the regularities outlined have already been recognized by \citet{BergerP1938}. However, the present data shows a number of deviations from Berger's analysis, which in most cases can be attributed to the greater quantity of data considered.\footnote{Berger himself recognizes the limits of his corpus and that the transcription of some of his second-hand data is rather dubious. Nevertheless, his work is a valuable point of departure. Apart from \citet{BergerP1938}, the forms cited in \citet{FelbergK1996} have been taken as an indication of where to look for regularity and variation. All forms stemming from those sources that were felt to be suspicious have been checked in elicitation.} This is based on the examination of some 1600 verbal bases for which perfective stems were available at the time of writing this section.

\subsubsection{Monosyllabic verbs}
% \todo{is the extra tab in fwile intended?}
With monosyllabic verbs, -\textit{ile} is suffixed. The general rules of vowel juxtaposition apply (\sectref{HiatusSolution}). Defective \textit{tɪ} (\sectref{defectiveti}) yields \textit{tile}, which is often reduced to [tʰi̯ɛ]. Likewise, \textit{jile} is often heard as [ɟi̯ɛ]. In both cases, \isi{stress} remains on the stem syllable.
\begin{exe}
%\begin{multicols}{2}
\ex
\begin{tabbing}
\textit{kya}x \=`dawn; cease\= to rain'x\= > \textit{kiile}\kill
\textit{pa}\>`give'\> > \textit{peele}\\
\textit{ja}\>`be(come)'\> > \textit{jile}\\
\textit{tɪ}\>`say'\> > \textit{tile} \\
\textit{twa}x\=`be plenty (esp. fish)'x\= > \textit{twile}\kill %Unsinnszeile für neue tabs
\textit{fwa}\>`die'\> > \textit{fwile} \\
\textit{gwa}\>`fall'\> > \textit{gwile} \\
\textit{kwa}\>`pay dowry'\> > \textit{kwile} \\% \columnbreak
\textit{lwa}\>`fight'\>\> > \textit{lwile} \\ 
\textit{mwa}\>`shave'\>\> > \textit{mwile} \\
\textit{nwa}\>`drink'\>\> > \textit{nwile} \\
\textit{swa}\>`spit; forgive'\>\> > \textit{swile} \\
\textit{twa}\>`be plenty (esp. fish)'\>\> > \textit{twile}\\
\textit{kya}\>`dawn; cease to rain'\>\> > \textit{kiile}\\
%\end{tabbing}
%\end{multicols}
%\end{exe}
%\clearpage
%\begin{exe}
%\sn
%\begin{tabbing}
%\textit{pya}x\=`be(come) burnt'x\= > \textit{piile}\kill
\textit{lya}\>`eat'\> > \textit{liile} \\
\textit{nia}\>`defecate'\> > \textit{niile} \\
\textit{pya}\>`be(come) burnt'\> > \textit{piile}\\
\textit{sya}\>`grind'\> > \textit{siile}
\end{tabbing}
\end{exe}

\subsubsection{Disyllabic verbs} \label{ImbricationDisyllabic}
Disyllabic verbs show by far the most complex variation. Suffixation of -\textit{ile}
can be considered the default case:
\begin{exe}
\ex
\begin{tabular}[t]{@{}>{\itshape}lll}
\textit{goga}&`kill'& > \textit{gogile}\\
\textit{keeta}&`watch'& > \textit{keetile}\\
\textit{konga}&`follow'& > \textit{kongile}\\
\textit{ʊla}&`buy'& > \textit{ʊlile}
\end{tabular}
\end{exe}


With disyllabic applicatives (that is, applicatives of monosyllabic roots) -\textit{iile}, with a long morpheme-initial vowel,\is{vowels!length} is suffixed.
\begin{exe}
\ex
\begin{tabbing}
\textit{nwela}x\=`defecate with/at/for'x\=\kill%unsinnige Zeile nur fuer tabs
\textit{peela}\>`give off'\> > \textit{peeliile}\\
\textit{jɪɪla}\>`be in a condition'\> > \textit{jɪɪliile}\\
\textit{lɪɪla}\>`eat with/at/for'\> > \textit{lɪɪliile}\\
\textit{niela}\>`defecate with/at/for'\> > \textit{nieliile}\\
\textit{syela}\>`grind with/at/for'\> > \textit{syeliile}\\
\textit{gwɪla}\>`fall with/to/for'\> > \textit{gwɪliile}\\
\textit{nwela}\>`drink with/at/for'\> > \textit{nweliile}
\end{tabbing}
\end{exe}

\largerpage
With disyllabic fossilized passives\is{passive!fossilized} (\sectref{FossilizedPassive}), -\textit{il}- is infixed before the glide and -\textit{e} is suffixed (\ref{exFossilizedPassivesImbrication}). These verbs thus occupy an intermediate position between suffixing and imbrication in the strict sense.\footnote{\citet{SchumannK1899} and \citet{BergerP1938} note that these verbs form their perfective stem with long -\textit{iilwe}.\is{vowels!length} This could not be confirmed and might be due to diatopic variation or due to confusion with their applicativized forms.} The same holds for one of the perfective stem variants of \textit{lɪɪgwa} \lq be eaten' (\ref{exlIIgwaSemiImbrication}).
\begin{exe}
\ex\label{exFossilizedPassivesImbrication}
\begin{tabbing}
\textit{nyonywa}x\=`be burdened'x\= > \textit{nyonyilwe}\kill
\textit{babwa}\>`be in pain'\> > \textit{babilwe}\\ 
\textit{gogwa}\>`dream'\> > \textit{gogilwe}\\
\textit{milwa}\>`drown'\> > \textit{mililwe}\\
\textit{nyonywa}\>`desire'\> > \textit{nyonyilwe}\\
\textit{syʊkwa}\>`miss sadly'\> > \textit{syʊkilwe}\\
\textit{tolwa}\>`be burdened'\> > \textit{tolilwe}
\end{tabbing}
\ex \label{exlIIgwaSemiImbrication}
\begin{tabbing}
\textit{nyonywa}x\=`be burdened'x\= > \textit{nyonyilwe}\kill
\textit{lɪɪgwa}\>`be eaten'\> > \textit{lɪɪgilwe} (also \textit{lɪɪgiigwe}, see below)
\end{tabbing}
\end{exe}

Verbs of the shapes CG\textit{al} and CG\textit{an} induce imbrication (\ref{exImbricationalaman}). One exception to this rule is attested (\ref{exCGAXloansNoImbrication}). The unusual sequence /ŋw/ indicates that the verb \textit{ng'wala} might be a \ili{Ndali} loan.\footnote{Cf. pairs such as Nyakyusa \textit{nwa}, \ili{Ndali} \textit{ŋwa} \lq drink' < PB *\textit{ɲwó}.} \citet{BergerP1938} further lists \textit{nywama} > \textit{nyweme}, thus CG\textit{am}. Of the speakers consulted, several did not know this verb at all. Those familiar with it unanimously gave \textit{nywamile} as their first answer. Some accepted \textit{nyweme} as a variant perfective stem, whereas it was rejected by others (\ref{exCGAXnywama}). The only other verb of the shape in question is \textit{kwama} \lq be(come) stuck', an obvious loan from Swahili, which has the perfective stem \textit{kwamile}  not *\textit{kweme}. As the examples in (\ref{exCGACnoImbrication}) illustrate, other CG\textit{a}C shapes do not induce imbrication.

\begin{exe}
\ex
\begin{tabbing}
\textit{nywama}x\=\lq resemble; be enough'x\=\kill  %unsinnige Zeile fur Tabulatoren
\textit{fwala}\>`wear; receive salary'\> > \textit{fwele}
\\\textit{syala}\>`remain'\> > \textit{syele}\label{exImbricationalaman}
\\\textit{fwana}\>\lq resemble; be enough'\> > \textit{fwene}
\\\textit{lwana}\>`quarrel'\> > \textit{lwene}
\end{tabbing}
\ex\label{exCGAXloansNoImbrication}
\begin{tabbing}
\textit{nywama}x\=\lq resemble; be enough'x\=\kill  %unsinnige Zeile fur Tabulatoren
\textit{ng'wala}\> \lq scratch with claws'\> > \textit{ng'walile} (not *\textit{ng'wele})
\end{tabbing}
\ex\label{exCGAXnywama}
\begin{tabbing}
\textit{nywama}x\=\lq resemble; be enough'x\=\kill  %unsinnige Zeile fur Tabulatoren
\textit{nywama}\>`enlarge (intr.); chase'\> > \textit{nywamile} (also \textit{nyweme})
\end{tabbing}
\ex\label{exCGACnoImbrication}
\begin{tabbing}
\textit{nywama}x\= \lq resemble; be enough'x\=\kill  %unsinnige Zeile fur Tabulatoren
\textit{fyata}\>`fasten'\> > \textit{fyatile} (not *\textit{fyete})
\\\textit{kwaba}\>`take'\> > \textit{kwabile} (not *\textit{kwebe})
\end{tabbing}
\end{exe}

The verb \textit{baala} \lq increase, thrive' shows variation between an imbricating and a suffixing form (\ref{exBaalaImbrication}). This must be considered an idiosyncrasy, as no other verb of the shape /Caal/ has an imbricating perfective stem (\ref{exCaalNoImbrication}).

\begin{exe}
\ex\label{exBaalaImbrication}
\begin{tabbing}
\textit{paala}x\= \lq get drunk, be drunk'x\=\kill  %unsinnige Zeile fur Tabulatoren
\textit{baala}\> \lq increase; thrive' \>> \textit{beele} / \textit{baalile}
\end{tabbing}
\ex\label{exCaalNoImbrication}
\begin{tabbing}
\textit{paala}x\= \lq get drunk, be drunk'x\= > \textit{paalile}x\=\kill  %unsinnige Zeile fur Tabulatoren
\textit{gaala}\> \lq get drunk, be drunk' \> > \textit{gaalile}\>(not *\textit{geele})\\
\textit{paala}\> \lq invite' \> > \textit{paalile}\>(not *\textit{peele})\\
\textit{saala}\> \lq be(come) happy' \> > \textit{saalile}\>(not *\textit{seele})
\end{tabbing}
\end{exe}

Two further disyllabic verbs must be considered irregular: \textit{manya} \lq know' and \textit{bona} \lq see'. These trigger imbrication although no other regularity can account for this. Further, \textit{bona} does not yield *\textit{bwine} as would be expected from the rules of vowel coalescence.
\begin{exe}
\ex
\begin{tabular}[t]{@{}>{\itshape}lll}
\textit{manya} &`know' & > \textit{meenye}
\\\textit{bona} & `see' & > \textit{bwene}
\end{tabular}
\end{exe}

\is{causative|(}
Disyllabic causatives trigger copying, yielding (C)(G)V\textit{siisye} if the base ends in /sy/ and (C)(G)V\textit{fiifye} if the base ends in /fy/. This holds for causatives\is{causative} of monosyllabic roots\is{root} formed with the long causative\textsubscript{2} \mbox{-\textit{ɪsi} }(\ref{exCopyingBiCausativeLong}), as well as for those causatives formed with the short causative\textsubscript{1} \mbox{-\textit{i}} on disyllabic verbs (\ref{exCopyingBiCausativeShort}); see \sectref{TwoCausatives} for a discussion of the two causatives. Causatives with base-final nasals also have a perfective stem of the shape (C)VN\textit{iisye} (\ref{exDisyllabicCausativesWithNasal}), which shows that synchronically speaking this is not a rule of reduplication,\is{reduplication} as it might appear at first sight. A historic scenario for this alternation is provided by \citet{HymanL2003b}. He argues that its origin most probably lies in imbrication plus a cyclic application of \isi{spirantization} through the causative\textsubscript{1} -\textit{i} (\sectref{Causative1}), thus yielding (C)(G)V\textit{siisye} for base-final spirantizing\is{spirantization} oral linguals. This then came to be re-analysed as a process of reduplication,\is{reduplication} yielding (C)(G)V\textit{fiifye} with base-final oral labials. Lastly, this turned into a phonological pattern requirement\is{templatic} for disyllabic causative stems to end in -\textit{ii}\{\textit{s},\textit{f}\}\textit{ye}, hence the extension to final nasals (and causatives of monosyllabic verbs, which are not discussed by Hyman).

Note that disyllabic causatives derived from the verbs of the shape CG\textit{a}\{\textit{l}, \textit{n}\} discussed above are excluded from this process. As with their underlying bases,\is{base} imbrication takes place (\ref{exNoImbricationDisyllabicanlm}). The causative of \textit{baala}, \textit{baasya} \lq increase (tr.)' shows variation just like its underlying root,\is{root} and is attested with both imbricating and copying forms (\ref{exImbricationBaasya}).

\begin{exe}
\ex Causatives of monosyllabic roots:\\
\label{exCopyingBiCausativeLong}
\begin{tabular}[t]{@{}>{\itshape}lll}
%\textit{gw-ɪsi-a}x\=`overturn; throw down'x\= > \textit{gwɪsiisye}\kill
\textit{lw-ɪsi-a}&`make fight'& > \textit{lwɪsiisye}
\\\textit{gw-ɪsi-a}&`overturn; throw down'& > \textit{gwɪsiisye}
\end{tabular}
\ex Disyllabic causatives with -\textit{i}:\\
\label{exCopyingBiCausativeShort}
\begin{tabular}[t]{@{}>{\itshape}llll}
%\textit{pyʊfya}x\=(\degree pyʊp-i-a)x\=`warm, heat up'x\= > \textit{pyʊfiifye}\kill
\textit{bosya}&(\degree bol-i-a)&`cause to rot'& > \textit{bosiisye}
\\\textit{pyʊfya}&(\degree pyʊp-i-a)&`warm, heat up'& > \textit{pyʊfiifye}
\end{tabular}
\ex Disyllabic causatives with final nasal:\\
\label{exDisyllabicCausativesWithNasal}
\begin{tabular}[t]{@{}>{\itshape}lll}
%\textit{taam-y-a}x\=`put out, extinguish'x\=\kill%Unsinnige Zeile nur fuer Tabulatoren
\textit{pon}-\textit{i}-\textit{a}&`greet; visit'& > \textit{poniisye}
\\\textit{an}-\textit{i}-\textit{a}&`ask'& > \textit{aniisye}
\\\textit{sim}-\textit{y}-\textit{a}&`put out, switch off'& > \textit{simiisye}
\\\textit{taam}-\textit{y}-\textit{a}&`trouble, persecute'& > \textit{taamiisye}
\end{tabular}
\ex Disyllabic causatives CG\textit{a\{l, n\}-i}:\\\label{exNoImbricationDisyllabicanlm}
\begin{tabular}[t]{@{}>{\itshape}llll}
%\textit{lwania}x\=(\degree lwan-i-a)x\=`confront; make quarrel'x\=\kill
\textit{lwasya}&(\degree lwal-i-a)&`nurse the sick, care for'& > \textit{lwesye}
\\\textit{syasya}&(\degree syal-i-a)&`leave over'& > \textit{syesye}
\\\textit{fwania}&(\degree fwan-i-a)&`match; reconcile'& > \textit{fwenie}
\\\textit{lwania}&(\degree lwan-i-a)&`confront; make quarrel'& > \textit{lwenie}
\end{tabular}
\ex Causative of \textit{baala}:\label{exImbricationBaasya}\\
\begin{tabular}[t]{@{}>{\itshape}llll}
\textit{baasya}&(\degree baal-i-a)&`increase (tr.)'& > \textit{beesye} / \textit{baasiisye}
\end{tabular}
\end{exe}
\is{causative|)}
Some other verbs of the shape (C)VCG form their perfective stems by copying (C\textsubscript{1})VC\textsubscript{2}G $\rightarrow$ (C\textsubscript{1})VC\textsubscript{2}-\textit{ii}-C\textsubscript{2}G-\textit{e}.\footnote{For \textit{bɪfwa}, \citet{BergerP1938} and \citet{FelbergK1996} list \textit{bɪfwifwe} as a variant, while for \textit{ibwa} \citet{BergerP1938} also has \textit{ibwibwe}; \citet{NurseD1979} has \textit{okya} > \textit{okyokye}, thus (C\textsubscript{1})(G)VC\textsubscript{2}G $\rightarrow$ (C\textsubscript{1})VC\textsubscript{2}GV-\textit{i}-C\textsubscript{2}G-\textit{e}. All the speakers consulted in the present study rejected these forms, which seems to be a case of diatopic\is{dialects} variation (both Felberg's and Berger's main sources stem from more southern varieties). \citet{FelbergK1996} further lists a copying stem for \textit{miimwa} \lq crave; envy', which was also rejected.}
\begin{exe}
\ex
\begin{tabular}[t]{@{}>{\itshape}llll}
\textit{bɪfwa}&`ripen'& > \textit{bɪfiifwe}
\\\textit{ibwa}&`forget'& > \textit{ibiibwe}
\\\textit{ikya}&`be(come) confident'& > \textit{ikiikye}
\\\textit{okya}&`grill, burn'& > \textit{okiikye}
\\\textit{lɪɪgwa}&`be eaten'& > \textit{lɪɪgiigwe} (also \textit{lɪɪgilwe}; see above)
\\\textit{peegwa}&`be given'& > \textit{peegiigwe}
\end{tabular}
\end{exe} 

\subsubsection{Tri- and polysyllabic verbs}
With verbs that have three or more syllables, imbrication is the default. %evtl je vokal nur eins
\begin{exe}
\ex%hier muss es tabbing sein, da das sonst mit dem Seitenumbruch nicht funktioniert. Das longtable-Paket scheitert daran, die Tabelle oben in dem Beispiel auszurichten
\begin{tabbing}
\textit{bʊmbʊlʊka}x\='be slackened, give in'x\=\kill%unsinnszeile für tabs
\textit{aganila}\>`meet'\> > \textit{aganiile} \\
\textit{gundamika}\>`bend over'\> > \textit{gundamiike}\\
\textit{guulɪla}\>`wait for'\> > \textit{guuliile}\\
\textit{itɪkɪla}\>`answer; approve'\> > \textit{itɪkiile}\\ 
\textit{geleka}\>`thatch, pile'\> > \textit{geliike}\\
\textit{koolela}\>`call; name; bid'\> > \textit{kooliile} \\
\textit{bagala}\>`carry on shoulder'\> > \textit{bageele} \\
\textit{fumbata}\>`(en)close'\> > \textit{fumbeete}\\
\textit{jobana}\>`converse'\> > \textit{jobeene}\\
\textit{honyoka}\>`be slackened; give in'\> > \textit{honywike}\\
\textit{kosomola}\>`cough'\> > \textit{kosomwile}\\
\textit{bʊmbʊlʊka}\>`get well, be healed'\> > \textit{bʊmbʊlwike}\\
\textit{fyʊtʊla}\>`pull out'\> > \textit{fyʊtwile}\\
\textit{fujula}\>`humiliate, dishonour'\> > \textit{fujwile}\\
\textit{amula}\>`answer'\> > \textit{amwile}
\end{tabbing}
\end{exe}

With the long\is{vowels!length} allomorph of the \isi{reciprocal} -\textit{aan} and its causativized/plur\-action\-al\is{causative}\is{pluractional} form \mbox{-(\textit{ɪs})\textit{aani}}, imbrication takes place:

\begin{exe}
\ex
\begin{tabular}[t]{@{}>{\itshape}lll}
\textit{bʊngaana}&`(be) assemble(d)'& > \textit{bʊngeene}\\
\textit{bʊngaania}&`gather'& > \textit{bʊngeenie}\\
\textit{lʊngaana}&`join together (intr.)'& > \textit{lʊngeene}\\
\textit{lʊngɪsaania}&`put together, arrange'& > \textit{lʊngɪseenie}\\
\textit{ongaana}&`be together, mixed'& > \textit{ongeene}\\
\textit{ongaania}&`mix, sum up'& > \textit{ongeenie}
\end{tabular}
\end{exe}

The verb \textit{ilaamwa} \lq disregard, doubt' shows some variation. An imbricating stem \textit{ileemwe} was observed, as well as \textit{ilaamwisye}. \citet{BergerP1938} observes a third hybrid variant \textit{ileemwisye}. With the speakers consulted, \textit{ileemwe} is the most frequent form.

All other verbs which have more than two syllables, a long rightmost vowel and that are not causatives\is{causative} form their perfective stems by suffixation of -\textit{ile}: %(\ref{exLongVowelNoImbrication}). 
\begin{exe}
\ex \label{exLongVowelNoImbrication}
\begin{tabbing}
\textit{niembeteela}x\=`wrap, conceal'x\= > \textit{niembeteelile}\kill
\textit{jɪgɪɪla}\>`shake (intr.)'\> > \textit{jɪgɪɪlile}\\
\textit{ogeela}\>`swim'\> > \textit{ogeelile}\\
\textit{niembeteela}\>`wrap, conceal'\> > \textit{niembeteelile}\\
\textit{tendeela}\>`peep, peek'\> > \textit{tendeelile}\\
\textit{kolooma}\>`grown'\> > \textit{koloomile}
\end{tabbing}
\end{exe}

With verbs that feature a base-final prenasalized plosive (recall that these predictably induce lengthening\is{vowels!length} in the preceding vowel), as well as with causatives\is{causative} having a long vowel\is{vowels!length} in the rightmost position (minus those containing the \isi{reciprocal} plus causative,\is{causative} as discussed above), variation is found. Typically, the former trigger suffixing of -\textit{ile} while the latter trigger copying, which yields \mbox{-C\textit{iisye}/}\mbox{-C\textit{iifye}}. These are the formations given by \citet{SchumannK1899} and \citet{BergerP1938} and the only ones attested in the textual data. However, at least for the verbs given in (\ref{exImbricationVariationNC}, \ref{exImbricationVariationLongCaus}), some speakers also have imbricating forms.\footnote{Given this variation and the variant forms of \textit{ilaamwa}, together with the fact that tri- and polysyllabic verbs ending in -\textit{aan(i)} induce imbrication, one might suspect a loosening constraint against imbrication with long and lengthened vowels\is{vowels!length} (in terms of moraic phonology: bimoraic vowels). \citet{FelbergK1996} lists \textit{bulunga} > \textit{bulungile} / \textit{bulwinge}, \textit{l[a]alʊʊsya} / \textit{l[a]alwsiye} (indication of vowel length\is{vowels!length} missing) and  \textit{palamaasya} > \textit{palamaasiisye} / \textit{palameesye}.}

\begin{exe}
\ex Copying with -VV\textit{\{s, f\}y}:\\
\label{exLongCausCopying}
\begin{tabular}[t]{@{}>{\itshape}lll}
malɪɪsya&`end, eliminate'& > \textit{malɪɪsiisye}
\\eneesya&`inspect, visit'& > \textit{eneesiisye}
\\balabaasya& \lq spread out (tr.); pretend'& > \textit{balabaasiisye}
\\syʊngʊʊsya&`rotate (tr.)'& > \textit{syʊngʊʊsiisye}
\\nyonyoofya&`attract, rouse desire'& > \textit{nyonyofiifye}
\end{tabular}
\ex Variation with -VNC:\label{exImbricationVariationNC}\\
\begin{tabular}[t]{@{}>{\itshape}lll}
%\textit{kung'unda}x\=`shake off; beat up; knock'x\= >\textit{kung'undile}x\= \=\kill
onanga& \lq destroy'& > \textit{onangile} / \textit{onenge}
\\tononda&`peck; dot up'& > \textit{tonondile} / \textit{tonwinde}
\\kung'unda&shake off; beat up'& > \textit{kung'undile} / \textit{kung'winde}
\\bulunga&`roll up, make round'& > \textit{bulungile} / \textit{bulwinge}
\end{tabular}
\ex Variation with -VV\textit{sy}:\\
\label{exImbricationVariationLongCaus}
\begin{tabular}[t]{@{}>{\itshape}lll}
%\textit{palamaasya}x\=`proclaim (<SWA)'x\= > \textit{palamaasiisye}x\= \kill%Unsinn, nur fuer Tabulatoren
palamaasya&`touch; grope'& > \textit{palamaasiisye} / \textit{palameesye}
\\tangaasya& \lq proclaim (<SWA)'& > \textit{tangaasiisye} / \textit{tangeesye}
\\laalʊʊsya& \lq ask'& > \textit{laalʊʊsiisye} / \textit{laalwisye}
\end{tabular}
\end{exe}

Lastly, partially reduplicated\is{reduplication} verbs that have one of the shapes \mbox{C\textsubscript{1}V\textsubscript{1}C\textsubscript{1}V\textsubscript{1}V\textsubscript{1}C\textsubscript{2}} or \mbox{C\textsubscript{1}G\textsubscript{1}V\textsubscript{1}C\textsubscript{1}G\textsubscript{1}V\textsubscript{1}C\textsubscript{2}}, where C\textsubscript{2} is an approximant or plosive, are treated as if disyllabic. The same holds for spirantized\is{spirantization} causatives\is{causative} (\sectref{Causative1}) thereof. Although for the first pattern this behaviour could also be motivated by the rightmost long vowel,\is{vowels!length} the second group shows that this is rather a function of the overall phonemic shape. It is a noteworthy fact that all the attested verbs in question can be considered onomatopoetic or de-ideophonic.\is{ideophones} A similar rule that blocks imbrication holds in \ili{Yao} P21 \citep[112, 120]{BergerP1938}. This suggests itself to be functionally motivated, in order to preserve the sound symbolism. For a discussion of how sound symbolism can motivate exceptions on the diachronic axis, see \citet[55f]{DimmendaalG2011}.
\begin{exe}
\ex Shape C\textsubscript{1}V\textsubscript{1}C\textsubscript{1}V\textsubscript{1}V\textsubscript{1}C\textsubscript{2}:
\\\begin{tabular}[t]{@{}>{\itshape}lll}
%\textit{ng'ong'oola}x\=`grimace at'x\=\kill
\textit{boboota}&`grunt'& > \textit{bobootile}
\\\textit{hohoola}&`jeer'& > \textit{hohoolile}
\\\textit{ng'ong'oola}&`grimace at'& > \textit{ng'ong'oolile}
\\\textit{nyunyuuta}&`whine'& > \textit{nyunyuutile}
\\\textit{nyinyiila}&`squint'& > \textit{nyinyiilile}
\end{tabular}
\ex Shape C\textsubscript{1}G\textsubscript{1}V\textsubscript{1}C\textsubscript{1}G\textsubscript{1}V\textsubscript{1}C\textsubscript{2}:
\\\begin{tabular}[t]{@{}>{\itshape}lll}
%\textit{mwemweka}x\=`blab, talk nonsense'x\=\kill%unsinnig, aber fuer Tabulatoren
\textit{bwabwata}&`blab, talk nonsense'& > \textit{bwabwatile}
\\\textit{bwabwaja}&`blab, talk nonsense'& > \textit{bwabwajile}
\\\textit{mwemweka}&`glitter'& > \textit{mwemwekile}
\end{tabular}
\ex Shape C\textsubscript{1}G\textsubscript{1}V\textsubscript{1}C\textsubscript{1}G\textsubscript{1}V\textsubscript{1}C\textsubscript{2}-\textit{i}\textsubscript{\textsc{caus}}:
\\\begin{tabular}[t]{@{}>{\itshape}lll}
\textit{mwemwesya}&`move around light'& > \textit{mwemwesiisye}
\\\textit{myamyasya}&`smoothen out'& > \textit{myamyasiisye}
\\\textit{i-ng'weng'wesya}&`grumble'& > \textit{i-ng'weng'wesiisye}
\\\textit{ng'wang'wasya}&`not do thoroughly'& > \textit{ng'wang'wasiisye}
\end{tabular}
\end{exe}
\is{imbrication|)}\is{aspect!perfective|)}

\label{PastNonPast}
\section{Synthetic present and past constructions} \label{SimpleConstructions}
In the following, present\is{tense!present} (non-past) and past\is{tense!past} tense constructions consisting of solely the inflected verb will be described. These are listed in \tabref{tableSimpleConstructions}.
\begin{table}[hh]
\setlength{\tabcolsep}{2pt}
\begin{tabularx}{\textwidth}{llll}
\lsptoprule
\footnotesize{Label} & \footnotesize{Shape} & \footnotesize{Example}\\
\midrule 
Simple present & \textsc{sm}\textsubscript{2}-\textit{kʊ}-\textsc{vb}-\textit{a} & \textit{tʊkʊjoba} & \lq we  speak' \\  
Negative present & \textsc{sm}-\textit{ti}-\textit{kʊ}-\textsc{vb}-\textit{a} &  \textit{tʊtikʊjoba} & \lq we do not speak'\\ 
Present perfective & \textsc{sm}-\textsc{vb}-\textit{ile} & \textit{tʊjobile} & \lq we have spoken'\\ 
Neg. present perfective & \textsc{sm}-\textit{ka}-\textsc{vb}-\textit{a} & \textit{tʊkajoba} & \lq we have not spoken'\\ 
Past perfective & \textsc{sm}-\textit{a}(\textit{lɪ})-\textsc{vb}-\textit{ile} & \textit{twajobile} & \lq we spoke' \\ 
Neg. past perfective & \textsc{sm}-\textit{ka}-\textit{a}(\textit{lɪ})-\textsc{vb}-\textit{ile} & \textit{tʊkaajobile} & \lq we did not speak' \\ 
Past imperfective & \textsc{sm}-\textit{a}-\textsc{vb}-\textit{aga} & \textit{twajobaga} & \lq we were speaking' \\ 
Neg. past imperfective & \textsc{sm}-\textit{ka}-\textit{a}-\textsc{vb}-\textit{aga} & \textit{tʊkaajobaga} & \lq we were not speaking' \\  
\lspbottomrule  
\end{tabularx}
\caption{Synthetic non-past/present and past tense constructions}\label{tableSimpleConstructions}
\end{table}
\subsection{Simple present}\label{Present}
\is{simple present|(}\is{tense!present|(}The simple present is formed by a subject prefix\is{subject marker} from the second series (see \sectref{SubjectConcords}) followed by a prefix \textit{kʊ}- in post-initial position, together with the final vowel -\textit{a}.
\begin{exe}
\ex\textit{tʊkʊjoba} \lq we speak / are speaking'
\end{exe}

The familiar label \textit{simple present} is applied to this construction for reasons of convenience. More precisely, this construction can be understood as the imperfective\is{aspect!imperfective} counterpart to the present perfective\is{aspect!perfective} (\sectref{PresentPerfective}). Depending on context and co-text, the simple present can have a continuous/progressive\is{aspect!progressive} reading (\ref{exPRScontinuous}). For a discussion of this reading vis-à-vis the periphrastic progressive,\is{aspect!progressive} see \sectref{Progressive}.

\begin{exe}
\ex \label{exPRScontinuous}\gll tʊ-kʊ-fw-a, jɪ-kʊ-tʊ-gog-a ɪ-n-galamu\\
\textsc{1pl}-\textsc{prs}-die-\textsc{fv} 9-\textsc{prs}-\textsc{1pl}-kill-\textsc{fv} \textsc{aug}-9-lion\\
\glt `We're dying, the lion is killing us.' [Chief Kapyungu]
\end{exe}

The availability of this continuous reading hinges on the availability of a pre-culminative phase, i.e. an extended Onset\is{phase!Onset phase} or Nucleus phase\is{phase!Nucleus phase} (see Chapter \ref{AspectualClassification}), or alternatively, as in (\ref{exPRScontinuous}), the availability of a series reading.

The simple present is found in some further functions that can be related to its continuous reading. These are illustrated in the following examples (terminology following \citealt[247]{BinnickR1991}). For use of the simple present as a narrative present,\is{narrative present} see \sectref{NarrativePresent}.

\begin{exe}
\ex Reportative:\\
\gll ii-peasi lɪ-mo \textbf{li}-\textbf{kʊ}-\textbf{satʊk}-\textbf{a} paa-si. \textbf{i}-\textbf{kʊ}-\textbf{sal}-\textbf{a} n=ʊ-kʊ-pugut-a n=ɪ-kɪ-tambala. \textbf{i}-\textbf{kʊ}-\textbf{bɪɪk}-\textbf{a} kangɪ n-kɪ-kapʊ\\
5-pear(<SWA) 5-one 5-\textsc{prs}-fall-\textsc{fv} 16-down 1-\textsc{prs}-pick-\textsc{fv} \textsc{com}=\textsc{aug}-15-shake\_off-\textsc{fv} \textsc{com}=\textsc{aug}-7-cloth 1-\textsc{prs}-put-\textsc{fv} again 18-7-basket\\
\glt `One pear falls on the ground. He picks it up and cleans it with a cloth. He puts it back into the basket.' [Elisha Pear Story]
\ex Performative:\\
\gll \textbf{n}-\textbf{gʊ}-\textbf{mm}-\textbf{oosy}-\textbf{a} ʊ-mw-ana ʊ-jʊ Joni\\ \textsc{1sg}-\textsc{prs}-1-baptize-\textsc{fv} \textsc{aug}-1-child \textsc{aug}-\textsc{prox.1} J.\\
\glt `I baptize [name] this child John.' [ET]
\end{exe}

The simple present is further used in habitual and generic statements:\is{aspect!habitual}\is{aspect!generic}
\begin{exe}
\ex \label{exPRSHABGEN1} \gll ɪ-n-gambɪlɪ si-ti-kʊ-j-a n=ɪ-n-dumbula m-mu-nda. ɪ-n-gambɪlɪ \textbf{tʊ}-\textbf{kʊ}-\textbf{si}-\textbf{lek}-\textbf{a} m-mi-piki\\
\textsc{aug}-10-monkey 10-\textsc{neg}-\textsc{prs}-be(come)-\textsc{fv} \textsc{com}=\textsc{aug}-10-heart 18-3?-inside\_of\_body \textsc{aug}-10-monkey \textsc{1pl}-\textsc{prs}-10-let-\textsc{fv} 18-4-tree\\
\glt \lq Monkeys don't have their hearts inside the body. Us monkeys, we leave them in the trees.' [Crocodile and Monkey]
\ex \label{exPRSHABGEN2} \gll po na lɪlɪno \textbf{li}-\textbf{kʊ}-\textbf{kol}-\textbf{a} \textbf{kʊkʊtɪ} \textbf{ii}-\textbf{sikʊ}. looli ɪ-n-gʊkʊ \textbf{kw}-\textbf{ag}-\textbf{a} \textbf{kʊkʊtɪ} \textbf{ka}-\textbf{balɪlo} si-lɪ paa-si, si-kʊ-lond-a ɪɪ-sindaano ɪ-jɪ sy-aly-asiime kʊ-ly-ebe\\
then \textsc{com} now/today 5-\textsc{prs}-grasp-\textsc{fv} every 5-day but \textsc{aug}-10-chicken \textsc{2sg.prs}-find-\textsc{fv} every 12-time 10-\textsc{cop} 16-down 10-\textsc{prs}-search-\textsc{fv} \textsc{aug}-needle(<SWA)(9) \textsc{aug}-\textsc{prox.9} 10-\textsc{pst}-borrow.\textsc{pfv} 17-5-crow\\
\glt \lq So even now it [Crow] takes them [little chicks] every day. As for the chickens, you find them all the time on the ground, searching for the needle that they had borrowed from Crow.' [Chickens and Crow]
\end{exe}

The simple present can also be used to refer to future eventualities, often within the same day.\is{futurate} In (\ref{exPRSFutureMonkeTortoise}), the speaker, Monkey, angrily soliloquises, announcing that Tortoise will pay back his overdue debts this very same day. In (\ref{exPRSFutureKillerWoman}), a strange woman has visited the speaker's wife and asked for breast milk. She was told to come back later and now the husband explain his plans to trap her, using the simple present with future reference. 
\begin{exe}
\ex \label{exPRSFutureMonkeTortoise}\gll ii-sikʊ lɪ-lɪ-ngɪ po a-al-iis-ile mwa=n-gambɪlɪ, a-al-iis-ile n=ɪ-ly-ojo \textup{\lq\lq}lɪlɪno \textbf{kʊ}-\textbf{m}-\textbf{b}-\textbf{a}=\textbf{ko} ɪɪ-heela j-angʊ, \textbf{kʊ}-\textbf{m}-\textbf{b}-\textbf{a}=\textbf{ko} ɪɪ-heela j-angʊ\textup{''} i-kʊ-job-a mu-n-jɪla\\
5-day 5-5-other then 1-\textsc{pst}-come-\textsc{pfv} matronym=9-monkey 1-\textsc{pst}-come-\textsc{pfv} \textsc{com}=\textsc{aug}-5-anger \phantom{\lq\lq}now/today \textsc{2sg.prs}-\textsc{1sg}-give-\textsc{fv}=17 \textsc{aug}-money(9) 9-\textsc{poss.1sg} \textsc{2sg.prs}-\textsc{1sg}-give-\textsc{fv}=17 \textsc{aug}-money(9) 9-\textsc{poss.1sg} 1-\textsc{prs}-speak-\textsc{fv} 18-9-path\\
\glt `Another day Mr. Monkey came, he came with anger. ``Today you're giving me my money, you're giving me my money'' he is saying on the way.' [Monkey and Tortoise]
\ex \label{exPRSFutureKillerWoman} \gll lɪnga iis-ile ʊ-ne \textbf{n}-\textbf{gw}-\textbf{i}-\textbf{fis}-\textbf{a} \textbf{n}-\textbf{gʊ}-\textbf{j}-\textbf{a} kʊʊ-sofu, ʊ-gwe ʊ-job-ege nagwe ʊkʊtɪ a-kʊ-p-e ɪ-kɪ-kombe gʊ-n-kam-il-e. ʊ-ne \textbf{n}-\textbf{gʊ}-\textbf{n}-\textbf{kol}-\textbf{a} ʊkʊtɪ a-m-bʊʊl-e ɪ-fi i-kʊ-bomb-el-a!\\
if/when 1.come-\textsc{pfv} \textsc{aug}-\textsc{1sg} \textsc{1sg}-\textsc{prs}-\textsc{refl}-hide-\textsc{fv} \textsc{1sg}-\textsc{prs}-be(come)-\textsc{fv} 17-room.9 \textsc{aug}-\textsc{2sg} \textsc{2sg}-speak-\textsc{ipfv.subj} \textsc{com}.1 \textsc{comp} 1-\textsc{2sg}-give-\textsc{subj} \textsc{aug}-7-cup \textsc{2sg}-1-milk-\textsc{appl}-\textsc{subj} \textsc{aug}-\textsc{1sg} \textsc{1sg}-\textsc{prs}-1-hold-\textsc{fv} \textsc{comp} 1-\textsc{1sg}-tell-\textsc{subj} \textsc{aug}-\textsc{prox.8} 1-\textsc{prs}-do-\textsc{appl}-\textsc{fv}\\
\glt `When she comes, I'll hide, I'll be in the bedroom, you talk with her so that she gives you the cup so that you can express milk for her. I'll catch her so that she tells me what she does with it!' [Killer woman]
\end{exe}
Rather than being a calendaric constraint, the present day is the default for the \isi{futurate} use of the simple present. However, the simple present can also be used to talk about eventualities later than the same day. This is common with plans (\ref{exPRSplanned}) and regular or scheduled eventualities (\ref{exPRSscheduled}). What is more, the simple present can be used to talk about a future eventuality with certainty (\ref{exPRSfuturate}); see also (\ref{exProcliticAavsPRS}) on p. \pageref{exProcliticAavsPRS}.

\begin{exe}
\ex \label{exPRSplanned} Context: The household help informs that she will not be present the next day.\\
\gll kɪ-laabo n-gʊ-sumuk-a kʊ-Tʊkʊjʊ\\
7-tomorrow \textsc{1sg}-\textsc{prs}-depart-\textsc{fv} 17-T.\\
\glt `Tomorrow I am heading to Tukuyu.' [overheard]
\ex \label{exPRSscheduled} \gll kɪ-laabo n=ʊ-lʊ-bʊnjʊ fiijo ii-sʊba li-kʊ-sook-a\\
7-tomorrow \textsc{com}=\textsc{aug}-11-morning \textsc{intens} 5-sun 5-\textsc{prs}-leave-\textsc{fv}\\
\glt `The sun rises early tomorrow morning.' [ET]
\ex \label{exPRSfuturate}
Context: Mbeya FC is playing against Simba SC tomorrow. You are sure that Mbeya FC will win.\\
\gll tʊ-kʊ-ba-tol-a kɪ-laabo\\
\textsc{1pl}-\textsc{prs}-2-defeat-\textsc{fv} 7-tomorrow\\
\glt \lq We are defeating them tomorrow (sic!).' [ET]
\end{exe}

Conversely, when specifically evoking a later period of the same day as the reference frame, the future\is{tense!future} \isi{proclitic} \textit{aa}= (\sectref{ProcliticAa}) is used. (\ref{exProcliticAaEvening}) illustrates this. Note that the proclitic is represented as \textit{a}= in this example, as it is followed by a prenasalized plosive and its length\is{vowels!length} is thus predictable. Likewise, addition of the future\is{tense!future} \isi{proclitic} changes (\ref{exPRSfuturate}) above to a mere prediction (\ref{exProcliticAaLessCertain}).
\begin{exe}
\ex \label{exProcliticAaEvening}
Context: Talking about the speaker's plans for the evening.\\
\gll na=a-ma-jolo a=n-gʊ-j-a pa-kʊ-bomb-a pa-ky-alo\\
\textsc{com}=\textsc{aug}-6-evening \textsc{fut}=\textsc{1sg}-\textsc{prs}-be(come)-\textsc{fv} 16-15-work-\textsc{fv} 16-8-field\\
\glt `In the evening I will be working in the field.' [ET]
\ex \label{exProcliticAaLessCertain}
Context: Mbeya FC is playing against Simba SC tomorrow. You think that Mbeya FC will win.\\
\gll aa=tʊ-kʊ-ba-tol-a kɪ-laabo\\
\textsc{fut}=\textsc{1pl}-\textsc{prs}-2-defeat-\textsc{fv} 7-tomorrow\\
\glt \lq We will defeat them tomorrow.' [ET]
\end{exe}

In past narrative discourse,\is{narrative} the simple present features as a \isi{narrative present} and in subordinate contexts; see \sectref{PRSasNonPST}. It is also found in the coda section\is{section!coda} of some narratives, with reference to the speaker-now, as in (\ref{exPRSHABGEN2}) above.
\is{simple present|)}\is{tense!present|)}
\subsection{Negative present}\label{NegPresent}
\is{negative|(}\is{tense!present|(}
The negative counterpart to the \isi{simple present} consists of the post-initial negative prefix \textit{ti} followed by the present prefix \textit{kʊ}- and the final vowel -\textit{a}. As the subject prefix\is{subject marker} is not directly adjacent to the present prefix, the series 1 prefixes are used; see \sectref{SubjectConcords}. Diachronically speaking, \textit{ti}-\textit{kʊ}- most likely goes back to the fusion of \textit{ta}- and \textit{ikʊ}-; see \sectref{SubjectConcords} on the vowel /i/ preceding the present prefix. The negative prefix is \textit{ta}- in all indicative constructions in neighbouring \ili{Ngonde} \citep[77]{KishindoP1999}, as well as in \ili{Ndali} \citep[108--116]{BotneR2008} and \ili{Sukwa} \citep[45--51]{KershnerT2002}.

\begin{exe}
\ex \textit{tʊtikʊjoba} \lq we do not speak / we are not speaking'
\end{exe}

Like its affirmative counterpart, the negative present has a continuous reading (\ref{exNegPRSPROG}), which also serves as the negative counterpart to the present progessive (\sectref{Progressive}).\is{aspect!progressive} It is also used for negative habituals and generics\is{aspect!habitual}\is{aspect!generic} (\ref{exNegPRSHabGen2}, \ref{exNegPRSHabGen3}).
\begin{exe}
\ex \label{exNegPRSPROG} \gll bʊle, a-lɪ pa-kʊ-ly-a? -- mma, \textbf{a}-\textbf{ti}-\textbf{kʊ}-\textbf{ly}-\textbf{a}\\
\textsc{q} 1-\textsc{cop} 16-15-eat-\textsc{fv} {} no 1-\textsc{neg}-\textsc{prs}-eat-\textsc{fv}\\
\glt \lq Is s/he eating?' -- \lq No, s/he is not eating.' [ET]
\ex \label{exNegPRSHabGen2} \gll ʊ-gʊʊso m-oolo. \textbf{a}-\textbf{ti}-\textbf{kʊ}-\textbf{bomb}-\textbf{a} n=ɪ-m-bombo na=jɪ-mo. i-kʊ-syʊngʊʊtɪl-a itolo\\
\textsc{aug}-your\_father(1) 1-lazy 1-\textsc{neg}-\textsc{prs}-work-\textsc{fv} \textsc{com}=\textsc{aug}-9-work \textsc{com}=9-one 1-\textsc{prs}-surround-\textsc{fv} just\\
\glt `Your father is lazy. He doesn't do any work. He just wanders around.' [Monkey and Tortoise] 
\ex \label{exNegPRSHabGen3} \gll n=ɪɪ-swi ɪ-si si-li=mo n-kɪ-siba mu-la, a-b-eene ka-aja a-ba ba-lɪ kɪfuki \textbf{ba}-\textbf{ti}-\textbf{kʊ}-\textbf{ly}-\textbf{a}, paapo a-ba-ndʊ ba-al-iibiile ba-a-jongiile paa$\sim$po\\
\textsc{com}=\textsc{aug}-fish(10) \textsc{aug}-\textsc{prox.10} 10-\textsc{cop}=18 18-7-pond 18-\textsc{dist} \textsc{aug}-2-owner 12-homestead \textsc{aug}-\textsc{prox.2} 2-\textsc{cop} near 2-\textsc{neg}-\textsc{prs}-eat-\textsc{fv} because \textsc{aug}-2-person 2-\textsc{pst}-sink.\textsc{pfv} 2-\textsc{pst}-disappear.\textsc{pfv} \textsc{redupl}$\sim$\textsc{ref.16}\\ 
\glt `And as for the fish in that pond, the people living near do not eat them, because people sank and disappeared there.' [Selfishness kills]
\end{exe}

The following examples illustrate the use of the negative present in negative futurates.\is{futurate}
\begin{exe}
\ex \gll n-sulumeenie fiijo paapo ʊlʊ n-iis-ile n-kʊ-kw-eg-a \textbf{ʊ}-\textbf{ti}-\textbf{kʊ}-\textbf{gomok}-\textbf{a} kangɪ\\
\textsc{1sg}-afflict.\textsc{pfv} \textsc{intens} because now \textsc{1sg}-come-\textsc{pfv} 18-15-\textsc{2sg}-take-\textsc{fv} \textsc{2sg}-\textsc{neg}-\textsc{prs}-return-\textsc{fv} again\\
\glt \lq I'm very sad, because this time that I've come to pick you up you won't return.' [Crocodile and Monkey]
\ex Context: According to contract we do not work the next day.\\
\gll kɪ-laabo \textbf{tʊ}-\textbf{ti}-\textbf{kʊ}-\textbf{bomb}-\textbf{a} ɪ-m-bombo\\
7-tomorow \textsc{1pl}-\textsc{neg}-\textsc{prs}-work-\textsc{fv} \textsc{aug}-9-work\\
\glt `Tomorrow we do not work.' [ET]
\end{exe}
\is{negative|)}\is{tense!present|)}

\subsection{Present perfective}\label{PresentPerfective}
\is{tense!present|(}\is{aspect!perfective|(}
\subsubsection{Formal makeup and overview of meaning}\label{PresentPerfectiveIntroduction}
The present perfective is formed with the perfective final suffix -\textit{ile} or one of its allomorphs; see \sectref{Imbrication}.

\begin{exe}
\ex \label{exPRSPFVEinstiegsbeispiel} \textit{tʊjobile} \lq we have spoken'
\end{exe}

The meaning of the present perfective depends on the aspectual class of the lexical verb. When used with non-inchoative verbs, it refers to a completed act (\ref{exPRSPFVnonInchoative}). With inchoative verbs,\is{inchoative verbs} the default reading of the present perfective is one of a present state (\ref{exPRSPFVInchoative}).

\begin{exe}
\ex \label{exPRSPFVnonInchoative} \gll mu-keet-ile leelo, mu-keet-ile leelo! n-iis-ile ne malafyale\\
\textsc{2pl}-watch-\textsc{pfv} now/but \textsc{2pl}-watch-\textsc{pfv} now/but \textsc{1sg}-come-\textsc{pfv} \textsc{1sg} chief(1)\\
\glt `You have seen now! You have seen now! I have come as a king.' [Hare and Hippo]
\ex \label{exPRSPFVInchoative} \gll a-kaleele\\
1-be(come)\_angry.\textsc{pfv}\\
\glt (Default reading:) \lq S/he is angry.' [ET]
\end{exe}

Accordingly, the present perfective is used with \isi{inchoative verbs} in a number of cases that translate as a simple present or present progressive\is{aspect!progressive} in English. The following are a few examples; for \isi{narrative present} uses see \sectref{NarrativePresent}.
\begin{exe}
\ex \label{exAlsAnteriorOfResult1}Stative:\\
\gll ee, nalooli \textbf{n}-\textbf{dʊ}-\textbf{gan}-\textbf{ile}\\
yes really \textsc{1sg}-11-love-\textsc{pfv}\\
\glt `Yes, I really love him [spider].' [Hare and Spider]
\ex \label{exAlsAnteriorOfResult2}Stative:\\
\gll \textbf{a}-\textbf{tʊʊgeele} pa-kɪ-kota, a-lɪ pa-kw-ɪmb-a ɪ-kɪ-tabʊ\\
1-get\_seated/sit.\textsc{pfv} 16-7-chair 1-\textsc{cop} 16-15-read-\textsc{fv} \textsc{aug}-7-book(<SWA)\\
\glt `He is sitting in a chair, reading a book.' [ET]
\ex Reportative:\\
\gll i-kʊ-kɪnd-a ʊ-n-nyambala n=ɪ-m-bene. \textbf{a}-\textbf{kol}-\textbf{ile} ɪɪ-kamba\\
1-\textsc{prs}-pass-\textsc{fv} \textsc{aug}-1-man \textsc{com}=\textsc{aug}-9-goat 1-grasp/hold-\textsc{pfv} \textsc{aug}-rope(9)(<SWA)\\
\glt \lq A man with a goat passes. He holds a rope.' [Elisha Pear Story]
\end{exe}

Although a present state reading is more common with inchoative verbs\is{inchoative verbs}, the present perfective also allows for a change-of-state reading (\ref{exPRSPFVInchoativeDynamic}).\footnote{\citet[127]{CraneTM2011} gives a pragmatic explanation as to why the stative reading is the default for inchoative verbs: ``With change-of-state verbs [inchoatives, BP], the implicature of continued resultant state is particularly salient. This implicature is easy to derive from general conversational principles of relevance. Use of a verb describing entry into a state, in general, is most relevant if the state holds at perspective time. For example, [in \ili{Totela} K41, BP] a verb like -\textit{taba} \lq become happy', requires no direct reference to the situation resulting in happiness, although the context may provide such information. Thus, when uttering \textit{ndataba} without much other context, a speaker is less likely to be referring to the rather ethereal process of obtaining happiness than to the ensuing happy state.''} However, to express that the resultant state of an inchoative verb held at a certain time in the past, the past perfective has to be used (\ref{exPRSPFVnotPSTstate}).

\begin{exe}
\ex \label{exPRSPFVInchoativeDynamic} Context: How did your father react when he heard the news this morning?\\
\gll pa-bw-andɪlo a-kaleele fiijo, ʊlʊ si-maliike\\
16-14-beginning 1-be(come)\_angry.\textsc{pfv} \textsc{intens} now 10-finish.\textsc{pfv}\\
\glt `First he got angry, but now the anger is gone.' [ET]
\ex \label{exPRSPFVnotPSTstate} \gll a-a-kaleele\\
1-\textsc{pst}-be(come)\_angry.\textsc{pfv}\\
\glt (Default reading:) \lq S/he was angry.' [ET]
\end{exe}

Like all present tense paradigms, the present perfective can be used with relative time reference in a number of subordinate contexts and as a \isi{narrative present} (\sectref{PRSasNonPST}). Given the right context, it can also be used to refer to a completed\is{completion} eventuality situated in future time. In (\ref{exPFVboFuture}) the future reference anchor established by the adverbial clause\is{subordinate clauses!temporal clause} is re-introduced through a stressed form of the class 14 referential demonstrative \textit{bo}. In (\ref{exPFVtajaliFuture}), context plus \textit{tajalɪ} \lq already' (a Swahili loan) licenses a future reading. Note that in many cases this is the only way of depicting a future state without referring to its inception.

\begin{exe}
\ex \label{exPFVboFuture}
\gll ʊ-lɪ n-sekele fiijo, kangɪ lɪnga ga-kɪnd-ile a-ma-sikʊ ma-nandɪ, ʊ-gwe \textbf{bo} \textbf{ʊ}-\textbf{fw}-\textbf{ile}\\
\textsc{2sg}-\textsc{cop} 1-thin \textsc{intens} again if/when 6-pass-\textsc{pfv} \textsc{aug}-6-day 6-little \textsc{aug}-\textsc{2sg} \textsc{ref.14} \textsc{2sg}-die-\textsc{pfv}\\
\glt `‎‎You are very tiny, also, when few days have passed, then you'll be dead.' [Mosquito and Ear]
\ex \label{exPFVtajaliFuture} Context: Your brother is late for dinner.\\
\gll lɪnga a-fik-ile ɪ-fi-ndʊ \textbf{tajalɪ} \textbf{fi}-\textbf{talaliile}\\
if/when 1-arrive-\textsc{pfv} \textsc{aug}-8-food already(<SWA) 8-cool.\textsc{pfv}\\
\glt `When he arrives, the food will be cold.' [ET]
\end{exe}

\subsubsection{Perfectivity as completion of the Nucleus}\label{PerfectivityCompletion}
\is{completion|(}The question now arises as to whether there is a common semantic core to the different readings and uses of the present perfective. \citet[43]{BotneR2010b} defines perfectivity in Bantu as ``an assertion about a time of the event subsequent to the endpoint of the event nucleus''. Recall from \sectref{AristotelianAspect} that Nucleus in Botne \& Kershner's framework is the characteristic act encoded in the verb. With \isi{inchoative verbs} this is the change-of-state; with most types of non-inchoatives it is the eventuality as such.



Botne's definition of perfectivity obviously differs from the more widespread one in theories of grammatical aspect, according to which ``perfectivity indicates the view of a situation as a whole, without distinction of the various separate phases that make up the situation'' \citep[16]{ComrieB1976}. It does, however, correspond to what \citet{BotneRKershnerT2000} call \lq\lq completive'' for the Zulu S42 suffix \mbox{-\textit{ile}}, a label that is also employed by \citet{BotneR2008} and \citet{KershnerT2002} for \mbox{-\textit{ite}} (allomorphs i.a. \mbox{-\textit{ile}}) in Nyakyusa's neighbours \ili{Ndali} and Sukwa.\il{Sukwa} A comparable use of \textit{completive} is already found in \citet{WelmersW1974}. \citet[118--142]{CraneTM2011}, in a lengthy discussion of the subject in \ili{Totela} K41, speaks of ``nuclear completion''. Botne's definition also comes close to \citeauthor{JohansonL2000}'s (\citeyear[29]{JohansonL2000}) \lq\lq postterminal viewpoint'', that \lq\lq envisag[es] the event after the transgression of its relevant limit''. Johnson's \lq\lq relevant limit'', however, hinges upon a conception of Aristotelian aspect similar to Sasse \& Breu's (see \sectref{AristotelianAspect}). In the present study, the term \textit{perfective} is preferred over \textit{completive}, as the latter is commonly used with a sense of ``to do something thoroughly and to completion, e.g. \textit{to shoot someone dead}, \textit{to eat up}'' \citep[318]{BybeePerkinsPaglucia1994}. What is more, the term \textit{perfective} stresses the opposition to a clearly imperfective category, an opposition which is central to the Nyakyusa TMA system.

Adopting the above definition of perfectivity, the different readings of the perfective in Nyakyusa find a unified explanation.\footnote{The following argumentation owes much to \citeauthor{CraneTM2011}'s (\citeyear{CraneTM2011}) ample discussion of the concept of completion of the Nucleus phase in \ili{Totela} K41, which shares a number of similarities with Nyakyusa.} With non-inchoatives, a post-Nucleus vantage point equals a vantage point following the situation as a whole. This is depicted in \figref{FigurePerfectiveActivity} for the activity verb \textit{moga} \lq dance'. As for inchoative verbs, both the stative and the change-of-state reading can be explained by a post-Nucleus vantage point. In the more common stative reading, the vantage point falls within the stative coda phase.\is{phase!Coda phase} For the present perfective used in main clauses this normally coincides with the time of speech, thus giving a present state reading. A change-of-state reading arises from a vantage point following the eventuality as a whole. That is to say, the perfective selects a time as following the Nucleus as the vantage point, but is vague as to the exact position of the eventuality. \figref{FigurePerfectiveInchoative} depicts the two perspectives for the transitional achievement \textit{kalala} \lq be(come) angry'.

\begin{figure} 
\begin{center}
\includegraphics{figures/GrafikActivityCompletive.eps}
\caption{Perfective with an activity verb}
\label{FigurePerfectiveActivity}
\end{center}
\end{figure}

\begin{figure} 
\begin{center}
\caption{\label{FigurePerfectiveInchoative}Perfective with an inchoative verb}
\subfigure[Stative reading]{
\includegraphics[width=0.47\textwidth]{figures/GrafikInchoativeStative.eps}}
\subfigure[Change-of-state reading]{
\includegraphics[width=0.47\textwidth]{figures/GrafikInchoativeChangeOfState.eps}}
\is{completion|)}
\end{center}
\end{figure}

 \newpage 
\subsubsection{Present perfective vs. anterior}\label{PRSPFVvsAnterior}
\is{aspect!anterior|(}It is opportune at this point to consider an alternative analysis of present perfective construction. At first glance, the Nyakyusa configuration \textit{ø}-\textsc{vb}-\textit{ile} resembles an \textit{anterior} or \textit{perfect}.\footnote{The term \textit{anterior}, as used by among others \citet{BybeePerkinsPaglucia1994}, is synonymous with the traditional \textit{perfect}, but has the advantage of not being easily confused with \textit{perfective}.} This is how it has been labelled in some of the previous accounts of Nyakyusa: \citet{SchumannK1899} speaks of a \lq\lq Perfectum'' and \citet{EndemannC1914} of a \lq\lq Perfekt''. Likewise, \citet{MwangokaNVoorhoeveJ1960b} use the label \lq\lq recent perfect''. The anterior (perfect) is also how this construction is in many cases conveniently translated into English. In fact, the Nyakyusa present perfective covers all of the sentences in \citeauthor{DahlOe1985}'s (\citeyear{DahlOe1985}) tense and aspect questionnaire that are identified as prototypical for the employment of the crosslinguistic category of anteriors.

An analysis as an anterior is also suggested by the criteria proposed in Nurse's pan-Bantu study of tense and aspect, in which he identifies the configuration \mbox{\textit{ø}-}\textsc{vb}\mbox{-\textit{ile}} as the most common marker of anteriors in his sample of Bantu languages \citep[156]{NurseD2008}. More importantly, \citet[95--99]{NurseD2008} proposes a number of criteria specific to the morphosyntactic make-up of Bantu languages in order to distinguish between anteriors on the one hand and near past perfectives on the other. Two of these criteria can readily be applied to Nyakyusa. The first one concerns the interaction of grammatical aspect with the aspectual potential of the lexical verb. Drawing on the concepts of consequent state and relevance, which are widely considered core components of anteriors, \citeauthor{NurseD2008} states:
\begin{quote}
For an action verb, for example, anterior represents a situation that is completed but relevant, whereas for a stative verb anterior represents the continuing state resulting from an action initiated in the past. \citep[73]{NurseD2008}
\end{quote}

\newpage 
As has been seen in \sectref{PresentPerfectiveIntroduction}, while a resultant state reading is indeed the default for the Nyakyusa present perfective with \isi{inchoative verbs} (Nurse’s statives), a dynamic change-of-state reading is also available. More example cases are given in the following. In (\ref{exInchoativeChangeOfStateOver}) the resultant state is explicitly cancelled for utterance time, leaving only the dynamic reading. In (\ref{exInchoativeChangeOfStateExistential}), the question is about a previous event of turning angry, not a present state.\footnote{See \sectref{MoSome} on the enclitic =\textit{mo}.} Lastly, in (\ref{exInchoativeChangeOfStateTemporalAdverb}) the event time adverbial phrase \lq last year' allows only for a change-of-state reading. While \lq to die' in English is often considered a punctual achievement (e.g. \citealt[54]{DowtyD1979}), its Nyakyusa equivalent \textit{fwa} is an inchoative verb\is{inchoative verbs} best translated as \lq to be moribund/die/be dead'; see \sectref{VerbalClassTransitionalAchievement}. Also note that lexical verbs which are non-inchoative but suggest the entry into a state do not depend upon a result persisting at utterance time, as with \textit{isa} \lq come' in (\ref{exPRSPFVsequence2}) below.

\begin{exe}
\ex \label{exInchoativeChangeOfStateOver}\gll ii-treni ly-ɪm-ile, lɪno li-kw-end-a kangɪ\\
5-train(<SWA) 5-stand/stop-\textsc{pfv} now 5-\textsc{prs}-walk/travel-\textsc{fv} again\\
\glt \lq The train stopped, but now it is moving again.' [ET]
\ex \label{exInchoativeChangeOfStateExistential} \gll a-kaleele=mo sikʊ baaba gw-ako?\\ 1-be(come)\_angry.\textsc{pfv}=some ever father(<SWA) 1-\textsc{poss.2sg}\\\glt `Has your father ever become angry?' [ET] %ist gecheckt <-lieber hier behalten, als 'weiteres BSP'
\ex \label{exInchoativeChangeOfStateTemporalAdverb} \gll a-fw-ile ɪ-ky-ɪnja ɪ-kɪ kɪ-kɪnd-ile\\
1-die-\textsc{pfv} \textsc{aug}-7-year \textsc{aug}-\textsc{prox.7} 7-pass-\textsc{pfv}\\
\glt `He died last year.' [ET]
\end{exe}

What is more, the present perfective does not feature a strong relevance component. Instead it is the default paradigm for relating recent non-progressive states-of-affairs. In doing so it can be freely used with sequential eventualities, which is a negative criterion for anteriors (\citealt[138]{DahlOe1985}; \citealt[54]{BybeePerkinsPaglucia1994}; \citealt[371]{LindstedtJ2000}). The following examples illustrate both points. In (\ref{exPRSPFVsequence1}) Tortoise asks his child for the whereabouts of a grindstone. Tortoise's child's answer consists of three clauses. In the first, the past perfective with inchoative \textit{kalala} \lq be(come) angry' refers to Mr. Monkey's previous state of mind, which has not only passed but is background information. In the two following clauses, the eventualities affecting the discourse topic (the grindstone) are summarized in the form of a minimal narrative using the present perfective. In (\ref{exPRSPFVsequence2}) a woman reports to her husband the recent happenings, again using two narrative clauses\is{clause types (Labov \& Waletzky)!narrative clause} with the present perfective.

\begin{exe}
\ex \label{exPRSPFVsequence1}
\gll mw-anangʊ, lʊ-lɪ koo=kʊʊgʊ ʊ-lw-ala lw-angʊ?\\
1-my\_child 11-\textsc{cop} \textsc{ref}.17=where \textsc{aug}-11-grindstone 11-\textsc{poss.1sg}\\
\glt `[Tortoise:] My child, where is my grindstone?'
\sn \gll aah, mwa=n-gambɪlɪ a-a-kaleele fiijo. \textbf{eeg}-\textbf{ile} ʊ-lw-ala lʊ-la, \textbf{a}-\textbf{taag}-\textbf{ile} m-mi-syanjʊ\\
 \textsc{interj} matronym=9-monkey 1-\textsc{pst}-be(come)\_angry.\textsc{pfv} \textsc{intens} 1.take-\textsc{pfv} \textsc{aug}-11-grindstone 11-\textsc{dist} 1-throw-\textsc{pfv} 18-4-bush\\
\glt \lq [Tortoise's child:] Aah, Mr. Monkey was very angry. He took that grindstone, he threw it into the bush.' [Monkey and Tortoise]
\ex \label{exPRSPFVsequence2} \gll \textbf{iis}-\textbf{ile}=\textbf{po} ʊ-mu-ndʊ jʊ-mo, a-a-sʊʊm-aga a-ma-beele. leelo \textbf{n}-\textbf{um̩}-\textbf{bʊʊl}-\textbf{ile} ʊkʊtɪ iis-e na=a-ma-jolo\\
1.come-\textsc{pfv}=16 \textsc{aug}-1-person 1-one 1-\textsc{pst}-beg-\textsc{ipfv} \textsc{aug}-6-breast\_milk now/but \textsc{1sg}-1-tell-\textsc{pfv} \textsc{comp} 1.come-\textsc{subj} \textsc{com}=\textsc{aug}-6-evening\\
\glt `Somebody came by, she was asking for breast milk. I told her to come back in the evening.' [Killer woman]
\end{exe} 

To sum up so far, the Nyakyusa present perfective neither necessarily brings about a persistent result, nor does it feature a strong relevance component. Now, recall from \sectref{TenseAspect} that \citeauthor{BotneRKershnerT2008} define tense in terms of inclusion and exclusion of the deictic locus. Being a present tense construction, not only does the vantage point evoked by the present perfective by default coincide with the time of speech, but the eventuality depicted is also situated within the same reference frame. Any notion of prevailing effects can thus be understood as a mere function of the eventuality being included within the here-and-now reality of the speech event, particularly when taking into consideration the opposition to its past tense\is{tense!past} counterpart \mbox{a(\textit{lɪ})-}\textsc{vb}\mbox{-\textit{ile}}. There is thus no need to assume any of the further components of meaning commonly assumed in the literature on anteriors such as the introduction of a \lq\lq perfect state'' (\citealt{MoensM1987}; \citealt{MoensMSteedmanM1988}), \lq\lq result state'' \citep{KampHReyleU1993} or a modal presupposition \citep{PortnerP2003};\footnote{Also note that these authors, in drawing on \citet{ReichenbachH1947}, take as a starting point the temporal ordering of an eventuality relative to a reference point (which for a present anterior equals time of speech), as in \citeauthor{ReichenbachH1947}'s famous formalisation \lq\lq E--R,S''. As seen in \sectref{PerfectivityCompletion}, the vantage point evoked rather follows the characteristic act (Nucleus)\is{phase!Nucleus phase} which may be followed by a resultant state phase (Coda).\is{phase!Coda phase}} see \citet{RitzM2012} for an overview of common theories of anteriors.

\largerpage
Nurse's second criterion concerns compound verb constructions. Anteriors, but not perfectives, should be found in these. The present perfective with \isi{inchoative verbs} can serve as the complement of the persistive aspect \isi{auxiliary} (\sectref{Persistive}),\is{aspect!persistive} which would speak in favour of a classification as an anterior. However, the same combinatory possibility also holds for the past perfective, which itself is most likely derived historically from a compound verb construction consisting of the \isi{copula} plus the present perfective (\sectref{PastPerfective}).

Furthermore, the notion of \lq still', as expressed through the persistive in Nyakyusa, is typically incompatible with anteriors (see \citealt{NedjalkovPJaxontovS1988} among others). To combine with `still' is instead said to be a hallmark of resultatives.\is{resultative} These are grammatical constructions that express a state brought about by a past situation. In this they differ from anteriors, which focus on the past situation itself (\citealt[134]{DahlOe1985}; \citealt[69]{BybeePerkinsPaglucia1994}). The Nyakyusa perfective in its stative reading asserts that the subject is in the resultant state of the act lexically encoded as the Nucleus, and thus comes closer to a \isi{resultative} than to an anterior. As seen in \sectref{PerfectivityCompletion} above, this resultative-like\is{aspect!resultative} reading falls out naturally from the analysis of \mbox{-\textit{ile}} as a marker of nuclear completion.\is{completion}\is{phase!Nucleus phase}

Lastly, consider the organization of the Nyakyusa TAM system. Within the present tense, the perfective contrasts primarily with the simple present.\is{simple present} While the perfective selects a time posterior to the right edge of the Nucleus phase,\is{phase!Nucleus phase} the simple present denotes the unfolding or future occurrence of a single eventuality. That is, it denotes a time before the endpoint of the Nucleus.\is{phase!Nucleus phase} In the past tense,\is{tense!past} the very same opposition is found between the past perfective -- which shows the same interaction with the lexical dimension -- and the past imperfective.\is{aspect!imperfective}\is{aspect!grammatical}

In sum, an analysis of the configuration \textit{ø}-\textsc{vb}-\textit{ile} as an anterior is contradicted by its semantics and distribution. Instead, its meaning and use, including any overlap with the crosslinguistic category of anterior, are readily explained by the aspectual notion of a post-Nucleus vantage point, together with the present tense \lq\lq denoting a primary, prevailing [\ldots] perspective'' \citep[153]{BotneRKershnerT2008}. What is more, postulating an analysis of \textit{ø}-\textsc{vb}-\textit{ile} as an anterior would preclude a compositional analysis of its past-marked counterpart \mbox{\textit{a}(\textit{lɪ})-}\textsc{vb}-\mbox{\textit{ile}}.\footnote{Unless both are analysed as anteriors, which would not only render this label vacuous, but is contradicted by the negative criteria of their compatibility with sequential events and persistive aspect.\is{aspect!persistive}} It would also miss the systematic parallel between the aspectual oppositions in the present and past tenses.
\is{aspect!anterior|)}
\subsubsection{Summary}
To sum up, the Nyakyusa present perfective depicts the \isi{completion} of an eventuality's Nucleus\is{phase!Nucleus phase} without dissociation to a distant reference frame. In this it forms part of a coherent grammatical system, centred around the notions of perfectivity as \isi{completion} of the Nucleus phase\is{phase!Nucleus phase} in the dimension of grammatical aspect,\is{aspect!grammatical} and association vs. dissociation in the dimension of tense.\is{tense}

Consistent with this analysis, in past tense \isi{narrative} discourse the present perfective is found in a few clearly determined environments: first, as a narrative present,\is{narrative present} where it provides information ancillary to the storyline and patterns with the \isi{simple present} and present copula;\is{copula} and second, with relative time reference in subordinate clauses.\is{subordinate clauses} These two kinds of uses are discussed separately in \sectref{PRSasNonPST}. Lastly, it features in the coda section\is{section!coda} of some narratives (\ref{PRSPFVCodaNarration}), where it refers to the speaker-now and is again in predictable alternation with the other present-tense paradigms.

\begin{exe}
\ex \label{PRSPFVCodaNarration} \gll a-ka-pango ka-mal-iike\\
\textsc{aug}-12-story 12-finish-\textsc{neut}.\textsc{pfv}\\
\glt \lq The story is over.' [Monster with Guitar]
\end{exe}
\is{tense!present|)}\is{aspect!perfective|)}

\subsection{Negative present perfective}\label{NEGPresentPerfective}
\is{tense!present|(}\is{negative|(}
The negative counterpart to the present perfective\is{aspect!perfective} consists of the negative post-initial prefix \textit{ka}- and the default final vowel -\textit{a}.

\begin{exe}
\ex \textit{tʊkajoba} \lq we have not spoken'
\end{exe}
With inchoative verbs,\is{inchoative verbs} this construction typically negates the resultant state (\ref{NegPrsPfv1}), while with non-inchoatives, the eventuality is typically understood not to have occurred (\ref{NegPrsPfv2}).

\begin{exe}
\ex \label{NegPrsPfv1}\gll \textbf{ba}-\textbf{k}-\textbf{ii}-\textbf{gan}-\textbf{a} ʊ-kʊ-bomb-el-a ɪ-fy-ombo ɪ-fy-a kw-asim-a kʊ-ba-palamani\\
2-\textsc{neg}-\textsc{refl}-like-\textsc{fv} \textsc{aug}-15-work-\textsc{appl}-\textsc{fv} \textsc{aug}-8-tool \textsc{aug}-8-\textsc{assoc} 15-borrow-\textsc{fv} 17-2-neighbour\\
\glt \lq They do not like to work with tools borrowed from neighbours' [Types of tools in the home] %inchoative
\ex  \label{NegPrsPfv2}\gll mma, \textbf{a}-\textbf{ka}-\textbf{job}-\textbf{a} bo ʊ-lʊ. a-t-ile \textup{\lq\lq}kalʊlʊ ʊ-jʊ n-heesya gw-ɪtʊ\textup{''}\\
no, 1-\textsc{neg}-speak-\textsc{fv} as \textsc{aug}-\textsc{prox.11} 1-say-\textsc{pfv} \phantom{\lq\lq}hare(1) \textsc{aug}-\textsc{prox.1} 1-foreigner 1-\textsc{poss.1pl}\\
\glt \lq No, he didn't speak like that. He said \lq\lq This hare's our guest.''' [Saliki and Hare]
%non-inchoative
\end{exe}
Examples such as (\ref{exImperfectiveParadox1}, \ref{exImperfectiveParadox2}) constitute a variation on \citeauthor{DowtyD1979}'s (\citeyear{DowtyD1979}) \lq\lq imperfective paradox''.\is{aspect!imperfective} These were judged redundant but not contradictory. This suggests that what is actually negated is the \isi{completion} of the Nucleus phase.\is{phase!Nucleus phase}

\begin{exe}
\ex \label{exImperfectiveParadox1} \gll a-ka-fik-a, leelo a-lɪ pa-kʊ-fik-a\\
1-\textsc{neg}-arrive-\textsc{fv} now/but 1-\textsc{cop} 16-15-arrive-\textsc{fv}\\
\glt \lq He has not arrived (yet), but he is arriving.' [ET]
\ex \label{exImperfectiveParadox2} \gll a-ka-kalal-a, leelo a-lɪ pa-kʊ-kalal-a\\
1-\textsc{neg}-be(come)\_angry-\textsc{fv} now/but 1-\textsc{cop} 16-15-be(come)\_angry-\textsc{fv}\\
\glt \lq He is not angry (yet), but he is about to get angry.' [ET]
\end{exe}
\is{tense!present|)}\is{negative|)}
\subsection{Past perfective}\label{PastPerfective}
\is{tense!past|(}\is{aspect!perfective|(}
\subsubsection{Formal makeup}
The past perfective consists of a prefix \textit{a}- and the suffix -\textit{ile} (\ref{exPSTPFValiConsonantInitial}); see \sectref{Imbrication} on the allomorphs of -\textit{ile}. Preceding a vowel, i.e. in vowel-initial stems (\ref{exPSTPFValiVowelInitial}) or the reflexive object marker (\ref{exPSTPFValiReflexive}), the prefix of the past perfective has the allomorph \textit{alɪ}-. The usual rules of pre-stem vowel contact apply; see \sectref{HiatusSolution}.

\begin{exe}
\ex \label{exPSTPFValiConsonantInitial}
\begin{tabbing}
\textit{twaliikeetile}x\=(\degree tʊ-alɪ-i-keet-ile)x\=\kill
\textit{twabombile}\>(\degree tʊ-a-bomb-ile)\>`we worked'\\
\textit{twakeetile}\>(\degree tʊ-a-keet-ile)\>`we watched'
\end{tabbing}
\ex \label{exPSTPFValiVowelInitial}
\begin{tabbing}
\textit{twaliikeetile}x\=(\degree tʊ-alɪ-i-keet-ile)x\=\kill
\textit{twaliisile}\>(\degree tʊ-alɪ-is-ile)\>`we came'\\
\textit{twalyɪmile}\>(\degree tʊ-alɪ-ɪm-ile)\>`we stopped'\\
\textit{twalyegile}\>(\degree tʊ-alɪ-eg-ile)\>`we took'\\
\textit{twalyagile}\>(\degree tʊ-alɪ-ag-ile)\>`we found'\\
\textit{twalyogile}\>(\degree tʊ-alɪ-og-ile)\>`we bathed'\\
\textit{twalyʊmile}\>(\degree tʊ-alɪ-ʊm-ile)\>`we dried'
\end{tabbing}
\ex
\label{exPSTPFValiReflexive}
\begin{tabbing}
\textit{twaliikeetile}x\=(\degree tʊ-alɪ-i-keet-ile)x\=\kill
\textit{twaliikeetile}\>(\degree tʊ-alɪ-i-keet-ile)\>`we looked at ourselves'\\
\textit{twaliipakile}\>(\degree tʊ-alɪ-i-pak-ile)\>`we painted ourselves' 
\end{tabbing}
\end{exe}
The same allomorph \textit{alɪ}- also surfaces preceding the object markers\is{object marker} of the first person singular (\ref{exPSTPFValiOM1SG}) and noun class 1 (\ref{exPSTPFValiOMNCL1}), regardless of their respective allomorphs.\footnote{See \sectref{sectionParticipantOM}, \ref{sectionNCLOM} on the morphophonemics of these prefixes.}

\pagebreak
\begin{exe}
\ex\label{exPSTPFValiOM1SG}
\begin{tabbing}
\textit{mwalɪndaagile}x\=(\degree mu-alɪ-ny-taag-ile)x\=\kill
\textit{mwalɪndaagile}\>(\degree mu-alɪ-ny-taag-ile)\>`you (pl.) threw me'\\
\textit{mwalɪmbʊʊlile}\>(\degree mu-alɪ-ny-bʊʊl-ile)\>`you (pl.) told me'\\
\textit{mwalɪɪmetile}\>(\degree mu-alɪ-ny-met-ile)\>`you (pl.) shaved me'\\
\textit{mwalɪɪsalile}\>(\degree mu-alɪ-ny-sal-ile)\>`you (pl.) chose me'\\
\textit{mwalɪɪnyaagile}\>(\degree mu-alɪ-ny-ag-ile)\>`you (pl.) found me'
\end{tabbing}
\ex\label{exPSTPFValiOMNCL1}
\begin{tabbing}
\textit{twalɪmmwagile}x\=(\degree tʊ-alɪ-mu-bʊʊl-ile)x\=\kill%sinnnlose Zeile, fuer Tabulatoren
\textit{twalɪntaagile}\>(\degree tʊ-alɪ-mu-taag-ile)\>`we threw him/her'\\
\textit{twalɪm̩bʊʊlile}\>(\degree tʊ-alɪ-mu-bʊʊl-ile)\>`we told him/her'\\
\textit{twalɪmmetile}\>(\degree tʊ-alɪ-mu-met-ile)\>`we shaved him/her'\\
\textit{twalɪnsalile}\>(\degree tʊ-alɪ-mu-sal-ile)\>`we chose him/her'\\
\textit{twalɪmmwagile}\>(\degree tʊ-alɪ-mu-ag-ile)\>`we found him/her'\\ 
\textit{twalɪmmootile}\>(\degree tʊ-alɪ-mu-ot-ile)\>`we invited him/her'
\end{tabbing}
\end{exe}

The sequence /lɪ/ indicates that diachronically the longer form of the prefix is derived from a serial construction involving the \isi{copula} \textit{lɪ} (see \citealt{BotneR1986}). From a synchronic perspective, this allomorphy finds a functional explanation: due to the rules of vowel coalescence (\sectref{HiatusSolution}),\is{vowels!hiatus solution} a prefix \textit{a}- would assimilate to the vowel of vowel-initial stems and the \isi{reflexive} prefix, resulting in forms homophonous with the present perfective. Similarly, any vowel preceding the first person singular object prefix surfaces as long\is{vowels!length} (\sectref{sectionParticipantOM}), while any vowel preceding the class 1 object prefix surfaces as short\is{vowels!length} (\sectref{sectionNCLOM}). Again, without the alternation \textit{a}-/\textit{alɪ}- this would result in formal identity of the present and past perfective with subjects of \isi{noun classes} 1, 2, 6, 12 and 16. Note that classes 1 and 2 include all human beings. Thus the alternation between \textit{a}- and \textit{alɪ}- serves to avoid a high degree of ambiguity.\footnote{Interestingly, where Nyakyusa has \textit{ø}-\textsc{vb}-\textit{ile} vs. \textit{a(lɪ)}-\textsc{vb}-\textit{ile}, the \ili{Ngonde} variety described by \citet[76f]{KishindoP1999} has an opposition between \textit{a}-\textsc{vb}-\textit{ile} and \textit{ali}-\textsc{vb}-\textit{ile}.}

\subsubsection{Overview of meaning}\label{PastPerfectiveOverviewMeaning}
The past perfective construes a state-of-affairs as situated in the conceptual past (i.e. past D-domain in \citeauthor{BotneRKershnerT2008}'s framework; see \sectref{Tense}) with its Nucleus phase\is{phase!Nucleus phase} completed.\is{completion} The aspectual notion of \isi{completion} is discussed in more detail in \sectref{PerfectivityCompletion}. As (\ref{exPSTPFValiOM1SG}, \ref{exPSTPFValiOMNCL1}) show, with non-inchoative verbs this gives a typical posterior reading. With inchoative verbs,\is{inchoative verbs} the default reading is one of a state holding at some contextually defined past moment:

\begin{exe}
\ex \gll a-a-kaleele\\
1-\textsc{pst}-be(come)\_angry.\textsc{pfv}\\
\glt (Default reading:) \lq S/he was angry.' [ET]
\end{exe}

However, a posterior, holistic perspective is also possible with inchoative verbs. In the opening of a \isi{narrative} given in (\ref{exPSTPFVinchoativePosterior1}), the past perfective with the inchoative verb \textit{lambalala} \lq lie down, sleep' depicts the eventuality as a whole, rather than the state of being asleep at a certain point in time. The same holds for the inchoative verb\is{inchoative verbs} \textit{gona} \lq rest, sleep' in (\ref{exPSTPFVinchoativePosterior2}).

\begin{exe}
\ex \label{exPSTPFVinchoativePosterior1}\gll po leelo ɪ-n-galamu j-aa-lɪ m-bine. \textbf{j}-\textbf{aa}-\textbf{lambaleele} a-ma-sikʊ ma-tatʊ n-nyumba\\
then now/but \textsc{aug}-9-lion 9-\textsc{pst}-\textsc{cop} 9-ill 9-\textsc{pst}-lie\_down.\textsc{pfv} \textsc{aug}-6-day 6-three 18-house(9)\\
\glt \lq Lion was ill. He slept for three days in his house.' [Lion and Tortoise]

\ex \label{exPSTPFVinchoativePosterior2}
\gll bo a-fik-ile kʊ-ka-aja \textbf{a}-\textbf{a}-\textbf{gon}-\textbf{ile}. n=ʊ-lʊ-bʊnjʊ a-lɪnkʊ-bʊʊk-a m-ma-tengele\\
as 1-arrive-\textsc{pfv} 17-12-homestead 1-\textsc{pst}-rest-\textsc{pfv} \textsc{com}=\textsc{aug}-11-morning 1-\textsc{narr}-go-\textsc{fv} 18-6-bush\\
\glt \lq When she arrived home, she slept. In the morning she went into the bush.' [Mfyage turns into a lion] %es sei denn, noch besseres bsp gehalten
\end{exe}

Note that the past perfective by itself does not denote sequential events. Consider the following plot summary of a story, which was given subsequent to the \isi{narrative} itself; Hare's running alone, mentioned for the first time in (\ref{exHareTugutuSummarySentences1and2}), only takes place after Tugutu's preparations (\ref{exHareTugutuSummarySentence4}, \ref{exHareTugutuSummarySentence5}). The flowchart in \figref{FigureRelativeOrderHareTugutuSummary} illustrates the relative order of events. The shunt to the right symbolizes that the \isi{negative} verb in (\ref{exHareTugutuSummarySentence3})  remains outside the sequence as a function of its negative polarity.

\begin{exe}
\ex \label{exHareTugutuSummary} \begin{xlist}
\ex\label{exHareTugutuSummarySentences1and2}\gll a-a-bop-ile mw-ene, a-a-bop-ile mw-ene kalʊlʊ\\
1-\textsc{pst}-run-\textsc{pfv} 1-self 1-\textsc{pst}-run-\textsc{pfv} 1-self hare(1)\\
\glt `He ran alone, Hare ran alone.'
\ex\label{exHareTugutuSummarySentence3}\gll mwa=n-dugutu a-ka-a-bop-ile=po\\
matronym=9-type\_of\_bird 1-\textsc{neg}-\textsc{pst}-run-\textsc{pfv}=\textsc{part}\\
\glt `Mr. Tugutu did not run at all.'
\ex\label{exHareTugutuSummarySentence4}\gll a-a-ba-paal-ile a-ba-nine\\
1-\textsc{pst}-2-invite-\textsc{pfv} \textsc{aug}-2-colleague\\
\glt \lq He had gathered companions.'
\ex\label{exHareTugutuSummarySentence5}\gll bo a-ba a-a-ba-bɪɪk-ile ʊ-kʊ-tɪ maelɪ jɪ-mo, maelɪ jɪ-mo, maelɪ jɪ-mo\\
\textsc{ref.2} \textsc{aug}-\textsc{prox.2} 1-\textsc{pst}-2-put-\textsc{pfv} \textsc{aug}-15-say mile(9)(<EN) 9-one mile(9) 9-one mile(9) 9-one\\
\glt \lq Those are the ones he placed, like one mile, one mile, one mile.'%gutes bsp cleft
\ex\label{exHareTugutuSummarySentence6}\gll po kalʊlʊ a-a-bop-ile mw-ene\\
then hare(1) 1-\textsc{pst}-run-\textsc{pfv} 1-self\\
\glt \lq So Hare ran alone.' [Hare and Tugutu]
\end{xlist}
\end{exe}

\begin{figure}[hbt]
\begin{center}
\includegraphics{figures/GrafikRelativeOrderHareTugutuPFV.eps}
\caption{Relative event order of (\ref{exHareTugutuSummary})}
\label{FigureRelativeOrderHareTugutuSummary}
\end{center}
\end{figure}
 
\subsubsection{The past perfective in narrative discourse}
\label{PastPerfectiveNarrative}\is{narrative|(}
\paragraph{Introduction}
While in many languages a perfective form is the principle paradigm for relating the storyline in narrative discourse (see i.a. \citealt{HopperP1979}; \citealt{FleischmanS1990}), this does not hold for Nyakyusa.\footnote{For a recent discussion of the relationship between TMA constructions and narrating in African languages see \citet{PayneDLShirtzS2015}.} As discussed in \sectref{NarrativeMarkersLicensingDependency}, none of the narrative texts in the corpus exclusively features the past perfective  in the recollection of storyline eventualities. Instead, Nyakyusa possesses two dedicated \isi{narrative markers} -- see \sectref{NarrativeMarkersApproximation} for an introduction -- which are employed in a collaborative manner together with the past perfective to fulfil this function. 

As was observed in the discussion of (\ref{exHareTugutuSummary}) above, the past perfective may be used with sequential events, but does not inherently encode any chronological ordering. Further, what is construed with the past perfective has an independent existence in discourse, quite contrary to the \isi{narrative tense} and subsecutive,\is{subsecutive} which are pragmatically dependent and thus embedded in a situation established by other means, usually through the use of a preceding past tense verb (\sectref{NarrativeMarkersLicensingDependency}). A series of mere past perfectives, such as the plot summary in (\ref{exHareTugutuSummary}) above, thus results in a loose enumeration of discrete events rather than forming a coherent narrative discourse of its own.

 
In the following, the usage of the past perfective in Nyakyusa narrative discourse will be outlined. In order to identify patterns of usage, it is necessary to first take a look at the typical composition of Nyakyusa narratives\is{narrative} (\sectref{PastPFVNarrativeDiscourseOpenings}--\ref{PastPFVNarrativeDiscourseSupportive}). Considering a set of occurrences not readily explained by notions of textual macro-structure (\sectref{PastPFVNarrativeDiscourseOtherDiscontinuities}), it will then be shown that the various uses of the past perfective in \isi{narrative} discourse form part of a larger coherent pattern of tense usage, whose common denominator is the notion of \isi{thematic continuity} (\sectref{PastPFVNarrativeDiscourseSummary}). The following analysis is based mainly on the oral narratives\is{narrative} in the corpus, but some additional examples were taken from written texts.

  
\paragraph{Opening a narrative}\label{PastPFVNarrativeDiscourseOpenings}
A narrative in Nyakyusa typically opens with an orientation section\is{section!orientation} of varying length. By definition  -- see \sectref{ToolsNarrativeAnalysis} -- the orientation section\is{section!orientation} depicts states-of-affairs that hold throughout the whole text and orient the listener in respect to the setting. In the expected division of labour with the other past tense paradigms, the past perfective is used here with inchoative verbs,\is{inchoative verbs} giving a stative reading (\ref{exPSTPFVOrientationSectionInchoative}), as well as with non-inchoatives, which have a pluperfect reading (\ref{exPSTPFVOrientationSectionPluPerfect}).

\begin{exe}
\ex \label{exPSTPFVOrientationSectionInchoative}
\gll a-a-li=ko kalʊlʊ n=ɪɪ-fubu. \textbf{ba}-\textbf{a}-\textbf{many}-\textbf{eene} fiijo. kangɪ b-angal-aga pamopeene ʊ-bʊ-sikʊ b-oosa\\
1-\textsc{pst}-\textsc{cop}=17 hare(1) \textsc{com}=\textsc{aug}-hippo(9) 2-\textsc{pst}-know-\textsc{recp.pfv} \textsc{intens} again 2-\textsc{pst}.be\_well-\textsc{ipfv} together \textsc{aug}-14-time 14-all\\
\glt \lq Once upon a time, Hare and Hippo were very much friends. And they used to be together the whole time.' [Hare and Hippo]

\ex \label{exPSTPFVOrientationSectionPluPerfect}
\gll ijolo ky-a-li=ko ɪ-ky-amba. pa-mwanya pa-ky-amba pa-a-lɪ pa-tengaamu itolo. \textbf{ba}-\textbf{a}-\textbf{jeng}-\textbf{ile}=\textbf{po} a-ba-ndʊ ba-tatʊ\\
old\_times 7-\textsc{pst}-\textsc{cop}=17 \textsc{aug}-7-mountain 16-high 16-7-mountain 16-\textsc{pst}-\textsc{cop} 16-peaceful just 2-\textsc{pst}-build-\textsc{pfv}=16 \textsc{aug}-2-person 2-three\\
\glt \lq Long ago there was a mountain. On top of the mountain it was all peaceful. Three people had built there.' [Selfishness kills]
\end{exe}%xxx vokallänge nochmal checken tenga(a?)mu

Following the orientation section,\is{section!orientation} one commonly finds a verb in one of the past tense paradigms that sets the stage for the first episode. In cases where there is no orientation section,\is{section!orientation} this constitutes the opening of the text. For an example of the past perfective in this environment, see (\ref{exNarrBeginningPlotUnequalNarr}) on p.\nobreakspace\pageref{exNarrBeginningPlotUnequalNarr}.

\paragraph{Textual boundaries}\label{PastPFVNarrativeDiscourseBoundaries}
In longer narratives, subsequent major units are commonly also delimited by one or more past tense verbs. This often implies a shift away from one of the narrative markers.\is{narrative markers} These verbs depict the state-of-affairs out of which the following course of events develops. Again, the past perfective, imperfective\is{aspect!imperfective} and \isi{copula} pattern together, depending on the aspectual profile\is{aspect!grammatical} of the state-of-affairs and the verb's Aristotelian aspect.\is{aspect!Aristotelian}

It is important to note at this point that the notions of boundary marking and staging operate within the textual component.\is{component of meaning!textual} An event that forms part of the semantic storyline may additionally be construed as the setting of a unit within the text if it depicts the circumstances under which following eventualities take place \citep{PayneD1992}. By virtue of its aspectual semantics,\is{aspect!grammatical} it is the past perfective that is employed in these cases. Further, it is typically not every episode within longer texts that starts out with a past tense. The following examples will illustrate both points.

In the text \lq\lq Monkey and Tortoise'', lazy but witty Tortoise sends his child to Monkey's to get food, pretending to buy it on credit. After obtaining the food, little Tortoise returns home. They enjoy their meals and never make the agreed payment (\ref{exMonkeyTortoiseFirstNoPaying}). Clause (\ref{exMonkeyTortoiseSentence1}) is the last one in the food-obtaining episode. The following clauses (\ref{exMonkeyTortoiseSentence2}--\ref{exMonkeyTortoiseSentences6and7}) are all marked for past tense and group together to describe a major discontinuity in action and time. Of these, (\ref{exMonkeyTortoiseSentences4and5}, \ref{exMonkeyTortoiseSentences6and7}), by virtue of the states-of-affairs they depict, as well as the fact that they include two negatives,\is{negative} give information that diverges from the main storyline. The eating in (\ref{exMonkeyTortoiseSentence2}), however, is no less eventive than the preceding walk home. Nevertheless, it is construed as part of the upcoming episode's setting.

\begin{exe}
\ex \label{exMonkeyTortoiseFirstNoPaying} \begin{xlist}
\ex \label{exMonkeyTortoiseSentence1} \gll po mw-ene jʊ-la gw-a kajamba a-lɪnkʊ-bʊʊk-a kʊ-ka-aja n=ɪ-fi-ndʊ fi-la\\
then 1-self 1-\textsc{dist} 1-\textsc{assoc} tortoise(1) 1-\textsc{narr}-go-\textsc{fv} 17-12-homestead \textsc{com}=\textsc{aug}-8-food 8-\textsc{dist}\\
\glt \lq Tortoise's child went home with that food.'
\ex \label{exMonkeyTortoiseSentence2}
\gll po n=ʊ-gwise po \textbf{ba}-\textbf{a}-\textbf{l}-\textbf{iile}\\
then \textsc{com}=\textsc{aug}-his\_father(1) then 2-\textsc{pst}-eat-\textsc{pfv}\\
\glt \lq  He and his father ate.'
\ex \label{exMonkeyTortoiseSentence3}
\gll po ba-a-hobwike fiijo a-ma-sikʊ ma-bɪlɪ\\
then 2-\textsc{pst}-be(come)\_happy.\textsc{pfv} \textsc{intens} \textsc{aug}-6-day 6-two\\
\glt \lq They were very happy for two days.'

\ex \label{exMonkeyTortoiseSentences4and5} \gll leelo bo gɪ-kɪnd-ile ɪ-mi-lʊngʊ mi-bɪlɪ, kajamba a-ka-a-buj-iisye. a-ka-a-bʊʊk-ile kw-a mwa=n-gambɪlɪ kʊ-kʊ-bɪɪk-a ɪɪ-heela\\
now/but as 4-pass-\textsc{pfv} \textsc{aug}-4-week 4-two tortoise(1) 1-\textsc{neg}-\textsc{pst}-return-\textsc{caus.pfv} 1-\textsc{neg}-\textsc{pst}-go-\textsc{pfv} 17-\textsc{assoc} matronym=9-monkey 17-15-put-\textsc{fv} \textsc{aug}-money(9)\\
\glt \lq When two weeks passed, Tortoise had not paid. ‎‎He had not gone to Monkey's to hand over the money.'

\ex  \label{exMonkeyTortoiseSentences6and7} \gll gw-a-kɪnd-ile ʊ-n̩-dʊngʊ gʊ-mo. gy-a-kɪnd-ile mi-bɪlɪ na=ka-mo n=ʊ-kʊ-bʊʊk-a kʊ-kʊ-bɪɪk-a\\
3-\textsc{pst}-pass-\textsc{pfv} \textsc{aug}-3-week 3-one 4-\textsc{pst}-pass-\textsc{pfv} 4-two \textsc{com}=12-one \textsc{com}=\textsc{aug}-15-go-\textsc{fv} 17-15-put-\textsc{fv}\\
\glt\lq One week passed. Two weeks passed, not once going to hand it over.'

\ex \label{exMonkeyTortoiseSentence8} \gll po mwa=n-gambɪlɪ a-a-tɪ \textup{\lq\lq}hee. po m-bʊʊk-e kw-a kajamba kʊ-kʊ-mel-a ɪɪ-heela sy-angʊ\textup{''}\\
as matronym=9-monkey 1-\textsc{subsec}-say \phantom{\lq\lq}\textsc{interj} then \textsc{1sg}-go-\textsc{subj} 17-\textsc{assoc} tortoise(1) 17-15-claim-\textsc{fv} \textsc{aug}-money(10) 10-\textsc{poss.1sg}\\
\glt \lq Then Monkey said \lq\lq Hee. ‎‎I shall go to Tortoise to demand my money.''{}'
\ex \label{exMonkeyTortoiseSentence9} \gll po a-lɪnkw-end-a, a-lɪnkw-end-a, a-lɪnkw-end-a\\
then 1-\textsc{narr}-walk/travel-\textsc{fv} 1-\textsc{narr}-walk/travel-\textsc{fv} 1-\textsc{narr}-walk/travel-\textsc{fv}\\
\glt \lq He walked and walked and walked.' [Monkey and Tortoise]
\end{xlist}
\end{exe}

With (\ref{exMonkeyTortoiseSentence8}, \ref{exMonkeyTortoiseSentence9}) the storyline continues. Tortoise makes an excuse and Monkey gives up on claiming his money that day. This episode is narrated entirely using the \isi{narrative markers} (plus one token of a narrative present).\is{narrative present} Now, after two more weeks, Monkey returns to Tortoise's only to hear another excuse and to go home again without his money. This episode, whose main events essentially repeat those of the preceding one, is not further delimited by a shift to the past tense (\ref{exMonkeyTortoiseNoStaging}).

\largerpage[2]
\begin{exe}
\ex \label{exMonkeyTortoiseNoStaging}
\begin{xlist}
\ex \label{exMonkeyTortoiseNoStagingSentence1} \gll kangɪ po bo gɪ-kɪnd-ile ɪ-mi-lʊngʊ mi-bɪlɪ, kangɪ \textbf{a}-\textbf{lɪnkʊ}-\textbf{buj}-\textbf{a} kw-a kajamba\\
again then as 4-pass-\textsc{pfv} \textsc{aug}-4-week 4-two again 1-\textsc{narr}-return-\textsc{fv} 17-\textsc{assoc} tortoise(1)\\
\glt \lq When two weeks had passed, he returned again to Tortoise's.'

\ex \gll bo i-kʊ-fik-a i-kʊ-mmw-ani-a ʊ-mw-anaake \lq\lq a-li=po ʊ-gʊʊso a-pa?''\\
as 1-\textsc{prs}-arrive-\textsc{fv} 1-\textsc{prs}-1-ask-\textsc{fv} \textsc{aug}-1-his\_child \phantom{\lq\lq}1-\textsc{cop}=16 \textsc{aug}-your\_father \textsc{aug}-\textsc{prox.16}\\
\glt \lq As he was arriving he asks his (Tortoise's) child \lq\lq Is your father here?''{}' [Monkey and Tortoise]
\end{xlist}
\end{exe}

In contrast, the next episode does start with a shift to the past tense (\ref{exMonkeyTortoiseStaging}). Note how the content of the first two clauses of this episode (\ref{exMonkeyTortoiseStagingSentence1}, \ref{exMonkeyTortoiseStagingSentence2}) mostly parallels the first clause of the preceding episode (\ref{exMonkeyTortoiseNoStagingSentence1}). The use of the past perfective instead of the \isi{narrative tense} coincides with a change in Monkey's attitude. This entails an important shift in the development of events. Instead of continuing the routine of the preceding two episodes, Tortoise finds himself needing to resort to a more elaborate trick to get away without paying his debts.
\begin{exe}
\ex\label{exMonkeyTortoiseStaging} 
\begin{xlist}
\ex \label{exMonkeyTortoiseStagingSentence1} \gll ii-sikʊ lɪ-lɪ-ngɪ po \textbf{a}-\textbf{al}-\textbf{iis}-\textbf{ile} mwa=n-gambɪlɪ\\
5-day 5-5-other then 1-\textsc{pst}-come-\textsc{pfv} matronym=9-monkey\\
\glt \lq Another day Mr. Monkey came.'
\ex \label{exMonkeyTortoiseStagingSentence2} \gll \textbf{a}-\textbf{al}-\textbf{iis}-\textbf{ile} n=ɪ-ly-ojo\\
1-\textsc{pst}-come-\textsc{pfv} \textsc{com}=\textsc{aug}-5-anger\\
\glt \lq He came with anger.'
\ex \gll \textup{\lq\lq}lɪlɪno kʊ-m-b-a=ko ɪɪ-heela j-angʊ, kʊ-m-b-a=ko ɪɪ-heela j-angʊ\textup{''} i-kʊ-job-a mu-n-jɪla\\
\phantom{\lq\lq}now/today \textsc{2sg.prs}-\textsc{1sg}-give-\textsc{fv}=17 \textsc{aug}-money(9) 9-\textsc{poss.1sg} \textsc{2sg.prs}-\textsc{1sg}-give-\textsc{fv}=17 \textsc{aug}-money(9) 9-\textsc{poss.1sg} 1-\textsc{prs}-speak-\textsc{fv} 18-9-path\\
\glt \lq\lq \lq Today you're giving me my money, you're giving me my money'', he is saying on his way.'
\ex\gll a-lɪnkw-a kʊ-fik-a pa-a kajamba\\
1-\textsc{narr}-go.\textsc{fv} 15-arrive-\textsc{fv} 16-\textsc{assoc} tortoise(1)\\
\glt \lq He arrived at Tortoise's.' [Monkey and Tortoise]
\end{xlist}
\end{exe}

\paragraph{Further supportive material}
\label{PastPFVNarrativeDiscourseSupportive}
The storyline may be sprinkled with occasional supportive information in the past tense, be it the past perfective, past imperfective\is{aspect!imperfective} or the past copula,\is{copula} or a \isi{narrative present} (see \sectref{NarrativePresent} on the latter). This may consist of embedded orientations, that is, states-of-affairs that hold throughout all or parts of the text. The past perfective is therefore used with inchoative verbs.\is{inchoative verbs} This is common with verbs depicting psychological states, as illustrated in (\ref{exPstPfvSupportiveInchoativeSentences2}). Note that with most \isi{inchoative verbs} the past perfective is the only paradigm that can denote the resultant state and not merely the change-of-state. 
\newpage
\begin{exe}
\ex Context: Monkey is waiting for Tortoise.

\begin{xlist}
\ex \gll a-a-j-ile kʊ-buj-a. \textup{\lq\lq}Hahaha mwa=n-gambɪlɪ ʊ-li=po. gw-is-ile, pole! n-gʊ-p-a lɪlɪno ɪɪ-heela sy-ako\textup{''}\\
1-\textsc{pst}-go-\textsc{pfv} 15-return-\textsc{fv} \phantom{\lq\lq}\textsc{interj} matronym=9-monkey \textsc{2sg}-\textsc{cop}=16 \textsc{2sg}-come-\textsc{pfv} sorry(<SWA) \textsc{1sg}-\textsc{prs}-give-\textsc{fv} now/today \textsc{aug}-money(10) 10-\textsc{poss.2sg}\\
\glt \lq He [Tortoise] went and returned. \lq\lq Hahaha, Mr. Monkey, you are here! You have come, sorry! I'm giving you your money now.''
\ex \label{exPstPfvSupportiveInchoativeSentences2}
 \gll po mwa=n-gambɪlɪ \textbf{a}-\textbf{a}-\textbf{hobwike} \textup{\lq\lq}haa \textup{(clapping hands)} mma po kʊ-m-b-a ahahaa\textup{''}\\
then matronym=9-monkey 1-\textsc{pst}-be(come)\_happy.\textsc{pfv}
\phantom{\lq\lq}\textsc{interj} {} no then \textsc{2sg.prs}-\textsc{1sg}-give-\textsc{fv} \textsc{interj}\\
\glt \lq Mr. Monkey was happy \lq\lq Haa (clapping hands), you are giving me [my money], ahahaa.'' [Monkey and Tortoise]
\end{xlist}
\end{exe}

The past perfective is also used for flashbacks. It is the only verbal paradigm attested in the case of flashbacks.

\begin{exe}
\ex Context: Tortoise's child and Monkey are waiting for Tortoise.
\begin{xlist}
\ex \gll ba-lɪnkʊ-m̩-bon-a bo a-fum-ile bo kʊ-la\\
2-\textsc{narr}-1-see-\textsc{fv} as 1-come\_from-\textsc{pfv} as 17-\textsc{dist}\\
\glt \lq They saw him [Tortoise] as he came from over there.' 

\ex \gll ʊ-tɪ \textbf{ba}-\textbf{alɪ}-\textbf{n}-\textbf{taag}-\textbf{ile} mu-m-mi-syanjʊ\\
\textsc{2sg}-say.\textsc{subj} 2-\textsc{pst}-1-throw-\textsc{pfv} 18-18-4-bush\\
\glt \lq Do you know, they had thrown him into the bush.'

\ex \gll po \textbf{a}-\textbf{a}-\textbf{j}-\textbf{ile} kw-end-a kʊ-la\\
then 1-\textsc{pst}-go-\textsc{pfv} 15-walk/travel-\textsc{fv} 17-\textsc{dist}\\
\glt \lq He had (gone and) had a walk there.'

\ex \gll \textbf{a}-\textbf{a}-\textbf{j}-\textbf{ile} kʊ-buj-a\\
1-\textsc{pst}-go-\textsc{pfv} 17-return-\textsc{fv}\\
\glt \lq He had gone and returned.' [Monkey and Tortoise]
\end{xlist}
\end{exe}

\paragraph{Further storyline events} \label{PastPFVNarrativeDiscourseOtherDiscontinuities}
As has been discussed thus far, the past perfective is frequently employed -- together with the other past tense paradigms -- in narrative discourse for setting the stage at major episode boundaries and for providing information ancillary to the storyline. In addition, it is commonly found in relating more storyline events, where its employment vis-à-vis the dedicated \isi{narrative markers} is not readily explained by the patterns described above. A closer examination shows that in these cases it normally depicts eventualities that are unpredictable, pivotal for the development of the plot and/or constitute a major change in the roles of participants.\footnote{\citet[8]{JonesLBJonesLK1979} define pivotal events as \lq\lq very crucial or significant events of a narrative [\ldots] when collected together, they form a high-level summary or abstract of the story''. This definition is reflected in \citet[90]{TomlinR1985} and \citet[5]{LongacreR1990}.} In \sectref{PastPFVNarrativeDiscourseSummary} below it will be shown that all these uses of the past perfective form part of one coherent pattern of employment of the past tense paradigms vis-à-vis the narrative markers.\is{narrative markers} It will be shown that this pattern is governed by the notion of thematic continuity.\is{thematic continuity}

The uses of the past perfective with further storyline events will be illustrated by walking through the narrative of \lq\lq Hare and Tugutu''. A summary of this story was given in (\ref{exHareTugutuSummary}) on p.\nobreakspace\pageref{exHareTugutuSummary} above. (\ref{exOpeningHareTugutu}) presents the first three clauses of the narrative. Instead of featuring a proper orientation section,\is{section!orientation} the two protagonists and the setting are introduced in the opening clause (\ref{exOpeningHareTugutuSentence1}). This is followed by a switch to the \isi{narrative tense} in (\ref{exOpeningHareTugutuSentence2}). The use of the past perfective in (\ref{exOpeningHareTugutuSentence3}), however, is conspicuous, as this example neither constitutes an episode boundary nor backgrounded material. However, the proposition given in this clause, Tugutu's challenging of Hare, is a major turning point in the story. It constitutes the raison d'être for the following race, around which the story revolves. Furthermore, apart from Tugutu's answer being somewhat unexpected, there is a major discontinuity in the relative roles of the participants. From this point on, Tugutu is the focal character driving the action. This stands in contrast to the first two clauses which centre around Hare.
\begin{exe}
\ex\label{exOpeningHareTugutu}
\begin{xlist}
\ex \label{exOpeningHareTugutuSentence1}\gll ii-sikʊ lɪ-mo, kalʊlʊ a-aly-ag-an-iile n=ii-tugutu\\
5-day 5-one hare(1) 1-\textsc{pst}-find-\textsc{recp}-\textsc{appl.}\textsc{pfv} \textsc{com}=5-type\_of\_bird\\
\glt \lq One day, Hare met with Tugutu (a type of bird).'
\ex \label{exOpeningHareTugutuSentence2}
\gll a-lɪnkʊ-tɪ \textup{\lq\lq}gwe, tugutu, ʊ-gwe! ʊ-ka-bagɪl-a ʊ-kʊ-n-gɪnd-a ʊ-ne ʊ-kʊ-bop-a, ʊ-ne n-gʊ-bop-a fiijo, m-bagiile ʊ-kʊ-kʊ-tol-a\textup{''}\\
1-\textsc{narr}-say \phantom{\lq\lq}\textsc{2sg} t.o.bird \textsc{aug}-\textsc{2sg} \textsc{2sg}-\textsc{neg}-be\_able-\textsc{fv} \textsc{aug}-15-\textsc{1sg}-surpass-\textsc{fv} \textsc{aug}-\textsc{1sg} \textsc{aug}-15-run-\textsc{fv} \textsc{aug}-\textsc{1sg} \textsc{1sg}-\textsc{prs}-run-\textsc{fv} \textsc{intens} \textsc{1sg}-be\_able.\textsc{pfv} \textsc{aug}-15-\textsc{2sg}-beat-\textsc{fv}\\
\glt \lq He said \lq\lq You, Tugutu, you! You can't beat me in running, I run fast, I can beat you.'''
\ex \label{exOpeningHareTugutuSentence3}\gll ii-tugutu \textbf{ly}-\textbf{a}-\textbf{t}-\textbf{ile} \textup{\lq\lq}mma. ʊ-ka-bagɪl-a ʊ-kʊ-n-dol-a kalʊlʊ ʊ-kʊ-bop-a. tʊ-bagiile pamo ʊ-kʊ-tol-an-a, ʊ-kʊ-fwan-a\textup{''}\\
5-t.o.bird 5-\textsc{pst}-say-\textsc{pfv} \phantom{\lq\lq}no \textsc{2sg}-\textsc{neg}-be\_able-\textsc{fv} \textsc{aug}-15-\textsc{1sg}-beat-\textsc{fv} hare(1) \textsc{aug}-15-run-\textsc{fv} \textsc{1pl}-be\_able.\textsc{pfv} maybe \textsc{aug}-15-beat-\textsc{recp}-\textsc{fv} \textsc{aug}-15-be(come)\_equal-\textsc{fv}\\
\glt \lq Tugutu said \lq\lq No. You, Hare, can't beat me in running. We can maybe have a tie.''{}' [Hare and Tugutu]
\end{xlist}
\end{exe}

  
The continuation of the story is given in (\ref{exHareTugutuPlacing}). Hare's answer in (\ref{exHareTugutuPlacingSentence1}), using the subsecutive,\is{subsecutive} still forms part of the challenging episode. With (\ref{exHareTugutuPlacingSentence2}--\ref{exHareTugutuPlacingSentence7}) the narrator shifts to the past perfective as he takes up a new line of events: Tugutu invites his companions and places them along the track. Leaving aside the parenthetic insertion in (\ref{exHareTugutuPlacingSentence4}), the events given are preliminary and decisive for the development of the plot. Rather than mere staging for the upcoming race, these can be considered to form a coherent episode of their own. 
%discontinuity: scene, auch temporal terms nur loosely connected, zudem anderes theme 
\largerpage
\begin{exe}
\ex \label{exHareTugutuPlacing}\begin{xlist}
\ex \label{exHareTugutuPlacingSentence1}\gll a-a-tɪ \textup{\lq\lq}mma popaa$\sim$po tw-and-e kɪ-laabo ʊ-kʊ-bop-a\textup{''}\\
1-\textsc{subsec}-say \phantom{\lq\lq}no \textsc{redupl}$\sim$then \textsc{1pl}-start-\textsc{subj} 7-tomorrow \textsc{aug}-15-run-\textsc{fv}\\
\glt \lq He (Hare) said \lq\lq Ok then, tomorrow let's start to run.''{}'
\ex \label{exHareTugutuPlacingSentence2} \gll po mwa=n-dugutu \textbf{a}-\textbf{a}-\textbf{bʊʊk}-\textbf{ile}\\
then matronym=9-t.o.bird 1-\textsc{pst}-go-\textsc{pfv}\\
\glt \lq Mr. Tugutu went.'
\ex\label{exHareTugutuPlacingSentence3} \gll \textbf{a}-\textbf{a}-\textbf{ba}-\textbf{paal}-\textbf{ile} a-ba-nine ba-haano\\
1-\textsc{pst}-2-invite-\textsc{pfv} \textsc{aug}-2-companion 2-five\\
\glt \lq He gathered five companions.'
\ex\label{exHareTugutuPlacingSentence4} \gll paapo ba-al-iitɪk-eene na kalʊlʊ ʊ-kʊ-bop-a a-ma-eli ma-haano\\
because 2-\textsc{pst}-agree-\textsc{recp.pfv} \textsc{com} hare(1) \textsc{aug}-15-run-\textsc{fv} \textsc{aug}-6-mile(EN) 6-five\\
\glt \lq Because they had agreed to run five miles.'
\ex\label{exHareTugutuPlacingSentence5} \gll bo ii-sikʊ ly-a kɪ-laabo, bo lɪ-fik-ile, mwa=n-dugutu \textbf{a}-\textbf{alɪ}-\textbf{m̩}-\textbf{bɪɪk}-\textbf{ile} mwa=n-dugutu n-nine pa-bw-andɪlo\\
as 5-day 5-\textsc{assoc} 7-tomorrow as 5-arrive-\textsc{pfv} matronym=9-t.o.bird 1-\textsc{pst}-1-put-\textsc{fv} matronym=9-t.o.bird 1-companion 16-14-start\\
\glt \lq When the next day arrived, Mr. Tugutu placed a fellow Mr. Tugutu at the start.'
\ex\label{exHareTugutuPlacingSentence6} \gll ɪɪ-maeli j-aa bʊ-bɪlɪ mwa=n-dugutu ʊ-jʊ-ngɪ \textup{[\ldots]}\\
\textsc{aug}-mile(9) 9-\textsc{assoc} 14-two matronym=9-t.o.bird \textsc{aug}-1-other\\
\glt \lq The second mile another Mr. Tugutu [\ldots]'
\ex\label{exHareTugutuPlacingSentence7}\gll na kʊ-bʊ-malɪɪkɪlo, maeli ga-a bʊ-haano, mwa=n-dugutu ʊ-jʊ-ngɪ\\
\textsc{com} 17-14-end mile 6-\textsc{assoc} 14-five matronym=9-t.o.bird \textsc{aug}-1-other\\
\glt \lq At the finish line, the fifth mile, another Mr. Tugutu.' [Hare and Tugutu]%\footnotemark
\end{xlist}
\end{exe}

\largerpage
The rest of the story consists of a detailed description of the race itself and the dialogue that takes place throughout the race, which are narrated entirely using the narrative markers,\is{narrative markers} with the exception of one flashback.
\paragraph{Summary and discussion}\label{PastPFVNarrativeDiscourseSummary}
The preceding sections have shown a number of recurring contexts in which the past perfective is employed within narrative discourse. It can be seen that it is used, as are the other past tense paradigms, in free\is{clause types (Labov \& Waletzky)!free clause} and restricted\is{clause types (Labov \& Waletzky)!restricted clause} clauses that provide information ancillary to the storyline. These clauses may be contained within a dedicated orientation section\is{section!orientation} or be distributed throughout the text. Consistent with the semantics of the past perfective, this mostly gives a stative reading with \isi{inchoative verbs} and a posterior one with non-inchoatives. Within the \isi{plot} proper, the latter can be employed for flashbacks. The past perfective is further used to set the stage at major boundaries within the text, again in a division of labour with the other past tense paradigms. This is common as a \lq backdrop' to the first episode of a text, and is also found in major units throughout narratives. Lastly, the past perfective is also found with storyline events that form important turning points.
\is{storyline ranks|(}
Note that this distribution of the past perfective defies \citeauthor{LongacreR1990}'s approach of storyline ranks, which is widely received in linguistic discourse studies, especially in the SIL tradition. According to Longacre, TMA categories within a given language can be ranked  in order of the degree to which the clauses they are found in either belong to the primary storyline, augment the storyline (pivotal events) or depart from it (e.g. secondary storyline, backgrounded actions, setting); also see \sectref{NarrativeMarkersTypologies}. The Nyakyusa past perfective figures all over Longacre's storyline ranks: it is used with ancillary material, i.e. below the storyline, with events that serve as staging for what follows, as well as with pivotal events, which have the highest degree of saliency in Longacre's framework. That is, the employment of TMA categories in Nyakyusa narrative discourse cannot be explained on the basis of storyline ranks.

Interestingly, in \citeauthor{LongacreR1990}'s own work one finds a case that very much parallels that of Nyakyusa. In a discussion of \ili{Avokaya} (East Sudanic, Nilo-Saharan), Longacre stipulates that the so-called \textit{narrative tense} is a marker of the primary storyline, while the perfect or perfective -- his use of the two terms is inconsistent -- belongs to a lower rank, the secondary storyline \citep[91--97]{LongacreR1990}. The latter paradigm, however, is used not only to stage and re-stage sequences of actions, but also to \lq\lq show that a particular action is not script-predictable''. Longacre goes on to conclude that in the first case we are dealing with events that are \lq\lq preliminary and preparatory for what follows [\ldots] but this usage grades off into one representing events that are not script-predictable, but rather of themselves shift a sequence of actions/events into the direction of a new script''.
\is{storyline ranks|)}

If not storyline concerns, then what is the conceptual category that governs Nyakyusa narrative discourse? \citet[120]{NurseD2008}, in a discussion of narrative markers in Bantu, states that \lq\lq [u]se of the special [narrative, BP] marker can be suspended and then deliberately reintroduced by the speaker to stress continuity.'' Put the other way around, use of verbal paradigms other than the narrative markers signals discontinuity. If one states Nurse's vague term of continuity more precisely as thematic continuity,\is{thematic continuity} all of the uses of the past perfective, as well as of the other past tense paradigms, find a common denominator. As \citet{GivonT1984} describes it, \isi{thematic continuity} is a conceptual notion that holds within (parts of) a text and is made up of four thematic dimensions: the three dimensions of time, place, action, i.e. the unity well-known from ancient Greek playwrights, plus a fourth one of participants. Languages may employ specific signals that continuity is significantly interrupted in at least one of these four dimensions \citep{DooleyRALevinsohnSH2000}. \tabref{TableThematicContinuityDiscontinuity} illustrates the dimensions of thematic continuity and their respective continuities and discontinuities.

\begin{table}[htb]
\begin{center}
\begin{tabularx}{\textwidth}{lXX}
\lsptoprule 
& \footnotesize{Continuity} & \footnotesize{Discontinuity}\\
\midrule
\footnotesize{time} & events separated by at most only small forward gaps & large forward gaps or events out of order\\
\footnotesize{place} & same place or continuous motion & discrete changes of place\\
\footnotesize{action} & all material of the same type & changes from one type of material to another \\
\footnotesize{participants} & same cast and same general roles vis-à-vis one other & discrete changes of cast or changes in relative roles\\
\lspbottomrule 
\end{tabularx}
\captionabove{Dimensions of thematic continuity/discontinuity (adapted from \citealt[19]{DooleyRALevinsohnSH2000})}\is{thematic continuity}
\label{TableThematicContinuityDiscontinuity}
\end{center}
\end{table}

The opening of a narrative, as well as an initial orientation section,\is{section!orientation} are discontinuous among all these dimensions with regard to the discursive context. Ancillary information distributed throughout the text is highly discontinuous with regard to the storyline, by virtue of the type of material it provides (dimension of action), as well as by being outside the temporal sequence (dimension of time). The latter also holds for flashbacks. Episodes and groups of episodes within a text are by definition major thematic groupings that deserve to be delimited by use of a past tense paradigm. Lastly, important turning points in the plot constitute major changes in at least one of the thematic dimensions.

To sum up then, as far as tense and \isi{narrative markers} are concerned, Nyakyusa narrative discourse is structured around the conceptual notions of \isi{thematic continuity} in the storyline (employment of the narrative markers)\is{narrative markers} and discontinuity (employment of the past tense paradigms). Comparable analyses focusing on \isi{thematic continuity} as the motivation for the choice of TMA categories in narrative discourse have been put forward by \citet{WattersD2002} for \ili{Kham} (Sino-Tibetan) and by \citet{RobarE2014} for Biblical Hebrew.\il{Biblical Hebrew} A similar line of reasoning is made by \citet{CarlsonR1992} for a number of West and East African languages. The past perfective occupies a privileged role in this interplay between past tense and \isi{narrative markers} as it can be employed for discontinuous events. It is important to note at this point that this employment of the TMA categories constitutes a discursive convention, which leaves the narrator with room for stylistic considerations. There is considerable variation across texts concerning how much of the storyline is carried by the past perfective vis-à-vis the narrative markers.\is{narrative markers} This is further discussed and illustrated in \sectref{NarrativeMarkersUseDistribution}.\is{narrative|)}
\is{tense!past|)}\is{aspect!perfective|)}
\subsection{Negative past perfective}
\label{NEGPstPFV}\is{tense!past|(}\is{aspect!perfective|(}\is{negative|(}
The negative past perfective is formed with the negative prefix \textit{ka}-, the past prefix \textit{a}- and the perfective suffix -\textit{ile} or one of its allomorphs (see \sectref{Imbrication}). As with the affirmative past perfective, the past prefix surfaces as \textit{alɪ}- preceding vowel-initial stems, the \isi{reflexive} object marker,\is{object marker} as well as the object markers of the first person singular and noun class 1; see \sectref{PastPerfective}.

\begin{exe}
\ex \textit{tʊkaajobile} \lq we did not work'
\end{exe}

With non-inchoative verbs, this construction gives a negative posterior/ho\-lis\-tic reading (\ref{exNegativePastPerfective1}). As such it is by far the most common form used in the \is{narrative} corpus to assert the non-occurrence of eventualities. With inchoative verbs,\is{inchoative verbs} all tokens in the corpus have a negative state reading, as illustrated in (\ref{exNegativePastPerfective2}, \ref{exNegativePastPerfective3}).

\begin{exe}
\ex \label{exNegativePastPerfective1}
\gll po jɪ-lɪnkʊ-bʊʊk-a ʊ-kʊ-sook-a=po. po leelo \textbf{jɪ}-\textbf{ka}-\textbf{a}-\textbf{bʊʊk}-\textbf{ile} pa-bʊ-tali\\
then 9-\textsc{narr}-go-\textsc{fv} \textsc{aug}-15-leave-\textsc{fv}=16 then but/now 9-\textsc{neg}-\textsc{pst}-go-\textsc{pfv} 16-14-long\\
\glt \lq It [snake] went and left. It did not go far.' [Python and woman]

\ex \label{exNegativePastPerfective2}
\gll ʊ-mw-ene kalʊlʊ \textbf{a}-\textbf{ka}-\textbf{a}-\textbf{meenye} ʊkʊtɪ ʊ-lw-ɪfi lʊ-nyeel-iile pa-n-sana paapo lw-a-lɪ lʊ-pepe\\
\textsc{aug}-1-self hare(1) 1-\textsc{neg}-\textsc{pst}-know.\textsc{pfv} \textsc{comp} \textsc{aug}-11-chameleon 11-jump-\textsc{appl.pfv} 16-3-waist because 11-\textsc{pst}-\textsc{cop} 11-light\\
\glt \lq Hare himself did not know that Chameleon had jumped at his hip because he was light.' [Hare and Chameleon]

\ex \label{exNegativePastPerfective3} \gll a-a-li=ko kajamba jʊ-mo. a-a-lɪ m-oolo fiijo. \textbf{a}-\textbf{ka}-\textbf{al}-\textbf{iigan}-\textbf{ile} ʊ-kʊ-bomb-a ɪ-m-bombo\\
1-\textsc{pst}-\textsc{cop}=17 tortoise(1) 1-one 1-\textsc{pst}-\textsc{cop} 1-lazy \textsc{intens} 1-\textsc{neg}-\textsc{pst}-like-\textsc{pfv} \textsc{aug}-15-work-\textsc{fv} \textsc{aug}-9-work\\
\glt \lq There was a certain tortoise. It was very lazy. It did not like to work.' [Monkey and Tortoise]
\end{exe}
\is{tense!past|)}\is{aspect!perfective|)}\is{negative|)}
\subsection{Past imperfective}\label{PastImperfective}
\is{tense!past|(}\is{aspect!imperfective|(}
\subsubsection{Formal makeup and overview of meaning}
The past imperfective is formed with the past prefix \textit{a}- and the imperfective suffix -\textit{aga}. See \sectref{AlternationsIPFVaga} for allomorphs of the latter.

\begin{exe}
\ex \textit{twajobaga} \lq we used to speak / we were speaking'
\end{exe}

\largerpage
This construction has the uses typically associated with a past imperfective category (\citealt{ComrieB1976}; \citealt{DahlOe1985}). It can give a past continuous reading (\ref{exPSTIPFVcontinuous1}, \ref{exPSTIPFVcontinuous2}) and can also give a broad range of past habitual or generic readings\is{aspect!habitual}\is{aspect!generic} (\ref{exPSTIPFVhab}, \ref{exPSTIPFVgen}).
\begin{exe}

\ex \label{exPSTIPFVcontinuous1}  
\gll a-pa ʊ-m-b-eele ɪ-fi-ndʊ. ɪ-n-jala \textbf{j}-\textbf{a}-\textbf{n}-\textbf{dʊm}-\textbf{aga}\\
\textsc{aug}-\textsc{prox.16} \textsc{2sg}-\textsc{1sg}-give-\textsc{pfv} \textsc{aug}-8-food \textsc{aug}-9-hunger 9-\textsc{pst}-\textsc{1sg}-bite-\textsc{ipfv}\\
\glt \lq Here you've given me food. I was hungry [lit. hunger was biting me].' [Lake Kyungululu]


\ex 
\label{exPSTIPFVcontinuous2} \gll fyobeene \textbf{gw}-\textbf{a}-\textbf{job}-\textbf{aga}, nalooli lʊ-kafu\\
therefore \textsc{2sg}-\textsc{pst}-speak-\textsc{ipfv} really 11-difficult\\
\glt `That is why you were speaking, it [the lock] truly is tough.' [Wage of the thieves]

\ex
\label{exPSTIPFVhab}
\gll n-kɪ-panga kɪ-la a-a-li=po ʊ-n-kiikʊlʊ ʊ-n-hɪɪji ʊ-jʊ \textbf{a}-\textbf{a}-\textbf{bomb}-\textbf{aga} ɪ-j-aa kw-ib-a ɪ-fi-ndʊ ɪ-fi ba-pɪɪj-ile a-ba-nine,\\
18-7-village 7-\textsc{dist} 1-\textsc{pst}-\textsc{cop}=16 \textsc{aug}-1-woman \textsc{aug}-1-thief \textsc{aug}-\textsc{prox.1} 1-\textsc{pst}-work-\textsc{ipfv} \textsc{aug}-9-\textsc{assoc} 15-steal-\textsc{fv} \textsc{aug}-8-food \textsc{aug}-\textsc{prox.8} 2-cook-\textsc{pfv} \textsc{aug}-2-companion\\
\glt `In that village there was a thieving woman, who used to steal the food the others had cooked,'\\
\sn  \gll \textbf{a}-\textbf{a}-\textbf{fyʊl}-\textbf{aga} mu-n-deko na muu-sefulɪla\\
1-\textsc{pst}-remove-\textsc{ipfv} 18-10-earthen\_pot \textsc{com} 18-cooking\_pot(9)(<SWA)\\
\glt \lq She used to take it out of earthen pots and cooking pots.'
\sn \gll pa-la lɪnga a-bʊʊk-ile n-k-ookol-a ʊ-m-ooto kʊ-ba-nine, lɪnga ba-lɪ pa-nja, \textbf{a}-\textbf{a}-\textbf{kuputul}-\textbf{aga} ɪ-fi-ndʊ n=ʊ-kʊ-fyʊl-a=mo fi-mo, \textbf{a}-\textbf{a}-\textbf{bʊʊk}-\textbf{aga} na=fyo kʊ-my-ake\\
16-\textsc{dist} if/when 1-go-\textsc{pfv} 18-15-fetch\_fire-\textsc{fv} \textsc{aug}-3-fire 17-2-companion, if/when 2-\textsc{cop} 16-outside 1-\textsc{pst}-uncover-\textsc{ipfv} \textsc{aug}-8-food \textsc{com}=\textsc{aug}-15-remove-\textsc{fv}=18 8-one 1-\textsc{pst}-go-\textsc{ipfv} \textsc{com}=\textsc{ref.8} 17-4-\textsc{poss.sg}\\
\glt \lq When she went to her neighbours to get fire, if they were outside she would uncover the food, take some out and go home with it.' [Thieving woman]
\end{exe} 
\exewidth{(100)}
\begin{exe}
\ex
\gll a-ba-nyambala \textbf{ba}-\textbf{a}-\textbf{fwal}-\textbf{aga} ɪ-n-gʊbo j-aa ng'ombe mu-no mu-n-sana\\
\textsc{aug}-2-man 2-\textsc{pst}-dress/wear-\textsc{ipfv} \textsc{aug}-9-skin 9-\textsc{assoc} cow(9) 18-\textsc{prox} 18-3-waist\\
\glt \lq The men wore a skin of a cow here at the waist.' [Clothing long ago]\label{exPSTIPFVgen}
\end{exe}

\subsubsection{Uses in narrative discourse}
Not surprisingly, in past narratives\is{narrative} the past imperfective is mainly used in the orientation section,\is{section!orientation} as exemplified in (\ref{exPSTIPFVhab}) above. To a lesser extent, it is found in free\is{clause types (Labov \& Waletzky)!free clause} or restricted\is{clause types (Labov \& Waletzky)!restricted clause} clauses that constitute embedded orientation. This is illustrated in (\ref{exPSTIPFVembeddedOriention}). %später querverweisen zu BSP die bei pst.pfv verwendet

\begin{exe}
\ex \label{exPSTIPFVembeddedOriention} 
\gll kɪ-laabo ɪ-kɪ-ngɪ Sokoni a-lɪnkʊ-bʊʊk-a kangɪ kʊ-kʊ-kʊng-ɪl-a ɪɪ-ng'ombe\\
7-tomorrow \textsc{aug}-7-other S. 1-\textsc{narr}-go-\textsc{fv} again 17-15-tie-\textsc{appl}-\textsc{fv} \textsc{aug}-10.cow\\
\glt `The next day, Sokoni went again to tie the cows.'
\sn \gll p-oope a-alɪ-mmw-ag-ile ʊ-n-kasi gw-a Pakyɪndɪ i-kʊ-bomb-a bo sila$\sim$si-la ɪ-sy-a m-ma-jolo\\
16-also 1-\textsc{pst}-1-find-\textsc{pfv} \textsc{aug}-1-wife 1-\textsc{assoc} P. 1-\textsc{prs}-work-\textsc{fv} as \textsc{redupl}$\sim$10-\textsc{dist} \textsc{aug}-10-\textsc{assoc} 18-6-evening\\
\glt `Again he found Pakyindi's wife doing just like the day before.'
\sn\gll looli ʊ-n̩-dʊme \textbf{a}-\textbf{m̩}-\textbf{bʊʊl}-\textbf{aga} \textbf{a}-\textbf{a}-\textbf{t}-\textbf{ɪgɪ} \textup{\lq\lq}n-gʊ-bʊʊk-a kʊ-kʊ-nyukul-a a-ma-jabʊ\textup{''}\\
but \textsc{aug}-1-husband 1.\textsc{pst}-1-tell-\textsc{ipfv} 1-\textsc{pst}-say-\textsc{ipfv} \phantom{\lq\lq}\textsc{1sg}-\textsc{prs}-go-\textsc{fv} 17-15-pull\_up-\textsc{fv} \textsc{aug}-6-cassava\\
\glt `But to her husband she always said ``I am going to harvest cassava.''{}' [Sokoni and Pakyindi]
\end{exe}

The past imperfective is also used in the staging of episodes within a text, as in (\ref{exPSTIPFVstaging}); see \sectref{PastPFVNarrativeDiscourseBoundaries} on staging in Nyakyusa \isi{narrative} discourse.
\begin{exe}
\ex \label{exPSTIPFVstaging}
\begin{xlist}
\ex \gll ʊ-mu-ndʊ jʊ-mo \textbf{a}-\textbf{a}-\textbf{tiim}-\textbf{aga} ɪɪ-ng'oosi\\
\textsc{aug}-1-person 1-one 1-\textsc{pst}-herd-\textsc{ipfv} \textsc{aug}-sheep(10)\\
\glt \lq A man was herding sheep.'
\ex \gll a-a-lɪ n=ɪ-m-bwa\\
1-\textsc{pst}-\textsc{cop} \textsc{com}=\textsc{aug}-9-dog\\
\glt \lq He had a dog.'
\ex \gll popaa$\sim$po ʊ-mu-ndʊ jʊ-la a-lɪnkʊ-bʊʊk-a pa-kɪ-syanjʊ pa-kʊ-tʊʊsy-a\\
\textsc{redupl}$\sim$then \textsc{aug}-1-person 1-\textsc{dist} 1-\textsc{narr}-go-\textsc{fv} 16-7-thicket 16-15-rest-\textsc{fv}\\
\glt \lq That man went into the thicket to rest.'
\end{xlist}
\end{exe}

\subsubsection{Modal uses}\label{PastImperfectiveModal}
Apart from its basic uses, the past imperfective is also found in the apodoses of counterfactual conditionals.\is{conditional}\footnote{For another strategy of marking the apodoses of counterfactuals see \sectref{ngali}.} Employed in this way, the past imperfective loses its temporal\is{tense} and aspectual\is{aspect!grammatical} specification. It can be used with a present\is{tense!present} or future\is{tense!future} reading (\ref{exPSTIPFVapodosisPRS1}, \ref{exPSTIPFVapodosisPRS2}) and in typical perfective\is{aspect!perfective} contexts (\ref{exPSTIPFVapodosisPFV}).

\begin{exe}
\ex \label{exPSTIPFVapodosisPRS1} \gll lɪnga n-aa-lɪ ne laɪsi n-aa-ba-tʊʊl-aga a-ba-londo\\
if/when \textsc{1sg}-\textsc{pst}-\textsc{cop} \textsc{1sg} president(<SWA) \textsc{1sg}-\textsc{pst}-2-help-\textsc{ipfv} \textsc{aug}-2-poor\\
\glt `If I were president, I would help/be helping the poor.' [ET]
\ex \label{exPSTIPFVapodosisPRS2} \gll lɪnga n-aa-lɪ jo ʊ-ne n-aa-bʊʊk-aga kɪ-laabo\\
if/when \textsc{1sg}-\textsc{pst}-\textsc{cop} \textsc{ref.1} \textsc{aug}-\textsc{1sg} \textsc{1sg}-\textsc{pst}-go-\textsc{ipfv} 7-tomorrow\\
\glt \lq If it were me, I would go tomorrow.' [ET]
\ex \label{exPSTIPFVapodosisPFV} \gll lɪnga tʊ-ka-aly-ag-ile ʊ-lw-ɪsi tw-a-fw-aga\\
if/when \textsc{1pl}-\textsc{neg}-\textsc{pst}-find-\textsc{pfv} \textsc{aug}-11-river \textsc{1pl}-\textsc{pst}-die-\textsc{ipfv}\\
\glt `If we had not found the river, we would have died.' [ET]
\end{exe}

Another modal\is{modality} use is attested in the following example:

\begin{exe}
\ex Context: The researcher has asked a language assistant if she is free in the following days.\\
\gll ee ka-lʊmbʊ n-dɪ na=a-ka-balɪlo. ʊ-gwe \textbf{gw}-\textbf{a}-\textbf{lond}-\textbf{aga} bo ndɪɪli tw-ag-an-il-e\\
yes 12-sibling\_of\_opposite\_sex \textsc{1sg}-\textsc{cop} \textsc{com}=\textsc{aug}-12-time \textsc{aug}-\textsc{2sg} \textsc{2sg}-\textsc{pst}-want-\textsc{ipfv} as when \textsc{1pl}-find-\textsc{recp}-\textsc{appl}-\textsc{subj}\\
\glt \lq Yes little brother, I have time. Whenever you want, let's meet.' [overheard]
\end{exe}
\subsection{Negative past imperfective}\label{NegPSTIPFV}\is{negative|(}
The negative counterpart to the past imperfective consists of the negative prefix \textit{ka}-, the past prefix \textit{a}- and the imperfective  suffix -\textit{aga}. See \sectref{AlternationsIPFVaga} for allomorphs of the latter.

\begin{exe}
\ex \textit{tʊkaajobaga} \lq we did not use to speak / we were not speaking'
\end{exe}

The uses of the negative past imperfective parallel those of its affirmative counterpart. It has a continuous/progressive reading (\ref{exNEGPASTIPFVcontinuous}), which also serves as the negative counterpart to the periphrastic past progressive\is{aspect!progressive} (\sectref{Progressive}). It is further used for negative past habituals and generics\is{aspect!habitual}\is{aspect!generic} (\ref{exNEGPSTIPFVhab}, \ref{exNEGPSTIPFVgen}).

\largerpage[2]
\begin{exe}
\ex \label{exNEGPASTIPFVcontinuous}\gll fiki gwe mw-inɪɪtʊ, gwe gw-a bʊ-tatʊ, gw-a-tʊ-bʊʊl-iile bo ʊ-seng-iigwe, \textbf{ʊ}-\textbf{ka}-\textbf{a}-\textbf{lek}-\textbf{aga} fiki tw-esa tʊ-seng-igw-e?\\
why \textsc{2sg} 1-our\_companion \textsc{2sg} 1-\textsc{assoc} 14-three \textsc{2sg}-\textsc{pst}-\textsc{1pl}-tell-\textsc{appl.pfv} as \textsc{2sg}-chop-\textsc{pass.pfv} \textsc{2sg}-\textsc{neg}-\textsc{pst}-let-\textsc{ipfv} why \textsc{1pl}-all \textsc{1pl}-chop-\textsc{pass}-\textsc{subj}\\
\glt `Why, our friend, you the third one, did you tell us when you were cut, why did you not let all of us be cut?' [Wage of the thieves]
\ex \label{exNEGPSTIPFVhab} \gll ʊ-mw-ana \textbf{a}-\textbf{k}-\textbf{end}-\textbf{aga}. ɪ-fy-ɪnja \textbf{a}-\textbf{k}-\textbf{end}-\textbf{aga}\\
\textsc{aug}-1-child 1-\textsc{neg}.\textsc{pst}-walk/travel-\textsc{ipfv} \textsc{aug}-8-year 1-\textsc{neg}.\textsc{pst}-walk/travel-\textsc{ipfv}\\
\glt `The child did not walk. For years it did not walk.' [Pregnant women]
\ex \label{exNEGPSTIPFVgen} \gll \textbf{ba}-\textbf{k}-\textbf{eeg}-\textbf{an}-\textbf{aga} ʊ-bw-egi bʊ-la bw-a kʊ-piny-a pamo kw-i-kanisa. b-aa-lɪ n=ʊ-bw-egi ʊ-bw-a kyenyeeji\\
2-\textsc{neg}-\textsc{pst}.marry-\textsc{recp}-\textsc{ipfv} \textsc{aug}-14-marriage 14-\textsc{dist} 14-\textsc{assoc} 15-tie-\textsc{fv} maybe 17-5-church(<SWA) 2-\textsc{pst}-\textsc{cop} \textsc{com}=\textsc{aug}-14-marriage \textsc{aug}-14-\textsc{assoc} informal\_type(<SWA)\\
\glt \lq They did not have weddings of that type where they tie the bond maybe at church. They had weddings of an informal type.' [Life and marriage long ago]  
\end{exe}

\is{tense!past|)}\is{aspect!imperfective|)}\is{negative|)}

\section{Periphrastic present and past constructions}
\label{ComplexConstructions}
In the following subsections, periphrastic present (non-past)\is{tense!present} and past tense\is{tense!past} constructions will be described. An overview of these is given in \tabref{TableComplexConstructions}. For ease of reading, the present and past forms of each paradigm will be discussed together in single sections (\sectref{Progressive}, \ref{Persistive}). Some further infrequent constructions will be discussed in \sectref{MinorConstructions}.

\begin{table}[H]
\setlength{\tabcolsep}{2pt}
\begin{tabularx}{\textwidth}{llll}
\lsptoprule
\footnotesize{Label} & \footnotesize{Shape} & \footnotesize{Example}\\
\midrule 
\multirow{2}{*}{Progressive} & \textsc{sm}-\textit{lɪ} \textit{pa}-\textit{kʊ}-\textsc{vb}-\textit{a} & \textit{tʊlɪ pakʊjoba} & \lq we are speaking' \\  
& \textsc{sm}-\textit{a}-\textit{lɪ} \textit{pa}-\textit{kʊ}-\textsc{vb}-\textit{a} & \textit{twalɪ pakʊjoba} & \lq we were speaking'\\ 
\multirow{2}{*}{Persistive} & \textsc{sm}-\textit{kaalɪ} (+ Verb) & \textit{tʊkaalɪ tʊkʊjoba} & \lq we still speak'\\ 
 & \textsc{sm}-\textit{a}-\textit{kaalɪ} (+ Verb) & \textit{twakaalɪ twajobaga} & \lq we were still speaking'\\
\lspbottomrule  
\end{tabularx}
\caption{Periphrastic non-past/present and past tense constructions}\label{TableComplexConstructions}
\end{table}

\subsection{Progressive}\label{Progressive}
\is{aspect!progressive|(}
The progressive consists of the \isi{copula} \textit{lɪ}\is{tense!present} and an \isi{infinitive} marked for \isi{locative} noun class 16. The past\is{tense!past} progressive is formed with the past prefix \textit{a}- on the copula.\is{copula}
\begin{exe}
\ex
\begin{xlist}
\ex \textit{tʊlɪ pakʊjoba} \lq we are speaking'
\ex \textit{twalɪ pakʊjoba} \lq we were speaking'
\end{xlist}
\end{exe}

There is no \isi{negative} counterpart to the progressive. Instead, the negative present\is{tense!present} (\sectref{NegPresent}) and the negative past\is{tense!past} imperfective\is{aspect!imperfective} (\sectref{NegPSTIPFV}) are used.

As the label \textit{progressive} suggests, this construction expresses that an eventuality is ongoing. Unlike the \isi{simple present} and the past imperfective,\is{tense!past}\is{aspect!imperfective} no habitual/generic\is{aspect!habitual}\is{aspect!generic} reading is available with the progressive.\footnote{\citet[168--174]{KershnerT2002} discusses a parallel construction in Sukwa,\il{Sukwa} for which she coins the label \lq\lq punctuated imperfectivity''. Contrasting this with the progressive reading of the \ili{Sukwa} equivalents of the simple present and past imperfective, she postulates that the periphrastic construction construes the subject as \lq\lq inside the event'' or \lq\lq engaged in the event'' in contrast with a mere unfolding of the eventuality. No indication of such a reading was found in the Nyakyusa data.} While frequently heard, use of the periphrastic progressive is far from obligatory. The \isi{simple present} and the past imperfective\is{tense!past}\is{aspect!imperfective} can also give a progressive reading  (\sectref{Present}, \ref{PastImperfective}).
In temporal clauses (\sectref{TemporalConditionalClauses}),\is{subordinate clauses!temporal clause} the progressive is used only infrequently and often retains a locational reading (\ref{exPROGtemporalclause1}, \ref{exPROGtemporalclause2}). The \isi{simple present} is the paradigm of choice for an ongoing eventuality in this context.

\begin{exe}
\ex \label{exPROGtemporalclause1}
\gll a-ba-ndʊ bo a-bo bi-kʊ-bʊʊk-a kʊ-kw-asim-a ɪ-fi-bombelo ɪ-fy-a kʊ-bomb-el-a ɪ-m-bombo \textbf{bo} a-b-iinaabo \textbf{ba}-\textbf{lɪ} \textbf{pa}-\textbf{kʊ}-\textbf{tʊʊsy}-\textbf{a}\\
\textsc{aug}-2-person as \textsc{aug}-\textsc{ref.2} 2-\textsc{prs}-go-\textsc{fv} 17-15-borrow-\textsc{fv} \textsc{aug}-8-tool \textsc{aug}-8-\textsc{assoc} 15-work-\textsc{appl}-\textsc{fv} \textsc{aug}-9-work as \textsc{aug}-2-their\_companion 2-\textsc{cop} 16-15-rest-\textsc{fv}\\
\glt \lq People like those go to borrow tools to do work with, when their fellows are resting.' [Types of tools in the home]
\ex \label{exPROGtemporalclause2}
\gll \textbf{lɪnga} \textbf{ba}-\textbf{lɪ} \textbf{pa}-\textbf{kʊ}-\textbf{sanuk}-\textbf{a} kʊʊ-nyuma kʊ-no k-oope ba-a-kyakyatɪl-aga fiijo bʊno$\sim$bʊ-no\\
if/when 2-\textsc{cop} 16-15-alter-\textsc{fv} 17-back(9) 17-\textsc{prox} 17-also 2-\textsc{pst}-move\_back\_and\_forth-\textsc{ipfv} \textsc{intens} \textsc{redupl}$\sim$14-\textsc{dem}\\
\glt \lq When they were turning back there they would move quickly back and forward like this.' [Custom of dancing]
\end{exe}

A further difference between the \isi{simple present} and the past imperfective\is{tense!past}\is{aspect!imperfective} on the one hand and the (past) progressive on the other concerns the interaction of the progressive with the lexical class of resultative achievements: with these verbs, the \isi{simple present} and the past imperfective\is{tense!past}\is{aspect!imperfective} are not available in a progressive reading, while the periphrastic progressive may be used with reference to the resultant state (see Chapter \ref{AspectualClassification}).\footnote{Note that \isi{locative} noun class 16 commonly denotes proximity (\sectref{NounClasses}). To summarize the interaction of the periphrastic progressive with different lexical classes as analysed in Chapter \ref{AspectualClassification}, one finds an ongoing/pre-change reading with those verbs that feature either an extended Nucleus\is{phase!Nucleus phase} or an extended Onset phase.\is{phase!Onset phase} Resultative achievements, however, feature a punctiliar Nucleus plus an extended Coda,\is{phase!Coda phase} but lack an Onset phase. One may now interpret the phase selection of the progressive as a metaphoric extension of the erstwhile \isi{locative} semantics: \lq proximity to X' → \lq proximity to culmination of the characteristic act (N)'. The post-change reading with resultative achievements can then be understood as a \lq\lq second-best choice'' where no other phase around the right edge of N is available.}
\is{aspect!progressive|)}
\subsection{Persistive}\label{Persistive}\is{aspect!persistive|(}
Persistive aspect denotes that a state-of-affairs continues to hold from an earlier point until a later point of reference, by default the moment of speech. Grammaticalized constructions for this \lq\lq still-tense'' are common in Bantu, although often overlooked \citep[45]{NurseD2008}.
In Nyakyusa, persistive aspect is expressed by a periphrastic construction consisting of the subject prefix and a persistive \isi{auxiliary} \textit{kaalɪ}.

\begin{exe}
\ex \label{exPersistiveIntroductory}\gll tʊ-kaalɪ tʊ-kʊ-job-a\\
\textsc{1pl}-\textsc{pers} \textsc{1pl}-\textsc{prs}-speak-\textsc{fv}\\
\glt `we still speak / we are still speaking'
\end{exe}

The persistive \isi{auxiliary} takes an \isi{infinitive} with the augment, an \isi{infinitive} additionally marked for either of \isi{locative} classes 16 and 18, or an inflected verb as its complement. It can also be used with nominal predicates and for \isi{locative} predication. Some of these combinations merit a short discussion.

With an \isi{infinitive} complement, the persistive denotes that the act encoded in the verb has not yet taken place (\lq yet to V'):
\begin{exe}
\ex \gll m-ba-kooliile ʊkʊtɪ m-ba-lagɪl-e a-ma-syʊ bo \textbf{n}-\textbf{gaalɪ} \textbf{ʊ}-\textbf{kʊ}-\textbf{fw}-\textbf{a}\\
\textsc{1sg}-\textsc{2pl}-call.\textsc{pfv} \textsc{comp} \textsc{1sg}-\textsc{2pl}-dictate-\textsc{subj} \textsc{aug}-6-word as \textsc{1sg}-\textsc{pers} \textsc{aug}-15(\textsc{inf})-die-\textsc{fv}\\
\glt `I've called you (pl.) to give you instructions before I die.' [Chief Kapyungu]
\ex \gll nsyɪsyɪ a-lɪnkʊ-m̩-bʊʊl-a kalʊlʊ a-lɪnkʊ-tɪ \textup{\lq\lq}ɪɪ-nyama jɪ-p-iile, is-aga tʊ-ly-ege!\textup{''} po kalʊlʊ a-lɪnkʊ-tɪ \textup{\lq\lq}taasi. \textbf{jɪ}-\textbf{kaalɪ} \textbf{ʊ}-\textbf{kʊ}-\textbf{py}-\textbf{a}\textup{''}\\
skunk(1) 1-\textsc{narr}-1-tell-\textsc{fv} hare(1) 1-\textsc{narr}-say \phantom{\lq\lq}\textsc{aug}-meat(9) 9-be(come)\_burnt-\textsc{pfv} come-\textsc{ipfv} \textsc{1pl}-eat-\textsc{ipfv.subj} then hare(1) 1-\textsc{narr}-say yet 9-\textsc{pers} \textsc{aug}-15(\textsc{inf})-be(come)\_burnt-\textsc{fv}\\
\glt `Skunk told Hare ``The meat is done, come let's eat!'' Hare said ``Later. It's not yet done.''{}' [Hare and Skunk]
\end{exe}

This is also the default interpretation for the persistive without an overt complement (\ref{exBarePersistive}). If a polar question contains the persistive plus a complement, a bare persistive in the answer is understood as being elliptic (\ref{exPersistiveQuestion}).

\begin{exe}
\ex \label{exBarePersistive} \gll ɪ-li-sikʊ ly-a kw-and-a a-a-bʊʊk-ile kʊ-kʊ-keet-a ɪ-fi-lombe muno fi-j-ɪɪl-iile ʊkʊtɪ kalɪ fi-bɪfiifwe pamo \textbf{fi}-\textbf{kaalɪ}\\
\textsc{aug}-5-day 5-\textsc{assoc} 15-begin-\textsc{fv} 1-\textsc{pst}-go-\textsc{pfv} 17-15-look-\textsc{fv} \textsc{aug}-8-maize whether 8-be(come)-\textsc{appl}-\textsc{pfv} \textsc{comp} \textsc{q} 8-ripen.\textsc{pfv} or 8-\textsc{pers}\\
\glt `On the first day he went to look at how the maize was looking to see if it had ripened yet or not.' [Thieving monkeys]
\ex \label{exPersistiveQuestion}
\gll bʊle, ʊ-kaalɪ kʊ-manyil-a? -- ee, \textbf{n}-\textbf{gaalɪ}\\
\textsc{q} \textsc{2sg}-\textsc{pers} \textsc{2sg.prs}-learn-\textsc{fv} {} yes \textsc{1sg}-\textsc{pers}\\
\glt \lq Are you still studying?' -- \lq Yes, I still am.' [overheard]
\end{exe}

\newpage
Apart from the bare infinitive,\is{infinitive} verbal nouns additionally marked for \isi{locative} classes 16 and 18 are attested. It is unclear whether these differ in meaning and in how far speaker preferences and diatopic\is{dialects} variation are involved.\footnote{For a comparable case with \isi{phasal verbs} see \sectref{VerbalNounsArguments}.}

\begin{exe}
\ex \label{exPers16INF}
\gll i-kʊ-j-a pa-kʊ-kwel-a kangɪ, paapo ɪ-kɪ-kapʊ kɪ-mo \textbf{kɪ}-\textbf{kaalɪ} \textbf{pa}-\textbf{kw}-\textbf{isʊl}-\textbf{a}\\
1-\textsc{prs}-be(come)-\textsc{fv} 16-15-climb-\textsc{fv} again because \textsc{aug}-7-basket 7-one 7-\textsc{pers} 16-15-be(come)\_full-\textsc{fv}\\
\glt \lq He is about to climb up again, because one basket is still empty.' [Elisha Pear Story]
\ex \label{exPers18INF}
\gll \textbf{ga}-\textbf{kaalɪ} \textbf{n}-\textbf{kʊ}-\textbf{kom}-\textbf{a}\\
6-\textsc{pers} 18-15-become\_ripe-\textsc{fv}\\
\glt \lq They [the bananas] are still not ripe.' [overheard]
\end{exe} 

With a \isi{simple present} (\sectref{Present}) or past imperfective\is{tense!past}\is{aspect!imperfective} (\sectref{PastImperfective}) complement, both the continuous and habitual/generic\is{aspect!habitual}\is{aspect!generic} readings are available; see (\ref{exPersistiveIntroductory}) above and (\ref{exPersPST1}) below. With inchoative verbs,\is{inchoative verbs} the persistive can take a complement inflected for perfective\is{aspect!perfective} aspect. This collocation denotes that the resultant state continues to hold. For further discussion see Chapter \ref{AspectualClassification}.\footnote{This argument is of course circular, as the possible collocation of persistive and perfective\is{aspect!perfective} is the major criterion for distinguishing inchoative verbs.\is{inchoative verbs}}

\begin{exe}
\ex \gll a-ba ba-ka-j-a ba-kɪlisiti \textbf{ba}-\textbf{kaalɪ} \textbf{b}-\textbf{ʊʊmɪɪliile} ʊ-lw-iho ʊ-lw-a kʊ-lond-a ʊ-kʊ-tiil-igw-a\\
\textsc{aug}-\textsc{prox.2} 2-\textsc{neg}-be(come)-\textsc{fv} 2-christian 2-\textsc{pers} 2-stick\_to.\textsc{pfv} \textsc{aug}-11-custom \textsc{aug}-11-\textsc{assoc} 15-want-\textsc{fv} \textsc{aug}-15-fear-\textsc{pass}-\textsc{fv}\\
\glt `Those that are not Christians stick to the tradition of wanting to be feared.' [Should she save a life\ldots]
\end{exe}

With the \isi{negative} counterpart to the present\is{tense!present} perfective\is{aspect!perfective} (\sectref{NEGPresentPerfective}) as a complement, the persistive denotes that the state-of-affairs encoded in the lexical verb still has not occurred. 

\begin{exe}
\ex \gll \textbf{a}-\textbf{kaalɪ} \textbf{a}-\textbf{ka}-\textbf{fik}-\textbf{a} pa-ka-aja\\
1-\textsc{pers} 1-\textsc{neg}-arrive-\textsc{fv} 16-12-homestead\\
\glt `S/he still has not arrived home.' [ET]
\ex \gll taata ʊ-ne nalooli ɪ-fy-ʊma \textbf{n}-\textbf{gaalɪ} \textbf{n}-\textbf{ga}-\textbf{kab}-\textbf{a} ɪɪ-sala ɪ-jɪ\\
my\_father \textsc{aug}-\textsc{1sg} truely \textsc{aug}-8-rich \textsc{1sg}-\textsc{pers} \textsc{1sg}-\textsc{neg}-get-\textsc{fv} \textsc{aug}-hour(9)(<SWA) \textsc{aug}-\textsc{prox.9}\\
\glt `Father [honorific], I still haven't obtained the brideprice.' [Man and his in-law] 
\end{exe}

The meaning of this combination (\lq still not V-ed') is obviously similar to that of the persistive with an \isi{infinitive} complement (\lq yet to V'). One difference can be found in pragmatics: when the pending state-of-affairs is expressed as counter to expectation or custom, it is the \isi{negative} perfective\is{aspect!perfective} that is used:
\begin{exe}
\ex \begin{xlist}
\ex \gll m-ma-jolo n-aa-lambaleele bo ʊ-n-kʊlʊ gw-angʊ \textbf{a}-\textbf{kaalɪ} \textbf{a}-\textbf{ka}-\textbf{fik}-\textbf{a} pa-ka-aja\\
18-6-evening \textsc{1sg}-\textsc{pst}-lie\_down.\textsc{pfv} as \textsc{aug}-1-elder\_sibling 1-\textsc{poss.sg} 1-\textsc{pers} 1-\textsc{neg}-arrive-\textsc{fv} 16-12-homestead\\
\glt `Yesterday I went to bed when my elder brother still had not come home.' (implies: his late arrival is unusual) [ET]
\ex \gll m-ma-jolo n-aa-lambaleele bo ʊ-n-kʊlʊ gw-angʊ \textbf{a}-\textbf{kaalɪ} \textbf{ʊ}-\textbf{kʊ}-\textbf{fik}-\textbf{a} pa-ka-aja\\
18-6-evening \textsc{1sg}-\textsc{pst}-lie\_down.\textsc{pfv} as \textsc{aug}-1-elder\_sibling 1-\textsc{poss.sg} 1-\textsc{pers} \textsc{aug}-15-arrive-\textsc{fv} 16-12-homestead\\
\glt `Yesterday I went to bed before my older brother arrived home.' [ET]
\end{xlist}
\end{exe}

The persistive can also take a predicate nominal as a complement with subjects other than the participants\is{discourse participants} (see \sectref{CopulaUse} on the use of the zero copula):\is{copula}
\begin{exe}
\ex \gll ba-kaalɪ ba-fumuke\\
2-\textsc{pers} 2-famous\\
\glt `They are still famous.' [ET]
\ex \gll a-ma-tunda ga-kaalɪ ma-nyaafu\\
\textsc{aug}-6-fruit 6-\textsc{pers} 6-tasty\\
\glt `The fruits are still tasty.' [ET]
\end{exe}

Interestingly, \isi{locative} predication, otherwise one of the cases where use of the \isi{copula} is obligatory, is also possible with the persistive (\ref{exPERSLOC0COP}). The \isi{copula} can be used here, though in most cases it is optional (\ref{exPERSLOCoptionalCOP}). Also see (\ref{exDistalKaSubordinateAbility}) on p.\nobreakspace\pageref{exDistalKaSubordinateAbility}.
\begin{exe}
\ex\label{exPERSLOC0COP} \gll po bo a-kaalɪ mu-n-jɪla po kajamba a-a-t-ile\\
then as 1-\textsc{pers} 18-9-path then tortoise(1) 1-\textsc{pst}-say-\textsc{pfv}\\
\glt `As he [Mr. Monkey] was still on the way, Tortoise said:' [Monkey and Tortoise]
\ex \label{exPERSLOCoptionalCOP} \gll ʊ-n-kiikʊlʊ a-kaalɪ (a-lɪ) paa-sokoni\\
\textsc{aug}-1-woman 1-\textsc{pers} (1-\textsc{cop}) 16-market(9)(<SWA)\\
\glt `The woman is still at the market.' [ET]
\end{exe}

The past persistive\is{tense!past} is formed with the past prefix \textit{a}- following the subject prefix\is{subject marker} (\ref{exPersPST1}, \ref{exPersPST2}). Note that this clearly distinguishes the persistive from the negated past copula,\is{negative}\is{tense!past}\is{copula} which shares the shape \textit{kaalɪ} with the present\is{tense!present} persistive (see \sectref{Copulae}).

\begin{exe}
\ex \label{exPersPST1}\gll tw-a-kaalɪ tw-a-ly-aga\\
\textsc{1pl}-\textsc{pst}-\textsc{pers} \textsc{1pl}-\textsc{pst}-eat-\textsc{ipfv}\\
\glt \lq We were still eating. / We still used to eat.' [ET]
\ex \label{exPersPST2}\gll tw-a-kaalɪ tw-a-kateele\\
\textsc{1pl}-\textsc{pst}-\textsc{pers} \textsc{1pl}-\textsc{pst}-be(come)\_tired.\textsc{ipfv}\\
\glt \lq We were still tired.' [ET]
\end{exe}
\is{aspect!persistive|)}
\subsection{Minor constructions}
\label{MinorConstuctions}In this subsection, a few less frequent constructions will be discussed. The collocation of the existential construction (\sectref{Existentials}) with the \isi{locative} class 18 enclitic plus the \isi{simple present} gives a reading of a constantly or consistently occurring eventuality (\ref{exExistential18PRS1}). In this, it can itself occur as the complement of the persistive\is{aspect!persistive} aspect auxiliary,\is{auxiliary} as illustrated in (\ref{exExistential18PRS2}); \citet[125]{NurseD1979} lists a comparable example. Its past\is{tense!past} counterpart is formed with the past imperfective\is{aspect!imperfective} (\ref{exExistential18PST}).

\largerpage[2]
\begin{exe}
\ex \label{exExistential18PRS1}\gll \textbf{a}-\textbf{li}=\textbf{mo} \textbf{i}-\textbf{kw}-\textbf{ib}-\textbf{a} ɪ-fi-lombe fy-angʊ\\
1-\textsc{cop}=18 1-\textsc{prs}-steal-\textsc{fv} \textsc{aug}-8-maize 8-\textsc{poss.1sg}\\
\glt \lq S/he constantly steals my maize.' [ET]
\ex \label{exExistential18PRS2}
\gll ʊ-saj-igw-ege n=ʊ-n-twa paapo na ʊlʊ \textbf{ʊ}-\textbf{kaalɪ} \textbf{ʊ}-\textbf{li}=\textbf{mo} \textbf{kʊ}-\textbf{bomb}-\textbf{a} ɪ-sy-a b-ooloolo. kangɪ lɪlɪno po ʊ-bomb-ile ɪ-n-gɪnd-ɪlɪsi kʊ-sy-a kw-and-a\\
\textsc{2sg}-bless-\textsc{pass}-\textsc{ipfv.subj} \textsc{com}=\textsc{aug}-1-lord because \textsc{com} now \textsc{2sg}-\textsc{pers} \textsc{2sg}-\textsc{cop}=18 \textsc{2sg.prs}-do-\textsc{fv} \textsc{aug}-10-\textsc{assoc} 14-kind again now/today then \textsc{2sg}-do-\textsc{pfv} \textsc{aug}-10-pass-\textsc{ints.agnr} 17-10-\textsc{assoc} 15-begin-\textsc{fv}\\
\glt \lq Blessed be thou of the lord, my daughter: for thou hast shewed more kindness in the latter end than at the beginning [lit. \ldots because you still continuously do kind things and now you have done things much more kind than those in the beginning]' (Ruth 3:10)
\ex \label{exExistential18PST} \gll a-a-li=mo a-a-tʊ-taamy-aga\\
1-\textsc{pst}-\textsc{cop}=18 1-\textsc{pst}-\textsc{1pl}-trouble-\textsc{ipfv}\\
\glt \lq S/he constantly annoyed us.' [ET]
\end{exe}

A parallel construction is formed with a noun class 16 (but not class 17) existential. This denotes a continuous single event or series of events. The following examples were both suggested during elicitation sessions directed at verb categorization (Chapter \ref{AspectualClassification}).

\begin{exe}
\ex\gll \textbf{n}-\textbf{di}=\textbf{po} \textbf{n}-\textbf{gʊ}-\textbf{ly}-\textbf{a} (ɪɪ-sala j-oosa)\\
\textsc{1sg}-\textsc{cop}=16 \textsc{1sg}-\textsc{prs}-eat-\textsc{fv} (\textsc{aug}-hour(<SWA) 9-all)\\
\glt \lq I have been eating continually (for a whole hour).' [ET]
\ex \gll ɪɪ-nyumba j-and-ile ʊ-kʊ-nyal-a, a-ba-ndʊ b-ingi \textbf{ba}-\textbf{li}=\textbf{po} \textbf{bi}-\textbf{kw}-\textbf{ingɪl}-\textbf{a}\\
\textsc{aug}-house(9) 9-begin-\textsc{pfv} \textsc{aug}-15-be(come)\_dirty-\textsc{fv} \textsc{aug}-2-person 2-many 2-\textsc{cop}=16 2-\textsc{prs}-enter-\textsc{fv}\\
\glt \lq The house has begun to get dirty, many people are entering.' [ET]
\end{exe}

Lastly, as discussed in \sectref{Progressive}, \ref{NarrativeTense}, \ref{ProspectiveKujapa}, the collocations of the \isi{copula} plus an \isi{infinitive} marked for \isi{locative} noun class 16 or 18 have become grammaticalized as the progressive,\is{aspect!progressive} the prospective/inceptive\is{aspect!prospective} (both class 16) and the \isi{narrative tense} (\sectref{NarrativeTense}) (class 18), respectively. An infinitive marked for \isi{locative} noun class 17, however, is very rare and maintains a primarily locative reading. \citet[23]{SchumannK1899} also notes that this is a \lq\lq very infrequent form'' (translated from the original German, BP) and \citet{EndemannC1914} mentions it only marginally. The only attested token in the data is (\ref{exCopInf17}). Also note (\ref{exCopInf17Busse}).

\begin{exe}
\ex \label{exCopInf17}Context: You are asked where your brother is.\\
\gll n-gw-ag-a kʊʊ-sala ɪ-si \textbf{a}-\textbf{lɪ} \textbf{kʊ}-\textbf{kʊ}-\textbf{kin}-\textbf{a}\\
\textsc{1sg}-\textsc{prs}-find-\textsc{fv} 17-hour(10)(<SWA) \textsc{aug}-\textsc{prox.10} 1-\textsc{cop} 17-15-play-\textsc{fv}\\
\glt \lq I think he is playing (= is where they play).' [ET]
\ex \label{exCopInf17Busse}
\gll leelo popaa$\sim$po n-aa-meenye ʊkʊtɪ taata \textbf{a}-\textbf{lɪ} \textbf{kʊ}-\textbf{kʊ}-\textbf{lɪm}-\textbf{a}\\
now/but \textsc{redupl}$\sim$then \textsc{1sg}-\textsc{pst}-know.\textsc{pfv} \textsc{comp} my\_father(1) 1-\textsc{cop} 17-15-farm-\textsc{fv}\\ \largerpage
\glt \lq I, however, knew that Father was farming.' (\citealt[204]{BusseJ1949}; orthography adapted)
\end{exe}

\label{MinorConstructions}
\section{The present as non-past}\label{PRSasNonPST}\is{tense!present|(}
Having discussed the individual present tense constructions, it is worth considering a number of cases in which all of the present tense paradigms pattern with references other than the time of speech, including the unmarked \isi{copula} (\sectref{Copulae}) and derived constructions. It is essential here to make a basic distinction between their use in main clauses on the one hand, and in subordinate clauses on the other hand.

With regard to the former, a widespread crosslinguistic phenomenon is the so-called narrative present,\is{narrative present} which is the employment of a present tense in past \is{narrative} discourse. While in a language such as \ili{English} this may be subsumed under a single category of \textit{present tense} uses, in Nyakyusa, with its pervasive system of aspectuality, this phenomenon encompasses a wider array of verbal paradigms. This will be discussed in \sectref{NarrativePresent}.

Regarding use in subordinate clauses, \citet{NurseD2008} exhorts researchers of Bantu languages not to be misled in their analysis by the sequence-of-tense rules commonly found in Western European languages, such as in the \ili{English} sentence \textit{When he had paid the car, he drove it home}, not *\textit{When he (has) paid the car he drove it home}. Discussing these sequence-of-tense rules, he observes that
\begin{quote}
No Bantu language with which I am familiar does what English does. Instead of shifting the tenses on the left one step further into the past, as English, Bantu languages would keep the forms on the left in the contexts on the right. \citep[159]{NurseD2008}
\end{quote}
For a more general discussion see \citet{CoverRTonhauserJ2015}. Tense use in subordinate contexts will be discussed in \sectref{PRSsubordinate}, where a further distinction will be made between different types of subordinate clauses.

As will become clear from the following discussion, in a number of subordinate contexts the present tense paradigms take their temporal reference from the matrix clause. Further, the \isi{simple present} (\sectref{Present}) may be used as a futurate,\is{futurate} and can be shifted to a future\is{tense!future} reference frame by the use of the future \isi{enclitic} \textit{aa}= (\sectref{ProcliticAa}). What is more, under certain conditions that will be discussed in their appropriate places, other paradigms such as the present perfective\is{aspect!perfective} (\sectref{PresentPerfective}) and the present or zero \isi{copula} (\sectref{Copulae}) may be used with reference to future state-of-affairs. What is termed \textit{present tense} throughout this study can thus well be understood as having non-past reference.\footnote{Cf. \citeauthor{KleinW1994}'s (\citeyear[124--128]{KleinW1994}) analysis of the \ili{German} present tense as expressing non-past.}
\subsection{Narrative present}\label{NarrativePresent}\is{narrative present|(}
As \citet[376]{FleischmanS1990} defines it, a narrative present is the primarily oral phenomenon of a present tense that is used in narratives,\is{narrative} where it refers to the past time of the storyworld. This device \lq\lq enables particular textual\is{component of meaning!textual} or expressive\is{component of meaning!expressive} effects because the meaning \lq simultaneous with S' [\ldots] is always open'' (p. 54). 

In Nyakyusa, given the basic division between inchoative\is{inchoative verbs} and non-inchoative verbs and its reflection in choice of grammatical aspect,\is{aspect!grammatical} as well as the morphologically divergent copulae,\is{copula} the phenomenon of narrative present encompasses a number of verbal paradigms. (\ref{exNarrativePresentPRS}) illustrates the use of the simple present.\is{simple present} Note how the use of a narrative present here forms part of a shift towards drama,\label{Drama}\footnote{\label{FNDrama}Drama, in the sense of \citet[43]{LongacreR1996}, is ``a very vivid style of discourse in which quotation formulas drop out and people speak out in multiple I-thou relations''.} characterized by the omission of an otherwise compulsory form of the quotative verb \textit{tɪ} (\sectref{defectiveti}). Example (\ref{exNarrativePresentNegPfv}) illustrates the use of the \isi{negative} counterpart to the present perfective\is{aspect!perfective} with the inchoative verb\is{inchoative verbs} \textit{manya} \lq know'.

\begin{exe}
\ex \label{exNarrativePresentPRS}
\gll po kalʊlʊ a-lɪnkʊ-lembʊk-a. \textbf{i}-\textbf{kʊ}-\textbf{kuut}-\textbf{a} \textup{\lq\lq}hɪhɪɪ. ba-n-gom-ile, ba-n-gom-ile, ba-n-gom-ile.\textup{''} po nsyɪsyɪ \textup{\lq\lq}jw-ani a-kʊ-kom-ile?\textup{''}\\
then hare(1) 1-\textsc{narr}-awake-\textsc{fv} 1-\textsc{prs}-cry-\textsc{fv} \phantom{\lq\lq}of\_crying 2-\textsc{1sg}-hit-\textsc{pfv} 2-\textsc{1sg}-hit-\textsc{pfv} 2-\textsc{1sg}-hit-\textsc{pfv} then skunk(1) \phantom{\lq\lq}1-who 1-\textsc{2sg}-hit-\textsc{pfv}\\
\glt \lq Then Hare woke up. He cries \lq\lq Hihii. They've beaten me, they've beaten me, they've beaten me.'' Skunk: \lq\lq Who's beaten you?'' [Hare and Skunk]
\ex \label{exNarrativePresentNegPfv}
\gll po ʊ-m-fimba gw-ake gw-a n-nyambala jʊ-la, a-lɪnkw-a kʊ-syɪl-a. a-lɪnkw-a kʊ-bɪɪk-ɪl-a ii-pumba n=ʊ-kʊ-syɪl-a kʊ-la\\
then \textsc{aug}-3-corpse 3-\textsc{poss.sg} 3-\textsc{assoc} 1-man 1-\textsc{dist} 1-\textsc{narr}-go.\textsc{fv} 15-bury-\textsc{fv} 1-\textsc{narr}-go.\textsc{fv} 15-put-\textsc{appl}-\textsc{fv} 5-grave \textsc{com}=\textsc{aug}-15-bury-\textsc{fv} 17-\textsc{dist}\\
\glt \lq The corpse of that man, he went and buried it. He went and put it in a grave and buried it there.'
\sn \gll ʊ-n-kasi \textbf{a}-\textbf{ka}-\textbf{many}-\textbf{a} ɪ-lɪ na ɪ-lɪ\\
 \textsc{aug}-1-wife 1-\textsc{neg}-know-\textsc{fv} \textsc{aug}-\textsc{prox.5} \textsc{com} \textsc{aug}-\textsc{prox.5}\\
\glt \lq His [killer's] wife does not know anything.' [Man and his in-law] 
\end{exe}

The narrative present is relatively infrequent in the corpus. With the exception of one text, in which nearly the entire peak episode is related in the narrative present, it hardly ever appears in consecutive clauses.

\citet{FleischmanS1990}, following \citet{BuffinJ1925}, distinguishes between two varieties of the narrative present: the ``action'' variety, used for narrating events, and the ``visualizing'' variety for descriptions. The majority of instances of the narrative present of the type discussed so far belong to the action variety. (\ref{exNarrativePresentNegPfv}) above is among the few exceptions. Furthermore, all occurrences of the action variety in the corpus feature the imperfective simple present.\is{aspect!imperfective}\is{simple present}

Concerning the visualizing variety of the narrative present, there is a reoccurring construction consisting of a present tense paradigm together with the interjection of surprise \textit{ngɪmba} (topolectual variant:\is{dialects} \textit{ndɪmba}) `behold, gosh'. This construction is in fact not limited to narratives,\is{narrative} but also found in past \isi{expository} texts. In the case of narratives,\is{narrative} the use of \textit{ngɪmba} plus narrative present often goes along with intrusions of the narrator, who typically foregrounds relevant information that is not transparent for at least one participant or the hearer. This is specifically the case with the secret intentions or attitudes of a protagonist. With this construction, a greater variety of verbal paradigms is encountered. (\ref{exNarrativePresentNgimbaPRS}--\ref{exNarrativePresentNgimbaCOP}) illustrate uses of the simple present,\is{simple present} the present perfective\is{aspect!perfective} and the zero copula,\is{copula} respectively.  

\begin{exe}
\ex \label{exNarrativePresentNgimbaPRS}
\gll bo bi-kʊ-gomok-a mu-n-jɪla, kalʊlʊ a-lɪnkʊ-jɪ-bʊʊl-a ɪɪ-fubu a-lɪnkʊ-tɪ \textup{\lq\lq}ba-kʊ-tuufiifye fiijo ʊ-gwe ʊkʊtɪ ʊ-lɪ n-nunu, looli fi-fy-ɪma fy-ako fi-kɪnd-ɪliile ʊ-bʊ-nywamu, ba-t-ile ʊ-pungusy-e\textup{''}.\\
as 2-\textsc{prs}-return-\textsc{fv} 18-9-path hare(1) 1-\textsc{narr}-9-tell-\textsc{fv} \textsc{aug}-hippo(9) 1-\textsc{narr}-say \phantom{\lq\lq}2-\textsc{2sg}-praise.\textsc{pfv} \textsc{intens} \textsc{aug}-\textsc{2sg} \textsc{comp} \textsc{2sg}-\textsc{cop} 1-good but 8-8-thigh 8-\textsc{poss.2sg} 8-pass-\textsc{ints}.\textsc{pfv} \textsc{aug}-14-big 2-say-\textsc{pfv} \textsc{2sg}-reduce-\textsc{subj}\\
\glt \lq As they were on the road returning, Hare told Hippo, ``They've praised you a lot, that you're a good person, but your thighs are too big, they have said you should lose weight.''{}'
\sn \gll looli \textbf{ngɪmba} kalʊlʊ \textbf{i}-\textbf{kʊ}-\textbf{lond}-\textbf{a} ɪɪ-nyama ɪ-j-aa kʊ-ly-a\\
but behold hare(1) 1-\textsc{prs}-want-\textsc{fv} \textsc{aug}-meat(9) \textsc{aug}-9-\textsc{assoc} 15-eat-\textsc{fv}\\
\glt  \lq But gosh, Hare wants meat for eating.' [Hare and Hippo]
\ex \label{exNarrativePresentNgimbaPFV}\gll kalʊlʊ a-lɪnkw-angal-a pala$\sim$pa-la kɪsita kʊ-lʊ-bon-a ʊ-lʊ-bʊbi.\\
hare(1) 1-\textsc{narr}-be\_well-\textsc{fv}  \textsc{redupl}$\sim$16-\textsc{dist} without 15-11-see-\textsc{fv} \textsc{aug}-11-spider \\
\glt `Hare stayed right there without seeing Spider.'
\sn \gll \textbf{ngɪmba} ʊ-lʊ-bʊbi \textbf{lʊ}-\textbf{bʊʊk}-\textbf{ile} kʊ-ka-aja kʊ-my-ake\\
behold \textsc{aug}-11-spider 11-go-\textsc{pfv} 17-12-homestead 17-4-\textsc{poss.sg}\\
\glt \lq Gosh, Spider has gone home.' [Hare and Spider]
\ex \label{exNarrativePresentNgimbaCOP}\gll a-a-kalang-ile ii-fumbɪ lɪ-la. \textbf{ngɪmba} ii-fumbɪ \textup{(}\textbf{ø}\textup{)} ly-a sota\\
1-\textsc{pst}-fry-\textsc{pfv} 5-egg 5-\textsc{dist} behold 5-egg (\textsc{cop}) 5-\textsc{assoc} python(9)\\
\glt \lq She fried that egg. Gosh, that egg is Python's.' [Python and woman]
\end{exe}

The interjection \textit{ngɪmba}/\textit{ndɪmba} is strongly associated with the present tense. It is otherwise only attested in the corpus as an exclamation of surprise or as a question tag in direct speech. It should be noted that exchanging the present tense paradigms in the above examples for their past tense\is{tense!past} counterparts would lead to a past-in-the-past reading.
\is{narrative present|)}

\subsection{Present tense in subordinate clauses}\label{PRSsubordinate}
In the following, the use of the present tense in the most common types of subordinate clauses which feature a finite indicative verb will be described. The focus is on past tense discourse.

\subsubsection{Temporal and conditional adverbial clauses}\label{TemporalConditionalClauses}\is{subordinate clauses!temporal clause|(}\is{subordinate clauses!conditional protasis|(}
Temporal adverbial clauses are most commonly introduced by the augmentless noun class 14 referential demonstrative \textit{bo}, which in this use will be glossed as \lq as' throughout this study. When it introduces temporal clauses \textit{bo} is unstressed and may be considered a proclitic to the verb phrase.

These clauses introduced by \textit{bo} can occur in either pre-verbal or post-verbal position, although the pre-verbal position clearly predominates in the corpus. The most common paradigms in temporal clauses are the present perfective\is{aspect!perfective} and the simple present.\is{simple present} The present perfective\is{aspect!perfective} is used when the state-of-affairs it describes is construed as completed before the one expressed in the matrix clause takes place (\ref{exBoPRSPFVsuccession}). With \isi{inchoative verbs} this typically gives a stative reading (\ref{exBoPRSPFVinchoative}). The simple present,\is{simple present} on the other hand, is commonly used with a continuous, simultaneous reading (\ref{exBoPRS}).
\begin{exe}
\ex \label{exBoPRSPFVsuccession}
\gll \textbf{bo} \textbf{a}-\textbf{mal}-\textbf{ile} ɪ-m-bombo j-aake a-a-sook-ile=po, a-a-bʊʊk-ile kʊ-my-ake\\
as 1-finish-\textsc{pfv} \textsc{aug}-9-work 9-\textsc{poss.sg} 1-\textsc{pst}-leave-\textsc{pfv}=16 1-\textsc{pst}-go-\textsc{pfv} 17-4-\textsc{poss.sg}\\
\glt \lq When he finished his work, he left and went home.' [Hare and Hippo]
\ex \label{exBoPRSPFVinchoative}
\gll po \textbf{bo} ɪ-li-ndʊ lɪ-la \textbf{lɪ}-\textbf{gon}-\textbf{ile} itolo ba-aly-eg-ile a-m-ɪɪsi ga-la\\
then as \textsc{aug}-5-monster 5-\textsc{dist} 5-rest-\textsc{pfv} just 2-\textsc{pst}-take-\textsc{pfv} \textsc{aug}-6-water 6-\textsc{dist}\\
\glt \lq When that monster was deeply asleep they took that water.' [Monster with guitar]
\ex \label{exBoPRS}
\gll po \textbf{bo} \textbf{i}-\textbf{kw}-\textbf{ɪmb}-\textbf{a} po jɪ-lɪnkʊ-tup-a\\
then as 1-\textsc{prs}-sing-\textsc{fv} then 9-\textsc{narr}-become\_fat-\textsc{fv}\\
\glt \lq As it [child] was singing, it [snake] became fat.' [Snake and children]
\end{exe}

At first glance it may seem that these clauses introduced by \textit{bo} express two conflicting kinds of temporal ordering between the matrix clause and the temporal clause: posteriority (\ref{exBoPRSPFVsuccession}) vs. simultaneity (\ref{exBoPRSPFVinchoative}, \ref{exBoPRS}). However, in both cases the time interval for which the matrix clause eventuality is asserted can be understood as concomitant with the state-of-affairs expressed in the adverbial clause. As discussed in \sectref{PresentPerfective}, perfective aspect\is{aspect!perfective} in Nyakyusa introduces a post-Nucleus\is{phase!Nucleus phase} perspective. With a perfective\is{aspect!perfective} non-inchoative verb in the temporal clause, as in (\ref{exBoPRSPFVsuccession}), the main clause eventuality is thus constrained to the post-time of the subordinate one. With an inchoative verb\is{inchoative verbs} in the temporal clause (\ref{exBoPRSPFVinchoative}), the main clause eventuality is concomitant with the resultant state. In the same fashion, adverbial clauses of anteriority feature either the persistive\is{aspect!persistive} (\sectref{Persistive}) in its \lq still to/not yet' reading (\ref{exBoPersistive}) or the \isi{negative} counterpart to the present perfective\is{aspect!perfective} (\ref{exBoNegPrsPfv}).
\begin{exe}
\ex \label{exBoPersistive}
\gll po jʊ-mo ʊ-jʊ a-lɪnkʊ-mmw-eg-a ʊ-mw-ana \textbf{bo} \textbf{a}-\textbf{kaalɪ} \textbf{ʊ}-\textbf{kʊ}-\textbf{piny}-\textbf{a} ɪ-ly-ʊndʊ\\
then 1-one \textsc{aug}-\textsc{prox.1} 1-\textsc{narr}-1-take-\textsc{fv} \textsc{aug}-1-child as 1-\textsc{pers} \textsc{aug}-15-bind-\textsc{fv} \textsc{aug}-5-thatching\_grass\\
\sn \lq One of them took her child before binding the grass.' [Throw away the child]
\ex \label{exBoNegPrsPfv}
\gll \textbf{bo} \textbf{a}-\textbf{ka}-\textbf{fik}-\textbf{a} pa-la a-lɪnkw-ag-an-il-a n=ʊ-mu-ndʊ pa-n-jɪla\\
as 1-\textsc{neg}-arrive-\textsc{fv} 16-\textsc{dist} 1-\textsc{narr}-find-\textsc{recp}-\textsc{appl}-\textsc{fv} \textsc{com}=\textsc{aug}-1-person 16-9-path\\
\glt \lq Before she arrived there, she met a person on the way.' [Throw away the child]
\end{exe}

\largerpage
(\ref{exTemporalClauseCopula}) illustrates the use of the \isi{copula} in a temporal clause. (\ref{exTemporalClauseNegative}) is an example featuring a negated verb,\is{negative} while (\ref{exTemporalClauseFuture}) illustrates reference to a future state-of-affairs.
\begin{exe}
\ex  \label{exTemporalClauseCopula} \gll \textbf{bo} \textbf{tʊ}-\textbf{lɪ} ba-niini tw-a-bʊʊk-ile kʊ-dalesalama\\
as \textsc{1pl}-\textsc{cop} 2-little \textsc{1pl}-\textsc{pst}-go-\textsc{pfv} 17-D.\\
\glt \lq When we were little we went to Dar es Salaam.'  [ET]
\ex \label{exTemporalClauseNegative}
\gll po \textbf{bo} \textbf{ba}-\textbf{ti}-\textbf{kʊ}-\textbf{j}-\textbf{aag}-\textbf{a} po ɪ-ly-ebe ly-al-iis-ile ʊ-kw-and-a ʊ-kʊ-kol-a ʊ-tw-ana tw-a n-gʊkʊ\\
then as 2-\textsc{neg}-\textsc{prs}-9-find-\textsc{fv} then \textsc{aug}-5-crow 5-\textsc{pst}-come-\textsc{pfv} \textsc{aug}-15-begin-\textsc{fv} \textsc{aug}-15-grasp-\textsc{fv} \textsc{aug}-12-child 12-\textsc{assoc} 10-chicken\\
\glt \lq As/while they were not finding it [needle], Crow came, beginning to catch the little children of the chickens.' [Chickens and Crow]
\ex \label{exTemporalClauseFuture} \gll \textbf{bo} \textbf{ga}-\textbf{kɪnd}-\textbf{ile} a-ma-sikʊ a-ma-longo ma-na a-ka-aja a-ka-a Ninibe ki-kʊ-pyut-igw-aga\\
as 6-pass-\textsc{pfv} \textsc{aug}-6-day \textsc{aug}-6-ten 6-four \textsc{aug}-12-village \textsc{aug}-12-\textsc{assoc} N. 12-\textsc{mod.fut}-ruin-\textsc{pass}-\textsc{mod.fut}\\
\glt \lq Yet forty days, and Nineveh shall be overthrown.' (Jonah 3:4)
\end{exe}

A less frequent kind of temporal clauses is introduced by the \isi{locative} noun class 17 proximal demonstrative \textit{kʊno}. All occurrences of this in the corpus are found in post-verbal positions. Again, the present tense paradigms are used, with temporal reference stemming from the matrix clause. Thus, in (\ref{exKunoAdverbialile}) the present perfective\is{aspect!perfective} with inchoative\is{inchoative verbs} \textit{kola} \lq grasp, hold' induces a stative reading, which is construed as concomitant with the eventuality of returning. Similarly in (\ref{exKunoAdverbialPFV}) the act of singing, in the simple present,\is{simple present} takes place at the same time as the act of going home.
\begin{exe}
\ex \label{exKunoAdverbialile} \gll a-ba-ndʊ ba-la ba-lɪnkʊ-buj-a kʊ-ka-aja, \textbf{kʊ}-\textbf{no} \textbf{ba}-\textbf{kol}-\textbf{ile} a-ma-boko ga-abo m-mi-tʊ\\
\textsc{aug}-2-person 2-\textsc{dist} 2-\textsc{narr}-return-\textsc{fv} 17-12-homestead 17-\textsc{prox} 2-hold-\textsc{pfv} \textsc{aug}-6-hand 6-\textsc{poss.pl} 18-4-head\\
\glt `‎‎Those people returned home with their hands on their heads.' [Thieving monkeys]
\ex \label{exKunoAdverbialPFV} \gll ba-lɪnkʊ-bʊʊk-a na=gyo kʊ-my-abo n=ʊ-lʊ-saalo ʊ-lʊ-nywamu fiijo \textbf{kʊ}-\textbf{no} \textbf{bi}-\textbf{kw}-\textbf{ɪmb}-\textbf{a} ɪɪ-nyɪmbo\\
2-\textsc{narr}-go-\textsc{fv} \textsc{com}=\textsc{ref.4} 17-4-\textsc{poss.pl} \textsc{com}=\textsc{aug}-11-happiness \textsc{aug}-11-big \textsc{intens} 17-\textsc{prox} 2-\textsc{prs}-sing-\textsc{fv} \textsc{aug}-song(10)\\
\glt `They went home with them [monkey's tails], very happy and singing songs.' [Thieving monkeys]
\end{exe}

A similar pattern of usage is found with adverbial clauses introduced by \textit{lɪnga} \lq if, when'. As the English translation suggests, these can receive a temporal as well as a \isi{conditional} reading. Again, one of the present paradigms is used, with temporal reference stemming from the main clause:

\begin{exe}
\ex \label{exLingaPSTGeneric}
\gll po \textbf{lɪnga} \textbf{ba}-\textbf{n}-\textbf{swɪl}-\textbf{ile} po ʊ-mw-ana a-a-j-aga na=a-ma-ka fiijo\\
then if/when 2-1-feed-\textsc{pfv} then \textsc{aug}-1-child 1-\textsc{pst}-be(come)-\textsc{ipfv} \textsc{com}=\textsc{aug}-6-strength \textsc{intens}\\
\glt \lq When they had fed it, the child would become very strong.' [Clothing long ago]
\ex \gll kangɪ \textbf{lɪnga} \textbf{kʊ}-\textbf{ga}-\textbf{keet}-\textbf{a} ma-tiitʊ\\
again if/when \textsc{2sg.prs}-6-watch-\textsc{fv} 6-black\\
\glt \lq When you look at it [water], it is black' [Selfishness kills]
\end{exe}

\begin{exe}
\ex \gll \textbf{lɪnga} \textbf{ga}-\textbf{fik}-\textbf{ile} a-ma-jolo ʊ-ka-suluk-ege paa-si\\
if/when 6--arrive-\textsc{pfv} \textsc{aug}-6-evening \textsc{2sg}-\textsc{itv}-descend-\textsc{ipfv.subj} 16-below\\
\glt \lq When the evening has come, then you can climb down.' [Mfyage turns into a lion]
\end{exe}

However, in the \isi{conditional} reading of \textit{lɪnga}, overt marking of past tense\is{tense!past} is possible, if the state-of-affairs in the \isi{conditional} clause is overtly construed in a past reference frame.\is{tense!past}

\begin{exe}
\ex \gll \textbf{lɪnga} ɪ-fi-ndʊ \textbf{fy}-\textbf{a}-\textbf{lɪ} paa-meesa ba-a-l-iile\\
if/when \textsc{aug}-8-food 8-\textsc{pst}-\textsc{cop} 16-table(9)(<SWA) 2-\textsc{pst}-eat-\textsc{pfv}\\
\glt \lq If the food was on the table [on that occasion] then they ate it.' [ET]
\ex \gll lɪlɪno \textbf{lɪnga} ʊ-jo ʊ-gwise gw-a n̩-dʊme \textbf{a}-\textbf{ka}-\textbf{alɪ}-\textbf{m̩}-\textbf{bonwile} ʊ-n-kasi gw-a mw-anaake, po i-kʊ-bomb-aga bʊle$\sim$bʊle?\\
now/today if/when \textsc{aug}-\textsc{ref.1} \textsc{aug}-his\_father(1) 1-\textsc{assoc} 1-husband 1-\textsc{neg}-\textsc{pst}-1-pay\_off.\textsc{pfv} \textsc{aug}-1-wife 1-\textsc{assoc} 1-his\_child then 1-\textsc{mod.fut}-do-\textsc{mod.fut} \textsc{redupl}$\sim$how\\
\glt `Nowadays, if the father of the husband has not paid the wife of his child, then what will she do?' [Should she save a life\ldots]
\end{exe}

Lastly, in adverbial clauses referring to iteratives, habituals\is{aspect!habitual} or generics,\is{aspect!generic} the temporal reference can be understood as relative to the singular events that make up these repeated occurrences. These types of adverbial clauses are often introduced by \textit{kʊkʊtɪ} \lq every' (\ref{exKukutiClause1}, \ref{exKukutiClause2}). (\ref{exAdverbialGenericAnteriority}) illustrates a temporal clause of anteriority relating to a past generic\is{aspect!generic} proposition. See also (\ref{exLingaPSTGeneric}) above.
\begin{exe}
\ex \label{exKukutiClause1}\gll \textbf{kʊkʊtɪ} \textbf{m}-\textbf{bomb}-\textbf{ile}=\textbf{po} panandɪ itolo ɪ-m-bombo n-gw-ag-a n-gateele\\
every \textsc{1sg}-work-\textsc{pfv}=\textsc{part} a\_little just \textsc{aug}-9-work \textsc{1sg}-\textsc{prs}-find-\textsc{fv} \textsc{1sg}-be(come)\_tired.\textsc{pfv}\\
\glt \lq Every time after working just a bit I find myself tired.' [ET]
\ex \label{exKukutiClause2} \gll \textbf{kʊkʊtɪ} \textbf{a}-\textbf{sulwike} paa-si a-a-j-aga ɪ-n-galamu\\
every 1-descend.\textsc{pfv} 16-down 1-\textsc{pst}-be(come)-\textsc{ipfv} \textsc{aug}-9-lion\\
\glt \lq Every time she went down to the ground she would become a lion.' [Mfyage turns into a lion]

\ex\label{exAdverbialGenericAnteriority}\gll \textbf{bo} \textbf{ba}-\textbf{kaalɪ} \textbf{ʊ}-\textbf{kw}-\textbf{and}-\textbf{a} ʊ-kʊ-mog-a ba-a-fwal-aga ɪ-my-enda ɪ-my-elu, pamo a-ma-golole a-m-eelu
\\
as 2-\textsc{pers} \textsc{aug}-15-begin-\textsc{fv} \textsc{aug}-15-dance-\textsc{fv} 2-\textsc{pst}-dress/wear-\textsc{ipfv} \textsc{aug}-4-cloth \textsc{aug}-4-white or \textsc{aug}-6-sheet \textsc{aug}-6-white\\
\glt \lq Before starting to dance, they would put on white clothes, or white sheets.' [Custom of dancing]
\end{exe}
\is{subordinate clauses!temporal clause|)}\is{subordinate clauses!conditional protasis|)}
\subsubsection{Complements of PCU verbs}\label{PRSnonpstPCU}\is{subordinate clauses!complement clause|(}
With verbs of perception, cognition and utterance (PCU verbs; terminology following \citealt{GivonT2001}), a pattern parallel to that of temporal adverbial clauses is found. In the following, verbs of perception and cognition will be discussed before turning to verbs of utterance, specifically information verbs.

The following examples illustrate the use of present tense paradigms in the clausal complements of perception and cognition verbs, with temporal reference relative to the state-of-affairs depicted in the matrix clause. Thus, the \isi{simple present} in (\ref{exPRSparadigmsInComplementPerception1}) and in the first complement clause of (\ref{exPRSparadigmsInComplementCognition1}) depicts a process unfolding at the same time as its perception. The same holds for the periphrastic present progressive\is{aspect!progressive} in the second complement clause of (\ref{exPRSparadigmsInComplementCognition2}). The present perfective\is{aspect!perfective} is used with \isi{inchoative verbs} in the second and third complement clauses of (\ref{exPRSparadigmsInComplementCognition1}) and accordingly gives a stative reading. Use of the present perfective\is{aspect!perfective} with non-inchoative verbs is illustrated in (\ref{exPRSparadigmsInComplementCognition2}, \ref{exPRSparadigmsInComplementPerception2}). Accordingly it denotes a completed eventuality whose result is perceived.
\begin{exe} %sortierung ist durchdacht, damit es sich auch im text gut fassen lässt.
\ex \label{exPRSparadigmsInComplementPerception1} \gll Sambʊka a-lɪnkʊ-kalal-a mu-n-dumbula ʊkʊtɪ \textbf{i}-\textbf{kʊ}-\textbf{n̩}-\textbf{dek}-\textbf{a} mw-ene\\
S. 1-\textsc{narr}-be(come)\_angry-\textsc{fv} 18-9-heart \textsc{comp} 1-\textsc{prs}-1-let-\textsc{fv} 1-only\\
\glt `Sambuka became angry in her heart that she [Asia] was leaving her alone.' [Juma, Asia and Sambuka]
\ex \label{exPRSparadigmsInComplementCognition1}\gll po a-lɪnkw-ag-a kajamba \textbf{i}-\textbf{kʊ}-\textbf{sook}-\textbf{a}, \textbf{a}-\textbf{fwele} ɪ-kɪ-tili, n=ii-koti ly-ake, \textbf{a}-\textbf{kol}-\textbf{ile} n=ɪ-n-gili j-aa kw-end-el-a, ɪ-kɪ-ngoti\\
then 1-\textsc{narr}-find-\textsc{fv} tortoise(1) 1-\textsc{prs}-leave-\textsc{fv} 1-dress/wear.\textsc{pfv} \textsc{aug}-7-hat \textsc{com}=5-coat(<SWA) 5-\textsc{poss.sg} 1-grasp-\textsc{pfv} \textsc{com}=\textsc{aug}-9-stick 9-\textsc{assoc} 15-walk/travel-\textsc{appl}-\textsc{fv} \textsc{aug}-7-walking\_stick\\
\glt `He [Monkey] found Tortoise coming out, wearing a hat and his coat, holding a stick for walking,  a walking stick.' [Monkey and Tortoise]
\ex \label{exPRSparadigmsInComplementCognition2} \gll bo a-fik-ile pa-la a-lɪnkʊ-sy-ag-a ɪ-n-gambɪlɪ i-haano \textbf{si}-\textbf{fi}-\textbf{fungamiile} ɪ-fi-lombe mu-n̩-gʊnda gw-ake, \textbf{si}-\textbf{lɪ} \textbf{pa}-\textbf{kʊ}-\textbf{ly}-\textbf{a}\\
as 1-arrive-\textsc{pfv} 16-\textsc{dist} 1-\textsc{narr}-10-find-\textsc{fv} \textsc{aug}-10-monkey 10-five 10-8-put\_pressure\_on.\textsc{pfv} \textsc{aug}-8-maize 18-3-field 3-\textsc{poss.sg} 10-\textsc{cop} 16-15-eat-\textsc{fv}\\
\glt `When he arrived there, he found five monkeys had devastated the maize in his field and were eating.' [Thieving monkeys]
\ex \label{exPRSparadigmsInComplementPerception2}%das ist klare perception
\gll ii-sikʊ lɪ-mo kalʊlʊ bo i-kʊ-jaat-a a-lɪnkʊ-fi-bon-a ɪ-fi-lombe mu-n̩-gʊnda \textbf{fi}-\textbf{bɪfiifwe}\\
5-day 5-one hare(1) as 1-\textsc{prs}-walk-\textsc{fv} 1-\textsc{narr}-8-see-\textsc{fv} \textsc{aug}-8-maize 18-3-field 8-ripen.\textsc{pfv}\\
\glt \lq One day Hare, while he was taking a walk, saw that the maize in the field was ripe.' [Saliki and Hare]
\end{exe}

Sometimes \textit{bo} \lq as' (see \sectref{TemporalConditionalClauses} above) follows a verb of perception:

\begin{exe}
\ex \label{exPRSparadigmsInComplementBo} \gll po a-a-pɪliike bo kʊ-kʊ-lɪl-a ``káa!''\\
then 1-\textsc{pst}-hear.\textsc{pfv} as 17-\textsc{prs}-sound-\textsc{fv} \phantom{\lq\lq}of\_sickle\_swinging\\
\glt `Then he heard it as there was a sound ``Káa!'' [of a sickle swinging]' [Wage of the thieves]
\end{exe}

Information verbs also follow the now familiar pattern. They differ, however, in their preferences as to the syntactic\is{syntax} structure of their complement. The complement most commonly consists of a headless relative clause,\is{subordinate clauses!relative clause} as in (\ref{exSpeechReport1}). Alternatively, complementation through the associative plus infinitive of \textit{tɪ} (see \sectref{defectiveti}) is attested (\ref{exSpeechReport2}). In both cases, within the relative clause\is{subordinate clauses!relative clause} the \isi{subject marker} typically is of noun class 10. This can be understood as referring to implicit \textit{ɪɪnongwa} \lq issue(s) (9/10)', an interpretation that is strengthened by example (\ref{exSpeechReport3}). Note that again the present perfective\is{aspect!perfective} is used in the relative clause.\is{subordinate clauses!relative clause}

\begin{exe}
\ex \label{exSpeechReport1} \gll a-lɪnkʊ-ba-bʊʊl-a \textbf{ɪ}-\textbf{si} \textbf{si}-\textbf{sookiile} pa-ka-aja pa-my-ake\\
1-\textsc{narr}-2-tell-\textsc{fv} \textsc{aug}-\textsc{prox.10} 10-happen.\textsc{pfv} 16-12-homestead 16-4-\textsc{poss.sg}\\
\glt \lq He told them what had happened in his house.' [Killer woman]
\ex \label{exSpeechReport2} \gll bo ba-gomwike kʊ-malafyale gw-abo, ba-alɪ-m-pangiile ɪ-sy-\textbf{a} \textbf{kʊ}-\textbf{tɪ} ʊ-n-nuguna \textbf{a}-\textbf{fug}-\textbf{ile} ɪ-n-galamu\\
as 2-return.\textsc{pfv} 17-chief(1) 2-\textsc{poss.pl} 2-\textsc{pst}-1-tell.\textsc{pfv} \textsc{aug}-10-\textsc{assoc} 15-say \textsc{aug}-1-younger\_sibling 1-tame-\textsc{pfv} \textsc{aug}-9-lion\\
\glt `When they returned to their chief, they told him that his younger brother had tamed a lion.' [Chief Kapyungu]
\ex \label{exSpeechReport3}
\gll ʊ-n-kʊlʊmba gw-abo ɪɪ-sofu jɪ-lɪnkw-igʊl-a ʊ-lʊ-komaano n=ʊ-kʊ-fi-bʊʊl-a ɪ-fi-nyamaana \textbf{ɪɪ}-\textbf{nongwa} \textbf{ɪ}-\textbf{jɪ} \textbf{jɪ}-\textbf{m}-\textbf{pel}-\textbf{ile} ʊ-kʊ-koolel-a ʊ-lʊ-komaano\\
\textsc{aug}-1-older 1-\textsc{poss.pl} \textsc{aug}-elephant(9) 9-\textsc{narr}-open-\textsc{fv} \textsc{aug}-11-meeting \textsc{com}=\textsc{aug}-15-8-tell-\textsc{fv} \textsc{aug}-8-animal \textsc{aug}-issue(9) \textsc{aug}-\textsc{prox.9} 9-1-make-\textsc{pfv} \textsc{aug}-15-call-\textsc{fv} \textsc{aug}-11-meeting\\
\sn \lq Their eldest, Elephant, opened the meeting and told the animals the reason that had made him call the meeting.' [Hare and Chameleon]
\end{exe}

Similarly to what was found for temporal clauses, in complements of PCU verbs which relate to an iterative, habitual\is{aspect!habitual} or generic\is{aspect!generic} proposition, the temporal perspective can be understood as relative to the individual sub-events:

\begin{exe}
\ex \gll po tw-a-many-aga ʊkʊtɪ kʊ-la ʊ-bw-ite \textbf{bʊ}-\textbf{kol}-\textbf{eene} ʊ-bw-a kʊ-mog-a\\
then \textsc{1pl}-\textsc{pst}-know-\textsc{ipfv} \textsc{comp} 17-\textsc{dist} \textsc{aug}-14-fight 14-hold-\textsc{recp.pfv} \textsc{aug}-14-\textsc{assoc} 15-dance-\textsc{fv}\\
\glt \lq Then we would know that a dancing competition was being held.' [Custom of dancing]
\end{exe}
\is{subordinate clauses!complement clause|)}
\subsubsection{Relative clauses}\label{PRSnonPSTRelativeClauses}\is{subordinate clauses!relative clause|(}\is{tense!past|(}
Relative clauses show a more complex picture. Within past narrative discourse, both present tense verbs and past tense verbs are encountered in relative clauses. In terms of temporal reference, the present tense is found both with its default meaning and with past time reference (present-in-the-past). Likewise, the past tense is encountered both as a concomitant past and as a past-in-the-past.

In the cases discussed in \sectref{NarrativePresent}--\ref{PRSnonpstPCU} above -- that is, relative clauses subordinate to a narrative present,\is{narrative present} embedded in a temporal clause\is{subordinate clauses!temporal clause} or modifying the perceived, cognized or uttered proposition of PCU verbs\is{subordinate clauses!complement clause} -- the use of a present tense paradigm is predictable through \isi{syntax} and semantics. Further straightforward cases are those relative clauses referring to timeless statements or past iteratives, habituals\is{aspect!habitual} and generics.\is{aspect!generic} These will be discussed further below. Excluding these predictable cases, there remains a number of alternations between present and past tense\is{tense!past} which are governed by pragmatic considerations, namely by the textual\is{component of meaning!textual} and expressive\is{component of meaning!expressive} components of meaning. The past tense, however, clearly predominates and is to be considered the default.

\is{information status|(}For a first approximation of the alternations of tenses, it is worth considering the activation status of the information given in the relative clauses in question; see \sectref{ToolsNarrativeAnalysis} on the categories of activation status.  All relative clauses that feature a present tense predicate with past time reference contain either old information, that is, both their head and the proposition they contain can be classified as either discourse-old/hearer-old, or information that is strongly inferable.

It follows from this generalization that all relative clauses containing brand new information relating to past time (the story-now) feature a past tense verb. A prototypical case is that of explicative relative clauses in the orientation section:\is{section!orientation}

\begin{exe}
\ex \label{exRelativeClausePSTorientation}%bsp pst relative clause orientation, new information
\gll ijolo n-k-iisʊ ky-a Tʊkʊjʊ, ba-a-li=ko a-ba-ndʊ \textbf{a}-\textbf{ba} \textbf{ba}-\textbf{a}-\textbf{lɪm}-\textbf{aga} ɪ-mi-gʊnda gy-abo kɪfuki na=a-ma-tengele\\
old\_times 18-7-land 7-\textsc{assoc} T. 2-\textsc{pst}-\textsc{cop}=17 \textsc{aug}-2-person \textsc{aug}-\textsc{prox.2} 2-\textsc{pst}-farm-\textsc{ipfv} \textsc{aug}-4-farm 4-\textsc{poss.pl} near \textsc{com}=\textsc{aug}-6-bush\\
\glt \lq Long ago in Tukuyu there were people who were farming their fields near the forest.' [Thieving monkeys] 
\end{exe}

\largerpage
This association between information status and tense marking is a one-way conditional. While brand new information invariably comes with the past tense, discourse-old/hearer-old or inferable propositions can also receive past tense marking:

\begin{exe}
\ex \label{exRelativeClausePSTalsoOldInformation}
Context: A monster had caught and killed a child.\\
\gll ba-lɪnkʊ-pɪlɪkɪsy-a a-ba-kamu ba-a mw-ana \textbf{jʊ}-\textbf{la} \textbf{ly}-\textbf{alɪ}-\textbf{n}-\textbf{kol}-\textbf{ile} ɪ-li-ndʊ\\
2-\textsc{narr}-listen-\textsc{fv} \textsc{aug}-2-relative 2-\textsc{assoc} 1-child 1-\textsc{dist} 5-\textsc{pst}-1-grasp-\textsc{pfv} \textsc{aug}-5-monster\\
\glt \lq The relatives of that child that the monster had caught listened.'
[Monster with Guitar]
\end{exe}
\is{information status|)}

Before taking a closer look at these tense alternations, the contexts which allow for them need to be narrowed down further. In semantic terms, what allows for the use of the present tense as a present-in-the-past is the lack of tense specification which goes together with the lack of overt morphological tense marking. \citet{FleischmanS1990} calls this the \lq\lq zero interpretation''. Consequently, if a relative clause depicts a state or an unfolding process that is situated prior to the state-of-affairs of the matrix clause, an overt past tense is required.\footnote{As \citet[84]{SmithC1997} notes, both stat(iv)es and progressives\is{aspect!progressive} depict a state-of-affairs that is stable and extends in time. Another logically possible case, that would be predicted to require a past tense, but which is not attested in the corpus, is that of habitual\is{aspect!habitual} or generic\is{aspect!generic} propositions relating to a previous reference frame, as in \textit{He carved with the same adze that his father had used to carve with}.} This is illustrated in (\ref{exRelPSTinPSTinchoative}--\ref{exRelPSTinPSTCop}) for an inchoative verb\is{inchoative verbs} with perfective aspect\is{aspect!perfective}, a non-inchoative verb with imperfective aspect\is{aspect!imperfective}, and the \isi{copula}, respectively.

\begin{exe}
\ex \label{exRelPSTinPSTinchoative}
Context: A dog had hunted and saved some leftover meat, which in the meantime has been eaten by another dog.\\
\gll ɪɪ-nine j-oope jɪ-lɪnkʊ-kong-a muu-nyuma pa-kɪ-syanjʊ \textbf{a}-\textbf{pa} \textbf{j}-\textbf{aa}-\textbf{syele} ɪɪ-nyama\\
\textsc{aug}-companion(9) 9-also 9-\textsc{narr}-follow-\textsc{fv} 18-back(9) 16-7-thicket \textsc{aug}-\textsc{prox.16} 9-\textsc{pst}-remain.\textsc{pfv} \textsc{aug}-meat(9)\\
\glt \lq The other dog followed behind to the thicket where the meat had been left.' [Dogs laughed at each other] 

\ex \label{exRelPSTinPSTipfv}
Context: Sambuka has deceived Juma by saying that her fiancé does not love her.\\
\gll Juma a-lɪnkʊ-swig-a fiijo kʊ-ma-syʊ a-ga \textbf{a}-\textbf{a}-\textbf{job}-\textbf{aga} Sambʊka\\
J. 1-\textsc{narr}-wonder-\textsc{fv} \textsc{intens} 17-6-word \textsc{aug}-\textsc{prox.6} 1-\textsc{pst}-speak-\textsc{ipfv} S.\\
\glt \lq Juma wondered much about the words that Sambuka had been saying.' [Juma, Asia and Sambuka]

\ex \label{exRelPSTinPSTCop}
Context: The late chief has split up his chiefdom between his two heirs.\\
\gll bo ka-kɪnd-ile a-ka-balɪlo ka-nandɪ ʊ-n-kʊlʊmba a-lɪnkʊ-lond-a ʊ-bʊ-nyafyale bo \textbf{ʊ}-\textbf{bʊ} \textbf{a}-\textbf{a}-\textbf{lɪ} na=bo ʊ-gwise\\
as 12-pass-\textsc{pfv} \textsc{aug}-12-time 12-little \textsc{aug}-1-older 1-\textsc{narr}-want-\textsc{fv} \textsc{aug}-14-chiefdom as \textsc{aug}-\textsc{prox.14} 1-\textsc{pst}-\textsc{cop} \textsc{com}=\textsc{prox.14} \textsc{aug}-his\_father\\
\glt \lq After a little while the elder brother wanted a chiefdom just like the one his father had had.' [Chief Kapyungu] % cop
\end{exe}

So far it has been established that the alternations between past and present tense are found in those relative clauses that act on old or inferred information and which do not require a preceding reference frame. A look at the relative frequency again reveals a clear preference for the past tense, which dominates by a factor of approximately 2.5 in the corpus. The present tense is thus to be considered the \lq\lq deviation from [the] default parameter setting'' \citep[64f]{HaspelmathM2006} for which specific pragmatic functions must be assumed.\footnote{\citet{HaspelmathM2006} discusses the numerous possible uses of the more traditional term \textit{markedness} and suggests abandoning it altogether in favour of a more specific terminology. In the case discussed here, two uses of markedness are in conflict: \lq\lq markedness as deviation from the default parameter setting'' vs. \lq\lq markedness as overt coding''.}

(\ref{exInvadersRelAlternations}) is a representative example of the alternation between present and past tense with a concomitant state. Perfective aspect\is{aspect!perfective} is here employed with an inchoative verb,\is{inchoative verbs} yielding a stative reading. In (\ref{exInvadersRelPRS}) the present perfective is used,\is{aspect!perfective} but in (\ref{exInvadersRelPST}) -- a few clauses later in same text and depicting essentially the same state-of-affairs -- the past perfective\is{aspect!perfective} is used. Two things are apparently going on in this example. First, the act of hiding in (\ref{exInvadersRelPRS}) immediately precedes this eventuality.\footnote{\textit{biibɪɪliile} may thus be paraphrased as \lq have just hidden and are in hiding'. See \sectref{PresentPerfective} for a more detailed discussion of the semantic interplay between perfective aspect\is{aspect!perfective} and inchoative verbs.\is{inchoative verbs}} Second, (\ref{exInvadersRelPRS}) depicts a moment of high tension: will the invaders notice the hidden locals? Note how the narrator employs repetition so as not to let this moment go unnoticed. The relative clause with the default past tense in (\ref{exInvadersRelPST}), however, merely serves to identify the patient of the stabbing.

\begin{exe}
\ex \label{exInvadersRelAlternations}\begin{xlist}
\ex Context: Invaders have come to a certain land. Three of the locals have run off and hidden below the straw on the fields.\label{exInvadersRelPRS}\\
\gll po leelo bo b-iibɪɪliile ba-lɪnkw-is-a a-ba-lʊgʊ. po ba-lɪnkw-end-a pa-mwanya pa-my-abo. ba-lɪnkw-end-a pa-my-abo pa-ba-ndʊ \textbf{a}-\textbf{ba} \textbf{b}-\textbf{iibɪɪliile} paa-si\\ %fall von foregrounding
then now/but as 2-hide\_at.\textsc{pfv} 2-\textsc{narr}-come-\textsc{fv} \textsc{aug}-2-enemy then 2-\textsc{narr}-travel/walk-\textsc{fv} 16-high 14-4-\textsc{poss.pl} 2-\textsc{narr}-walk/travel-\textsc{fv} 16-4-\textsc{poss.pl} 16-2-person \textsc{aug}-\textsc{prox.2} 2-hide\_at.\textsc{pfv} 16-below\\
\glt \lq ‎‎So when they had hidden, the invaders came. They walked on top of them. They walked on top of the people that were hidden below.'
\ex Context: One of the invaders has heard one of the hidden locals speaks.\\
\label{exInvadersRelPST} \gll a-a-las-ile paa-si. a-alɪ-n̩-das-ile \textbf{jʊ}-\textbf{la} \textbf{a}-\textbf{al}-\textbf{iibɪɪliile} paa-si\\
1-\textsc{pst}-stab-\textsc{pfv} 16-below 1-\textsc{pst}-1-stab-\textsc{pfv} 1-\textsc{dist} 1-\textsc{pst}-hide\_at.\textsc{pfv} 16-below\\
\glt \lq He drove the spear downwards. He stabbed that one that was hidden below.' [Invaders]
\end{xlist}
\end{exe}

The preceding examples feature perfective aspect\is{aspect!perfective} with an inchoative verb.\is{inchoative verbs} A specifically intriguing case is the alternation between the past and present perfective\is{aspect!perfective} with non-inchoative verbs. Independent of tense, perfective\is{aspect!perfective} aspect with these verbs yields a posterior vantage point (see \sectref{PresentPerfective}). There is thus the choice of which temporal perspective\is{tense} to apply to the preceding event (\sectref{Tense}). Example (\ref{exRelPFVDogs1}) illustrates the use of the present perfective vis-à-vis the past perfective with a non-inchoative verb. The eating of the remaining meat, construed with the present perfective\is{aspect!perfective} (\ref{exRelPFVDogs1sentence1}), takes place in the episode preceding the two dogs' encounter; while the act of hunting, which is construed in the past perfective,\is{aspect!perfective} is more remote in temporal and textual\is{component of meaning!textual} terms (\ref{exRelPFVDogs1sentence3}). Further, it is the act of eating that allows the story's central conflict to develop. Lastly, in (\ref{exRelPFVDogs1sentence3}) the \isi{negative} counterpart to the present perfective\is{aspect!perfective} is employed, which as an evaluative device works on the expressive component:\is{component of meaning!expressive}  it is the second dog's imprudence that will have fatal consequences and that constitutes the story's theme. As \citet[159]{FleischmanS1990} points out, \isi{negative} predicates in narratives are evaluative, as they entail an unrealized alternative scenario. 

\begin{exe}
\ex \label{exRelPFVDogs1}Context: A dog has hunted and left some meat uneaten, which in the meantime has been eaten by another dog.
\begin{xlist}
\ex 
\label{exRelPFVDogs1sentence1}
\gll jɪ-lɪnkw-ag-an-il-a n=ɪ-m-bwa \textbf{ɪ}-\textbf{jɪ} \textbf{jɪ}-\textbf{l}-\textbf{iile} ɪɪ-nyama\\
9-\textsc{narr}-find-\textsc{recp}-\textsc{appl}-\textsc{fv} \textsc{com}=\textsc{aug}-9-dog \textsc{aug}-\textsc{prox.9} 9-eat-\textsc{pfv} \textsc{aug}-meat(9)\\
\glt \lq He met the dog that had eaten the meat.'
\ex\gll jɪ-lɪnkʊ-jɪ-laalʊʊsy-a jɪ-lɪnkʊ-tɪ, \textup{\lq\lq}mw-inangʊ, ʊ-sumwike kʊʊgʊ?\textup{''}\\
9-\textsc{narr}-9-ask-\textsc{fv} 9-\textsc{narr}-say \phantom{\lq\lq}1-my\_companion \textsc{2sg}-depart.\textsc{pfv} where\\
\glt \lq He [dog who has eaten the meet] asked \lq\lq My friend, where are you going?''{}'

\ex\label{exRelPFVDogs1sentence3}
\gll ɪ-m-bwa \textbf{ɪ}-\textbf{jɪ} \textbf{j}-\textbf{aa}-\textbf{fwɪm}-\textbf{ile} jɪ-lɪnkʊ-tɪ \textup{\lq\lq}n-sumwike kʊ-kʊ-malɪɪsy-a ɪɪ-nyama j-angʊ\textup{''}\\
\textsc{aug}-9-dog \textsc{aug}-\textsc{prox.9} 9-\textsc{pst}-hunt-\textsc{pfv} 9-\textsc{narr}-say \phantom{\lq\lq}\textsc{1sg}-depart.\textsc{pfv} 17-15-end-\textsc{fv} \textsc{aug}-meat(9) 9-\textsc{poss.1sg}\\
\glt \lq The dog that had hunted said \lq\lq I am going to finish my meat.''{}'

\ex \label{exRelPFVDogs1sentence4}
\gll ɪ-m-bwa \textbf{ɪ}-\textbf{jɪ} \textbf{jɪ}-\textbf{ka}-\textbf{fwɪm}-\textbf{a}=\textbf{po} jɪ-lɪnkʊ-tɪ \textup{\lq\lq}ɪɪ-nyama ɪ-jɪ gw-a-syesye n-d-iile ʊ-ne\textup{''}\\
\textsc{aug}-9-dog \textsc{aug}-\textsc{prox.9} 9-\textsc{neg}-hunt-\textsc{fv}=\textsc{part} 9-\textsc{narr}-say \phantom{\lq\lq}\textsc{aug}-meat(9) \textsc{aug}-\textsc{prox.9} \textsc{2sg}-\textsc{pst}-remain.\textsc{caus.pfv} \textsc{1sg}-eat-\textsc{pfv} \textsc{aug}-\textsc{1sg}\\
\glt \lq The dog that had not hunted said \lq\lq The meat you left over, I ate it.''{}' [Dogs laughed at each other]
\end{xlist}
\end{exe}

Immediateness as well as evaluation also appear to be relevant in the following example:%mal sehen, ob wer sagt, ich solle da mehr zu sagen

\begin{exe}
\ex Context: A woman has cooked chicken for her guests. While she was outside fetching water, a thieving woman stole most of the food.\\
\gll Ngateele a-al-iinogwine fiijo kangɪ sy-alɪ-m̩-bab-ile mu-n-dumbula, paapo a-al-iib-ɪl-iigwe ɪ-fi-ndʊ \textbf{ɪ}-\textbf{fi} \textbf{a}-\textbf{ba}-\textbf{pɪɪj}-\textbf{iile} a-ba-heesya ba-ake\\
N. 1-\textsc{pst}-think.\textsc{pfv} \textsc{intens} again 10-\textsc{pst}-1-hurt-\textsc{pfv} 18-9-heart because 1-\textsc{pst}-steal-\textsc{appl}-\textsc{pass.pfv} \textsc{aug}-8-food \textsc{aug}-\textsc{prox.8} 1-2-cook-\textsc{appl.pfv} \textsc{aug}-2-foreigner 2-\textsc{poss.sg}\\
\glt \lq Ngateele thought much and it hurt her in her heart, because she was robbed of the food she had cooked for her guests.' [Thieving woman]
\end{exe}

To summarize then, in past tense \isi{narrative} discourse relative clauses that refer to the story-now and that introduce new information\is{information status} invariably feature a past tense verb. In relative clauses that contain old or inferred information, the past tense is the default. The present tense may, however, be 
employed to foreground and evaluate.\footnote{One may object that relative clauses are inherently backgrounded as a function of their syntactic status. However, as shown i.a. by \citeauthor{FleischmanS1985} (\citeyear{FleischmanS1985}; \citeyear[ch. 6]{FleischmanS1990}) among others, grounding is best understood as a cluster concept. It follows that salience within a text constitutes a spectrum or continuum rather than a binary opposition. What is more, the syntactic status of a clause is only one among various factors determining the relative salience of the state-of-affairs it describes. Information provided in a relative clause thus possesses a relative salience, which can be modulated by the choice of TMA paradigm among other means. Fleischman further notes a close conceptual connection between grounding as a means of textual\is{component of meaning!textual} organisation and evaluation as an expressive\is{component of meaning!expressive} device.} In the case of stative predicates or continuous/progressive aspect,\is{aspect!progressive} this requires the depicted state-of-affairs to be concomitant with the one expressed in the matrix clause. With perfective aspect,\is{aspect!perfective} the preceding event or entrance into a new state is typically close by and/or of direct relevance to the storyline.

Other instances of present tense relative clauses in a past tense environment are subject- and object-relative clauses of past iteratives, habituals\is{aspect!habitual} and generics.\is{aspect!generic} This patterns with what has been found for temporal clauses\is{subordinate clauses!temporal clause} and complements\is{subordinate clauses!complement clause} of PCU verbs. In (\ref{exPRSnonPSTipfv1}) there are a present tense \isi{copula} and a \isi{simple present} in the subject- and object-relative clauses, respectively. These describe states-of-affairs concomitant with the one expressed in the main clause. In (\ref{exPRSnonPSTipfv2}), the present perfective\is{aspect!perfective} in the object-relative clause construes the occurrences of cooking as taking place before the occurrences of stealing.
\begin{exe}
\ex \label{exPRSnonPSTipfv1} \gll \textbf{a}-\textbf{ba} \textbf{ba}-\textbf{lɪ} kɪfuki n=ii-tengele ba-a-tumul-aga n=ʊ-kʊ-b-ʊʊl-ɪkɪsy-a \textbf{a}-\textbf{ba} \textbf{bi}-\textbf{kʊ}-\textbf{ga}-\textbf{lond}-\textbf{a}\\
\textsc{aug}-\textsc{prox.2} 2-\textsc{cop} near \textsc{com}=5-bush 2-\textsc{pst}-cut-\textsc{ipfv} \textsc{com}=\textsc{aug}-15-2-buy-\textsc{caus.appl}-\textsc{fv} \textsc{aug}-\textsc{prox.2} 2-\textsc{prs}-6-want-\textsc{fv}\\
\glt \lq Those who were near to the bush would cut it [grass] and sell it to those who wanted it.' [Nyakyusa houses of long ago]
\ex \label{exPRSnonPSTipfv2} \gll n-kɪ-panga kɪ-la a-a-li=po ʊ-n-kiikʊlʊ ʊ-n-hɪɪji ʊ-jʊ a-a-bomb-aga ɪ-j-aa kw-ib-a ɪ-fi-ndʊ \textbf{ɪ}-\textbf{fi} \textbf{ba}-\textbf{pɪɪj}-\textbf{ile} a-ba-nine\\
18-7-village 7-\textsc{dist} 1-\textsc{pst}-\textsc{cop}=16 \textsc{aug}-1-woman \textsc{aug}-1-thief \textsc{aug}-\textsc{prox.1} 1-\textsc{pst}-work-\textsc{ipfv} \textsc{aug}-9-\textsc{assoc} 15-steal-\textsc{fv} \textsc{aug}-8-food \textsc{aug}-\textsc{prox.8} 2-cook-\textsc{pfv} \textsc{aug}-2-companion\\
\glt \lq In that village there was a thieving woman, who used to steal the food the others had cooked.' [Thieving woman]
\end{exe}
\is{tense!past|)}

Lastly, the present tense paradigms feature with their default meaning in relative clauses that depict timeless states-of-affairs. Thus, in (\ref{exPRSnonPSTRelativeTimeless1}), the referential demonstrative serves as an emphatic copulative\is{copula} (see \sectref{CopulaUse}) and in (\ref{exPRSnonPSTRelativeTimeless2}) the \isi{simple present} has a generic\is{aspect!generic} reading.

\begin{exe}
\ex \label{exPRSnonPSTRelativeTimeless1}\gll ɪɪ-sofu j-aa-jɪ-kooliile ɪ-n-galamu \textbf{ɪ}-\textbf{jɪ} \textbf{jo} j-aa kɪ-bɪlɪ\\
\textsc{aug}-elephant(9) 9-\textsc{pst}-9-call.\textsc{pfv} \textsc{aug}-9-lion \textsc{aug}-\textsc{prox.9} \textsc{ref.9} 9-\textsc{assoc} 7-two\\
\glt \lq Elephant called Lion, who is the second [in rank].' [Hare and Chameleon]
\ex \label{exPRSnonPSTRelativeTimeless2}\gll kangɪ ga-a-li=ko na=a-ma-laasi \textbf{a}-\textbf{ga} \textbf{bi}-\textbf{kʊ}-\textbf{tem}-\textbf{a} ʊ-bw-alwa \textbf{ʊ}-\textbf{bʊ} \textbf{bi}-\textbf{kʊ}-\textbf{tɪ} ʊ-bʊ-laasi\\
again 6-\textsc{pst}-\textsc{cop}=17 \textsc{com}=\textsc{aug}-6-bamboo \textsc{aug}-\textsc{prox.6} 2-\textsc{prs}-tap-\textsc{fv} \textsc{aug}-14-alcohol \textsc{aug}-\textsc{prox.14} 2-\textsc{prs}-say \textsc{aug}-14-bamboo\_beer\\
\glt \lq Also there was bamboo from which they tap this beer they call bamboo beer.' [Nyakyusa houses of long ago]
\end{exe}
\is{subordinate clauses!relative clause|)}
\is{tense!present|)}

