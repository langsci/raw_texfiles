\section{Demonstratives}\label{sec:6}



\subsection{Overview}\label{sec:6.1}


Ik’s \textsc{demonstratives} grammatically point to a referent. In the case of \textsc{nominal} demonstratives, the referent is an entity named by a noun, whereas \textsc{adverbial} demonstratives point to a scene or situation of some sort, encoded by a whole clause. The Ik nominal demonstratives are all \textsc{enclitics} that come just after their host (the referent), as in \textit{ámá=nà} ‘this person’. Because the locative adverbial demonstratives function as adverbs, they tend to come at the end of the clause they are modifying. Unlike demonstrative pronouns (see \sectref{sec:5.6}), spatial and temporal demonstratives are not nouns and never take case endings.




\subsection{Spatial demonstratives (\textsc{dem})}\label{sec:6.2}


Ik’s \textsc{spatial} demonstratives locate their referent in physical space in degrees of distance from the speaker. For singular referents, there are three degrees of distance: \textsc{proximal} (near), \textsc{medial} (relatively near/far), and \textsc{distal} (more distant). For plural referents, the language inexplicably only distinguishes between proximal and distal. The singular demonstratives are usually translated into English as ‘this’ and ‘that’ and the plural ones as ‘these’ or ‘those’. \tabref{tab:dem:spat} below presents the whole set of spatial nominal demonstratives. Notice that in their final forms (\textsc{ff}), their final vowels may be whispered or omitted altogether:


\begin{table}
\caption{Ik spatial demonstratives}
\label{tab:dem:spat}
\begin{tabularx}{\textwidth}{XXXXX}
\lsptoprule
& Singular &  & Plural & \\
\midrule
& \textsc{nf} & \textsc{ff} & \textsc{nf} & \textsc{ff}\\
Proximal & =na & =na (=n) & =ni & =ni (=n)\\
Medial & =ne & =ne (=n) &  & \\
Distal & =ke & =ke (=kᵉ) & =ki & =ki (=kⁱ)\\
\lspbottomrule
\end{tabularx}
\end{table}

Spatial demonstratives usually directly follow their referent, as in \REF{ex:dem:1}-\REF{ex:dem:2}:




\ea\label{ex:dem:1}
\gll Eakwóó   ɗa=\textbf{n}. \\
man:\textsc{cop}  this.one:\textsc{nom}=\textsc{dem.sg.prox}    \\
\glt ‘This one is a man.’ 
\z




\ea\label{ex:dem:2}
\gll Káwese   koto   ríʝá=\textbf{ke}. \\
cut:\textsc{sps}   then   forest=\textsc{dem.sg.dist}    \\
\glt ‘And then that forest over there was cut down.’ 
\z






\subsection{Temporal demonstratives (\textsc{dem.pst})}\label{sec:6.3}


The \textsc{temporal} demonstratives, by contrast, locate their referent in five periods of time: \textsc{non-past} (present and future), \textsc{recent} past (earlier today), \textsc{removed} past (yester-, last), \textsc{remote} past (a while ago before yesterday), and \textsc{remotest} past (long ago). Ik has both singular and plural temporal nominal demonstratives, and these are listed below in \tabref{tab:dem:temp}. These temporal demonstratives are usually translated into English as ‘this’ and ‘that’ in the singular, and ‘these’ and ‘those’ in the plural, but with a sense of time rather than physical location. Recall from \tabref{tab:pro:rel} that Ik’s relative pronouns are identical in form to the temporal demonstratives in \tabref{tab:dem:temp}, except that because relative pronouns never occur before a pause, they lack the final forms (\textsc{ff}) of those in \tabref{tab:dem:temp}:


\begin{table}
\caption{Ik temporal demonstratives}
\label{tab:dem:temp}


\begin{tabularx}{\textwidth}{lXXXX}
\lsptoprule

& Singular &  & Plural & \\
& \textsc{nf} & \textsc{ff} & \textsc{nf} & \textsc{ff}\\
\midrule
Non-past & =na & =n & =ni & =n\\
Recent past & =náa & =nákᵃ & =níi & =níkⁱ\\
Removed past & =sɨna & =sɨn & =sini & =sin\\
Remote past & =nótso & =nótso & =nútsu & =nútsu\\
Remotest past & =noo & =nokᵒ & =nuu & =nukᵘ\\
\lspbottomrule
\end{tabularx}
\end{table}
Just like spatial demonstratives, temporal demonstratives directly follow the noun they refer to, as example sentences \REF{ex:dem:3}-\REF{ex:dem:4} illustrate:




\ea\label{ex:dem:3}
\gll Ráʝéte     ɗɨ=\textbf{nák\ᵃ}.\\
return:\textsc{ven:imp}   one:\textsc{obl=dem.sg.rec}\\
\glt ‘Give back the earlier one.’ 
\z




\ea\label{ex:dem:4}
\gll Gaana   kaɨna=\textbf{nótso}       Lopíar{\Í}\'{ɛ}     zùkᵘ. \\
bad:\textsc{3sg}   year:\textsc{nom}=\textsc{dem.sg.rem}   Lopiar.\textsc{gen} very    \\
\glt `That year (a while back) of Lopiar was very bad.’
\z  





\subsection{Anaphoric demonstratives (\textsc{anaph})}\label{sec:6.4}

\largerpage[2]
The \textsc{anaphoric} demonstratives locate their referent not in space or time \textit{per se} but rather in \textit{shared communicative context}. In other words, they point back to a referent that has either been mentioned already in the same discourse or is already known by both speaker and hearer by some other means. Ik has a singular and a plural \isi{anaphoric} demonstrative which are enclitics that have the same form in both non-final and final environments (i.e., their final vowels are not omitted). These invariant \isi{anaphoric} demonstratives, translated into English as ‘that’ in the singular and ‘those’ in the plural, are presented in \tabref{tab:dem:anaph}:


\begin{table}
\caption{Ik \isi{anaphoric} demonstratives}
\label{tab:dem:anaph}
\begin{tabularx}{\textwidth}{XX}
\lsptoprule
Singular & Plural\\
\midrule
=déé & =díí\\
\lspbottomrule
\end{tabularx}
\end{table}

Ik \isi{anaphoric} demonstratives also directly follow their referents, as in \REF{ex:dem:5}-\REF{ex:dem:6}:




\ea\label{ex:dem:5}
\gll Itíóna     ɲatala=\textbf{déé}. \\
be.important:\textsc{3sg}   tradition:\textsc{nom}=\textsc{anaph.sg}    \\
\glt ‘That tradition (already discussed) is important.’ 
\z




\ea\label{ex:dem:6}
\gll Atsa=noo     roɓa=\textbf{díí}            Sópìàᵒ. \\
come:\textsc{3sg}=\textsc{pst}   people:\textsc{nom}=\textsc{anaph.pl}  Ethiopia:\textsc{abl}    \\
\glt ‘Those people (already mentioned) came from Ethiopia.’ 
\z






\subsection{Adverbial demonstratives}\label{sec:6.5}
\subsubsection{Overview}\label{sec:6.5.1}

Besides the three types of nominal demonstratives described above, Ik also has a set of \textsc{adverbial} demonstratives that involve both locative and \isi{anaphoric} locative reference. Unlike the nominal demonstratives, the adverbial demonstratives are technically nouns themselves in that they are marked for case and can take their own nominal demonstratives. Their function, however, is adverbial.


\subsubsection{Locative adverbial demonstratives}\label{sec:6.5.2}

The first type of adverbial demonstrative, the \textsc{locative adverbial} demonstrative, locates the state or event expressed in a clause in physical space. Ik has three sets of such demonstratives. As shown in \tabref{tab:dem:locadv}, Sets 1 and 2 are built on degree of distance, while Set 3, in addition to degree of distance, is also split into singular and plural. These demonstratives are usually translated into English as ‘here’, ‘there’, and ‘over there’, depending on relative distance:


\begin{table}
\caption{Ik locative adverbial demonstratives}
\label{tab:dem:locadv}


\begin{tabularx}{\textwidth}{XXX}
\lsptoprule

& \multicolumn{1}{X}{Set 1} & Set 2\\
\midrule
Proximal & \multicolumn{1}{X}{} & náxánà- (=na)\\
Medial & \multicolumn{1}{X}{nédì- (=ne)} & \\
Distal & \multicolumn{1}{X}{kédì- (ke)} & k{\Í}xánà- (=ke)\\
\midrule
\multicolumn{1}{X}{Set 3} & Singular & Plural\\
\midrule
Proximal & naí- (=na) & nií- (=ni)\\
Medial & naí- (=ne) & \\
Distal & k\'{ɔ}\'{ɔ} (=ke) & kií- (=ke)\\
\lspbottomrule
\end{tabularx}
\end{table}
Examples \REF{ex:dem:7}-\REF{ex:dem:8} illustrate the locative adverbial demonstratives:




\ea\label{ex:dem:7}
\gll Ɨ{tá{\Í}a=bee}     k{\Í}xánee=kᵉ. \\
reach:\textsc{1sg}=\textsc{pst}   there=\textsc{dem.sg.dist}    \\
\glt ‘I reached there yesterday.’ 
\z




\ea\label{ex:dem:8}
\gll Ƙ{aini}   dzígwaa   naíé=ne. \\
go:\textsc{seq}   trade:\textsc{acc}   there=\textsc{dem.sg.med}    \\
\glt ‘And they went to do trade just right there.’ 
\z




\subsubsection{Anaphoric locative demonstratives}\label{sec:6.5.3}

The second type of Ik adverbial demonstrative are called the \textsc{anaphoric} \textsc{locatives}, which are nouns with a demonstrative function. Like the locative nominal demonstratives, these demonstratives point to a specific place – or metaphorically, a specific time – while also signifying anaphorically that that place or time is already known, either from earlier in the discourse or for some other reason. Ik has two such demonstratives with roughly the same meaning: \textit{tsʼ\'{ɛ}d\'{ɛ}-} and \textit{tʉmɛd\'{ɛ}-}, both of which are typically translated as ‘there’ or more rarely `then'. Because these words are technically nouns, \tabref{tab:dem:locanaph} presents them in a full case declension, while examples \REF{ex:dem:9}-\REF{ex:dem:10} illustrate them in sentences.


\begin{table}
\caption{Case declension of \isi{anaphoric} locative demonstratives}
\label{tab:dem:locanaph}


\begin{tabularx}{.66\textwidth}{XXX}
\lsptoprule

& ‘there’ & ‘there’\\
\midrule
\textsc{nom} & tsʼ\'{ɛ}da & tʉmɛda\\
\textsc{acc} & tsʼ\'{ɛ}d\'{ɛ}á & tʉmɛd\'{ɛ}á\\
\textsc{dat} & tsʼ\'{ɛ}d\'{ɛ}\'{ɛ} & tʉmɛd\'{ɛ}\'{ɛ}\\
\textsc{gen} & tsʼ\'{ɛ}d\'{ɛ}\'{ɛ} & tʉmɛd\'{ɛ}\'{ɛ}\\
\textsc{abl} & tsʼ\'{ɛ}d\'{ɔ}\'{ɔ} & tʉmɛd\'{ɔ}\'{ɔ}\\
\textsc{ins} & tsʼ\'{ɛ}dɔ & tʉmɛdɔ\\
\textsc{cop} & tsʼ\'{ɛ}d\'{ɔ}\'{ɔ} & tʉmɛd\'{ɔ}\'{ɔ}\\
\textsc{obl} & tsʼ\'{ɛ}d\'{ɛ} & tʉmɛd\'{ɛ}\\
\lspbottomrule
\end{tabularx}
\end{table}



\ea\label{ex:dem:9}
\gll Ƙ{aa=noo}   óŋora=ʝɨɨ     tsʼ\'{ɛ}d\'{ɛ}\'{ɛ}. \\
go:\textsc{3sg=pst}   elephant(s):\textsc{nom}=also   there:\textsc{dat}    \\
\glt ‘Even the elephants went there (place already mentioned).’ 
\z




\ea\label{ex:dem:10}
\gll Pɛl\'{ɛ}mʉɔ   saa     tʉmɛd\'{ɔ}\'{ɔ}. \\
appear:\textsc{seq}   others:\textsc{nom}   there:\textsc{abl}    \\
\glt ‘And others appeared from there (place already known).’ 
\z




