\section{Pronouns}\label{sec:5}



\subsection{Overview}\label{sec:5.1}


\textsc{Pronouns} ‘stand in’ for nouns that are not explicitly mentioned. Most Ik pronouns are free-standing words, but the subject-agreement pronominals and the dummy pronominal are suffixes that are bound to verbs (and so are treated in §8 on verbs). In a sentence, free pronouns are handled just like nouns in that they take case and modifiers. The free pronouns discussed in this section fall into the following nine categories: personal, impersonal possessum, indefinite, \isi{interrogative}, demonstrative, relative, \isi{reflexive}, distributive, and cohortative.




\subsection{Personal pronouns}\label{sec:5.2} 
\largerpage[2]



\begin{table}[p]
\caption{Ik personal pronouns}
\label{tab:pro:pers1}


\begin{tabularx}{.5\textwidth}{XXX}
\lsptoprule

\textsc{1sg} & \'{ɲ}cì- & ‘I’\\
\textsc{2sg} & bì- & ‘you’\\
\textsc{3sg} & ntsí- & ‘(s)he/it’\\
\textsc{1pl.exc} & ŋgó- & ‘we’\\
\textsc{1pl.inc} & ɲjíní- & ‘we all’\\
\textsc{2pl} & bìtì- & ‘you all’\\
\textsc{3pl} & ńtí- & ‘they’\\
\lspbottomrule
\end{tabularx}
\end{table}

\begin{table}[p]
\caption{Case declension of Ik personal pronouns}
\label{tab:pro:pers2}
\begin{tabularx}{\textwidth}{XXXXXXXXXXXlXXX}
\lsptoprule
 & \multicolumn{2}{c}{ ‘I’} &&& \multicolumn{2}{c}{ ‘you’} & \multicolumn{2}{c}{ ‘(s)he/it’} \\
& \textsc{nf} & \textsc{ff} &&& \textsc{nf} & \textsc{ff} & \textsc{nf} & \textsc{ff} \\
\midrule
\textsc{nom} & \'{ŋ}kà  & \'{ŋ}kᵃ   &&& bìà & bì   & ntsa  & ntsᵃ   \\
\textsc{acc} & \'{ɲ}cìà & \'{ɲ}cìkᵃ &&& bìà & bìkᵃ & ntsíá & ntsíkᵃ \\ 
\textsc{dat} & \'{ɲ}cìè & \'{ɲ}cìkᵉ &&& bìè & bìkᵉ & ntsíé & ntsíkᵉ \\ 
\textsc{gen} & \'{ɲ}cìè & \'{ɲ}cì   &&& bìè & bì   & ntsíé & ntsí   \\ 
\textsc{abl} & \'{ɲ}cùò & \'{ɲ}cù   &&& bùò & bù   & ntsúó & ntsú   \\ 
\textsc{ins} & \'{ŋ}kò  & \'{ŋ}kᵒ   &&& bùò & bù   & ntso  & ntsᵒ   \\ 
\textsc{cop} & \'{ɲ}cùò & \'{ɲ}cùkᵒ &&& bùò & bùkᵒ & ntsúó & ntsúkᵒ \\ 
\textsc{obl} & \'{ɲ}cì  & \'{ɲ}cⁱ   &&& bì  & bì   & ntsi  & ntsⁱ   \\ 
\\
& \multicolumn{2}{c}{ ‘we’} & \multicolumn{2}{c}{ ‘we all’} & \multicolumn{2}{c}{ ‘you all’} & \multicolumn{2}{c}{ ‘they’}\\
& \textsc{nf} & \textsc{ff} & \textsc{nf} & \textsc{ff} & \textsc{nf} & \textsc{ff} & \textsc{nf} & \textsc{ff}\\
  \midrule
\textsc{nom} &ŋgwa & ŋgwᵃ & ɲjíná & ɲjín & bìtà & bìtᵃ & ńtá & ńtᵃ\\
\textsc{acc} &ŋgóá & ŋgókᵃ & ɲjíníà & ɲjíníkᵃ & bìtìà & bìtìkᵃ & ńtíà & ńtíkᵃ\\
\textsc{dat} &ŋgóé & ŋgókᵉ & ɲjíníè & ɲjíníkᵉ & bìtìè & bìtìkᵉ & ńtíè & ńtíkᵉ\\
\textsc{gen} &ŋgóé & ŋgóᵉ & ɲjíníè & ɲjíní & bìtìè & bìtì & ńtíè & ńtí\\
\textsc{abl} &ŋgóó & ŋgó & ɲjínúò & ɲjín\'{u} & bìtùò & bìtù & ńtúò & ńtú\\
\textsc{ins} &ŋgo & ŋgᵒ & ɲjínó & ɲjínᵒ & bìtò & bìtᵒ & ńtó & ńtᵒ\\
\textsc{cop} &ŋgóó & ŋgókᵒ & ɲjínúò & ɲjínúkᵒ & bìtùò & bìtùkᵒ & ńtúò & ńtúkᵒ\\
\textsc{obl} &ŋgo & ŋgᵒ & ɲjíní & ɲjín & bìtì & bìtⁱ & ńtí & ńtⁱ\\

\lspbottomrule
\end{tabularx}
\end{table}
Ik \textsc{personal pronouns} represent the various grammatical persons that can be referred to in a sentence. The name is slightly misleading in that the pronouns can also denote nonpersonal, inanimate entities expressed by ‘it’ and ‘they’ (when referring to things). The Ik personal pronoun system operates along three axes: person (1, 2, 3), number (\textsc{sg}, \textsc{pl}), and \isi{clusivity} (\textsc{exc}, \textsc{inc}). The ‘first person’ refers to ‘I’ and ‘we’, the second to ‘you’, and the third to ‘she’, ‘he’, ‘it’, and ‘they’. ‘Number’ (singular or plural) obviously has to do with whether the entity is one or more than one. And ‘\isi{clusivity}’ (\textsc{exclusive} or \textsc{inclusive}) indicates whether the addressee of the speech is \textit{ex}cluded from or \textit{in}cluded in the reference of ‘we’. \tabref{tab:pro:pers1} presents the seven Ik personal pronouns in their lexical root forms, while \tabref{tab:pro:pers2} presents the same but in their full case declension:
\newpage 

 

\subsection{Impersonal possessum pronoun (\textsc{pssm})}\label{sec:5.3}


Ik also has a special pronoun whose only function is to represent a \textsc{possessum}, that is, generally, an entity associated with another entity (a \textsc{possessor}). This pronoun has the form \textit{ɛn{\Í}-} and must be bound to another noun or pronoun as the last element in a compound construction. It is \textsc{impersonal} in that it communicates nothing about the \isi{possessor} or the possessum except for the relationship of \isi{possession} itself. The impersonal possessum pronoun can be used in a compound construction with personal pronouns or other nouns. \tabref{tab:pro:impers1} shows \textit{ɛn{\Í}-} in conjunction with all seven personal pronouns. It can also be used with full nouns (including deverbalized verbal infinitives) as the compound’s first element. This type of possessive construction is illustrated in \tabref{tab:pro:impers2}:


\begin{table}
\caption{Ik impersonal possessum with pronouns}
\label{tab:pro:impers1}


\begin{tabularx}{\textwidth}{XXX}
\lsptoprule

ɲj-\'{ɛ}n{\Í}- & I-\textsc{possessum} & ‘mine’\\
bi-\'{ɛ}n{\Í}- & you-\textsc{possessum} & ‘yours’\\
nts-\'{ɛ}n{\Í}- & (s)he/it-\textsc{possessum} & ‘hers/his/its’\\
ŋgó-\'{ɛ}n{\Í}- & we-\textsc{possessum} & ‘ours’\\
ɲjíní-\`{ɛ}n{\Ì}- & we all-\textsc{possessum} & ‘all of ours’\\
biti-ɛn{\Í}- & you all-\textsc{possessum} & ‘all of yours’\\
ńtí-\`{ɛ}n{\Ì}- & they-\textsc{possessum} & ‘theirs’\\
\lspbottomrule
\end{tabularx}
\end{table}



\begin{table}
\caption{Ik impersonal possessum with nouns}
\label{tab:pro:impers2}


\begin{tabularx}{\textwidth}{XXX}
\lsptoprule

aɗoni-ɛn{\Í}- & to be three-\textsc{pssm} & ‘the third time’\\
cɨkám\'{ɛ}-\'{ɛ}n{\Í}- & women-\textsc{pssm} & ‘the women’s’\\
ɦyɔ-ɛn{\Í}- & cattle-\textsc{pssm} & ‘the cattle’s’\\
Icé-\'{ɛ}n{\Í}- & Ik-\textsc{pssm} & ‘the Ik’s’\\
ɲɔt\'{ɔ}-\'{ɛ}n{\Í}- & men-\textsc{pssm} & ‘the men’s’\\
roɓe-ɛn{\Í}- & people-\textsc{pssm} & ‘the people’s’\\
wicé-\'{ɛ}n{\Í}- & children-\textsc{pssm} & ‘the children’s’\\
\lspbottomrule
\end{tabularx}
\end{table}



\subsection{Indefinite pronouns}\label{sec:5.4}


Pronouns that are \textsc{indefinite} stand for other entities but with a certain degree of in\isi{definiteness} or vagueness. All but one of the Ik indefinite pronouns are based on the root \textit{kɔn{\Í}-} ‘one’ or its plural counterpart \textit{kíní-} ‘more than one’. The other one that is not based on these roots is \textit{saí-} ‘some more/other’, a root that may not actually belong with this set but is included on the basis of its English translation. \tabref{tab:pro:indef} provides a rundown of these Ik indefinite pronouns:


\begin{table}
\caption{Ik indefinite pronouns}
\label{tab:pro:indef}


\begin{tabularx}{\textwidth}{XXl}
\lsptoprule

kɔn{\Í}- & one & ‘another, some (\textsc{sg})’\\
k\'{ɔ}n-áí- & one-place & ‘somewhere (else)’\\
k\'{ɔ}n{\Í}-\'{ɛ}n{\Í}- & one-\textsc{possessum} & ‘a(n), some (\textsc{sg})’\\
kɔn{\Í}-ámà- & one-person & ‘somebody, someone’\\
k\'{ɔ}n-\'{ɔ}mà- & one-\textsc{singulative} & ‘some unknown person’\\
kíní-ámá- & many-person & ‘some unknown people’\\
kíní-\'{ɛ}n{\Í}- & many-\textsc{possessum} & ‘some (\textsc{pl})’\\
saí- & some & ‘some more, some other’\\
\lspbottomrule
\end{tabularx}
\end{table}



\subsection{Interrogative pronouns}\label{sec:5.5}


The role of \textsc{interrogative} pronouns is to query the identity of the entity they represent. As a result, they are used to form questions. All but one of the Ik \isi{interrogative} pronouns incorporate the ancient northeastern African \isi{interrogative} \isi{particle} \textit{*nd-/nt-}, and the one that does not has the form \textit{ìsì-} ‘what’. The seven Ik \isi{interrogative} pronouns are presented in \tabref{tab:pro:inter}. Note that in the table's first column, forms are hyphenated when there is a hypothesis as to their internal morphological composition, which is  reflected in the second column:


\begin{table}
\caption{Ik \isi{interrogative} pronouns}
\label{tab:pro:inter}


\begin{tabularx}{\textwidth}{XXX}
\lsptoprule

ìsì- & what & ‘what?’\\
nd-aí- & ?-place & ‘where?’\\
ǹdò- & who & ‘who?’\\
ńt- & ? & ‘where?’\\
ńt\'{ɛ}-\'{ɛ}n{\Í}- & ?-\textsc{possessum} & ‘which (\textsc{sg})’\\
ńtí-\'{ɛ}n{\Í}- & ?-\textsc{possessum} & ‘which (\textsc{pl})’\\
ńtí & ? & `how?'\\
\lspbottomrule
\end{tabularx}
\end{table}

\newpage 
In the formation of a question, Ik \isi{interrogative} pronouns fill the same slot as the nouns they are representing. It is common for the \isi{interrogative} pronoun to be ‘fronted’: moved for emphasis to the first place in the sentence. When this happens, the \isi{interrogative} pronoun takes the \isi{copulative case} (see \sectref{sec:7.8}), as exemplified in sentences \REF{ex:pro:1}-\REF{ex:pro:2}. Both demonstrated word orders are perfectly acceptable. For more on how questions are formed in Ik, please see \sectref{sec:10.4.3}.




\ea\label{ex:pro:1}
  \ea
  \gll B\'{ɛ}ɗ{\Í}dà   \textbf{ìs}?      \\
want:\textsc{2sg}   what:\textsc{nom}    \\ 
  \glt ‘You want what?
  \ex
  \gll \textbf{Isio}     b\'{ɛ}ɗ{\Î}dᵃ? \\
  what:\textsc{cop}   want:2\textsc{sg}    \\
  \glt ‘What do you want?
  \z
\z





\ea\label{ex:pro:2}
  \ea
  \gll Ia     \textbf{ndaík\ᵉ}?    \\
be:\textsc{3sg} where:\textsc{dat}       \\ 
  \glt ‘It is where?’      
  \ex
  \gll \textbf{Ndaíó}   iâdᵉ? \\
  where:\textsc{cop}   be:\textsc{3sg:dp}    \\
  \glt ‘Where is it?’
  \z
\z







\subsection{Demonstrative pronouns}\label{sec:5.6}


Ik also has a set of \textsc{demonstrative} pronouns that referentially ‘demonstrate’ or point to an entity. They are all based on either the singular form \textit{ɗɨ{}-} ‘this (one)’ or the plural form \textit{ɗi-} ‘these (ones)’ that differ formally only in regard to their vowel (/ɨ/ versus /i/). The Ik demonstrative pronoun system is divided in three categories based on spatial distance from the speaker: 1) \textsc{proximal}, meaning near the speaker, 2) \textsc{medial}, meaning a relatively medium distance from the speaker, and 3) \textsc{distal}, meaning relatively far from the speaker. The medial and distal forms, for both singular and plural, consist of the root \textit{ɗɨ{}-/ɗi-} preceded by the cliticized distal demonstratives \textit{kɨ} ‘that’ (derived from \textit{ke}) for singular and \textit{ki} ‘those’ for plural. The only difference between the medial and distal pronouns is the tone pattern whereby the medial form has a high tone on the last \isi{syllable}, while the distal form does not. \tabref{tab:pro:dem1} presents these pronouns in their six lexical forms, while \tabref{tab:pro:dem2} gives their full case declensions. Note that the medial and distal forms are indistinguishable except in the \textsc{nom}, \textsc{ins}, and \textsc{obl} cases:


\begin{table}
\caption{Ik demonstrative pronouns}
\label{tab:pro:dem1}


\begin{tabularx}{\textwidth}{XXXXX}
\lsptoprule

& Singular &  & Plural & \\
\midrule
Proximal & ɗɨ\'{}- & ‘this’ & ɗi\'{}- & ‘these’\\
Medial & kɨɗ{\Í}- & ‘that’ & kiɗ\'{i}- & ‘those\\
Distal & kɨɗɨ\'{}- & ‘that’ & kiɗi\'{}- & ‘those’\\
\lspbottomrule
\end{tabularx}
\end{table}

\begin{table}
\caption{Case declensions of the demonstrative pronouns}
\label{tab:pro:dem2}


\begin{tabularx}{\textwidth}{XXXXXXX}
\lsptoprule

& \multicolumn{2}{X}{ Proximal} & \multicolumn{2}{X}{ Medial} & \multicolumn{2}{X}{ Distal}\\

& \textsc{sg} & \textsc{pl} & \textsc{sg} & \textsc{pl} & \textsc{sg} & \textsc{pl}\\
\midrule
\textsc{nom} & ɗa & ɗa & kɨɗá & kiɗá & kɨɗa & kiɗa\\
\textsc{acc} & ɗ{\Í}á & ɗíá & kɨɗ{\Í}á & kiɗíá & kɨɗ{\Í}á & kiɗíá\\
\textsc{dat} & ɗ\'{ɛ}\'{ɛ} & ɗíé & kɨɗ\'{ɛ}\'{ɛ} & kiɗíé & kɨɗ\'{ɛ}\'{ɛ} & kiɗíé\\
\textsc{gen} & ɗ\'{ɛ}\'{ɛ} & ɗíé & kɨɗ\'{ɛ}\'{ɛ} & kiɗíé & kɨɗ\'{ɛ}\'{ɛ} & kiɗíé\\
\textsc{abl} & ɗ\'{ɔ}\'{ɔ} & ɗúó & kɨɗ\'{ɔ}\'{ɔ} & kiɗúó & kɨɗ\'{ɔ}\'{ɔ} & kiɗúó\\
\textsc{ins} & ɗɔ & ɗo & kɨɗ\'{ɔ} & kiɗó & kɨɗɔ & kiɗo\\
\textsc{cop} & ɗ\'{ɔ}\'{ɔ} & ɗúó & kɨɗ\'{ɔ}\'{ɔ} & kiɗúó & kɨɗ\'{ɔ}\'{ɔ} & kiɗúó\\
\textsc{obl} & ɗɨ & ɗi & kɨɗ{\Í} & kiɗí & kɨɗɨ & kiɗi\\
\lspbottomrule
\end{tabularx}
\end{table}



\subsection{Relative pronouns (\textsc{rel})}\label{sec:5.7}


The role of \textsc{relative} pronouns is to introduce a \isi{relative clause}: a clause embedded in a \isi{main clause} to specify the reference of an entity in the \isi{main clause}. One of the most fascinating features of the Ik \isi{relative pronoun} system is that it is tensed. That is, it is able to encode the time period at which the statement contained in the \isi{relative clause} holds or held true. The five time periods covered by these pronouns are 1) \textsc{non-past}, 2) \textsc{recent past} (earlier today), 3) \textsc{removed past} (yester-, last), 4) \textsc{remote past} (a while ago), and 5) \textsc{remotest past} (long ago).

The Ik relative pronouns are all enclitics based on the proto-demonstratives \textit{na} ‘this’ and \textit{ni} ‘these’ (see \sectref{sec:6.2} below). Those proto-forms are identical to the non-past relative pronouns \textit{na} ‘that/which’ and \textit{ni} ‘that/which (\textsc{pl})’ shown in \tabref{tab:pro:rel}. The remaining tensed relative pronouns are built from the proto-forms with a variety of ancient prefixes and suffixes such as \textit{sɨ-/si-} and \textit{-tso/-tsu}.


\begin{table}
\caption{Ik relative pronouns}
\label{tab:pro:rel}


\begin{tabularx}{\textwidth}{XXXX}
\lsptoprule

& Singular & Plural & \\
\midrule
Non-past & =na & =ni & ‘that/which {\dots}’\\
Recent past & =náa & =níi & ‘that/which {\dots}’\\
Removed past & =sɨna & =sini & ‘that/which {\dots}’\\
Remote past & =nótso & =nútsu & ‘that/which {\dots}’\\
Remotest past & =noo & =nuu & ‘that/which {\dots}’\\
\lspbottomrule
\end{tabularx}
\end{table}
As shown in examples \REF{ex:pro:3}-\REF{ex:pro:4}, no matter where an Ik \isi{relative clause} (\textsc{rc}) appears in a sentence, the \isi{relative pronoun} will introduce it as the first element in the clause. The entity in the \isi{main clause} that the \isi{relative clause} is modifying – called the \textsc{common argument} – must be the last word before the \isi{relative clause}. As a \isi{clitic}, the \isi{relative pronoun} attaches to the \isi{common argument}. To learn more about the syntax of relative clauses, please see \sectref{sec:10.3.2}.




\ea\label{ex:pro:3}
\gll Atsáá     ceka=[\textbf{náa}       ƙwaatetᵃ]\textsc{\textsubscript{rc}}. \\
come:\textsc{3sg:prf}   woman:\textsc{nom}=\textsc{rel:sg} give.birth:\textsc{3sg}    \\
\glt ‘The woman [who gave birth today] has come.’ 
\z




\ea\label{ex:pro:4}
\gll Tɔŋ\'{ɔ}lano     rie=[\textbf{sini}     detí]\textsc{\textsubscript{rc}}. \\
slaughter:\textsc{hort}   goats:\textsc{obl}=\textsc{rel:pl}   bring:\textsc{1sg}    \\
\glt ‘Let’s slaughter the goats [that I brought yesterday].’ 
\z






\subsection{Reflexive pronoun}\label{sec:5.8}


Ik has a \textsc{reflexive} pronoun that ‘reflects’ the impact of a verb back onto the subject of the verb. In other words, with the \isi{reflexive}, the subject and object of an action are the same entity. The Ik \isi{reflexive} pronoun has the form \textit{as{\Í}-} in the singular and \textit{ás{\Í}kà-} in the plural, translated as ‘-self’ and ‘-selves’, respectively.


These \isi{reflexive} pronouns are used extensively to make \textsc{semi-transitive} verbs: verbs falling between transitive and \isi{intransitive}. For example, while the verb \textit{ídzòn} ‘to discharge, emit’ is \isi{intransitive} and the verb \textit{ídzès} ‘to discharge, emit, shoot’ is transitive, the verb \textit{ídzesa as{\Í}} ‘to shoot across (literally `to shoot -self’)’ is ‘semi-transitive’ because the subject and object of the shooting are the same entity. The full case declensions of the singular and plural \isi{reflexive} pronouns are given below in \tabref{tab:pro:refl}, and example sentences \REF{ex:pro:5}-\REF{ex:pro:6} illustrate both the \isi{reflexive} and the semi-transitive usages of these special pronouns:




\ea\label{ex:pro:5}
\gll Kwatsítúƙoe     \textbf{as}. \\
small:\textsc{caus:comp:imp}   self:\textsc{obl}    \\
\glt ‘Humble yourself (lit: make yourself small).’ 
\z




\ea\label{ex:pro:6}
\gll Ƙaio     dzúíka   ɨt{\Í}ɗ{\Í}ɗátie     \textbf{ás{\Í}kàk\ᵃ}. \\
go:\textsc{seq}   thieves:\textsc{nom}   sneak:\textsc{3pl:sim} selves:\textsc{acc}    \\
\glt ‘And the thieves went slinking away (lit: sneaking themselves).’ 
\z



\begin{table}
\caption{Case declensions of the \isi{reflexive} pronouns}
\label{tab:pro:refl}


\begin{tabularx}{\textwidth}{XXXXX}
\lsptoprule

& Singular &  & Plural & \\
& \textsc{nf} & \textsc{ff} & \textsc{nf} & \textsc{ff}\\
\midrule
\textsc{nom} & asa & as & ás{\Í}kà & ás{\Í}kᵃ\\
\textsc{acc} & as{\Í}á & as{\Í}kᵃ & ás{\Í}kàà & ás{\Í}kàkᵃ\\
\textsc{dat} & as{\Í}\'{ɛ} & as{\Í}k\ᵋ & ás{\Í}k\`{ɛ}\`{ɛ} & ás{\Í}kàk\ᵋ\\
\textsc{gen} & as{\Í}\'{ɛ} & as{\Í} & ás{\Í}k\`{ɛ}\`{ɛ} & ás{\Í}kà\ᵋ\\
\textsc{abl} & as\'{ʉ}\'{ɔ} & as\'{ʉ} & ás{\Í}k\`{ɔ}\`{ɔ} & ás{\Í}kàᵓ\\
\textsc{ins} & asɔ & asᵓ & ás{\Í}k\`{ɔ} & ás{\Í}kᵓ\\
\textsc{cop} & as\'{ʉ}\'{ɔ} & as\'{ʉ}kᵓ & ás{\Í}k\`{ɔ}\`{ɔ} & ás{\Í}kàkᵓ\\
\textsc{obl} & asɨ & as & ás{\Í}kà & ás{\Í}kᵃ\\
\lspbottomrule
\end{tabularx}
\end{table}


