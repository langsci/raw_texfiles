% =====================================
% Circle
% =====================================

   \def\Circled#1{\raisebox{.7pt}{\textcircled{\raisebox{-.8pt} 
{{\scriptsize #1}}}}}
 \def\Circledd#1{\raisebox{.6pt}{\textcircled{\raisebox{-.8pt} 
{{\scriptsize #1}}}}}
% =====================================



\newenvironment{letter}[1]{\phantomsection  %
\addcontentsline{toc}{section}{#1}%
\begin{multicols}{2}[%
% \addvspace{1.5\baselineskip}%
\scalebox{1.5}{\bfseries\Huge #1}%
% \hline
% \addvspace{.1\baselineskip}
% \hline
% \addvspace{1\baselineskip}
]%
}%
{%
\end{multicols}%
}




% =====================================
% Class based parsing
% =====================================
% \newenvironment{newentry}{\hspace*{.05\textwidth}
% % % 			  \begin{minipage}{.44\textwidth}%
% 			  \setlength{\parindent}{-.05\textwidth}%
% 			  }%
% 			  {%
% % 			  \end{minipage}
% 			  \setlength{\parindent}{.00\textwidth}
% 			  }

\newcommand{\newentry}{\par\smallskip\hangindent=6pt} 
 
\newcommand{\homograph}[1]{\hspace*{-1.5mm}%
			  \parbox{1.5mm}{\raggedleft \textsuperscript{#1}}}%
% 			  \parbox[1mm]{%
% 				\begin{tabular}{r}#1\end{tabular}%
% 			  }%
% 			  }}
\newcommand{\ipa}[1]{{ [#1]}}
% \newcommand{\hyperipa}[2]{{[\hypertarget{#1}{#2}]}}
\newcommand{\pos}[1]{ {\itshape #1}.}
\newcommand{\synpos}[1]{ {\itshape #1}.}
\newcommand{\headword}[1]{{\bfseries #1}\markboth{#1}{#1}}
% % \newcommand{\hyperheadword}[2]{{\bfseries \hypertarget{#1}{#2}}\markboth{#2}{#2}}
\newcommand{\sensenr}[1]{ {\itshape\large  #1}}
\newcommand{\definition}[1]{ {#1}. }
\newcommand{\noperioddefinition}[1]{ {#1} }
\newcommand{\hyperdefinition}[2]{\hypertarget{#1}{#2}}
\newcommand{\lsgloss}[1]{ {#1}.}
% \newcommand{\exnr}[1]{{(#1)}}
\newcommand{\startexample}{ •}
\newcommand{\exnr}[1]{}
\newcommand{\vernacular}[1]{{}}
\newcommand{\hypervernacular}[2]{\hypertarget{}{ \slshape }}
\newcommand{\modvernacular}[1]{{ \slshape #1}}
\newcommand{\hypermodvernacular}[2]{{ \slshape\mdseries \hypertarget{#1}{#2}}}
\newcommand{\trs}[1]{{ #1}}
\newcommand{\hypertrs}[2]{\hypertarget{#1}{ #2}}
\newcommand{\sciname}[1]{{ (\textit{#1})}.}
\newcommand{\usage}[1]{ [\textit{#1}].}
\newcommand{\plural}[1]{ \textit{pl.} \textbf{#1}.}
\newcommand{\literalmeaning}[1]{ \textit{Lit.} `#1'. }
\newcommand{\JBtype}[1]{$\xrightarrow{\makebox[.4cm]{\tiny #1}}$}
\newcommand{\error}[1]{{\color{red}{#1}}}
\newcommand{\etymology}{}
\newcommand{\etymologyform}[1]{{ <\textit{#1}}}
\newcommand{\etymologysrc}[1]{ (#1)}
\newcommand{\etymologygloss}[1]{ `#1'}
% % \newcommand{\fixme}{\todo[inline]{FIXME}}
% % \newcommand{\fixpos}{{\color{orange}{pos?}}}
\newcommand{\fixpron}{{\color{green}{[IPA?]}}}
\renewcommand{\fixpron}{}
% % \newcommand{\fixdef}{{\color{red}{definition?}}}

\newcommand{\typeant}[1]{ \textit{Ant:} \textbf{#1} }
% \newcommand{\typeasmlvar}[1]{#1}
\newcommand{\typecf}[1]{ \textit{Cf:} \textbf{#1} }
\newcommand{\typecntrof}[1]{ (\textit{cont. var.} \textbf{#1}) }
\newcommand{\typecntrvarof}[1]{ (\textit{cont. var. of} \textbf{#1}) }
\newcommand{\typedialvarGu}[1]{ (\textit{Gu. var.} \textbf{#1}) }
\newcommand{\typedialvarMo}[1]{ (\textit{Mo. var.} \textbf{#1}) }
\newcommand{\typedialvarofGu}[1]{  (\textit{Gu. var. of} \textbf{#1}) }
\newcommand{\typedialvarofMo}[1]{  (\textit{Mo. var. of} \textbf{#1}) }
\newcommand{\typeempty}[1]{#1}
\newcommand{\typeEnum}[1]{ \textit{Enum:} \textbf{#1} }
% \newcommand{\typefrgnvar}[1]{#1}
% \newcommand{\typefrgnvarof}[1]{#1}
% \newcommand{\typeinflvar}[1]{#1}
% \newcommand{\typeinflvarof}[1]{#1}
\newcommand{\typePl}[1]{ (\textit{Pl. var.} \textbf{#1})}
\newcommand{\typesyn}[1]{ \textit{Syn:} \textbf{#1} }
\newcommand{\typeSynT}[1]{ \textit{SynT:} \textbf{#1} }
\newcommand{\typevar}[1]{ (\textit{var.} \textbf{#1}) }
\newcommand{\typevarof}[1]{ (\textit{var. of} \textbf{#1}) }
\newcommand{\definitioncloser}{.}
\newcommand{\etymologycloser}{.}

 

\newcommand{\higha}{{$^{\text{a}}$}}
\newcommand{\highe}{{$^{\text{e}}$}}
\newcommand{\highE}{{$^{\text{ɛ}}$}}
\newcommand{\highI}{{$^{\text{ɨ}}$}}
\newcommand{\higho}{{$^{\text{o}}$}}
\newcommand{\highO}{{$^{\text{ɔ}}$}}
\newcommand{\highu}{{$^{\text{u}}$}}
\newcommand{\highU}{\textsf{{\hspace{.5pt}ᶶ}}}

 
\newcommand{\ᵃ}{\textsf{{\hspace{.5pt}ᵃ}}}
\newcommand{\ᵋ}{\textsf{{\hspace{.5pt}ᵋ}}} 
\newcommand{\ᵉ}{\textsf{{\hspace{.1pt}ᵉ}}} 	
\newcommand{\ⁱ}{{\raisebox{-.65mm}{{\textsuperscript{\sffamily\scriptsize i}}}}}
\newcommand{\ᴵ}{\textsf{{\hspace{.5pt}ᴵ}}}
\newcommand{\ᶤ}{\textsf{{\hspace{.5pt}ᶤ}}} 
\newcommand{\ᵒ}{\textsf{{\hspace{.5pt}ᵒ}}}
\newcommand{\ᵓ}{\textsf{{\hspace{.5pt}ᵓ}}}
\newcommand{\ᵘ}{\textsf{{\hspace{.5pt}ᵘ}}}
\newcommand{\ᶶ}{\textsf{{\hspace{.5pt}ᶶ}}}
\newcommand{\ꜜ}{\textsf{ꜜ\hspace*{-2pt}}}

\newcommand{\glosses}[1]{}
\newcommand{\citationform}[1]{\markboth{#1}{#1}\textbf{#1}.~}
\newcommand{\lexicalunit}[1]{ (#1) }
\newcommand{\nobreaklexicalunit}[1]{\mbox{\lexicalunit{#1}}}

 



\renewcommand{\root}[1]{ (#1) }
\newcommand{\scientificname}[1]{\textit{#1}. } 
\newcommand{\note}[1]{#1 }  




\newcommand{\dicobox}[1]{\medskip\parbox{.90\textwidth}{\par\smallskip\hangindent=6pt#1}\\\medskip}


% \newcommand{\tablevspace}{\\[-.5em]}

% \newcommand{\I}{\ipabar{i}{.5ex}{1}{}{}}
\newcommand{\I}{\sout{{\i}}}
\newcommand{\Í}{\sout{\'{\i}}}
\newcommand{\Ì}{\sout{\`{\i}}}
\newcommand{\Î}{\sout{\^{\i}}}
\newcommand{\Ú}{\sout{Ú}}


\newcommand{\noteseealso}[1]{See also \textit{#1}.}
\newcommand{\notealsocalled}[1]{Also called \textit{#1}.}
\newcommand{\notealsopronounced}[1]{Also pronounced as \textit{#1}. }
\newcommand{\notefrom}[1]{From #1.}
\newcommand{\notecompare}[1]{Compare with \textit{#1}.}

\newcommand{\ETC}{etc}
\newcommand{\SP}{sp}
\newcommand{\SPP}{spp}