\addchap{Preface}
When I first heard about the Ik back in September 2005, I was thoroughly intrigued. Here was a people just emerging from the mists of time, from that now dark and shrouded realm of African prehistory. Judging by appearances, their journey had not been easy. Their story spoke of great suffering in the form of sickness, suppression, starvation, and slaughter. And yet, somehow, there they were, limping into the 21\textsuperscript{st} century as survivors of conditions most of us cannot imagine. Having grown up in a safe and serene community in the American South, I thought the Ik seemed stranger than fiction. People like this actually exist out there? I found myself wanting to know more about them, wanting to know who they are. Subconsciously I sensed that anyone who could endure what they had endured could perhaps teach me something about being truly human. 

My quest to know the Ik has led me down a winding path to the present. Over the years I have been frustrated by my inability to enter fully into their world, to see reality through their eyes. More than once I wished I were an anthropologist, so I could get a better grasp of their essence as a people. But time and time again, life steered me right back to the language – to Icétôd. I gradually learned to accept that their language is a doorway to their spirit, and that as a linguist, I could only open the door for others, and point the way to the Promised Land while I remain at the threshold. This book can act as a key to that door, a key that has been carefully shaped and smoothened by hands tired yet trembling with purpose.

Living in Ikland has taught me a lot about being human, but not in the way I expected. It was not by becoming ‘one with the people’ that I learned what it is like to survive subhuman conditions. And it was not physical starvation, or sickness, or slaughter that I was forced to endure. No, I was spared those things. Yet all the same, in Ikland I became acquainted with spiritual starvation, social sickness, and the wholesale slaughter of my cultural, religious, and intellectual idols. And just as the Ik have learned that life does not consist in ‘bread alone’, nor in health, nor in security – but can carry on living with dignity and humanity – I have learned that at the rock bottom of my soul, where my self ends and the world begins, there is where Life resides. That realization is my ‘pearl of great price’ for which I have sold everything else and would do it all over again. 
