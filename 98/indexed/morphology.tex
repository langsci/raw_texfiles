\section{Morphology}\label{sec:3}
 
\subsection{Overview}\label{sec:3.1}
\largerpage

\textsc{morphology} is the system by which a language grammar makes words. While the preceding chapter introduced meaningful sound units (phonemes), the pres\-ent chapter describes larger meaningful units called \textsc{morphemes}. Icétôd exhibits three types of morpheme: word, affix, and \isi{clitic}. A \textsc{word} is defined as a free morpheme that can meaningfully stand alone. An \textsc{affix} is a bound morpheme that must attach to a word to maintain its integrity. Affixes are indicated in this grammar by a hyphen before (and sometimes after) them, as in \{-án-\}, the \isi{stative adjectival} suffix. A \textsc{clitic} is a hybrid: in some constructions it acts like a word standing alone, while in other constructions, it attaches to a word like an affix. Clitics may be marked in this grammar by an equals sign, as in \{=kì\} ‘those’. 

Traditionally, languages are described as having \textsc{word classes}, that is, categories of morphemes that have certain characteristics. These classes include the familiar major ones like ‘nouns’ and ‘verbs’ but often several others as well. For the purposes of this grammar sketch, free-standing words and clitics are considered ‘words’, while affixes are not. In Icétôd, thirteen word classes are recognized and include the following: nouns, pronouns, demonstratives, quantifiers, numerals, prepositions, verbs, adverbs, ideophones, interjections, nursery words, complementizers, and connectives (or conjunctions). Each of these word classes is briefly introduced in the following subsections, while a full list of Icétôd affixes can be found later in Appendix A.
 
\subsection{Nouns}\label{sec:3.2}


\textsc{Nouns} and verbs make up the language’s only two \textsc{open} word classes, meaning that they may have new members added to them. Nouns make up roughly 47\% of the total Icétôd lexicon. Noun roots can be short, like \textit{eí-} ‘stomach contents’, or long like \textit{ɲákaɓɔɓwáátá-} ‘finger ring’, but they all have at least two syllables in their root form. This structural condition is necessary because some case suffixes delete the last vowel of the noun root when they affix to it. All Icétôd nouns, without a single exception, end in a vowel in their root forms. Noun roots are represented throughout this book with hyphenated forms, indicating that in actual Icétôd speech, any noun must have at least a case suffix. In addition to case, nouns may take \isi{singulative} or \isi{plurative} suffixes and may be joined with other nouns to make compound nouns. §4 is devoted to expounding on Icétôd nouns.
 
\subsection{Pronouns}\label{sec:3.3}


\textsc{Pronouns} form a \textsc{closed} word class, incapable of admitting new members. They ‘stand in’ for nouns whose specific names need not always be mentioned or repeated. Pronouns make up less than 1\% of the Icétôd lexicon and yet have great grammatical importance. Most Icétôd pronouns are \textsc{free}, capable of standing on their own, while others are inextricably \textsc{bound} to verbs. They may be \textsc{personal}, capable of specifying grammatical person, or \textsc{impersonal}. Other categories of pronoun include: indefinite, \isi{interrogative}, demonstrative, relative, and \isi{reflexive}. §5 is devoted to describing the various kinds of pronouns in Icétôd.
 
\subsection{Demonstratives}\label{sec:3.4}


\textsc{Demonstratives} form another closed word class, admitting no new members. They ‘demonstrate’ nouns by ‘pointing them out’, referring to them spatially, temporally, or discursively. They too make up less than 1\% of the lexicon. Many Icétôd demonstratives have been analyzed as clitics: They seem sometimes to act like separate words, and yet in terms of \isi{vowel harmony}, they act like suffixes. As clitics, they may be written connected to words in linguistic writing (with =), whereas in non-linguistic writing, they are written separately. For example, the phrase ‘these trees’ would be written as \textit{dakwítína=ni} in linguistic publications and as \textit{dakwítína ni} elsewhere. Icétôd has four kinds of demonstrative: spatial, temporal, \isi{anaphoric}, and locative adverbial – all of which are discussed in §6.
 
\subsection{Quantifiers}\label{sec:3.5}


As their name implies, \textsc{quantifiers} ‘quantify’ the nouns that precede them. That is, they are separate words that follow nouns and convey the general quantity of the noun in terms of allness, bothness, fewness, or manyness. Specific, numeric quantity is expressed by the numerals, which are the topic of the next subsection. Icétôd quantifiers sometimes act more like numerals by directly following the noun they modify without an intervening \isi{relative pronoun}, as in \textit{wika ƙwaɗ{\ᵉ}} ‘few children’. But other times they act more like \isi{adjectival} verbs by taking a \isi{relative pronoun} between them and the noun they modify, for example, \textit{wika ni ƙwaɗ{\ᵉ}} ‘children that (are) few’. In the former function as numerals, they have a distinct, perhaps more ancient root, as in \textit{ƙwàɗè}, whereas in their function as \isi{adjectival} verbs, they have a truncated root in a verbal \isi{infinitive}, in this case \textit{ƙwàɗ-òn} ‘to be few’. The eight known Icétôd quantifiers are given in \tabref{tab:morph:quant}:


\begin{table}
\caption{Icétôd quantifiers}
\label{tab:morph:quant}


\begin{tabularx}{.66\textwidth}{XXl}
\lsptoprule

Non-final & Final & \\
\midrule
ɗàŋ{\Ì}ɗàŋ{\Ì} & ɗàŋ{\Ì}ɗàŋ & ‘all, entire, whole’\\
mùɲù & mùɲ & ‘all, entire, whole’\\
mùɲùmùɲù & mùɲùmùɲ & ‘all, entire, whole’\\
ts{\Í}ɗ{\Ì} & ts{\Í}ɗ\ᶤ & ‘all, entire, whole’\\
ts{\Í}ɗɨts{\Í}ɗɨ & ts{\Í}ɗɨts{\Í}ɗ\ᶤ & ‘all, entire, whole’\\
ɡáí & ɡáí & ‘both’\\
ƙwàɗè & ƙwàɗ{\ᵉ} & ‘few’\\
kòmà & kòm & ‘many’\\
\lspbottomrule
\end{tabularx}
\end{table}



\subsection{Numerals}\label{sec:3.6}


\textsc{Numerals} convey the specific number of the noun they modify. Icétôd has a quinary or ‘base-5’ counting system, meaning that it has individual words for the numbers 1-5 and then builds numbers 6-9 by adding the appropriate number to 5, as in \textit{tude ńda kiɗi tsʼagús} ‘five and those four’, which is 9. The number 10 is technically not a numeral, but rather, a noun: \textit{toomíní-}. Icétôd numerals directly follow the noun they modify, without an intervening \isi{relative pronoun}. Just as the quantifiers \textit{ƙwàɗè} ‘few’ and \textit{kòmà} ‘many’ can function as verbs, the numerals 1-5 can also function as verbs. \tabref{tab:morph:num} presents Icétôd numerals 1-9:


\begin{table}
\caption{Icétôd numerals}
\label{tab:morph:num}


\begin{tabularx}{\textwidth}{lXXX}
\lsptoprule

\# & Non-final & Final & \\
\midrule
1 & k\`{ɔ}nà & k\`{ɔ}n & ‘one’\\
2 & lèɓètsè & lèɓèts{\ᵉ} & ‘two’\\
3 & àɗè & àɗ{\ᵉ} & ‘three’\\
4 & tsʼagúsé & tsʼagús & ‘four’\\
5 & tùdè & tùd{\ᵉ} & ‘five’\\
6 & tude ńdà k\`{ɛ}ɗ{\Ì} kɔn & {\dots} ńdà k\`{ɛ}ɗ{\Ì} kɔn & ‘five and one’\\
7 & tude ńda kiɗi léɓètsè & {\dots} ńda kiɗi léɓèts{\ᵉ} & ‘five and two’\\
8 & tude ńdà kìɗì àɗè & {\dots} ńdà kìɗì àɗ{\ᵉ} & ‘five and three’\\
9 & tude ńda kiɗi tsʼagúsé & {\dots} ńda kiɗi tsʼagús & ‘five and four’\\
\lspbottomrule
\end{tabularx}
\end{table}

  
To form numbers 11-19, Icétôd builds off the noun \textit{toomíní-} ‘ten’ and then repeats the quinary system shown in \tabref{tab:morph:num}. For example, the number 17 is expressed as \textit{toomín ńda kiɗi túde ńda kiɗi léɓèts{\ᵉ}} ‘ten and those five and those two’. Then, after 19, the numbers 20, 30, 40, etc. are based on the compound \textit{toomín-ékù-} ‘ten-eye’, as in \textit{toomínékwa léɓèts{\ᵉ}} ‘ten-eye two’, which is 20. The numbers for 100 (\textit{ŋam{\Í}á{\Ì}-}) and 1,000 (\textit{álìfù-}) have both been borrowed from Swahili.
 
\subsection{Prepositions}\label{sec:3.7}


\textsc{Prepositions} are usually small particles ‘pre-posed’, that is, put in front of a noun to indicate what its relationship is to another noun or to the wider sentence in which it occurs. Many of the functions that prepositions fulfill in other languages are handled by cases in Icétôd (see §7). However, Icétôd still has a very small, closed group of prepositions that somehow have survived the hegemony of case. Nonetheless, they interact closely with case as each \isi{preposition} selects the case that its noun head (or host) must take. \tabref{tab:morph:prep} presents all the known Icétôd prepositions with their meanings and the cases they require on nouns:


\begin{table}
\caption{Icétôd prepositions}
\label{tab:morph:prep}


\begin{tabularx}{\textwidth}{XXX}
\lsptoprule

Preposition & Meaning & Case required\\
\midrule
nàpèì & ‘from, since’ & \textsc{ablative}\\
ɗ{\Í}tá & ‘as, like’ & \textsc{genitive}\\
n\'{ɛ}\'{ɛ} & ‘from, through’ & \textsc{genitive}\\
akán{\Í} & ‘until, up to’ & \textsc{oblique}\\
àk{\Ì}l\`{ɔ} & ‘instead of’ & \textsc{oblique}\\
gònè & ‘until, up to’ & \textsc{oblique}\\
ikóteré & ‘because of’ & \textsc{oblique}\\
ńdà & ‘and, with’ & \textsc{oblique}\\
pákà & ‘until, up to’ & \textsc{oblique}\\
tònì & ‘even’ & \textsc{oblique}\\
\lspbottomrule
\end{tabularx}
\end{table}
The following example sentences \REF{ex:morph:1}-\REF{ex:morph:8} offer an opportunity to see the prepositions from \tabref{tab:morph:prep} in a variety of natural language contexts:


 
\ea\label{ex:morph:1}
\gll napei Kaaɓ\'{ɔ}ŋʉɔ     \textbf{páka}   awᵃ  \\
from   Kaabong:\textsc{abl}   up.to   home:\textsc{obl}    \\
\glt ‘from Kaabong up to home’ 
\z

\ea\label{ex:morph:2}
\gll Gógese   tufúlá       \textbf{ɗ{\Í}tá}   rié. \\
peg:\textsc{pass}   field.rat:\textsc{nom}   like   goat:\textsc{gen}    \\
\glt ‘And the field rat is pegged up like a goat.’ 
\z





\ea\label{ex:morph:3}
\gll Atsía     \textbf{n\'{ɛ}\'{ɛ}}   Tímuaƙwɛɛ  nɛ. \\
come:\textsc{1sg}   from   Timu:inside:\textsc{gen}  that    \\
\glt ‘I’m coming from within Timu there.’ 
\z




\ea\label{ex:morph:4}
\gll Hoɗuƙot{\ᵉ},   \textbf{akɨlɔ}     cɛ\'{ɛ}s\'{ʉ}ƙɔt\ᶤ. \\
set.free:\textsc{imp}   instead.of   killing:\textsc{obl}    \\
\glt ‘Set (him) free instead of killing (him).’ 
\z




\ea\label{ex:morph:5}
\gll Duƙotuo   \textbf{gone}   hoo     déé. \\
take:\textsc{seq}   up.to   hut:\textsc{obl}   that    \\
\glt ‘And she took (it) up to that hut.’ 
\z




\ea\label{ex:morph:6}
\gll Ƙáátaa   Tábayɛɛ   \textbf{ikóteré}   ɲ\`{ɛ}ƙ{\ᵋ}. \\
go:\textsc{3pl:prf}   West:\textsc{dat}   because.of   hunger:\textsc{obl}    \\
\glt ‘They’ve gone west because of hunger.’ 
\z




\ea\label{ex:morph:7}
\gll tɛwɛɛsa     kɔlɨl{\Í}\'{ɛ}       \textbf{ńda}   lomuƙeⁱ \\
sow:\textsc{inf:nom}   cucumber:\textsc{gen}   and   squash:\textsc{obl}   \\
\glt ‘the sowing of cucumber and squash’ 
\z




\ea\label{ex:morph:8}
\gll \textbf{toni}   Pakóíce     ʝɨk,   góƙánɨk\^{ɛ}d{\ᵋ} \\
even  Turkanas:\textsc{obl}   also   seated:\textsc{ips:sim:dp}    \\
\glt ‘even the Turkanas as well, (were) staying there’ 
\z






\subsection{Verbs}\label{sec:3.8}


\textsc{Verbs} comprise the second of Icétôd’s two large open word classes. Like nouns, Icétôd verbs make up approximately 48\% of the lexicon. Verb roots can be short like \textit{ó-} ‘call’, long like \textit{gwɛrɛʝ\'{ɛ}ʝ-} ‘be coarse’, or reduplicated like \textit{diridír-} ‘be sugary’ and \textit{ɨpɨr{\Í}p{\Í}r-} ‘drill’. Verb roots are represented throughout this book with hyphenated forms, indicating that in actual Icétôd speech, any verb must have at least one suffix. That minimal suffix may be a subject-agreement suffix or a tense-aspect-mood (TAM) suffix like an \isi{imperative} or optative. Icétôd verb stems can stand alone as an independent, self-contained clause and can have many suffixes strung together, as in \textit{soƙórítiísínàk\ᵃ} ‘we all have clawed’ and \textit{zeikááƙotinîd{\ᵉ}} ‘and they all grew large there’. Among the many suffixes that can derive nouns from verbs or inflect verbs for different meanings, there are: deverbatives, subject-agreement markers, directionals, the dummy pronominal, modals, aspectuals, voice and valency changers, and adjectivals. All these verb-related topics (and others) are treated more fully later on in §8.




\subsection{Adverbs}\label{sec:3.9}


\textsc{Adverbs} make up a catch-all category of words that modify verbs or whole clauses. The sixty-or-so Icétôd adverbs make up less than 1\% of the total lexicon. They include ‘manner’ adverbs like \textit{hɨ{\Í}ʝ\'{ɔ}} ‘slowly’ and \textit{zùkù} ‘very’, epistemic adverbs like \textit{tsábò} ‘apparently’ and \textit{tsamʉ} ‘of course’, and general adverbs like \textit{ɛɗá} ‘only’ and \textit{naɓó} ‘again’. Other important categories of adverbs are the tense-marking adverbs, \isi{certainty} and contingency markers, and the conditional-hypothetical adverbs. All these types of Icétôd adverbs are discussed later in §9.




\subsection{Ideophones}\label{sec:3.10}


\textsc{Ideophones} form a word class that is characterized by highly expressive words that denote physical phenomena like color, motion, sound, shape, volume, etc. They are often ‘sound-symbolic’ or onomatopoeic. That means that their very sound as they are pronounced evokes the physical perception they signify. For example, the \isi{ideophone} \textit{bùlùƙ\ᵘ} means ‘the sound something makes when dropping into water’, like ‘splashǃ’ or ‘kersplunkǃ’ in English. At present, one hundred forty Icétôd ideophones (1.6\% of total) have been recorded, but there are certainly many more in the language. And they are probably continually created. \tabref{tab:morph:ideo} offers a sample of the colorful variety of Icétôd ideophones on record.


\begin{table}
\caption{Icétôd ideophones}
\label{tab:morph:ideo}


\begin{tabularx}{\textwidth}{XX}
\lsptoprule

Animal sounds & \\
\midrule
bèrrr & ‘baaaǃ’\\
buúù & ‘moooǃ’\\
ƙútú & ‘cluckǃ’\\
% \midrule
\tablevspace
Other sounds & \\
\midrule
ɓɛkɛ & ‘snapǃ’\\
g\`{ʉ}l\`{ʉ}ʝ\`{ʉ} & ‘gulpǃ’\\
pùsù & ‘plopǃ’\\
% \midrule
\tablevspace
Colors & \\
\midrule
pàkì & ‘pure white’\\
tíkí & ‘pitch black’\\
tsònì & ‘blood red’\\
% \midrule
\tablevspace
Attributes & \\
\midrule
ɓa & ‘unliftably heavy’\\
dùù & ‘very deep’\\
ts\`{ɛ}k\`{ɛ} & ‘completely full’\\
\lspbottomrule
\end{tabularx}
\end{table}



\subsection{Interjections}\label{sec:3.11}


Like adverbs, \textsc{interjections} form a bit of a catch-all word class. Interjections include any word that expresses emotions or mental states of any kind, usually outside the grammar of a sentence. The roughly thirty Icétôd interjections that have been recorded make up less than 1\% of the total lexicon. Icétôd interjections may consist of a single word like \textit{aaii} ‘ouchǃ’ or \textit{wúlù} ‘yikesǃ’ or a short phrase like \textit{wika ni} ‘these kids (I tell you)ǃ’ or \textit{t{\Í}ɔ ʝ\'{ɔ}\`{ɔ}} ‘there, there (it’s okay)ǃ’. Several of the other interjections on record are provided in \tabref{tab:morph:interj}.


\begin{table}
\caption{Icétôd interjections}
\label{tab:morph:interj}


\begin{tabularx}{.5\textwidth}{Xl}
\lsptoprule

ee Ɲakuʝᵃ & ‘oh my Godǃ’\\
ee/éé & ‘yeah, yes’\\
hà & ‘whateverǃ’\\
maráŋ & ‘fine, okayǃ’\\
ɲɔto ni & ‘these guys (I tell you)ǃ’\\
ne & ‘here you goǃ’\\
ńtóo(n)dó & ‘nah, no’\\
otí & ‘whoaǃ’\\
wóí & ‘aahhǃ’\\
yóói & ‘uh-huh {\dots} sureǃ’\\
\lspbottomrule
\end{tabularx}
\end{table}



\subsection{Nursery words}\label{sec:3.12}


\textsc{Nursery} \textsc{words} make up a small class of one-word expressions that act as commands or encouragements to babies or toddlers to do something. The ten Icétôd nursery words on record are lain out in \tabref{tab:morph:nurs} with English glosses:


\begin{table}
\caption{Icétôd nursery words}
\label{tab:morph:nurs}


\begin{tabularx}{.66\textwidth}{lll}
\lsptoprule

bubú & ‘nighty-night’ & for going to sleep\\
ɓá & ‘yummy’ & for eating\\
dɪ & ‘poo’ & for defecating\\
dʉʉd\'{ʉ} & ‘sitty-sit’ & for sitting down\\
kó & ‘wa-wa’ & for drinking water\\
kɔk\'{ɔ} & ‘no-no’ & for not touching\\
kukú & ‘up-up’ & for riding on mother’s back\\
kwàà & ‘pee’ & for urinating\\
mamá & ‘yum-yum’ & for eating\\
nʉʉn\'{ʉ} & ‘yum-yum’ & for breastfeeding\\
\lspbottomrule
\end{tabularx}
\end{table}



\subsection{Complementizers}\label{sec:3.13}


\textsc{Complementizers} are words that introduce \isi{reported speech} or thought. For example, in the English sentence ‘She said that she agrees’, the word \textit{that} is the \isi{complementizer} that introduces that reported statement \textit{she agrees}. Icétôd has only two complementizers. One of them, \textit{tòìm\`{ɛ}nà-} ‘that’, is technically a noun and thus belongs in the noun word class. But because of its function, it is dealt with here. The word \textit{tòìm\`{ɛ}nà-}, a compound of the verb \textit{tód-} ‘speak’ and \textit{mɛná-} ‘words’, is used with a variety of speaking and thinking verbs. The second Icétôd \isi{complementizer}, \textit{tàà}, is a probably a derivative of the verb \textit{kʉta} ‘(s)he says’ that has been reduced over time. Even now it is usually used after the verb \textit{k\`{ʉ}t-} ‘say’. Example \REF{ex:morph:9} shows how \textit{tòìm\`{ɛ}nà-} is used in a sentence to introduce the clause \textit{mɨt{\Í}da bɔnán} ‘you are an orphan’. And example \REF{ex:morph:10} shows the \isi{complementizer} \textit{tàà} introducing the clause \textit{iya ɲjíníkiʝa k\'{ɔ}\'{ɔ}kɛ} ‘our land is over there’:




\ea\label{ex:morph:9}
\gll Hyeíá   \textbf{toimɛna}   mɨt{\Í}da   bɔnán. \\
know:\textsc{1sg}   that:\textsc{nom}   be:\textsc{2sg}   orphan:\textsc{obl}    \\
\glt ‘I know that you are an orphan.’ 
\z




\ea\label{ex:morph:10}
\gll Kʉta   \'{ɲ}cie   \textbf{taa}   ia     ɲjíníkiʝa    k\'{ɔ}\'{ɔ}kɛ. \\
say:\textsc{3sg}   I:\textsc{dat}   that   be:\textsc{3sg}  we:land:\textsc{nom} there    \\
\glt ‘He says to me that our land is over there.’ 
\z






\subsection{Connectives}\label{sec:3.14}


\textsc{Connectives} or ‘conjunctions’ are words whose function is to join together other words, phrases, or clauses. If they are \textsc{coordinating} connectives like \textit{ńdà} ‘and’, then they join grammatical units of equal status, like a word to a word, or an independent clause to another independent clause. Whereas if they are \textsc{subordinating} connectives like \textit{na} ‘if’, then they join grammatical units of unequal status, usually a dependent clause to an independent one. Even though their role is to link grammatical units, not all of them come between the units they link. Many come before both, often as the first word in the sentence. Icétôd has roughly eight coordinating connectives and thirty subordinating ones – making up less than 1\% of the lexicon. The coordinating connectives are presented in \tabref{tab:morph:coordconn}, while \tabref{tab:morph:subordconn} offers a sampling of the subordinating connectives:


\begin{table}
\caption{Icétôd coordinating connectives}
\label{tab:morph:coordconn}


\begin{tabularx}{.66\textwidth}{lX}
\lsptoprule

kèɗè & ‘or’\\
k{\Í}ná & ‘and then, so then, then’\\
kòrì & ‘or’\\
kòtò & ‘and, but, so, then, therefore’\\
m{\Í}sɨ {\dots} m{\Í}sɨ {\dots} & ‘either {\dots} or {\dots}’\\
náàtì & ‘and then’\\
naɓó & ‘furthermore, moreover’\\
ńdà & ‘and’\\
\lspbottomrule
\end{tabularx}
\end{table}
The following natural-language examples illustrate three of the more commonly used coordinating connectives: \textit{kèɗè}, \textit{kòtò}, and \textit{ńdà}. In example \REF{ex:morph:11}, the connective \textit{kèɗè} ‘or’ joins two equal constituents, the nouns \textit{Tábayɔɔ} and \textit{Fetíékù}. In \REF{ex:morph:12}, the connective \textit{kòtò} ‘and, but, then,’ links two independent but semantically related clauses, and in \REF{ex:morph:13}, the connective \textit{ńdà} ‘and’ connects two \isi{passive} clauses:




\ea\label{ex:morph:11}
\gll Tábayɔɔ   \textbf{keɗe}   Fetíékù? \\
West:\textsc{abl}   or   East:\textsc{abl}    \\
\glt ‘From the West or from the East?’ 
\z



\ea\label{ex:morph:12}
\gll Ɨ\'{ʉ}mʉƙɔtɨakôd{\ᵉ}, moo     \textbf{koto}   sáɓánɨ   ínw{\ᵃ}  \\
marry.forcibly:\textsc{1sg:seq:dp}  not:\textsc{seq}   but   kill:\textsc{ips}   animal:\textsc{nom}  \\
\glt ‘And from there I took (her) away as my wife, but no animal was killed.’  
\z



\ea\label{ex:morph:13}
\gll Sáɓese   basaúr   \textbf{ńda}   kotsana   cue. \\
kill:\textsc{sps}  eland:\textsc{nom}   and   fetch:\textsc{ips} water:\textsc{nom}    \\
\glt ‘Elands were killed, and water was fetched.’ 
\z

In contrast to the coordinating connectives shown in \tabref{tab:morph:coordconn} and examples \REF{ex:morph:11}-\REF{ex:morph:13}, \textit{sub}ordinating connectives join units of unequal status, usually a subordinate (dependent) clause to a main one. \tabref{tab:morph:subordconn} provides a sample of the thirty Icétôd subordinating connectives, while examples \REF{ex:morph:14}-\REF{ex:morph:16} below illustrate the function of some of these connectives in a few natural-language environments.


\begin{table}
\caption{Icétôd subordinating connectives}
\label{tab:morph:subordconn}


\begin{tabularx}{.66\textwidth}{XX}
\lsptoprule

átà & ‘even (if)’\\
ɗ\`{ɛ}m\`{ʉ}s\`{ʉ} & ‘before, unless, until’\\
ikóteré & ‘because’\\
kán{\Ì} & ‘in order that, so that’\\
m{\Í}s{\Ì} & ‘if, whether’\\
na= & ‘if, when’\\
náà & ‘when (earlier today)’\\
nàpèì & ‘since’\\
n\'{ɛ}\'{ɛ} & ‘if, when’\\
nòò & ‘when (long ago)’\\
nótsò & ‘when (a while ago)’\\
pákà & ‘until’\\
s{\Ì}nà & ‘when (yester-)’\\
tònì & ‘even’\\
\lspbottomrule
\end{tabularx}
\end{table}
In example \REF{ex:morph:14} below, the subordinating connective \textit{ɗ\`{ɛ}m\`{ʉ}s\`{ʉ}} ‘before, unless, until’ introduces a dependent clause that connects semantically to the following independent one. The same grammatical structure is also evident in \REF{ex:morph:15} and \REF{ex:morph:16}, where the connectives \textit{m{\Í}s{\Ì}} ‘if, whether’ and \textit{na} ‘if, when’ set off short dependent clauses that logically lead into main clauses that follow them:



\ea\label{ex:morph:14}
  \ea
  \gll \textbf{Ɗ{ɛmʉsʉ}}   Pakóíce     deti   riékᵃ, \\
before   Turkanas:\textsc{obl}   bring   goats:\textsc{acc}    \\ 
  \glt ‘Before the Turkanas brought goats,
  \medskip
  \ex
  \gll isio     noo   ŋábìàn? \\
  what:\textsc{cop}   \textsc{pst3}   wear:\textsc{plur:ips}    \\
  \glt  what was typically worn?’ 
  \z
\z




\ea\label{ex:morph:15}
\gll \textbf{M{\Í}sɨ}   ɨtáána   basaúrék{\ᵉ},   sáɓes. \\
if   reach:\textsc{ips}   eland:\textsc{dat}   kill:\textsc{sps}   \\
\glt ‘If they reach the eland, it is killed.’ 
\z




\ea\label{ex:morph:16}
\gll \textbf{Na}   átsik{\ᵉ},       z\'{ɛ}ƙw\'{ɛ}tɔɔ   nayéé   na. \\
 when   come:\textsc{3sg:sim}   sit:\textsc{3sg:seq}   here   this   \\
\glt ‘When she came, she sat down here.’ 
\z




