\addchap{Acknowledgements}

Compiling a dictionary such as this one is a massive undertaking, far more so than I ever imagined it would be. Although I myself spent many hours, days, and months working alone on this project, a whole host of people put me in a position to do so. And it is here that I wish to acknowledge and thank them all.

First, I want to express a heartfelt \textit{Ɨ}\textit{lákásʉƙɔt{\Í}àà zùk\ᵘ} to all the Ik people of the Timu Forest area for welcoming us into their community and patiently putting up with the long process of a foreigner trying to learn their language. To the following Ik men and women, I give thanks for their participation in a word-collecting workshop that took place in October 2009, during which roughly 7,000 Ik words were amassed: Ariko Hillary, Kunume Cecilia, Lochul Jacob, Lokure Jacob, Longoli Philip, Losike Peter, Lotuk Vincent, Nakiru Rose, Nangoli Esther, †Ngiriko Fideli, Ngoya Joseph, Ochen Simon Peter, Sire Hillary, and Teko Gabriel. 

A second group of Ik men are sincerely thanked for giving me a clearer view of the Ik sound system and for helping me edit several hundred words during an \isi{orthography} workshop in April 2014: Amida Zachary, Dakae Sipriano, Lokauwa Simon, Lokwameri Sylvester, Lomeri John Mark, Longoli Philip, Longoli Simon, Lopeyok Simon, †Lopuwa Paul, and Lotuk Paul. One of those men, Longoli Philip, deserves special thanks for the years he spent as my main guide into the grammar and lexicon of his mother tongue. The number and quality of entries in this book are owed in large part to his skillful labors. Four other men – Lojere Philip, †Lochiyo Gabriel, Lokwang Hillary, and †Lopuwa Paul – also deserve my thanks for teaching me bits of the language at various points along the journey.

But it is another group of Ik men that I wish to give special honor. These are the ones who for an entire year went with me through every word in this dictionary to refine their spellings and define their meanings. They include the respectable elders Iuɗa Lokauwa, Locham Gabriel, and Lemu Simon, as well as our translators Kali Clement, Lotengan Emmanuel, and Lopeyok Simon. The three elders not only shared their intimate knowledge of the language with me but also befriended me with a grace and humility that can only come with age. Every moment I spent with them was a blessing I will never forget. As they said, if I ever come back, I should ask if those old men are still around. I pray they are.

Although teaching foreigners Icétôd has traditionally been the domain of men, I wish to bring special attention to two Ik women who, through their resilient friendship and lively conversation, greatly enhanced my grasp of the language. These are the highly esteemed Nachem Esther and Nakiru ‘Akóóro’ Rose.

Next, I want to gratefully mention those in the long line of linguists who worked on Icétôd and – in person or publication – passed their knowledge down to me: Fr. J. P. Crazzolara who wrote the first recorded grammatical description of the language; A. N. Tucker whose series of articles on Icétôd expanded my knowledge considerably; Fritz Serzisko who penned several insightful articles and books on Icétôd and Kuliak; Bernd Heine who wrote numerous works on Icétôd and Kuliak and authored a grammar sketch and dictionary of the language (1999); Richard Hoffman who studied the grammar and lexicon, devised an \isi{orthography}, and tirelessly supported language development efforts on behalf of the Ik; Christa K\"{o}nig who wrote several articles and an entire book on the Icétôd case system; Ron Moe who helped me lead a word-collection workshop; Keith Snider who trained me in tone analysis; Kate Schell who collected dozens of hours of recorded Ik texts; and Dusty Hill who supervised me.

Another group I wish to thank are our friends and family members whose generous and faithful donations have made it possible for us to live and work in Uganda since 2008. It has been a privilege to be financially supported in doing long-term work on the Ik language, and I do not take that for granted. 

My sincere thanks also go the reviewers and proofreaders who helped me improve this manuscript, to Monika Feinen for drawing up a lovely map of Ikland (\figref{fig:1}), and lastly to Sebastian Nordhoff, whose \isi{patient} help and technical expertise in manuscript preparation I could never have done without.

I also want to thank my dear family: my two adopted Ik daughters, Kaloyang Mercy and Lemu Immaculate, and my wife Amber Dawn. Their loving presence enabled me to carry out this long work in an otherwise isolated and often very lonely environment. The existence of this book is owed in large measure to Amber’s innumerable sacrifices big and small. It came into being at great cost to her. For that and many other reasons, I thank her from the bottom of my heart.

Lastly, I want to praise the God whose Word became flesh
-- ὁ λόγος σὰρξ ἐγένετο --
making a linguistic cosmos where my mind and the Ik language could collide and radiate bright rays of new knowledge out into the world.
