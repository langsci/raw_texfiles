\chapter{Tone assignment rules and the division of utterances into tone groups}
\label{chap:toneassignmentrulesandthedivisionoftheutteranceintotonegroups}

\is{tone group|(}

The present chapter sets out and discusses (i)~the rules of tone assignment, {\linebreak}whereby
\is{form!surface}surface phonological tones can be derived from the \is{form!underlying}underlying tones, and (ii)~the principles underlying the division of the utterance into tone groups, a~key unit for tonal processes. One of the first readers of this book (Boyd Michailovsky, who went through the first draft in 2012) pointed out that, to him, the really engaging part of the study was in this chapter, where the archipelago of tables of the earlier chapters gives way to linguistic analysis in the full sense: conveying a~feel for the \is{stylistics}stylistic choices open to the speakers (\sectref{sec:thedivisionofutterancesintotonegroups}"=\ref{sec:casesofbreachoftonalgroupingandconsequencesforthesystem}), and thereby shedding light on Yongning Na morphotonology in use. 

The unit within which tonal processes apply in Yongning Na is
referred to here as a~\is{tone group|textbf}\textit{tone group}. Its boundaries are indicated in transcriptions by means of the International Phonetic
Alphabet symbol ‘\ipa{|}’. Successive tone groups are entirely independent from the point of view of
their phonological tones: tones never spread or otherwise influence one another across tone"=group
boundaries, i.e.\ tonal computation takes place independently for
successive tone groups.\footnote{From a~phonetic point of view, successive tone groups are linked to one another through a~variety of phenomena, from the most local~-- such as tonal {coarticulation}~-- to the most global: in particular, \textit{declination} at the level of utterances and of higher"=level discourse units. But these intonational
phenomena, discussed in Chapter~\ref{chap:fromsurfacephonologicalformstophoneticrealizationintonationandtonalimplementation}, do not affect the phonological tonal string of the utterance: conceptually, {intonation} needs to be
distinguished from the processes whereby the surface phonological string of a~tone group
obtains.} 

In a~model of prosodic hierarchy such as the universal model proposed by \citet{selkirk1986} and
\citet{nesporetal1986}, made up of utterance phrase ⊃
%{$\supset$} 
intonational phrase ⊃ phonological phrase ⊃ phonological word ⊃
      foot ⊃ syllable ⊃ mora, tone groups may be considered as
      constituting one \textit{phonological phrase} each. The term “tone group” is
      nonetheless used here in preference to “phonological phrase”, for several reasons. First and foremost, the defining characteristic of this phonological unit is that it serves as the domain of tonal processes. Na does not have segmental rules
      such as the lenition of word"=medial consonants observed in other languages of the area, such as Qiang
      (\citealt[31–32]{lapollaetal2003a}) and \ili{Shixing} \citep[12–13]{chirkova2009}, which provide evidence
      for the phonological word as a~prosodic domain. Another reason for favouring the label “tone group” over “phonological phrase” is that, in Yongning Na, two levels are plausible candidates for identification with the level of the “phonological phrase”: the tone group, and the tonal phrase, discussed further below.

 In Yongning Na, the tone group is the highest unit for tonal computation, and the syllable is the smallest unit: the tone"=bearing unit at the surface phonological level. In between these two levels, I~propose to distinguish the following additional levels:
 
 \begin{itemize}
 	\item{the lexical word, to which tone categories are lexically associated;}
 	\item{the tonal word: a~combination of lexical words, such as noun plus verb in S+V or O+V
 		combinations, and noun plus noun in compounds;}
 	\item{the tonal phrase: a~tonal word plus any added clitics and affixes.}
 \end{itemize}

Using the lexical word //\ipa{kʰv̩˩mi˩}// ‘dog' as an example, a~tonal word is shown in (\ref{ex:hitdog}): the object"=plus"=verb combination ‘to hit a~dog'. A~tonal phrase is shown in (\ref{ex:hitdogPFV}): it consists of the tonal word (\ref{ex:hitdog}) plus the {perfective} suffix //\ipa{-ze˧\textsubscript{b}}//.

\begin{exe}
	\ex
	\label{ex:hitdog}
	\ipaex{kʰv̩˩mi˩ ti˥}\\
	\gll kʰv̩˩mi˩				ti˩\textsubscript{a}\\
	dog		to\_tap/to\_hit\_gently\\
	\glt ‘to hit a~dog (gently)'
\end{exe}

\begin{exe}
	\ex
	\label{ex:hitdogPFV}
	\ipaex{kʰv̩˩mi˩ ti˥-ze˩}\\
	\gll kʰv̩˩mi˩		ti˩\textsubscript{a}		-ze˧\textsubscript{b}\\
	dog		to\_tap/to\_hit\_gently			\textsc{pfv}\\
	\glt ‘has hit a~dog (gently)'
\end{exe}

A~different possibility would be to use “prosodic stem” instead of \textit{tonal word}, and “prosodic word” instead of \textit{tonal phrase}. This would clear the way for use of “phonological phrase” instead of \textit{tone group}. But an issue is that the tonal word as characterized here tends to include more material than the ``prosodic stem'', which ``usually coincides with the morphological stem'' (\citealt[17]{vandevelde2008b}; see also \citealt{downing2015}). Therefore, the terms “prosodic stem”, “prosodic word” and “phonological phrase” are not used in the present volume. It is hoped that the description proposed here in terms of \textit{lexical words}, \textit{tonal words}, \textit{tonal phrases} and \textit{tone groups} is explicit enough to be fully intelligible and translatable, as it were, into various theoretical frameworks.

This chapter starts out from the phonological part of the picture: the tone"=to"=syllable association rules (\sectref{sec:asummaryoftonetosyllableassociationrules}). The discussion then moves on to the topic of the division of the utterance into tone groups, which links up with \is{stylistics}stylistic issues (\sectref{sec:thedivisionofutterancesintotonegroups}).

\section{A summary of tone"=to"=syllable association rules}
\label{sec:asummaryoftonetosyllableassociationrules}

\is{tone rules|textbf}

This section summarizes the tone"=to"=syllable association rules of Yongning Na. The description is based on the notion of \is{derivation!tonal|textbf}derivation, from an~\is{form!underlying|textbf}underlying level to
a~surface phonological\is{form!surface|textbf} level.\footnote{The notion of derivation from underlying representations to surface phonological representations is a~great help in phonological analysis, to the point that it may not seem to require special justification. But this notion has come under criticism from various quarters: in the view of some authors, instead of derivation, it is more satisfactory to reason in terms of sets of (surface) forms that stand in close relationship with one another. \citet{hyman2015} concludes from his review of this issue that underlying representations should be maintained as a~central tool of phonological analysis. Of course this does not mean that underlying representations constitute an adequate tool for exploring all the questions that linguists may be interested in raising, or that they constitute the final word in phonological description. The ultimate aim is to approach the actual processes taking place in the speaker’s brain. When formulating generalizations, such as the tone rules proposed below, efforts were made to keep in mind issues of psychological (cognitive) plausibility. Mid- and long"=term perspectives for modelling with more elaborate tools are briefly mentioned in this volume's conclusion (Chapter~\ref{chap:conclusion}).} For instance, the LM and LH lexical tone categories are underlyingly
distinct, but when a~word is spoken \is{form!in isolation}in isolation, the opposition of LM and LH is neutralized at the
surface phonological level due to a~contextual \isi{neutralization} of the M and H levels: M and H do not contrast with each other in contexts
where they appear in tone"=group"=final position and are preceded by a~L tone. In places
where there is potential for confusion between underlying phonological level and surface
phonological level, double slashes are used for transcriptions at the //underlying\is{form!underlying} phonological
level//, as against simple slashes for the /surface\is{form!surface} phonological level/. This is visually
unattractive, but desperate tones call for desperate measures.

 
\subsection{The phonological tone rules}
\label{sec:alistoftonerules}

\is{tone rules|textbf}

The tone"=to"=syllable association rules yield the surface phonological tones of a~given tone
group, on the basis of its underlying tones. Seven phonological tone rules have been found to apply in Yongning Na. For ease of reference, a~list is provided first, before the analysis and discussion.

	\begin{enumerate}[leftmargin=2cm, itemsep=0pt, labelwidth=\widthof{Rule~1:}]%[topsep=12pt, partopsep=0pt]
		\item[Rule~1:] L tone spreads progressively (“left"=to"=right”) onto syllables that are unspecified for tone.
		\item[Rule~2:] Syllables that remain unspecified for tone after the application of Rule 1 receive M tone.
		\item[Rule~3:] In tone"=group"=initial position, H and M are neutralized to M.
		\item[Rule~4:] The syllable following a~H-tone syllable receives L tone.
		\item[Rule~5:] All syllables following a~H.L or M.L sequence receive L tone.
		\item[Rule~6:] In tone"=group"=final position, H and M are neutralized to H if they follow a~L tone.
		\item[Rule~7:] If a~tone group only contains L tones, a~post"=lexical H tone is added to its last syllable.
	\end{enumerate}

%%Version below: flush left at margin, with broad vertical spacing between lines.
%\begin{enumerate}[leftmargin=!,labelwidth=\widthof{Rule~1:}]
%	\item[Rule~1:] L tone spreads progressively (‘left"=to"=right’) onto syllables that are unspecified for tone.
%	\item[Rule~2:] Syllables that remain unspecified for tone after the application of Rule 1 receive M tone.
%	\item[Rule~3:] In tone"=group"=initial position, H and M are neutralized to M.
%	\item[Rule~4:] The syllable following a~H-tone syllable receives L tone.
%	\item[Rule~5:] All syllables following a~H.L or M.L sequence receive L tone.
%	\item[Rule~6:] In tone"=group"=final position, H and M are neutralized to H if they follow a~L tone.
%	\item[Rule~7:] If a~tone group only contains L tones, a~post"=lexical H tone is added to its last syllable.
%\end{enumerate}

The following paragraphs explain the motivation for positing these seven {\linebreak}phonological rules.

The tone association rules for each of the lexical tone categories have been set out in the course of Chapter~\ref{chap:thelexicaltonesofnouns}, e.g.~describing in \sectref{sec:afloatinghtonewithcomparativeevidencepointingtoitsorigin} and \sectref{sec:wordfinalandmorphologicalnucleusfinalHtones} the association of tone to syllables for the three types of High tones:
H\#, \#H and H\$. The same rules hold for the tones of more complex entities \textit{(tonal words)} such as \is{compounds}compound
nouns or \is{numerals}numeral"=plus"=classifier phrases. Unless otherwise specified, the tone pattern associates to the tonal word syllable by syllable
(“left"=to"=right”), one tone level after another. When there are fewer syllables than tone levels,
two levels associate to the last syllable.

Thus, L tone and M tone associate to the first syllable of the tonal word. For LM tone, the first
syllable receives L, and the second receives M. For LH, the first syllable likewise receives L, and
the second receives H. These four tonal categories (L, M, LM and LH) are the simplest in terms of
tone"=to"=syllable association. The other tone categories (\#H, MH\#, H\$, L\#, LM+MH\#, LM+\#H, and
H\#) all have a~specific syllabic \is{anchorage}anchoring, described in \sectref{sec:aphonologicalanalysisofthetonecategoriesofnouns}, and reflected in the special symbols used in the present
transcription. 

In the case of the mixed tone categories LM+MH\# and LM+\#H, their first part (LM) attaches to
syllables following the usual rules (like a~simple LM tone) and their second part (MH\#
or \#H, respectively) attaches as indicated by the added diacritic. Thus, for LM+MH\# tone, the
first syllable receives L, the second receives M, and the last receives a~MH \is{tonal contour}contour. In the special
case where only two syllables are available, the MH \is{tonal contour}contour associates to the second syllable,
overriding its M tone. In the case of tone category LM+\#H, the first syllable likewise receives a~L
tone and the second a~M tone; the H level associates to the first syllable following the tonal
phrase, if a~suitable carrier is available. (These two tone categories never associate to
a~\is{monosyllables}monosyllable.)

At this point in the tone"=to"=syllable mapping process, some syllables remain toneless. For instance,
the lexical disyllable //\ipa{dɑ.ʝi˩}// ‘mule’, the \is{compounds}compound //\ipa{po.{\allowbreak}lo-ɬi.pi˩}// ‘ram’s ear’, the
\is{numerals}numeral"=plus"=classifier //\ipa{gv̩-ʂɯ˩}// ‘nine times’ and the object"=plus"=verb combination
//\ipa{ɖɤ.mi ʑi˩}// ‘to grab a~fox’, all of which have L\# tone, are only specified for tone on their
last syllable. Tonal nuclei carrying L tone constitute a mirror image of this situation: they are only specified for tone on their \textit{first} syllable. For instance, the noun //\ipa{v̩˩dze}// ‘bird’, the \is{compounds}compound //\ipa{kʰv̩˩mi"=hṽ̩}// ‘dog’s hair’, the \is{numerals}numeral"=plus"=classifier
//\ipa{so˩-dze}// ‘three pairs’, and the object"=plus"=verb combination //\ipa{li˩ tɕʰi}// ‘to sell
tea’ all carry L tone, which attaches to their first syllable. Their other syllables receive a~surface phonological tone through the phonological rules set out below.

First, L tone spreads, and toneless syllables receive M tone by default.

\begin{enumerate}[leftmargin=2cm, itemsep=0pt, labelwidth=\widthof{Rule~1:}]%[topsep=12pt, partopsep=0pt]
%\begin{enumerate}[leftmargin=!,labelwidth=\widthof{Rule~1:}]
\item[Rule~1:] L tone spreads progressively (“left"=to"=right”) onto syllables that are unspecified for tone.
\item[Rule~2:] Syllables that remain unspecified for tone after the application of Rule 1 receive M tone.
\end{enumerate}

The {phrasing} of Rule 2 makes it explicit that these rules need to be ordered: if Rule 2 applied
before Rule 1, there would be no tonally unspecified syllables left for L tone to spread over.

Taking up the above examples, Rule~1 yields the following results:

\begin{enumerate}[itemsep=-1mm]
	\item[] //\ipa{v̩˩dze}// → /\ipa{v̩˩dze˩}/ ‘bird’ 
	\item[] //\ipa{kʰv̩˩mi"=hṽ̩}// → /\ipa{kʰv̩˩mi˩-hṽ̩˩}/ ‘dog’s hair’
	\item[] //\ipa{so˩-dze}// → /\ipa{so˩-dze˩}/ ‘three
	pairs’ 
	\item[] //\ipa{li˩ tɕʰi}// → /\ipa{li˩ tɕʰi˩}/ ‘to sell tea’
\end{enumerate}

{\largerpage}

Application of Rule~2 yields: 

\begin{enumerate}[itemsep=-1mm]
	\item[] //\ipa{dɑ.ʝi˩}// → /\ipa{dɑ˧ʝi˩}/ ‘mule’
	\item[] //\ipa{po.lo-ɬi.pi˩}// → /\ipa{po˧lo˧-ɬi˧pi˩}/ ‘ram’s
	ear’
	\item[] //\ipa{gv̩-ʂɯ˩}// → /\ipa{gv̩˧-ʂɯ˩}/ ‘nine times’
	\item[] //\ipa{ɖɤ.mi ʑi˩}// → /\ipa{ɖɤ˧mi˧ ʑi˩}/ ‘to
	grab a~fox’
\end{enumerate}

%Application of Rule~1 yields:\\
%//\ipa{v̩˩dze}// > /\ipa{v̩˩dze˩}/ ‘bird’~~~~~~~~~~~~~~~~~~~~//\ipa{kʰv̩˩mi"=hṽ̩}// > /\ipa{kʰv̩˩mi˩-hṽ̩˩}/ ‘dog’s hair’\\
%//\ipa{so˩-dze}// > /\ipa{so˩-dze˩}/ ‘three
%pairs’~~~{\kern1pt}//\ipa{li˩ tɕʰi}// > /\ipa{li˩ tɕʰi˩}/ ‘to sell tea’
%
%Application of Rule~2 yields:\\
%//\ipa{dɑ.ʝi˩}// > /\ipa{dɑ˧ʝi˩}/ ‘mule’~~~~~~~~~~~~~~~~~~//\ipa{po.lo-ɬi.pi˩}// > /\ipa{po˧lo˧-ɬi˧pi˩}/ ‘ram’s
%ear’\\
%//\ipa{gv̩-ʂɯ˩}// > /\ipa{gv̩˧-ʂɯ˩}/ ‘nine times’~~~{\kern1pt}//\ipa{ɖɤ.mi ʑi˩}// > /\ipa{ɖɤ˧mi˧ ʑi˩}/ ‘to
%grab a~fox’

After application of Rules~1 and 2, there remain no toneless syllables. (In Yongning Na, it is an exceptionless observation that each syllable carries tone at the surface phonological level.) The rules that apply next refer to the boundaries of the tone group: tonal oppositions are neutralized in certain positions within the tone group (Rules 3-6), and a~repair rule adds a~H tone on the last syllable of the tone group if the whole group only contains L tones (Rule 7).

\begin{enumerate}[leftmargin=2cm, itemsep=0pt, labelwidth=\widthof{Rule~1:}]%[topsep=12pt, partopsep=0pt]
%\begin{enumerate}[leftmargin=!,labelwidth=\widthof{Rule~1:}]
	\item[Rule~3:] In tone"=group"=initial position, H and M are neutralized to M.
	\item[Rule~4:] The syllable following a~H-tone syllable receives L tone.
	\item[Rule~5:] All syllables following a~H.L or M.L sequence receive L tone.
	\item[Rule~6:] In tone"=group"=final position, H and M are neutralized to H if they follow a~L tone.
	\item[Rule~7:] If a~tone group only contains L tones, a~post"=lexical H tone is added to its last syllable.
\end{enumerate}

Rule 6 is proposed on the basis of the observation that there is no opposition between H and M on a~tone"=group"=final syllable when the
syllables that precede it carry L tone. This also applies to contours: LH and LM contours are neutralized to LH in
tone"=group"=final position. Thus, a~tone"=group"=final syllable following a~L-tone syllable can only
have one of the following tones at the surface phonological level: L, H, LH, or MH. There is no opposition between L...L.LH and
L...L.LM, any more than between L...L.H and L...L.M. In transcriptions, it appeared advisable to adhere to the principle of
providing a~\is{form!surface}surface transcription of tone, with no more tonal oppositions than are really present at
the phonological surface. This required choosing one of two alternatives: transcribing the product of \isi{neutralization} of M and H as M, or as H. A phonological reason for transcribing the product of the \isi{neutralization} of H and M in this context as H (rather than M) was that it appeared more appropriate to represent a~two"=term opposition by
means of the two extreme values of the tone scale (L and H). 

An annoying consequence is that, in transcriptions,
the same word can appear with a~L.M pattern in certain positions, and with L.H in others, even though from a~phonetic point of view the realizations may be indistinguishable. The paradox is carried to an extreme in (\ref{ex:mushcome}).

\begin{exe}
	\ex
	\label{ex:mushcome}
	\ipaex{mo˩kv̩˧ tʰv̩˧-kv̩˩-ze˩. {\kern2pt}|{\kern2pt} mo˩kv̩˥!}\\
	\gll mo˩kv̩\#˥	tʰv̩˧\textsubscript{a}	-kv̩˧˥	 -ze˧\textsubscript{b}		mo˩kv̩\#˥\\
	meadow\_mushroom	to\_grow	\textsc{abilitive}	\textsc{pfv}	meadow\_mushroom\\
	\glt ‘[Starting in the third month,] meadow mushrooms grow. Meadow mushrooms!’ (Mushrooms.149)
\end{exe}

In example (\ref{ex:mushcome}), the second occurrence of the word //\ipa{mo˩kv̩\#˥}// is transcribed as /\ipa{mo˩kv̩˥}/, with a~H tone on the last syllable, because of its tone"=group"=final position, whereas the first occurrence is transcribed as /\ipa{mo˩kv̩˧}/. Phonetically, on the other hand, the second occurrence is realized \textit{lower} than the first, due to utterance"=final lowering.\footnote{For a~discussion of intonational factors that come into play in the phonetic realization of tones, see Chapter~\ref{chap:fromsurfacephonologicalformstophoneticrealizationintonationandtonalimplementation}.} I can't say that I am perfectly happy with this transcription, but it is principled and structured, and the result is unequivocal. Readers are invited to think of more elegant solutions. 

Rule~7 (“If a~tone group only contains L tones, a~post"=lexical H tone is added to its last syllable”) is proposed on the basis of the observation that there are no tone groups that carry no other tone than L. It is through application of Rule~7 that L-tone expressions receive a~final rising \is{tonal contour}contour when spoken \is{form!in isolation}in isolation: when the post"=lexical H tone is added to a~syllable that carries a~L tone, this results in a~LH \is{tonal contour}contour on that syllable. Taking up the same examples as
above, application of Rule~7 is as follows: 

\begin{enumerate}[itemsep=-1mm]
	\item[] /\ipa{v̩˩dze˩}/ → /\ipa{v̩˩dze˩˥}/ ‘bird’ 
	\item[] /\ipa{kʰv̩˩mi˩"=hṽ̩˩}/ → /\ipa{kʰv̩˩mi˩-hṽ̩˩˥}/ ‘dog’s hair’
	\item[] /\ipa{so˩-dze˩}/ → /\ipa{so˩-dze˩˥}/ ‘three
	pairs’ 
	\item[] /\ipa{li˩ tɕʰi˩}/ → /\ipa{li˩ tɕʰi˩˥}/ ‘to sell tea’\footnote{At one point, the added tone was analyzed as M \citep{michaud2008c}. The choice between M and H may seem to be
		a~non"=issue, since LM and LH are neutralized in this position. However, the M tone in Yongning Na is not
		phonologically active: it does not spread, float or otherwise reassociate. A~tone rule\is{tone rules} such as that which affects all-L tone groups is therefore much more likely to involve the H
		tone than the M tone. In two"=tone systems, addition of a~final H tone in domains having only L tone is common: it is attested in Lhasa {Tibetan} \citep[498-499]{sun1997}, {Japanese} \citep[19]{haraguchi1999}, the {Bantu} languages Matengo and Kimatuumbi \citep[415]{odden2005}, and
		{Shixing} \citep{chirkovaetal2009}. In a~three"=tone system, describing the added tone as M could cause confusion for cross"=language comparisons. For these reasons, the post"=lexical tone of Yongning Na is analyzed here as a~H tone.}
\end{enumerate}

\subsection{About the ordering of rules}
\label{sec:abouttheorderingofrules}

As mentioned above, the rules need to be ordered. If Rule 7, which adds a~H tone to all-L sequences,
applied before Rule 2, which assigns M tone to toneless syllables, a~sequence such as ‘has
not come’, made up of the negation \is{prefixes}prefix, \mbox{/\ipa{mɤ˧-}/}, and the verb /\ipa{tsʰɯ˩\textsubscript{a}}/ ‘to come.\textsc{pst}’, would
have only L tones (/\ipa{mɤ-tsʰɯ˩}/) when Rule 7 applied, and would receive a~final H tone
($\ddagger${\kern2pt}\ipa{mɤ-tsʰɯ˩˥}) before receiving a~M tone on its first syllable ($\ddagger${\kern2pt}\ipa{mɤ˧-tsʰɯ˩˥}). Its realization as /\ipa{mɤ˧-tsʰɯ˩}/ illustrates the fact that Rule 7 applies after all the other rules. 

%Rule 2 also applies before Rule 5, as shown by the sequence
%/\ipa{mɤ˧-tsʰɯ˩-sɯ˩}/ ‘has not come yet’, made up of the negation \is{prefixes}prefix, /\ipa{mɤ˧-}/, the verb
%/\ipa{tsʰɯ˩\textsubscript{a}}/ ‘to come.\textsc{pst}’, and /\ipa{sɯ˧}/ ‘yet; first’. The levelling"=down of the M tone of
%/\ipa{sɯ˧}/ is a~result of the application of Rule 5, “All syllables following a~H.L or M.L sequence receive L tone”; at the point where it applies, the negation \is{prefixes}prefix must therefore be supposed to bear a~M
%tone.

Rules 3 and 4 (“H and M are neutralized to M in tone"=group"=initial position”, and “A syllable following
a~H-tone syllable receives L tone”) likewise need to be ordered. If Rule 4 applied first,
an~underlying sequence such as //\ipa{dzɯ˥-bi˧}// ‘will eat’ would have its second syllable lowered
to L, yielding /\ipa{dzɯ˥-bi˩}/, and finally /\ipa{dzɯ˧-bi˩}/, by application of Rule 3. The
observed surface phonological pattern is /\ipa{dzɯ˧-bi˧}/, with M tone on the second
syllable, not L tone.


\subsection{A discussion of alternative formulations}
\label{sec:adiscussionofalternativeformulations}

The generalizations formulated here in the form of Rules~1-7 could also be captured through other formulations. For instance, it may seem simpler to collapse Rules 4 and 5 into one rule to the
effect that “H tone can only be followed by L tones”. By the application of that rule, all syllables
following a~H-tone syllable would receive L tone. But a~further rule would then have to be formulated
specifically for the M.L sequence: “All syllables following a~M.L sequence receive L tone”. The choice
to adopt the present formulation for Rule 5 (“All syllables following a~H.L or M.L sequence receive L tone”) is based on the intuition that the same mechanism is at play in both cases: H.L and M.L are
both stepping"=down sequences, from a~higher tone level to a~lower one. The generalization is that
stepping"=down sequences can only be followed by L tones.

This could also be phrased as a~static constraint: “There can be no trough within a~tone group”, or
“A tone cannot be surrounded by higher tones within a~tone group”. This would rule out sequences
such as $\ddagger${\kern2pt}MLH, $\ddagger${\kern2pt}MLM or $\ddagger${\kern2pt}MHLM. However, such a~static constraint would not provide any information on
how the offending sequences are avoided or repaired in Yongning Na, whereas Rule~5 is explicit on this
point: the tones which would result in such sequences are all lowered to L.

Rule~4 precludes {\dots}H.H{\dots} sequences, so the H tone can be said to be \isi{culminative}. In this
light, the H tone might also be analyzed as a~HL \is{tonal contour}contour. In the early stages of analysis of the tone system of Yongning Na, I~attempted an~account in which the underlying phonological entities were not tones, but steps up and down the three"=level tonal
score: up one level, from L to M or M to H; or down from H or M to L. Proposals
in this vein have been made for \il{Japanese|textbf}Japanese, where the fall from H to L can be considered as one phonological entity (which can be called a~tonal accent), rather than a~succession of two distinct phonological entities \citep[1399]{kubozono2012}. Under
a~tonal account, the tonal accents of \ili{Japanese} dialects have to be represented as a~sequence of two
tones, which is uneconomical. A~“dynamic treatment of tone” was also attempted for Igbo (Niger"=Congo family): such is the
title of Mary Clark’s dissertation \citep{clark1976}. In the case of Yongning Na, if HL were
reinterpreted as a~“high fall”, ML could be called a~“low fall”; LM would be a~“low rise”, and MH a
“high rise”. However, the existence of contours on a~single syllable argues in favour of a~tonal
analysis of the Yongning Na system. The dynamic treatment, which was abandoned for Igbo, does not
appear promising for Yongning Na either. (This is of course not to say that it may not prove useful
for other languages.)

Rule 4 above also precludes {\dots}H.M{\dots} sequences. This can be described as a~\isi{neutralization}
(to H.L) of the contrast between H.M and H.L.


\subsection{Implications for the tones of sentence"=final particles}
\label{sec:implicationsforthetonesofsentenceparticles}

In Yongning Na as in numerous East Asian languages, sentence"=final particles play a~major role in
conveying evidentiality and speaker attitude. In tonal languages, these particles can have lexical tone (e.g.~in \ili{Vietnamese}), but there is a~cross"=linguistic tendency for an evolution towards reduced tonal distinctions or tonelessness (e.g.~in \ili{Mandarin}). The fact that these particles are located at the end of a~sentence results in strong intonational modification. In Yongning Na, sentence"=final position implies tone"=group"=final position; this exerts a~strong influence on the particles' tone patterns, because the tone of the last syllable in a tone group is often determined by the tones that precede. If the tone group contains a~H tone, or a~M.L
sequence, all following tones are lowered to L, through Rules 4 and 5. For instance, the lexical M tone of
final particle //\ipa{mæ˧}// (expressing obviousness) is lowered to L in (\ref{ex:fertil}) because all tones following the M.L
sequence /\ipa{le˧-ʁæ˩}/ are lowered to L by a~{phonological rule} (Rule~5). 

\begin{exe}
	\ex
	\label{ex:fertil}
	\ipaex{ʈʂe˧ ʈʂʰɯ˧ {\kern2pt}|{\kern2pt} le˧-ʁæ˩-ɲi˩ mæ˩.}\\
	\gll ʈʂe˥	ʈʂʰɯ˧	le˧-	ʁæ˩\textsubscript{a}		-ɲi˩		mæ˧\\
	earth	\textsc{top}	\textsc{accomp}		to\_melt/to\_fall\_apart	\textsc{certitude}		\textsc{obviousness}\\
	\glt ‘The clods of earth fall apart / the clods of earth melt [into the water].’ (Agriculture.54)
\end{exe}

Example (\ref{ex:fertil}) illustrates the majority case. But the final particle can also receive a~H tone projected by a~\is{tonal contour}contour
tone lexically attached to the preceding syllable, as in (\ref{ex:wearfelt}), where the lexical M tone of the
affirmative //\ipa{mæ˧}// is replaced by a~H tone (hence /\ipa{mæ˥}/) through reassociation of the H part of the MH \is{tonal contour}contour of the
{{abilitive}} //\ipa{-kv̩˧˥}//.

\begin{exe}
	\ex
	\label{ex:wearfelt}
	\ipaex{ʐæ˩sɯ˩˥ {\kern2pt}|{\kern2pt} -dʑo˩ {\kern2pt}|{\kern2pt} ʈʂʰɯ˧ne˧-ʝi˥ {\kern2pt}|{\kern2pt}
		tʰi˧-mv̩˧-kv̩˧ mæ˥.}\\
	\gll ʐæ˩sɯ˩		dʑo˥		ʈʂʰɯ˧ne˧-ʝi˥	tʰi˧-		mv̩˧\textsubscript{a}		-kv̩˧˥		mæ˧\\
	felt	\textsc{top}	thus		\textsc{dur}		to\_put\_on		\textsc{abilitive}		\textsc{obviousness}\\
	\glt ‘This is how we used to wear felt.’ (Sister3.74)
\end{exe}

It is only after transcribing ten texts, containing more than 150 examples of //\ipa{mæ˧}//, that
this particle was finally observed in a~context where the preceding syllables did not impose a~tone
on it: 
%example (\ref{ex:medicinesstom}).

\begin{exe}
	\ex
	\label{ex:medicinesstom}
	\ipaex{hu˧mi˧-ʈʂʰæ˧ɣɯ˧ ʈʰɯ˧˥ {\kern2pt}|{\kern2pt} le˧-qʰwɤ˧-ze˧-mæ˧!{\kern2pt}|{\kern2pt}}\\
	\gll hu˧mi˥\$		ʈʂʰæ˧ɣɯ\#˥	ʈʰɯ˩\textsubscript{b}	le˧-	qʰwɤ˧\textsubscript{b}	-ze˧\textsubscript{b}		mæ˧\\
	stomach		medicine		to\_drink	\textsc{accomp}		to\_heal		\textsc{pfv}		\textsc{obviousness}\\
	\glt ‘[nowadays, the diseased person] drinks
	medicines for the stomach, and [they] are healed, aren’t they!’ (Healing.66)
\end{exe}

The phrase
/\ipa{le˧-qʰwɤ˧-ze˧}/ ({\textsc{accomp}}-to\_heal-{\textsc{pfv}}) does not contain any \is{tonal contour}contour or
\is{floating tone}floating tone that would associate to a~following syllable, and it does not contain a~H tone,
or a~M.L sequence, which would impose a~L tone on following syllables. The tone carried by the
sentence"=final particle //\ipa{mæ˧}// in this context is M. This makes it clear that the particle does not carry
a~lexical //L//, //H// or \mbox{//MH//} tone, and is to be interpreted as carrying \mbox{//M//} tone.

Likewise, the tone of the reported"=speech particle //\ipa{tsɯ˧˥}// is affected by the tones that
precede it within the tone group in a~vast majority of cases. In a~set of ten narratives, its
realization with MH tone on the surface is observed in only eight cases out of more than three
hundred. 
%As for the affirmative final particle //\ipa{mv̩˧}//, in almost all examples it follows the
%  reported"=speech particle //\ipa{tsɯ˧˥}//, which determines its surface phonological tone.

To determine the lexical tones of sentence"=final particles, elicitation proved a~valuable complement to
observations from recorded texts. Here is an~example: that of the final particle /\ipa{mo˩}/,
conveying invitation. This particle cannot appear right after the verb, in a~$\ddagger${\kern2pt}V+\ipa{mo˩}
construction. Invitation is expressed as (\ref{ex:pleaseV}), exemplified with the verb ‘to eat’ in (\ref{ex:pleaseeat}). 

\begin{exe}
	\ex
	\label{ex:pleaseV}
	\ipaex{ɖɯ˧-V-ɻ̍˥ mo˩}\\
	\gll ɖɯ˧-		V		-ɻ̍˥	mo˩\\
	\textsc{delimitative}		{\textit{target verb}}	\textsc{inchoative}		\textsc{disc.ptcl:invitation}\\
	\glt ‘Please go ahead and V a~little!’
\end{exe}

\begin{exe}
	\ex
	\label{ex:pleaseeat}
	\ipaex{ɖɯ˧-dzɯ˧-ɻ̍˥ mo˩!}\\
	\gll ɖɯ˧-		dzɯ˥		-ɻ̍˥	mo˩\\
	\textsc{delimitative}		to\_eat	\textsc{inchoative}		\textsc{disc.ptcl:invitation}\\
	\glt ‘Please have some/please eat some [of it]!’
\end{exe}

\tabref{tab:thetonepatternsoftheconstruction} shows a~set of elicited data. The particle /\ipa{mo˩}/ carries L tone in all six cases. With verbs with H, L and MH tones, this
L tone carried by the particle derives phonologically from the tonal sequence that precedes. Within
a~tone group, a~syllable following a~H tone can only have L tone (as is the case after H and MH
tones); likewise, a~syllable following a~M.L sequence can only carry L tone. In the case of M tones,
however, the preceding tone sequence (M.M.M) does not impose such a~phonological constraint: it does not rule out any of L, M, H or MH on the final syllable. The L tone observed on the surface is therefore
to be attributed to the lexical specification of the particle, hence analysis as //\ipa{mo˩}//, with lexical L tone.

\begin{table}%[t]
\caption{\label{tab:thetonepatternsoftheconstruction}The tone patterns of the /\ipa{ɖɯ˧-V-ɻ̍˥ mo˩}/ construction.}
\begin{tabularx}{\textwidth}{ l@{\hspace{40pt}} Q Q l@{\hspace{40pt}} }
  \lsptoprule
	tone & example & meaning & \ipa{/ɖɯ˧-V-ɻ̍˥ mo˩/}\\\midrule
	H & \ipa{dzɯ˥} & to eat & \ipa{ɖɯ˧-dzɯ˧-ɻ̍˥ mo˩}\\ 
	M\textsubscript{a} & \ipa{hwæ˧\textsubscript{a}} & to buy & \ipa{ɖɯ˧-hwæ˧-ɻ̍˧ mo˩}\\ 
	M\textsubscript{b} & \ipa{tɕʰi˧\textsubscript{b}} & to sell & \ipa{ɖɯ˧-tɕʰi˧-ɻ̍˧ mo˩}\\ 
	L\textsubscript{a} & \ipa{bæ˩\textsubscript{a}} & to sweep & \ipa{ɖɯ˧-bæ˩-ɻ̍˩ mo˩}\\ 
	L\textsubscript{b} & \ipa{ʐwɤ˩\textsubscript{b}} & to speak & \ipa{ɖɯ˧-ʐwɤ˩-ɻ̍˩ mo˩}\\ 
	MH & \ipa{lɑ˧˥} & to strike & \ipa{ɖɯ˧-lɑ˧-ɻ̍˥ mo˩}\\
   \lspbottomrule
\end{tabularx}
\end{table}





%%%subsec:8-1-3
%\subsection[Illustration: deriving surface tone for isolated words]{A simple illustration: deriving the surface tone pattern of words spoken in isolation}
%\label{sec:asimpleapplicationderivingthesurfacetonepatternofwordsspokeninisolation}
%
%
%A simple illustration of the tone rules consists in deriving the tones of words spoken \is{form!in isolation}in isolation from their underlying tone category. 




\section{The division of utterances into tone groups}
\label{sec:thedivisionofutterancesintotonegroups}

\is{boundary (between tone groups)|textbf}

The division of the utterance into tone groups is a~central part of Na \isi{prosody}. Although there are some general tendencies in the division of utterances into tone groups, and
a~few hard"=and"=fast rules, there are often several possibilities open to the speaker; different
divisions into tone groups have different implications in terms of prominence of the various
components. Prominence (conveying \isi{information structure}) and \isi{phrasing} (reflecting syntactic
structure) interact in the division of an~utterance into tone groups. There is therefore no
one"=to"=one {correspondence} between syntactic structure and the division into tone groups.

Tone groups can have highly different syntactic compositions. A~tone group may consist of a~single syllable: \is{monosyllables}monosyllabic nouns and verbs spoken \is{form!in isolation}in isolation constitute a~tone group on their own. Personal pronouns can associate with other words but often appear on their own, as in (\ref{ex:iwillbuildabridge}).

\begin{exe}
	\ex
	\label{ex:iwillbuildabridge}
	\ipaex{njɤ˧ {\kern2pt}|{\kern2pt} tso˩-bi˩-zo˩-ʝi˩˥.}\\
	\gll njɤ˩	tso˩\textsubscript{a}	bi˧\textsubscript{c}		-zo˧-ʝi˧\\
	1\textsc{sg}	to\_build	to\_go	to\_have\_to\\
	\glt ‘I shall go and build [a bridge].’ (Renaming.13)
\end{exe}

Tone groups are often longer, however: e.g.~\is{compounds}compound noun and \is{numerals}numeral"=plus"=classifier phrase, as in (\ref{ex:themotherandthedaughterthetwoofthem}); noun phrase and affixed verb, as in (\ref{ex:therewasnofood}); or \is{numerals}numeral"=plus"=classifier phrase and affixed verb, as in (\ref{ex:willyoubuildabridge}).
\begin{exe}
  \ex
  \label{ex:themotherandthedaughterthetwoofthem}
  \ipaex{ə˧mi˧-mv̩˩ ɲi˩-kv̩˩}\\
  \gll ə˧mi˧	mv̩˩ 		ɲi˧-kv̩˧˥\\
  mother	daughter	two-\textsc{clf}.persons\\
  \glt ‘the mother and the daughter, the two of them’ (Lake4.93)

  \ex
  \label{ex:therewasnofood}
  \ipaex{dzɯ˧-di˧ mɤ˧-dʑo˧˥}\\
  \gll dzɯ˥	-di	mɤ˧	dʑo˧\\
  to\_eat	\textsc{nmlz}	\textsc{neg}	\textsc{exist}\\
  \glt ‘there was no food’ (Seeds2.69)

  \ex
  \label{ex:willyoubuildabridge}
  \ipaex{dzo˧ {\kern2pt}|{\kern2pt} ɖɯ˧-pɤ˩ tso˩ ə˩-bi˩?}\\
  \gll dzo˩	ɖɯ˧-pɤ˩	tso˩\textsubscript{a} 		ə-˩		-bi˧\\
  bridge	one-\textsc{clf}		to\_build		\textsc{interrog}	\textsc{imm\_fut}\\
  \glt ‘will [you] build a~bridge?’ (Renaming.10)
\end{exe}

The present description starts with the simplest case: that of morphemes which always constitute
a~tone group on their own.


\subsection[Morphemes that constitute a~tone group on their own]{Morphemes that always constitute a~tone group on their own}
\label{sec:someelementsalwaysconstituteatonegroupontheirown}

Some morphemes always constitute a~tone group on their own. They could be referred to as \textit{tonal standalones}. These include the gap"=filler /\ipa{tʰi˩˥}/
‘(and) so, (and) then’; the
contrastive topic marker /\ipa{-no˧˥}/; /\ipa{wɤ˩˥}/ ‘again; also’, which in quite a~few cases does not have its full
lexical meaning and is close to a~simple gap"=filler; and the \is{intensifiers}intensifier /\ipa{ɖwæ˧˥}/ ‘very’. The first three happen to appear in succession in (\ref{ex:Housebuilding144}): /\ipa{tʰi˩˥ | -no˧˥ | wɤ˩˥}/.

\begin{exe}
	\ex
	\label{ex:Housebuilding144}
	\ipaex{tʰi˧-gv̩˩-se˩-dʑo˩ | tʰi˩˥ | -no˧˥ | wɤ˩˥ | qwɤ˧ tʰi˧-gv̩˩.}\\
	\gll tʰi˧-		gv̩˩\textsubscript{a}	-se˩	-dʑo˥	tʰi˩˥	-no˧˥	wɤ˩˥		qwɤ˧	tʰi˧-	gv̩˩\textsubscript{a}\\
	\textsc{dur}		to\_make/to\_build	\textsc{completion}			\textsc{top}	then	\textsc{cntr.top}		again/also	fire\_pit		\textsc{dur}	to\_make/to\_build\\
	\glt ‘After one has finished to build [the cupboard], well, one builds the fire pit!’ (Housebuilding.144)
\end{exe}

One could speculate that /\ipa{tʰi˩˥}/ ‘(and) so, (and) then’ and /\ipa{wɤ˩˥}/ ‘again’ are favoured
as gap"=fillers because of their properties with respect to tone"=group divisions. The gap"=filler
/\ipa{tʰi˩˥}/ appears in most sentences in the narratives told by consultant F4: more than 1,500
occurrences among twenty narratives. The gap"=filler /\ipa{wɤ˩˥}/ appears more than 120 times among the
same twenty narratives. These two items may owe part of their conspicuous success as gap"=fillers to the
phonological fact that they demarcate tone groups clearly. Since they always constitute a~tone group
on their own, they create a~pause in the computation of tone sequences.

But one may just as well hypothesize the inverse causal link: that these words tended to be set off from the
rest of the utterance due to their function as gap"=fillers, eventually resulting in the present
situation where they systematically constitute tone groups on their own. An argument in favour of
this hypothesis is provided by items that are in the process of entering the set of \textit{tonal standalones}. The adverb //\ipa{ɖɯ˧ njɤ˧}// ‘continuously, ceaselessly’ is a~case in point. It was elicited in
association with verbs exemplifying the six tone categories of verbs, yielding the results shown in
\tabref{tab:continuously}. In narratives, however, the adverb is always followed by a~tone"=group \is{boundary (between tone groups)}boundary, as in (\ref{ex:theelderswouldalwayssay}), where
//\ipa{ɖɯ˧ njɤ˧}// ‘continuously, ceaselessly’ and //\ipa{ʐwɤ˩}\textsubscript{b}// ‘to say’ are not integrated in
the same tone group. (There are more than thirty examples in the first twenty texts recorded.)
\begin{exe}
	\ex
	\label{ex:theelderswouldalwayssay}
	\ipaex{ hĩ˧mo˥=ɻæ˩-ɳɯ˩   {\kern2pt}|{\kern2pt} ɖɯ˧-njɤ˧ {\kern2pt}|{\kern2pt} ʐwɤ˩-kv̩˩˥ {\kern2pt}|{\kern2pt} mæ˩ ({\dots})}\\
	\gll hĩ˧mo˥=ɻæ˩-ɳɯ˩	ɖɯ˧-njɤ˧	ʐwɤ˩\textsubscript{a}	-kv̩˧˥		mæ˧\\
	elders=\textsc{pl-a}		constantly	to\_say	\textsc{abilitive}	\textsc{obviousness}\\
	\glt ‘The elders would always say{\dots}’ (Dog2.32)
\end{exe}


\begin{table}%[t]
\caption{\label{tab:continuously}The tone patterns of phrases made up of the adverb //\ipa{ɖɯ˧ njɤ˧}// ‘continuously, ceaselessly’ followed by a~verb.}
\begin{tabularx}{\textwidth}{ l@{\hspace{30pt}} Q Q l@{\hspace{30pt}} Q }
  \lsptoprule
	tone & example & meaning & result & tone pattern\\\midrule
	H & \ipa{dzɯ˥} & to eat & \ipa{ɖɯ˧-njɤ˧ dzɯ˧} & M.M.M\\ 
	M\textsubscript{a} & \ipa{hwæ˧\textsubscript{a}} & to buy & \ipa{ɖɯ˧-njɤ˧ hwæ˩} & M.M.L\\ 
	M\textsubscript{b} & \ipa{tɕʰi˧\textsubscript{b}} & to sell & \ipa{ɖɯ˧-njɤ˧ tɕʰi˧} & M.M.M\\ 
	L\textsubscript{a} & \ipa{dze˩\textsubscript{a}} & to cut & \ipa{ɖɯ˧-njɤ˧ dze˧˥} & M.M.MH\\ 
	L\textsubscript{b} & \ipa{ʐwɤ˩\textsubscript{b}} & to speak & \ipa{ɖɯ˧-njɤ˧ ʐwɤ˧˥} & M.M.MH\\ 
	MH & \ipa{lɑ˧˥} & to strike & \ipa{ɖɯ˧-njɤ˧ lɑ˧˥} & M.M.MH\\ 
\lspbottomrule
\end{tabularx}
\end{table}


Using the context of this narrative, it was attempted to combine the adverb with the verb, but the
consultant judged this wrong, even when truncating the sentence after the main verb: $\ddagger${\kern2pt}\ipa{ɖɯ˧ njɤ˧
  ʐwɤ˧˥}. This judgment highlights the fact that the data in \tabref{tab:continuously} was elicited at a~push: in
the present state of the language, these expressions verge on the incorrect, and the adverb is well advanced on
its way towards the status of \textit{tonal standalone}. This example illustrates how easily different data
collection methods can lead to different conclusions. The combination of different types of data,
collected with suitable precautions, appears indispensable for cumulative progress in
research.\footnote{See the set of themed articles \textit{How to Study a~Tone Language} edited by Steven Bird \& Larry Hyman, in volume 8 of the journal
  \textit{Language Documentation and Conservation} (2014). A~case of diverging notations in
  a~level"=tone language is analyzed by \citet{roux2003}, leading to similar recommendations for
  precautions in the investigation method. Some views on this topic are set out in
  \citet{niebuhretal2015}.}

A discourse factor that arguably plays a~leading role in the evolution of the adverb //\ipa{ɖɯ˧
  njɤ˧}// ‘continuously, ceaselessly’ is the emphasis that is associated with
it from a~semantic"=pragmatic point of view. This adverb sometimes carries
\isi{emphatic stress} in narratives. The scenario would thus be one of generalization (\isi{lexicalization}) of intonational emphasis.


\subsection{Topicalized constituents always end a~tone group}
\label{sec:atonegroupboundaryisalwaysfoundaftertopicalizedphrases}


A~tone group \is{boundary (between tone groups)}boundary is always found after topicalized constituents. In detail, the situation is as follows: 
\begin{itemize}
\item the topic marker /\ipa{-dʑo˥}/ always terminates a~tone group. No \is{exceptions}exception has been found among 2,000 examples from narratives.
\item the topic marker /\ipa{ʈʂʰɯ˧}/ likewise terminates a~tone group, except in the many cases where it is followed by another topic marker: /\ipa{-dʑo˥}/.
\item the contrastive topic marker /\ipa{-no˧˥}/ always constitutes a~tone group on its own, as mentioned above and illustrated by (\ref{ex:asforthecatithasalifespanoffourfiveyears}).
\end{itemize}
\begin{exe}
  \ex
  \label{ex:asforthecatithasalifespanoffourfiveyears}
  \ipaex{hwɤ˧li˧˥ {\kern2pt}|{\kern2pt} -no˧˥, {\kern2pt}|{\kern2pt} ʐv̩˧kʰv̩˩-ŋwɤ˩kʰv̩˩. {\kern2pt}|{\kern2pt}}\\
  \gll hwɤ˧li˧˥	-no˧˥	ʐv̩˧kʰv̩˩		ŋwɤ˧kʰv̩˩\\
  cat		\textsc{cntr.top}	four.years	five.years\\
  \glt ‘As for the cat, [it has a~lifespan of] four or five years.’ (Dog2.84. Context: the previous
  discussion hinges on the dog’s lifespan, and the speaker now moves on to the topic of cats.)
\end{exe}


\subsection[Options left to the speaker]{Options left to the speaker in the division into tone groups}
\label{sec:optionslefttothespeakerinthedivisionintotonegroups}

Apart from the cases presented in \sectref{sec:someelementsalwaysconstituteatonegroupontheirown}--\ref{sec:atonegroupboundaryisalwaysfoundaftertopicalizedphrases}, the speaker generally has several options. They may choose to integrate large chunks of speech into
a~single tone group; or they may divide the utterance into a~number of
tone groups, with the \is{stylistics}stylistic effect of emphasizing these individual
components one after the other. This parallels observations about the
\isi{intonation} of numerous languages, e.g.~observations about Russian and German by \citet[204]{karcevskij1931}:
“Within certain limits, it is possible to change the position of the
rhythmic breaks that separate a~sentence into
parts”.\footnote{\textit{Original text}: Dans certaines limites, nous
  pouvons déplacer les anti"=cadences séparant les membres de la
  phrase.} An interesting characteristic of Yongning Na is that this
division exerts a~strong influence on tone, since tonal
processes never apply across tone"=group junctures.

For instance, /\ipa{dzɯ˧-di˧˥}/ ‘things to eat; food’, from /\ipa{dzɯ˥}/ ‘to eat’ and the
{nominalizer} /\ipa{-di˩}/, can combine with /\ipa{mɤ˧-dʑo˧}/ ‘there isn’t’ to mean ‘there isn’t any
food, there is nothing to eat’; the noun and the negated verb can either be integrated into one
tone group, as /\ipa{dzɯ˧-di˧ mɤ˧-dʑo˧˥}/, or separated, as /\ipa{dzɯ˧-di˧˥ {\kern2pt}|{\kern2pt} mɤ˧-dʑo˧}/. The latter
option is illustrated by (\ref{ex:therewasnothingtoeatandnothingtodrink}):

\begin{exe}
  \ex
  \label{ex:therewasnothingtoeatandnothingtodrink}
  \ipaex{dzɯ˧-di˧˥   {\kern2pt}|{\kern2pt}   mɤ˧-dʑo˧,   {\kern2pt}|{\kern2pt}   ʈʰɯ˩-di˩˥  {\kern2pt}|{\kern2pt}  mɤ˧-dʑo˧!}\\
  \gll dzɯ˥	-di˩	mɤ˧-	dʑo˧\textsubscript{b}	ʈʰɯ˩\textsubscript{b}	-di˩	mɤ˧-	dʑo˧\textsubscript{b}\\
  to\_eat	\textsc{nmlz}	\textsc{neg}	\textsc{exist}	to\_drink	\textsc{nmlz}	\textsc{neg}
  \textsc{exist}\\
  \glt ‘[Before mankind had learnt to grow crops], there was nothing to eat and nothing to drink!’
  (Seeds2.67)
\end{exe}

In
(\ref{ex:therewasnothingtoeatandnothingtodrink}), separating the noun phrase /\ipa{dzɯ˧-di˧˥}/ ‘food’ and the negated \is{existentials}existential verb /\ipa{mɤ˧-dʑo˧}/ ‘there
isn’t’ into two tone groups has the effect of emphasizing the two noun
phrases, /\ipa{dzɯ˧-di˧˥}/ ‘food; things to eat’ and /\ipa{ʈʰɯ˩-di˩}/ ‘drink; beverage’. 
%This could
%be analyzed as a~case of \isi{focalization}, and transcribed as /\ipa{dzɯ˧-di˧˥ F {\kern2pt}|{\kern2pt} mɤ˧-dʑo˧, {\kern2pt}|{\kern2pt} ʈʰɯ˩-di˩˥
%  F {\kern2pt}|{\kern2pt} mɤ˧-dʑo˧}/, where the symbol ‘F’ indicates intonational \isi{focalization}. The presence of
%a~tone"=group \is{boundary (between tone groups)}boundary before the negation \is{prefixes}prefix could then be interpreted as a~consequence of
%\isi{focalization}.
The following sentence in the story repeats the statement ‘There was no food’, continuing the same
strategy of bringing out the noun phrase ‘food’, this time with the topic marker /\ipa{-dʑo˥}/ (\ref{ex:nofood}). The effect is to emphasize how dire the situation was getting.

\begin{exe}
	\ex
	\label{ex:nofood}
	\ipaex{dzɯ˧-di˧˥ {\kern2pt}|{\kern2pt} -dʑo˩, {\kern2pt}|{\kern2pt} mɤ˧-dʑo˧-ɲi˥ tsɯ˩ {\kern2pt}|{\kern2pt} mv̩˩!}\\
	\gll dzɯ˥	-di˩				-dʑo˥				mɤ˧-			dʑo˧\textsubscript{b}			-ɲi˩							tsɯ˧˥				mv̩˧\\
		to\_eat		\textsc{nmlz}	\textsc{top}	\textsc{neg}	\textsc{exist}	\textsc{certitude}	\textsc{rep}	\textsc{affirm}\\
	\glt  ‘As for food, it’s said that there was none!’	(Seeds2.68)
\end{exe}

Then the narrator recapitulates as (\ref{ex:astherewasnothingtoeat}):

\begin{exe}
  \ex
  \label{ex:astherewasnothingtoeat}
  \ipaex{dzɯ˧-di˧ mɤ˧-dʑo˧˥  {\kern2pt}|{\kern2pt}  -dʑo˩  {\kern2pt}|{\kern2pt}  tʰi˩˥ {\dots}}\\
  \gll dzɯ˥	-di˩	mɤ˧-	dʑo˧\textsubscript{b}	-dʑo˥	tʰi˩˥\\
  to\_eat	\textsc{nmlz}	\textsc{neg}	\textsc{exist}	\textsc{top}	so/then\\
  \glt  ‘As there was nothing to eat, {\dots}’ (the narrative moves on to: ‘there were some
  exceptional, smart people, who stood up and did something about it’) (Seeds2.69)
\end{exe}

At this juncture, ‘there was no food’ is integrated into one tone group, and followed by the
topic marker /\ipa{-dʑo˥}/. This provides an~exemplary illustration of the integration of larger
chunks of information into one tone group as this information changes status from new to
old and backgrounded.

Long tone groups within which phonological and morphosyntactic tone rules are allowed full play, undisturbed by local
intrusions of pragmatic phenomena of emphasis, yield a~\is{stylistics}stylistic effect of carefully constructed,
poised, stately speech. Conversely, in lively speech, tone"=group boundaries are inserted here and there to highlight the word or
phrase that precedes. Even \isi{function words} can be emphasized in this way, as in (\ref{ex:itissaidthatonthatoccasionthewholefamilywillkowtow}).
\begin{exe}
  \ex
  \label{ex:itissaidthatonthatoccasionthewholefamilywillkowtow}
  \ipaex{ɬo˧pv̩˥ ti˩-kv̩˩ {\kern2pt}|{\kern2pt} tsɯ˧˥ {\kern2pt}|{\kern2pt} mv̩˩!}\\
  \gll ɬo˧pv̩˥	ti˩\textsubscript{a}	-kv̩˧˥		tsɯ˧˥	mv̩˧\\
  kow"=tow	to\_hit	\textsc{abilitive}	\textsc{rep}	\textsc{affirm}\\
  \glt ‘It is said that [on that occasion, the whole family] will kow"=tow!’ (Sister3.138)
\end{exe}

A simpler formulation would be /\ipa{ɬo˧pv̩˥ ti˩-kv̩˩ tsɯ˩ {\kern2pt}|{\kern2pt} -mv̩˩}/. The formulation in
(\ref{ex:itissaidthatonthatoccasionthewholefamilywillkowtow}) emphasizes the reported"=speech
particle /\ipa{tsɯ˧˥}/. This evidential particle is used whenever the speaker only has indirect knowledge of the situation at issue, and therefore it appears over and over again in narratives. But in the
context of (\ref{ex:itissaidthatonthatoccasionthewholefamilywillkowtow}), it takes on its full
meaning, because the narrator never witnessed the ritual that she describes. The emphasis laid on
the evidential particle in the context of this sentence is one of the manifestations of speaker F4’s
concern to adhere to truthfulness and precision.

The \is{stylistics}stylistic choices made by a~speaker can be appraised against a~background of general tendencies,
outlined below.


\subsection[Some general tendencies]{Some general tendencies in the division into tone groups}
\label{sec:somegeneraltendenciesinthedivisionintotonegroups}


\subsubsection{The role of the morphological complexity of constituents}
\label{sec:theroleofthemorphologicalcomplexityofconstituents}


The degree of internal complexity of the successive constituents of an~utterance is among the parameters that influence its division into tone groups. A~verb without prefixes or suffixes is
usually just one syllable long, and easily associates with a~preceding adverb or
noun. For instance, /\ipa{kʰɯ˧tsʰɤ˧ ʈʂʰe˧}/ ‘to stretch out [one’s] legs’, from /\ipa{kʰɯ˧tsʰɤ˧˥}/
‘leg’ and /\ipa{ʈʂʰe˧\textsubscript{b}}/ ‘to stretch out’, constitutes one tonal word, whose output tone is
determined by the tone rules that apply in subject"=plus"=verb phrases (see~\sectref{sec:subjectandverb}). When
an~adverb is inserted, it can be integrated into the tone group, as in (\ref{ex:stretchdown}), where there is only one tone group for the subject, the adverb //\ipa{mv̩˩tɕo˧}//
‘downward’, and the verb.
\begin{exe}
	\ex
	\label{ex:stretchdown}
	\ipaex{kʰɯ˧tsʰɤ˧
		mv̩˥tɕo˩ ʈʂʰe˩}\\
	\gll kʰɯ˧tsʰɤ˧˥		mv̩˩tɕo˧		ʈʂʰe˧\textsubscript{b}\\
	leg		downward	to\_stretch\\
	\glt ‘to stretch (one’s) leg downward’
\end{exe}

But the adverb often marks the beginning of a~new tone group, as in (\ref{ex:stretchdown2}), where the subject is in a~different tone group from the adverb and verb.

\begin{exe}
	\ex
	\label{ex:stretchdown2}
	\ipaex{kʰɯ˧tsʰɤ˧˥ {\kern2pt}|{\kern2pt} mv̩˩tɕo˧ ʈʂʰe˧}\\
	\gll kʰɯ˧tsʰɤ˧˥		mv̩˩tɕo˧		ʈʂʰe˧\textsubscript{b}\\
	leg		downward	to\_stretch\\
	\glt ‘to stretch (one’s) leg downward’
\end{exe}

Like directional adverbs, \is{numerals}numeral"=plus"=classifier phrases often mark the beginning of a~new group, as in (\ref{ex:sheetofpaper}). But
they can also be integrated into a~single tone group with a~preceding noun, as in (\ref{ex:motheranddaughter}). 

\begin{exe}
	\ex
	\label{ex:sheetofpaper}
	\ipaex{ʂv̩˧{$\sim$}ʂv̩˧˥ {\kern2pt}|{\kern2pt} ɖɯ˧-pʰæ˧˥}\\
	\gll ʂv̩˧{$\sim$}ʂv̩˧˥	ɖɯ˧-pʰæ˧˥\\
	paper		one-\textsc{clf}.flat\_objects\\
	\glt ‘a sheet of paper’
\end{exe}

\begin{exe}
	\ex
	\label{ex:motheranddaughter}
	\ipaex{ə˧mi˧-mv̩˩ ɲi˩-kv̩˩}\\
	\gll ə˧mi˧	mv̩˩˥		ɲi˧-kv̩˧˥\\
	mother		daughter	two-\textsc{clf}.persons\\
	\glt  ‘the mother and her daughter’ \textit{Literally:} ‘mother and
		daughter, the two’ (Tiger.11, 51, and Lake4.93, 96-98, 125)
\end{exe}

The expression in (\ref{ex:motheranddaughter}) is a~special case: /\ipa{ə˧mi˧-mv̩˩}/ is a~coordinative compound that means ‘mother and daughter’; addition of the expression /\ipa{ɲi˧-kv̩˧˥}/, ‘two’ plus the classifier for persons, is obviously not to be understood as an~instance of counting mother"=and"=daughter pairs. The \is{numerals}numeral"=plus"=classifier
{\linebreak}phrase does not serve the usual purpose of providing a figure: in this
context, it serves an~anaphoric function. The expression can be paraphrased as ‘these two: the mother and the daughter’. 

Demonstrative"=plus"=classifier\is{classifiers} phrases are commonly integrated with a~preceding noun. For instance,
in the first version of the Lake story, the same two characters, a~mother and her daughter, are
referred to as /\ipa{ə˧mi˧ ʈʂʰɯ˧-v̩˧ lɑ˩ {\kern2pt}|{\kern2pt} mv̩˩ ʈʂʰɯ˩-v̩˩˥}/,
‘that mother and that daughter’: see (\ref{ex:thatmotherandthatdaughter}). 
%(On this topic, see the discussion of classifiers in Chapter~\ref{chap:classifiers}.)

\begin{exe}
  \ex
  \label{ex:thatmotherandthatdaughter}
  \ipaex{ə˧mi˧ ʈʂʰɯ˧-v̩˧ lɑ˩ {\kern2pt}|{\kern2pt} mv̩˩ ʈʂʰɯ˩-v̩˩˥}\\
  \gll ə˧mi˧ 	ʈʂʰɯ˧-v̩˧		lɑ˧	mv̩˩˥		ʈʂʰɯ˧-v̩˧\\
  mother 	\textsc{dem}-\textsc{clf}.individual	and	daughter	\textsc{dem}-\textsc{clf}\\
  \glt ‘that mother and that daughter’ (Lake.52).
\end{exe}


\subsubsection{The role of information structure: Considerations of prominence}
\label{sec:theroleofinformationstructureconsiderationsofprominence}


Information structure also influences the division into tone groups, in a~way which is often difficult to disentangle from the influence of morphological complexity. Consider (\ref{ex:itssaidthatonemustnteatdogmeat}): 
\begin{exe}
  \ex
  \label{ex:itssaidthatonemustnteatdogmeat}
  \ipaex{kʰv̩˩mi˩-ʂe˩˥, {\kern2pt}|{\kern2pt} dzɯ˧ mɤ˧-ɖo˧ pi˧-zo˥!}\\
  \gll kʰv̩˩mi˩-ʂe˩	dzɯ˥	mɤ˧-	ɖo˧\textsubscript{a}		pi˥	-zo\\
  dog"=meat	to\_eat	\textsc{neg}	ought\_to	to\_say	\textsc{advb}\\
  \glt ‘It’s said that one mustn’t eat dog meat! / It’s said that dog meat is something one must not
  eat!’ (Dog2.37)
\end{exe}

In (\ref{ex:itssaidthatonemustnteatdogmeat}), the noun phrase ‘dog meat’ is set into relief by
constituting a~tone group on its own. Despite the absence of a~morphemic indication that it is
topicalized, such as use of a~topic marker /\ipa{ʈʂʰɯ˧}/ or /\ipa{-dʑo˥}/, it clearly has the
status of topic. In this context, tonal integration with a~following verb would not be stylistically
appropriate.

Likewise, in (\ref{ex:intheoldtimesonewouldntusuallyletdogsgooutside}), the {adverbial} ‘outside’,
/\ipa{ə˩pʰo˩}/, constitutes a~tone group on its own. Another option would be to integrate it tonally
with the following verb. (The tonal paradigms for \textit{spatial adverbial}+\textit{verb} combinations
are set out in Chapter~\ref{chap:verbsandtheircombinatoryproperties}.) In this context, integration into a~single tone group would be
stylistically acceptable. Separation
into two tone groups has the effect of providing the information gradually, giving the impression
that the speaker is constructing the utterance as she is saying it, rather than delivering long, carefully
preplanned chunks of speech.

\begin{exe}
  \ex
  \label{ex:intheoldtimesonewouldntusuallyletdogsgooutside}
  \ipaex{ə˧ʝi˧-ʂɯ˥ʝi˩-dʑo˩, {\kern2pt}|{\kern2pt} kʰv̩˧-ʈʂʰɯ˧-dʑo˩, {\kern2pt}|{\kern2pt} dʑɤ˩˥ {\kern2pt}|{\kern2pt} ə˩pʰo˩˥ {\kern2pt}|{\kern2pt} kʰɯ˧ mɤ˥-kv̩˩!}\\
  \gll ə˧ʝi˧-ʂɯ˥ʝi˩	dʑo˥	kʰv̩˥	-ʈʂʰɯ˥	-dʑo˥	dʑɤ˩˥		ə˩pʰo˩ kʰɯ˧˥	mɤ˧-	-kv̩˧˥\\
  in\_the\_past	\textsc{top}	dog	\textsc{top}	\textsc{top}	\textsc{ints}	outside
  to\_let	\textsc{neg}	\textsc{abilitive}\\
  \glt ‘In the old times, one wouldn’t usually let dogs go outside! / In the old times, dogs weren’t usually allowed to leave the house!’ (Dog2.75)
\end{exe}

It is uncommon for a~verb preceded by the {accomplished} \is{prefixes}prefix /\ipa{le˧-}/ to interact tonally with
a~preceding noun phrase. In Caravans.191, for instance, ‘the uncle comes back’ is realized as
/\ipa{ə˧v̩˧˥ {\kern2pt}|{\kern2pt} le˧-tsʰɯ˩}/, not as /\ipa{ə˧v̩˧ le˥-tsʰɯ˩}/, although the latter form is also
acceptable. Cases where tonal interaction does take place are characterized by a~strong degree of
semantic givenness, as in (\ref{ex:therearrivedonepersonthentwothenthree}):
\begin{exe}
  \ex
  \label{ex:therearrivedonepersonthentwothenthree}
  \ipaex{ɖɯ˧-v̩˧ le˧-tsʰɯ˩, {\kern2pt}|{\kern2pt} ɲi˧-kv̩˧ le˧-tsʰɯ˧˥, {\kern2pt}|{\kern2pt} so˩-kv̩˩ le˩-tsʰɯ˩˥.}\\
  \gll ɖɯ˧-v̩˧	le˧-	tsʰɯ˩\textsubscript{a}	ɲi˧-kv̩˧˥	so˩-kv̩˩\\
  one-\textsc{clf}.individual	\textsc{accomp}	to\_come.\textsc{pst}	two-\textsc{clf}	three-\textsc{clf}\\
  \glt ‘One person came, then two, then three.’ (Field notes: explanation proposed by
  consultant F4 during a~discussion of Lake4.126)
\end{exe}
 
It would not be incorrect to say /\ipa{ɖɯ˧-v̩˧ {\kern2pt}|{\kern2pt} le˧-tsʰɯ˩, {\kern2pt}|{\kern2pt} ɲi˧-kv̩˧˥ {\kern2pt}|{\kern2pt} le˧-tsʰɯ˩, {\kern2pt}|{\kern2pt} so˩-kv̩˩˥ {\kern2pt}|{\kern2pt}
  le˧-tsʰɯ˩}/, but this would be inappropriate in a~context where the emphasis is on the count (one person, then two, then three), not on the verb. There would be no point in setting
the subject apart from the verb, hence the division into three tone groups, rather than six.

When an~explanation is added as an~afterthought, a~relatively long sequence of syllables can be
integrated into one tone group, as in (\ref{ex:fromchengduinthepastsilk}), where the last tone
group contains ten syllables.
\begin{exe}
  \ex
  \label{ex:fromchengduinthepastsilk}
  \ipaex{jɤ˧ŋɤ˧-dʑo˧, {\kern2pt}|{\kern2pt} ə˧ʝi˧-ʂɯ˥ʝi˩, {\kern2pt}|{\kern2pt} hæ̃˩-bɑ˥lɑ˩! {\kern2pt}|{\kern2pt} hæ̃˩-bɑ˥lɑ˩-bɑ˩lɑ˩ le˩-po˩ jo˩-kv̩˩ mæ˩!}\\
  \gll jɤ˧ŋɤ˧		-dʑo˥	ə˧ʝi˧-ʂɯ˥ʝi˩	hæ̃˩-bɑ˥lɑ˩	hæ̃˩-bɑ˥lɑ˩-bɑ˩lɑ˩  le˧-		po˧˥		jo˩		-kv̩˧˥
  mæ˧\\
  Chengdu	\textsc{top}	in\_the\_past	silk		silk\_clothes \textsc{accomp}	to\_bring	to\_come
  \textsc{abilitive}	\textsc{obviousness}\\
  \glt ‘From Chengdu, in the past{\dots} Silk!! [The people who went on caravans] would bring back
  silk clothing [from their journeys to Chengdu]!’ (Caravans.104-105)
\end{exe}

In a~context where the narrator is explaining which goods used to be transported by caravan, the
essential information is already given in the word ‘silk’. The portion of sentence /\ipa{le˩-po˩-jo˩-kv̩˩-mæ˩}/ ‘[they] would bring back’ is added as an~explanation; its integration in the same tone group as the preceding noun, ‘silk’, results in a~levelling down of all of its
tones to L, reflecting its status as backgrounded information.

As many as twelve syllables are bunched together in (\ref{ex:elders3}):

\begin{exe}
	\ex
	\label{ex:elders3}
	\ipaex{ə˧ʑi˧˥, {\kern2pt}|{\kern2pt} ɖɯ˩mɑ˧-ɬɑ˩tsʰo˩
		pi˩-hĩ˩ ɖɯ˩-v̩˩ dʑo˩-ɲi˩ tsɯ˩ {\kern2pt}|{\kern2pt} mv̩˩.}\\
	\gll ə˧ʑi˧˥		ɖɯ˩mɑ˧-ɬɑ˩tsʰo˩		pi˥		-hĩ˥ ɖɯ˧	v̩˧		dʑo˧\textsubscript{a}	-ɲi˩		tsɯ˧˥		mv̩˧\\
	grandmother		\textsc{given\_name}	to\_say		\textsc{rel/nmlz}	one	\textsc{clf}.individual		\textsc{exist}	\textsc{certitude}	\textsc{rep}	\textsc{affirm}\\
	\glt ‘[Among] women elders, it is said that there was one by the
	name of Ddeema Lhaco.’ (Elders3.11)
\end{exe}

The speaker lays considerable emphasis on the person’s name,
Ddeema Lhaco. All the rest of the sentence follows as a~strongly backgrounded
accompaniment to this name. Phonologically, the name and all that follows are integrated into
one tone group, with the result that all the syllables from the third to the twelfth and last
are lowered to L.

As a~last example, consider (\ref{ex:youCAME}).

\begin{exe}
	\ex
	\label{ex:youCAME}
	\ipaex{dzo˧ {\kern2pt}|{\kern2pt} le˧-gv̩˩ {\kern2pt}|{\kern2pt} tʰi˧-tɕɯ˥ {\kern2pt}|{\kern2pt} tʰi˩˥ {\kern2pt}|{\kern2pt} no˧ {\kern2pt}|{\kern2pt} le˧-tsʰɯ˩-ɲi˩-ze˩-mæ˩, {\kern2pt}|{\kern2pt} ə˩-gi˩! {\kern2pt}|{\kern2pt} hĩ˧ ɖɯ˧-v̩˧ mɤ˧-tsʰɯ˩! {\kern2pt}|{\kern2pt} no˩ le˩-tsʰɯ˩-ɲi˥-ze˩ mæ˩!}\\
	\gll dzo˩		le˧-	gv̩˩\textsubscript{b}	tʰi˧-	tɕɯ˥	tʰi˩˥	no˩	le˧-	tsʰɯ˩\textsubscript{a}	-ɲi˩		-ze˧\textsubscript{b}		mæ˧	ə˩-gi˩		hĩ˥	ɖɯ˧		v̩˧		mɤ˧-		tsʰɯ˩\textsubscript{a}	no˩	le˧-	tsʰɯ˩\textsubscript{a}	-ɲi˩		-ze˧\textsubscript{b}		mæ˧\\
	bridge		\textsc{accomp}	to\_build		\textsc{dur}	to\_put		then 2\textsc{sg}	\textsc{accomp}		to\_come	\textsc{certitude}	\textsc{pfv}	\textsc{affirm}		isn't\_it	person	one	\textsc{clf}.individual		\textsc{neg}	to\_come	2\textsc{sg}	\textsc{accomp}		to\_come	\textsc{certitude}	\textsc{pfv}	\textsc{affirm}\\
	\glt ‘After the bridge is built, and left there [=and the person who built it waits for someone to cross], you come along! [=someone comes along: you, for instance!] [For a long time] nobody comes, [but at last] you come along!’ (Renaming.17)
\end{exe}

The same syntactic structure, ‘you came along’, is realized as
two tone groups: /\ipa{no˧ {\kern2pt}|{\kern2pt} le˧-tsʰɯ˩ ɲi˩-ze˩ mæ˩}/, then repeated as a~single tone group:
/\ipa{no˩ le˩-tsʰɯ˩ ɲi˥-ze˩ mæ˩}/. This provides an exemplary illustration of how tone groups tend to be longer when the speaker assumes that the semantic content is already familiar
to the listener.




\subsection[Extreme cases of tonal integration]{Extreme cases of tonal integration: Set phrases and proverbs}
\label{sec:extremecasesoftonalintegrationsetphrasesandproverbs}



\subsubsection{Tonal integration in set phrases}
\label{sec:tonalintegrationinsetphrases}

Set phrases constitute a~typical case of integration. For instance, there are formulae that recapitulate which of the animals symbolizing the twelve Terrestrial Branches have special affinities with one another. These subsets are used in fortune"=telling: the year of birth serves as a~basis on which one predicts whether or not an individual will be able to relate harmoniously with another. Among the animals that succeed one another in the twelve"=year cycle, there are four sets of three, shown in (\ref{ex:SerpentOxRooster})-(\ref{ex:TigerHorseDog}).

\begin{exe}
	\ex
	\label{ex:SerpentOxRooster}
	\ipaex{bv̩˧ʐv̩˧ ʝi˧ {\kern2pt}|{\kern2pt} æ̃˩ so˥-kʰv̩˩}\\
	\gll bv̩˧ʐv̩˧	ʝi˥	æ̃˩˧	so˩		kʰv̩˧˥\textsubscript{a}\\
	snake		ox	chicken	three	\textsc{clf}.years\\
	\glt ‘the three years of the Snake, the Ox, and the Rooster’
\end{exe}

\begin{exe}
	\ex
	\label{ex:DragonApeRat}
	\ipaex{mv̩˧gv̩˧ ʑi˧˥ {\kern2pt}|{\kern2pt} hwɤ˧ so˧-kʰv̩˥}\\
	\gll mv̩˧gv̩˧	ʑi˩˥	hwɤ˧	so˩		kʰv̩˧˥\textsubscript{a}\\
	dragon		ape		rat		three	\textsc{clf}.years\\
	\glt ‘the three years of the Dragon, the Ape, and the Rat’
	%\footnote{The Na twelve"=year cycle has the Cat instead of the Rat. Various cases of replacement are observed in the history of the cycle's geographical travels, such as a~change from buffalo to ox when the cycle was borrowed into \il{Sinitic}Chinese culture from an Austroasiatic origin, i.e.\ from a~more Southern environment to a~more Northern environment; this change was later reverted (from ox to buffalo) when the \il{Sinitic}Chinese cycle was borrowed into \ili{Vietnamese}, and thence into Khmer \citep{coedes1935,ferlus2010,ferlus2013b}.}
\end{exe}

\begin{exe}
	\ex
	\label{ex:RabbitPigSheep}
	\ipaex{tʰo˧li˧ bo˩ {\kern2pt}|{\kern2pt} jo˩ so˩-kʰv̩˩˥}\\
	\gll tʰo˧li˧	bo˩˧		jo˩		so˩		kʰv̩˧˥\textsubscript{a}\\
	rabbit		pig			sheep	three	\textsc{clf}.years\\
	\glt ‘the three years of the Rabbit, the Pig, and the Sheep’
\end{exe}

\begin{exe}
	\ex
	\label{ex:TigerHorseDog}
	\ipaex{lɑ˧ {\kern2pt}|{\kern2pt} ʐwæ˧ {\kern2pt}|{\kern2pt} kʰv̩˧ {\kern2pt}|{\kern2pt} so˩-kʰv̩˩˥}\\
	\gll lɑ˧		ʐwæ˥	kʰv̩˥		so˩		kʰv̩˧˥\textsubscript{a}\\
	tiger		horse	dog	three	\textsc{clf}.years\\
	\glt ‘the three years of the Tiger, the Horse, and the Dog’
\end{exe}


%\begin{itemize}
%\item /\ipa{bv̩˧ʐv̩˧ {\kern2pt}|{\kern2pt} ʝi˧ {\kern2pt}|{\kern2pt} æ˩˥}/ are grouped as /\ipa{bv̩˧ʐv̩˧, ʝi˧, {\kern2pt}|{\kern2pt} æ˩-so˥-kʰv̩˩}/ ‘the three years of the Serpent, the Ox, and the Rooster’;
%\item /\ipa{mv̩˧gv̩˧ {\kern2pt}|{\kern2pt} ʑi˩˥ {\kern2pt}|{\kern2pt} hwɤ˧˥}/ are grouped as /\ipa{mv̩˧gv̩˧ ʑi˧˥ {\kern2pt}|{\kern2pt} hwɤ˧ so˧-kʰv̩˥}/ ‘the three years of the Dragon, the Ape, and the Rat’; 
%\item /\ipa{tʰo˧li˧ {\kern2pt}|{\kern2pt} bo˩˥ {\kern2pt}|{\kern2pt} jo˩˥}/ are grouped as /\ipa{tʰo˧li˧-bo˩ {\kern2pt}|{\kern2pt} ʝo˩-so˩kʰv̩˩˥}/ ‘the three years of the Rabbit, the Pig, and the Sheep’;
%\item /\ipa{lɑ˧ {\kern2pt}|{\kern2pt} ʐwæ˧ {\kern2pt}|{\kern2pt} kʰv̩˧}/ are grouped as /\ipa{lɑ˧, {\kern2pt}|{\kern2pt} ʐwæ˧, {\kern2pt}|{\kern2pt} kʰv̩˧ {\kern2pt}|{\kern2pt} so˩-kʰv̩˩˥}/ ‘the three years of the Tiger, the Horse, and the Dog’.
%\end{itemize}


The tone grouping is not exactly the same in these four phrases. In (\ref{ex:SerpentOxRooster}), (\ref{ex:DragonApeRat}) and (\ref{ex:RabbitPigSheep}), the phrase is divided into two tone groups, instead of the four tone groups that would obtain if each animal name were said separately. This yields the expected effect of stronger integration than would be found in a~list in which each item stood in a~tone group of its own. The first tone group contains the names of two animals; the rest of the phrase, making up another tone group, contains the third animal name and the expression ‘three years’. Each of the two tone groups has three syllables, creating an~effect of rhythmic balance. The tone that obtains by association of two animal names conforms to the patterns observed in coordinative compounds (as set out in \sectref{sec:coordinativecompounds}): disyllabic M-tone noun plus {monosyllabic} H-tone noun yields \mbox{//\#H//} tone (//\ipa{bv̩˧ʐv̩˧ ʝi\#˥}//), hence a~surface form with M tone: /\ipa{bv̩˧ʐv̩˧ ʝi˧}/.

For ease of comparison, the forms that would obtain if each noun stood in a~tone group of its own are shown as (\ref{ex:SerpentOxRoosterCUT}), (\ref{ex:DragonApeRatCUT}) and (\ref{ex:RabbitPigSheepCUT}). 

\begin{exe}
	\ex
	\label{ex:SerpentOxRoosterCUT}
	\ipaex{bv̩˧ʐv̩˧ {\kern2pt}|{\kern2pt} ʝi˧ {\kern2pt}|{\kern2pt} æ̃˩˥ {\kern2pt}|{\kern2pt} so˩-kʰv̩˩˥}\\
	\gll bv̩˧ʐv̩˧	ʝi˥	æ̃˩˧	so˩		kʰv̩˧˥\textsubscript{a}\\
	snake		ox	chicken	three	\textsc{clf}.years\\
	\glt \textit{modified from (\ref{ex:SerpentOxRooster}) by placing each noun in a separate tone group}
\end{exe}

\begin{exe}
	\ex
	\label{ex:DragonApeRatCUT}
	\ipaex{mv̩˧gv̩˧ {\kern2pt}|{\kern2pt} ʑi˩˥ {\kern2pt}|{\kern2pt} hwɤ˧ {\kern2pt}|{\kern2pt} so˩-kʰv̩˩˥}\\
	\gll mv̩˧gv̩˧	ʑi˩˥	hwɤ˧	so˩		kʰv̩˧˥\textsubscript{a}\\
	dragon		ape		rat		three	\textsc{clf}.years\\
	\glt \textit{modified from (\ref{ex:DragonApeRat}) by placing each noun in a separate tone group}
\end{exe}

\begin{exe}
	\ex
	\label{ex:RabbitPigSheepCUT}
	\ipaex{tʰo˧li˧ {\kern2pt}|{\kern2pt} bo˩˥ {\kern2pt}|{\kern2pt} jo˩˥ {\kern2pt}|{\kern2pt} so˩-kʰv̩˩˥}\\
	\gll tʰo˧li˧	bo˩˧		jo˩		so˩		kʰv̩˧˥\textsubscript{a}\\
	rabbit		pig			sheep	three	\textsc{clf}.years\\
	\glt \textit{modified from (\ref{ex:RabbitPigSheep}) by placing each noun in a separate tone group}
\end{exe}

In (\ref{ex:TigerHorseDog}), on the other hand, there is an odd number of syllables, making it impossible to build two tone groups that have the same number of syllables, and to arrive at a~symmetrical structure as in (\ref{ex:SerpentOxRooster}), (\ref{ex:DragonApeRat}) and (\ref{ex:RabbitPigSheep}). The solution chosen is to keep the three
animal names separate, each constituting a~distinct tone group, hence yielding four tone
groups in total. This illustrates the role played by rhythmic factors. (For an~introduction to the notoriously complex domain of linguistic rhythm, see \citealt{niebuhr2009b, cummins2012, house2012}.)


\subsubsection{Tonal integration in proverbs}
\label{sec:tonalintegrationinproverbs}

Proverbs are also typical instances of tightly"=knit tonal integration. An example is shown in
(\ref{ex:thepoormustnotborrowmoneytheshinbonemustnotreceivewounds}).
\begin{exe}
  \ex
  \label{ex:thepoormustnotborrowmoneytheshinbonemustnotreceivewounds}
  \ipaex{hĩ˧dzɑ˧ {\kern2pt}|{\kern2pt} ɖʐe˧ {\kern2pt}|{\kern2pt} tʰɑ˧-ʝi˥, {\kern2pt}|{\kern2pt} ɻ̍̃˧ko˩ mi˩ tʰɑ˩-tʰv̩˩.}\\
  \gll hĩ˥	dzɑ˥	ɖʐe˧	tʰɑ˧-	ʝi˥	ɻ̍̃˧ko˩	mi˧	tʰɑ˧-	tʰv̩˧\textsubscript{a}\\
  person	poor	money	\textsc{proh}	to\_borrow	shinbone	wound	\textsc{proh}	to\_get\\
  \glt ‘The poor must not borrow money; the shinbone must not receive wounds.’
\end{exe}

The proverb’s argument is that one must beware of hitting fragile spots. The listener is presumed to
know that a~blow to the shin is especially painful, and to imagine, by {analogy}, how hard it is for
a~poor person to reimburse a~loan plus added interest. The sequence /\ipa{ɻ̍̃˧ko˩ mi˩ tʰɑ˩-tʰv̩˩}/,
‘the shinbone must not receive wounds’, is integrated into one tone group, with the \is{stylistics}stylistic
effect of presenting it as a~self"=evident fact (an established truth), not a~statement coined on the
fly by the speaker, in which case the division into tone groups would have been /\ipa{ɻ̍̃˧ko˩ {\kern2pt}|{\kern2pt} mi˧ tʰɑ˧-tʰv̩˧}/ or /\ipa{ɻ̍̃˧ko˩ {\kern2pt}|{\kern2pt} mi˧ {\kern2pt}|{\kern2pt} tʰɑ˧-tʰv̩˧}/.

It is highly revealing that, even in the case of proverbs and set phrases, the speaker retains
a~latitude of choice in the division of the utterance into tone groups. The comparison of different
versions of the same story by the same speaker yields a~wealth of examples. For instance, the saying ‘If you see a~tiger, it means your father is going
to die; if you see a~panther, it means your mother is going to die’, which is at the heart of the
narrative Tiger, is divided into four tone groups in (\ref{ex:seetigerdie4}), three groups in (\ref{ex:seetigerdie3}), and two groups in (\ref{ex:seetigerdie}). 

\begin{exe}
	\ex
	\label{ex:seetigerdie4}
	\ipaex{ʐæ˩ do˥ {\kern2pt}|{\kern2pt} ə˧mi˧ ʂɯ˧; {\kern2pt}|{\kern2pt} lɑ˧ do˩, {\kern2pt}|{\kern2pt} ə˧dɑ˧ ʂɯ˧.}\\
	\gll ʐæ˩˥			do˩\textsubscript{b}			ə˧mi˧		ʂɯ˧\textsubscript{a}		lɑ˧		do˩\textsubscript{b}		ə˧dɑ˥\$		ʂɯ˧\textsubscript{a}\\
	panther		to\_see	mother		to\_die		tiger			to\_see					father		to\_die\\
	\glt ‘If you see a~panther, it means your mother is going to die; if you see a~tiger, it means your father is going
	to die.’ (Tiger.10; Tiger2.5, 13)
\end{exe}

\begin{exe}
	\ex
	\label{ex:seetigerdie3}
	\ipaex{ʐæ˩ do˥ {\kern2pt}|{\kern2pt} ə˧mi˧ ʂɯ˩, {\kern2pt}|{\kern2pt} lɑ˧ do˩ ə˩dɑ˩ ʂɯ˩.}\\
	\gll ʐæ˩˥			do˩\textsubscript{b}			ə˧mi˧		ʂɯ˧\textsubscript{a}		lɑ˧		do˩\textsubscript{b}		ə˧dɑ˥\$		ʂɯ˧\textsubscript{a}\\
	panther		to\_see	mother		to\_die		tiger			to\_see					father		to\_die\\
	\glt ‘If you see a~panther, it means your mother is going to die; if you see a~tiger, it means your father is going
	to die.’ (Tiger.50)
\end{exe}

\begin{exe}
	\ex
	\label{ex:seetigerdie}
	\ipaex{lɑ˧ do˩ ə˩dɑ˩ ʂɯ˩, {\kern2pt}|{\kern2pt} ʐæ˩ do˥ ə˩mi˩ ʂɯ˩.}\\
	\gll lɑ˧		do˩\textsubscript{b}		ə˧dɑ˥\$		ʂɯ˧\textsubscript{a}		ʐæ˩˥			do˩\textsubscript{b}			ə˧mi˧		ʂɯ˧\textsubscript{a}\\
	tiger			to\_see					father		to\_die		panther		to\_see	mother		to\_die\\
	\glt ‘If you see a~tiger, it means your father is going
	to die; if you see a~panther, it means your mother is going to die.’ (Tiger.31; Tiger2.111)
\end{exe}

As in examples (\ref{ex:therewasnothingtoeatandnothingtodrink})--(\ref{ex:itissaidthatonthatoccasionthewholefamilywillkowtow}) and (\ref{ex:therearrivedonepersonthentwothenthree})--(\ref{ex:youCAME}), the \is{stylistics}stylistic nuance is
that, the greater the number of tone groups, the more attention is drawn to the individual
components of the sentence. The {phrasing} in (\ref{ex:seetigerdie4}) is found at first occurrence, towards the beginning of the story; later occurrences tend to have fewer tone groups, as in (\ref{ex:seetigerdie3}) and (\ref{ex:seetigerdie}). Example (\ref{ex:seetigerdie3}) is an interesting intermediate case, in which the ‘panther/mother’ and ‘tiger/father’ pairs are not treated symmetrically. The context is a~journey on the mountain, during which a~mother and her daughter encounter a~tiger; the father is not present. The interpretation of the asymmetry is clear: emphasis is laid on the omen concerning the mother. The other part of the proverb, mentioning the father, is less relevant; it is not entirely omitted (which could compromise recognition of the proverb), but it is backgrounded by integrating it into one tone group.\footnote{The order of the two clauses, ‘panther/mother+tiger/father’ or ‘tiger/father+panther/mother’, does not exert a~direct influence on the division into tone groups. Both orders are acceptable: the order found in  (\ref{ex:seetigerdie4})-(\ref{ex:seetigerdie3}) is not necessarily a~departure from a~canonical order exemplified by (\ref{ex:seetigerdie}), or the other way round.}

As a~final example, let us examine (\ref{ex:heavensee}).

\begin{exe}
	\ex
	\label{ex:heavensee}
	\ipaex{hĩ˧ ɳɯ˩ mɤ˩-do˩, {\kern2pt}|{\kern2pt} mv̩˧ ɳɯ˩ {\kern2pt}|{\kern2pt} do˩˥!}\\
	\gll hĩ˥	ɳɯ˧	mɤ˧-	do˩\textsubscript{b}		mv̩˥	ɳɯ˧		do˩\textsubscript{b}\\
	person/human	\textsc{a}		\textsc{neg}	to\_see			sky/heavens		\textsc{a}		to\_see\\
	\glt ‘People may not see, but the \textit{heavens} see!’
\end{exe}

This saying is used as a~reminder that other people’s gaze is not the touchstone of good conduct,
and that one’s actions should be guided by the same rules whether seen or unseen. The most common
realization of this saying is (\ref{ex:heavensee}), where the first part
(‘What people do not see’) is integrated into one tone group, whereas the second is divided
into two. This emphatically brings out the verb /\ipa{do˩\textsubscript{b}}/ ‘to see, to observe’, which being on its
own in the tone group receives a~final H tone and is realized with a~rise, /LH/, following Rule
7: “If a~tone group only contains L tones, a~post"=lexical H tone is added to its last
syllable”. (Variants are found in the narrative Reward.28, 36, 62 and 114.)


\subsection{Two cases of resistance to tonal integration}
\label{sec:acaseofresistancetotonalintegration}

Some expressions resist integration into
one tone group. As mentioned in~\sectref{sec:theroleofthenumberofsyllables}, tonal changes in compounding are only observed when the second term (the
head) has fewer than three syllables. A~typical example is the proper name “Lake Lugu”, shown in example (\ref{ex:lugulake}) of Chapter~\ref{chap:compoundnouns}. As for the phrase /\ipa{sɑ˧ {\kern2pt}|{\kern2pt} zo˩bv̩˥ɭɯ˩}/, meaning ‘the universe, the whole world’ (Mountains.69), it is also perceived as composed of two distinct parts,
/\ipa{sɑ˧}/ and /\ipa{zo˩bv̩˥ɭɯ˩}/, even though the first syllable, /\ipa{sɑ˧}/, is no longer
intelligible by itself, and does not appear on its own. The \is{trisyllables}trisyllable /\ipa{zo˩bv̩˥ɭɯ˩}/ can be employed on its own, to mean ‘the universe’, like the four"=syllable expression. The existence of
this \is{trisyllables}{trisyllabic} form may be part of the reason why the four"=syllable form /\ipa{sɑ˧ {\kern2pt}|{\kern2pt} zo˩bv̩˥ɭɯ˩}/ ‘the
universe’ does not get integrated into a~single tone group. (If such integration took place, it
would yield a~M.L.L.L tone pattern, \ipa{$\dagger$sɑ˧-zo˩bv̩˩ɭɯ˩}, by application of Rule 5: “All syllables following a~H.L or M.L sequence receive L tone”.)

On the other hand, the same argument cannot be invoked to explain the absence of tonal integration in the Na word for ‘field penny"=cress’, a~foetid plant with round flat pods (\textit{Thlaspi
  arvense}). It is called /\ipa{ʁv̩˧=bv̩˥ {\kern2pt}|{\kern2pt} v̩˩tsʰɤ˩˥}/, literally ‘the crane’s vegetable’. One would expect this expression to be treated as one tone group, since it constitutes one lexical item. But the
fact that the name can still be transparently analyzed as a~\isi{possessive} construction probably
contributes to slowing down its phonological integration into one tone group. Importantly, this plant is not commonly used, and the word therefore has low frequency in discourse, limiting the pressure towards phonetic"=phonological \isi{simplification}.


\subsection{Illustration: Sample derivations}
\label{sec:samplederivation}

This section recapitulates some of the mechanisms described in this section by providing sample \is{derivation!tonal}derivations for two sentences from transcribed texts. 

\begin{exe}
	\ex
	\label{ex:ahornofwater}
	\ipaex{dʑɯ˧ {\kern2pt}|{\kern2pt} ɖɯ˧-qʰv̩˧tʰv̩˧ tʰi˩-kʰɯ˩.}\\
	\gll dʑɯ˩	ɖɯ˧-qʰv̩˧tʰv̩\#˥	tʰi˧-	kʰɯ˧˥\\
	water	one-\textsc{clf}.hornful	\textsc{dur}	to\_put\\
	\glt ‘[People who travel all day] put a~hornful of water [in their bag, so as to have something to
	drink].’ (Tiger2.51)
\end{exe}

In example (\ref{ex:ahornofwater}), the noun ‘water’ constitutes a~tone group on its own, hence its realization with M tone, following the regular pattern shown in \tabref{tab:thelexicaltonesofmonosyllabicnouns}. The second tone group
contains five syllables: /\ipa{ɖɯ˧-qʰv̩˧tʰv̩˧ tʰi˩-kʰɯ˩}/. First, the tone of the
\is{numerals}numeral"=plus"=classifier phrase is determined, following the table"=lookup rules set out in Chapter~\ref{chap:classifiers} (\tabref{tab:hornfuls}). This yields //\ipa{ɖɯ˧-qʰv̩˧tʰv̩\#˥}// ‘one hornful’. Association of this object with the prefixed verb //\ipa{tʰi˧-kʰɯ˧˥}// ‘to put (into)’ also operates as table lookup (see Chapter~\ref{chap:verbsandtheircombinatoryproperties}). This yields
the surface phonological string, M.M.M.L.L. Thus, knowledge of a~tone group's internal morphosyntactic structure and of the tones of its constituting elements suffices to arrive at the surface tone sequence.

%%subsubsec:8-2-7-2
%\subsubsection{Movement and unfolding of the MH tone}
%\label{sec:movementandunfoldingofthemhtone}
%
%Example (\ref{ex:itissaidthatshespatouttheegg}) contains two tone groups. 
%
%\begin{exe}
%  \ex
%  \label{ex:itissaidthatshespatouttheegg}
%  \ipaex{æ̃˩ʁv̩˩˥ {\kern2pt}|{\kern2pt} gɤ˩-pʰi˧ le˧-tsʰɯ˧-tsɯ˥}\\
%  \gll æ̃˩ʁv̩˩	gɤ˩-	pʰi˧˥	le˧-	tsʰɯ˩	tsɯ˧˥\\
%  egg	upward	to\_spit	\textsc{accomp}	to\_come	\textsc{rep}\\
%  \glt  ‘It is said that [she] spat out the egg!’ (BuriedAlive2.143)
%\end{exe}
%
%The derivation of the second of these tone groups, /\ipa{gɤ˩-pʰi˧ le˧-tsʰɯ˧-tsɯ˥}/, can be presented as follows: 
%\begin{enumerate}
%	\item[(i)] Lexical tones: /\ipa{gɤ˩- pʰi˧˥ le˧- tsʰɯ˩ -tsɯ˧˥}/
%	\item[(ii)] First level of grouping: /\ipa{{gɤ˩-pʰi˧˥} {le˧-tsʰɯ˩} -tsɯ˧˥}/
%	\item[(iii)] Left"=to"=right computation: the MH pattern of /\ipa{pʰi˧˥}/ spreads rightwards, up to the syllable /\ipa{tsʰɯ˩}/: /\ipa{{gɤ˩-pʰi˧ le˧-tsʰɯ˧˥} -tsɯ˧˥}/
%	\item[(iv)]	Unfolding of the MH \is{tonal contour}contour, overriding (replacing) the lexical tone of the final particle: /\ipa{gɤ˩-pʰi˧ le˧-tsʰɯ˧-tsɯ˥}/
%\end{enumerate}
%
%
As a~second sample \is{derivation!tonal}derivation, let us consider a~slightly more complex example:  (\ref{ex:whenthebigbrothercamebacktheyoungersisterdidntrecognizehim}).

%\subsubsection{Derivation of an~entire sentence: Sister.49}
\label{sec:anentiresentencesister49}

\begin{exe}
  \ex
  \label{ex:whenthebigbrothercamebacktheyoungersisterdidntrecognizehim}
  \ipaex{ə˧mv̩˧˥ {\kern2pt}|{\kern2pt} le˧-tsʰɯ˩ {\kern2pt}|{\kern2pt} tʰi˩˥, {\kern2pt}|{\kern2pt} go˧mi˧ ɳɯ˧ {\kern2pt}|{\kern2pt} ə˧mv̩˧˥ {\kern2pt}|{\kern2pt} mɤ˧-sɯ˥ tsɯ˩ {\kern2pt}|{\kern2pt} mv̩˩!}\\
  \gll ə˧mv̩˧˥		le˧-		tsʰɯ˩\textsubscript{a}	tʰi˩˥		go˧mi˧ ɳɯ˧	ə˧mv̩˧˥		mɤ˧-	sɯ˥		tsɯ˧˥	mv̩˧\\
  elder\_sibling	\textsc{accomp}	to\_come.\textsc{pst}	gap\_filler:well	younger\_sister
  \textsc{a}/\textsc{top}
  elder\_sibling	\textsc{neg}	to\_know	\textsc{rep}	\textsc{affirm}\\
  \glt ‘[When] the big brother came back, the younger sister didn’t recognize him!’ (Sister.49)
\end{exe}

In (\ref{ex:whenthebigbrothercamebacktheyoungersisterdidntrecognizehim}), the group \ipa{{\kern2pt}|{\kern2pt} ə˧mv̩˧˥
{\kern2pt}|{\kern2pt}} ‘elder sibling’ simply consists of a~noun, so that the levels of the tonal word
and tone group coincide. In the absence of suffixes or final particles, the noun's lexical tone, MH, is
realized on the last syllable of the tonal word, which is also the last syllable of the tone
group.
%In the group {\kern2pt}|{\kern2pt} ə˧mv̩˧-ɳɯ˥ {\kern2pt}|{\kern2pt} ‘by the elder sibling’, the rising \is{tonal contour}contour (MH) of /\ipa{ə˧mv̩˧˥}/ ‘elder sibling’ projects its H part onto the \is{suffixes}suffix.
The tone pattern of the group \ipa{{\kern2pt}|{\kern2pt}le˧-tsʰɯ˩ {\kern2pt}|{\kern2pt}} ‘came back’ obtains as described in \tabref{tab:accomplishedpfvcompletion}.
The word /\ipa{tʰi˩˥}/ ‘(and) then, (and) so’ always constitutes a~tone group on its own, as explained in \sectref{sec:someelementsalwaysconstituteatonegroupontheirown}.
The tone pattern of the group \ipa{{\kern2pt}|{\kern2pt}go˧mi˧ ɳɯ˧{\kern2pt}|{\kern2pt}} ‘by the sister’ obtains as described in \tabref{tab:topicfull}.
In the group \ipa{{\kern2pt}|{\kern2pt} mɤ˧-sɯ˥ tsɯ˩ {\kern2pt}|{\kern2pt}}, the verb ‘to know, to recognize’,
//\ipa{sɯ˥}//, is flanked by the {negation} \is{prefixes}prefix //\ipa{mɤ˧-}//
and the sentence"=final particle //\ipa{tsɯ˧˥}// ({reported speech}). The {negation} \is{prefixes}prefix surfaces with its lexical M tone, the verb with its H tone, and the
sentence"=final particle, being preceded by a~H tone, receives L through Rule 4.

The final particle //\ipa{mv̩˧}// ({affirmative}) was already encountered in various examples in which it appears after a~tone"=group break, and carries L tone: \mbox{/\ipa{{\kern2pt}|{\kern2pt} mv̩˩}/.} It does not constitute a~well"=formed tone group, since a~tone group cannot contain only L tones. The special case of this particle, analyzed here as \textit{extrametrical}, brings us to the topic of breaches of tonal grouping, which introduce extrametrical syllables into the system.


\section[Cases of breach of tonal grouping and their consequences]{Cases of breach of tonal grouping and their consequences for the system}
\label{sec:casesofbreachoftonalgroupingandconsequencesforthesystem}

\is{extrametricality|textbf}

Breach of tonal grouping occurs when non"=final syllables come to carry a~\is{tonal contour}contour. This causes following syllables to become extrametrical. In turn, this has consequences for the tonal system: considerations of analytical consistency lead to positing extrametrical syllables even in some contexts where they are not preceded by a~tonal \is{tonal contour}contour, as in the case of the affirmative final particle /\ipa{{\kern2pt}|{\kern2pt} mv̩˩}/ in example (\ref{ex:whenthebigbrothercamebacktheyoungersisterdidntrecognizehim}).

\subsection[Non"=final contours as a~stylistic option]{The stylistic option of realizing a~contour on a~word in non"=final position }
\label{sec:thestylisticoptionofrealizingacontouronawordinnonfinalposition}

Syllables that are not in final position within the tone group cannot carry a~\is{tonal contour}contour (\mbox{//MH//}, \mbox{//LM//} or \mbox{//LH//}). This is an~important part
of the definition of the tone group as a~phonological unit. But this rule is at odds with a~\is{stylistics}stylistic
device whereby a~word is emphasized by cutting short the tone group immediately after it. This
device suspends tonal calculation, and allows the realization of a~\is{tonal contour}contour on the emphasized word,
as in (\ref{ex:byrowingrowingrowingtheyescapedtheymanagedtoescape}).
\begin{exe}
  \ex
  \label{ex:byrowingrowingrowingtheyescapedtheymanagedtoescape}
  \ipaex{le˧-tsɑ˧˥, {\kern2pt}|{\kern2pt} le˧-tsɑ˧˥, {\kern2pt}|{\kern2pt} le˧-tsɑ˧˥ {\kern2pt}|{\kern2pt} -kwɤ˩tɕɯ˩, {\kern2pt}|{\kern2pt} le˧-lv̩˧˥!}\\
  \gll le˧-		tsɑ˧˥		-kwɤ˧tɕɯ˥	le˧-		lv̩˧˥\\
  \textsc{accomp}	to\_row		because	\textsc{accomp}	to\_escape\\
  \glt ‘By rowing, rowing, rowing, they managed to escape!’ (Lake3.59)
\end{exe}

The context to this example is highly emotional: a~mother and her daughter are rowing for their
lives, struggling against the flood that has come over the plain where they lived, now suddenly
become a~lake. The verb ‘to row’ is repeated, and the sentence chopped into short tone groups, as if mimicking the oar chopping into the water in a~rapid succession of rowing strokes. The
verb is strongly articulated phonetically, each time with its lexical rising tone, and the \is{conjunctions}conjunction
/\ipa{-kwɤ˩tɕɯ˩}/ is tacked on at the end as if it were an~afterthought. 
%(The hyphen after the tone group
%\is{boundary (between tone groups)}boundary (/\ipa{{\dots} {\kern2pt}|{\kern2pt} -kwɤ˩tɕɯ˩}/) serves as an~indication that the syllables at issue are
%extrametrical, and do not constitute a~full"=fledged tone group on their own.) 
It would be possible
to integrate the \is{conjunctions}conjunction with the preceding tone group, as in (\ref{ex:byrowingMODIFIED}), with the expected
unfolding of the MH \is{tonal contour}contour over the verb and the first syllable of the \is{conjunctions}conjunction. But this
deliberate, neatly structured \is{variants}variant would be stylistically inappropriate in this context.



\begin{exe}
	\ex
	\label{ex:byrowingMODIFIED}
	\ipaex{le˧-tsɑ˧˥, {\kern2pt}|{\kern2pt} le˧-tsɑ˧˥, {\kern2pt}|{\kern2pt} le˧-tsɑ˧-kwɤ˥tɕɯ˩, {\kern2pt}|{\kern2pt} le˧-lv̩˧˥!}\\
	\gll le˧-		tsɑ˧˥		-kwɤ˧tɕɯ˥	le˧-		lv̩˧˥\\
	\textsc{accomp}	to\_row		because	\textsc{accomp}	to\_escape\\
	\glt ‘By rowing, rowing, rowing, they managed to escape!’ \textit{Modified from} (\ref{ex:byrowingrowingrowingtheyescapedtheymanagedtoescape})
\end{exe}

An example using the same \is{conjunctions}conjunction as in (\ref{ex:byrowingrowingrowingtheyescapedtheymanagedtoescape}), but where the expected division into tone groups is
respected, and where the expected process of unfolding of a~\is{tonal contour}contour tone takes place, is found in (\ref{ex:buuurp}).
\begin{exe}
  \ex
  \label{ex:buuurp} 
  \ipaex{lo˩dʑo˥ {\kern2pt}|{\kern2pt} ʈʂʰɯ˧ne˧-ʝi˥ {\kern2pt}|{\kern2pt} mv̩˩tɕo˧ pʰv̩˧-kwɤ˥tɕɯ˩-ɳɯ˩, {\kern2pt}|{\kern2pt} ``qʰʰʰ{\dots}ə!” {\kern2pt}|{\kern2pt} pi˧ tsɯ˩ {\kern2pt}|{\kern2pt} mv̩˩.~{\kern2pt}|{\kern2pt}}\\
  \gll lo˩dʑo˥	ʈʂʰɯ˧ne˧-ʝi˥	mv̩˩tɕo˧	pʰv̩˧˥	-kwɤ˧tɕɯ˥	-ɳɯ qʰʰʰ{\dots}ə!			pi˥	tsɯ˧˥	mv̩˧\\
  bracelet	thus	downward	to\_take\_off	because		\textsc{top} onomatopoeia:burp!		to\_say
  \textsc{rep}	\textsc{affirm}\\
  \glt ‘When [the man] took off [the buried woman’s bracelets], like this, [the corpse made
    a~gurgling sound]: Buuurp!’ (BuriedAlive2.48)
\end{exe}

This is the only example found so far where the H part of a~verb’s MH \is{tonal contour}contour reassociates to the
\is{conjunctions}conjunction //\ipa{-kwɤ˧tɕɯ˥}//, as against many examples where this \is{tonal contour}contour surfaces as such on the
verb prior to this \is{conjunctions}conjunction (for instance in Dog.49, Tiger.46, BuriedAlive3.65, Caravans.80, Sister.50,
Sister3.133, Seeds2.34, Renaming.18 and Funeral.51). Example (\ref{ex:buuurp}) is just enough evidence
to show that a~realization with \is{tonal contour}contour \is{tone spreading}spreading is possible. Contour unfolding might once have
been the norm, and the realization with a~\is{tonal contour}contour on the verb might have originally been
a~conspicuous \is{stylistics}stylistic effect, but the latter is now much more common than the former, to the point
that the realization with \is{tonal contour}contour unfolding is now a~stylistically marked option.

Realizations of contours in non"=final position are not uncommon (e.g.~Housebuilding.71, 98, 100), each time with a~distinct \is{stylistics}stylistic twist. For instance, in (\ref{ex:bite}), avoidance of \is{tonal contour}contour unfolding has an effect of emphasis, making the statement more eloquent by emphasizing the indication of {certitude} carried by the particle /\ipa{-ɲi˩}/. Contour unfolding (/\ipa{le˧-ʈʰæ˧-ɲi˥}/) would be less forceful in this context. 

\begin{exe}
	\ex
	\label{ex:bite} 
	\ipaex{“mɤ˧-dʑɤ˩ ɲi˩, {\kern2pt}|{\kern2pt} ə˧-sɯ˥! {\kern2pt}|{\kern2pt} ə˧ʝi˧-ʂɯ˥ʝi˩, {\kern2pt}|{\kern2pt} ə˧mi˧-mv̩˩ ɲi˩-kv̩˩, {\kern2pt}|{\kern2pt} zo˩no˥, {\kern2pt}|{\kern2pt} ɬi˧di˩-di˩mi˩ qo˩ dzi˩, {\kern2pt}|{\kern2pt} le˧-wo˥ {\kern2pt}|{\kern2pt} le˧-hɯ˩-zo˩, {\kern2pt}|{\kern2pt} ə˧ʑi˧-ə˧pʰv̩˧-ki˥ {\kern2pt}|{\kern2pt} le˧-hɯ˩-dʑo˩, {\kern2pt}|{\kern2pt} ʈʂʰɯ˧ne˧-ʝi˥, {\kern2pt}|{\kern2pt} ʐæ˩ ɳɯ˥ {\kern2pt}|{\kern2pt} le˧-ʈʰæ˧˥ {\kern2pt}|{\kern2pt} lɑ˧ ɳɯ˧ {\kern2pt}|{\kern2pt} \textbf{le˧-ʈʰæ˧˥ {\kern2pt}|{\kern2pt} -ɲi˩}” {\kern2pt}|{\kern2pt} pi˧-zo˩!}\\
	\gll mɤ˧-	dʑɤ˩\textsubscript{b}	ɲi˩		ə˩-		sɯ˥		ə˧ʝi˧-ʂɯ˥ʝi˩									ə˧mi˧		mv̩˩˥	ɲi˧-kv̩˧˥		zo˩no˥		ɬi˧di˩-di˩mi˩		qo˧		dzi˩\textsubscript{a}		le˧-wo˥le˧-hɯ˩		-zo		ə˧ʑi˧˥		ə˧pʰv̩˧		-ki˧	le˧-		hɯ˧\textsubscript{c}		-dʑo˥			ʈʂʰɯ˧ne˧-ʝi˥	ʐæ˩˥		ɳɯ˧	le˧-		ʈʰæ˧˥		lɑ˧		ɳɯ˧		\textbf{le˧-}	\textbf{ʈʰæ˧˥}		\textbf{-ɲi˩}		pi˥		-zo\\
	\textsc{neg}	good	\textsc{cop}	\textsc{interrog}	to\_know		once\_upon\_a\_time		mother		daughter	two-\textsc{clf}.persons	well	Yongning\_plain		inside		to\_dwell		went\_back		\textsc{advb}	grandmother		grandmother’s\_brother		 \textsc{all}		\textsc{accomp}		to\_go.\textsc{pst}		\textsc{top}	thus		panther	\textsc{a}	\textsc{accomp}			to\_bite 		tiger		\textsc{a}	\textsc{accomp}			\textbf{to\_bite}		\textbf{\textsc{certitude}}		to\_say		\textsc{advb}\\
	\glt ‘[Elders would tell stories about the tiger eating people,] saying: “That is very bad, you know! Once upon a time, a~mother and her daughter who lived in the Yongning plain went to see the grandmother and her brothers; and they were actually attacked by the big cats!”’ \textit{Literally:} ‘they were bitten by the panther, by the tiger’ (Tiger.51)
\end{exe}

A~second \is{stylistics}stylistic consequence of the absence of \is{tonal contour}contour unfolding in (\ref{ex:bite}) is the symmetry between the clauses /\ipa{ʐæ˩ ɳɯ˥ | le˧-ʈʰæ˧˥}/ ‘(a/the) panther bit’ and /\ipa{lɑ˧ ɳɯ˧ | le˧-ʈʰæ˧˥}/ ‘(a/the) tiger bit’, which would be decreased if the tone pattern were different (/\ipa{le˧-ʈʰæ˧˥}/ in the first case, and /\ipa{le˧-ʈʰæ˧-ɲi˥}/ in the second). This symmetry is crucial to connect the story to the Na saying ‘If you see a~tiger, it means your father is going to die; if you see a~panther, it means your mother is going to die’ (example (\ref{ex:seetigerdie}) above). The story is about a~tiger killing a young woman's mother, so there is an~apparent mismatch between the saying (which associates \textit{panther} and \textit{mother} on the one hand, and \textit{tiger} and \textit{father} on the other) and the story (\textit{tiger kills mother}). Mention of both panther and tiger in (\ref{ex:bite}) does not make literal sense, as there is no panther in the story. But this parallel mention clarifies that the seeming opposition between \textit{panther} and \textit{tiger} is not relevant, and guides towards the more general interpretation that is relevant to the story: that seeing a~big cat is a~bad omen, foreboding the death of a~parent. Example (\ref{ex:bite}) provides a clear hint that the panther and tiger in the saying are no more distinct from each other than \textit{cats} and \textit{dogs} in the {English} saying \textit{it's raining cats and dogs}.

\newpage 
Another construction for which the set of narratives contains examples without the unfolding of
a~rising \is{tonal contour}contour is /\ipa{ʈʂʰɯ˧ne˧ gv̩˧˥}/, combining the adverb ‘thus, in this way’ with the verb ‘to
take place, to occur’, as in (\ref{ex:mygrandmotherknewabouteverything}).
\begin{exe}
  \ex
  \label{ex:mygrandmotherknewabouteverything}
  \ipaex{ʈʂʰɯ˧-ʑi˧˥ {\kern2pt}|{\kern2pt} -dʑo˩, {\kern2pt}|{\kern2pt} ʈʂʰɯ˧ne˧ gv̩˧˥ {\kern2pt}|{\kern2pt}
  -ɲi˩! {\kern2pt}|{\kern2pt} tʰv̩˧-ʑi˧˥ {\kern2pt}|{\kern2pt} -dʑo˩, {\kern2pt}|{\kern2pt} ʈʂʰɯ˧ne˧ gv̩˧˥ {\kern2pt}|{\kern2pt} -ɲi˩!}\\
  \gll ʈʂʰɯ˧-ʑi˧˥	-dʑo˥	ʈʂʰɯ˧ne˧	gv̩˧\textsubscript{c}		-ɲi˩		tʰv̩˧-ʑi˧˥\\
  \textsc{dem.prox}-\textsc{clf}.households		\textsc{top}	thus to\_take\_place		\textsc{certitude}	 \textsc{dem.dist}-\textsc{clf}.households\\
  \glt ‘This is what happened to this household! And this is what happened to that household!' (Elders3.44. Context: the narrator reports how her grandmother used to teach children the proper way to behave by taking real"=life examples from events that had occurred within the community.) 
\end{exe}

In this example, a~tone"=group \is{boundary (between tone groups)}boundary is found after both occurrences of the phrase /\ipa{ʈʂʰɯ˧ne˧
  gv̩˧˥}/. The effect is to set this phrase into relief. The following morpheme also stands out by not being incorporated within the same tone group. This morpheme, /\ipa{-ɲi˩}/, is a~grammaticalized form of the \isi{copula}, used to convey “an epistemic strategy that marks a~high degree of certitude” \citep[497]{lidz2010}. The context helps clarify what happens here: this is a~passage in reported speech, in which the narrator adopts the tone of voice of her grandmother, whom she views as the highest authority on traditional lore. 

\begin{quotation}
	My grandmother knew about everything! In the old times, she would also tell us stories about people in the village, and what we
	must learn from them: “This household, this is what happened to them! That household, this is
	what happened to them! One must develop habits of doing
	good! One mustn’t do wrong!” (Elders3.44-45) 
\end{quotation}

The \is{stylistics}stylistic choice of adding a~tone"=group \is{boundary (between tone groups)}boundary before the morpheme indicating {certitude} conveys the assertiveness of the character to whom this passage in reported speech is assigned. Realization as /\ipa{ʈʂʰɯ˧ne˧ gv̩˧˥ {\kern2pt}|{\kern2pt} -ɲi˩}/ rather than /\ipa{ʈʂʰɯ˧ne˧ gv̩˧-ɲi˥}/
conveys an authoritative attitude. On the other hand, when the phrase ‘This is how it happened’ is told more casually, as
an~introduction to a~narrative, it constitutes one tone group: see (\ref{ex:howithappened}). A~related {phrasing}, which likewise constitutes a~set phrase and is therefore integrated into
one tone group, is shown in (\ref{ex:howitwouldhappen}).

\begin{exe}
	\ex
	\label{ex:howithappened}
	\ipaex{ʈʂʰɯ˧ne˧ gv̩˧-ɲi˥ tsɯ˩ mv̩˩.}\\
	\gll ʈʂʰɯ˧ne˧	gv̩˧\textsubscript{c}		-ɲi˩		tsɯ˧˥	mv̩˧\\
	thus		to\_take\_place		\textsc{certitude}	 \textsc{rep}		\textsc{affirm}\\
	\glt ‘This is how it happened.’ (BuriedAlive.1)
\end{exe}

\begin{exe}
	\ex
	\label{ex:howitwouldhappen}
	\ipaex{ʈʂʰɯ˧ne˧ gv̩˧-kv̩˥.}\\
	\gll ʈʂʰɯ˧ne˧	gv̩˧\textsubscript{c}		-kv̩˧˥\\
	thus		to\_take\_place		\textsc{abilitive}\\
	\glt ‘This is how it would happen.’ (Sister3.149, Tiger.2)
\end{exe}

Example (\ref{ex:afterherushedawayhisyoungerdaughtercriedhereyesout}) shows that the \is{stylistics}stylistic device whereby a~tone group is cut short after a~certain word can be applied as early as the first syllable of a~sentence. A more strongly integrated formulation would be /\ipa{pʰo˩-hɯ˩-kwɤ˩tɕɯ˥-lɑ˩}/, without any special
emphasis on the verb ‘to flee/to rush’.

\begin{exe}
	\ex
	\label{ex:afterherushedawayhisyoungerdaughtercriedhereyesout}
	\ipaex{pʰo˩˥ {\kern2pt}|{\kern2pt} hɯ˧-kwɤ˧tɕɯ˥-lɑ˩ {\kern2pt}|{\kern2pt} tʰi˩˥, {\kern2pt}|{\kern2pt} go˧mi˧ ʈʂʰɯ˧-v̩˧-dʑo˩, {\kern2pt}|{\kern2pt} le˧-ŋv̩˩, {\kern2pt}|{\kern2pt} le˧-ŋv̩˩, {\kern2pt}|{\kern2pt} le˧-ŋv̩˩, {\kern2pt}|{\kern2pt}
		le˧-ŋv̩˩, {\kern2pt}|{\kern2pt} le˧-ŋv̩˩-zo˩!}\\
	\gll pʰo˩\textsubscript{a}		hɯ˧\textsubscript{c}		-kwɤ˧tɕɯ˥-lɑ˩		tʰi˩˥ go˧mi˧	ʈʂʰɯ˥	v̩˧	-dʑo˥	le˧-	ŋv̩˩	-zo\\
	to\_flee/to\_rush	to\_go.\textsc{pst}	after		then younger\_sister
	\textsc{dem}.\textsc{prox}
	\textsc{clf}	\textsc{top}	\textsc{accomp}	to\_cry	\textsc{advb}\\
	\glt ‘After he rushed away, [his] younger daughter cried her eyes out!’ (Sister3.68)
\end{exe}


\subsection[The emergence of extrametrical syllables]{Consequences for the tone system: The emergence of extrametrical syllables}
\label{sec:consequencesforthetonesystemtheemergenceofextrametricalsyllables}

The phenomenon whereby a~tone group is cut short after a~certain word (noun or verb) has consequences for the general architecture of the tone system. In cases where the portion of tone
group that is cut off from the verb can stand on its own as a~tone group, the tenets of the system
remain unaffected, such as in example (\ref{ex:thechinesehaneatdogmeat}).

\begin{exe}
  \ex
  \label{ex:thechinesehaneatdogmeat}
  \ipaex{hæ˧, {\kern2pt}|{\kern2pt} kʰv̩˩mi˩-ʂe˩˥ {\kern2pt}|{\kern2pt} dzɯ˧-kv̩˩!}\\
  \gll hæ˧		kʰv̩˩mi˩-ʂe˩	dzɯ˥		-kv̩˧˥\\
  Chinese	dog\_meat	to\_eat		\textsc{abilitive}\\
  \glt ‘The Chinese (Han) eat dog meat!’ (Field notes, 2012)
\end{exe}

Example (\ref{ex:thechinesehaneatdogmeat}) lays emphasis on ‘dog meat’. In the Na world view, dogs
and men are close friends: the dog agreed to exchange its sixty"=year lifespan with the thirteen"=year
lifespan that had initially been granted to man (see the narrative Dog). Eating dog meat is
therefore taboo among the Na, and the fact that some other ethnic groups do eat dog meat comes to
them as a~shock. An unmarked {phrasing} of (\ref{ex:thechinesehaneatdogmeat}) would be /\ipa{hæ˧ {\kern2pt}|{\kern2pt}
  kʰv̩˩mi˩-ʂe˩ dzɯ˩-kv̩˥}/, in which a~single tone group spans the object and verb, and tonal
computation takes place.

On the other hand, when particles or conjunctions are left stranded, as in
(\ref{ex:byrowingrowingrowingtheyescapedtheymanagedtoescape}), they do not constitute a~well"=formed tone group
on their own. The rules recapitulated in \sectref{sec:alistoftonerules}, such as the addition of a~final H tone to all-L
sequences, do not apply to them~-- otherwise one would expect a~final rising \is{tonal contour}contour in (\ref{ex:byrowingrowingrowingtheyescapedtheymanagedtoescape}):
\ipa{$\dagger$le˧-tsɑ˧˥ {\kern2pt}|{\kern2pt} -kwɤ˩tɕɯ˩˥}. Nor are these stranded syllables integrated into the following
tone group.

Several options for analysis are open here. One would be to consider that, at some
phonological level, the division into tone groups is in fact left unchanged. This would entail that
a~\is{tonal contour}contour can be realized in non"=final position within a~tone group, an~implication which contradicts headlong the definition of the tone group used under the present analysis. A~preferred option is to consider that the emphasis laid on a~word,
and the consequent realization of a~\is{tonal contour}contour on that word, modifies the utterance’s division into
tone groups, and that the syllables left stranded acquire extrametrical status. The notion of
extrametricality rescues the general rule which serves as one of the key criteria for the
definition of the tone group as a~phonological unit, i.e.\ that contours only appear
tone"=group"=finally. 

Consider example (\ref{ex:pluckbutton}):
\begin{exe}
  \ex
  \label{ex:pluckbutton}
  \ipaex{pv̩˩ɭɯ˥ {\kern2pt}|{\kern2pt} ɖʐɤ˧˥ {\kern2pt}|{\kern2pt} ki˩ tsɯ˩ {\kern2pt}|{\kern2pt} mv̩˩.}\\
  \gll pv̩˩ɭɯ˥	ɖʐɤ˧˥		ki˧		tsɯ˧˥	mv̩˧\\
  button	to\_pluck	to\_give		\textsc{rep}	\textsc{affirm}\\
  \glt ‘It is said that [he] plucked a~[button from his jacket] and gave it [to the child].~/ He
  plucked a~button and gave it [to the child].’ (Renaming.23)
\end{exe}

At least three \is{stylistics}stylistic options are open here. The most tightly"=knit would involve a~single tone group:
/\ipa{ɖʐɤ˧ ki˥ tsɯ˩}/.\footnote{For the sake of simplicity, the {affirmative} final particle is left out of the analysis; it will be discussed in \sectref{sec:furtherexamplesofextrametricalelements}.} The most analytic would involve two full"=fledged tone groups: /\ipa{ɖʐɤ˧˥
  {\kern2pt}|{\kern2pt} ki˧ tsɯ˧˥}/. The third one, found in (\ref{ex:pluckbutton}), is intermediate: the verb /\ipa{ɖʐɤ˧˥}/ ‘to pluck’ is realized with its lexical MH
\is{tonal contour}contour, as if it were tone"=group"=final, and the syllables that follow are all lowered to L, as if
they belonged to the preceding tone group. The syllables /\ipa{ki˩ tsɯ˩}/  are extrametrical: they do not constitute a~full"=fledged tone group on their own.

This range of \is{stylistics}stylistic \isi{variation} is a~salient characteristic of Yongning Na.\Hack{\break} Among other potential
consequences for the evolution of the tone system, extrametrical syllables at the end of a~tone
group may tend to become affiliated to the following tone group instead, in cases where the sequence
of (surface)\is{form!surface} tones allows for this reinterpretation. A~case in point is the highly frequent sequence of topic marker //\ipa{-dʑo˥}// and discourse marker //\ipa{tʰi˩˥}// ‘so, then’. The latter
makes up a~tone group on its own, as was mentioned in \sectref{sec:someelementsalwaysconstituteatonegroupontheirown}. However, in the narratives
recorded by consultant F4, it is not preceded by any perceived pause, whereas there tends to be a~pause
before the topic marker, so that the two morphemes are pronounced
in quick succession. There is thus a~discrepancy between two levels: that of the division into tone
groups, on the one hand, and that of linguistic rhythm, on the other. Now, the topic marker //\ipa{-dʑo˥}// most often surfaces as /\ipa{-dʑo˩}/, due to the presence of a~H tone earlier on in the tone group, and the sequence of /\ipa{-dʑo˩}/ and /\ipa{tʰi˩˥}/ would constitute a~well"=formed tone group. The tone sequence L.LH can be the surface realization of underlying //L.LH//, or of underlying //L//. One may speculate that the
high discourse frequency of the /\ipa{-dʑo˩ tʰi˩˥}/ sequence, which on the phonological surface looks like a~tightly"=knit tone group, paves the
way for its reinterpretation as one tone group. Such reinterpretation is especially tempting in contexts such as (\ref{ex:actual fact}).

\begin{exe}
	\ex
	\label{ex:actual fact}
	\ipaex{{\dots} gɯ˩-ʝi˥ {\kern2pt}|{\kern2pt}
		-dʑo˩ {\kern2pt}|{\kern2pt} tʰi˩˥ {\dots}}\\
	\gll gɯ˩-ʝi˥	-dʑo˥		tʰi˩˥\\
	really		\textsc{top}		then\\
	\glt ‘{\dots} in actual fact, {\dots}’ (Mountains.58)
\end{exe}

As explained in Appendix A (\sectref{sec:articulatoryreductionreducedformsandtheirlexicalization}), the expression /\ipa{gɯ˩-ʝi˥}/ ‘really, truly’ (from /\ipa{gɯ˩}/ ‘authentic, true’) is well on its way towards reduction to a~\is{monosyllables}monosyllable: to my ears, it sounds like [\ipa{gi˩˥}] except when hyperarticulated. In (\ref{ex:actual fact}), for instance, /\ipa{gɯ˩-ʝi˥}/ sounds very much like a~\is{monosyllables}monosyllable carrying a~rising \is{tonal contour}contour: [\ipa{gi˩˥}]. Now, a~rising \is{tonal contour}contour signals the end of a~tone group. While the topic marker that follows, /\ipa{-dʑo˩}/, does not constitute a~well"=formed tone group, the sequence /\ipa{-dʑo˩ tʰi˩˥}/ would constitute one. To labour the point: although in data from speaker F4 the division into tone groups is clearly /\ipa{{\dots}-dʑo˩ {\kern2pt}|{\kern2pt} tʰi˩˥} {\kern2pt}|{\kern2pt}/, this sequence could easily be interpreted by a~language learner as a~L"=tone group: /\ipa{{\kern2pt}|{\kern2pt} -dʑo˩ tʰi˩˥ {\kern2pt}|{\kern2pt}}/.

\newpage 
\subsection{Further examples of extrametrical elements}
\label{sec:furtherexamplesofextrametricalelements}

Additional language"=internal evidence for resorting to the concept of
extrametricality in the description of the Na tone system comes from the {affirmative} particle //\ipa{-mv̩˧}// and the expression /\ipa{ə˩-gi˩}/ ‘isn’t it!’, ‘right!’

The
{affirmative} particle //\ipa{-mv̩˧}// cannot host a~H level from a~preceding
{reported"=speech} particle //\ipa{tsɯ˧˥}//: the sequence is realized as /\ipa{tsɯ˧˥ mv̩˩}/, not
\ipa{$\ddagger${\kern2pt}tsɯ˧ mv̩˥} or \ipa{$\ddagger${\kern2pt}tsɯ˧˥ {\kern2pt}|{\kern2pt} mv̩˧}. This is a~case that seems to be best handled in terms of extrametricality.

The expression /\ipa{ə˩-gi˩}/ ‘isn’t it!’, ‘right!’ is commonly tagged at the end of an~utterance. Two observations suggest that this expression constitutes a~tone group on
its own. First, a~preceding LH or MH \is{tonal contour}contour does not unfold over it, as would be expected inside a~tone group
(Caravans.257, 287; Housebuilding.113; Mountains.159; Sister3.86). Secondly, the expression /\ipa{ə˩-gi˩}/ is often preceded by a~short
(perceived) pause. On the other hand, the fact that the expression /\ipa{ə˩-gi˩}/ only contains L tones
implies that it does \textit{not} constitute a~tone group on its own, otherwise it would be realized as
/\ipa{ə˩-gi˩˥}/ (following Rule~7). The latter, /\ipa{ə˩-gi˩˥}/, is well"=formed and attested in the narratives, but it is a~full"=fledged {question} (‘Is it true?’), whereas /\ipa{ə˩-gi˩}/ is more phatic, almost a~gap"=filler. For these reasons, the expression /\ipa{ə˩-gi˩}/ is here treated as extrametrical. In the transcriptions,
it is preceded by a~tone group \is{boundary (between tone groups)}boundary, to reflect the fact that it does not interact tonally with
what precedes it.

To sum up: the tone group may be interrupted after the last syllable of
a~word (generally, but not exclusively, a~verb or noun), leaving some syllables stranded. These syllables are described as having
extrametrical status.


\subsection[Deviant tone patterns and Mandarin loanwords]{Deviant tone patterns and Mandarin loanwords: Does the existence of extrametrical syllables facilitate the introduction of loanwords with a~non"=final rising tone?}
\label{sec:extrametricalconsolidates}

The phenomena described above in terms of \textit{\isi{extrametricality}} are clearly marginal. Yet they pave the way for increasingly significant changes to the tone system as a~whole. They introduce unusual tone patterns which may become consolidated through loanwords: once a~pattern exists in the language, however peripheral it may be, it is available for accommodating foreign combinations of sounds. 

\newpage 
For instance, the
gap"=filler \textit{jiùshi} \zh{就是}, ‘quite right; exactly, precisely, just’ is borrowed as
/\ipa{tɕo˧˥ʂɯ˩}/, with a~word"=internal MH \is{tonal contour}contour.\footnote{Note that the \is{loanwords}borrowing is from \il{Mandarin!Southwestern|textbf}Southwestern Mandarin, where the syllable /\ipa{tɕo}/ \zh{就} carries a rising tone, not a falling tone as in \il{Mandarin!Standard}Standard Mandarin \citep[on Southwestern Mandarin:][]{guiyunnanese2001, pinson2008}.} At first blush, this contravenes a~basic phonotactic rule of Na: the restriction of contours to tone"=group"=final position. On the other hand, the process of emphasis described in \sectref{sec:thestylisticoptionofrealizingacontouronawordinnonfinalposition} introduces tone"=group"=internal contours, which have now become habitually associated with some morphemes. The existence of these rising contours arguably facilitated the introduction of {Mandarin} loanwords with this tone pattern. In turn, loanwords contribute to the gradual spread of the previously deviant phonotactic pattern.

To carry the argument one step further, assuming that the emphatic value of tone"=group"=internal contours predates the \is{loanwords}borrowing of the gap"=filler \textit{jiùshi} \zh{就是} ‘exactly’ as /\ipa{tɕo˧˥ʂɯ˩}/, this expressive value may well have facilitated the retention of the rising tone at \is{loanwords}borrowing. Emphasis is well"=suited to this item. A~hint of emphasis or insistence is not inappropriate for a~gap"=filler: it can help convey to the addressee that the speaker wishes to keep their~speech turn open. The word /\ipa{tɕo˧˥ʂɯ˩}/ is also used as a~rejoinder (‘Exactly!’), in Yongning Na as in \ili{Mandarin}. In this usage too, a~touch of emphasis is welcome, highlighting the intended message of convergence of viewpoints between the interlocutors. To put it differently, the item's lexical tone in \ili{Mandarin} is congruent with its expressive interpretation in terms of Na \isi{prosody}.

A~cross"=linguistic analogue to this situation is found in the success of the \ili{Vietnamese} \is{loanwords}loanword \textit{nhà quê} in \ili{French}. The original \ili{Vietnamese} is a~derogatory term: ‘yokel, hayseed, country bumpkin, backwoods person’. It was borrowed into \ili{French} as \textit{niakoué} (also spelt as \textit{niacoué}) as a~derogatory term for the \ili{Vietnamese}, and later also for the Chinese. Lexical tones were lost in the process of \is{loanwords}borrowing, but the vowels and consonants match exactly: \ili{Vietnamese} /\ipa{ɲa.kwe}/ was borrowed as /\ipa{ɲa.kwe}/. Initial \ipa{ɲ} has expressive value in \ili{French}, as shown by a~quick list of items that contain it: \textit{gnan"=gnan} ‘mawkish, mushy’, \textit{gniaf} ‘cobbler’, \textit{gn(i)ard} ‘child’, \textit{gn(i)ouf} ‘prison’, \textit{gnognot(t)e} ‘worthless stuff’, \textit{gnolle} ‘futile person’, \textit{gnôle} ‘alcohol, hard stuff’, \textit{gnon} ‘blow’, \textit{niaque (gnaque)} ‘combativeness’ are all slang words. The bad guy in Lyon's puppet theatre is revealingly named \textit{Gnafron}. The only \is{exceptions}exception in the list is an~Italian \is{loanwords}loanword, \textit{gnocchi}, which apparently managed to gain integration despite the slangy flavour of its initial \ipa{ɲ}. Seen in this light, \ili{Vietnamese} \textit{nhà quê} /\ipa{ɲa.kwe}/ presumably owes some of its success in \ili{French} to the overtones of its initial consonant.

Returning to Yongning Na, word"=initial contours in Chinese words used by consultant F4 are not restricted to expressively loaded words. The word for ‘television’ is a~case in point.\footnote{Remember that the borrowings are from Southwestern {Mandarin}, where tone values are almost the reverse of Standard (Beijing) Mandarin, so that the syllable \textit{diàn} \zh{电} in \textit{diànshì} \zh{电视} ‘television’ carries a rising tone, not a falling tone as in Standard Mandarin.} Realizations \is{form!in isolation}in isolation fluctuate between /\ipa{tjɤ˩˥ʂɯ˧}/ (with a~tone pattern unattested in Yongning Na, apart from Chinese words), /\ipa{tjɤ˧ʂɯ˧}/ (suggesting an underlying M or \#H tone), and /\ipa{tjɤ˩ʂɯ˧˥}/ (unambiguously pointing to an underlying LM+MH\# tone). The noun's tonal behaviour in context also fluctuates. The tones of /\ipa{tjɤ˩ʂɯ˧ li˥}/ ‘to watch television’, /\ipa{tjɤ˩ʂɯ˧-qo˥}/ ‘on TV’ and /\ipa{tjɤ˩ʂɯ˧ ɲi˥}/ ‘is \mbox{(a/the)} TV’ would suggest that the noun carries either LM+\#H tone or LM+MH\# tone (in light of the data set out in \sectref{sec:encliticsthatcarrymtonewhenfollowingamtonenoun}, \sectref{sec:objectandnonprefixedverb}, and \sectref{sec:overviewofthesystem}, respectively), but a~\is{tonal contour}contour is sometimes realized on the first syllable: realizations such as /\ipa{tjɤ˩˥ʂɯ˧ ɲi˥}/ ‘is \mbox{(a/the)} TV’ are observed.

This is part of a~general state of flux characterizing recent Chinese \isi{loanwords}, also evidenced by fluctuation in the rhyme of the word for ‘television’, between /\ipa{jɤ}/ (which complies with Na phonotactics) and /\ipa{je}/ (which does not, and is phonetically closer to the Chinese model). These words remain perceived by consultant F4 as Chinese words, and have not acquired a~stable Na form, but their presence paves the way for changes in the morphotonological system.

\section{Concluding note}
\label{sec:concludingremark}

The division of an~utterance into tone groups plays a~central role in conveying \isi{phrasing} and
prominence. In this respect, the Na facts appear closely parallel to the division of
sentences into intonational groups in \ili{English} (or \ili{French})~-- extensively studied languages, for which a~wealth of references is available (on \ili{French}, see for instance \citealt{vaissiere1975,dicristo1998,rossi1999,martin2015}). A
striking characteristic of Na is the constant interaction between these intonational choices and the
language’s tonal processes.

\is{tone group|)}
