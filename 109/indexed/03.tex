\chapter{Compound nouns}
\label{chap:compoundnouns}

% indexing 'compound' for whole chapter
\is{compounds|(}

Tonal processes applying within the noun phrase constitute a~major part of the Yongning Na tone
system. They also shed light on evolutionary processes. In Na, as in other {Sino"=Tibetan} languages
that have undergone considerable \isi{phonological erosion} (such as \ili{Tujia} \zh{土家语}, \ili{Bai} \zh{白语}, \ili{Namuyi} \zh{纳木依语}, or \ili{Shixing} \zh{史兴语}), many
roots that used to be phonologically distinct have become \is{homophony}homophonous. As a~consequence, there
exists a~strong tendency towards \isi{disyllabification}. The study of synchronic tonal processes reveals which processes~-- such as compounding and {affixation}~-- feed into which categories of
\is{disyllables}disyllabic nouns. It also brings out, by contrast, those disyllabic nouns whose tones are different from what one would expect in view of currently productive rules. In turn, this draws attention to these
outlier nouns, raising the issue of where they got their tone from~-- whether they date back to
a~time when different tone rules applied, for instance.

Tonal phenomena taking place within the noun phrase in Na will be presented in the following order:
compounds (this chapter); {numeral}"=plus"=classifier phrases (Chapter~\ref{chap:classifiers}); and combinations between
nouns and grammatical morphemes (Chapter~\ref{chap:combinationsofnounswithgrammaticalwords}).

Compounding is a~highly productive word formation process in Na. “Compounding is the
prevalent morphological process” \citep[344]{lidz2010}. This is also true of many other languages of East and Southeast Asia (on {Sinitic}: \citealt[passim]{arcodia2012}). \is{compounds!determinative|textbf}Determinative compounds (the \textit{tatpuruṣa} compounds of Sanskrit grammar) are
more common than coordinative compounds\is{compounds!coordinative|textbf} (Sanskrit \textit{dvandva}). In Yongning Na,
determinative compounds, such as ‘tiger’s skin’, and coordinative compounds, such as ‘mother and
daughter’, do not follow the same tone rules. For instance, the determinative compound ‘nanny goat’s
back’, /\ipa{tsʰɯ˧mi˧-gv̩˧dv̩˥}/, carries H\# tone (a final H tone), whereas the coordinative compound
‘father and mother’, /\ipa{ə˧dɑ˧-ə˧mi\#˥}/, carries tone \#H (a \is{floating tone}floating H tone), even though the
input tones are the same: both ‘nanny goat’ and ‘father’ have H\$ tone, and both ‘mother’ and ‘back’
have M tone. Determinative compounds and coordinative compounds are therefore presented separately.



\section{Determinative compound nouns. Part I: The main facts}
\label{sec:determinativecompoundnouns}

In Yongning Na, the
order of constituents in determinative compounds is determiner plus head, as is generally the case in \il{Sino-Tibetan}Sino"=Tibetan
\citep{michailovsky2011}.

In some tonal languages, \isi{possessive} constructions (genitival syntagms) and compounds (complex
lexemes) are distinguished by their tone patterns. In Kita Malinke (\ili{Mande} branch of Niger"=Congo), for instance, ‘the meat of the cow’ is
/\ipa{mìsí sùbû}/, and ‘beef, cow meat’ is /\ipa{mìsì-súbú}/ \citep{creisselsetal1993}. The
latter is characterized by tonal compactness (\textit{compacité tonale}): the tone pattern of the
compound is determined by that of its first component, which is the determiner. Similarly, in
Yongning Na, no tonal change takes place in \isi{possessive} constructions, whereas tonal changes take
place in compounds~-- although the tone changes are more complex than in Malinke, as will be explained further down. In Na, the two constructions are conspicuously different: in \isi{possessive}
constructions, the \isi{possessive} /\ipa{=bv̩˧}/ is added after the determiner, before the head,
e.g.~/\ipa{hwɤ˧li˧˥}/ ‘cat’, /\ipa{ɬv̩˧˥}/ ‘brains’, /\ipa{hwɤ˧li˧=bv̩˥ {\kern2pt}|{\kern2pt} ɬv̩˧˥}/ ‘brains of the
cat’. The first noun~-- the determiner~-- and the \isi{possessive} particle form a~single \isi{tone group}. As
for the second noun (the head), its tone pattern remains the same as \is{form!in isolation}in isolation. By contrast, the tones of compounds are not simply the concatenation of those of their constituents. The present
analysis progresses in increasing order of abstraction, from the \is{form!surface}surface phonological patterns of
compounds to the underlying system.

Determinative compounds are sometimes divided into “free” and “fixed” combinations. The former consist of two nouns that are not
habitually associated, e.g.~/\ipa{gi˧nɑ˧mi˧-njɤ˥ɭɯ˩}/ ‘bear’s eye’: the two nouns are combined into
a~noun phrase in the context of a~given utterance. The
latter constitute lexicalized combinations, e.g.~/\ipa{ʑi˩hṽ̩\#˥}/ ‘body hair (of humans)’, literally ‘monkey’s hair’. There is a~cross"=linguistic tendency for compounds to stray away from regular morphophonological
patterns and from the semantics that one would expect on the basis of their
constituting elements. But the meaning may
be specialized whereas the phonological form
remains undistinguishable from that of a~newly coined compound; conversely, the phonological form may
be irregular whereas the meaning is as expected on a~flatly synchronic basis. 

\begin{quotation}
	Thus in Zarma, \textit{háw bíì} /ox/black/ is a~syntagm with a~perfectly regular form, which would
	be expected to mean ‘black ox’, but which refers to the buffalo~-- an~animal that resembles the ox,
	and whose colour is black. In \textit{cùrò bíì}, one easily recognizes \textit{cúrò} ‘bird’ and
	\textit{bíì} ‘black’, but the meaning is ‘guinea fowl’; in this case, semantic specialization is
	accompanied by a~tonal irregularity: a~syntagm meaning ‘black bird’ would be expected to have the
	form \textit{cúrò bíì}.~\citep[121]{creissels1991}\footnote{\textit{Original text}: Ainsi en zarma, \textit{háw bíì} /bœuf/noir/ est un syntagme de
		formation parfaitement régulière dont on attendrait qu’il signifie «~bœuf noir~», mais qui
		désigne le buffle (animal semblable au bœuf et de couleur noire). Dans \textit{cùrò bíì}, nous
		reconnaissons facilement \textit{cúrò} «~oiseau~» et \textit{bíì} «~noir~», mais la signification
		est «~pintade~» ; dans ce cas, le figement sémantique s’accompagne d’une irrégularité tonale : le
		syntagme signifiant «~oiseau noir~» serait \textit{cúrò bíì}.}
\end{quotation}

The lack of straightforward match between semantic regularity and morphophonological
similarity needs to be taken into account when exploring a~tone
system. Semantics cannot be used as the sole criterion to tease apart
morphophonologically irregular compounds. The distinction made in this chapter is therefore not between “fixed” and “free” combinations, but between regular
and irregular combinations. From a~morphotonological point of view, the relevant parameter is
whether the tone pattern of a~compound follows productive rules or not.

\subsection{The role of the number of syllables}
\label{sec:theroleofthenumberofsyllables}

Tonal changes in compounding are only observed when the second term~-- the head~-- has fewer than
three syllables, i.e.\ in combinations of the form σ+σ, σ+σσ, σσ+σ, σσ+σσ, σσσ+σ, or
σσσ+σσ. Otherwise no tone change takes place (for instance in σ+σσσ, σσ+σσσ, or σσσ+σσσ). What matters is thus not the total number of syllables
of the resulting compound, but the number of syllables of the head. An illustration is provided by examples (\ref{ex:lugulake}--\ref{ex:bearseye}).

\begin{exe}
  \ex
  \begin{xlist}
    \ex \label{ex:lugulake}
    \ipaex{lo˧ʂv̩˩ {\kern2pt}|{\kern2pt} -hi˩nɑ˧mi\#˥}\\
	\gll lo˧ʂv̩˩		hi˩nɑ˧mi\#˥\\
	Luoshui~(village~name)		lake\\
    \glt ‘Lake Lugu’ (\textit{literally:} ‘the lake of Luoshui’)
    \ex \label{ex:yongningplain}
    \ipaex{ɬi˧di˩-di˩mi˩}\\
	\gll ɬi˧di˩		di˧mi˧\\
		Yongning~(place~name)		large\_plain\\
    \glt ‘Yongning plain’
    \ex \label{ex:bearseye}
    \ipaex{gi˧nɑ˧mi˧-njɤ˥ɭɯ˩}\\
	\gll gi˧nɑ˧mi\#˥		njɤ˩ɭɯ˧\\
	bear		eye\\
    \glt ‘bear’s eye’
  \end{xlist}
\end{exe}

The place names in (\ref{ex:lugulake}) and (\ref{ex:yongningplain}) have the same syntactic structure: /\ipa{lo˧ʂv̩˩}/ (Chinese: Luòshuǐ \zh{落水}) is the name of a~village on the shore of Lake Lugu, and /\ipa{ɬi˧di˩}/ is the name of Yongning. The relationship in both cases is between determiner and
head: ‘the lake of /\ipa{lo˧ʂv̩˩}/’, ‘the plain of /\ipa{ɬi˧di˩}/’. In (\ref{ex:lugulake}), both parts of the
compound retain their lexical tones: /\ipa{lo˧ʂv̩˩}/ ‘Luoshui’ and /\ipa{hi˩nɑ˧mi\#˥}/ ‘lake’. The
compound ‘Lake Lugu’, /\ipa{lo˧ʂv̩˩ {\kern2pt}|{\kern2pt} -hi˩nɑ˧mi\#˥}/, is to be analyzed as consisting of two tone
groups (this is reflected by the symbol ‘\ipa{|}’, which indicates a~\isi{tone group} boundary); if it constituted one \isi{tone group}, its tone pattern would be $\ddagger${\kern2pt}\ipa{lo˧ʂv̩˩-hi˩nɑ˩mi˩},
by application of Rule 5: “All syllables following a~H.L or M.L sequence receive L tone”. (For
a~list of the tone rules, see \sectref{sec:alistoftonerules}.) In (\ref{ex:yongningplain}), the expected
tone change takes place, by application of Rule 5: the lexical tone of ‘plain’ is M
(/\ipa{di˧mi˧}/), but in the context of this compound it is lowered to L. Example (\ref{ex:bearseye}), from /\ipa{gi˧nɑ˧mi\#˥}/
‘bear’ and /\ipa{njɤ˩ɭɯ˧}/ ‘eye’, illustrates the fact that tonal change takes place in compounds
with a~three"=syllable determiner, provided that the head comprises no more than two syllables.

It is clear from the data set out below (\tabref{tab:differencesbetweenspeakersf4andm21} and following
tables) that heads undergo more tonal changes than determiners in compounding. In particular, there
are numerous cases where a~H tone that originates lexically on the determiner associates to
the last syllable of the head. If this happens in a~σσ+σ compound, the distance between the syllabic position to which the tone is lexically attached and the syllabic position of the tone in the surface phonology is no greater than one syllable. In a~σσ+σσ compound, this distance increases to two syllables. 
%(In this rule"=of"=thumb calculation, the H tone is considered to be associated to the lexical word's last syllable: remember that H tones never appear on a~word"=initial syllable.) 
The same process applying to a~σσ+σσσ compound would become rather unwieldy, as it would result in a~H tone moving three syllables away from its original position in the lexical representation. This is by no means a~cognitive impossibility: staggeringly complex tonal phenomena are firmly attested in the world's languages, and by some estimates the Alawa dialect of Yongning Na would rate as a~complex system. Still, it seems clear that in this instance the observed pattern (dividing the compound into two parts) is a~strategy to bypass tonal computation, thereby avoiding a~source of complexity. The asymmetrical state of affairs whereby σσσ+σσ compounds undergo tonal change, but σσ+σσσ compounds do not, makes intuitive sense in light of the amount of tonal computation that the latter require. 

There is thus a~preference (in this particular language and dialect) for tonal processes that do not result in tonal movements of more than two syllables at a~time. Long"=distance movement of tones is avoided. This observation will be taken up in \sectref{sec:longdistancedispreferred}. 


\subsection{How the tone patterns were collected}
\label{sec:howthetonepatternswerecollected}
\largerpage[-1] %longdistance
Some lexicalized compounds can
be found in narratives, e.g.~/\ipa{ə˧mi˧-ʁæ˧ʈv̩˥}/ {\linebreak}‘mother’s neck’ in Tiger2.86. Others were encountered during lexical elicitation sessions, and are consigned in my dictionary of Yongning Na \citep{michauddict2015}. In order to obtain all possible tonal combinations of determiner and head,
systematic elicitation was also used. The main language consultant, F4, was reluctant to
accept semantically implausible combinations. She gradually understood that the unusual combinations
that I put forward were designed to obtain a~particular sequence of tones; she nonetheless retained
a~strong commitment to a~common"=sense use of language. In the consultant’s view, compounds such as ‘flea’s back’ and ‘flea’s liver’ did not stretch
plausibility too far, and I gratefully recorded them. But she would definitely not have accepted
combinations such as ‘chief’s beetle’s kidney basket’ (to cite an~example used by \citealt{hyman2007a}, taken
up in \citealt[225]{evans2010}). 
Compounds that did not make sufficiently good sense, such as ‘woman’s blood’, were produced either
as \isi{possessive} constructions~-- an~{English} equivalent would be ‘blood of a/the woman’, as opposed to the
desired ‘woman’s blood’~-- or as an~ungrammatical expression consisting in the mere juxtaposition
of the citation forms of the two words. 

It was nonetheless possible to obtain all combinations in the end, by searching
through the word list to arrive at the least implausible combinations, and discussing possible
contexts with the consultant. In the case of ‘woman’s blood’, her argument was that there is nothing specific about the blood of women, and thus no need to distinguish it from that of men: women’s blood is no more and no less than human blood. To get round this issue, the imagined context was that a~man"=eating demon feels an~urge to drink \textit{woman’s} \textit{blood}. Each combination was then checked multiple times by using different example words for the same
tonal combination, and eliciting tokens several times during different elicitation sessions. 
This made the elicitation process slower than it would have been with a~consultant who
readily agreed to create any combination. 
%The elicitation of data from four speakers about this particular point of the tonal
%grammar extended over the first three field trips, from 2006 to 2008. 
A~consultant’s {conservative} behaviour may have some
advantages, however: one might have suspicions that a~consultant whose imagination runs free
from the trammels of common sense could occasionally take similar liberties with the language’s
ordinary rules.

Transcribed recordings of over 1,500 compounds are available online from the Pangloss Collection (references: DetermCompounds1 through DetermCompounds16), providing, for each compound, the input tones and the lexical forms of the determiner and the head, as shown in \figref{fig:CompOnPangloss}. In all cases where examples are also
found in texts, the tone
patterns are identical with those obtained through elicitation. \tabref{tab:examplewordsusedtoelicitbodypartcompoundnouns} provides an~example word for each tonal category of noun used to build compounds referring to body parts.

\begin{figure}
	\includegraphics[width=\textwidth]{figures/CompOnPangloss.png}
	\caption{First lines of the document DetermCompounds1 as displayed in the online interface.}
	\label{fig:CompOnPangloss}
\end{figure}

\begin{table}[t]
\caption{Example words used to elicit body"=part compound nouns.}
\begin{tabularx}{\textwidth}{ l Q Q Q l }
\lsptoprule
	tone & determiners & meaning & heads & meaning\\\midrule
	LM & \ipa{bo˩˧} & pig & \ipa{ɣɯ˩˧} & skin\\
	M & \ipa{lɑ˧} & tiger & \ipa{bv̩˧} & intestine\\
	L & \ipa{jo˩} & sheep & \ipa{mɤ˩} & fat\\
	\#H & \ipa{ʐwæ˥} & horse & \ipa{sɤ˥} & blood\\
	MH\# & \ipa{ʈʂʰæ˧˥} & deer & \ipa{ɬv̩˧˥} & brains\\ \addlinespace \hdashline \addlinespace
	M & \ipa{po˧lo˧} & ram & \ipa{gv̩˧dv̩˧} & back\\
	\#H & \ipa{ʐwæ˧zo\#˥} & colt & \ipa{ɲi˧gɤ\#˥} & nose, snout\\
	MH\# & \ipa{hwɤ˧li˧˥} & cat & \ipa{qv̩˧ʈʂæ˧˥} & throat\\
	H\$ & \ipa{hwɤ˧mi˥\$} & she"=cat & \ipa{hu˧mi˥\$} & stomach\\
	L & \ipa{kʰv̩˩mi˩} & dog & \ipa{nv̩˩mi˩} & heart\\
	L\# & \ipa{dɑ˧ʝi˩} & mule & \ipa{ɬi˧pi˩} & ear\\
	LM+MH\# & \ipa{õ˩dv̩˧˥} & wolf & \ipa{ʝi˩ʈʂæ˧˥} & waist\\
	LM+\#H & \ipa{nɑ˩hĩ\#˥} & Naxi person & \ipa{njæ˩qʰæ\#˥} & eye sand, rheum\\
	LM & \ipa{æ˩mi˧} & hen & \ipa{njɤ˩ɭɯ˧} & eye\\
	LH & \ipa{bo˩ɬɑ˥} & boar & \ipa{hi˩ʐæ˥} & uvula\\
	H\# & \ipa{hwæ˧tsɯ˥} & rat & \ipa{ʁæ˧ʈv̩˥} & neck\\
\lspbottomrule
\end{tabularx}
\label{tab:examplewordsusedtoelicitbodypartcompoundnouns}
\end{table}

\subsection[Surface phonological tone patterns]{The facts: Surface phonological tone patterns}
\label{sec:thefacts}
\largerpage[-1]

The tone patterns of compound nouns in Alawa are set out in Tables~\ref{tab:surfacecompounds}a--e as a~function of the tones
of their constituting elements. The tone of the determiner is indicated in the leftmost column, and the tone of the head in the top row. For instance, ‘tiger’ /\ipa{lɑ˧}/ carries lexical M and ‘skin’ /\ipa{ɣɯ˩˧}/ carries lexical LM; the tone of the compound ‘tiger’s skin’ can be found at the intersection of row M and column LM in Table~\ref{tab:surfacemonosyllabicmonosyllables}. The information ‘M.L’ provided in the cell at the intersection of row M and column LM indicates
that the surface phonological tone of the compound at issue is M.L: /\ipa{lɑ˧-ɣɯ˩}/ ‘tiger’s skin’. Tables~\ref{tab:surfacemonosyllabicmonosyllables} and
\ref{tab:surfacemonosyllabicdisyllables} present monosyllabic heads,
and Tables~\ref{tab:surfacedisyllabicmonosyllables} and \ref{tab:surfacedisyllabicdisyllables} disyllabic
heads. 

% Is this page break relevant? To verify on final proofs.
%\Hack{\newpage}
Like simple nouns, compounds need to be elicited in at least two contexts to bring out their lexical tone
categories, because the opposition between tone categories such as \mbox{//M//} and \mbox{//\#H//} is neutralized \is{form!in isolation}in isolation. 
Compound nouns, like simple nouns, were elicited (i)~\is{form!in isolation}in isolation and (ii)~in frame (\ref{ex:carrierthisisatheREP}).\footnote{Frame (\ref{ex:carrierthisisatheREP}) is reproduced from example (\ref{ex:carrierthisisathe}) of Chapter~\ref{chap:thelexicaltonesofnouns}. In (\ref{ex:carrierthisisatheREP}), the lexical tones of the {demonstrative} and of the {copula} are indicated in the interlinear glosses, building on the findings reported in the course of Chapter~\ref{chap:thelexicaltonesofnouns}. The tone"=group {boundary} separating the {demonstrative} from what follows is also indicated, by means of the vertical bar symbol ‘\ipa{|}’. No tone is indicated for the {copula} in the surface phonological representation (the first line of the example) because its surface tone changes according to the tone category of the target noun. }

 \begin{exe}
 	\ex
 	\label{ex:carrierthisisatheREP}
	\ipaex{ʈʂʰɯ˧ {\kern2pt}|{\kern2pt} {\_\_\_\_\_\_\_\_\_} {\kern2pt}ɲi.}\\
 	\gll ʈʂʰɯ˥ {\_\_\_\_\_\_\_\_\_} ɲi˩\\
 	\textsc{dem.prox} \textit{{target item}}	\textsc{cop}\\
 	\glt	‘This is \mbox{(a/the)} \ipa{{\_\_\_\_\_\_\_\_\_}}.’
 \end{exe}

 \newpage 
In cases where the \isi{copula} surfaces with its lexical L tone, only one tone pattern is indicated in the corresponding cell in the table. For instance, ‘tiger’s skin’ is /\ipa{lɑ˧-ɣɯ˩}/ (tone: /M.L/), and the \isi{copula} surfaces with L tone after this compound: /\ipa{lɑ˧-ɣɯ˩ ɲi˩}/ ‘is tiger’s skin’. The information provided in Table~\ref{tab:surfacemonosyllabicmonosyllables} in the cell at the intersection of row M and column LM is therefore simply ‘M.L’: the fact that only one pattern is provided means that this pattern is unchanged when a~\isi{copula} is added. On the other hand, in cases where the \isi{copula} bears a~tone other than its lexical L tone, the tonal string that is obtained when adding the \isi{copula} is indicated after a~comma. For instance, the information ‘M.M, M.M.H’ provided at the intersection of row M and column M in Table~\ref{tab:surfacemonosyllabicmonosyllables} indicates that the tonal string of the
compound at issue is M.M when said \is{form!in isolation}in isolation, e.g.~/\ipa{lɑ˧-bv̩˧}/ ‘tiger’s intestine’, and that
addition of a~\isi{copula} yields a~M.M.H pattern: /\ipa{lɑ˧-bv̩˧ ɲi˥}/ ‘is tiger’s intestine’.\footnote{In \ref{tab:surfacemonosyllabicmonosyllables}, there are no examples of simple M.M: all disyllabic M.M compounds plus {copula} yield M.M.H, so that the ‘M.M.H’ part in ‘M.M, M.M.H’ may seem redundant. But this piece of information is not superfluous, because the surface tone pattern M.M on a~disyllable always needs disambiguation in Yongning Na: /M.M/ constitutes the {neutralization} of underlying \mbox{//M//} and \mbox{//\#H//} tones, illustrated in Chapter~\ref{chap:thelexicaltonesofnouns} by the nouns //\ipa{po˧lo˧}// ‘ram’ and //\ipa{ʐwæ˧zo\#˥}// ‘colt’, respectively.}

When there are tonal variants, alternatives are separated by slashes. For instance, the indication
‘M.L.L/M.M.H’ in row \#H, column H\$ of \tabref{tab:surfacedisyllabicmonosyllables} means that
these compounds can have either of two patterns: M.L.L or M.M.H, e.g. /\ipa{ʐwæ˧-hu˩mi˩}/ or
/\ipa{ʐwæ˧-hu˧mi˥}/ for ‘horse’s stomach’. For the sake of typographical economy, sequences of four M tones (M.M.M.M) have been
abbreviated to ‘M{\dots}M’. Adjacent cells with identical contents have been
grouped by using boxes in dashed lines. For instance, the tone pattern of any disyllabic compound with a~L"=tone determiner is /L.LH/, hence all the cells in the ‘L' row in Table~\ref{tab:surfacemonosyllabicmonosyllables} are grouped into one box delimitated by dashed lines, containing the indication ‘L.LH'. This process has not been pushed to an~extreme, however.
%rows and columns have not been systematically rearranged in order to maximize the number of merged cells, and adjacent cells have not been merged in cases where it appeared that a~{merger} made the table less easy to read and did not yield relevant insights. For instance, in the bottom left-hand corner of Table~\ref{tab:surfacemonosyllabicmonosyllables}, it did not appear appropriate to merge cells in the ‘H’ and ‘MH’ rows, as the reasons for the compounds carrying a~/M.H/ tone pattern were likely to be different in these two rows (as set out in \sectref{sec:Htonedems} and \sectref{sec:MHtonedems}, respectively). 
The purpose of Table~\ref{tab:surfacemonosyllabicmonosyllables} is to set out the facts in a~legible and unambiguous way; this is only one step towards the long"=term goal of arriving at a~full"=fledged linguistic model, with new ways of modelling regularities and irregularities within paradigms
\citep{sagotetal2013}. 

In view of the rarity of three"=syllable nouns, only one three"=syllable determiner was used:
/\ipa{gi˧nɑ˧mi\#˥}/ ‘bear’ (tone: \#H). The data is set out in
\tabref{tab:surfacetriisyllabic}.

%Table placement: centering in remaining space would be cool. I don't know how to do this, so chose an ugly solution: to add blank lines. 
%Placing the table at bottom does not look good.
% Result to be checked on proofs.
%~\newline ~
%~\newline  ~
%~\newline  ~

\begin{subtables}
\label{tab:surfacecompounds}
\begin{table}[p]%[b]
\caption{\label{tab:surfacemonosyllabicmonosyllables}Surface phonological representation of the tones of
  compound nouns. Monosyllabic determiner and monosyllabic head. Leftmost column: tone of determiner; top row: tone of head.}
%{\renewcommand{\arraystretch}{1.35}
{\renewcommand{\arraystretch}{1.2}
\begin{tabularx}{\textwidth}{ Q Q l Q P{21mm} Q }
\lsptoprule
	tone & LH; LM & M & L & H & MH\\ \midrule
	LM & \tikzmark{0a}L.M &  & \hspace*{\fill}\tikzmark{0e} & \tikzmark{1a}L.M, L.M.H & \tikzmark{2a}L.MH\\
	LH & L.H & L.L & L.H & \hspace*{\fill}\tikzmark{1e} &  \hspace*{\fill}\tikzmark{2e}\\
	M & M.L & M.M, M.M.H & M.L & M.M, M.M.H & M.MH\\
	L & \tikzmark{3a}L.LH &  &  &  &  \hspace*{\fill}\tikzmark{3e}\\
	H & M.H & \tikzmark{4a}M.M, M.M.H &  & \hspace*{\fill}\tikzmark{4e} & M.L\\
	MH & \tikzmark{5a}M.H &  &  \hspace*{\fill}\tikzmark{5e} & \tikzmark{6a}M.H, M.M.H & \hspace*{\fill}\tikzmark{6e}\\
\lspbottomrule
\end{tabularx}}
\DrawBox{0a}{0e}
\DrawBox{1a}{1e}
\DrawBox{2a}{2e}
\DrawBox{3a}{3e}
\DrawBox{4a}{4e}
\DrawBox{5a}{5e}
\DrawBox{6a}{6e}
\end{table}

%{\largerpage[3]}

\begin{table}%[p]%[t]
\caption{\label{tab:surfacemonosyllabicdisyllables}Surface phonological representation of the tones of
  compound nouns. Disyllabic determiner and monosyllabic head. Leftmost column: tone of determiner; top row: tone of head.}
%{\renewcommand{\arraystretch}{1.35}
{\renewcommand{\arraystretch}{1.2}
\begin{tabularx}{\textwidth}{ l Q Q Q P{21mm} Q }
\lsptoprule
	tone & LH; LM & M & L & H & MH\\ \midrule
	M & M.M.L & M.M.M, M.M.M.H & M.M.L & M.M.M, M.M.M.H & M.M.L\\
	\#H & M.M.H & \tikzmark{1a}\hbox{M.M.M, M.M.M.H} &  & \hspace*{\fill}\tikzmark{1e} & M.M.L\\
	MH\# & M.M.H & \tikzmark{2a}M.M.MH &  &  \hspace*{\fill}\tikzmark{2e} & M.M.H\\
	H\$ & M.M.H & M.M.M, M.M.M.H & M.M.H, M.M.M.H & M.M.M, M.M.M.H & M.H.L\\
	L & L.L.H & \tikzmark{3a}L.L.LH &  & \hspace*{\fill}\tikzmark{3e} & L.L.H\\
	L\# & \tikzmark{4a}M.L.L &  &  &  &  \hspace*{\fill}\tikzmark{4e}\\
	LM+MH\# & L.M.H & L.M.MH & \tikzmark{5a}\hbox{L.M.H, L.M.M.H} & & \hspace*{\fill}\tikzmark{5e}\\
	LM+\#H & L.M.H & L.M.M, L.M.M.H & L.M.H & L.M.M, L.M.M.H & L.M.H\\
	LM & L.M.L & L.M.M & L.M.L & L.M.M, L.M.M.H & L.M.MH\\
	LH & \tikzmark{6a}L.H.L &  &  &  & \hspace*{\fill}\tikzmark{6e}\\
	H\# & \tikzmark{7a}M.H.L &  &  &  &  \hspace*{\fill}\tikzmark{7e}\\
\lspbottomrule
\end{tabularx}}
\DrawBox[dashed]{1a}{1e}
\DrawBox[dashed]{2a}{2e}
\DrawBox[dashed]{3a}{3e}
\DrawBox[dashed]{4a}{4e}
\DrawBox[dashed]{5a}{5e}
\DrawBox[dashed]{6a}{6e}
\DrawBox[dashed]{7a}{7e}
\end{table}


\begin{sidewaystable}%[p!]
\caption{\label{tab:surfacedisyllabicmonosyllables}Surface phonological representation of the tones of compound nouns. Monosyllabic determiner and disyllabic head. Leftmost column: tone of determiner; top row: tone of head.}
{\renewcommand{\arraystretch}{1.35}
\begin{tabularx}{\textwidth}{ Q Q Q Q Q Q Q Q Q }
\lsptoprule
	tone & M & \#H & MH\# & H\$ & L & L\# & LM+MH\#; LM+\#H; LM; LH & H\#\\\midrule
	LM; LH & L.M.M & L.M.M, L.M.M.H & L.M.MH~/ L.H.L & L.M.H, L.M.M.H & L.M.L & L.M.L~/ L.L.H & L.H.L ($=$L.M.L) & L.M.H~/ L.L.H\\
	M & M.M.M & M.M.M, M.M.M.H & M.M.MH & M.M.H, M.M.M.H & M.L.L & M.M.L & M.L.L & M.M.H\\
	L & \tikzmark{1a}L.L.LH &  &  \hspace*{\fill}\tikzmark{1e} & L.L.H & L.L.LH & L.L.H & L.H.L & L.L.H\\
	\#H & M.M.H & M.M.M, M.M.M.H & M.H.L & M.L.L~/ M.M.H & M.H.L &
   M.M.H & M.H.L & M.M.H\\ 
	MH\# & M.M.H & M.M.M, M.M.M.H & M.M.MH & M.H.L & M.H.L & M.M.H & M.H.L & M.M.H\\
\lspbottomrule
\end{tabularx}}
\DrawBox[dashed]{1a}{1e}
\end{sidewaystable}


\begin{sidewaystable}[p]
\caption{\label{tab:surfacedisyllabicdisyllables}Surface phonological tones of compound nouns. Disyllabic determiner and disyllabic head. Leftmost column: tone of determiner; top row: tone of head.}
{\renewcommand{\arraystretch}{1.6}
{\fontsize{9}{10.75}\selectfont
\begin{tabularx}{\textwidth}{ l P{16mm} Q Q P{19mm} P{15mm} Q P{23mm} P{16mm} }
\lsptoprule
	tone & M & \#H & MH\# & H\$ & L & L\# & LM+MH\#;\par LM+\#H; LM; LH & H\#\\\midrule
	M & M{\dots}M & M{\dots}M,\par M{\dots}M.H & M{\dots}MH & M.M.M.H,\par M{\dots}M.H~/ M.M.H.L & M.M.L.L & M.M.M.L & M.M.L.L & M.M.M.H\\
	\#H & \tikzmark{1a}M.M.M.H & M{\dots}M,\par M{\dots}M.H & M.M.H.L & M.M.M.H,\par M{\dots}M.H~/ M.M.H.L~/ M.M.M.H & M.M.H.L & M.M.M.H & M.M.H.L & M.M.M.H\\
	MH\# &  & M{\dots}MH & M.M.H.L & M.M.H.L~/ M.M.M.H & M.M.H.L & M.M.M.H & M.M.H.L & M.M.M.H\\
	H\$ &  ~\par\hspace*{\fill}\tikzmark{1e} & M{\dots}M,\par M{\dots}M.H & \tikzmark{2a}\hbox{M.M.H.L~/ M.H.L.L} & ~\par\hspace*{\fill}\tikzmark{2e} & M.M.H.L & M.M.M.H & M.M.H.L & M.M.M.H\\
	L & L.L.L.H & L.L.L.LH & L.L.H.L & L.L.L.H & L.L.H.L & L.L.L.H & L.L.H.L & L.L.L.H\\
	L\# & M.L.L.L & \tikzmark{3a}M.L.L.L &  &  &  &  &  &  \hspace*{\fill}\tikzmark{3e}\\
	L+MH\# & \tikzmark{4a}L.M.M.H & \tikzmark{5a}L.M.M.M, L.M.M.M.H & \tikzmark{6a}L.M.H.L &  & ~\par\hspace*{\fill}\tikzmark{6e} & \tikzmark{7a}L.M.M.H & \tikzmark{8a}L.M.H.L & \tikzmark{9a}L.M.M.H\\
	LM+\#H &  \hspace*{\fill}\tikzmark{4e} &  & L.M.H.L & L.M.M.H, &
   L.M.H.L &  \hspace*{\fill}\tikzmark{7e} & \hspace*{\fill}\tikzmark{8e} &\\
	LM & L.M.M.M & \hspace*{\fill}\tikzmark{5e} & L.M.M.MH & L.M.M.M.H & L.M.L.L & L.M.M.L & L.M.L.L & \hspace*{\fill}\tikzmark{9e}\\
	LH & \tikzmark{10a}L.H.L.L &  &  &  &  &  &  & \hspace*{\fill}\tikzmark{10e}\\
	H\# & \tikzmark{11a}M.H.L.L &  &  &  &  &  &  &
   \hspace*{\fill}\tikzmark{11e}\\
\lspbottomrule
\end{tabularx}}
\DrawBox[dashed]{1a}{1e}
\DrawBox[dashed]{2a}{2e}
\DrawBox[dashed]{3a}{3e}
\DrawBox[dashed]{4a}{4e}
\DrawBox[dashed]{5a}{5e}
\DrawBox[dashed]{6a}{6e}
\DrawBox[dashed]{7a}{7e}
\DrawBox[dashed]{8a}{8e}
\DrawBox[dashed]{9a}{9e}
\DrawBox[dashed]{10a}{10e}
\DrawBox[dashed]{11a}{11e}}
\end{sidewaystable}

% %Table 2d
% %Table 2b. bottom in manuscript
% \begin{sidewaystable}[p]
% \caption{\label{tab:surfacedisyllabicdisyllables}surface phonological representation of the tone patterns of compound nouns. Part b. Disyllabic head and determiner.}
% \scalebox{0.85}{
% {\renewcommand{\arraystretch}{1.35}
% \begin{tabularx}{\textwidth}{ l P{16mm} Q Q Q P{15mm} Q Q P{16mm} }
% \lsptoprule
% 	tone & M & \#H & MH\# & H\$ & L & L\# & LM+MH\#; LM+\#H; LM; LH & H\#\\\midrule
% 	M & M{\dots}M & M{\dots}M, M{\dots}M.H & M{\dots}MH & M.M.M.H, M{\dots}M.H~/ M.M.H.L & M.M.L.L & M.M.M.L & M.M.L.L & M.M.M.H\\
% 	\#H & \tikzmark{1a}M.M.M.H & M{\dots}M, M{\dots}M.H & M.M.H.L & M.M.M.H, M{\dots}M.H~/ M.M.H.L~/ M.M.M.H & M.M.H.L & M.M.M.H & M.M.H.L & M.M.M.H\\
% 	MH\# &  & M{\dots}MH & M.M.H.L & M.M.H.L~/ M.M.M.H & M.M.H.L & M.M.M.H & M.M.H.L & M.M.M.H\\
% 	H\$ &  ~\par\hspace*{\fill}\tikzmark{1e} & M{\dots}M, M{\dots}M.H & \tikzmark{2a}\hbox{M.M.H.L~/ M.H.L.L} & ~\par\hspace*{\fill}\tikzmark{2e} & M.M.H.L & M.M.M.H & M.M.H.L & M.M.M.H\\
% 	L & L.L.L.H & L.L.L.LH & L.L.H.L & L.L.L.H & L.L.H.L & L.L.L.H & L.L.H.L & L.L.L.H\\
% 	L\# & M.L.L.L & \tikzmark{3a}M.L.L.L &  &  &  &  &  &  \hspace*{\fill}\tikzmark{3e}\\
% 	L+MH\# & \tikzmark{4a}L.M.M.H & \tikzmark{5a}L.M.M.M, L.M.M.M.H & \tikzmark{6a}L.M.H.L &  & ~\par\hspace*{\fill}\tikzmark{6e} & \tikzmark{7a}L.M.M.H & \tikzmark{8a}L.M.H.L & \tikzmark{9a}L.M.M.H\\
% 	LM+\#H &  \hspace*{\fill}\tikzmark{4e} &  & L.M.H.L & L.M.M.H, &
%    L.M.H.L &  \hspace*{\fill}\tikzmark{7e} & \hspace*{\fill}\tikzmark{8e} &\\
% 	LM & L.M.M.M & \hspace*{\fill}\tikzmark{5e} & L.M.M.MH & L.M.M.M.H & L.M.L.L & L.M.M.L & L.M.L.L & \hspace*{\fill}\tikzmark{9e}\\
% 	LH & \tikzmark{10a}L.H.L.L &  &  &  &  &  &  & \hspace*{\fill}\tikzmark{10e}\\
% 	H\# & \tikzmark{11a}M.H.L.L &  &  &  &  &  &  &
%    \hspace*{\fill}\tikzmark{11e}\\
% \lspbottomrule
% \end{tabularx}}
% \DrawBox[dashed]{1a}{1e}
% \DrawBox[dashed]{2a}{2e}
% \DrawBox[dashed]{3a}{3e}
% \DrawBox[dashed]{4a}{4e}
% \DrawBox[dashed]{5a}{5e}
% \DrawBox[dashed]{6a}{6e}
% \DrawBox[dashed]{7a}{7e}
% \DrawBox[dashed]{8a}{8e}
% \DrawBox[dashed]{9a}{9e}
% \DrawBox[dashed]{10a}{10e}
% \DrawBox[dashed]{11a}{11e}}
% \end{sidewaystable}

\begin{table}%[t]
\caption{\label{tab:surfacetriisyllabic}Surface phonological
  representation of the tones of compound nouns. Compounds with a~{trisyllabic} \#H-tone determiner: ‘bear’+body
  part.}
{\setlength\tabcolsep{5pt}
\begin{tabularx}{\textwidth}{ l l l Q }
\lsptoprule
	\multicolumn{2}{l}{head} & \multicolumn{2}{l}{compound}\\ \cmidrule(r){1-2}\cmidrule(l){3-4}
	form  & tone & surface form & surface tone\\ \midrule
	\ipa{ɣɯ˩˧} ‘skin’ & LM & \ipa{gi˧nɑ˧mi˧-ɣɯ˥} & M.M.M.H, M.M.M.H.L\\
	\ipa{bv̩˧} ‘intestine’ & M & \ipa{gi˧nɑ˧mi˧-bv̩˧} & M.M.M.M, M{\dots}M.H\\
	\ipa{mɤ˩} ‘grease’ & L & \ipa{gi˧nɑ˧mi˧-mɤ˧˥} & M.M.M.MH\\
	\ipa{sɤ˥} ‘blood’ & H & \ipa{gi˧nɑ˧mi˧-sɤ˧} & M.M.M.M, M{\dots}M.H\\
	\ipa{ɬv̩˧˥} ‘brains’ & MH & \ipa{gi˧nɑ˧mi˧-ɬv̩˩} & M.M.M.L\\ \addlinespace \hdashline \addlinespace
	\ipa{gv̩˧dv̩˧} ‘back’ & M & \ipa{gi˧nɑ˧mi˧-gv̩˧dv̩˧} & M.M.M.M.M, M{\dots}M.H\\
	\ipa{ɲi˧gɤ˧} ‘nose’ & \#H & \ipa{gi˧nɑ˧mi˧-ɲi˧gɤ˧} & M.M.M.M.M, M{\dots}M.H\\
	\ipa{qv̩˧ʈʂæ˧˥} ‘throat’ & MH\# & \ipa{gi˧nɑ˧mi˧-qv̩˥ʈʂæ˩} & M.M.M.H.L\\
	\ipa{hu˧mi˥\$} ‘stomach’ & H\$ & \ipa{gi˧nɑ˧mi˧-hu˧mi˥} & M.M.M.M.H, M{\dots}M.H\\
	\ipa{nv̩˩mi˩} ‘heart’ & L & \ipa{gi˧nɑ˧mi˧-nv̩˥mi˩} & M.M.M.H.L\\
	\ipa{ɬi˧pi˩} ‘ear’ & L\# & \ipa{gi˧nɑ˧mi˧-ɬi˧pi˥} & M.M.M.M.H\\
	\ipa{ʝi˩ʈʂæ˧˥} ‘waist’ & LM+MH\# & \ipa{gi˧nɑ˧mi˧-ʝi˥ʈʂæ˩} & M.M.M.H.L\\
	\ipa{njɤ˩ɭɯ˧} ‘eye’ & LM & \ipa{gi˧nɑ˧mi˧-njɤ˥ɭɯ˩} & M.M.M.H.L\\
	\ipa{ʁæ˧ʈv̩˥} ‘neck’ & H\# & \ipa{gi˧nɑ˧mi˧-ʁæ˧ʈv̩˥} & M.M.M.M.H\\
\lspbottomrule
\end{tabularx}}
\end{table}
\end{subtables}

% The subsection below needs to start on a clean page, after all floats have been unpiled.
\clearpage

\subsection{Analysis into underlying tone patterns}
\label{sec:analysisintounderlyingtonepatterns}

The surface\is{form!surface} tonal strings found on compounds, as reported in Tables~\ref{tab:surfacecompounds}a--e, can in most cases be reduced to simple tone categories, such as //L//, \mbox{//LM//} or \mbox{//LH//}, by examining them in light of the phonological \is{tone rules}tone rules set out in \sectref{sec:alistoftonerules}.
% \footnote{For ease of reference, the seven rules are reproduced here: 
	% \begin{enumerate}[leftmargin=2cm, itemsep=0pt, labelwidth=\widthof{Rule~1:}]%[topsep=12pt, partopsep=0pt]
%		\begin{enumerate}[leftmargin=!,labelwidth=\widthof{Rule~1:}]
		% \item[Rule~1:] L tone spreads progressively (‘left"=to"=right’) onto syllables that are unspecified for tone.
		% \item[Rule~2:] Syllables that remain unspecified for tone after the application of Rule 1 receive M tone.
		% \item[Rule~3:] In tone"=group"=initial position, H and M are neutralized to M.
		% \item[Rule~4:] The syllable following a~H-tone syllable receives L tone.
		% \item[Rule~5:] All syllables following a~H.L or M.L sequence receive L tone.
		% \item[Rule~6:] In tone"=group"=final position, H and M are neutralized to H if they follow a~L tone.
		% \item[Rule~7:] If a~{tone group} only contains L tones, a~post"=lexical H tone is added to its last syllable.
	% \end{enumerate}} 
Here are two examples. 

\begin{itemize}
	\item{The
		sequence /L.LH/ can be interpreted as the realization of a~simple //L// tone: it spreads over the two
		syllables of the compound, yielding //L.L//, and this sequence is further supplemented by a~post"=lexical
		H tone due to Rule 7 (“If a~{tone group} only contains L tones, a~post"=lexical H tone is added to its
		last syllable”).}
	\item{The sequence /M.M/ in /\ipa{lɑ˧-bv̩˧}/ ‘tiger’s intestine’ could be the realization of underlying \mbox{//M//} or \mbox{//\#H//} (recall that \mbox{//\#H//} is a~H tone that is {floating}, and can only be realized after the lexical item at issue). The fact
		that the \isi{copula} receives a~H tone when it follows this compound (/\ipa{lɑ˧-bv̩˧ ɲi˥}/ ‘is tiger’s intestine’) reveals that
		the \is{form!underlying}underlying tone pattern is \#H.}
\end{itemize}

The result of analysis is presented in Tables~\ref{tab:abstractcompounds}a--e, which contain all the information
required to generate the \is{form!surface}surface phonological patterns of compounds, following the standard tone"=to"=syllable association
rules.
% set out in~\sectref{sec:alistoftonerules}. 

In the description of these patterns, reference needs to be made to a~\is{juncture (inside a tone group)}juncture that is internal to the
\isi{tone group}: one that separates the determiner from the head. This \is{juncture (inside a tone group)}juncture is indicated by the
symbol ‘--’; the same symbol will be used in the description of numeral-plus-classifier phrases in Chapter~\ref{chap:classifiers}. Thus, --L refers to a~L tone attaching to the second part of an expression. 
For example, the output --L indicated in \tabref{tab:abstractdisyllabicdisyllables} for a~disyllabic determiner carrying M tone (first row) and a~disyllabic head carrying L tone (sixth column) means that the determinative compound carries L tone on its second part, i.e.\ after the \is{juncture (inside a tone group)}juncture between the two nouns. For instance, the combination of /\ipa{po˧lo˧}/ ‘ram’ and /\ipa{nv̩˩mi˩}/ ‘heart’ yields a~compound with L tone on its second part, hence /\ipa{po.lo.nv̩˩mi}/ (L tone associates to the first syllable after the \is{juncture (inside a tone group)}juncture between the two nouns). Association of tones to the other three syllables then follows from the regular \is{tone rules}phonological tone rules of Yongning Na: the L tone spreads onto the following syllable (by Rule~1), hence /\ipa{po.lo.nv̩˩mi˩}/; and the first two syllables receive M tone (by Rule~2), yielding /\ipa{po˧lo˧-nv̩˩mi˩}/. A~step"=by"=step representation of this simple example is provided in \figref{fig:toneLround}. A~similar representation is provided in \figref{fig:tonepoundHround} for \#H-- (a~\is{floating tone}floating H tone attaching to the first part of the expression): the \#H tone is inserted before the \is{juncture (inside a tone group)}juncture between the two nouns, i.e.\ it associates to the second syllable of the compound. From there, the H tone attaches to the following syllable (this is the defining property of the \#H tone category, which associates \textit{after} the syllable to which it is lexically attached), i.e.\ on the third syllable. Finally, the first and second syllables receive M tone by Rule 2, and the fourth syllable receives L tone by Rule~4. 

\begin{figure}
	\caption{First illustration of the anchoring of tones relative to a~morpheme break inside a~complex expression (notation: ‘--’): step"=by"=step representation of tonal association of the --L tone of the compound /\ipa{po˧lo˧-nv̩˩mi˩}/ ‘rat's stomach'.}
	\begin{tikzpicture}
	\node (1) at (0.2,-0.5) {M};
	\node (4) at (3.5,-0.5) {L};
	
	\node (2) at (-0.3,-1.5) {σ};
	\node (3) at (0.7,-1.5) {σ};
	\node (5) at (3,-1.5) {σ};
	\node (6) at (4,-1.5) {σ};
	
	\node [anchor=mid] (s1l) at (0.2,-2) {/\ipa{po.lo}/ ‘ram’};
	%  \node (s1ll) at (0.5,-2.5) {lexical tone: MH\#};
	
	\node [anchor=mid] (s1lll) at (3.6,-2) {/\ipa{nv̩.mi}/ ‘heart'};
	%  \node (s1llll) at (4,-2.5) {lexical tone: L};
	
	\node[text width=40mm] (s1) at (-3,-0.75) {Stage 1:\\ input nouns};
	
	
	
	\node (12) at (1.5,-4) {--L};
	
	\node (22) at (0,-5.5) {σ};
	\node (32) at (1,-5.5) {σ};
	\node (52) at (2,-5.5) {σ};
	\node (7) at (3,-5.5) {σ};
	
	\node[text width=40mm] (s2) at (-3,-4.75) {Stage 2:\\ morphotonological rules\\ determine the tone\\ of the compound\\ (see \tabref{tab:abstractdisyllabicdisyllables})};
	
	
	
	
	\node (9) at (1.5,-7) {L};

	\node (23) at (0,-8.5) {σ};
	\node (33) at (1,-8.5) {σ};
	\node (53) at (2,-8.5) {σ};
	\node (8) at (3,-8.5) {σ};
	
	\node[text width=40mm] (s3) at (-3,-7.75) {Stage 3:\\ association of L tone\\ to its specified locus:\\ after the \is{juncture (inside a tone group)}juncture\\between the two nouns};
	
	% arrow from L tone: 
	\draw[decoration={markings,mark=at position 1 with
		{\arrow[scale=2,>=stealth]{>}}},postaction={decorate}] (9) -- (53);
	
	
	\node (44) at (2,-10) {L};
	
	\node (24) at (0,-11.5) {σ};
	\node (34) at (1,-11.5) {σ};
	\node (54) at (2,-11.5) {σ};
	\node (8) at (3,-11.5) {σ};
	
	\node[text width=40mm] (s4) at (-3,-10.5) {Stage 4:\\ assignment of L tone\\ by {phonological rule}:\\ L"=tone spreading};
	
	\draw (44) -- (54);	
	\draw[decoration={markings,mark=at position 1 with
		{\arrow[scale=2,>=stealth]{>}}},postaction={decorate}] (44) -- (8);	
	
	\node (14) at (0,-13) {M};
	\node (64) at (1,-13) {M};
	\node (44) at (2,-13) {L};
	\node (91) at (3,-13) {L};
	
	\node (24) at (0,-14.5) {σ};
	\node (34) at (1,-14.5) {σ};
	\node (54) at (2,-14.5) {σ};
	\node (92) at (3,-14.5) {σ};
	
	\node[text width=40mm] (s4) at (-3,-13.5) {Stage 5:\\ addition of M tones\\ to toneless syllables\\ (by Rule 2)};
	
	\draw[decoration={markings,mark=at position 1 with
		{\arrow[scale=2,>=stealth]{>}}},postaction={decorate}] (14) -- (24);	
	\draw[decoration={markings,mark=at position 1 with
		{\arrow[scale=2,>=stealth]{>}}},postaction={decorate}] (64) -- (34);	
	\draw (44) -- (54);
	\draw (91) -- (92);
	\end{tikzpicture}
	\label{fig:toneLround}
\end{figure}


\begin{figure}
	\caption{Second illustration of the anchoring of tones relative to a~morpheme break inside a~complex expression (notation: ‘--’): step"=by"=step representation of tonal association of the \#H-- tone of the compound /\ipa{hwɤ˧li˧"=qv̩˥ʈʂæ˩}/ ‘cat's sound / cat sounds'.}
	\begin{tikzpicture}
	\node (1) at (0.2,-0.5) {M};
	\node (4) at (3.5,-0.5) {L};
	
	\node (2) at (-0.3,-1.5) {σ};
	\node (3) at (0.7,-1.5) {σ};
	\node (5) at (3,-1.5) {σ};
	\node (6) at (4,-1.5) {σ};
	
	\node [anchor=mid] (s1l) at (0.2,-2) {/\ipa{hwɤ.li}/ ‘cat’};
	%  \node (s1ll) at (0.5,-2.5) {lexical tone: MH\#};
	
	\node [anchor=mid] (s1lll) at (3.6,-2) {/\ipa{qv̩.ʈʂæ}/ ‘sound'};
	%  \node (s1llll) at (4,-2.5) {lexical tone: L};
	
	\node[text width=40mm] (s1) at (-3,-0.75) {Stage 1:\\ input nouns};
	
	
	
	\node (12) at (1.5,-4) {\#H--};
	
	\node (22) at (0,-5.5) {σ};
	\node (32) at (1,-5.5) {σ};
	\node (52) at (2,-5.5) {σ};
	\node (7) at (3,-5.5) {σ};
	
	\node[text width=40mm] (s2) at (-3,-4.75) {Stage 2:\\ morphotonological rules\\ determine the tone\\ of the compound\\ (see \tabref{tab:abstractdisyllabicdisyllables})};
	
	
	\node (9) at (1.5,-7) {\#H};
	
	\node (23) at (0,-8.5) {σ};
	\node (33) at (1,-8.5) {σ};
	\node (53) at (2,-8.5) {σ};
	\node (8) at (3,-8.5) {σ};
	
	\node[text width=40mm] (s3) at (-3,-7.75) {Stage 3:\\ the \is{floating tone}floating H tone\\ (\#H) is inserted\\ before the \is{juncture (inside a tone group)}juncture\\between the two nouns};
	
	% arrow from L tone: 
	\draw[decoration={markings,mark=at position 1 with
		{\arrow[scale=2,>=stealth]{>}}},dashed,postaction={decorate}] (9) -- (33);
	
	% attempting a bent line
	%	\draw [->] (33.north) to [out=150,in=30] (53.north)
	
	\node (44) at (1,-10) {H};
	
	\node (24) at (0,-11.5) {σ};
	\node (34) at (1,-11.5) {σ};
	\node (54) at (2,-11.5) {σ};
	\node (8) at (3,-11.5) {σ};
	
	\node[text width=40mm] (s4) at (-3,-10.5) {Stage 4:\\ the floating H tone\\ gets anchored\\to the next syllable};
	
	\draw[decoration={markings,mark=at position 1 with
		{\arrow[scale=2,>=stealth]{>}}},postaction={decorate}] (44) -- (54);	
	
	\node (14) at (0,-13) {M};
	\node (64) at (1,-13) {M};
	\node (44) at (2,-13) {H};
	\node (91) at (3,-13) {L};
	
	\node (24) at (0,-14.5) {σ};
	\node (34) at (1,-14.5) {σ};
	\node (54) at (2,-14.5) {σ};
	\node (92) at (3,-14.5) {σ};
	
	\node[text width=40mm] (s4) at (-3,-13.5) {Stage 5:\\ addition of L after H\\ (by Rule 4),\\ and of M tones\\ to toneless syllables\\ (by Rule 2)};
	
	\draw[decoration={markings,mark=at position 1 with
		{\arrow[scale=2,>=stealth]{>}}},postaction={decorate}] (14) -- (24);	
	\draw[decoration={markings,mark=at position 1 with
		{\arrow[scale=2,>=stealth]{>}}},postaction={decorate}] (64) -- (34);	
	\draw[decoration={markings,mark=at position 1 with
		{\arrow[scale=2,>=stealth]{>}}},postaction={decorate}] (91) -- (92);	
	\draw (44) -- (54);
	\end{tikzpicture}
	\label{fig:tonepoundHround}
\end{figure}


Four tonal categories of head nouns always behave the same way in compounds: the opposition among LM, LH,
LM+\#H and LM+MH\# is neutralized. These tone categories for heads are therefore pooled together in Tables~\ref{tab:abstractcompounds}a--e. Among determiners, the opposition between LH and LM on monosyllables is
neutralized; accordingly, these two tones are also pooled together in the tables.

% Suitability of clearpage command below to be checked on proofs
%\clearpage

\begin{subtables}
\label{tab:abstractcompounds}

\begin{table}[p]%[h!!]%[t!]
\caption{\label{tab:abstractmonosyllabicmonosyllables}The underlying tonal categories of compound
  nouns. Monosyllabic determiner and monosyllabic head. The tone of the determiner is indicated in the leftmost column, and the tone of the head in the top row.}
{\renewcommand{\arraystretch}{1.35}
\begin{tabularx}{\textwidth}{ Q  Q  Q  Q  Q  Q }
\lsptoprule
	tone & LM; LH & M & L & H & MH\\ \midrule
	LM; LH & LH & LM & LH & LM+\#H & LM+MH\#\\
	M & --L & \#H & --L & \#H & MH\#\\
	L & \tikzmark{1a}L &  &  &  & \hspace*{\fill}\tikzmark{1e}\\
	H & \#H-- & \tikzmark{2a}\#H &  &
   \hspace*{\fill}\tikzmark{2e} & --L\\
	MH & \tikzmark{3a}H\# &  &  \hspace*{\fill}\tikzmark{3e} & \tikzmark{4a}H\$ & \hspace*{\fill}\tikzmark{4e}\\
\lspbottomrule
\end{tabularx}}
\DrawBox[dashed]{1a}{1e}
\DrawBox[dashed]{2a}{2e}
\DrawBox[dashed]{3a}{3e}
\DrawBox[dashed]{4a}{4e}
\end{table}

\begin{table}%[p!]
\caption{\label{tab:abstractmonosyllabicdisyllables}The underlying tonal categories of compound
  nouns. Disyllabic determiner and monosyllabic head. The tone of the determiner is indicated in the leftmost column, and the tone of the head in the top row.}
{\renewcommand{\arraystretch}{1.35}
\begin{tabularx}{\textwidth}{ Q P{18mm} Q Q Q Q }
\lsptoprule
	tone & LH; LM & M & L & H & MH\\ \midrule
	M & --L & \#H & --L & \#H & \tikzmark{1a}--L\\
	\#H & \tikzmark{99a}H\# & \tikzmark{2a}\#H &  &
   \hspace*{\fill}\tikzmark{2e} & \hspace*{\fill}\tikzmark{1e}\\
	MH\# & \hspace*{\fill}\tikzmark{99e} & \tikzmark{3a}MH\# &  &  \hspace*{\fill}\tikzmark{3e}& H\#\\
	H\$ & \#H-- & \#H & H\$ & \#H & H\#--\\
	L & L+H\# & \tikzmark{4a}L &  & \hspace*{\fill}\tikzmark{4e} & L+H\#\\
	L\# & \tikzmark{5a}L\#-- &  &  &  & \hspace*{\fill}\tikzmark{5e}\\
	LM+MH\# & \tikzmark{6a}LM+MH\#-- & LM+MH\# & \tikzmark{7a}LM+H\$ &  & \hspace*{\fill}\tikzmark{7e}\\
	LM+\#H &  \hspace*{\fill}\tikzmark{6e}& LM+\#H & LM+H\# & \tikzmark{8a}LM+\#H & LM+H\#\\
	LM & LM--L & LM & LM--L &
   \hspace*{\fill}\tikzmark{8e} & LM+MH\#\\
	LH & \tikzmark{9a}LH &  &  &  & \hspace*{\fill}\tikzmark{9e}\\
	H\# & \tikzmark{10a}H\#-- &  &  &  & \hspace*{\fill}\tikzmark{10e}\\ 
\lspbottomrule
\end{tabularx}}
\DrawBox[dashed]{1a}{1e}
\DrawBox[dashed]{2a}{2e}
\DrawBox[dashed]{3a}{3e}
\DrawBox[dashed]{4a}{4e}
\DrawBox[dashed]{5a}{5e}
\DrawBox[dashed]{6a}{6e}
\DrawBox[dashed]{7a}{7e}
\DrawBox[dashed]{8a}{8e}
\DrawBox[dashed]{9a}{9e}
\DrawBox[dashed]{10a}{10e}
%\DrawBox[dashed]{11a}{11e}
\DrawBox[dashed]{99a}{99e}
\end{table}


\begin{sidewaystable}%[p]
\caption{\label{tab:abstractdisyllabicmonosyllables}The underlying tonal categories of compound nouns. Monosyllabic determiner and disyllabic head. The tone of the determiner is indicated in the leftmost column, and the tone of the head in the top row.}
{\renewcommand{\arraystretch}{1.35}
{\fontsize{9}{10.75}\selectfont
\begin{tabularx}{\textwidth}{ l P{7mm} P{12mm} Q l P{10mm} P{22mm} P{16mm} P{26mm} }
\lsptoprule
	tone & M & \#H & MH\# & H\$ & L & L\# & LM+MH\#; LM+\#H; LM; LH & H\#\\\midrule
	LM; LH & LM & LM+\#H  & LM+MH\#~/ L+\#H-- & LM+H\$ & L+\#H-- & L+\#H--~/ L+H\# & L+\#H-- & LM+H\#~/ L+H\#\\
	M & M & \#H & MH\# & H\$ & --L & --L\# & --L & H\#\\
	L & \tikzmark{1a}L &  & \hspace*{\fill}\tikzmark{1e} & L+H\# & L & L+H\# & L+\#H-- & L+H\#\\
	\#H & \tikzmark{2a}H\# & \tikzmark{3a}\#H & \#H-- & --L~/ H\# & \tikzmark{4a}\#H-- & \tikzmark{5a}H\# & \tikzmark{6a}\#H-- & H\#\\
	MH & \hspace*{\fill}\tikzmark{2e} & \hspace*{\fill}\tikzmark{3e} &
   MH\# & \#H-- & \hspace*{\fill}\tikzmark{4e} & \hspace*{\fill}\tikzmark{5e} & \hspace*{\fill}\tikzmark{6e} & \#H\\
\lspbottomrule
\end{tabularx}}
\DrawBox[dashed]{1a}{1e}
\DrawBox[dashed]{2a}{2e}
\DrawBox[dashed]{3a}{3e}
\DrawBox[dashed]{4a}{4e}
\DrawBox[dashed]{5a}{5e}
\DrawBox[dashed]{6a}{6e}
}
\end{sidewaystable}

\begin{sidewaystable}[p]
\caption{\label{tab:abstractdisyllabicdisyllables}The underlying tonal categories of compound nouns. Disyllabic determiner and disyllabic head. The tone of the determiner is indicated in the leftmost column, and the tone of the head in the top row.}
{\renewcommand{\arraystretch}{1.65}
{\fontsize{10}{10.75}\selectfont
\begin{tabularx}{\textwidth}{ l@{\hspace{16pt}} P{12mm} P{12mm} P{16mm} P{22mm} P{20mm} P{12mm} Q P{12mm} }
  \lsptoprule tone & M & \#H & MH\# & H\$ & L & L\# & LM+MH\#;\hack{\par} LM+\#H;\hack{\par} LM; LH & H\#\\\midrule
  M & M & \tikzmark{1a}\#H & MH\# & H\$ / \#H-- & --L & --L\# & --L & \tikzmark{2a}H\#\\
  \#H & \tikzmark{3a}H\# & \hspace*{\fill}\tikzmark{1e} & \tikzmark{4a}\#H-- & H\$ / \#H-- / H\# & \tikzmark{17a}\#H-- & \tikzmark{15a}H\# & \#H-- &\\
  MH\# &  & MH\# & \hspace*{\fill}\tikzmark{4e} & \#H-- / H\# &  &  & MH\#-- &\\
  H\$ & \hspace*{\fill}\tikzmark{3e} & \#H & \tikzmark{16a}\#H-- / H\#-- & \hspace*{\fill}\tikzmark{16e} & \hspace*{\fill}\tikzmark{17e} & \hspace*{\fill}\tikzmark{15e} & \#H-- &  \hspace*{\fill}\tikzmark{2e}\\
  L & L+H\# & L & L+H\# & L+H\# & L+\#H-- & L+H\# & L+\#H-- & L+H\#\\
  L\# & \tikzmark{14a}L\#-- &  &  &  &  &  &  & \hspace*{\fill}\tikzmark{14e}\\
  LM+MH\# & \tikzmark{5a}LM+H\# & \tikzmark{12a}LM+\#H & \tikzmark{13a}LM+MH\#-- & LM+MH\#--/ H\# & LM+MH\#-- & \tikzmark{11a}LM+H\# & \tikzmark{10a}LM+MH\#-- & \tikzmark{9a}LM+H\#\\
  LM+\#H & \hspace*{\fill}\tikzmark{5e} &  & \hspace*{\fill}\tikzmark{13e} & \tikzmark{8a}LM--H\$ & LM+\#H-- & \hspace*{\fill}\tikzmark{11e} & \hspace*{\fill}\tikzmark{10e} &\\
  LM & LM-- &  \hspace*{\fill}\tikzmark{12e} & LM+MH\# & \hspace*{\fill}\tikzmark{8e} & LM--L & LM--L\# & LM--L & \hspace*{\fill}\tikzmark{9e}\\
  LH & \tikzmark{6a}LH &  &  &  &  &  &  & \hspace*{\fill}\tikzmark{6e}\\
  H\# & \tikzmark{7a}H\#-- &  &  &  &  &  &  & \hspace*{\fill}\tikzmark{7e}\\
  \lspbottomrule
\end{tabularx}
\DrawBox[dashed]{1a}{1e}
\DrawBox[dashed]{2a}{2e}
\DrawBox[dashed]{3a}{3e}
\DrawBox[dashed]{4a}{4e}
\DrawBox[dashed]{5a}{5e}
\DrawBox[dashed]{6a}{6e}
\DrawBox[dashed]{7a}{7e}
\DrawBox[dashed]{8a}{8e}
\DrawBox[dashed]{9a}{9e}
\DrawBox[dashed]{10a}{10e}
\DrawBox[dashed]{11a}{11e}
\DrawBox[dashed]{12a}{12e}
\DrawBox[dashed]{13a}{13e}
\DrawBox[dashed]{14a}{14e}
\DrawBox[dashed]{15a}{15e}
\DrawBox[dashed]{16a}{16e}
\DrawBox[dashed]{17a}{17e}
}}
\end{sidewaystable}

\begin{table}%[t]
\caption{\label{tab:abstracttrisyllabic}Examples and underlying tonal categories of compound nouns with a~{trisyllabic}, \#H-tone determiner: ‘bear’+body part.}
\begin{tabularx}{\textwidth}{ l l Q Q }
\lsptoprule
	\multicolumn{2}{l}{head} & \multicolumn{2}{l}{compound}\\
   \cmidrule(r){1-2} \cmidrule(l){3-4}
	form  & tone & underlying form & underlying tone\\\midrule
	\ipa{ɣɯ˩˧} ‘skin’ & LM & \ipa{gi˧nɑ˧mi˧-ɣɯ˥} & H\#\\
	\ipa{bv̩˧} ‘intestine’ & M & \ipa{gi˧nɑ˧mi˧-bv̩\#˥} & \#H\\
	\ipa{mɤ˩} ‘grease’ & L & \ipa{gi˧nɑ˧mi˧-mɤ˧˥} & MH\#\\
	\ipa{sɤ˥} ‘blood’ & H & \ipa{gi˧nɑ˧mi˧-sɤ\#˥} & \#H\\
	\ipa{ɬv̩˧˥} ‘brains’ & MH & \ipa{gi˧nɑ˧mi˧-ɬv̩˩} & L\#\\ \addlinespace \hdashline \addlinespace
	\ipa{gv̩˧dv̩˧} ‘back’ & M & \ipa{gi˧nɑ˧mi˧-gv̩˧dv̩\#˥} & \#H\\
	\ipa{ɲi˧gɤ˧} ‘nose’ & \#H & \ipa{gi˧nɑ˧mi˧-ɲi˧gɤ\#˥} & \#H\\
	\ipa{qv̩˧ʈʂæ˧˥} ‘throat’ & MH\# & \ipa{gi˧nɑ˧mi˧-qv̩˥ʈʂæ˩} & \#H--\\
	\ipa{hu˧mi˥\$} ‘stomach’ & H\$ & \ipa{gi˧nɑ˧mi˧-hu˧mi˥\$} & H\$\\
	\ipa{nv̩˩mi˩} ‘heart’ & L & \ipa{gi˧nɑ˧mi˧-nv̩˥mi˩} & \#H--\\
	\ipa{ɬi˧pi˩} ‘ear’ & L\# & \ipa{gi˧nɑ˧mi˧-ɬi˧pi˥} & H\#\\
	\ipa{ʝi˩ʈʂæ˧˥} ‘waist’ & LM+MH\# & \ipa{gi˧nɑ˧mi˧-ʝi˥ʈʂæ˩} & \#H--\\
	\ipa{njɤ˩ɭɯ˧} ‘eye’ & LM & \ipa{gi˧nɑ˧mi˧-njɤ˥ɭɯ˩} & \#H--\\
	\ipa{ʁæ˧ʈv̩˥} ‘neck’ & H\# & \ipa{gi˧nɑ˧mi˧-ʁæ˧ʈv̩˥} & H\#\\
\lspbottomrule
\end{tabularx}
\end{table}
\end{subtables}


% The subsection below needs to start on a clean page, after all floats have been unpiled.
\clearpage

\section{Determinative compound nouns. Part II: Discussion}
\label{sec:determinativecompoundnounsII}
\label{sec:aboutproductivetonerulesincompounding}

Let us now proceed to an~analysis of the patterns presented above. At this point, it may be useful to speculate about possible ways in which the tones of the determiner and head could combine. Some languages, such as {Mandarin}, have no tone change at all in compounds, showing that tone change is not necessary to compounding. A~first theoretical possibility would thus be the simple concatenation of the input tones. Under such a~configuration, given the \textit{determiner"=head} order of Yongning Na compounds, the tone of the determiner would express
itself first, and given the limitations on tone sequences within a~\isi{tone group} (summarized as Rules
1--7), this would leave little room for the tone of the head to express itself. Since a~H
tone precludes the expression of any other
tone on following syllables (by phonological rules 4 and 5), determiners carrying a~lexical tone containing a~H level (i.e.\ one of \#H, H\$, MH\#, LM+MH\#, LM+\#H and H\#) would \is{neutralization}neutralize all tonal oppositions on head nouns. Likewise, the L level in the lexical categories L and L\# would spread
rightward all the way to the end of the compound noun. The tone of the head could only
express itself when the determiner has M tone: M plus L would yield --L, M plus \#H would yield \#H, and so on.

A~second possibility would be the complete \isi{neutralization} of tonal oppositions on the determiner, or on the head. Neutralization of oppositions on the determiner would seem odd, because the linear order is determiner"=first, and processes of tone \is{tone spreading}spreading and tone reassociation in Yongning Na are mostly perseverative (towards following syllables, not preceding syllables), so that the loss of tonal oppositions among determiners would drastically reduce the number of tone patterns on compounds at the surface phonological level. Neutralization of tonal oppositions on the head is attested in the neighbouring language \ili{Shixing} (also known as Xumi), where only the tone of the determiner expresses
itself, i.e.\ all tonal oppositions on the head are neutralized, and the 3×3 tonal combinations among nouns boil down to three patterns on compounds \citep{chirkovaetal2009}. %The existence of a~greater number of lexical tones in Yongning Na may
%explain in part why such massive \isi{neutralization} did not take place in this language. 

A~third possibility would be the assignment of a~replacive tone in compounds. This is reported to be widespread in the \ili{Mande} subgroup of the Niger"=Congo family, e.g.~in Dan-Gwɛɛtaa \citep{vydrin2016} and Kpelle (\cites{welmers1969}[132]{welmers1973}[239]{konoshenko2014syntax}). In \ili{Naish} (the lower"=level subgroup within Sino"=Tibetan to which Yongning Na belongs), on the other hand, no case of replacive tone has been observed to date.
%\footnote{The closest one gets to such phenomena in \ili{Naish} is found in the \ili{Laze} language, where compounds with input tones M and H, M and L, and M and MH tend to adopt a~H.H tone pattern as they lexicalize \citep{michaud2008a}. However, the H.H pattern is not the only one carried by compounds: if the tone of the head is L, the compound receives a different tone pattern.}

The state of
affairs found in Yongning Na bears some similarities to the first and second theoretical possibilities: there is a~tendency for output tone patterns to consist of the concatenation of input tones, and there is a~measure of \isi{neutralization} of tonal oppositions (output tone patterns are fewer than input combinations). On the one hand, about half the patterns can be straightforwardly explained in terms of successive association of the two input tones, followed by
application of the general rules governing the adjustment of successive tones within a~tone
group. On the other hand, it was not found possible to capture the other half of the patterns by
means of a~set of rules.

The tone patterns of determinative compounds thus present a~composite picture (play on words intended). Since not all the tone patterns of compounds follow either of the three theoretical possibilities mentioned above, it must be acknowledged that compounds have \is{tone rules}tone rules of their own, which differ subtly from successive association of the two input tones. 

This conclusion may come as a~slight disappointment to the linguist, whose job is to
account for all observations through a~model that is as simple and elegant as possible. Here, as
in many other domains of the Yongning Na tone system, it is clear that the gap between the lexical tones of
words and their tonal realizations in context is not simply a~matter of sandhi rules operating on
a~phonological level. (Some clarifications about the terms ‘{tone sandhi}’, ‘morphotonology’ and ‘tonal morphology’ are set out in \sectref{sec:definitionofterms}.) The behaviour of the three"=syllable noun ‘bear’, /\ipa{gi˧nɑ˧mi\#˥}/, as
a~determiner is a~case in point: it patterns almost like disyllabic \#H-tone nouns, such as ‘colt’,
/\ipa{ʐwæ˧zo\#˥}/, but not quite. When the head is a~L-tone {monosyllable}, the output tone is MH\#,
instead of \#H when the determiner is disyllabic.

In view of these observations, the present discussion of the combinations in Tables~\ref{tab:abstractcompounds}a--e is arranged by increasing degree of complexity. 

The simpler cases can be described as follows: the tone of the determiner expresses itself first, and after that the tone of the head expresses itself to the extent allowed by the tones already assigned. The tone patterns of compounds in which the tone of the determiner is H\#, LH or L\# are so simple as to appear trivial: in these three cases, only the tone of the determiner expresses itself. When the determiner has H\# tone (a H tone associated to its last syllable), tonal oppositions on the head are
neutralized, irrespective of the number of syllables: see the last rows of Tables~\ref{tab:abstractcompounds}a--e. The compound
carries H\#--, i.e.\ a~H tone on the last syllable of the determiner. This can be interpreted as the
result of the straightforward association of the H\# tone to the determiner: H on its last syllable,
and M on its first syllable, by default. The lowering of all the following tones to L results from
Rules 4 and 5: “A syllable following a~H-tone syllable receives L tone” and “All syllables
following a~H.L or M.L sequence receive L tone”. This is shown in \figref{fig:toneMHcomp}, using ‘rat's stomach' (/\ipa{hwæ˧tsɯ˥-hu˩mi˩}/) as an example. In this compound, no trace is left of the H\$ tone (the ‘flea' tone) carried by the head.

\begin{figure}[p]
	\caption[{A hypothesis about how H\# tone on the determiner associates to the entire compound.}]{A hypothesis about how H\# tone on the determiner associates to the entire compound. Example: /\ipa{hwæ˧tsɯ˥-hu˩mi˩}/ ‘rat's stomach'.}
	\begin{tikzpicture}
	\node (1) at (0.2,-0.5) {H\#};
	\node (4) at (3.5,-0.5) {H\$};
	
	\node (2) at (-0.3,-1.5) {σ};
	\node (3) at (0.7,-1.5) {σ};
	\node (5) at (3,-1.5) {σ};
	\node (6) at (4,-1.5) {σ};
	
	\node [anchor=mid] (s1l) at (0.2,-2) {/\ipa{hwæ.tsɯ}/ ‘rat’};
	%  \node (s1ll) at (0.5,-2.5) {lexical tone: MH\#};
	
	\node [anchor=mid] (s1lll) at (3.6,-2) {/\ipa{hu.mi}/ ‘stomach'};
	%  \node (s1llll) at (4,-2.5) {lexical tone: L};
	
	\node[text width=40mm] (s1) at (-3,-0.75) {Stage 1:\\ input};
	
	
	
	\node (12) at (0.5,-4) {H\#};
	\node (42) at (2.5,-4) {H\$};
	
	\node (22) at (0,-5.5) {σ};
	\node (32) at (1,-5.5) {σ};
	\node (52) at (2,-5.5) {σ};
	\node (7) at (3,-5.5) {σ};
	
	\node[text width=40mm] (s2) at (-3,-4.75) {Stage 2:\\ \is{anchorage}anchoring of H\# to\\ its
		phonologically\\ specified locus};
	
	\draw[decoration={markings,mark=at position 1 with
		{\arrow[scale=2,>=stealth]{>}}},postaction={decorate}] (12) -- (32);
	
	
	
	\node (9) at (0,-7) {M};
	\node (13) at (1,-7) {H};
%	\node (63) at (1.5,-7) {H};
	\node (43) at (2.5,-7) {H\$};
	
	\node (23) at (0,-8.5) {σ};
	\node (33) at (1,-8.5) {σ};
	\node (53) at (2,-8.5) {σ};
	\node (8) at (3,-8.5) {σ};
	
	\node[text width=40mm] (s3) at (-3,-7.75) {Stage 3:\\ addition of default\\ M tone, by Rule~2.\\ H\$ remains unassociated\\ (and is deleted)};
	
	\draw (13) -- (33);
%	\draw[decoration={markings,mark=at position 1 with {\arrow[scale=2,>=stealth]{>}}},postaction={decorate}] (63) -- (53);
% M tone: 
	\draw[decoration={markings,mark=at position 1 with
		{\arrow[scale=2,>=stealth]{>}}},postaction={decorate}] (9) -- (23);
	
	
	\node (14) at (0,-10) {M};
	\node (64) at (1,-10) {H};
	\node (44) at (2.5,-10) {L};
	
	\node (24) at (0,-11.5) {σ};
	\node (34) at (1,-11.5) {σ};
	\node (54) at (2,-11.5) {σ};
	\node (8) at (3,-11.5) {σ};
	
	\node[text width=40mm] (s4) at (-3,-10.5) {Stage 4:\\ assignment of L tone\\ by Rules 4 and 5.};
	
	\draw[decoration={markings,mark=at position 1 with
		{\arrow[scale=2,>=stealth]{>}}},postaction={decorate}] (44) -- (54);
	\draw[decoration={markings,mark=at position 1 with
		{\arrow[scale=2,>=stealth]{>}}},postaction={decorate}] (44) -- (8);
	\draw (14) -- (24);
	\draw (64) -- (34);
	
	
	\node (14) at (0,-13) {M};
	\node (64) at (1,-13) {H};
	\node (44) at (2,-13) {L};
	\node (91) at (3,-13) {L};
	
	\node (24) at (0,-14.5) {σ};
	\node (34) at (1,-14.5) {σ};
	\node (54) at (2,-14.5) {σ};
	\node (92) at (3,-14.5) {σ};
	
	\node[text width=40mm] (s4) at (-3,-13.5) {Stage 5:\\ resulting surface-\\ phonological tone};
	
	\draw (14) -- (24);
	\draw (64) -- (34);
	\draw (44) -- (54);
	\draw (91) -- (92);
	\end{tikzpicture}
	\label{fig:toneMHcomp}
\end{figure}


The same analysis can be extended to the two other
tone categories of disyllables after which all tonal oppositions are neutralized: LH and L\#. Application of one of these tone patterns to the first part of the compound (the determiner) precludes any
tone other than L on the following syllables, by Rules 4 and 5. The L\# tone pattern of the latter results in association of a~L tone to the second syllable of the determiner, whose first syllable receives M by default; this M.L sequence
precludes, again, any tones other than L on the next syllables, by Rule~5.

In the other cases, which constitute a~majority, the tonal oppositions on the head are not
entirely neutralized: the tone of the compound cannot be arrived at without knowledge of the tone of the head noun. This observation casts doubt on the adequacy of the representation proposed in \figref{fig:toneMHcomp}, which assumes determiner"=driven tonal association. If the process were one of step"=by"=step association of the tones of the determiner, then of the head, one would expect \isi{neutralization} of all tonal oppositions on the head when the determiner carries MH\# tone. On the {analogy} of \figref{fig:toneMHcomp}, one would expect the behaviour shown in \figref{fig:toneMHcompFALSE}, i.e.\ that the tone pattern for ‘cat's ear’ would be $\ddagger${\kern2pt}\ipa{hwɤ˧li˧-ɬi˥pi˩}. But the observed tone is H\#: /\ipa{hwɤ˧li˧-ɬi˧pi˥}/. 

\begin{figure}[p]
	\caption{Tone"=to"=syllable association expected for ‘cat's ear’ under the mistaken hypothesis of determiner"=driven tone association:  $\ddagger${\kern2pt}\ipa{hwɤ˧li˧-ɬi˥pi˩}, as contrasted with the observed pattern: /\ipa{hwɤ˧li˧-ɬi˧pi˥}/.}
	\begin{tikzpicture}
	\node (1) at (0.5,-0.5) {MH\#};
	\node (4) at (3,-0.5) {L\#};
	
	\node (2) at (0,-1.5) {σ};
	\node (3) at (1,-1.5) {σ};
	\node (5) at (2.5,-1.5) {σ};
	\node (91) at (3.5,-1.5) {σ};
	
	\node [anchor=mid] (s1l) at (0.5,-2) {/\ipa{hwɤ.li}/ ‘cat’};
	%  \node (s1ll) at (0.5,-2.5) {lexical tone: MH\#};
	
	\node [anchor=mid] (s1lll) at (3,-2) {/\ipa{ɬi.pi}/ ‘ear’};
	%  \node (s1llll) at (4,-2.5) {lexical tone: L};
	
	\node[text width=40mm] (s1) at (-3,-0.75) {Stage 1:\\ input};
	
	
	
	\node (12) at (0.5,-4) {MH\#};
	\node (42) at (2.5,-4) {L\#};
	
	\node (22) at (0,-5.5) {σ};
	\node (32) at (1,-5.5) {σ};
	\node (52) at (2,-5.5) {σ};
	\node (90) at (3,-5.5) {σ};
	
	\node[text width=40mm] (s2) at (-3,-4.75) {$\ddagger${\kern2pt}Stage 2:\\ \is{anchorage}anchoring of MH\# to\\ its
		phonologically\\ specified locus};
	
	\draw[decoration={markings,mark=at position 1 with
		{\arrow[scale=2,>=stealth]{>}}},postaction={decorate}] (12) -- (32);
	
	
	
	\node (13) at (1,-7) {M};
	\node (63) at (1.5,-7) {H};
	\node (43) at (2.5,-7) {L\#};
	
	\node (23) at (0,-8.5) {σ};
	\node (33) at (1,-8.5) {σ};
	\node (53) at (2,-8.5) {σ};
	\node (92) at (3,-8.5) {σ};
	
	\node[text width=40mm] (s3) at (-3,-7.75) {$\ddagger${\kern2pt}Stage 3:\\ one-to-one mapping\\ of levels to available syllables};
	
	\draw[decoration={markings,mark=at position 1 with
		{\arrow[scale=2,>=stealth]{>}}},postaction={decorate}] (13) -- (33);
	\draw[decoration={markings,mark=at position 1 with
		{\arrow[scale=2,>=stealth]{>}}},postaction={decorate}] (63) -- (53);
	
	
	\node (14) at (0,-10) {M};
	\node (64) at (1,-10) {M};
	\node (44) at (2,-10) {H};
	\node (97) at (3,-10) {L};
	
	\node (24) at (0,-11.5) {σ};
	\node (34) at (1,-11.5) {σ};
	\node (54) at (2,-11.5) {σ};
	\node (94) at (3,-11.5) {σ};
	
	\node[text width=40mm] (s4) at (-3,-10.5) {$\ddagger${\kern2pt}Stage 4:\\ addition of default M,\\ and assignment\\ of final L by\\ phonological rule};
	
	\draw[decoration={markings,mark=at position 1 with
		{\arrow[scale=2,>=stealth]{>}}},postaction={decorate}] (14) -- (24);
	\draw[decoration={markings,mark=at position 1 with
		{\arrow[scale=2,>=stealth]{>}}},postaction={decorate}] (97) -- (94);
	\draw (64) -- (34);
	\draw (44) -- (54);
	
	
	\node (14) at (0,-13) {M};
	\node (64) at (1,-13) {M};
	\node (44) at (2,-13) {M};
	\node (96) at (3,-13) {H};
	
	\node (24) at (0,-14.5) {σ};
	\node (34) at (1,-14.5) {σ};
	\node (54) at (2,-14.5) {σ};
	\node (95) at (3,-14.5) {σ};
	
	\node[text width=40mm] (s4) at (-3,-13.5) {\textit{Observed pattern:\\ final H tone.}};
	
	\draw (14) -- (24);
	\draw (64) -- (34);
	\draw (44) -- (54);
	\draw (96) -- (95);
	\end{tikzpicture}
	\label{fig:toneMHcompFALSE}
\end{figure}

Note that the double daggers $\ddagger$ added to the labels ‘Stage 2', ‘Stage 3' and ‘Stage 4' in \figref{fig:toneMHcompFALSE} aim to emphasize that this representation is only proposed to bring out the \textit{ad hoc} nature of the representation in \figref{fig:toneMHcomp}: tone association in compounds is not always determiner"=driven. The representation in \figref{fig:toneMHcomp} nonetheless appears to correspond to a~reality, but one that is specific to H\# tone. This tone is \is{anchorage}anchored onto a~word's final syllable; in compounding, when the first element of the compound (the determiner) carries H\# tone, this tone remains moored onto the final syllable of the first element in the compound.

A~few of the combinations in Tables~\ref{tab:abstractcompounds}a--e appear counter"=intuitive in terms of the input tones. For
instance, a~M-tone determiner plus a~M-tone monosyllabic head combine to a~compound with a~\is{floating tone}floating H tone, \#H. This result could go so far as to cast doubt
on the correctness of the analysis of the tone category of the two components of the compound as M,
since this tone is expected to be inactive (see~\sectref{sec:analysisofmasadefaulttone}). However, while
a~disyllabic determiner with M tone and a~monosyllabic head with M tone likewise yield a~compound
with \#H tone, a~compound made of a~M-tone determiner and a~disyllabic M-tone head carries a~simple
M tone. The analysis of the lexical category as M does not appear mistaken: the unexpected \#H
output is not the result of a~general {phonological rule} of Na whereby any combination of two M tones
would produce a~\is{floating tone}floating H; it results from the \textit{morpho}{phonological rules} that apply in this specific
syntactic construction, and which need to be specified for each combination of input tones (hence their presentation in table form, as Tables~\ref{tab:abstractcompounds}a--e).

%{\largerpage} %%May be useful here
The following discussion of the more {complex tone} patterns is arranged by tone of the determiner.

\subsection{LM"=tone determiners}

A LM tone on the determiner results in the assignment of L on the first syllable of the compound,
and M on its second syllable, in all cases. A~relatively high number of tone sequences are allowed after
/L.M/; accordingly, some of the tone categories of the
head manifest themselves in full. Over three syllables, one may observe /L.M.L/, /L.M.M/, and /L.M.H/. Over four syllables, $\ddagger${\kern2pt}L.M.L.M and $\ddagger${\kern2pt}L.M.L.H are ruled out because the sequence /M.L/ can only be followed by /L/, by virtue
of Rule 5 (“All syllables following a~H.L or M.L sequence receive L tone”). The tones that are compatible with the realization of an~initial LM pattern are
observed to manifest themselves: LM plus \#H, LM plus MH\#, LM plus H\$ and LM plus
H\# are realized as such~-- a~concatenation of the two input tones. As expected, a~M tone on the
head has no effect on the final tone pattern, which is simply LM. Likewise, the combinations LM plus \#L and
LM plus H\# for quadrisyllabic compounds surface as such.

Some of the \is{form!surface}surface patterns are analytically indeterminate. For instance, the result of the
combination of a~monosyllabic LM determiner and a~monosyllabic MH\# head is L.MH
(e.g.~/\ipa{bo˩-ɬv̩˧˥}/ ‘pig’s brains’). This could be analyzed as L--MH\#: L on first part, and
MH\# on second part. Or it could be analyzed as LM+MH\#: the LM portion of the pattern yields L on
the first syllable and M on the second, and the MH\# portion of the pattern expresses itself by the assignment of a~MH \is{tonal contour}contour to the last syllable,
in this case also the second syllable. Both analyses are equivalent insofar as they generate the same
output, but it appears simpler to describe this tone pattern as the concatenation of the two input
tones, analyzing it as //LM+MH\#//. Under this analysis, the pattern is the same for \is{trisyllables}{trisyllabic} compounds and
quadrisyllabic compounds (σσ+σ and σσ+σσ); the different surface patterns (L.M.MH for σσ+σ, and
L.M.M.H for σσ+σσ) result straightforwardly from the general rules of tone"=to"=syllable mapping
summarized in~\sectref{sec:asummaryoftonetosyllableassociationrules}.

In the case of a~disyllabic compound with a~LM-tone determiner, a~L or LM tone on the head cannot express itself, since the
determiner’s LM \is{tonal contour}contour has already projected its endpoint (M) on the second syllable. While the M
tone can in some respects be considered as default (see~\sectref{sec:analysisofmasadefaulttone}), as the endpoint
of a~LM \is{tonal contour}contour it counts as a~fully specified tone. Such cases are not typologically infrequent;
this point will be returned to in the discussion in Chapter~\ref{chap:arealandtypologicaldiscussion}. In derivational terms,
this interpretation of the Yongning Na facts could be phrased as follows: at the point when the tone
of the head could come into play, both syllables of the compound are already specified for tone,
resulting in the \isi{neutralization} of LM, M and L as the second components of σ+σ compounds with
a~LM determiner.

On the other hand, if the compound has three or four syllables, a~L or LM tone on the head can
express its L tone, resulting in L.M.L on three"=syllable compounds, and in L.M.L.L on four"=syllable
compounds (by virtue of Rule 5: “All syllables following a~H.L or M.L sequence receive L tone”).

To sum up, the tones of all compounds with a~LM determiner obtain through the concatenation of that
of their two components, modified by subsequent application of the phonological rules that apply to tone
groups (as recapitulated in Chapter~\ref{chap:toneassignmentrulesandthedivisionoftheutteranceintotonegroups}).

\subsection{M-tone determiners}
\label{sec:Mtonedet}

After M"=tone determiners, the L and LM tones on heads are
neutralized due to Rule 5 (“All syllables following a~H.L or M.L sequence receive L tone”). Apart from these two, one would expect the tone of the second component of the compound to express itself
fully, because M behaves in some respects as a~default tone (see \sectref{sec:analysisofmasadefaulttone}) and is expected to be phonologically neutral (inert). This prediction is not entirely realized, however. As mentioned at the outset of this chapter, the tone pattern \#H (a~\is{floating tone}floating H tone) that results from the combination of a~M-tone determiner with a~monosyllabic M-tone
head does not conform to the regularities observed for the other combinations. The \#H--
\is{variants}variant for disyllabic M-tone determiner plus H\$-tone head is likewise unexplained. A~third
unexpected pattern is the --L output of the combination of disyllabic M with monosyllabic MH\#, which
one would expect to surface as MH\#, as is the case in each of the other combinations, i.e.\  σσ+σσ, σ+σσ, and σ+σ. These three cases confirm an observation made above: some combinations are not simply the product of a~set of rules applying
throughout the tone system. As a~child, F4 (the main consultant for the present study) apparently learnt a~great number of tone
patterns individually, acquiring morphotonology in a~comparable way to children learning the
morphology of \ili{Rgyalrongic} or \ili{Kiranti} languages, to cite two subgroups of Sino"=Tibetan that have
flamboyant morphology \citep{michailovsky1975a,vandriem1990,sun2000a,jacques2004}. But it may be too late to investigate children's acquisition of Yongning Na morphotonology in the form documented in this volume: in the mid"=2010s, school"=age children in the Alawa village were exposed to a~great deal of {Mandarin}, and the chances that much morphotonology would be passed down to them seemed small. (The influence of {bilingualism} with {Mandarin} constitutes the topic of \sectref{sec:theinfluenceofbilingualismwithchinese}.)

\subsection[L-tone determiners]{L-tone determiners and what they reveal for the analysis of the head noun}

The first of the seven phonological tone rules of Yongning Na is that L tone spreads progressively (“left"=to"=right”) onto syllables that are unspecified
for tone. The tone patterns of compounds with L-tone determiners provide an~interesting
testing"=ground for determining whether or not the lexical tones which have an~initial M tone in their surface
form are specified for tone on their first syllable: if they were, that initial M tone would
be expected to block L-tone \is{tone spreading}spreading; on the other hand, if that syllable is unspecified for tone,
it should receive a~L tone through \is{tone spreading}spreading.

The observed patterns lend support to the analysis of the disyllables with High tone (\#H, H\$, and
H\#) as unspecified for tone on the first syllable: if the head has one of these tones, a~L tone on
the determiner spreads onto the first syllable of the head. 

The same analysis can be extended to the L\# tone, a~type of L tone that associates in word"=final position. L plus L\# yields H\#,
e.g.~/\ipa{kʰv̩˩mi˩}/ ‘dog’ and /\ipa{ɬi˧pi˩}/ ‘ear’ yield /\ipa{kʰv̩˩mi˩-ɬi˩pi˥}/ ‘dog’s ear’, where
the L tone on the first syllable of the head is analyzed as resulting from L-tone \is{tone spreading}spreading. (More
below about the H part in this compound's tone pattern.)

The weight of this argument is admittedly decreased by the fact that the tone patterns after
a~L-tone determiner cannot be generated through the application of a~set of general rules. It is not
entirely clear to what extent the processes involving L tone in this particular morphosyntactic context
(determinative compounds) relate to the general rule of L-tone \is{tone spreading}spreading. If the determiner is
monosyllabic and combines with another monosyllable, all oppositions are neutralized; the compound
carries L tone. In the other three length combinations (σσ+σσ, σσ+σ, σ+σσ), the picture is more
complex. A~\is{floating tone}floating H tone (\#H) on the head is always disregarded, and the result is L. In
combination with a~M-tone head, the result is L except for σσ+σσ which yields L+H\#: a~sequence of L
tones and a~final H (surface phonological tone sequence: L.L.L.H). The presence of a~final H is not due to a~general rule preventing the
L tone from \is{tone spreading}spreading more than one syllable to its right: for instance, the compound
//\ipa{jo˩-gv̩˩dv̩˩}// ‘sheep’s back’ (input tones: L and \#H) carries a~simple L tone, which spreads over the two syllables of
the head. (It surfaces as /\ipa{jo˩-gv̩˩dv̩˩˥}/, with a~final rise, due to post"=lexical H-tone
addition: Rule 7.) A~combination of L and L on disyllables also yields a~result that is unexpected under
the hypothesis that the tones of the determiner and head are simply concatenated: L+\#H-- (surface
form: L.L.H.L).

The behaviour of L-tone determiners in compounds resists phonological generalizations, just like that of M-tone determiners (studied in the previous section: \sectref{sec:Mtonedet}). For instance, in some cases it could seem as if \isi{dissimilation} were at play. In the
example /\ipa{kʰv̩˩mi˩-nv̩˥mi˩}/ ‘dog’s heart’, from a~L plus L input, there is a~H tone on the first
syllable of the head. This looks like a~case of \isi{dissimilation} whereby the L tone on the head
dissimilates to an~initial H tone. But a~L plus L input yields a~simple L output for words of other lengths (σσ+σ, σ+σ and σ+σσ), clarifying that there is no general (phonological) mechanism of \isi{dissimilation} between successive L tones. Here again, it appears that the tone combination rules are learnt individually.


\subsection{H-tone determiners: \#H, H\# and H\$}
\label{sec:Htonedems}

For the H tones (\#H, H\# and H\$) as for the L tone, there are differences in the tone of the compound
depending on the number of syllables of the head noun. Attempts at generating the tones of these
compounds from the input tones on the basis of a~set of rules were unsuccessful. A~general
observation can be made nonetheless:

\begin{quotation}
A \is{floating tone}floating H (\#H) and a~tone"=group"=final H (H\$) are never observed to reassociate more than
one syllable to their right.
\end{quotation}

This regularity is specific to determinative compounds. It does not hold in other parts of the
morphotonology, witness example (\ref{ex:byLatami}), where H\$ tone moves two syllables away from
the word to which it is lexically attached (the family name /\ipa{lɑ˧tʰɑ˧mi˥\$}/ ‘Latami’).

\begin{exe}
	\ex
	\label{ex:byLatami}
	\ipaex{lɑ˧tʰɑ˧mi˧=ɻ̍˧ ɳɯ˥}\\
	\gll lɑ˧tʰɑ˧mi˥\$		=ɻ̍˩		ɳɯ˧\\
	Latami~(family~name)	\textsc{associative}	\textsc{a}\\
	\glt ‘by the Latamis’ (Source: field notes.)
\end{exe}

In combinations with a~monosyllabic head, the \#H tone is preserved in six cases out of ten. As this
tone attaches at the end of the word that carries it, this amounts to a~one"=syllable shift to the
right from its original position. In combinations with a~disyllabic head, the \#H tone is never
present in the output, as if it could not move more than one syllable away from its original
position without changing its nature. Eight of the sixteen combinations have a~fixed,
word"=final H tone instead (H\#).

In this light, the H\# tone that occurs on H\#-plus-H\# combinations (with a~disyllabic head) is to be interpreted as originating in the H\# tone of the head, not of the determiner.

The above generalization also captures the fact that the H\$ tone never surfaces on compounds with
a~disyllabic head. On the other hand, no hypothesis can be proposed as to why it surfaces when the head is a~L-tone
monosyllable but does not surface in association with any other monosyllabic head. Interestingly, the seven
combinations that have two or three variants all involve a~H\$-tone head noun, pointing to the relatively greater instability of this tone category as compared with the other types of H tones.

Compounds with a~\#H-tone determiner have the same output tone whether the determiner is
monosyllabic or disyllabic. This is taken as a~confirmation of the initial hypothesis that the tone
category of monosyllables illustrated by /\ipa{ʐwæ\#˥}/ ‘horse’ and the category of disyllables
illustrated by /\ipa{gi˧zɯ\#˥}/ ‘little brother’ are phonologically identical. For the sake of
typographical simplicity, and in the absence of an~opposition among different types of H tones on
monosyllabic nouns, ‘horse’ is transcribed as /\ipa{ʐwæ˥}/ rather than /\ipa{ʐwæ\#˥}/,
omitting the information on the segmental anchoring of its H tone.

\subsection{MH"=tone determiners}
\label{sec:MHtonedems}

In determinative compounds, MH tone, like other tones containing a~H level, is not observed to move
more than one syllable to the right. In σσ+σ (that is, when the head is monosyllabic and the determiner disyllabic),
a~MH tone on the determiner moves onto the last syllable of the compound. In σ+σ, on the other hand,
the MH tone does not move as a~whole: it appears to remain associated to the determiner, and to project
its H level onto the head~-- except when the head has a~\#H or MH tone. Again, those are simply
piecemeal observations: no set of rules can be proposed to generate the tones of these compounds
from the input tones.

\subsection{Determiners carrying LM+MH\# tone or LM+\#H tone}

The behaviour of LM+MH\# and LM+\#H when they appear on determiners provides evidence for their
phonological analysis. These two
lexical categories unfold as L.M over the first two syllables of polysyllabic
compounds. This offers support for the analysis of these tones proposed in the previous chapter (\sectref{sec:overviewofthesystem}): both contain a~LM pattern. If LM+MH\# were analyzed as L+MH\#, dispensing with the first M in the expression ‘LM+MH\#’, its initial L level would be
expected to spread, creating a~sequence of Ls, followed by a~final MH \is{tonal contour}contour. Likewise, reanalysis of LM+\#H as L+\#H would not at all be promising, despite the apparent gain in descriptive economy: if this tone's first part were a~simple L tone, its behaviour in compounds would be incomprehensible.

Apart from this observation, the interpretation of individual combinations is not
straightforward. The cases that seem to make good sense in terms of the input tones do not greatly
outnumber those that seem opaque. For instance, LM+MH\# followed by a~{monosyllable} with \#H or MH
yields a~pattern comprising a~tone"=group"=final H tone (H\$), exactly like a~combination of
monosyllables with MH on the determiner and \#H or MH on the head. But the parallel ends here: if
the head is a~L-tone {monosyllable}, a~MH"=tone determiner yields a~final H tone (H\#), whereas
a~LM+MH\# determiner yields a~tone"=group"=final H (H\$).


\subsection[Cases of neutralization of tonal oppositions on the head]{About cases of neutralization of tonal oppositions on the head}
\label{sec:abouttheneutralizationoftonaloppositionsonthehead}

Four tonal categories of disyllabic head nouns always behave in the same way: the oppositions among
LM, LH, LM+\#H and LM+MH\# are neutralized. This \isi{neutralization} is not a~direct result of the application of
the seven phonological tone rules set out in \sectref{sec:alistoftonerules} and listed at several places in this volume (including the ‘Quick reference’ section). In
principle, one could imagine a~morphotonological rule whereby L tone on the determiner and LM+MH\#
on the head would yield L--LM+MH\# on the compound, through the simple concatenation of the
two tones. Thus, ‘sheep’s waist’ (input tones: L and LM+MH\#) would not be /\ipa{jo˩-ʝi˥ʈʂæ˩}/, but $\ddagger${\kern2pt}\ipa{jo˩-ʝi˩ʈʂæ˧˥}. Such a~compound would not violate conditions on well"=formedness. What may be the reasons why it is not attested? 

As observed at the outset of \sectref{sec:determinativecompoundnounsII}, in the
simpler cases the tone of the determiner expresses itself first, and after that the tone of the head
expresses itself to the extent allowed by the tones already assigned. The four categories LM, LH,
LM+\#H and LM+MH\# all have an~initial L tone; in almost all cases, expression of this L tone on the head
results in the creation of a~/H.L/ or /M.L/ sequence at the \is{juncture (inside a tone group)}juncture between the determiner and the head. Taking the simple example of a~M-tone determiner,
such as /\ipa{po˧lo˧}/ ‘ram’, and a~LM+MH\#-tone head, such as /\ipa{ʝi˩ʈʂæ˧˥}/ ‘waist’, tone
assignment can be hypothesized to take place as follows: 

\begin{enumerate}[label=(\roman*)]
	\item  the M tone of the
	determiner associates first; this yields M tone on the first two syllables of the compound /\ipa{po.lo-ʝi.ʈʂæ}/
	‘ram’s waist’, hence /\ipa{po˧lo˧-ʝi.ʈʂæ}/;
	
	\item  since the tone of the determiner is M, a~tone that
	does not spread, the tone of the head can express itself, by “left"=to"=right” association of its tone
	pattern; its first syllable receives L, through association of the first tone level in the LM+MH\#
	pattern, yielding /\ipa{po˧lo˧-ʝi˩ʈʂæ}/;
	
	\item  the last syllable receives L tone through application
	of one of the phonological rules that govern tone association in the Alawa dialect of Yongning Na, namely Rule 5: “All syllables following a~H.L or M.L sequence receive L tone”. This yields /\ipa{po˧lo˧-ʝi˩ʈʂæ˩}/.\footnote{Another way of thinking of it would be to consider that the lexical tone pattern of the head gets associated in full, yielding $\ddagger${\kern2pt}\ipa{po˧lo˧-ʝi˩ʈʂæ˧˥}, and that this expression, which does not constitute
	a~well"=formed tone sequence due to the presence of a~trough in the middle (the L in the M.M.L.MH
	sequence), is repaired by deletion of the final MH sequence and its replacement by L. Rule 5 could
	then be rephrased as Rule 5’: “All syllables following a~H.L or M.L sequence are lowered to L”. Rule
	5’ would apply after association of the entire LM+MH\# tone pattern, whereas under the present
	account, Rule 5 applies as soon as the M.L sequence is created, i.e.\ as soon as a~L tone associates to
	the syllable /{\dots}\ipa{-ʝi˩}{\dots}/. Since both views have the same practical implications, the choice of one or the other can be made freely in view of one’s theoretical
	preferences.}
\end{enumerate}

The situation illustrated by /\ipa{po˧lo˧-ʝi˩ʈʂæ˩}/ ‘ram’s waist’ is widespread: cases where the tone pattern of the compound can be analyzed as the result of the successive association of the tones of the determiner and the head (with any adjustments required by the phonological tone rules). How come the combination of input tones L and LM+MH\# does not follow this general pattern? 

Let us hypothesize that, at an earlier historical stage, input L and LM+MH\# for a~σ+σσ string yielded L.L.MH (phonologically: L+MH\#) by successive association of the two input tones. At that stage, supposing that the rest of the system was as it is now, the output L+MH\# must have been an outlier. Among the tone patterns of compound nouns with a L-tone determiner, it was the only one that contained anything other than L and H tones, because it was the only combination in which a~tone pattern beginning in L could express itself fully on the head without contravening phonological rules. Apart from this one combination, the opposition between LM, LH, LM+\#H and LM+MH\# tones was always neutralized on compounds' head nouns by application of Rule~5: “All syllables following
a~H.L or M.L sequence receive L tone”. Thus, this combination was the only \isi{counterexample} to a~local pattern of \isi{neutralization}. It appears possible that speakers modified this unique output (*L+MH\#) by \isi{analogy} with the more common pattern. 

In other words, the scenario is at follows. First, an almost complete pattern of \isi{neutralization} resulted from a~{phonological rule}. The \is{exceptions}exception was *L+MH\#, from input L and LM+MH\#. Later, \isi{neutralization} was made exceptionless (in this corner of Yongning Na morphotonology) by a~process of \isi{analogy}: the one and only \is{exceptions}exception was regularized. This change created greater uniformity in the inventory of tone patterns of compounds, as it resulted in the current pattern of full \isi{neutralization} of LM, LH, LM+\#H and LM+MH\# tones on compounds' head nouns. On the other hand, it detracted from the regularity of the correspondences between input tones and output tones, by adding an \is{exceptions}exception to the pattern of successive association of the tones of the determiner and the head. (The simplicity of the successive"=association pattern makes it pleasing to the phonologist, but apparently less so to the speakers of the language: the simplicity of rules is one thing; the simplicity of output forms is another matter.) The new output for the combination of input L and LM+MH\# was more similar to the tones of other compounds, but it had to be learnt individually.

This is admittedly a~purely speculative scenario. But formulating this kind of hypotheses can be useful in order to see the tone combination rules as a~coherent whole, rather than a~mere collection of pieces of information. Successive association of the two input tones in compounds may never have been an exceptionless rule at any {diachronic} stage: on the contrary, cases where the output tone obtains by successive association of the input tones may conceivably have been \textit{fewer} in number at some point in the past than they are now. But it is nonetheless useful to reason in terms of one prototype (in this case: successive association), listing and analyzing the combinations that do not obey this simple pattern, and looking for {structural} factors that may shed light on these cases. Further progress in this strand of research will require broader dialectal coverage than is available at present. The general topic of the {diachronic} dynamics of the Yongning Na tone system is taken up in Chapter~\ref{chap:yongningnatonesinadynamicsynchronicperspective}.


\subsection{A tendency to avoid long"=distance movement of tones}
\label{sec:longdistancedispreferred}

It was noted at the outset of this chapter, when discussing compounds
of five syllables and more, that there was in Yongning Na a~tendency to avoid
processes that would result in long"=distance movement of tone, namely
reassociation of a~H tone more than two syllables away from the word
to which it is lexically attached. This tendency is confirmed by
observations about the tone patterns that result from compounding. A
\is{floating tone}floating H tone (\#H) on the determiner plus a~M tone on the head
yields a~final H tone (H\#) on the compound, not a~\is{floating tone}floating H tone
(\#H). A~consequence is that the H tone does not move further away, as a~{floating} tone would do. To propose an~impressionistic description of
the process: the H tone only floats once; at the outcome of the
compounding process, its mode of association does not retain any
potential for further movement. It appears highly significant that the
only σσ+σσ compounds that carry a~\is{floating tone}floating H tone result from an~input
where the head had this tone in the first place, i.e.\ cases where the
\is{floating tone}floating H tone does not move in the process of compounding. It seems
as if a~\is{floating tone}floating H tone loses its ability to {float} when it becomes
modified by compounding.

Likewise, when the determiner carries H\$ or MH\#, the H level in the lexical tone tends to remain close to the determiner's last syllable, even though the
nature of this H level's \is{anchorage}anchoring is sometimes modified in compounding (as set out in \sectref{sec:Htonedems}). 
%For instance, H\$ tone yields H\$, \#H--, \#H or H\#-- on compounds (with a~range of different modes of association) depending on the tone
%of the head.

The association of L tone to long stretches of syllables does not
constitute a~\isi{counterexample} to the tendency to avoid long"=distance
movement of tones. This is tone \is{tone spreading}spreading,
neutralizing all tonal oppositions on a~portion of the tone
group; it is not an~instance of tonal movement (reassociation of a~tone
away from the word to which it is lexically attached). 
%The process is
%summarized in Chapter~\ref{chap:toneassignmentrulesandthedivisionoftheutteranceintotonegroups} as Rule 5: “All syllables following a~H.L or M.L sequence receive L tone”.

\subsection{Brief remarks about slips of the tongue}
\label{sec:slipsofthetongue}

Hesitations, variants, and tonal slips of the tongue can offer insights into the tone system. A~full"=fledged study of this captivating topic would require more fine"=grained tools than
have been employed so far. The {boundary} between an~acceptable \is{variants}variant and a~commonly occurring \is{mistakes}mistake is
not altogether clear, and the main consultant’s judgments sometimes wavered between one and the
other. While the greatest care was exercised to verify the data, the initial dichotomy between
\is{mistakes}mistakes and acceptable variants would need to be followed up by specific experiments
to ascertain the degree of acceptability of variants along a~precise scale. (On the notion of
gradient acceptability, see \citealt{kirbyetal2007,coetzee2008constraints,goldrick2011}.) An attempt at quantifying and analyzing \is{mistakes}mistakes is proposed in the discussion of numeral"=plus"=classifier phrases (\sectref{sec:aboutmistakenrealizationsintherecordings}); for compounds, only some preliminary remarks can be offered here.

To explore slips of
the tongue, it is useful to take into account, for each of the surface phonological tone patterns, (i)~the rules from whose application it can result, and (ii)~the morphosyntactic constructions in which it is attested. For instance, the L tone category is not attested in any σσ+σ determinative compound (see \tabref{tab:abstractmonosyllabicdisyllables}), so that cases where a~σσ+σ compound is realized erroneously with a~L tone cannot be put down to \isi{analogy} with other σσ+σ compounds. (Examples include the realization of ‘dog’s brains’ as $\ddagger${\kern2pt}\ipa{kʰv̩˩mi˩-ɬv̩˩} instead of
/\ipa{kʰv̩˩mi˩-ɬv̩˥}/ in the recording DetermCompounds12.) It may be relevant here that σ+σσ compounds with the same tonal input yield a~L tone (see \tabref{tab:abstractdisyllabicmonosyllables}): the slip of the tongue may be an~example of interference between the tone rules that apply on compounds with different syllabic patterns. But the erroneous pattern could also result from the mistaken application of a~{phonological rule} of tone \is{tone spreading}spreading, Rule~1: “L tone spreads progressively onto syllables that are unspecified for tone” (see \sectref{sec:alistoftonerules}). In that case, the \is{mistakes}mistake would consist in pairing the words together with a~solely phonological adjustment instead of a~morphophonological one.

Tonal slips of the tongue (“slips of the larynx”?) in the recorded Na data could also be adduced to argue that there is a~special closeness between certain pairs (or subsets) of tonal categories. The M tone category seems more liable to confusions
with \#H than with other tones. For instance, errors involving /\ipa{lɑ˧}/ ‘tiger’ consist in the
substitution of the tone pattern expected for a~\#H-tone word: $\ddagger${\kern2pt}\ipa{lɑ˧-ɬv̩˩} ‘tiger’s brains’,
instead of /\ipa{lɑ˧-ɬv̩˧˥}/, and $\ddagger${\kern2pt}\ipa{lɑ˧-hu˩mi˩} ‘tiger’s stomach’, instead of
/\ipa{lɑ˧-hu˧mi˥\$}/. (There are two instances of both of these mistakes; the recordings are: DetermCompounds6
and DetermCompounds7.) There are also instances of substitutions among H\$, MH\# and H\#, e.g.~$\ddagger${\kern2pt}\ipa{hwɤ˧li˧-sɤ˥}
instead of /\ipa{hwɤ˧li˧-sɤ˧˥}/ for ‘cat’s blood’, and $\ddagger${\kern2pt}\ipa{hwɤ˧li˧-ɬv̩˧˥} instead of
/\ipa{hwɤ˧li˧-ɬv̩˥}/ for ‘cat’s brains’. Finally, the speaker sometimes gets confused between the LM+H\# and
LM+MH\# tones.



\subsection{Perspectives for comparison across speakers}
\label{sec:crossspeakerdifferences}
\largerpage

The entire data set discussed here was provided by the consultant of reference, F4. Data was also
elicited from three other speakers: F5, who is F4’s daughter"=in"=law; M21, a~relative of F4,
belonging to the same generation; and M23, who is M21’s son. All three are less proficient
speakers than F4, due to long stays away from Yongning in the case of M21, and to a~generation gap
in the case of F5 and M23, both proficient speakers of \il{Mandarin!Southwestern}Southwestern Mandarin.

Unsurprisingly, some cross"=speaker differences are observed. Analysis of this data is crucial to understanding the dynamics of the tone system. Some of the differences reflect the fact that the nouns at issue belong in different categories in the speech of the other consultants. For instance, ‘flea’ is /\ipa{kv̩˧ʂe˥\$}/ in F4’s speech (LH
tone), whereas in M21’s speech it is /\ipa{kv̩˧ʂe\#˥}/ (\#H tone). Likewise, ‘boar’ is /\ipa{bo˩ɬɑ˥}/
in F4’s speech (LH tone), whereas in M21’s speech it fluctuates between /\ipa{bo˩ɬɑ˥}/ and
/\ipa{bo˩ɬɑ˧˥}/ (LM+MH\# tone). The difference in tones for ‘boar’s nose’ between the two speakers~-- /\ipa{bo˩ɬɑ˥-ɲi˩gɤ˩}/ for F4 (tone: LH), and /\ipa{bo˩ɬɑ˧ɲi˧gɤ\#˥}/ (tone: LM+\#H) for M21~-- is interpreted
not to have originated in a~difference in the rules that determine the tonal output
for compounds, but in a~difference in input tones. Both compounds follow the
regularities summarized in Tables~\ref{tab:abstractcompounds}a--e, but the input combination is different.

This situation requires a~full checkup of each consultant's tone system and lexicon as a~preliminary to the selection of compound nouns for elicitation. On this basis, cross"=speaker differences in the tone rules can be brought out. 

To begin with a~comparison within close age groups, \tabref{tab:differencesbetweenspeakersf4andm21} presents two differences
between F4 and M21.


\begin{table}%[t]
\caption{Differences between speakers F4 and M21 in the tones of compounds.}
{\renewcommand{\arraystretch}{1.15}
\begin{tabularx}{\textwidth}{ P{26mm} P{27mm} P{27mm} Q }
%\begin{tabularx}{\textwidth}{ P{15mm}@{\hspace{5mm}} P{15mm} P{10mm} Q }
\lsptoprule
	input tones & \multicolumn{3}{c}{output tones, with example compounds}\\
	 & F4 & M21 & meaning\\ \midrule
	M+M (σ+σ) & \#H  & L\# &\\ 
	 &  \ipa{lɑ˧-bv̩\#˥} & \ipa{lɑ˧-bv̩˩} & tiger’s intestine\\ 
	\addlinespace \hdashline \addlinespace
	\#H+H\$ (σσ+σσ) & H\#, H\$, or \#H--  & --L  &\\
	  &  \ipa{ʐwæ˧zo˧-hu˧mi˥} &  \ipa{ʐwæ˧zo˧-hu˩mi˩} & colt’s stomach\\
\lspbottomrule
\end{tabularx}}
\label{tab:differencesbetweenspeakersf4andm21}
\end{table}

The output that is obtained when combining two M tones over monosyllables is different for the two speakers, and this output is odd in both cases: why not use a~simple concatenation of
the input tones, yielding M tone for the compound, as happens for the other M+M compounds (σσ+σσ,
σσ+σ and σ+σσ)? For both F4 and M21, the output differs from what would be expected
as default. The combination of input \#H and H\$ over disyllables (σσ+σσ) is also different for the two speakers, who
both refused the other’s \is{variants}variant when I tried it on them. 

An interesting characteristic of these different patterns is that they nonetheless have a~family resemblance: M21’s patterns are not that
different from F4’s. M21 has --L tone on the phrase ‘colt’s stomach’ (σσ+σσ); this same --L tone is found in F4’s data, when {monosyllabic} input nouns with these tones are combined into a~compound (σ+σ). Such observations suggest that subtle processes may be at play, whereby individuals bound by social ties tend towards a~degree of convergence. One could speculate that, in cases where speakers wish to promote a~feeling of community, for instance in relaxed discussions with a~relative, they tend to accommodate to their interlocutor's tone patterns, occasionally adopting new patterns so as to emphasize linguistic common ground over differences. Since one and the same surface tone pattern is open to several phonological interpretations (e.g.~/M.M/ may be the realization of underlying \mbox{//M//} or \mbox{//\#H//}), this process of accommodation can result in a~proliferation of divergent \is{variants}variant forms from one speaker to another. Speakers' adoption of tone patterns from their customary interlocutors could thus explain the extent of the observed pool of morphotonological \isi{variation}. By a~related process, speakers may tend to select from within this pool of \isi{variation} those patterns that they feel will be most accessible to their addressee: avoiding, among possible variants, those that are felt to be most sharply at variance with the addressee's linguistic habits. Conversely, self"=assertive speakers who want to distance themselves from the addressee could favour linguistic patterns that they feel are most different from those used by the addressee: this would be a~possible path whereby a~\is{variants}variant acquires prominence and eventually comes to be generalized by that speaker. 

To test these hypotheses, one could examine dialogues, for instance comparing F4's tone patterns in conversations with different family members~-- whose tone patterns will need to be described with the greatest possible precision. This is a~perspective for future research; it holds promise of bringing out mechanisms that play a~key role in the evolution of the tone system, thereby shedding light on the system's synchronic outlook. 

\subsection{Exceptional items}
\label{sec:exceptionalitems}

Some compounds possess lexical tones that differ from those that would be expected on the basis of
their constituting elements. The irregularity can be due to the determiner, as in the first four lines of \tabref{tab:compoundsdiffer}, or to the head, as in its last line. These examples
are discussed one by one below.

\begin{sidewaystable}%[t]
    \caption{Compounds whose tones differ from those that would result from the application of the synchronic tone rules.}
{\renewcommand{\arraystretch}{1.35}
\begin{tabularx}{\textwidth}{P{50mm} P{30mm} P{30mm} l l l }
    \lsptoprule
 observed compounds & tone pattern & expected pattern & irregular word & meaning & tone\\ 
	\midrule
 \ipa{nɑ˩hĩ˧-kʰɯ˥dʑi˩} ‘Naxi leggings’; \ipa{nɑ˩hĩ˧-bɑ˧lɑ˥} ‘Naxi clothes’ & LM+\#H--; \newline LM+H\#-- & LM+MH\#--; \newline LM+\#H-- & 	\ipa{nɑ˩hĩ\#˥} & Naxi & LM+\#H\\
 \ipa{ɲi˧gɤ˧-dʑɯ˧˥} ‘mucus’ & MH\# & \#H & \ipa{ɲi˧gɤ\#˥} & nose & \#H\\
 \ipa{mv̩˧-ʁo˥} ‘sky’ & H\# & \#H & 	\ipa{mv̩˥} & sky & \#H\\
 \ipa{ʝi˧bv̩˧˥} ‘cows’ stable’ & MH\# & \#H & \ipa{ʝi˥} & ox & \#H\\
\addlinespace \hdashline \addlinespace \ipa{lv̩˧mi˧-tsɑ˩bɤ˩} ‘fine sand’; \ipa{qʰɑ˧dze˧-tsɑ˩bɤ˩} ‘sweetcorn flour’; \ipa{dze˧ɭɯ˧-tsɑ˩bɤ˩} ‘wheat flour’ & --L & M & \ipa{tsɑ˧bɤ˧} & powder & M\\
    \lspbottomrule
\end{tabularx}}
\label{tab:compoundsdiffer}
\end{sidewaystable}

\subsubsection{The noun ‘Naxi’}
\label{sec:thenounnaxi}

The noun ‘Naxi’ (/\ipa{nɑ˩hĩ\#˥}/, tone: LM+\#H) yields irregular results in two quadrisyllabic compounds (see the recording DetermCompounds16):
\begin{enumerate}[label=(\roman*)]
\item with MH\# tone: /\ipa{nɑ˩hĩ˧-kʰɯ˥dʑi˩}/ ‘Naxi leggings’. The regular tone pattern would be
  $\dagger$\ipa{nɑ˩hĩ˧-kʰɯ˧dʑi˧˥} (underlying tone: LM+MH\#--); however, this pattern is
  not acceptable. The observed tone is LM+\#H--. An example of the regular tone pattern
  is /\ipa{nɑ˩hĩ˧-ŋwɤ˧pʰæ˧˥}/ ‘Naxi tile’.
\item with L tone: /\ipa{nɑ˩hĩ˧-bɑ˧lɑ˥}/ ‘Naxi clothes’. The regular tone pattern would be
  $\dagger$\ipa{nɑ˩hĩ˧-bɑ˥lɑ˩} (underlying tone: LM+\#H--, which can also be described as
  LM+MH\#--), but it is not acceptable. The observed tone is LM+H\#. An example of
  the regular tone pattern is /\ipa{nɑ˩hĩ˧-sɯ˥tʰi˩}/ ‘Naxi knife’.
\end{enumerate}
‘Naxi’ has a~special status in Yongning Na. On the one hand, it refers to an~ethnic group perceived
as distinct from the Na: the Naxi of the Lijiang plain, some of whom settled in Yongning
since the early 20\textsuperscript{th} century but still retain their distinct costumes and language (on the cultural divide between the Naxi and Na, see Appendix B, in particular the end of \sectref{sec:shih19932010andweng1993}). On the other hand, it
is made up of the \isi{endonym} of the Na, compounded with the word for ‘person, human being’, so that its
independence from ‘Na’ is problematic. The exceptional treatment of this noun may have to do with
the perceived necessity of handling the term in such a~way as to attempt to avoid its perception as
a~compound meaning ‘Na person’.


\subsubsection{The nouns ‘nose’ and ‘hair’}
\label{sec:thenounsnoseandhair}

‘Nasal mucus’ is /\ipa{ɲi˧gɤ˧-dʑɯ˧˥}/ (tone: MH\#); on the basis of the input nouns, /\ipa{ɲi˧gɤ\#˥}/ ‘nose’ and
/\ipa{dʑɯ˩}/ ‘water’, the expected tone would be \#H (/†\ipa{ɲi˧gɤ˧-dʑɯ\#˥}/). ‘Hair (on the head)’,
/\ipa{ʁo˧hṽ̩˧˥}/, has the same tone (MH\#); for this word, too, the expected tone would be \#H: /†\ipa{ʁo˧hṽ̩\#˥}/ (the input nouns both have H tone: /\ipa{ʁo˥}/ ‘head; top’ and /\ipa{hṽ̩˥}/ ‘hair’).

The \isi{etymology} of these two words is self"=evident. On the other hand, mucus is not just
a~kind of water (‘nose"=water’). The difference between hair on the head and on the body may seem smaller,
but many languages have different roots for the two, e.g.~{French} and Lao \citep[187-188]{enfield2006body}. In Na, 
the compounds referring to ‘hair (on the head)’ and ‘body hair’ are lexicalized compounds (‘body hair’ is /\ipa{ʑi˩hṽ\#˥}/, literally ‘ape hair’). The irregular tone patterns of these two words may reflect an~early \isi{lexicalization}. The discrepancy between the tones of these items and the output of the currently
productive tone rules for compounds may be due to different \is{tone rules}tone rules that applied at the time when these compounds were created. Or they may result from an~evolution of the compounds away from the regular tone
pattern, triggered by the perception of their status as lexical units rather than compounds. This second possibility might sound less plausible than the first, but item"=by"=item tone
change accompanying \isi{lexicalization} is a~salient characteristic of the tone system of \il{Laze|textbf}Laze,
a~language closely related to Na,\footnote{There are four lexical tones for {Laze} monosyllables (for predicates: H, M, L and MH; for nouns: H, M, L and a~floating H tone); in
	theory, this could yield as many as sixteen tone patterns over disyllables, but only seven are
	observed. Tone
	changes occur as {lexicalization} takes place. Numerous combinations of two input tones other than H yield H+H. For instance, ‘dog’ is /\ipa{kʰɯ˧}/, and ‘to beat’ is /\ipa{ɖɯ˩}/. Their combination should
	yield M.L, but it comes out as H.H. This is a~key process in the lexical
	integration of disyllables in {Laze} \citep{michaud2008a,michaud2009a,michaudetal2012c}.} so this
possibility should not be lightly dismissed.

\subsubsection{The noun ‘flour, powder’}
\label{sec:thenounflourpowder}

The word for ‘powder, flour’ is /\ipa{tsɑ˧bɤ˧}/, with M tone. According to the synchronically
productive rules, the combination of this word with /\ipa{lv̩˧mi˧}/ ‘stone’, /\ipa{qʰɑ˧dze˧}/
‘sweetcorn’ and /\ipa{dze˧ɭɯ˧}/ ‘wheat’ should yield a~simple M-tone output,
e.g.~$\dagger$\ipa{lv̩˧mi˧-tsɑ˧bɤ˧} for ‘fine sand’. But the observed forms, shown in \tabref{tab:powder}, all carry a~M.M.L.L tone pattern, corresponding to underlying --L (a~L tone on the second part of the compound). 

\begin{table}%[t]
	\caption{Irregular compounds with /\ipa{tsɑ˧bɤ˧}/ ‘flour, powder’.}
	{\renewcommand{\arraystretch}{1.35}
		\begin{tabularx}{\textwidth}{ P{21mm} P{30mm} P{28mm} Q }
			%\begin{tabularx}{\textwidth}{ P{15mm}@{\hspace{5mm}} P{15mm} P{10mm} Q }
			\lsptoprule
			determiner & \multicolumn{3}{c}{compound with /\ipa{tsɑ˧bɤ˧}/ ‘flour, powder’ as head noun}\\
			& expected tone: M & attested tone:  --L  & meaning\\ \midrule
			/\ipa{lv̩˧mi˧}/ ‘stone’ & $\dagger$\ipa{lv̩˧mi˧-tsɑ˧bɤ˧}  & \ipa{lv̩˧mi˧-tsɑ˩bɤ˩} & fine sand\\ 
			/\ipa{qʰɑ˧dze˧}/ ‘sweetcorn’ &  $\dagger$\ipa{qʰɑ˧dze˧-tsɑ˧bɤ˧} & \ipa{qʰɑ˧dze˧-tsɑ˩bɤ˩} & sweetcorn flour\\ 
			/\ipa{dze˧ɭɯ˧}/ ‘wheat’ &  $\dagger$\ipa{dze˧ɭɯ˧-tsɑ˧bɤ˧} &  \ipa{dze˧ɭɯ˧-tsɑ˩bɤ˩} & wheat flour\\
			\lspbottomrule
		\end{tabularx}}
		\label{tab:powder}
	\end{table}

The principle applied in the present study is that two lexical items of like phonological structure whose tones differ in at least one morphosyntactic context need to be recognized as belonging to different lexical tone categories. Mechanical application of this principle would lead to recognition 
%\rephrase{of the word}%
%{of a word such as }
of /\ipa{tsɑ˧bɤ˧}/ ‘flour, powder’ as the sole example of an umpteenth (twelfth) tonal category of disyllabic nouns. But it makes much more sense to try to explain the irregular behaviour of the compounds in \tabref{tab:powder} by other factors.

The word /\ipa{tsɑ˧bɤ˧}/ is a~\ili{Tibetan} \is{loanwords}loanword (from \textit{rtsam pa} ‘roasted flour’). The lowering of the tones on the latter part of the compounds in \tabref{tab:powder} echoes observations about a~larger set of words of \ili{Tibetan} origin: given names. These compounds will be referred to for short as ‘\ili{Tibetan} compounds’; this topic will be taken up in the analysis of proper names in \sectref{sec:anindependentsetoffactscompoundgivennamesandtermsofaddress}.

\subsubsection{The noun ‘sky’}
\label{sec:thenounsky}

The disyllabic form of the noun ‘sky’ is /\ipa{mv̩˧ʁo˥\$}/. The \is{monosyllables}monosyllabic root for ‘sky’ is
/\ipa{mv̩˥}/; if the second syllable were /\ipa{ʁo˥}/ ‘head; top’, one would expect the compound to
have \#H tone (on the basis of the regularities set out in Tables~\ref{tab:surfacecompounds}a--e). But the second
syllable could also be the \is{postpositions}postposition ‘on’ – itself likely to be grammaticalized from ‘head’. This \is{postpositions}postposition is not in common use anymore (the common form is //\ipa{bi˩}// ‘on; at’: see \sectref{sec:ltoneencliticspluralandassociativeplural}); this is an obstacle to establishing its lexical tone and studying its combinatory properties.


\section{Coordinative compounds}
\label{sec:coordinativecompounds}

\subsection{The main facts}
\label{sec:themainfactscoordinativecompounds}

In the closely related language \ili{Naxi}, the tones of coordinative compounds are simply the
concatenation of those of their constituents, e.g.~/\ipa{ɲi˧nv̩˩-jæ˧kæ˩zɯ˧}/ ‘wife and husband’ from /\ipa{ɲi˧nv̩˩}/ ‘wife’ and /\ipa{jæ˧kæ˩zɯ˧}/ ‘husband’.\footnote{This tonally inert compound is of little phonological interest; on the other hand, it has some ethnolinguistic interest. As the {Naxi} of Lijiang like to point out, this compound places the wife in front of the
husband, in contradiction of Confucian principles.} In Yongning Na, on the other hand, coordinative compounds are tonally active, to an~extent
comparable with determinative compounds.

Coordinative compounds are less common than determinative compounds, however, and less easy to
elicit systematically. Syntactically, coordinative constructions can be applied to any pair of nouns. This is exemplified by the names of public houses in \ili{English}. Combinations like “Fox and Hounds” or “Dog and Duck” refer to hunting traditions; others, such as “Bear and
Ragged Staff”, refer to heraldry. Once the pattern is established, new coordinative combinations can be created at will, such as the humorous “Snail and
Salad”, where the relationship between the two terms~-- and their relationship to the food served in
the pub~-- is offered to the customer’s fancy. In Yongning Na, any two nouns can be coordinated by
means of the \is{conjunctions}conjunction /\ipa{lɑ˧}/ ‘and’, but coordinative compound nouns are not as easy to coin: the two nouns must refer to entities that are commonly paired together.

Three sources were found: pairs of animal names of the two sexes, and their offspring, such as ‘ewe
and ram’, ‘ram and ewe’, ‘ewe and lamb’, and ‘ram and lamb’; pairs of kinship terms, such as ‘uncle
and nephew’ and ‘mother and daughter’; and successive numerals followed by the same classifier, such
as ‘two or three years’, ‘five or six months’ or ‘four or five days’. A~broad sample of the first
two sets can be found in the online recording CoordCompounds; the third set is found in
CoordCompounds2. The consultant (F4) preferred to remain within the bounds of common
sense, and non"=matching pairs such as ‘mother and nephew’ or ‘grandmother and brother’
were avoided. The data is set out in Tables \ref{tab:coordinativecompoundsmono}-\ref{tab:examplesofcoordinativeHpound}. As elsewhere, a~slash separates
variants. Some tone patterns that were proposed by the investigator and refused by the consultant
are indicated, with a~double dagger $\ddagger$, in the output column: for instance, in view of the existence of two
variants for the combination /\ipa{ʐv̩˩ɬi˩-ŋwɤ˩ɬi˩˥}/{\kern2pt}\ipa{≈}{\kern2pt}/\ipa{ʐv̩˩ɬi˩-ŋwɤ˥ɬi˩}/ ‘four or five months’, whose input nouns both have L tone, it was attempted to apply the L tone pattern of the \is{variants}variant /\ipa{ʐv̩˩ɬi˩-ŋwɤ˩ɬi˩˥}/ to other combinations of two L-tone nouns, such as
‘nephews and nieces’, /\ipa{ze˩v̩˩-ze˥mi˩}/. The fact that the L {variant} is not possible for these
expressions ($\ddagger${\kern2pt}\ipa{ze˩v̩˩-ze˩mi˩˥}) is indicated through the mention ‘($\ddagger${\kern2pt}L)’ in the output column.

Three suffixes appear repeatedly in the table: the female/{\allowbreak}{augmentative} \is{suffixes}suffix /\ipa{-mi˩}/, the male
\is{suffixes}suffix /\ipa{-pʰv̩˥}/, and the child/{\allowbreak}male/{\allowbreak}{diminutive} suffixe /\ipa{-zo˥}/. These suffixes are discussed in more detail in~\sectref{sec:thegendersuffixes}.

Most examples are quadrisyllabic, from two input disyllables (σσ+σσ). The two disyllabic examples
(σ+σ) at the top of \tabref{tab:coordinativecompoundsmono} are written without a~hyphen, on the basis of the intuition that
they are more strongly integrated than the others. Trisyllabic examples are of the structure
\textit{disyllable plus monosyllable} (σσ+σ), showing a~preference for coordinative compounds
where the first term has at least as many syllables as the second. Hexasyllabic compounds can be
created, e.g.~/\ipa{ŋwɤ˩-ɬi˩mi˩-qʰv̩˥-ɬi˩mi˩}/ (Dog2.64) ‘the fifth and sixth months’. 

%Table placement: centering in remaining space would be good. I don't know how to do this, so chose an ugly solution: to add blank lines. 
%Placing the table at bottom does not look good.
%Result to be checked on proofs.
~\newline ~
~\newline  ~
~\newline  ~

\setlength{\defaultaddspace}{2.5pt}
\begin{subtables}
\begin{table}%[t]
  \caption{Coordinative compounds of fewer than four syllables, arranged by input tones.}
  \begin{tabularx}{\textwidth}{ l Q l l }
    \lsptoprule
  	compound & meaning & input & output\\ \midrule
	\ipa{mv̩˧di˧˥} & universe (‘sky’+‘earth’) & H and LM & MH\#\\ \addlinespace \hdashline \addlinespace
	\ipa{zo˧mv̩˥} & child (‘son’+‘daughter’) & H and LH & H\#\\ \addlinespace \hdashline \addlinespace
	\ipa{ə˧mi˧-mv̩˩} & mother and daughter & M and LH & --L\\ \addlinespace \hdashline \addlinespace
	\ipa{ə˧mi˧-zo\#˥} & mother and son & M and H & \#H\\ \addlinespace \hdashline \addlinespace
	\ipa{ə˧dɑ˧-mv̩˥} & father and daughter & H\$ and LH & H\#\\ \addlinespace \hdashline \addlinespace
	\ipa{ə˧dɑ˧-zo\#˥} & father and son & H\$ and H & \#H\\
\lspbottomrule
  \end{tabularx}
\label{tab:coordinativecompoundsmono}
\end{table}

\begin{table}%[t]
  \caption{Quadrisyllabic compounds with M as the first input tone.}
  \begin{tabularx}{\textwidth}{ l Q l l }
    \lsptoprule
  	compound & meaning & input & output\\ \midrule
	\ipa{ə˧pʰv̩˧-ʐv̩˧v̩\#˥} & great"=uncle and great"=nephews & M and M & \#H\\
	\ipa{ə˧pʰv̩˧-ʐv̩˧mi\#˥} & great"=uncle and great"=nieces &  &\\
	\ipa{ə˧si˧-ə˧pʰv̩\#˥} & 3\textsuperscript{rd}-generation ancestors &  &\\
	\ipa{ə˧si˧-ʐv̩˧mi\#˥} & (great-)grandmother and granddaughters &  &\\
	\ipa{ə˧si˧-ʐv̩˧v̩\#˥} & great"=grandmother and grandsons &  &\\ \addlinespace \hdashline \addlinespace
	\ipa{gv̩˧dv̩˧-gv̩˧mi˧} & (human) body & M and M & M\\
	\ipa{jo˧mi˧-po˧lo˧} & ewe and ram &  &\\
	\ipa{ɖʐwæ˧mi˧-ɖʐwæ˧pʰv̩˧} & male and female sparrow &  &\\
	\ipa{ɖɯ˧ɬi˧-ɲi˧ɬi˧} & one or two months &  &\\ \addlinespace \hdashline \addlinespace
	\ipa{ɲi˧ɬi˧-so˩ɬi˩} & two or three months & M and M & --L\\ \addlinespace \hdashline \addlinespace
	\ipa{bæ˧mi˧-bæ˧pʰv̩\#˥} & female and male duck & M and \#H & \#H\\
	\ipa{bæ˧mi˧-bæ˧zo\#˥} & female duck and duckling &  &\\
	\ipa{bv̩˧mi˧-bv̩˧zo\#˥} & female yak and yak calf &  &\\ \addlinespace \hdashline \addlinespace
   \ipa{ʂɯ˧ɬi˧-hõ˧ɬi\#˥} & seven or eight months & M and H\$ & \#H\\ \addlinespace \hdashline \addlinespace
	\ipa{ə˧mi˧-ze˩mi˩} & aunt and niece & M and L & --L\\
	\ipa{ə˧mi˧-ze˩v̩˩} & aunt and nephew &  &\\
	\ipa{bv̩˧mi˧-bv̩˩ʂwæ˩} & female and male yak &  &\\
	\ipa{dzo˧mi˧-dzo˩pʰv̩˩} & female and male lizard &  &\\
	\ipa{so˧ɬi˧-ʐv̩˩ɬi˩} & three or four months &  &\\
   \lspbottomrule
  \end{tabularx}
\label{tab:examplesofcoordinativecompoundsarrangedbyinputtones}
\end{table}

%Table 6c
%Table 6. in manuscript
\begin{table}[p]
  \caption{Quadrisyllabic compounds with \#H as the first input tone.}
\scalebox{0.92}{
  \begin{tabularx}{\textwidth}{ P{31mm} Q l l }
    \lsptoprule
  	compound & meaning & input & output\\ \midrule
	\ipa{gi˧zɯ˧-go˧mi\#˥} & little brothers and sisters & \#H and M & \#H\\
	\ipa{bæ˧zo˧-bæ˧mi\#˥} & duckling and female duck  &  &\\
	\ipa{bæ˧pʰv̩˧-bæ˧mi\#˥} & male duck and female duck  &  &\\ \addlinespace \hdashline \addlinespace
	\ipa{tsʰɯ˧zo˧-to˧qɑ˥} & kids and little nanny goats & \#H and M & H\#\\ \addlinespace \hdashline \addlinespace
	\ipa{ʐv̩˧v̩˥-ʐv̩˩mi˩} & grandchildren & \#H and \#H & H\#--\\ \addlinespace \hdashline \addlinespace
	\ipa{hwɤ˧pʰv̩˧-hwɤ˧zo\#˥} / \ipa{hwɤ˧pʰv̩˧-hwɤ˥zo˩} & tom"=cat and kitten & \#H and \#H & \#H~/ \#H--\\
	\ipa{ho˧mi˧-ho˧pʰv̩\#˥} / \ipa{ho˧mi˧-ho˥pʰv̩˩} & female and male pheasant &  &\\
	\ipa{dʑi˧mi˧-dʑi˧zo\#˥} / \ipa{dʑi˧mi˧-dʑi˥zo˩} & female and baby buffalo &  &\\
	\ipa{dʑi˧zo˧-dʑi˧mi\#˥} / \ipa{dʑi˧zo˧-dʑi˥mi˩} & baby and female buffalo &  &\\
	\ipa{lɑ˧mi˧-lɑ˧pʰv̩\#˥} / \ipa{lɑ˧mi˧-lɑ˥pʰv̩˩} & female and male tiger &  &\\
	\ipa{lɑ˧mi˧-lɑ˧zo\#˥} / \ipa{lɑ˧mi˧-lɑ˥zo˩} & female and baby tiger &  &\\
	\ipa{ʐv̩˧ɲi˧-ŋwɤ˧ɲi\#˥} / \ipa{ʐv̩˧ɲi˧-ŋwɤ˥ɲi˩} & four or five days &  &\\ \addlinespace \hdashline \addlinespace
	\ipa{ʁv̩˧pʰv̩˧-ʁv̩˧mi\#˥} & male and female crane & \#H and MH & \#H\\ \addlinespace \hdashline \addlinespace
	\ipa{hwɤ˧pʰv̩˧-hwɤ˧mi˥} & tom"=cat and she"=cat & \#H and H\$ & H\#\\
	\ipa{hwɤ˧zo˧-hwɤ˧mi˥} & cats: kitten and parents &  &\\ \addlinespace \hdashline \addlinespace
	\ipa{ŋwɤ˧ɲi˧-qʰv̩˩ɲi˩} & five or six days & \#H and H\$ & --L\\
	\ipa{ʂɯ˧ɲi˧-hõ˩ɲi˩} & seven or eight days &  &\\ \addlinespace \hdashline \addlinespace
	\ipa{ʐwæ˧zo˧-ʐwæ˥mi˩}~/ \ipa{ʐwæ˧zo˧-ʐwæ˧mi˥} & colt and mare & \#H and L & \#H--/ H\#\\
	\ipa{pʰɤ˧pʰv̩˧-pʰɤ˥mi˩} / \ipa{pʰɤ˧pʰv̩˧-pʰɤ˧mi˥} & male and female hyena &  &\\ \addlinespace \hdashline \addlinespace
	\ipa{gv̩˧ɲi˧-tsʰe˩ɲi˩} / \ipa{gv̩˩ɲi˩-tsʰe˩ɲi˥} & nine or ten days & \#H and L & --L~/ L+H\#\\ \addlinespace \hdashline \addlinespace
	\ipa{kʰv̩˧zo˥-kʰv̩˩mv̩˩} / \ipa{kʰv̩˧zo˧-kʰv̩˧mv̩˥} & male and female puppies & \#H and H\# & H\#-- / H\#\\
	\lspbottomrule
  \end{tabularx}}
\label{tab:examplesofcoordinativecFLOATINGH}
\end{table}

\begin{table}%[t]
  \caption{Quadrisyllabic compounds with MH\# as the first input tone.}
  \begin{tabularx}{\textwidth}{ P{29mm} Q l l }
    \lsptoprule
  	compound & meaning & input & output\\ \midrule
   \ipa{ə˧ʑi˧-ə˧pʰv̩˧˥} & elders, grandparents & MH\# and M & MH\#\\ \addlinespace \hdashline \addlinespace
   \ipa{æ˧mv̩˧-go˧mi˥} & sisters, female siblings & MH\# and M & H\#\\ \addlinespace \hdashline \addlinespace
   \ipa{ə˧ʑi˧-ʐv̩˥v̩˩} & grandmother and grandsons & MH\# and \#H & \#H--\\
	\ipa{ə˧ʑi˧-ʐv̩˥mi˩} & grandmother and granddaughter &  &\\
	\ipa{æ˧mv̩˧-gi˥zɯ˩} & brethren, brothers &  &\\ \addlinespace \hdashline \addlinespace
	\ipa{ɖɯ˧kʰv̩˧-ɲi˥kʰv̩˩} & one or two years & MH\# and MH\# & MH\#--\\ \addlinespace \hdashline \addlinespace
	\ipa{ə˧v̩˧-ze˥v̩˩} & uncle and nephew & MH\# and L & MH\#--\\
	\ipa{ə˧v̩˧-ze˥mi˩} & uncle and niece &  &\\
	\ipa{zo˧hṽ̩˧-mv̩˥zo˩} & descendants &  &\\ \addlinespace \hdashline \addlinespace
	\ipa{ɲi˧kʰv̩˧-so˧kʰv̩˥} ($\ddagger${\kern2pt}\ipa{ɲi˧kʰv̩˧-so˥kʰv̩˩}) & two or three years & MH\# and L & H\#\\ \addlinespace \hdashline \addlinespace
	\ipa{ʂɯ˧kʰv̩˧-hõ˥kʰv̩˩} / \ipa{ʂɯ˧kʰv̩˧-hõ˧kʰv̩˥} & seven or eight years & MH\# and H\# & MH\#-- / H\#\\
   \lspbottomrule
  \end{tabularx}
\label{tab:examplesofcoordinativeMH}
\end{table}

%Table 6e
%Table 6. in manuscript
\begin{table}[p]
  \caption{Quadrisyllabic compounds with H\$ as the first input tone.}
\scalebox{0.95}{
  \begin{tabularx}{\textwidth}{ P{33mm} Q l P{15mm} }
    \lsptoprule
  	compound & meaning & input & output\\ \midrule
	\ipa{tsʰɯ˧mi˧-po˧lo˥ } & nanny goat and billy goat & H\$ and M & H\#\\ \addlinespace \hdashline \addlinespace
	\ipa{ə˧dɑ˧-ə˧mi\#˥ } & father and mother, parents & H\$ and M & \#H\\ \addlinespace \hdashline \addlinespace
	\ipa{qʰv̩˧ɬi˥-ʂɯ˩ɬi˩ } & six or seven months & H\$ and M & H\#--\\ \addlinespace \hdashline \addlinespace
	\ipa{ə˧ɲi˧tsʰi˧ɲi\#˥} & these days & H\$ and \#H & \#H\\
	\ipa{ə˧ʝi˧-tsʰi˧ʝi\#˥} & these years &  &\\
	\ipa{ʈʂʰæ˧mi˧-ʈʂʰæ˧zo\#˥} & doe and stag &  &\\ \addlinespace \hdashline \addlinespace
	\ipa{hwɤ˧mi˧-hwɤ˧zo\#˥} / \ipa{hwɤ˧mi˧-hwɤ˧zo˥\$} & she"=cat and kitten & H\$ and \#H & \#H~/ H\$\\ \addlinespace \hdashline \addlinespace
   \ipa{hwɤ˧mi˧-hwɤ˥pʰv̩˩} / \ipa{hwɤ˧mi˧-hwɤ˧pʰv̩\#˥ / hwɤ˧mi˧-hwɤ˧pʰv̩˥\$} & she"=cat and tom"=cat &
   H\$ and \#H & \#H-- / \#H~/ H\$\\ \addlinespace \hdashline \addlinespace
   \ipa{qʰv̩˧ɲi˥-ʂɯ˩ɲi˩} / \ipa{qʰv̩˧ɲi˧-ʂɯ˥ɲi˩ / qʰv̩˧ɲi˧-ʂɯ˧ɲi\#˥} & six or seven days & H\$ and \#H & H\#-- / \#H-- / \#H\\
	\ipa{($\ddagger${\kern2pt}qʰv̩˧ɲi˧-ʂɯ˧ɲi˥\$)} &  &  &\\
	\ipa{hõ˧ɲi˥-gv̩˩ɲi˩} /  & eight or nine days &  &\\
	\ipa{hõ˧ɲi˧-gv̩˥ɲi˩} /  &  &  &\\
	\ipa{hõ˧ɲi˧-gv̩˧ɲi\#˥} &  &  &\\
	(\ipa{$\ddagger${\kern2pt}hõ˧ɲi˧-gv̩˧ɲi˥\$}) &  &  &\\ \addlinespace \hdashline \addlinespace
   \ipa{ɲi˧ɲi˧-so˧ɲi˥} ($\ddagger${\kern2pt}\ipa{ɲi˧ɲi˥-so˩ɲi˩}) & two or three days & H\$ and MH\# & H\#\\ \addlinespace \hdashline \addlinespace
	\ipa{ɖɯ˧ɲi˧-ɖɯ˥hɑ̃˩} ($\ddagger${\kern2pt}\ipa{ɖɯ˧ɲi˧-ɖɯ˧hɑ̃˥}) & one day and one night & & \#H--\\ \addlinespace \hdashline \addlinespace
	\ipa{ʈʂʰæ˧mi˧-ʈʂʰæ˧zo\#˥} & doe and fawn & H\$ and H\$ & \#H\\
	\ipa{ɖɯ˧ɲi˧-ɲi˧ɲi\#˥} & one or two days &  &\\ \addlinespace \hdashline \addlinespace
	\ipa{hõ˧ɬi˥-gv̩˩ɬi˩} & eight or nine months & H\$ and L & H\#--\\
	\ipa{ɲi˧ɲi˥} {\kern2pt}|{\kern2pt} \ipa{-so˩ɲi˩˥} & two or three days &  &\\ \addlinespace \hdashline \addlinespace
	\ipa{ə˧ʝi˧-ʂɯ˥ʝi˩} & in the past & H\$ and LM+\#H & \#H--\\
\lspbottomrule
  \end{tabularx}}
\label{tab:examplesofcoordinativeDOLLAR}
\end{table}

\begin{table}%[t]
  \caption{Quadrisyllabic compounds with L as the first input tone.}
  \begin{tabularx}{\textwidth}{ P{30mm} Q l l }
    \lsptoprule
  	compound & meaning & input & output\\ \midrule
	\ipa{bv̩˩ʂwæ˩-bv̩˥mi˩} / \ipa{bv̩˩ʂwæ˩-bv̩˩mi˩} & male yak and female yak & L and M & L~/
   L+\#H--\\
   \ipa{kɤ˩pʰv̩˩-kɤ˩mi˥} / \ipa{kɤ˩pʰv̩˩-kɤ˥mi˩} & male and female falcon &  &\\
	\ipa{dzo˩pʰv̩˩-dzo˩mi˩} / \ipa{dzo˩pʰv̩˩-dzo˥mi˩} & male and female lizards &  &\\ \addlinespace \hdashline \addlinespace
   \ipa{mv̩˩zo˩-ə˥mi˩} / \ipa{mv̩˩zo˩-ə˩mi˥} & young woman and (her) mother & L and M & L+\#H-- / L+H\#\\
	\ipa{gv̩˩ɬi˩-tsʰe˥ɬi˩} / \ipa{gv̩˩ɬi˩-tsʰe˩ɬi˥} & nine or ten months &  &\\ \addlinespace \hdashline \addlinespace
	\ipa{mv̩˩zɯ˩-ni˥mi˩} & brothers and sisters & L and \#H & L+\#H--\\ \addlinespace \hdashline \addlinespace
\ipa{ʐwæ˩mi˩-ʐwæ˩zo˩} & mare and colt & L and \#H & L\\
	\ipa{so˩ɲi˩-ʐv̩˩ɲi˩} & three or four days &  &\\ \addlinespace \hdashline \addlinespace
   \ipa{ŋwɤ˩ɬi˩-qʰv̩˥ɬi˩} & five or six months & L and H\$ & \#H--\\ \addlinespace \hdashline \addlinespace
	\ipa{ʝi˩mi˩-ʐɤ˥qo˩} & cow and calf & L and L & L+\#H-- ($\ddagger${\kern2pt}L)\\
	\ipa{ze˩v̩˩-ze˥mi˩} & nephews and nieces &  &\\
	\ipa{ʝi˩bv̩˩-ʝi˥mi˩} & bull and cow &  &\\
	\ipa{pɤ˩mi˩-pɤ˥pʰv̩˩} & female and male frog &  &\\
	\ipa{pʰɤ˩mi˩-pʰɤ˥zo˩} & female hyena and hyena pup  &  &\\ \addlinespace \hdashline \addlinespace
	\ipa{ʐv̩˩ɬi˩-ŋwɤ˩ɬi˩˥} / \ipa{ʐv̩˩ɬi˩-ŋwɤ˥ɬi˩} & four or five months & L and L & L~/ L+\#H--\\
	\ipa{so˩kʰv̩˩-ʐv̩˩kʰv̩˥} & three or four years & L and L\# & L+H\#\\
   \lspbottomrule
  \end{tabularx}
\label{tab:examplesofcoordinativeL}
\end{table}


%Table 6g
%Table 6. in manuscript
\begin{table}%[t]
\caption{Quadrisyllabic compounds with L\# as the first input tone.}
  \begin{tabularx}{\textwidth}{ Q Q P{20mm} l }
    \lsptoprule
  	compound & meaning & input & output\\ \midrule
	\ipa{ʐwæ˧sɯ˩-ʐwæ˩zo˩} & stallion and colt & L\# and \#H & L\#--\\ \addlinespace \hdashline \addlinespace
	\ipa{ʐwæ˧sɯ˩-ʐwæ˩mi˩} & stallion and mare & L\# and L & L\#--\\
	\ipa{gv̩˧kʰv̩˩-tsʰe˩kʰv̩˩} & nine or ten years &  &\\ \addlinespace \hdashline \addlinespace
	\ipa{ʐv̩˧kʰv̩˩-ŋwɤ˩kʰv̩˩} & four or five years & L\# and L\# & L\#--\\ \addlinespace \hdashline \addlinespace
	\ipa{ŋwɤ˧kʰv̩˩-qʰv̩˩kʰv̩˩} & five or six years & L\# and H\# & L\#--\\
\lspbottomrule
  \end{tabularx}
\label{tab:examplesofcoordinativeLpound}
\end{table}

\begin{table}%[t]
  \caption{Quadrisyllabic compounds with LM as the first input tone.}
  \begin{tabularx}{\textwidth}{ l Q P{26mm} l }
    \lsptoprule
  	compound & meaning & input & output\\ \midrule
   \ipa{ɑ˩ʁo˧-ʑi˧dv̩˧} & the household & LM and M & LM--\\ \addlinespace \hdashline \addlinespace
	\ipa{pv̩˩tsɯ˧-pv̩˥mi˩} & small and large combs & LM+MH\# and L & LM+MH\#--\\
	\ipa{pɤ˩tɕi˧-pɤ˥mi˩} & tadpole &  &\\ \addlinespace \hdashline \addlinespace
	\ipa{æ˩mi˧-æ˧ʂwæ˥} & hen and cock & LM and H\# & LM+H\#\\
	\ipa{æ˩mi˧-æ˧tsɯ˥} & hen and chicks &  &\\
	\ipa{bo˩mi˧-bæ˧bv̩˥} & sow and piglets &  &\\ \addlinespace \hdashline \addlinespace
	\ipa{ʐæ˩pʰv̩˧-ʐæ˩mi˩} & male and female leopard & LM and LM+\#H & LM--L\\ \addlinespace \hdashline \addlinespace
	\ipa{dv̩˩mi˧-dv̩˥pʰv̩˩} & female and male weasels & LM+\#H and LM & LM+\#H--\\ \addlinespace \hdashline \addlinespace
   \ipa{ɑ˩mi˧-ɑ˥pʰv̩˩} & female and male goose  & \multirow{2}{26mm}{LM+\#H and LM+\#H} & LM+\#H--\\
   \ipa{ɖɯ˩zo˧-ɖɯ˥mi˩} & female and male mule  &  &\\
   \lspbottomrule
  \end{tabularx}
\label{tab:examplesofcoordinativeLM}
  \end{table}

\begin{table}%[t]
  \caption{Compounds of four to six syllables with H\# as the first input tone.}
  \begin{tabularx}{\textwidth}{ l Q l l }
    \lsptoprule
  	 compound & meaning & input & output\\ \midrule
   \ipa{qʰv̩˧kʰv̩˥-ʂɯ˩kʰv̩˩} & six or seven years & H\# and MH\# & H\#--\\ \addlinespace \hdashline \addlinespace
	\ipa{hõ˧kʰv̩˥-gv̩˩kʰv̩˩} & eight or nine years & H\# and L\# & H\#--\\ \addlinespace \hdashline \addlinespace
	\ipa{æ˧ʂwæ˥-æ˩mi˩} & cock and hen & H\# and LM & H\#--\\
	\ipa{ŋwɤ˩ɬi˩mi˩-qʰv̩˥ɬi˩mi˩} & the fifth and sixth months & &\\
\lspbottomrule
\end{tabularx}
\label{tab:examplesofcoordinativeHpound}
\end{table}
\end{subtables}
  
% The subsection below needs to start on a clean page, after all floats have been unpiled.
\clearpage

\subsection{Discussion: Tonal variability and lexical diversity}
\label{sec:discussiontonalvariabilityandsemanticlexicaldiversity}


The existence of variants was already observed for some determinative compounds (see Tables~\ref{tab:abstractcompounds}a--e), but the overall
proportion of combinations that have variants is low: 7 combinations out of 257, all of which have
H\$ tone on the head noun. For coordinative compounds, on the other hand, less regularity is
observed. There are not only tonal variants, but compounds with identical input tones and
different outputs. Among quadrisyllabic compounds, two different outputs (on different examples) are
found for no fewer than six tonal combinations: those with input tones M and M, \#H and M, MH\# and
M, H\$ and M, H\$ and \#H, and L and \#H. A~seventh combination, \#H and \#H, even has three
different outputs. For quadrisyllabic compounds, the proportion of tone combinations with two or
three different outputs is about one out of four.

The general picture is thus one of great tonal diversity. But coordinative compounds are not
simply characterized by a~general looseness of their tone patterns, whereby two or three tonal
variants would be acceptable for any combination of input tones. In cases where the same tonal input yields different tones in different coordinative compounds, it was attempted to substitute one for the other. For instance, input H\$ and M nouns yield a~compound with \#H tone in the case of ‘little brothers and little sisters’, //\ipa{gi˧zɯ˧-go˧mi\#˥}//, but a~compound with H\# tone in the case of ‘kids and little nanny goats’, //\ipa{tsʰɯ˧zo˧-to˧qɑ˥}//; it was therefore attempted to substitute tones \#H and H\# for each other on coordinative compounds. These attempted variants were rejected by the consultant, as shown in \tabref{tab:braveattempts}. Clearly, each coordinative compound comes to carry a~habitual tone pattern to the exclusion of others. Cross-speaker comparison (within the family, then extending the comparison to other micro"=dialects) will be necessary to explore the paths of development of idiosyncratic preferences. 

\begin{table}%[t]
	\caption{Attempted tonal variants for coordinative compounds.}
	\fittable{
	\begin{tabular}{llllll}
		\lsptoprule
	    meaning & input tones &	\multicolumn{2}{l}{attested form} & \multicolumn{2}{l}{attempted variant}\\ \cmidrule(lr){1-2}\cmidrule(lr){3-4}\cmidrule(lr){5-6}
%		 & full form & tone & attempted {variant} & tone\\\midrule
		little brothers and little sisters & H\$ and M & \ipa{gi˧zɯ˧-go˧mi\#˥} & \#H & $\ddagger${\kern2pt}\ipa{gi˧zɯ˧-go˧mi˥} & H\#\\
		father and mother & H\$ and M & \ipa{ə˧dɑ˧-ə˧mi\#˥} & \#H & $\ddagger${\kern2pt}\ipa{ə˧dɑ˧-ə˧mi˥} & H\#\\
		kids and little nanny goats & H\$ and M & \ipa{tsʰɯ˧zo˧-to˧qɑ˥} & H\# & $\ddagger${\kern2pt}\ipa{tsʰɯ˧zo˧-to˧qɑ\#˥} & \#H\\
		nanny goat and billy goat & H\$ and M & \ipa{tsʰɯ˧mi˧-po˧lo˥} & H\# & $\ddagger${\kern2pt}\ipa{tsʰɯ˧mi˧-po˧lo\#˥} & \#H\\
		ancestors & MH\# and M & \ipa{ə˧ʑi˧-ə˧pʰv̩˧˥} & MH\# & $\ddagger${\kern2pt}\ipa{ə˧ʑi˧-ə˧pʰv̩˥} & H\#\\
		sisters & MH\# and M & \ipa{æ˧mv̩˧-go˧mi˥} & H\# & $\ddagger${\kern2pt}\ipa{æ˧mv̩˧-go˧mi˧˥} & MH\#\\
		\lspbottomrule
	\end{tabular}
	}
	\label{tab:braveattempts}
\end{table}

The diversity of tone patterns relates in subtle ways to the semantic diversity of coordinative compounds. Importantly, it is not always
possible to arrive at the meaning of coordinative compounds simply on the basis of their two
constituents. For instance, /\ipa{hwɤ˧zo˧-hwɤ˧mi˥}/, made up of ‘kitten’ and ‘she"=cat’, does not
mean ‘kitten and she"=cat’ (the child and the mother), but refers to cats in general, as
a~species. The terms for male and female puppies, /\ipa{kʰv̩˧zo\#˥}/ and
/\ipa{kʰv̩˧mv̩\#˥}/ respectively, have come to be used as names for human newborns: an~unlovely name is purposedly
chosen to repel demons who may be lurking around to take their lives.\footnote{This is known in Chinese as “milk name” (\zh{乳名} \textit{rǔmíng}), and constitutes one of the types of “names intended to avoid attracting the unwanted attention of gods, sparing the name"=bearers the misfortunes wrought by the god's wrath or jealousy” \citep[118]{chen2016}.} The real name is only given
after a~couple of months, or sometimes as late as one full year after birth. The two terms
/\ipa{kʰv̩˧zo\#˥}/ and /\ipa{kʰv̩˧mv̩\#˥}/, and their compound /\ipa{kʰv̩˧zo˥-kʰv̩˩mv̩˩}/, have become
culturally specialized and have ceased to be used to refer to real puppies. There is thus a~broad range of situations, from elicited
combinations which the consultant may never have conceptualized before (such as ‘male and female
jackal’) to highly lexicalized expressions. This sheds indirect light on the observed tonal diversity. However, the overall number of examples
is too small to determine with confidence which outputs are currently productive. The logical next step to investigate this issue would be to test the degree of \isi{lexicalization} of the various compounds. 

The tone patterns observed to date are summarized in
Tables~\ref{tab:thetonepatternsofcoordinativecompoundscombinationswithmonosyllabicsecondnoun}
and
\ref{tab:thetonepatternsofcoordinativecompoundscombinationsamongdisyllables}. When
two different (and mutually exclusive) patterns are observed over
different compounds, these patterns are separated by a~semi"=colon. In
cases of free \isi{variation} (over the same compound), the patterns are
separated by a~slash. In order to save space, tone categories for
which there are no examples are simply omitted from the table. 

\begin{subtables}
\begin{table}%[t]
\caption{\label{tab:thetonepatternsofcoordinativecompoundscombinationswithmonosyllabicsecondnoun}The tones of coordinative compounds with {monosyllabic} second noun. A~{question} mark indicates that no example was found.}
\begin{tabularx}{.75\textwidth}{ l@{\hspace{7mm}} l@{\hspace{7mm}} l@{\hspace{7mm}} Q Q }
\lsptoprule
	\multirow{2}{12mm}[-1.6mm]{type of 1\textsuperscript{st}~noun} & \multirow{2}{11mm}[-1.6mm]{tone of 1\textsuperscript{st}~noun} & \multicolumn{3}{l}{tone of 2\textsuperscript{nd} noun}\\ \cmidrule{3-5}
	 &  & LM  & L  & \#H\\\midrule
	monosyllables & \#H & MH\# & H\# & ?\\ \addlinespace \hdashline \addlinespace
	disyllables & M & ? & --L & \#H\\
	 & H\$ & ? & H\# & \#H\\
\lspbottomrule
\end{tabularx}
\end{table}


\begin{sidewaystable}[p]
\caption{\label{tab:thetonepatternsofcoordinativecompoundscombinationsamongdisyllables}The tones of coordinative compounds consisting of two disyllables. A~{question} mark indicates that no example was found.}
{\renewcommand{\arraystretch}{1.35}
{\setlength\tabcolsep{5pt}
\begin{tabularx}{\textwidth}{ Q l l l l l l l l }
\lsptoprule
& 2\textsuperscript{nd} noun\\ \cmidrule{2-9}
	tone of 1\textsuperscript{st}~noun & M & \#H & MH\# & H\$ & L & LM+\#H & LM & H\#\\\midrule
	M & M; \#H  & \#H & ? & ? & --L & ? & ? & ?\\
	\#H & H\#; \#H & H\#--; H\#/\#H-- & \#H & H\# & \#H-- / H\# & ? & ? & H\# / H\#--\\
	MH\# & MH\#; H\# & MH\#-- & ? & ? & MH\#-- & ? & ? & ?\\
	H\$ & H\#; \#H & \#H; \#H-- / H\# & \#H-- & \#H & ? & \#H-- & ? & ?\\
	L & L+H\# / L+\#H-- & L; L+\#H-- & ? & ? & L+\#H-- & ? & ? & ?\\
	L\# & ? & L\#-- & ? & ? & L\#-- & ? & ? & ?\\
	LM+MH\# & ? & ? & ? & ? & LM+MH\#-- & ? & ? & ?\\
	LM+\#H & ? & ? & ? & ? & ? & LM+MH\#-- & LM+MH\#-- & ?\\
	LM & LM-- & ? & ? & ? & ? & LM--L & ? & LM+H\#\\
	LH & ? & ? & ? & ? & ? & ? & ? & ?\\
	H\# & ? & ? & ? & ? & ? & ? & H\#-- & ?\\
\lspbottomrule
\end{tabularx}}}
\end{sidewaystable}
\end{subtables}


Sixty percent of the combinations are identical with those found on
determinative compounds. There is a~considerable proportion of combinations for which no example was found; to the cells containing a~{question} mark in the table must be added the empty columns, which are omitted
from the table. Only 35 combinations were observed, out of a~theoretically possible
223 (6×6 for σ+σ compounds, 11×6 for σσ+σ compounds, and 11×11 for σσ+σσ compounds). Since there is nothing to restrict input combinations, the gaps in Tables~\ref{tab:thetonepatternsofcoordinativecompoundscombinationswithmonosyllabicsecondnoun}
and
\ref{tab:thetonepatternsofcoordinativecompoundscombinationsamongdisyllables} must be considered accidental. This is due in part to the limitations of available materials, but the scarcity of examples, and the diversity of their tone patterns, also provides food for morphotonological thought: coordinative compounds are much less common than
determinative compounds. It may be that there is no such thing as a~full"=fledged set of productive \is{tone rules}tone combination rules governing their tone patterns, and that newly minted coordinative compounds are based on \isi{analogy} with the best example at hand: a~lexicalized compound perceived (on grounds that may fluctuate) as liable to the same tone rules. A~touch of expressivity may also be involved in the process: greater weight placed on one of the two elements in the compound, on \is{stylistics}semantic"=stylistic grounds, could contribute to the selection of one tone pattern rather than another. This would constitute a~clue to the observed diversity. 


\subsection{Compound given names and terms of address}
\label{sec:anindependentsetoffactscompoundgivennamesandtermsofaddress}

Compound given names and terms of address constitute two specific areas in the morphotonology of Yongning Na: they do not follow the regularities brought out above for other coordinative compounds.

In Yongning, given names are of \ili{Tibetan} origin. They consist of a~combination of two disyllabic names. For instance, /\ipa{ʝi˧tɕi˧-ɖɯ˩mɑ˩}/ is made up of /\ipa{ʝi˧tɕi˧}/ and /\ipa{ɖɯ˩mɑ\#˥}/; /\ipa{ɖɯ˩ɖʐɯ˧-tsʰɯ˩-ɻ̍˩}/ is a~combination of /\ipa{ɖɯ˩ɖʐɯ˧}/ and /\ipa{tsʰɯ˧ɻ̍\#˥}/. \tabref{tab:Names} provides a~list: disyllabic names, and attested combinations. The corresponding forms in Written \ili{Tibetan} were proposed by Nathan Hill and Tsering Samdrup (p.c.\ 2016), and remain to be confirmed by eliciting the written forms from a~monk in Yongning. Question marks indicate uncertain identifications. A~dash ‘--’ indicates that no forms are found in the set of recorded texts (and additional data will be necessary).

\begin{table}[p!!]
	\caption{Yongning Na given names and identifications with {Tibetan} names proposed by Nathan Hill and Tsering Samdrup.}
	{\renewcommand{\arraystretch}{1.13}
		\begin{tabularx}{\textwidth}{ P{17mm} P{35mm} Q }
		\lsptoprule
		name & \ili{Tibetan} & attested combinations\\ \midrule
		\ipa{ɖɯ˩ɖʐɯ˧} & Rdo rje & \ipa{ɖɯ˩ɖʐɯ˧-tsʰɯ˩ɻ̍˩}, \ipa{ɖɯ˩ɖʐɯ˧-ɬɑ˩-tsʰo˩}\\
		\ipa{ɖɯ˩mɑ\#˥} & Sgrol ma & \ipa{ɖɯ˩mɑ˧-ɬɑ˩tsʰo˩}, \ipa{ɖɯ˩mɑ˧-pv̩˩ʈʰɯ˩}\\
		\ipa{dʑɤ˩tsʰi\#˥} & Bde skyid? & \ipa{dʑɤ˩tsʰi˥-ɖɯ˩mɑ˩}, \ipa{dʑɤ˩tsʰi˥-pv̩˩ʈʰɯ˩}\\
		\ipa{gv̩˧mɑ˧} & ? & \ipa{gv̩˧mɑ˧-tsʰɯ˩ɻ̍˩}\\
		\ipa{ʝi˧ʂɯ˥} & Ye shes & \ipa{ʝi˧ʂɯ˥-ti˩ɖo˩}\\
		\ipa{ʝi˧tɕi˧} & Yid ches? & \ipa{ʝi˧tɕi˧-ɖɯ˩mɑ˩}, \ipa{ʝi˧tɕi˧-ɬɑ˩mv̩˩}\\
		\ipa{kɤ˧zo\#˥} & Skal bzang? & \ipa{kɤ˧zo˧-tsʰɯ˩ɻ̍˩}\\
		\ipa{ki˧zo\#˥} & ? & \ipa{ki˧zo˧-ɖɯ˩mɑ˩}, \ipa{ki˧zo˧-ɬɑ˩mv̩˩}\\
		\ipa{lɑ˩mɑ˩} & Bla ma & --\\
		\ipa{ɬɑ˧mv̩˥\$} & Lha mo & --\\
		\ipa{ɬɑ˧tsʰo\#˥} & Lha mtsho & --\\
		\ipa{nɑ˧dʑi\#˥} & Rnam rgyal & --\\
		\ipa{ɲi˩mɑ\#˥} & Nyi ma & --\\
		\ipa{no˩bv̩˧} & Nor bu & \ipa{no˩bv̩˧-tsʰɯ˩ɻ̍˩}\\
		\ipa{no˧no˧} & ? & \ipa{no˧no˧-ɖɯ˩mɑ˩}\\
		\ipa{pæ˩pʰæ˧˥} & Spen pa? & --\\
		\ipa{pi˧mɑ˧} & Padma & \ipa{pi˧mɑ˧-ɬɑ˩mv̩˩}, \ipa{pi˧mɑ˧-ɬɑ˩tsʰo˩}\\
		\ipa{pʰi˧tsʰo\#˥} & Phun tshogs & \ipa{pʰi˧tsʰo˧-ɖɯ˩ɖʐɯ˩}\\
		\ipa{pv̩˩ʈʰɯ˧} & Bu phrug? Bu khrid?  & --\\
		\ipa{ɻ̍˩ʈʂʰe\#˥} & Rin chen & \ipa{ɻ̍˩ʈʂʰe˧-ɖɯ˩mɑ˩}, \ipa{ɻ̍˩ʈʂʰe˧-tsʰɯ˩ɻ̍˩}\\
		\ipa{tɑ˩dʑɤ\#˥} & Dar rgyes? & --\\
		\ipa{ʈæ˧ʂɯ˧} & Bkra shis & \ipa{ʈæ˧ʂɯ˧-ɖɯ˩mɑ˩}, \ipa{ʈæ˧ʂɯ˧-ɬɑ˩mv̩˩}, \ipa{ʈæ˧ʂɯ˧-pæ˩pʰæ˩}, \ipa{ʈæ˧ʂɯ˧-ʈæ˩ʈv̩˩}, \ipa{ʈæ˧ʂɯ˧-tsʰi˩ti˩}\\
		\ipa{ʈæ˩ʈv̩\#˥} & Dgra 'dul? & --\\
		\ipa{tɕʰi˧ɖv̩\#˥} & Spyi 'dul? & --\\
		\ipa{ti˧ɖo˥} & ? & --\\
		\ipa{tsʰi˧ti\#˥} & ? & --\\
		\ipa{tsʰɯ˧ɻ̍\#˥} & Tshe ring & \ipa{tsʰɯ˧ɻ̍˧-lɑ˩mv̩˩}, \ipa{tsʰɯ˧ɻ̍˧-ɖɯ˩mɑ˩}, \ipa{tsʰɯ˧ɻ̍˧-ɬɑ˩mv̩˩}, \ipa{tsʰɯ˧ɻ̍˧-pʰi˩tsʰo˩}\\
		\lspbottomrule
	\end{tabularx}}
	\label{tab:Names}
\end{table}

% The full name of F4’s grandmother was /\ipa{ɖɯ˩mɑ˧-pv̩˩ʈʰɯ˩}/ (compounded from the LM"=tone names /\ipa{ɖɯ˩mɑ˧}/ and /\ipa{pv̩˩ʈʰɯ˧}/). Her siblings addressed her as /\ipa{pv̩˩ʈʰɯ˧}/, using the second part of the given name (as it is pronounced on its own, and not as it appears in the compound, where both of its syllables carry L tone), and her nephews and nieces, as well as other persons from the village, addressed her respectfully as /\ipa{ə˧mi˧-pv̩˩ʈʰɯ˩}/, i.e.\ using the term of address for aunts. The lowering of the last syllable in /\ipa{ə˧mi˧-pv̩˩ʈʰɯ˩}/ is due to the general prohibition  prohibits tonal troughs such as $\ddagger${\kern2pt}M.M.L.M within a~tone group.

The tones of the second name are lowered to L in all cases, even when they could in theory be expressed without contravening any {phonological rule}. For instance, /\ipa{ɖɯ˩ɖʐɯ˧}/ and /\ipa{tsʰɯ˧ɻ̍\#˥}/ could combine as $\ddagger${\kern2pt}\ipa{ɖɯ˩ɖʐɯ˧-tsʰɯ˧ɻ̍\#˥}, by successive association of the tones of the two nouns. The expression $\ddagger${\kern2pt}\ipa{ɖɯ˩ɖʐɯ˧-tsʰɯ˧ɻ̍\#˥} would not violate phonological conditions of well"=formedness: there exist some four-syllable expressions that carry just this tone pattern, such as /\ipa{nɑ˩bɑ˧-ʁɑ˧ɭɯ\#˥}/ (the name of a~mountain). The lowering to L of the second part of compound names only applies to given names, not to compound names made up of a~term of address and a~two"=syllable given name. For instance, a~woman named /\ipa{ki˧zo\#˥}/ may be addressed as /\ipa{ə˧mi˧-ki˧zo\#˥}/ (“Mother \ipa{ki˧zo\#˥}, Aunt \ipa{ki˧zo\#˥}”) by her nephews and nieces. This term of address is not realized as $\ddagger${\kern2pt}\ipa{ə˧mi˧-ki˩zo˩}, as would be the case if it were treated in the same way as compound given names. So the lowering of the last two syllables in given names, such as /\ipa{ɖɯ˩ɖʐɯ˧-tsʰɯ˩-ɻ̍˩}/ and all the other examples in \tabref{tab:Names}, is puzzling: it cannot be put down to the application of tone rules that apply throughout the Yongning Na tone system. Nor can it be explained as the result of a~process of \isi{lexicalization} that would have taken place at an earlier stage in the language's history: the lowering process clearly has the status of an exceptionless synchronic rule, applying to all compound given names. Speakers retain a~clear awareness of the two parts of the compound names as distinct components, each of which has a~lexical tone of its own. One of the two parts of the given name serves as the usual term of address; it may be the first or (less commonly) the second. For instance, the 
%full given name of M18 is /\ipa{ʈæ˧ʂɯ˧-tsʰɯ˩ɻ̍˩}/, but people address him as /\ipa{ʈæ˧ʂɯ˧}/. The 
name that was bestowed on me by a~priest of the Yongning monastery is /\ipa{ʝi˧ʂɯ˥-ti˩ɖo˩}/; F4 chose to call me by the second part of that compound given name: /\ipa{ti˧ɖo˥}/, which she pronounced with its own lexical tone, not as a~stump retrieved from the compound (in which case she would have said it with L tone).\footnote{F4's son (M18) argued that the shortened name should be /\ipa{ʝi˧ʂɯ˥}/, but F4 maintained her initial choice, providing no further explanation than that she preferred /\ipa{ti˧ɖo˥}/. The choice depends in part on considerations of homonymy within the extended family: there are so few names that homonymy is a~real issue, and concerns about inappropriate use of proper names runs deep in Asian cultures. This is evidenced by the taboo on using the names of important characters (emperors, but also one's elders) in China and neighbouring countries, a~practice known as \textit{bìhuì} \zh{避讳} \citep[studied in detail by][]{adamek2012}. In Yongning Na, the concern is not limited to elders, as illustrated by the following anecdote. In the course of telling the story BuriedAlive2, consultant F4 realized that the name of the woman protagonist, \ipa{/ɻ̍˩ʈʂʰe˧-ɖɯ˩mɑ˩}/, happened to be that of a~member of her household. The consultant immediately substituted another name, /\ipa{no˩bv̩˧-tsʰɯ˩ɻ̍˩}/, exempt from association with family members. The character in the story has a~less than glorious role; the coincidence of names threatened to build an unwanted association between the family member and the character's shameful behaviour. The concern was such that F4 would have preferred me to delete the recording of this version of the story altogether, so as to leave no trace of this unfortunate coincidence of names. But we had already spent much time working on the transcription, so I~was not overjoyed at the prospect of outright deletion of the files. Fortunately, the consultant found comfort in the explicit disclaimer which is present on the recording, to the effect that the character's name was /\ipa{no˩bv̩˧-tsʰɯ˩ɻ̍˩}/ and \textit{not} /\ipa{ɻ̍˩ʈʂʰe˧-ɖɯ˩mɑ˩}/, and she agreed that we should complete the transcription and allow access to this linguistic document.} This is one of many pieces of evidence demonstrating that compounds such as /\ipa{ʝi˧ʂɯ˥-ti˩ɖo˩}/ are readily decomposable. 

Should the tonal lowering of the latter part of compoun given names be considered as another instance of a~tone rule applying in a~highly specific morphosyntactic context: a~tone combination rule that holds in given names, and nowhere else~-- not even in other sets of proper names, such as place names?

Interestingly, a~similar process of lowering is found in compounds involving the word for ‘powder, flour’: /\ipa{tsɑ˧bɤ˧}/ (as reported in \sectref{sec:thenounflourpowder}). According to the synchronically
productive rules, the combination of this word with /\ipa{lv̩˧mi˧}/ ‘stone’, /\ipa{qʰɑ˧dze˧}/
‘sweetcorn’ and /\ipa{dze˧ɭɯ˧}/ ‘wheat’ should yield a~simple M-tone output,
e.g.~$\dagger$\ipa{lv̩˧mi˧-tsɑ˧bɤ˧} for ‘fine sand’. But the observed forms, shown in \tabref{tab:powder}, all carry a~M.M.L.L tone pattern, corresponding to underlying --L (a~L tone on the second part of the compound), just like the compound given names in \tabref{tab:Names}. The label ‘\ili{Tibetan} compounds’ is proposed for these compounds. From a~synchronic point of view, this label may seem invalid, since none of the consultants has any knowledge of \ili{Tibetan} (either spoken or written) and hence any awareness of \ili{Tibetan} loanwords as such. The hypothesis here is that at some earlier point in time~-- one to four centuries ago?~-- speakers of Yongning Na who had some knowledge of \ili{Tibetan} (a~smattering would have been enough) applied a~specific tonal treatment to compounds containing \ili{Tibetan} words, by imitation of what they perceived as the \ili{Tibetan} pattern. \ili{Tibetan} was a~prestige language in Yongning at least since the fourteenth century, and up until the mid"=twentieth century (see Appendix B, \sectref{sec:historicaloutline}), so it is not unlikely that processes of imitation of prosodic patterns took place. This may have happened at the time when the \ili{Tibetan} words were borrowed, or at a~later stage, when speakers of Yongning Na who were aware of the status of these words as \ili{Tibetan} in origin made efforts to copy (what they felt to be) \ili{Tibetan} prosodic patterns. Cases of contact with a~prestigious language can result in a~range of unusual changes by imitation, including hypercorrections: see \citet[99-103]{meillet1936} on a~possible case concerning {Germanic} and {Romance}, and \citet{ferlus2001eng} on cases in the {Vietic} subgroup of Austroasiatic; a~highly speculative application to {Tibetan} has also been proposed \citep{ferlus2003a}. 

Supposing that imitation of \ili{Tibetan} was at play in these compounds, it remains to be explained why lowering tone to L on the latter part of compounds had a~\ili{Tibetan} ring to Na ears, and at which point in history the adoption of this prosodic pattern took place. For want of a~command of \ili{Tibetan}, I~am not in a~position to investigate topics of historical contact with this language~-- one of many topics that remain for future investigations.



\section{Compound nouns containing adjectives}
\label{sec:compoundnounscontainingadjectives}

The tonal categories of \is{adjectives}adjectives will be brought out in \sectref{sec:adjectivesasdistinctfromverbs}. As a~background to the discussion in the present section, here is a~preview of the results: the four tonal categories of adjectives are L, M, H, and MH, with a~subdivision among L-tone items, distinguishing L\textsubscript{a} and L\textsubscript{b}.

{\largerpage} % To avoid 1 stranded line at end of page + section

Compounds containing adjectives are a~difficult topic in Yongning Na, not least because they are the product of \isi{lexicalization}, and cannot be elicited systematically by asking consultants to coin compounds on the fly to bring out a~full set of synchronic rules, as can be done for \textit{noun plus noun} compounds. As a~preliminary to the study of lexicalized compounds, it is useful to present adjectival phrases.

\subsection{A productive construction: \textsc{N}+\textsc{Adj}+{relativizer}}
\label{sec:productiveconstruction}

%
%Liberty Lidz notes that “the constituent order for Na adjectival phrases is
%\textsc{N}+\textsc{Adj}, which is consistent with Na’s OV constituent order” \citep[215]{lidz2010}, as in (\ref{ex:averybigfish}).
%
%\begin{exe}
%  \ex
%  \label{ex:averybigfish}
%  \glll {ni³³ zɔ³³} dɯ⁵⁵ ʐwæ¹³ dɯ³³ mi³¹\\
%  fish big \textsc{ints} one \textsc{cls}\\
%  \zh{鱼} \zh{大} \zh{很} \zh{一} \zh{量词}\\
%  \glt ‘a very big fish’ (example~187 from \citealt[215]{lidz2010})
%\end{exe}

In Yongning Na, the association of \is{adjectives}adjectives to nouns is realized by
the construction \textsc{N}+\textsc{Adj}+{relativizer}/{nominalizer}
\mbox{/\ipa{-hĩ˥}/}. For instance, //\ipa{ɖɯ˩\textsubscript{a}}// ‘big’\footnote{As explained on the first page of the introduction, morpheme"=level transcriptions indicate lexical tone by means of tone symbols supplemented by
	subscript letters \textsubscript{a} \textsubscript{b} \textsubscript{c} to distinguish subcategories of lexical tones. The {subcategories} for verbs and adjectives are set out in \tabref{tab:Utonesofverbs} of \sectref{sec:overview}. The pound symbol \# is also part of the apparatus to transcribe the different categories of lexical tones, as was explained in \sectref{sec:afloatinghtonewithcomparativeevidencepointingtoitsorigin}.} yields //\ipa{ɖɯ˩-hĩ˩}// ‘(which is) big’, and //\ipa{ʂɯ˧˥}// ‘new’ yields //\ipa{ʂɯ˧-hĩ˥\$}// ‘(which is) new’. These are added after the noun as
a~separate \isi{tone group}, as in (\ref{ex:bigfish})-(\ref{ex:newbowl}).

\begin{exe}
	\ex
	\label{ex:bigfish}
	\ipaex{ɲi˧zo˧ {\kern2pt}|{\kern2pt} ɖɯ˩-hĩ˩˥}\\
	\gll ɲi˧zo˧	ɖɯ˩\textsubscript{a}	-hĩ˥\\
	fish	large		\textsc{nmlz}\\
	\glt ‘big fish’
\end{exe}

\begin{exe}
	\ex
	\label{ex:newcloth}
	\ipaex{pʰi˧ {\kern2pt}|{\kern2pt} ʂɯ˧-hĩ˥}\\
	\gll pʰi˧	ʂɯ˧˥		-hĩ˥\\
	linen\_cloth		new		\textsc{nmlz}\\
	\glt ‘brand new linen cloth’
\end{exe}

\begin{exe}
	\ex
	\label{ex:newladle}
	\ipaex{tɕʰo˩˥ {\kern2pt}|{\kern2pt} ʂɯ˧-hĩ˥}\\
	\gll tɕʰo˩˧		ʂɯ˧˥		-hĩ˥\\
	ladle		new		\textsc{nmlz}\\
	\glt ‘new ladle’
\end{exe}

\begin{exe}
	\ex
	\label{ex:newbowl}
	\ipaex{qʰwɤ˧˥ {\kern2pt}|{\kern2pt} ʂɯ˧-hĩ˥}\\
	\gll qʰwɤ˧˥		ʂɯ˧˥		-hĩ˥\\
	bowl		new		\textsc{nmlz}\\
	\glt ‘new bowl’
\end{exe}


No tonal interaction takes place between the noun and \is{adjectives}adjective. If an~\is{intensifiers}intensifier is
substituted for the relativizer, the construction becomes a~statement, as in (\ref{ex:fishisbig}). The construction (\ref{ex:fishisbig2}) likewise means ‘the fish is big’.

\begin{exe}
	\ex
	\label{ex:fishisbig}
	\ipaex{ɲi˧zo˧ {\kern2pt}|{\kern2pt} ɖɯ˧ {\kern2pt}|{\kern2pt} ʐwæ˩˥.}\\
	\gll ɲi˧zo˧		ɖɯ˩\textsubscript{a}		ʐwæ˩\\
	fish		large		\textsc{ints}\\
	\glt ‘The fish is very big.’
\end{exe}

\begin{exe}
	\ex
	\label{ex:fishisbig2}
	\ipaex{ɲi˧zo˧ {\kern2pt}|{\kern2pt} ɖɯ˧.}\\
	\gll ɲi˧zo˧		ɖɯ˩\textsubscript{a}\\
	fish		large\\
	\glt ‘The fish is big.’
\end{exe}


In example (\ref{ex:averybigfish}), the
{numeral}"=plus"=classifier phrase has the effect of nominalizing a~construction that would otherwise
mean ‘the fish is/was really big’, rather than ‘a very big fish’.

\begin{exe}
  \ex
  \label{ex:averybigfish}
  \glll {ni³³ zɔ³³} dɯ⁵⁵ ʐwæ¹³ dɯ³³ mi³¹\\
  fish big \textsc{ints} one \textsc{cls}\\
  \zh{鱼} \zh{大} \zh{很} \zh{一} \zh{量词}\\
  \glt ‘a very big fish’ (example~187 from \citealt[215]{lidz2010}; her glosses and tone marking. \textsc{cls}: classifier; \ipa{⁵⁵}: High tone; \ipa{³³}: Mid tone; \ipa{³¹}: Low"=falling tone; \ipa{¹³}: Low"=rising tone.)
\end{exe}

\largerpage[-2] %longdistance 
The \isi{word order} Noun+Adjective in a~noun phrase always signals a~lexicalized item: to use a~textbook example from {English}, a~phrasal, nonlexicalized ‘black bird' as distinct from ‘blackbird' requires a~{relativizer} (examples are provided a~few lines below). For instance, //\ipa{ʐɯ˧nɑ˩}//, from //\ipa{ʐɯ˧}//
‘liquor/spirits’ and //\ipa{nɑ˩\textsubscript{b}}// ‘black’, does not mean ‘black liquor, liquor of a~black colour’, but
refers to a~specific type of strong, high"=quality spirits. This is a~disyllabic noun, requiring an entry of its own in the dictionary; it is not a~phrasal construction (a~description associating
a~quality to the entity that the noun refers to). The nouns //\ipa{ə˧mi˧-ɖɯ˩}// and
//\ipa{ə˧mi˧-tɕi˩}//, referring to the mother’s older sisters and younger sisters respectively, are
lexical units, even though they can still be transparently analyzed as made up of //\ipa{ə˧mi˧}//
‘mother’ plus the adjectives //\ipa{ɖɯ˩\textsubscript{a}}// ‘large’ and //\ipa{tɕi˩\textsubscript{a}}// ‘small’. 
%The conceptual
%difference among aunts (mother’s older sisters and younger sisters) is clear in Na culture, witness
%the existence of distinct terms of address: /\ipa{ə˧jɤ˩}/ for ‘mother’s older sister’ and
%/\ipa{ə˧tɕi˩}/ for ‘mother’s younger sister.

There also exist compounds with the adjectives ‘big’ and ‘small’ for maternal uncles, //\ipa{ə˧v̩˧˥}//: //\ipa{ə˧v̩˧-tɕi˥}// for ‘mother's younger brother’, and //\ipa{ə˧v̩˧-ɖɯ˧˥}// for ‘mother's elder brother’. However, constructions with the
{relativizer} \mbox{//\ipa{-hĩ˥}//} are more common: to clarify
whether one is referring to the mother’s elder brother or younger brother, the former is called /\ipa{ə˧v̩˧˥ {\kern2pt}|{\kern2pt} ɖɯ˩-hĩ˩˥}/, and the latter /\ipa{ə˧v̩˧˥
  {\kern2pt}|{\kern2pt} tɕi˩-hĩ˩˥}/, as shown in (\ref{ex:unc1}--\ref{ex:unc2}). 

\begin{exe}
	\ex 
	\begin{xlist}
		\ex
		\label{ex:unc1}
		\ipaex{ə˧v̩˧˥ {\kern2pt}|{\kern2pt} ɖɯ˩-hĩ˩˥}\\
		\gll 	ə˧v̩˧˥		ɖɯ˩\textsubscript{a}	-hĩ˥\\
		uncle		big		\textsc{nmlz}\\
		\glt ‘mother's elder brother, elder maternal uncle'
		
		\ex
		\label{ex:unc2}
		\ipaex{ə˧v̩˧˥ {\kern2pt}|{\kern2pt} tɕi˩-hĩ˩˥}\\
		\gll 	ə˧v̩˧˥		tɕi˩\textsubscript{a}	-hĩ˥\\
		uncle		small		\textsc{nmlz}\\
		\glt ‘mother's younger brother, younger maternal uncle' (Caravans.75, 76, 78, 79, 177--179, 196, 259, Elders3.23, 31, 32)
	\end{xlist}
\end{exe}



The study of successive occurrences within the same text confirms that the construction with the
{relativizer}/{nominalizer} \mbox{//\ipa{-hĩ˥}//}, although it may seem cumbersome, is the
standard construction to associate an~adjective to a~noun. This construction is not followed by
a~synthetic, compact \textsc{N}+\textsc{Adj} construction at later occurrences. For instance, example (\ref{ex:bigchildB}) follows example (\ref{ex:bigchildA}) at a~distance of a~few sentences inside the same narrative, and both contain the same construction with nominalizer.

\begin{exe}
	\ex
	\label{ex:bigchildA}
	\ipaex{mv̩˩zo˩˥ {\kern2pt}|{\kern2pt} ɖɯ˩-hĩ˩˥, {\kern2pt}|{\kern2pt} zo˧mv̩˥ {\kern2pt}|{\kern2pt} ɖɯ˩-hĩ˩˥ {\kern2pt}|{\kern2pt} ɖɯ˧-ɭɯ˧ dʑo˩ tsɯ˩.}\\
	\gll mv̩˩zo˩	ɖɯ˩\textsubscript{a}	-hĩ˥	zo˧mv̩˥		ɖɯ˩\textsubscript{a}	-hĩ˥	ɖɯ˧-ɭɯ˧		dʑo˩\textsubscript{b}	tsɯ˧˥\\
	young\_lady		large	\textsc{nmlz}	child	large	\textsc{nmlz}	one-\textsc{clf}	\textsc{exist}	\textsc{rep}\\
	\glt ‘It is said that [this couple] had a~big girl, a~big child (=a child who thought
	very seriously for her age).’ (Reward.59)
\end{exe}

\begin{exe}
	\ex
	\label{ex:bigchildB}
	\ipaex{mv̩˩zo˩˥ {\kern2pt}|{\kern2pt} ɖɯ˩-hĩ˩-ki˥ ({\dots})}\\
	\gll mv̩˩zo˩	ɖɯ˩\textsubscript{a}	-hĩ˥	 -ki˧\\
	young\_lady		large		\textsc{nmlz}	\textsc{dat}\\
	\glt ‘to his elder daughter, [the father
	said{\dots}]’ (Reward.65)
\end{exe}

This is a~notable difference from \ili{Naxi}. For instance, ‘important person, great personage; adult’ in
\ili{Naxi} is /\ipa{hi˧-ɖɯ˩}/ ‘person’+‘big’. 
%This appears in the text Weresow: /\ipa{hi˧-ɖɯ˩ wɑ˩ ji˥, ɖɯ˧-mə˞˩ ʈʂʰu˩-be˧ kæ˧ le˧˥-tsʰɯ˩}/ ‘being an~important person, [the to"=mba priest] left [the
%  place where he had conducted a~ritual] rather early’. 
In Yongning Na, the same concept is:
/\ipa{ɖɯ˩-hĩ˩˥}/, ‘big’ plus relativizer, with ‘person’ as the implicit referent (e.g.~Sister.13, 14,
34, Sister3.31, 36, 38, 41, and BuriedAlive3.5).

\subsection{Lexicalized compounds of  \textsc{N}+\textsc{Adj} structure}
\label{sec:lexicalizedcompoundsofnadjstructure}

The adjectives that appear in lexicalized combinations with nouns in the examples provided above are
//\ipa{nɑ˩\textsubscript{b}}// ‘black, dark’, //\ipa{ɖɯ˩\textsubscript{a}}// ‘large’ and //\ipa{tɕi˩\textsubscript{a}}// ‘small’. Is it a~coincidence that
‘black’ is also the adjective used in the textbook example of \ili{English} \textit{black bird} and
\textit{blackbird}? The compound noun \textit{blackbird} refers to \textit{Turdus merula}, a~species
of thrush, while the combination of noun and adjective \textit{black bird} refers to any bird of
a~black colour. The former, \textit{blackbird}, carries stress on the first element of the compound
(for short: “first"=element stress”), whereas the latter, \textit{black bird}, carries last"=element
stress: primary stress on \textit{bird}. The meaning of \textit{black bird} can be deduced from the
meaning of its elements and the meaning of the construction, whereas the meaning of
\textit{blackbird} cannot be arrived at on the basis of the meaning of the elements. First"=element stress is
generally interpreted as a~marker of degree of {lexicalization}. It has been observed, in a~study of
\ili{English}, that “the number of adjectives that work in the way that \textit{black} does in our
\textit{exemple}-\textit{type} seems to be very restricted” \citep[9]{bauer2004}. Examples are
shown in \tabref{tab:typesofadjectivesthatappearincompoundnounsinenglish}. But at the end of a~quest using corpus"=query tools to explore hypotheses about the relevance of factors such as the frequency of the particular collocations and contrasting patterns of premodification, the author concludes that the gaps in \ili{English} adjective"=plus"=noun compounds are likely to be accidental. In Na, as in \ili{English}, adjectives that appear in compound nouns do not constitute a~closed set.


\begin{table}%[t]
\caption{Types of adjectives that appear in compound nouns in {English} \citep[from][9]{bauer2004}.}
{\renewcommand{\arraystretch}{1.35}
\begin{tabularx}{\textwidth}{ Q Q Q }
\lsptoprule
	type of adjectives & examples & example compounds\\\midrule
	some colour adjectives & \textit{black}, \textit{blue}, \textit{brown}, \textit{green}, \textit{grey}, \textit{red}, \textit{white} & \textit{blackboard}, \textit{blue"=tit},
   \textit{brownstone}, \textit{greenfly}, \textit{greyhound}, \textit{redfish}, \textit{whiteboard}\\
   \textit{grand} in words of family relationships & \textit{grand} & \textit{grandfather}\\
   a~miscellaneous set of {monosyllabic} gradable adjectives & \textit{broad}, \textit{dry}, \textit{free}, \textit{hard}, \textit{hot}, \textit{mad},
   \textit{small}, \textit{sweet} (among others) & \textit{broadcloth}, \textit{dry"=cell}, \textit{freepost}, \textit{hardboard}, \textit{hotbed}, \textit{madman},
   \textit{small"=arm}, \textit{sweetcorn}\\
   a~small set of non"=gradable {monosyllabic} adjectives & \textit{blind}, \textit{dumb}, \textit{first}, \textit{quick} (= ‘alive’),
   \textit{square}, \textit{whole} & \textit{blindside}, \textit{dumbcluck}, \textit{first"=day}, \textit{quicksand}, \textit{squaresail}, \textit{wholestitch}\\ a~very
   small number of disyllabic adjectives & \textit{bitter}, \textit{narrow}, \textit{silly} & \textit{bitter"=cress}, \textit{narrow"=boat},
   \textit{sillyseason}\\
\lspbottomrule
\end{tabularx}}
\label{tab:typesofadjectivesthatappearincompoundnounsinenglish}
\end{table}

\largerpage[-3]
In {English}, there is {variation} across speakers (and even for one and the same speaker) in judgments about stress patterns, and in stress assignment in actual speech. In Yongning Na, on the other hand, noun"=plus"=adjective compounds are conspicuously different from attributive constructions, since the latter comprise a~{relativizer}. This makes it easy to identify these compounds in Yongning Na. Their tonal analysis, however, is not straightforward. Examples are shown in tabular form, arranged by the tone of the adjective: L\textsubscript{a} in \tabref{tab:adjective-plus-nouncompoundtoneLa}, L\textsubscript{b} in \tabref{tab:black}, M in \tabref{tab:adjective-plus-nouncompoundtoneM}, and H in \tabref{tab:adjective-plus-nouncompoundtoneH}.\footnote{The	subscript letters \textsubscript{a} \textsubscript{b} \textsubscript{c} added to the tones serve to distinguish subcategories of lexical tones. The {subcategories} for verbs and adjectives are set out in \tabref{tab:Utonesofverbs} (\sectref{sec:overview}).} (No compounds with MH-tone adjectives have yet been observed.) All these items are lexicalized: for instance, the phrase /\ipa{ʈʂʰæ˧nɑ˥}/ refers to a~legendary stag, which only spirits are able to hunt down; it is thus different from an~attributive
construction (‘black"=coloured deer’).

\begin{table}%[t]
	\caption{Examples of compounds containing the L\textsubscript{a}"=tone adjectives /\ipa{mo˩\textsubscript{a}}/ ‘old’, /\ipa{ɖɯ˩\textsubscript{a}}/ ‘large’, /\ipa{tɕi˩\textsubscript{a}}/ ‘small’, and /\ipa{pʰv̩˩\textsubscript{a}}/ ‘white’. Note that no {monosyllabic} form is attested synchronically for ‘stone’ and ‘ard’.}
	\begin{tabularx}{\textwidth}{ l l P{27mm} l l Q }
		\lsptoprule
		\multicolumn{3}{l}{head noun} & \multicolumn{3}{l}{compound}\\
		\cmidrule(r){1-3} \cmidrule(l){4-6}
		form & tone & meaning & form & tone & meaning\\\midrule
		\ipa{hĩ˥} & H & person & \ipa{hĩ˧mo˥} & H\# & elderly person\\
		\ipa{ʐwæ˥} & H & horse & \ipa{ʐwæ˧mo˥} & H\# & old horse\\
		\ipa{si˥} & H & wood & \ipa{si˧mo˥} & H\# & old wood, old tree\\
		\ipa{lv̩˧mi˧} & ? & stone & \ipa{lv̩˧mo˥} & H\# & old stones\\
		\ipa{tsʰo˩} & L & human being & \ipa{tsʰo˩mo˩} & L & old man\\
		\ipa{æ˩gv̩˩} & ? & ard\footnote{The ard, also known as scratch plough, is the type of ploughing implement used in Yongning. Unlike the plough, the ard has a~symmetrical share that traces a shallow furrow but does not invert the soil \citep{haudricourtetal1955}.} & \ipa{æ˩mo˥} & LH & used ard (out of use)\\ 
		\addlinespace \hdashline \addlinespace
		\ipa{ʁo˥} & H & head & \ipa{ʁo˧ɖɯ˧˥} & MH\# & tadpole\\
		\ipa{zo˥} & H & son & \ipa{zo˧ɖɯ˧} & M & eldest son\\
		\ipa{mv̩˩˥} & LH & daughter & \ipa{mv̩˩ɖɯ˩} & L\# & eldest daughter\\
		\ipa{ə˧mi˧} & M & mother & \ipa{ə˧mi˧-ɖɯ˩} & L\# & mother's elder sister\\
		\ipa{ə˧v̩˧˥} & MH\# & maternal uncle & \ipa{ə˧v̩˧-ɖɯ˧˥} & MH\# & mother's elder brother\\
		\ipa{ə˧bo˥\$} & H\$ & paternal uncle & \ipa{ə˧bo˧-ɖɯ˧˥} & MH\# & father's elder brother\\
		\addlinespace \hdashline \addlinespace
		\ipa{mv̩˩˥} & LH & daughter & \ipa{mv̩˩tɕi˥} & LH & youngest daughter\\
		\ipa{zo˥} & H & son & \ipa{zo˧tɕi˥} & H\# & youngest son\\
		\ipa{ə˧mi˧} & M & mother & \ipa{ə˧mi˧-tɕi˩} & L\# & mother's younger sister\\
		\ipa{ə˧v̩˧˥} & MH\# & maternal uncle & \ipa{ə˧v̩˧-tɕi˥} & H\# & mother's younger brother\\
		\ipa{ə˧bo˥\$} & H\$ & paternal uncle & \ipa{ə˧bo˧-tɕi˥} & H\# & father's younger brother\\
		\addlinespace \hdashline \addlinespace
		\ipa{tɕɯ˧} & M & cloud &  \ipa{tɕɯ˧pʰv̩˩} & L\# & white cloud\\
		\lspbottomrule
	\end{tabularx}
	\label{tab:adjective-plus-nouncompoundtoneLa}
\end{table}


\begin{table}%[t]
\caption{Examples of compounds containing the L\textsubscript{b}"=tone adjective /\ipa{nɑ˩\textsubscript{b}}/ ‘black’.}
\begin{tabularx}{\textwidth}{ l l P{27mm} l l Q }
\lsptoprule
	\multicolumn{3}{l}{head noun} & \multicolumn{3}{l}{compound}\\
   \cmidrule(r){1-3} \cmidrule(l){4-6}
	form & tone & meaning & form & tone & meaning\\\midrule
	\ipa{hṽ̩˥} & H & hair & \ipa{hṽ̩˧nɑ˩} & L\# & wild animal\\
	\ipa{ʂe˥} & H & meat & \ipa{ʂe˧nɑ˩} & L\# & lean meat\\
	\ipa{si˥} & H & wood & \ipa{si˧nɑ˥} & H\# & deep forest\\
	\ipa{kʰv̩˥} & H & dog & \ipa{kʰv̩˧nɑ˥} & H\# & dog \textit{(in formal speech)}\\
	\ipa{tɕʰi˥} & H & thorn & \ipa{tɕʰi˧nɑ˥} & H\# & prinsepia\\
	\ipa{ʐɯ˧} & M & liquor/spirits & \ipa{ʐɯ˧nɑ˩} & L\# & high"=quality spirits\\
	\ipa{njɤ˩˥} & LH & eye & \ipa{njɤ˧nɑ˩} & L\# & eyeball\\
	\ipa{ʈʂʰæ˧˥} & MH & deer & \ipa{ʈʂʰæ˧nɑ˥} & H\# & legendary black stag\\
\lspbottomrule
\end{tabularx}
\label{tab:black}
\end{table}

\begin{table}%[t]
	\caption{Examples of compounds containing the M"=tone adjectives /\ipa{pv̩˧}/ ‘dry’, /\ipa{bæ˧}/ ‘stupid’, /\ipa{tʰi˧}/ ‘clever’, /\ipa{tsʰi˧}/ ‘hot’ and /\ipa{ʂæ˧}/ ‘long’.}
	\begin{tabularx}{\textwidth}{ l l P{27mm} l l Q }
		\lsptoprule
		\multicolumn{3}{l}{head noun} & \multicolumn{3}{l}{compound}\\
		\cmidrule(r){1-3} \cmidrule(l){4-6}
		form & tone & meaning & form & tone & meaning\\\midrule
	\ipa{hɑ˥} & H & food & \ipa{hɑ˧pv̩˩} & L\# & dry cooked rice (as opposed to gruel)\\
		\addlinespace \hdashline \addlinespace
	\ipa{zo˥} & H & son & \ipa{zo˧bæ˩} & L\# & idiot\\
		\addlinespace \hdashline \addlinespace
	\ipa{mv̩˩˥} & LH & daughter & \ipa{mv̩˩tʰi˩} & L & clever woman\\
		\addlinespace \hdashline \addlinespace
	\ipa{dʑɯ˩} & L & water & \ipa{dʑɯ˩tsʰi˩} & L & hot water\\
		\addlinespace \hdashline \addlinespace
	\ipa{zɯ˧} & M & life & \ipa{zɯ˧ʂæ˧} & M & long life\\
		\lspbottomrule
	\end{tabularx}
	\label{tab:adjective-plus-nouncompoundtoneM}
\end{table}

\begin{table}
	\caption{Examples of compounds containing the H"=tone adjectives /\ipa{qʰæ˥}/ ‘cold’ and /\ipa{ɖæ˥}/ ‘short’.}
	\begin{tabularx}{\textwidth}{ l l P{27mm} l l Q }
		\lsptoprule
		\multicolumn{3}{l}{head noun} & \multicolumn{3}{l}{compound}\\
		\cmidrule(r){1-3} \cmidrule(l){4-6}
		form & tone & meaning & form & tone & meaning\\\midrule
	\ipa{dʑɯ˩} & L & water &  \ipa{dʑɯ˩qʰæ˩} & L & cold water\\
		\addlinespace \hdashline \addlinespace
	\ipa{zɯ˧} & M & life & \ipa{zɯ˧ɖæ\#˥} & \#H & short life\\
		\lspbottomrule
	\end{tabularx}
	\label{tab:adjective-plus-nouncompoundtoneH}
\end{table}

\clearpage 
In terms of tone, the compounds exhibit some diversity. Of the five compounds that relate to {monosyllabic} roots with H tone, three have
L\# tone, and two have H\# tone. This is not related to any obvious structural property of the
compounds: the compound /\ipa{hṽ̩˧nɑ˩}/, literally ‘black hair’, does not refer to a~type of hair,
but to ‘wild animal’, referring by synecdoche to the \textit{possessor} of dark
hair;\footnote{Interestingly, the association of darker fur with wildness (less disposition to
  domestication) is confirmed by scientific studies of animal domestication \citep{trut1999}. The
  phenomenon is apparently due (at least in part) to links between levels of stress in an~individual
  and amount of melanine, itself reflected in darker fur \citep{burchilletal1986}: individuals in
  the wild experience greater stress. Wild yaks have
  darker hair than domestic yaks \citep{leslieetal2009}.} on the other hand, the compound /\ipa{ʂe˧nɑ˩}/,
literally ‘dark meat’, refers to a~sort of meat (lean meat). (In the Indian linguistic tradition, /\ipa{hṽ̩˧nɑ˩}/ ‘black hair’ for ‘wild animal’ would be referred to as a~\textit{bahuvrīhi} compound: a~compound that denotes a~referent by a~certain characteristic.) Semantically, there is no salient
difference either. In the L\#-tone compounds for ‘dark hair’ and ‘lean meat’ the \is{adjectives}adjective
/\ipa{nɑ˩\textsubscript{b}}/ can be argued to have a~literal interpretation as ‘black, dark’: traditionally, pigs
and cattle were only slaughtered once a~year, so that fresh meat was the {exception}; the norm for
lean meat was the preserved sort, with a~dark brown colour. By contrast, prinsepia (/\ipa{tɕʰi˧nɑ˥}/) is not black or
dark"=coloured, and /\ipa{kʰv̩˧nɑ˥}/ for ‘dog’ carries no hint of hair colour, so it may be argued
that one of the two tone patterns corresponds to a~semantically bleached use of the adjective. But
this is less clear in the case of /\ipa{si˧nɑ˥}/, ‘wood’+‘dark’, for ‘deep forest’: here the
semantic indication of darkness seems present. Analysis of this issue is made more difficult by the rarity of lexical items sharing the same tone as ‘black’, //\ipa{nɑ˩\textsubscript{b}}//: the only other example observed to date is //\ipa{dʑɤ˩\textsubscript{b}}// ‘good’ (see \sectref{sec:adjectivesasdistinctfromverbs}).

A~similar lack of one"=to"=one {correspondence} between input tones and output tone is also found for compounds containing the adjective /\ipa{ɖɯ˩\textsubscript{a}}/ ‘large, big’. The compound nouns /\ipa{zo˧ɖɯ\#˥}/
‘eldest son’ and /\ipa{ʁo˧ɖɯ˧˥}/ ‘tadpole’ (literally ‘big head’) have different tones (\#H and
MH\#, respectively), although both are made up of a~noun root that has H lexical tone and the
adjective /\ipa{ɖɯ˩\textsubscript{a}}/ ‘big’.

To venture speculative hypotheses about the origin of these variegated tone patterns, first,
they may belong to different historical layers, and hence reflect different tone rules, which applied at different {diachronic} stages. For instance, ‘deep forest’, ‘dog’ and ‘prinsepia’ might be earlier than ‘wild animal’ and
‘lean meat’, as the compounds look less transparent semantically. The morpheme /\ipa{nɑ}/ could be analyzed as a~\is{suffixes}suffix in some cases, and as an adjective in others, on the {analogy} of the analysis of the morpheme /\ipa{mɔ¹³}/ by \citet[182]{lidz2010}, distinguishing the adjective ‘old’ from its use as a~\is{suffixes}suffix meaning ‘dear (indicating respect)’.

Another possibility is that some of these compounds are not based on the association of
a~{monosyllabic} noun with a~{monosyllabic} adjective, but constitute a~reduced form of longer words. For instance, the syllable /\ipa{si˧}/ in /\ipa{si˧nɑ˥}/ could result from the truncation of
the disyllable /\ipa{si˧ɕi˧˥}/, meaning ‘forest’.\footnote{\citet[50-52]{creissels1982} reports cases of synchronic {variation} in {Mandinka} where nouns that occur more frequently in compounds than on their own tend to get extracted from compounds with the changed tone that they carry inside these compounds. The changed tone eventually replaces the original lexical tone. Such processes, which operate on a word"=by"=word basis, detract from the regularity of tonal correspondences across dialects, greatly complicating {diachronic} comparison and {reconstruction}.} Evidence for the possibility to truncate this disyllable comes from /\ipa{tʰo˧ɕi˧˥}/ ‘pine forest’, which combines a~{monosyllable} for ‘pine’ with the \textit{second} syllable of /\ipa{si˧ɕi˧˥}/ ‘forest’. In the case of ‘eldest son’ and ‘tadpole’, such a~process appears
rather implausible, as these disyllables seem to have a~straightforward link to the {monosyllabic} nouns ‘son’ and
‘head’, respectively. If one nonetheless tries to push this hypothesis, one might hypothesize that
‘tadpole’ was built on the basis of a~disyllabic noun, which in principle could still be present in
another dialect.

A third possibility is that the adjective is not the same in all of these words: in synchrony, there
exists an adjective /\ipa{nɑ˥}/ (not \is{homophony}homophonous with /\ipa{nɑ˩\textsubscript{b}}/ ‘black, dark’) which means ‘important, serious (e.g.~a~wound)’, and this
adjective, or some other adjective pronounced [\ipa{nɑ}], may have provided the second syllable in
some adjectival compounds. This does not apply to ‘eldest son’ and ‘tadpole’, where the adjective
seems recognizably identical~-- unless interpretation of ‘tadpole’ as ‘big head’ is not the correct etymology,
but ‘big head’ seems fitting enough as the name of this small and amusing animal, which is an~obvious
candidate for frequent replacement through expressive coinages.

So far, consistency in the tone patterns of adjectival compounds seems limited to synchronically trivial
patterns. For instance, it does not come as a~surprise that Mid"=tone /\ipa{ʐɯ˧}/ ‘liquor/spirits’ and Low"=tone
/\ipa{nɑ˩\textsubscript{b}}/ ‘black’ yield a~compound with M+L surface tone pattern, /\ipa{ʐɯ˧nɑ˩}/: this looks
like a~case of simple concatenation. The same tone pattern is also found with another adjective that has
the same lexical tone, /\ipa{dʑɤ˩\textsubscript{b}}/ ‘good’. Disyllabic /\ipa{kɯ˧ dʑɤ˩}/ was easily extracted from the compound /\ipa{kɯ˧ dʑɤ˩ hɑ̃˩ dʑɤ˩}/ ‘auspicious
day’, from /\ipa{kɯ˧}/ ‘star’ and /\ipa{hɑ̃˧˥}/ ‘evening, night’ (a
term used to count days). (The L\# tone pattern of /\ipa{kɯ˧ dʑɤ˩}/, literally ‘good star’, results in the following two syllables receiving L tone,
through Rule 5.) 
The tone pattern of /\ipa{kɯ˧ dʑɤ˩}/ ‘good star’ is the same as that of /\ipa{ʐɯ˧nɑ˩}/ ‘high"=quality liquor’. This could suggest that both compounds belong to the same historical layer, but more examples would be necessary to investigate this issue further.

To sum up: in view of the limited number of examples found to date, and of their heterogeneity, it does not
appear illuminating to pool them all into a~table summarizing the tonal output of
\textsc{N}+\textsc{Adj} compounds. Provisionally, examples are simply listed in \tabref{tab:adjective-plus-nouncompoundtoneLa}, arranged by adjective, by decreasing number of examples.

%Table 10
%Table 10. in manuscript
%\begin{table}[t]
%\caption{Further examples of adjective"=plus"=noun compounds.}
%\begin{tabularx}{\textwidth}{ l Q l P{15mm} l P{32mm} }
%\lsptoprule
%	\multicolumn{2}{l}{noun} & \multicolumn{2}{l}{adjective} & \multicolumn{2}{l}{compound}\\\midrule
%\lspbottomrule
%\end{tabularx}
%\end{table}

As a~general observation, it can be seen that the patterns in \is{adjectives}noun"=plus"=adjective compounds are not identical to those in noun"=plus"=verb combinations (described in Chapter~\ref{chap:verbsandtheircombinatoryproperties}). For instance, ‘hot water’, /\ipa{dʑɯ˩tsʰi˩}/, has L tone, from an~input of L and M on the noun and adjective respectively; in noun"=plus"=verb combinations (either object plus verb or subject plus verb) with the same tonal input, the output is M.

The {diachronic} trend is for \is{monosyllables}monosyllabic nouns to become less frequent, being replaced by disyllables. As disyllabic forms of nouns become lexicalized, the tonal {correspondence} between the noun"=plus"=adjective combination and the root ceases to be accessible to the speakers of the language, making the tone patterns of disyllables more vulnerable to replacement (through \isi{analogy} or contact among dialects) than in cases where the root still exists as a~{monosyllable}. Thus, in the current state of the language ‘stone’ and ‘ard’ are only attested as disyllables, except in compounds with the adjective ‘old’, where they are found in \is{monosyllables}monosyllabic form. This raises the issue of whether it makes sense to extract a~{monosyllable} from the disyllabic compounds with ‘old’. The root for ‘stone’ could be \is{comparative method (historical linguistics)}reconstructed with a~H tone, as *\ipa{lv̩˥}, on the basis of this root's tonal behaviour when it stands in an adjectival compound: the compound /\ipa{lv̩˧mo˥}/ ‘old stone’ carries H\# tone, like the adjectival compounds created by adding ‘old’ to the H-tone monosyllables ‘person’, ‘horse’ and ‘wood’. But closer examination shows that ‘old stone’ is not in common use in Yongning Na, and has no clear meaning of its own. It is a~recent coinage, only found in a~saying~-- example (\ref{ex:whydontyou})~-- where it serves as a~parallel to /\ipa{si˧mo˥}/ ‘old wood’; its tone pattern also follows that of ‘old wood’. (Use of the symbol ‘F’ for ‘Focalization’ in the sentence"=level transcription is explained in \sectref{sec:focalization}.)

\begin{exe}
	\ex
	\label{ex:whydontyou}
		\ipaex{lv̩˧mo˥ F {\kern2pt}|{\kern2pt} dʑɯ˧ {\kern2pt}|{\kern2pt} le˧-qv̩˩; {\kern2pt}|{\kern2pt} si˧mo˥ F {\kern2pt}|{\kern2pt} le˧-dze˩ kv̩˩! {\kern2pt}|{\kern2pt} no˧ F {\kern2pt}|{\kern2pt} ə˧tse˧ {\kern2pt}|{\kern2pt} le˧-ʂɯ˧ mɤ˧-tʰɑ˧˥ {\kern2pt}|{\kern2pt} di˩!}\\
		\gll lv̩˧mo˥		dʑɯ˩	le˧					qv̩˩\textsubscript{a}	si˧mo˥		le˧-	dze˩\textsubscript{a}	-kv̩˧˥		no˩		ə˧tse˧	le˧-	ʂɯ˧\textsubscript{a}	mɤ˧-	tʰɑ˧˥ 	di˩\textsubscript{a}\\
		old.stones		water	\textsc{accomp}	to\_carry\_away			old\_wood	\textsc{accomp}	to\_cut		\textsc{abilitive}		\textsc{2sg}	\textsc{interrog.}why	\textsc{accomp}		to\_die		\textsc{neg}	\textsc{permissive}	\textsc{exist.spatial}\\
		\glt ‘Old stones are carried away by the stream; and old wood gets chopped down! And you, why won't you die?' (Field notes. Context: jeering an elderly person. Na tradition assigns human beings a~lifespan of sixty years; people getting past seventy are considered to be well past their expected lifespan.)
\end{exe}

Further analysis will require gathering more examples, sorting them into sets according to their tone patterns, identifying the historical layers that they belong to, and examining their process of formation. As a~first step in this direction, the following paragraph discusses items that are currently on the verge of \isi{lexicalization}.

\subsection{\textsc{N}+\textsc{Adj} combinations in the process of lexicalization}
\label{sec:nadjitemscurrentlyintheprocessoflexicalization}

In between adjectival constructions such as /\ipa{ə˧v̩˧˥ {\kern2pt}|{\kern2pt} ɖɯ˩-hĩ˩˥}/ ‘mother’s elder brother’ (example (\ref{ex:unc1}) above) on the one hand, and lexical items such as /\ipa{ə˧v̩˧-ɖɯ˧˥}/ (also meaning ‘mother’s elder brother’: see \tabref{tab:adjective-plus-nouncompoundtoneLa}) on the other, there are cases that offer insights into the process of \isi{lexicalization}. ‘Elderly person’ is /\ipa{hĩ˧mo˥}/, from /\ipa{hĩ˥}/ ‘person’ and /\ipa{mo˩\textsubscript{a}}/ ‘old’. In a~set of twenty texts, this noun appears fifteen times, always in the plural, as /\ipa{hĩ˧mo˥=ɻæ˩}/; the fact that it is followed by a~\is{clitics}clitic shows that it is a~full"=fledged noun. But there is a~higher number of occurrences (twenty"=three) of /\ipa{hĩ˧ mo˥-hĩ˩}/, which also means ‘elderly person’, again from \mbox{/\ipa{hĩ˥}/} ‘person’ and /\ipa{mo˩\textsubscript{a}}/ ‘old’, but with addition of the relativizer \mbox{/\ipa{-hĩ˥}/}. This is not quite like the adjectival construction presented in \sectref{sec:productiveconstruction}: in that construction, the noun constitutes a~\isi{tone group} on its own, e.g.~/\ipa{tɕʰo˩˧ {\kern2pt}|{\kern2pt} ʂɯ˧-hĩ\#˥}/ ‘new ladle’ (example (\ref{ex:newladle}) above), whereas ‘elderly person’ is realized as /\ipa{hĩ˧ mo˥-hĩ˩}/, in one \isi{tone group}. At a~push, it would be possible to say /\ipa{hĩ˧ {\kern2pt}|{\kern2pt} mo˩-hĩ˩˥}/ ‘a person that is old’, but this is judged decidedly awkward in the contexts where /\ipa{hĩ˧ mo˥-hĩ˩}/ is attested. The interpretation that can be proposed is that /\ipa{hĩ˧mo˥}/ is on its way towards \isi{lexicalization}~-- as evidenced by the tonal interaction between its two constituting morphemes~-- but the perception of its second syllable as an~adjective remains strong enough for the relativizer to be commonly added after it. The impossibility of adding the agent marker ($\ddagger${\kern2pt}\ipa{hĩ˧mo˥ ɳɯ˩}) or the topic marker ($\ddagger${\kern2pt}\ipa{hĩ˧mo˥ ʈʂʰɯ˩})  shows that /\ipa{hĩ˧mo˥}/ is not fully lexicalized yet. It is compulsory to add an~intervening plural or relativizer: /\ipa{hĩ˧mo˥=ɻæ˩ ɳɯ˩}/, /\ipa{hĩ˧mo˥=ɻæ˩ ʈʂʰɯ˩}/, /\ipa{hĩ˧ mo˥-hĩ˩ ɳɯ˩}/, and /\ipa{hĩ˧ mo˥-hĩ˩ ʈʂʰɯ˩}/.

\newpage 
For purposes of synchronic description, the notations adopted are /\ipa{hĩ˧mo˥{\allowbreak}=ɻæ˩}/ and /\ipa{hĩ˧ mo˥-hĩ˩}/. In /\ipa{hĩ˧mo˥=ɻæ˩}/, the sequence /\ipa{hĩ˧mo˥}/ is transcribed as a~lexical unit, with no hyphen or blank space between its two syllables. In /\ipa{hĩ˧ mo˥-hĩ˩}/, the first syllable is analyzed as a~noun, and separated by a~blank space from the adjective that follows. This notational distinction aims to draw attention to the versatility of disyllables made up of a~noun and an~adjective. To take another example of this phenomenon, /\ipa{zo˧bæ˩}/, from /\ipa{zo˥}/ ‘son; man’ and /\ipa{bæ˧}/ ‘stupid; dumb (unable to speak)’, has clearly nominal uses, meaning ‘dumb man; stupid man’. More than twenty examples are found in the Lake narrative, one of whose main protagonists is a~dumb person. The noun can be followed by the agent adposition: /\ipa{zo˧bæ˩ ɳɯ˩}/ (Lake3.29, Lake4.24); there is no need for an~intervening {relativizer{\slash}nominalizer} for quantization purposes, witness examples (\ref{ex:onedumb})--(\ref{ex:thatdumb}).

 

\begin{exe}
	\ex
	\label{ex:onedumb}
	\ipaex{zo˧bæ˩ ɖɯ˩-v̩˩}\\
	\gll zo˧bæ˩		ɖɯ˧-v̩˧\\
	dumb\_person	one-\textsc{clf}.individual\\
	\glt ‘a dumb person’ (Lake4.4)
\end{exe}

\begin{exe}
	\ex
	\label{ex:thisdumb}
	\ipaex{zo˧bæ˩ ʈʂʰɯ˩-v̩˩}\\
	\gll zo˧bæ˩		ʈʂʰɯ˥	v̩˧\\
	dumb\_person	\textsc{dem.prox}	\textsc{clf}.individual\\
	\glt ‘this dumb person’ (Lake4.6)
\end{exe}

\begin{exe}
	\ex
	\label{ex:thatdumb}
	\ipaex{zo˧bæ˩ tʰv̩˩-v̩˩}\\
	\gll zo˧bæ˩		tʰv̩˥	v̩˧\\
	dumb\_person	\textsc{dem.dist}	\textsc{clf}.individual\\
	\glt ‘that dumb person’ (Lake4.12-14)
\end{exe}

The expression /\ipa{zo˧bæ˩}/ can also appear right in front of a~verb, as in /\ipa{zo˧bæ˩ {\kern2pt}|{\kern2pt} go˩bo˧ di˧˥}/ ‘the dumb man drove cattle’ (Lake4.19). But in addition to such typically nominal uses, the word also has predicative (adjectival) uses: in a~context of self"=deprecation where someone accuses himself of being stupid, /\ipa{zo˧bæ˩}/, another person may comfort him by saying (\ref{ex:notdumb}).

\begin{exe}
	\ex
	\label{ex:notdumb}
	\ipaex{mɤ˧-zo˧bæ˩!}\\
	\gll mɤ˧-		zo˧bæ˩\\
	\textsc{neg}	stupid\\
	\glt ‘[No, you are] not stupid!’
\end{exe}

The antonym of /\ipa{zo˧bæ˩}/ in this adjectival sense is /\ipa{zo˧tʰi˧}/ ‘clever; clever person’, which has the same structure, from /\ipa{zo˥}/ ‘son; man’ and /\ipa{tʰi˧}/ ‘able; capable; clever; sharp’. The two words have different tones (L\# tone for /\ipa{zo˧bæ˩}/, vs.\ M tone for /\ipa{zo˧tʰi˧}/) despite their constituting morphemes having the same tones. This tonal difference alerts us to the possibility that the two words may have different time depths. Syntactically, the two words also differ: it is not possible to say $\ddagger${\kern2pt}\ipa{mɤ˧-zo˧tʰi˧} ‘not clever’, on the {analogy} of /\ipa{mɤ˧-zo˧bæ˩}/ ‘not stupid’ (\ref{ex:notdumb}). The first syllable of these two words is bleached enough for them to be used as adjectives for men and women alike, but in their nominal use they can only refer to men: /\ipa{zo˧bæ˩}/ means ‘stupid man’, and /\ipa{zo˧tʰi˧}/ ‘clever man’; the corresponding words for women are /\ipa{mv̩˩-bæ˧mi˩}/ ‘stupid woman’ and /\ipa{mv̩˩tʰi˩}/ ‘clever woman’.


\subsection{A lexicalized compound of \textsc{Adj}+N structure}
\label{sec:alexicalizedcompoundofadjnstructure}

So far, only one lexicalized compound of \textit{adjective plus noun} structure has been found: /\ipa{pv̩˧lv̩˧}/
‘nonirrigated farmland; dry land’, clearly related to /\ipa{pv̩˧}/ ‘dry’ and /\ipa{lv̩˧}/
‘field’. This word shows no phonological signs of antiquity, such as a~difference in consonant,
vowel or tone from its etymological components. But \is{comparative method (historical linguistics)}{comparative evidence} suggests that it may have
some time depth: it is also found in \ili{Naxi}, as /\ipa{pv̩˩ɭɯ˧}/, with the same \textsc{Adj}+\textsc{N}
structure, the same meaning, and the same transparency in terms of its components (in \ili{Naxi}, ‘dry’ is
/\ipa{pv̩˩}/, and ‘field’ is /\ipa{ɭɯ˧}/). \ili{Naxi} has another \textsc{Adj}+\textsc{N} compound
containing ‘dry’: /\ipa{pv̩˩dy˩}/ ‘dry land (as opposed to water)’ (\citealt[55]{pinsonetal2012}; note that there
is a~typographical error in the phonetic transcription, /\ipa{pv̩˩dv̩˩}/, which should be /\ipa{pv̩˩dy˩}/, as
indicated by the orthographic transcription).


\subsection{A lexicalized compound of V+\textsc{Adj} structure}
\label{sec:alexicalizedcompoundofvadjstructure}

As a~final observation about compounds, this paragraph discusses the only lexicalized compound of \textsc{V}+\textsc{Adj} structure observed so far: /\ipa{tsʰo˧ɖɯ˩}/ ‘group dance’, a~dance that can involve from ten
to about a~hundred people. Its components are /\ipa{tsʰo˧\textsubscript{b}}/ ‘to jump’ (in a~nominalized reading)
and /\ipa{ɖɯ˩\textsubscript{a}}/ ‘large’. An alternative interpretation whereby /\ipa{ɖɯ˩\textsubscript{a}}/ would have an~{adverbial}
reading (‘jumping a~lot’) would be implausible, because /\ipa{ɖɯ˩\textsubscript{a}}/ does not have attested
{adverbial} uses.

This is not a~productive construction: it is not possible to create compounds such as ‘banquet’ from
the verb ‘to eat’ and the adjective ‘large’, for instance. Still, the existence of the word /\ipa{tsʰo˧ɖɯ˩}/ ‘group dance’ can be
interpreted as evidence of an~occasional permeability of word classes. Across languages, the
distinction between nouns and verbs may be more or less stringent (\citealt{launey1994}, \textit{passim}). In \ili{Naish}
languages, there are some borderline items. For example, Na /\ipa{kɤ˧ʈʂɯ˩}/,
like \ili{Naxi} /\ipa{kɯ˧ʈʂɯ˩}/, has both verbal and nominal uses, namely ‘to speak’ and ‘speech; language’ in \ili{Naxi} and ‘to tell’ and ‘speech’ in Na.


\section{Concluding note}
\label{sec:concludingnotes}

The\is{complexity} complexity of the tone combination rules in various types of compounds in the Alawa dialect of Yongning Na, and the existence of exceptions to these synchronic rules, arguably shed light on the considerable diversity of
patterns from one dialect to the next, and even among different speakers from the same village
(not to mention the existence of variants within a~single idiolect). A~few compounds happen to carry
the same tone sequence as would the simple juxtaposition of their constituting elements. Most carry
a~tone pattern that reflects their status as compounds; among these, the nature of the compound
(determinative or coordinative) can also be identified from the tone pattern in a~few cases, whereas in the others it
must be arrived at on the basis of semantic information.

% indexing 'compound' for whole chapter
\is{compounds|)}
