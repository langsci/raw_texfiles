\chapter{Vowels and consonants} 
\label{chap:appendixa}
\label{chap:vowelsandconsonants}

% This chapter presents the segmental phonology of Yongning Na; it does not constitute a~necessary preliminary to the discussion of tones, since there are no restrictions on co-occurrence of tones and segments, and there are no special classes of segments triggering synchronic tonal processes (as is the case with depressor consonants in {Bantu}, e.g.~in the Ikalanga language: see \citealt{hymanetal1998}). The information in this chapter is background knowledge, explaining the notation used for vowels and consonants.

\epigraph{{\dots}~without a~good sense of how languages vary, not only in terms of the symbolic units such as phonemes and allophones but also in the details of their phonetic implementation, we have little hope of understanding the possible range of language. Endangered languages in particular represent an~important but often-ignored source of information about what is possible in Language.}{Richard Wright, University of Washington Phonetics Laboratory website}

%Command \noindent added to avoid having a first indent in cases where a paragraph starts after an epigraph without an intervening title.
{\noindent}In Yongning Na, there are no restrictions on co-occurrence of tones and segments, and no special classes of segments triggering synchronic tonal processes (as is the case with depressor consonants in \ili{Bantu}, e.g.~in the Ikalanga language: see \citealt{hymanetal1998}). So it did not appear appropriate to interpose a~presentation of the language's segmental phonetics and phonology between the reader and the book's central topic~-- tone.

This Appendix offers an opportunity to smuggle into this tonal study a~free-standing overview of the vowels and consonants of Yongning Na. Choices made at phonemicization are discussed in some detail, laying emphasis on those areas where phonemic analysis is cracking at the seams. Not a~few of these observations could open into \is{experimental phonetics}experimental phonetic/phonological studies, building on the availability of many hours of transcribed and annotated recordings.


%\section[Introductory note]{Introductory note: no phonological alternations; rich {coarticulatory} phenomena}
%\section{Introductory note}
\label{sec:introductionnophonologicalalternationsrichcoarticulatoryphenomena}

The vowels and consonants of Yongning Na are phonologically inert:
they are not involved in synchronic phonological rules and processes. In this respect, Yongning Na
is at the opposite end of the typological continuum from a~language such as Kifuliiru (\ili{Bantu}), which
has (i)~a~range of phonological rules, such as the strengthening of /\ipa{h}/, /\ipa{l}/ and
/\ipa{r}/ to a~plosive when preceded by a~nasal, 
%and the total \isi{assimilation} of a~non"=high,
%non"=back vowel (/\ipa{e}/ or /\ipa{a}/) to a~following vowel at morpheme boundaries within the
%word, e.g.~/\ipa{a}/+/\ipa{u}/$\rightarrow$/\ipa{uu}/, 
and (ii)~morphological rules, such as the
deletion of final consonants in {resultative} verb forms \citep[37--96]{vanotterloo2011}.

When examining vowels and consonants in Yongning Na, one's attention is drawn instead to their {coarticulation} patterns. Due to dramatic \isi{phonological erosion} since proto"=\il{Sino-Tibetan}Sino"=Tibetan \citep{jacquesetal2011}, syllabic structure in
Yongning Na has collapsed down to (C)(G)V+T, where C is a~consonant, G a~glide~-- with a~severely restricted
distribution~--, V a~syllable nucleus, and T represents tone; the brackets indicate that C and G are
optional. Coarticulation constitutes a~salient part of a~language’s sound system \citep{keating1990, kuhnertetal1999}. Structural
approaches to phonological systems predict cross"=linguistic differences in phenomena of
\isi{coarticulation} and articulatory reduction. To take an~example, the extent to which the palatalizing
influence of high, front vowels makes itself felt depends in part on the number, nature and
functional yield of existing phonemic oppositions: in Na, which contrasts /\ipa{ki}/ and
/\ipa{tɕi}/, the range of allophonic \isi{variation} of /\ipa{ki}/ can safely be predicted to be narrower
than in \ili{Naxi}, which does not have this contrast.

Phonetic studies confirm that \isi{coarticulation} is language"=specific: it is sensitive to phonological
inventory size and to the phonological distribution of contrasts \citep[162]{dicanio2012}. In the description that follows,
special attention is paid to \isi{coarticulation}, allophonic \isi{variation}, and phenomena of
articulatory reduction. 


\section{Consonant and vowel charts}
\label{sec:ConsonantAndVowelChart}

\tabref{tab:theinitialsofyongningna} presents consonants, and
\figref{fig:therhymesofyongningna} presents rhymes.\footnote{This Appendix focuses on the Alawa dialect. Systematic comparison across dialects is not attempted here.} The chart of rhymes includes syllabic /\ipa{ɻ̍}{\kern2pt}/ and /\ipa{v̩}/, discussed in the following section (\sectref{sec:consonantalnuclei}). They are respectively placed in the lower
right"=hand and upper right"=hand areas of the chart as a~rough approximation of their articulatory
characteristics. The following rhymes are not shown on the chart in order to avoid overcrowding:

\begin{itemize}
	\item  rhymes that contain a~glide (discussed in \sectref{sec:apresentationofonglideswithahypothesisaboutadiachroniconsetofhardeningofinitialglides}): /\ipa{wæ}/, /\ipa{wɑ}/, /\ipa{wɤ}/, /\ipa{jæ}/, /\ipa{jɤ}/ and /\ipa{jo}/
	\item the nasal vowels
	/\ipa{ĩ}/, /\ipa{ṽ̩}/, /\ipa{w̃ɤ}/,\footnote{In the rhymes /\ipa{w̃ɤ}/ and /\ipa{w̃æ}/, the diacritic for nasality is placed on
		the glide rather than on the final vowel because the degree of phonetic nasalization decreases throughout
		the course of the rhyme. About nasal rhymes in Yongning Na, see \sectref{sec:nasalrhymes}.} /\ipa{æ̃}/ and /\ipa{ɑ̃}/, all appearing only after /\ipa{h}/
	\item the rhymes /\ipa{ɻ̍̃}{\kern2pt}/ and /\ipa{w̃æ}/, which always constitute syllables on their own (i.e.\ do not combine with any initial)
	\item the nasal vowel /\ipa{õ}/ appearing after /\ipa{h}/ or on
	its own, i.e.\ in the syllables /\ipa{hõ}/ and /\ipa{õ}/
\end{itemize}

{\setlength\tabcolsep{4pt}
	\begin{table}[h!!]
		\caption{The initials of Yongning Na.}
%		{\fontsize{9}{10.75}\selectfont
		\begin{tabularx}{\textwidth}{ l P{14mm} l P{14mm} l l l l}
			\lsptoprule
			& bilabial~/ labio"=dental & dental & alveolo"=palatal & retroflex & velar & uvular & glottal\\\midrule
			plosive & \ipa{pʰ p b}  & \ipa{tʰ t d} & & \ipa{ʈʰ ʈ ɖ} & \ipa{kʰ k g} & \ipa{qʰ q} & \ipa{ʔ}\\
			affricate &   & \ipa{tsʰ ts dz} & \ipa{tɕʰ tɕ dʑ} & \ipa{ʈʂʰ ʈʂ ɖʐ}  & & &\\
%			nasal & \ipa{m} & \ipa{n} & \ipa{ɲ} & \ipa{ɳ} & \ipa{ŋ} &\\
			nasal & \ipa{m}  & \ipa{n} &  \ipa{ɲ} &  \ipa{ɳ} & \ipa{ŋ} & &\\
			fricative &  \ipa{f} & \ipa{s z} & \ipa{ɕ ʑ} & \ipa{ʂ ʐ} & & \ipa{ʁ} & \ipa{h}\\
			lateral  &  & \ipa{ɬ l} & & & & &\\
			approximant &  &  &  & \ipa{ɻ} &  &  &\\\lspbottomrule
		\end{tabularx}
%				}
		\label{tab:theinitialsofyongningna}
	\end{table}}
	
	\begin{figure}[h!!]
		\caption{The rhymes of Yongning Na.}
		\begin{tikzpicture}[scale=1.25]
		\large
		\tikzset{
			vowel/.style={fill=white, anchor=mid, text depth=0ex, text height=1ex},
			dot/.style={circle,fill=black,minimum size=0.6ex,inner sep=0pt,outer sep=-1pt},
		}
		\coordinate (hf) at (0,3); % high front
		\coordinate (hb) at (4,3); % high back
		\coordinate (lf) at (2,0); % low front
		\coordinate (lb) at (4,0); % low back
		\def\V(#1,#2){barycentric cs:hf={(3-#1)*(2-#2)},hb={(3-#1)*#2},lf={#1*(2-#2)},lb={#1*#2}}
		
		% Draw the horizontal lines first.
		\draw (\V(0,0)) -- (\V(0,2));
		\draw (\V(1,0)) -- (\V(1,2));
		\draw (\V(2,0)) -- (\V(2,2));
		\draw (\V(3,0)) -- (\V(3,2));
		
		% Place all the unrounded-rounded pairs next, on top of the horizontal lines.
		\path (\V(0,0))     node[vowel, left] (close) {\ipa{i}} node[vowel, right] {} node[dot] {};
		\path (\V(0,1))     node[vowel, left] {} node[vowel, right] {} node[dot] {};
		\path (\V(0,2))     node[vowel, left] {\ipa{ɯ}} node[vowel, right] {\ipa{u}} node[dot] {};
		\path (\V(0.5,0.4)) node[vowel, left] {} node[vowel, right] {} node[ ] {};
		\path (\V(0.5,1.6)) node[vowel, left] {\ipa{v̩}} node[vowel, right] {} node[ ] {};
		\path (\V(1,0))     node[vowel, left] (closemid) {\ipa{e}} node[vowel, right] {} node[dot] {};
		\path (\V(1,1))     node[vowel, left] {} node[vowel, right] {} node[dot] {};
		\path (\V(1,2))     node[vowel, left] {\ipa{ɤ}} node[vowel, right] {\ipa{o}} node[dot] {};
		\path (\V(2,0))     node[vowel, left] (openmid) {} node[vowel, right] {} node[dot] {};
		\path (\V(2,1))     node[vowel, left] {} node[vowel, right] {} node[dot] {};
		\path (\V(2,2))     node[vowel, left] {} node[vowel, right] {} node[dot] {};
		\path (\V(2.5,0))   node[vowel, left] {\ipa{æ}} node[vowel, right] {} node[   ] {};
		\path (\V(2.5,1.6)) node[vowel, left] {\ipa{ɻ̍}} node[vowel, right] {} node[ ] {};
		\path (\V(3,0))     node[vowel, left] (open) {} node[vowel, right] {} node[dot] {};
		\path (\V(3,2))     node[vowel, left] {\ipa{ɑ}} node[vowel, right] {} node[dot] {};
		
		\node[scale=0.8] [left=of close] {Close};
		\node[scale=0.8] [left=of closemid] {Close-mid};
		\node[scale=0.8] [left=of openmid] {Open-mid};
		\node[scale=0.8] [left=of open] {Open};
		
		\node[scale=0.8] () at (0,3.75) {Front};
		\node[scale=0.8] () at (2,3.75) {Central};
		\node[scale=0.8] () at (4,3.75) {Back};
		
		% Draw the vertical lines.
		\draw (\V(0,0)) -- (\V(3,0));
		\draw (\V(0,1)) -- (\V(3,1));
		\draw (\V(0,2)) -- (\V(3,2));
		
		% Place the unpaired symbols last, on top of the vertical lines.
		\path (\V(1.5,1))   node[vowel]       {\textbf{ə}};
		%\path (\V(2.5,1))   node[vowel]       {};
		\end{tikzpicture}
		\label{fig:therhymesofyongningna}
	\end{figure}
	
\newpage 	
The notation used throughout this book, and in the transcription of texts, is phonemic in
orientation. However, a~concern for phonetic transparency led to indicate the empty"=onset fillers [{\kern1.3pt}\ipa{ʝ}], [\ipa{ɣ}] and [\ipa{w}] in transcriptions, even though they are not phonemically contrastive (as explained in \sectref{sec:smoothphoneticonsets}) and hence not included in \tabref{tab:theinitialsofyongningna}. 

All the possible vowel nuclei following each initial are listed in Tables \ref{tab:InNucl} and \ref{tab:InNucl2}. In the interest of space, syllables with nasal rhymes are not shown. These tables only provide a~first"=pass view of Na phonotactics: no indication about lexical frequency is provided, only a~binary indication of presence or absence of the combination at issue. A~syllable is considered as present if it is firmly attested in Na vocabulary, i.e.\ excluding combinations that have been introduced through lexical borrowings, vowel harmony and expressive processes, no matter how widely attested these are in the present"=day state of the language. Thus, combinations of dental stops with the vowel /\ipa{æ}/ are indicated as nonexistent, because the only attestations are in examples such as /\ipa{tʰæ˩tsɯ˧}/ ‘jar’, borrowed from {Mandarin} \textit{tánzi} \zh{坛子}, as well as /\ipa{tæ˧ɻæ˩}/ ‘Adam's apple’ and /\ipa{læ˧dæ˧qæ˥}/ ‘armpit’, where it may result from vowel harmony with the second syllable. Likewise, combinations of retroflex stops with the vowel /\ipa{ɑ}/ are indicated as nonexistent, because the only attestations are in the examples /\ipa{ʈʂʰɑ˧lɑ˧}/ ‘to chat’, /\ipa{qv̩˧ɻ̍˧-ʈʂʰɑ˧nɑ˥\#}/ (the name of a~mountain) and /\ipa{ʈʂɑ˧tɑ˥}/ ‘written sign, tracing’, three words in which the first syllable's /\ipa{ɑ}/ vowel  may result from vowel harmony with the second syllable. Abstracting away from phenomena that can be identified as {structural} \is{gap-filling}gap"=filling highlights empty slots in the table, and cases of complementary distribution. Comments about the inventory of syllables are set out in \sectref{sec:commentsabouttheinventoryofsyllables}.
%\todo{move split between first and esecond table up}
\begin{table}
	\caption{Inventory of attested combinations of initials and rhymes, leaving aside marginal words (loanwords, expressive words{\dots}). First part.}
	\label{tab:InNucl}
	\begin{tabular}{llllllllllllllll}
		\lsptoprule
		& \ipa{i} & \ipa{ɯ} & \ipa{u} & \ipa{v̩} & \ipa{e} & \ipa{ɤ} & \ipa{o} & \ipa{æ} & \ipa{ɑ} & \ipa{wæ} & \ipa{wɑ} & \ipa{wɤ} & \ipa{jæ} & \ipa{jɤ} & \ipa{jo}\\ \midrule
		\ipa{Ø} & \checkmark & \checkmark & \checkmark & \checkmark & \xmark & \xmark & \checkmark & \checkmark & \checkmark & \xmark & \xmark & \checkmark & \xmark & \checkmark & \checkmark\\
		\ipa{pʰ} & \checkmark & \xmark & \xmark & \checkmark & \checkmark & \checkmark & \checkmark & \checkmark & \xmark & \xmark & \xmark & \xmark & \xmark & \xmark & \xmark\\
		\ipa{p} & \checkmark & \xmark & \xmark & \checkmark & \xmark & \checkmark & \checkmark & \checkmark & \xmark & \xmark & \xmark & \xmark & \xmark & \xmark & \xmark\\
		\ipa{b} & \checkmark & \xmark & \xmark & \checkmark & \xmark & \checkmark & \checkmark & \checkmark & \xmark & \xmark & \xmark & \xmark & \xmark & \xmark & \xmark\\
		\ipa{m} & \checkmark & \xmark & \xmark & \checkmark & \xmark & \checkmark & \checkmark & \checkmark & \xmark & \xmark & \xmark & \xmark & \xmark & \xmark & \xmark\\
		\ipa{tʰ} & \checkmark & \xmark & \xmark & \checkmark & \xmark & \xmark & \checkmark & \xmark & \checkmark & \xmark & \xmark & \xmark & \xmark & \xmark & \xmark\\
		\ipa{t} & \checkmark & \xmark & \xmark & \checkmark & \xmark & \checkmark & \checkmark & \xmark &  \checkmark & \xmark & \xmark & \xmark & \xmark & \xmark & \xmark\\
		\ipa{d} & \checkmark & \xmark & \xmark & \checkmark & \xmark & \checkmark & \checkmark & \xmark &  \checkmark & \xmark & \xmark & \xmark & \xmark & \xmark & \xmark\\
		\lspbottomrule
	\end{tabular}
\end{table}

\begin{table}
	\caption{Inventory of attested combinations of initials and rhymes, leaving aside marginal words (loanwords, expressive words{\dots}). Second part.}
	\label{tab:InNucl2}
	\begin{tabular}{llllllllllllllll}
		\lsptoprule
		& \ipa{i} & \ipa{ɯ} & \ipa{u} & \ipa{v̩} & \ipa{e} & \ipa{ɤ} & \ipa{o} & \ipa{æ} & \ipa{ɑ} & \ipa{wæ} & \ipa{wɑ} & \ipa{wɤ} & \ipa{jæ} & \ipa{jɤ} & \ipa{jo}\\ \midrule
		\ipa{tsʰ} & \checkmark & \checkmark & \xmark & \xmark & \checkmark & \checkmark & \checkmark & \xmark &  \checkmark & \xmark & \xmark & \xmark & \xmark & \xmark & \xmark\\
		\ipa{ts} & \checkmark & \checkmark & \xmark & \xmark & \checkmark & \checkmark & \checkmark & \xmark &  \checkmark & \xmark & \xmark & \xmark & \xmark & \xmark & \xmark\\
		\ipa{dz} & \checkmark & \checkmark & \xmark & \xmark & \checkmark & \checkmark & \checkmark & \xmark &  \checkmark & \xmark & \xmark & \xmark & \xmark & \xmark & \xmark\\
		\ipa{n} & \checkmark & \xmark & \xmark & \checkmark & \checkmark & \xmark & \checkmark & \xmark &  \checkmark & \xmark & \xmark & \xmark & \xmark & \xmark & \xmark\\
		\ipa{s} & \checkmark & \checkmark & \xmark & \xmark & \checkmark & \checkmark & \checkmark & \xmark &  \checkmark & \xmark & \xmark & \xmark & \xmark & \xmark & \xmark\\
		\ipa{z} & \checkmark & \checkmark & \xmark & \xmark & \checkmark & \xmark & \checkmark & \xmark &  \checkmark & \xmark & \xmark & \xmark & \xmark & \xmark & \xmark\\
		\ipa{ɬ} & \checkmark & \xmark & \xmark & \checkmark & \xmark & \xmark & \checkmark & \xmark &  \checkmark & \xmark & \xmark & \xmark & \xmark & \xmark & \xmark\\
		\ipa{l} & \checkmark & \xmark & \xmark & \checkmark & \checkmark & \xmark & \checkmark & \xmark &  \checkmark & \xmark & \xmark & \xmark & \xmark & \xmark & \xmark\\
		\ipa{tɕʰ} & \checkmark & \checkmark & \xmark & \xmark & \xmark & \checkmark & \checkmark & \xmark & \xmark & \xmark & \xmark & \xmark & \xmark & \xmark & \xmark\\
		\ipa{tɕ} & \checkmark & \checkmark & \xmark & \xmark & \xmark & \checkmark & \checkmark & \xmark & \xmark & \xmark & \xmark & \xmark & \xmark & \xmark & \xmark\\
		\ipa{dʑ} & \checkmark & \checkmark & \xmark & \xmark & \xmark & \checkmark & \checkmark & \xmark & \xmark & \xmark & \xmark & \xmark & \xmark & \xmark & \xmark\\
		\ipa{ɲ} & \checkmark & \xmark & \xmark & \xmark & \xmark & \xmark & \xmark & \xmark & \xmark & \xmark & \xmark & \xmark & \xmark & \xmark & \xmark\\
		\ipa{ɕ} & \checkmark & \checkmark & \xmark & \xmark & \xmark & \xmark & \xmark & \xmark & \xmark & \xmark & \xmark & \xmark & \xmark &  \checkmark &  \checkmark\\
		\ipa{ʑ} & \checkmark & \xmark & \xmark & \xmark & \xmark & \xmark & \xmark & \xmark & \xmark & \xmark & \xmark & \xmark & \xmark & \xmark & \xmark\\
		\ipa{ʈʰ} & \checkmark & \checkmark & \xmark & \xmark & \xmark & \checkmark & \xmark & \checkmark & \xmark & \xmark & \xmark & \xmark & \xmark & \xmark & \xmark\\
		\ipa{ʈ} & \checkmark & \checkmark & \xmark & \checkmark & \xmark & \checkmark & \xmark & \checkmark & \xmark & \xmark & \xmark & \xmark & \xmark & \xmark & \xmark\\
		\ipa{ɖ} & \checkmark & \checkmark & \xmark & \checkmark & \xmark & \checkmark & \checkmark & \checkmark & \xmark & \xmark & \xmark & \xmark & \xmark & \xmark & \xmark\\
		\ipa{ʈʂʰ} & \checkmark & \checkmark & \xmark & \checkmark & \checkmark & \checkmark & \checkmark & \checkmark & \xmark & \xmark & \xmark & \xmark & \xmark & \xmark & \xmark\\
		\ipa{ʈʂ} & \checkmark & \checkmark & \xmark & \checkmark & \checkmark & \checkmark & \checkmark & \checkmark & \xmark & \xmark & \xmark & \xmark & \xmark & \xmark & \xmark\\
		\ipa{ɖʐ} & \checkmark & \checkmark & \xmark & \checkmark & \checkmark & \checkmark & \checkmark & \checkmark & \xmark & \xmark & \xmark & \xmark & \xmark & \xmark & \xmark\\
		\ipa{ɳ} & \xmark & \checkmark & \xmark & \checkmark & \xmark & \xmark & \xmark & \checkmark & \xmark & \xmark & \xmark & \xmark & \xmark & \xmark & \xmark\\
		\ipa{ʂ} & \checkmark & \checkmark & \xmark & \checkmark & \checkmark & \checkmark & \checkmark & \checkmark & \xmark & \checkmark & \xmark & \checkmark & \xmark & \xmark & \xmark\\
		\ipa{ʐ} & \checkmark & \checkmark & \xmark & \checkmark & \checkmark & \checkmark & \checkmark & \checkmark & \xmark & \checkmark & \xmark & \checkmark & \xmark & \xmark & \xmark\\
		\ipa{ɻ} & \checkmark & \xmark & \xmark & \xmark & \xmark & \xmark & \xmark & \checkmark & \xmark & \checkmark & \xmark & \xmark & \xmark & \xmark & \xmark\\
		\ipa{kʰ} & \checkmark & \checkmark & \xmark & \checkmark & \xmark & \checkmark & \checkmark & \xmark & \xmark & \xmark & \xmark & \checkmark & \xmark & \xmark & \xmark\\
		\ipa{k} & \checkmark & \checkmark & \xmark & \checkmark & \xmark & \checkmark & \checkmark & \xmark & \xmark & \xmark & \xmark & \checkmark & \xmark & \xmark & \xmark\\
		\ipa{g} & \checkmark & \checkmark & \xmark & \checkmark & \xmark & \checkmark & \checkmark & \xmark & \xmark & \xmark & \xmark & \checkmark & \xmark & \xmark & \xmark\\
		\ipa{ŋ} & \checkmark & \xmark & \xmark & \checkmark & \xmark & \checkmark & \xmark & \xmark & \xmark & \xmark & \xmark & \checkmark & \xmark & \xmark & \xmark\\
		\ipa{qʰ} & \checkmark & \xmark & \xmark & \checkmark & \xmark & \xmark & \checkmark & \checkmark & \checkmark & \checkmark & \xmark & \checkmark & \xmark & \xmark & \xmark\\
		\ipa{q} & \checkmark & \xmark & \xmark & \checkmark & \xmark & \xmark & \checkmark & \checkmark & \checkmark & \checkmark & \xmark & \checkmark & \xmark & \xmark & \xmark\\
		\ipa{ʁ} & \checkmark & \xmark & \xmark & \checkmark & \xmark & \xmark & \checkmark & \checkmark & \checkmark & \checkmark & \xmark & \checkmark & \xmark & \xmark & \xmark\\
		\ipa{h} & \checkmark & \checkmark & \checkmark & \xmark & \xmark & \checkmark & \checkmark & \checkmark & \checkmark & \checkmark & \xmark & \checkmark & \xmark & \xmark & \xmark\\
		\ipa{f} & \xmark & \xmark & \xmark & \checkmark & \xmark & \xmark & \xmark & \xmark & \xmark & \xmark & \xmark & \xmark & \xmark & \xmark & \xmark\\
		\lspbottomrule
	\end{tabular}
\end{table}

\clearpage
%Page typesetting will need verification on proofs to ensure that the tables+figures are in good succession, without too much empty space, and not too much distance between text and tables


	\section{Syllable nuclei: Vowels and syllabic consonants}
	\label{sec:thesyllablenucleivowelsandsyllabicconsonants}
	
	\subsection{Consonantal nuclei}
	\label{sec:consonantalnuclei}
	
	In Na, some consonantal sounds function as syllable nuclei. In the International Phonetic Alphabet such
	sounds are called ‘syllabic’. Fricative sounds with syllabic value
	are an~areal characteristic. They are common in neighbouring \ili{Yi} languages, which
	belong to the {Nasoid} subgroup of Loloish (also known as Ngwi; see \citealt[70]{bradley1979}), as well as in \ili{Lizu} \citep{chirkovaetal2012}, Ersu \citep{chirkovaersu2015} and Pumi \citep[52]{daudey2014}. Local
	dialects of {Mandarin} also have a~fricative syllable nucleus, /\ipa{fv̩}/: for instance ‘lake’, Standard {Mandarin}
	\textit{hú} \zh{湖}, is pronounced\il{Mandarin!Southwestern} [\ipa{fv̩}] \citep[on Southwestern Mandarin: see][]{guiyunnanese2001, pinson2008}. In the absence of coda consonants in Na, ‘consonantal
	nuclei’ are also referred to in this volume as ‘consonantal rhymes’. Yongning Na has the following consonantal
	rhymes: [\ipa{v̩}], [\ipa{z̩}], [\ipa{ʐ̍}{\kern2pt}], [\ipa{ɻ̍}{\kern2pt}] and [\ipa{ɻ̍̃}{\kern2pt}].
	

	\subsubsection{The voiced fricative /\ipa{v̩}/}
	\label{sec:thevoicedfricative}
	
	The voiced fricative /\ipa{v̩}/ can only appear as a~rhyme, not as an~initial. The
	friction for /\ipa{v̩}/ is weaker than in \ili{Naxi}. There is some amount of formant movement towards
	a~central articulation: [\ipa{və}]. Since friction noise is slight, even at the beginning of the
	rhyme, this /\ipa{v̩}/ could be described as close to an~approximant, [\ipa{ʋ}]. Martinet makes
	a~distinction between fricatives proper, which have a~firm articulation and are characterized by
	fricative noise, and spirants, which have a~relaxed articulation, tending towards a~vowel"=like
	aperture (\citealt[24]{martinet1956}; \citeyear{martinet1981}). In terms of this distinction, the articulation of /\ipa{v̩}/ in
	Yongning Na is spirant rather than fricative. After bilabial initials, /\ipa{v̩}/ has a~tendency
	towards trilling (though, again, less markedly so than in \ili{Naxi}): /\ipa{bv̩}/ is realized close to
	[\ipa{ʙ̩}], /\ipa{pv̩}/ close to [\ipa{pʙ̩}], and /\ipa{pʰv̩}/ close to [\ipa{pʰʙ̩}].
	
	In the absence of strong friction, the rhyme /\ipa{v̩}/ can be difficult to distinguish from the high
	back vowel /\ipa{o}/, especially after consonants that exert similar {coarticulatory} effects on the
	rhyme. Uvular stops result in a~retracted (backed) articulation of both /\ipa{v̩}/ and /\ipa{o}/,
	making the opposition between syllables such as /\ipa{qv̩}/ and /\ipa{qo}/ difficult to hear at
	first. Minimal pairs such as /\ipa{mæ˧qv̩˩}/ ‘tail’ vs.\ /\ipa{mæ˧qo˩}/ ‘end; back’ constitute handy materials for learners to practise this opposition.
	
	From a~\is{comparative method (historical linguistics)}{diachronic} point of view, the rhyme /\ipa{v̩}/ in Yongning Na (as also in \ili{Naxi}) is hypothesized to originate in *\ipa{u} \citep{jacquesetal2011}. Seen in this light, the process of fricativization is less advanced in Yongning Na than in Lijiang \ili{Naxi}. One could find among \ili{Naish} languages a~continuum between [\ipa{u}]-like and [\ipa{v̩}]-like realizations, through approximant [\ipa{ʋ̩}]. Such cases serve as reminders that the choice of symbols from the International Phonetic Alphabet does not tell the full story about phonological systems, as Martinet reminds us.
	
	\begin{quotation}
		One wonders whether the habit of constantly operating with graphic notations does not make some linguist[s] deaf to the gradual shifts which any painstaking observation can reveal. If one has been taught, not only that phonological systems are made up of discrete units, but also that these units are basically the same in all languages, ({\dots}) one can hardly avoid concluding that no change can take place except by means of jumps from one unit or allophone to another. Only those who know that linguistic identity does not imply physical sameness, can accept the notion that discreteness does not rule out infinite variety and be thus prepared to perceive the gradualness of phonological shifts. \citep[25]{martinet1988}
	\end{quotation}
	

	\subsubsection{Apicalized vowels}
	\label{sec:apicalizedvowels}
	
{\largerpage}

	Apicalized vowels are found “on the tip of many tongues” across \il{Sino-Tibetan}Sino"=Tibetan \citep{baron1974}, and Na is no \is{exceptions}exception. The vowel  /\ipa{ɯ}/ has fricative allophones after dental and retroflex fricatives and affricates; this situation is identical to that found in \ili{Naxi}. For instance,  /\ipa{tsʰɯ˧˥}/ ‘goat’
	is realized  [\ipa{tsʰz̩˧˥}], and  /\ipa{ɖʐɯ˥}/ ‘market, city’ is realized  [\ipa{ɖʐʐ̍˥}]. In phonetic transcription, it is customary in Asian studies to use the symbols coined by Chao Yuen"=ren:
	[\ipa{ɿ}] and [\ipa{ʅ}{\kern2pt}], respectively. The whole set proposed by Chao Yuen"=ren is shown in \tabref{tab:apicalized}; symbol descriptions are from \citet[80, 89-90]{pullumetal1986}.
	
	\begin{table}[h]
		\caption{Chao Yuen-ren's symbols for apicalized vowels, and closest equivalents in the International Phonetic Alphabet (IPA).}
		\begin{tabularx}{\textwidth}{ llll }
			\lsptoprule
			sound & symbol & symbol description & IPA\\\midrule
			plain apical vowel & [\ipa{ɿ}] & long"=leg turned iota & \ipa{z̍}\\
			retroflex apical vowel & [\ipa{ʅ}{\kern2pt}] & right-tail turned iota & \ipa{ʐ̍}\\
			rounded plain
			apical vowel &  [\ipa{ʮ}] & curvy turned h & \ipa{z̹̍}\\
			rounded retroflex apical vowel &  [\ipa{ʯ}{\kern2pt}]  & right"=tail curvy turned h & \ipa{ʐ̹̍}\\
			\lspbottomrule
		\end{tabularx}
		\label{tab:apicalized}
	\end{table}
	
%\Hack{\newpage}

	The reason that rounded apical vowels are not found in Na is that (i)~there is no rounding opposition among front vowels, and hence no source for a~process of apicalization such as *\ipa{y}~> [\ipa{ʮ}], and (ii)~among back vowels, the close vowel *\ipa{u} fricativized to /\ipa{v̩}/, as described above (\sectref{sec:thevoicedfricative}). 
	
	In addition to an~opposition between plain and retroflex apical vowels, there is in Yongning Na an~opposition between /\ipa{i}/ and another high front
	vowel after the alveolopalatal initials /\ipa{dʑ}/, /\ipa{tɕ}/ and /\ipa{tɕʰ}/. The first set~-- syllables /\ipa{dʑi}/, /\ipa{tɕi}/ and /\ipa{tɕʰi}/~-- has moderate friction during
	the initial, and the rhyme is not strongly apicalized. 

{\largerpage}
	
	The second set consists of syllables that have
	an~apicalized rhyme and a~slightly more central vowel; these syllables can be approximated as [\ipa{dʑ̍}], [\ipa{tɕʑ̍}] and [\ipa{tɕʰʑ̍}], respectively. If one were to abstract from the phonetic
	friction found on the rhymes of these syllables, a~possible phonetic
	notation would be as [\ipa{dʑɪ}], [\ipa{tɕɪ}] and [\ipa{tɕʰɪ}]. The phonemic analysis adopted here
	is as allophones of /\ipa{ɯ}/, hence /\ipa{dʑɯ}/, /\ipa{tɕɯ}/ and /\ipa{tɕʰɯ}/. This choice is based
	on structural considerations (complementary distribution), not on phonetics, and the degree of
	validity of this choice remains to be tested experimentally, e.g.~by investigating to what extent
	priming\footnote{Priming is a~memory effect in which exposure to one stimulus (the \textit{prime}) influences the response to another stimulus.} by canonical realizations of the vowel /\ipa{ɯ}/ (as in /\ipa{ʈʰɯ˩}/ ‘to drink’) affects
	reaction times in perceptual tests. Phonetically, the syllables /\ipa{dʑɯ}/, /\ipa{tɕɯ}/ and /\ipa{tɕʰɯ}/ are not easy to distinguish
	from /\ipa{dzɯ}/, /\ipa{tsɯ}/ and /\ipa{tsʰɯ}/, which are shown in \tabref{tab:alveolopal} with their realizations in Chao notation and in International Phonetic Alphabet.
	
	\begin{table}
		\caption{\label{tab:alveolopal}Syllables with initial dental affricate and high, unrounded vowel /\ipa{ɯ}/, and their phonetic realizations.}
%		{\renewcommand{\arraystretch}{1.35}	
			\begin{tabularx}{\textwidth}{ P{30mm} l Q }
				\lsptoprule
				phonemic analysis & Chao notation & International Phonetic Alphabet\\\midrule
				\ipa{dzɯ} & [\ipa{dzɿ}] & [\ipa{dz̩}]\\
				\ipa{tsɯ} & [\ipa{tsɿ}] & [\ipa{tsz̩}]\\
				\ipa{tsʰɯ} & [\ipa{tsʰɿ}] & [\ipa{tsʰz̩}]\\
				\lspbottomrule
			\end{tabularx}
		\end{table}
		
	An added complexity is related to Voice Onset Time oppositions among alveolopalatal
	initials. Syllables with an~unvoiced or aspirated initial, /\ipa{tɕ}/ or /\ipa{tɕʰ}/, have stronger
	affrication than those with initial /\ipa{dʑ}/. This has {coarticulatory} consequences on the
	following vowel, so that both /\ipa{i}/ and /\ipa{ɯ}/ are somewhat less open after /\ipa{tɕ}/ and
	/\ipa{tɕʰ}/ than after /\ipa{dʑ}/.\footnote{A recording (PalatalizedApicalized) was specifically conducted for words containing one of the following syllables:
		/\ipa{dʑi}/, /\ipa{dʑɯ}/, /\ipa{tɕi}/, /\ipa{tɕɯ}/, /\ipa{tɕʰi}/ and /\ipa{tɕʰɯ}/.}
		

	\subsubsection{Rhotic rhymes}
	\label{sec:rhoticrhymes}
	
	Further consonantal rhymes are /\ipa{ɻ̍}{\kern2pt}/ and its nasalized counterpart /\ipa{ɻ̍̃}{\kern2pt}/. From a~phonetic
	point of view, the rhyme /\ipa{ɻ̍}{\kern2pt}/ does not display the considerable lowering of the third formant
	which is the tell"=tale characteristic of rhotic vowels \citep[313]{ladefogedetal1996}, such as \ili{Naxi}
	/\ipa{ə˞}{\kern1.5pt}/. The diacritic indicating syllabic status distinguishes this rhyme from the consonant
	phoneme /\ipa{ɻ}{\kern2pt}/ described in \sectref{sec:lateralsandandtheretroflexapproximant}. As a~rhyme,
	/\ipa{ɻ̍}{\kern2pt}/ can appear on its own or preceded by a~retroflex fricative or affricate. Examples of
	syllables containing a~/\ipa{ɻ̍}{\kern2pt}/ rhyme preceded by a~retroflex affricate are presented in
	\tabref{tab:someexamplesillustratingthephonemiccontrastbetweenandafterretroflexfricativesandaffricates},
	together with syllables made up of the same consonant followed by the rhyme /\ipa{v̩}/.
	

	\begin{table}%[t]
		\caption{Some examples illustrating the phonemic contrast between  /\ipa{ɻ̍}{\kern2pt}/ and  /\ipa{v̩}/ after
			retroflex fricatives and affricates.}
		{\renewcommand{\arraystretch}{1.35}
			\begin{tabularx}{.75\textwidth}{ l Q l }
				\lsptoprule
				& \ipa{v̩} & \ipa{ɻ̍}\\\midrule
				
				
				\ipa{ʈʂʰ} & /\ipa{ʈʂʰv̩˧}/ ‘breakfast’,  /\ipa{ʈʂʰv̩˧ɻ̍˥\$}/ ‘ant’,  \par /\ipa{bv̩˧ʈʂʰv̩˧}/ ‘cymbals’ & /\ipa{ʈʂʰɻ̍˧˥}/ ‘lungs’\\ 
				\ipa{ʈʂ} & /\ipa{mv̩˧ʈʂv̩˥}/ ‘wrinkles’ & /\ipa{ʈʂɻ̍˥}/ ‘to cough’\\ 
				\ipa{ɖʐ} & \multicolumn{2}{c}{\textit{no contrasts observed}\footnote{Roselle Dobbs (p.c.\ 2016) reports that this contrast does exist in the Lataddi dialect. Using her notations, in which /\ipa{u}/ corresponds to Alawa /\ipa{v̩}/, and /\ipa{v̩}/ to Alawa /\ipa{ɻ̍}/, the contrast is exemplified by /\ipa{ʐu˥˩}/ ‘four’ (homophone: ‘delicious’) vs.\ /\ipa{ʐv̩˥˩}/ ‘horse’.}}\\ 
				\ipa{ʈʰ  ʈ  ɖ  tʰ  t  d} & \multicolumn{2}{c}{\textit{no contrasts observed}}\\ 
				\ipa{ʂ} & /\ipa{ʂv̩˧ɖv̩˧}/ ‘to think’ & /\ipa{ʂɻ̍˧˥}/ ‘full’\\ 
				\ipa{ʐ} & \multicolumn{2}{c}{\textit{no contrasts observed}}\\
				\lspbottomrule
			\end{tabularx}}
			\label{tab:someexamplesillustratingthephonemiccontrastbetweenandafterretroflexfricativesandaffricates}
		\end{table}
		
		
		Under the present analysis, the rhyme /\ipa{ɻ̍}{\kern2pt}/ has a~nasal counterpart, /\ipa{ɻ̍̃}{\kern2pt}/. Only two items
		have been found so far: ‘bone’, /\ipa{ɻ̍̃˥}/, which also has the meaning of
		‘stem’; and ‘helpless, impoverished, troubled’, /\ipa{ɻ̍̃{\kern1pt}˥}/. Notation as a~nasal rhotic vowel,
		/\ipa{ə̃˞}{\kern1.5pt}/ or /\ipa{œ̃˞}{\kern1.5pt}/, could also be acceptable; the symbol /\ipa{ɻ̍̃}{\kern2pt}/ was chosen because the
		rhymes transcribed as /\ipa{ɻ̍}{\kern2pt}/ and /\ipa{ɻ̍̃}{\kern2pt}/ have phonetic similarities and plausibly stand in
		a~relation of structural opposition, the one oral, the other nasal.
		
		The word for ‘bone’, here transcribed as /\ipa{ɻ̍̃˥}/, already puzzled earlier investigators. It is transcribed phonetically as [\ipa{ʔɱɹ}]
		and analyzed phonemically as /\ipa{ŋv̩ɹ}/ by \citet[25]{fu1983}. In this phonetic
		transcription, the sequence of two symbols is not to be understood as a~succession of two sounds. Here, as in his other publications, Fu Maoji uses the symbol /\ipa{ɹ}/ to indicate a~rhotic quality
		of the vowel. Phonemic analysis as /\ipa{ŋv̩ɹ}/ does appear to be an~attractive
		solution: nasality would be due to a~nasal onset, and rhoticity to a~rhotic, fricative rhyme,
		/\ipa{v̩ɹ}/. However, in the attested syllable /\ipa{ŋv̩}/, {coarticulation} between the initial and
		rhyme does not go so far as to result in complete coalescence: the syllable retains two parts, a~nasal onset
		[\ipa{ŋ}], with complete oral closure, and a~rhyme [\ipa{ṽ̩}], which is partly nasalized phonetically but without oral
		closure. A~syllable that is identical with /\ipa{ŋv̩}/ except for a~feature of rhoticity would be
		expected to begin similarly by a~nasal consonant (with complete oral closure). For this reason, this
		syllable is not analyzed here as a~rhotic counterpart to /\ipa{ŋv̩}/ (with nasal onset and oral
		rhyme), but as a~monophonemic syllable consisting simply of a~nasal rhyme. As for its analysis as
		/\ipa{ɻ̍̃}{\kern2pt}/ or /\ipa{ṽ̩}/, both options are open, since there is no opposition between these on
		onsetless syllables. On the basis of phonetic considerations, I believe that notation as /\ipa{ɻ̍̃ }/
		is more adequate synchronically.
		
		
		\subsubsection{Potential for the creation of new syllabic consonants and monophonemic syllables}
		\label{sec:potentialforthecreationofnewsyllabicconsonantsandmonophonemicsyllables}
		
		In syllables of simple CV (consonant+vowel) segmental structure, the consonant and vowel paired
		together become strongly coarticulated; their features tend to be realized over the syllable as
		a~whole. %Numerous examples are found in the \ili{Yi} (Lolo) branch of \il{Sino-Tibetan}Sino"=Tibetan languages. While the
%		impetus for \isi{monosyllabicization} can safely be hypothesized to have come from Old \il{Sinitic}Chinese, which
%		influenced~-- directly or indirectly~-- languages of the \il{Sino-Tibetan}Sino"=Tibetan, \il{Tai-Kadai}Tai"=Kadai, Hmông"=Miên and
%		Austroasiatic families, segmental depletion has reached a~more extreme stage in \ili{Yi} than within
%		\ili{Sinitic} itself (as pointed out by \citealt{haudricourt1991}). 
		In \ili{Naish} languages, {coarticulation}
		in CV monosyllables tends to create compact units that become less and less tractable to
		a~straightforward analysis into two distinct phonemes, until the syllable becomes monophonemic.
		
		In addition to apicalized vowels (\sectref{sec:apicalizedvowels}), syllabic nasals can
		also be interpreted in this light. “In various Loloish languages some or all of the nasals occur as
		syllabics. In most such cases the \is{comparative method (historical linguistics)}{diachronic} source is syllables with a~nasal initial and a~high
		vowel; sometimes one dialect has nasal syllabics where others have nasals plus a~high vowel. This
		could be called rhyme"=gobbling” (\citealt[150]{bradley1989}; see also
		\citealt[8]{bjorverud1998}). Yongning Na reveals an~intermediate stage: the syllable /\ipa{mv̩}/ is
		phonetically realized as [\ipa{m̩}] except in careful (hyperarticulated) speech, but /\ipa{nv̩}/
		and /\ipa{ŋv̩}/ are pronounced as [\ipa{nʋ̩}] and [\ipa{ŋʋ̩}] respectively, retaining an~oral portion
		after the initial nasal. 
%		Other syllables that tend towards articulation as one single sound include
%		/\ipa{kʰɯ}/, which is often devoiced when it carries a~L tone: [\ipa{kʰɯ̥}]. The entire syllable is
%		realized as a~[\ipa{kʰ}] that adopts the lip and tongue configuration of the vowel /\ipa{ɯ}/. 
		In the
		present state of the language, these are simply phonetic variants; they are mentioned to illustrate
		the potential for the further evolution of highly \is{phonological erosion}eroded syllables in Yongning Na.
				

		\subsection{Close vowels}
		\label{sec:closevowels}
		
		\tabref{tab:ie} shows that the close vowels /\ipa{i}/ and /\ipa{e}/ 
		contrast after dental fricatives and affricates,
		e.g.~/\ipa{dzi˩}/ ‘to sit’ vs.\ /\ipa{dze˩}/ ‘to fly’, /\ipa{tsi˩}/ ‘to boil’ vs.\ /\ipa{tse˩}/ ‘to
		lock’, /\ipa{tsʰi˥}/ ‘classifier for animal skins’ vs.\ /\ipa{tsʰe˥}/ ‘salt’, and /\ipa{si˥}/ ‘wood’
		vs.\ /\ipa{se˥}/ ‘to walk’. The syllable /\ipa{zi}/ has only been found in /\ipa{tsʰi˧zi\#˥}/ ‘highland
		barley’ and /\ipa{lɑ˧zi˥}/ ‘painter’; instances of /\ipa{ze}/ are more numerous. 
		
		\begin{table}%[t]
			\caption{Distribution of the close vowels /\ipa{i}/ and /\ipa{e}/.}
			\begin{tabularx}{\textwidth}{  l@{\hspace{12mm}} Q Q }
				\lsptoprule
				initial & \ipa{i}  & \ipa{e}\\\midrule
				Ø & \ipa{ʝi˥} ‘ox’ & --\\
				\ipa{b p pʰ} & \ipa{bi˥} ‘snow’ & --\\
				\ipa{m} & \ipa{mi˧} ‘wound’ & --\\
				\ipa{d t tʰ} & \ipa{di˧˥} ‘to hunt’ & --\\
				\ipa{dz} & \ipa{dzi˩} ‘to sit’ & \ipa{dze˩} ‘to fly’\\
				\ipa{ts} & \ipa{tsi˩} ‘to boil’ & \ipa{tse˩} ‘to lock’\\
				\ipa{tsʰ} & \ipa{tsʰi˥} ‘\textsc{clf}.animal skins’ & \ipa{tsʰe˥} ‘salt’\\
				\ipa{s} & \ipa{si˥} ‘wood’ & \ipa{se˥} ‘to walk’\\
				\ipa{z} & \ipa{tsʰi˧zi\#˥} ‘highland barley’ & \ipa{ze˩mi˩} ‘niece’\\
				\ipa{n} & \ipa{ni˥} ‘amaranth’ & \ipa{-ne} ‘as, like’\\
				\ipa{l} & \ipa{li˧\textsubscript{a}} ‘to look’ & \ipa{le˧-} \textsc{accomplished}\\
				\ipa{ɬ} & \ipa{ɬi˥} ‘to rest’ & --\\
				\ipa{dʑ tɕ tɕʰ} & \ipa{tɕi˥} ‘to shake’ & --\\
				\ipa{ɕ ʑ} & \ipa{ɕi˧} ‘rice’ & --\\
				\ipa{ɲ} & \ipa{ɲi˥} ‘to listen’ & --\\
				\ipa{ɖ ʈ ʈʰ} & \ipa{ʈi˩\textsubscript{a}} ‘to get up’ & --\\
				\ipa{ɖʐ ʈʂ ʈʂʰ} & -- &  \ipa{ʈʂe˥} ‘earth’\\
				\ipa{ʐ ʂ} & -- &  \ipa{ʐe˥} ‘arrow’\\
%				\ipa{ɭ} & -- & --\\
				\ipa{ʁ ɻ} & -- & --\\
				\ipa{g k kʰ} & \ipa{gi˥} ‘to owe’ & --\\
				\ipa{h} & \ipa{hi˩˥} ‘rain’ & --\\
				\lspbottomrule
			\end{tabularx}
			\label{tab:ie}
		\end{table}
		
		After the dental nasal /\ipa{n}/ and the dental lateral /\ipa{l}/, the close vowels /\ipa{i}/ and
		/\ipa{e}/ are marginally contrastive. The syllable /\ipa{ne}/ is only found in a~grammatical
		morpheme, /\ipa{-ne}/ ‘as, like’, which appears in the interrogative /\ipa{qʰɑ˩ne˩}/ ‘how’, the
		manner demonstratives /\ipa{ʈʂʰɯ˧ne˧-ʝi˥}/ (proximal) and /\ipa{tʰv̩˧ne˧-ʝi˥}/ (distal), and
		constructions such as /\ipa{tɕʰɤ˧ɲi˧-ne˧-ʝi˥}/ ‘every day; repeatedly, all the time’. The
		recognition of the morpheme /\ipa{ne}/ was delayed by the fact that the highly frequent expression
		/\ipa{ʈʂʰɯ˧ne˧-ʝi˥}/ ‘thus, in this way’ (occurring over 500 times in 25 texts) is pronounced very close to [\ipa{ʈʂʰɯ˧ni˧˥}], and hence
		was initially transcribed as \ipa{$\ddagger${\kern2pt}ʈʂʰɯ˧ni˧˥}. (This adverb is analyzed further in the section devoted to on"=glides, \sectref{sec:smoothphoneticonsets}.) Another marginal case where it appears reasonable to posit a~/\ipa{e}/ vowel distinct from /\ipa{i}/
		is in combination with initial /\ipa{l}/. The syllable /\ipa{li}/ is common, appearing in about
		thirty words. A~[\ipa{le}] syllable appears in the {accomplished} \is{prefixes}prefix /\ipa{le˧-}/, and in
		derived items, such as [\ipa{njɤ˧le˧gv̩\#˥}] ‘daytime’, which is perceived by consultant F4 as meaning literally ‘the
		day is flowing/going by’, as shown in (\ref{ex:daytime}).\footnote{Assuming that the etymological glossing in (\ref{ex:daytime}) is correct, the
			change of the first syllable from /{\kern1pt}\ipa{ɲi}/ to /\ipa{njɤ}/ remains to be accounted for.}
		
			\begin{exe}
				\ex
				\label{ex:daytime}
				\ipaex{njɤ˧le˧gv̩\#˥}\\
				\gll ɲi˥ le˧- gv̩˧\textsubscript{c}\\
				day		\textsc{accomp}		to\_flow/to\_go\_by\\
				\glt ‘daytime’, interpreted by consultant F4 as having the literal meaning ‘the
				day is flowing/going by’
			\end{exe}
		
		In F4's idiolect, the second syllable of ‘daytime’ is perceived as the {accomplished} \is{prefixes}prefix, and therefore does not constitute evidence for /\ipa{le}/ as
		an~attested syllable in nouns. Moreover, the form /\ipa{njɤ˧le˧gv̩\#˥}/ is only one of many avatars of the word for ‘daytime’ among Na dialects, some of which have /\ipa{ɬi}/ and not /\ipa{le}/ as a~middle syllable (/\ipa{ɲi33-ɬi31 ku33}/ in \citealt[297]{lidz2010}). It appeared interesting to dwell on this example nonetheless, as it provides an insight into processes whereby marginal combinations of initials and rhymes can enter the lexicon.
		
		To sum up, the opposition of /\ipa{e}/ and /\ipa{i}/ is restricted to syllables that have a~dental initial (most examples have a~fricative or affricate). This is one of the many cases
		where a~phonemic opposition is found in highly restricted contexts; in \is{Praguian phonology}Praguian terms, these
		constitute extreme cases of \isi{neutralization}. This issue is taken up in the discussion of the
		inventory of syllables, in \sectref{sec:commentsabouttheinventoryofsyllables}.
		
		Recognition of the opposition between /\ipa{i}/ and /\ipa{e}/ after dental fricatives and affricates was delayed by the fact that, in this context, both /\ipa{i}/ and /\ipa{e}/ are apicalized, which makes their phonetic difference a~very fine
		one. Apicalization is stronger for /\ipa{ɯ}/ than for /\ipa{i}/. Another difference is that the lips
		are stretched for /\ipa{si}/, /\ipa{dzi}/, /\ipa{tsi}/ and /\ipa{tsʰi}/. In the first
		transcriptions, the contrast between the two apicalized vowels /\ipa{i}/ and /\ipa{ɯ}/ was
		overlooked, and the syllables above were mistakenly transcribed as $\ddagger${\kern2pt}\ipa{sɯ} (for ‘wood’),
		$\ddagger${\kern2pt}\ipa{zɯ} (for ‘barley’), $\ddagger${\kern2pt}\ipa{dzɯ} (for ‘to sit’), $\ddagger${\kern2pt}\ipa{tsɯ} (for ‘to boil’) and
		$\ddagger${\kern2pt}\ipa{tsʰɯ} (for the classifier for animal skins). 
		
		In the many contexts where the opposition
		between /\ipa{i}/ and /\ipa{e}/ is neutralized, the transcription follows their phonetic
		realization, which is closer to [\ipa{e}] after retroflex fricatives and affricates, and to
		[\ipa{i}] in all other contexts (after bilabials, velars, uvulars, retroflex stops and fricatives, laterals,
		nasals, alveolopalatals, and glottals).
		
		As for close back vowels, [\ipa{o}] and [\ipa{u}] are only contrastive after initial /\ipa{h}/. Examples of /\ipa{ho}/ include /\ipa{tɑ˧ho˧}/ ‘together’, /\ipa{ho\#˥}/
		‘porridge, gruel’,	/\ipa{ho\#˥}/ ‘partridge’, /\ipa{qo˩ho˧˥}/ ‘wicker box’, /\ipa{ho˧˥}/ ‘to sip’, /\ipa{ho˩\textsubscript{a}}/ ‘correct’, and \mbox{/\ipa{-ho˩}/} \textsc{desiderative}. Examples of /\ipa{hu}/ are fewer;
		they include /\ipa{hu˧mi˥\$}/ ‘stomach’, /\ipa{hu˥}/ ‘to wait’, and /\ipa{hu˧˥}/ ‘to miss, to long
		for’. Phonetically, there is
		stronger friction in the initial for /\ipa{ho}/ than for /\ipa{hu}/, which could be approximated phonetically as
		[\ipa{hu}] vs.\ [\ipa{χo}]. This suggests an alternative phonemic analysis, dispensing with the opposition between /\ipa{u}/ and /\ipa{o}/,  positing instead an opposition between /\ipa{h}/ and /\ipa{χ}/, and rewriting /\ipa{hu}/ and /\ipa{ho}/ as /\ipa{ho}/ and /\ipa{χo}/. The phonetic difference in vowel quality appears more salient than the difference in the initial, however, hence the choice to interpret the vowel difference as phonemic. 
		
		Another way to reinterpret the syllable [\ipa{hu}] and eliminate the /\ipa{u}/ phoneme altogether would be to grant phonemic status to initial [\ipa{f}], analyzing the syllables [\ipa{fv̩}], [\ipa{hu}] and [\ipa{ho}] as /\ipa{fv̩}/, /\ipa{hv̩}/ and /\ipa{ho}/, respectively (Roselle Dobbs, p.c.\ 2016): see \sectref{sec:theglottalfricativeandthesound}. 
		
		After all initials other than /\ipa{h}/, there is no opposition between close and close"=mid rounded back vowels,
		[\ipa{o}] and [\ipa{u}]. Phonetic realizations are close to [\ipa{o}] after dentals, velars and
		uvulars, and more often close to [\ipa{u}] after the other
		consonants. In all the contexts where the opposition is neutralized, the notation chosen is /\ipa{o}/: this appeared less cumbersome than notation as an \isi{archiphoneme} /\ipa{O}/ (or /\ipa{U}/). 
		
		The two sounds [\ipa{o}] and [\ipa{u}] may have stronger phonemic status in the speech of younger
		speakers, whose increasing proficiency in \il{Mandarin!Standard}Standard {Mandarin} makes them familiar with a~phonemic
		/\ipa{u}/ (contrasting with /\ipa{oʷ}/ and /\ipa{ʷo}/; the \textit{Pinyin} transcription of these three
		vowels is: \textit{u}, \textit{ou}, \textit{uo}). \il{Mandarin!Southwestern}Southwestern {Mandarin} does not exert direct
		pressure in this direction, however, since /\ipa{u}/ is fricativized in this dialect of {Mandarin}. For
		instance, the word /\ipa{tsʰu}/ ‘vinegar’ is pronounced [\ipa{tsʰv̩}]. This word, which is in common
		use in Yongning, is accordingly pronounced as [\ipa{tsʰv̩˩˥}] in the speech of the older speaker F4.
		
		
		\subsection{A neutral vowel:  /\ipa{ə}/}
		\label{sec:aneutralvowel}
		
		A~clarification is necessary concerning the use of the phonetic symbol /\ipa{ə}/. In their \citeyear{heetal1985} book about \ili{Naxi}, which includes a~word list for Yongning Na, He Jiren \& Jiang Zhuyi use the symbol /\ipa{ə}/ for two different vowels: a~back unrounded vowel, /\ipa{ɤ}/, realized as [\ipa{ɣɤ}] in an onsetless syllable; and a~neutral vowel, /\ipa{ə}/, which always constitutes a~syllable on its own, harmonizes with the following syllable’s vowel, and is realized with an initial glottal stop \citep[130]{michaud2013b}.\footnote{A~likely origin for this confusion is the use of /\ipa{ə}/ as the official phonetic equivalent of the letter \textit{e} of the \textit{Pinyin} romanization of Standard {Mandarin}: /\ipa{ə}/ was used for a~back unrounded vowel. The ‘ram’s horn’ vowel symbol, /\ipa{ɤ}/, or the turned v, /\ipa{ʌ}/, constitute more felicitous choices than the neutral vowel /\ipa{ə}/, in view of phonetic realizations, which are “\ipa{ɤ}-like or \ipa{ʌ}-like” \citep[42]{association1949}. Linguists trained in mainland China in the first decades of the People's Republic of China tended to base their transcriptions on this system, reproduced in dictionaries and textbooks until the turn of the twenty"=first century. Things have now changed, thanks to more accurate descriptions distinguishing a~vowel /\ipa{ɤ}/ from a~vowel /\ipa{ə}/, such as \citet[110]{waisumetal2003}.} This confusing usage was adopted for the official phonetic transcription for \ili{Naxi}, which remains influential.\footnote{In his manuscript lexicographic notes, He Jiren used the turned v, /\ipa{ʌ}/, and not the neutral vowel, /\ipa{ə}/, but the symbol /\ipa{ə}/ was reintroduced when the data was edited for publication as a~dictionary \citep{heetal2011}.} 
		
		In the present system, the symbol /\ipa{ə}/ is only used to transcribe the {interrogative} particle, and the first syllable of some lexical words,
		where it can plausibly be analyzed as a~\is{prefixes}prefix. In particular, it is present in kinship terms referring to one's elders, such as
		/\ipa{ə˧mi˧}/ ‘mother’, realized as [\ipa{e˧mi˧}]; /\ipa{ə˧mɑ˧}/ ‘mother (\textsc{vocative})’,
		[\ipa{ɑ˧mɑ˧}]; and /\ipa{ə˧v̩˧˥}/ ‘uncle’, [\ipa{ɤ˧v̩˧˥}]. Realizations are close to [\ipa{æ}]
		before /\ipa{æ}/ and apicalized allophones of /\ipa{ɯ}/; to [\ipa{ɛ}] before /\ipa{i}/, /\ipa{ĩ}/
		and /\ipa{e}/; to [\ipa{ɑ}] before /\ipa{ɑ}/ and /\ipa{wɤ}/; and to [\ipa{ɤ}] before /\ipa{ɤ}/,
		/\ipa{o}/, /\ipa{ɯ}/, /\ipa{jɤ}/, /\ipa{jo}/ and /\ipa{ɻ̍}{\kern2pt}/. Before /\ipa{v̩}/, the neutral vowel
		/\ipa{ə}/ harmonizes differently depending on whether there is an~intervening consonant or not: it
		is realized close to [\ipa{ɤ}] when immediately followed by /\ipa{v̩}/, as in /\ipa{ə˧v̩˧˥}/ ‘uncle’
		(phonetic approximation: [\ipa{ɤ˧v̩˧˥}]), and it is close to [\ipa{æ}] when the /\ipa{v̩}/ is preceded
		by a~consonant, as in /\ipa{ə˧pʰv̩˧}/ ‘mother’s uncle; male ancestor of the second generation’ and
		/\ipa{ə˧mv̩˩}/ ‘elder sibling’ (phonetic approximation: [\ipa{æ˧pʰv̩˧}] and [\ipa{æ˧mv̩˩}]).\footnote{This phonemic
			analysis of the kinship {prefix} was suggested by Roselle Dobbs.}
		
		Note, however, that phonetic (incomplete) vowel harmony is not restricted to the vowel transcribed
		as /\ipa{ə}/. This topic is taken up in \sectref{sec:anoteonvowelharmony}.
		
		No phonetic difference could be found between the [\ipa{ɑ}] realization of /\ipa{ə}/ before
		/\ipa{ɑ}/, on the one hand, and the realization of the /\ipa{ɑ}/ vowel phoneme. For instance, the
		clan name [\ipa{ɑ˧lɑ˧}] is phonemicized as /\ipa{ə˧lɑ˧}/, but an~interpretation as /\ipa{ɑ˧lɑ˧}/
		cannot be ruled out. 
		
		\subsection{Nasal rhymes}
		\label{sec:nasalrhymes}
		
		\subsubsection{Nasal rhymes after the glottal /\ipa{h}/}
		\label{sec:nasalrhymesaftertheglottal}
		
		Yongning Na has a~relatively large inventory of nasal rhymes: it comprises /\ipa{ĩ}/, /\ipa{ṽ̩}/,
		/\ipa{õ}/, /\ipa{w̃ɤ}/, /\ipa{æ̃}/ and /\ipa{ɑ̃}/, and the syllables /\ipa{ɻ̍̃}{\kern2pt}/ and /\ipa{w̃æ}/. The
		first six are found after /\ipa{h}/, where they contrast neatly with their non"=nasal counterparts:  /\ipa{hi}/--/\ipa{hĩ}/,  /\ipa{hv̩}/--/\ipa{hṽ̩}/,  /\ipa{ho}/--/\ipa{hõ}/,  /\ipa{hwɤ}/--/\ipa{hw̃ɤ}/,  /\ipa{hæ}/--/\ipa{hæ̃}/, and  /\ipa{hɑ}/--/\ipa{hɑ̃}/. Examples are provided in \tabref{tab:examplesofinitialsyllablesthatarepartofacorrelationofnasality}.
		
		\begin{table}%[t]
			\caption{Examples of  /\ipa{h}/-initial syllables that are part of a~correlation of nasality.}
			{\renewcommand{\arraystretch}{1.35}
				\begin{tabularx}{\textwidth}{ Q P{80mm} }
					\lsptoprule
					oral rhyme & nasal rhyme\\ \midrule
					/\ipa{hi˥}/ ‘tooth’  & /\ipa{hĩ˥}/ ‘man’; /\ipa{hĩ˧˥}/ ‘to stand’\\ 
					/\ipa{hv̩˧}/ ([\ipa{fv̩˧}]) ‘to like’ & /\ipa{hṽ̩˩}/ ‘red’; /\ipa{hṽ̩˥}/ ‘hair’; /\ipa{nv̩˧hṽ̩˩}/ ‘kidney bean’; /\ipa{dʑi˧hṽ̩˥\$}/ ‘clothes’; /\ipa{hṽ̩˧{$\sim$}hṽ̩˧}/ ‘to stir"=fry’\\ 
					/\ipa{ho˧˥}/ ‘to sip’; /\ipa{ho˥}/ ‘to wait’ & /\ipa{hõ˧˥}/ ‘eight’; /\ipa{hõ˧}/ ‘to go (\textsc{imperative})’\\ 
					/\ipa{hwɤ˩}/ ‘to pass over, to hand over’ & /\ipa{hw̃ɤ˩}/ ‘late’\\ 
					/\ipa{hæ˧}/ ‘Chinese’; /\ipa{hæ˧˥}/ ‘lime’; /\ipa{hæ˧pɤ˧}/ ‘plait’  & /\ipa{hæ̃˧}/ ‘wind’; /\ipa{hæ̃˩}/ ‘gold’\\ 
					/\ipa{hɑ˥}/ ‘food’ & /\ipa{hɑ̃˧˥}/ ‘night’\\
					\lspbottomrule
				\end{tabularx}}
				\label{tab:examplesofinitialsyllablesthatarepartofacorrelationofnasality}
			\end{table}
			
			Diachronically, these syllables illustrate a~process of transfer of nasality from
			a~syllable"=initial consonant cluster to a~following vowel. This process is attested in several
			languages of Asia. In Kam"=Sui (\il{Tai-Kadai}Tai"=Kadai family), Sandong Sui lost the stop part of the original
			cluster: the stop+nasal clusters *\ipa{km-}, *\ipa{kn-}, *\ipa{tn-} and *\ipa{kɲ-} merged with the
			preglottalized *\ipa{ˀm-}, *\ipa{ˀn-} and *\ipa{ˀɲ-} initials. The latter are preserved in Sui,
			e.g.~/\ipa{ˀma¹}/ ‘vegetables’ and /\ipa{ˀma³}/ ‘flexible’, both corresponding to a~proto"=Kam"=Sui *\ipa{ˀm}
			initial \citep[251--252]{ferlus1996c}. Lakkia preserved the initial stop, while the nasal underwent
			lenition, nasalizing the following vowel in the process, e.g.~/\ipa{kũːi}/ ‘bear’, from a~stop+nasal cluster *\ipa{km-}. Northern Sui dialects (Pandong \zh{潘洞}
			and Yang’an \zh{阳安}) illustrate a~possibility for the later evolution of glottal+nasal onsets. Distinctive nasality is transferred onto the following vowel, and only the glottal remains, yielding
			[\ipa{ʔṼ}] or [\ipa{h̰Ṽ}]; the entire syllable is nasal, including the initial glottal sound
			\citep[176]{haudricourt1967}. This is exactly parallel to the facts of Yongning Na and other \ili{Naish}
			languages, as brought out in
			\tabref{tab:comparativevocabularyforfivewordsinrgyalrongandinnaxinaandlaze}. (This table is taken up from
			a~cross"=linguistic (\is{panchronic phonology}{panchronic}) analysis of historical transfers of nasality between consonantal
			onset and vowel: \citet{michaudetal2012b}.) \ili{Japhug} is a~{conservative} \il{Sino-Tibetan}Sino"=Tibetan language that preserves a~broad range of initial clusters; after correspondences among \ili{Naish} languages have been established, comparison with such {conservative} languages suggests hypotheses for fleshing out reconstructions \citep[470-471]{jacquesetal2011}.
			
			\begin{table}%[t]
				\caption{Comparative data pointing to the development of nasality in {Naish} from earlier */\ipa{rN-}/ onsets.}
				\begin{tabularx}{\textwidth}{ l@{\hspace{10mm}} l@{\hspace{10mm}} Q Q l@{\hspace{10mm}} }
					\lsptoprule
					& \ili{Japhug} & Fengke \ili{Naxi} & Yongning Na & \ili{Laze}\\ \midrule
					red & \ipa{ɣɯrni} & \ipa{hỹ˩} & \ipa{hṽ̩˩} &  --\\
					to stand & \ipa{rma} & \ipa{hỹ˩˧} & \ipa{hĩ˧˥} & \ipa{hĩẽ˥}\\
					person & \ipa{tɯ-rme} & \ipa{hĩ˧} & \ipa{hĩ˥} & \ipa{hĩ˧}\\
					body hair & \ipa{tɤ-rme} & \ipa{hṽ̩˥} & \ipa{hṽ̩˥} & \ipa{hṽ̩˩}\\
					to stir"=fry & \ipa{rŋu} & -- & \ipa{hṽ̩˧{$\sim$}hṽ̩˧} & --\\
					two & \ipa{ʁnɯs} & \ipa{ɲi˩˧} & \ipa{ɲi˥} & \ipa{ɲi˧}\\
					\lspbottomrule
				\end{tabularx}
				\label{tab:comparativevocabularyforfivewordsinrgyalrongandinnaxinaandlaze}
			\end{table}
			
			%% %Table 4.
			%% \begin{table}%[t]
			%% \caption{Comparative data pointing to the development of nasality in \ili{Naish} from earlier  /\ipa{*rN-}/ onsets.}
			%%   \begin{tabularx}{\textwidth}{ l l l l l Q Q }
			%% \lsptoprule
			%% 	 & red & to stand & person & body hair & to stir"=fry & two\\ \midrule
			%% 	\ili{Japhug} & \ipa{ɣɯrni} & \ipa{rma} & \ipa{tɯ-rme} & \ipa{tɤ-rme} & \ipa{rŋu} & \ipa{ʁnɯs}\\ 
			%% 	Fengke \ili{Naxi} & \ipa{hỹ˩} & \ipa{hỹ˩˧} & \ipa{hĩ˧} & \ipa{hṽ̩˥} & -- & \ipa{ɲi˩˧}\\ 
			%% 	Yongning Na & \ipa{hṽ̩˩} & \ipa{hĩ˧˥} & \ipa{hĩ˥} & \ipa{hṽ̩˥} & \ipa{hṽ̩˧{$\sim$}hṽ̩˧} & \ipa{ɲi˥}\\ 
			%% 	\ili{Laze} & -- & \ipa{hĩẽ˥} & \ipa{hĩ˧} & \ipa{hṽ̩˩} & -- & \ipa{ɲi˧}\\
			%%    \lspbottomrule
			%% \end{tabularx}
			%% \label{tab:comparativevocabularyforfivewordsinrgyalrongandinnaxinaandlaze}
			%% \end{table}
			
			\tabref{tab:comparativevocabularyforfivewordsinrgyalrongandinnaxinaandlaze} brings out
			a~\is{comparative method (historical linguistics)}correspondence between the /\ipa{h̰Ṽ}/ syllables of Yongning Na and etyma with initial
			/\ipa{rm-}/ or /\ipa{rn-}/ in \ili{Japhug}. This leads to the hypothesis that the /\ipa{h̰Ṽ}/ syllables found in \ili{Naish} languages originate in earlier *CNV syllables. The hypothesis that the nasal vowels
			found in some \il{Sino-Tibetan}Sino"=Tibetan languages could be attributed to the influence of syllable"=initial
			nasals was already expressed by \citet{huang1991a}. On the other hand, no hypotheses had been
			proposed until \citet{michaudetal2012b} as to which specific sequences of phonemes were involved in the change.
			
			The last example in \tabref{tab:comparativevocabularyforfivewordsinrgyalrongandinnaxinaandlaze},
			‘two’, illustrates the preservation in \ili{Naxi}, Na and \ili{Laze} of nasals that originate in onsets other
			than */\ipa{rN-}/. It appears reasonable to hypothesize that the *CN- onsets that led to vowel
			nasalization all went through a~*/\ipa{sN-}/ stage. (For general phonetic reflections on this topic see
			\citealt[233]{ohalaetal1993}: “children learning \ili{English} sometimes pronounce target \textit{sm} and
			\textit{sn} clusters as voiceless nasals”.)
			
			In front of nasal rhymes, /\ipa{h}/ is nasalized; the lowered velum prevents the buildup of
			intra"=oral pressure required for a~strong friction noise. Since the entire syllable is nasalized,
			another option for phonemic analysis would be to posit a~nasalized glottal fricative,
			/\ipa{h̰}/, contrasting with plain /\ipa{h}/. A~review about “possible and impossible segments” \citep{walkeretal1999} includes /\ipa{h̰}/ among the set of “possible segments” because it is firmly attested as a~phoneme contrasting with /\ipa{h}/ in at least two languages: Kwangali (\ili{Bantu}; \citealt[132-133]{ladefogedetal1996}) and Seimat (Austronesian; \citealt{blust1998}). In the case of Yongning Na, the decision to interpret nasality as a~characteristic of the vowel rather than the consonant is based on the observation that nasal vowels are also found in a~small set of
			syllables that do not have an~initial glottal fricative.
			
			
			\subsubsection{Onsetless nasal syllables:  /\ipa{õ}/,  /\ipa{æ̃}/ and  /\ipa{ɻ̍̃}/}
			\label{sec:onsetlessnasalsyllables}
			
			Among onsetless nasal syllables, /\ipa{õ}/ contrasts with /\ipa{o}/, /\ipa{æ̃}/ with /\ipa{æ}/, and
			/\ipa{ɻ̍̃}{\kern2pt}/ with /\ipa{ɻ̍}{\kern2pt}/. Examples of /\ipa{o}/ (realized with a~glide onset, [\ipa{wo}]) include
			/\ipa{wo˥}/ ‘hard’; examples of /\ipa{õ}/ include /\ipa{õ˧˥}/ ‘(one)self’, /\ipa{õ˧ʈʂwɤ˧}/
			‘mosquito’ and /\ipa{õ˩dv̩˧˥}/ ‘wolf’. Synchronically, initial glottalization contributes to a~clear
			phonetic contrast between /\ipa{o}/, realized [\ipa{wo}], and /\ipa{õ}/, realized
			[\ipa{ʔõ}]. Examples of /\ipa{ɻ̍̃}{\kern2pt}/ and /\ipa{ɻ̍}{\kern2pt}/ were presented above
			(\sectref{sec:rhoticrhymes}). The syllable /\ipa{æ̃}/ has a~glottalized onset: [\ipa{ʔæ̃}]. A
			non"=nasalized syllable [\ipa{ʔæ}] is also found in the system, and is analyzed as /\ipa{æ}/,
			i.e.\ recognizing the phonemic status of both /\ipa{æ}/ and /\ipa{æ̃}/. Examples of /\ipa{æ̃}/ are
			relatively numerous: it is found in words such as ‘chicken’, ‘bronze’, ‘plough’ and ‘soul’. Among
			disyllables, examples include /\ipa{æ̃˩zɯ˩}/ ‘agate’ vs.\ /\ipa{æ˩gv̩˩}/ ‘ard’;\footnote{The ard, also known as scratch plough, is the type of ploughing implement used in Yongning. Unlike the plough, the ard has a~symmetrical share that traces a shallow furrow but does not invert the soil \citep{haudricourtetal1955}.} /\ipa{æ̃˩-mi˧}/ ‘hen’
			and /\ipa{æ˩mi˧-ʁwɤ\#˥}/ ‘the village of A-Mi’ form a~quasi"=minimal pair.
			
			From an~evolutionary point of view, the glottal onset of [\ipa{ʔõ}] and [\ipa{ʔæ̃}] may be due to the
			same phenomenon of hardening of empty onsets which results in the presence of an~initial [\ipa{ʁ}]
			in words such as /\ipa{ʁwɤ˥}/ ‘village’, corresponding to \ili{Naxi} /\ipa{wɤ˧}/ (more about this
			phenomenon below,
			\sectref{sec:apresentationofonglideswithahypothesisaboutadiachroniconsetofhardeningofinitialglides}). 
			
			
			\subsubsection{Phonemic analysis of the onsetless nasal syllable  [\ipa{w̃æ}]}
			\label{sec:phonemicanalysisoftheonsetlessnasalsyllable}
			
			The syllable realized as [\ipa{ʔw̃æ}] appears in a~single word: ‘to swell, to
			inflate (e.g.~the belly is swollen)’, [\ipa{ʔw̃æ˧}]. It does not have a~non"=nasal counterpart. In
			the absence of an~opposition with an~oral syllable [\ipa{wæ}], nasality and initial glottalization
			might be considered the product of implementation rules, rather than as part of the phoneme’s
			definition. From this perspective, phonemic analysis could be simply /\ipa{wæ}/. However, the {Mandarin}
			syllables \textit{wa}, \textit{wan} and \textit{wang} (as in \textit{wáng} \zh{王}, a~common family
			name) are pronounced as [\ipa{ʁwæ}] by consultant F4, rather than as [\ipa{ʔw̃æ}]. If the underlying,
			phonemic form of syllables such as ‘to swell’ were simply /\ipa{wæ}/, one would expect these {Mandarin}
			forms to fall into the /\ipa{wæ}/ category (phonetically [\ipa{ʔw̃æ}]), not into the /\ipa{ʁwæ}/
			category. The transcription used here for ‘to swell’ is therefore /\ipa{w̃æ}/, granting phonological status to
			nasality, which plays a~role in the vowel system of Yongning Na, but not to initial glottalization,
			which does not. To sum up the analysis: the syllable /\ipa{w̃æ}/, realized as [\ipa{ʔw̃æ}], contrasts
			with /\ipa{wæ}/, realized as [\ipa{ʁwæ}] (see
			\sectref{sec:theinitialvoiceduvularfricativeasaphonemicizedemptyonsetfiller}).
			
			
			\subsubsection{The syllable /\ipa{ĩ}/}
			\label{sec:thesyllable}
			
			The syllable /\ipa{ĩ}/ appears in one single word, the \is{interjections}interjection ‘Yes!’, ‘Okay!’, used as
			a~response to an~instruction given by a~person having authority (either one’s elder, or another
			person detaining authority): /\ipa{ĩ˧}/ (phonetically: [\ipa{ʔĩ˧}]). This item is described as
			having a~mid tone, but could also be analyzed as a~toneless \is{interjections}interjection generally realized on
			a~level pitch.
			
			
			\subsubsection{The nasal rhyme /\ipa{õ}/ appears as a~variant in two Tibetan loanwords}
			\label{sec:thenasalrhymeappearsasavariantontwotibetanloanwords}
			
			Two \ili{Tibetan} loanwords preserve a~nasal rhyme /\ipa{õ}/ alternating with oral realizations:
			/\ipa{o}/. The examples are /\ipa{tsʰo˧pæ\#˥}/{\kern2pt}\ipa{≈}{\kern2pt}/\ipa{tsʰõ˧{\allowbreak}pæ\#˥}/ ‘head of
			caravan’ (compare Written \ili{Tibetan} \textit{tshong} \textit{pa} ‘merchant’), and a~formula of blessing:
			/\ipa{lɑ˧mɑ˧-ko˧ʈʂʰo˧}/{\kern2pt}\ipa{≈}{\kern2pt}/\ipa{lɑ˧mɑ˧-ko˧ʈʂʰõ˧}/.\footnote{Nathan Hill (p.c.\ 2016) proposes that the Na word /\ipa{ko˧ʈʂʰo˧}/{\kern2pt}\ipa{≈}{\kern2pt}/\ipa{ko˧ʈʂʰõ˧}/ is a~shortened form of \textit{dkon mchog gsum}, Sanskrit \textit{triratna}: ‘the Three Jewels: Buddha, dharma (the teaching), and saṅgha (the monastic order, or community)’. Na /\ipa{lɑ˧mɑ˧}/ is clearly from \textit{bla ma} ‘spiritual leader, great teacher (at the monastery)’; it is not obvious why this term is paired with the Three Jewels into a~formula of blessing: \textit{bla ma dkon mchog gsum}, ‘the Lama and the Three Jewels’ or ‘the Lama's Three Jewels’. Nathan Hill's suggestion is that ‘lama’ may be used here to stand for the Three Roots \textit{(tsa sum)} of the {Tibetan} Buddhist tradition. These Three Roots, \textit{lama}, \textit{yidam} and \textit{khandroma}, are more abstract and recondite than the Three Jewels, and hence difficult to remember for people who have not received training at a~monastery. Inside the set of three, the teacher, \textit{lama}, is not only the first: it is also impersonated by a~living person (a~high"=ranking monk), which makes the notion readily understandable. This could explain why \textit{lama} rather than one of the other two Roots would appear in a~(somewhat garbled) formula of blessing.}
			On the
			other hand, the word for ‘monastery’, /\ipa{go˧bɤ˩}/ (compare Written \ili{Tibetan} \textit{dgon pa}), does not have a~nasalized \is{variants}variant \ipa{$\ddagger${\kern2pt}gõ˧bɤ˩}. Tibetan \is{loanwords}borrowings would definitely warrant in"=depth study, distinguishing different layers and different degrees of integration into the Na language.
			

			\subsection{The open vowels  /\ipa{ɑ}/ and  /\ipa{æ}/, and the vowel  /\ipa{ɤ}/}
			\label{sec:theopenvowelsandandthevowel}
			
			Na, \ili{Naxi} and \ili{Laze} all have an~opposition between two open vowels. For \ili{Naxi}, some authors transcribe
			them as /\ipa{a}/ vs.\ /\ipa{æ}/ (for instance \citealt{heetal1989}), others as /\ipa{ɑ}/ vs.
			/\ipa{æ}/ \citep{fangetal1995}, others still as /\ipa{ɑ}/ vs.\ /\ipa{a}/ \citep{fu1981, heetal1985, pinsonetal2012}. These choices are all fine from a phonemic point of view, since the relevant structural fact is the distinction between two low vowels. But use of the symbol /\ipa{a}/ can cause some confusion for linguists who consult data from several sources, hence my decision to adopt a~notation as /\ipa{ɑ}/ vs.\ /\ipa{æ}/ for all \ili{Naish}
			languages, avoiding the symbol /\ipa{a}/.\footnote{For similar reasons of consistency across descriptions of {Naish} languages, the vowel symbol /\ipa{ɯ}/ is used for Yongning Na, {Naxi} and {Laze} despite slight differences in pronunciation. In Yongning Na, this vowel is articulated less to the back than the {Naxi} vowel
			transcribed with the same symbol. From a~phonetic perspective, one could argue in favour of using
			the symbol /\ipa{ɨ}/ for Yongning Na.} Phonetically, the vowel transcribed as /\ipa{ɑ}/ is clearly a~back vowel in \ili{Naxi}, close
			to cardinal [\ipa{ɑ}], whereas in Na and \ili{Laze} it is closer to [\ipa{a}]. I had not noticed any
			cross"=language difference in the pronunciation of the vowel transcribed as /\ipa{æ}/ until a~native speaker of Lijiang \ili{Naxi}, Mu Yanjuan \zh{木艳娟}, explained to me (p.c.\ 2016) that in her view the front low vowel in \ili{Naxi} is clearly [\ipa{a}] and notation as [\ipa{æ}] is definitely inappropriate.
			
			In Yongning Na, the sounds [\ipa{ɑ}] and [\ipa{ɤ}] are contrastive in some contexts. Although it is clear from their
			present"=day distribution that the opposition used to be neutralized in most contexts, this opposition has been reinforced through processes of \is{gap-filling}gap"=filling. (About the notion of structural \is{gap-filling}gap"=filling, see \sectref{sec:theinitialvoiceduvularfricativeasaphonemicizedemptyonsetfiller}.)
			
			After labials, only [\ipa{ɤ}] and not [\ipa{ɑ}] is found (as shown in \tabref{tab:InNucl}), except (i)~in loanwords and (ii)~in contexts where the vowel is likely to have been changed by vowel harmony. Thus, the combination /\ipa{mɑ}/
			(contrasting with /\ipa{mɤ}/) has been introduced by \ili{Tibetan} borrowings, as in the names
			/\ipa{ɖɯ˩mɑ\#˥}/, /{\kern1pt}\ipa{ɲi˩mɑ\#˥}/ and /\ipa{gv̩˧mɑ˧}/, and in /\ipa{lɑ˧mɑ˧}/ ‘priest, lama’,
			/\ipa{mɑ˩ɳɯ˧-do˥bv̩˩}/ ‘Mani wall’, and /\ipa{mɑ˧pʰv̩˧}/
			‘butter’ (the second syllable is a~Na adjective meaning ‘white’). Somewhat paradoxically, the first
			syllable of /\ipa{mɑ˧pʰv̩˧}/
			‘butter’ is perceived as semantically and phonemically different from /\ipa{mɤ˩}/
			‘animal fat’, even though ‘butter tea’ is /\ipa{mɤ˩ɬi˩}/. The combination /\ipa{mɑ}/ is also found
			in the term of address /\ipa{ə˧mɑ˧}/, ‘Mum, mother’. Finally, there is a~clan name,
			/\ipa{lɑ˩mɑ˩}/, which also contains this syllable; it may be \ili{Tibetan} in origin. In the remaining two items, the syllable /\ipa{mɑ}/ may result from vowel harmony:
			/\ipa{mɑ˩dzɑ˩}/ ‘solid ink’, and /\ipa{mɑ˧tsɑ˥}/ ‘origin, cause’. With bilabial stops, /\ipa{bɑ˩}/ appears in /\ipa{bɑ˩lɑ˩}/ ‘jacket, upper outer garment’, where it could result from vowel harmony, and in the final particle /\ipa{bɑ˩˥}/, which serves a~function comparable to that of {question}"=tags in \ili{English} and belongs to the language's expressive margins. The syllable /\ipa{pɑ}/ is found in /\ipa{pɑ˧tɕɤ˧}/ ‘plantain (a~species of banana)’, which is a~\is{loanwords}loanword from {Mandarin} (\textit{bājiāo} \zh{芭蕉}).
			
			On the other hand, /\ipa{ɑ}/ and /\ipa{ɤ}/ are contrastive after affricates and fricatives,
			e.g.~/\ipa{tsɑ˧}/ ‘busy’ vs.\ /\ipa{tsɤ˧}/ ‘greedy’ and /\ipa{sɑ˥}/ ‘hemp, \textit{Cannabis sativa}’
			vs.\ /\ipa{sɤ˥}/ ‘blood’.
			
			Only  [\ipa{ɤ}] is found after velars, and only  [\ipa{ɑ}] after uvulars, e.g.~/\ipa{kɤ˧˥}/ ‘to knock (on the door)’ vs.\ /\ipa{qɑ˧˥}/ ‘to help’.
			
			After /\ipa{h}/, there is an~opposition, witness /\ipa{hɑ˩}/
			‘to open (one’s eyes)’ vs.\ /\ipa{hɤ˩}/ ‘to dry beside or over a~fire’.
			
			
			
			\subsection{A note on phonetic diphthongization}
			\label{sec:diphthongization}
			
			There is some diphthongization in the phonetic realization of the Na phonemes written as simple vowels, /\ipa{i}/, /\ipa{e}/, /\ipa{æ}/, /\ipa{ɯ}/, /\ipa{ɤ}/, /\ipa{o}/ and /\ipa{ɑ}/. There is also some formant movement during the syllabic consonants /\ipa{ɻ̍}{\kern2pt}/ and /\ipa{v̩}/. Formants in Na are not as stable as in, say, Northern (Parisian) \ili{French}, whose {conservative} varieties provide a~canonical illustration of the four-way contrast in openness among the simple vowels /\ipa{i}/, /\ipa{e}/, /\ipa{ɛ}/, /\ipa{a}/ and /\ipa{u}/, /\ipa{o}/, /\ipa{ɔ}/, /\ipa{ɑ}/. 
			
			Phonetic diphthongization in Na is at its clearest for the vowel /\ipa{e}/, realized phonetically close to [\ipa{ej}], as noted by \citet[63, 96]{lidz2010}. This realization of /\ipa{e}/ is a~point of similarity with \ili{Naxi}, a~language in which the simple vowels are otherwise stable. Phonetic diphthongization appears to date back to a~century at least, as Bonin, an explorer of the turn of the twentieth century, transcribed the \ili{Naxi} word /{\kern1pt}\ipa{ɲi˧}/ ‘two’ as \textit{ngié}, and Bacot, another explorer, transcribed the \ili{Naxi} name of the city of Lijiang, /\ipa{ʝi˧gv̩˧dy˩}/, with a~\textit{yé} for /\ipa{ʝi˧}/ \citep[3]{bacot1913}. Bacot indicates that “each [simple] vowel and its diphthongs [diphthongized variants] are interchangeable” \citep[28]{bacot1913},\footnote{\textit{Original text:} chaque voyelle et ses diphtongues sont interchangeables.} suggesting that diphthongization was not considerable \citep{michaudetal2010}.
			

			\subsection{A note on vowel harmony}
			\label{sec:anoteonvowelharmony}
			
			Anticipatory vowel harmony (‘right"=to"=left’ harmony) is a~salient phonetic tendency in connected
			speech in all the \ili{Naish} languages studied so far. For instance, in \ili{Laze}, /\ipa{ʝi˧dy˧}/ ‘family’ is
			sometimes realized close to [{\kern1.3pt}\ipa{ʝy˧dy˧}]. This is not a~phonological phenomenon: vowel oppositions
			on the first syllable of disyllables are not neutralized. But this phonetic tendency becomes
			lexicalized on some disyllables. For instance, in some \ili{Naxi} dialects, including A"=sher, ‘pigswill’ is /\ipa{bu˩-hɑ˧}/ (from /\ipa{bu˩}/ ‘pig’
			and /\ipa{hɑ˧}/ ‘food’); in other dialects, including
			Nda"=le, it has become /\ipa{bɑ˩-hɑ˧}/. (This phenomenon is reported by \citealt[11]{he1985} but
			without mention of the dialects concerned.)
			
			In detail, this phenomenon is highly language- and dialect"=specific. Among the three \ili{Naish}
			languages studied so far (\ili{Laze}, Na and \ili{Naxi}), \ili{Naxi} is least prone to the \isi{lexicalization} of such
			phenomena, and \ili{Laze} most prone to it. The \isi{lexicalization} of vowel harmony sometimes goes hand in
			hand with other processes such as the voicing of intervocalic voiceless consonants. A~typical
			example in \ili{Laze} is /\ipa{ʂie˧-lie˧mie˧}/ ‘seventh month’, from /\ipa{ʂɯ˧}/ ‘seven’ and
			/\ipa{ɬie˧mie˧}/ ‘month’: in addition to the change in the vowel of the first syllable, note the
			voicing of /\ipa{ɬ}/ to~/\ipa{l}/.
			
			In Na, the phonetic tendency towards regressive vowel harmony is especially strong for the vowel
			/\ipa{æ}/. For instance, /\ipa{ŋwɤ˧}/ ‘five’ plus the monetary unit /\ipa{mæ˩\textsubscript{a}}/ is pronounced
			[\ipa{ŋwæ˧-mæ˥}] ‘five yuan’, although in careful (hyperarticulated) speech the pronunciation is
			[\ipa{ŋwɤ˧-mæ˥}]. Also, \isi{function words} are more susceptible to vowel harmony~-- this is one of the
			manifestations of their overall weaker realization. Here are two examples.
			
			\begin{enumerate}[label=(\roman*)]
				\item  The
				{accomplished} \is{prefixes}prefix /\ipa{le˧-}/ is realized close to [\ipa{læ}] when the vowel of the following
				verb is /\ipa{æ}/ or an~apical vowel.
				
				\item  The {negation} \is{prefixes}prefix
				/\ipa{mɤ˧-}/ is realized close to [\ipa{mɑ}] when the vowel of the following verb is an~apical
				vowel.
				
			\end{enumerate}
			
			Vowel harmony for these two prefixes is so strong that at one point (2008--2011) I transcribed them as /\ipa{lə˧-}/ and  /\ipa{mə˧-}/, with a~neutral vowel. But other morphemes such as the {durative} \is{prefixes}prefix /\ipa{tʰi˧-}/ also exhibit some vowel harmony. Pending the results of future phonetic studies of the relative degree of vowel harmony for a~broad range of grammatical words, I chose to stop granting special phonemic status to the vowel in the
			{accomplished} and {negation} prefixes. This topic is well worth investigating further. It links up with topics of {diachronic} phonology: vowel harmony has the potential to introduce new phonotactic combinations into the phonological system, through sporadic \isi{lexicalization} of harmonized phonetic forms. For
			instance, syllables containing a~dental stop followed by /\ipa{æ}/, such as /\ipa{læ}/, /\ipa{tæ}/
			and /\ipa{tʰæ}/, are scarce in Na, and most of them appear to originate in vowel harmony (see
			\sectref{sec:retroflexstopsandaffricates}).
			

			\section{Initial consonants}
			\label{sec:consonants}

			\subsection{On"=glides}
			\label{sec:smoothphoneticonsets}
			\label{sec:apresentationofonglideswithahypothesisaboutadiachroniconsetofhardeningofinitialglides}
			
			The high vowels /\ipa{i}/, /\ipa{ɯ}/ and /\ipa{o}/ have phonetic on"=glides: they are realized as
			[{\kern1.3pt}\ipa{ʝi}], [\ipa{ɣɯ}] and [\ipa{wo}]. 
			
			The amount of friction in the realization of /\ipa{i}/ led to the transcription of the on"=glide as a~fricative, [{\kern1.3pt}\ipa{ʝ}], rather than an approximant, [\ipa{j}]. The latter is used in the transcription of the complex rhymes /\ipa{jo}/, /\ipa{jɤ}/ and /\ipa{jæ}/, which are realized with less friction.
			
			The phonetic realization of the syllable /\ipa{ɯ}/ could also
			be transcribed with an~approximant initial: [\ipa{ɰɯ}]; this is the choice made in a~dictionary of
			\ili{Naxi} \citep{pinsonetal2012}, a~language where the phonemic analysis of this syllable is the same as in Yongning Na and the phonetic
			realization does not sound any different to me.
			
			The phonetic on"=glides in [{\kern1.3pt}\ipa{ʝi}], [\ipa{ɣɯ}] and [\ipa{wo}] appeared salient enough to be
			indicated in transcriptions, departing slightly from considerations of notational economy. However, they do not involve as much change in vowel quality (formant movement) as is found for the
			phonemic on"=glides in the complex rhymes, /\ipa{jo}/, /\ipa{jɤ}/, /\ipa{jæ}/, /\ipa{wɤ}/, /\ipa{w̃ɤ}/,
			/\ipa{wæ}/, /\ipa{w̃æ}/, /\ipa{wɑ}/ and /\ipa{w̃ɤ}/. Onsetless
			syllables tend to coalesce phonetically with the preceding syllable inside polysyllabic words and tightly"=knit polysyllabic
			expressions. For instance, I mistakenly believed for several years that the manner adverbials ‘in this
			way, thus’ and ‘in that way’ were /\ipa{ʈʂʰɯ˧ni˧˥}/ and /\ipa{tʰv̩˧ni˧˥}/, respectively; these are in
			fact /\ipa{ʈʂʰɯ˧ne˧ ʝi˥}/ and /\ipa{tʰv̩˧ne˧ ʝi˥}/, made up of a~manner {adverbial} (/\ipa{ʈʂʰɯ˧ne˧˥}/, /\ipa{tʰv̩˧ne˧˥}/) followed by the verb ‘to do’, /\ipa{ʝi˥}/ (as mentioned in \sectref{sec:closevowels}). Realization of the \is{trisyllables}{trisyllabic}
			structure of these phrases did not come from phonetic evidence, but from their tonal
			behaviour. Consider example (\ref{ex:thisishowtheyusedtosing}), shown here as initially transcribed:
			\begin{exe}
				\ex
				\label{ex:thisishowtheyusedtosing}
				\ipaex{ʈʂʰɯ˧ni˧˥ {\kern2pt}|{\kern2pt} gwɤ˩-ɲi˥ mæ˩!}\\
				\gll ʈʂʰɯ˧ni˧˥ gwɤ˩ -ɲi˩ mæ˧\\
				in\_this\_way to\_sing \textsc{certitude} \textsc{obviousness}\\
				\glt ‘This is how [people] used to sing!’ (Caravans.51, 53, 57)
			\end{exe}
			
			If the notation /\ipa{ʈʂʰɯ˧ni˧˥}/ were correct, it should be possible to link up the {adverbial}
			together with the following verb phrase into a~single \isi{tone group}, as \ipa{$\ddagger${\kern2pt}ʈʂʰɯ˧ni˧ gwɤ˥-ɲi˩-mæ˩}.\footnote{About the division of the utterance into tone groups, see Chapter~\ref{chap:toneassignmentrulesandthedivisionoftheutteranceintotonegroups}.} But this \is{stylistics}stylistic option is not open: the manner \is{demonstratives}demonstrative cannot project a~H tone onto a~following
			syllable, as would be expected if it carried MH\# tone. The solution to this puzzle is that the {adverbial} phrase ‘thus, in this way’ is \is{trisyllables}{trisyllabic} (/\ipa{ʈʂʰɯ˧ne˧-ʝi˥}/), and remains so despite the strong phonetic {coarticulation} between its last two
			syllables. The correct notation of
			(\ref{ex:thisishowtheyusedtosing}) is as in (\ref{ex:thisishowtheyusedtosingGOOD}). 
			
\begin{exe}
	\ex
	\label{ex:thisishowtheyusedtosingGOOD}
	\ipaex{ʈʂʰɯ˧ne˧ ʝi˥ {\kern2pt}|{\kern2pt} gwɤ˩-ɲi˥ mæ˩!}\\
	\gll ʈʂʰɯ˧ne˧˥ 		ʝi˥		gwɤ˩ -ɲi˩ mæ˧\\
	in\_this\_way to\_do		to\_sing \textsc{certitude} \textsc{obviousness}\\
	\glt ‘This is how [people] used to sing!’ (Caravans.51, 53, 57)
\end{exe}

			Now moving on to other vowels: /\ipa{æ}/ and /\ipa{ɑ}/ may begin either with a~glottal stop (hard phonetic onset) or with soft phonation:
			[\ipa{ɦæ}]{\kern2pt}\ipa{≈}{\kern2pt}[\ipa{ʔæ}], [\ipa{ɦɑ}]{\kern2pt}\ipa{≈}{\kern2pt}[\ipa{ʔɑ}]. Concerning the choice of one type of onset or the other, a~study of glottal stops before word"=initial vowels in American \ili{English} concludes that “full glottal stops [\ipa{ʔ}] are predicted overwhelmingly by prominence and \isi{phrasing}” \citep[20]{garellek2012}, and reviews a~list of factors that would constitute a~good starting"=point for a~future phonetic study of this topic in Yongning Na:
			
			\begin{quotation}
				{\dots} the factors that promote the occurrence of word"=initial glottalization ({\dots}) may be segmental, lexical, prosodic, or sociolinguistic. In \ili{English}, segmental factors include hiatus (V\#V) environments and word"=initial back vowels are found to glottalize more frequently than non"=back vowels. As for lexical factors, content words exhibit more frequent glottalization than \isi{function words}. Sociolinguistically, women are known to use glottalization more than men. Prosodically, the presence of stress and/or a~pitch accent on the word"=initial vowel, as well as a~larger {juncture} with the preceding word, are known to promote glottalization. Researchers working on other languages have found additional factors that promote the occurrence of word"=initial glottalization, including presence of a~preceding pause as well as speech rate and low vowel quality for German. \citep[1-2]{garellek2012}\footnote{Citations have been removed from this quotation.}
			\end{quotation}
			
			Unlike in \ili{Naxi}, where /\ipa{ɤ}/ is realized as /\ipa{ɣɤ}/, in Yongning Na the vowel /\ipa{ɤ}/ never
			constitutes a~syllable on its own. As explained in \sectref{sec:theopenvowelsandandthevowel}, the
			opposition between /\ipa{ɤ}/ and /\ipa{ɑ}/ is restricted to a~few contexts; synchronically, it may
			be considered to be neutralized in onsetless syllables.
			
			No distinct onset portion is found for  /\ipa{v̩}/, realized simply as  [\ipa{v̩}]. Likewise, the rhyme /\ipa{ɻ̍}{\kern2pt}/ is not
			separated from the preceding rhyme by a~glide or glottal stop. For instance, in /\ipa{bv̩˧ɻ̍\#˥}/
			‘fly (the insect)’ the /\ipa{v̩}/ and /\ipa{ɻ̍}{\kern2pt}/ follow each other: [\ipa{bv̩ɻ̍}{\kern2pt}].
			
			The following paragraph discusses syllables that canonically begin with a~glottal stop. 
				
			
			\subsection{Initial glottal stop}
			\label{sec:hardphoneticonsets}
			
			
			The rhymes /\ipa{ə}/ and /\ipa{ɻ̍̃}{\kern2pt}/ begin with a~phonetic glottal stop when said \is{form!in isolation}in isolation, and
			also word"=medially in hyperarticulated realizations: e.g.~/\ipa{ʂæ˩ɻ̍̃˩}/ ‘bone’ is realized as
			[\ipa{ʂæ˩ʔɻ̍̃˩˥}] in careful speech, and as [\ipa{ʂæ˩ɻ̍̃˩˥}] in casual speech.
			
			The morpheme /\ipa{u˧}/
            %\rephrase{,}{ is } % Felicitous suggestion by Sebastian Nordhoff, to achieve a typographically balanced line
            is an {associative} first person \is{pronouns}pronoun that appears in the pronouns /\ipa{u˧=ɻ̍˩}/ ‘my clan, my people’ and /\ipa{u˧=ɻæ˩}/ ‘us (as opposed to them)’. It
            %\rephrase{,}{.    It }% Felicitous suggestion by Sebastian Nordhoff, to achieve a typographically balanced line
            is pronounced with an initial glottal stop, hence [\ipa{ʔu˧=ɻ̍˩}] and [\ipa{ʔu˧=ɻæ˩}]. The morpheme /\ipa{u˧}/ is the only instance of this syllable that has been observed; but one example is enough to require recognition of this syllable as distinct from the syllable realized as [\ipa{wo}]{\kern2pt}\ipa{≈}{\kern2pt}[\ipa{wu}], found in words such as //\ipa{wo˥}// ‘hard’, //\ipa{wo˩kɤ\#˥}// ‘swing’ and //\ipa{wo˩˥}// ‘turnip leaves’. Several solutions are open here, such as the following three (which could be combined in various ways).
			
			\begin{enumerate}[label=(\roman*)]
				\item The syllable [\ipa{ʔu}] could be phonemicized as /\ipa{u}/, and [\ipa{wo}]{\kern2pt}\ipa{≈}{\kern2pt}[\ipa{wu}] as /\ipa{wu}/, granting phonemic status to the initial glide. 
				\item The syllable [\ipa{ʔu}] could be phonemicized as /\ipa{ʔu}/, and [\ipa{wo}]{\kern2pt}\ipa{≈}{\kern2pt}[\ipa{wu}] as /\ipa{u}/, granting phonemic status to the initial glottal stop. 
				\item The syllable [\ipa{ʔu}] could be phonemicized as /\ipa{u}/, and [\ipa{wo}]{\kern2pt}\ipa{≈}{\kern2pt}[\ipa{wu}] as /\ipa{o}/, ascribing the different initials to phonotactic rules of assignment of empty"=onset fillers.
			\end{enumerate}
			
			The most important cue to this opposition appears to lie in the initial, and would suggest granting phonemic status to the initial glottal stop or the labial"=velar approximant. The provisional choice made here consists in considering the approximant as an empty"=onset filler, and the glottal stop as a~phonemic initial, hence its inclusion in \tabref{tab:theinitialsofyongningna}. Clearly, such cases of phonemic ambiguity hold potential for changes to the system; the example of initial /\ipa{ʁ}/, examined in the next paragraph, exemplifies the introduction of new consonants from phonemic reinterpretation of empty"=onset fillers.
			
			\subsection{Initial /\ipa{ʁ}/ as a~phonemicized empty"=onset filler}
			\label{sec:theinitialvoiceduvularfricativeasaphonemicizedemptyonsetfiller}
			
			
			The situation of /\ipa{wɤ}/, /\ipa{wæ}/, /\ipa{o}/, /\ipa{ɑ}/ and /\ipa{æ}/ is especially
			interesting. These rhymes can combine with an~initial voiced uvular fricative,
			/\ipa{ʁ}/. (Phonetically, this /\ipa{ʁ}/ is weakly articulated, and can be mistaken for /\ipa{w}/ in
			some hypo"=articulated tokens.) Some of the words at issue derive diachronically from
			onsetless syllables, such as ‘village’, /\ipa{ʁwɤ˧}/, corresponding to /\ipa{wɤ˧}/ in \ili{Naxi} and
			\ili{Laze}, and its \is{homophony}homophone ‘mountain’, /\ipa{ʁwɤ˧}/, likewise corresponding to /\ipa{wɤ˧}/ in \ili{Naxi}
			(where it means ‘hill, hillock’); others from syllables with an~initial velar or uvular cluster,
			such as ‘sword’, /\ipa{ʁæ˧mi˧}/, corresponding to \ili{Naxi} /\ipa{ŋgæ˩}/.\footnote{A process of hardening of
			initial glides has been reported in Zeluo {Ersu}, where /\ipa{w}-/ is sometimes pronounced with frication, as
			[\ipa{ɣʷ-}] \citep{sun1982}. Synchronically, among the last speakers of the language, there is variation between [\ipa{ʁ}] and [\ipa{w}] sounds \citep{chirkovaduoxu2014}.}
			
			The hardening of soft onsets should in principle result in the absence of any syllable realized as
			[\ipa{wɤ}], [\ipa{wæ}], [\ipa{o}], [\ipa{ɑ}] or [\ipa{æ}], since these became [\ipa{ʁwɤ}],
			[\ipa{ʁwæ}], [\ipa{ʁo}], [\ipa{ʁɑ}] and [\ipa{ʁæ}]. This is true for /\ipa{wæ}/: only [\ipa{ʁwæ}] is
			attested, not [\ipa{wæ}] (e.g.~/\ipa{ʁwæ˥}/ ‘left, leftward’, corresponding to \ili{Naxi} /\ipa{wæ˧}/ and
			\ili{Laze} /\ipa{væ˧}/). For the other syllables, however, there are oppositions between syllables with
			and without a~/\ipa{ʁ}/ onset: a~process of structural \is{gap-filling}gap"=filling has taken place. Progress in the
			etymological study of individual examples will be necessary to understand the various processes
			whereby onsetless syllables were reintroduced into the system. The hardening of empty"=onset fillers
			must date back a~relatively long way, judging from the number of onsetless syllables that currently
			exist: e.g., for /\ipa{wɤ}/, examples include the classifier for loads, /\ipa{wɤ˩\textsubscript{b}}/; the noun ‘serf, slave’, /\ipa{wɤ˧}/; the adverb ‘again, anew’, /\ipa{wɤ˩˥}/; the verbs ‘to depend on, to rely on’,
			/\ipa{wɤ˩\textsubscript{b}}/ and ‘to detour past, to bypass’,
			/\ipa{wɤ˩{$\sim$}wɤ˩}/; and the final exclamative particle /\ipa{wɤ˧}/, conveying obviousness.\footnote{The voiced uvular fricative /\ipa{ʁ}/ is not uncommon in this linguistic area,
				but with a~widely different phonemic status from one language to another. For instance, in {Lizu}
				\citep{chirkovaetal2012}, it is an~allophone of the voiced velar fricative /\ipa{ɣ}/ in front of
				/\ipa{ɐ}/ and /\ipa{wɐ}/, e.g.~/\ipa{ɣɐ˥˩}/ [\ipa{ʁɐ˥˩}] ‘needle’, /\ipa{ɣwɐ˥˩}/ [\ipa{ʁuɐ˥˩}] ‘to
				thunder’.}
			
			The change from proto"=\ili{Naish} *\ipa{j} to Na /\ipa{ʑ}/ may be part of the same process of hardening. ‘To
			sleep’, Na /\ipa{ʑi˧˥}/, corresponds to \ili{Naxi} /\ipa{ʝi˥}/ (phonemically a~simple /\ipa{i}/, to which
			an~empty"=onset"=filler gets added) and \ili{Laze} /\ipa{zi˩}/; the \is{comparative method (historical linguistics)}reconstruction proposed in \citet{jacquesetal2011} is *\ipa{jip}.
			
			
			\subsection{Velar and uvular stops}
			\label{sec:velaranduvularstops}
			
			
			Velar and uvular stops are in complementary distribution, except in front of  /\ipa{v̩}/,
			/\ipa{wɤ}/, and  /\ipa{o}/, where they are contrastive. Examples of the syllables  /\ipa{kv̩}/,
			/\ipa{qv̩}/,  /\ipa{kʰv̩}/,  /\ipa{qʰv̩}/,  /\ipa{ko}/,  /\ipa{qo}/,  /\ipa{kʰo}/ and  /\ipa{qʰo}/ are
			provided in (\ref{ex:velaranduvularstops}). A~single instance of  /\ipa{qi}/ (contrasting with  /\ipa{ki}/) has also
			been found.
			
\Hack{\newpage}

			\begin{exe}
				\ex \label{ex:velaranduvularstops}
				\begin{xlist}
					\extab
					\begin{tabularx}{112mm}{ P{14mm} P{27mm} P{15mm} Q }
						~~~/\ipa{qv̩}/ &  & ~~~/\ipa{kv̩}/ &\\
						\ipa{qv̩˩˥} & handle & \ipa{kv̩˥} & garlic\\ 
						\ipa{qv̩˧˥} & to frighten & \ipa{kv̩˧˥} & to be able to\\ 
						\ipa{qv̩˧ʈʂæ˧˥} & throat & \ipa{kv̩˧ʈʂɯ˥\$} & nail\\ 
						\ipa{mæ˧qv̩˩} & tail & \ipa{kv̩˧dʑɯ˧˥} & tent\\
					\end{tabularx}
					
					\Hack{\vspace*{.5\baselineskip}}  
					\extab
					\begin{tabularx}{112mm}{ P{14mm} P{27mm} P{15mm} Q }
						~~~/\ipa{qʰv̩}/ &  & ~~~/\ipa{kʰv̩}/ &\\
						\ipa{qʰv̩˧˥} & six & \ipa{mv̩˧kʰv̩˧˥} & smoke\\ 
						\ipa{qʰv̩˧} & horn & \ipa{kʰv̩˥} & to cut (grass)\\ 
						\ipa{qʰv̩˧} & hole & \ipa{kʰv̩˥} & dog\\ 
						\ipa{bv̩˩qʰv̩˩} & conch shell & \ipa{kʰv̩˧˥} & year\\
					\end{tabularx}
					
					\Hack{\vspace*{.5\baselineskip}}
					\extab
					\begin{tabularx}{112mm}{ P{14mm} P{27mm} P{15mm} Q }
						~~~/\ipa{qo}/ &  & ~~~/\ipa{ko}/ &\\
						\ipa{qo˥} & to love & \ipa{ko˥} & hill\\ 
						\ipa{qo˩ho˧˥} & bamboo box & \ipa{ko˩} & to bask\\ 
						\ipa{qo˩qɑ˩} & mountain pass & \ipa{ko˩ɖʐo˩} & flail\\ 
						\ipa{-qo˧} & inside & \ipa{mæ˩ko˥} & harness\\
					\end{tabularx}
					
					% \Hack{\newpage}
					\Hack{\vspace*{.5\baselineskip}}
					\extab
					\begin{tabularx}{112mm}{ P{14mm} P{27mm} P{15mm} Q }
						~~~/\ipa{qʰo}/ &  & ~~~/\ipa{kʰo}/ &\\
						\ipa{qʰo˧˥} & to kill & \ipa{kʰo˥} & to spread (e.g.~a~sheet)\\ 
						\ipa{qʰo˧lo˧} & wheel & \ipa{kʰo˧lo˧} & prayer wheel\\ 
						\ipa{qʰo˩tv̩˧˥} & tree stump & \ipa{tse˧kʰo˩} & sanctuary\\ 
						\ipa{qʰo˩mv̩˩} & straw hat & \ipa{hæ̃˧kʰo˧} & princess, young lady\\
					\end{tabularx}
					
					\Hack{\vspace*{.5\baselineskip}}
					\extab
					\begin{tabularx}{112mm}{ P{14mm} P{27mm} P{15mm} Q }
						~~~/\ipa{qi}/ &  & ~~~/\ipa{ki}/ &\\
						\ipa{qi˧qi˧} & originally, at first & \ipa{ki˧} & to give\\
					\end{tabularx}
				\end{xlist}
			\end{exe}
			
			This situation is unsurprising in areal context. In \ili{Lizu}, velar and uvular stops only contrast before
			/\ipa{o}/, e.g.~/\ipa{ko˨˧}/ ‘to beg’ vs.\ /\ipa{qo˥˩}/ ‘hole, pit’ \citep{chirkovaetal2012}. 
			
			From a~\is{comparative method (historical linguistics)}{diachronic} point of view, uvulars have various possible origins \citep[782-783]{sun2003a}; in view of cognates with uvular initials in Rgyalrong, a~{conservative} language, the current analysis is that Na uvulars have great historical depth \citep[492]{jacquesetal2011}.
			
			
			\subsection{Retroflex stops and affricates}
			\label{sec:retroflexstopsandaffricates}
			
			Yongning Na has (i)~an~opposition between dental and retroflex affricates, with a~high functional
			load, and (ii)~an~opposition between dental and retroflex stops and nasals, but only in front of
			/\ipa{i}/, /\ipa{æ}/, /\ipa{v̩}/ and /\ipa{o}/. Examples include: /\ipa{ʈʰi˩}/ ‘tired’ vs.
			/\ipa{tʰi˩}/ ‘to plane (wood)’; /\ipa{ʈi˩}/ ‘to get up’ vs.\ /\ipa{ti˩}/ ‘to knock, to tap
			(lightly)’; /\ipa{ʈæ˧bɤ˧}/ ‘Buddhist priest’ vs.\ /\ipa{tæ˧pv̩˩}/ ‘thin, skinny’; /\ipa{ɖo˧\textsubscript{a}}/ ‘to
			allow; ought; to have to’ vs.\ /\ipa{do˥}/ ‘to climb’. After retroflex consonants, /\ipa{o}/ is realized
			close to [\ipa{u}].
			
			The consonants transcribed as retroflex are articulated much less to the back than canonical
			retroflex sounds such as those of Tamil or Nepali \citep{khatiwada2009}; a~palatographic study may reveal that they are
			postalveolars rather than retroflexes. This is an area where there is clearly room for improvement of the International Phonetic Alphabet: separate symbols are needed for postalveolar and dental sounds, as emphasized by \citet[21-30]{ladefogedetal1996}. This is not merely an issue of fine detail in phonetic transcription, to be handled through the addition of diacritics: the absence of distinct symbols for postalveolars can lead to the widespread adoption of retroflex symbols for non"=dental sounds, creating no small amount of typological confusion and introducing a~bias in cross"=language studies of phonological inventories. In the absence of distinct International Phonetic Alphabet
			symbols for postalveolar stops, the provisional solution chosen here is to use IPA symbols for retroflex sounds. 
			
			
			\subsection{Laterals  /\ipa{l}/ and  /\ipa{ɬ}/, and retroflexes /\ipa{ɭ}/ and  /\ipa{ɻ }/}
			\label{sec:lateralsandandtheretroflexapproximant}
			
			
			The laterals /\ipa{l}/ and /\ipa{ɬ}/ are contrastive in Yongning Na, as demonstrated by pairs such
			as /\ipa{li˧\textsubscript{a}}/ ‘to look’ vs.\ /\ipa{ɬi˧}/ ‘month’ and /\ipa{lo˧˥}/ ‘thick’ vs.\ /\ipa{ɬo˧˥}/
			‘deep’. Phonetically, the voiced lateral /\ipa{l}/ has a~broad perimeter of allophonic \isi{variation}. It
			is realized as retroflex in front of /\ipa{ɯ}/, e.g.~in the classifier for round objects such as
			bowls and grains (also serving as generic classifier), /\ipa{ɭɯ˧\textsubscript{b}}/. The phonemic analysis for this
			syllable was only arrived at after the greatest hesitations: a~broad range of options was
			considered, including [\ipa{ɻɯ}], [\ipa{ɬɯ}], [\ipa{lv̩}], [\ipa{ly}], [\ipa{ɭ{\kern1pt}y}] and syllabic
			[\ipa{ɭ}{\kern2.5pt}] or [\ipa{ɹ}]. The entire syllable is articulated loosely: the initial is close to
			an~approximant, and the vowel quality is not precise, so that the syllable often resembles
			a~monophonemic [\ipa{ɭ}{\kern2.5pt}].
			
			In front of /\ipa{v̩}/, the voiced lateral /\ipa{l}/ is slightly retroflex. In all contexts,
			/\ipa{l}/ is accompanied by some friction. This characteristic is at its clearest before the high
			front vowel /\ipa{i}/, but it is also observed before open vowels, including /\ipa{ɑ}/ and
			/\ipa{æ}/. Phonetically, it may therefore be more adequate to transcribe this allophone as [\ipa{ɮ}]
			rather than [\ipa{l}]. Phonologically, it would be economical to consider that the two laterals are
			distinguished solely by the feature of voicing; this also argues in favour of a~notation as
			/\ipa{ɮ}/.
			
			The compromise choice made here for the sake of simplicity consists in using a~notation as /\ipa{l}/
			rather than /\ipa{ɮ}/. On the other hand, a~retroflex initial is used in the transcription of the
			syllable [\ipa{ɭɯ}] (phonemically: /\ipa{lɯ}/) to reflect what I perceive as a~great phonetic
			distance between the realization of /\ipa{l}/ in this context and in all others; it appeared better to keep the
			transcriptions close to the surface forms.
			
			These observations on the allophonic \isi{variation} of /\ipa{l}/ shed indirect light on the distribution
			of the retroflex approximant /\ipa{ɻ}{\kern2pt}/: this phoneme may well have originated as an~allophone of /\ipa{l}/
			which drifted to such a~phonetic distance that it opened a~structural gap that was later filled
			through borrowings and processes of vowel harmony. The present"=day /\ipa{ɻ}{\kern2pt}/ initial of Yongning Na
			only appears in the syllables /\ipa{ɻæ}/ and /\ipa{ɻwæ}/, which correspond to the syllables
			/\ipa{læ}/ and /\ipa{lwæ}/ in \ili{Laze}, e.g.~‘to shout, to cry’: Na /\ipa{ɻwæ˥}/, \ili{Laze} /\ipa{lwæ˧}/, and
			‘seed’: Na /\ipa{ɻæ˩˥}/, \ili{Laze} /\ipa{læ˩}/. Roselle Dobbs (p.c.\ 2013) indicates that in villages that
			belong to the area referred to in Na as /\ipa{lɑ˧tʰɑ˧-di˧˥}/, to the east of Lake Lugu, these items
			retain the lateral initial. Synchronically, /\ipa{ɻæ}/ contrasts with /\ipa{læ}/, but the latter
			only appears (i)~in borrowings, such as /\ipa{læ˧tsɯ˥}/ ‘chili peppers’, from Southwestern {Mandarin}
			\zh{辣子} [\ipa{la.tsɿ}], (ii)~in the {accomplished} \is{prefixes}prefix /\ipa{le˧-}/, whose phonetic
			realizations, determined by the vowel of the following verb, include [\ipa{læ}], and (iii)~in words
			where the /\ipa{æ}/ could have resulted from vowel harmony, e.g.~/\ipa{læ˧ʁæ˥}/
			‘raven’. (Regressive vowel harmony, which becomes sporadically lexicalized, is a~salient phonetic
			tendency in Na: see \sectref{sec:anoteonvowelharmony}.) As for the syllable /\ipa{ɻwæ}/, there
			exists no lateral counterpart /\ipa{lwæ}/.\footnote{A tantalizingly similar synchronic situation is found in {Lizu}, where the /\ipa{ɹ}/ phoneme only occurs before
			/\ipa{æ}/, /\ipa{ə}/, and /\ipa{wæ}/, e.g.~/\ipa{ɹæ˥˩}/ ‘yak’, /\ipa{ɹə˨˧}/ ‘to laugh’, and /\ipa{ɹwæ˨˧}/
			‘chicken’ \citep{chirkovaetal2012}. The diachronic origin seems different, however, as {Ersu} and {Duoxu} evidence rather points to an earlier *\ipa{r}, as discussed in \citet{yu2012}. Surface similarities in sound patterns tend to arise from areal convergence, as well as from cross"=linguistic ({panchronic}) regularities in phonological {erosion} processes.}
			
			It can therefore be hypothesized that present"=day
			/\ipa{ɻæ}/ and /\ipa{ɻwæ}/ originated\is{comparative method (historical linguistics)} in earlier *\ipa{læ} and *\ipa{lwæ}, which became
			phonetically closer to [\ipa{ɻæ}] and [\ipa{ɻwæ}], thus leaving these phonetic slots empty; the
			[\ipa{læ}] slot was then occupied by other syllables. 
			
			These {diachronic} reflections
			do not detract from the synchronic phonemic status of /\ipa{ɻ}{\kern2pt}/.
			
			
			\subsection{The glottal fricative /\ipa{h}/ and the sound  [\ipa{f}]}
			\label{sec:theglottalfricativeandthesound}
			
			Na has a~glottal fricative /\ipa{h}/. At an~earlier stage of the language, the sound [\ipa{f}] can
			be hypothesized to have been entirely absent, since early \ili{Mandarin} borrowings with initial [\ipa{f}]
			in the donor language were reinterpreted as having initial /\ipa{h}/. For instance, ‘method, solution’, \zh{办法}
			(Standard {Mandarin}: \textit{bànfǎ}) was borrowed as /\ipa{pæ˧˥hwɤ˧}/.
			
			The sound [\ipa{f}] appears in more recent layers of borrowings, however: e.g.~/\ipa{fæ˧}/ for ‘direction’, \zh{方}
			(Standard {Mandarin}: \textit{fāng}) and /\ipa{fɑ˩\textsubscript{a}}/ for ‘to ferment’, \zh{发(酵)}
			(Standard {Mandarin}: \textit{fā}). A~plausible scenario is
			that /\ipa{h}/ in front of oral rhymes came to be realized in Na with a~friction source at a~point
			in the vocal tract determined by the following vowel, e.g.~palatal before /\ipa{i}/ and
			labial"=dental before /\ipa{v̩}/, hence [\ipa{çi}] and [\ipa{fv̩}]. Once the sound [\ipa{f}] had thus been introduced into Na (on a~phonetic level), the way was paved for the introduction of [\ipa{f}]-initial
			loanwords. In the present state of the language, speakers of Yongning Na have no problem pronouncing a~[\ipa{f}] sound in front of any rhyme. This can be taken as evidence that the sound [\ipa{f}] is no longer perceived by the speakers as
			an~allophone of /\ipa{h}/. It is well"=known that allophones that drift far apart acquire psychophonetic independence from one another, witness the well"=documented cases of German [\ipa{ç}]
			and [\ipa{x}], and Standard {Mandarin} [\ipa{ɕ}] and [\ipa{x}]. As the psychological reality of the
			underlying unity among allophones wanes, the resistance against structural \is{gap-filling}gap"=filling decreases.
			
			In view of this situation, and also in order to keep the transcriptions close to the surface forms,
			the syllable [\ipa{fv̩}] is transcribed as such, rather than pushing phonemicization to an extreme and analyzing it as /\ipa{hv̩}/. Under a~flatly synchronic
			analysis that includes \ili{Mandarin} borrowings, the sound [\ipa{f}] needs to be granted phonemic
			status, hence its inclusion in \tabref{tab:theinitialsofyongningna}. 
			
			This could lead to a~modified treatment of the syllables [\ipa{fv̩}] and [\ipa{hu}]. The former could be phonemicized as /\ipa{fv̩}/, and the latter as /\ipa{hv̩}/; this would eliminate the /\ipa{u}/ phoneme altogether (Roselle Dobbs, p.c.\ 2016). This is one of the points of Yongning Na phonemics which are open to several analyses.
			
			The syllable [\ipa{çi}], phonemicized as /\ipa{hi}/, contrasts with /\ipa{ɕi}/.
			

			\section{Comments about the inventory of syllables}
			\label{sec:commentsabouttheinventoryofsyllables}
			
			The inventory of attested combinations of initials and rhymes provided at the outset of this Appendix (in Tables \ref{tab:InNucl} and \ref{tab:InNucl2}) reveals that numerous phonemic oppositions are found in highly restricted contexts
			in Yongning Na. A~similar situation is found in
			\ili{Naxi} \citep{michaud2006c}. The strict application of principles of
			\is{Praguian phonology}Praguian synchronic description leads to an~analysis of these
			phenomena as extreme cases of \textit{neutralization} of phonemic contrasts. For instance, in Yongning Na the opposition between nasal and
			oral vowels is neutralized in all contexts except after glottal
			initials. It may appear counter"=intuitive to speak of \is{neutralization|textbf}neutralization
			here: it is more usual to use this notion to describe cases
			where a~thoroughgoing contrast disappears in a~restricted environment,
			e.g.~in Trubetzkoy’s classical example: \ili{French} /\ipa{e}/ and /\ipa{ɛ}/ contrast
			only in open syllables, the opposition being neutralized in closed
			syllables.
			\begin{quotation}
				In {French} ({\dots}) an opposition between \ipa{e} and \ipa{ɛ} only occurs word"=finally in open syllables, e.g.~\textit{les} ‘definite article.\textsc{pl}’ vs.\ \textit{lait} ‘milk’ and \textit{allez} ‘to go.2\textsc{pl}’ vs.\ \textit{allait} ‘to go.3\textsc{sg.pst}’. In all other positions the occurrence
				of \ipa{e} and \ipa{ɛ} is predictable: \ipa{ɛ} occurs in closed syllables, \ipa{e} in open. These
				two vowels must thus be considered two phonemes in open"=syllable"=final position, and combinatory
				variants of a~single phoneme in all other positions. We call such oppositions
				\textit{neutralizable} oppositions, the positions in which the \isi{neutralization} takes place
				\textit{positions of neutralization}, and those positions where the opposition is relevant
				\textit{positions of relevance}.~\citep[78]{trubetzkoy1969}\footnote{Some modifications to the translation were made by Roselle Dobbs. \textit{Original text}: Im
					Französischen kommen aber \ipa{e} und \ipa{ɛ} nur im offenen Auslaute als Glieder einer
					phonologisch"=distinktiven Opposition vor (\textit{les"=lait}, \textit{allez"=allait}); in den
					übrigen Stellungen ist das Vorkommen von \ipa{e} und \ipa{ɛ} mechanisch geregelt (in gedeckter
					Silbe \ipa{ɛ}, in ungedeckter \ipa{e}), so daß diese zwei Vokale nur im offenen Auslaut as zwei
					Phoneme, in den übrigen Stellungen dagegen als kombinatorische Varianten eines einzigen Phonems
					gewertet werden müssen. Der phonologische Gegensatz ist also im Französischen in gewissen
					Stellungen a{\kern2pt}u{\kern2pt}f{\kern2pt}g{\kern2pt}e{\kern2pt}h{\kern2pt}o{\kern2pt}b{\kern2pt}e{\kern2pt}n. Solche Oppositionen nennen wir
					a{\kern2pt}u{\kern2pt}f{\kern2pt}\-h{\kern2pt}e{\kern2pt}b{\kern2pt}\-b{\kern2pt}a{\kern2pt}r; jene
					Lautstellungen, in denen die Aufhebung erfolgt,
					A{\kern2pt}u{\kern2pt}f{\kern2pt}\-h{\kern2pt}e{\kern2pt}\-b{\kern2pt}u{\kern2pt}n{\kern2pt}g{\kern2pt}s{\kern2pt}\-s{\kern2pt}t{\kern2pt}e{\kern2pt}l{\kern2pt}l{\kern2pt}u{\kern2pt}n{\kern2pt}g{\kern2pt}e{\kern2pt}n,
					jene, wo die Opposition relevant ist,
					R{\kern2pt}e{\kern2pt}l{\kern2pt}e{\kern2pt}v{\kern2pt}a{\kern2pt}n{\kern2pt}z{\kern2pt}\-s{\kern2pt}t{\kern2pt}e{\kern2pt}l{\kern2pt}\-l{\kern2pt}u{\kern2pt}n{\kern2pt}g{\kern2pt}e{\kern2pt}n. \citep[70]{trubetzkoy1939}}
			\end{quotation}
			
			Importantly, the term \textit{neutralization} should not be understood in a~dynamic sense, whereby the
			opposition once existed and later has been neutralized. It has a~static, flatly synchronic
			application (\citealt[257--259]{martinet1969}; \citeyear[87--89]{martinet1970}). It is not unusual for a~synchronic formulation to be the
			reverse image of a~{diachronic} perspective. For instance, describing the synchronic stage during which \il{Sinitic}Chinese contrasted three tones (A, B and C) on non"=obstruent"=final syllables, it can be said that
			the tonal opposition was neutralized on obstruent"=final syllables (described as belonging in
			a~fourth category: D), although this opposition never actually existed on these syllables.
			
			In \is{Praguian phonology}Praguian phonology, phonemes have relations of opposition with one another, and are marked for features as
			a~consequence of these relations. If the features shared by two phonemes are not found in any other
			phoneme, the opposition is bilateral and neutralizable; otherwise it is multi"=lateral and not
			neutralizable. Stated differently, only where there is opposition is a~feature contrastive. An
			opposition is neutralized whenever one member does not occur in a~specific environment. The product
			of \isi{neutralization} is referred to as an~\is{archiphoneme|textbf}archiphoneme. (For a~book"=length treatment of \isi{neutralization} from a~variety of theoretical perspectives, see \citealt{silverman2012}.)
			
			In phonological transcription, archiphonemes can be set in capitals, but this gets visually
			cumbersome in cases where positions of \isi{neutralization} are more numerous than
			positions of relevance. Notations using archiphonemes are also more abstract than notations
			containing phonetic symbols: interpreting them requires a~knowledge of the language's
			phonotactics. For these reasons, notations in terms of archiphonemes are not used in this volume.
			
			From a~dynamic point of view, gaps in the inventory of syllables provide structural hints about past
			evolutions and current tensions within the system.
			

			\subsection{Combinations of a~dental stop and  /\ipa{æ}/ vowel seem recent}
			\label{sec:combinationsofadentalstopandvowelareprobablyrecent}
			
			Combinations consisting of a~dental stop followed by /\ipa{æ}/ are scarce. The only example for
			/\ipa{dæ}/ is /\ipa{læ˧dæ˧qæ˥}/ ‘armpit’; the two examples for /\ipa{tæ}/ are /\ipa{tæ˧ɻæ˩}/ ‘Adam’s
			apple, oesophagus’ and /\ipa{tæ˧pv̩˩}/ ‘thin, skinny’; the only example for /\ipa{tʰæ}/ is
			/\ipa{tʰæ˧ɻæ˩}/ ‘book’. All of these except /\ipa{tæ˧pv̩˩}/ ‘thin, skinny’ can be explained as
			resulting from vowel harmony. Na /\ipa{tʰæ˧ɻæ˩}/ ‘book’ corresponds to \ili{Laze} /\ipa{tʰɑ˧ɹ˧}/ and \ili{Naxi}
			/\ipa{tʰe˧ɣɯ˧}/; the vowel \is{comparative method (historical linguistics)}correspondence /\ipa{e:æ:ɑ}/ is otherwise unattested, reinforcing the
			hypothesis that vowel harmony or some other \is{exceptions}exception"=causing force was at play here. Na
			/\ipa{tæ˧ɻæ˩}/ ‘Adam’s apple, oesophagus’ corresponds to Labai Na /\ipa{tɑ˧ɻ̍˧}/, again an~irregular
			\is{comparative method (historical linguistics)}correspondence, as the regular correspondences are simply \ipa{æ::æ} and \ipa{ɑ::ɑ}. 
			
			
			\subsection{A marginal combination: Dental stop plus /\ipa{ɤ}/}
			\label{sec:dentalstopplus}
			
			Few words contain a~dental stop plus /\ipa{ɤ}/; the only two attested combinations are /\ipa{dɤ}/ and
			/\ipa{tɤ}/ (no /\ipa{tʰɤ}/). There is an~extra"=distal \is{demonstratives}demonstrative: ‘way over there’, realized
			as /\ipa{dɤ˧˥-qo˧}/ or /\ipa{dɤ˥˧-qo˧}/, with the locative /\ipa{qo}/ as its second syllable (like
			in /\ipa{ʈʂʰɯ˧-qo˧}/ ‘here’ and /\ipa{tʰv̩˧-qo˧}/ ‘there’). The pitch of the first syllable reflects
			the intended distance: a~realization with a~mild rise, which could be transcribed as /\ipa{dɤ˧˥-qo˧}/, points to a~less distant place than a~realization with a~super"=high, decreasing pitch, which could be transcribed as
			/\ipa{dɤ˥˧-qo˧}/. The same phenomena are observed for /\ipa{dɤ˧˥tʰv̩˧qo˧}/{\kern2pt}\ipa{≈}{\kern2pt}/\ipa{dɤ˥˧tʰv̩˧-qo˧}/
			(same meaning, with added distal \is{demonstratives}demonstrative) and
			/\ipa{dɤ˧˥tʰv̩˧-gi\#˥}/{\kern2pt}\ipa{≈}{\kern2pt}/\ipa{dɤ˥˧tʰv̩˧-gi\#˥}/ ‘that side, way over there’ (for further details, see \sectref{sec:towardsthelossoflexicaltoneonsomegrammaticalwordsthroughhabitualintonationalmodifications}). The expressive load
			of these phrases goes a~great distance towards explaining \is{irregularities}oddities in the phonemes and tone of their
			first syllable, where the expressivity is at its strongest.
			
			All the other examples of syllables /\ipa{dɤ}/ and /\ipa{tɤ}/ are found with a~following /\ipa{ɻ̍}{\kern2pt}/: /\ipa{hṽ̩˧-dɤ˧ɻ̍\#˥}/ ‘clumsy’, /\ipa{õ˧-dɤ˧ɻ̍˧}/ ‘fundamentally’ (from /\ipa{õ˧˥}/
			‘(one)self’), /\ipa{ʂɯ˧-tɤ˧ɻ̍˧}/ ‘smooth (e.g.~carefully planed wood)’ and /\ipa{dʑɯ˩-tɤ˩ɻ̍˥}/
			‘humid, moist’ (from /\ipa{dʑɯ˩}/ ‘water’). These words contain a~phonetic sequence which sounds like a~trilled rhyme. They were initially
			transcribed as /\ipa{dr̩}/ and /\ipa{tr̩}/, adding another unit to the inventory of rhotic
			rhymes. This analysis appears correct for the phonological system of speaker F5 (F4’s
			daughter"=in"=law): when repeating the phrase [\ipa{hṽ̩˧dr̩˧{$\sim$}hṽ̩˧dr̩˧-zo˥}] ‘clumsily’ very slowly, she
			still syllabifies the phonetic sequence [\ipa{dr}] as one syllable. In the speech of F4, though, phonetic realization is slightly less packed together, ranging between [\ipa{dər}] and [\ipa{dəɻ}{\kern2pt}] for the syllable with a~voiced initial, and between [\ipa{tər}] and [\ipa{təɻ}{\kern2pt}] for the syllable with an unvoiced initial. A~phonological argument demonstrating that there are two syllables, not one, comes from the noun /\ipa{ho˧dʑɯ˧tɤ˥ɻ̍˩}/ ‘paste, starch’, literally
			‘watery gruel’, derived from /\ipa{ho˥}/ ‘porridge’ and /\ipa{dʑɯ˩-tɤ˩ɻ̍˥}/ ‘humid, moist’. The noun
			carries H tone on its penultimate syllable and L tone on its final syllable. No HL falling tone (or
			any other falling tone, HM or ML) is ever observed on a~single syllable in Yongning Na, so one is
			led to conclude that there are two syllables here: /\ipa{tɤ.ɻ̍}{\kern2pt}/. 
			
			Interestingly, the two tokens containing /\ipa{tɤ}/ have a~\is{variants}variant with /\ipa{dɤ}/:
			/\ipa{ʂɯ˧-dɤ˧ɻ̍˧}/ for ‘smooth’, and /\ipa{dʑɯ˩-dɤ˩ɻ̍˥}/ for ‘humid, moist’. The reverse is not true:
			the two tokens containing /\ipa{dɤ}/ do not have a~\is{variants}variant with /\ipa{tɤ}/; it is not acceptable to
			say \ipa{$\ddagger${\kern2pt}hṽ̩˧-tɤ˧ɻ̍\#˥} for ‘clumsy’, or \ipa{$\ddagger${\kern2pt}õ˧-tɤ˧ɻ̍˧} for ‘fundamentally’. This suggests
			that /\ipa{-dɤ.ɻ̍}{\kern2pt}/ was, at one stage, a~\is{suffixes}suffix used to \is{derivation!morphological}derive \is{adjectives}adjectives (also used adverbially).
			This \is{suffixes}suffix must have ceased to be productive quite some time ago, since two of the four examples
			underwent a~separate phonetic evolution. It may never have been highly productive.
			

			\subsection{After alveolopalatals, is the rhyme /\ipa{o}/ or /\ipa{jo}/?}
			\label{sec:afteralveolopalatalstwooptionsforanalysisand}
			
			The syllables transcribed as /\ipa{tɕʰo}/, /\ipa{tɕo}/ and /\ipa{dʑo}/ could also be analyzed as
			/\ipa{tɕʰjo}/, /\ipa{tɕjo}/ and /\ipa{dʑjo}/, with a~/\ipa{-jo}/ rhyme. The \is{comparative method (historical linguistics)}correspondence between
			Na /\ipa{dʑo}/ and \ili{Naxi} /\ipa{gy}/ (phonetically: [\ipa{ɟy}]) suggests that the initial became
			palatalized in Na by a~following high front vowel or glide. From a~synchronic point of view,
			however, it seemed more appropriate to transcribe these syllables as composed of an~alveolopalatal
			initial followed by a~back vowel.
			
			
			\subsection{Phonemic status of the~retroflex nasal}
			\label{sec:apossiblereanalysisdispensingwithaphonemicretroflexnasal}
			
			There is one single instance of /\ipa{ɳv̩}/: /\ipa{ɳv̩˥}/ ‘to sniff; to get to know (news)’, often
			used in the negative: /\ipa{mɤ˧-ɳv̩˥}/ ‘[I] don’t know’. This syllable contrasts with /\ipa{nv̩}/,
			e.g.~/\ipa{nv̩˥}/ ‘to bury’. Another possible analysis would be as [\ipa{ɳɻ̍}{\kern2pt}], phonemically /\ipa{nɻ̍{\kern2pt}}/, in which case one could dispense with positing a~/\ipa{ɳ}{\kern1pt}/ consonant phoneme contrasting
			with /\ipa{n}/: retroflex realizations would be conditioned by a~following /\ipa{ɯ}/ or /\ipa{ɻ̍}{\kern2pt}/,
			the combinations /\ipa{nɯ}/ and /\ipa{nɻ̍}{\kern2pt}/ being realized as [\ipa{ɳɯ}] and [\ipa{ɳɻ̍}{\kern2pt}],
			respectively. In the absence of a~phonemic opposition, and given the phonetic proximity between
			these two rhymes in a~retroflex context, this interpretation is not absurd. The next logical step in
			that perspective would be to reinterpret /\ipa{ɖv̩}/, /\ipa{ʈv̩}/ and /\ipa{ʈʰv̩}/ as /\ipa{dɻ̍}{\kern2pt}/,
			/\ipa{tɻ̍}{\kern2pt}/ and /\ipa{tʰɻ̍}{\kern2pt}/. This does not represent a~real economy, however, since there is an~opposition between dentals
			and retroflexes in front of other vowels (e.g.~/\ipa{ʈi}/ vs.\ /\ipa{ti}/). The choice to transcribe
			as /\ipa{ɳv̩}/ is based on my auditory impression that, in the present state of the system, the rhyme
			is closer to [\ipa{v̩}] than to [\ipa{ɻ̍}{\kern2pt}].
			
			\subsection{The palatal nasal}
			\label{sec:palatalnasal}
			
			The palatal nasal [{\kern2pt}\ipa{ɲ}] only appears in the syllable [{\kern2pt}\ipa{ɲi}], suggesting a~reanalysis as an allophone of one of the other nasal initials in the system: /\ipa{m}/, /\ipa{n}/, /\ipa{ɳ{\kern1pt}}/ or /\ipa{ŋ}/. The most plausible analysis from a~language-independent perspective would be phonemicization as /\ipa{ŋi}/, in view of the well-documented palatalizing effects of high front vowels on velar consonants. 
			
			This analysis is possible in principle, in the absence of a~syllable [\ipa{ŋi}] in the syllabic inventory of Yongning Na. In \ili{Naxi}, analysis of palatal initials as allophones of velars is an attractive solution, because it applies throughout the system: [\ipa{cʰi}], [\ipa{ci}], [{\kern2pt}\ipa{ɟi}], [{\kern2pt}\ipa{ɲɟi}] and [{\kern2pt}\ipa{ɲi}] can be analyzed as /\ipa{kʰi}/, /\ipa{ki}/, /\ipa{gi}/, /\ipa{ŋgi}/ and /\ipa{ŋi}/ \citep[14]{michailovskyetal2006}. However, detailed examination of the \ili{Naxi} lexicon shows that this amounts to an internal \is{comparative method (historical linguistics)}reconstruction rather than a~synchronic phonemic analysis, as some expressive coinages have now filled the structural gaps left empty by the palatalization of velars \citep[7]{michaudetal2015c}. In Na, it is less tempting to phonemicize [{\kern2pt}\ipa{ɲi}] as /\ipa{ŋi}/, as there are phonemic combinations of velar initials with the vowel /\ipa{i}/, which are not strongly palatalized. The notation adopted therefore remains close to the surface form: /{\kern1pt}\ipa{ɲi}/.
			
			
			\subsection{Syllables introduced by Mandarin borrowings}
			\label{sec:syllablesintroducedbychineseborrowings}
			
			\ili{Mandarin} borrowings hold potential for bringing considerable changes to the phonotactics of Na
			syllables; in particular, they introduce many new combinations of vowels with glides. The
			overall situation is comparable to that of \ili{Naxi}. A~young {Naxi} from Dadong \zh{大东}, He Likun \zh{和丽昆}, did
			an~inventory of the syllables present in his own speech, and found that recent {Mandarin} \isi{loanwords}
			account for about 150 of the syllables that he uses when speaking \ili{Naxi} \citep{michaudetal2015c}. He
			Likun is basically bilingual in {Mandarin}, a~situation which is common among young {Naxi} and Na
			people. On the other hand, the main consultant for Yongning Na is thirty"=eight years older than He Likun, and
			her knowledge of {Mandarin} is limited. In her speech, there is a~tension between a~general tendency to
			integrate loanwords into the Na phonological system and occasional efforts at getting closer to the “correct” pronunciation in \ili{Mandarin} (either \il{Mandarin!Southwestern}Southwestern
			{Mandarin} or \il{Mandarin!Standard}Standard {Mandarin}, depending on the addressee). This source of instability needs to be recognized when transcribing {Mandarin} \is{loanwords}borrowings: they
			typically possess both (i)~an~adapted form, conforming to Na phonotactics and phonetics, and
			(ii)~forms that are closer to \ili{Mandarin}, and which depart from Na phonotactics and phonetics. For
			instance, in the absence of a~rounding opposition for front vowels, \ili{Mandarin} [\ipa{y}] is borrowed as
			[\ipa{i}]: the \ili{Mandarin} \textit{zájūn} \zh{杂菌} [\ipa{tsa.tɕyn}] ‘mixed mushrooms’ is pronounced
			/\ipa{tsɑ˩tɕi˩}/. But the consultant is aware of the phonetic distance between [\ipa{tɕyn}] and
			[\ipa{tɕi}], and is able to make efforts towards rounding of the front vowel, getting close to
			[\ipa{tɕɥe}]{\kern2pt}\ipa{≈}{\kern2pt}[\ipa{tɕɥi}]. 
			
			The competing pressures towards adaptation to the Na system and faithfulness to \ili{Mandarin}
			pronunciation can sometimes be observed in the lexicon. The \is{loanwords}Chinese word for ‘Westerners, foreigners’, \textit{yáng} \zh{洋}, has three forms in Yongning Na: /\ipa{jɤ˩}/, /\ipa{je˩}/ and /\ipa{ʐe˩}/. The most common form is /\ipa{jɤ˩}/, as in /\ipa{jɤ˩ho˧}/ ‘matches’, from \textit{yánghuǒ} \zh{洋火} and /\ipa{jɤ˩jo\#˥}/ ‘potato’, from \textit{yángyù} \zh{洋芋}. The second, /\ipa{je˩}/, appears in /\ipa{je˩ʐe˧}/, from \textit{yángrén} \zh{洋人}
			‘Westerner’. In turn, this word appears in a~modified form in ‘wild cotton flowers’, /\ipa{ʐe˩ʐe˧-bæ˩bæ˩}/. This is a~distortion of /\ipa{je˩ʐe˧-bæ˩bæ˩}/, literally ‘Westerners’ flower’, which
			is still an~acceptable \is{variants}variant. The borrowed syllable /\ipa{je}/ in /\ipa{je˩ʐe˧}/ \zh{洋人}
			‘Westerner’ is Naicized by identifying it with a~syllable that is well"=attested in Na (/\ipa{ʐe}/),
			taking occasion of its presence in the immediate vicinity: as the second syllable of ‘wild cotton flowers’. Unsurprisingly, the \is{stylistics}stylistic effect of Naicization is to sound
			more local, playing on a~sense of closeness among speakers of Na, whereas the more faithful rendering of the \ili{Mandarin} original sounds more modern and forward"=looking.
			
			
			\section{Articulatory reduction: Reduced forms and their lexicalization}
			\label{sec:articulatoryreductionreducedformsandtheirlexicalization}
			
			
			Phenomena of articulatory reduction pave the way for the \isi{lexicalization} of new forms, sometimes
			resulting in the creation of new syllabic combinations. Some salient examples are presented below.
			
			The verb /\ipa{ʝi˥}/ ‘to do’ is prone to reduction. Reduction is well on its way towards
			\isi{lexicalization} for /\ipa{gɯ˩ ʝi˥}/ ‘really, truly’ (from /\ipa{gɯ˩}/ ‘authentic, true’), realized as
			[\ipa{gi˩˥}] except when hyperarticulated. Phonetic reduction is common, e.g.~/\ipa{no˧ {\kern2pt}|{\kern2pt} ə˧tso˧
				ʝi˧-bi˧}/ ‘What are you going to do?’
			(2\textsc{sg}-\textsc{interrog}:what"=to\_do-\textsc{imm.fut}) is commonly
			realized in a~hypo"=articulated way that can be approximated as [\ipa{no˧ ə˧tsɤ˧bi˧}].
			
			The {relativizer} /\ipa{hĩ˥}/ is articulated much more weakly than the lexical word
			/\ipa{hĩ˥}/ ‘human being, person’. The initial fricative is often strongly reduced: it gets voiced
			throughout. Before a~voiced stop, realizations as a~nasal consonant (nasal stop) are observed, as in
			\figref{fig:anillustrationofthereductionoftherelativizertoanasalconsonant}, which shows a~spectrogram of (\ref{ex:reallyhappy}). 
			
						\begin{exe}
							\ex
							\label{ex:reallyhappy}
							\ipaex{ɖwæ˧˥ {\kern2pt}|{\kern2pt} fv̩˧-hĩ˧ ɖɯ˧-v̩˧ ɲi˩!}\\
							\gll ɖwæ˧˥	fv̩˧	-hĩ˥	ɖɯ˧		v̩˧		ɲi˩\\
							very	happy	\textsc{rel}	one		\textsc{clf}.individual		\textsc{cop}\\
							\glt ‘(S)he is really happy!’ (Source: field notes.)
						\end{exe}
						
			The spectrogram shows that the sequence /\ipa{hĩ˧ ɖɯ˧}/ is realized phonetically close to [\ipa{nɖɯ˧}], as if the syllable /\ipa{hĩ˥}/ were realized as prenasalization of the following stop. This is not a~categorical process: the claim that /\ipa{hĩ˥}/ becomes categorically changed to /\ipa{n}/ in this context would be up against insuperable phonotactic difficulties, since Na does not have a~series of prenasalized stops (hence no /\ipa{nɖ}{\kern2pt}/ initial) and /\ipa{n}/ on its own is not a~well-formed syllable, as it lacks a~rhyme and a~tone. Instead, the reduction process is phonetic; annotating the reduced realization of /\ipa{hĩ˥}/ in \figref{fig:anillustrationofthereductionoftherelativizertoanasalconsonant} as [\ipa{n}] is simply a~convenient shorthand notation. Underlying specifications can still leave traces in the overall articulatory gesture, even though the targets normally thought of as primary are not being achieved \citep[272]{nolan1992}. With this reservation, it is clear that the reduction is strong; Roselle Dobbs (p.c.\ 2014) indicates that some younger speakers are not aware that the
			{relativizer} /\ipa{hĩ˥}/ is present in contexts such as the one shown in \figref{fig:anillustrationofthereductionoftherelativizertoanasalconsonant}, and that they tend to omit it altogether.
			
			\begin{figure}%[t]
				\includegraphics[width=.9\textwidth]{figures/fvndeevnhi/fvndeevnhiF5.eps}
				\caption{An illustration of the reduction of the relativizer /\ipa{hĩ}/ to a~nasal consonant. Top: phonetic transcription; bottom: phonemic transcription. Speaker: F5.}
				\label{fig:anillustrationofthereductionoftherelativizertoanasalconsonant}
			\end{figure}
			
			
			The phrase /\ipa{ʈʰæ˧mi˧-ɳɯ˩}/ ‘really, actually’ is generally reduced to a~\is{monosyllables}monosyllable with a~\is{lengthening}long rhyme, which can be approximated as [\ipa{ʈʰææ̃˧}], sometimes with a~trace of the final L tone of the full
			expression: [\ipa{ʈʰæ˧æ̃˩}]. In the absence of a~length opposition among vowels, the reduced form is
			unlikely to become lexicalized.
			
			The proximal \is{demonstratives}demonstrative /\ipa{ʈʂʰɯ˥}/ in association with the associative plural \is{clitics}clitic /\ipa{=ɻæ˩}/
			yields /\ipa{ʈʂʰɯ˧=ɻæ˥\$}/ ‘these things, this sort of things’, and the distal \is{demonstratives}demonstrative yields
			/\ipa{tʰv̩˧=ɻæ˥\$}/ ‘those things’. These disyllabic forms are strongly coalescent. Regressive vowel
			harmony is strong, yielding instances resembling [\ipa{ʈʂʰæ˧=ɻæ˥}], e.g.~in
			Caravans.153 and Agriculture.109. Together with the weakening of the consonant /\ipa{ɻ}{\kern2pt}/ (which, as an~approximant, is vowel-like in the first place), this leads to realizations that often resemble a~\is{monosyllables}monosyllable, [\ipa{ʈʂʰæ˧˥}] or
			[\ipa{ʈʰæ˧˥}]. Examples include Caravans.160,
			165, Mountains.83, 109, Funeral.190, and BuriedAlive3.50.
			
			The exclamative final particle, /\ipa{wɤ˧}/, which conveys obviousness, tends to fuse with
			a~preceding /\ipa{-ɲi˩}/ (expressing certitude). In all fourteen occurrences found in F4’s transcribed narratives, the combination /\ipa{-ɲi˩ wɤ˧}/ has its final M tone depressed to L by application of Rule~5 (see \sectref{sec:alistoftonerules}), and it is realized phonetically close to a~monosyllable:
			[\ipa{-ɲo˩}]. This phenomenon is also highly frequent in the speech of M21.
			

			\section{Expressive coinages and phonostylistic observations}
			\label{sec:expressivecoinagesandmore}
			
\index{expressive coinages|(}

To conclude this chapter about Na phonemes (vowels and consonants), it appeared interesting to mention expressive coinages, and some \is{phonostylistics}phonostylistic observations.
			
			\subsection{Onomatopoeia and ideophones}
			\label{sec:onomatopoeics}

\index{onomatopoeia|(}

			Onomatopoeia constitute one aspect of expressive (phonaesthetic) coinages, {\linebreak}which also comprise interjections,
			calling sounds, and ideophones. All of these present interesting morphological and phonological
			specificities. “Of the 446 known onsets in \ili{Japhug}, forty"=five clusters (including thirty"=five
			two"=consonant and eleven three"=consonant clusters) are exclusively attested in ideophones or
			ideophonic verbs” \citep[264]{jacques2013c}. Expressive coinages tend to have a~lilt of their own, but they
			also undergo a~continuous attraction from the language's phonological system, tending to
			their integration into the language's phonological categories: “ideophones fill gaps in the distribution of segments
			within rhymes that have been caused by sound changes” \citep[267]{jacques2013c}. A~classical case of structural \is{gap-filling}gap"=filling is found in
			\ili{Vietnamese}, where the /\ipa{ɔŋ}/ and /\ipa{oŋ}/
			rhymes underwent an evolution whereby lip rounding was shuffled from the vowel to the consonant, as an~added final
			labial closure: the result can be approximated as [\ipa{ʌɔŋ͡m}] and [\ipa{ɤoŋ͡m}]. The slots left empty by this phonetic evolution were filled by onomatopoeic coinages, and by loanwords (\citealt{haudricourt1952b}; \citealt[21]{henderson1985}; \citealt[143]{michaud2004a}). 
			
			He
			Likun, a~native speaker of \ili{Naxi}, went through each cell in a~table of possible combinations of
			initials and rhymes in the Pianding dialect of \ili{Naxi}, determining (by introspection) whether the
			combination was attested, and in which words. The results were supplemented by examining a~word list
			of about 3,000 words. He identified more than fifteen syllables that are only attested in
			onomatopoeic words \citep{michaudetal2015c}. 			
			
			Onomatopoeia are no less abundant in Na than in \ili{Naxi}. But they are scarce in
			the set of transcribed narratives, as could be expected of relatively formal monologues. Other data
			collection methods, such as recording lively conversations, will be required to explore the wealth
			of expressive phenomena found in Yongning Na. Here are three examples.
			
			\begin{enumerate}[label=(\roman*)]
				\item The noise of a~shock between two hard objects, for instance the sound of an~axe hitting a~tree trunk (‘Bang!’), is rendered as [\ipa{bõ}]. This syllable contravenes Na phonotactics, as nasal rhymes do not normally appear after stops. 
				\item The onomatopoeia for rumbling sounds, for instance the sound of heavy loads carried over a~wooden floor, or the noise of lorries, is a~prolonged [\ipa{ʐ}{\kern2pt}]. This sound is unlike the syllable /\ipa{ʐɯ}/. The latter is a~full"=fledged syllable, which surfaces with an~apicalized vowel, as [\ipa{ʐʐ̩}{\kern2pt}]: the beginning of the syllable is more consonant"=like, and the end more vowel"=like. This is the reason that Chao Yuen"=ren advocated the use of special symbols for apicalized sounds (see \tabref{tab:apicalized}). In his system, the syllable would be transcribed as /\ipa{ʐʅ}{\kern2pt}/. In the onomatopoeic form for rumbling sound, on the other hand, friction is sustained from beginning to end, hence transcription as  [\ipa{ʐʐʐ}{\kern2pt}].
				\item The hissing noise of water that comes in contact with red"=hot metal or incandescent wood (‘Pssshhh!’) is [\ipa{ʈʂʰɻ}{\kern2pt}]. The combination of initial and rhyme used to transcribe this onomatopoeia is attested in some lexical items, but its phonetic realization does not exactly match that of the syllable /\ipa{ʈʂʰɻ}{\kern2pt}/ of lexical items. To reflect this special status, a~possible transcription is [\ipa{ʈʂʰɻɻɻ}{\kern2pt}].
			\end{enumerate}
			
\index{onomatopoeia|)}

%			Because the present volume's focus is on morphotonology, little attention has been devoted to the
%			rich linguistic field of expressives in Yongning Na. But ultimately, this field is not without
%			relevance to tone and \isi{intonation}.
%			
%			Investigation of this topic is
%			envisaged at the stage of experimental study of fine phonetic detail in Na tone and \isi{intonation}~--
%			a~topic about which some initial observations are set out in Chapter~\ref{chap:fromsurfacephonologicalformstophoneticrealizationintonationandtonalimplementation}.
			
			\subsection{Phonostylistic observations}
			\label{sec:phonostylisticobservations}
			\label{sec:liproundingandprotrusionwithdemonstrativeproximalvalue}
			\label{sec:palatalizationconveyingatenderemotion}
			Expressivity is not limited to specific areas of the lexicon, such as ideophones. The “appeal function” of speech \citep{buhler1934} is constantly present. The study of this function~-- examining how speakers shape their utterances with
			a~view to evoking a~certain response on the part of the hearer~-- features prominently in the
			programme of phonological research set out by \citet[14]{trubetzkoy1969} (original text: \citealt{trubetzkoy1939}), who coined the term
			‘phonostylistics’ (for a~review: \citealt{leon1969}). The term ‘psycho"=phonetics’ used by
			\citet{fonagy1983} is less specific and therefore perhaps less appropriate, although it has the
			advantage of bringing out the considerable breadth of this strand of research: studying how phonetic details convey the speaker’s communicative purposes. If \isi{intonation} is “a symptom of how we
			feel about what we say and how you feel when you say it” \citep[1]{bolinger1989}, phonostylistics is
			part and parcel of \isi{intonation} studies.
			
			But this fascinating topic is best investigated through experimental phonetic study, whereas the
			present volume essentially focuses on lexical tone and morphotonology; the approved order of business consists
			in postponing the study of expressive phenomena until the stage when the more central facts of
			a~language's linguistic structure have been clarified. Discussion is therefore deferred to
			\is{experimental phonetics}experimental phonetic studies to be conducted in future. Let us simply mention two salient cases of
			modification of vowels and consonants for expressive effects in Na.
			
			
%			\begin{description}
%				\item[Lip rounding and protrusion with \is{demonstratives}demonstrative (proximal) value:] the vowel /\ipa{ɯ}/ has neither lip rounding nor lip protrusion. It acquires lip protrusion when the
%				phrase /\ipa{ʈʂʰɯ˧-ɭɯ˧}/ ‘this one’ (proximal \is{demonstratives}demonstrative plus generic classifier) is used as
%				a~real"=world \is{demonstratives}demonstrative, pointing to an~object within sight. The speaker’s face points in the
%				direction of the object, and lip protrusion functions as part of the gesture of pointing. It is
%				often accompanied by an~upward movement of the chin, further reinforcing the pointing gesture.
%				\item[Palatalization conveying a~tender emotion:] the adjective /\ipa{ɳɯ˧ɕi˩}/ ‘lovely’ can be pronounced close to /\ipa{ni˧ɕi˩}/. This
%				child"=speech"=like \is{variants}variant has iconic value: palatalization, narrowing the vocal tract, is
%				associated with smallness \citep[22--23]{fonagy1983}. The realization of this cross"=linguistic
%				tendency is facilitated in Na by the tendency towards regressive vowel harmony (\sectref{sec:anoteonvowelharmony}).
%			\end{description}
			
The first is \textit{lip rounding and protrusion with \is{demonstratives}demonstrative (proximal) value}. Pointing with the lips is used in Na culture. (For details about this gesture, see \citealt{enfield2001}.) When the lip"=pointing gesture is used during speech, lip rounding and lip protrusion get superimposed onto the speech production gestures. For instance, the vowel /\ipa{ɯ}/, which has neither lip rounding nor lip protrusion, acquires lip protrusion when the
				phrase /\ipa{ʈʂʰɯ˧-ɭɯ˧}/ ‘this one’ (proximal \is{demonstratives}demonstrative plus generic classifier) is said while lip"=pointing to an~object within sight. 
				
The second case is that of \textit{palatalization conveying a~tender emotion}. The adjective /\ipa{ɳɯ˧ɕi˩}/ ‘lovely’ can be pronounced close to /\ipa{ni˧ɕi˩}/. This
				child"=speech"=like \is{variants}variant has iconic value: palatalization, narrowing the vocal tract, is
				associated with smallness \citep[22--23]{fonagy1983}. The realization of this cross"=linguistic
				tendency is facilitated in Na by the tendency towards regressive vowel harmony (\sectref{sec:anoteonvowelharmony}).

%			\subsubsection*{Lip rounding and protrusion with \is{demonstratives}demonstrative (proximal) value}
%			\label{sec:liproundingandprotrusionwithdemonstrativeproximalvalue}
%			
%			
%			The vowel /\ipa{ɯ}/ has neither lip rounding nor lip protrusion. It acquires lip protrusion when the
%			phrase /\ipa{ʈʂʰɯ˧-ɭɯ˧}/ ‘this one’ (proximal \is{demonstratives}demonstrative plus generic classifier) is used as
%			a~real"=world \is{demonstratives}demonstrative, pointing to an~object within sight. The speaker’s face points in the
%			direction of the object, and lip protrusion functions as part of the gesture of pointing. It is
%			often accompanied by an~upward movement of the chin, further reinforcing the pointing gesture.
%			
%			
%			
%			%subsec:2-4-6
%			\subsubsection*{Palatalization conveying a~tender emotion}
%			\label{sec:palatalizationconveyingatenderemotion}
%			
%			
%			The adjective /\ipa{ɳɯ˧ɕi˩}/ ‘lovely’ can be pronounced close to /\ipa{ni˧ɕi˩}/. This
%			child"=speech"=like \is{variants}variant has iconic value: palatalization, narrowing the vocal tract, is
%			associated with smallness \citep[22--23]{fonagy1983}. The realization of this cross"=linguistic
%			tendency is facilitated here by the tendency towards regressive vowel \isi{assimilation} found in Na and
%			other \ili{Naish} languages.
%			
%			
			
			\subsection{Expressive uses of reduplication}
			\label{sec:thereduplicationofnonlexicalwords}
			
			Reduplication serves various grammatical functions in Yongning Na, as also in \ili{Naxi} \citep[30–33]{heetal1985}. Despite having neatly grammaticalized uses, such as lending reciprocal value to verbs, it retains an~expressive dimension, especially in its sporadic application to parts of speech other than verbs and nouns. This is reflected in irregular tone patterns~-- and thus this Appendix finally returns to the book's central topic: tone.
			
			The phrase /\ipa{qʰɑ˧~ɲi˧}/ ‘how many days’ reduplicates to /\ipa{qʰɑ˧~ɲi˧{$\sim$}qʰɑ˩~ɲi˩}/ ‘thus and so
			many days’ (Healing.29; the context is the following: a~priest of the Na religion diagnoses the
			number of days of rituals it will take to cure a~person’s disease). The tone pattern is
			not the same as in \is{numerals}numeral"=plus"=determiner phrases, where the expected output would be M tone
			throughout the phrase (see Chapter~\ref{chap:classifiers}).
			
			The interrogative /\ipa{ə˧tso˧}/ reduplicates to /\ipa{ə˧tso˧{$\sim$}ə˧tso˥}/ (Dog.48). This
			is also an~unexpected pattern.
			
			The word /\ipa{zo˧{$\sim$}zo˧-mv̩˧{$\sim$}mv̩˥}/ ‘thingummy’ looks a~lot like it could be
			the product of \isi{reduplication}, perhaps as a~playful manipulation over /\ipa{zo˧mv̩˥}/ ‘child’. A~more
			common word for ‘thing’ is /\ipa{tso˧tso\#˥}/, which may originate in a~reduplication of the
			nominalizer /\ipa{tso}/. Both of the above nouns combine into
			/\ipa{tso˧{$\sim$}tso˧-zo˧{$\sim$}zo˧-mv̩˧{$\sim$}mv̩˥}/ ‘thingummies, stuff’,
			suggesting that they are currently perceived as having a~similar
			internal structure.
			
			The reduplicated form /\ipa{zɯ˧{$\sim$}zɯ˧}/ for ‘life, existence’ is more frequent than
			{monosyllabic} /\ipa{zɯ˧}/, but both are in common use.
			
			The /\ipa{lv̩.lv̩}/ portion in /\ipa{bi˧-lv̩˧{$\sim$}lv̩˥}/ ‘snowflake’ and
			/\ipa{dzo˧-lv̩˧{$\sim$}lv̩˥}/ ‘hailstone’ looks like a~reduplicated form of the classifier for kernels, /\ipa{lv̩˧}/ \citep[xxxiv]{lidz2010}.
			
			The phrase /\ipa{tɕɤ˧{$\sim$}tɕɤ˧}/ ‘right at the moment that{\dots}’ looks clearly like
			a~reduplicated form, but the \textit{simplex} (non"=reduplicated) form could not be recovered.
			
			Several four"=syllable onomatopoeic expressions of the form ABAB were observed, all of them with
			a~L.L.M.M tone pattern: /\ipa{tsɯ˩qwæ˩{$\sim$}{\allowbreak}tsɯ˧qwæ˧}/ ‘crashing sound, for instance the sound of timber falling down’
			(Housebuilding.243), /\ipa{zɯ˩gɯ˩{$\sim$}{\allowbreak}zɯ˧gɯ˧}/ ‘boom!’ (sound of heavy shock against a~door: Tiger.15),
			/\ipa{ʐɯ˩ʐɤ˩{$\sim$}{\allowbreak}ʐɯ˧ʐɤ˧}/ ‘sound of tearing leaves to pieces’ (FoodShortage2.37), and
			/\ipa{ɕi˩hwɑ˩{$\sim$}{\allowbreak}ɕi˧hwɑ˧}/ and /\ipa{ʐɯ˩ʁæ˩{$\sim$}{\allowbreak}ʐɯ˧ʁæ˧}/, both describing the dizziness of {\linebreak}a~character
			under a~dazzling moonlight (Reward.17 and Reward.68). These expressions do not have an~identifiable \textit{simplex} form.
            
            \index{expressive coinages|)}