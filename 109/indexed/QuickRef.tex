%\addchap{Reference section: a~recapitulation of the main facts about the tone system}
\addchap{For quick reference: Lexical tones and~main~tone~rules}

% Add missing right-page heading
\rohead{For quick reference: Lexical tones and main tone rules}


\begin{refsection}
	
This volume is organized in analytical order: setting out facts, and gradually advancing towards an analysis. This mode of exposition replicates the progression of analysis during fieldwork, working up from the surface
facts. The aim is to allow the reader to evaluate the analysis step by step, and to reflect on
possible alternatives, rather than proposing a~complete analysis from a~top"=down perspective. A~drawback of this choice is that the main facts about the tone system are not grouped in one place, and can be difficult to look up. The present reference section is the place to go to re"=check (i)~the inventory of tones for nouns and verbs, (ii)~the meaning of the custom notations used for the tone categories of Yongning Na (H\#, \#H and the like), (iii)~the list of phonological tone rules, and (iv)~three sets of tables presenting combination rules: those that hold in determinative compound nouns, subject"=plus"=verb combinations, and object"=plus"=verb combinations, respectively.

\section*{The tones of nouns and verbs}
\label{sec:tonesofN}

	The copula and the possessive are used as tests to determine the lexical tone of a~noun, because their surface tone changes according to the tone category of the target word. The use of different contexts of elicitation can be likened to the use of chemicals in the processing of photographic films: in the same way as a~blend of chemicals is required to convert the latent image to a~visible image, several contexts need to be used to reveal underlying tone. Together, the copula and the possessive suffice to bring out all the categories of nouns. For nominal classifiers (\tabref{tab:TONECLREF}), nine tonal categories were brought out by examining tonal patterns in association with numerals. For verbs (\tabref{tab:UtonesofverbsREF}), the seven categories are arrived at by piecing together evidence from four contexts: in isolation; with a~preceding \textsc{negation} prefix, /\ipa{mɤ˧-}/ or accomplished prefix, /\ipa{le˧-}/; and in association with the object ‘a~bit’, /\ipa{ɖɯ˧-kʰwɤ˥\$}/.

\begin{subtables}
	\label{tab:thelexicaltonesofmonosyllabicanddisyllabicnounsBIS}
	\begin{table}[h]
		\caption{\label{tab:thelexicaltonesofmonosyllabicnounsBIS}The lexical tone categories of monosyllabic nouns. From \sectref{sec:overviewofthesystem}.}
		\begin{tabularx}{\textwidth}{ P{20mm} P{19mm} Q Q P{19mm} Q }
			\lsptoprule
			analysis & in isolation & +\textsc{cop} & +\textsc{poss} & //example// & meaning\\ \midrule
			// LM // & LH & L+H & L+H & \ipa{bo˩˧}  & pig\\
			// LH // & LH & L+H & L+H & \ipa{ʐæ˩˥}  & leopard\\
			// M // & M & M+L & M+M & \ipa{lɑ˧} & tiger\\
			// L // & M & L+LH & L+M & \ipa{jo˩} & sheep\\
			// \#H // & M & M+H & M+M & \ipa{ʐwæ˥} & horse\\
			// MH\# // & MH & M+H & M+H & \ipa{ʈʂʰæ˧˥} & deer\\
			\lspbottomrule
		\end{tabularx}
	\end{table}
% Addition of 2 blank lines for neat page break
%\vspace{15mm}

	
	\begin{table}[t]
		\caption{\label{tab:thelexicaltonesofdisyllabicnounsBIS}The lexical tone categories of disyllabic nouns. From \sectref{sec:overviewofthesystem}.}
		\begin{tabularx}{\textwidth}{ P{21mm} l Q Q P{19mm} Q }
			\lsptoprule
			analysis & in isolation & +\textsc{cop} & +\textsc{poss} & //example// & meaning\\ \midrule
			// M // & M.M & M.M+L & M.M+M & \ipa{po˧lo˧} & ram\\
			// \#H // & M.M & M.M+H & M.M+M & \ipa{ʐwæ˧zo\#˥} & colt\\
			// MH\# // & M.MH & M.M+H & M.M+H & \ipa{hwɤ˧li˧˥} & cat\\
			// H\$ // & M.H & M.M+H & M.M+M & \ipa{kv̩˧ʂe˥\$} & flea\\
			// H\# // & M.H & M.H+L & M.H+L & \ipa{hwæ˧ʈʂæ˥} & squirrel\\
			// L // & L.LH & L.L+H & L.L+H & \ipa{kʰv̩˩mi˩} & dog\\
			// L\# // & M.L & M.L+L & M.L+L & \ipa{dɑ˧ʝi˩} & mule\\
			//LM+MH\#// & L.MH & L.M+H & L.M+H & \ipa{õ˩dv̩˧˥} & wolf\\
			//LM+\#H// & L.M & L.M+H & L.M+M & \ipa{nɑ˩hĩ\#˥} & Naxi\\
			// LM // & L.M & L.M+L & L.M+M & \ipa{bo˩mi˧} & sow\\
			// LH // & L.M & L.M+L & L.M+L & \ipa{bo˩ɬɑ˥} & boar\\
			\lspbottomrule
		\end{tabularx}
	\end{table}
\end{subtables}


\begin{table}[h]
	\caption{One example of each of the nine tonal categories of monosyllabic classifiers. From \sectref{sec:howthetonalcategorieswerebroughtoutandlabelled}.}
	\begin{tabularx}{\textwidth}{ Q Q l }
		\lsptoprule
		classifier & tone & description: classifier for{\dots}\\ \midrule
		\ipa{ɖwæ˥} & H\textsubscript{a} & steps (of stairs)\\
		\ipa{ɲi˥} & H\textsubscript{b} & days\\
		\ipa{hɑ̃˧˥} & MH\textsubscript{a} & nights\\
		\ipa{kv̩˧˥} & MH\textsubscript{b} & people, persons\\
		\ipa{nɑ˧} & M\textsubscript{a} & tools\\
		\ipa{dzi˧} & M\textsubscript{b} & pairs of non"=separable objects, e.g.~shoes\\
		\ipa{dze˩} & L\textsubscript{a} & pairs of separable objects, e.g.~pots, bottles\\
		\ipa{dzi˩} & L\textsubscript{b} & trees, bamboo\\
		\ipa{ʐɤ˩} & L\textsubscript{c} & lines, patterns (in weaving or drawing)\\
		\lspbottomrule
	\end{tabularx}
	\label{tab:TONECLREF}
\end{table}

\begin{table}[h]
	\caption{The seven tonal categories of monosyllabic verbs: analysis into H, M, L and LH tones. From \sectref{sec:overview}.}
	\label{tab:UtonesofverbsREF}
	{\renewcommand{\arraystretch}{1.20}
	\begin{tabularx}{\textwidth}{ P{8mm} P{25mm} P{20mm} Q Q  P{20mm} }
			% {\textheight}{ l@{\hspace{10mm}} l@{\hspace{10mm}} Q l@{\hspace{10mm}} Q l@{\hspace{10mm}} Q }
			\lsptoprule
			tone & example & in isolation & \textsc{neg} & \textsc{accomp} & V+‘a~bit’\\ \midrule
			H &  \ipa{dzɯ˥} ‘to eat’ & \tikzmark{0a}M & M.H & M.H & \lshadedcell M.M.M\\ 
			M\textsubscript{a} & \ipa{hwæ˧\textsubscript{a}} ‘to buy’ &  \tikzmark{1a} & \tikzmark{1b}M.M & \tikzmark{1c}M.M & \shadedcell M.H.L\\
			M\textsubscript{b} & \ipa{tɕʰi˧\textsubscript{b}} ‘to sell’ & \tikzmark{2a} & \hspace*{\fill}\tikzmark{2b} & \hspace*{\fill}\tikzmark{2c} & \lshadedcell M.M.M\\
			M\textsubscript{c}  & \ipa{bi˧\textsubscript{c}} ‘to go’ & \hspace*{\fill}\tikzmark{3a} & \hspace*{\fill}\tikzmark{3b} & \tikzmark{3c}M.L & n.a.\\ 
			L\textsubscript{a} & \ipa{dze˩\textsubscript{a}} ‘to cut’ & \tikzmark{4a}LH & \tikzmark{4b}M.L & \hspace*{\fill}\tikzmark{4c} & M.M.H\\
			L\textsubscript{b}  & \ipa{ʈʰɯ˩\textsubscript{b}} ‘to drink’ & \hspace*{\fill}\tikzmark{5a} & \hspace*{\fill}\tikzmark{5b} & \hspace*{\fill}\tikzmark{5c} & M.M.MH\\ 
			MH &  \ipa{lɑ˧˥} ‘to strike’ & \tikzmark{6a}MH & M.MH & M.MH & \shadedcell M.H.L\\
			\lspbottomrule
	\end{tabularx}}
	\DrawBox{0a}{3a}
	\DrawBox{4a}{5a}
	\DrawBox{4b}{5b}
	\DrawBox{3c}{5c}
	\DrawBox{1b}{3b}
	\DrawBox{1c}{2c}
\end{table}

\clearpage
\section*{Notational conventions for tones}
\label{sec:ToneNotation}
%Notational conventions for tones were set out briefly in the ‘Abbreviations and conventions’ section; they are recapitulated here in somewhat greater detail. 

\begin{table}[H]
	{\renewcommand{\arraystretch}{1.35}
		\begin{tabularx}{\textwidth}{ @{}l Q } 
			L & Low tone\\
			M & Mid tone\\
			H & High tone\\
			\# & word boundary; by extension: the boundary of the entire expression to which a~tone pattern is associated (e.g.~a~compound noun or a~numeral"=plus"=determiner phrase)\\
			H\# & final High tone: this H gets anchored on the last syllable of the expression to which it is associated\\
			\#H & floating High tone: this H gets anchored \textit{after} the word boundary, i.e.\ on a~following morpheme, if one is available and can serve as a~host. In some contexts, the floating H tone does not surface (for want of syllabic association) but lowers following tones to L.\\
			MH\# & final Mid"=to"=High tone, realized either as a~rising contour on the last syllable of the expression, or, where the morphotonological context allows, as M tone on the last syllable of the expression and H tone on (the first syllable of) the morpheme that follows\\
			H\$ & a~type of H tone (analyzed in \sectref{sec:wordfinalandmorphologicalnucleusfinalHtones}) exemplified by the noun ‘flea’, and  nicknamed the ‘flea’ tone or ‘gliding’ tone. When a~word carrying this tone is pronounced in isolation, the H tone associates to its last syllable: /\ipa{kv̩˧ʂe˥}/ ‘flea’ has a~M.H tone sequence at the surface phonological level. When the copula is added, the result is /\ipa{kv̩˧ʂe˧ ɲi˥}/ ‘is \mbox{(a/the)} flea’, with H tone on the copula. When the noun is followed by the possessive, no H tone reaches the phonological surface: the observed form is /\ipa{kv̩˧ʂe˧=bv̩˧}/ ‘of \mbox{(a/the)} flea’, with M tone on both syllables of the
			noun and also on the possessive.\\
			-- & In the representation of the tones of expressions containing two (or more) morphemes, such as compound nouns, the symbol -- refers to the morpheme break. Thus, --L refers to a~L tone that attaches after the morpheme break, i.e., in the case of a~compound noun, on the head noun. LM--L indicates that the morpheme before the morpheme break (i.e., in a~compound, the determiner) gets LM tone and the noun after the break gets L tone.\\
		\end{tabularx}}
	\end{table}

\clearpage

%\section*[The seven phonological tone rules]{The seven phonological tone rules (from \sectref{sec:alistoftonerules})}
\section*{The seven phonological tone rules (from \sectref{sec:alistoftonerules})}
\label{sec:tonerules}

\begin{enumerate}[leftmargin=2cm, itemsep=0pt, labelwidth=\widthof{Rule~1:}]%[topsep=12pt, partopsep=0pt]
	\item[Rule~1:] L tone spreads progressively (“left"=to"=right”) onto syllables that are unspecified for tone.
	\item[Rule~2:] Syllables that remain unspecified for tone after the application of Rule 1 receive M tone.
	\item[Rule~3:] In tone"=group"=initial position, H and M are neutralized to M.
	\item[Rule~4:] The syllable following a~H-tone syllable receives L tone.
	\item[Rule~5:] All syllables following a~H.L or M.L sequence receive L tone.
	\item[Rule~6:] In tone"=group"=final position, H and M are neutralized to H if they follow a~L tone.
	\item[Rule~7:] If a~tone group only contains L tones, a~post"=lexical H tone is added to its last syllable.
\end{enumerate}

\clearpage
%\Hack{\newpage}

\section*{Combination rules in determinative compounds,\\ S+V combinations, and O+V combinations}
% \largerpage[2]
%\section*[Tone combination rules in compounds, S+V, and O+V]{Combination rules in determinative compounds, ~~~~~~~~~~~S+V combinations, and O+V combinations}
\label{sec:tonesofCOMPOUNDS}
% {\largerpage[3]}
%As explained as part of the notational conventions for tones, the symbol -- refers to a~morpheme break inside a~complex expression such as a~compound noun. Thus, --L refers to a~L tone that attaches after the morpheme break, i.e.\ on the second noun (the head noun), and LM--L indicates that the noun before the morpheme break (i.e.\ the determiner) gets LM tone and the noun after the break (the head) gets L tone. 

\begin{subtables}
	\label{tab:abstractcompoundsREF}
	
	\begin{table}%[h!!]%[t!]
	\small
		\caption{\label{tab:abstractmonosyllabicmonosyllablesREF}The underlying tonal categories of σ+σ compound
			nouns. Leftmost column: tone of determiner; top row: tone of head. From \sectref{sec:analysisintounderlyingtonepatterns}.}
		{\renewcommand{\arraystretch}{1.35}
			\begin{tabularx}{\textwidth}{ Q  Q  Q  Q  Q  Q }
				\lsptoprule
				tone & LM; LH & M & L & H & MH\\ \midrule
				LM; LH & LH & LM & LH & LM+\#H & LM+MH\#\\
				M & --L & \#H & --L & \#H & MH\#\\
				L & \tikzmark{1a}L &  &  &  & \hspace*{\fill}\tikzmark{1e}\\
				H & \#H-- & \tikzmark{2a}\#H &  &
				\hspace*{\fill}\tikzmark{2e} & --L\\
				MH & \tikzmark{3a}H\# &  &  \hspace*{\fill}\tikzmark{3e} & \tikzmark{4a}H\$ & \hspace*{\fill}\tikzmark{4e}\\
				\lspbottomrule
			\end{tabularx}}
			\DrawBox[dashed]{1a}{1e}
			\DrawBox[dashed]{2a}{2e}
			\DrawBox[dashed]{3a}{3e}
			\DrawBox[dashed]{4a}{4e}
		\end{table}
		
		\begin{table}%[p!]
		\small
			\caption{\label{tab:abstractmonosyllabicdisyllablesREF}The underlying tonal categories of σσ+σ compound
				nouns. Leftmost column: tone of determiner; top row: tone of head. From \sectref{sec:analysisintounderlyingtonepatterns}.}
			{\renewcommand{\arraystretch}{1.35}
				\begin{tabularx}{\textwidth}{ Q P{18mm} Q Q Q Q }
					\lsptoprule
					tone & LH; LM & M & L & H & MH\\ \midrule
					M & --L & \#H & --L & \#H & \tikzmark{1a}--L\\
					\#H & \tikzmark{99a}H\# & \tikzmark{2a}\#H &  &
					\hspace*{\fill}\tikzmark{2e} & \hspace*{\fill}\tikzmark{1e}\\
					MH\# & \hspace*{\fill}\tikzmark{99e} & \tikzmark{3a}MH\# &  &  \hspace*{\fill}\tikzmark{3e}& H\#\\
					H\$ & \#H-- & \#H & H\$ & \#H & H\#--\\
					L & L+H\# & \tikzmark{4a}L &  & \hspace*{\fill}\tikzmark{4e} & L+H\#\\
					L\# & \tikzmark{5a}L\#-- &  &  &  & \hspace*{\fill}\tikzmark{5e}\\
					LM+MH\# & \tikzmark{6a}LM+MH\#-- & LM+MH\# & \tikzmark{7a}LM+H\$ &  & \hspace*{\fill}\tikzmark{7e}\\
					LM+\#H &  \hspace*{\fill}\tikzmark{6e}& LM+\#H & LM+H\# & \tikzmark{8a}LM+\#H & LM+H\#\\
					LM & LM--L & LM & LM--L &
					\hspace*{\fill}\tikzmark{8e} & LM+MH\#\\
					LH & \tikzmark{9a}LH &  &  &  & \hspace*{\fill}\tikzmark{9e}\\
					H\# & \tikzmark{10a}H\#-- &  &  &  & \hspace*{\fill}\tikzmark{10e}\\ 
					\lspbottomrule
				\end{tabularx}}
				\DrawBox[dashed]{1a}{1e}
				\DrawBox[dashed]{2a}{2e}
				\DrawBox[dashed]{3a}{3e}
				\DrawBox[dashed]{4a}{4e}
				\DrawBox[dashed]{5a}{5e}
				\DrawBox[dashed]{6a}{6e}
				\DrawBox[dashed]{7a}{7e}
				\DrawBox[dashed]{8a}{8e}
				\DrawBox[dashed]{9a}{9e}
				\DrawBox[dashed]{10a}{10e}
%				\DrawBox[dashed]{11a}{11e}
				\DrawBox[dashed]{99a}{99e}
	\end{table}
			
			
\begin{sidewaystable}%[p]
\caption{\label{tab:abstractdisyllabicmonosyllablesREF}The underlying tonal categories of σ+σσ compound nouns. Leftmost column: tone of determiner; top row: tone of head. From \sectref{sec:analysisintounderlyingtonepatterns}.}
{\renewcommand{\arraystretch}{1.35}
	{\fontsize{10}{11}\selectfont
		\begin{tabularx}{\textwidth}{ l P{7mm} P{12mm} Q l P{10mm} P{22mm} P{16mm} P{26mm} }
			\lsptoprule
			tone & M & \#H & MH\# & H\$ & L & L\# & LM+MH\#; LM+\#H; LM; LH & H\#\\\midrule
			LM; LH & LM & LM+\#H  & LM+MH\#~/ L+\#H-- & LM+H\$ & L+\#H-- & L+\#H--~/ L+H\# & L+\#H-- & LM+H\#~/ L+H\#\\
			M & M & \#H & MH\# & H\$ & --L & --L\# & --L & H\#\\
			L & \tikzmark{1a}L &  & \hspace*{\fill}\tikzmark{1e} & L+H\# & L & L+H\# & L+\#H-- & L+H\#\\
			\#H & \tikzmark{2a}H\# & \tikzmark{3a}\#H & \#H-- & --L~/ H\# & \tikzmark{4a}\#H-- & \tikzmark{5a}H\# & \tikzmark{6a}\#H-- & H\#\\
			MH & \hspace*{\fill}\tikzmark{2e} & \hspace*{\fill}\tikzmark{3e} &
			MH\# & \#H-- & \hspace*{\fill}\tikzmark{4e} & \hspace*{\fill}\tikzmark{5e} & \hspace*{\fill}\tikzmark{6e} & \#H\\
			\lspbottomrule
		\end{tabularx}
							} % for 'font'
		\DrawBox[dashed]{1a}{1e}
		\DrawBox[dashed]{2a}{2e}
		\DrawBox[dashed]{3a}{3e}
		\DrawBox[dashed]{4a}{4e}
		\DrawBox[dashed]{5a}{5e}
		\DrawBox[dashed]{6a}{6e}
	}
\end{sidewaystable}

\begin{sidewaystable}[p]
	\caption{\label{tab:abstractdisyllabicdisyllablesREF}The underlying tonal categories of σσ+σσ compound nouns. Leftmost column: tone of determiner; top row: tone of head. From \sectref{sec:analysisintounderlyingtonepatterns}.}
	{\renewcommand{\arraystretch}{1.65}
		{\fontsize{10}{10.75}\selectfont
			\begin{tabularx}{\textwidth}{ l@{\hspace{16pt}} P{12mm} P{12mm} P{16mm} P{22mm} P{20mm} P{12mm} Q P{12mm} }
				\lsptoprule tone & M & \#H & MH\# & H\$ & L & L\# & LM+MH\#;\hack{\par} LM+\#H;\hack{\par} LM; LH & H\#\\\midrule
				M & M & \tikzmark{1a}\#H & MH\# & H\$ / \#H-- & --L & --L\# & --L & \tikzmark{2a}H\#\\
				\#H & \tikzmark{3a}H\# & \hspace*{\fill}\tikzmark{1e} & \tikzmark{4a}\#H-- & H\$ / \#H-- / H\# & \tikzmark{17a}\#H-- & \tikzmark{15a}H\# & \#H-- &\\
				MH\# &  & MH\# & \hspace*{\fill}\tikzmark{4e} & \#H-- / H\# &  &  & MH\#-- &\\
				H\$ & \hspace*{\fill}\tikzmark{3e} & \#H & \tikzmark{16a}\#H-- / H\#-- & \hspace*{\fill}\tikzmark{16e} & \hspace*{\fill}\tikzmark{17e} & \hspace*{\fill}\tikzmark{15e} & \#H-- &  \hspace*{\fill}\tikzmark{2e}\\
				L & L+H\# & L & L+H\# & L+H\# & L+\#H-- & L+H\# & L+\#H-- & L+H\#\\
				L\# & \tikzmark{14a}L\#-- &  &  &  &  &  &  & \hspace*{\fill}\tikzmark{14e}\\
				LM+MH\# & \tikzmark{5a}LM+H\# & \tikzmark{12a}LM+\#H & \tikzmark{13a}LM+MH\#-- & LM+MH\#--/ H\# & LM+MH\#-- & \tikzmark{11a}LM+H\# & \tikzmark{10a}LM+MH\#-- & \tikzmark{9a}LM+H\#\\
				LM+\#H & \hspace*{\fill}\tikzmark{5e} &  & \hspace*{\fill}\tikzmark{13e} & \tikzmark{8a}LM--H\$ & LM+\#H-- & \hspace*{\fill}\tikzmark{11e} & \hspace*{\fill}\tikzmark{10e} &\\
				LM & LM-- &  \hspace*{\fill}\tikzmark{12e} & LM+MH\# & \hspace*{\fill}\tikzmark{8e} & LM--L & LM--L\# & LM--L & \hspace*{\fill}\tikzmark{9e}\\
				LH & \tikzmark{6a}LH &  &  &  &  &  &  & \hspace*{\fill}\tikzmark{6e}\\
				H\# & \tikzmark{7a}H\#-- &  &  &  &  &  &  & \hspace*{\fill}\tikzmark{7e}\\
				\lspbottomrule
			\end{tabularx}
			\DrawBox[dashed]{1a}{1e}
			\DrawBox[dashed]{2a}{2e}
			\DrawBox[dashed]{3a}{3e}
			\DrawBox[dashed]{4a}{4e}
			\DrawBox[dashed]{5a}{5e}
			\DrawBox[dashed]{6a}{6e}
			\DrawBox[dashed]{7a}{7e}
			\DrawBox[dashed]{8a}{8e}
			\DrawBox[dashed]{9a}{9e}
			\DrawBox[dashed]{10a}{10e}
			\DrawBox[dashed]{11a}{11e}
			\DrawBox[dashed]{12a}{12e}
			\DrawBox[dashed]{13a}{13e}
			\DrawBox[dashed]{14a}{14e}
			\DrawBox[dashed]{15a}{15e}
			\DrawBox[dashed]{16a}{16e}
			\DrawBox[dashed]{17a}{17e}
		}}
	\end{sidewaystable}
\end{subtables}

%\clearpage
%\section{Subject"=plus"=verb combinations}
%\label{sec:tonesofSVcomb}

\begin{sidewaystable}[p]
	\caption{\label{tab:thetonepatternsofsubjectREF}The tone patterns of subject"=plus"=verb combinations, in
		surface phonological transcription. From \sectref{sec:thefactssubjectandverb}.}
	\begin{tabularx}{\textheight}{ l@{\hspace{6mm}} Q l@{\hspace{6mm}} l@{\hspace{6mm}} l@{\hspace{6mm}} l@{\hspace{6mm}} Q }
		\lsptoprule
		& tone of verb & & & & &\\ \cmidrule{2-7}	
		tone of noun & H & M\textsubscript{a} & M\textsubscript{b} & L\textsubscript{a} & L\textsubscript{b} & MH\\ \midrule
		LM, LH & L.H & L.M+M & L.M+M & L.H & L.H & L.MH\\
		M & M.M+L & M.M+M & M.M+M & M.L & M.L & M.MH\\
		L & M.M+L & L.L  & M.M+M & L.L & L.L~/ M.L & L.L\\
		H & M.M+L & M.M+L & M.M+L & M.MH & M.MH & M.L\\
		MH & M.H & M.H & M.H & M.MH & M.MH & M.H\\ \addlinespace \hdashline \addlinespace
		M & M.M.M+L & M.M.M+M & M.M.M+M & M.M.L & M.M.L & M.M.MH\\
		\#H & M.M.M+L & M.M.M+L & M.M.M+L & M.M.MH & M.M.MH & M.M.L\\
		MH\# & M.M.MH & M.M.MH & M.M.MH & M.M.MH & M.M.MH & M.M.H\\
		H\$ & M.M.M+L & M.M.M+L & M.M.M+L~/ M.M.M+H & M.M.MH & M.M.MH & M.H.L\\
		L & L.L.L & L.L.L & L.L.L & L.L.L & L.L.L & L.L.H\\
		L\# & M.L.L & M.L.L & M.L.L & M.L.L & M.L.L & M.L.L\\
		LM+MH\# & L.M.M+L & L.M.M+L & L.M.M+L & L.M.MH & L.M.MH & L.M.H\\
		LM+\#H & L.M.M+L & L.M.M+M & L.M.M+M & L.M.L & L.M.MH & L.M.MH\\
		LM & L.M.M+L & L.M.M+M & L.M.M+M & L.M.L & L.M.L & L.M.MH\\
		LH & L.H.L & L.H.L & L.H.L & L.H.L & L.H.L & L.H.L\\
		H\# & M.H.L & M.H.L & M.H.L & M.H.L & M.H.L & M.H.L\\
		\lspbottomrule
	\end{tabularx}
\end{sidewaystable}

%\clearpage
%\section{Object"=plus"=verb combinations}
%\label{sec:tonesofOVcomb}


\begin{sidewaystable}[p]
	\caption{\label{tab:thetonepatternsofobjectREF}The tone patterns of object"=plus"=verb combinations. From \sectref{sec:thefactsobjectandnonprefixedverb}.}
	{\renewcommand{\arraystretch}{1.1}
		\begin{tabularx}{\textheight}{ l@{\hspace{6mm}} Q l@{\hspace{6mm}} l@{\hspace{6mm}} l@{\hspace{6mm}} l@{\hspace{6mm}} Q }
			\lsptoprule
			& tone of verb\\\cmidrule{2-7}
			tone of noun & H & M\textsubscript{a} & M\textsubscript{b} & L\textsubscript{a} & L\textsubscript{b} & MH\\ \midrule
			LM & L.M+L & L.M+M & L.M+M & L.M+L & L.M+L & L.MH\\
			LH & L.L / L.H & L.H & L.L / L.H & L.H & L.H / L.L & L.MH\\
			M & M.M+L & M.M+M & M.M+M & M.L & M.L & M.MH\\
			L & L.L & M.M+M & M.M+M / L.L & L.L & L.L / M.L & L.L\\
			H & M.M+L & M.L & M.M+L & M.H & M.MH & M.L\\
			MH & M.H & M.H & M.H & M.H & M.MH & M.H\\  \addlinespace \hdashline \addlinespace
			M & M.M.M+L & M.M.M+M & M.M.M+M & M.M.L & M.M.L & M.M.MH\\
			\#H & M.M.M+L & M.M.L & M.M.M+L & M.M.H & M.M.MH & M.M.L\\
			MH\# & M.M.MH & M.M.H+L & M.M.MH & M.M.H & M.M.MH & M.M.H\\
			H\$ & M.M.M+L & M.H.L & M.M.M+L & M.M.H & M.M.MH & M.H.L\\
			L & L.L.L & L.L.H & L.L.L & L.L.H & L.L.L & L.L.H\\
			L\# & M.L.L & M.L.L & M.L.L & M.L.L & M.L.L & M.L.L\\
			LM+MH\# & L.M.M+L & L.M.H & L.M.M+L & L.M.H & L.M.MH & L.M.H\\
			LM+\#H & L.M.M+L & L.M.L & L.M.M+L & L.M.H & L.M.L / L.M.MH & L.M.L\\
			LM & L.M.M+L & L.M.M+M & L.M.M+M & L.M.L & L.M.L & L.M.MH\\
			LH & L.H.L & L.H.L & L.H.L & L.H.L & L.H.L & L.H.L\\
			H\# & M.H.L & M.H.L & M.H.L & M.H.L & M.H.L & M.H.L\\
			\lspbottomrule
		\end{tabularx}}
	\end{sidewaystable}

\end{refsection}