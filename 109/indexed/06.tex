\chapter{Verbs and their combinatory properties}
\label{chap:verbsandtheircombinatoryproperties}

This chapter discusses the tones of verbs and adjectives, and their combinatory properties.

\section{The lexical tones of verbs}
\label{sec:thelexicaltonesofverbs}

\subsection{Overview}
\label{sec:overview}

For \is{monosyllables}monosyllabic verbs, which are an overwhelming majority, seven tonal categories have come to light, as shown in
\tabref{tab:thelexicaltonesofverbs}. 

%% Using subtables to obtain numbering as 1a and 1b.
%\begin{subtables}

\begin{table}[h!!]
\caption{The seven tonal categories of {monosyllabic} verbs: behaviour in four different contexts.}
\label{tab:thelexicaltonesofverbs}
{\renewcommand{\arraystretch}{1.20}
\begin{tabularx}{\textwidth} { P{25mm} P{20mm} Q Q  P{20mm} }
	% {\textheight}{ l@{\hspace{10mm}} l@{\hspace{10mm}} Q l@{\hspace{10mm}} Q l@{\hspace{10mm}} Q }
\lsptoprule
	 example & \is{form!in isolation}in isolation & \textsc{neg} & \textsc{accomp} & V+‘a~bit’\\ \midrule
	  \ipa{dzɯ} ‘to eat’ & \tikzmark{0a}M & M.H & M.H & \lshadedcell M.M.M\\ 
	 \ipa{hwæ} ‘to buy’ &  \tikzmark{1a} & \tikzmark{1b}M.M & \tikzmark{1c}M.M & \shadedcell M.H.L\\
	 \ipa{tɕʰi} ‘to sell’ & \tikzmark{2a} & \hspace*{\fill}\tikzmark{2b} & \hspace*{\fill}\tikzmark{2c} & \lshadedcell M.M.M\\
	 \ipa{bi} ‘to go’ & \hspace*{\fill}\tikzmark{3a} & \hspace*{\fill}\tikzmark{3b} & \tikzmark{3c}M.L & n.a.\\ 
	\ipa{dze} ‘to cut’ & \tikzmark{4a}LH & \tikzmark{4b}M.L & \hspace*{\fill}\tikzmark{4c} & M.M.H\\
	\ipa{ʈʰɯ} ‘to drink’ & \hspace*{\fill}\tikzmark{5a} & \hspace*{\fill}\tikzmark{5b} & \hspace*{\fill}\tikzmark{5c} & M.M.MH\\ 
	\ipa{lɑ} ‘to strike’ & \tikzmark{6a}MH & M.MH & M.MH & \shadedcell M.H.L\\
\lspbottomrule
\end{tabularx}}
\DrawBox{0a}{3a}
\DrawBox{4a}{5a}
\DrawBox{4b}{5b}
\DrawBox{3c}{5c}
\DrawBox{1b}{3b}
\DrawBox{1c}{2c}
\end{table}

%\end{subtables}

%The subset of intransitive verbs within the M tone category labelled as ‘M\textsubscript{c}’ will be discussed further below, \sectref{sec:tonemcasubsetoffiveintransitiveverbswithinthemtonecategory}.
The four contexts shown in \tabref{tab:thelexicaltonesofverbs} are: (i)~\is{form!in isolation}in isolation, (ii)~with the {negation} \is{prefixes}prefix, (iii)~with the {accomplished} \is{prefixes}prefix, and (iv)~with /\ipa{ɖɯ˧-kʰwɤ˥\$}/ ‘a piece’ or
/\ipa{ɖɯ˧-ʈʰɤ˥\$}/ ‘a drop’ as an~object (meaning ‘to V a~bit’).\footnote{The two {numeral}"=plus"=classifier phrases ‘a
	piece’ and ‘a drop’ have the same tone: H\$. The tones of
	{numeral}"=plus"=classifier phrases are analyzed in Chapter~\ref{chap:classifiers}, and the H\$ tone in \sectref{sec:wordfinalandmorphologicalnucleusfinalHtones}.} These four contexts were chosen because their combination reveals all seven categories, even though
each context taken individually only distinguishes three or four. Neutralizations are reflected in cells that share the same tone pattern in a~given column; they are indicated by a~dashed box if the cells are adjacent, and by shading if they are nonadjacent. 

\begin{figure}[h!!]
	\includegraphics[width=\textwidth]{figures/ms/1027ABit.jpg}
	\caption{Field notes (2007). At the beginning of a~work session, the consultant cheerfully said /\ipa{wɤ˩˥~| ɖɯ˧-kʰwɤ˧ ʐwɤ˧{$\sim$}ʐwɤ˥}/ ‘Let's have another chat!’ This construction, ‘to V a~bit’, was written down, and elicited with various verbs. It turned out to be a~useful test for the tones of verbs.}
	\label{fig:msABIT}
\end{figure}

While the data in \tabref{tab:thelexicaltonesofverbs} demonstrates the existence of seven distinct categories, it leaves several
analytic possibilities open. Observations made in the analysis of the tonal categories of nouns (in Chapter~\ref{chap:thelexicaltonesofnouns}) clarify that the diversity of tonal patterns is highest after an initial M tone, since in that position there is no phonological prohibition against H, M, L or MH (only LM and LH are ruled out, by Rule~5, about which see \sectref{sec:alistoftonerules}). The behaviour of the seven classes of verbs after a~M"=tone \is{prefixes}prefix thus appeared as a~promising guide to their underlying tone. But a~complication here is that not all M"=tone prefixes yield the same tonal results in association with verbs. Specifically, ‘to go’ yields /\ipa{mɤ˧-bi˧}/ with the {negation} \is{prefixes}prefix (tone pattern: M.M), and /\ipa{le˧-bi˩}/ with the {accomplished} \is{prefixes}prefix (tone pattern: M.L). After examining a~set of M"=tone prefixes, such as the {prohibitive} /\ipa{tʰɑ˧}-/ and the {durative} /\ipa{tʰi˧}-/, it appeared that the {accomplished} \is{prefixes}prefix is an~outlier in this respect. The {negation} was chosen as a~reference \is{prefixes}prefix to test the phonological nature of the tones of verbs. 

It was observed that, following the {negation} \is{prefixes}prefix, a~verb carries one of four tones: H, M, L, and MH. These are interpreted as four main categories. Within these, some \is{subcategories of lexical tones}subcategories are identified based on their different behaviour in other contexts. There are three \is{subcategories of lexical tones}subcategories among M-tone verbs: M\textsubscript{a}, M\textsubscript{b} and M\textsubscript{c}, and two \is{subcategories of lexical tones}subcategories among L-tone verbs: L\textsubscript{a} and L\textsubscript{b}. The seven categories distinguished in \tabref{tab:thelexicaltonesofverbs} are therefore labelled as H, M\textsubscript{a}, M\textsubscript{b}, M\textsubscript{c}, L\textsubscript{a}, L\textsubscript{b} and MH, as shown in \tabref{tab:Utonesofverbs}, which has the same contents as \tabref{tab:thelexicaltonesofverbs} plus an indication of the underlying tones.

\begin{table}[h!!]
	\caption{The seven tonal categories of {monosyllabic} verbs: analysis into H, M, L and LH tones.}
	\label{tab:Utonesofverbs}
	{\renewcommand{\arraystretch}{1.20}
		\begin{tabularx}{\textwidth}{ P{8mm} P{25mm} P{20mm} Q Q  P{20mm} }
			% {\textheight}{ l@{\hspace{10mm}} l@{\hspace{10mm}} Q l@{\hspace{10mm}} Q l@{\hspace{10mm}} Q }
			\lsptoprule
			tone & example & \is{form!in isolation}in isolation & \textsc{neg} & \textsc{accomp} & V+‘a~bit’\\ \midrule
			H &  \ipa{dzɯ˥} ‘to eat’ & \tikzmark{0a}M & M.H & M.H & \lshadedcell M.M.M\\ 
			M\textsubscript{a} & \ipa{hwæ˧\textsubscript{a}} ‘to buy’ &  \tikzmark{1a} & \tikzmark{1b}M.M & \tikzmark{1c}M.M & \shadedcell M.H.L\\
			M\textsubscript{b} & \ipa{tɕʰi˧\textsubscript{b}} ‘to sell’ & \tikzmark{2a} & \hspace*{\fill}\tikzmark{2b} & \hspace*{\fill}\tikzmark{2c} & \lshadedcell M.M.M\\
			M\textsubscript{c}  & \ipa{bi˧\textsubscript{c}} ‘to go’ & \hspace*{\fill}\tikzmark{3a} & \hspace*{\fill}\tikzmark{3b} & \tikzmark{3c}M.L & n.a.\\ 
			L\textsubscript{a} & \ipa{dze˩\textsubscript{a}} ‘to cut’ & \tikzmark{4a}LH & \tikzmark{4b}M.L & \hspace*{\fill}\tikzmark{4c} & M.M.H\\
			L\textsubscript{b}  & \ipa{ʈʰɯ˩\textsubscript{b}} ‘to drink’ & \hspace*{\fill}\tikzmark{5a} & \hspace*{\fill}\tikzmark{5b} & \hspace*{\fill}\tikzmark{5c} & M.M.MH\\ 
			MH &  \ipa{lɑ˧˥} ‘to strike’ & \tikzmark{6a}MH & M.MH & M.MH & \shadedcell M.H.L\\
			\lspbottomrule
		\end{tabularx}}
		\DrawBox{0a}{3a}
		\DrawBox{4a}{5a}
		\DrawBox{4b}{5b}
		\DrawBox{3c}{5c}
		\DrawBox{1b}{3b}
		\DrawBox{1c}{2c}
	\end{table}

\largerpage
Realizations \is{form!in isolation}in isolation, which only distinguish three subsets, make sense in light of this analysis. H tone is
realized as M due to the \isi{neutralization} of H and M in tone"=group"=initial position (Rule 3; see the list of phonological tone rules in~\sectref{sec:alistoftonerules}). M tones (M\textsubscript{a}, M\textsubscript{b} and M\textsubscript{c}) are straightforwardly realized as M. Tones L\textsubscript{a} and L\textsubscript{b} are both realized as LH due
to the post"=lexical addition of a~H tone to all-L tone groups (by Rule~7); L-tone verbs thus behave unlike L-tone nouns, which surface with M tone \is{form!in isolation}in isolation, as explained in Chapter~\ref{chap:thelexicaltonesofnouns}. This is
one of many pieces of evidence showing that the tone system of Yongning Na is not based only on a~set of
phonological rules that apply across"=the"=board in all contexts, but also has morphotonological rules: \is{tone rules}tone rules that are specific to a~given morphological context.

Tones M\textsubscript{a} and M\textsubscript{b} yield the same
tone pattern in association with the {negation} \is{prefixes}prefix, as do L\textsubscript{a} and L\textsubscript{b}, but they are distinguished in the
fourth context. Conversely, the tone pairs \{M\textsubscript{a}, MH\} and \{M\textsubscript{b}, H\} yield the same tonal pattern when associated
with the object ‘a piece’/‘a drop’ but are distinguished after the {negation} \is{prefixes}prefix. Finally, tone category M\textsubscript{c} has a~different behaviour from M\textsubscript{a} and M\textsubscript{b} after the {accomplished} \is{prefixes}prefix.

Examples of predicates of the seven categories are presented in \tabref{tab:examplesofthesixcategoriesofverbs}.

\begin{table}%[t]
\caption{\label{tab:examplesofthesixcategoriesofverbs}
	Examples of the seven categories of verbs.}
{\renewcommand{\arraystretch}{1.35}
\begin{tabularx}{\textwidth}{ l Q }
\lsptoprule
	tone & examples\\ \midrule
	H & \ipa{dzɯ˥} ‘to eat’, \ipa{bv̩˥} ‘to divide’, \ipa{ʝi˥} ‘to do’, \ipa{se˥} ‘to walk’, \ipa{ʈʂʰæ˥} ‘to wash’\\
	M\textsubscript{a} & \ipa{hwæ˧\textsubscript{a}} ‘to buy’, \ipa{hõ˧\textsubscript{a}} ‘to go away.\textsc{imp}’, \ipa{ki˧\textsubscript{a}} ‘to give’, \ipa{li˧\textsubscript{a}}~‘to~watch’, \ipa{mæ˧\textsubscript{a}}~‘to~catch hold of’\\
	M\textsubscript{b} & \ipa{tɕʰi˧\textsubscript{b}} ‘to sell’, \ipa{ɖɯ˧\textsubscript{b}} ‘to obtain’,  \ipa{ɖʐæ˧\textsubscript{b}} ‘to ride’, \ipa{pʰæ˧\textsubscript{b}} ‘to fasten’, \ipa{ɲi˧\textsubscript{b}} ‘to need’\\
	M\textsubscript{c} & \ipa{bi˧\textsubscript{c}} ‘to go’, \ipa{hɯ˧\textsubscript{c}} ‘to go.\textsc{pst}’, \ipa{gv̩˧\textsubscript{c}} ‘to go by (of
	time)’, \ipa{ʝi˧\textsubscript{c}} ‘to come’, \ipa{pv̩˧\textsubscript{c}} ‘to chant’\\
	L\textsubscript{a} & \ipa{dze˩\textsubscript{a}} ‘to cut’, \ipa{bæ˩\textsubscript{a}} ‘to sweep’, \ipa{ti˩\textsubscript{a}} ‘to hit (gently)’, \ipa{tɕi˩\textsubscript{a}}~‘to write’, \ipa{dzi˩\textsubscript{a}} ‘to sit’\\
	L\textsubscript{b} & \ipa{ʈʰɯ˩\textsubscript{b}} ‘to drink’, \ipa{dɑ˩\textsubscript{b}} ‘to weave’, \ipa{do˩\textsubscript{b}} ‘to see’, \ipa{mɤ˩\textsubscript{b}} ‘to eat food in powder form, typically tsamba (roasted flour)’, \ipa{ʐwɤ˩\textsubscript{b}} ‘to speak’\\
	MH & \ipa{ɕjɤ˧˥} ‘to try; to taste’, \ipa{gɤ˧˥} ‘to carry on one’s shoulder’, \ipa{lɑ˧˥} ‘to strike’, \ipa{tɕɤ˧˥} ‘to boil’, \ipa{ʐv̩˧˥} ‘to sew (clothes)’\\
\lspbottomrule
\end{tabularx}}
\end{table}

The labels used for the sets of categories \{L\textsubscript{a}, L\textsubscript{b}\} and \{M\textsubscript{a}, M\textsubscript{b}, M\textsubscript{c}\} are
deliberately abstract, for want of decisive evidence about the phonological nature of the categories
at issue. The letters are assigned on the basis of relative frequency in the lexicon. Among L tones, the ‘to cut’ type is about five times as frequent as the ‘to drink’ type. Among M tones, the ‘to buy’ type is twice as frequent as the ‘to sell’ type, and the ‘to go’ type is infrequent.

 The ‘to cut’ and ‘to drink’ types, labelled here as L\textsubscript{a} and L\textsubscript{b}, are both analyzed as containing a~L tone level,
 since they are realized with L tone after the {negation} \is{prefixes}prefix. There is limited evidence on the
 phonological nature of the difference between the L\textsubscript{a} and L\textsubscript{b} categories. One of the two could be analyzed as a~simple L tone, and the other as a~\is{tonal contour}contour (LM or LH), on the {analogy} of nouns, but this would be
 arbitrary, since there is no compelling evidence that either of these categories consists of
 a~\is{tonal contour}contour. The apparent economy gained from using labels similar to those of nouns would come
 together with high costs in terms of descriptive adequacy. On nouns, the \mbox{//LM//} and \mbox{//LH//} \is{tonal contour}contour tones surface as such \is{form!in isolation}in isolation (where they are neutralized to /LH/), unlike the //L// tone, which surfaces
 as /M/ \is{form!in isolation}in isolation. There is no such difference between the L\textsubscript{a} and L\textsubscript{b} categories of verbs, which
 both surface with a~/LH/ \is{tonal contour}contour \is{form!in isolation}in isolation.


\subsection{About subsets of M"=tone verbs}
\label{sec:tonemcasubsetoffiveintransitiveverbswithinthemtonecategory}

M"=tone verbs, defined as those that are realized with M tone after the {negation} \is{prefixes}prefix, fall into three subsets: M\textsubscript{a}, M\textsubscript{b} and M\textsubscript{c}. It would be satisfactory from the point of view of economy of description to reserve the label M for one of the three, and to assign to the others labels selected from the inventory of tone categories of nouns, such as \#H. No evidence has
been found to support such identifications, however, hence the choice to adopt noncommittal abstract labels with subscript letters.

Tone category M\textsubscript{c} hosts five verbs that behave
like M-tone verbs of the M\textsubscript{a} category except in a~few contexts, such as when preceded by the {accomplished} \is{prefixes}prefix, /\ipa{le˧}-/. These
are /\ipa{bi˧\textsubscript{c}}/ ‘to go’, and its past form /\ipa{hɯ˧\textsubscript{c}}/; /\ipa{gv̩˧\textsubscript{c}}/ ‘to go by, to flow, to fly (of
time)’; /\ipa{ʝi˧\textsubscript{c}}/ ‘to come’; and /\ipa{pv̩˧\textsubscript{c}}/ ‘to chant, to perform (a sacrifice, a~ritual,
a~festival)’. With the {accomplished} \is{prefixes}prefix and the {perfective} \is{suffixes}suffix, the pattern
is M.L.L, as shown in (\ref{ex:went})"=(\ref{ex:prayed}); likewise for the other three verbs: /\ipa{le˧-hɯ˩-ze˩}/, /\ipa{le˧-gv̩˩-ze˩}/, and /\ipa{le˧-ʝi˩-ze˩}/. This contrasts with the other verbs of the M tone category (~i.e.\ the M\textsubscript{a} and M\textsubscript{b} subtypes), which carry M tone after the
{accomplished} \is{prefixes}prefix. 

\begin{exe}
	\ex
	\label{ex:went}
	\ipaex{le˧-bi˩-ze˩}\\
	\gll le˧-		bi˧\textsubscript{c}	-ze˧\\
	\textsc{accomp}		to\_go	\textsc{pfv}\\
	\glt ‘[she/he/they{\dots}] went’
\end{exe}



\begin{exe}
	\ex
	\label{ex:prayed}
	\ipaex{le˧-pv̩˩-ze˩}\\
	\gll le˧-		pv̩˧\textsubscript{c}	-ze˧\\
	\textsc{accomp}		to\_chant	\textsc{pfv}\\
	\glt ‘[she/he/they{\dots}] chanted’
\end{exe}

The difference between M\textsubscript{a} and M\textsubscript{c} is not related to verb valency: all M\textsubscript{c}"=tone verbs have intransitive uses, but one of the five (‘to chant’) can also be used transitively; conversely, not all intransitive verbs belong in this tonal category, witness ‘to die’, /\ipa{ʂɯ˧\textsubscript{a}}/, which yields /\ipa{le˧-ʂɯ˧}/ (Sister3.11, 95).

Since these five verbs have the same behaviour as M\textsubscript{a}"=tone verbs in most contexts, they could be described as a~subset of the M\textsubscript{a} category; a~further diacritic could be added to their tone
label, yielding something such as M\textsubscript{a}’ (M\textsubscript{a} prime). However, it appeared less awkward typographically
to label them as M\textsubscript{c}, a~third \is{subcategories of lexical tones}subcategory within M tones. 

The tonal behaviour of verbs in category M\textsubscript{c} calls for analysis. An~observation that may be relevant is that the {accomplished} \is{prefixes}prefix
/\ipa{le˧}-/ in association with these verbs can carry special semantic connotations. With these verbs,
/\ipa{le˧}-/ can carry the meaning ‘back/to return’, as in (\ref{ex:nodinner})~and (\ref{ex:stones}):
\begin{exe}
  \ex
  \label{ex:nodinner}
  \ipaex{le˧-bi˩-dʑo˩, {\kern2pt}|{\kern2pt} ʈʂʰwɤ˧ {\kern2pt}|{\kern2pt} ɖɯ˧ mɤ˧-kv̩˧ tsɯ˥ {\kern2pt}|{\kern2pt} mv̩˩!}\\
  \gll le˧-	bi˧\textsubscript{c}	-dʑo˥	ʈʂʰwɤ˥	ɖɯ˧	mɤ˧-	-kv̩˧˥	tsɯ˧˥	mv̩˧\\
  \textsc{accomp}	to\_go	\textsc{top}	dinner	to\_get	\textsc{neg}	\textsc{abilitive}
  \textsc{rep}	\textsc{affirm}\\
  \glt ‘If [the daughter] goes back [to her mother’s home after marriage, she] cannot have dinner
  there.’ [She must not stay there for the night, she has to go back to her new home before
    evening.] (Sister3.116)

  \ex
  \label{ex:stones}
  \ipaex{lv̩˧mi˧ so˩-ʈv̩˩ pɤ˩{$\sim$}pɤ˩! {\kern2pt}|{\kern2pt} le˧-bi˩-ze˩!}\\
  \gll lv̩˧mi˧	so˩-ʈv̩˩	pɤ˧˥	{$\sim$}	le˧-	bi˧\textsubscript{c}	-ze˧\\
  stone	a\_few	to\_carry	\textsc{activity}	\textsc{accomp}	to\_go	\textsc{pfv}\\
  \glt  ‘Tonight, I’ll bring (back) a~couple of stones! I’m going back!’ (Reward.77. Context: a~man
  is compelled by his spouse to go and steal in order to support the family; as he is about to steal
  sweetcorn, he decides to refrain from stealing; instead, he fills his basket with stones and goes back.)
\end{exe}

This meaning is not always present, however, and the tone pattern is the same when the meaning is
‘to go, to set off’, as in
(\ref{ex:ahihi}), where the movement is clearly away from a~familiar place and towards an~unfamiliar one.

\largerpage
\begin{exe}
  \ex
  \label{ex:ahihi}
  \ipaex{“æ.hi.hi!” pi˧, {\kern2pt}|{\kern2pt} le˧-bi˩-zo˩-kv̩˩ {\kern2pt}|{\kern2pt} tsɯ˧˥ {\kern2pt}|{\kern2pt} mv̩˩!}\\
  \gll æ.hi.hi		pi˥	le˧-	bi˧\textsubscript{c}	-zo˧	-kv̩˧˥	tsɯ˧˥	mv̩˧\\
  \textsc{intj}	to\_say	\textsc{accomp}	to\_go	\textsc{obligative}	\textsc{abilitive}
  \textsc{rep}	\textsc{affirm}\\
  \glt ‘[The mother, uncles, aunts and other relatives of the deceased wife’s family shout out a~cry
    of defiance:] “A-hi"=hi!” and they set off [towards the husband’s house]!’ (Sister1.81)
\end{exe}

To sum up: for want of a~principled explanation for the tonal behaviour of these five verbs, it appeared best to set up a~distinct synchronic tone category for them: M\textsubscript{c}.


\subsection{Adjectives as distinct from verbs}
\label{sec:adjectivesasdistinctfromverbs}

\is{adjectives|textbf}
\is{adjectives|(}
The issue whether adjectives constitute a~distinct part of speech has been raised for various languages of Southeast Asia: for a~review and discussion, with special focus on \il{Tai-Kadai}Tai, see \citet{post2008adjectives}. “Some Tibeto-Burman languages have a definable adjective category, usually only marginally distinguishable from nouns or from verbs” \citep[41]{delancey2015adjectival}. Yongning Na is one of these languages: in Na, adjectives behave in most respects like verbs, i.e.\ as \is{stative verbs|textbf}stative verbs, but they have some tonal specificities which require that they be recognized as a~formally distinct class of words. 

Four main tonal categories for \is{monosyllables}monosyllabic adjectives were found: L, M, H, and MH. The L tone
category must further be split into two \is{subcategories of lexical tones}subcategories. The L\textsubscript{b} category only contains
two examples: /\ipa{dʑɤ˩\textsubscript{b}}/ ‘good’ and /\ipa{nɑ˩\textsubscript{b}}/ ‘black, dark’. Details on their behaviour in context are
provided in the course of this chapter.

Importantly, the MH category of adjectives and the MH category of verbs do not always have the same
tonal behaviour: examples are provided in \sectref{sec:ahtonesuffixtherelativizernominalizer} below. Likewise, the L\textsubscript{a} and L\textsubscript{b} categories of
adjectives are not fully parallel to the L\textsubscript{a} and L\textsubscript{b} categories of verbs in terms of their tonal
behaviour. As for the M category of adjectives, no evidence was found for a~division into \is{subcategories of lexical tones}subcategories corresponding to the M\textsubscript{a}, M\textsubscript{b} and M\textsubscript{c} categories set up for verbs. These differences
between the tone system of adjectives and that of verbs exemplify the morphosyntactic
ramifications of tone in Yongning Na.

There also exist \is{disyllables}disyllabic adjectives. Examples include /\ipa{pʰv̩˧ɖɯ˧˥}/ ‘expensive’, from /\ipa{pʰv̩˧}/ ‘price’ and
/\ipa{ɖɯ˩\textsubscript{a}}/ ‘large’, and /\ipa{lo˩ɖɯ˧}/ ‘generous’, from /\ipa{lo˩˧}/ ‘hand’ and, again, /\ipa{ɖɯ˩\textsubscript{a}}/ ‘large’.

In view of the range of tone categories found on nouns, verbs and adjectives in Yongning Na, it is
no wonder that they yield a~wealth of diverse patterns when combined among themselves, and combined
with grammatical morphemes. The structure of the Na verb phrase as schematized by \citet[350–351]{lidz2010} comprises: Manner
adverb~-- Verb complex~-- Causative~-- Intensifier~-- Tense/aspect and modal particles, and auxiliary
verbs~-- Quotative evidential. Additionally, the verb can be preceded by spatial indications such as
‘forward’/’backward’ and ‘upward’/‘downward’. The verb complex may be a~lexical verb, an~\is{existentials}existential
verb, a~\isi{copula}, or a~serial verb construction, which may take a~verbal \is{prefixes}prefix (or two prefixes, in the
case of the {durative} \is{prefixes}prefix followed by the {negation} \is{prefixes}prefix). The following sections explore the verb phrase's morphotonology. The topic of adverbials is not addressed in this chapter, because adverbials seldom interact with verbs; for a~discussion, see~\sectref{sec:someelementsalwaysconstituteatonegroupontheirown}.
\is{adjectives|)}

\section{Reduplication}
\label{sec:reduplication}

Reduplication of verbs is a~highly productive process in Yongning Na. From a~semantic point of view, \isi{reduplication} can convey various types of divergence from the prototype of the action or activity referred to by the verb. If the verb refers to an action, the reduplicated form can warp it towards representation as an activity, as in~(\ref{ex:costume}).  

\begin{exe}
	\ex
	\label{ex:costume}
	\ipaex{jo˧-lo˥dʑo˩ tʰi˩-kʰɯ˩{$\sim$}kʰɯ˩. {\kern2pt}|{\kern2pt} hæ̃˩-lo˩pv̩˩ tʰi˥-kʰɯ˩{$\sim$}kʰɯ˩.}\\
	\gll jo˥	lo˩dʑo˥		tʰi˧-		kʰɯ˧˥	{$\sim$}		hæ̃˩	lo˩pv̩˧˥ 	tʰi˧-		kʰɯ˧˥	{$\sim$}\\
	jade	bracelet	\textsc{dur}	to\_put		\textsc{activity}	gold		ring		\textsc{dur}	to\_put		\textsc{activity}\\
	\glt ‘They adorned her with jade bracelets. They adorned her with gold rings.’ (BuriedAlive2.29. Context: a bride is being prepared for a wedding.\footnote{In view of the ethnological reports about Na family structure reviewed in Appendix B (\sectref{sec:anthropologicalresearchthefascinationofnafamilystructure}), it may come as a~surprise that there should be folk stories about weddings in Yongning. Arranged marriages and difficult relationships between brides and mothers"=in"=law are typical of Confucian cultures (see e.g.~\citealt{rileyInterwoven1994, chanviolence2008}), not of pre-1950s Yongning. But the argument of a~story's fit to a~particular culture runs both ways: a~tale can be appealing because it resonates with one's cultural background, or on the contrary, it can gain appeal from exoticism. The 1990s soap opera Ke Wang \zh{渴望} (‘Yearnings’), which was a~roaring success throughout China \citep[a~phenomenon studied by][]{wangkewang1992}, was also popular in Yongning, where the Na gathered at the homes of television owners to watch the story of Liu Huifang, the ideal daughter"=in"=law (this is narrated by F4 in the document entitled Evenings). Part of the soap opera's appeal to a~Na audience may have been due to the novelty of the social relationships that it stages. A~young {Naxi} woman from the Lijiang plain explained to me in 2010 that she loved watching Chinese soap operas “to see how Han people live”, i.e.\ out of curiosity for cultural habits which she sees as different from {Naxi} custom. A~study in comparative mythology and folklore in this area of the Himalayas would be necessary to trace the origin and development of stories such as that of the unhappy daughter"=in"=law who gets buried alive, a~story also known to {Laze} consultant F7 (the {Laze} version is also available online). The thrills and spills of this story make it a~good candidate for adoption by anyone who likes a~good yarn.})
\end{exe}

The \textit{simplex} form /\ipa{jo˧-lo˥dʑo˩ tʰi˩-kʰɯ˩}/ means ‘to put a~jade bracelet (on someone's wrist)’. It depicts a~well"=delimited, unique event: in the absence of a~distinction between singular and plural, the interpretation of the noun phrase tends to be as singular, unless a~\is{numerals}numeral"=plus"=classifier phrase is added, as in~(\ref{ex:putsimple}).  

\begin{exe}
	\ex
	\label{ex:putsimple}
	\ipaex{jo˧-lo˥dʑo˩{\kern2pt}|{\kern2pt}ɲi˧-ɭɯ˧ tʰi˧-kʰɯ˧˥}\\
	\gll jo˥	lo˩dʑo˥		ɲi˧		ɭɯ˧\textsubscript{b}	tʰi˧-		kʰɯ˧˥\\
	jade	bracelet	two		\textsc{clf}	\textsc{dur}		to\_put\\
	\glt ‘to put two jade bracelets’
\end{exe}

By contrast, (\ref{ex:costume}) means that the bride is adorned with jade bracelets and gold rings by family members. The reduplicated form leads to a~plural interpretation of the nouns, presumably because successive acts of adorning are not performed with the same object: if you adorn someone more that once you would be doing this with yet another ornament, not taking one off to put it on again. 

The use of \isi{reduplication} illustrated by~(\ref{ex:costume}) is glossed as \textsc{activity}. It represents the process denoted by the verb as an activity, rather than a~neatly delimited event. In the text from which (\ref{ex:costume}) is extracted, verb \isi{reduplication} emphasizes the generosity of the family members as a~unified social entity. They collectively go through the prescribed steps for the marriage ritual, giving generous measure of the prescribed offerings; each family member's individual gesture partakes in a~collective activity. This builds a~contrast with later events, when the young woman, feeling estranged at her in"=laws’ home, becomes oblivious of good manners and lapses into anti"=social behaviour (solitary gluttony).

The surface semantic effect of \textsc{activity} is close to that of the {progressive} \is{suffixes}suffix /\ipa{-dʑo˧}/. For instance, the reduplicated form of /\ipa{ʑi˧˥}/ ‘to sleep’ in~(\ref{ex:everyoneasleep}) could be replaced by a~\textit{simplex} form followed by the {progressive} \is{suffixes}suffix /\ipa{-dʑo˧}/, as in~(\ref{ex:everyoneasleepprog}). 

\begin{exe}
	\ex
	\label{ex:everyoneasleep}
	\ipaex{hĩ˧ {\kern2pt}|{\kern2pt} ɖɯ˧-tɑ˧˥ {\kern2pt}|{\kern2pt} le˧-ʑi˩{$\sim$}ʑi˩.}\\
	\gll hĩ˥		ɖɯ˧-tɑ˧˥	le˧-			ʑi˧˥	{$\sim$}\\
	person		all		\textsc{accomp}		to\_sleep	\textsc{activity}\\
	\glt ‘Everyone was asleep.’ (BuriedAlive2.94)
\end{exe}

\begin{exe}
	\ex
	\label{ex:everyoneasleepprog}
	\ipaex{hĩ˧ {\kern2pt}|{\kern2pt} ɖɯ˧-tɑ˧˥ {\kern2pt}|{\kern2pt} le˧-ʑi˧-dʑo˥.}\\
	\gll hĩ˥		ɖɯ˧-tɑ˧˥	le˧-			ʑi˧˥	-dʑo˧\\
	person		all		\textsc{accomp}		to\_sleep	\textsc{prog}\\
	\glt ‘Everyone was asleep.’
\end{exe}

“When \isi{stative verbs} reduplicate, one gets a~reading of added intensity, while
reduplicating non-{stative} verbs gives a~reading of reciprocity of action, or a~semantics of
back-and-forth” \citep[373]{lidz2010}. An example is shown in~(\ref{ex:likedeachother}), where the reduplicated adjective ‘pleased’ serves as an~understatement for ‘in love’ (‘they were in love with each other’).

\begin{exe}
	\ex
	\label{ex:likedeachother}
	\ipaex{ʁo˧dɑ˧, {\kern2pt}|{\kern2pt} no˧bv̩˥-tsʰɯ˩ɻ̍˩ lɑ˩ {\kern2pt}|{\kern2pt} ʈʂʰɯ˧=zɯ˩ | fv̩˧{$\sim$}fv̩˩-ɲi˩!}\\
	\gll ʁo˧dɑ˧		no˧bv̩˥-tsʰɯ˩ɻ̍˩	lɑ˧	ʈʂʰɯ˧=zɯ˩		fv̩˧	{$\sim$} -ɲi˩\\
	before		proper\_name	and		\textsc{3du}	pleased		\textsc{recp}		\textsc{certitude}\\
	\glt ‘Before [my daughter married your son]{\dots} Nobbu Ci’er and [my daughter]{\dots} they used to like each other!’ (BuriedAlive2.136. Context: the unhappy wife's mother explains to the husband's mother what the matter is with her daughter.)
\end{exe}

The online texts contain more than a~thousand examples of \isi{reduplication}. This volume about morphotonology is not the right place to exploit these rich materials to delve any further into the study of the semantic values of \isi{reduplication} in Yongning Na, to study the “paradoxes of fragmentation” \citep{francois2004} that arise through divergence from \textit{simplex} values, or to look for semantic and \is{stylistics}stylistic common denominators to \isi{reduplication} as it applies to different parts of speech.\footnote{Some insights into these topics are found in \citet[372-373, 385, 438, 440]{lidz2010}. Additionally, see \citet{michaudetal2007b} about {reduplication} in {Naxi}.} Instead, the focus here is on the morphotonological patterns of \isi{reduplication}.

\tabref{tab:thetonepatternsofreduplicatedverbsinyongningna} presents the tone patterns of reduplicated verbs in Yongning Na. Two of these verbs are not the same as those in Tables \ref{tab:thelexicaltonesofverbs}-\ref{tab:Utonesofverbs}: ‘to go' does not easily lend itself to \isi{reduplication}, and ‘to chant' was therefore substituted; likewise, ‘to drink' is much less commonly used in reduplicated form than ‘to speak'. For each table in the present chapter, a~choice of example verbs is made (mostly from the list in \tabref{tab:examplesofthesixcategoriesofverbs}) with a~view to avoiding examples that grate and jar on native ears for semantic"=syntactic or pragmatic reasons. Crucially, the seven tonal categories of verbs are homogeneous: the tone patterns of all the verbs in each class are consistent. In other words, the examples ‘to chant' and ‘to speak' in \tabref{tab:thetonepatternsofreduplicatedverbsinyongningna} would behave in the same way as ‘to go' and ‘to drink' if they were in \tabref{tab:Utonesofverbs}, and vice versa. 

The patterns in \tabref{tab:thetonepatternsofreduplicatedverbsinyongningna} are valid for all the verbs for which a~reduplicated form could be elicited. The only apparent \is{exceptions}exception is the disyllabic verb //\ipa{wɤ˩}{\allowbreak}\ipa{{$\sim$}wɤ˩}// (surface form:
/\ipa{wɤ˩{$\sim$}wɤ˩˥}/) ‘to detour past, to bypass’, which looks like an unmistakeable case of \isi{reduplication}, but whose L tone does not correspond to any of the patterns found for the other verbs. At a~push, this verb can be used in {monosyllabic} form, as in (\ref{ex:bypassed}). 

\newpage 
% 
\begin{exe}
	\ex
	\label{ex:bypassed}
	\ipaex{le˧-wɤ˩-ze˩}\\
	\gll le˧-	??wɤ˩		-ze˧\\
	\textsc{accomp}		to\_bypass	\textsc{pfv}\\
	\glt ‘[She/he/they] bypassed’ (elicited example; the {question} marks ‘??’ are intended as an indication of the problematic status of the {monosyllabic} verb form)
\end{exe}

In view of the M.L.L tone pattern of example (\ref{ex:bypassed}), the \is{monosyllables}monosyllable must be analyzed as carrying lexical L tone: //\ipa{wɤ˩}//. But this \is{monosyllables}monosyllable seems to obtain by truncation of the disyllable, rather than the other way round. Thus, there are strong reasons to consider that the disyllabic form //\ipa{wɤ˩{$\sim$}wɤ˩}// ‘to bypass’ is not synchronically derived from a~{monosyllabic} form, but belongs to the small and heterogeneous set of disyllabic verbs found in Yongning Na. Consequently, this verb does not constitute a~\isi{counterexample} to the generalizations shown in \tabref{tab:thetonepatternsofreduplicatedverbsinyongningna}.

\begin{table}%[t]
\caption{\label{tab:thetonepatternsofreduplicatedverbsinyongningna}The tone patterns of reduplicated verbs.}
\begin{tabularx}{\textwidth}{ l@{\hspace{5mm}} l@{\hspace{5mm}} l@{\hspace{5mm}} l@{\hspace{4mm}} l@{\hspace{3mm}} l@{\hspace{3mm}} }
\lsptoprule
	tone & example & meaning & \isi{reduplication} & surface tone & underlying tone\\ \midrule
	H & \ipa{dzɯ˥} & to eat & \ipa{dzɯ˧{$\sim$}dzɯ\#˥} & M.M & \#H\\
	M\textsubscript{a} & \ipa{hwæ˧\textsubscript{a}} & to buy & \ipa{hwæ˥{$\sim$}hwæ˩} & M.L & H--\\
	M\textsubscript{b} & \ipa{tɕʰi˧\textsubscript{b}} & to sell & \ipa{tɕʰi˧{$\sim$}tɕʰi˧} & M.M & M\\
	M\textsubscript{c} & \ipa{pv̩˧\textsubscript{c}} & to chant & \ipa{pv̩˥{$\sim$}pv̩˩} & M.L & M\\
	L\textsubscript{a} & \ipa{dze˩\textsubscript{a}} & to cut & \ipa{dze˧{$\sim$}dze˥} & M.H & H\#\\
	L\textsubscript{b} & \ipa{ʐwɤ˩\textsubscript{b}} & to speak & \ipa{ʐwɤ˥{$\sim$}ʐwɤ˩} & M.L & H--\\
	MH & \ipa{lɑ˧˥} & to strike & \ipa{lɑ˩{$\sim$}lɑ˧˥} & L.MH & LM+MH\#\\
\lspbottomrule
\end{tabularx}
\end{table}


\begin{table}%[t]
\caption{Surface phonological representation of reduplicated verbs in two different carrier phrases.}
	\label{tab:reduplicatedverbsinacarrierphrase}
	\begin{tabularx}{\textwidth}{ l@{\hspace{6mm}} l@{\hspace{6mm}} l@{\hspace{6mm}} l@{\hspace{6mm}} Q }
		\lsptoprule
		tone & example & meaning & \textsc{accomp}+\textsc{redupl} & ‘to V a~little’\\ \midrule
		H & \ipa{dzɯ˥} & to eat & \ipa{le˧-dzɯ˧{$\sim$}dzɯ˧} & \ipa{ɖɯ˧-dzɯ˧{$\sim$}dzɯ˧-ɻ̍˥}\\
		M\textsubscript{a} & \ipa{hwæ˧\textsubscript{a}} & to buy & \ipa{le˧-hwæ˥{$\sim$}hwæ˩} & \ipa{ɖɯ˧-hwæ˥{$\sim$}hwæ˩-ɻ̍˩}\\
		M\textsubscript{b} & \ipa{tɕʰi˧\textsubscript{b}} & to sell & \ipa{le˧-tɕʰi˧{$\sim$}tɕʰi˧} & \ipa{ɖɯ˧-tɕʰi˧{$\sim$}tɕʰi˧-ɻ̍˩}\\
		M\textsubscript{c} & \ipa{pv̩˧\textsubscript{c}} & to chant & \ipa{le˧-pv̩˥{$\sim$}pv̩˩} & \ipa{ɖɯ˧-pv̩˧{$\sim$}pv̩˥-ɻ̍˩}\\
		L\textsubscript{a} & \ipa{bæ˩\textsubscript{a}} & to sweep & \ipa{le˧-bæ˧{$\sim$}bæ˥} & \ipa{ɖɯ˧-bæ˧{$\sim$}bæ˥-ɻ̍˩}\\
		L\textsubscript{b} & \ipa{ʐwɤ˩\textsubscript{b}} & to speak & \ipa{le˧-ʐwɤ˥{$\sim$}ʐwɤ˩} & \ipa{ɖɯ˧-ʐwɤ˥{$\sim$}ʐwɤ˩-ɻ̍˩}\\
		MH & \ipa{lɑ˧˥} & to strike & \ipa{le˧-lɑ˩{$\sim$}lɑ˩} & \ipa{ɖɯ˧-lɑ˩{$\sim$}lɑ˩-ɻ̍˩}\\
		\lspbottomrule
	\end{tabularx}
\end{table}

The tonal
string that obtains \is{form!in isolation}in isolation is indicated in the “surface tone” column in \tabref{tab:thetonepatternsofreduplicatedverbsinyongningna}. The underlying
tone (indicated in the last column, and also used for transcribing the reduplicated expressions in the fourth column) was arrived at by examining the behaviour of reduplicated expressions in different contexts. (The corresponding recordings are VerbReduplObj and
VerbReduplObj2.) \tabref{tab:reduplicatedverbsinacarrierphrase} sets out the tone patterns that obtain in frames (\ref{ex:accompred}) and (\ref{ex:delimredincho}). Frame (\ref{ex:delimredincho}) is one of the few contexts where M\textsubscript{a} and M\textsubscript{c} yield different results. 

\newpage 

%
\begin{exe}
	\ex
	\label{ex:accompred}
	\ipaex{le˧-V{$\sim$}V}\\
	\gll le˧-		V					{$\sim$}\\
	\textsc{accomp}		\textit{target~verb}				\textsc{red}\\
	\glt ‘to V’
\end{exe}

\begin{exe}
	\ex
	\label{ex:delimredincho}
	\ipaex{ɖɯ˧-V{$\sim$}V-ɻ̍˩}\\
	\gll ɖɯ˧-		V					{$\sim$}		-ɻ̍˩\\
	\textsc{delimitative}		\textit{target~verb}				\textsc{red}		\textsc{inchoative}\\
	\glt ‘to V a~little’
\end{exe}

Finally, \tabref{tab:reduplicatedverbswiththehtoneobjectthings} shows the result of associating the object /\ipa{tso˧{$\sim$}tso˧}/ ‘thing’ to a~reduplicated verb. 

 
Interpretation of the tone pattern for H"=tone verbs is guided by the data in \tabref{tab:reduplicatedverbsinacarrierphrase}. The presence of a~H tone on the last syllable of the expression /\ipa{ɖɯ˧-dzɯ˧{$\sim$}dzɯ˧-ɻ̍˥}/ ‘to eat a~little’, from /\ipa{dzɯ˥}/ ‘to eat’, suggests that the reduplicated expression contains a~H tone. Since this H tone does not surface when the expression is said \is{form!in isolation}in isolation (where the reduplicated verb surfaces as /\ipa{dzɯ˧{\allowbreak}{$\sim$}dzɯ˧}/), the tonal category cannot be //H--// (an initial H tone), nor \mbox{//H\#//} (a~final H tone), nor \mbox{//H\$//} (a~\textit{gliding} H tone: see \sectref{sec:wordfinalandmorphologicalnucleusfinalHtones}). On the other hand, its behaviour is consistent with interpretation as \mbox{//\#H//} (a~\textit{floating} H tone), and this hypothesis is therefore adopted in \tabref{tab:thetonepatternsofreduplicatedverbsinyongningna}. 


\begin{table}[t]
	\caption{\label{tab:reduplicatedverbswiththehtoneobjectthings}Reduplicated verbs with the \#H-tone object ‘things’.}
	\begin{tabularx}{\textwidth}{ l@{\hspace{6mm}} l@{\hspace{6mm}} l@{\hspace{6mm}} l@{\hspace{6mm}} Q }
		\lsptoprule
		tone & example & meaning & \isi{reduplication} & ‘things’+reduplicated V\\ \midrule
		H & \ipa{dzɯ˥} & to eat & \ipa{dzɯ˧{$\sim$}dzɯ\#˥} & \ipa{tso˧{$\sim$}tso˧ dzɯ˧{$\sim$}dzɯ˧}\\
		M\textsubscript{a} & \ipa{hwæ˧\textsubscript{a}} & to buy & \ipa{hwæ˥{$\sim$}hwæ˩} & \ipa{tso˧{$\sim$}tso˧ hwæ˧{$\sim$}hwæ˥}\\
		M\textsubscript{b} & \ipa{tɕʰi˧\textsubscript{b}} & to sell & \ipa{tɕʰi˧{$\sim$}tɕʰi˧} & \ipa{tso˧{$\sim$}tso˧ tɕʰi˧{$\sim$}tɕʰi˧}\\
		L\textsubscript{a} & \ipa{bæ˩\textsubscript{a}} & to sweep & \ipa{bæ˧{$\sim$}bæ˥} & \ipa{tso˧{$\sim$}tso˧ bæ˧{$\sim$}bæ˥}\\
		L\textsubscript{b} & \ipa{ʐwɤ˩\textsubscript{b}} & to speak & \ipa{ʐwɤ˥{$\sim$}ʐwɤ˩} & \ipa{tso˧{$\sim$}tso˧ ʐwɤ˧{$\sim$}ʐwɤ˥}\\
		MH & \ipa{lɑ˧˥} & to strike & \ipa{lɑ˩{$\sim$}lɑ˧˥} & \ipa{tso˧{$\sim$}tso˧ lɑ˥{$\sim$}lɑ˩}\\
		\lspbottomrule
	\end{tabularx}
\end{table}

\newpage
The /M.L/ surface pattern in the \isi{reduplication} of M\textsubscript{a}"=tone verbs (e.g.~/\ipa{hwæ˧\textsubscript{a}}/ ‘to buy’ → /\ipa{hwæ˧{$\sim$}hwæ˩}/) could be the realization of various underlying tones, such as //L\#// (a~final L tone) or //H--// (a~H tone associated to the first part of the reduplicated expression, i.e.\ its first syllable). Relevant evidence comes from the contexts shown in \tabref{tab:reduplicatedverbsinacarrierphrase}, where the first syllable of the reduplicated expression carries H tone, guiding towards an interpretation of the underlying pattern as //H--// and not //L\#//.\footnote{An~initial H tone can never surface as such, due to the {neutralization} of //H// and \mbox{//M//} in tone"=group"=initial position: this is referred to as Rule~3 (see \sectref{sec:alistoftonerules}).}

Reduplication of M\textsubscript{b}"=tone verbs is simplest: the reduplicated expression carries M tone \is{form!in isolation}in isolation (e.g.~/\ipa{tɕʰi˧\textsubscript{b}}/ ‘to sell’ → /\ipa{tɕʰi˧{$\sim$}tɕʰi˧}/), and it is not modified in the frames shown in \tabref{tab:reduplicatedverbsinacarrierphrase} and \tabref{tab:reduplicatedverbswiththehtoneobjectthings}. This strongly suggests an analysis of the reduplicated verb as carrying M tone.

Analysis for the remaining three tones is also straightforward. L\textsubscript{a} tone reduplicates to a~pattern with a~H tone attached to its last syllable (see \tabref{tab:thetonepatternsofreduplicatedverbsinyongningna}), hence analysis as \mbox{//H\#//}. The analysis for L\textsubscript{b} tone is identical to that for H set out above. Finally, the underlying tone //LM+MH\#//postulated for the expression reduplicated from a~MH"=tone verb corresponds to its surface form, without any added complexities.
 
\largerpage[-1] 
The relationship between the tones of \is{monosyllables}monosyllabic verbs and that of reduplicated expressions is another matter. Why is it that a~final H tone is found in reduplicated L\textsubscript{a}-tone verbs, for instance? Why is it that MH"=tone verbs get an initial L at \isi{reduplication}? Phonology does not provide answers to these questions: the tone patterns found in reduplicated forms cannot be derived from those of the simple forms by phonological rules. The
correspondences between simple and reduplicated tones need to be learnt; they
constitute a~component of the tonal grammar of Yongning Na.\footnote{In {Naxi}, {reduplication} likewise lacks a~clear phonological pattern: H reduplicates to H.M, M to M.M, and L to M.L \citep[10-11]{he1987}. These patterns can be analyzed as originating diachronically in total {reduplication}: H → H.H, M → M.M, L → L.L \citep{michaudetal2007c}. In Yongning Na, on the other hand, no such simple historical scenario can be proposed at present. This is an area where data from neighbouring dialects seems necessary for progress in the {diachronic} analysis.}

It was mentioned above that disyllabic verbs are rare and constitute a~heterogeneous set. Only one case of \isi{reduplication} of a~disyllabic verb was observed: /\ipa{ʂv̩˧{$\sim$}ʂv̩˧ɖv̩˧{$\sim$}ɖv̩˧˥}/
‘pensively; with a~heavy heart’ (Reward.49), from /\ipa{ʂv̩˧ɖv̩˧}/ ‘to think; to miss’. This is the only example of
a~{correspondence} between a~M tone on the \textit{simplex} form and a~MH\# tone on the reduplicated
expression. It looks
like a~unique creation, rather than the result of a~productive pattern.


\section{Prefixes}
\label{sec:prefixes}

\subsection{M-tone prefixes}
\label{sec:mtoneprefixes}

The prefixes described in this section can be interpreted either as M-tone prefixes, or as toneless
prefixes that receive M by default; no evidence was found that they are specified for tone. They are
referred to as M-tone prefixes for convenience.

The most common verbal prefixes carrying M tone are negative /\ipa{mɤ˧}-/, prohibitive /\ipa{tʰɑ˧}-/, durative /\ipa{tʰi˧}-/ and accomplished /\ipa{le˧}-/.

\begin{quotation}
	The \textsc{accomplished} \is{prefixes}prefix \ipa{lə³³-} is used to give a~reading of
	accomplishment to a~verb with lexical aspect of ongoing state, process, or liminality. ({\dots})
	The \textsc{durative} \is{prefixes}prefix \ipa{tʰɯ³³-} [in the Alawa dialect, studied in this volume: /\ipa{tʰi˧}/] is used to give a~reading of
	ongoing action to verbs with lexical aspect of process or liminality. \citep[345]{lidz2010}
\end{quotation}
  
These prefixes all have the same behaviour (apart from the exceptional case of M\textsubscript{c}-tone verbs in association with
/\ipa{le˧-}/, as reported in \sectref{sec:tonemcasubsetoffiveintransitiveverbswithinthemtonecategory}). A~less common \is{prefixes}prefix, /\ipa{mv̩˧-}/, conveying imminence,
has a~behaviour of its own, described further below.

\tabref{tab:thetonesofverbsinassociationwithamtoneprefix} presents the tonal behaviour of the most common prefixes. With a~view to ease of reference,
redundant data is provided for three prefixes: {durative}, {prohibitive}, and {negation}, all
with identical tone patterns. Some of the combinations are found in the following online recordings: VerbProhib, VerbProhib2, VerbDurative and AccompPfv.

\begin{table}%[t]
\caption{\label{tab:thetonesofverbsinassociationwithamtoneprefix}The tones of verbs in association with a~M-tone prefix.}
\begin{tabularx}{\textwidth}{ l@{\hspace{7mm}} l@{\hspace{7mm}} l@{\hspace{7mm}} l@{\hspace{7mm}} l@{\hspace{7mm}} l@{\hspace{7mm}} }
\lsptoprule
	tone & example & meaning & \textsc{durative} & \textsc{prohibitive} & \textsc{negation}\\ \midrule
	H & \ipa{dzɯ˥} & to eat & \ipa{tʰi˧-dzɯ˥} & \ipa{tʰɑ˧-dzɯ˥} & \ipa{mɤ˧-dzɯ˥}\\
	M\textsubscript{a} & \ipa{hwæ˧\textsubscript{a}} & to buy & \ipa{tʰi˧-hwæ˧} & \ipa{tʰɑ˧-hwæ˧} & \ipa{mɤ˧-hwæ˧}\\
	M\textsubscript{b} & \ipa{tɕʰi˧\textsubscript{b}} & to sell & \ipa{tʰi˧-tɕʰi˧} & \ipa{tʰɑ˧-tɕʰi˧} & \ipa{mɤ˧-tɕʰi˧}\\
	M\textsubscript{c} & \ipa{pv̩˧\textsubscript{c}} & to chant & \ipa{tʰi˧-pv̩˧} & \ipa{tʰɑ˧-pv̩˧} & \ipa{mɤ˧-pv̩˧}\\
	L\textsubscript{a} & \ipa{bæ˩\textsubscript{a}} & to sweep & \ipa{tʰi˧-bæ˩} & \ipa{tʰɑ˧-bæ˩} & \ipa{mɤ˧-bæ˩}\\
	L\textsubscript{b} & \ipa{ʐwɤ˩\textsubscript{b}} & to speak & \ipa{tʰi˧-ʐwɤ˩} & \ipa{tʰɑ˧-ʐwɤ˩} & \ipa{mɤ˧-ʐwɤ˩}\\
	MH & \ipa{lɑ˧˥} & to strike & \ipa{tʰi˧-lɑ˧˥} & \ipa{tʰɑ˧-lɑ˧˥} & \ipa{mɤ˧-lɑ˧˥}\\
\lspbottomrule
\end{tabularx}
\end{table}

After a~M-tone \is{prefixes}prefix, the lexical tones M, H, L and MH have a~straightforward realization. The
\is{subcategories of lexical tones}subcategories M\textsubscript{a} and M\textsubscript{b} are neutralized in this context; likewise for L\textsubscript{a} and L\textsubscript{b}.

\tabref{tab:accomplishedpfvcompletion} sets out the facts for the {accomplished} \is{prefixes}prefix /\ipa{le˧}-/ in association with the
perfective, \mbox{/\ipa{-ze˧}/}, and the morpheme indicating completion: \mbox{/\ipa{-se˩}/}. \tabref{tab:accomplishedcauscertitudecop} lists the patterns that obtain when the verb is followed by the {causative} /\ipa{-tsæ˧}/ and
{certitude} morpheme /\ipa{-ɲi˩}/. 

\begin{table}%[t]
\caption{\label{tab:accomplishedpfvcompletion}Tone patterns in constructions consisting of verbs with the {accomplished} M-tone prefix and a~{perfective} or {completion} morpheme.}
\begin{tabularx}{\textwidth}{ l@{\hspace{6mm}} l l l Q Q }
\lsptoprule
	tone & example & meaning & \textsc{accomp} & \textsc{accomp}+V+\textsc{pfv} & \textsc{accomp}+V+{\allowbreak}\textsc{completion}\\ \midrule
	H & \ipa{dzɯ˥} & to eat & \ipa{le˧-dzɯ˥} & \ipa{le˧-dzɯ˥-ze˩} & \ipa{le˧-dzɯ˧-se˥}\\
	M\textsubscript{a} & \ipa{hwæ˧\textsubscript{a}} & to buy & \ipa{le˧-hwæ˧} & \ipa{le˧-hwæ˧-ze˧} & \ipa{le˧-hwæ˧-se˩}\\
	M\textsubscript{b} & \ipa{tɕʰi˧\textsubscript{b}} & to sell & \ipa{le˧-tɕʰi˧} & \ipa{le˧-tɕʰi˧-ze˧} & \ipa{le˧-tɕʰi˧-se˩}\\
	M\textsubscript{c} & \ipa{bi˧\textsubscript{c}} & to go & \ipa{le˧-bi˩} & \ipa{le˧-bi˩-ze˩} & \ipa{le˧-bi˩-se˩}\\
	L\textsubscript{a} & \ipa{bæ˩\textsubscript{a}} & to sweep & \ipa{le˧-bæ˩} & \ipa{le˧-bæ˩-ze˩} & \ipa{le˧-bæ˩-se˩}\\
	L\textsubscript{b} & \ipa{ʐwɤ˩\textsubscript{b}} & to speak & \ipa{le˧-ʐwɤ˩} & \ipa{le˧-ʐwɤ˩-ze˩} & \ipa{le˧-ʐwɤ˩-se˩}\\
	MH & \ipa{lɑ˧˥} & to strike & \ipa{le˧-lɑ˧˥} & \ipa{le˧-lɑ˧-ze˥} & \ipa{le˧-lɑ˧-se˥}\\
\lspbottomrule
\end{tabularx}
\end{table}

\begin{table}%[t]
\caption{\label{tab:accomplishedcauscertitudecop}Tone patterns in constructions consisting of verbs with the {accomplished} M-tone prefix and {causative}+{certitude} morphemes.}
\begin{tabularx}{\textwidth}{ l l l l Q }
\lsptoprule
	tone & example & meaning & \textsc{accomp} & \textsc{accomp}+V+\textsc{caus}+\textsc{certitude}\\ \midrule
	H & \ipa{dzɯ˥} & to eat & \ipa{le˧-dzɯ˥} & \ipa{le˧-dzɯ˧-tsæ˥-ɲi˩}\\
	M\textsubscript{a} & \ipa{hwæ˧\textsubscript{a}} & to buy & \ipa{le˧-hwæ˧} & \ipa{le˧-hwæ˧-tsæ˧-ɲi˩}\\
	M\textsubscript{b} & \ipa{tɕʰi˧\textsubscript{b}} & to sell & \ipa{le˧-tɕʰi˧} & \ipa{le˧-tɕʰi˧-tsæ˧-ɲi˩}\\
	M\textsubscript{c} & \ipa{pv̩˧\textsubscript{c}} & to chant & \ipa{le˧-pv̩˧} & \ipa{le˧-pv̩˧-tsæ˧-ɲi˩}\\
	L\textsubscript{a} & \ipa{bæ˩\textsubscript{a}} & to sweep & \ipa{le˧-bæ˩} & \ipa{le˧-bæ˩-tsæ˩-ɲi˩}\\
	L\textsubscript{b} & \ipa{ʐwɤ˩\textsubscript{b}} & to speak & \ipa{le˧-ʐwɤ˩} & \ipa{le˧-ʐwɤ˩-tsæ˩-ɲi˩}\\
	MH & \ipa{lɑ˧˥} & to strike & \ipa{le˧-lɑ˧˥} & \ipa{le˧-lɑ˧-tsæ˥-ɲi˩}\\
\lspbottomrule
\end{tabularx}
\end{table}

The {certitude} morpheme is invariantly low in \tabref{tab:accomplishedcauscertitudecop}, in keeping with its lexical L tone; this reveals that there is no \is{floating tone}floating H tone in any of the expressions that precede it. This is important information for arriving at the underlying tone of the expressions with surface M tones, exemplified by /\ipa{le˧-hwæ˧}/ and
/\ipa{le˧-tɕʰi˧}/: if they carried a~\is{floating tone}floating H tone, the following morpheme, /\ipa{-ɲi˩}/, would surface with H
tone. The data in \tabref{tab:accomplishedcauscertitudecop} also offers a~new illustration of the behaviour that is characteristic of the MH and H tones, respectively illustrated by
/\ipa{le˧-lɑ˧˥}/ and /\ipa{le˧-dzɯ˥}/. For MH-tone verbs in this construction, the M tone remains on the syllable to which it is lexically associated, and its H part is projected to the next syllable (the {causative}). (This is obligatorily followed by L tone on the {certitude} \is{suffixes}suffix, due to Rule~4: “The syllable following a~H-tone syllable receives L tone”.) In the case of H-tone verbs, the H tone no longer appears on the syllable to which it is lexically associated. Instead it associates to the {causative}, resulting in a surface phonological output that is identical to that for MH tone.

\largerpage
A related set of facts is presented in \tabref{tab:tonepatternsofthevnegvconstruction}: the construction V-\textsc{neg}-V, as in example (\ref{ex:whetheritisactuallythecasewedontknow}).\footnote{Two variants are possible, the one integrated into one single {tone group}, as in
	(\ref{ex:whetheritisactuallythecasewedontknow}), the other divided into two groups: /\ipa{ɲi˩˥ {\kern2pt}|{\kern2pt}
		mɤ˧-ɲi˩, {\kern2pt}|{\kern2pt} mɤ˧-ɳv̩˥}!/ (same meaning and morphemic composition as (\ref{ex:whetheritisactuallythecasewedontknow})). In the latter case,
	the first syllable has the same tone as in isolation, and the negated form has the tone indicated in
	\tabref{tab:thetonesofverbsinassociationwithamtoneprefix} above. These V-\textsc{neg}-V constructions are typically followed by /\ipa{mɤ˧-ɳv̩˥}/
	‘[I/we] don’t know’ or /\ipa{mɤ˧-do˩}/ ‘[we] don’t know/can’t see for ourselves’, sometimes with
	{focalization} of the V \textsc{neg}-V portion, e.g.~/\ipa{hwæ˧ mɤ˧-hwæ˧ F {\kern2pt}|{\kern2pt}
		mɤ˧-do˩}/ ‘[I/we] don’t know whether [they] bought [it/some] or not’. (For a~discussion of intonational {focalization}, transcribed as ‘F’, see \sectref{sec:focalization}.) In addition to the elicited data in \tabref{tab:tonepatternsofthevnegvconstruction}, there are some examples in texts, e.g.~in
	BuriedAlive3.133, Seeds2.85 and Dog.59.}


	
\begin{exe}
  \ex
  \label{ex:whetheritisactuallythecasewedontknow}
  \ipaex{ɲi˩ mɤ˥-ɲi˩, {\kern2pt}|{\kern2pt} mɤ˧-ɳv̩˥!}\\
  \gll ɲi˩	mɤ˧-	ɲi˩	mɤ˧-	ɳv̩˥\\
  \textsc{cop}	\textsc{neg}	\textsc{cop}	\textsc{neg}	to\_know/to\_get\_to\_know\\
  \glt ‘Whether it is actually the case{\dots} we don’t know!’ (Context: two persons discuss what
  a~third person has said.)
\end{exe}

\clearpage 
\begin{table}%[t]
\caption{\label{tab:tonepatternsofthevnegvconstruction}Tone patterns of the V-\textsc{neg}-V construction.}
\begin{tabularx}{\textwidth}{ l@{\hspace{7mm}} l@{\hspace{7mm}} l@{\hspace{5mm}} Q Q }
\lsptoprule
	tone & example & meaning & V \textsc{neg} V & V | \textsc{neg} V\\ \midrule
	H & \ipa{dzɯ˥} & to eat & \ipa{dzɯ˧ mɤ˧-dzɯ˧} & \ipa{dzɯ˧ {\kern2pt}|{\kern2pt} mɤ˧-dzɯ˥}\\
	M\textsubscript{a} & \ipa{hwæ˧\textsubscript{a}} & to buy & \ipa{hwæ˧ mɤ˧-hwæ˧} & \ipa{hwæ˧ {\kern2pt}|{\kern2pt} mɤ˧-hwæ˧}\\
	M\textsubscript{b} & \ipa{tɕʰi˧\textsubscript{b}} & to sell & \ipa{tɕʰi˧ mɤ˧-tɕʰi˧} & \ipa{tɕʰi˧ {\kern2pt}|{\kern2pt} mɤ˧-tɕʰi˧}\\
	M\textsubscript{c} & \ipa{pv̩˧\textsubscript{c}} & to chant & \ipa{pv̩˧ mɤ˧-pv̩˧} & \ipa{pv̩˧ {\kern2pt}|{\kern2pt} mɤ˧-pv̩˧}\\
	L\textsubscript{a} & \ipa{gɯ˩\textsubscript{a}} & to be true & \ipa{gɯ˩ mɤ˩-gɯ˥} & \ipa{gɯ˩˥ {\kern2pt}|{\kern2pt} mɤ˧-gɯ˩}\\
	L\textsubscript{b} & \ipa{ʐwɤ˩\textsubscript{b}} & to speak & \ipa{ʐwɤ˩ mɤ˥-ʐwɤ˩} & \ipa{ʐwɤ˩˥ {\kern2pt}|{\kern2pt} mɤ˧-ʐwɤ˩}\\
	MH & \ipa{lɑ˧˥} & to strike & \ipa{lɑ˧ mɤ˥-lɑ˩} & \ipa{lɑ˧˥ {\kern2pt}|{\kern2pt} mɤ˧-lɑ˧˥}\\
\lspbottomrule
\end{tabularx}
\end{table}

  
As mentioned at the outset of this section, the \is{prefixes}prefix /\ipa{mv̩˧}-/, conveying imminence, is
infrequent: only one example is found in the first twenty"=five transcribed narratives. The
consultant was not comfortable pairing this \is{prefixes}prefix with verbs into a~disyllabic expression, and
proposed instead the three constructions (\ref{ex:rightaway}), (\ref{ex:durimmin}) and (\ref{ex:accompneg}). The results are shown in \tabref{tab:tonepatternsfortheprefixandarelatedconstruction}.

\begin{exe}
	\ex
	\label{ex:rightaway}
	\ipaex{mv̩˧-V-bi˧}\\
	\gll mv̩˧-	V		-bi˧\\
	\textsc{imminence}			\textit{target~verb}		\textsc{imm\_fut}\\
	\glt ‘will V right away’
\end{exe}

\begin{exe}
	\ex
	\label{ex:durimmin}
	\ipaex{tʰi˧-mv̩˧-V}\\
	\gll tʰi˧-		mv̩˧-	V\\
	\textsc{dur}		\textsc{imminence}		\textit{target~verb}\\
	\glt ‘is going to V up’
\end{exe}

\begin{exe}
	\ex
	\label{ex:accompneg}
	\ipaex{le˧-mɤ˧-V}\\
	\gll le˧-		mɤ˧-	V\\
	\textsc{accomp}		\textsc{neg}		\textit{target~verb}\\
	\glt ‘does not V / not to V’
\end{exe}

The results for H- and MH"=tone verbs are far from trivial. As with other systematically elicited
combinations, it appears safer to wait until further confirmation can be obtained (ideally from
texts) before attempting an~interpretation.

\begin{table}[h]
  \caption{\label{tab:tonepatternsfortheprefixandarelatedconstruction}Tone patterns of verbs with the prefix /\ipa{mv̩˧}-/, and an antonymic construction.}
{\setlength\tabcolsep{4.5pt}
\begin{tabularx}{\textwidth}{ l l l Q Q Q }
\lsptoprule
	tone & example & meaning & \ipa{mv̩˧-}V\ipa{-bi˧} & \ipa{tʰi˧-mv̩˧-}V & \ipa{le˧-mɤ˧-}V\\ \midrule
	H & \ipa{dzɯ˥} & to eat & \ipa{mv̩˧-dzɯ˧-bi˧} & \ipa{tʰi˧-mv̩˧-dzɯ˧} & \ipa{le˧-mɤ˧-dzɯ˧}\\
	M\textsubscript{a} & \ipa{hwæ˧\textsubscript{a}} & to buy & \ipa{mv̩˧-hwæ˧-bi˧} & \ipa{tʰi˧-mv̩˧-hwæ˧} & \ipa{le˧-mɤ˧-hwæ˧}\\
	M\textsubscript{b} & \ipa{tɕʰi˧\textsubscript{b}} & to sell & \ipa{mv̩˧-tɕʰi˧-bi˧} & \ipa{tʰi˧-mv̩˧-tɕʰi˧} & \ipa{le˧-mɤ˧-tɕʰi˧}\\
	M\textsubscript{c} & \ipa{pv̩˧\textsubscript{c}} & to chant & \ipa{mv̩˧-pv̩˧-bi˧} & \ipa{tʰi˧-mv̩˧-pv̩˧} & \ipa{le˧-mɤ˧-pv̩˧}\\
	L\textsubscript{a} & \ipa{bæ˩\textsubscript{a}} & to sweep & \ipa{mv̩˧-bæ˧-bi˩} & \ipa{tʰi˧-mv̩˧-bæ˩} & \ipa{le˧-mɤ˧-bæ˩}\\
	L\textsubscript{b} & \ipa{ʐwɤ˩\textsubscript{b}} & to speak & \ipa{mv̩˧-ʐwɤ˧-bi˩} & \ipa{tʰi˧-mv̩˧-ʐwɤ˩} & \ipa{le˧-mɤ˧-ʐwɤ˩}\\
	MH & \ipa{lɑ˧˥} & to strike & \ipa{mv̩˧-lɑ˩-bi˩} & \ipa{tʰi˧-mv̩˧-lɑ˧˥} & \ipa{le˧-mɤ˧-lɑ˧˥}\\
\lspbottomrule
\end{tabularx}}
\end{table}


\subsection{L-tone prefixes}
\label{sec:ltoneprefixes}
\largerpage

The yes/no interrogative (polar interrogative) /\ipa{ə˩}-/ illustrates the case of L-tone prefixes. This interrogative is
segmentally bleached, consisting of a~neutral vowel undergoing strong regressive vowel harmony. From a~tonal point of view, on the other hand, it has a~specification of its own. \tabref{tab:thetonesofverbsinassociationwithaltoneprefix} sets out
the facts.

\begin{table}[h]
\caption{\label{tab:thetonesofverbsinassociationwithaltoneprefix}The tones of verbs in association with a~L-tone prefix.}
\begin{tabularx}{.8\textwidth}{ Q Q Q Q }
\lsptoprule
	tone & example & meaning & interrogative\\ \midrule
	H & \ipa{dzɯ˥} & to eat & \ipa{ə˧-dzɯ˥}\\
	M\textsubscript{a} & \ipa{hwæ˧\textsubscript{a}} & to buy & \ipa{ə˩-hwæ˧}\\
	M\textsubscript{b} & \ipa{tɕʰi˧\textsubscript{b}} & to sell & \ipa{ə˩-tɕʰi˧}\\
	M\textsubscript{c} & \ipa{pv̩˧\textsubscript{c}} & to chant & \ipa{ə˩-pv̩˧}\\
	L\textsubscript{a} & \ipa{bæ˩\textsubscript{a}} & to sweep & \ipa{ə˩-bæ˩}\\
	L\textsubscript{b} & \ipa{ʐwɤ˩\textsubscript{b}} & to speak & \ipa{ə˩-ʐwɤ˩}\\
	MH & \ipa{lɑ˧˥} & to strike & \ipa{ə˧-lɑ˥}\\
\lspbottomrule
\end{tabularx}
\end{table}

The realizations of tones M and L after a~L-tone \is{prefixes}prefix reflect the lexical tones in a~straightforward way. That of tones H and MH,
on the other hand, is non"=trivial: both have the same pattern, in which the tone of the \is{prefixes}prefix is
raised to M, and the verb carries H. The sequence /L.MH/ would not contravene Yongning Na phonotactics, so there is no phonological reason why the interrogative followed by a \mbox{//MH//}"=tone verb should not yield /L.MH/: for instance  $\dagger$\ipa{ə˩-lɑ˧˥} (intended meaning: ‘does (s)he strike’). The sequence /L.H/ is also fine from a~phonological point of view.

\newpage 
The origin of the raising of the interrogative morpheme's L tone in front of H-tone and MH-tone verbs constitutes a~{diachronic} puzzle in view of the scarcity of cases of modification of a~tone by that of a~following morpheme in the Alawa dialect of Yongning Na. In this language variety, tone modification tends to be progressive (unlike vowel harmony, a~phonetic tendency which works in the other direction: see Appendix A, \sectref{sec:anoteonvowelharmony}). On the other hand, if any morpheme is likely to be modified by the tone of a~following word, a~schwa \is{prefixes}prefix is surely the most likely, given its phonetic and phonological lightness. The realization of schwa prefixes would warrant an \is{experimental phonetics}experimental phonetic study in future, to investigate their \isi{duration}, fundamental frequency, formant frequencies and other parameters. Synchronic patterns of phonetic \isi{variation} may help formulate hypotheses about how these prefixes' current morphotonological patterns came about.


\subsection{The marking of spatial orientation on verbs}
\label{sec:themarkingofspatialorientationonverbs}

Extensive marking of orientation on verbs is found among Na"=Qiangic languages. In particular, \ili{Rgyalrongic} languages “have a~whole array of verbal orientation prefixes, which are
obligatorily present on all perfective and {imperative} verb forms” (\citealt[180]{sun2000a}; see also \citealt{lin2002},
and \citealt{jacques2011b} on Tangut). The three distinct pairs of directions described by J. Sun for \ili{Rgyalrongic} are: eastward (i.e.\ in the
direction of the rising sun) vs.\ westward; upstream vs.\ downstream; and uphill (upward) vs.\ downhill
(downward). The system found in \ili{Shixing} (Xumi) comprises two productive pairs of (non"=obligatory)
orientation prefixes, only one of which corresponds semantically with the \ili{Rgyalrongic} system: upward
vs.\ downward, the other being inward vs.\ outward. \ili{Shixing} also displays traces of a~third pair:
hither and thither, found in a~set construction meaning ‘to V back and forth’
\citep{chirkova2009}.

This conspicuous characteristic is sometimes awarded the status of criterion for phylogenetic classification, e.g.~in
proposals by \citet[105]{matisoff2004}. But cross"=dialect and cross"=language differences reveal that
orientation systems are no less prone to change than other areas of a~language’s structure. Under the hypothesis that directional prefixes have great historical depth in \il{Sino-Tibetan}Sino"=Tibetan,
\ili{Naish} must be hypothesized to have lost them. In \ili{Naish}, topographically"=based spatial
deixis is not marked through an~obligatory \is{prefixes}prefix on verbs. Indications of orientation, such as /\ipa{mv̩˩tɕo˧}/ ‘downward’ in (\ref{ex:downwd}), may more
properly be called orientation adverbials.
\newpage 

\begin{exe}
	\ex
	\label{ex:downwd}
	\ipaex{mv̩˩tɕo˧ mɤ˧-hɯ˧}\\
	\gll mv̩˩tɕo˧	mɤ˧-	hɯ˧\textsubscript{c}\\
		downward	\textsc{neg}		to\_go.\textsc{pst}\\
	\glt ‘[The dog, who had come to sit on the wooden platform
	close to the fire pit, where dogs are not allowed] did not/would not get down!’ (Example from
	a~discussion about Sister3.22.)
\end{exe}

The only \is{monosyllables}monosyllabic indications of orientation in common use, for which one could claim the status
of prefixes, are /\ipa{gɤ˩}-/ ‘upward’ and /\ipa{mv̩˩}-/ ‘downward’. For instance, /\ipa{ʂo˥}/ ‘to
reap, to gather in’ can be used in association with /\ipa{gɤ˩}-/ ‘upward’ to mean ‘to reap in, to
bring back to the house and into the granary’: /\ipa{gɤ˩-ʂo˥}/. These \is{monosyllables}monosyllabic prefixes also
appear as part of set constructions, such as /\ipa{gɤ˩-V} {\kern2pt}|{\kern2pt} \ipa{mv̩˩-V}/ ‘to V in all directions’, as in (\ref{ex:blows}):

\begin{exe}
  \ex
  \label{ex:blows}
  \ipaex{ʈʂʰɯ˧ne˧-ʝi˥ {\kern2pt}|{\kern2pt} gɤ˩-dɑ˧˥, {\kern2pt}|{\kern2pt} mv̩˩-dɑ˧˥, {\kern2pt}|{\kern2pt} gɤ˩-dɑ˧˥, {\kern2pt}|{\kern2pt} mv̩˩-dɑ˧˥, {\kern2pt}|{\kern2pt} ({\dots}) ɖɯ˧-so˩ ʂɯ˩
  ʝi˩ tsɯ˩ {\kern2pt}|{\kern2pt} mv̩˩! {\kern2pt}|}\\
  \gll ʈʂʰɯ˧ne˧-ʝi˥	gɤ˩-	dɑ˧˥	mv̩˩-		dɑ˧˥	ɖɯ˧-so˩ ʂɯ˩		ʝi˥	tsɯ˧˥	mv̩˧\\
  thus		upward	to\_strike	downward	to\_strike	several	times		to\_do	\textsc{rep}
  \textsc{affirm}\\
  \glt ‘He would give blows high and low (= hither and thither), again and again!~/ He would strike
  blows in all directions, again and again!’ (Healing.38. Context: an~exorcist is performing
  a~ritual.)
\end{exe}

In this context, the prefixes retain to some extent their literal meaning of ‘upward’ and
‘downward’: the exorcist’s blows with his sword are aimed high up, then down (close to the ground), then up again, and so on. But in this construction, the spatial indications take up a~broader meaning, summoning
up the swift, dance"=like movements of the exorcist fighting with an~invisible cohort of demons
surrounding him. Repetition of the prefixed verb (/\ipa{gɤ˩-dɑ˧˥}, {\kern2pt}|{\kern2pt} \ipa{mv̩˩-dɑ˧˥}, {\kern2pt}|{\kern2pt}
\ipa{gɤ˩-dɑ˧˥}, {\kern2pt}|{\kern2pt} \ipa{mv̩˩-dɑ˧˥}/) participates to loosening the exact indication of
spatial orientation, yielding a~meaning of ‘in all directions’ rather than ‘up and down’.

When an~orientation \is{prefixes}prefix is separated from the verb by other prefixes, it can constitute a~tone
group on its own, as in (\ref{ex:afteronehasprayed}). In this example, the \is{prefixes}prefix's //L// tone surfaces as /LH/ due to the prohibition of all-L tone groups: a~{phonological rule} referred to as Rule~7 (see \sectref{sec:asummaryoftonetosyllableassociationrules}) repairs all-L tone groups by addition of a~post"=lexical H tone to their last syllable, which, in the case of the first \isi{tone group} of example (\ref{ex:afteronehasprayed}), also happens to be the \textit{first} syllable.
%
\begin{exe}
  \ex
  \label{ex:afteronehasprayed}
  \ipaex{{\dots} gɤ˩˥ {\kern2pt}|{\kern2pt} le˧-ʈʂʰo˧-se˥-dʑo˩ {\kern2pt}|{\kern2pt} tʰi˩˥ {\dots}}\\
  \gll gɤ˩-		le˧-		ʈʂʰo˥	-se˩		-dʑo˥	tʰi˩˥\\
  upward		\textsc{accomp}	to\_pray	\textsc{completion}	\textsc{top}	then\\
  \glt ‘after one has prayed [\textit{literally}: prayed up (to the ancestors)]’ (Dog2.54)
\end{exe}

Judging from available texts, {monosyllabic} /\ipa{gɤ˩}-/ ‘upward’ is more
frequent than /\ipa{mv̩˩}-/ ‘downward’. The two prefixes /\ipa{gɤ˩}-/ ‘upward’ and /\ipa{mv̩˩}-/ ‘downward’ have the same tonal behaviour.
%, shown in \tabref{tab:spatialmonoFULL}. 
The disyllabic expressions /\ipa{gɤ˩tɕo˧}/
‘upward’ and /\ipa{mv̩˩tɕo˧}/ ‘downward’ are generally preferred. Monosyllabic forms are tentatively
labelled here as prefixes (and transcribed accordingly: with a~following hyphen) and disyllabic forms
as adverbials, but in the absence of language"=internal criteria to distinguish the two, the divide
is not as clear as this choice of terms suggests. From a~tonal point of view, aside from uncommon cases such as (\ref{ex:afteronehasprayed}), orientation prefixes belong in the same \isi{tone group} as the following verb. On the other hand, orientation adverbials often
constitute a~\isi{tone group} on their own, as discussed in~\sectref{sec:someelementsalwaysconstituteatonegroupontheirown}. The data in \tabref{tab:spatialFULL} concerns cases when the orientation \is{prefixes}prefix or {adverbial} is integrated into the same \isi{tone group} as the verb. The {perfective} \is{suffixes}suffix /-\ipa{ze˧}/ serves to reveal whether the verb carries H tone (which results in lowering of the \is{suffixes}suffix to L), M tone (which leaves the \is{suffixes}suffix unaffected), or MH tone (which results in the association of the H part of the \is{tonal contour}contour to the \is{suffixes}suffix). For the sake of clarity, the surface forms are provided in full, even though in many cases the tonal behaviour of the {perfective} \is{suffixes}suffix \mbox{/\ipa{-ze˧}/} derives straightforwardly from the tone pattern of the non"=suffixed expression through the application of the seven phonological tone rules recapitulated in \sectref{sec:alistoftonerules}. For
instance, L.H can only be followed by L, by virtue of Rule~4 (“A syllable following a~H-tone syllable receives L tone''), so the L.H expression /\ipa{gɤ˩-se˥}/ ‘to walk up(ward)’ predictably yields L.H+L in (\ref{ex:walkedup}).

\begin{exe}
	\ex
	\label{ex:walkedup}
	\ipaex{gɤ˩-se˥-ze˩}\\
	\gll gɤ˩-		se˥		-ze˧\\
	\textsc{upward}		to\_walk	\textsc{pfv}\\
	\glt ‘walked up(ward)’
\end{exe}

The original data is found in the online document SpatialOrientation. The verb /\ipa{bi˧\textsubscript{c}}/ ‘to go’ was accidentally omitted in the recording. This verb can combine with the disyllabic
orientation adverbials, but not with {monosyllabic} /\ipa{gɤ˩}-/ ‘upward’ and /\ipa{mv̩˩}-/
‘downward’. The data is shown in (\ref{ex:goforward})"=(\ref{ex:goupward}). Note that the \is{variants}variant $\ddagger${\kern2pt}\ipa{jo˩lo˩ bi˥} for (\ref{ex:goright}) is not acceptable.

\begin{exe}
	\ex
	\label{ex:goforward}
	\ipaex{ʁo˧dɑ˧ bi˧(-ze˧)}\\
	\gll ʁo˧dɑ˧		bi˧\textsubscript{c}		-ze˧\\
	forward		to\_go		\textsc{pfv}\\
	\glt ‘to go forward’
\end{exe}

\begin{exe}
	\ex
	\label{ex:goleft}
	\ipaex{ʁwæ˧gi˧ bi˧(-ze˧)}\\
	\gll ʁwæ˧gi˧		bi˧\textsubscript{c}		-ze˧\\
	leftward		to\_go		\textsc{pfv}\\
	\glt ‘to go to the left’
\end{exe}

\begin{exe}
	\ex
	\label{ex:goright}
	\ipaex{jo˩lo˩ bi˩}\\
	\gll jo˩lo˩		bi˧\textsubscript{c}\\
	rightward		to\_go\\
	\glt ‘to go to the right’
\end{exe}

\begin{exe}
	\ex
	\label{ex:gobackward}
	\ipaex{ʁo˧tʰo˩ bi˩}\\
	\gll ʁo˧tʰo˩	bi˧\textsubscript{c}\\
	backward		to\_go\\
	\glt ‘to go backward’
\end{exe}

\begin{exe}
	\ex
	\label{ex:goupward}
	\ipaex{gɤ˩tɕo˧ bi˧(-ze˧)}\\
	\gll gɤ˩tɕo˧	bi˧\textsubscript{c}		-ze˧\\
	upward		to\_go		\textsc{pfv}\\
	\glt ‘to go upward’
\end{exe}

%‘Leftward’ and ‘rightward’ are expressed by disyllabic /\ipa{ʁwæ˧gi\#˥}/ or /\ipa{ʁwæ˧lo˥}/ ‘to the
%left’, and /\ipa{jo˩gi˩}/ or /\ipa{jo˩lo˩}/ ‘to the right’. 

%The following subtables of \tabref{tab:spatialFULL} set out the data for the adverbials /\ipa{ɬo˧tɑ˧}/ ‘to the side’, /\ipa{ʁwæ˧gi\#˥}/ and /\ipa{ʁwæ˧lo˥}/ ‘to the
%left’, /\ipa{jo˩gi˩}/ and /\ipa{jo˩lo˩}/ ‘to the right’, /\ipa{ʁo˧dɑ˧}/
%‘forward, to the front’ and /\ipa{ʁo˧tʰo˩}/ ‘backward, to the back’.

%The example verbs used are /\ipa{se˥}/ ‘to walk’, /\ipa{li˧\textsubscript{a}}/ ‘to look’, /\ipa{tsi˧\textsubscript{b}}/ ‘to set, to install’, /\ipa{bi˧\textsubscript{c}}/
%‘to go’, /\ipa{kwɤ˩\textsubscript{a}}/ ‘to throw’, /\ipa{ɻ̍˩\textsubscript{b}}/ ‘to turn’, and /\ipa{mi˧˥}/
%‘to push’. 

\begin{subtables}
	\label{tab:spatialFULL}
	\begin{table}%[h!]
		\caption{\label{tab:spatialmonoFULL}The tonal behaviour of verbs after indications of spatial orientation: {monosyllabic}
			prefixes.}
		\begin{tabularx}{\textwidth}{ l@{\hspace{2mm}} l@{\hspace{2mm}} l@{\hspace{2mm}} Q@{\hspace{2mm}} Q@{\hspace{1mm}} }
		\lsptoprule
		tone & example & meaning & ‘upward’ \is{prefixes}prefix & ‘downward’ \is{prefixes}prefix\\ \midrule
		%Use of kerning to obtain splitting of the contents of the last two columns over two lines: more symmetrical to the eye.
		H & \ipa{se˥} & to walk & \ipa{gɤ˩-se˥, gɤ˩-se˥-ze˩} & \ipa{mv̩˩-se˥, mv̩˩-se˥-ze˩}\\
		M\textsubscript{a} & \ipa{li˧\textsubscript{a}} & to look & \ipa{gɤ˩-li˧, gɤ˩-li˧-ze˧} & \ipa{mv̩˩-li˧, mv̩˩-li˧-ze˧}\\
		M\textsubscript{b} & \ipa{tsi˧\textsubscript{b}} & to set & \ipa{gɤ˩-tsi˧, gɤ˩-tsi˧-ze˧} & \ipa{mv̩˩-tsi˧, mv̩˩-tsi˧-ze˧}\\
		L\textsubscript{a} & \ipa{kwɤ˩\textsubscript{a}} & to throw & \ipa{gɤ˩-kwɤ˥, gɤ˩-kwɤ˥-ze˩} & \ipa{mv̩˩-kwɤ˥, mv̩˩-kwɤ˥-ze˩}\\
		L\textsubscript{b} & \ipa{ɻ̍˩\textsubscript{b}} & to turn & \ipa{gɤ˩-ɻ̍˥, gɤ˩-ɻ̍˥-ze˩} & \ipa{mv̩˩-ɻ̍˥, mv̩˩-ɻ̍˥-ze˩}\\
		MH & \ipa{mi˧˥} & to push & \ipa{gɤ˩-mi˧˥, gɤ˩-mi˧-ze˥} & \ipa{mv̩˩-mi˧˥, mv̩˩-mi˧-ze˥}\\
		\lspbottomrule
		\end{tabularx}
	\end{table}
	
	\begin{table}
		\caption{\label{tab:spatialdiFULLFRONT}The tonal behaviour of verbs after indications of spatial orientation: {adverbial} //\ipa{ɬo˧tɑ˧}// ‘to the side’.}
		\begin{tabularx}{\textwidth}{ l@{\hspace{12mm}} l@{\hspace{12mm}} l@{\hspace{12mm}} Q }
			\lsptoprule
			tone & example & meaning & with {adverbial} ‘to the side’\\ \midrule
			%Use of kerning to obtain splitting of the contents of the last two columns over two lines: more symmetrical to the eye.
			H & \ipa{se˥} & to walk & \ipa{ɬo˧tɑ˧ se˧, ɬo˧tɑ˧ se˧-ze˩}\\
			M\textsubscript{a} & \ipa{li˧\textsubscript{a}} & to look & \ipa{ɬo˧tɑ˧ li˧, ɬo˧tɑ˧ li˧-ze˧}\\
			M\textsubscript{b} & \ipa{tsi˧\textsubscript{b}} & to set & \ipa{ɬo˧tɑ˧ tsi˧, ɬo˧tɑ˧ tsi˧-ze˩}\\
			L\textsubscript{a} & \ipa{kwɤ˩\textsubscript{a}} & to throw & \ipa{ɬo˧tɑ˧ kwɤ˩, ɬo˧tɑ˧ kwɤ˩-ze˩}\\
			L\textsubscript{b} & \ipa{ɻ̍˩\textsubscript{b}} & to turn & \ipa{ɬo˧tɑ˧ ɻ̍˩, ɬo˧tɑ˧ ɻ̍˩-ze˩}\\
			MH & \ipa{mi˧˥} & to push & \ipa{ɬo˧tɑ˧ mi˧˥, ɬo˧tɑ˧ mi˧-ze˥}\\
			\lspbottomrule
		\end{tabularx}
		\end{table}

		\begin{table}%[h!]
		\caption{\label{tab:spatialdiFULLLFRONTBACK}The tonal behaviour of verbs after indications of spatial orientation: adverbials //\ipa{ʁo˧dɑ˧}// ‘forward’ and //\ipa{ʁo˧tʰo˩}// ‘backward’.}
		\begin{tabularx}{\textwidth}{ l@{\hspace{7mm}} l@{\hspace{7mm}} l@{\hspace{7mm}} Q Q }
			\lsptoprule
			tone & example & meaning & ‘forward’ & ‘backward’\\ \midrule
			%Use of kerning to obtain splitting of the contents of the last two columns over two lines: more symmetrical to the eye.
			H & \ipa{se˥} & to walk & \ipa{ʁo˧dɑ˧ se˧,{\kern8pt} ʁo˧dɑ˧ se˧-ze˩} & \ipa{ʁo˧tʰo˩ se˩,{\kern8pt} ʁo˧tʰo˩ se˩-ze˩}\\
			M\textsubscript{a} & \ipa{li˧\textsubscript{a}} & to look & \ipa{ʁo˧dɑ˧ li˧,{\kern8pt} ʁo˧dɑ˧ li˧-ze˧} & \ipa{ʁo˧tʰo˩ li˩,{\kern8pt} ʁo˧tʰo˩ li˩-ze˩}\\
			M\textsubscript{b} & \ipa{tsi˧\textsubscript{b}} & to set & \ipa{ʁo˧dɑ˧ tsi˧,{\kern8pt} ʁo˧dɑ˧ tsi˧-ze˩} & \ipa{ʁo˧tʰo˩ tsi˩,{\kern8pt} ʁo˧tʰo˩ tsi˩-ze˩}\\
			L\textsubscript{a} & \ipa{kwɤ˩\textsubscript{a}} & to throw & \ipa{ʁo˧dɑ˧ kwɤ˩,{\kern8pt} ʁo˧dɑ˧ kwɤ˩-ze˩} & \ipa{ʁo˧tʰo˩ kwɤ˩,{\kern8pt} ʁo˧tʰo˩ kwɤ˩-ze˩}\\
			L\textsubscript{b} & \ipa{ɻ̍˩\textsubscript{b}} & to turn & \ipa{ʁo˧dɑ˧ ɻ̍˩,{\kern14pt} ʁo˧dɑ˧ ɻ̍˩-ze˩} & \ipa{ʁo˧tʰo˩ ɻ̍˩,{\kern14pt} ʁo˧tʰo˩ ɻ̍˩-ze˩}\\
			MH & \ipa{mi˧˥} & to push & \ipa{ʁo˧dɑ˧ mi˧˥,{\kern8pt} ʁo˧dɑ˧ mi˧-ze˥} & \ipa{ʁo˧tʰo˩ mi˩,{\kern8pt} ʁo˧tʰo˩ mi˩-ze˩}\\
			\lspbottomrule
		\end{tabularx}
	\end{table}

	\begin{table}%[h!]
		\caption{\label{tab:spatialdiFULLLEFT}The tonal behaviour of verbs after indications of spatial orientation: //\ipa{ʁwæ˧lo˥}// ‘leftward’ and //\ipa{ʁwæ˧-gi\#˥}// ‘to the left side’.}
		\begin{tabularx}{\textwidth}{ l@{\hspace{7mm}} l@{\hspace{7mm}} l@{\hspace{7mm}} Q Q }
			\lsptoprule
			tone & example & meaning & ‘leftward’  & ‘to the left side’\\ \midrule
			%Use of kerning to obtain splitting of the contents of the last two columns over two lines: more symmetrical to the eye.
			H & \ipa{se˥} & to walk & \ipa{ʁwæ˧lo˥ se˩,{\kern8pt} ʁwæ˧lo˥ se˩-ze˩} & \ipa{ʁwæ˧-gi˧ se˧,{\kern8pt} ʁwæ˧-gi˧ se˧-ze˩}\\
			M\textsubscript{a} & \ipa{li˧\textsubscript{a}} & to look & \ipa{ʁwæ˧lo˥ li˩,{\kern8pt} ʁwæ˧lo˥ li˩-ze˩} & \ipa{ʁwæ˧-gi˧ li˩,{\kern8pt} ʁwæ˧-gi˧ li˩-ze˩}\\
			M\textsubscript{b} & \ipa{tsi˧\textsubscript{b}} & to set & \ipa{ʁwæ˧lo˥ tsi˩,{\kern8pt} ʁwæ˧lo˥ tsi˩-ze˩} & \ipa{ʁwæ˧-gi˧ tsi˧,{\kern8pt} ʁwæ˧-gi˧ tsi˧-ze˩}\\
			L\textsubscript{a} & \ipa{kwɤ˩\textsubscript{a}} & to throw & \ipa{ʁwæ˧lo˥ kwɤ˩,{\kern8pt} ʁwæ˧lo˥ kwɤ˩-ze˩} & \ipa{ʁwæ˧-gi˧ kwɤ˥,{\kern8pt} ʁwæ˧-gi˧ kwɤ˥-ze˩}\\
			L\textsubscript{b} & \ipa{ɻ̍˩\textsubscript{b}} & to turn & \ipa{ʁwæ˧lo˥ ɻ̍˩,{\kern14pt} ʁwæ˧lo˥ ɻ̍˩-ze˩} & \ipa{ʁwæ˧-gi˧ ɻ̍˥,{\kern14pt} ʁwæ˧-gi˧ ɻ̍˥-ze˩}\\
			MH & \ipa{mi˧˥} & to push & \ipa{ʁwæ˧lo˥ mi˩,{\kern8pt} ʁwæ˧lo˥ mi˩-ze˩} & \ipa{ʁwæ˧-gi˧ mi˩,{\kern8pt} ʁwæ˧-gi˧ mi˩-ze˩}\\
			\lspbottomrule
		\end{tabularx}
	\end{table}
	
	\begin{table}%[h!]
		\caption{\label{tab:spatialdiFULLRIGHT}The tonal behaviour of verbs after indications of spatial orientation: //\ipa{jo˩lo˩}// ‘rightward’ and //\ipa{jo˩-gi˩}// ‘to the right side’.}
		\begin{tabularx}{\textwidth}{ l@{\hspace{7mm}} l@{\hspace{7mm}} l@{\hspace{7mm}} Q Q }
			\lsptoprule
			tone & example & meaning & ‘rightward’  & ‘to the right side’\\ \midrule
			%Use of kerning to obtain splitting of the contents of the last two columns over two lines: more symmetrical to the eye.
			H & \ipa{se˥} & to walk & \ipa{jo˩lo˩ se˩˥, {\kern12pt} jo˩lo˩ se˩-ze˥} & \ipa{jo˩-gi˩ se˩˥, {\kern8pt} jo˩-gi˩ se˩-ze˥}\\
			M\textsubscript{a} & \ipa{li˧\textsubscript{a}} & to look & \ipa{jo˩lo˩ li˥,{\kern19pt} jo˩lo˩ li˥-ze˩} & \ipa{jo˩-gi˩ li˥, {\kern14pt} jo˩-gi˩ li˥-ze˩}\\
			M\textsubscript{b} & \ipa{tsi˧\textsubscript{b}} & to set & \ipa{jo˩lo˩ tsi˩˥, jo˩lo˩ tsi˩-ze˥}{\kern2pt}\ipa{≈}{\kern2pt}\ipa{jo˩lo˩ tsi˥, jo˩lo˩ tsi˥-ze˩} & \ipa{jo˩-gi˩ tsi˥, jo˩-gi˩ tsi˥-ze˩}{\kern2pt}\ipa{≈}{\kern2pt}\ipa{jo˩-gi˩ tsi˩˥, jo˩-gi˩ tsi˩-ze˥}\\
			L\textsubscript{a} & \ipa{kwɤ˩\textsubscript{a}} & to throw & \ipa{jo˩lo˩ kwɤ˥,{\kern8pt} jo˩lo˩ kwɤ˥-ze˩} & \ipa{jo˩-gi˩ kwɤ˥,{\kern8pt} jo˩-gi˩ kwɤ˥-ze˩}\\
			L\textsubscript{b} & \ipa{ɻ̍˩\textsubscript{b}} & to turn & \ipa{jo˩lo˩ ɻ̍˥,{\kern22pt} jo˩lo˩ ɻ̍˥-ze˩} & \ipa{jo˩-gi˩ ɻ̍˥,{\kern14pt} jo˩-gi˩ ɻ̍˥-ze˩}\\
			MH & \ipa{mi˧˥} & to push & \ipa{jo˩lo˩ mi˥,{\kern14pt} jo˩lo˩ mi˥-ze˩} & \ipa{jo˩-gi˩ mi˥,{\kern8pt} jo˩-gi˩ mi˥-ze˩}\\
			\lspbottomrule
		\end{tabularx}
	\end{table}
\clearpage
	\begin{table}%[h!]
		\caption{\label{tab:spatialdiFULLLUPDOWN}The tonal behaviour of verbs after indications of spatial orientation: orientation adverbials //\ipa{gɤ˩tɕo˧}// ‘upward’ and //\ipa{mv̩˩tɕo˧}// ‘downward’.}
		\begin{tabularx}{\textwidth}{ l@{\hspace{7mm}} l@{\hspace{7mm}} l@{\hspace{7mm}} Q Q }
			\lsptoprule
			tone & example & meaning & ‘upward’  &  ‘downward’\\ \midrule
			%Use of kerning to obtain splitting of the contents of the last two columns over two lines: more symmetrical to the eye.
			H & \ipa{se˥} & to walk & \ipa{gɤ˩tɕo˧ se˧,{\kern8pt} gɤ˩tɕo˧ se˧-ze˩} & \ipa{mv̩˩tɕo˧ se˧,{\kern8pt} mv̩˩tɕo˧ se˧-ze˩}\\
			M\textsubscript{a} & \ipa{li˧\textsubscript{a}} & to look & \ipa{gɤ˩tɕo˧ li˧, {\kern8pt} gɤ˩tɕo˧ li˧-ze˧} & \ipa{mv̩˩tɕo˧ li˧,{\kern8pt} mv̩˩tɕo˧ li˧-ze˧}\\
			M\textsubscript{b} & \ipa{tsi˧\textsubscript{b}} & to set & \ipa{gɤ˩tɕo˧ tsi˧, {\kern8pt} gɤ˩tɕo˧ tsi˧-ze˧} & \ipa{mv̩˩tɕo˧ tsi˧, {\kern8pt} mv̩˩tɕo˧ tsi˧-ze˧}\\
			L\textsubscript{a} & \ipa{kwɤ˩\textsubscript{a}} & to throw & \ipa{gɤ˩tɕo˧ kwɤ˩ ,{\kern8pt} gɤ˩tɕo˧ kwɤ˩-ze˩} & \ipa{mv̩˩tɕo˧ kwɤ˩, {\kern8pt} mv̩˩tɕo˧ kwɤ˩-ze˩}\\
			L\textsubscript{b} & \ipa{ɻ̍˩\textsubscript{b}} & to turn & \ipa{gɤ˩tɕo˧ ɻ̍˩,{\kern14pt} gɤ˩tɕo˧ ɻ̍˩-ze˩} & \ipa{mv̩˩tɕo˧ ɻ̍˩,{\kern14pt} mv̩˩tɕo˧ ɻ̍˩-ze˩}\\
			MH & \ipa{mi˧˥} & to push & \ipa{gɤ˩tɕo˧ mi˧˥,{\kern8pt} gɤ˩tɕo˧ mi˧-ze˥} & \ipa{mv̩˩tɕo˧ mi˧˥,{\kern8pt} mv̩˩tɕo˧ mi˧-ze˥}\\
			\lspbottomrule
		\end{tabularx}
	\end{table}

% Keep the line below at end of WHOLE SET of subtables with full forms
\end{subtables}

 
All the data shown in \tabref{tab:spatialFULL} boils down to the underlying tone patterns summarized in \tabref{tab:spatialABSTRACT}. The tone indicated after a~‘+’ sign is that carried by the
{perfective} \mbox{/\ipa{-ze˧}/} as it appears after the directional"=plus"=verb combination at issue. For example, the notation L.M+H for
the combination of /\ipa{gɤ˩}-/ ‘upward’ and a~verb with M\textsubscript{b} tone indicates that the pattern is
/\ipa{gɤ˩-tɕi˧}/, /\ipa{gɤ˩-tɕi˧-ze˥}/ ‘to set in an~upward direction’. 

An especially interesting aspect of this data is the lowering of the {perfective} \is{suffixes}suffix (which carries a~lexical M tone: /\mbox{/\ipa{-ze˧}/}/) after expressions such as /\ipa{mv̩˩tɕo˧ se˧}/ ‘to go downward’, yielding /\ipa{mv̩˩tɕo˧ se˧-ze˩}/ ‘(s)he went downward’. This contrasts with expressions such as /\ipa{mv̩˩tɕo˧ li˧}/ ‘to look downward’, after which the {perfective} surfaces with its lexical M tone: /\ipa{mv̩˩tɕo˧ li˧-ze˧}/ ‘(s)he looked downward’. The lowering influence of an overt H tone on following tones within a~\isi{tone group} is a~hard"=and"=fast {phonological rule}; but the expression /\ipa{mv̩˩tɕo˧ se˧}/ ‘to go downward’ does not contain a~H tone at the surface phonological level. At a~deeper level, referred to in this volume as the \textit{underlying} phonological level, the expressions /\ipa{mv̩˩tɕo˧ se˧}/ ‘to go downward’ and /\ipa{mv̩˩tɕo˧ li˧}/ ‘to look downward’ must be analyzed as carrying different tones, since they behave differently with the same \is{suffixes}suffix. Since tone lowering is characteristic of H tones, it seems reasonable to posit a~\is{floating tone}floating H in the underlying representation of /\ipa{mv̩˩tɕo˧ se˧}/ ‘to go downward’, thus: //\ipa{mv̩˩tɕo˧ se\#˥}//. If the \is{suffixes}suffix /\mbox{/\ipa{-ze˧}/}/ surfaced with H~tone in any of the combinations in \tabref{tab:spatialABSTRACT}, that H~tone would need to be recognized as the manifestation of a~\is{floating tone}floating H tone from the directional"=plus"=verb expression; this would contradict headlong the interpretation proposed here: that \is{floating tone}floating H tones on such expressions only manifest themselves through \textit{lowering} of a~following tone. Importantly, the \is{suffixes}suffix /\mbox{/\ipa{-ze˧}/}/ does \textit{not} surface with H~tone in any of the combinations in \tabref{tab:spatialABSTRACT}. This is taken as confirmation of the present analysis, in light of which the data in \tabref{tab:spatialABSTRACT} is rewritten in \tabref{tab:spatialUNDERL}, indicating the underlying tone patterns. 

In \tabref{tab:spatialUNDERL}, reference needs to be made to the \is{juncture (inside a tone group)}juncture between the directional morpheme and the verb (similar to junctures internal to \is{numerals}numeral"=plus"=classifier phrases and \is{compounds}compound nouns, studied in previous chapters). This morpheme break is indicated by the symbol ‘--’. Thus, the indication ‘L\#--’ refers to a~final L tone (L\#) attaching before the morpheme break. For instance, association of L\#-- to the syllable sequence /\ipa{ʁo.tʰo.li}/ ‘to look back(ward)’ requires identification of the morpheme break that follows the directional /\ipa{ʁo.tʰo}/ ‘backward’: /\ipa{ʁo.tʰo~-- li}/. The L tone associates to the last syllable of the first part of the expression, yielding /\ipa{ʁo.tʰo˩}/, and spreads (by Rule~1) to the following syllable, hence /\ipa{ʁo.tʰo˩ li˩}/. Finally, the expression's first syllable receives tone M (by Rule~2), hence /\ipa{ʁo˧tʰo˩ li˩}/. This is shown in \figref{fig:lookback}.

\begin{figure}
	\caption{Illustration of the anchoring of tones relative to an internal juncture (notation: ‘--’): representation of association of L\#-- tone to the expression /\ipa{ʁo.tʰo~--~li}/ to yield /\ipa{ʁo˧tʰo˩ li˩}/ ‘to look back'.}
	\begin{tikzpicture}

	
	\node (9) at (1.5,-7) {L\#--};
	
	\node (23) at (0,-8.5) {σ};
	\node (33) at (1,-8.5) {σ};
	\node (3333) at (1.5,-8.5) {--};
	\node (53) at (2,-8.5) {σ};
	\node [anchor=mid] (1553) at (0,-9) {\ipa{ʁo}};
	\node [anchor=mid] (1653) at (1,-9) {\ipa{tʰo˩}};
	\node [anchor=mid] (1753) at (2,-9) {\ipa{li}};
	
	\node[text width=40mm] (s3) at (-3,-7.75) {Stage 1:\\ association of L tone\\ to its specified locus:\\ before the \is{juncture (inside a tone group)}juncture\\between the two parts of the expression};
	
	% arrow from L tone: 
	\draw[decoration={markings,mark=at position 1 with
		{\arrow[scale=2,>=stealth]{>}}},postaction={decorate}] (9) -- (33);
	
	
	\node (44) at (1,-10) {L};
	
	\node (24) at (0,-11.5) {σ};
	\node (34) at (1,-11.5) {σ};
	\node (54) at (2,-11.5) {σ};
	\node [anchor=mid] (1500) at (0,-12) {\ipa{ʁo}};
	\node [anchor=mid] (1600) at (1,-12) {\ipa{tʰo˩}};
	\node [anchor=mid] (1700) at (2,-12) {\ipa{li˩}};
	
	\node[text width=40mm] (s4) at (-3,-10.5) {Stage 2:\\ assignment of L tone\\ by {phonological rule}:\\ L"=tone spreading};
	
	\draw (44) -- (34);	
	\draw[decoration={markings,mark=at position 1 with
		{\arrow[scale=2,>=stealth]{>}}},postaction={decorate}] (44) -- (54);	
	
	\node (14) at (0,-13) {M};
	\node (64) at (1,-13) {L};
	\node (44) at (2,-13) {L};
	
	\node (24) at (0,-14.5) {σ};
	\node (34) at (1,-14.5) {σ};
	\node (54) at (2,-14.5) {σ};
	\node [anchor=mid] (2000) at (0,-15) {\ipa{ʁo˧}};
	\node [anchor=mid] (2100) at (1,-15) {\ipa{tʰo˩}};
	\node [anchor=mid] (2200) at (2,-15) {\ipa{li˩}};
	
	\node[text width=40mm] (s4) at (-3,-13.5) {Stage 3:\\ addition of M tone\\ to the remaining\\ toneless syllable};
	
	\draw[decoration={markings,mark=at position 1 with
		{\arrow[scale=2,>=stealth]{>}}},postaction={decorate}] (14) -- (24);	
	\draw (64) -- (34);
	\draw (44) -- (54);
	\end{tikzpicture}
	\label{fig:lookback}
\end{figure}


\begin{subtables}
	\label{tab:spatialABSTRACT}
	\begin{table}[h!]%[t]
		\caption{\label{tab:spatialmono}The surface tone patterns of verbs after indications of spatial orientation: {monosyllabic}
			prefixes.}
		\begin{tabularx}{\textwidth}{ P{14mm} l@{\hspace{8mm}} l@{\hspace{8mm}} Q Q Q Q }
			\lsptoprule
			\multirow{2}{14mm}{tone of prefix} & \multicolumn{6}{l}{tone of verb}\\ \cmidrule{2-7}
			& H & M\textsubscript{a} & M\textsubscript{b} & L\textsubscript{a} & L\textsubscript{b} & MH\\ \midrule
			L & L.H & L.M+M & L.M+M & L.H & L.H & L.MH\\
			\lspbottomrule
		\end{tabularx}
	\end{table}
	
	\begin{sidewaystable}[p]
		\caption{\label{tab:spatialdi}The surface tone patterns of verbs after indications of spatial orientation: disyllabic orientation adverbials.}
		{\renewcommand{\arraystretch}{1.15}  
			\begin{tabularx}{\textheight}{ l Q Q Q Q l@{\hspace{6mm}} l@{\hspace{6mm}} Q }
				\lsptoprule
				\multirow{2}{14mm}{tone of prefix} & \multicolumn{7}{l}{tone of verb}\\ \cmidrule{2-8}
				& H & M\textsubscript{a} & M\textsubscript{b} & M\textsubscript{c} & L\textsubscript{a} &
				L\textsubscript{b} & MH\\ \midrule
				M & M.M.M+L & M.M.M+M & M.M.M+L & M.M.M+M & M.M.L & M.M.L & M.M.MH\\
				\#H & M.M.M+L & M.M.L & M.M.M+L & M.M.M+L & M.M.H & M.M.H & M.M.L\\
				L & L.L.L & L.L.H & L.L.H / L.L.L & L.L.L & L.L.H & L.L.H & L.L.H\\
				L\# & M.L.L & M.L.L & M.L.L & M.L.L & M.L.L & M.L.L & M.L.L\\
				LM & L.M.M+L & L.M.M+M & L.M.M+M & L.M.M+M & L.M.L & L.M.L & L.M.MH\\
				H\# & M.H.L & M.H.L & M.H.L & M.H.L & M.H.L & M.H.L & M.H.L\\
				\lspbottomrule
			\end{tabularx}}
		\end{sidewaystable}
	\end{subtables}
	
\begin{subtables}
	\label{tab:spatialUNDERL}
	\begin{table}[h!]%[t]
		\caption{\label{tab:spatialmonoUNDERL}The underlying tone patterns of verbs after indications of spatial orientation: {monosyllabic}
			prefixes.}
		\begin{tabularx}{\textwidth}{ P{14mm} l@{\hspace{8mm}} l@{\hspace{8mm}} Q Q Q Q }
			\lsptoprule
			\multirow{2}{14mm}{tone of prefix} & \multicolumn{6}{l}{tone of verb}\\ \cmidrule{2-7}
			& H & M\textsubscript{a} & M\textsubscript{b} & L\textsubscript{a} & L\textsubscript{b} & MH\\ \midrule
			L & L.H & L.M & L.M & L.H & L.H & L.MH\\
			\lspbottomrule
		\end{tabularx}
	\end{table}
	
	\begin{sidewaystable}[p]
		\caption{\label{tab:spatialdiUNDERL}The underlying tone patterns of verbs after indications of spatial orientation: disyllabic orientation adverbials.}
		{\renewcommand{\arraystretch}{1.15}  
			\begin{tabularx}{\textheight}{ l Q Q Q Q l@{\hspace{6mm}} l@{\hspace{6mm}} Q }
				\lsptoprule
				\multirow{2}{14mm}{tone of prefix} & \multicolumn{7}{l}{tone of verb}\\ \cmidrule{2-8}
				& H & M\textsubscript{a} & M\textsubscript{b} & M\textsubscript{c} & L\textsubscript{a} &
				L\textsubscript{b} & MH\\ \midrule
				M & \tikzmark{A1}\#H & M & \tikzmark{C1}\#H & M & \tikzmark{E1}L\# & \hspace*{\fill}\tikzmark{F1} & MH\#\\
				\#H & \hspace*{\fill}\tikzmark{A2} & L\# & \hspace*{\fill}\tikzmark{C2} & \tikzmark{D2}\#H & \tikzmark{A3}H\# & \hspace*{\fill}\tikzmark{B3} & \#H\\
				L & L & L+H\# & L+H\# / L & L & \tikzmark{E3}L+H\# &  & \hspace*{\fill}\tikzmark{G3}\\
				L\# & \tikzmark{CROCHERON}L\#-- &  &  &  &  &  & \hspace*{\fill}\tikzmark{CROCHE}\\
				LM & LM+\#H & \tikzmark{Y1}LM & & \hspace*{\fill}\tikzmark{Y9} & \tikzmark{LMLdeb}LM+L\# & \hspace*{\fill}\tikzmark{LMLfin} & LM+MH\#\\
				H\# & \tikzmark{Z1}H\#-- & &  &  & & & \hspace*{\fill}\tikzmark{Z9}\\
				\lspbottomrule
			\end{tabularx}}
		\DrawBox[dashed]{A1}{A2}
		\DrawBox[dashed]{A3}{B3}
		\DrawBox[dashed]{E1}{F1}
		\DrawBox[dashed]{E3}{G3}
		\DrawBox[dashed]{C1}{C2}
		\DrawBox[dashed]{CROCHERON}{CROCHE}
		\DrawBox[dashed]{LMLdeb}{LMLfin}
		\DrawBox[dashed]{Y1}{Y9}
		\DrawBox[dashed]{Z1}{Z9}
		\end{sidewaystable}
	\end{subtables}
	
	
A further intricacy is that the behaviour of /\ipa{ɑ˩pʰo˩}/ ‘outside’ is not fully identical with
that of /\ipa{jo˩gi˩}/ and /\ipa{jo˩lo˩}/ ‘to the right’, even though these three spatial indications have the same
lexical tone. In association with /\ipa{tʰv̩˧\textsubscript{a}}/ ‘come out’, /\ipa{ɑ˩pʰo˩}/ ‘outside’ yields
/\ipa{ɑ˩pʰo˩ tʰv̩˩}/ ‘to go outside, to get outside’, instead of the expected /\ipa{†ɑ˩pʰo˩
  tʰv̩˥}/. The latter form is not acceptable as a~\is{variants}variant.

  \largerpage
Closer examination of this issue reveals yet another \is{irregularities}{oddity}: different verbs that belong in the same
tonal category, M\textsubscript{a}, have different tone patterns when associated with /\ipa{ɑ˩pʰo˩}/ ‘outside’. ‘To
look outside’ (from /\ipa{li˧\textsubscript{a}}/ ‘to look’) is /\ipa{ɑ˩pʰo˩ li˥}/, and $\ddagger${\kern2pt}\ipa{ɑ˩pʰo˩ li˩} is not
an~acceptable \is{variants}variant. The verb /\ipa{tʰv̩˧\textsubscript{a}}/ ‘come out’ is an~outlier: it is the only M\textsubscript{a}-tone verb
yielding a~L.L.L tone pattern in association with /\ipa{ɑ˩pʰo˩}/ ‘outside’.

In view of the fact that L.L.L and L.L.H are both acceptable variants for /\ipa{jo˩gi˩}/ and
/\ipa{jo˩lo˩}/ ‘to the right’ followed by a~M\textsubscript{a}-tone verb, one can venture the speculation that the
same pattern of \isi{variation} once existed for /\ipa{ɑ˩pʰo˩}/ ‘outside’. Under this hypothesis, the
L.L.L \is{variants}variant must have become dominant for ‘to go outside, to get outside’, to the extent that the form /\ipa{ɑ˩pʰo˩ tʰv̩˩}/ came to be regarded as the only correct one. 

\newpage 
But even if one chooses to treat the combination /\ipa{ɑ˩pʰo˩ tʰv̩˩}/ ‘to go outside, to get outside’ as a~lexicalized {oddity}, the behaviour of /\ipa{ɑ˩pʰo˩}/ ‘outside’ is still different from that of /\ipa{jo˩gi˩}/ and
/\ipa{jo˩lo˩}/ ‘to the right’: see \tabref{tab:thetonalbehaviourofverbsinassociationwithoutside}. Here again, knowledge of the input tones is not sufficient to generate the tone patterns.

\begin{table}%[t]
\caption{\label{tab:thetonalbehaviourofverbsinassociationwithoutside}The tonal behaviour of verbs in association with /\ipa{ɑ˩pʰo˩}/ ‘outside’.}
\begin{tabularx}{\textwidth}{ l@{\hspace{8mm}} l@{\hspace{8mm}} l@{\hspace{8mm}} Q }
\lsptoprule
	tone of verb & example & meaning of verb & tone pattern\\ \midrule
	H & \ipa{ɑ˩pʰo˩ se˩} & to walk & L.L.L\\
	M\textsubscript{a} & \ipa{ɑ˩pʰo˩ li˥} & to look & L.L.H\\
	M\textsubscript{a} (exceptional) & \ipa{ɑ˩pʰo˩ tʰv̩˩} & to get/go & L.L.L\\
	M\textsubscript{b} & \ipa{ɑ˩pʰo˩ hõ˩} & to go.\textsc{imperative} & L.L.L\\
	M\textsubscript{c} & \ipa{ɑ˩pʰo˩ bi˩} & to go & L.L.H\\
	L\textsubscript{a} & \ipa{ɑ˩pʰo˩ kwɤ˥} & to throw & L.L.H\\
	L\textsubscript{b} & \ipa{ɑ˩pʰo˩ pʰv̩˥} & to move around & L.L.H\\
	MH & \ipa{ɑ˩pʰo˩ ʑi˥} & to sleep & L.L.H\\
\lspbottomrule
\end{tabularx}
\end{table}


The interrogative /\ipa{zo˩qo˧}/ ‘where’ has the same behaviour as /\ipa{gɤ˩tɕo˧}/ ‘upward’ and
/\ipa{mv̩˩tɕo˧}/ ‘downward’, as shown in \tabref{tab:thetonalbehaviourofverbsinassociationwithwhere}. In association with /\ipa{zo˩qo˧}/ ‘where’, the verb /\ipa{tʰv̩˧}/, whose association with
/\ipa{ɑ˩pʰo˩}/ ‘outside’ yields an~unexpected pattern (see \tabref{tab:thetonalbehaviourofverbsinassociationwithoutside}), is not any different from the
other M\textsubscript{a}-tone verbs, yielding /\ipa{zo˩qo˧} \ipa{tʰv̩˧}(\ipa{-ze˧})/.

\begin{table}%[t]
\caption{\label{tab:thetonalbehaviourofverbsinassociationwithwhere}The tonal behaviour of verbs in association with /\ipa{zo˩qo˧}/ ‘where’, with added information about a~following {perfective} morpheme.}
\begin{tabularx}{\textwidth}{ Q l@{\hspace{8mm}} l@{\hspace{8mm}} l }
\lsptoprule
	tone of verb & example & meaning of verb & tone pattern\\ \midrule
	H & \ipa{zo˩qo˧ se˧(-ze˩)} & to walk & L.M.M+L\\
	M\textsubscript{a} & \ipa{zo˩qo˧ ʂe˧(-ze˧)} & to look for & L.M.M+M\\
	M\textsubscript{b} & \ipa{zo˩qo˧ pʰæ˧(-ze˧)} & to attach, to fasten & L.M.M+M\\
	M\textsubscript{c} & \ipa{zo˩qo˧ hɯ˧(-ze˧)} & to go.\textsc{pst} & L.M.M+M\\
	L\textsubscript{a} & \ipa{zo˩qo˧ dzi˩} & to sit; to live & L.M.L\\
	L\textsubscript{b} & \ipa{zo˩qo˧ ɻ̍˩} & to turn toward & L.M.L\\
	MH & \ipa{zo˩qo˧ lɑ˧˥} & to strike, to hit & L.M.MH\\
\lspbottomrule
\end{tabularx}
\end{table}

 

%\subsubsection{On the morphosyntactic analysis of locative constituents}

\newpage 
In principle, the tonal behaviour of locative constituents in Yongning Na could shed light on their morphosyntactic
treatment. Various configurations are attested cross"=linguistically. For instance, in Central \ili{Bantu} agreement with the
verb reveals a~typologically uncommon pattern: “a locative noun phrase in preverbal position
can be analyzed as the grammatical subject” \citep[34]{creissels2011}, whereas in Northern Sotho, in
the absence of such evidence, the construction is better analyzed as “an impersonal construction
with a~preposed locative constituent” \citep{zerbian2006b}. But the Yongning Na tone patterns presented above are not fully identical with those of any other construction. In particular, they differ from subject"=plus"=verb as well as from object"=plus"=verb
constructions.

Let us now turn from pre"=verbal elements to post"=verbal elements.


\section{Monosyllabic postverbal morphemes}
\label{sec:verbalsuffixesandverbserializationmonosyllabicelements}

Verbs appear after their objects. They can be followed by various morphemes, such as suffixes,\footnote{\citet[349]{lidz2010} proposed, on the basis of data from the Luoshui dialect, that “[s]uffixation is not attested on verbs in Na”; but for the dialect under description here, it appears easiest to recognize a~few postverbal morphemes as suffixes, e.g.~the {perfective} \mbox{/\ipa{-ze˧}/} (analyzed by Lidz (\citeyear[424]{lidz2010}) as a~postverbal particle). In transcriptions, a~hyphen is also used in cases where a~verb's grammaticalized use appears distinct enough from the verb's original meaning to warrant separate recognition, for instance the {immediate future}, /\ipa{-bi˧}/, grammaticalized from the non"={imperative} form of ‘to go’, /\ipa{bi˧\textsubscript{c}}/. The decision to treat a~morpheme as a~serialized verb or a~{suffix} can be difficult. My use of hyphens fluctuated in the years of preparation of this volume, and there remain some inconsistencies in the transcriptions of texts.} serialized verbs, postpositions, and discourse particles, which will be referred to here by the cover term ‘postverbal morphemes'. 

From the point of view of tone patterns, a~key fact is that not all morphemes following verbs have the same tonal behaviour, suggesting that they have lexical tones of their own. Distributional analysis constitutes a~first step in approaching these tones: finding out how many tonal categories there are for a~given part of speech. After morphemes have been sorted into tonal sets, the next step consists in analyzing the underlying tones, as for nouns and verbs. On the one hand, in Yongning Na all syllables at the surface phonological level carry level tones (H, M, L, and combinations into low"=rising and mid"=rising contours), so it makes sense to hypothesize that the lexical tone categories all have these level tones as their building"=blocks, and to attempt to pinpoint the phonological nature of the tonal categories of \isi{function words}, as was done above for nouns and verbs. On the other hand, one should keep in mind the structural fact that the tone systems for different word classes are not identical. Beyond differences in the number of tones (among monosyllables: seven categories for verbs, five for adjectives, six for free nouns, nine for classifiers{\dots}), the behaviour of these tones in context varies across word classes, even for highly similar classes such as verbs and adjectives (as pointed out in \sectref{sec:adjectivesasdistinctfromverbs}). As a consequence, it is not possible to carry over the tonal analysis of a~verb to its grammaticalized counterpart. For instance, the morpheme indicating completion, \mbox{/\ipa{-se˩}/}, is clearly a~grammaticalized use of the verb /\ipa{se˩}/ ‘to finish, to complete'. But this fact does not by itself warrant the conclusion that the grammaticalized morpheme carries the same lexical tone as the verb. The process of \isi{grammaticalization} can be accompanied by a~change of lexical tone: this is evidenced by the example of classifiers, studied in Chapter~\ref{chap:classifiers}. Returning to postverbal morphemes, cases where they can straightforwardly be identified as originating in a~lexical verb only offer indirect hints on tonal identity. The evidence on the tone categories of the grammatical morphemes studied in this chapter is therefore assembled piece by piece, and there remains room for further progress in the analysis. The reader will be reminded now and then that a~given label (say, L tone) given to two different parts of speech (such as a~verb and a~\is{suffixes}suffix) cannot be said to refer to the same phonological entity: instead, those are two distinct morphophonological entities, which appear similar enough (at the present stage in the analysis) to warrant the use of the same label from among the set of phonologically distinctive tonal levels.

Monosyllabic elements will be discussed first, before getting on to disyllabic postpositions and combinations of affixes.

\subsection[L tone]{L-tone postverbal morphemes}
\label{sec:ltonesuffixesandserializedverbs}

\subsubsection{Main facts}
\label{sec:mainfactsaboutlsuffix}

\largerpage[-2]
As mentioned a~few lines above, the morpheme indicating completion, \mbox{/\ipa{-se˩}/}, is a~grammaticalized use of the verb /\ipa{se˩}/ ‘to finish, to complete'. This does not by itself constitute sufficient evidence that its lexical tone is L, but its tonal behaviour after M-tone verbs (where it surfaces with L tone) points in the same direction, and the tonal category in which it belongs is therefore labelled as L. Since the tonal behaviour of the {desiderative} morpheme is the same as that of the morpheme indicating completion, the same tonal label is applied, interpreting the {desiderative} as \mbox{/\ipa{-ho˩}/}. Data is shown in \tabref{tab:thepatternsofltonetenseaspectmoodsuffixes}, also including data for the morpheme /\ipa{-sɯ˩}/ ‘yet’ (in the negative construction ‘not yet’). Other items belonging to the same class (L tone) include the {inchoative}, /\ipa{-ɻ̍˩}/, and the morpheme /\ipa{-dze˩}/ ‘to remain; to be left over’. The latter is only observed in /\ipa{dzɯ˧-dze˥}/ ‘left over after eating’ and /\ipa{ʈʰɯ˩-dze˩}/ ‘left over after drinking (or smoking)’.\footnote{I have not been able to come up with better {English} translations for these \textit{noncompletion resultatives}, which refer to the state of affairs that results from an action that stopped short of completion (incomplete consumption of an object). {Mandarin} Chinese has handy equivalents: \textit{chī shèng de} \zh{吃剩的} for ‘left over after eating’ and \textit{hē shèng de} \zh{喝剩的} for ‘left over after drinking’.}

\begin{table}%[t]
\caption{\label{tab:thepatternsofltonetenseaspectmoodsuffixes}The patterns of L-tone tense"=aspect"=modality morphemes.}
\begin{tabularx}{\textwidth}{ l@{\hspace{4mm}} l@{\hspace{4mm}} l@{\hspace{4mm}} l@{\hspace{3mm}} P{25mm}@{\hspace{3mm}} Q@{\hspace{1mm}} }
\lsptoprule
	tone & example & meaning & \textsc{completion} & \textsc{desiderative} & not yet\\ \midrule
	H & \ipa{dzɯ˥} & to eat & \ipa{dzɯ˧-se˥} & \ipa{dzɯ˧-ho˥} & \ipa{mɤ˧-dzɯ˧-sɯ˥}\\
	M\textsubscript{a} & \ipa{hwæ˧\textsubscript{a}} & to buy & \ipa{hwæ˧-se˩} & \ipa{hwæ˧-ho˩} & \ipa{mɤ˧-hwæ˧-sɯ˩}\\
	M\textsubscript{b} & \ipa{tɕʰi˧\textsubscript{b}} & to sell & \ipa{tɕʰi˧-se˩} & \ipa{tɕʰi˧-ho˩} & \ipa{mɤ˧-tɕʰi˧-sɯ˩}\\
	M\textsubscript{c} & \ipa{pv̩˧\textsubscript{c}} & to chant & \ipa{pv̩˧-se˩} & \ipa{pv̩˧-ho˩} & \ipa{mɤ˧-pv̩˧-sɯ˩}\\
	L\textsubscript{a} & \ipa{bæ˩\textsubscript{a}} & to sweep & \ipa{bæ˩-se˩} & \ipa{bæ˩-ho˩} & \ipa{mɤ˧-bæ˩-sɯ˩}\\
	L\textsubscript{b} & \ipa{ʐwɤ˩\textsubscript{b}} & to speak & \ipa{ʐwɤ˩-se˩} & \ipa{ʐwɤ˩-ho˩} & \ipa{mɤ˧-ʐwɤ˩-sɯ˩}\\
	MH & \ipa{lɑ˧˥} & to strike & \ipa{lɑ˧-se˥} & \ipa{lɑ˧-ho˥} & \ipa{mɤ˧-lɑ˧-sɯ˥}\\
\lspbottomrule
\end{tabularx}
\end{table}


\subsubsection{A nominalizing suffix}
\label{sec:!nominalization}

It has been observed in a~study of the Luoshui dialect of Yongning Na that “\ipa{di³³} ‘earth; place’ grammaticalized into a~locative nominalizer, and then further
grammaticalized into a~purposive nominalizer” \citep[184]{lidz2010}. For the Alawa dialect, the nominalizing morpheme is assigned L tone (hence /\ipa{-di˩}/) on the basis of its behaviour after
M-tone verbs. It behaves differently from the L-tone
morphemes discussed in the previous paragraph: /\ipa{dzɯ˧-di˧˥}/ ‘food, things for eating’ vs.\ /\ipa{dzɯ˧-ho˥}/ ‘will eat’; /\ipa{ʈʰæ˧-di˧˥}/
‘thing to bite (e.g.~toy given to teething babies)’ vs.\ /\ipa{ʈʰæ˧-ho˥}/ ‘will bite’. This observation is taken as confirmation that the nominalizing morpheme is to be analyzed as a~\is{suffixes}suffix, belonging in a~morphotonological class that is distinct from that of serialized verbs. The data is
set out in \tabref{tab:thetonalbehaviourofthenominalizingsuffix}. 

\begin{table}%[t]
	\caption{\label{tab:thetonalbehaviourofthenominalizingsuffix}The tonal behaviour of the nominalizing suffix /\ipa{-di˩}/.}
	\begin{tabularx}{\textwidth}{ l@{\hspace{5mm}} l@{\hspace{5mm}} l@{\hspace{5mm}} l@{\hspace{5mm}} Q }
		\lsptoprule
		tone & example & meaning & nominalizer & meaning\\ \midrule
		H & \ipa{dzɯ˥} & to eat & \ipa{dzɯ˧-di˧˥} & thing to eat, food\\
		M\textsubscript{a} & \ipa{hwæ˧\textsubscript{a}} & to buy & \ipa{hwæ˧-di˩} & thing to buy, product\\
		M\textsubscript{b} & \ipa{tɕʰi˧\textsubscript{b}} & to sell & \ipa{tɕʰi˧-di˩} & thing to sell, commodity\\
		M\textsubscript{c} & \ipa{pv̩˧\textsubscript{c}} & to chant & \ipa{pv̩˧-di˩} & thing to chant, ritual\\
		L\textsubscript{a} & \ipa{dze˩\textsubscript{a}} & to cut & \ipa{dze˩-di˩} & thing to cut, e.g.~knife\\
		L\textsubscript{b} & \ipa{ʈʰɯ˩\textsubscript{b}} & to drink & \ipa{ʈʰɯ˩-di˩} & thing to drink, beverage\\
		MH & \ipa{ʈʰæ˧˥} & to bite & \ipa{ʈʰæ˧-di˧˥} & thing to bite (for infant, dog{\dots})\\
		\lspbottomrule
	\end{tabularx}
\end{table}


As an~aside, consultant M21 has different tone patterns for the L tone: L.H
(e.g.~/\ipa{dze˩-di˥}/ and /\ipa{ʈʰɯ˩-di˥}/), and not L.L. This could result from \isi{analogy} with
the L-tone morphemes described in \sectref{sec:mainfactsaboutlsuffix}, such as the morpheme indicating completion, \mbox{/\ipa{-se˩}/}, which takes H tone
after a~L-tone verb.
  

\subsection[M tone]{M-tone postverbal morphemes}
\label{sec:mtonesuffixes}

The {imperative},
/\ipa{-hõ˧}/, is grammaticalized from the {imperative} form of the verb ‘to go’, /\ipa{hõ˧\textsubscript{a}}/; and the {immediate future}, /\ipa{-bi˧}/, from the non"={imperative} form of ‘to go’, /\ipa{bi˧\textsubscript{c}}/. This constitutes a~first hint that the tonal category of these two morphemes could be labelled as M. A~stronger argument comes from their behaviour in association with the seven tonal categories of verbs, shown in \tabref{tab:thetonalbehaviourofmtonesuffixes}. The table also shows data for the experiential /\ipa{-dʑɯ˧}/, which has the same tonal behaviour in all cases. The tone patterns after a~verb with M, L or MH tone seem characteristic of a~neutral tone: the postverbal morpheme allows the verb's MH tone to unfold over it, and the verb's L tone to spread over it; after a~M-tone verb, the surface result is M, which is consistent with the phonologically inert nature of M tone, already mentioned in \sectref{sec:analysisofmasadefaulttone}. To preview observations set out in \sectref{sec:suffixesprecededbythenegation}, another test confirms the appropriateness of M tone as a~label for the tonal category to which these morphemes belong: after the {negation} \is{prefixes}prefix, they surface with a~M tone. 

Other morphemes with the same
tonal behaviour include /\ipa{-ɖo˧}/ ‘must, have to’, the {volitive} /\ipa{-tso˧}/, the {obligative}
/\ipa{-zo˧}/, and the {causative} /\ipa{-tsæ˧}/.

{\setlength\tabcolsep{5.5pt}
\begin{table}%[t]
\caption{\label{tab:thetonalbehaviourofmtonesuffixes}The tonal behaviour of M-tone morphemes.}
\begin{tabularx}{\textwidth}{ l l l l l Q l }
\lsptoprule
	tone & example & meaning & \textsc{imm\_fut} & experiential & imperative & ana\-lysis\\ \midrule
	H & \ipa{dzɯ˥} & to eat & \ipa{dzɯ˧-bi˧} & \ipa{dzɯ˧-dʑɯ˧} & \ipa{dzɯ˧-hõ˧} & M\\
	M\textsubscript{a} & \ipa{hwæ˧\textsubscript{a}} & to buy & \ipa{hwæ˧-bi˧} & \ipa{hwæ˧-dʑɯ˧} & \ipa{hwæ˧-hõ˧} & M\\
	M\textsubscript{b} & \ipa{tɕʰi˧\textsubscript{b}} & to sell & \ipa{tɕʰi˧-bi˧} & \ipa{tɕʰi˧-dʑɯ˧} & \ipa{tɕʰi˧-hõ˧} & M\\
	M\textsubscript{c} & \ipa{pv̩˧\textsubscript{c}} & to chant & \ipa{pv̩˧-bi˧} & \ipa{pv̩˧-dʑɯ˧} & \ipa{pv̩˧-hõ˧} & M\\
	L\textsubscript{a} & \ipa{bæ˩\textsubscript{a}} & to sweep & \ipa{bæ˩-bi˩} & \ipa{bæ˩-dʑɯ˩} & \ipa{bæ˩-hõ˩} & L\\
	L\textsubscript{b} & \ipa{ʐwɤ˩\textsubscript{b}} & to speak & \ipa{ʐwɤ˩-bi˩} & \ipa{ʐwɤ˩-dʑɯ˩} & \ipa{ʐwɤ˩-hõ˩} & L\\
	MH & \ipa{lɑ˧˥} & to strike & \ipa{lɑ˧-bi˥} & \ipa{lɑ˧-dʑɯ˥} & \ipa{lɑ˧-hõ˥} & H\#\\
\lspbottomrule
\end{tabularx}
\end{table}}


\largerpage[-1]
In addition to the surface phonological forms, \tabref{tab:thetonalbehaviourofmtonesuffixes} proposes an~analysis of the underlying tone,
in the last column. This analysis is based on the tone patterns when the {causative} /\ipa{-tsæ˧}/ is
added after the {immediate future} /\ipa{-bi˧}/, as shown in \tabref{tab:immediatefuturecausativecopula}. The \isi{copula}, in its use to
convey {certainty}, was also added, as a~further test to reveal the underlying tonal categories. In every case, the \isi{copula} carries L, i.e.\ its lexical tone. This shows that there is no \is{floating tone}floating H tone in any of the verb phrases. The first three expressions are therefore interpreted as having M tone. The underlying tone category leading to the realizations /\ipa{lɑ˧-bi˥}/ ‘will strike’ and
/\ipa{lɑ˧-bi˧-tsæ˥}/ (not $\ddagger${\kern2pt}\ipa{lɑ˧-bi˥-tsæ˩}) ‘cause to strike’ is interpreted as H\#.


\begin{table}[t]
\caption{\label{tab:immediatefuturecausativecopula}Tone patterns of V+\textsc{immediate future}+ \textsc{causative}+\textsc{copula}.}
\begin{tabularx}{\textwidth}{ l@{\hspace{9mm}} Q Q l }
\lsptoprule
	tone & example & meaning & \ipa{-bi˧-tsæ˧-ɲi˩}\\ \midrule
	H & \ipa{dzɯ˥} & to eat & \ipa{dzɯ˧-bi˧-tsæ˧-ɲi˩}\\
	M\textsubscript{a} & \ipa{hwæ˧\textsubscript{a}} & to buy & \ipa{hwæ˧-bi˧-tsæ˧-ɲi˩}\\
	M\textsubscript{b} & \ipa{tɕʰi˧\textsubscript{b}} & to sell & \ipa{tɕʰi˧-bi˧-tsæ˧-ɲi˩}\\
	M\textsubscript{c} & \ipa{pv̩˧\textsubscript{c}} & to chant & \ipa{pv̩˧-bi˧-tsæ˧-ɲi˩}\\
	L\textsubscript{a} & \ipa{bæ˩\textsubscript{a}} & to sweep & \ipa{bæ˩-bi˩-tsæ˥-ɲi˩ ($\ddagger${\kern2pt}bæ˩-bi˩-tsæ˩˥-ɲi˩)}\\
	L\textsubscript{b} & \ipa{ʐwɤ˩\textsubscript{b}} & to speak & \ipa{ʐwɤ˩-bi˩-tsæ˥-ɲi˩ ($\ddagger${\kern2pt}ʐwɤ˩-bi˩-tsæ˩˥-ɲi˩)}\\
	MH & \ipa{lɑ˧˥} & to strike & \ipa{lɑ˧-bi˧-tsæ˥-ɲi˩ ($\ddagger${\kern2pt}lɑ˧-bi˧-tsæ˧˥-ɲi˩)}\\
\lspbottomrule
\end{tabularx}
\end{table}


\begin{table} 
\caption{\label{tab:sametonalcategoryofmorphemesasinprevioustablewithnegation}Tone patterns of V+\textsc{negation} prefix+\textsc{immediate future}.}
\begin{tabularx}{\textwidth}{ l@{\hspace{7mm}} l@{\hspace{7mm}} l@{\hspace{7mm}} l@{\hspace{7mm}} Q }
\lsptoprule
	tone & example & meaning & V \textsc{neg"=imm\_fut} & V \textsc{neg"=imm\_fut"=pfv}\\ \midrule
	H & \ipa{dzɯ˥} & to eat & \ipa{dzɯ˧ mɤ˧-bi˧} & \ipa{dzɯ˧ mɤ˧-bi˧-ze˩}\\
	M\textsubscript{a} & \ipa{hwæ˧\textsubscript{a}} & to buy & \ipa{hwæ˧ mɤ˧-bi˧} & \ipa{hwæ˧ mɤ˧-bi˧-ze˧}\\
	M\textsubscript{b} & \ipa{tɕʰi˧\textsubscript{b}} & to sell & \ipa{tɕʰi˧ mɤ˧-bi˧} & \ipa{tɕʰi˧ mɤ˧-bi˧-ze˧}\\
	M\textsubscript{c} & \ipa{pv̩˧\textsubscript{c}} & to chant & \ipa{pv̩˧ mɤ˧-bi˧} & \ipa{pv̩˧ mɤ˧-bi˧-ze˧}\\
	L\textsubscript{a} & \ipa{bæ˩\textsubscript{a}} & to sweep & \ipa{bæ˩ mɤ˩-bi˩} & \ipa{bæ˩ mɤ˩-bi˩-ze˥}\\
	L\textsubscript{b} & \ipa{ʐwɤ˩\textsubscript{b}} & to speak & \ipa{ʐwɤ˩ mɤ˩-bi˩} & \ipa{ʐwɤ˩ mɤ˩-bi˩-ze˥}\\
	MH & \ipa{lɑ˧˥} & to strike & \ipa{lɑ˧ mɤ˧-bi˧} & \ipa{lɑ˧ mɤ˧-bi˧-ze˩}\\
\lspbottomrule
\end{tabularx}
\end{table}

\largerpage[-1]  %longdistance
It is reassuring to be able to report that these M-tone morphemes have the same behaviour when preceded by the {negation} \is{prefixes}prefix,
/\ipa{mɤ˧}-/, e.g.~in /\ipa{V mɤ˧-bi˧}/ ‘is not going to V’, /\ipa{V mɤ˧-ɖo˧}/ ‘ought not to V’, and
/\ipa{V mɤ˧-zo˧}/ ‘must not V’. \tabref{tab:sametonalcategoryofmorphemesasinprevioustablewithnegation} shows examples with /\ipa{V mɤ˧-bi˧}/, and also with
an~added {perfective}, \mbox{/\ipa{-ze˧}/}. M-tone morphemes are not without complexities, however. The verb /\ipa{mæ˧}/ ‘to achieve’ carries M tone; but when it appears in serial verb
constructions, where it indicates that an~action achieved its goal, its behaviour is not fully identical to that
of the M-tone morphemes in \tabref{tab:thetonalbehaviourofmtonesuffixes}. The {perfective}, \mbox{/\ipa{-ze˧}/}, has the same tonal behaviour as /\ipa{-mæ˧}/. The behaviour of these two morphemes is recapitulated in \tabref{tab:thebehaviouroftonelesssuffixe}. The tone patterns are in most
respects like the M-tone morphemes in \tabref{tab:thetonalbehaviourofmtonesuffixes}, but the pattern after a~H-tone verb is
M.L and not M.M. This justifies setting up a~distinct tone category, distinguishing between M\textsubscript{a} tone and M\textsubscript{b} tone for postverbal morphemes. In the case of classifiers~-- about which see Chapter~\ref{chap:classifiers}~--, a~total of nine tonal categories came to light. It would not be particularly surprising to have to set up a~comparable number of categories for affixes. In view of the greater morphosyntactic diversity of affixes as compared with classifiers, the number could even be higher. Obviously, the added subscript letters are simply a~means of classifying: they do not reflect an~analysis of the categories. An analysis would be better conducted with a fuller
inventory of morphemes; it should be borne in mind that the current size of the dictionary \citep{michauddict2015} is on the order of 3,000 words, i.e.\ only a~small part of the full richness of the language's lexicon. In the present chapter, the postverbal morphemes are simply arranged into broad classes (L, M, H, and MH), reporting the encountered internal diversity of these classes, and setting up provisional subclasses, such as M\textsubscript{a} and M\textsubscript{b}. Subclasses M\textsubscript{a} and M\textsubscript{b} of M-tone postverbal morphemes contain the following items: 

\begin{itemize}
	\item{M\textsubscript{a} tone: \textsc{imperative} /\ipa{-hõ˧\textsubscript{a}}/, \textsc{immediate future} /\ipa{-bi˧\textsubscript{a}}/, \textsc{experiential}\Hack{\break} /\ipa{-dʑɯ˧\textsubscript{a}}/, /\ipa{-ɖo˧\textsubscript{a}}/ ‘must, have to’, \textsc{volitive} /\ipa{-tso˧\textsubscript{a}}/, \textsc{obligative}
		/\ipa{-zo˧\textsubscript{a}}/, and \textsc{causative} /\ipa{-tsæ˧\textsubscript{a}}/}
	\item{M\textsubscript{b} tone: \textsc{perfective} /\ipa{-ze˧\textsubscript{b}}/, and /\ipa{mæ˧\textsubscript{b}}/ ‘to achieve’}
\end{itemize}

\begin{table}%[t]
	\caption{\label{tab:thebehaviouroftonelesssuffixe}The behaviour of postverbal morphemes belonging to a~second {subcategory} of M tones: M\textsubscript{b}.}
	\begin{tabularx}{\textwidth}{ l@{\hspace{12mm}} Q Q Q Q }
		\lsptoprule
		tone & example & meaning & perfective & to achieve\\ \midrule
		H & \ipa{dzɯ˥} & to eat & \ipa{dzɯ˧-ze˩} & \ipa{dzɯ˧-mæ˩}\\
		M\textsubscript{a} & \ipa{hwæ˧\textsubscript{a}} & to buy & \ipa{hwæ˧-ze˧} & \ipa{hwæ˧-mæ˧}\\
		M\textsubscript{b} & \ipa{tɕʰi˧\textsubscript{b}} & to sell & \ipa{tɕʰi˧-ze˧} & \ipa{tɕʰi˧-mæ˧}\\
		M\textsubscript{c} & \ipa{pv̩˧\textsubscript{c}} & to chant & \ipa{pv̩˧-ze˧} & \ipa{pv̩˧-mæ˧}\\
		L\textsubscript{a} & \ipa{bæ˩\textsubscript{a}} & to sweep & \ipa{bæ˩-ze˩} & \ipa{bæ˩-mæ˩}\\
		L\textsubscript{b} & \ipa{ʐwɤ˩\textsubscript{b}} & to speak & \ipa{ʐwɤ˩-ze˩} & \ipa{ʐwɤ˩-mæ˩}\\
		MH & \ipa{lɑ˧˥} & to strike & \ipa{lɑ˧-ze˥} & \ipa{lɑ˧-mæ˥}\\
		\lspbottomrule
	\end{tabularx}
\end{table}

\largerpage[-1] %long distance
In the data shown in \tabref{tab:sametonalcategoryofmorphemesasinprevioustablewithnegation}, the H component found in the lexical categories H and MH
(illustrated by /\ipa{dzɯ˥}/ ‘to eat’ and /\ipa{lɑ˧˥}/ ‘to strike’, respectively) does not \is{form!surface}surface as such, but it results in
the lowering of the perfective morpheme /\ipa{-ze˧\textsubscript{b}}/, no fewer than three syllables distant from the
verb. The phrase /\ipa{dzɯ˧ mɤ˧-bi˧-ze˩}/ ‘will not eat anymore’ constitutes one tone
group, and the underlying presence of a~H tone in this group makes itself felt by lowering the {perfective} morpheme despite the intervening syllables. The same phenomenon is observed when there is no intervening
{negation} \is{prefixes}prefix, as shown in \tabref{tab:withoutintnegation}.

\begin{table}%[t]
	\caption{Same data as in previous table, without intervening
		{negation} prefix.}
	\label{tab:withoutintnegation}
	\begin{tabularx}{\textwidth}{ l@{\hspace{9mm}} l@{\hspace{9mm}} l@{\hspace{9mm}} Q }
		\lsptoprule
		tone & example & meaning & V+\textsc{immediate future}+\textsc{perfective}\\ \midrule
		H & \ipa{dzɯ˥} & to eat & \ipa{dzɯ˧ bi˧-ze˩}\\
		M\textsubscript{a} & \ipa{hwæ˧\textsubscript{a}} & to buy & \ipa{hwæ˧ bi˧-ze˧}\\
		M\textsubscript{b} & \ipa{tɕʰi˧\textsubscript{b}} & to sell & \ipa{tɕʰi˧ bi˧-ze˧}\\
		M\textsubscript{c} & \ipa{pv̩˧\textsubscript{c}} & to chant & \ipa{pv̩˧ bi˧-ze˧}\\
		L\textsubscript{a} & \ipa{bæ˩\textsubscript{a}} & to sweep & \ipa{bæ˩ bi˩-ze˥}\\
		L\textsubscript{b} & \ipa{ʐwɤ˩\textsubscript{b}} & to speak & \ipa{ʐwɤ˩ bi˩-ze˥}\\
		MH & \ipa{lɑ˧˥} & to strike & \ipa{lɑ˧ bi˥-ze˩}\\
		\lspbottomrule
	\end{tabularx}
\end{table}

In these contexts, the {perfective} morpheme //\ipa{-ze˧\textsubscript{b}}// carries different tones depending on the lexical tone of the verb~-- even in the case of H tone, even though this lexical tone surfaces neither on the verb itself nor on the morpheme that follows it (the {immediate future}). For H-tone verbs, the {perfective} morpheme //\ipa{-ze˧\textsubscript{b}}// is lowered to L in each of (\ref{ex:haveeaten}), (\ref{ex:willeat}) and (\ref{ex:willnoteat}).

%
\begin{exe}
	\ex
	\label{ex:haveeaten}
	\ipaex{dzɯ˧-ze˩}\\
	\gll dzɯ˥		-ze˧\textsubscript{b}\\
	to\_eat		\textsc{pfv}\\
	\glt ‘have eaten’
\end{exe}

\begin{exe}
	\ex
	\label{ex:willeat}
	\ipaex{dzɯ˧-bi˧-ze˩}\\
	\gll dzɯ˥		-bi˧\textsubscript{a}		-ze˧\textsubscript{b}\\
	to\_eat		\textsc{imm.fut}		\textsc{pfv}\\
	\glt ‘will eat’
\end{exe}

\begin{exe}
	\ex
	\label{ex:willnoteat}
	\ipaex{dzɯ˧ mɤ˧-bi˧-ze˩}\\
	\gll dzɯ˥		mɤ˧-	-bi˧\textsubscript{a}		-ze˧\textsubscript{b}\\
	to\_eat		\textsc{neg}	\textsc{imm.fut}		\textsc{pfv}\\
	\glt ‘will not eat’
\end{exe}

Tones that are present \is{form!underlying}underlyingly but only manifest themselves in a~restricted set of contexts constitute a~salient aspect of the Yongning Na tone system. The surface phonological notation of ‘going to buy’
as /\ipa{hwæ˧-bi˧}/ and ‘going to eat’ as /\ipa{dzɯ˧-bi˧}/, with the same tone pattern, does not
reveal the underlying presence of a~H tone in the former phrase. The manifestation of the H tone of
the verb is not straightforward: it does not lower a~M-tone postverbal morpheme (witness /\ipa{dzɯ˧-bi˧}/, ‘going
to eat’), but it lowers the last syllable in (\ref{ex:willeat}). 

The lowering effect of overt H tones is an~exceptionless phonological regularity, formulated as Rule~4: “The syllable following a~H-tone syllable receives L tone” (\sectref{sec:alistoftonerules}). In some morphosyntactic contexts, exemplified by \tabref{tab:sametonalcategoryofmorphemesasinprevioustablewithnegation}, H tones exert a~similar influence even though they remain \is{floating tone}floating, i.e.\ not associated to a~syllable. Such complexities cannot be summarized through a~small set of rules. This state of affairs explains the abundance of tables in this chapter, and in this volume as a~whole. 

Further data on M-tone postverbal morphemes is shown in \tabref{tab:anotherpostverbalelementsoverwhichamhtonecannotunfoldthedurative}, illustrating the behaviour of the {progressive} /\ipa{-dʑo˧}/. The affirmative particle /\ipa{mv̩˧}/ has the same behaviour. An interesting peculiarity is that a~MH \is{tonal contour}contour on the verb does not unfold onto these morphemes. A~MH"=tone verb preceding these morphemes is realized with a~MH \is{tonal contour}contour, e.g.~/\ipa{mɤ˧-lɑ˧˥ {\kern2pt}|{\kern2pt} -dʑo˩}/
for ‘to strike’. Following the same guiding principle as before, this difference in tonal behaviour requires setting up a~third descriptive \is{subcategories of lexical tones}subcategory of M tones among postverbal morphemes: M\textsubscript{c}.

\begin{table}%[t]
	\caption{Tonal behaviour of the {progressive}, illustrating a~third {subcategory} of M tones among postverbal morphemes: M\textsubscript{c}.}
	\label{tab:anotherpostverbalelementsoverwhichamhtonecannotunfoldthedurative}
	\begin{tabularx}{\textwidth}{ l@{\hspace{9mm}} l@{\hspace{9mm}} l@{\hspace{9mm}} Q }
		\lsptoprule
		tone & example & meaning & \textsc{dur}+V+\textsc{prog}: ‘is currently V-ing’\\ \midrule
		H & \ipa{dzɯ˥} & to eat & \ipa{tʰi˧-dzɯ˥-dʑo˩}\\
		M\textsubscript{a} & \ipa{hwæ˧\textsubscript{a}} & to buy & \ipa{tʰi˧-hwæ˧-dʑo˧}\\
		M\textsubscript{b} & \ipa{tɕʰi˧\textsubscript{b}} & to sell & \ipa{tʰi˧-tɕʰi˧-dʑo˧}\\
		M\textsubscript{c} & \ipa{pv̩˧\textsubscript{c}} & to chant & \ipa{tʰi˧-pv̩˧-dʑo˧}\\
		L\textsubscript{a} & \ipa{bæ˩\textsubscript{a}} & to sweep & \ipa{tʰi˧-bæ˩-dʑo˩}\\
		L\textsubscript{b} & \ipa{ʐwɤ˩\textsubscript{b}} & to speak & \ipa{tʰi˧-ʐwɤ˩-dʑo˩}\\
		MH & \ipa{lɑ˧˥} & to strike & \ipa{tʰi˧-lɑ˧˥-dʑo˩} ($\ddagger${\kern2pt}\ipa{tʰi˧-lɑ˥-dʑo˩})\\
		\lspbottomrule
	\end{tabularx}
\end{table}


\subsection[H tone]{H-tone postverbal morphemes}
\label{sec:ahtonesuffixtherelativizernominalizer}

Two postverbal morphemes are tentatively placed under the heading of “H-tone postverbal morphemes”: the {relativizer/nominalizer} \mbox{//\ipa{-hĩ˥}//} and the {topic} marker //\ipa{-dʑo˥}//. The argument is as follows.



The {relativizer/nominalizer} \mbox{//\ipa{-hĩ˥}//} behaves in many tonal contexts like the M-tone postverbal morphemes presented in \sectref{sec:mtonesuffixes}, such as the {immediate future} \mbox{/\ipa{-bi˧}/.} But the {relativizer/nominalizer} cannot host the H part of
a~MH \is{tonal contour}contour from the preceding verb: the result is M.M, as exemplified in (\ref{ex:whostrikes}), contrasting with the M.H pattern found for M-tone morphemes, illustrated in (\ref{ex:goingtostrike}). The {relativizer} must therefore be considered to belong to a~tonal category distinct from
those labelled above as L and M. 

\begin{exe}
	\ex
	\label{ex:whostrikes}
	\ipaex{lɑ˧-hĩ˧}\\
	\gll lɑ˧˥		-hĩ˥\\
	to\_strike		\textsc{relativizer/nominalizer}\\
	\glt ‘who strikes’
\end{exe}

\begin{exe}
	\ex
	\label{ex:goingtostrike}
	\ipaex{lɑ˧-bi˥}\\
	\gll lɑ˧˥		-bi˧\\
	to\_strike		\textsc{imm.fut}\\
	\glt ‘is going to strike’
\end{exe}

In the task of assigning labels to tonal categories, I make the (debatable) assumption that the tone systems of different parts of speech all rest on the same primitives (H, M, and L levels) and on combinations among these primitives (MH, LM and LH). An argument for identifying the tonal category of postverbal morphemes exemplified by the {relativizer/nominalizer} as H comes from the hypothesis that this morpheme derives from the noun
for ‘person, human being’, which has H tone: /\ipa{hĩ˥}/.\footnote{About the {grammaticalization} of this noun, see Lidz (\citeyear[164, 183]{lidz2010}).} This argument is not decisive, however: as was pointed out above, in the process of \isi{grammaticalization}, a~morpheme enters a~tonal subsystem that differs from that of its original word class, and can come to carry a~tone that is widely different from that of the original word. Another argument is that, when associated with the {relativizer/nominalizer} \mbox{//\ipa{-hĩ˥}//}, the adjectives in tone category L\textsubscript{b} (such as /\ipa{dʑɤ˩\textsubscript{b}}/
‘good’ and /\ipa{nɑ˩\textsubscript{b}}/ ‘black, dark’) yield a~L.H
pattern, as shown in \tabref{tab:thetonalbehaviourofhtonesuffixes}. But this argument is not decisive either: the rules are morphotonological, not simply phonological, so that a~H tone in the output cannot be taken as conclusive evidence of a~H tone in the input. The existence of a~LH tonal category for disyllabic nouns may function as an attractor, funnelling, as it were, various morphological combinations towards a~LH output. 

Adjectives and verbs behave quite differently in association with the {relativizer/nominalizer} \mbox{//\ipa{-hĩ˥}//}. The tone patterns of verbs and corresponding adjectives are the same only for one tone category out of five (namely L\textsubscript{a}). This comes as a~surprise in view of the fact that, in many other contexts, the tonal behaviour of verbs and adjectives is pretty much the same. This could have to do with semantic"=syntactic differences between the uses of the morpheme \mbox{//\ipa{-hĩ˥}//} with verbs and with adjectives: the morpheme can be categorized as a~{nominalizer} when suffixed to an adjective, and a~{relativizer} when suffixed to a~verb. 

\begin{table}%[t]
	\caption{\label{tab:thetonalbehaviourofhtonesuffixes}The tonal behaviour of the {relativizer{\slash}nominalizer}, analyzed as having a~lexical H tone.}
	\begin{tabularx}{\textwidth}{ l@{\hspace{10mm}} l@{\hspace{10mm}} l@{\hspace{10mm}} l@{\hspace{10mm}} l }
		\lsptoprule
		word class & tone & example & meaning & with \textsc{relativizer}\\ \midrule
		verbs & H & \ipa{dzɯ˥} & to eat & \ipa{dzɯ˧-hĩ˧}\\
		& M\textsubscript{a} & \ipa{hwæ˧\textsubscript{a}} & to buy & \ipa{hwæ˧-hĩ˧}\\
		& M\textsubscript{b} & \ipa{tɕʰi˧\textsubscript{b}} & to sell & \ipa{tɕʰi˧-hĩ˧}\\
		& M\textsubscript{c} & \ipa{bi˧\textsubscript{c}} & to go & \ipa{bi˧-hĩ˧}\\
		& L\textsubscript{a} & \ipa{bæ˩\textsubscript{a}} & to sweep & \ipa{bæ˩-hĩ˩}\\
		& L\textsubscript{b} & \ipa{ʐwɤ˩\textsubscript{b}} & to speak & \ipa{ʐwɤ˩-hĩ˩}\\
		& MH & \ipa{lɑ˧˥} & to strike & \ipa{lɑ˧-hĩ˧}\\ \midrule
		adjectives & H & \ipa{bi˥} & shallow & \ipa{bi˧-hĩ\#˥}\\
		& M & \ipa{tɕi˧} & sour & \ipa{tɕi˧-hĩ\#˥}\\
		& L\textsubscript{a} & \ipa{hṽ̩˩\textsubscript{a}} & red & \ipa{hṽ̩˩-hĩ˩}\\
		& L\textsubscript{b} & \ipa{dʑɤ˩\textsubscript{b}} & good & \ipa{dʑɤ˩-hĩ˥}\\
		& MH & \ipa{tʰɑ˧˥} & sharp & \ipa{tʰɑ˧-hĩ}˥\$\\
		\lspbottomrule
	\end{tabularx}
\end{table}

The topic marker //\ipa{-dʑo˥}// commonly occurs after verb phrases, as well as after noun phrases. On the basis of its behaviour after M-tone verbs (as after M-tone nouns, which were presented in the previous chapter), the topic marker is (provisionally) analyzed as carrying a~lexical H tone. Data on its behaviour in context is presented in \tabref{tab:thetonalbehaviourofthetopicmarkerwithverbs}. As in \tabref{tab:anotherpostverbalelementsoverwhichamhtonecannotunfoldthedurative}, the MH tone does not unfold over the following morpheme: it is not correct to say $\ddagger${\kern2pt}\ipa{mɤ˧-lɑ˧-dʑo˥}. 

Importantly, the tone patterns for the topic marker, shown in Tables~\ref{tab:thetonalbehaviourofthetopicmarkerwithverbs} and \ref{tab:thetonalbehaviourofthetopicmarkerwithadjectives}, are different from those of the {relativizer}, shown in \tabref{tab:thetonalbehaviourofhtonesuffixes}. The tone patterns for the topic marker show stronger surface similarity to those for the {progressive} morpheme, shown in \tabref{tab:anotherpostverbalelementsoverwhichamhtonecannotunfoldthedurative}, which is analyzed as carrying lexical M tone. The only difference between {topic} and {progressive} is that, when these morphemes combine with M-tone verbs, the {topic} marker gets H tone, whereas the {progressive} gets M tone. The logical thing to do would be to recognize two \is{subcategories of lexical tones}subcategories of H-tone postverbal elements (H\textsubscript{a} and H\textsubscript{b}), one exemplified by the {relativizer}, and the other by the {topic} marker. But categories containing only one item are not immensely useful. The categorization of morphemes as belonging to tonal category L, M, H or MH, as proposed in \sectref{sec:ltonesuffixesandserializedverbs}-\sectref{sec:mhtonesuffixes} of the present chapter, is likely to require in"=depth revision in future, in light of a~fuller picture of the tonal behaviour of postverbal elements. It appears too early at present to harden (hypostatize) the freshly established tonal categories of grammatical morphemes into a~seemingly neat and tidy system. Yongning Na is replete with morphotonological anfractuosities; the present chapter only constitutes a~step towards an orderly inventory and analysis, using “L tone”, “M tone”, “H tone” and “MH tone” as convenient first"=pass labels.

\begin{subtables}
%	\label{tab:thetonalbehaviourofthetopicmarker}
	\begin{table}%[t]
		\caption{\label{tab:thetonalbehaviourofthetopicmarkerwithverbs}The tonal behaviour of the topic marker with verbs.}
		\begin{tabularx}{\textwidth}{ l@{\hspace{11mm}} l@{\hspace{11mm}} l@{\hspace{11mm}} l@{\hspace{11mm}} Q }
			\lsptoprule
			tone & example & meaning & V+\textsc{top} & \textsc{neg}+V+\textsc{top}\\ \midrule
			H & \ipa{dzɯ˥} & to eat & \ipa{dzɯ˧-dʑo˩} & \ipa{mɤ˧-dzɯ˥-dʑo˩}\\
			M\textsubscript{a} & \ipa{hwæ˧\textsubscript{a}} & to buy & \ipa{hwæ˧-dʑo˥} & \ipa{mɤ˧-hwæ˧-dʑo˥}\\
			M\textsubscript{b} & \ipa{tɕʰi˧\textsubscript{b}} & to sell & \ipa{tɕʰi˧-dʑo˥} & \ipa{mɤ˧-tɕʰi˧-dʑo˥}\\
			M\textsubscript{c} & \ipa{bi˧\textsubscript{c}} & to go & \ipa{bi˧-dʑo˥} & \ipa{mɤ˧-bi˧-dʑo˥}\\
			L\textsubscript{a} & \ipa{bæ˩\textsubscript{a}} & to sweep & \ipa{bæ˩˥-dʑo˩} & \ipa{mɤ˧-bæ˩-dʑo˩}\\
			L\textsubscript{b} & \ipa{ʐwɤ˩\textsubscript{b}} & to speak & \ipa{ʐwɤ˩˥-dʑo˩} & \ipa{mɤ˧-ʐwɤ˩-dʑo˩}\\
			MH & \ipa{lɑ˧˥} & to strike & \ipa{lɑ˧˥-dʑo˩} & \ipa{mɤ˧-lɑ˧˥-dʑo˩}\\
			\lspbottomrule
		\end{tabularx}
	\end{table}
	
	\begin{table}%[t]
		\caption{\label{tab:thetonalbehaviourofthetopicmarkerwithadjectives}The tonal behaviour of the topic marker with adjectives.}
		\begin{tabularx}{\textwidth}{ l@{\hspace{11mm}} l@{\hspace{11mm}} l@{\hspace{11mm}} l@{\hspace{11mm}} Q }
			\lsptoprule
			tone & example & meaning & \textsc{Adj}+\textsc{top} & \textsc{neg}+\textsc{Adj}+\textsc{top}\\ \midrule
			H & \ipa{bi˥} & shallow & \ipa{bi˧-dʑo˩} & \ipa{mɤ˧-bi˥-dʑo˩}\\
			M & \ipa{tɕi˧} & sour & \ipa{tɕi˧-dʑo˥} & \ipa{mɤ˧-tɕi˧-dʑo˥}\\
			L\textsubscript{a} & \ipa{hṽ̩˩\textsubscript{a}} & red & \ipa{hṽ̩˩˥-dʑo˩} & \ipa{mɤ˧-hṽ̩˩-dʑo˩}\\
			L\textsubscript{b} & \ipa{dʑɤ˩\textsubscript{b}} & good & \ipa{dʑɤ˩˥-dʑo˩} & \ipa{mɤ˧-dʑɤ˩-dʑo˩}\\
			MH & \ipa{tʰɑ˧˥} & sharp & \ipa{tʰɑ˧˥-dʑo˩} & \ipa{mɤ˧-tʰɑ˧˥-dʑo˩}\\
			\lspbottomrule
		\end{tabularx}
	\end{table}
\end{subtables}


\subsection[MH tone]{MH"=tone postverbal morphemes}
\label{sec:mhtonesuffixes}

Following the same exploratory method as outlined above in the discussion of the “L”, “M” and “H” sets of postverbal morphemes, a~“MH” set is postulated, likewise based on fragmentary evidence. The {abilitive} /\ipa{-kv̩˧˥}/ derives from the verb /\ipa{kv̩˧˥}/ ‘to be able to'; this link, together with similarities in tonal behaviour, leads to adoption of the label “MH” for the tone category of the {abilitive}, and of other morphemes that share its tonal behaviour. Examples are provided in \tabref{tab:thetonalbehaviourofmh}: in addition to the {abilitive}, /\ipa{-kv̩˧˥}/, they are the {permissive}, /\ipa{-tʰɑ˧˥}/, and the {causative}, /\ipa{kʰɯ˧˥}/. Other MH"=tone postverbal elements include the reported"=speech particle /\ipa{tsɯ˧˥}/. The MH tone surfaces as such after M, in keeping with the general phonological tendency that the M tone does not interfere with
following tones. The MH tone is lowered to L after H; the interpretation proposed is that the H tone does not surface as such due to the
\isi{neutralization} of M and H in tone"=group"=initial position (this is formulated in \sectref{sec:alistoftonerules} as Rule~3: “In tone"=group"=initial position, H and M are neutralized to M''), but that this H tone is present underlyingly and lowers
all following tones to L (through Rules 4 and 5). A~L tone on the verb spreads over a~MH"=tone postverbal morpheme (e.g.\ //\ipa{bæ˩\textsubscript{a}}// → //\ipa{bæ˩-kv̩˩}// ‘is apt to sweep'), as does
the H part of a~MH \is{tonal contour}contour (//\ipa{lɑ˧˥}// → //\ipa{lɑ˧-kv̩˥}// ‘is apt to strike'), delinking the MH tone on the postverbal morpheme.

\begin{table}%[t]
 \caption{\label{tab:thetonalbehaviourofmh}The tonal behaviour of MH"=tone morphemes after verbs and adjectives.}
{\setlength\tabcolsep{4.5pt}
\begin{tabularx}{\textwidth}{ l l l l l l Q }
\lsptoprule
	word class & tone & example & meaning & \textsc{abilitive} & \textsc{permissive} & \textsc{causative}\\ \midrule
	verbs &	H & \ipa{dzɯ˥} & to eat & \ipa{dzɯ˧-kv̩˩} & \ipa{dzɯ˧-tʰɑ˩} & \ipa{dzɯ˧ kʰɯ˩}\\
	& M\textsubscript{a} & \ipa{hwæ˧\textsubscript{a}} & to buy & \ipa{hwæ˧-kv̩˧˥} & \ipa{hwæ˧-tʰɑ˧˥} & \ipa{hwæ˧ kʰɯ˧˥}\\
	& M\textsubscript{b} & \ipa{tɕʰi˧\textsubscript{b}} & to sell & \ipa{tɕʰi˧-kv̩˧˥} & \ipa{tɕʰi˧-tʰɑ˧˥} & \ipa{tɕʰi˧ kʰɯ˧˥}\\
	& M\textsubscript{c} & \ipa{pv̩˧\textsubscript{c}} & to chant & \ipa{pv̩˧-kv̩˧˥} & \ipa{pv̩˧-tʰɑ˧˥} & \ipa{pv̩˧ kʰɯ˧˥}\\
	& L\textsubscript{a} & \ipa{bæ˩\textsubscript{a}} & to sweep & \ipa{bæ˩-kv̩˩} & \ipa{bæ˩-tʰɑ˩} & \ipa{bæ˩ kʰɯ˩}\\
	& L\textsubscript{b} & \ipa{ʐwɤ˩\textsubscript{b}} & to speak & \ipa{ʐwɤ˩-kv̩˩} & \ipa{ʐwɤ˩-tʰɑ˩} & \ipa{ʐwɤ˩ kʰɯ˩}\\
	& MH & \ipa{lɑ˧˥} & to strike & \ipa{lɑ˧-kv̩˥} & \ipa{lɑ˧-tʰɑ˥} &
   \ipa{lɑ˧ kʰɯ˥}\\ \midrule
	adjectives & H & \ipa{bi˥} & shallow & \ipa{bi˧-kv̩˩} & \ipa{bi˧-tʰɑ˩} & \ipa{bi˧ kʰɯ˩}\\
	& M & \ipa{tsʰi˧} & hot & \ipa{tsʰi˧-kv̩˧˥} & \ipa{tsʰi˧-tʰɑ˧˥} & \ipa{tsʰi˧ kʰɯ˧˥}\\
	& L\textsubscript{a} & \ipa{ɖɯ˩a} & large & \ipa{ɖɯ˩-kv̩˩} & \ipa{ɖɯ˩-tʰɑ˩} & \ipa{ɖɯ˩ kʰɯ˩}\\
	& L\textsubscript{b} & \ipa{dʑɤ˩\textsubscript{b}} & good & \ipa{dʑɤ˩-kv̩˧˥} & \ipa{dʑɤ˩-tʰɑ˥} & \ipa{dʑɤ˩ kʰɯ˥}\\
	& MH & \ipa{tʰɑ˧˥} & sharp & \ipa{tʰɑ˧-kv̩˥} & \ipa{tʰɑ˧-tʰɑ˥} & \ipa{tʰɑ˧ kʰɯ˥}\\
\lspbottomrule
\end{tabularx}}
\end{table}

The phrase /\ipa{dʑɤ˩ kʰɯ˥}/ (‘good’+\textsc{causative}) is in common use as a~blessing on special
occasions such as the New Year and the rite of passage into adulthood. It could be translated as
‘Best wishes!’ or ‘Let there be good/happiness!’ Its L.H tone pattern is also observed on combinations that do not constitute set phrases, showing that it is not an \is{exceptions}exception. The other phrases in \tabref{tab:thetonalbehaviourofmh}, /\ipa{bi˧ kʰɯ˩}/
(‘shallow’+\textsc{causative}), /\ipa{ɖɯ˩ kʰɯ˩˥}/
(‘large’+\textsc{causative}), /\ipa{tsʰi˧ kʰɯ˧˥}/ (‘hot’+\textsc{causative}) and /\ipa{tʰɑ˧ kʰɯ˥}/
(‘sharp’+\textsc{causative}), all have straightforward causative meanings: ‘to make shallow’, e.g.~to
cause the level of water in a~field to become shallow by decreasing the flow of water sent into it; ‘to enlarge’, e.g.~to
increase the size of a~farm by adding another building; ‘to heat up’; and ‘to sharpen’.

\tabref{tab:thetonalbehaviourofmh} shows that not all postverbal elements grouped under the provisional heading “MH” have exactly the same tonal behaviour. A~difference in tone pattern between the {causative} and {abilitive} constructions is found in association with L\textsubscript{b}-tone adjectives. (This difference has been verified through elicitation, on several occasions; examples are found in Caravans.203 and ComingOfAge2.61, 72, 73, 91, 92.) The pattern is L.MH in /\ipa{dʑɤ˩-kv̩˧˥}/ (‘good’+\textsc{abilitive}), and L.H in /\ipa{dʑɤ˩ kʰɯ˥}/ (‘good’+\textsc{causative}). This requires the recognition of at least two \is{subcategories of lexical tones}subcategories of MH"=tone postverbal morphemes. As explained above, it appears too early at present to propose a~final inventory. The labels “L tone”, “M tone”, “H tone” and “MH tone” are used here for a~first"=pass inventory.


\section{Disyllabic postverbal morphemes}
\label{sec:disyllabicsuffixes}


\subsection{M.H tone}
\label{sec:mhtone}

A first tonal category of \is{disyllables}disyllabic postverbal elements is illustrated by /\ipa{-kwɤ˧{\allowbreak}tɕɯ˥}/ ‘after; as;
because’. This \is{postpositions}postposition mostly appears as part of a~{trisyllabic} expression: /\ipa{-kwɤ˧tɕɯ˥-lɑ˩}/. This expression is
transcribed with a~hyphen before the syllable /\ipa{-lɑ˩}/ because this last syllable can be
detached from the other two. In texts, out of 140 examples, ten are without /\ipa{-lɑ˩}/, as in (\ref{ex:lake35354}). (The other examples are Lake3.54,
59, 67, Sister.34, Sister3.133, Caravans.80, 137, Renaming.18, and
BuriedAlive2.48.) 



\begin{exe}
	\ex
	\label{ex:lake35354}
	\ipaex{bo˩-gv̩˥, {\kern2pt}|{\kern2pt} bo˩-hɑ˥ ki˩-hĩ˩=bv̩˩, {\kern2pt}|{\kern2pt} ʈʂʰwæ˩-ne˩˥ {\kern2pt}|{\kern2pt} ɖɯ˧-ɭɯ˧ {\kern2pt}|{\kern2pt} dʑo˧-kwɤ˧tɕɯ˥, {\kern2pt}|{\kern2pt} ʈʂʰɯ˧-qo˧ {\kern2pt}|{\kern2pt} tʰi˧-dzi˩-kwɤ˩tɕɯ˩, {\kern2pt}|{\kern2pt} tɕʰo˩˥ {\kern2pt}|{\kern2pt} ɖɯ˧-nɑ˧ {\kern2pt}|{\kern2pt} tʰi˧-po˧ tsɯ˥ {\kern2pt}|{\kern2pt} mv̩˩.}\\
	\gll bo˩-gv̩˥		bo˩-hɑ\#˥		ki˧\textsubscript{a}	-hĩ˥	=bv̩˧		ʈʂʰwæ˩	-ne		ɖɯ˧-ɭɯ˧		dʑo˧\textsubscript{b}	-kwɤ˧tɕɯ˥		ʈʂʰɯ˧-qo˧	tʰi˧-		dzi˩\textsubscript{a}		-kwɤ˧tɕɯ˥	tɕʰo˩˧		ɖɯ˧-nɑ˧		tʰi˧-	po˧˥	-tsɯ˧˥		mv̩˧\\
	pig\_manger		pig\_feed		to\_give	\textsc{nmlz}	\textsc{poss}		boat~(\textit{loan:~Chinese}~\zh{船})		like	one-\textsc{clf}		\textsc{exist}	as			here		\textsc{dur}	to\_sit		as	ladle	one-\textsc{clf}.tools	\textsc{dur}	to\_bring		\textsc{rep}	\textsc{affirm}\\
	\glt ‘As there was a~pig manger, [you know,] the thing for giving swill, that was like a~boat (=that had the shape of a~boat), as [they] sat [in this manger]{\dots} it is said that they brought a~ladle [with them].’ (Lake3.53"=54)
\end{exe}

Thus, while addition of /\ipa{-lɑ}/ (presumably the morpheme //\ipa{lɑ˧}//, meaning ‘and, also’) is
a~well"=established habit, the expression can be employed without it. No special nuance of
meaning was found, except that the formulation without /\ipa{-lɑ}/ is felt to be more
pithy and economical. Example (\ref{ex:lake35354}) clarifies that the presence or absence of /\ipa{-lɑ˩}/ is not conditioned by the semantic interpretation of the \is{postpositions}postposition as meaning either ‘when; after’ or ‘because, since’: among the two occurrences in (\ref{ex:lake35354}), both without an accompanying /\ipa{-lɑ˩}/, the \is{postpositions}postposition /\ipa{-kwɤ˧tɕɯ˥}/ has a~causal reading (‘since, because’) at first occurrence, and a~temporal reading (‘as, when’) at second occurrence. 

The {monosyllabic} form $\ddagger${\kern2pt}\ipa{-kwɤ˧} is not attested in this dialect, even though it is reported in another hamlet of the Yongning plain, Walabie (\ipa{ʁwɤ˧lɑ˩-bi˩}; Chinese: \zh{瓦拉片}) (Roselle Dobbs, p.c.\ 2016).

\begin{table}%[t]
\caption{\label{tab:thetonalbehaviourofafterbecause}The tonal behaviour of /\ipa{-kwɤ˧tɕɯ˥}(\ipa{-lɑ˩})/‚ ‘after; because’.}
\begin{tabularx}{\textwidth}{ Q l@{\hspace{8mm}} Q Q P{40mm} }
\lsptoprule
	word class & tone & example & meaning & V+‘after; because’\\ \midrule
	verbs & H & \ipa{dzɯ˥} & to eat & \ipa{dzɯ˧-kwɤ˩tɕɯ˩(-lɑ˩)}\\
	& M\textsubscript{a} & \ipa{hwæ˧\textsubscript{a}} & to buy & \ipa{hwæ˧-kwɤ˧tɕɯ˥(-lɑ˩)}\\
	& M\textsubscript{b} & \ipa{tɕʰi˧\textsubscript{b}} & to sell & \ipa{tɕʰi˧-kwɤ˧tɕɯ˥(-lɑ˩)}\\
	& M\textsubscript{c} & \ipa{bi˧\textsubscript{c}} & to go & \ipa{bi˧-kwɤ˧tɕɯ˥(-lɑ˩)}\\
	& L\textsubscript{a} & \ipa{bæ˩\textsubscript{a}} & to sweep & \ipa{bæ˩-kwɤ˩tɕɯ˥(-lɑ˩)}\\
	& L\textsubscript{b} & \ipa{ʐwɤ˩\textsubscript{b}} & to speak & \ipa{ʐwɤ˩-kwɤ˩tɕɯ˥(-lɑ˩)}\\
	& MH & \ipa{lɑ˧˥} & to strike & \ipa{lɑ˧˥-kwɤ˩tɕɯ˩-(lɑ˩) /
     lɑ˧-kwɤ˥tɕɯ˩-(lɑ˩)}\\ \midrule
	adjectives & H & \ipa{bi˥} & shallow & \ipa{bi˧-kwɤ˩tɕɯ˩(-lɑ˩)}\\
	& M & \ipa{tɕi˧} & sour & \ipa{tɕi˧-kwɤ˧tɕɯ˥(-lɑ˩)}\\
	& L\textsubscript{a} & \ipa{hṽ̩˩\textsubscript{a}} & red & \ipa{hṽ̩˩-kwɤ˩tɕɯ˥(-lɑ˩)}\\
	& L\textsubscript{b} & \ipa{dʑɤ˩\textsubscript{b}} & good & \ipa{dʑɤ˩˥-kwɤ˧tɕɯ˥(-lɑ˩)}\\
	& MH & \ipa{tʰɑ˧˥} & sharp & \ipa{tʰɑ˧˥-kwɤ˩tɕɯ˩(-lɑ˩)}\\
\lspbottomrule
\end{tabularx}
\end{table}

The lexical tone of /\ipa{-kwɤ˧tɕɯ˥}/ is deduced from its behaviour in association with the adjective
/\ipa{dʑɤ˩\textsubscript{b}}/ ‘good’. The observed pattern is /\ipa{dʑɤ˩˥-kwɤ˧tɕɯ˥-lɑ˩}/, which does not constitute
a~well"=formed \isi{tone group} (since it contains two H tones) and must therefore be analyzed as
a~sequence of two tone groups: /\ipa{dʑɤ˩˥ {\kern2pt}|{\kern2pt} -kwɤ˧tɕɯ˥-lɑ˩}/. In this context, the
tones carried by /\ipa{-kwɤ˧tɕɯ˥}/, namely /M.H/, must be supposed to reflect the underlying tone of the expression: a~final H tone, i.e.\ \mbox{//H\#//}. As for the added morpheme, interpreted as //\ipa{lɑ˧}//,  meaning ‘and, also’, it receives L tone through Rule~4 (“A syllable following a~H-tone syllable receives L tone”).

The sequence /\ipa{dʑɤ˩˥-kwɤ˧tɕɯ˥-lɑ˩}/ ‘because/since [it is] good’ illustrates the
existence of cases in which a~rising \is{tonal contour}contour is realized on the verb, and does not unfold over the
postverbal expression. A~second case in point is with MH"=tone verbs and adjectives, e.g.~/\ipa{lɑ˧˥-kwɤ˩tɕɯ˩(-lɑ˩)}/
‘because/since [someone] beat [something]’. Variants in which the MH \is{tonal contour}contour unfolds over the first
syllable of the following morpheme (/\ipa{lɑ˧-kwɤ˥tɕɯ˩(-lɑ˩)}/) are considered acceptable, and
there is one example in a~text (BuriedAlive2.48), but the majority case is with the \is{tonal contour}contour
sitting on the verb. 

Since contours are only realized tone"=group"=finally, this suggests that there is a~tone"=group \is{boundary (between tone groups)}boundary before the \is{postpositions}postposition //\ipa{-kwɤ˧tɕɯ˥}// ‘as, because’. Such behaviour is not unparalleled in this dialect. For instance, the contrastive topic marker //\ipa{-no˧˥}// and the
word /\ipa{tʰi˩˥}/ ‘then’ always mark the beginning of a~new \isi{tone group}. This interpretation would be
consistent with the fact that the syllable preceding //\ipa{-kwɤ˧tɕɯ˥}// ‘as, because’ tends to be \is{lengthening}lengthened~-- a~cue to the presence of an intonational {boundary}.\footnote{On articulatory, acoustic and perceptual cues to intonational boundaries, see \citet{byrdElastic2003}.} In early transcriptions, a~comma was used
to reflect a~perceived pause, transcribing e.g.~/\ipa{le˧-tsɑ˧˥, {\kern2pt}|{\kern2pt} -kwɤ˩tɕɯ˩}/ ‘because [they] rowed’ (Lake3.59).

But a~difficulty with this analysis is that after some of the tonal categories of verbs, the tone patterns that are observed on //\ipa{-kwɤ˧tɕɯ˥}// ‘because’ suggest that
the verb and the postverbal element are part of the same \isi{tone group}. Spreading of a~L tone from the verb onto the following syllable demonstrates that the verb and its postverbal element are integrated into the same \isi{tone group}. 

One possible way of handling this would be to postulate that the division into tone groups is
determined by the lexical tone of the words at issue: there would be one single \isi{tone group} (e.g.~/{\kern2pt}|{\kern2pt}~\ipa{hwæ˧-kwɤ˧tɕɯ˥-lɑ˩}~{\kern2pt}|{\kern2pt}/ ‘when [she/he] buys’) except with a~verb carrying MH tone (or an~adjective carrying MH or L\textsubscript{b}): \mbox{/\ipa{{\kern2pt}|{\kern2pt} lɑ˧˥}} \ipa{{\kern2pt}|{\kern2pt} -kwɤ˩tɕɯ˩-lɑ˩ {\kern2pt}|{\kern2pt}}/, ‘when [she/he] strikes’. But there are more difficulties here. The first is that some lexical tones leave both solutions open. The MH tone allows two variants, the one with the two morphemes in different tone groups (no tonal interaction: /\ipa{lɑ˧˥{\kern2pt}|{\kern2pt} -kwɤ˧tɕɯ˥}/) and the other with integration into the same group (the H part of the MH \is{tonal contour}contour associates to the following syllable, and triggers a~lowering of the following tone to L, hence /\ipa{lɑ˧-kwɤ˥tɕɯ˩}/). As for the case of verbs carrying lexical M tone, it is simply impossible to determine the underlying division of the expression into tone groups on the basis of the surface tones, because the pattern is /M.M.H/ in both cases, and could be interpreted as made up of either one or two tone groups: \ipa{{\kern2pt}|{\kern2pt}}M.M.H\ipa{{\kern2pt}|{\kern2pt}} or \ipa{{\kern2pt}|{\kern2pt}} M \ipa{{\kern2pt}|{\kern2pt}} M.H \ipa{{\kern2pt}|}, e.g.~\ipa{{\kern2pt}|{\kern2pt} hwæ˧-kwɤ˧tɕɯ˥-lɑ˩ {\kern2pt}|{\kern2pt}} or \ipa{{\kern2pt}|{\kern2pt} hwæ˧ {\kern2pt}|{\kern2pt} -kwɤ˧tɕɯ˥-lɑ˩ {\kern2pt}|{\kern2pt}} for ‘because (she/he) buys’. Since the M tone does not exert an~influence on the following tone, the surface phonological output is the same in both cases. 

Another difficulty is that the situation of the MH tone (for verbs and adjectives) and that of the L\textsubscript{b} tone (for adjectives) are different. The tones
of the postverbal element are all lowered to L when following a~MH"=tone verb, whereas they surface unscathed after the rising \is{tonal contour}contour on a~L\textsubscript{b}-tone adjective. Lowering to L after
a~preceding H tone level is a~{phonological rule} in the Alawa dialect of Yongning Na (Rule~4: “The syllable following a~H-tone syllable receives L tone”; see \sectref{sec:alistoftonerules}); the tone rules operate within the \isi{tone group}, never across
a~tone"=group \is{juncture (inside a tone group)}juncture. The fact that /\ipa{kwɤ.tɕɯ-lɑ}/ is lowered to L
after a~MH"=tone verb (e.g.~/\ipa{lɑ˧˥-kwɤ˩tɕɯ˩-lɑ˩}/ for ‘to strike’) strongly suggests that the
expression /\ipa{kwɤ.tɕɯ-lɑ}/ does not make up an~independent \isi{tone group}. The sequence
/\ipa{-kwɤ˩tɕɯ˩-lɑ˩}/ in /\ipa{lɑ˧˥-kwɤ˩tɕɯ˩-lɑ˩}/ ‘when (she/he) strikes’ is not
a~well"=formed \isi{tone group}, since it only contains L tones.

This special situation is analyzed as the result of a~\is{stylistics}stylistic process of emphasis, which became habitually associated with certain morphemes. The analysis is set out as part of the discussion of the \isi{tone group} as a~key phonological unit in Yongning Na (\sectref{sec:thedivisionofutterancesintotonegroups}).


\subsection{M.L tone}
\label{sec:mltone}

Two categories of disyllabic postverbal elements have a~tantalizingly similar behaviour. They are illustrated
by the \is{postpositions}postposition /\ipa{-ʁo˧to˩}/ ‘on top of; during’ and the adverb /\ipa{-pʰæ˧di˩}/ ‘as if/it seems that’, as shown in
\tabref{tab:thebehaviouroftwodisyllabicsuffixesanalyzedascarryingmltone}. Some examples from texts are shown below, illustrating cases where /\ipa{-pʰæ˧di˩}/ is preceded by a verb with H tone (\ref{ex:shortage68}), L\textsubscript{a} tone (\ref{ex:tiger14}) and L\textsubscript{b} tone (\ref{ex:buriedalive24}), as well as by a~postverbal element carrying MH tone (\ref{ex:dog108}).

 \begin{exe}
 	\ex
 	\label{ex:shortage68}
 	\ipaex{tʰi˩˥ {\kern2pt}|{\kern2pt} hĩ˧=ɻæ˥-dʑo˩ {\kern2pt}|{\kern2pt} wɤ˩˥ {\kern2pt}|{\kern2pt} ʝi˧kʰv̩˥-dʑo˩ | mv̩˧-pʰæ˧di˥!}\\
 	\gll tʰi˩˥		hĩ˥	=ɻæ˩		-dʑo˥			wɤ˩˥		ʝi˧kʰv̩˥	-dʑo˥				mv̩˥					-pʰæ˧di˩\\
 	then	people		\textsc{pl}	\textsc{top}	again	some		\textsc{top}	to\_understand		it\_seems\_that\\
 	\glt ‘It seems that some people understood in the end!' (FoodShortage2.72)
 \end{exe}
 
\begin{exe}
  	\ex
  	\label{ex:tiger14}
  	\ipaex{mv̩˩ ʈʂʰɯ˩-v̩˩˥ {\kern2pt}|{\kern2pt} qʰwɤ˩-pʰæ˩di˥-dʑo˩ {\kern2pt}|{\kern2pt} tʰi˩˥ {\kern2pt}|{\kern2pt} kʰi˧ {\kern2pt}|{\kern2pt} tʰi˧-tv̩˧ tsɯ˥ {\kern2pt}|{\kern2pt} mv̩˩.}\\
  	\gll mv̩˩˥		ʈʂʰɯ˥				v̩˧								qʰwɤ˩\textsubscript{a}	-pʰæ˧di˩		-dʑo˥				tʰi˩˥	kʰi˥		tʰi˧-					tv̩˧˥		tsɯ˧˥			mv̩˧\\
  	daughter	\textsc{dem}		\textsc{clf}.individual		clever	it\_seems\_that		\textsc{top}	then	door	\textsc{dur}		to\_support	\textsc{rep}	\textsc{affirm}\\
  	\glt ‘That girl was clever, as it turned out: she propped herself against the door. / That girl reacted smartly: she immediately propped herself against the door [so as to keep the tiger out].' (Tiger.14)
\end{exe}
  
\begin{exe}
   	\ex
   	\label{ex:buriedalive24}
   	\ipaex{le˧-ʂɯ˧ le˧-nv̩˥-dʑo˩ {\kern2pt}|{\kern2pt} tʰi˩˥ {\kern2pt}|{\kern2pt} hĩ˧ {\kern2pt}|{\kern2pt} ɬo˧tɑ˧ {\kern2pt}|{\kern2pt} wɤ˩˥ {\kern2pt}|{\kern2pt} ɖɯ˧-v̩˧ ɳɯ˧ {\kern2pt}|{\kern2pt} do˩-pʰæ˩di˥ tsɯ˩ {\kern2pt}|{\kern2pt} mv̩˩!}\\
	 \gll le˧-		ʂɯ˧\textsubscript{a}	le˧-	nv̩˥		-dʑo˥		tʰi˩˥		hĩ˥		ɬo˧tɑ˧		wɤ˩˥		ɖɯ˧	v̩˧		ɳɯ˧	do˩\textsubscript{b}	-pʰæ˧di˩		 tsɯ˧˥		-mv̩˧\\
   	\textsc{accomp}	to\_die		\textsc{accomp}	to\_bury	\textsc{top}	then	person	to\_the\_side	again	one		\textsc{clf}.individual	\textsc{a}	to\_see			it\_seems\_that			\textsc{rep}	\textsc{affirm}\\
   	\glt ‘When she died and was buried{\dots} apparently, someone close by had seen [what had really happened to her]!' (BuriedAlive2.24)
\end{exe}
   
\begin{exe}
    	\ex
    	\label{ex:dog108}
    	\ipaex{kʰv̩˩mi˩ lɑ˥ {\kern2pt}|{\kern2pt} ɖʐv̩˧nɑ˥mi˩ ʈʂʰɯ˩-dʑo˩ {\kern2pt}|{\kern2pt} ɖɯ˧-pi˧˥ {\kern2pt}|{\kern2pt} tʰi˧-kv̩˧-pʰæ˥di˩ mæ˩!}\\
    	\gll kʰv̩˩mi˩	lɑ˧	ɖʐv̩˧nɑ˥mi˩		ʈʂʰɯ˧		-dʑo˥		ɖɯ˧-pi˧˥	tʰi˧				-kv̩˧˥		-pʰæ˧di˩			mæ˧\\
    	dog			also	heron	\textsc{top}	\textsc{top}	a\_little		skilful		\textsc{abilitive}		it\_seems\_that	\textsc{obviousness}\\
    	\glt ‘The Dog and the Heron were rather talented, it seems!' (Dog2.108)
\end{exe}

{\setlength\tabcolsep{4.5pt}
\begin{table}%[t]
\caption{\label{tab:thebehaviouroftwodisyllabicsuffixesanalyzedascarryingmltone}The behaviour of two disyllabic suffixes analyzed as carrying M.L tone.}
\begin{tabularx}{\textwidth}{ l l l l Q }
\lsptoprule
	tone & example & meaning & V+\ipa{/-pʰæ˧di˩/} ‘as if’ & V+\ipa{/-ʁo˧to˩/} ‘during’\\ \midrule
	H & \ipa{dzɯ˥} & to eat & \ipa{dzɯ˧-pʰæ˧di˥} & \ipa{dzɯ˧-ʁo˧to˩}\\
	M\textsubscript{a} & \ipa{hwæ˧\textsubscript{a}} & to buy & \ipa{hwæ˧-pʰæ˧di˩} & \ipa{hwæ˧-ʁo˧to˩}\\
	M\textsubscript{b} & \ipa{tɕʰi˧\textsubscript{b}} & to sell & \ipa{tɕʰi˧-pʰæ˧di˩} & \ipa{tɕʰi˧-ʁo˧to˩}\\
	M\textsubscript{c} & \ipa{pv̩˧\textsubscript{c}} & to chant & \ipa{pv̩˧-pʰæ˧di˩} & \ipa{pv̩˧-ʁo˧to˩}\\
	L\textsubscript{a} & \ipa{bæ˩\textsubscript{a}} & to sweep & \ipa{bæ˩-pʰæ˩di˥} & \ipa{bæ˩-ʁo˩to˥}\\
	L\textsubscript{b} & \ipa{ʐwɤ˩\textsubscript{b}} & to speak & \ipa{ʐwɤ˩-pʰæ˩di˥ } & \ipa{ʐwɤ˩-ʁo˩to˥}\\
	MH & \ipa{lɑ˧˥} & to strike & \ipa{lɑ˧-pʰæ˥di˩ / lɑ˧-pʰæ˧di˥} & \ipa{lɑ˧-ʁo˥to˩}\\
\lspbottomrule
\end{tabularx}
\end{table}}


The only difference between the two expressions is found in association with H-tone verbs. This
difference is enough to require recognition of the tones of the postpositions \ipa{/-pʰæ˧di˩/} ‘as if’ and \ipa{/-ʁo˧to˩/} ‘during’ as belonging to two different morphotonological categories. The difference may be one between postpositions and adverbs; it could reflect the internal structure of the disyllabic expressions; or, in the absence of any such synchronic conditioning, it may have to be recognized as a~difference between tonal categories, like the various other distinctions transcribed in this volume by means of subscript letters added to the tone (e.g.~in the M\textsubscript{a}, M\textsubscript{b} and M\textsubscript{c} \is{subcategories of lexical tones}subcategories of M-tone verbs). This is one of many issues requiring further examination.


\section{Combinations of postverbal morphemes}
\label{sec:combinationsbetweenaffixes}

\subsection{Postverbal morphemes preceded by the {negation} prefix}
\label{sec:suffixesprecededbythenegation}

It is common for a~verb to be separated from a~following morpheme by the {negation} \is{prefixes}prefix, as
in example (\ref{ex:wasnotabletotakeoff}).
\begin{exe}
  \ex
  \label{ex:wasnotabletotakeoff}
  \ipaex{{\dots} pʰv̩˧ mɤ˥-tʰɑ˩-ɲi˩ ho˩ mæ˩!}\\
  \gll pʰv̩˧˥	mɤ˧	tʰɑ˧˥	-ɲi˩	ho˩	mæ˧\\
  to\_take\_off	\textsc{neg}	\textsc{permissive}	\textsc{certitude}	\textsc{desiderative}
  \textsc{obviousness}\\
  \glt ‘[I] was not able to take off [the bracelets]!' (BuriedAlive2.88)
\end{exe}

One could expect the tonal behaviour of such expressions to be computed progressively
(“left"=to"=right”) in some simple way. But as is often the case in Yongning Na morphotonology, no simple algorithm can be
proposed that would account for all the data. A~telling example is that of MH"=tone verbs, such as
/\ipa{pʰv̩˧˥}/ ‘to take off’. In example (\ref{ex:wasnotabletotakeoff}), the H part of the MH \is{tonal contour}contour reassociates to the
{negation} \is{prefixes}prefix, yielding /\ipa{pʰv̩˧-mɤ˥}{\dots}/ and resulting in the lowering of all following tones to
L (through Rules 4 and 5). When the postverbal morpheme carries M tone, on the other hand, the H part of the MH
\is{tonal contour}contour is not present in the surface phonological form, e.g.~in /\ipa{pʰv̩˧ mɤ˧-bi˧}/ ‘will
not take off’. Unfolding of the MH \is{tonal contour}contour does not take place: $\ddagger${\kern2pt}\ipa{pʰv̩˧-mɤ˥-bi˩} is not an acceptable \is{variants}variant. The data is therefore set out here in static tabular form, rather than as a~set of rules.



\subsubsection{M-tone postverbal morphemes preceded by the {negation}}

\tabref{tab:thetonepatternsofcombinationsbetweenverbnegationandmtoneortonelesssuffixes} shows combinations of a~verb, a {negation} \is{prefixes}prefix, and a~morpheme that carries M\textsubscript{a} or M\textsubscript{b} tone. The example morphemes are the {immediate future}, /\ipa{bi˧\textsubscript{a}}/, and /\ipa{mæ˧\textsubscript{b}}/ ‘to succeed, to
achieve’. Note that the verb used to illustrate the M\textsubscript{c} tone category in \tabref{tab:thetonepatternsofcombinationsbetweenverbnegationandmtoneortonelesssuffixes} is /\ipa{ʝi˧\textsubscript{c}}/ ‘to come' rather than
/\ipa{bi˧\textsubscript{c}}/ ‘to go', to avoid the form /\ipa{bi˧ mɤ˧-bi˧}/ ‘am not going to go'. This expression is well"=formed, but rather confusing because \isi{homophony} of the verb and the postverbal morpheme makes the entire expression \is{homophony}homophonous with a~V \textsc{neg}-V construction, meaning ‘whether [she/he{\dots}] goes or not'.

{\setlength\tabcolsep{4pt}
	\begin{table}%[t]
	\caption{\label{tab:thetonepatternsofcombinationsbetweenverbnegationandmtoneortonelesssuffixes}Combinations of verb, {negation} prefix and M-tone morpheme.}
	\begin{tabularx}{\textwidth}{ l@{\hspace{6mm}} l@{\hspace{6mm}} l@{\hspace{6mm}} l@{\hspace{6mm}} Q }
	\lsptoprule
		tone & example & meaning & ‘am not going to V’  & ‘cannot manage to V’\\
		 &  &  & (tone: M\textsubscript{a}) & (tone: M\textsubscript{b})\\ \midrule
		H & \ipa{dzɯ˥} & to eat & \ipa{dzɯ˧ mɤ˧-bi˧} & \ipa{dzɯ˧ mɤ˧-mæ˩}\\
		M\textsubscript{a} & \ipa{hwæ˧\textsubscript{a}} & to buy & \ipa{hwæ˧ mɤ˧-bi˧} & \ipa{hwæ˧ mɤ˧-mæ˧}\\
		M\textsubscript{b} & \ipa{tɕʰi˧\textsubscript{b}} & to sell & \ipa{tɕʰi˧ mɤ˧-bi˧} & \ipa{tɕʰi˧ mɤ˧-mæ˧}\\
		M\textsubscript{c} & \ipa{ʝi˧\textsubscript{c}} & to come & \ipa{ʝi˧ mɤ˧-bi˧} & \ipa{ʝi˧ mɤ˧-mæ˧}\\
		L\textsubscript{a} & \ipa{dze˩\textsubscript{a}} & to cut & \ipa{dze˩ mɤ˩-bi˩˥} & \ipa{dze˩ mɤ˩-mæ˥}\\
		L\textsubscript{b} & \ipa{ʈʰɯ˩\textsubscript{b}} & to drink & \ipa{ʈʰɯ˩ mɤ˩-bi˩˥} & \ipa{ʈʰɯ˩ mɤ˩-mæ˥}\\
		MH & \ipa{ʈʰæ˧˥} & to bite & \ipa{ʈʰæ˧ mɤ˧-bi˧} & \ipa{ʈʰæ˧ mɤ˥-mæ˩}\\
	\lspbottomrule
	\end{tabularx}
	\end{table}
}


\subsubsection{MH"=tone postverbal morphemes preceded by \textsc{neg} or \textsc{prohib}}

This paragraph presents three constructions that share the same tone patterns: /\ipa{V mɤ˧-tʰɑ˧˥}/, V-\textsc{neg}-\textsc{permissive}:  ‘[one] must not V’; /\ipa{V mɤ˧-kʰɯ˧˥}/, V-\textsc{neg}-\textsc{caus}: ‘not to let [someone] V’; and /\ipa{V tʰɑ˧-kʰɯ˧˥}/, V-\textsc{prohib}-\textsc{caus}: ‘do not cause to V’, ‘do not let [someone] V’. Examples from texts include (\ref{ex:whensomeonediesdonotclothe})-(\ref{ex:difflength}).

%Used to be a list. But it took up a lot of space on the page. In case this is restored: to make lines closer to one another: \vspace{-3mm} 
%\begin{itemize}
%	\item{/\ipa{V mɤ˧-tʰɑ˧˥}/, V-\textsc{neg}-\textsc{permissive}:  ‘[one] must not V’}
%	\item{/\ipa{V mɤ-kʰɯ˧˥}/, V-\textsc{neg}-\textsc{causative}: ‘not to let [someone] V’}
%	\item{/\ipa{V tʰɑ˧-kʰɯ˧˥}/, V-\textsc{prohibitive}-\textsc{causative}: ‘do not cause to V’, ‘do not let [someone] V’}
%\end{itemize}

\begin{exe}
  \ex
  \label{ex:whensomeonediesdonotclothe}
  \ipaex{le˧-ʂɯ˧-dʑo˧, {\kern2pt}|{\kern2pt} dʑi˧hṽ̩˥ {\kern2pt}|{\kern2pt} mv̩˧ mɤ˧-kʰɯ˧˥! {\kern2pt}|{\kern2pt}}\\
  \gll le˧-	ʂɯ˧	-dʑo˧	dʑi˧hṽ̩˥\$	mv̩˧		mɤ˧-	-kʰɯ˧˥\\
  \textsc{accomp}	to\_die	\textsc{prog}	clothes	to\_put\_on	\textsc{neg}	\textsc{caus}\\
  \glt ‘When someone dies, [we] do not clothe [the corpse]!’ (BuriedAlive3.58)

  \ex
  \label{ex:burncorpse}
  \ipaex{so˩ɲi˩-so˩hɑ̃˥ {\kern2pt}|{\kern2pt} qæ˧˥ {\kern2pt}|{\kern2pt} -dʑo˩ {\kern2pt}|{\kern2pt} tʰi˩˥, {\kern2pt}|{\kern2pt} le˧-qæ˧˥, {\kern2pt}|{\kern2pt} mv̩˩-mɤ˩-tʰɑ˥!}\\
  \gll so˩ɲi˩-so˩hɑ̃˥	qæ˧˥	-dʑo˥	tʰi˩˥	le˧-		qæ˧˥	 mv̩˩			mɤ˧-	-tʰɑ˧˥\\
  3\_days\_and\_nights	to\_burn	\textsc{top}	then	\textsc{accomp}	to\_burn to\_consume/to\_burn\_up
  \textsc{neg}	possible\\
  \glt ‘The corpse was burnt [on the pyre] for three days and three nights, but it was not possible
  to burn it up!’ (Sister3.93)

  \ex
  \label{ex:difflength}
  \ipaex{ʂæ˧ɖæ˧ {\kern2pt}|{\kern2pt} di˩-tʰɑ˩-kʰɯ˥!}\\
  \gll ʂæ˧ɖæ˧ di˩\textsubscript{a}	-tʰɑ˧˥	-kʰɯ˧˥\\
  differences\_in\_length		\textsc{exist}	\textsc{prohib}	\textsc{caus}\\
  \glt ‘[The tree trunks] must not be different lengths! / There must not be differences in length!’ (Housebuilding.19. Context: selecting trees that will be felled as lumber for building a~house.)

\end{exe}

The data for all tone categories of nouns is shown in Tables \ref{tab:mustnotvdonotlet} and  \ref{tab:cannotv}, bringing out the full identity among the tone patterns for the three constructions.

\begin{table}%[t]
\caption{\label{tab:mustnotvdonotlet}The tone patterns of /\ipa{V mɤ˧-tʰɑ˧˥}/ ‘[one] must not V’ and /\ipa{V tʰɑ˧-kʰɯ˧˥}/ ‘do not let [someone] V/do not cause to V’.}
\begin{tabularx}{\textwidth}{ l@{\hspace{7mm}} l@{\hspace{7mm}} l@{\hspace{7mm}} l@{\hspace{7mm}} Q }
\lsptoprule
	tone & example & meaning & ‘[one] must not V’ & ‘do not cause to V’\\ \midrule
	H & \ipa{dzɯ˥} & to eat & \ipa{dzɯ˧ mɤ˧-tʰɑ˩} & \ipa{dzɯ˧ tʰɑ˧-kʰɯ˩}\\
	M\textsubscript{a} & \ipa{hwæ˧\textsubscript{a}} & to buy & \ipa{hwæ˧ mɤ˧-tʰɑ˧˥} & \ipa{hwæ˧ tʰɑ˧-kʰɯ˧˥}\\
	M\textsubscript{b} & \ipa{tɕʰi˧\textsubscript{b}} & to sell & \ipa{tɕʰi˧ mɤ˧-tʰɑ˧˥} & \ipa{tɕʰi˧ tʰɑ˧-kʰɯ˧˥}\\
	M\textsubscript{c} & \ipa{bi˧\textsubscript{c}} & to go & \ipa{bi˧ mɤ˧-tʰɑ˧˥} & \ipa{bi˧ tʰɑ˧-kʰɯ˧˥}\\
	L\textsubscript{a} & \ipa{dze˩\textsubscript{a}} & to cut & \ipa{dze˩ mɤ˩-tʰɑ˥} & \ipa{dze˩ tʰɑ˩-kʰɯ˥}\\
	L\textsubscript{b} & \ipa{ʈʰɯ˩\textsubscript{b}} & to drink & \ipa{ʈʰɯ˩ mɤ˩-tʰɑ˥} & \ipa{ʈʰɯ˩ tʰɑ˩-kʰɯ˥}\\
	MH & \ipa{ʈʰæ˧˥} & to bite & \ipa{ʈʰæ˧ mɤ˥-tʰɑ˩} & \ipa{ʈʰæ˧ tʰɑ˥-kʰɯ˩}\\
\lspbottomrule
\end{tabularx}
\end{table}



The tones for \textsc{V/\textsc{Adj}"=neg"=abilitive}, /\ipa{V mɤ˧-kv̩˧˥}/ ‘[one] cannot V’, are entirely identical with those for
/\ipa{V mɤ˧-tʰɑ˧˥}/ ‘[one] must not V’ and /\ipa{V tʰɑ˧-kʰɯ˧˥}/ ‘do not cause to V’, as shown in
\tabref{tab:cannotv}, which also includes adjectives.

\begin{table}%[t]
\caption{\label{tab:cannotv}The tone patterns of V/\textsc{Adj}+\textsc{neg"=abilitive}.}
\begin{tabularx}{\textwidth}{ l@{\hspace{9mm}} l@{\hspace{9mm}} l@{\hspace{9mm}} l@{\hspace{9mm}} Q }
\lsptoprule
	 word class & tone & example & meaning & V/\textsc{Adj}+\textsc{neg"=abilitive}\\ \midrule
	verbs & H & \ipa{dzɯ˥} & to eat & \ipa{dzɯ˧ mɤ˧-kv̩˩}\\
	 & M\textsubscript{a} & \ipa{hwæ˧\textsubscript{a}} & to buy & \ipa{hwæ˧ mɤ˧-kv̩˧˥}\\
	 & M\textsubscript{b} & \ipa{tɕʰi˧\textsubscript{b}} & to sell & \ipa{tɕʰi˧ mɤ˧-kv̩˧˥}\\
	 & M\textsubscript{c} & \ipa{bi˧\textsubscript{c}} & to go & \ipa{bi˧ mɤ˧-kv̩˧˥}\\
	 & L\textsubscript{a} & \ipa{bæ˩\textsubscript{a}} & to sweep & \ipa{bæ˩ mɤ˩-kv̩˥}\\
	 & L\textsubscript{b} & \ipa{ʐwɤ˩\textsubscript{b}} & to speak & \ipa{ʐwɤ˩ mɤ˩-kv̩˥}\\
	 & MH & \ipa{lɑ˧˥} & to strike & \ipa{lɑ˧ mɤ˥-kv̩˩}\\ \midrule
	adjectives & H & \ipa{bi˥} & shallow & \ipa{bi˧ mɤ˧-kv̩˩}\\
	 & M & \ipa{tsʰi˧} & hot & \ipa{tsʰi˧ mɤ˧-kv̩˧˥}\\
	 & L\textsubscript{a} & \ipa{ɖɯ˩\textsubscript{a}} & large & \ipa{ɖɯ˩ mɤ˩-kv̩˥}\\
	 & L\textsubscript{b} & \ipa{dʑɤ˩\textsubscript{b}} & good & \ipa{dʑɤ˩ mɤ˧-kv̩˧˥}\\
	 & MH & \ipa{tʰɑ˧˥} & sharp & \ipa{tʰɑ˧ mɤ˥-kv̩˩}\\
\lspbottomrule
\end{tabularx}
\end{table}


\subsubsection{L-tone postverbal morphemes preceded by {negation} prefix}

\begin{table}%[t]
\caption{\label{tab:willnotv}The tone patterns of the construction /V \ipa{mɤ˧-ho˩}/ ‘will not V’.}
\begin{tabularx}{.75\textwidth}{ l@{\hspace{9mm}} l@{\hspace{9mm}} l@{\hspace{9mm}} Q }
\lsptoprule
	tone & example & meaning & ‘will not V’\\ \midrule
	H & \ipa{dzɯ˥} & to eat & \ipa{dzɯ˧ mɤ˧-ho˥}\\
	M\textsubscript{a} & \ipa{hwæ˧\textsubscript{a}} & to buy & \ipa{hwæ˧ mɤ˧-ho˩}\\
	M\textsubscript{b} & \ipa{tɕʰi˧\textsubscript{b}} & to sell & \ipa{tɕʰi˧ mɤ˧-ho˩}\\
	M\textsubscript{c} & \ipa{bi˧\textsubscript{c}} & to go & \ipa{bi˧ mɤ˧-ho˩}\\
	L\textsubscript{a} & \ipa{dze˩\textsubscript{a}} & to cut & \ipa{dze˩ mɤ˥-ho˥}\\
	L\textsubscript{b} & \ipa{ʈʰɯ˩\textsubscript{b}} & to drink & \ipa{ʈʰɯ˩ mɤ˩-ho˥}\\
	MH & \ipa{ʈʰæ˧˥} & to bite & \ipa{ʈʰæ˧ mɤ˧-ho˥}\\
\lspbottomrule
\end{tabularx}
\end{table}

The data concerning L-tone postverbal morphemes preceded by the {negation} \is{prefixes}prefix is set out in \tabref{tab:willnotv}. In the expressions in \tabref{tab:willnotv}, the verb is in initial position; following an exceptionless {phonological rule} stated in \sectref{sec:alistoftonerules} as Rule~3, “in tone"=group"=initial position, H and M are neutralized to M”, the verb can only receive one of
two tones: L or M. Thus L-tone verbs appear with their lexical L tone, and
all others (M, H and MH) surface with M tone.

Tone assignment on the second and third syllables cannot be summarized in terms of a~set of phonological rules. Some data
subsets show regularities, however. When the verb that constitutes the first
syllable of the expression has M tone, the third morpheme (that following the {negation} \is{prefixes}prefix)
surfaces with its lexical tone, i.e.\ the last morpheme's tone is unchanged. For instance, the MH-tone {abilitive} /\ipa{-kv̩˧˥}/ carries MH tone in /\ipa{hwæ˧ mɤ˧-kv̩˧˥}/ ‘cannot buy’, and the L-tone {desiderative} \mbox{/\ipa{-ho˩}/} carries L tone in /\ipa{hwæ˧ mɤ˧-ho˩}/ ‘will not buy’. This relates to the tendency for M to behave as a~neutral tone: M~tone does not place any
restriction on the tonal level that follows. 

When the initial morpheme has H tone, on the other hand, this H tone
precludes H tone on any of the syllables that follow, even though it does not surface as such. This results in the replacement
of a~following MH tone by L, hence /\ipa{dzɯ˧ mɤ˧-kv̩˩}/ ‘cannot eat’, not $\ddagger${\kern2pt}\ipa{dzɯ˧ mɤ˧-kv̩˧˥}. It would theoretically be possible for the H part of the MH
\is{tonal contour}contour of the third morpheme to be deleted, yielding $\ddagger${\kern2pt}\ipa{dzɯ˧ mɤ˧-kv̩˧}. But the data suggests that Yongning Na treats lexical tones
as {unitary} in this respect: either the MH tone is compatible with what precedes, and it surfaces; or
it is not, and it is lowered to L. 

As for the cases where the initial morpheme has L tone, it is an~open
question why the L tone does not spread all the way to the third syllable, yielding
//$\dagger$\ipa{dze˩ mɤ˩-tʰɑ˩}// ‘one must not cut’ (which would be realized on the surface as
/$\dagger$\ipa{dze˩ mɤ˩-tʰɑ˩˥}/, through the application of Rule 7: “If a~\isi{tone group} only contains L tones,
a~post"=lexical H tone is added to its last syllable”). It does not seem to be the case that the H
tone observed in /\ipa{dze˩ mɤ˩-tʰɑ˥}/ is what remains of the MH lexical tone after its initial
portion (M) is removed: such a~process of truncation of the first portion of a~MH tone is unattested anywhere else in the language. Rather, it looks
as if the MH tone of the third syllable were deleted by a~morphotonological rule, leaving it toneless, at which point phonology takes over: Rule 7 results in
assignment of a~H tone to the toneless syllable.


\subsection[Postverbal morphemes preceded by the interrogative]{Postverbal morphemes preceded by the interrogative particle: Tonal oppositions are neutralized}
\label{sec:theneutralizationoftonaloppositionsonmorphemesfollowingtheinterrogativeparticle}


The interrogative particle is analyzed as carrying L tone. The data in \tabref{tab:thetonepatternforvinterrogativesuffix} reveals a~complete
\isi{neutralization} of tonal oppositions on the morphemes placed after the interrogative particle. The
three morphemes shown in the table have different tones: the {immediate future} /\ipa{-bi˧\textsubscript{a}}/ has M\textsubscript{a} tone; the {desiderative} \mbox{/\ipa{-ho˩}/} has L
tone; and the {permissive} /\ipa{-tʰɑ˧˥}/ has MH tone. The tone patterns for /\ipa{mæ˧\textsubscript{b}}/, ‘to manage to’, are entirely identical. (They are not shown in \tabref{tab:thetonepatternforvinterrogativesuffix}.)

{\setlength\tabcolsep{4.5pt}
\begin{table}%[t]
\caption{\label{tab:thetonepatternforvinterrogativesuffix}The tone patterns for V+\textsc{interrogative}+postverbal morpheme.}
\begin{tabularx}{\textwidth}{ l l l Q l Q }
\lsptoprule
	tone & example & meaning & ‘going to V?’ & ‘will V?’ & ‘can V?’\\ \midrule
	H & \ipa{dzɯ˥} & to eat & \ipa{dzɯ˧ ə˩-bi˩} & \ipa{dzɯ˧ ə˩-ho˩} & \ipa{dzɯ˧ ə˩-tʰɑ˩}\\
	M\textsubscript{a} & \ipa{hwæ˧\textsubscript{a}} & to buy & \ipa{hwæ˧ ə˧-bi˥} & \ipa{hwæ˧ ə˧-ho˥} & \ipa{hwæ˧ ə˧-tʰɑ˥}\\
	M\textsubscript{b} & \ipa{tɕʰi˧\textsubscript{b}} & to sell & \ipa{tɕʰi˧ ə˧-bi˥} & \ipa{tɕʰi˧ ə˧-ho˥} & \ipa{tɕʰi˧ ə˧-tʰɑ˥}\\
	M\textsubscript{c} & \ipa{bi˧\textsubscript{c}} & to go & \ipa{bi˧ ə˧-bi˥} & \ipa{bi˧ ə˧-ho˥} & \ipa{bi˧ ə˧-tʰɑ˥}\\
	L\textsubscript{a} & \ipa{dze˩\textsubscript{a}} & to cut & \ipa{dze˩ ə˩-bi˥} & \ipa{dze˩ ə˩-ho˥} & \ipa{dze˩ ə˩-tʰɑ˥}\\
	L\textsubscript{b} & \ipa{ʈʰɯ˩\textsubscript{b}} & to drink & \ipa{ʈʰɯ˩ ə˩-bi˥} & \ipa{ʈʰɯ˩ ə˩-ho˥} & \ipa{ʈʰɯ˩ ə˩-tʰɑ˥}\\
	MH & \ipa{ʈʰæ˧˥} & to bite & \ipa{ʈʰæ˧ ə˥-bi˩} & \ipa{ʈʰæ˧ ə˥-ho˩} & \ipa{ʈʰæ˧ ə˥-tʰɑ˩}\\
\lspbottomrule
\end{tabularx}
\end{table}}


\section{Morphemes surrounding adjectives}
\label{sec:combinationsofadjectiveswithgrammaticalmorphemes}

\subsection{Addition of reduplicated suffixes to adjectives}
\label{sec:thereduplicationofadjectives}

Although adjectives behave in the same way as verbs in many respects (i.e.\ as \isi{stative verbs}), as explained in \sectref{sec:adjectivesasdistinctfromverbs},
they do not reduplicate in the same way. Adjectives are intensified by \is{suffixes}suffixation of a~reduplicated
syllable that does not carry any meaning of its own. In some cases, the \is{monosyllables}monosyllabic form of the adjective has fallen into disuse: for instance, /\ipa{bæ˩-lɑ˩{$\sim$}lɑ˥}/ ‘limp, flabby (e.g.~of meat without bones)’ and /\ipa{bæ˩-ʁwæ˩{$\sim$}ʁwæ˥}/ ‘loose (of knot)’ point to a~{monosyllabic} adjective *\ipa{bæ˩} ‘loose, limp’, but in the present state of the language this
root is not attested on its own. Another lexical peculiarity is that some such expressions have shorter variants: for instance, ‘short (of persons)’, /\ipa{to˩ʈɯ˩{$\sim$}ʈɯ˥}/, has a~\is{variants}variant /\ipa{to˩ʈɯ˩}/. Comparison with closely related dialects will be necessary to verify whether the disyllable is a~shortened version of the \is{trisyllables}{trisyllable}: it might also be that a~disyllabic adjective /\ipa{to˩ʈɯ˩}/ underwent expressive \isi{reduplication} of its second syllable. Evidence from other dialects would also be necessary to find out what was the lexical tone of the \is{monosyllables}monosyllabic root from which the longer expressions are derived. All the examples of expressions made up of an adjective and reduplicated \is{suffixes}suffix observed so far carry the same tone pattern (L.L.H), but there are too few examples to tell whether this is a~morphotonological hallmark of \textit{\textsc{Adj}+reduplicated suffix} expressions, or mere coincidence.

\subsection{Demonstrative and intensive constructions}
\label{sec:demonstrativeandintensiveconstructions}

In Yongning Na, in addition to the construction with the {relativizer}
\mbox{/\ipa{-hĩ˥}/}, adjectives are often used in \is{demonstratives}demonstrative or intensive constructions such as (\ref{ex:thusadj}).

\begin{exe}
	\ex
	\label{ex:thusadj}
	\ipaex{ʈʂʰɯ˧-{\_\_\_\_\_\_\_\_\_}-gv̩˧}\\
	\gll 	ʈʂʰɯ˥					{\_\_\_\_\_\_\_\_\_}			gv̩˧\\
	\textsc{dem.prox}		\textit{{target adjective}}	to\_be/to\_become\\
	\glt ‘thus \textsc{Adj}’ (e.g.~‘thus big’, ‘thus thick’)
\end{exe}

The morpheme /\ipa{gv̩˧}/, grammaticalized from a~verb meaning ‘to be; to become’, indicates the degree
of a~quality, as exemplified in (\ref{ex:hisnose}).

\begin{exe}
	\ex
	\label{ex:hisnose}
	\ipaex{ʈʂʰɯ˧ {\kern2pt}|{\kern2pt} no˧ {\kern2pt}|{\kern2pt} ɲi˧gɤ˧ {\kern2pt}|{\kern2pt} ʂwæ˧ mɤ˧-gv̩˧.}\\
	\gll 	ʈʂʰɯ˥			no˩				ɲi˧gɤ\#˥	ʂwæ˧	mɤ˧-			gv̩˧\\
			3\textsc{sg}	2\textsc{sg}	nose		tall	\textsc{neg}	to\_be/to\_become\\
	\glt ‘Her nose is not as prominent as yours.’ (Source: field notes.)
\end{exe}

In the \is{demonstratives}demonstrative
construction ‘thus \textsc{Adj}’, all tonal oppositions on the adjective are neutralized, as shown in \tabref{tab:thisadj}. Examples are found in narratives, e.g.~Sister.28 and BuriedAlive3.144. 

\begin{table}[h]%[t!]
	\caption{\label{tab:thisadj}Demonstrative construction /\ipa{ʈʂʰɯ˧}-\textsc{Adj}-\ipa{gv̩˧}/ ‘thus \textsc{Adj}’.}
	\begin{tabularx}{\textwidth}{ l l l l@{\hspace{4mm}} Q Q }
		\lsptoprule
		tone & example & meaning & ‘thus \textsc{Adj}’ & meaning & tone pattern\\ \midrule
		H & \ipa{ɖæ˥} & short & \ipa{ʈʂʰɯ˧-ɖæ˧-gv̩˧} & thus short & M.M.M\\
		M & \ipa{hwɤ˧} & broad & \ipa{ʈʂʰɯ˧-hwɤ˧-gv̩˧} & thus broad & M.M.M\\
		L\textsubscript{a} & \ipa{tɕi˩\textsubscript{a}} & short & \ipa{ʈʂʰɯ˧-tɕi˧-gv̩˧} & thus short & M.M.M\\
		MH & \ipa{ɬo˧˥} & deep & \ipa{ʈʂʰɯ˧-ɬo˧-gv̩˧} & thus deep & M.M.M\\
		\lspbottomrule
	\end{tabularx}
\end{table}

The same M...M tone pattern is observed in cases where the adjective is reduplicated,
e.g.~/\ipa{ʈʂʰɯ˧-ɖɯ˧{$\sim$}ɖɯ˧}\ipa{-gv̩˧}/ ‘thus big’ (Agriculture.55), from /\ipa{ɖɯ˩\textsubscript{a}}/ ‘big’ (see
also FoodShortage.43, 85).\footnote{Related constructions include: \textsc{dem}+\textsc{Adj}+\textsc{augmentative} /-mi˩/,
	e.g.~/\ipa{ʈʂʰɯ˧-ʂwæ˧-mi˧-zo˥}/ ‘thus tall’, where /\ipa{-zo}/ is an {adverbializer} (Dog.12), and the
	intensive construction /\ipa{qʰɑ˧}- \textsc{Adj} \ipa{-mi˧}/ ‘so \textsc{Adj}’, e.g.~/\ipa{qʰɑ˧-ɖɯ˧-mi˧-hĩ˧}/ ‘thus
	huge, extremely big’ (Lake3.28), where \mbox{/\ipa{-hĩ˥}/} is the {relativizer}/{nominalizer}.}

Tonal oppositions are likewise neutralized in the construction ‘as \textsc{Adj} (as)’,
/\ipa{tʰɑ˧-}\textsc{reduplicated}~\textsc{Adj}\ipa{-gv̩˩}/, shown in \tabref{tab:asadjas}.

\begin{table}[h]%[t!]
	\caption{\label{tab:asadjas}Tone patterns in the construction ‘as \textsc{Adj} as’.}
	\begin{tabularx}{\textwidth}{ l l l l@{\hspace{3mm}} l l }
		\lsptoprule
		tone & example & meaning & ‘as \textsc{Adj} as’ & meaning & tone pattern\\ \midrule
		H & \ipa{ɖæ˥} & short & \ipa{tʰɑ˧-ɖæ˥{$\sim$}ɖæ˩-gv̩˩} & as short as & M.H.L.L\\
		M & \ipa{hwɤ˧} & broad & \ipa{tʰɑ˧-hwɤ˥{$\sim$}hwɤ˩-gv̩˩} & as broad as & M.H.L.L\\
		L\textsubscript{a} & \ipa{tɕi˩\textsubscript{a}} & short & \ipa{tʰɑ˧-tɕi˥{$\sim$}tɕi˩-gv̩˩} & as short as & M.H.L.L\\
		MH & \ipa{ɬo˧˥} & deep & \ipa{tʰɑ˧-ɬo˥{$\sim$}ɬo˩-gv̩˩} & as deep as & M.H.L.L\\
		\lspbottomrule
	\end{tabularx}
\end{table}

In contrast to the data in Tables \ref{tab:thisadj} and \ref{tab:asadjas}, tonal oppositions are not fully neutralized in the construction in (\ref{ex:thusadj2}), which means ‘thus \textsc{Adj}’, much like (\ref{ex:thusadj}).\footnote{The analysis of the morpheme /\ipa{-ɻ̍˩}/ in (\ref{ex:thusadj2}) remains uncertain: it may be the {inceptive/inchoative} morpheme //\ipa{-ɻ̍˧}//.} As shown in \tabref{tab:thusadj2}, adjectives of MH lexical tone category stand out by their distinct tone pattern in (\ref{ex:thusadj2}): M.M.H, vs.\ M.M.M for all other tone categories of adjectives.

\begin{exe}
	\ex
	\label{ex:thusadj2}
	\ipaex{ʈʂʰɯ˧-{\_\_\_\_\_\_\_\_\_}-ɻ̍˧}\\
	\gll 	ʈʂʰɯ˥					{\_\_\_\_\_\_\_\_\_}			ɻ̍˧\\
	\textsc{dem.prox}		\textit{{target adjective}}	\textsc{inceptive?}\\
	\glt ‘thus \textsc{Adj}’ (e.g.~‘thus big’, ‘thus thick’)
\end{exe}

\begin{table}[h]%[t!]
	\caption{\label{tab:thusadj2}Tone patterns in the construction /\ipa{ʈʂʰɯ˧}-\textsc{Adj}-\ipa{ɻ̍˧}/ ‘thus \textsc{Adj}’.}
	\begin{tabularx}{\textwidth}{ l l l l@{\hspace{4mm}} Q Q }
		\lsptoprule
		tone & example & meaning & ‘thus \textsc{Adj}’ & meaning & tone pattern\\ \midrule
		H & \ipa{bv̩˥} & thick & \ipa{ʈʂʰɯ˧-bv̩˧-ɻ̍˧} & thus thick & M.M.M\\
		M & \ipa{fv̩˧} & happy & \ipa{ʈʂʰɯ˧-fv̩˧-ɻ̍˧} & thus happy & M.M.M\\
		L\textsubscript{a} & \ipa{ɖɯ˩\textsubscript{a}} & large & \ipa{ʈʂʰɯ˧-ɖɯ˧-ɻ̍˧} & thus large & M.M.M\\
		MH & \ipa{hæ˧˥} & supple & \ipa{ʈʂʰɯ˧-hæ˧-ɻ̍˥} & thus supple & M.M.H\\
		\lspbottomrule
	\end{tabularx}
\end{table}


\section{Object followed by non"=prefixed verb}
\label{sec:objectandnonprefixedverb}

Cross"=linguistically, the object is the nominal argument that has strongest syntactic association to the
verb. Their tie is so strong that it has even been used as a~defining property of the notion of
object \citep[38]{creissels1991}. From a~morphotonological point of view, it is reasonable to
hypothesize that there will be at least as much tonal change in the association of an~object and
a~verb as in the association of a~subject and a~verb (studied below, \sectref{sec:subjectandverb}).

The association of an~object and a~verb in Yongning Na could in theory be ambiguous with that of a~subject and
a~verb. In Na, there is no indexing of the subject or object on verbs, no case marking, and no fixed position of the subject and object relative to the verb: the verb appears after all noun phrases, except in the case of constituents tacked on at the end of the utterance as an afterthought. The \is{postpositions}postposition that forces an~interpretation of a~noun as agent (/\ipa{ɳɯ˧}/) is not
compulsory. Different tone rules applying in S+V and O+V could
be one of the means to disambiguate, but in practice S+V (without intervening morphemes) is relatively uncommon,
and disambiguation is generally effected through context. 

When eliciting object"=plus"=verb combinations, the phrases were contextualized so as to avoid reinterpretation as subject"=verb combinations. For instance, the association of ‘wolf’ and ‘to eat’
immediately suggests agent role for ‘wolf’. A~subject noun phrase was therefore added to
lead to an~interpretation that corresponds to the desired pattern, with ‘wolf’ as object: see (\ref{ex:wolfate}). (The added noun phrase does not influence the tone of the object"=plus"=verb combination, because it is followed by a tone"=group boundary.) 

\begin{exe}
	\ex
	\label{ex:wolfate}
	\ipaex{lɑ˧ ɳɯ˧ {\kern2pt}|{\kern2pt} õ˩dv̩˧ dzɯ˧-ze˩.}\\
	\gll lɑ˧	ɳɯ˧	õ˩dv̩˧˥	dzɯ˥		-ze˧\textsubscript{b}\\
	tiger		\textsc{a}		wolf	to\_eat		\textsc{pfv}\\
	\glt ‘The tiger ate the wolf.’
\end{exe}

A~sample of the elicited data is shown in \tabref{tab:sampleofofoplusv}, with nouns illustrating the various tone categories as objects of the L\textsubscript{b}-tone verb /\ipa{do˩\textsubscript{b}}/ ‘to see'. The same results obtain when using \is{numerals}numeral"=plus"=classifier phrases instead of disyllabic nouns. For instance, /\ipa{ɖɯ˧-bæ˧}/ ‘something’, made up of the \is{numerals}numeral ‘one’ and a~classifier for sorts of things, behaves tonally in the same way as disyllabic lexical units carrying the same tone (M), such as /\ipa{po˧lo˧}/ ‘ram’ and /\ipa{qæ˧do˧}/ ‘timber’; and combinations with
/\ipa{ɖɯ˧-kʰwɤ˥\$}/ ‘a piece’ yield the same tone patterns as those with /\ipa{dʑi˧hṽ̩˥\$}/ ‘clothes’. 

\begin{table}%[t!]
	\caption{\label{tab:sampleofofoplusv}A sample of object plus verb combinations: the verb /\ipa{do˩\textsubscript{b}}/ ‘to see'.}
	\begin{tabularx}{\textwidth}{ l l l l@{\hspace{11mm}} Q }
		\lsptoprule
		tone of N & example & meaning & resulting phrase & tone pattern\\ \midrule
		LM & \ipa{bo˩˧} & pig & \ipa{bo˩ do˧}  & L.M+L\\
		LH & \ipa{ʐæ˩˥} & leopard & \ipa{ʐæ˩ do˥} & L.H\\
		M & \ipa{lɑ˧} & tiger & \ipa{lɑ˧ do˩}  & M.L\\
		L & \ipa{jo˩} & sheep & \ipa{jo˩ do˩˥}  & L.L\\
		H & \ipa{ʐwæ˥} & horse & \ipa{ʐwæ˧ do˧˥}   & M.MH\\
		MH & \ipa{ʈʂʰæ˧˥} & deer & \ipa{ʈʂʰæ˧ do˧˥}   & M.MH\\ \addlinespace \hdashline \addlinespace
		M & \ipa{po˧lo˧} & ram & \ipa{po˧lo˧ do˩}  & M.M.L\\
		\#H & \ipa{ʐwæ˧zo\#˥} & colt & \ipa{ʐwæ˧zo˧ do˧˥}   & M.M.MH\\
		MH\# & \ipa{hwɤ˧li˧˥} & cat & \ipa{hwɤ˧li˧ do˧˥}  & M.M.MH\\
		H\$ & \ipa{kv̩˧ʂe˥\$} & flea & \ipa{kv̩˧ʂe˧ do˧˥}  & M.M.MH\\
		L & \ipa{kʰv̩˩mi˩} & dog & \ipa{kʰv̩˩mi˩ do˩˥}  & L.L.L\\
		L\# & \ipa{dɑ˧ʝi˩} & mule & \ipa{dɑ˧ʝi˩ do˩} & M.L.L\\
		LM+MH\# & \ipa{õ˩dv̩˧˥} & wolf & \ipa{õ˩dv̩˧ do˧˥} & L.M.MH\\
		LM+\#H & \ipa{nv̩˩tɕʰi\#˥} & fine chaff & \ipa{nv̩˩tɕʰi˧ do˧˥}  & L.M.MH\\
		LM & \ipa{bo˩mi˧} & sow & \ipa{bo˩mi˧ do˩}  & L.M.L\\
		LH & \ipa{bo˩ɬɑ˥} & boar & \ipa{bo˩ɬɑ˧ do˩}  & L.H.L\\
		H\# & \ipa{hwæ˧ʈʂæ˥} & squirrel & \ipa{hwæ˧ʈʂæ˥ do˩}  & M.H.L\\
		\lspbottomrule
	\end{tabularx}
	\end{table}

\subsection{The facts}
\label{sec:thefactsobjectandnonprefixedverb}


\tabref{tab:thetonepatternsofobjectverbcombinations} presents the tone rules that apply in object"=plus"=verb phrases. Recordings
available online include (i)~some combinations among monosyllables: ObjectVerb, (ii)~a~full set,
except for L\textsubscript{b}-tone verbs: ObjectVerb2, and (iii)~L\textsubscript{b}-tone verbs: ObjectVerb3. 

The notation adopted in \tabref{tab:thetonepatternsofobjectverbcombinations} requires disambiguation for sequences ending in a~M tone, namely
M.M and L.M: object"=plus"=verb phrases realized with one of these patterns \is{form!in isolation}in isolation may have
different underlying patterns. Consider (\ref{ex:buytea}) and (\ref{ex:buychicken}): 

\begin{exe}
	\ex
	\label{ex:buytea}
	\ipaex{li˩ hwæ˧}\\
	\gll li˩˥	hwæ˧\textsubscript{a}\\
	tea		to\_buy\\
	\glt ‘to buy tea’
\end{exe}

\begin{exe}
	\ex
	\label{ex:buychicken}
	\ipaex{æ̃˩ hwæ˧}\\
	\gll æ̃˩˧	hwæ˧\textsubscript{a}\\
	chicken		to\_buy\\
	\glt ‘to buy chicken’
\end{exe}

The phrases /\ipa{li˩ hwæ˧}/ ‘to buy tea’ and /\ipa{æ̃˩ hwæ˧}/ ‘to
buy chicken’ both surface with a~/L.M/ pattern. But they yield different results with a~following {perfective} morpheme, /\ipa{-ze˧\textsubscript{b}}/, as shown in (\ref{ex:buyteaPFV})"=(\ref{ex:buychickenPFV}).

\begin{exe}
	\ex
	\label{ex:buyteaPFV}
	\ipaex{li˩ hwæ˧-ze˩}\\
	\gll li˩˥	hwæ˧\textsubscript{a}		-ze˧\textsubscript{b}\\
	tea		to\_buy		\textsc{pfv}\\
	\glt ‘bought tea’
\end{exe}

\begin{exe}
	\ex
	\label{ex:buychickenPFV}
	\ipaex{æ̃˩ hwæ˧-ze˧}\\
	\gll æ̃˩˧	hwæ˧\textsubscript{a}		-ze˧\textsubscript{b}\\
	chicken		to\_buy		\textsc{pfv}\\
	\glt ‘bought chicken’
\end{exe}


\largerpage
\tabref{tab:thetonepatternsofobjectverbcombinations} therefore contains information on the tonal
realization of a~following {perfective} morpheme, /\ipa{-ze˧\textsubscript{b}}/, where relevant: the tone pattern of
‘to buy tea’, an~example of the combination of input tones \mbox{//LH//} (on the noun) and \mbox{//M//} (on the verb), is
transcribed in the table as /L.M+L/, and that of ‘to buy chicken’ simply as /L.M/. The pattern /M.M+L/ is
likewise distinguished from /M.M+M/. The tone on the {accomplished} \is{prefixes}prefix is preceded by a~‘+’ sign.

In the cases where the tone of the postverbal element results straightforwardly from the
phonological rules of tone association of Yongning Na (as recapitulated in Chapter~\ref{chap:toneassignmentrulesandthedivisionoftheutteranceintotonegroups}), no information is
indicated in the table. These cases are the following: when the last tone of the object"=plus"=verb phrase
is /H/ or /L/, /\ipa{-ze˧\textsubscript{b}}/ always receives L tone; and when it is /MH/,
/\ipa{-ze˧\textsubscript{b}}/ receives the /H/ part of the \is{tonal contour}contour.

As in the other tables, a~slash separates variants. For instance, for ‘has sold
leopards’ (input: \mbox{//LH//} and \mbox{//M\textsubscript{b}//}), two variants are acceptable, as shown in (\ref{ex:soldpanther}).

\begin{exe}
	\ex
	\label{ex:soldpanther}
	\ipaex{/ʐæ˩ tɕʰi˥-ze˩/~≈~/ʐæ˩
		tɕʰi˩-ze˥/}\\
	\gll ʐæ˩˥ tɕʰi˧\textsubscript{b}		-ze˧\\
	leopard		to\_sell		\textsc{pfv}\\
	\glt ‘sold leopards’
\end{exe}

For //L// and \mbox{//M\textsubscript{b}//}, two patterns are also possible: L.L, and M.M+M, as shown in (\ref{ex:soldsheep}).
%There are other examples, such as /\ipa{kʰɯ˩ tɕʰi˩-ze˥}/ ≈ /\ipa{kʰɯ˧ tɕʰi˧-ze˧}/ ‘has sold thread’.

\begin{exe}
	\ex
	\label{ex:soldsheep}
	\ipaex{/jo˩ tɕʰi˩-ze˥/~≈~/jo˧ tɕʰi˧-ze˧/}\\
	\gll jo˩ tɕʰi˧\textsubscript{b}		-ze˧\textsubscript{b}\\
	sheep		to\_sell		\textsc{pfv}\\
	\glt ‘has sold sheep’
\end{exe}

If the input is \mbox{//LH//} and //H//, two patterns are also possible: L.L, and L.H, as illustrated by (\ref{ex:hasdugearth}) and (\ref{ex:haseatenbuff}).

\begin{exe}
	\ex
	\label{ex:hasdugearth}
	\ipaex{/di˩ dv̩˩-ze˥/~≈~/di˩ dv̩˥-ze˩/}\\
	\gll di˩˥	dv̩˥		-ze˧\textsubscript{b}\\
	earth		to\_dig		\textsc{pfv}\\
	\glt ‘has dug earth’
\end{exe}

\begin{exe}
	\ex
	\label{ex:haseatenbuff}
	\ipaex{/(lɑ˧ ɳɯ˧ {\kern2pt}|{\kern2pt}) tʰɑ˩ dzɯ˩-ze˥/~≈~/(lɑ˧ ɳɯ˧ {\kern2pt}|{\kern2pt}) tʰɑ˩ dzɯ˥-ze˩/}\\
	\gll lɑ˧	ɳɯ˧	tʰɑ˩˥	dzɯ˥		-ze˧\textsubscript{b}\\
	tiger		\textsc{a}	buffalo		to\_eat		\textsc{pfv}\\
	\glt ‘(the
	tiger) has eaten (a) buffalo’
\end{exe}

Finally, a~note concerning the L.M.L / L.M.MH output (from input //LM+\#H// and //L\textsubscript{b}//): examples include (\ref{ex:seechaff}), (\ref{ex:seenugg}) and (\ref{ex:seeNaxi}). The consultant
expressed a~preference for the L.M.MH realization.

\begin{exe}
	\ex
	\label{ex:seechaff}
	\ipaex{/nv̩˩tɕʰi˧ do˧˥/~≈~/nv̩˩tɕʰi˧ do˩/}\\
	\gll nv̩˩tɕʰi\#˥ do˩\textsubscript{b}\\
	fine\_chaff		to\_see\\
	\glt ‘to see fine chaff’
\end{exe}

\begin{exe}
	\ex
	\label{ex:seenugg}
	\ipaex{/pi˩ti˧ do˧˥/~≈~/pi˩ti˧ do˩/}\\
	\gll pi˩ti\#˥ do˩\textsubscript{b}\\
	nugget	to\_see\\
	\glt ‘to see nuggets (of silver)’
\end{exe}

\begin{exe}
	\ex
	\label{ex:seeNaxi}
	\ipaex{/nɑ˩hĩ˧ do˧˥/~≈~/nɑ˩hĩ˧ do˩/}\\
	\gll nɑ˩hĩ\#˥ do˩\textsubscript{b}\\
	Naxi	to\_see\\
	\glt ‘to see (the) Naxi’
\end{exe}


\begin{sidewaystable}[p]
\caption{\label{tab:thetonepatternsofobjectverbcombinations}The tone patterns of object"=plus"=verb combinations.}
{\renewcommand{\arraystretch}{1.1}
\begin{tabularx}{\textheight}{ l@{\hspace{6mm}} Q l@{\hspace{6mm}} l@{\hspace{6mm}} l@{\hspace{6mm}} l@{\hspace{6mm}} Q }
\lsptoprule
 & tone of verb\\\cmidrule{2-7}
	tone of noun & H & M\textsubscript{a} & M\textsubscript{b} & L\textsubscript{a} & L\textsubscript{b} & MH\\ \midrule
		LM & L.M+L & L.M+M & L.M+M & L.M+L & L.M+L & L.MH\\
	LH & L.L / L.H & L.H & L.L / L.H & L.H & L.H / L.L & L.MH\\
	M & M.M+L & M.M+M & M.M+M & M.L & M.L & M.MH\\
	L & L.L & M.M+M & M.M+M / L.L & L.L & L.L / M.L & L.L\\
	H & M.M+L & M.L & M.M+L & M.H & M.MH & M.L\\
	MH & M.H & M.H & M.H & M.H & M.MH & M.H\\  \addlinespace \hdashline \addlinespace
	M & M.M.M+L & M.M.M+M & M.M.M+M & M.M.L & M.M.L & M.M.MH\\
	\#H & M.M.M+L & M.M.L & M.M.M+L & M.M.H & M.M.MH & M.M.L\\
	MH\# & M.M.MH & M.M.H+L & M.M.MH & M.M.H & M.M.MH & M.M.H\\
	H\$ & M.M.M+L & M.H.L & M.M.M+L & M.M.H & M.M.MH & M.H.L\\
	L & L.L.L & L.L.H & L.L.L & L.L.H & L.L.L & L.L.H\\
	L\# & M.L.L & M.L.L & M.L.L & M.L.L & M.L.L & M.L.L\\
	LM+MH\# & L.M.M+L & L.M.H & L.M.M+L & L.M.H & L.M.MH & L.M.H\\
	LM+\#H & L.M.M+L & L.M.L & L.M.M+L & L.M.H & L.M.L / L.M.MH & L.M.L\\
	LM & L.M.M+L & L.M.M+M & L.M.M+M & L.M.L & L.M.L & L.M.MH\\
	LH & L.H.L & L.H.L & L.H.L & L.H.L & L.H.L & L.H.L\\
	H\# & M.H.L & M.H.L & M.H.L & M.H.L & M.H.L & M.H.L\\
\lspbottomrule
\end{tabularx}}
\end{sidewaystable}

The following paragraphs offer observations about the tone patterns of object"=plus"=noun combinations, and how they relate to other parts of the morphotonological system. 


\subsection[Evidence of the opposition between \mbox{//LM//} and \mbox{//LH//}]{Object"=plus"=verb combinations reveal the tonal opposition between \mbox{//LM//} and \mbox{//LH//} {monosyllabic} nouns}
\label{sec:usefulnessofelicitingobjectverbcombinationstodistinguishtheLMandLHcategoriesofverbs}

The opposition between the \mbox{//LM//} and \mbox{//LH//} categories of nouns (illustrated by //\ipa{bo˩˧}//
‘pig’ and //\ipa{ʐæ˩˥}// ‘leopard’) is neutralized in most contexts. Among object"=plus"=verb combinations,
two contexts distinguish them: their association with a~M-tone verb, e.g.~/\ipa{bo˩ hwæ˧-ze˧}/ ‘bought pigs’ vs.\ /\ipa{ʐæ˩ hwæ˧-ze˩}/ ‘bought leopards’, and with a~H-tone verb,
e.g.~/\ipa{bo˩ dzɯ˧}/ ‘to eat pigs’ vs.\ /\ipa{ɣɯ˩ dzɯ˩˥}/ ‘to eat skin’. (‘Skin' /\ipa{ɣɯ˩˥}/ appeared as a~semantically more suitable noun than ‘leopard’ as the object of the verb ‘to eat’; but /\ipa{ʐæ˩ dzɯ˩˥}/ ‘to eat leopards’ is syntactically well"=formed nonetheless.) 

\subsection{About tonal variants}
\label{sec:abouttonalvariants}

There are interesting fine details in the use of tonal variants. Any L-tone {monosyllabic} noun acting
as object of a~L\textsubscript{b}-tone verb can yield a~L.L pattern: for instance, /\ipa{ɬi˩ ʑi˩˥}/ ‘to grab/seize
a~roebuck’, /\ipa{mv̩˩ ʑi˩˥}/ ‘to grab/seize (a/one’s) daughter’, /\ipa{ɬi˩ do˩˥}/ ‘to see
a~roebuck’, and /\ipa{mv̩˩ do˩˥}/ ‘to see (a/one’s) daughter’. On the other hand, the M.L \is{variants}variant is only
observed for some nouns: it is possible to say /\ipa{ɬi˧ ʑi˩}/ ‘to grab/seize a~roebuck’ and
/\ipa{ɬi˧ do˩}/ ‘to see a~roebuck’, but not $\ddagger${\kern2pt}\ipa{mv̩˧ ʑi˩} (intended meaning: ‘to grab/seize
(a/one’s) daughter’) or $\ddagger${\kern2pt}\ipa{mv̩˧ do˩} (intended meaning: ‘to see (a/one’s) daughter’).

In some cases, it is possible to distinguish one \is{variants}variant that is more common for frequently
occurring combinations of words. For instance, with a~L-tone object and a~LM"=tone verb, there are
two variants (M.M+M and L.L), but the former \is{variants}variant is more common for the expression ‘to drink water’: it is customary to say /\ipa{dʑɯ˧ ʈʰɯ˧}/. The latter \is{variants}variant, /\ipa{dʑɯ˩ ʈʰɯ˩˥}/, is understandable, but sounds weird.  Conversely, the elicited phrase ‘to grab sheep’ yields /\ipa{jo˩ ʑi˩˥}/. The analysis proposed here is that object"=plus"=verb combinations with this pair of input tones (L and LM) tend to acquire a~M.M+M tone pattern as a~result of high frequency of occurrence (this amounts to \isi{lexicalization}). If this analysis is correct, the reason why the consultant did not choose the \is{variants}variant /\ipa{jo˧ ʑi˧}/ for ‘to grab sheep’ is because this tone pattern carries a~hint that the activity at issue is common~-- a~part of everyday routine. This is slightly weird from a~semantic point of view, because grabbing sheep is not part of the consultant's activity as a~farmer: there are no sheep on her farm.  The M.M+M \is{variants}variant /\ipa{jo˧ ʑi˧}/ could make sense
in the context of sheep shearing, for instance, where sheep"=grabbing is part of the shearing routine.


\subsection{Exceptional combinations}
\label{sec:exceptionalcombinations}


Two \mbox{//LH//} nouns, ‘leopard’ and ‘monkey’, yield a~L.M+L pattern in combination with a~//H//-tone verb,
instead of the expected L.L pattern, as shown in (\ref{ex:haseatenleopards}) and (\ref{ex:haseatenmonkeys}).

\begin{exe}
	\ex
	\label{ex:haseatenleopards}
	\ipaex{ʐæ˩ dzɯ˧-ze˩~~~(†ʐæ˩ dzɯ˩-ze˥)}\\
	\gll ʐæ˩˥	dzɯ˥		-ze˧\textsubscript{b}\\
	leopard		to\_eat		\textsc{pfv}\\
	\glt ‘has eaten leopards’
\end{exe}

\begin{exe}
	\ex
	\label{ex:haseatenmonkeys}
	\ipaex{ʑi˩ dzɯ˧-ze˩~~~(†ʑi˩ dzɯ˩-ze˥)}\\
	\gll ʑi˩˥	dzɯ˥		-ze˧\textsubscript{b}\\
	monkey		to\_eat		\textsc{pfv}\\
	\glt ‘has eaten monkeys’
\end{exe}

The combination of ‘to eat’ and ‘leopard’ is semantically odd, as
leopards are predators rather than game, but that with ‘monkey’ is fine, and the same result was
obtained in several elicitation sessions. This is provisionally analyzed as an~exceptional
pattern: another of the complexities that need to be learnt individually.

In \tabref{tab:thetonepatternsofobjectverbcombinations}, two variants were indicated for the combination of a~\mbox{//LH//} noun and a~\mbox{//M\textsubscript{b}//} verb: L.L and L.H. For some combinations of words, both patterns are acceptable: for ‘has sold leopards’, it is
possible to say /\ipa{ʐæ˩ tɕʰi˥-ze˩}/, as well as /\ipa{ʐæ˩ tɕʰi˩-ze˥}/. But some other combinations
have apparently lexicalized with one tone pattern or the other: ‘brought in the harvest’ can only be /\ipa{bæ˩
  ʂo˥-ze˩}/, not /†\ipa{bæ˩ ʂo˩-ze˥}/, whereas ‘to eat skin’ (with the same input tones) can only be
/\ipa{ɣɯ˩ dzɯ˩-ze˥}/, not /†\ipa{ɣɯ˩ dzɯ˥-ze˩}/. It may be that combinations tend to receive a~L.H
pattern as they lexicalize, as suggested above. Sporadic tone change accompanying \isi{lexicalization} is well"=attested cross"=linguistically. Or it may be that L.H is an~older pattern, and is therefore more common on lexicalized combinations, whereas L.L is innovative.

Two further exceptions corresponding to highly lexicalized combinations are /\ipa{kʰv̩˧ ʂæ˧˥}/ ‘to hunt’,
literally ‘to lead a~dog’, from //\ipa{kʰv̩˥}// ‘dog’ and //\ipa{ʂæ˧˥}// ‘to lead along’, and /\ipa{mv̩˧ tsʰi˧˥}/
‘to light a~fire’, from  //\ipa{mv̩˥}// ‘fire’ and //\ipa{tsʰi˧˥}// ‘to light’. The productive,
regular pattern would yield a M.L tone sequence, not M.MH: \mbox{/†\ipa{kʰv̩˧}} \ipa{ʂæ˩}/ and /\ipa{†mv̩˧ tsʰi˩}/. The two expressions ‘to hunt’ and ‘to light a~fire’ are interpreted as remnants of a~tone pattern that used to be productive at an earlier historical stage.


\subsection[Noun plus copula behaves tonally like O+V]{Noun plus copula behaves tonally like object plus verb}
\label{sec:nounsplusthecopulabehavetonallylikeobjectverbcombinations}

Data on the behaviour of the \isi{copula} //\ipa{ɲi˩\textsubscript{a}}// after each tonal category of nouns was set out in Chapter~\ref{chap:thelexicaltonesofnouns}. It
coincides exactly with the behaviour of L\textsubscript{a}-tone verbs in object"=plus"=verb combinations, and not
with the behaviour of L\textsubscript{a}-tone verbs in subject"=plus"=verb combinations. This observation about
tones provides a~strong indication about the morphosyntactic status of the \isi{copula} in Yongning Na.


\subsection{Interrogative pronoun and verb}
\label{sec:interrogativepronounandverb}

The combination of an~interrogative \is{pronouns}pronoun and a~verb is a~special case of the combination of
object and verb. The tonal patterns are not identical, however: see \tabref{tab:whichwhichplacewhere}. Surprisingly, the three interrogative pronouns in \tabref{tab:whichwhichplacewhere}, which are currently hypothesized to belong in the
same lexical tone category, namely \mbox{//LM//}, show different tonal behaviours. These differences among the
three pronouns were checked carefully; for instance, it has been verified that $\ddagger${\kern2pt}\ipa{ze˩bæ˧ lɑ˧˥}
is not an~acceptable \is{variants}variant for ‘strike which sort [of things]?’, any more than $\ddagger${\kern2pt}\ipa{ze˩gɤ˧ lɑ˥} for ‘strike which place?’.

{\setlength\tabcolsep{5.5pt}
\begin{table}%[t]
\caption{\label{tab:whichwhichplacewhere}The tonal behaviour of three interrogative pronouns preceding a~verb: /\ipa{ze˩bæ˧}/ ‘which sort’, /\ipa{ze˩gɤ˧}/ ‘which place’, and /\ipa{zo˩qo˧}/ ‘where’.}
\begin{tabularx}{\textwidth}{ l l l l l Q }
\lsptoprule
	tone & example & meaning & which & which place & where\\ \midrule
	H & \ipa{dzɯ˥} & to eat & \ipa{ze˩bæ˧ dzɯ˧} & \ipa{ze˩gɤ˧ dzɯ˧} & \ipa{zo˩qo˧ dzɯ˧}\\
	M\textsubscript{a} & \ipa{hwæ˧\textsubscript{a}} & to buy & \ipa{ze˩bæ˧ hwæ˩} & \ipa{ze˩gɤ˧ hwæ˧} & \ipa{zo˩qo˧ hwæ˧}\\
	M\textsubscript{b} & \ipa{tɕʰi˧\textsubscript{b}} & to sell & \ipa{ze˩bæ˧ tɕʰi˧} & \ipa{ze˩gɤ˧ tɕʰi˧} & \ipa{zo˩qo˧ tɕʰi˧}\\
	M\textsubscript{c} & \ipa{pv̩˧\textsubscript{c}} & to chant & \ipa{ze˩bæ˧ pv̩˩} & \ipa{ze˩gɤ˧ pv̩˧} & \ipa{zo˩qo˧ pv̩˧}\\
	L\textsubscript{a} & \ipa{bæ˩\textsubscript{a}} & to sweep & \ipa{ze˩bæ˧ bæ˥} & \ipa{ze˩gɤ˧ bæ˩} & \ipa{zo˩qo˧ bæ˩}\\
	L\textsubscript{b} & \ipa{ʈʰɯ˩\textsubscript{b}} & to drink;  & \ipa{ze˩bæ˧ ʈʰɯ˧˥} & \ipa{ze˩gɤ˧ ʈʰɯ˩} & \ipa{zo˩qo˧ ʈʰɯ˩}\\
	 & \ipa{ʐwɤ˩\textsubscript{b}} & to speak & \ipa{ze˩bæ˧ ʐwɤ˧˥} & \ipa{ze˩gɤ˧ ʐwɤ˩} & \ipa{zo˩qo˧ ʐwɤ˩}\\
	MH & \ipa{lɑ˧˥} & to strike & \ipa{ze˩bæ˧ lɑ˥} & \ipa{ze˩gɤ˧ lɑ˧˥} & \ipa{zo˩qo˧ lɑ˧˥}\\
\lspbottomrule
\end{tabularx}
\end{table}}

 

\section{Object and prefixed verb}
\label{sec:objectandprefixedverb}

When an~object associates to a~prefixed verb, they can form a~single \isi{tone group}, as in
(\ref{ex:putonclothes}), or the prefixed verb may be preceded by a~\isi{tone group} \is{boundary (between tone groups)}boundary, as in
(\ref{ex:theelderbrotherputoncoarsefeltcloak}).


\begin{exe}
  \ex
  \label{ex:putonclothes}
  \ipaex{bɑ˩lɑ˩ tʰi˥-mv̩˩}\\
  \gll bɑ˩lɑ˩		tʰi˧-	mv̩˧\textsubscript{a}\\
  clothes		\textsc{dur}	to\_put\_on\\
  \glt ‘to put on clothes’ (ComingOfAge2.37)

  
  \ex
  \label{ex:theelderbrotherputoncoarsefeltcloak}
  \ipaex{ə˧mv̩˧ ɳɯ˥, {\kern2pt}|{\kern2pt} ʐæ˩sɯ˩˥ {\kern2pt}|{\kern2pt} tʰi˧-mv̩˧}\\
  \gll ə˧mv̩˧˥		ɳɯ˧	ʐæ˩sɯ˩		tʰi˧	mv̩˧\textsubscript{a}\\
  elder\_sibling	\textsc{a}	coarse\_felt	\textsc{dur}	to\_put\_on\\
  \glt ‘the elder brother put on [his] coarse felt cloak’ (Sister3.57)
\end{exe}

The same \is{stylistics}stylistic choice is open for \is{numerals}numeral"=plus"=classifier phrases in object position. Example (\ref{ex:bowlofwine}) illustrates a~case of integration of object (‘a~bowl’) and verb (‘to pour’). The length of the \isi{tone group} is not the reason for separating the prefixed verb from the object as two tone groups: in (\ref{ex:bowlofwine}) we have a~\isi{tone group} comprising six syllables, but in (\ref{ex:theelderbrotherputoncoarsefeltcloak}) if the second \isi{tone group} were not divided it would be just four syllables. Motivations for the placement of tone"=group boundaries are discussed in detail in \sectref{sec:thedivisionofutterancesintotonegroups}.

\begin{exe}
  \ex
  \label{ex:bowlofwine}
  \ipaex{ʐɯ˧ {\kern2pt}|{\kern2pt} ɖɯ˧-qʰwɤ˧ tʰi˥-pʰv̩˩ tsɯ˩ {\kern2pt}|{\kern2pt} mv̩˩.}\\
  \gll ʐɯ˧		ɖɯ˧	qʰwɤ˧˥		tʰi˧-	pʰv̩˧˥	  tsɯ˧˥	mv̩˧\\
  liquor/spirits		one	\textsc{clf}.bowls	\textsc{dur}	to\_pour	  \textsc{rep}	\textsc{affirm}\\
  \glt ‘it is said that [she] poured a~bowl of liquor [for her brother].’ (Sister3.41)
\end{exe}

When the object and prefixed verb are integrated into one \isi{tone group}, it seems at first glance
as if the adjustment were purely phonological, the computation of tones within the group proceeding
from the beginning (“left"=to"=right”). In example (\ref{ex:bowlofwine}), for instance, it looks as if
the MH \is{tonal contour}contour on the \is{numerals}numeral"=plus"=classifier unfolds over the prefixed verb: /\ipa{ɖɯ˧-qʰwɤ˧˥}/
‘one bowl(ful)’ plus /\ipa{tʰi˧-pʰv̩˧˥}/ ‘to pour’ would be analyzed as yielding /\ipa{ɖɯ˧-qʰwɤ˧
  tʰi˥}{\dots}/ by unfolding of the MH \is{tonal contour}contour, the H part of the noun phrase’s MH \is{tonal contour}contour
reassociating to the {durative} \is{prefixes}prefix //\ipa{tʰi˧}-// (hence /\ipa{tʰi˥}-/). The final result /\ipa{ɖɯ˧-qʰwɤ˧ tʰi˥-pʰv̩˩}/ would
obtain by application of Rules 4 and 5: “A syllable following a~H-tone syllable receives L tone”,
and “All syllables following a~H.L or M.L sequence receive L tone”. A~similar analysis can be
extended to almost all cases~-- but not quite all of them, leading to the conclusion (now familiar to the reader) that in this part of the system too, the adjustment between tones is not purely phonological. The association of an~object and a~prefixed verb therefore
still belongs within morphophonology, although only a~small step would be required for
it to become a~purely phonological process: simplifying the handful of forms that do not currently
obtain on the basis of phonological rules. 

The facts are set out and discussed in detail below.


\subsection{The facts}
\label{sec:thefactsobjectandprefixedverb}

The data is arranged by tone. The behaviour of tones M\textsubscript{a}, M\textsubscript{b} and M\textsubscript{c} is identical, as is that
of tones L\textsubscript{a} and L\textsubscript{b}; the data presented is therefore limited to one M-tone verb (\tabref{tab:objectsplusmtoneverbsprefixedbythedurative})
and one L-tone verb
(\tabref{tab:objectsplusltoneverbsprefixedbythedurative}). The patterns that do not conform with the regularities
discussed below (\sectref{sec:dataanalysisobjectandprefixedverb}) are shaded in gray.

{%\setlength\tabcolsep{4pt}
\begin{table}[t]
\caption{\label{tab:objectsplusmtoneverbsprefixedbythedurative}Objects plus the M-tone verb \ipa{hwæ˧\textsubscript{a}} ‘to buy’ prefixed by the {durative} /\ipa{tʰi˧}-/.}
\begin{tabularx}{\textwidth}{ l l l Q l }
\lsptoprule
	tone & head & meaning & \ipa{hwæ˧\textsubscript{a}} ‘to buy’ & tone pattern\\ \midrule
	LM & \ipa{bo˩˧} & pig & \ipa{bo˩ tʰi˧-hwæ˧} & L.M.M\\
	LH & \ipa{mv̩˩˥} & daughter & \ipa{mv̩˩ tʰi˥-hwæ˩} & L.H.L\\
	M & \ipa{lɑ˧} & tiger & \ipa{lɑ˧ tʰi˧-hwæ˧} & M.M.M\\
	L & \ipa{jo˩} & sheep & \shadedcell \ipa{jo˧ tʰi˧-hwæ˧} & \shadedcell M.M.M\\
	\#H & \ipa{hĩ˥} & human being & \ipa{hĩ˧ tʰi˩-hwæ˩} & M.L.L\\
	MH\# & \ipa{tsʰɯ˧˥} & goat & \ipa{tsʰɯ˧ tʰi˥-hwæ˩} & M.H.L\\ \addlinespace \hdashline \addlinespace
	M & \ipa{po˧lo˧} & ram & \ipa{po˧lo˧ tʰi˧-hwæ˧} & M.M.M.M\\
	\#H & \ipa{ʐwæ˧zo\#˥} & colt & \ipa{ʐwæ˧zo˧ tʰi˩-hwæ˩} & M.M.L.L\\
	MH\# & \ipa{hwɤ˧li˧˥} & cat & \ipa{hwɤ˧li˧ tʰi˥-hwæ˩} & M.M.H.L\\
	H\$ & \ipa{kv̩˧ʂe˥\$} & flea & \ipa{kv̩˧ʂe˥ tʰi˩-hwæ˩} & M.H.L.L\\
	L & \ipa{kʰv̩˩mi˩} & dog & \ipa{kʰv̩˩mi˩ tʰi˥-hwæ˩} & L.L.H.L\\
	L\# & \ipa{dɑ˧ʝi˩} & mule & \ipa{dɑ˧ʝi˩ tʰi˩-hwæ˩} & M.L.L.L\\
	LM+MH\# & \ipa{v̩˩tsʰɤ˧˥} & vegetables & \ipa{v̩˩tsʰɤ˧ tʰi˥-hwæ˩} & M.M.H.L\\
	LM+\#H & \ipa{ɑ˩mi\#˥} & goose & \ipa{ɑ˩mi˧ tʰi˩-hwæ˩} & L.M.L.L\\
	LM & \ipa{bo˩mi˧} & sow & \ipa{bo˩mi˧ tʰi˧-hwæ˧} & L.M.M.M\\
	LH & \ipa{bo˩ɬɑ˥} & boar & \ipa{bo˩ɬɑ˥ tʰi˩-hwæ˩} & L.H.L.L\\
	H\# & \ipa{kʰv̩˧nɑ˥} & dog & \ipa{kʰv̩˧nɑ˥ tʰi˩-hwæ˩} & M.H.L.L\\
\lspbottomrule
\end{tabularx}
\end{table}}


\begin{table}%[t]
\caption{\label{tab:objectsplushtoneverbsprefixedbythedurative}Objects plus the H-tone verb \ipa{dzɯ˥} ‘to eat’ prefixed by the {durative} /\ipa{tʰi˧}-/.}
\begin{tabularx}{\textwidth}{ l l l Q l }
\lsptoprule
	tone & head & meaning & \ipa{dzɯ˥} ‘to eat’ & tone pattern\\ \midrule
	LM & \ipa{bo˩˧} & pig & \ipa{bo˩ tʰi˧-dzɯ˥} & L.M.H\\
	LH & \ipa{mv̩˩˥} & daughter & \ipa{mv̩˩ tʰi˥-dzɯ˩} & L.H.L\\
	M & \ipa{lɑ˧} & tiger & \ipa{lɑ˧ tʰi˧-dzɯ˥} & M.M.H\\
	L & \ipa{jo˩} & sheep & \shadedcell \ipa{jo˩ tʰi˩-dzɯ˩˥} & \shadedcell L.L.LH\\
	\#H & \ipa{hĩ˥} & human being & \ipa{hĩ˧ tʰi˩-dzɯ˩} & M.L.L\\
	MH\# & \ipa{tsʰɯ˧˥} & goat & \ipa{tsʰɯ˧ tʰi˥-dzɯ˩} & M.H.L\\ \addlinespace \hdashline \addlinespace
	M & \ipa{po˧lo˧} & ram & \ipa{po˧lo˧ tʰi˧-dzɯ˥} & M.M.M.H\\
	\#H & \ipa{ʐwæ˧zo\#˥} & colt & \ipa{ʐwæ˧zo˧ tʰi˩-dzɯ˩} & M.M.L.L\\
	MH\# & \ipa{hwɤ˧li˧˥} & cat & \ipa{hwɤ˧li˧ tʰi˥-dzɯ˩} & M.M.H.L\\
	H\$ & \ipa{kv̩˧ʂe˥\$} & flea & \ipa{kv̩˧ʂe˥ tʰi˩-dzɯ˩} & M.H.L.L\\
	L & \ipa{kʰv̩˩mi˩} & dog & \ipa{kʰv̩˩mi˩ tʰi˥-dzɯ˩} & L.L.H.L\\
	L\# & \ipa{dɑ˧ʝi˩} & mule & \ipa{dɑ˧ʝi˩ tʰi˩-dzɯ˩} & M.L.L.L\\
	LM+MH\# & \ipa{v̩˩tsʰɤ˧˥} & vegetables & \ipa{v̩˩tsʰɤ˧ tʰi˥-dzɯ˩} & L.M.H.L\\
	LM+\#H & \ipa{ɑ˩mi\#˥} & goose & \shadedcell \ipa{ɑ˩mi˧ tʰi˥-dzɯ˩} & \shadedcell L.M.H.L\\
	LM & \ipa{bo˩mi˧} & sow & \ipa{bo˩mi˧ tʰi˧-dzɯ˥} & L.M.M.H\\
	LH & \ipa{bo˩ɬɑ˥} & boar & \ipa{bo˩ɬɑ˥ tʰi˩-dzɯ˩} & L.H.L.L\\
	H\# & \ipa{kʰv̩˧nɑ˥} & dog & \ipa{kʰv̩˧nɑ˥ tʰi˩-dzɯ˩} & M.H.L.L\\
\lspbottomrule
\end{tabularx}
\end{table}

% \clearpage

\begin{table}%[t]
\caption{\label{tab:objectsplusltoneverbsprefixedbythedurative}Objects plus the L-tone verb \ipa{di˩\textsubscript{a}} ‘to have’ prefixed by the {durative} /\ipa{tʰi˧}-/.}
\begin{tabularx}{\textwidth}{ l@{\hspace{6mm}} l@{\hspace{6mm}} l@{\hspace{6mm}} Q l }
\lsptoprule
	tone & head & meaning & \ipa{di˩\textsubscript{a}} ‘to have’ & tone pattern\\ \midrule
	LM & \ipa{bo˩˧} & pig & \ipa{bo˩ tʰi˧-di˩} & L.M.L\\
	LH & \ipa{mv̩˩˥} & daughter & \shadedcell \ipa{mv̩˩ tʰi˩-di˥} & \shadedcell L.L.H\\
	M & \ipa{lɑ˧} & tiger & \ipa{lɑ˧ tʰi˧-di˩} & M.M.L\\
	L & \ipa{jo˩} & sheep & \shadedcell \ipa{jo˧ tʰi˧-di˩} & \shadedcell M.M.L\\
	\#H & \ipa{hĩ˥} & human being & \shadedcell \ipa{hĩ˧ tʰi˧-di˥} & \shadedcell M.M.H\\
	MH\# & \ipa{tsʰɯ˧˥} & goat & \shadedcell \ipa{tsʰɯ˧ tʰi˧-di˥} &
   \shadedcell M.M.H\\ \addlinespace \hdashline \addlinespace
	M & \ipa{po˧lo˧} & ram & \ipa{po˧lo˧ tʰi˧-di˩} & M.M.M.L\\
	\#H & \ipa{ʐwæ˧zo\#˥} & colt & \shadedcell \ipa{ʐwæ˧zo˧ tʰi˧-di˥} & \shadedcell M.M.M.H\\
	MH\# & \ipa{hwɤ˧li˧˥} & cat & \shadedcell \ipa{hwɤ˧li˧ tʰi˧-di˥} & \shadedcell M.M.M.H\\
	H\$ & \ipa{kv̩˧ʂe˥\$} & flea & \shadedcell \ipa{kv̩˧ʂe˧ tʰi˧-di˥} & \shadedcell M.M.M.H\\
	L & \ipa{kʰv̩˩mi˩} & dog & \shadedcell \ipa{kʰv̩˩mi˩ tʰi˩-di˥} & \shadedcell L.L.L.H\\
	L\# & \ipa{dɑ˧ʝi˩} & mule & \ipa{dɑ˧ʝi˩tʰi˩-di˩} & M.L.L.L\\
	LM+MH\# & \ipa{v̩˩tsʰɤ˧˥} & vegetables & \shadedcell \ipa{v̩˩tsʰɤ˧ tʰi˧-di˥} & \shadedcell L.M.M.H\\
	LM+\#H & \ipa{ɑ˩mi\#˥} & goose & \shadedcell \ipa{ɑ˩mi˧ tʰi˧-di˥} & \shadedcell L.M.M.H\\
	LM & \ipa{bo˩mi˧} & sow & \ipa{bo˩mi˧ tʰi˧-di˩} & L.M.M.L\\
	LH & \ipa{bo˩ɬɑ˥} & boar & \ipa{bo˩ɬɑ˥ tʰi˩-di˩} & L.H.L.L\\
	H\# & \ipa{kʰv̩˧nɑ˥} & dog & \ipa{kʰv̩˧nɑ˥ tʰi˩-di˩} & M.H.L.L\\
\lspbottomrule
\end{tabularx}
\end{table}

\begin{table}%[t]
\caption{\label{tab:objectsplusmhtoneverbsprefixedbythedurative}Objects plus the MH"=tone verb \ipa{ʈʰæ˧˥} ‘to bite’ prefixed by the {durative} /\ipa{tʰi˧}-/.}
\begin{tabularx}{\textwidth}{ l l l@{\hspace{6mm}} Q l }
\lsptoprule
	tone & head & meaning & \ipa{ʈʰæ˧˥} ‘to bite’ & tone pattern\\ \midrule
	LM & \ipa{bo˩˧} & pig & \ipa{bo˩ tʰi˧-ʈʰæ˧˥} & L.M.MH\\
	LH & \ipa{mv̩˩˥} & daughter & \ipa{mv̩˩ tʰi˥-ʈʰæ˩} & L.H.L\\
	M & \ipa{lɑ˧} & tiger & \ipa{lɑ˧ tʰi˧-ʈʰæ˧˥} & M.M.MH\\
	L & \ipa{jo˩} & sheep & \shadedcell \ipa{jo˩ tʰi˩-ʈʰæ˩˥} & \shadedcell L.L.LH\\
	\#H & \ipa{hĩ˥} & human being & \ipa{hĩ˧ tʰi˩-ʈʰæ˩} & M.L.L\\
	MH\# & \ipa{tsʰɯ˧˥} & goat & \ipa{tsʰɯ˧ tʰi˥-ʈʰæ˩} & M.H.L\\ \addlinespace \hdashline \addlinespace
	M & \ipa{po˧lo˧} & ram & \ipa{po˧lo˧ tʰi˧-ʈʰæ˧˥} & M.M.M.MH\\
	\#H & \ipa{ʐwæ˧zo\#˥} & colt & \ipa{ʐwæ˧zo˧ tʰi˩-ʈʰæ˩} & M.M.L.L\\
	MH\# & \ipa{hwɤ˧li˧˥} & cat & \ipa{hwɤ˧li˧ tʰi˥-ʈʰæ˩} & M.M.H.L\\
	H\$ & \ipa{kv̩˧ʂe˥\$} & flea & \ipa{kv̩˧ʂe˥ tʰi˩-ʈʰæ˩} & M.H.L.L\\
	L & \ipa{kʰv̩˩mi˩} & dog & \ipa{kʰv̩˩mi˩ tʰi˥-ʈʰæ˩} & L.L.H.L\\
	L\# & \ipa{dɑ˧ʝi˩} & mule & \ipa{dɑ˧ʝi˩ tʰi˩-ʈʰæ˩} & M.L.L.L\\
	LM+MH\# & \ipa{v̩˩tsʰɤ˧˥} & vegetables & \ipa{v̩˩tsʰɤ˧ tʰi˥-ʈʰæ˩} & L.M.H.L\\
	LM+\#H & \ipa{ɑ˩mi\#˥} & goose & \ipa{ɑ˩mi˧ tʰi˩-ʈʰæ˩} & L.M.L.L\\
	LM & \ipa{bo˩mi˧} & sow & \ipa{bo˩mi˧ tʰi˧-ʈʰæ˧˥} & L.M.M.MH\\
	LH & \ipa{bo˩ɬɑ˥} & boar & \ipa{bo˩ɬɑ˥ tʰi˩-ʈʰæ˩} & L.H.L.L\\
	H\# & \ipa{kʰv̩˧nɑ˥} & dog & \ipa{kʰv̩˧nɑ˥ tʰi˩-ʈʰæ˩} & M.H.L.L\\
\lspbottomrule
\end{tabularx}
\end{table}



Examples in texts are abundant: the {durative} \is{prefixes}prefix /\ipa{tʰi˧}-/ appears over 700 times in twenty
texts, illustrating a~broad range of combinations. In example (\ref{ex:putashes}), this \is{prefixes}prefix appears after a LH-tone noun, a~context in which it gets a~H tone. Other interesting examples are found in Funeral.69,
108, 190, 238, 253, Healing.103, Housebuilding.40, 110, 121, 217, 239, Mountains.7, 88, 161,
Reward.40, 73, Seeds2.51, 62 and Sister3.41, 95.

\newpage 
\begin{exe}
	\ex
	\label{ex:putashes}
	\ipaex{lwɤ˩ tʰi˥-kʰɯ˩ {\kern2pt}|{\kern2pt} tɕɤ˧-kv̩˥ mæ˩, {\kern2pt}|{\kern2pt} ə˧ʝi˧-ʂɯ˥ʝi˩!}\\
	\gll lwɤ˩˥		tʰi˧-			kʰɯ˧˥		tɕɤ˧˥		-kv̩˧˥					mæ˧							ə˧ʝi˧-ʂɯ˥ʝi˩\\
	ashes		\textsc{dur}	to\_put	 	to\_boil	\textsc{abilitive}	\textsc{obviousness}		in\_the\_old\_times\\
	\glt ‘One would add ashes and boil [linen thread], in the old times!’ (FoodShortage.71)
\end{exe}


\subsection{Data analysis}
\label{sec:dataanalysisobjectandprefixedverb}


The most straightforward cases are presented first, followed by the more complex ones.


\subsubsection{The tone of the verb expresses itself when the noun phrase has LM or M tone}
\label{sec:thetoneoftheverbexpressesitselfwhenthenounphrasehastonelmorm}


As observed in \sectref{sec:dataanalysisobjectandprefixedverb} of Chapter~\ref{chap:thelexicaltonesofnouns}, M tends to behave as an~inert tone in Yongning Na: it does not
spread or otherwise affect following tones. The behaviour of M-tone nouns in association with
a~prefixed verb is in keeping with this observation: when the noun phrase has M tone, the prefixed
verb surfaces with the same tones as \is{form!in isolation}in isolation. The same is true of LM"=tone noun phrases.


\subsubsection{Tonal oppositions on verbs are neutralized after a~disyllabic noun phrase with L\#, LH or H\# tone}
\label{sec:tonaloppositionsonverbsareneutralizedafteradisyllabicnounphrasewithtonellhandh}

Tones L\#, LH and H\# on disyllables preclude any tone other than L on syllables that follow within
the \isi{tone group}, by application of Rules 4 and 5: “A syllable following a~H-tone syllable receives L
tone”, and “All syllables following a~H.L or M.L sequence receive L tone”. All tonal oppositions on
verbs are therefore neutralized after a~disyllabic noun phrase carrying one of these three tones.


\subsubsection{Patterns that cannot be fully explained by phonological regularities}
\label{sec:commentsonpatternsthatcannotbefullyexplainedonaphonologicalbasis}


The patterns discussed in the two preceding paragraphs can be explained fully on the basis of phonological regularities that apply throughout the system. This is not true of all patterns, however. The overall proportion is that about three combinations out of four follow a~set of phonological tendencies. This figure appears high enough to formulate the tentative hypothesis that these tendencies represent the default case, and that the remaining cases are likely to be learnt individually. 

\largerpage
The phonological tendencies are as follows.

\begin{enumerate}[label=(\roman*), itemsep=0pt]
\item Tone \mbox{//LH//} on a~{monosyllabic} noun projects its H portion onto the next syllable, in this case the {durative} \is{prefixes}prefix /\ipa{tʰi˧}/. 
\item Tones \mbox{//\#H//} and //LM+\#H// do not overtly express their H tone, which remains \is{floating tone}floating, but is not deleted: this \is{floating tone}floating H tone lowers the tones of the following syllables to L. 
\item Tone \mbox{//H\$//} gets docked on the last syllable of the noun phrase, resulting in a~lowering of the tone of following syllables to L, by Rules 4 and 5: “A syllable following a~H-tone syllable receives L tone”, and “All syllables following a~H.L or M.L sequence receive L tone”.
\item Tones \mbox{//MH\#//} and //LM+MH\#// project their final H level onto the following syllable. 
\end{enumerate}

Seen in this light, the combinations that follow the general tendencies could be described as regular, and the others as irregular. The patterns for //L//-tone verbs are irregular because they contravene tendencies (i), (ii) and (iv). (The behaviour of L-tone verbs is analyzed further in the next section, \sectref{sec:thebehaviourofltoneverbsattemptingageneralization}.) The pattern for a~//LM+\#H// noun and a~//H//-tone verb is also irregular: in view of tendency (ii), one would expect L.M.L.L,  instead of the observed L.M.M.H: see (\ref{ex:eatgoose}).

\begin{exe}
	\ex
	\label{ex:eatgoose}
	\ipaex{ɑ˩mi˧ tʰi˧-dzɯ˥~~~~(†ɑ˩mi˧ tʰi˩-dzɯ˩)}\\
	\gll ɑ˩mi\#˥			tʰi˧-			dzɯ˥\\
	goose	\textsc{dur}		to\_eat\\
	\glt ‘eating a~goose’
\end{exe}

The issue of regularity and irregularity in morphotonological paradigms will be taken up again in the typological discussion in Chapter~\ref{chap:arealandtypologicaldiscussion}. 


\subsubsection{The behaviour of L-tone verbs: Attempting a~generalization}
\label{sec:thebehaviourofltoneverbsattemptingageneralization}

L-tone verbs are those that exhibit the greatest proportion of irregular patterns: patterns that do
not follow the four phonological tendencies described in the paragraph that precedes. It appears possible to attempt a~generalization
nonetheless. One way to describe what happens in this particular morphosyntactic context is as follows: when the noun phrase contains a~H tone, this H tone moves all the way to the last syllable of the resulting verb phrase, unless it is unmovably fixed to one of the syllables.

This generalization reflects the identical tonal treatment of the prefixed verb after a~disyllable
with \mbox{//H\#//} or \mbox{//LH//} tone. When a~\mbox{//H\#//} or \mbox{//LH//} tone is lexically associated to a~disyllabic noun, a~H tone is unmovably attached to the last syllable of the noun. On the other hand, tones \mbox{//\#H//}, //LM+\#H//, \mbox{//H\$//}, \mbox{//MH\#//}
and //LM+MH\#// share the property of containing a~H tone whose syllabic \is{anchorage}anchoring is context"=sensitive.

An apparent \isi{counterexample} to this generalization is \mbox{//LH//} tone: disyllabic \mbox{//LH//}"=tone nouns yield
L.H.L.L, but {monosyllabic} \mbox{//LH//}"=tone nouns yield L.L.H, as in (\ref{ex:seedaughter}).

\begin{exe}
	\ex
	\label{ex:seedaughter}
	\ipaex{mv̩˩ tʰi˩-do˥}\\
	\gll mv̩˩˥		tʰi˧-			do˩\textsubscript{b}\\
	daughter		\textsc{dur}		to\_see\\
	\glt ‘to see \mbox{(a/the)}
	daughter’
\end{exe}

At this point, one may consider that this difference of treatment demonstrates clearly
that the morphotonology at play here is irregular, and that the analyst should refrain from devising
clever accounts that are more systematic and neater than the facts. The above regularity would then need to be abandoned. Experience in learning and speaking the language suggests otherwise, however, and encourages the investigator to venture some more speculations about how the observed patterns relate to one another. 

The different treatment of \mbox{//LH//} tone on {monosyllabic} and disyllabic nouns could be related
to the fact that, although the tonal category is identical, its association to syllables is
different for monosyllables and disyllables. When the noun has two syllables, the H part of the
\mbox{//LH//} tone is unmovably attached to the second syllable, making this tone category an~easy
one for the language learner. In the case of monosyllables, on the other hand, for want of
a~sufficient number of syllables (which constitute the tone"=bearing unit in Yongning Na), there is some pressure for the H part to reassociate to
a~later syllable. In various morphosyntactic contexts, the H part of \mbox{//LH//} does not surface on the
noun to which it is lexically associated. To sum up, there is a~hard"=and"=fast syllabic
\is{anchorage}anchoring for \mbox{//LH//} tone when it associates to a~disyllable, whereas on a~\is{monosyllables}monosyllable its \is{anchorage}anchoring is looser. Viewed in this light, it does not come as a~great surprise that \mbox{//LH//} on a~{monosyllabic} noun
should pattern as one of the movable H tones (alongside \mbox{//\#H//}, //LM+\#H//, \mbox{//H\$//}, \mbox{//MH\#//} and //LM+MH\#//) rather than as one of the unmovable H tones.

This generalization is \textit{ad hoc}, in the sense that it is based on the data for this specific
morphosyntactic context. But it does not appear absurd to consider that it has psychological
reality, allowing learners to memorize the patterns for tones \mbox{//\#H//}, //LM+\#H//, \mbox{//H\$//}, \mbox{//MH\#//} and //LM+MH\#// (as well as \mbox{//LH//}"=tone monosyllables) all at one go.

Still on a~speculative note, this analysis predicts high cross"=dialect \isi{variation} for //L//-tone verbs. Cases that do not conform to regularities (i-iv) as set out in \sectref{sec:commentsonpatternsthatcannotbefullyexplainedonaphonologicalbasis} are especially numerous for //L//-tone verbs; the force of \isi{analogy} would tend to simplify the system by
eliminating these irregularities. This is an~empirical {question} to investigate on the basis of data from other dialects.


\section{Subject and verb}
\label{sec:subjectandverb}

Tonal interaction between subject and verb is illustrated by
(\ref{ex:snowrain}--\ref{ex:guestpriest}). The verb ‘to fall’ carries different tones in (\ref{ex:itissnowing}) and (\ref{ex:itisraining}), as does the verb ‘to come’ in (\ref{ex:theguesthasarrived}) and (\ref{ex:thepriesthasarrived}).

\begin{exe}
  \ex
  \label{ex:snowrain}
  \begin{xlist}
    \ex
  \label{ex:itissnowing}
  \ipaex{bi˧ gi˧-ze˩.}\\
  \gll bi˥	gi˥	-ze˧\textsubscript{b}\\
  snow	to\_fall	\textsc{pfv}\\
  \glt 		‘It has snowed.’

  \ex
  \label{ex:itisraining}
  \ipaex{hi˩ gi˩-ze˥.}\\
  \gll hi˩˧	gi˥	-ze˧\textsubscript{b}\\
  rain	to\_fall	\textsc{pfv}\\
  \glt 		‘It has rained.’
  \end{xlist}

  \ex
  \label{ex:guestpriest}
  \begin{xlist}
  \ex
  \label{ex:theguesthasarrived}
  \ipaex{hĩ˧bæ˧ tsʰɯ˧-ze˥.}\\
  \gll hĩ˧bæ\#˥	tsʰɯ˩\textsubscript{a}	-ze˧\textsubscript{b}\\
  guest		to\_come.\textsc{pst}		\textsc{pfv}\\
  \glt 		‘The guest has come. / The guests have come.’

  \ex
  \label{ex:thepriesthasarrived}
  \ipaex{dɑ˧pɤ˧ tsʰɯ˩-ze˩.}\\
  \gll dɑ˧pɤ˧		tsʰɯ˩\textsubscript{a}	-ze˧\textsubscript{b}\\
  priest		to\_come.\textsc{pst}	\textsc{pfv}\\
  \glt 		‘The priest has come. / The priests have come.’
  \end{xlist}
\end{exe}


Subject"=plus"=verb phrases are relatively infrequent in Yongning Na. This is due in part to the high
frequency of post-nominal morphemes and verbal prefixes (which separate the subject from the verb in the linear ordering of the sentence) and, for transitive verbs, to the subject"=object"=verb \isi{word order}. When the verb is preceded by a~particle, such as the {accomplished} \is{prefixes}prefix
/\ipa{le˧}-/ or the {durative} \is{prefixes}prefix /\ipa{tʰi˧}-/, the noun and the verb belong to two different tone
groups, and they do not interact. Compare, for instance, the realizations of the MH"=tone verb
/\ipa{qæ˧˥}/ ‘to burn’ in /\ipa{mv̩˧ {\kern2pt}|{\kern2pt} le˧-qæ˧-ze˥}/ ‘the fire burned’ and /\ipa{mv̩˧ qæ˩}/ ‘the fire
burns’.

The elicited data analyzed in this section is based on intransitive verbs, so as to avoid possible
confusions between S+V and O+V constructions. Following the same procedure as for
nouns, two contexts were used to arrive at underlying tone categories: S+V, and
S+V+{perfective}. For instance, (\ref{ex:theguesthasarrived}) without the \textsc{pfv} yields /\ipa{hĩ˧-bæ˧
  tsʰɯ˧˥}/ ‘the guests arrive’. The tone pattern for this combination of subject and predicate can therefore be described as /M.M.MH/, and further analyzed as \mbox{//MH\#//}: a~MH \is{tonal contour}contour associating to the last
syllable.


\subsection{The facts}
\label{sec:thefactssubjectandverb}

For systematic elicitation, the following verbs were used: /\ipa{se˥}/ ‘to walk’, /\ipa{ʂɯ˧\textsubscript{a}}/ ‘to die’, /\ipa{tsʰo˧\textsubscript{b}}/
‘to jump’, /\ipa{tsʰɯ˩\textsubscript{a}}/ ‘to come.\textsc{pst}’, /\ipa{ʐwɤ˩\textsubscript{b}}/ ‘to speak’, and
/\ipa{bæ˧˥}/ ‘to run’. The nouns used were kinship terms and animal names. \tabref{tab:thetonepatternsofsubjectplusverbcombinationsinsurfacephonologicaltranscription} presents the
results. The \mbox{//LM//} and \mbox{//LH//} tone categories of monosyllables always yield the same output, so they are pooled
together in the table.

When the subject"=plus"=verb combination ends in a~/H/ or /L/ tone, the {perfective} morpheme carries
/L/ tone. When it ends in a~/MH/ tone, the postverbal morpheme receives the /H/ part of the \is{tonal contour}contour. When it ends on
a~/M/ tone, the tone of the postverbal morpheme cannot be predicted; in these cases, it is indicated in the table, preceded by a~‘+’
sign.

The recording SubjectVerb contains all of the combinations in \tabref{tab:thetonepatternsofsubjectplusverbcombinationsinsurfacephonologicaltranscription}.


\begin{sidewaystable}[p]
\caption{\label{tab:thetonepatternsofsubjectplusverbcombinationsinsurfacephonologicaltranscription}The tone patterns of subject"=plus"=verb combinations, in
  surface phonological transcription.}
\begin{tabularx}{\textheight}{ l@{\hspace{6mm}} Q l@{\hspace{6mm}} l@{\hspace{6mm}} l@{\hspace{6mm}} l@{\hspace{6mm}} Q }
\lsptoprule
& tone of verb & & & & &\\ \cmidrule{2-7}	
tone of noun & H & M\textsubscript{a} & M\textsubscript{b} & L\textsubscript{a} & L\textsubscript{b} & MH\\ \midrule
	LM, LH & L.H & L.M+M & L.M+M & L.H & L.H & L.MH\\
	M & M.M+L & M.M+M & M.M+M & M.L & M.L & M.MH\\
	L & M.M+L & L.L  & M.M+M & L.L & L.L~/ M.L & L.L\\
	H & M.M+L & M.M+L & M.M+L & M.MH & M.MH & M.L\\
	MH & M.H & M.H & M.H & M.MH & M.MH & M.H\\ \addlinespace \hdashline \addlinespace
	M & M.M.M+L & M.M.M+M & M.M.M+M & M.M.L & M.M.L & M.M.MH\\
	\#H & M.M.M+L & M.M.M+L & M.M.M+L & M.M.MH & M.M.MH & M.M.L\\
	MH\# & M.M.MH & M.M.MH & M.M.MH & M.M.MH & M.M.MH & M.M.H\\
	H\$ & M.M.M+L & M.M.M+L & M.M.M+L~/ M.M.M+H & M.M.MH & M.M.MH & M.H.L\\
	L & L.L.L & L.L.L & L.L.L & L.L.L & L.L.L & L.L.H\\
	L\# & M.L.L & M.L.L & M.L.L & M.L.L & M.L.L & M.L.L\\
	LM+MH\# & L.M.M+L & L.M.M+L & L.M.M+L & L.M.MH & L.M.MH & L.M.H\\
	LM+\#H & L.M.M+L & L.M.M+M & L.M.M+M & L.M.L & L.M.MH & L.M.MH\\
	LM & L.M.M+L & L.M.M+M & L.M.M+M & L.M.L & L.M.L & L.M.MH\\
	LH & L.H.L & L.H.L & L.H.L & L.H.L & L.H.L & L.H.L\\
	H\# & M.H.L & M.H.L & M.H.L & M.H.L & M.H.L & M.H.L\\
\lspbottomrule
\end{tabularx}
\end{sidewaystable}


About one fourth of the tone patterns for subject"=plus"=verb phrases differ from the corresponding
object"=plus"=verb phrases. Among identical combinations are those where the noun has a~//L\#// or \mbox{//H\#//}
tone, since these fixed"=position tones lower the tones of all following syllables to L (by application of Rules~4 and 5: see \sectref{sec:alistoftonerules}).

As with other types of combinations, such as \is{numerals}numeral"=plus"=classifier and object"=plus"=verb, the tone
patterns in \tabref{tab:thetonepatternsofsubjectplusverbcombinationsinsurfacephonologicaltranscription} cannot be obtained on the basis of a~set of phonological tone rules. Among the more
surprising patterns is \mbox{//LH//} plus \mbox{//LM//}, yielding L.M.M+H, as in /\ipa{bo˩ɬɑ˧ ʈʰɯ˧-ze˥}/ ‘the boar drank’. The noun has a~\mbox{//LH//} lexical
tone, which would be expected to cause \isi{neutralization} of all tonal contrasts on the following morpheme (here, on the verb). The five
other combinations involving a~\mbox{//LH//}"=tone noun do indeed yield L.H.L, but that with a~//L\textsubscript{b}//"=tone verb yields L.M.MH. This pattern
is conspicuously unrelated to the phonological tendencies observed in the language.

In subject"=verb combinations as elsewhere, the //L.M// and //L.H// sequences are neutralized on the
surface. Notation as //L.H// is interchangeable with //L.M+L//. The former was chosen for the sake of
descriptive simplicity: it only requires two tone symbols, and it corresponds to one of the tones
attested on disyllabic nouns. (The argument for using //L.H// rather  than //L.M.L// as a~label for this category of disyllabic nouns was set out in \sectref{sec:twooptionsforanalysislmvslhorlmvslml}.)

An especially interesting issue is how the {perfective} acquires its tone after a~M-tone verb. The interpretation proposed here is that structural \is{gap-filling}gap"=filling in the tonal paradigm has taken place, with far"=reaching consequences for the system. This issue will be analyzed in the chapter devoted to the analysis of Yongning Na in dynamic perspective: Chapter \ref{chap:yongningnatonesinadynamicsynchronicperspective}, \sectref{sec:howthesuffixacquiresitslmorhtoneafteramtoneverb}. 


\subsection{Variants resulting from a~division into two tone groups}
\label{sec:variantsresultingfromadivisionintotwotonegroups}


Some deviant patterns are observed in recorded data when the association of subject and verb is, as
it were, incomplete: both the subject and the verb receive the tone that they would have in
isolation. For example, ‘the tiger jumped’ can be realized as /\ipa{lɑ˧ {\kern2pt}|{\kern2pt} tsʰo˧-ze˩}/ instead of the regular /\ipa{lɑ˧ tsʰo˧-ze˧}/. This amounts to a~division into two tone groups, as
transcribed by the vertical bar \ipa{{\kern2pt}|{\kern2pt}}. There are borderline cases, however, without a~true division into
two tone groups. If the two parts (the subject on the one hand, the verb plus its \is{suffixes}suffix on the other) were really treated as two tone groups, a~post"=lexical H tone would be added to all-L groups such as the subject /\ipa{bo˩}/ ‘pig’ in (\ref{ex:pigjumped}) and the verb phrase
/\ipa{tsʰɯ˩-ze˩}/ ‘came’ in (\ref{ex:sheeparrived}).

\begin{exe}
	\ex
	\label{ex:pigjumped}
	\ipaex{bo˩ {\kern2pt}|{\kern2pt} tsʰo˧-ze˩.}\\
	\gll bo˩˧	tsʰo˩\textsubscript{b}	-ze˧\textsubscript{b}\\
	pig		to\_jump		\textsc{pfv}\\
	\glt ‘The pig jumped.’
\end{exe}

\begin{exe}
	\ex
	\label{ex:sheeparrived}
	\ipaex{jo˧ {\kern2pt}|{\kern2pt} tsʰɯ˩-ze˩.}\\
	\gll jo˩	tsʰɯ˩\textsubscript{a}	-ze˧\textsubscript{b}\\
	sheep		to\_come.\textsc{pst}		\textsc{pfv}\\
	\glt ‘The sheep have come.’
\end{exe}
		
Such variants exist for all combinations. When
subject and verb are thus separated, it is possible to make a~pause before the verb. The \is{stylistics}stylistic
effect is to emphasize the subject. The topic of \is{stylistics}stylistic options in the division of the utterance
into tone groups, which has great importance in Yongning Na \isi{prosody}, is addressed in Chapter~\ref{chap:toneassignmentrulesandthedivisionoftheutteranceintotonegroups}.

When the demonstratives //\ipa{ʈʂʰɯ˥}// and //\ipa{tʰv̩˥}// appear in subject position, they are always separated from the verb by a \isi{tone group} \is{boundary (between tone groups)}boundary, as illustrated by example (\ref{ex:hehasarrived}).

\begin{exe}
	\ex
	\label{ex:hehasarrived}
	\ipaex{ʈʂʰɯ˧ {\kern2pt}|{\kern2pt} tsʰɯ˩-ze˩˥!}\\
	\gll ʈʂʰɯ˥ 	tsʰɯ˩\textsubscript{a}		-ze˧\textsubscript{b}\\
	3\textsc{sg}		to\_come.\textsc{pst}	\textsc{pfv}\\
	\glt ‘(S)he has come!’
\end{exe}

This is unlike H-tone nouns, which tend to be integrated in the same \isi{tone group} as the verb, as in /\ipa{ʐwæ˧ tsʰɯ˧˥}/ ‘the horse came’, from //\ipa{ʐwæ˥}// ‘horse’ plus the same verb ‘to come’ as in (\ref{ex:hehasarrived}). The peculiar status of demonstratives was also noted when exploring other nooks and crannies of the Yongning Na morphotonological system (see \sectref{sec:encliticsthatcarrymtonewhenfollowingamtonenoun}, in particular).

\newpage 
\subsection[Noun plus existential verb behaves tonally like S+V]{Nouns plus the \is{existentials}existential verb /\ipa{dʑo˩\textsubscript{b}}/ behave tonally like subject"=verb combinations}
\label{sec:nounsplustheexistentialverbbehavetonallylikesubjectverbcombinations}

The \is{existentials}existential verb /\ipa{dʑo˩\textsubscript{b}}/ patterns like other L\textsubscript{b}-tone verbs. The association of a~noun and
the \is{existentials}existential behaves tonally like the combination of a~subject and a~verb (shown in \tabref{tab:thetonepatternsofsubjectplusverbcombinationsinsurfacephonologicaltranscription}
above). The data is shown in \tabref{tab:thetoneoftheexistentialverbinassociationwithanoun}.

\begin{table}%[t]
\caption{\label{tab:thetoneoftheexistentialverbinassociationwithanoun}The tone of the existential verb /\ipa{dʑo˩\textsubscript{b}}/ in association with a~noun.}
\begin{tabularx}{\textwidth}{ l@{\hspace{7mm}} l@{\hspace{7mm}} l@{\hspace{7mm}} Q l }
\lsptoprule
	example & meaning & tone & with \is{existentials}existential & tone pattern\\ \midrule
	\ipa{bo˩˧} & pig & LM & \ipa{bo˩ dʑo˥(-ze˩)} & L.H\\
	\ipa{ʐæ˩˥} & leopard & LH & \ipa{ʐæ˩ dʑo˥(-ze˩)} & L.H\\
	\ipa{lɑ˧} & tiger & M & \ipa{lɑ˧ dʑo˩} & L.M\\
	\ipa{jo˩} & sheep & L & \ipa{ɬi˧ dʑo˩} & M.L\\
	\ipa{ʐwæ˥} & horse & H & \ipa{ʐwæ˧ dʑo˧˥} & M.MH\\
	\ipa{ʈʂʰæ˧˥} & deer & MH & \ipa{ʈʂʰæ˧ dʑo˧˥} & M.MH\\
	\addlinespace \hdashline \addlinespace
 	\ipa{ɖɤ˧mi˧} & fox & M & \ipa{ɖɤ˧mi˧ dʑo˩} & M.M.L\\
	\ipa{ʐwæ˧zo\#˥} & colt & \#H & \ipa{ʐwæ˧zo˧ dʑo˧˥} & M.M.MH\\
	\ipa{hwɤ˧li˧˥} & cat & MH\# & \ipa{hwɤ˧li˧ dʑo˧˥} & M.M.MH\\
	\ipa{hwɤ˧mi˥\$} & she"=cat & H\$ & \ipa{hwɤ˧mi˧ dʑo˧˥} & M.M.MH\\
	\ipa{kʰv̩˩mi˩} & dog & L & \ipa{kʰv̩˩mi˩ dʑo˩˥} & L.L.L\\
	\ipa{dɑ˧ʝi˩} & mule & L\# & \ipa{dɑ˧ʝi˩ dʑo˩} & M.L.L\\
	\ipa{õ˩dv̩˧˥} & wolf & LM+MH\# & \ipa{õ˩dv̩˧ dʑo˧˥} & L.M.MH\\
	\ipa{nɑ˩hĩ\#˥} & Naxi & LM+\#H & \ipa{nɑ˩hĩ˧ dʑo˧˥} & L.M.MH\\
	\ipa{bo˩mi˧} & sow & LM & \ipa{bo˩mi˧ dʑo˩} & L.M.L\\
	\ipa{bo˩ɬɑ˥} & boar & LH & \ipa{bo˩ɬɑ˥ dʑo˩} & L.H.L\\
	\ipa{hwæ˧tsɯ˥} & rat & H\# & \ipa{hwæ˧tsɯ˥ dʑo˩} & M.H.L\\
\lspbottomrule
\end{tabularx}
\end{table}

A subtle difference between the \is{existentials}existential verb /\ipa{dʑo˩\textsubscript{b}}/ and other L\textsubscript{b}-tone verbs is that the
association of other verbs with a~L-tone subject noun yields L.L, with a~M.L \is{variants}variant for some nouns
only, whereas the \is{existentials}existential verb only allows the M.L pattern (for all nouns carrying lexical L
tone).



