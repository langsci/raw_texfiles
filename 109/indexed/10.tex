\chapter{Typological perspectives}
\label{chap:arealandtypologicaldiscussion}

%\subsection{Typology and universals}
%\label{sec:typologyanduniversals}
%
%Linguistic typology is central to the series ``Studies in diversity linguistics'' which hosts this volume. The ``Aims and scope'' of the series fit snugly with the approach adopted here: 
%
%\begin{quotation} 
%This book series will publish book"=length studies on individual less"=widely studied languages (especially, but not only reference grammars), as well as works in broadly comparative typological linguistics that takes into account the world"=wide diversity of human languages. Work on individual languages and broadly comparative work is of a~different nature, but this book series sees the two as closely related: comparative studies need in"=depth work on individual languages from around the world to build on, and descriptive work is done best in a~comparative perspective.
%\end{quotation}

The following statement sets a~possible stage for typological work:

\begin{quotation} [L]anguages may differ at virtually all levels in their process of categorisation~-- not only in
	how they group sounds into emic categories (phonemes) but also in the way their particular
	constraints group these phonemes into meta"=categories (classes of phonemes). These constraints, in
	turn, have to be defined system"=internally, even when they derive from such supposedly universal
	parameters as sonority. \citet[129]{haspelmath2007} reminds us that “structural categories of language
	are language"=particular, and we cannot take pre"=established, \textit{a~priori} categories for
	granted”. Such a~stance does not rule out the possibility of universal generalisations, but
	entails that they can only be based on the empirical study of language"=internal structures, and
	the acknowledgment of cross"=linguistic diversity.~\citep[429]{francois2010}
\end{quotation}

%Command \noindent added to avoid having an indent. Proofreader suggestion: since this sentence continues the argument, it is better not to indent. 
{\noindent}In this citation, François reminds us that every language has its own emic categories, which can only be discovered through an in"=depth, language"=internal analysis. Universals should not be assumed aprioristically: they are to be investigated empirically, and explored through the careful comparison of languages.​ ​This view does not amount to a~relativistic claim​ that each language calls for its own distinct concepts​: rather, it opens up a~programme for comparative work that emphasizes cross"=language similarities in functional terms, and in terms of evolutionary potential, instead of static characteristics. 


\section{Tonal typology}
\label{sec:typologicalperspectives}

In this section, clarifications are provided about the typological distinction between “level tones” and “complex tones”, which constitutes the background to the classification of Yongning Na tones as “level tones” (\sectref{sec:typologicalbackgroundtotheclassificationofyongningnatonesasleveltones}). Some reflections are then set out (\sectref{sec:tonalrules}) concerning the typological profile of Na \isi{prosody} as shaped by the tone rules described in the preceding chapters. 

\subsection{Level tones and phonetically complex tones}
\label{sec:typologicalbackgroundtotheclassificationofyongningnatonesasleveltones}

\is{level tones|textbf}
\is{level tones|(}
\is{complex tones|(}

In what can be broadly termed as “Africanist” usage, “\is{level tones}level tone”
refers to \textit{a~tone that is
  defined simply by a~discrete level of relative pitch}. Level tones (a phrase used interchangeably with “tonal levels”) are monodimensional: they are defined along a~single parameter, F\textsubscript{0}. This is unlike segments, which are defined along intersecting phonetic parameters, such as voicing, nasality, place of articulation, etc. For want of free combinability of multiple properties, level tones are not further analyzed in terms of features: level tones constitute phonological primitives (\citealt[20]{clementsetal2011}; \citealt{hyman2011c}).

Level"=tone systems have two to five levels
of relative pitch: L vs.\ H; L vs.\ M vs.\ H; L vs.\ M vs.\ H vs T(op); or B(ottom) vs.\ L vs.\ M vs.\ H
vs.\ T(op). Systems with more than three levels are relatively uncommon (Bariba:
\citealt{welmers1952}; Bench, also known as Gimira: \citealt{wedekind1983,wedekind1985}). One single case
of a~six"=level system has been reported: Chori \citep{dihoff1977}, for which a~reanalysis is possible
\citep{odden1995}. Bariba, Bench and Chori are spoken in Subsaharan Africa, an area where level tones
are especially common. However, level"=tone representations have proved useful beyond the Subsaharan
area, for which they were initially developed (on languages of the Americas:
\citealt{gomezimbert2001}; \citealt{hargusetal2005}; \citealt{gironhiguitaetal2007};
\citealt{michael2010}; on languages of Asia: \citealt{ding2001};
\citealt{hymanetal2002a,hymanetal2004}; \citealt{donohue2003,donohue2005}; \citealt{evans2008a};
\citealt{jacques2011a}). In Yongning Na, the morphotonological alternations studied in
the preceding chapters provide overwhelming evidence for a~level"=tone analysis. \is{tonal contour|textbf}In level"=tone
systems, a~phonetic {contour} is \textit{the realization of two or more level tones on a~single
  syllable}. The contours are phonologically decomposable; the observed movement in F\textsubscript{0} is the result
of interpolation between the successive levels.

There are some languages for which attempts at the decomposition of contours into levels have been
less successful, however, to the point of casting doubt on the relevance of decomposition for these
languages. In the Austroasiatic and \il{Tai-Kadai}Tai"=Kadai language families, no convincing evidence is found for
the decomposition of contours into simpler units \citep[e.g.][639]{morey2014}. 
%%Quotation not fully appropriate here because it focuses on BINARY tone.
%\begin{quotation}
%	I do not find the idea of binary features necessary or helpful in analysing the languages that I
%	have worked on. In these languages, I do not believe that reducing the analysis of tones
%	to a~binary choice of H and L will assist in the understanding of the tonal system.
%\end{quotation}
In the field of \ili{Sinitic} languages (\il{Sinitic}Chinese dialects), Chao Yuen"=ren’s work on \ili{Mandarin} in the
early 20\textsuperscript{th} century \citep{chao1929,chao1933} brought out the complexities
of its tone system. Following sustained exchanges with Chao Yuen"=ren, Kenneth Pike proposed
a~typological divide between (i)~register"=tones, defined simply in terms of
discrete pitch levels, and (ii)~{contour}"=tones, about which he concludes: “the glides of a~{contour}
system must be treated as unitary tonemes and cannot be broken down into end points which constitute
lexically significant contrastive pitches” \citep[10]{pike1948}. Later studies have emphasized the
importance of phonation"=type characteristics. In some prosodic systems,
phonation types are simply a~low"=level phonetic characteristic that occasionally accompanies tone (see, for example, an investigation into the effect of
creaky voice on Cantonese tonal perception: \citealt{yuetal2014}). In others, they
are a~distinctive feature orthogonal to tone, as in the Oto"=Manguean languages Mazatec
\citep{garelleketal2011} and Trique \citep{dicanio2012}. Finally, in a~third type of system,
phonation"=type characteristics are part and parcel of the definition of tones. Experimental studies
of this third type of tone system include \citet{rose1982,rose1989a,rose1990} for the Wu branch of
\ili{Sinitic}, \citet{edmondsonetal2001} for \ili{Yi} and Bai, \citet{mazaudonetal2008} for \ili{Tamang}, and
\citet{andruskietal2000}, \citet{andruskietal2004}, \citet{kuang2013} for \ili{Hmong}. 

\begin{quotation}
	Languages such as Black Miao and \ili{Vietnamese} highlight the difficulty of drawing a~line in 
	the sand separating ‘tone’ languages from ‘register’ languages. This problem is even more 
	strikingly illustrated by \ili{Burmese} ({\dots}), which has been described both as a~register system 
	\citep[e.g.][]{bradley1982, jones1986} and as a~tone language \citep[e.g.][]{watkins2001a, gruber2011} ({\dots}). \citet{gruber2011} has shown that glottalisation, creakiness and the presence of a~high pitch target are all important perceptual cues, thus demonstrating that, much like \ili{Vietnamese} or \il{Hmong}Black Miao, \ili{Burmese} should not be analyzed in terms of pitch or \is{phonation types}phonation type alone, nor is it straightforward to decide which property is the primary acoustic cue to the contrast. \citep[194]{brunelleetal2016}
\end{quotation}

Pike’s two"=way typology of tone,
while it emphasizes typologically relevant properties of those languages which he was able to take
into consideration, has some limitations when it comes to characterizing tones such as those of \ili{Vietnamese}. In the \ili{Vietnamese} system, the tones contrast with one another through a~set of characteristics that include specific phonation types in addition to
the time course of F\textsubscript{0}
\citep{alves1995,mixdorffetal2003,brunelleetal2010,nguyenetal2013,macetal2015}; for such tones, characterization as “{contour} tones” sounds underspecific.
For this reason, the term “complex tones” is used here in preference to Pike’s “{contour} tones”. To
recapitulate the terms used in the present discussion: \textit{complex"=tone systems} are distinguished from \textit{level"=tone systems} (based on
discrete levels of relative pitch). \is{complex tones|textbf}Complex tones include
Pike’s category of “{contour} tones”, with the explicit addition of tones that comprise phonation"=type
characteristics.\footnote{Note that this differs from the definition used in the \textit{World Atlas of
  Language Structures}, where “complex” refers to the number of oppositions, not to the nature of the
  tones: “[t]he languages with tones are divided into those with a~simple tone system~-- essentially
  those with only a~two"=way basic contrast, usually between high and low levels~-- and those with
  a~more complex set of contrasts” \citep{Maddieson2011}.}

Under this set of definitions, “{contour}” refers to a~unitary {contour}: a~tone defined phonologically
in terms of an~overall template specifying the time course of F\textsubscript{0} over the tone"=bearing
unit. Phonologically unitary {contour} tones are encoded as an~overall shape: “there are no objective
reasons to decompose \ili{Vietnamese} tone contours into level tones or to reify phonetic properties like
high and low pitch into phonological units such as H and L” (\citealt[94]{brunelle2009a}; see also
\citealt{brunelleetal2010,kirby2010}). In this type of system, the term “\is{level tones}level tone” is used to
refer to \textit{a tone that does not exhibit any salient fluctuations in F\textsubscript{0}}. For instance,
\ili{Mandarin} tone 1 and \ili{Vietnamese} tone 1 (orthographic \textit{ngang}) can be referred to as “level
tones” because, unlike the other tones of \ili{Mandarin} and \ili{Vietnamese}, their F\textsubscript{0} curve is relatively
stable in the course of the syllable. This does not entail that they are phonologically defined by
a~discrete level of relative pitch (on \ili{Mandarin}: see \citealt{xuetal2001}).

The two types of phonological \is{tonal contour}contour tones~-- sequences of levels on the one hand, unitary contours
on the other~-- can be \is{phonetic realization of tones}
phonetically indistinguishable, so that phonetic observation must be related
to functional"=structural levels of description. The evidence for distinguishing the two types of
contours is morphotonological. A~rising {contour} in a~\ili{Bantu} language will typically exhibit
phonological behaviour showing that it consists of a~low \is{level tones}level tone followed by a~high \is{level tones}level tone
\citep{clementsetal1984,clementsetal2007}. In Yongning Na, too, there is a~wealth of evidence for the analysis of \is{tonal contour}contour tones into sequences of level
tones. From a~typological point of view, instead of positing that all tones can be decomposed into
levels, it is at least as reasonable to adopt the opposite standpoint, viewing contours as
nondecomposable units unless there is positive evidence to the contrary (Nick Clements, p.c.\ 2008).


Tonal systems thus based on levels (tone heights) are relatively unusual in \il{Sino-Tibetan}Sino"=Tibetan, but not
unheard of. Examples include \ili{Pumi} \citep{matisoff1997a,ding2006,jacques2011a,daudey2014}, Cone Tibetan \citep{sun2003b,jacques2014b}, Mianchi Qiang
\citep{evans2008a}, \ili{Shixing}~/ Xumi \citep{chirkovaetal2009}, Hakha Lai \citep{hymanetal2002a},
Kuki"=Thaadow \citep{hyman2010b}, and the Lataddi dialect of Na \citep{dobbsetal2016}. 
 
The distinction
between level tones and complex tones is proposed as a~rule"=of"=thumb distinction; it aims to
bring attention to a~considerable amount of interesting Asian data that is likely to lie below the
radar of some prosodic typologists and which deserves to be more widely appreciated. Needless to say, the two"=way distinction between level tones and complex tones is by no means
water"=tight: there are borderline situations. 
\is{level tones|)}
\is{complex tones|)}

The following subsection attempts to convey a~feel for the organization of the Na prosodic system by pointing out some consequences of its tone rules for the outlook of this level"=tone system.


\subsection[Typological profile of Na prosody]{Typological profile of Na prosody as shaped by the tone rules}
\label{sec:tonalrules}

\is{tone rules}

One of the salient characteristics of Yongning Na is the partial \isi{neutralization} of lexical oppositions when words are said \is{form!in isolation}in isolation. This does not appear to have far"=reaching implications for the organization of the entire tonal system, however: such \isi{neutralization} is observed in numerous prosodic systems which differ widely from one another, such as \ili{Japanese} \citep{kubozono1993}, San Juan Quiahije Chatino
(Oto"=Manguean family) \citep[91]{cruz2011}, and Sotho and Tswana (\ili{Bantu}) \citep{creisselsetal1997, zerbianetal2010b, zerbian2016}. 

On the other hand, the levelling rules of Yongning Na (Rules 4 and 5), whereby all tones following a~H tone, or a~M.L sequence, are lowered to L, have far"=reaching consequences for \is{form!surface}surface phonological tone sequences. These two rules result in the \isi{neutralization} of tonal oppositions over large portions of tone groups~-- a~massive phenomenon of levelling that is reminiscent of stress systems in which all syllables following a~major stress are de"=stressed. This rule alone makes Yongning Na tone strikingly different from the extensive set of tone systems called `terraced"=\is{level tones}level tone languages' \citep{armstrong1968}. `Terracing' refers to two processes of categorical shift in register, affecting all following tones: \textit{downstep}, a~distinctive lowering; and \textit{upstep}, a~distinctive raising. An important consequence of terracing is that tones belonging to the same `terrace' hang together more closely than the successive tones in a~language which, like Na, does not have \isi{downstep} or upstep. To use an image from weaving, one could say that tones belonging to the same terrace are tightly knit together; to use an image from woodwork, they could be said to be pegged together. This intuition is reflected in Nick Clements's proposed treatment of `terracing' in terms of a~``tone level frame''. ``Within this framework, terracing is seen to be the result of ({\dots}) processes applying to the tone level frame itself, rather than directly to individual tones'' \citep[538]{clements1979}. `Terracing' places constraints on the range of fundamental frequency within which the tonal levels are realized, as shifts in register are distinctive. It makes a~major contribution to shaping surface phonological tone sequences and their phonetic realization. Clements points out a~key factor: categorical shift in register can take place more than one time in a~\isi{tone group}. 

\begin{quotation}
	[An] important feature of tone terracing, at least in the case of \isi{downstep}, is that there is no limit on the number of register lowerings that may occur within a~\isi{tone group}; the only limit is the external one imposed by the lexical, grammatical, or phonological factors that govern the occurrence of \isi{downstep}. \citep[540]{clements1979}
\end{quotation}

%Command \noindent added to avoid having an indent. Proofreader suggestion: since this sentence continues the argument, it is better not to indent. 
{\noindent}This leads to preplanning strategies that can get highly elaborate: brilliant speakers anticipate the amount of downsteps that will be required in a~long utterance, raise the initial pitch accordingly, and manage successive lowerings all the way to the end of the utterance. Less talented orators need to reset their F\textsubscript{0} when successive downsteps make them hit bottom before they reach the end of a~\isi{tone group} \citep{rialland2001}.

In Yongning Na, there is no \isi{downstep}, and hence no need for such long"=distance preplanning. Whenever a~\isi{tone group} contains a~H tone, this tone serves as the \isi{tone group}'s climax. In terms of information processing, in cases where a~H tone is identified, following syllables in the \isi{tone group} contain no more tonal information: there is nothing to expect but a~sequence of phonological L tones. Phonetically, the pitch gradually lands towards its floor value; the realization of the F\textsubscript{0} curve on the portion of the \isi{tone group} that follows the H tone is free from the trammels of categorical precision.

To summarize, the `tone level frame' (Nick Clements's term to refer to the tone space at a~given point in an~utterance) is subject to far fewer constraints in Yongning Na than in `terracing' tone languages. The absence of \isi{downstep} or upstep in Yongning Na, and the culminative nature of its H tone, go a~long way towards explaining the different feel of its \isi{prosody} compared to that of `terracing' tone languages. There are simply fewer possible tone sequences in Yongning Na than in, for instance, Yala (Ikom), which has H, M and L tones, like Na, but also has the downstepped counterparts !H, !M and !L, and allows the full range of their combinations \citep{armstrong1968}. In Yongning Na, there is no contrast between a~fall from M to L and one from H to L. This gives a~greater range of phonetic freedom than in languages where the tonal space is more crowded. (The greater phonetic freedom found in Na is exploited for intonational purposes, as explained in \sectref{sec:pragmaticintonation}.)

{\largerpage}

Like \isi{downstep}, \textit{downdrift}~-- the gradually lower phonetic realization of phonologically identical tones separated by a~lower tone~-- is absent in Yongning Na, for the same reason: the two sequences of a~higher tone and a~lower one are H.L and M.L, both of which constitute a~descent to the lowest phonological level, and these sequences can only be followed by L tones (by Rule~5), thus prohibiting sequences such as $\ddagger${\kern2pt}M.L.M, $\ddagger${\kern2pt}H.L.H or $\ddagger${\kern2pt}H.M.H. This is another important trait of the typological profile of Yongning Na.


\section{Assessing the complexity of the Na tone system}
\label{sec:morphophonologicalcomplexity}

\is{complexity|textbf}

In comparison with \ili{Naxi} and \ili{Laze}, its immediate siblings in the \ili{Naish} subgroup of \il{Sino-Tibetan}Sino"=Tibetan (\sectref{sec:thepositionofnaandnaxiwithinsinotibetan}), Na presents a~high degree of tonal complexity. It has more lexical tone categories, and greater morphotonological complexity.\footnote{These two variables~-- the number of tonal contrasts, and the number of tonal rules~-- are proposed as the two main dimensions of tonal complexity in an article discussing methods for measuring the degree of complexity of a~tone systems \citep{konoshenko2014}.} A~task for the future will consist in assessing this complexity by modelling regularities and irregularities in the paradigms that make up Na morphotonology. %In this endeavour, it will be possible to build on advances in computational tools (for an example in the field of segmental morphology: \citealt{sagotetal2013}). 
As a~stepping"=stone towards this mid- and long"=term goal, some dimensions of this complexity are recapitulated below and compared with other languages, not on the basis of phylogenetic or areal closeness but of synchronic typological similarities.

\subsection{Partly regular morphotonology}
\label{sec:partlyregularmorphotonology}

Partly regular morphological paradigms are cross"=linguistically widespread. Examples include the inflection of transitive verbs in Dinka \citep[8]{andersen1993} (while certain inflections are marked by a~particular toneme for all
verbs alike, other inflections are specific to particular classes of verbs) and the inflection of interrogative pronouns in the Australian language Angu\-thimri \citep[172]{crowley1981}. 

Within the tone system of Na, \is{numerals}numeral"=plus"=classifier\is{classifiers} phrases
constitute an area where tone patterns have proliferated. The description set out in Chapter~\ref{chap:classifiers} brought to light no fewer than nine tonal categories for {monosyllabic} classifiers, as opposed to five for {monosyllabic} nouns. Furthermore, the tone patterns of these nine categories of classifiers in combination with numerals need to be learnt: they do not follow from synchronically regular rules. While this complexity is not as spectacular as that found in the Ahmao language (\ili{Hmong}"=Mien family), where classifiers have “12
basic forms, each displaying a~complex cluster of meanings” \citep{gerneretal2009}, the Na data may
nonetheless have a~contribution to make to typological generalizations, as it shows that the tones of classifiers can be more complex than those of nouns. 


\subsection{Nouns and verbs: A~comparable degree of complexity?}
\label{sec:limitationsontonaloppositions}

Keeping in mind that some types of nouns (especially classifiers) are more complex than others in terms of their tone categories, it seems that there is no conspicuous imbalance between Na nouns and verbs in terms of tonal complexity. This differs from tonal systems in \ili{Bantu}, and in the Niger"=Congo family at large, where verbs display less diversity of tone categories than nouns. Many \ili{Bantu} languages have two tonal types of verbs (irrespective of their number of syllables), versus three or more types of {monosyllabic} nouns and an~even greater number for nouns of two syllables and more \citep[183]{creissels1994}. Gbeya, Kissi, Baoulé and Urhobo do not have tonal oppositions among verbs at all \citep[184]{creissels1994}.

\subsection{More progressive spreading than regressive spreading: A~typologically common pattern}
\label{sec:propagationanticipation}

Under the analysis proposed here, Yongning Na has a~phonological tone rule (Rule~1) whereby L tone spreads progressively (‘left"=to"=right’) onto syllables that are unspecified for tone. The H tone does not spread, in the sense of associating to several syllables in a~row: there can only be one H tone per \isi{tone group}. Despite this, the presence of a~H tone does influence the following tones in the group: they all get lowered to L. Although the morphotonological rules are context"=specific, and cannot be summarized in terms of a~set of phonological rules, they also reveal an~overall tendency towards progressive tone \is{tone spreading}spreading, rather than the other way round. For instance, averaging over the entire set of tone rules applying in determinative compounds (\sectref{sec:determinativecompoundnounsII}), the determiner (which comes first) makes a~larger contribution than the head to the tone of the entire \is{compounds}compound. Seen in this light, Na clearly has more progressive \is{tone spreading}spreading than regressive \is{tone spreading}spreading of tone. 

This is a~typologically unsurprising pattern. Regressive tone \is{tone spreading}spreading is well"=attested, for instance in Tswana and Odienné Dioula, and in Kikwere as analyzed by \citet[177-178]{odden1998b}, but progressive tone \is{tone spreading}spreading is more common \citep[206-207]{creissels1994}. 

A separate but not wholly unrelated observation is that H tone in Yongning Na has a~tendency to be realized late within a~\isi{tone group}. A~H tone never associates to the first syllable within a~\isi{tone group}. \mbox{//H\#//} tone associates to the last syllable of the root; so does \mbox{//H\$//}, but it typically glides from this position to a~later syllable (\sectref{sec:wordfinalandmorphologicalnucleusfinalHtones}); and \mbox{//\#H//} tone never associates to the word to which it is lexically attached, only to a~later syllable (\sectref{sec:afloatinghtonewithcomparativeevidencepointingtoitsorigin}). This overall tendency is relatively common cross"=linguistically. Late realization of H targets is more common than early realization: ``perseverative tone \is{tone spreading}spreading phonologises the tendency of tone targets to be realized late'' \citep[19]{hyman2007d}. The {diachronic} developments leading to the present diversity of final H tones in Yongning Na seem to follow tendencies that are well"=attested across languages. This is a~case where synchronic complexity is not particularly surprising when viewed from the perspective of the typology of language change (i.e., from a~\textit{panchronic} perspective: see \sectref{sec:theoreticalbackdrop}).

\subsection{The dual status of the M tone is not a~typological rarity}
\label{sec:thestatusofmtone}

Under the present description, the M tone in Yongning Na has two facets. On the one hand it is a~full"=fledged, phonologically specified tone: the M element in tone categories such as LM and MH cannot be omitted. LM contrasts with L, and MH\# with H\#. On the other hand, the M tone serves as a~default tone: by Rule 2, M tone is assigned to syllables that remain toneless after Rule 1 (L-tone \is{tone spreading}spreading onto toneless syllables) has applied. 

Na is not an~isolated case in this respect. In
Yorùbá, too, the M tone has a dual status. M tone is not lexically specified: the only two lexical tones are L and H,
but following its insertion through default"=tone assignment rules, M behaves as a~specified
tone in \is{derivation!tonal}derivations \citep{akinlabi1985}. 


\section[Tonal vs.\ non"=tonal intonation]{Intonational typology: Tonal vs.\ non"=tonal intonation}
\label{sec:intonationaltypology}

\is{tonal intonation|textbf}

This last section of the typological discussion is devoted to intonational typology. The argument is that \textit{tonal intonation} and \textit{non"=tonal intonation} need to be carefully distinguished, and that Yongning Na does not have \isi{tonal intonation}. This is not a~particularly complex argument, and its conclusion seems fully clear to me, but the present state of befuddlement in the field of intonation studies requires step"=by"=step exposition of the typological premises.

\subsection{Instances of intonational tones in the world’s languages}
\label{sec:instancesofintonationaltonesintheworldslanguages}

There are some well"=established cases where \isi{intonation} is encoded by tones that are treated on a~par
with lexical and morphological tones: in some tonal languages, tone can serve as a~marker for
functions at the phrasal level. These will be referred to as \is{intonational tones|textbf}\textit{intonational tones}. This
extension of the notion of tone beyond its primary meaning (lexical and morphological tone) is made
in view of the structural similarities between lexical and morphological tone, on the one hand, and
certain intonational phenomena, on the other hand. It does by no means amount to a~broadening of the
concept of tone to intonational phenomena in general, as is the case in some versions of
autosegmental"=metrical models of \isi{intonation}.

Firstly, tone may indicate sentence mode. “The most commonly encountered cases involve a~tonal means
to distinguish interrogatives from declaratives. In Hausa, a~L is added after the rightmost lexical
H in a~yes/no {question}, fusing with any pre"=existing lexical L that may have followed the rightmost
H ({\dots}). As a~result, lexical tonal contrasts are neutralized. In statements, [\ipa{káì}] ‘head’ is
tonally distinct from [\ipa{káí}] ‘you [masculine]’. But at the end of a~yes/no {question}, they are
identical, consisting of an~extra-H gliding down to a~raised L” \citep[61]{hymanetal2000}. The Hausa
example is described as a~case of \is{intonational tones}intonational tone, not a~case of superimposition of
an~intonational {contour} onto an~underlyingly unchanged tone sequence.

Secondly, tone may serve the function of \isi{phrasing}. In some languages, certain junctures of the utterance are
characterized by the addition of \is{boundary tone|textbf}boundary tones, which, though introduced by post"=lexical rules, are
integrated into the tone sequence of the utterance on a~par with lexical tones. L.\ Hyman (p.c.\ 2012) points out that such phenomena are “rampant in African tone systems”: for instance, the phrase"=final \isi{boundary tone} of Luganda acts just like any other H tone, except that it is inserted into the tonal string later than the lexical tones. Any
sequence of preceding toneless moras will be raised to that H level (though there has to remain at
least one L before it). For example, /\ipa{omulimi}/ ‘farmer’ is pronounced all-L as subject of a~sentence
(/\ipa{òmùlìmì}/), but at the end of an~utterance marked by this H\% it is pronounced L-H-H-H: /\ipa{òmúlímí}/. The phrase"=final \isi{boundary tone} of Luganda is transcribed as H\%, where the ‘\%’ sign,
representing a~\is{boundary (between tone groups)}boundary, is a~functional indication of the tone’s origin.

A~third intonational function that may be served by tone is to convey prominence. A~clear example of
\is{intonational tones}intonational tone (a~tone of intonational origin) is encountered in \ili{Naxi}: a~word that carries
lexical L or M tone on its last syllable can be focused by the addition of a~H tone that aligns at the
right edge of the word, causing the tone of the last syllable to become rising
\citep[72]{michaud2006d}.

In order to understand how intonational tones emerge and evolve, it appears interesting to examine
not only clear"=cut cases such as those reviewed in this paragraph, but also doubtful cases of
\is{intonational tones}intonational tone.


\subsection{Doubtful cases of intonational tone: Crossing the fine line between intonation and tone?}
\label{sec:doubtfulcasesofintonationaltonescrossingthefinelinebetweenintonationandtone}

\is{intonational tones}

Scholars have long been aware of the phonetic similarities between \isi{intonation} and tones. In the mid-17\textsuperscript{th} century, the European authors who devised a~Latin"=based
writing system for \ili{Vietnamese} \citep{derhodes1651} had to develop a~notation for a~six"=way tonal
contrast. One of the tones was left unmarked, grave and acute accents were used for two others, and
the~tilde for a~fourth one. For the remaining two tones, symbols from sentence"=level punctuation were
used: the full stop was added (below the vowel) to indicate tone 4 (orthographic \textit{nặng}) on
the basis of the perceived similarity between its final glottal constriction and the intonational
expression of \textit{finality}; and the {question} mark (in reduced form, on top of the vowel) was
used for tone 5 (orthographic \textit{hỏi}) due to its final rise
\citep{haudricourt2010b}. To the authors of this system, the newly coined tone marks served as mnemonic
cues to the pronunciation of tone, via an~{analogy} with \isi{intonation} in \ili{Romance} languages. 

When Chao Yuen-ren devised a~system of “tone-letters” some three centuries later \citep{chao1930}, he proposed it as a~tool to transcribe \isi{intonation}, as well as tones. Examples of application to \ili{English} \isi{intonation} were offered, distinguishing various ways of saying \textit{Yes} and \textit{Where does he live}.
%\footnote{
The original article is entirely composed in International Phonetic Alphabet, as was the standard for the journal \textit{Le Maître phonétique}. For convenience, this excerpt from \citet[26]{chao1930} is given in {English} orthography.
%} 

\begin{quotation}
	\begin{tabular}{lll}
			42 & \ipa{jes}\reflectbox{\ipa{˨˦}} & Ordinary affirmation.\\
			51 & \ipa{jes}\reflectbox{\ipa{˩˥}} & Of course.\\
			24 & \ipa{jes}\reflectbox{\ipa{˦˨}} & Go on, I'm anxious to hear the rest of it.\\
			13 & \ipa{jes}\reflectbox{\ipa{˧˩}} & I'm listening.\\
			15 & \ipa{jes}\reflectbox{\ipa{˥˩}} & But, ---.\\
			11 & \ipa{ɦjes}\reflectbox{\ipa{˩˩}} & I understand of course.\\
			44 & \ipa{j\~eˑs}\reflectbox{\ipa{˦˦}} & It's all right, although you made a mess of it.\\
			55 & \ipa{j\~eˑs}\reflectbox{\ipa{˥˥}} & I heard all about that sort of thing.\\
			351 & \ipa{jes}\reflectbox{\ipa{˩˥˧}}  & I should be most delighted.\\
			3513 & \ipa{jes}\reflectbox{\ipa{˧˩˥˧}} & So far as that's concerned, only ---.\\
	\end{tabular}	
	\newline
	\vspace{0.2cm}
	\newline
	\begin{tabular}{lllll}
			\ipa{ʍɛə}\reflectbox{\ipa{˥˥}} & \ipa{dəz}\reflectbox{\ipa{˦˦}} & \ipa{i:}\reflectbox{\ipa{˧˧}} & \ipa{liv}\reflectbox{\ipa{˩˧}} & Ordinary interrogation.\\
			\ipa{ʍɛə}\reflectbox{\ipa{˩˩}} & \ipa{dəz}\reflectbox{\ipa{˧˧}} & \ipa{i:}\reflectbox{\ipa{˦˦}} & \ipa{liv}\reflectbox{\ipa{˥˥}} & Where did you say he lived?\\
			\ipa{ʍɛə}\reflectbox{\ipa{˩˩}} & \ipa{dəz}\reflectbox{\ipa{˨˨}} & \ipa{i:}\reflectbox{\ipa{˦˦}} & \ipa{liv}\reflectbox{\ipa{˩˥}} & No matter where he eats.\\
			\ipa{ʍɛə}\reflectbox{\ipa{˥˧}} & \ipa{dəz}\reflectbox{\ipa{˧˧}} & \ipa{i:}\reflectbox{\ipa{˩˩}} & \ipa{liv}\reflectbox{\ipa{˥˧}} & I didn't ask{\dots}, I asked {\em how} he lived.\\
			\ipa{ʍɛə}\reflectbox{\ipa{˩˩}} & \ipa{dəz}\reflectbox{\ipa{˦˦}} & \ipa{i:}\reflectbox{\ipa{˥˥}} & \ipa{liv}\reflectbox{\ipa{˥˩}} & Don't you know where he lives?\\
	\end{tabular}
\end{quotation}

%\begin{figure}%[t]
%	\includegraphics[width=0.7\textwidth]{figures/Chao1930.jpg}
%	\caption{Examples given by Chao Yuen-ren of use of “tone-letters” for the transcription of {intonation} (“tone-values”) and tones (“tonemes”) \citep[26-27]{chao1930}.}
%	\label{fig:chao1930toneinto}
%\end{figure} 

Under Chao's proposal, there is no ambiguity as to whether \is{tone letters}tone letters are used for tones (which Chao calls “tonemes”, to bring out the {analogy} with {\linebreak}phonemes, with which they share a~distinctive function) or for \isi{intonation} (what he calls “tone values”). “Each tone-letter consists of a vertical reference line ({\dots}), to which a simplified time-pitch curve of the tone represented is attached, for tonemes to the left of the line, and for tone-values to its right” \citep[24-25]{chao1930}.\footnote{The symbols to transcribe {intonation}, with the stylized pitch curve to the right of the reference bar, were not taken up in the International Phonetic Alphabet.} Chao's proposal to use similar tools for the transcription of intonational differences in {English} and tonal oppositions in Cantonese underlines their similarity in phonetic form. In Chao's examples ({English}, Cantonese and {Tibetan}), the distinction between lexical tone and {intonation} is clear, but there are other cases where a~language’s lexical tones are reported to serve intonational purposes. Phake (\il{Tai-Kadai}Tai"=Kadai
language family) exemplifies the diversity of situations found in Asian languages.

\subsubsection{The expression of negation and sentence mode in Phake}

Phake, a~\il{Tai-Kadai}Tai"=Kadai language of Assam (India), has six lexical tones, and cases of “changed tones”
\citep[234–240]{morey2008}. Three processes are reported. 

\begin{enumerate}[label=(\roman*)]
	\item  If a~verb has the second tone
	(High falling), it changes to rising when negated. This rising tone is identical in form to the
	language's sixth lexical tone; speakers of the language perceive such tone change to be categorical in nature (tonal replacement). This process also appears to be extending to verbs carrying other tones (S. Morey, p.c.\ 2013). 
	
	\item  According to
	observations made in the 1960s and 1970s, changing the lexical tone of the last syllable in
	a~sentence to the sixth tone (a~rising tone) would express a~{question} \citep{banchob1987}. More
	recent fieldwork reports the same phenomenon, but instead of identifying the “changed tone” with one
	of the six lexical tones, it is suggested that it is “a special questioning tone ({\dots}). This
	questioning tone first rises and then falls, and here is arbitrarily notated as 7”
	\citep[234]{morey2008}.
	
	\item  Finally, an~eighth tone is reported: an \textit{imperative tone} “that exhibited
	glottal constriction and creaky voice” \citep[239]{morey2008}. 
\end{enumerate}

%(i)~If a~verb has the second tone
%(High falling), it changes to rising when negated. This rising tone is identical in form to the
%language's sixth lexical tone; speakers of the language perceive such tone change to be categorical in nature (tonal replacement). This process also appears to be extending to verbs carrying other tones (S. Morey, p.c.\ 2013). 
%
%(ii)~According to
%observations made in the 1960s and 1970s, changing the lexical tone of the last syllable in
%a~sentence to the sixth tone (a rising tone) would express a~{question} \citep{banchob1987}. More
%recent fieldwork reports the same phenomenon, but instead of identifying the “changed tone” with one
%of the six lexical tones, it is suggested that it is “a special questioning tone ({\dots}). This
%questioning tone first rises and then falls, and here is arbitrarily notated as 7”
%\citep[234]{morey2008}. 
%
%(iii) Finally, an~eighth tone is reported: an “{imperative} tone”, “that exhibited
%glottal constriction and creaky voice” \citep[239]{morey2008}. 

Observation (ii)~can be reinterpreted
in terms of \isi{neutralization} of tonal oppositions: it does not appear implausible that {question}
\isi{intonation} in Phake would override the lexical tone of the sentence’s last syllable in questions. Likewise, {imperative} \isi{intonation} has a~salient influence on some tense"=aspect"=modality markers, which may go so far
as to override their lexical tone. As observed by \citet{martinet1957}, “the fluctuating needs of communication and expression are
reflected more directly and immediately in \isi{intonation} than in any other section of the phonic
system”. The \is{phonation types}phonation type associated to {imperative} mode~-- a~contraction of
the laryngeal sphincter, to convey an~attitude of authority~-- appears to have a~clear iconic
motivation \citep[see][113--126]{fonagy1983}. The “{imperative} tone” of Phake reflects what might be a cross"=linguistic \textit{command intonation}: short, sharp, high.\footnote{Stephen Morey (p.c.\ 2016) reports that a~similar “{imperative} tone” occurs in Tai Khamti, a nearby language, which has five citation tones. When the first attempt to mark tones in Khamti was made in the 1990s, tone marks for features such as {imperative} were made, and these replace the citation tone mark in texts. Whether lexical distinctions are fully neutralized in these cases remains to be investigated experimentally.}

It is perhaps significant that “changed tones” are reported in an~area where the dominant languages
are non"=tonal. Speakers of Phake are also fluent in Assamese, a~non"=tonal language, which may create
a~pressure towards the \isi{simplification} of the Phake tone system, e.g.~through \isi{neutralization} of tonal
contrasts in some contexts. Overall, it would seem that \isi{intonation} does not easily win the day over
lexical tone. Some experimental evidence on this topic comes from a~study of the Austroasiatic
language Kammu, one of few languages with two dialects whose only major phonological difference is
the presence or absence of lexical tones. A~comparison of the two dialects concludes that the
intonational systems of the two Kammu dialects are basically identical, and that the main
differences between the dialects are adaptations of \isi{intonation} patterns to the lexical tones when
the identities of the tones are jeopardized \citep{karlssonetal2012}.

\subsubsection{Mandarin interjections: A~case of spurious tonal identification}

The treatment of the \is{interjections}interjection /\ipa{a}/ (transcribed as \zh{啊} in Chinese writing) in a~learn\-er’s
dictionary of Standard \ili{Mandarin} offers a~clear case of spurious tonal identification. This
dictionary treats the \is{interjections}interjection as if it had lexical tone, and sets up four distinct entries for
it, corresponding to the four tones of \ili{Mandarin}: with tone 1, the \is{interjections}interjection would mean
“speaker gets to know something pleasant”; with tone 2, it would signal a~“call for repetition”;
with tone 3, “surprise or disbelief”; and with tone 4, the “sudden realization of something”
(\citealt{huangfu1994}, entry “a”). This categorization is based on phonetic similarities between
the pitch patterns of the four tones and intonational variants of the \is{interjections}interjection, as recapitulated
in \tabref{tab:interjectiona}.

%done
\begin{sidewaystable}[p!]
\caption{Phonetic basis for the four"=way categorization of the nuances expressed by the interjection
/\ipa{a}/ in {Mandarin}, as proposed in some dictionaries.}
{\renewcommand{\arraystretch}{1.35}
\begin{tabularx}{\textheight}{ l Q P{37mm} Q P{18mm} Q }
  \lsptoprule
  tone & characterization in~dictionary & example & translation of~example & F\textsubscript{0} on \is{interjections}interjection & canonical realization of tone\\\midrule
  1 & “speaker gets to know something pleasant” & \zh{啊!我考过了!} \par \textit{ā! wǒ kǎo guò-le!} & Wow! I
  passed the exam! & overall high F\textsubscript{0} & level, in the upper part of the speaker’s range\\
  2 & “call for repetition” & \zh{啊,是吗?} \par \textit{á, shì ma?} & Oh, is that right? & rising & rising\\
  3 & “surprise or disbelief” & \zh{啊?你在这儿干吗?} \par \textit{ǎ? nǐ zài zhèr gànmá?}  & Huh? What are you
  doing here? & falling"=rising & falling from mid"=low to lowest, with final rise \is{form!in isolation}in isolation\\
  4 & “sudden realization of something” & \zh{啊,现在我知道了。}  \par \textit{à, xiànzài wǒ
  zhīdào"=le.} & Aha! Now I understand. & falling & sharply falling, from high starting"=point\\ 
  \lspbottomrule
\end{tabularx}}
\label{tab:interjectiona}
\end{sidewaystable}

There is in fact a~considerable phonetic difference between the four"=way division of the \ili{Mandarin}
tonal space, on the one hand, and on the other hand the intonational gradations in the realization of interjections. Interestingly, the
authors of the dictionary gloss the “tone-4” realization of the \is{interjections}interjection /\ipa{a}/ as the
“\textit{sudden} realization of something” (emphasis added). The \is{interjections}interjection /\ipa{a}/ can just as well
convey the realization of something, without any hint of suddenness (\citealt{lin1972}, entry
“\zh{啊}”). The F\textsubscript{0} of the \is{interjections}interjection decreases gradually, in a~manner that does not resemble tone
4 (an abruptly falling tone). The mention of suddenness was probably added because the intonational
signalling of this extra nuance tends to shorten the \is{interjections}interjection, thereby creating greater surface
similarity with tone 4. From the point of view of linguistic functions, there should be no
confusion: the phonetic realization of interjections in \ili{Mandarin} is purely intonational,
“with varying, indeterminate accent, like \ili{English} \textit{Oh!} \textit{ah!} \textit{aha!}” (\citealt{lin1972}, entry “\zh{啊}”). \ili{Mandarin} interjections bypass tonal coding; the \is{interjections}interjection /\ipa{a}/ has
a~wide range of possible realizations, and of expressive effects. The four entries set up for this
\is{interjections}interjection in the dictionary single out four of these realizations, and grant them separate status
merely because they happen to be phonetically close to the language’s four lexical tones. This
example illustrates the potential for a~misinterpretation of intonational phenomena as tonal.


\subsection{Conditioning factors for the development of intonational tones}
\label{sec:conclusionaboutthepresenceorabsenceofintonationaltonesasatypologicalparameter}

In light of the above survey, it appears that the presence or absence of intonational tones is
a~typological parameter: a~parameter that varies from language to language.

The issue of whether a~language has \isi{tonal intonation} or not may appear as a~non"=issue to
researchers who use autosegmental"=metrical models, since these models operate with the same
concepts~-- among which tone plays a~key role~-- for all languages, as a~matter of
definition. To some extent, this is an~issue of choice of terms: there often exist straightforward
equivalences between observations couched in tonal and non"=tonal terms. For instance, in their
description of \isi{phrasing} in \ili{French}, \citet[49–51]{fougeronetal1998} explain that they use notations
as H* tone and H\% tone (or L\% tone) respectively as equivalents for Delattre’s
(\citeyear{delattre1966a}) \textit{minor continuation} and \textit{major continuation}. Such equivalences allow for converting from one framework to another~-- but only from
a~language"=internal point of view. When it comes to typology, use of the same labels with
widely different meanings for different languages creates difficulties. Upon close examination, it
appears that the labels H\% and L\% as used for, say, Kinande, \ili{French}, \ili{Vietnamese} and Bemba refer to different phenomena in each case. In Kinande, the H\% which marks the end of a~phrase is a~\textit{bona fide}
tone, which interacts with tones of lexical origin, for instance by causing \isi{neutralization} of certain lexical
tone oppositions on nouns when they are said \is{form!in isolation}in isolation \citep[558]{hyman2014}. In \ili{French}, in the
absence of tones at the lexical level (and at the morphological level), there is no language"=internal
evidence to decide whether the phenomenon at issue is tonal or not, so H\% can be considered to be
equivalent to \textit{major continuation}, with added information on phonetic realization. For
\ili{Vietnamese}, “rising final pitch movements” are labelled as H\% by \citet{haetal2010} for theory"=internal motivations, not on the basis of structural similarities
between the lexical tones of \ili{Vietnamese} and the intonational phenomena at issue. Finally, in Bemba, which like Kinande belongs to the vast \ili{Bantu} group within Niger"=Congo, the authors of a~description of this language's \isi{intonation} \citep{kula2016} posit \isi{boundary tone}s but remain noncommittal as to the extent to which these entities are really tonal. Two of these \isi{boundary tone}s attach to left boundaries (H- and L-) and two attach to right boundaries (H\% and L\%). The left"=edge \isi{boundary tone}s H- and L- constitute a~device to refer to “global effects of pitch \isi{range expansion} and compression”; using the conceptual framework advocated in the present volume, they clearly seem non"=tonal. The right"=edge boundary tones (\mbox{H\%} and L\%) look as if they could be similar to the intonational tones reported in Kinande. The authors take a~cautious stand: “[i]t remains to
be investigated whether this \isi{boundary tone} replaces the lexical tone” \citep[331]{kula2016}. To sum up, the generalized use of \is{boundary tone}boundary"=tone labels such as H\% and L\% may appear economical from a~theoretical point of view, but it tends to veil typological differences \citep{ladd2008}, often leaving readers hard put to figure out whether \isi{intonation} in the language at issue is actually encoded by tones. 

The motivation for using tonal labels for intonational phenomena is reminiscent of the use of the feature /ATR/ (Advanced Tongue Root) to describe the four"=way opposition in a~vowel
system containing four degrees of vowel height: /\ipa{i-e-ɛ-a}/. This device obviates the need for a~multi"=valued /open/ feature for vowels,
considered uneconomical under certain phonological analyses \citep{calabrese2000}. The pinch comes when typological
considerations come in: should \ili{French} and other \ili{Romance} languages be
included in cross"=linguistic studies of ATR phenomena? A~common"=sense answer is that it would seem best to
begin by identifying a~core set of languages that uncontroversially possess ATR systems (a~crucial
phonological test being the presence of ATR vowel harmony), and to apply due caution when
considering extensions of the concept beyond this core domain.

The current vogue of tonal models of \isi{intonation} as applied for the most diverse languages entails an~indirect benefit for specialists of \isi{intonation}, as it can lead linguists engaged in intonational descriptions of tonal languages to raise explicitly the issue of the degree of similarity of intonational tones with the other tones found in the language. For instance, a~study of Tanacross Athabaskan, a~two"=tone language, uses the tonal notations advocated in the framework of intonational phonology, proposing four \isi{intonation} contours for four utterance types: H* \mbox{L\%} for declaratives, H* \mbox{H\%} for polar questions (yes/no interrogatives), L* L\% for imperatives, and H+L* L\% for open questions (wh"=questions) \citep[263]{holton2005}. The author espouses the logic that underpins these notations: that tone and \isi{intonation} are of the same nature at a~certain phonological level: ``[b]oth tone and \isi{intonation} can be viewed as strings of binary tone values with certain associations between the tone values and the segmental tone bearing units'' \citep[267]{holton2005}. This opens into the {question} of whether intonational tones partake in the language's tone rules. In Tanacross Athabaskan there is a~phenomenon of tone \is{tone spreading}spreading which is conditional on the stem's tone: the tone of the stem syllable affects the assignment of tone to preceding syllables. The author raises the issue of whether the final L tone found in declaratives has an~influence over tone \is{tone spreading}spreading: in principle, an~added tone in the sequence could modify the application of tone \is{tone spreading}spreading. The observation is that the pre"=stem tone spread constraint is sensitive to the underlying lexical tone, and is unaffected by the hypothesized intonational `tone' \citep[270]{holton2005}. The search for categorical effects of intonational `tones' on the tonal string yields a~clear conclusion in the negative. In my view, this settles the issue: Tanacross Athabaskan does not have \isi{tonal intonation}. This conclusion sheds light in retrospect on the author's caveat that ``[t]he notation consists of two types of `pitch"=accents', or tones (not to be confused with lexical tones discussed in the previous section)'' \citep[263]{holton2005}. Thanks to these analyses, it is possible to arrive at a~clear understanding of the facts. But clearing the confusion and uncertainty created by the use of autosegmental"=metrical notations can be a~tough struggle. To call two things by the same name, assert their identity at some level, and insist that they need to be kept distinct is to ask a~lot of the reader. 

In most East and Southeast Asian languages, the
available literature suggests that \isi{intonation} is not implemented by the addition of
tones in the way described for Kinande, Hausa, or Luganda and {Naxi} (\sectref{sec:instancesofintonationaltonesintheworldslanguages}). 
The widely"=studied case
of \il{Mandarin!Standard}Standard Mandarin provides a~clear example. {Mandarin} has salient intonational phenomena, which have
a~strong influence on the phonetic realization of tones, to the extent of making the automatic
recognition of tone in continuous speech a~technological challenge. But these intonational phenomena do not
affect the phonological identity of the lexical tones. Instead, \isi{intonation} is superimposed on tone
sequences. From the point of view of linguistic structure, \isi{intonation} remains on an~altogether
different plane from tones: it does not modify the phonological sequence of tones, even in cases
where it exerts a~considerable influence on their phonetic realization. This has been studied since
the pioneering work of \citet{chao1929}. 
%Relevant evidence on this issue comes from the field of
%speech synthesis: some specialists choose to specify (i)~full templates of the time course of F\textsubscript{0} for
%each lexical tone, and (ii)~a “strength coefficient” for each syllable
%\citep{kochanskietal2003a,kochanskietal2003b}. The strength coefficient, which correlates with
%informational prominence, plays a~major role in the final shape of the synthesized F\textsubscript{0} curve. This
%synthesis system provides indirect evidence that, 
%although intonational
%parameters interact with the phonetic realization of tone, they do not modify the underlying
%phonological sequence of tones: there is no insertion or deletion of tones. 
The informational
prominence of a~syllable is reflected in local phenomena of curve expansion and \isi{lengthening} on the
target syllable, as well as some modifications in supraglottal articulation. Conversely, a~degree of
phonetic reduction is found on other syllables, including post"=focus compression of F\textsubscript{0}
range (see in particular \citealt{xu1999}).

It seems intuitively clear that multi\is{level tones}level tone systems (e.g.~Ngamambo, Wobe) cannot allow the type of intonational
flexibility in the realization of tone which is pervasive in {Mandarin} or \ili{Vietnamese}, because such
flexibility would jeopardize the identification of the utterance’s tonal string. The need to distinguish a~wide range of categorically different sequences makes it less economical to encode information about \isi{phrasing}
and prominence as modulations of F\textsubscript{0} superimposed on the tonal string. This creates a~tendency to favour other means to convey \isi{phrasing} and prominence: either by
integration into the tonal string (i.e., \textit{intonational tones} as defined here), or by the use
of non"=intonational means, such as \isi{word order} or topicalization and \isi{focalization} morphemes. 

Experimental verification of such hypotheses is greatly complicated by the multifarious
differences among the languages to be compared. It is hoped that the availability of an~increasing number of monographs such as the present one can contribute to gradual clarification of these typological perspectives.
