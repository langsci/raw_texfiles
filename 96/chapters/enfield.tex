\documentclass[output=paper]{langsci/langscibook}
\ChapterDOI{10.5281/zenodo.573780}

\author{N. J. Enfield\affiliation{University of Sydney}}
\title{Dependencies in language} 
\maketitle
\begin{document}

% Table \ref{table: if then} shows some of the wide ranging \textit{if-then} statements that can be found in language.
\noindent Consider the if-then statements about language listed in Table \ref{table: if then} (overleaf).
\newcolumntype{L}[1]{>{\raggedright\let\newline\\\arraybackslash\hspace{0pt}}m{#1}}
\begin{table}
 
\begin{tabularx}{\textwidth}{L{0.7\textwidth}L{0.2\textwidth}}
\lsptoprule
 If the verb comes before the object in a language, then that language probably has prepositions and not postpositions & \citet{GreenbergMeaningfulElements}\\
\tablevspace
 If a speaker has just heard a passive construction, then they are more likely to produce one now& \citet{Pickering2008}\\
\tablevspace
 In \ili{Estonian}, if the verb ‘to be’ is negated, then no distinctions in person or number may be marked& \citet[63]{Aikhenvald1998}\\
\tablevspace
 If a conceptual theme is expressed in multiple different semantic systems of a language, then that theme will be of cultural importance to speakers of the language& \citet{Hale1986}\\
\tablevspace
 If a language has three places of articulation in fricatives, then it has at least three places of articulation in stops& \citet[154]{Lass1984}\\
\tablevspace
 If a transitive clause in \ili{Hindi} is not in perfective aspect, then no ergative marking may occur&  \citet[239]{Kellogg1893}\\
\tablevspace
 If a language expresses manner and path of motion separately in its lexical semantics, then speakers of the language will express manner and path separately in their gestures& \citet{ozyurek2007}\\
\tablevspace
 If there is a voicing contrast in stops, then /t/ and /k/ are present&\citet{Sherman1975}\\
\tablevspace
If a child has not yet learned to produce and comprehend pointing gestures, then she will not acquire language&  \citet{Tomasello2008}\\
\tablevspace
 If a specific structure is highly embedded in language-specific grammatical structures, then it is less likely to be borrowed into an unrelated language& \citet[69]{Thomason2001}\\
 \lspbottomrule
\end{tabularx}
\caption{Some of the \textit{if-then} statements found in language}
\label{table: if then}
\end{table}

%\FloatBarrier

\is{timescales} \is{causality} Each of these statements implies a kind of dependency between systems or structures in language (and sometimes with systems or structures outside of language), though the statements invoke different timescales, and imply different types of causal relation. Do these statements -- and the many more that exist like them -- belie a unified notion of dependency in language? Or do they merely point to family resemblances among loosely related concepts? Here are some of the (non-exclusive) ways in which we might mean that A is dependent on B:

\begin{itemize}

% \textbf{1.} 
\item To state a rule concerning A one must refer to B

% \textbf{2.}
\item When a process affects B, it will necessarily affect A

% \textbf{3.}
\item The existence of B is a condition for the existence of A

% \textbf{4.} 
\item The existence of B is a cause of the existence of A

% \textbf{5.} 
\item A cannot be expressed without also expressing B

% \textbf{6.} 
\item If B is the case, A is also likely to be the case
\end{itemize}

It is important to define dependency clearly, because the notion of dependency in language is central to our understanding of key questions in our discipline. These questions include: How are linguistic sub-systems related? Are there constraints on language change? How are languages learned by infants? How is language processed in the brain? What is the relation between language and social context? \is{language change} \is{acquisition}

This book explores the question of dependency in language with case studies and reviews from across the language sciences. Despite the importance of the concept of dependency in our work, its nature is seldom defined or made explicit. What kinds of dependencies exist among language-related systems, and how do we define and explain them in natural, causal terms? 

\section{Condition}

One important kind of relation that can define a dependency between co-occur\-ring features is the relation of condition. This is where the existence of B is a condition for the existence of A. It is where A would not be observed were B not also observed. Clear examples are when B is a medium for A. For instance, without phonation, there can be no pitch contrast. Pitch contrast depends on phonation, because the existence of phonation is what makes pitch contrast possible. Similarly, in turn, without pitch contrast, there can be no systems of lexical tone. Note that conditional dependency cannot be paraphrased in terms of cause. \is{phonation} \is{pitch} \is{tone} We can say that if \ili{Thai} speakers did not have phonation they would not have lexical tone. We cannot say that \ili{Thai} speakers have lexical tone because they have phonation. Dependence in this conditional sense defines the relations between nested framings of language as a form of human action, as in Austin’s ladder \is{Austin's ladder} that links all types of linguistic act from the phonetic to the perlocutionary (\citealt{Austin1962}; see also \citeauthor{Clark1996} \citeyear{Clark1996}: 146; \citeauthor{Enfield2013agencyenchrony} \citeyear{Enfield2013agencyenchrony}: 91-92).

Conditional dependency introduces collateral effects \citep{Sidnell2012collateral}. If A is conditionally dependent on B, then A cannot be expressed without also expressing, implying, or revealing B, regardless of whether this was wanted; thus the expression of B is a collateral effect of the intention to express A. An example comes from the expressive use of the hands \is{gesture} in sign language (or co-speech hand gesture). If a person wants to use their hands to show the speed at which something moved, they are forced to show movement \textit{in a certain direction} (e.g., North, South, North-Northeast, etc.), regardless of any intention to depict \is{depiction} or reveal directional information. In this case, the depiction of direction of motion is a collateral effect of the depiction of speed of motion. 

\section{Cause}

\is{causality} \is{synchrony}  A second important kind of relation underlying dependency is that of cause. A problem with positing dependency relations among synchronic structures in language is that often no causal link between the two synchronic structures is posited at all \citep[201]{Clark1984}. We are familiar with proposals of connections between language, culture, and thought, but explicit causal paths are seldom posited. What would it take to establish that there is a causal relation between a linguistic feature and a cultural value (in either direction)? First, consider how a grammatical feature comes to exist in a language in the first place. Grammatical properties of languages mostly come about by means of invisible hand processes \citep[Bk 4 Ch 2]{Smith1776}. \is{invisible hand processes} This means that the causes of these effects are distributed through tiny steps in a massive process of diffusion of innovation in populations, a process that no person can directly guide. The outcomes of the process need not bear any direct relation to the beliefs, goals, or intentions that individuals have had in producing the original behaviour. 

\is{genetics} \is{enchrony} But this does not mean those things were not caused by people’s behaviour. To discover and define those causes, one needs the microgenetic and enchronic and historical frames together, and one needs to allow that those frames be independent. This is not to say that such a relation of direct link between individuals' internal behavior and linguistic structures is impossible. It is merely to say that if a pattern is observed in language, it is not necessarily the case that it is there or like that because people wanted it to be there or like that. What I have just described is a type of causal disconnect between individual intentions and aggregate outcomes that is inherent to the causality involved in diachronic processes. \is{diachrony} These diachronic processes are, at base, actuated by the contributions of individuals. But they cannot be consummated by individuals. Rather they accumulate at the population level in ways that are beyond individuals’ reach. 

\is{diachrony} \is{ontogeny} \is{conventionalization} There is a further type of causal disconnect that should be pointed out here, which concerns the distinction between diachronic and ontogenetic framings of causal explanation of a linguistic structure. If I observe that a person has conventionalized a certain linguistic structure, and if I ask why this has happened, one explanation is ontogenetic: she speaks like that because her peers and elders spoke like that when she was learning her language. Her reasons for speaking that way might simply be ``this is how we speak": when learning a language, infants apply a kind of docility principle \is{docility principle} \citep{Simon1990} by which they follow the practices of their community without questioning why things are done in the way that they are done. This strategy is efficient and adaptive. In this way one person’s reasons for speaking in a certain way may have ontogenetic explanations (and of course with relation to specific instances of speaking, they may have enchronic and microgenetic explanations), yet they may be completely disconnected from the diachronic explanations for why those structures came to be used in that infant’s community in the first place. \citet{Simpson2002} argues that if innovations and extensions of meaning can be generated out of cultural values, they will not spring directly into grammar. \is{inference} \is{discourse} Rather they will spring from patterns of inference, and patterns of discourse usage, and it is these patterns, in turn, that may later lead to a grammatical ``structuration" of cultural ideas \citep[see also][]{Evans2003,Blythe2013}. But importantly, we see here how there is a chain from microgenetic and enchronic processes to diachronic processes, and then to ontogenetic processes, through which the kinds of individual beliefs, goals, and motivations that we typically associate with cultural values get delinked from higher-level/cultural systems such as languages. In this way, a correlation between a grammatical structure in my language and a set of beliefs or values in my culture does not entail a causal relation in the sense that is usually understood, namely a direct causal relation.

\section{Frames and biases}

\is{biases} If we are going to understand dependency, we need to focus on the underlying dynamics of causal/conditional relations. One reason dependency is understudied in linguistics is that most of our questions begin with statements in a synchronic frame. But this is the one frame that fails to draw our attention to causes and conditions, because it is the one frame that brackets out time. Analyses of synchronically framed facts are accountable to a transmission criterion \citep{Enfield2014,Enfield2015}: if a trait is there, it has survived, in the sense that it has successfully passed through all the filters that might otherwise have blocked its diffusion and maintenance in a speech community. 

\is{transmission} \is{biases} \is{synchrony} \is{diachrony} To provide a natural, causal account for dependencies in language systems, we need to be explicit about the ontology of the transmission biases that define the causes and conditions we invoke. We need to specify how the abstract notion of a synchronic system has come to be instantiated in reality. It is not enough to describe a piece of language structure, a linguistic (sub)system, or a pattern of variance in language. We must ask why it is that way. One way to answer this is to find what has shaped it. ``Everything is the way it is because it got that way", as biologist D’Arcy Thompson is supposed to have said (cf. \citealt{Thompson1917}; see \citealt{Bybee2010}: 1). The aim is to explain structure by asking how structure is created through use \citep{Croft2004}. If we are going to do this systematically and with clarity, a central conceptual task is to define the temporal-causal frames within which we articulate our usage-based accounts \citep[see][9-21]{Enfield2014}. Some of those frames are well established: in a diachronic frame, population-level dynamics of variation and social diffusion provide biases in a community’s conventionalization of structure; in a microgenetic frame, \is{conventionalization} \is{cognition} sub-second dynamics of psychological processing, including heuristics of economy and efficiency, provide biases in the emergence of structure in utterances; in an ontogenetic frame, principles of learning, whether social, statistical, or otherwise, provide biases in the individual’s construction of a repertoire of linguistic competence in the lifespan; and in an enchronic frame, the interlocking of goal-directed, linguistically-constructed actions and responses in structured sequences in social interaction. These frames vary widely in kind and in scale, but we need to keep them all in the picture at once. It is only by looking at the broader ecology of causal/conditional frames in language that we will we have any hope of solving the puzzles of dependency in language.

\section{Questions}

Here are some of the fundamental questions about dependency that kicked off the agenda for the collaboration that led to this book:\footnote{ The project that produced this book began with a retreat titled ``Dependencies among Systems of Language", held on June 4-7, 2014 in the Ardennes, at Château de la Poste, Maillen, Belgium. I gratefully acknowledge funding from the European Research Council through grant 240853 ``Human Sociality and Systems of Language Use". I also thank the participants, including the authors, as well as Balthasar Bickel, Claire Bowern, and Martin Haspelmath for their contribution.} 

\begin{itemize}
\item Some have tried to explain Greenbergian dependencies with reference to microgenetic or cognitive processes (appealing to ideas such as ease, economy, and harmony); To what extent have they succeeded? Why hasn’t this work in psychology made a greater impact in linguistic typology?

\item Others have tried to explain dependencies with reference to diachronic processes (where, to be sure, microgenetic processes are often causally implied); To what extent have they succeeded? Are these accounts different from pure processing accounts (given that there must be a causal account of linkage between individual processing biases and the emergence of community conventions)?

\item Dependencies can be shown to hold in the application of rules and operations in different grammatical subsystems -- e.g., the presence or absence of negation will often determine whether marking will be made in other systems, such as person/number/transitivity-related marking; what is the causal nature of such dependencies? How are they explained?

\item There are numerous interfaces between lexical, grammatical, and perceptual/cognitive systems. What dependencies are implied?

\item What are the knowns and unknowns of causal dependency in language? What is the state of the art? In what ways are the different notions of dependency related? Can we best make progress with these questions by taking an interdisciplinary approach?
\end{itemize}

Many further questions arose in the collaborations and discussions that ensued. Each of the chapters of the book addresses these questions in one way or another. None of the questions receives a final answer. It is hoped that this book makes some progress, and helps to sharpen these questions for further consideration as our knowledge, methods, and understanding of language develop.

{\sloppy
\printbibliography[heading=subbibliography,notkeyword=this]
}
\end{document} 
