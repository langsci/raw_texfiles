\documentclass[output=paper,colorlinks,citecolor=brown]{langscibook}
\ChapterDOI{10.5281/zenodo.15274555}
\author{Kari Kinn\orcid{}\affiliation{University of Bergen} and Michael T. Putnam\orcid{}\affiliation{Penn State University}}
\title[Introduction and theoretical preliminaries]
      {The syntax of North American Norwegian: Introduction and theoretical preliminaries}
\abstract{This chapter serves as an introduction to the present volume. We provide a brief background on the Norwegian language in North America, while also highlighting how research on this language variety can be situated into the wider research program on heritage languages. A secondary function of this introduction chapter is to introduce and outline the core desiderata of syntactic theory in accordance with the Minimalist Program \citep{chomsky95}, and the basic syntactic architecture used in most of the chapters.}


\IfFileExists{../localcommands.tex}{
   \addbibresource{../localbibliography.bib}
   % add all extra packages you need to load to this file

\usepackage{tabularx,multicol}
\usepackage{url}
\urlstyle{same}

\usepackage{listings}
\lstset{basicstyle=\ttfamily,tabsize=2,breaklines=true}

\usepackage{langsci-basic}
\usepackage{langsci-optional}
\usepackage{langsci-lgr}
\usepackage{langsci-osl}
% \usepackage{./langsci/styles/langsci-lgr}
% \usepackage{./langsci/styles/langsci-osl}
% \usepackage{langsci-gb4e}

\usepackage{tikz}
\usetikzlibrary{patterns,calc}
\pgfdeclarepatternformonly{south east lines}{\pgfqpoint{-0pt}{-0pt}}{\pgfqpoint{3pt}{3pt}}{\pgfqpoint{3pt}{3pt}}{
    \pgfsetlinewidth{0.6pt}
    \pgfpathmoveto{\pgfqpoint{0pt}{3pt}}
    \pgfpathlineto{\pgfqpoint{3pt}{0pt}}
    \pgfpathmoveto{\pgfqpoint{.2pt}{-.2pt}}
    \pgfpathlineto{\pgfqpoint{-.2pt}{.2pt}}
    \pgfpathmoveto{\pgfqpoint{3.2pt}{2.8pt}}
    \pgfpathlineto{\pgfqpoint{2.8pt}{3.2pt}}
    \pgfusepath{stroke}}
    
\usepackage{stmaryrd}
\usepackage{wasysym}
\usepackage{multirow}
\usepackage{caption}
\usepackage{subcaption}
\usepackage{mathrsfs}
\usepackage{qtree}

\usepackage{linguex}


   %pminos do not split footnotes
% \interfootnotelinepenalty=10000 %Footnote in Laporte chapters has to be split SN


%\DeclareIndexNameFormat{default}{%
%\nameparts{#1}%
%\usebibmacro{index:name}%
%{\index[names]}%
%{\namepartfamily}%
%{\namepartgiveni}%
% {}% L1
% {}% L2
%{\namepartprefix}% generates spurious space L3
%{\namepartsuffix}% generates spurious space L4
%}

%  {\DeclareIndexNameFormat{default}{%
%     \usebibmacro{index:name}{\index[names]}{#1}{#3}{#5}{#7}}}

%\DeclareIndexNameFormat{default}{%
%  \usebibmacro{index:name}{\sindex[nom]}{#1}{#3}{#5}{#7}}

%\DeclareIndexNameFormat{default}{%
%  \usebibmacro{index:name}{\sindex[person]}{#1}{#3}{#5}{#7}}
%\DeclareIndexNameFormat{default}{%
%\nameparts{#1} \usebibmacro{index:name}{\sindex[person]]}{\namepartfamily}{‌​\namepartgiven}{\nam‌​epartprefix}{\namepa‌​rtsuffix}}

%\newcommand{\smiley}{:)}

%\renewbibmacro*{index:name}[5]{%
%\usebibmacro{index:entry}{#1}%
%{\iffieldundef{usera}{}{\thefield{usera}\actualoperator}\mkbibindexname{#2}{#3}{#4}{#5}}}

% \newcommand{\noop}[1]{}

%remove for final
%\overfullrule=1mm

\newcommand{\tobi}[2]}}
\renewcommand{\S}[1]{\tobi{#1}{\textsc{*}}}

% this volume references
% puts: [this volume]
% already defined: \citetv
%\newcommand{\citepv}[1]{(\citeauthor{#1} \citeyear*{#1} [this volume])}
\newcommand{\citealtv}[1]{\citeauthor{#1} \citeyear*{#1} [this volume]}

%parentheses around example number
\newcommand{\pref}[1]{(\ref{#1})}

% in-text examples

\newcommand{\lnex}[1]{\textit{#1}} %target lang word
\newcommand{\lnlit}[1]{(lit.: `#1')} %literal reading
\newcommand{\lnlat}[1]{(#1)} % latinization
\newcommand{\lntrans}[1]{`#1'} %translation
\newcommand{\lnexl}[2]%
{\lnex{#1}{} \lnlat{#2}} % ex with latinization
\newcommand{\lnexlat}[3]{\lnex{#1}{} \lnlat{#2}{} \lntrans{#3}} % ex with latinization and tranl.

%ch01
\newcommand{\co}[1]{\mbox{\textbf{#1}}}

%ch09

\newcommand{\cyrbulg}[1]{\begin{otherlanguage*}{bulgarian}#1\end{otherlanguage*}}


%ch10
\newcommand{\nlp}{{\small NLP}}
\newcommand{\mwe}{{\small MWE}}
\newcommand{\rae}{{\small RAE}}
\newcommand{\lvc}{{\small LVC}}
\newcommand{\pos}{{\small P}o{\small S}}
%\newcommand{\todo}[1]{ \textcolor{red}{#1} }

%\renewcommand{\labelenumi}{\theenumi}
%\ainamefmt{{vv}{ll}{, ff}{, jj}} % fullname

\newcommand{\biberror}[1]{{\color{red}#1}}

\newcommand{\osenovaitem}{--~}
   %% hyphenation points for line breaks
%% Normally, automatic hyphenation in LaTeX is very good
%% If a word is mis-hyphenated, add it to this file
%%
%% add information to TeX file before \begin{document} with:
%% %% hyphenation points for line breaks
%% Normally, automatic hyphenation in LaTeX is very good
%% If a word is mis-hyphenated, add it to this file
%%
%% add information to TeX file before \begin{document} with:
%% %% hyphenation points for line breaks
%% Normally, automatic hyphenation in LaTeX is very good
%% If a word is mis-hyphenated, add it to this file
%%
%% add information to TeX file before \begin{document} with:
%% \include{localhyphenation}
\hyphenation{
    Beck-man
    Ngu-yen
    back-chan-nel
    back-chan-nels
    mo-not-o-nous
    ste-reo-typ-i-cal
}

\hyphenation{
    Beck-man
    Ngu-yen
    back-chan-nel
    back-chan-nels
    mo-not-o-nous
    ste-reo-typ-i-cal
}

\hyphenation{
    Beck-man
    Ngu-yen
    back-chan-nel
    back-chan-nels
    mo-not-o-nous
    ste-reo-typ-i-cal
}

   \boolfalse{bookcompile}
   \togglepaper[23]%%chapternumber
}{}

\begin{document}
\maketitle

\section{Introduction} \label{intro}


Norwegian as spoken in North America has been an object of study since the early 20th century. Early accounts (e.g.  \citealt{Flom1900, flom1903gender, flom1926englishloanwords, Flaten1900}) focused mainly on vocabulary (particularly on loanwords) and noticeably less on structural elements of the language, such as syntax.\footnote{\citet{flom1903gender}; however, considers his findings on the gender of loanwords in a wider, theoretical context.} In his seminal work, \citet{haugen1953} includes a chapter on ``the grammar of loanwords'',   which covers grammatical gender, inflection and compounding, but there is not much discussion of phrase- or clause-level syntax, apart from a brief section stating that Norwegian word order is similar to English and that English loanwords are mostly used in Norwegian sentences ``in the position to which its word class entitle[s] it'' \citep[457--458]{haugen1953}. Other scholars have, however, continued the tradition where Haugen and his predecessors left off, and in this volume, we present some of the insights and analyses that have been reached in recent studies on the \mbox{(morpho-)}syntax of North American Norwegian (NAmNo), especially over the last years. 

The past few decades have borne witness to a significant increase in research centered on bi- and multi-competent individuals and populations. As a result of this upsurge in research on bi/multilingualism, there has been a growing awareness and interest in the sociolinguistic conditions in which these languages are spoken, the real-time processing demands of competing languages, and the underlying mental representations that constitute these grammars. Extra-territorial varieties of particular languages, such as Norwegian in North America, are commonly considered \textit{heritage languages}. A widely used definition of a \textit{heritage language} is the following provided by \citet[156]{rothman09} (for further discussions, see e.g. \citealt{polinsky18}):

\begin{quote}
    ... a language spoken at home or otherwise readily available to young children, and crucially this language is not a dominant language of the larger (national) society.
\end{quote}

The extra-territorial varieties and the conditions in which they exist today have now been grafted into a larger, more general research program on heritage languages that includes foci on language documentation, corpus linguistics, theoretical analysis, and to a more limited extent, psycholinguistic/experimental studies \citep{adamou21}. More generally, research on extra-territorial Norwegian has enhanced our understanding of general tendencies exhibited by heritage varieties of Germanic languages \textit{a priori} \citep{jannejoe15,jannemike20,jannejoe21,pageput15}. The vast majority of Germanic heritage languages spoken today are \textit{moribund}; i.e., the final or penultimate generation of speakers represents the final one that possesses significant proficiency in said heritage language. Extra-territorial variants of Norwegian also certainly fall into this category; however, in spite of their moribund status, the research conducted on heritage Norwegian has provided unique insights into the continued development of bi/multilingual grammars in the twilight of their existence. 

Echoing the significant increase of research conducted on bi\slash multi\hyp competence individuals and communities over the past few decades, research focusing on NAmNo witnessed a Renaissance of sorts in the earlier 1990s with pioneering research by \citet{hjelde1992trondsk}. Hjelde's early work focused mainly on the Norwegian dialect of \emph{Trøndsk} in North America. It was, however, followed by a wave of more general interest for NAmNo, in which the role of Janne Bondi Johannessen can hardly be overestimated. Johannessen initiated a  series of field trips to the US and Canada from 2010 onward (see, e.g., \citealt{johannessenlaake2012tomyter} for some early findings from this period). Speech data collected during these trips were made available for the research community in the Corpus of American Nordic Speech (CANS, \citealt{johannessen2015cans}).\footnote{The first version of the corpus was called the \emph{Corpus of American Norwegian Speech}; the  change from \emph{Norwegian} to \emph{Nordic} reflects the fact that the corpus has been expanded several times and now includes data from Swedish heritage speakers.} CANS is a transcribed and  morphologically tagged corpus of spontaneous speech which has facilitated research in a number of areas. Important works on the syntax of NAmNo that appeared recently after, or during, the creation of CANS include (to name just very few) \citet{larssonjohannessen2015incomplete}  and \citet{westergaardlohndal2019verbsecond} on verb placement; \citet{anderssenwestergaard2012tospraklighet}, \citet{anderssenetal2018cross-linguistic}, and  \citet{vanbaal2020thesis} on possessives and (double) definiteness; \citet{riksem2017phd} and \citet{riksemetal2019languagemixing} on syntactic patterns of language mixing, and \citet{eidekristinmelumhjeldearnstein} on expressions of modality. Nascent research has now begun on heritage Norwegian spoken in Latin America (especially Argentina, \citealt{kinnetal2024latamnoNLT, kinnetal2024argnowila}), necessitating the distinction \textit{Latin} and \textit{North} American Norwegian.

\begin{sloppypar}
Despite the rich and growing literature on the grammatical attributes of NAmNo, these findings have not yet been integrated into a collective whole. The principle disadvantage of the absence of a centralized work on elements of NAmNo is that it obfuscates continued progress on research that targets the grammar of this heritage variety~-- especially from a formal perspective. The main aim of this reference guide is threefold: First, most of the individual chapters provide a detailed overview of the current state of previously researched elements of the syntax of NAmNo. Second, some of the chapters explore topics that to date have not been as intensively and thoroughly researched as others, providing insights and starting points for ongoing and future research in these areas. Third, these chapters demonstrate how the formal analysis of the syntactic properties of heritage languages such as NAmNo can make important and lasting contributions to theory\hyp building efforts. Recent works by \citet{robertaetal21, robertamike23} have highlighted the importance of the symbiotic relationship between data and findings from (moribund) heritage languages (such as NAmNo) and rigorous theoretical analysis. The chapters found in this volume contribute to the important ongoing research on the nature of syntax born and maintained in these settings. To wit, these findings make a strong case for overall sturdiness of heritage language syntax \citep{lohndal2021} (as opposed to morphology; e.g. \citealt{putnametal2021}); however, there are a number of domains of syntax proper that \textit{do} display some element of (ongoing) change. Additionally, to the best of our knowledge, this reference guide is the first of its kind to combine detailed summaries of these findings from one heritage grammar with the primary purpose of contributing to a unified treatment of its syntax.

The remainder of this introductory chapter has the following structure: in \sectref{theory}, we present our theoretical preliminaries. \sectref{sec:clausesandnominals} maps out the basic structure of clauses and nominals in homeland Norwegian. In \sectref{sec:speaker,corpusdata,glossing}, we provide an overview of the empirical foundations for research on NAmNo (speakers and corpus data), as well as glossing conventions. \sectref{overview} is an overview of the chapters in the present volume.  
\end{sloppypar}


\section{Theoretical preliminaries} \label{theory}

In this section we sketch out a detailed overview of the architectural design and theoretical desiderata that guide the analyses found in the individual chapters of this volume. We collective adopt a \textit{generative} approach to grammar, which entails formal and explicit accounts of structure. The contributions in this volume provide analyses of syntactic~-- including morphosyntactic~-- phenomena that adhere to the core principles of the Minimalist Program \citep{chomsky95}. In this section, we outline the fundamental architectural assumptions and operations germane to Minimalist analysis. Since its inception over 25 years ago, the Minimalist Program has undergone further revision and development. Although one could argue that there are different “camps”, or “schools of thought”, that now exist under the heterodox umbrella of Minimalism, there still exists a high degree of cohesion and congruence amongst the majority of those who make use of some version of this program. Our contributions largely embrace the concept of an “open Universal Grammar (UG)” \citep{lightfoot2020}, which eschews the need for a pre-determined finite set of parameters that guide the acquisition process.\footnote{This does not entail that parameters do not exist, but rather it interprets them as second-order objects that emerge from core principles. See also \citet{roberts2019parameterhierarchies}.}

\subsection{Architectural considerations}
A guiding principle of Minimalist syntax is that the computational capacity of syntax is to generate structures that can be interpreted phonologically and semantically. Under this condition, syntax~-- and Language more generally~-- can be understood as the optimal solution to satisfy interpretation requirements of sound and meaning. Syntactic operations construct licit and legible syntactic structures which are then interpreted by external interfaces. This operation is known as \emph{Spell-Out}, and the external interfaces are commonly referred to as \emph{Logical Form} (LF) and \emph{Phonological Form} (PF) respectively. This modular model of linguistic competence is captured in the \emph{T-model} shown in \REF{t-model}: 

\ea\label{t-model}
\begin{forest}
[Syntax
    [Spell-Out
        [PF][LF]
    ]    
]
\end{forest}
\z
\noindent The Minimalist approach to syntax is a derivational, proof-theoretical one (as opposed to a representational, model-theoretical one); henceforth, structures are built in a piecemeal fashion. We assume a universal ordering of particular “domains” consisting of anchoring functional heads which are found in all human languages. These core heads form a “syntactic spine” of sorts \citep{ramsven2014,grohmann2003,putnam2020OFOH}. For our purposes, we focus on three particular core domains represented in the syntax, which we outline in \REF{triple} below: 

\ea \label{triple}
\ea \textit{v}P $\rightarrow$ Event semantics
\ex TP $\rightarrow$ Situational semantics 
\ex CP $\rightarrow$ Propositions; Information structure
\z
\z

The domains introduced in \REF{triple} are built in a bottom-up fashion, starting with \textit{v}P, followed by TP, and concluding with CP. Although there may be additional information (represented by formal features on syntactic projections) within these domains, these three domains are generally understood to be essential to syntactic structure cross-linguistically. Given that the Minimalist Program embraces a derivational approach to syntax, it is not uncommon that particular sub-units of structure may be selected to be interpreted from larger structures. Certain units are often held to have a privileged status as forming a completed semantic (propositional) units. These units are referred to as \emph{Phases}. Healthy debate continues concerning exactly which derivational units constitute a phase (and which do not), and in some respects, whether or not phases should be a part of the Minimalist Program moving forward.\footnote{For those interested in contrasting perspectives on the nature and size of phases, see \citet{abels12}, \citet{epseely06}, \citet{stroikput13} and \citet{boskovic2014phases}. Even though these perspectives may diverge (significantly) from one another, they all highlight the importance of cyclic derivations in formal syntax.} We do not engage in this debate further here. Note that although the spine shown in \REF{triple} refers to domains  found in clauses, other types of phrases will be built according to similar principles. A particularly important type of phrase is constituted by nominals. \citet{abney1987englishNP} is one of several authors who points out parallelisms between clauses and nominals; this work is also an important study in the debate about what constitutes the head of a nominal; the determiner, making the phrase a DP, or the noun, making the phrase an NP? Some long-standing contributions to this discussion are, e.g., \citet{Longobardi2010}, \citet{szabolcsi1994} and \citet{boscovic2005}.  For an overview of the debate, and references to recent works, see \citet{blumelholler2022newperspectives}.



In addition to these basic architectural assumptions, we find it prudent to briefly mention how these impact how we conceptualize bi- and multilingual grammars. Evidence from cognitive neuroscience and psycholinguistic research on the nature of bilingual grammars abounds confirming the integrated nature of bi- and multilingual language and cognition -- although the linguistic knowledge of a bi- or multilingual speaker can be conceptualized as a set of parallel grammars, these grammars exist within a single system and not in isolation  (\citealt{green2013}, \citealt{putnam2018} and references therein, \citealt{aboh2015}). In light of this evidence, we adopt the well-supported approach of “shared syntax”, i.e., computational mechanisms; we assume, following \citet{lohndal2021} and others, that basic structure-building operations are largely immune to attrition and decay in heritage grammar syntax. The general consensus that has emerged in the literature on this topic is that syntactic change or attrition occurs most commonly when syntax interfaces with morphophonology, i.e., “at PF”, and with semantics to a lesser extent. For further discussion of common ``outcomes'' in heritage language grammars, see \citet{polinsky18}. 




\subsection{Structuring-building and feature-valuing operations}

In this section we present the basic structure-building and feature-valuing operations that are assumed in Minimalist analysis. Here we review the basic tenets of Merge (\sectref{Merge-subsection}), feature valuation via Agree (\sectref{Agree-subsection}), and common concepts associated with realizational/late-insertion models (such as \textit{Distributed Morphology}) at the syntax-morphology interface (\sectref{smurf-subsection}). 



\subsubsection{Merge} \label{Merge-subsection}

Complex syntactic structures are composed of iterative instances of well-formed smaller units of structure. \emph{Merge} is an operation which combines two syntactic objects, yielding a more complex one as in \REF{merge-tree1}.\footnote{For an easily accessible overview of the basic mechanics of \emph{Merge}, see \citet[Section 3.3]{adger03}.}

\ea
\label{merge-tree1}
\begin{forest}
[Z
  [X] [Y]
]
\end{forest}
\z

The operation Merge unifies the syntactic objects X and Y and creates a new syntactic object Z. Two additional comments are in order here: First, the variable Z in \REF{merge-tree1} should be understood as a variable that could equate with X and Y. The reason here for differentiating “Z” from “X” or “Y” is simply for ease of exposition.\footnote{We do not engage with ongoing discussions concerning the challenges associated with the projection of syntactic structure in this introduction (see e.g. \citealt{chomsky13} for further discussion).} Second, the syntax does not concern itself with the linear order of the terminal nodes of the tree structure in \REF{merge-tree1}. Both orderings~-- [\textsubscript{Z} X Y ] or [\textsubscript{Z} Y X ]~-- are licit syntactic objects. 

The simple Merge-operation is quite powerful, allowing for the iterative recursive structure-building of simple and complex syntactic objects. Debate persists as to the motivation of Merge, i.e., whether it applies “freely” \citep{boeckx13,eks22}, or whether it is constrained in some way. We do not contribute further to this debate in this introduction, but do wish to point out that both schools of thought interpret Merge as the fundamental structure-building operation in Minimalist analysis. Traditionally, Merge has be motivated and constrained by the need to check \emph{formal features} in local configurations, i.e., sisterhood relations in syntactic trees, as in \REF{merge-tree1}. In this structure, the syntactic object X is endowed with an \textit{interpretable} feature. Its sister node, Y, has an \textit{uninterpretable} matching feature (indicated by the \textit{u}-diacritic). When features “match” in a local configuration (sisterhood), the interpretable feature checks, or values, the uninterpretable one. This ensures the interpretability of syntactic structures. 

\ea
\label{merge-tree2}
\begin{forest}
[Z
    [Y{[\sout{\textit{u}}F]}][X{[F]}]
]
\end{forest}
\z

Instances of “Second-Merge” (to borrow a term from \citealt[109]{adger03}) lead to a structural configuration in which the \emph{maximal projection} (XP) is \textit{extended} via a bar-level projection (X\textsuperscript{$\prime$}). This state of affairs is illustrated in \REF{xbar-tree}, in which the domain of the maximal project includes (i) a Spec(ifier), (ii) a bar-level projection, (iii) the head (or \emph{minimal projection}) of the phrase, and (iv) a complement. The specifier and complement are considered to be equidistant from their governing head.


\ea
\label{xbar-tree}
\begin{forest}
[ZP
    [Spec(ifier)]
           [Z\textsuperscript{$\prime$}
                [Z] [Comp(lement) 
                ]
                ] 
                ]
\end{forest}
\z

\subsubsection{Agree} \label{Agree-subsection}

The valuing of formal features can also take place in non-sisterhood configurations. For example, in the Norwegian example in \REF{ex:risingprices}, the plural form of the definite determiner, \emph{de}, is chosen in order to agree with the plural noun \emph{prisene} `the prices'; the present participle \emph{økende} `rising' intevenes between the two, and they are not sisters. 


\ea\label{ex:risingprices}
\gll de økende prisene \\
the.\textsc{pl} rising price.\textsc{def.pl} \\
\glt `the rising prices'
\z

%\ex. \gll Husene som st{\aa}r der borte er store. \\
 %         houses.\textsc{def} \textsc{comp} stand there away are big.\textsc{pl} \\
  %        `The houses standing over there are big.' \label{bighouse}

This state of affairs forces us to formulate how feature valuing can take place beyond sisterhood relations. The operation according to Minimalist parlance responsible for this feature check is referred to as \emph{Agree}. The tree structure in \REF{house-tree} illustrates how this plays out. In this structure, the syntactic head X possesses an uninterpretable feature [\emph{u}F] and functions as a \emph{Probe} that searches for a suitable \emph{Goal}, i.e., another syntactic head that possesses a matching, interpretable feature [F]. The syntactic head Y represents a licit Goal for X.\footnote{There is healthy, ongoing debate on the exact nature of the directionality of Agree as to whether or not it should (always) take place downward \citep{diercks20,carstensdiercks13,omer13,wurmbrand14}, upward \citep{heddde12,bjorkhedde19}, or perhaps in either direction \citep{bejar09}. In this introduction, we adopt a downward approach to Agree.} 



\ea\label{house-tree}
\begin{forest} 
[A
    [X{[\sout{\textit{u}}F]},name=x]
    [B
        [Z][C
            [Y{[F]},name=y][\dots]
        ]
    ]
]
\draw[->] (x) to[out=south,in=west] (y);
\end{forest}
\z

Three additional comments concerning the operation Agree are in order. First, although Agree takes place between two syntactic heads that are not sisters, there is still a demand to ensure that the distance between the Probe and the Goal is not too substantial. Agree is commonly constrained by a \emph{c-command} relationship, so long as there are no intervening syntactic heads between the potential Probe and Goal bearing an identical matching feature. In \REF{house-tree}, the syntactic head Z is not guilty of preventing an Agree relationship between X and Y since it does not have a matching feature (the tree structure could be applied to example \REF{ex:risingprices}, where \emph{økende} `rising' does not have a number feature). Second, in addition to c-command, additional locality constraints, such as phase boundaries, may prevent an Agree relation to hold between a potential Probe and Goal. Third, Agree may be accompanied by movement of the Goal. Whether or not Agreement between certain features is accompanied by movement is a matter of cross\hyp linguistic, parametric variation; in formal terms, this can be conceptualized as a second-order feature specification (a movement\hyp triggering feature that may or may not be associated with the Probe, see \citealt{adgersvenonius2011features} and references therein). Examples of movement in Norwegian are shown in \sectref{sec:clausesandnominals}. Movement of a syntactic element can be conceived of as a subtype of Merge: Internal Merge. 



\subsubsection{The syntax-morphology interface} \label{smurf-subsection}

A number of contributions in this volume adopt the stance that morphology is the result of post-syntactic operations. Such approaches separate morphology from syntax, and  are referred to as \textit{late-insertion} or \textit{realizational} models. An example of such a model is \textit{Distributed Morphology} \citep{embick2007}. Late-insertion models of morphology are compatible with Minimalist desiderata. Syntactic objects consisting of features are associated with phonological material, i.e., \emph{exponency}, through a series of correspondence rules formally known as \emph{Vocabulary Items}, as shown in \REF{vocabitem}: 

\ea
\label{vocabitem} 
Vocabulary Item \citep[9]{embick2015}:\medskip\\ 
\begin{tabular}{@{}ccc@{}}
$[\alpha\beta\gamma]$ &  $\longleftrightarrow$ & $\underbrace{/X/}$\\
\textit{synsem features} & & \textit{phonological exponents}\\ 
\end{tabular}
\z

The association of synsem features can be either one-to-one or one-to-many. A domain in which a late-insertion model has modeled findings in NAmNo is with respect to “mixed” Determiner Phrases (DPs), those consisting of an element from Norwegian and another from English \citep{riksem2017phd,riksemetal2019languagemixing,lohnput21,lohnput24}. Consider the tree structure for the mixed DP \textit{ei field} `a field' in \figref{lohnputree} (from \citealt[16]{lohnput21}), which consists of a Norwegian determiner and an English noun. Determining exponency is an additive function, with all features present in the syntax contributing to establishing a connection with the proper exponent.

\begin{figure}[h]
\caption{Mixed DP, from \citet{LohndalPutnam2021}.}
\label{lohnputree} 
\begin{forest}
[DP
    [D{[\textsc{def:}\textcolor{blue}{$-$Def}; \textsc{num:}\textcolor{blue}{$+$Sg}; \textsc{gen:}\textcolor{blue}{Fem}]}
        [\textcolor{blue}{\textit{ei}}]
    ]
    [DefP
        [Def{[\textsc{def:}$-$Def]}][NumP
            [Num{[\textsc{num:}$+$Sg]}][\textit{n}P
                [\textit{n}{[\textsc{gen:}Fem]}][$\sqrt{\textnormal{\textsc{field}}}$]
            ]
        ]
    ]
]
\end{forest}
\end{figure}

The realization of the indefinite article \textit{ei} `a' is based on the cumulative feature values of \textsc{def}, \textsc{num}, and \textit{n} (which determines \textsc{gen}(der) in this configuration). 

\section{Clauses and nominals in European Norwegian: Some starting points }\label{sec:clausesandnominals}
Studying heritage languages often involves implicit or explicit comparison to the homeland variety of the same language.\footnote{This  applies to heritage languages that have gained their heritage status through migration. Many indigenous minority languages can also be classified as heritage languages; in these cases, no additional homeland variety exists \citep[Chap. 1]{polinsky18}.} In this volume, the terms \emph{European Norwegian }(abbreviated \emph{EurNo}) and \emph{homeland Norwegian} are used interchangeably to refer to Norwegian as spoken in Norway.\footnote{The term \emph{European Norwegian} was chosen because of its parallelism with  \emph{North American Norwegian}; both labels refer to  the part of the world  in which the language is spoken, and both can be easily abbreviated (\emph{EurNo} and \emph{NAmNo}). Admittedly, \emph{European} is somewhat imprecise; however, more  precise alternatives (such as \emph{Norwegian Norwegian}, which is perhaps the most accurate label in theory) would introduce other sorts of complications. } Importantly, many research questions require a more precisely defined baseline for comparison, as  synchronic and diachronic variation in the homeland (and in the immigrant settlements) can greatly affect both the outcomes in the heritage language and how findings are interpreted (see \citealt[Chap. 1]{polinsky18} and, for NAmNo specifically, e.g. \citetv{chapters/vanbaal}, \citetv{chapters/eide} and \citetv{chapters/larsson} and references therein). Still, the homeland variety can be a useful starting point, and in the present section we describe some core syntactic properties of EurNo, analyzed within the theoretical framework introduced in \sectref{theory}. We limit our attention to some of the most basic properties of clauses and nominals; clauses are treated in \sectref{subsec:verbplacement}, with  main focus on verb placement, while nominals are treated in \sectref{subsec:nominals}. Further details will be given in the individual chapters; for a general overview of the syntax of EurNo, see, e.g., \citet{faarlund2019syntax}. 

\subsection{Verb placement in main and embedded clauses}\label{subsec:verbplacement}
In EurNo main clauses, the finite verb obligatorily moves out of the \emph{v}P, via T, to a position in the left periphery of the clause; this is commonly referred to as V-to-C movement. This movement, combined with a requirement that one constituent moves to the preverbal position (see e.g. \citealt{holmberg2015V2}), yields the Verb Second (V2) property, which is  characteristic  of EurNo (and most other Germanic languages, apart from English). There has been debate as to exactly which features trigger verb movement to the left periphery; however, it is a common assumption  that V2 is related to clause type. V2 is a main clause phenomenon, and the finite verb seems to occupy the same position as complementizers in embedded clauses (see further discussion below). The choice of preverbal constituent is to a great extent related to information structure, which is encoded in the CP domain (see \ref{triple}).  

V2 entails, informally, that the finite verb must be preceded by one -- and only one -- constituent. Some examples with different preverbal constituents are given in \REF{ex:mainclauses} (a subject in \REF{ex:frontedsubj}; an object in \REF{ex:frontedobject} and an adverbial in \REF{ex:preverbaladv}). The syntactic structure of \REF{ex:frontedobject} is sketched in \figref{tree:V2-homeland} (positions  from which an element has moved are marked by strikethrough).\footnote{Thus, strikethrough is used slightly differently here from in the structure in \REF{house-tree}, where it marks matching of features.} 

\ea \label{ex:mainclauses}
\ea \label{ex:frontedsubj}
\gll Jeg leste hele boka. \\
     I read whole book.\textsc{def} \\
\glt `I read the whole book.'
\ex \label{ex:frontedobject}
\gll Den boka har jeg ikke lest. \\
     that book.\textsc{def} have I not read\\
\glt `That book I haven't read.' \\
\ex \label{ex:preverbaladv}
\gll Neste uke skal jeg lese flere bøker. \\
     next week shall I read more books \\
\glt `I will read more books next week.'\\
\z
\z
\begin{figure}
\caption{Syntactic structure of example (\ref{ex:frontedobject}).}
\label{tree:V2-homeland} 
\begin{forest}
[CP
    [\textit{Den boka}][C$'$
        [C\\\textit{har}][TP
            [\textit{jeg}][T$'$
                [T\\\textit{\sout{har}}][NegP
                    [Neg\\\textit{ikke}][\textit{v}P
                        [\textit{v}\\\textit{\sout{har}}][\textit{v}P
                        [\textit{\sout{jeg} lest \sout{den boka}}, roof]
                        ]
                    ]
                ]
            ]
        ]
    ]
]
\end{forest}
\end{figure}

%\Tree [.CP {\emph{Den boka}}  [.C\1 [.C\\{\emph{har}} ] [.TP [.\emph{jeg} ] [.T\1 [.T\\\sout{\emph{har}} ] [.NegP [.Neg\\\emph{ikke} ] [.\emph{v}P [.\emph{v}\\\sout{\emph{har}} ] \qroof{\emph{\sout{jeg} lest \sout{den boka}}}.\emph{v}P ] ] ] ] ] ]




\noindent As shown in \figref{tree:V2-homeland}, the finite verb \emph{har} `have' (in this case an auxiliary) moves from the \emph{v}P-domain, through T and to C.\footnote{For simplicity, we treat auxiliaries as elements adjoined to \emph{v}P (like, e.g., \citealt{faarlund2019syntax}), although other analyses are available.} The preverbal position is filled by the fronted object \emph{den boka} `that book', which also originates within \emph{v}P.   The non-finite verb \emph{lest} `read' remains \emph{in situ} in a \emph{v}P-internal position. As the tree structure indicates, the subject \emph{jeg} `I' has moved from its externally merged, \emph{v}P-internal position past the negation \emph{ikke} to the specifier of T. An important discussion in Germanic syntax has centered around whether subject-initial V2 clauses (such as \REF{ex:frontedsubj}) have the same structure as non-subject-initial V2 clauses. As the subject precedes the verb, subject-initial V2 clauses are linearly compatible with verb movement to T instead of C. For Norwegian, the analysis whereby the verb always moves to C in main clauses (sometimes referred to as the ``symmetric'' analysis of V2) has  been mostwidely adopted (e.g., \citealt{faarlund2019syntax}), although some authors have also argued for an asymmetric analysis whereby the verb stays in T in subject-initial clauses (\citealt{holmberg2015V2} provides an overview of the  discussion; cf. also \citetv{chapters/anderssen}). 





Embedded clauses have the same basic spine as main clauses (\emph{v}P-TP-CP). However, they differ in terms of verb placement. In embedded clauses, the finite verb generally does not  move;  it remains in situ, inside \emph{v}P. The \emph{v}P-internal position is evidenced by the relative order of the finite verb and negation and other sentence adverbials: in embedded clauses, the verb normally  follows sentence adverbials, which are taken to mark the boundary between \emph{v}P and TP (e.g., \citealt{platzack2011fantastiska}).\footnote{This is the traditional view in Scandinavian syntax; see \citet{wiklundetal2009rethinking} for an alternative approach. Although  V-in-situ is the general pattern in embedded clauses, certain clause types optionally allow verb movement; see \citet{ringstad2019embedded} for a recent corpus study and, e.g., \citet{julien2015force} and references therein for in-depth discussion of the conditions and structural analysis of this phenomenon. Verb movement in embedded clauses in homeland Scandinavian is often labeled embedded V2, reflecting the view that embedded clauses with verb movement are structurally similar to main clauses.}
 Cf. \REF{ex:embedded}:

\ea\label{ex:embedded}
\ea\label{ex:embedded-a}
\gll Hvis jeg \textbf{ikke} \textbf{leser} boka, blir jeg ikke klok. \\
If I not read book.\textsc{def} become I not wise \\
\glt `If I don't read the book, I will not become wise.'
\ex
\gll Jeg angret på at jeg \textbf{aldri} \textbf{leste} boka. \\
I regretted on that I never read book.\textsc{def} \\
\glt `I regretted that I never read the book.'
\ex
\gll Dette er boka som jeg \textbf{heldigvis} \textbf{leste}. \\
This is book.\textsc{def} that I thankfully read \\
\glt `This is the book that I thankfully read.'
\z
\z

\noindent The structure of of the embedded clause in \REF{ex:embedded-a} is shown in \figref{tree:embedded-homeland}.

\begin{figure}
\caption{V-in-situ in embedded clauses in Norwegian.}
\label{tree:embedded-homeland}
\begin{forest}
[CP
    [C\\\textit{hvis}][TP
        [\textit{jeg}][T$'$
            [T][NegP
                [Neg\\\textit{ikke}][\textit{v}P
                    [\textit{\sout{jeg} leser den boka},roof]
                ]
            ]
        ]
    ]
]
\end{forest}
\end{figure}
%\Tree [.CP [.C\\{\emph{hvis}} ] [.TP [.\emph{jeg} ] [.T\1 [.T ] [.NegP [.Neg\\\emph{ikke} ] \qroof{\emph{\sout{jeg} leser den boka}}.\emph{v}P  ] ] ] ] 

The pattern whereby the finite verb moves to C in main clauses but remains in a lower position in embedded clauses is traditionally analyzed in terms of competition for the C position: in embedded clauses, this position is occupied by a complementizer, which means that the verb cannot move there (this account is often attributed to \citealt{denbesten1983interaction}). On the most common view, EurNo is assumed not to  have independent verb movement to T; i.e., the verb does not move to T unless it also moves further up to C (e.g., \citealt{faarlund2019syntax}). However, this  has not always been the case; Old Norse had V-to-T movement in embedded clauses (evidenced by the fact that the verb would precede sentence adverbials) \citep{faarlund2004syntax}, and this  word order is still the main rule in Modern Icelandic \citep[43]{thrainsson2007syntax}. 

\subsection{Nominals}\label{subsec:nominals}
The chapters in this book adopt the DP hypothesis \citep{abney1987englishNP}, which implies that (argumental) nominal phrases are headed by a determiner, or more precisely, a functional projection D, which takes the further lexical and functional projections  of the nominal  in its complement position. A simplified sketch is given in \REF{tree:simplifiedDP}:

\ea
\label{tree:simplifiedDP}
\begin{forest}
[DP
    [\dots][D$'$
        [D][NP]
    ]
]
\end{forest}
\z

%\Tree  [.DP [.... ] [.D\1 [.D ] \qroof.NP ] ]

One of the phenomena that calls for a more elaborate syntactic structure of nominals in EurNo is definiteness marking. Norwegian marks definite nouns with a suffix (\emph{-en} for masculine nouns, \emph{-a} for feminine nouns and \emph{-et} for neuters). This suffix can co-occur with prenominal determiners/demonstratives (see \REF{ex:det+suffix}), which means that it cannot be (externally) merged in D; it must  have a position further down in the structure. 

\ea \label{ex:det+suffix}
    \ea 
    \label{ex:det+suffix-a}
    \gll \textbf{Denne} \textbf{bok-a} er spennende \\
    this book-\textsc{def} is exciting \\
    \glt `This book is exciting.'
    \ex 
    \label{ex:double-def}
    \gll Jeg vil ha \textbf{den} røde \textbf{bok-a} \\
         I want have the red book-\textsc{def}\\
    \glt `I want the red book.'
    \z
\z

\noindent The DP-related chapters in this volume take (versions of) Julien's (\citeyear{julien2002determiners}, \citeyear{julien2005nominals}) analysis  of nominals in Scandinavian as their starting point. \citet{julien2002determiners} proposes a functional projection ArtP, which hosts the definite suffix.\footnote{The terminology used for this projection varies; ArtP corresponds to DefP in \citet{lohnput21} (see \figref{lohnputree}); \citet{julien2005nominals} uses the label \emph{n}P, which we will not use, as it could be confused with the nominal categorizer situated lower down in the structure. } ArtP is located below D, but above number features, the categorial feature N and the lexical root. The noun Agrees with the features in Num and Art and moves up to  Art; the structure of the nominal \emph{denne boka} `this book' \REF{ex:det+suffix-a} is sketched in \figref{tree:introducingart}.\footnote{\citet{julien2005nominals} and others propose that demonstratives are  placed in a head higher than D; we abstract away from that here. See \citet[n. 17]{kinnlarsson2022pronominal} for discussion.}

\begin{figure}
\begin{floatrow}
\ffigbox
    {\begin{forest}
        [DP
            [D\\\textit{denne}][ArtP
                [Art\\\textit{boka}][NumP
                    [Num\\\textit{\sout{boka}}][NP
                        [\textit{\sout{boka}},roof]
                    ]
                ]
            ]
        ]
    \end{forest}}
    {%
        \caption{Syntactic structure of DP in example (\ref{ex:det+suffix-a}).}
        \label{tree:introducingart}
    }

\ffigbox
    {\begin{forest}
        [DP
            [D][CardP
                [Card][$\alpha$P
                    [$\alpha$][ArtP
                        [Art][NumP
                            [Num][NP]
                        ]
                    ]
                ]
            ]
        ]
    \end{forest}}
    {%
        \caption{Overview of Norwegian nominal phrase \citep{Julien2005}.}
        \label{Tree:overviewDP}
    }
\end{floatrow}
\end{figure}

%\Tree [.DP [.D\\\emph{denne} ] [.ArtP  [.Art\\\emph{boka}  ] [.NumP [.Num\\\emph{\sout{boka}} ]  \qroof{\emph{\sout{boka}}}.NP ] ] ]   
%\Tree [.DP [.D ] [.CardP [.Card ] [.$\alpha$P [.$\alpha$ ] [.ArtP [.Art ] [.NumP [.Num ] \qroof{}.NP ] ] ] ] ]

Further functional categories within the DP are $\alpha${}P, which hosts adjectival phrases, and CardP, which hosts the indefinite article, numerals and weak quantifiers. 
An overview of the nominal phrase is given in \figref{Tree:overviewDP} (based on \citealt[267]{julien2002determiners}).\footnote{Note that CardP and $\alpha$P are only present when they contain lexical material.}

An important property of EurNo (and Mainland Scandinavian more widely) is that  the D projection must generally be phonologically  realized (\citealt{julien2002determiners}, \citeyear{julien2005nominals}). This requirement can be fulfilled in two ways: by insertion of an overt determiner in the D head, or by phrasal movement of material up to the specifier of D.  For illustration, cf. the nominals in  \REF{ex:forillustrationofid}:

% % % \begin{multicols}{2}
\ea \label{ex:forillustrationofid}
\ea\label{ex:denneboka}
\gll denne boka \\
    this book.\textsc{def} \\
\glt `this book'
\ex \label{ex:boka}
\gll boka \\
    book.\textsc{def}\\
\glt `the book'
\z
\z
% % % \end{multicols}

The syntactic structures are sketched in \REF{tree:boka_denneboka} (14a is a simplified version of Figure 4):
\newpage
\ea \label{tree:boka_denneboka}
\begin{multicols}{2}
\ea\label{tree:denneboka}
\begin{forest}
[DP
    [D\\\textit{denne}][ArtP
        [\textit{boka},roof]
    ]
]
\end{forest}
%\Tree [.DP [.D\\\emph{denne} ] \qroof{\emph{boka}}.ArtP ] 

\ex\label{tree:boka} 
\begin{forest}
[DP
    [ArtP
        [\textit{boka},roof]
    ][D$'$
        [D][\sout{ArtP}
            [\textit{\sout{boka}},roof]
        ]
    ]
]
\end{forest}
\z
\end{multicols}
\z
%\Tree [.DP \qroof{\emph{boka}}.ArtP  [.D\1 [.D ] \qroof{\sout{\emph{boka}}}.\sout{ArtP} ] ]  

In \REF{tree:denneboka}, D is identified by the overt determiner \emph{denne} `this'; in \REF{tree:boka}, D is identified by a phrase (ArtP with all its contained material) moving to its specifier. 

Having discussed some basic properties of clauses and nominals in EurNo and how they are analyzed, we now turn to NAmNo and the available linguistic data for this heritage variety. 



\section{Speakers, corpus data and glossing conventions}\label{sec:speaker,corpusdata,glossing}

NAmNo is a heritage language with a relatively long history. Many of the speakers who were recorded during the field trips in the 2010s (see \sectref{intro}) can trace their Norwegian heritage three or four generations back (this is evident from the CANS metadata). The present-day speakers have generally acquired Norwegian in the home in their childhood; on the historical context, the speakers' linguistic backgrounds and profiles as language users, see further \citealt{eidehjelde2023modelling} and \textcitetv{chapters/hjelde}. Norwegians moving to the US is, of course, not strictly a phenomenon of the past -- there is still scattered migration, and, consequently, children being born into families where Norwegian is spoken in the home. In this volume, however, we primarily deal with with the language of speakers who have ties to the waves of migration in the 19th and early 20th century.  

Most of the chapters in this volume use CANS as their main source of NAmNo data. In its current version (version 3.1), CANS comprises 729,000 word tokens of spontaneous speech produced by 246 individuals of Norwegian heritage in the US and Canada. This is considerably more than the first version described by \citet{johannessen2015cans}, which included 131,000 word tokens by 36 speakers.

Most of the data in CANS was recorded in 2010 or later. However, the most recent version  also includes older data collected by Didrik Arup Seip, Ernst Selmer and Einar Haugen in the 1930s and 1940s, as well as some data collected by Arnstein Hjelde in the 1980s/1990s. This facilitates diachronic studies of NAmNo as a heritage language, and as some of the speakers in the older parts of the corpus are 1st generation speakers, it can also help establish the best possible baseline for comparison with NAmNo as spoken today. Most of the chapters in this volume mainly describe present-day NAmNo from a synchronic perspective; however, several of the  authors also make use of the older data in CANS in their discussions (see in particular the chapters by van Baal, Larsson \& Kinn, Eide, Riksem \& Nygård and Putnam \& Søfteland). 

CANS has two levels of transcription; one semi-phonetic (orthophonic,  using the Latin alphabet), and one orthographic (Bokmål standard). In this volume, linguistic examples are mostly rendered in the orthographic transcription unless there are particular reasons to use the semi-phonetic level (in Eide's chapter; however, the examples are generally rendered in semi-phonetic form). Short pauses are marked with \#, as shown in \REF{ex:glossing_conventions}; longer pauses are marked with \#\#. 

\ea\label{ex:glossing_conventions}
\ea
\gll kanskje \# e sønnen lever på farmen\\
 maybe {} eh son.\textsc{def} livs on farm.\textsc{def} \\
\glt `maybe the son lives on the farm' (harmony\_MN\_01gk)
\ex \label{ex:glossing_names}
\gll vi kan gå åt E3 \# med hun	F1	og \# og mannen hennes e \# M4 \\
we can go to E3 {} with she F1 and {} and husband.\textsc{def} her eh {} M4 \\
\glt `we can go to E3 with F1 and... and her husband M4' (coon\_valley\_WI\_03gm)
\z
\z

Every speaker in the corpus is identified by a code consisting of their place of residence at the time of recording, a number (01, 02 etc.) and a combination of the letters \emph{u/g} and \emph{m/k} (cf. \ref{ex:glossing_conventions}). The letter \emph{u} is used for speakers under the age of 50, whereas \emph{g} is for speakers above that age. \emph{M} means that the speaker is male; \emph{k} means female. If the speakers refer to other people by name, these names are rendered with codes (M1, M2 etc. for male names, F1, F2 etc. for female names, E1, E2 etc. for last names); cf. \REF{ex:glossing_names}. 

In the next section, we provide an overview of the chapters in this volume. 


\section{Overview of chapters} \label{overview}
As stated in \sectref{intro}, the chapters in this volume aim to provide an easily accessible overview of the research that has been done on the syntax of North American Norwegian to date -- in addition to some new observations that have not been presented elsewhere before.  As will become evident, some topics have been more extensively studied than others, and  different authors take different angles, both in their implementation of the theoretical framework, in their reasoning about the baseline question (i.e., what the heritage variety can be most meaningfully compared to; cf. \sectref{sec:clausesandnominals}), and in their methods. This is a reflection of the state of the art and  the relative diversity of approaches that exist in the field of heritage-language research, even among researchers working within the generative framework. With this as a starting point, it should be clear that there are still many avenues to explore in the syntax of NAmNo, and we hope that this volume as a whole can be a guide towards the most productive directions. 

The volume is divided into four parts. The first part consists of the present introduction and a chapter by Arnstein Hjelde, who delivers an overview of emigration from Norway to North America in the 19th and 20th centuries, and the establishment and characteristics of the immigrant communities in which Norwegian has been spoken as a heritage language up until today. This chapter serves as a backdrop to the remaining chapters, and as a reminder of the  conditions under which North American Norwegian speakers have lived their lives, which inevitably has had consequences for the language. 


The second part of the volume concerns the nominal domain. Brita Ramsevik Riksem and Mari Nygård discuss agreement in number, gender, and definiteness within the DP. This chapter places particular emphasis on a type of nominals which has been observed in NAmNo for a long time, namely ``mixed'' phrases, with lexical material from both Norwegian and English. The chapter explores the theoretical implications of such mixing.  Definiteness is also investigated in Yvonne van Baal's chapter, but from a different perspective: rather than examining the extent to which the elements in the DP, such as determiners, agree, van Baal considers whether determiners and suffixes  expressing definiteness are present at all. Van Baal shows that the least stable contexts for definiteness marking are those which require double definiteness (both a prenominal determiner and a suffix) in EurNo. Interestingly, the most common pattern of innovation in this context is one that does not involve convergence with English. In her chapter, van Baal discusses the consequences of this and other findings regarding definiteness. The final chapter on the nominal domain, written by Kari Kinn, discusses DP-internal possessive constructions, of which (European) Norwegian has quite a large range. The distribution of some of these constructions in NAmNo has been been studied in previous research; this chapter, which is mainly empirically oriented, includes new data on additional possessive constructions and shows that most of the options for expressing possessive relations seem to have been retained in the heritage language. 

In the third part of the volume, we turn to the properties of clauses and the domains of the syntactic spine, \emph{v}P, TP, and CP (see \ref{triple}). Kristin M. Eide's chapter explores morphosyntactic expressions of tense, modality and aspect, categories that are traditionally associated with the TP-domain. One of Eide's key findings is the general trend of the retention of EurNo-like patterns of tense, modality, and aspect, while illustrating ways in which NAmNo shows divergence from its European counterpart through innovation. Mike Putnam and Åshild Søfteland discuss the structure of non-finite complement clauses (infinitives and gerunds). An important question about non-finite complements regards their syntactic size -- a core proposal in the chapter is that NAmNo speakers avoid bare TPs as complements, which is consistent with what is observed in other Germanic varieties, except English. Putnam and Søfteland discuss the implications of this for heritage language syntax.  The chapter by Merete Anderssen, Helene R. Jensberg, Terje Lohndal, Björn Lundquist and Marit Westergaard centers on finite clauses, more specifically on verb placement in main and embedded clauses. The chapter shows that in main clauses, V2, which is characteristic for Germanic languages apart from English, generally remains robust in NAmNo. In embedded clauses, on the other hand, verb placement is considerably more vulnerable, with a strong tendency for the verb to appear in a higher position than in EurNo, before negation and adverbials. The authors discuss possible reasons for these patterns and argue that several factors interact to yield the instability observed in embedded clauses. In the final chapter in this part of the volume, Ida Larsson and Kari Kinn discuss argument placement, more precisely the position of subjects and objects relative to negation (subject shift and object shift), and the position of objects relative to verb particles. Larsson and Kinn take the position that argument placement is largely stable over time, although there are fluctuations within the limits of the baseline grammar. Notably, there is an increased preference for subject-initial clauses (also described in Anderssen et al.'s chapter) which is argued to have certain knock-on effects for argument placement further down in the clause. 

The fourth part of the book comprises one chapter, written by Putnam \& Kinn. This chapter includes concluding remarks and points out directions for future research on NAmNo. 

\section*{Abbreviations}

\begin{multicols}{2}
\begin{tabbing}
MMMM \= Article\kill
ArtP \> Article Phrase \\
CANS \> Corpus of American \\ \> Nordic Speech \\
CP \> Complementizer Phrase \\
\textsc{def} \> Definite \\
DP \> Determiner Phrase \\
\textsc{gen} \> Gender \\
EurNo  \> European Norwegian\\
Fem \> Feminine \\
LF \> Logical Form \\
NAmNo \> North American Norwegian\\
\textsc{num} \> Number \\
PF \> Phonological Form \\
\textsc{prog} \> Progressive (aspect) \\
Sg \> Singular\\
VP \> Verb Phrase 
\end{tabbing}
\end{multicols}

\section*{Acknowledgements}
We would like to acknowledge and thank all of the contributors to this volume for their efforts. The majority of these authors also assisted in the reviewing of the individual chapters, so we thank them and the external reviewers for their help in improving and strengthening this volume as a whole. A tremendous amount of this research has come to life through fruitful discussions and exchanges in connection with WILA (Workshop on Immigrant Languages in the Americas). We thank all those affiliated with WILA past and present for the collaborative spirit and enjoyable conferences over the years. WILA was founded by Joe Salmons and Janne Bondi Johannessen, and although we're still fortunate to benefit from Joe's contributions to the field, we would like to recognize and remember Janne and dedicate this volume to her memory. In that same spirit, we thank all of our informants past, present, and future for not only sharing their language with us, but also for providing the unique opportunity of inviting us into their homes and into their lives. We're grateful to have these opportunities and to form these unique bonds. Finally, we would like to thank the Research Council of Norway (project 301114) for generous financial support and our research assistant Nikolai H. Krohne.


\printbibliography[heading=subbibliography,notkeyword=this]


\end{document}
