\documentclass[output=paper,colorlinks,citecolor=brown]{langscibook}
\ChapterDOI{10.5281/zenodo.15274576}
\author{Michael T. Putnam\orcid{}\affiliation{Penn State University} and Kari Kinn\orcid{}\affiliation{University of Bergen}}
\title{Epilogue}
\abstract{This chapter concludes the present volume. It synthesizes the findings of the previous chapters and points out some  under-researched areas and directions for future research.}

\IfFileExists{../localcommands.tex}{
   \addbibresource{../localbibliography.bib}
   % add all extra packages you need to load to this file

\usepackage{tabularx,multicol}
\usepackage{url}
\urlstyle{same}

\usepackage{listings}
\lstset{basicstyle=\ttfamily,tabsize=2,breaklines=true}

\usepackage{langsci-basic}
\usepackage{langsci-optional}
\usepackage{langsci-lgr}
\usepackage{langsci-osl}
% \usepackage{./langsci/styles/langsci-lgr}
% \usepackage{./langsci/styles/langsci-osl}
% \usepackage{langsci-gb4e}

\usepackage{tikz}
\usetikzlibrary{patterns,calc}
\pgfdeclarepatternformonly{south east lines}{\pgfqpoint{-0pt}{-0pt}}{\pgfqpoint{3pt}{3pt}}{\pgfqpoint{3pt}{3pt}}{
    \pgfsetlinewidth{0.6pt}
    \pgfpathmoveto{\pgfqpoint{0pt}{3pt}}
    \pgfpathlineto{\pgfqpoint{3pt}{0pt}}
    \pgfpathmoveto{\pgfqpoint{.2pt}{-.2pt}}
    \pgfpathlineto{\pgfqpoint{-.2pt}{.2pt}}
    \pgfpathmoveto{\pgfqpoint{3.2pt}{2.8pt}}
    \pgfpathlineto{\pgfqpoint{2.8pt}{3.2pt}}
    \pgfusepath{stroke}}
    
\usepackage{stmaryrd}
\usepackage{wasysym}
\usepackage{multirow}
\usepackage{caption}
\usepackage{subcaption}
\usepackage{mathrsfs}
\usepackage{qtree}

\usepackage{linguex}


   %pminos do not split footnotes
% \interfootnotelinepenalty=10000 %Footnote in Laporte chapters has to be split SN


%\DeclareIndexNameFormat{default}{%
%\nameparts{#1}%
%\usebibmacro{index:name}%
%{\index[names]}%
%{\namepartfamily}%
%{\namepartgiveni}%
% {}% L1
% {}% L2
%{\namepartprefix}% generates spurious space L3
%{\namepartsuffix}% generates spurious space L4
%}

%  {\DeclareIndexNameFormat{default}{%
%     \usebibmacro{index:name}{\index[names]}{#1}{#3}{#5}{#7}}}

%\DeclareIndexNameFormat{default}{%
%  \usebibmacro{index:name}{\sindex[nom]}{#1}{#3}{#5}{#7}}

%\DeclareIndexNameFormat{default}{%
%  \usebibmacro{index:name}{\sindex[person]}{#1}{#3}{#5}{#7}}
%\DeclareIndexNameFormat{default}{%
%\nameparts{#1} \usebibmacro{index:name}{\sindex[person]]}{\namepartfamily}{‌​\namepartgiven}{\nam‌​epartprefix}{\namepa‌​rtsuffix}}

%\newcommand{\smiley}{:)}

%\renewbibmacro*{index:name}[5]{%
%\usebibmacro{index:entry}{#1}%
%{\iffieldundef{usera}{}{\thefield{usera}\actualoperator}\mkbibindexname{#2}{#3}{#4}{#5}}}

% \newcommand{\noop}[1]{}

%remove for final
%\overfullrule=1mm

\newcommand{\tobi}[2]}}
\renewcommand{\S}[1]{\tobi{#1}{\textsc{*}}}

% this volume references
% puts: [this volume]
% already defined: \citetv
%\newcommand{\citepv}[1]{(\citeauthor{#1} \citeyear*{#1} [this volume])}
\newcommand{\citealtv}[1]{\citeauthor{#1} \citeyear*{#1} [this volume]}

%parentheses around example number
\newcommand{\pref}[1]{(\ref{#1})}

% in-text examples

\newcommand{\lnex}[1]{\textit{#1}} %target lang word
\newcommand{\lnlit}[1]{(lit.: `#1')} %literal reading
\newcommand{\lnlat}[1]{(#1)} % latinization
\newcommand{\lntrans}[1]{`#1'} %translation
\newcommand{\lnexl}[2]%
{\lnex{#1}{} \lnlat{#2}} % ex with latinization
\newcommand{\lnexlat}[3]{\lnex{#1}{} \lnlat{#2}{} \lntrans{#3}} % ex with latinization and tranl.

%ch01
\newcommand{\co}[1]{\mbox{\textbf{#1}}}

%ch09

\newcommand{\cyrbulg}[1]{\begin{otherlanguage*}{bulgarian}#1\end{otherlanguage*}}


%ch10
\newcommand{\nlp}{{\small NLP}}
\newcommand{\mwe}{{\small MWE}}
\newcommand{\rae}{{\small RAE}}
\newcommand{\lvc}{{\small LVC}}
\newcommand{\pos}{{\small P}o{\small S}}
%\newcommand{\todo}[1]{ \textcolor{red}{#1} }

%\renewcommand{\labelenumi}{\theenumi}
%\ainamefmt{{vv}{ll}{, ff}{, jj}} % fullname

\newcommand{\biberror}[1]{{\color{red}#1}}

\newcommand{\osenovaitem}{--~}
   %% hyphenation points for line breaks
%% Normally, automatic hyphenation in LaTeX is very good
%% If a word is mis-hyphenated, add it to this file
%%
%% add information to TeX file before \begin{document} with:
%% %% hyphenation points for line breaks
%% Normally, automatic hyphenation in LaTeX is very good
%% If a word is mis-hyphenated, add it to this file
%%
%% add information to TeX file before \begin{document} with:
%% %% hyphenation points for line breaks
%% Normally, automatic hyphenation in LaTeX is very good
%% If a word is mis-hyphenated, add it to this file
%%
%% add information to TeX file before \begin{document} with:
%% \include{localhyphenation}
\hyphenation{
    Beck-man
    Ngu-yen
    back-chan-nel
    back-chan-nels
    mo-not-o-nous
    ste-reo-typ-i-cal
}

\hyphenation{
    Beck-man
    Ngu-yen
    back-chan-nel
    back-chan-nels
    mo-not-o-nous
    ste-reo-typ-i-cal
}

\hyphenation{
    Beck-man
    Ngu-yen
    back-chan-nel
    back-chan-nels
    mo-not-o-nous
    ste-reo-typ-i-cal
}

   \boolfalse{bookcompile}
   \togglepaper[23]%%chapternumber
}{}

\begin{document}
\maketitle

\section{Synthesis and summary of findings}

The previous chapters in this book deliver a detailed overview of a number of empirical domains and theoretical issues  relevant to the syntax of North American Norwegian. Although the topics covered in this volume are impressive and reflective of the research that has been carried out on this language over the course of many decades, these chapters do not represent an exhaustive treatment of the syntactic structure of this language. In all fairness, a comprehensive overview of the entirety of topics related to the syntax of North American Norwegian is simply not attainable. This state of affairs simultaneously represents a unique opportunity, while issuing a significant challenge to researchers moving forward. In many respects, this scenario is reminiscent of the traditional “carrot-and-stick” analogy. The “carrot” represents that abundance of remaining empirical domains and opportunities to advance and test theoretical axioms and proposals, while the “stick” relates to the fact that the remaining eldest and penultimate generation of North American Norwegian speakers in all likelihood represents the final generation(s) of highly proficient speakers. Although these challenges are certainly not unique to research on this heritage language \citep{DAlessandro2021}, they signal a shift in the research paradigm from limited fieldwork + corpus mining to exclusively investigating this language through the lens of what forms and constructions exist in CANS. 

One salient point that is consistently present throughout these chapters is the robust nature of the core elements of syntax \citep{Lohndal2021}, and beyond that, there are no signs of an extraordinary collapse or contraction of North American Norwegian syntax \citep{Bousquette2020}. Furthermore, despite arguments in favor of a “hybrid”, or “integrated” syntactic system \citep{Aboh2015,PutnamCarlsonReitter2018}, there is an abundance of evidence of a clear Norwegian-based syntax vs. an English-based variant. To mention but some examples, most speakers retain DP-internal agreement in grammatical gender \parencitetv{chapters/riksem}, uniquely Norwegian possessive constructions are not replaced by more English-like constructions \parencitetv{chapters/kinn}, and verb second is generally robust \parencitetv{chapters/anderssen}. These are findings consistent both with Minimalist theorizing and the individual implementations of this larger program in the individual chapters. In short, the analyses found in this volume are strongly supportive of a syntactico-centric model of grammar as found in the Minimalist Program. 

If, as argued above, structure-building properties associated with Norwegian grammar are largely unaffected, we should expect that the locus of language change -- either through the influence of English or due to the language's status as a moribund heritage language -- should be confined to functional heads and their interpretation at the external interfaces  \citep{LohndalPutnam2021,LohndalPutnam2024,PutnamEtAl2019}. Concerning the phenomena that appear to be most susceptible to instability\slash change, three immediately come to mind, i.e., (i) the licensing of double definiteness, with the omission of the pre-nominal determiner (\citealt{vanBaal2020,vanBaal2024Definiteness,chapters/vanbaal} [this volume]), (ii) the overgeneralization of verb movement in embedded clauses in contexts and structural positions that are not licit in homeland Norwegian \parencitetv{chapters/anderssen}, and (iii) the occasional generation of “additional” and “less” structure in non-finite complement verb phrases \citep{PutnamSøfteland2022}. It is worth mentioning that these three syntactic properties are generally considered to be acquired relatively late in L1 acquisition, which provides support for maturational constraints on heritage language acquisition and the subsequent quality of representations that stem from qualitative and quantitative differences in input \citep{Montrul2008,Montrul2016,PerezCortesEtAl2019}. 





\section{Lingering questions and under-researched domains}

Despite the fact that the chapters in this reference guide provide the most state-of-the-art assessment of the syntax of North American Norwegian, there are a number of empirical domains that remain under-researched and under-explored. In our closing comments, we point to a few of these areas, with the hope of encouraging continued research into the syntactic developments underway in North American Norwegian. One domain that has received little attention thus far in research on this heritage language pertains to issues related to A\textsuperscript{$\prime$}-dependencies (i.e., filler-gap dependencies more generally). Structures that fit under this description are basic \textit{wh}-movement phenomena, ellipsis, and relative clauses. A persistent challenge associated with researching A\textsuperscript{$\prime$}-dependencies in moribund heritage grammars such as North American Norwegian is the difficulty in finding enough speakers who can successfully participate in judgment tasks or any other experiments \citep{DAlessandro2021}. 

Another “long-distance” phenomenon, albeit as an instance of an A\hyp dependency, worthy of further study concerns the properties of anaphoric binding in North American Norwegian.  Preliminary observations of reflexive possessives \parencitetv{chapters/kinn} suggest stability of Norwegian binding patterns, but also some interesting variation. Further inquiries into anaphoric binding should not only include the syntactic configurations that regulate the antecedent--anaphor or antecedent--pronoun relations, but also the position of these anaphors\slash pronouns. The realization of these forms has the potential to reveal significant insights about this facet of the language. Two final areas of syntax stand out at this juncture to us as suitable candidates for fruitful exploration: First, argument structure and grammatical voice alternations (e.g., passive, anticausatives, etc.) deserve more targeted attention moving forward. Finally, despite nascent work on non-finite clauses of North American Norwegian summarized by \textcitetv{chapters/putnam}, there remains a number of issues, such as a more thorough investigation of translational equivalents to gerund-like structures, that warrant additional research. The primary takeaway from this volume as a whole is the increased need for interpreting findings from heritage language research through the lens of generative grammar. For example, an appeal to representational economy as initially set forth by \citet{Scontras2018} (but see \citealt{Putnam2025} for a updated assessment of its application in heritage language syntax) has assisted many of the authors of this volume in establishing testable hypotheses and predictions in connection with observed changes. The axioms found in current generative approaches to syntactic change have thus proven to be valuable to not only interpreting empirical findings, but also in theory-building efforts -- especially those that integrate heritage language data in larger theoretical discussions and debates. 

%\textcolor{blue}{What remains to be said: 
%\begin{itemize}
%    \item Highlight and discussion of any major `red threads' that we observed in these papers (with an eye towards theoretical issues and theory-building efforts)  
%    \item Something on representational economy?  
%\end{itemize}
%}

As mentioned, future research on NAmNo will primarily rely on the existing CANS data. While there are still many important questions that can be explored through the CANS data alone, the corpus can also be used in different ways. One perspective that stands out is a comparative perspective. To date, NAmNo is the only heritage variety of Norwegian that has been extensively researched from both a synchronic and diachronic perspective. This is a typical situation in the sense that it instantiates a dyad in which English is the dominant language  \citep{Scontras2020}. However, heritage Norwegian, like many other heritage languages, is not confined to the US or other English-speaking countries. Recently, research on Norwegian as a heritage language in Latin America has begun (\citealt{KinnEtAl2024, KinnHjeldeLundStokka2024}). It will be interesting and exciting to compare findings on the syntax of North American vs. Latin American Norwegian in the years to come. This can enhance our understanding of the effects of  different majority languages on syntax. 

In closing, although we have reached the likely twilight of remaining (highly) proficient speakers of North American Norwegian, the research on this vernacular and its important contributions to syntactic change continues to move forward. In our humble opinion, there is no more fitting way to honor the legacy and heritage of those who spoke, and those who continue to proudly speak, North American Norwegian.

\printbibliography[heading=subbibliography,notkeyword=this]

\end{document}
