\addchap{Preface}
\begin{refsection}

When an area of scientific inquiry warrants the creation of an edited volume, this volume usually comes into existence for one of two reasons: First, the volume brings together cutting-edge research on a particular topic that scholars are actively engaged with. Second, the volume functions as a capstone of sorts for excellent research that has been carried out on a topic by well-established scholars. We’re of the opinion that this volume before you speaks to both of the reasons. We intentionally chose to refer to this as a reference guide rather than a handbook, because although we, as a community of scholars, have made sizeable gains in our collective and individual research to better understand the structural properties of North American Norwegian, much important research remains to be carried out (as we outline and mention in the Epilogue to this volume). 

One of the primary motivations behind the creation of this reference guide is to celebrate the continuation of research on the Norwegian language in North America. The pioneering research of Prof. Einar \citeauthor{Haugen1953}, in particular his two-volume work entitled \textit{The Norwegian language in America: A study in bilingual behavior} (\citeyear{Haugen1953, Haugen1969}), has had a profound and lasting impact on both of us, both at the early stages of our careers into our current positions. It is an honor, and an equally ardent responsibility, to continue this tradition, as we are approaching the 200th anniversary of the departure of Restauration from Stavanger, which marks the beginning of Norwegian emigration to North America. The important research on North American Norwegian was rekindled in the late 1980s and early 1990s by scholars such as Arnstein Hjelde and later on by Janne Bondi Johannessen. It cannot be overstated how much we, as researchers of this variety of Norwegian, owe to both of them. Unfortunately, Janne is no longer with us, but we are confident that those who read this and were fortunate enough to know (of) her and her research will sense how she inspired us all. In addition to contributing to Haugen’s legacy, the contributions in this volume also contribute to ongoing research in theoretical syntax, most notably, in the Minimalist framework. In this respect, we hope this work serves as an addendum to Jan Terje \citeauthor{Faarlund2019}’s \textit{The syntax of Mainland Scandinavian} (\citeyear{Faarlund2019}) and related works. 

Since the 1990s, the study of North American Norwegian has benefited from detailed research from myriad perspectives. The creation of the Corpus or American Nordic Speech (CANS) at the University of Oslo, under the direction of Janne, has now enabled anyone anywhere with the opportunity to both enjoy this language, as well as research its structural traits. In addition to the creation of CANS, our understanding of the properties of North American Norwegian have increased due to the numerous theses, dissertations, book chapters, and articles that have focused on properties of this language. The contributors to this reference guide are individuals who have researched, and continue to research, aspects of North American Norwegian. We are thankful for their hard work and patience with us throughout the revision and publication process. 

To the best of our knowledge, this volume is the first of its kind to take a detailed look at various aspects of the syntax of a diasporic heritage variety such as North American Norwegian. We, alongside the contributors to this volume, believe that we have reached a point in the collective research on this variety to take stock, and, in some respects, to celebrate how far we’ve advanced our understanding of not only the syntactic properties of this heritage variety, but also how these findings have been instrumental in theory-building efforts. As we transition into a new and exciting era of research on the structural aspects of heritage Norwegian spoken in Latin America, the experience and knowledge gained thus far through our individual and collective efforts on investigating the North American variant will undoubtedly aid these efforts. We gratefully acknowledge the support of the Research Council of Norway, grant 301114, which has enabled us to expand the geographical scope of  research on heritage Norwegian, and also to delve deeper into the diachronic dimension of North American Norwegian by including recordings by Einar Haugen in CANS. As will be evident from several of the chapters in this book, the older recordings can provide new insights as to the interpretation of the findings from the present-day North American Norwegian speakers. The findings discussed in detail in this volume constitute our current understanding of many of these phenomena. However, discussions continue, and much exciting work is yet to be carried out.\bigskip\\
\noindent Kari Kinn \& Mike Putnam\\
Bergen, Norway \& University Park, PA\\
April 2025 

{\sloppy\printbibliography[heading=subbibliography]}
\end{refsection}
