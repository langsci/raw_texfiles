\documentclass[output=paper,colorlinks,citecolor=brown]{langscibook}
\ChapterDOI{10.5281/zenodo.15274570}
\author{Michael T. Putnam\orcid{}\affiliation{Penn State University \& The University of Greenwich} and Åshild Søfteland\orcid{}\affiliation{Østfold University College}}
\title{Non-finite complementation in North American Norwegian}
\abstract{In this chapter we take a closer look at the structure of non-finite clauses in North American Norwegian (NAmNo), building upon our ongoing work in this area \citep{Softelandetal2021,putsoft,putsoft23}. Although in many respects NAmNo infinitives and gerunds model what we find in European Norwegian, deviations from the latter show a unified trajectory. These different structures avoid the projection of “naked” TP-complements, which are commonly avoided in most Germanic languages except English \citep{christo2020}. After reviewing the general properties of non-finite complementation in NAmNo, we briefly turn to how these findings contribute to broader debates in heritage language syntax, most notably, with respect to the notion of \textit{Representational Economy} as originally formulated by \citet{scontras2018}.}

\IfFileExists{../localcommands.tex}{
   \addbibresource{../localbibliography.bib}
   \usepackage{langsci-optional}
\usepackage{langsci-gb4e}
\usepackage{langsci-lgr}

\usepackage{listings}
\lstset{basicstyle=\ttfamily,tabsize=2,breaklines=true}

%added by author
% \usepackage{tipa}
\usepackage{multirow}
\graphicspath{{figures/}}
\usepackage{langsci-branding}

   
\newcommand{\sent}{\enumsentence}
\newcommand{\sents}{\eenumsentence}
\let\citeasnoun\citet

\renewcommand{\lsCoverTitleFont}[1]{\sffamily\addfontfeatures{Scale=MatchUppercase}\fontsize{44pt}{16mm}\selectfont #1}
  
   %% hyphenation points for line breaks
%% Normally, automatic hyphenation in LaTeX is very good
%% If a word is mis-hyphenated, add it to this file
%%
%% add information to TeX file before \begin{document} with:
%% %% hyphenation points for line breaks
%% Normally, automatic hyphenation in LaTeX is very good
%% If a word is mis-hyphenated, add it to this file
%%
%% add information to TeX file before \begin{document} with:
%% %% hyphenation points for line breaks
%% Normally, automatic hyphenation in LaTeX is very good
%% If a word is mis-hyphenated, add it to this file
%%
%% add information to TeX file before \begin{document} with:
%% \include{localhyphenation}
\hyphenation{
affri-ca-te
affri-ca-tes
an-no-tated
com-ple-ments
com-po-si-tio-na-li-ty
non-com-po-si-tio-na-li-ty
Gon-zá-lez
out-side
Ri-chárd
se-man-tics
STREU-SLE
Tie-de-mann
}
\hyphenation{
affri-ca-te
affri-ca-tes
an-no-tated
com-ple-ments
com-po-si-tio-na-li-ty
non-com-po-si-tio-na-li-ty
Gon-zá-lez
out-side
Ri-chárd
se-man-tics
STREU-SLE
Tie-de-mann
}
\hyphenation{
affri-ca-te
affri-ca-tes
an-no-tated
com-ple-ments
com-po-si-tio-na-li-ty
non-com-po-si-tio-na-li-ty
Gon-zá-lez
out-side
Ri-chárd
se-man-tics
STREU-SLE
Tie-de-mann
}
   \boolfalse{bookcompile}
   \togglepaper[23]%%chapternumber
}{}

\begin{document}
\maketitle

\section{Non-finite complementation and the role of T in Germanic}\label{putnam-intro}
When one investigates the syntax of Germanic languages from a generative perspective, one particular domain in which the past and present languages and dialects of this family differ is in relation to the properties of the functional head T \citep{bobjonas96,bobthrain98}. A significant number of syntactic phenomena, e.g., C/TP-expletives, subject-to-subject raising, scrambling, object shift, (in)coherent infinitives, sluicing, (a)symmetric V2, etc., hinge on the parametric variation of the nature of T and whether or not Spec,TP represents a licit final subject position in a particular language. English and European Norwegian principally differ in their preference for licensing T, with European Norwegian T not functioning as a subject position in most instances (whereas English allows this). This key distinction leads to a number of contrasts that are highly relevant for comparative research focusing on the English--Norwegian dyad. As a case in point, consider Exceptional Case-Marked verbs. \citet{christo2020} suggest that Germanic languages have two different subtypes of ECM verbs, namely, ECM1 and ECM2 (examples in (\ref{chris1}) are from \citealt[393]{christo2020}). They argue that ECM1-predicates appear in all Germanic languages with causative and perception verbs (\ref{ECM1e} and \ref{ECM1n}). An important feature of ECM1-predicates is that they do not co-occur with an infinitival marker, which suggests that the structures they occur in are smaller than TP (likely \textit{v}P). ECM2-predicates contrast with their ECM1-counterparts with respect to their presence of the infinitival marker (\ref{ECM2}). 

\begin{exe}
\item \label{chris1}
    \begin{xlist}
    \item \label{ECM1e} I saw/let him steal my ice cream. \hfill[ECM1]
    \item \label{ECM1n} Jeg lot ham stjele iskremen min. \hfill[ECM1]
    \item \label{ECM2} Leo believes/expects me to like ice cream. \hfill[ECM2]
    \end{xlist}
\end{exe}

Germanic languages vary with respect to whether or not they license ECM2-predicates. European Norwegian  does not license ECM2-predicates with predicates such as \textit{say/claim} and \textit{believe}, and only licenses them in a restricted sense with \textit{consider} and \textit{expect}, as shown in (\ref{2ECM2}--\ref{goodNoECM2}). 

\begin{exe}
\item \label{2ECM2}
     \gll *Jeg forventer ham \aa {} drepe musen. \\
          I expect him \textsc{inf} kill mouse.\textsc{def} \\
         \glt Intended: `I expect him to kill the mouse.' \hfill[ECM2] \\
         \citep[394]{christo2020}
\end{exe}

Only under exceptional circumstances do we find ECM2-predicates in European Norwegian. According to \citet{lodrup2008}, these are only possible with “dislocated” subjects of individual-level predicates, as shown in (\ref{goodNoECM2}).\footnote{For a more detailed discussion of the restricted contexts in which Norwegian can potentially license ECM2-predicates, the reader is referred to \citet{lodrup2002,lodrup2008}.}

\begin{exe}
\item \gll Ingen forventer \textbf{l{\ae}rere} {\aa} v{\ae}re perfekte. \\
         nobody expects teachers \textsc{inf} be perfect \\
        \glt `Nobody expects teachers to be perfect.' \hfill[ECM2] \\
         \citep[3]{lodrup2002}\label{goodNoECM2}

\end{exe} 

In this chapter, we review and expand upon our ongoing work on non-finite complementation in North American Norwegian (NAmNo) \citep{Softelandetal2021,putsoft,putsoft23}. Although core operations and properties of the syntax of heritage languages are generally held to be quite sturdy \citep{lohndal2021}, our research to date in this domain shows that some degree of syntactic optionality and derivation is present \citep{silvia2011}. Following recent arguments put forth by \citet{putPDtoappear}, we classify differences in NAmNo non-finite complement clauses from what we would expect to find in European Norwegian, including relevant dialectal variation,\footnote{See Section \ref{section-shrink} for a discussion of \textit{baseline} in our case here.} as well as American English, as an instance of \textit{Representational Economy} \citep{scontras2018,polinsky2020}. In their own words, \citet[3]{scontras2018} define the process and outcome of \textit{Representational Economy} as follows:

\begin{quote}
    ...does the heritage grammar reduce or augment structure relative to the native baseline? The pressures driving each outcome are interestingly different. Finding reduced structure relative to the native baseline, heritage speakers will have likely prioritized representational economy, restructuring their grammar in favor of lighter-weight linguistic representations. Less articulated, more parsimonious structures (e.g., structures with fewer explicit agreement features or syntactic projections) could ease the load on working memory and might therefore be preferred to their fully-articulated brethren. 
\end{quote}

The validity of \textit{Representational Economy} and its relevance for our treatment of non-finite complementation in NAmNo rests on two factors, namely, (i) the integrated nature of the cognitive architecture of bilinguals \citep{putnam2018} and its resulting hybrid representations \citep{aboh2015}, and (ii) the unique contrast in the Norwegian--English dyad with respect to how (Spec,)T(P) is realized \citep{sconput20}. Although we find the proposal of \textit{Representational Economy}, and the notion of \textit{clausal-shrinking} to be largely on track, in this chapter we also investigate instances of apparent \textit{clausal-stretching}, i.e., situations in which the heritage grammar (in this case, NAmNo) appears to require \textit{additional} syntactic structure to license a particular phenomena. Based on these empirical observations, we offer a revised definition of \textit{Representational Economy} at the conclusion of this chapter.\footnote{It is worth noting that our call to revise the definition of \textit{Representational Economy} as introduced by \citet{scontras2018} does not refute their original definition, nor does it mean that we advance any claims on sentence processing being ``shallow'' \citep{ferreira02,sanford02,clahsen06}. Rather, we merely seek to clarify that \textit{Representational Economy} may in some instances imply additional functional projections in heritage representations when compared with homeland varieties.}

%Reducing this to the crux of the matter, grammars with hybrid representations attempt to generate structures that would not (radically) conflict with the tendencies of one of the source grammars, resulting in both clause-\textit{shrinking} and clause-\textit{stretching}. 

This chapter has the following structure: In \sectref{section-shrink} we review recent work on NAmNo non-finite clauses, demonstrating that alongside the dominant trend of structures that strongly resemble expected forms from European Norwegian, we find instances of both clause-\textit{shrinking} and clause-\textit{stretching}. We then investigate a number of properties associated with NAmNo non-finite clauses in general (i.e., A-movement) in \sectref{section-nonfinite}, arguing that the lack of “true” ECM-predicates and instances of subject-to-subject raising support our analysis in terms of Representational Economy. Finally, we provide a brief treatment of important theoretical points in \sectref{section-theory}, and list a number of domains that are fertile for ongoing research in \sectref{outro}. 

\section{Shrinking and expanding computational domains} \label{section-shrink}

In this section we highlight and summarize our recent research on non-finite clauses in NAmNo \citep{Softelandetal2021,putsoft,putsoft23}. In particular, we expand upon two salient trends that exist in non-finite clauses in NAmNo; namely (i) instances of clausal “shrinking” as well as (ii) examples of clausal “stretching”. In \sectref{imvariation} we take a closer look at the realization and position of infinitival markers in NAmNo. Building upon the findings of \citet{Softelandetal2021} and \citet{putsoft}, we comment on forms that resemble European Norwegian and those that differ from it, and in some cases, from English too. We turn to the rise of \textit{wh}-infinitives in NAmNo in \sectref{whinfs}, structures which are not attested in older, traditional varieties of European Norwegian, but commonly found in American English. 

It is difficult to identify baseline populations in heritage language research (see \citealt[Section 1.4]{polinsky18} and \citealt{roberta21}), which is a challenge we also face here. Most of the speakers in our samples from CANS report a mixed dialectal background and the Norwegian-speaking language communities they have been part of have often been multilectal (see \citetv{chapters/hjelde} for an overview of the dialectal landscape in NAmNo). In previous work we have discussed where the use of the infinitival marker /te/ in NAmNo could possibly come from, and one likely explanation is that it is a feature inherited from traditional rural dialects in European Norwegian (see \citealt{Softelandetal2021} for an overview of the most relevant structural variation in European Norwegian in our case here). Sometimes we refer to these dialectal variants in terms of a \textit{baseline}, even though this is somewhat problematic, especially when it comes to 3rd and 4th generations of speakers. At some points in the text we also refer to, apparently in comparison with, the modern written standard Bokmål. This does not mean that we consider Bokmål a baseline in itself for our heritage speakers, but the structure we refer to when doing so will often be the most common variant in European Norwegian today.

\subsection{Variation in infinitival marker placement} \label{imvariation}

As pointed out to us by an anonymous reviewer, arriving at a definitive position for infinitival markers in European Norwegian is a difficult and arduous task. In fact, the most tenable view in the literature is that the infinitival marker can occupy \textit{multiple} structural positions in Germanic languages \citep{faaretal97,faarlund2007,faarlund2003reanalyse,faarlund2015,wilder1988,wurmbrand2001,Wurmbrand2014Tense,wurmlohn2020,christo2020,Christensen2005,aafarli2003}. This is determined by the position of the non-finite verb in the clause alongside other requirements of the predicate. NAmNo appears to exhibit the same options for the placement of infinitival markers as in European Norwegian varieties \citep{putsoft,putsoft23,Softelandetal2021}. The tree structure in \figref{inftree} summarizes the various positions where the infinitival marker \textit{\aa} (and other lexical variants) can occur in the clausal architecture of both European Norwegian and NAmNo. 

\begin{figure}
\caption{Structural positions where \textit{\aa} can occur in Norwegian} 
\label{inftree}
\begin{forest}
[PP
    [\textit{\textbf{til}}]
       [C/FinP
          [\textit{\textbf{\aa}}]
             [TP
                [\textit{\textbf{\aa}}]
                   [\textit{v}P
                       [\textit{\textbf{\aa}}] 
                         [VP
                            [V] [...]
                          ] 
                    ]
                ]
          ]
]
\end{forest}
\end{figure}


Let's now turn to the empirical data that suggest that the infinitival marker in NAmNo can in fact occupy these multiple functional projections; consider the following examples of non-finite clauses in NAmNo (from the CANS corpus, described thoroughly in the introduction of this volume):\footnote{All examples from CANS here and in the following are rendered in standard orthography (Bokmål), except for the infinitival marker \textit{te}, alone or in combination with \textit{\aa}, as shown in (\ref{introexample}). The equivalent to \textit{te} in standard written Norwegian would have been 'til', but in almost all cases with this (former) preposition used as (part of) an infinitival marker, it is pronounced without the 'l' -- mostly as /te/ and occasionally as /ti/ or with other vowels. See \citet{Softelandetal2021} for details.}

\begin{exe}

\item \label{introexample}

\begin{xlist}

\item \gll jeg har ingenting \textbf{å} gjøre her i byen \\
    I have nothing \textsc{inf} do here in town.\textsc{def}\\
    \glt `I have nothing \textbf{to} do here in town.'\hfill (outlook\_SK\_06gm)
    \label{introexample1} 

\item \gll jeg er for gammel \textbf{te} \textbf{å} reise  \\
    I am too old \textsc{prp} \textsc{inf} travel\\
    \glt `I'm too old \textbf{to} go.' \hfill(hatton\_ND\_02gm)
    \label{introexample2}

\item \gll åh det er så godt \textbf{te} komme heim att  \\
    oh it is so good \textsc{inf} come home again \\
    \glt `Oh, it's so good \textbf{to} come home again' \hfill(sunburg\_MN\_11gk)
    \label{introexample3}

\item \gll jeg har lært \textbf{te} \textbf{å} lese noe norsk  \\
    I have learnt \textsc{prp} \textsc{inf} read some Norwegian \\
    \glt `I have learned \textbf{to} read some Norwegian.' \hfill(fargo\_ND\_02gm)
    \label{introexample4}

\item \gll når jeg var stor nok \textbf{å} kjøre bundle-wagon  \\
    when I was big enough \textsc{inf} drive bundle-wagon \\
    \glt `When I was big enough \textbf{to} drive a bundle wagon.' \hfill(decorah\_IA\_02gm)
    \label{introexample5}

\end{xlist}

\end{exe}

As attested in the examples in (\ref{introexample}), the realization of infinitival markers displays a certain degree of allomorphic variation in NAmNo, i.e., \textit{\aa} and \textit{te} in particular. For a more detailed discussion of this allomorphic distribution, see \citet{Softelandetal2021} and \citet[Section 2.2]{putsoft}. While (\ref{introexample1}) and (\ref{introexample2}) follow the most regular patterns in European Norwegian, (\ref{introexample3}) likely stems from (older, rural) dialectal patterns. Of particular interest, however, are examples (\ref{introexample4}) and (\ref{introexample5}), which are not considered to be found in European Norwegian (including dialectal variation). Although the vast majority of non-finite phrases follow what is found in European Norwegian, the existence of structures such as (\ref{introexample4}) and (\ref{introexample5}) requires deeper investigation. 

Following \citet{putsoft}, we comment on innovative structures in NAmNo that exhibit “more” structure, and, contrariwise, “less” structure than what is found in European Norwegian. First, let's take a closer look at forms with additional structure; i.e., those in which only a simple infinitival marker (either only \textit{\aa} or only \textit{te}) is expected. The NAmNo examples in (\ref{50sss}) exhibit a structural unit that can be analyzed as either [PP + infinitival marker] or a double infinitival marker, in structures where this is not attested in European Norwegian.

\begin{exe}

\item \label{50sss}

\begin{xlist}

\item \gll det er fin ting \textbf{te} \textbf{å} ha stor family  \\
        it is fine thing \textsc{prp} \textsc{inf} have big family \\
         \glt `It's a good thing \textbf{to} have a big family.' \hfill(saskatoon\_SK\_14gk)\label{ex57}

\item \gll jeg har lært \textbf{te} \textbf{å} lese noe norsk  \\
        I have learned \textsc{prp} \textsc{inf} read some Norwegian\\
         \glt `I have learned \textbf{to} read Norwegian to some extent.' \hfill(fargo\_ND\_02gm)\label{ex52}

\item \gll jeg skulle like \textbf{te} \textbf{å} være kan hende i Fargo \\
        I should like \textsc{prp} \textsc{inf} be may happen in Fargo \\
         \glt `I would like \textbf{to} be, maybe, in Fargo.' \hfill(portland\_ND\_02gk) \label{ex53}


\end{xlist}

\end{exe}

In contrast, we also encounter non-finite clauses  that display \textit{reduced} structure, i.e., those in which only the infinitival marker \textit{\aa} appears in NAmNo tokens when most or all European Norwegian varieties would require the combination of \textit{til \aa} or \textit{for \aa} (or other prepositions + \textit{\aa}). Consider the examples in (\ref{60sss}), where the structure that would be \textit{til \aa} in the written standard Bokm{\aa}l and most modern/urban dialects in European Norwegian (it could be only \textit{te} in some traditional/rural dialects) is simply realized as a bare infinitival marker \textit{\aa}:

\begin{exe}

\item \label{60sss}

\begin{xlist}

\item \gll men de er bedre enn meg \textbf{å} snakke og forstå norsk  \\
        but they are better than me \textsc{inf} speak and understand Norw.\\
         \glt `But they are better than me \textbf{to} speak and understand Norwegian.' \hfill(spring\_grove\_MN\_09gm)\label{ex61}

\item \gll jeg er ikke vant \textbf{å} snakke med du da veit du \\
        I am not used \textsc{inf} speak with you then you know \\
         \glt `I'm not used \textbf{to} talk to you, you know.' \hfill(sunburg\_MN\_10gm)\label{ex125}

\end{xlist}

\end{exe}

\citet{putsoft} analyze these pairs of non-finite clauses in NAmNo as instances of \textsc{hyperextension}, i.e., instances in which heritage speakers generate structures that are not found in homeland varieties of the heritage language, yet avoid converging with equivalent structures of the majority language \citep{kupisch2014adjective,putnamhoff21}. The primary culprit that results in structures that sometimes display additional or reduced clausal structure in connection with non-finite domains is the avoidance of lexicalizing complement clauses with a “naked” TP; i.e., a TP that can be a selected argument of a verb in a matrix clause. This is possible in English, i.e., with ECM-verbs in sentences such as \textit{I expect him to win}. In contrast, this is not possible in European Norwegian. The `solution' that some NAmNo speakers have adopted to circumvent the occurrence of “naked” TPs is the inclusion of prepositions as in (\ref{50sss}) (see the tree structure in \figref{inftree} for an illustration) in embedded clauses, in which non-finite complement clauses function as objects of said preposition. It is also worth noting that in English, non-finite clauses cannot function as objects of a preposition; however, in dialects of Norwegian, these structures occur frequently. 

\subsection{\textit{Wh}-infinitives} \label{whinfs}

We observe a similar trend to expand the computational domain of infinitival phrases in the context of \textit{wh}-infinitives in NAmNo. Consider the contrast between English and Norwegian exemplified in (\ref{ex1}). While English allows for modality to be represented, or \textit{lexicalized}, on either the \textit{wh}-element or the complementizer (\ref{fine1}) (commonly referred to as \textit{covert modality}; \citealt{bhatt1999,groen1982,portner1997}), modality must be lexicalized as a modal verb in traditional European Norwegian (cf. \ref{fine2}--\ref{fine3}). 

\begin{exe}

\item \label{ex1}

\begin{xlist}

\item I don't know [what to do.] \label{fine1} \hfill[English]

\item\label{fine2} \gll *Jeg veit ikke [hva {\aa} gj{\o}re.] \\
         I know not what \textsc{inf} do \\
         \glt Intended: `I don't know what to do.' \hfill[trad. European Norwegian]

\item \label{fine3} \gll Jeg veit ikke hva jeg skal/kan/m{\aa} gj{\o}re. \\
         I know not what I shall/can/must do \\
         \glt `I don't know what I should do.' 
 
\end{xlist}

\end{exe}

\citet{putsoft23} show that several NAmNo speakers seem to adopt the English-like strategy of licensing \textit{wh}-infinitives; see the examples in (\ref{firstamno}).\footnote{\citet{putsoft23} also mention the connection between the ability (or lack thereof) to license \textit{wh}-infinitives with subject and non-subject infinitival relatives. (See \citealt[Section 3.3]{putsoft23} for an overview of these data.) We do not consider infinitival relatives further in this review chapter and leave this interesting puzzle for future research.} 

\begin{exe} 

\item \label{firstamno}

\begin{xlist}


\item \label{first1} \gll jeg l{\ae}rte i skolen \textbf{{\aa}ssen} \textbf{{\aa}} snakke engelsk \\
        I learnt in school how \textsc{inf} speak English \\
         \glt `I learnt how to speak English in school.' \hfill(hatton\_ND\_01gm)

\item \label{first2} \gll men jeg lærte aldri \textbf{{\aa}ssen} \textbf{{\aa}} lese norsk \\
        but I learnt never how \textsc{inf} read Norwegian \\
         \glt `But I never learnt how to read Norwegian.' \hfill(coon\_valley\_WI\_08gm)

\end{xlist}
\end{exe}

Although the actual token number of items that display this English-like pattern is relatively low ($n=22$ spoken by 21 different speakers in CANS version 3.1\footnote{For half of these speakers we find examples of the regular Norwegian pattern as well, see appendix 1 in \citet{putsoft23}.}), the low frequency in the corpus does not diminish the novelty and importance of these forms \citep{roberta21}. After searching through the (modern) Norwegian part of the \textit{Nordic Dialect Corpus} (NDC, approx. 2 million tokens) and the \textit{LIA -- Corpus of older Norwegian dialect recordings} (LIA, approx. 3.5 million tokens) to investigate whether or not \textit{wh}-infinitives can be found in spoken (older) European Norwegian\footnote{See references and corpus webpages for details:  \citet{NDC} \url{http://www.tekstlab.uio.no/nota/scandiasyn/index.html}; \citet{LIAcorpus} \url{http://tekstlab.uio.no/LIA/index.html}.}, \citet{putsoft23} report only one single example exhibiting a \textit{wh}-infinitive, reinforcing the innovative nature of these structures in NAmNo. Nevertheless, we want to briefly mention here that recent research has discussed a possible emerging appearance of \textit{wh}-infinitives in European Norwegian \citep{sundekris2018}. Most young Norwegians today have a high competence in English, and possibilities of structural transfer are discussed in this work, initially building on data from a large web-corpus. Some of the young people in Sunde \& Kristofferen's study accept \textit{wh}-infinitives in grammaticality judgements, but none of the adults do, supporting an analysis that this is a new development in Norwegian, and at least not part of a baseline in any way for the NAmNo speakers. The fact that some European Norwegian adolescents find these types of structures acceptable, and possibly use them in speech and/or informal writing, opens the door to a broad range of research questions that should be pursued in more detail in future studies.

In addition to the standard English-like patterns observed in (\ref{firstamno}), we also find \textit{wh}-infinitives that exhibit a combination of the particle\slash preposition \textit{te} \emph{and} the infinitival marker \textit{{\aa}} (cf. \citealp{putsoft}): 


\begin{exe} 

\item \label{CANSexm} 

\begin{xlist} 

\item \gll veit ikke \textbf{hva} \textbf{te} \textbf{{\aa}} snakke omm \\
        know not what \textsc{prp} \textsc{inf} talk about \\
         \glt `(I) don't know what to talk about.' \hfill(saskatoon\_SK\_07gk)
         
         (Trad. Europ. Norw.: (Jeg) veit ikke hva jeg/vi skal/kan snakke om.)


\item \label{CANSexh} \gll han veit ikke \textbf{hva} for en \textbf{te} \textbf{{\aa}} sette først \\
        he knows not what for one \textsc{prp} \textsc{inf} put first \\
         \glt `He doesn't know which one to put first.' \hfill(sunburg\_MN\_01gm)
         
         (Trad. Europ. Norw.: Han veit ikke hva for en han skal/b{\o}r sette f{\o}rst.)

\end{xlist}
         
\end{exe}

Adopting a Spanning-analysis, \citet{putsoft23} suggest that the rise of covert modality in NAmNo can be understood as a preference for lexicalizing modality, represented as the modal ◇\textsubscript{\textit{D},$\rightarrow$}, in the C-domain, rather than allowing the marker to be realized independently as a modal verb lower in the tree structure (in ModP). To appreciate the impact that these structures have on our call for the inclusion of \textit{clausal-stretching}, let's discuss these examples in a bit more detail. We assume the underlying structure in \figref{putsofttree1} for embedded questions that license modality. 

In European Norwegian, which traditionally only allows embedded \textit{wh}\hyp questions to appear with overt modality, as illustrated above in example (\ref{fine3}), the \textit{wh}-item moves to C and the modal element is spelled-out, or lexicalized, as a modal verb.

\begin{figure}[h]
\begin{floatrow}
\ffigbox
    {\begin{forest}
        [CP 
           [C]
              [TP
                [T]
                  [MoodP
                    [{◇\textsubscript{\textit{D},$\rightarrow$}} ]
                      [VP 
                        [V] [wh
                        ] 
                        ]
                        ]  
                        ]
                        ]
    \end{forest}%
    }
    {%
        \caption{Functional heads in an embedded CP}
        \label{putsofttree1}
    }
    
\ffigbox
    {\begin{forest}
        [CP 
           [C, name=c]
              [TP
                [T]
                  [MoodP
                    [{◇\textsubscript{\textit{D},$\rightarrow$}} ]
                      [VP 
                        [V] [wh, name=wh
                        ] 
                        ]
                        ]  
                        ]
                        ]
    \draw[->, overlay] (wh) to[out=south west,in=south] (c);                
    \end{forest}}
    {%
        \caption{\textit{wh}-movement and the lexicalization of a modal verb}
        \label{putsofttree2}
    }
\end{floatrow}
\end{figure}

This situation is different whenever covert modality, i.e., whenever the head of MoodP is not lexicalized independently as a modal verb, is realized (as in English). In these cases, the \textit{wh}-item participates in “roll-up” movement, incorporating the contents of MoodP, T, and C. \citet{putsoft23} call this complex lexical item a “span”, in line with current theoretical discussions surrounding late-insertion approaches to the syntax--morphology interface \citep{svenonius2020span}. 

This state of affairs shows that restructuring strategies in heritage language syntax may not always result in “smaller” computational domains, but optionally could result in “expanded” structures (see \citealt{lohnput23} for a recent discussion of this phenomenon). The more complex span, i.e., \textit{wh}-element in \figref{putsofttree3}, involves the semantic content of multiple functional heads, and as a result, more syntactic structure. 

\begin{figure}[h]
    \begin{forest}
        [CP 
           [C, name=c]
              [TP
                [T, name=t]
                  [MoodP
                    [{◇\textsubscript{\textit{D},$\rightarrow$}}, name=d ]
                      [VP 
                        [V] [wh, name=wh
                        ] 
                        ]
                        ]  
                        ]
                        ]
    \draw[->, overlay] (wh) to[out=south west,in=south] (d);
    \draw[->, overlay] (d) to[out=west,in=south] (t); 
    \draw[->, overlay] (t) to[out=west,in=south] (c);     
    \end{forest}
    \caption{``Roll-up" movement associated with covert modality} 
    \label{putsofttree3}
\end{figure}

\section{The (current) status of TP in NAmNo} \label{section-nonfinite}

Based on our review of the status of non-finite clauses in Germanic and their classification as defective TP-projections in the contemporary languages (e.g., European Norwegian) and the empirical puzzles it presents, we can introduce some initial predictions concerning the licensing of non-finite clauses in NAmNo. The main point we wish to convey here is that how T is represented and realized in NAmNo is still likely to be fundamentally different than its status in American English. Based on this hypothesis, we make the following predictions in (\ref{predictions}):

\begin{exe}

\item Predictions: \label{predictions}

\begin{xlist}

\item We anticipate that instances and approximations of English raising (\sectref{raising}) and Exceptional Case-Marked (ECM) verbs (\sectref{ecmsection}) in NAmNo will be either (i) rare or (ii) structurally distinct from what is found in European Norwegian.
\item Based on the long-standing assumption that \textsc{pro} in control-structures requires a CP-projection, we should expect to regularly find related examples in NAmNo; see \sectref{control}. 
\item Based on the assumption that gerunds require less structure than the projection of T (i.e., these are usually analyzed as VPs), we predict that \textit{defective clausal} gerunds may occur in NAmNo somewhat frequently; see \sectref{gerundives}. 

\end{xlist}

\end{exe}

In the remainder of this section, we review data from the CANS corpus to put these predictions to the test. 

\subsection{Raising} \label{raising}

The Norwegian Reference Grammar (NRG, \citealt[1027]{faaretal97}) says that most constructions with raising verbs in Norwegian belong to the written domain (and mostly Bokmål, not Nynorsk). \citet[2]{lodrup2002} opposes ``the impression that raising from infinitival complements is a marginal phenomenon in Norwegian". He lists a lot of different raising predicates, in active and passive form, with and without the preposition \textit{til} preceding the infinitival marker \textit{\aa}. This point notwithstanding, NRG's overall comment is highly relevant for the NAmNo context: Most of the raising predicates in \citet{lodrup2002} are truly from the (formal) written domain and likely not part of the NAmNo variety, such as the Latin loanwords \textit{dedusere} `deduce', \textit{erklære} `state, declare', \textit{estimere}, `estimate', \textit{rapporteres} `be reported'. Many of them are (Low) German loanwords, previously or still banned in the Nynorsk written standard, such as \textit{anslå} `estimate', \textit{beregne} `estimate', \textit{forekomme} `seem', \textit{antydes} `be suggested'.\footnote{The treatment of raising in \citet[Section 11.4]{aafarli2003} is based on this formal register that includes these verbs.}

The very few examples we have found of these formal/writing-style raising predicates in the CANS corpus are either from modern, younger speakers that have partly lived in Norway (they have a different profile than most of the older heritage speakers, both in input and use of Norwegian), or from the oldest recordings of 1st generation immigrants:


\begin{exe} 

\item \label{raisingFormal}

\begin{xlist}

\item \label{raisingOld1} \gll og så var det mye andre regler som syntes \_ være kanskje nokså strenge \\
        and so were there much other rules that appeared \_ be maybe quite strict \\
         \glt `There were many other rules that appeared to be quite strict.' \hfill(viroqua\_WI\_04gm, 1942, 1st generation)

\item \label{raisingOld2} \gll og så blei jeg oppfordret te åsså komme ned da de hadde åttiårsfest \\
        and so was I encouraged \textsc{prp} \textsc{inf} come down when they had 80-years-party \\
         \glt `I was encouraged to come down when they had a birthday party.' \hfill(viroqua\_WI\_04gm, 1942, 1st generation)

\end{xlist}
\end{exe}


As reported by \citet{lodrup2002} and \citet{faarlund19}, there is a small number of raising constructions, with verb + prep. `til' (often pronounced /te/), that are frequently used in Norwegian and part of the domain of colloquial speech. These collocations are also present in the CANS data; the phrase \textit{komme til å} `will, is going to' is especially frequent (grammaticalized as a regular expression for future tense), and \textit{få X til å} `get, make' as well. There are also some examples of \textit{se ut til å} `seem' in the corpus.


\begin{exe} 

\item \label{raisingamno}

\begin{xlist}


\item \label{raising1} \gll det kom te \textbf{{\aa}} bli langt følge bakafor \\
        there came \textsc{prp} \textsc{inf} be long trail behind \\
         \glt `There would be a long trail behind.' \hfill(westby\_WI\_01gm)

\item \label{raising2} \gll hvis du kan få han te \textbf{{\aa}} snakke norsk \\
        if you can get him \textsc{prp} \textsc{inf} speak Norwegian \\
         \glt `..if you can get him to speak Norwegian.' \hfill(saskatoon\_SK\_01gk)

\item \label{raising3} \gll hun så ut te \textbf{{\aa}} være nokså kjekk jente \\
        she looked out \textsc{prp} \textsc{inf} be quite nice girl \\
         \glt `She seemed to be a nice girl.' \hfill(coon\_valley\_WI\_06gm)

\end{xlist}
\end{exe}




\subsection{Control} \label{control}

Infinitive constructions with control verbs are very frequent, following mostly the patterns from European Norwegian (exceptions are mentioned in §\ref{section-shrink}, cf. \cite{putsoft}). In (\ref{controlamno}) we see four regular examples with \textit{like} `like' and \textit{prøve} `try', varying between \textit{\aa} and \textit{te} as the infinitival marker\footnote{See \citet{putsoft} for details on variation in infinitival markers in constructions like these.}: 

\begin{exe} 

\item \label{controlamno}

\begin{xlist}


\item \label{control1} \gll de liker \textbf{{\aa}} komme heim for jul \\
        they like \textsc{inf} come home for Christmas \\
         \glt `They like \textbf{to} come home for Christmas.' \hfill(sunburg\_MN\_05gk)

\item \label{control2} \gll jeg liker \textbf{te} gå der og se hvor foreldrene s- de kom ifra \\
        I like \textsc{inf} go there and see where parents.DEF s- they came from \\
         \glt `I like \textbf{to} go there and see where the(ir) parents came from.' \hfill(sunburg\_MN\_08gk)
         
\item \label{control3} \gll de prøvde \textbf{{\aa}} ta norsken ifra oss \\
        they tried \textsc{inf} take Norwegian.\textsc{def} from us \\
         \glt `They tried \textbf{to} take the Norwegian language away from us.' \hfill(albert\_lea\_MN\_01gk)
         
\item \label{control4} \gll omtrent tre fire plasser jeg prøvde \textbf{te} komme oppover her \\
        about three four places I tried \textsc{inf} come up here \\
         \glt `(There were) about three or four places here where I tried \textbf{to} get up.' \hfill(coon\_valley\_WI\_17gm)
         

\end{xlist}
\end{exe}

The examples from CANS in (\ref{OBJcontrol}) are object control predicates:

\begin{exe} 

\item \label{OBJcontrol}

\begin{xlist}


\item \label{OBJcontrol1} \gll dem ville hjelpe meg \textbf{te} lære engelsk \\
        they would help me \textsc{inf} learn English \\
         \glt `They wanted to help me \textbf{to} learn English.' \hfill(sunburg\_MN\_18gk)

\item \label{OBJcontrol2} \gll og så han lærte deg \textbf{{\aa}} kjøre bil \\
        and then he learnt you \textsc{inf} drive car \\
         \glt `And so he taught you how \textbf{to} drive a car.' \hfill(saskatoon\_SK\_01gk)

\item \label{OBJcontrol4} \gll jeg trur de sendte oss te \textbf{{\aa}} ha moro \\
        I believe they sent us \textsc{prp} \textsc{inf} have fun \\
         \glt `I believe they sent us \textbf{to} have fun.' \hfill(coon\_valley\_WI\_03gm)
         
                  
\end{xlist}
\end{exe}


Instances of general control in which an arbitrary \textsc{pro} subject is licensed are also present in the CANS data, as shown by the examples in (\ref{arbitrarycontrol}):\footnote{Note that (\ref{arbit3}) could be an example of clausal stretching (added preposition).} 

\begin{exe} 

\item \label{arbitrarycontrol}

\begin{xlist}

         
\item \label{arbit2} \gll men det tar så mye lenger \textbf{{\aa}} gjøre alt \\
        but it takes so much longer \textsc{inf} do all \\
         \glt `But it takes so much longer \textbf{to} do it all.' \hfill(coon\_valley\_WI\_04gm)

\item \label{arbit3} \gll det tar nesten tre timer \textbf{te} \textbf{{\aa}} skrive bare ett brev \\
        it takes almost three hours \textsc{prp} \textsc{inf} write only one letter \\
         \glt `It takes almost tree hours \textbf{to} write only one letter.' \hfill(fargo\_ND\_01gm)

\item \label{arbit4} \gll det er ikke så verst \textbf{{\aa}} være bestefar \\
        it is not so bad \textsc{inf} be grandpa \\
         \glt `It's not so bad \textbf{to} be a grandpa.' \hfill(coon\_valley\_WI\_06gm)

                  
\end{xlist}
\end{exe}

%NB: In (b), Bokmål would have only å, not te å. PS: I'm not sure if all these are control constructions?


It is interesting to note that the most frequent non-finite phrases in NAmNo are \textit{tough}-constructions, as represented in (\ref{mostfrequentinf}). These constructions are generally held to have a \textsc{pro} subject that is bound by an operator (Op) in Spec,CP (for an example of a contemporary analysis of \textit{tough}-constructions according to Minimalist parlance, see \citealt{hicks09}). Given the high frequency for which these occur in European Norwegian, it comes as little surprise that \textit{tough}-constructions are commonplace in NAmNo. 

\begin{exe} 

\item \label{mostfrequentinf}

\begin{xlist}


\item \label{infadj1} \gll det var hardt \textbf{{\aa}} sove når det var så lyst \\
        it was hard \textsc{inf} sleep when it was so bright\\
         \glt `It was hard \textbf{to} sleep when it was so bright.' \hfill(sunburg\_MN\_05gk)
         
\item \label{infadj2} \gll det er så moro \textbf{{\aa}} ha besøk fra Norge \\
        it is so fun \textsc{inf} have visit from Norway\\
         \glt `It is so much fun \textbf{to} have visitors from Norway.' \hfill(blair\_WI\_02gm)

\item \label{infnoun1} \gll det er ingenting \textbf{{\aa}} gjøre i Hatton mer \\
        there is nothing \textsc{inf} do in Hatton more \\
         \glt `There isn't anything \textbf{to} do in Hatton anymore.' \hfill(fargo\_ND\_03gm)
         
         \item \label{infnoun2} \gll det er ikke mye \textbf{{\aa}} høre om Casablanca \\
        it is not much \textsc{inf} hear about Casablanca \\
         \glt `There isn't much \textbf{to} hear about Casablanca.' \hfill(coon\_valley\_WI\_02gm)
                  
\end{xlist}
\end{exe}

\subsection{ECM} \label{ecmsection}

When it comes to ECM-predicates, European Norwegian only licenses ECM1-type constructions. Our efforts to mine CANS for similar  constructions confirm that NAmNo overwhelmingly follows this pattern as well. In (\ref{ECMamno}) we see examples with the matrix verbs \textit{se} `see', \textit{høre} `hear', and \textit{late} `let', which are canonically associated with ECM1-predicates:

\begin{exe} 

\item \label{ECMamno}

\begin{xlist}


\item \label{ECMamno1} \gll men jeg har aldri sett dem spise blåbær \\
        but I have never seen them eat blueberries \\
         \glt `But I have never seen them eat blueberries.' \hfill(stillwater\_MN\_01gm)

\item \label{ECMamno2} \gll så jeg har ikke hørt han synge \\
        so I have not heard him sing \\
         \glt `So I haven't heard him sing.' \hfill(sunburg\_MN\_05gk)

\item \label{ECMamno3} \gll han lot oss bruke bilen deres for fem dager \\
        he let us use car.\textsc{def} theirs for five days \\
         \glt `He let us use their car for five days.' \hfill(sunburg\_MN\_03gm)

         
\end{xlist}
\end{exe}

Our search efforts through CANS have thus far not been successful in finding ECM2-predicates in NAmNo, which confirms the predictions we introduced at the beginning of this section, namely, that these structures should be absent or exceedingly rare in NAmNo. This is due to the fact that ECM2-predicates project a TP-layer and not a CP-layer. 


\subsection{Gerunds} \label{gerundives}

Finally, we turn to gerunds, which, to date, have not received detailed treatment in NAmNo. Gerunds can be divided into two sub-classes: (i) \textit{clausal gerunds}, and (ii) \textit{defective clausal gerunds}. Consider the contrast between the control predicate licensing \textsc{pro} in (\ref{ECG11}) and the clausal gerund in (\ref{ECG12}).

\begin{exe}

\item \label{ECG1}

\begin{xlist}

\item Carol worried about [\textsc{pro} being late for dinner.] \label{ECG11}

\item Carol worried about [Jim being late for dinner.] \label{ECG12}

\end{xlist}

\end{exe}

\citet[16]{pires2007} suggests that clausal gerunds can be analyzed as TPs. As shown in (\ref{ECG2}), clausal gerunds (in English) can appear as (i) complements to verbs (\ref{ECG21}), as (ii) complements to prepositions (\ref{ECG22}--\ref{ECG23}), and as (iii) phrases in “subject position” (\ref{ECG24}). Clausal gerunds appear in “Case positions”, i.e., positions in which arguments receive structural Case.

\begin{exe}

\item \label{ECG2}

\begin{xlist}

\item Mary favored [Bill taking care of her land]. \label{ECG21}
\item Susan worried about [Mark being late for dinner]. \label{ECG22}
\item Sylvia wants to find a new house without [Anna helping her]. \label{ECG23}
\item\relax [Sue showing up at the game] was a surprise to everybody. \label{ECG24}

\end{xlist}

\end{exe}

Associating clausal gerunds with the notion of Case positions also explains why these can occur as complements to prepositions (\ref{casep2}), unlike finite and non-finite clauses (\ref{casep1}). Clausal gerunds in English thus behave similarly to possessive \textit{-ing} and DPs, as shown in (\ref{casep3}) (data from \citealt[21]{pires2007}).

\begin{exe}

\item \label{caseposition}

\begin{xlist}

\item *Mary talked about [(that) John moved out/John to move out]. \label{casep1}

\item Mary talked about [John moving out]. \label{casep2}

\item Mary talked about [John's moving out/John's move]. \label{casep3}

\end{xlist}

\end{exe}

In contrast, \textit{defective clausal gerunds} appear to have a reduced syntactic structure. They can appear as (gerund) complements of aspectualizers (e.g., \textit{start, finish,} \textit{keep}) and verbs such as \textit{try} and \textit{avoid} %form a distinct class 
\citep[70]{pires2007}:

\begin{exe}

\item \label{defectCGEnglish1}

\begin{xlist}

\item Mary\textsubscript{j} started/finished/continued [\textit{e}\textsubscript{j} reading the newspaper]. 
\item Bill\textsubscript{j} tried [\textit{e}\textsubscript{j} talking to his boss].
\item Philip\textsubscript{j} avoids [\textit{e}\textsubscript{j} driving on the freeway].

\end{xlist}


\end{exe}

Due to (i) their lack of independence regarding tense and aspect in relation to the matrix clause and (ii) the questionable status of \textsc{pro}, these  \textit{defective clausal gerunds} are generally considered to be relatively small, i.e., VPs.

European Norwegian does not license either \textit{clausal} or \textit{defective clausal gerunds} on par with English. Consider the following examples in (\ref{gerundsEngNo}); while English allows both a gerund (\ref{EngGerund}) and an infinitive (\ref{EngINF}) in this context, European Norwegian only allows an infinitive (\ref{NoINF}):

\begin{exe} 

\item \label{gerundsEngNo}

\begin{xlist}


\item \label{EngGerund} He started reading. 

\item \label{EngINF} He started to read. 
         
\item \label{NoINF} \gll Han begynte å lese. \\
                         he began \textsc{inf} read \\
                         \glt `He began to read.' 


\end{xlist}
\end{exe}

Based on a cursory search through CANS data, NAmNo follows the pattern of European Norwegian to a large extent. This is is illustrated in the data provided in (\ref{gerundivesamnotarget}), which show examples with matrix verbs \textit{like} `like', \textit{prøve} `try', \textit{begynne} `begin, start' and \textit{slutte} `stop, quit':


\begin{exe} 

\item \label{gerundivesamnotarget}

\begin{xlist}


\item \label{gerundivestarget1} \gll hun likte \textbf{{\aa}} snakke norsk \\
        she liked \textsc{inf} speak Norwegian \\
         \glt `She liked speaking Norwegian.' \hfill(stillwater\_MN\_01gm)

\item \label{gerundivestarget2} \gll jeg prøvde \textbf{{\aa}} spare så mye mjølk som jeg kunne  \\
        I tried \textsc{inf} save as much milk that I could  \\
         \glt `I tried saving as much milk as I could.' \hfill(westby\_WI\_01gm)
         
\item \label{gerundivestarget3} \gll de må begynne \textbf{{\aa}} skrive noen brev \\
        they must begin \textsc{inf} write some letters \\
         \glt `They have to start writing some letters.' \hfill(coon\_valley\_WI\_10gm)

\item \label{gerundivestarget4} \gll etter jeg slutta \textbf{{\aa}} arbeide så kom jeg her en eller to eller tre dager i uka \\
        after I stopped \textsc{inf} work so came I here one or two or three days a week  \\
         \glt `After I stopped working, I came here one or two or three days a week.' \hfill(decorah\_IA\_01gm)

\end{xlist}
\end{exe}

Once again, despite the prevalence of the European Norwegian pattern with respect to gerund-like equivalents in NAmNo, there exist a suitable number of examples in CANS that lack the required infinitival marker.\footnote{It is worth mentioning that the infinitival marker is optional in European Norwegian in a limited number of environments, especially when there is also negation (\textit{ikke}, `not'). NRG \citep[995]{faaretal97} gives the following examples: \textit{Du behøver ikke (å) komme} (You don't need to come), \textit{Vi trenger ikke (å) gå så tidlig} (We don't have to leave so early), \textit{Jeg orker ikke (å) høre på deg} (I can't bear to listen to you), \textit{Han gadd ikke (å) prøve en gang} (He didn't even bother to try), \textit{Han freista (å) få igjen pusten} (He tried to catch  his breath).} The examples in (\ref{gerundivesamno}) reflect an English-like pattern of gerund-like VPs that lack an infinitival marker: 


\begin{exe} 
\item \label{gerundivesamno}

\begin{xlist}

\item \label{gerundives1} \gll hun likte \_ ha moro \\
        she liked \_ have fun \\
         \glt `She liked having fun.' \hfill(sunburg\_MN\_04gk)

\item \label{gerundives2} \gll jeg prøvde \_ snakke litt norsk med han \\
        I tried \_ speak some Norwegian with him \\
         \glt `I tried speaking Norwegian with him.' \hfill(coon\_valley\_WI\_12gm)

\item \label{gerundives3} \gll nå må vi begynne \_ snakke norsk \\
        now must we start \_ speak Norwegian \\
         \glt `Now we must start speaking Norwegian.' \hfill(outlook\_SK\_09gm)
         
\item \label{gerundives4} \gll hvilket år var det du slutta \_ mjølke? \\
        what year was it you stopped \_ milk \\
         \glt `What year did you stop milking?' \hfill(hatton\_ND\_02gm)

\end{xlist}
\end{exe}

A common semantic feature in these examples in  (\ref{gerundivesamno}) is that they resemble English gerunds in involving some degree of aspectual complementation of the VP, e.g., \textit{stop smoking, continue reading, start raining,} etc. \citep{freed79}. Although this development warrants additional research in the future, even at this nascent state it is worth mentioning that while we see the emergence of \textit{defective clausal gerunds} in NAmNo, we have not yet found any examples of \textit{clausal gerunds} in CANS. A similar development has also recently been noticed and discussed in Pennsylvania Dutch, another Germanic heritage language (see \citealt{putPDtoappear} for details). 

\subsection{Section summary} \label{subsect-summary3}

Our succinct review of non-finite complementation in NAmNo, based primarily on CANS data, supports our predictions introduced at the beginning of this section. ECM2-predicates are not found in the extant CANS data; there is only evidence of ECM1-predicates. There is also only scant evidence of subject-to-subject raising in NAmNo, aside from a small number of fixed phrases. Subject control is highly frequent and object control is also present, once again confirming our initial predictions. Our incipient treatment of gerunds in NAmNo appears to echo recent findings in Pennsylvania Dutch \citep{putPDtoappear}, showing that while \textit{defective clausal gerunds} are possible in NAmNo, \textit{clausal gerunds}, as defined by \citet{pires2007}, are not attested in the extant CANS data. In the next section we provide some theoretical guidelines into how formal analyses of non-finite complementation can shed light on the trajectory of these developments. 


\section{Theoretical relevance: Representational Economy} \label{section-theory}

Research to date on non-finite complementation in NAmNo and other heritage languages largely confirms previous proposals concerning the relative sturdiness of these syntactic systems \citep{polinsky18,lohndal2021}. With this being said, examples that differ from (most or all varieties of) European Norwegian display interesting and somewhat predictable patterns. To better illustrate the theoretical relevance of these forms, consider Table \ref{Englishdefectstable}, which summarizes the size of the clausal projections associated with each type of non-finite clause.

%\newpage
\begin{table}
    \begin{tabular}{ll}
    \lsptoprule
      Non-finite clause   & Projection \\\midrule
      ECM1                & \textit{v}P \\
      Clausal Gerund (CG) & TP \\
      Defective CG        & VP (or “defective” TP) \\
      ECM2                & TP \\
      Raising             & TP \\
      Control             & CP \\
    \lspbottomrule
    \end{tabular}
    \caption{Non-finite clauses and their projections}
    \label{Englishdefectstable}
\end{table}

In our review of the general properties and proposed underlying structures of non-finite clauses in Sections~\ref{section-shrink} and~\ref{section-nonfinite}, one factor that stands out is that NAmNo syntax collectively avoids projecting non-finite complement clauses that project a TP. We illustrate this comparison of English, NAmNo, and European Norwegian in Table \ref{summary1}, which clearly shows that NAmNo has overwhelmingly retained a Norwegian-like syntax when it comes to non-finiteness.


\begin{table}
    \begin{tabular}{l ccc}
    \lsptoprule
                         &         &       & European\\
      {}                 & English & NAmNo & Norwegian \\\midrule
     ECM1                & \ding{51} & \ding{51} & \ding{51} \\
     Clausal Gerund (CG) & \ding{51} & \ding{55} & \ding{55} \\
     Defective CG        & \ding{51} & \ding{51} & \ding{55} \\
     ECM2                & \ding{51} & \ding{55} & \ding{55} \\
     Raising             & \ding{51} & \ding{55} & \ding{55} \\
     Control             & \ding{51} & \ding{51} & \ding{51} \\
     \lspbottomrule
    \end{tabular}
    \caption{English--NAmNo--European Norwegian non-finite clauses}
    \label{summary1}
\end{table}

We now turn to the theoretical relevance of these findings. Firstly, decades of theoretical, experimental, and hybrid research on bi- and multi-component populations supports the integrated nature of these cognitive and grammar systems \citep{putnam2018,aboh2015}. Second, if we embrace an integrated view of cognition and grammar of bi- and multi-competent populations, distinctions that are unique to the dyad of both\slash all grammar systems should receive elevated attention \citep{sconput20}. In this particular dyad, i.e., American English--NAmNo, the Norwegian-based representations display a strong tendency to avoid projecting “naked” TPs as non-finite complements, as argued most recently by \citet{putsoft,putsoft23}. The differences that cannot be directly attributed to American English can be classified as instances of \textsc{hypercorrection} \citep{kupisch2014adjective,putnamhoff21}, e.g., instances where heritage speakers of NAmNo are generating Norwegian-source representations and want to avoid projecting English\hyp like “naked” TP-finite complement clauses. 

An important takeaway from this discussion of non-finite complementation in NAmNo is that underlying representations and the integrated nature of bi-/multi-lingual grammars matter a great deal. In fact, avoiding significant conflict between diametrically opposed elements of generated representations from both source grammars would seem to be a tacit goal of such a system in order to facilitate language production and comprehension. In a related treatment of non-finite clauses in Pennsylvania Dutch, \citet{putPDtoappear} advances a revised version of the notion of Representational Economy originally suggested by \citet{scontras2018} and \citet{polinsky2020}, which we provide in (\ref{fixedRE}): 

\begin{exe}

\item \label{fixedRE} Representational Economy (revised):\\
 In HL-syntax, reduce syntactic structure (i.e., computational domains) in order to achieve maximal processing efficiency, \textit{or avoid generating projections that result in representational conflict between the two source grammars in the dyad.}

\end{exe}

Although the reduction, or \textit{shrinking}, of domains is the most readily applied strategy in HL-syntax, the revised definition of Representational Economy provided in (\ref{fixedRE}) opens the possibility of clausal-\textit{stretching} in order to maximize efforts to avoid a conflict in dyadic representations (see e.g. \citealt{lohnput23} for a treatment of the possibility of instances of clause-\textit{stretching} in agglutinating HLs). 

\section{Conclusion} \label{outro}

Our primary goal in this chapter was twofold: First, we provided an overview of non-finite complementation in NAmNo situated in current Minimalist theorizing of these structures. Second, we demonstrated that although NAmNo has retained a largely European Norwegian-like syntactic system with respect to non-finiteness, differences from this model provide an important and unique perspective into the grammar of heritage bilinguals. Nuanced non-finite forms in the CANS corpus overwhelmingly avoid projecting “naked” TP-complements, which are perfectly acceptable in English, but not in European Norwegian. This state of affairs led to a revised version of the Representational Economy metric introduced by \citet{putPDtoappear} in recent work. In summary, mental representations must continue to play a dominant role in ongoing research on developments in HL-syntax research -- in NAmNo and beyond. 

Two additional points are in order before ending this chapter. First, our nascent research on gerunds in NAmNo could be a productive and fertile empirical domain that warrants further research. Second, continued research with the remaining speakers of NAmNo should dedicate a watchful eye to these non-finite structures, because if they begin to appear more frequently in NAmNo, this could signal radical shifts in other elements of grammar, most notably, its V2-behavior. This, of course, is based on the assumption that these mental representations are not simply individual constructions (as assumed by some), but are actually a series of interconnected underlying structures that are responsible for generating any and all forms found in a grammar. 

%\nocite{putsan2013,putsoft,lohndal2021,harbert07,faarlund19,cans,aboh2015,bous2020,silvia2011,faarlund2007,faarlund2015,beukema1989,wurmbrand2001,Wurmbrand2014Tense,salzman2019,giusti86,giusti91,wilder1988,Christensen2005,aafarli2003,wurmlohn2020,Softelandetal2021,ojea2008,aagaard2016participles,huus2018distribusjonen,faarlund2003reanalyse,lodrup2008,johnson1994,putsoft23,morckINFmark}

%\is{Cognition} %add "Cogntion" to subject index for this page

%\ea
%\gll cogito                           ergo      sum\\
%     think.\textsc{1sg}.\textsc{pres} therefore \textsc{cop}.\textsc{1sg}.\textsc{pres}\\
%\glt `I think therefore I am.'
%\z
%\il{Latin} %add "Latin" to language index for this page


\section*{Abbreviations}

\begin{tabbing}
MMMM \= Corpus\kill
%BM \> Norwegian Bokmål \\
CANS \> Corpus of American Nordic Speech \\
\textsc{def} \> Definite \\
ECM \> Exceptional Case-Marked \\
\textsc{inf} \> Infinitival marker \\
LIA \> Corpus of older Norwegian dialect recordings \\
%L2 \> Second language \\
NAmNo \> North American Norwegian \\
NDC \> Nordic Dialect Corpus \\
NRG \> Norwegian Reference Grammar \\
PD \> Pennsylvania Dutch\\
%\textsc{prog} \> Progressive (aspect) \\
\textsc{prp} \> Preposition/particle \\
\end{tabbing}

\section*{Acknowledgements}
We are indebted to two anonymous reviewers for their critical questions, comments, and observations that forced us to improve and clarify key facets of our treatment of non-finite complementation in NAmNo. Finally, we would like to especially thank Kari Kinn and Terje Lohndal for helpful comments on an earlier draft of this chapter. We are also thankful for the support from the Research Council of Norway, grant 301114. The usual disclaimers apply. 


\printbibliography[heading=subbibliography,notkeyword=this]

\end{document}
