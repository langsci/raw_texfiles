\documentclass[output=paper,colorlinks,citecolor=brown]{langscibook}
\ChapterDOI{10.5281/zenodo.15274566}
\author{Kari Kinn\orcid{}\affiliation{University of Bergen}}
\title{Possession in determiner phrases in North American Norwegian}
\abstract{This chapter discusses DP-internal possessive constructions, of which there are several in European  (homeland) Norwegian (EurNo). It is shown that North American Norwegain (NAmNo) generally has retained the different options for expressing DP-internal possession, and that analyses previously proposed for EurNo can be extended to NAmNo. There are some differences with respect to the \emph{distribution} of the options. Notably, however, most speakers display no particular preference for patterns that converge with the majority language English.}


\IfFileExists{../localcommands.tex}{
   \addbibresource{../localbibliography.bib}
   \usepackage{langsci-optional}
\usepackage{langsci-gb4e}
\usepackage{langsci-lgr}

\usepackage{listings}
\lstset{basicstyle=\ttfamily,tabsize=2,breaklines=true}

%added by author
% \usepackage{tipa}
\usepackage{multirow}
\graphicspath{{figures/}}
\usepackage{langsci-branding}

   
\newcommand{\sent}{\enumsentence}
\newcommand{\sents}{\eenumsentence}
\let\citeasnoun\citet

\renewcommand{\lsCoverTitleFont}[1]{\sffamily\addfontfeatures{Scale=MatchUppercase}\fontsize{44pt}{16mm}\selectfont #1}
  
   %% hyphenation points for line breaks
%% Normally, automatic hyphenation in LaTeX is very good
%% If a word is mis-hyphenated, add it to this file
%%
%% add information to TeX file before \begin{document} with:
%% %% hyphenation points for line breaks
%% Normally, automatic hyphenation in LaTeX is very good
%% If a word is mis-hyphenated, add it to this file
%%
%% add information to TeX file before \begin{document} with:
%% %% hyphenation points for line breaks
%% Normally, automatic hyphenation in LaTeX is very good
%% If a word is mis-hyphenated, add it to this file
%%
%% add information to TeX file before \begin{document} with:
%% \include{localhyphenation}
\hyphenation{
affri-ca-te
affri-ca-tes
an-no-tated
com-ple-ments
com-po-si-tio-na-li-ty
non-com-po-si-tio-na-li-ty
Gon-zá-lez
out-side
Ri-chárd
se-man-tics
STREU-SLE
Tie-de-mann
}
\hyphenation{
affri-ca-te
affri-ca-tes
an-no-tated
com-ple-ments
com-po-si-tio-na-li-ty
non-com-po-si-tio-na-li-ty
Gon-zá-lez
out-side
Ri-chárd
se-man-tics
STREU-SLE
Tie-de-mann
}
\hyphenation{
affri-ca-te
affri-ca-tes
an-no-tated
com-ple-ments
com-po-si-tio-na-li-ty
non-com-po-si-tio-na-li-ty
Gon-zá-lez
out-side
Ri-chárd
se-man-tics
STREU-SLE
Tie-de-mann
}
   \boolfalse{bookcompile}
   \togglepaper[23]%%chapternumber
}{}

\begin{document}
\tikzset{every tree node/.style={align=center,anchor=north}}
\maketitle

\section{Introduction}\label{kinn-intro}
\begin{sloppypar}
This chapter gives an overview of DP-internal possessive constructions in North American Norwegian (NAmNo). Both in the homeland variety European Norwegian (EurNo) and in NAmNo, DP-internal possessive constructions cover all of the central semantic relationships that are typical for DP-internal possession cross\hyp linguistically: ownership, part-whole relationships and kinship \citep[263--264]{dixon2010basic}. Other relationships between two entities can also be expressed by possessives; for example, with deverbal nouns, a possessor can have a role parallel to that of the subject of the corresponding verb  (e.g., \citealt{chomsky1970remarks}). In what follows,  possessive constructions expressing many different semantic relations will be included; however, the discussion will mainly focus on the syntax rather than the meaning.
\end{sloppypar}


EurNo has a relatively large set of possessive constructions. Examples of the main types are given in (\ref{ex:initialpossessiveconstructions}):

\begin{multicols}{2}
\ea \label{ex:initialpossessiveconstructions}
\ea \label{ex:minsykkel} \gll \textbf{min} sykkel \\
my bike\\
\glt `my bike'\\
\ex \label{ex:sykkelenmin} \gll sykkel-en \textbf{min} \\
bike-\textsc{def} my \\
\glt `my bike'\\
\ex \label{ex:olassykkel} \gll \textbf{Ola-s} sykkel \\
Ola-\textsc{poss} bike \\
\glt `Ola's bike' \\
\ex \label{ex:sykkelentilola} \gll sykkel-en \textbf{til} \textbf{Ola} \\
bike-\textsc{def} to Ola \\
\glt `Ola's bike' \\
\ex \label{ex:olasinsykkel} \gll \textbf{Ola} \textbf{sin} sykkel\\
Ola \textsc{poss} bike \\
\glt `Ola's bike' \\
\z
\z
\end{multicols}

\noindent In (\ref{ex:minsykkel}), the possessor is expressed by a prenominal possessive determiner. In (\ref{ex:sykkelenmin}), there is also a possessive determiner; however, its position is postnominal, and the noun expressing the possessee is equipped with a definite suffix. In (\ref{ex:olassykkel}), the possessor role is expressed by a noun to which a possessive clitic \emph{-s} is attached. In (\ref{ex:sykkelentilola}), the possessor is expressed by a PP introduced by the preposition \emph{til} `to'. Example  (\ref{ex:olasinsykkel}) illustrates the so-called possessor doubling construction, in which the possessive marker \emph{sin} ``doubles'' a noun with a possessor role. On Julien's \citeyearpar{julien2005nominals} analysis, which is adopted here, possessors are generated in the specifier of NP; postnominal possessors are derived by movement of the noun past this position. Prenominal possessors move to a higher position below D, but above adjectives and ordinals (see further in \sectref{sec:prenom_postnom}). 


The chapter builds on previous studies on possessive constructions in NAmNo, notably \citet{AnderssenWestergaard2012, WestergaardAnderssen2015, AnderssenEtAl2018} and \citet{Kinn2021}. These works do, however, not cover the whole range of possessive constructions shown in (\ref{ex:initialpossessiveconstructions}), and the chapter will offer new observations from the Corpus of American Nordic Speech (CANS, \citealt{Johannessen2015CANS}) to shed some (initial) light on understudied aspects of possession in DPs in NAmNo.  It will be shown that most of the numerous ways of expressing possession in EurNo are attested in NAmNo; the overall picture that emerges is one of syntactic stability.  However, as will be evident, there are some differences with respect to the distribution of the different options.

The chapter is structured as follows: \sectref{sec:prenom_postnom} discusses possessive determiners. \sectref{sec:s-gen} discusses the possessive clitic -\emph{s}, while \sectref{sec:PP-poss} discusses constructions with PP possessors. \sectref{sec:poss_doubling} deals with the possessor doubling construction, and \sectref{sec:special} discusses some special patterns associated with relational nouns. \sectref{sec:theoreticalrelevance} briefly considers the theoretical relevance of some of the observations in the chapter, while \sectref{sec:conclusion} contains concluding remarks. 


\section{Possessive determiners}\label{sec:prenom_postnom}
\largerpage[2]

This section deals with constructions involving possessive determiners, which are very common both in EurNo and NAmNo. Possessive determiners can be either prenominal or postnominal; the syntactic structure and distribution of these options are discussed in \sectref{subsec:prenompostnomdet}. \sectref{subsec:reflexive} discusses reflexive possessive determiners, and  \sectref{subsec:poss+prop} is about possessive determiners with proper names. 

\subsection{Prenominal and postnominal possessive determiners}\label{subsec:prenompostnomdet}


\subsubsection{Overview and syntactic structure}
An overview of  Norwegian possessive determiners is given in Table \ref{table:possdet} (based on \citealt[31]{faarlund2019syntax}).\footnote{I follow \citet{Faarlund2019} in using the term possessive \emph{determiner} rather than possessive \emph{pronoun}, which is used by, e.g., \citet{Julien2005}.} The determiner forms  are  written Bokmål standard forms, which are the ones used in the orthographic transcriptions in CANS. In the spoken homeland dialects, as well as in NAmNo, a variety of dialect forms are used.

\begin{table}
\begin{tabular}{ *5{l} }  
    \lsptoprule
    \multicolumn{5}{l}{\textit{Singular}}\\
    1st & 2nd & 3rd m & 3rd f & 3rd refl\\\midrule
    \emph{min} (\textsc{m,f}) & \emph{din} (\textsc{m,f})  & \emph{hans} & \emph{hennes} & \emph{sin} (\textsc{m,f})\\
    \emph{mitt} (\textsc{n}) & \emph{ditt} (\textsc{n})& & & \emph{sitt} (\textsc{n})\\    
    \emph{mine} (\textsc{pl}) &  \emph{dine} (\textsc{pl})& & & \emph{sine} (\textsc{pl})\\\midrule    
    \multicolumn{5}{l}{\textit{Plural}}\\
    1st & 2nd & \multicolumn{2}{c}{3rd} & 3rd refl \\\midrule
    \emph{vår}  (\textsc{m,f}) & \emph{deres} & \multicolumn{2}{c}{\emph{deres}} & \emph{sin} (\textsc{m,f})\\
    \emph{vårt} (\textsc{n})  & & & & \emph{sitt} (\textsc{n}) \\
    \emph{våre} (\textsc{pl}) & & & & \emph{sine} (\textsc{pl}) \\
    \lspbottomrule
\end{tabular}
\caption{Possessive determiners in  EurNo (Bokmål standard).}
\label{table:possdet}
\end{table}

As is evident from Table \ref{table:possdet}, the set of possessive determiners distinguishes between singular and plural, and three grammatical persons, depending on the properties of the possessor.\footnote{Many spoken dialects do not have the syncretism between the 2nd and 3rd pl. which appears in Table \ref{table:possdet}. Some common forms in the 3rd pl. in homeland dialects and in CANS are, e.g., \emph{demmes}, \emph{demmses} and \emph{domms}.} There are separate reflexive possessive determiners in the 3rd person; these are discussed in further detail in \sectref{subsec:reflexive}. Most of the possessive determiners  show agreement in gender and number with the possessee, 
except \emph{hans} (3rd sg. m.), \emph{hennes} (3rd sg. f.) and \emph{deres} (2nd pl./3rd pl.),   which are uninflected \citep[31]{faarlund2019syntax}. On agreement in NAmNo, see further in \textcitetv{chapters/riksem}. 


\begin{sloppypar}
Possessive determiners can be either prenominal or postnominal. When the determiner is prenominal, the noun must be indefinite, both in EurNo and NAmNo.\footnote{A query in CANS for possessive determiners followed by a noun yielded virtually no exceptions to this in NAmNo.} NAmNo examples (all taken from CANS, v. 3.1) of prenominal possessive determiners are given in (\ref{ex:prenom}):
\end{sloppypar}

\ea \label{ex:prenom}
\ea \label{ex:minfar} \gll \textbf{min} far \\
    my.\textsc{m} father\\
   \glt  `my father' (spring\_grove\_MN\_09gm)\\
\ex \label{ex:hennesjobb} \gll \textbf{hennes} jobb \\
her job \\
\glt `her job' (seattle\_WA\_03gm)\\   
\ex \label{ex:vårepenger} \gll \textbf{vår-e} penger \\
our-\textsc{pl} money.\textsc{pl} \\
\glt `our money' (blair\_WI\_02gm) \\
\z 
\z 

When the possessive determiner is postnominal, the possessed noun must appear in the \emph{definite} form, i.e., with a definite suffix.  This is the general rule in both EurNo and NAmNo (certain kinship nouns are an exception, see \sectref{sec:special}). NAmNo examples of postnominal possessive determiners are given in (\ref{ex:postnom}):
\ea \label{ex:postnom}
\ea \label{ex:farmenmin} \gll farm-en \textbf{min} \\
farm-\textsc{def.m} my\textsc{.m}\\
\glt `my farm' (blair\_WI\_01gm)\\
\ex \label{landetmitt} \gll land-et \textbf{mitt} \\
land-\textsc{def.n} my.\textsc{n} \\
\glt `my land' (coon\_valley\_WI\_02gm) \\
\ex \gll fel-a \textbf{mi} \\
fiddle-\textsc{def.f} my.\textsc{f} \\
\glt `my fiddle' (coon\_valley\_WI\_06gm)\\
\ex hund-ene \textbf{deres}\\
dogs-\textsc{def.pl} their\\
\glt `their dogs' (westby\_WI\_09gm)\\
\z 
\z 



\noindent In EurNo, prenominal and postnominal possessive determiners have different syntactic and information\hyp structural properties. Lødrup (\citeyear{lødrup2011norwegian}, \citeyear{lødrup2012forholdet}) shows that postnominal possessors behave like weak pronouns (in the sense of \citealt{cardinaletteistarke1999typology}) and are used for topical information, whereas prenominal possessives (at least in the spoken language) behave like strong pronouns and generally involve focal information. For NAmNo, there are, as of yet, no systematic studies of the information-structural properties of prenominal vs. postnominal possessive determiners (see \citealt[327]{anderssenwestergaard2012tospraklighet} for discussion); thus, it is not entirely clear if their distribution follows  the same 
patterns. However, as mentioned, it is clear that both options exist (on their quantitative distribution, see further below). 

In terms of syntactic structure, both prenominal and postnominal possessive determiners in NAmNo are generally compatible with the analyses proposed by \citet{julien2005nominals}. Postnominal possessive determiners can be analyzed as sketched in \figref{tree:postnom} (based on example (\ref{ex:farmenmin}) above and \citealt[163]{julien2005nominals}).\footnote{Labelling the possessive determiner as a DP is a slight simplification (\citealt[162]{Julien2005}). \citet{julien2005nominals} provides a more elaborate analysis of the lower part of the DP structure, and she uses the label \emph{n}P instead of ArtP. The label ArtP is, however, consistent with \citet{julien2002determiners}, and it is used in this chapter to avoid confusion with the nominalizer (n) in DM and related frameworks; see also \textcitetv{chapters/introduction}.}

\vfill
\begin{figure}[H]
\Tree [.DP [.\emph{farmen min} ] [.D\1 [.D ] [.ArtP  [.Art\\\emph{\sout{farmen}} ] [.NP [.DP\\\emph{\sout{min}} ] [.N\1 [.N\\\emph{\sout{farm}} ] ] ] ] ] ]
\caption{Possessive construction with postnominal possessive determiner, following \citet{julien2005nominals}.} 
\label{tree:postnom}
\end{figure} 
\vfill\pagebreak

In this derivation, the possessive determiner is  base\hyp generated in Spec-NP. The noun moves across the possessive, to Art, which hosts the definite article; this yields the postnominal position. In order to make the D-layer visible, the whole ArtP moves to Spec-DP.






The analysis of the prenominal possessive in (\ref{ex:minfar}) \emph{min far} `my father' is sketched in \figref{tree:prenom} (based on \citealt[208]{julien2005nominals}, details omitted).


\begin{figure} \caption{Possessive construction with prenominal possessive determiner, following \citet{julien2005nominals}.} \label{tree:prenom} 
\Tree [.DP [.\emph{min} ] [.D\1 [.D ] [.PossP [.\emph{\sout{min}} ] [.Poss\1 [.Poss ] [.ArtP [.\emph{\sout{min}} ] [.Art\1 [.Art\\\emph{far} ] [.NP [.DP\\\emph{\sout{min}} ] [.N\1 [.N\\\emph
{\sout{far}} ] ] ] ] ] ] ] ] ] 

\end{figure}


In \figref{tree:prenom}, the possessive determiner is also base\hyp generated in Spec-NP, above the noun, which means that it starts out in a prenominal position. However, the noun moves out of NP, past the base-generated position of the possessive determiner and up to Art. For the correct word order to be derived, the possessive determiner must sit in a higher position than this.\footnote{\label{footnote:prenom}Although the noun formally has the indefinite form with prenominal possessive determiners, the nominal phrase  behaves like a definite. If an adjective is included, it carries a definite suffix (see \citetv{chapters/riksem}). Moreover, DPs with prenominal possessive determiners are inappropriate in existential constructions, just like definite nominals.  These are well-known facts of EurNo  which motivate movement from N to Art, and I am not aware of evidence that NAmNo is different.  An NAmNo example including an adjective in the definite form is given in (\ref{ex:sittnorskehjem}):


\ea \label{ex:sittnorskehjem} 
\gll sitt norsk-e hjem\\
his.\textsc{refl} Norwegian-\textsc{def} home\\
\glt `his Norwegian home' (fargo\_ND\_02gm)\\
\z 



} Julien's \citeyearpar{julien2005nominals} proposal is a Poss projection below DP, which is associated with Focus; recall that prenominal possessors (at least in EurNo) are generally associated with focal information. From Spec-PossP, the possessive determiner moves up to Spec-DP to make the D-domain visible. 

As mentioned, the information-structural properties of possessive determiners in NAmNo are still understudied; further inquiries into this area would be required to establish more firmly the independent motivation for the higher Poss position with a Focus feature in NAmNo. However, it is clear that prenominal possessives must be located higher than Spec-NP, where they are base\hyp generated, and outside  ArtP. Empirical evidence for this comes from constructions in which an adjective is involved: Like in EurNo, the prenominal possessive precedes adjectives, as illustrated in the NAmNo example in (\ref{ex:prenom+adj}) (see also footnote \ref{footnote:prenom}):

\ea \label{ex:prenom+adj} \gll min yngste bror\\
my.\textsc{m} youngest.\textsc{def} brother\\
\glt `my youngest brother' (fargo\_ND\_06gm)
\z 

\noindent Adjectival phrases are located in a layer labelled $\alpha$P above ArtP (see \citetv{chapters/introduction}), which means that elements preceding adjectives must be in a high position.
If it turns out that there is no motivation for a dedicated high Poss position in NAmNo, one could possibly assume that the prenominal possessive determiner moves directly to Spec-DP (via Spec-ArtP) (this would bear resemblance to the analysis that \citealt[32--33]{faarlund2019syntax} gives for Swedish and Danish).

\subsubsection{Distribution of prenominal and postnominal possessive determiners}
The distribution of prenominal vs. postnominal possessive determiners in NAmNo is highly interesting from a language-contact perspective, as a potential area of  cross-linguistic influence (CLI) on syntax. The constructions involving postnominal possessive determiners do not have a direct equivalent in English, whereas prenominal possessive determiners converge with the English pattern (cf. examples (\ref{ex:prenom}--\ref{ex:postnom}) with translations). \citet{anderssenwestergaard2012tospraklighet} and \citet{westergaardanderssen2015word} show that postnominal possessives are the most frequent option in EurNo, occuring at a proportion of around 75\% in corpora of spoken language, whereas prenominal possessives constitute around 25\%.  If CLI was affecting these possessive determiners, one would expect overproduction of prenominal possessives. However, overproduction of prenominal possessives is not the predominant pattern in NAmNo. \citet{anderssenwestergaard2012tospraklighet, westergaardanderssen2015word} and \citet{anderssenetal2018cross-linguistic} show that there is considerable inter-speaker variation; notably, however, a large majority of speakers overuse the pattern with \emph{postnominal} possessives compared to the homeland baseline, while only a minority displays a preference for prenominal possessives. \citet{anderssenetal2018cross-linguistic} argue that the overuse of prenominal possessive determiners (CLI) is  associated with low proficiency, whereas the inclination to use postnominal possessives (to an even greater extent than in homeland Norwegian) is described as ``cross-linguistic overcorrection'' (CLO), a notion attributed to \citet{kupisch2014overcorr}, whereby bilinguals overstress what is \emph{different} between their languages. \citet{anderssenetal2018cross-linguistic} tentatively explain CLO in terms of ``overinhibition'' of structures from the majority language, which also affects similar structures in the heritage language. 

The conclusion that a majority of NAmNo speakers overuse postnominal possessive determiners is based on a comparison between present-day NAmNo and present-day EurNo; \citet{anderssenetal2018cross-linguistic} included 50 NAmNo speakers recorded in 2010 or later; their homeland baseline material is based on adults in an acquisition corpus collected in Tromsø, and the NoTa corpus, collected in Oslo in the early 2000s. A different perspective is taken by \citet{eidehjelde2023input}, who also discuss possessives (among other phenomena), and who include older data both from NAmNo and EurNo and study the development over time. Like \citet{anderssenetal2018cross-linguistic}, they find  inter-speaker variation. In their oldest data set (Einar Haugen's recordings from the 1930s and 1940s), they also observe group-level variation between speakers from Coon Valley and Westby and the rest of the speakers: In Haugen's recordings from Coon Valley and Westby (places with substantial Norwegian settlements which are well represented also in the present-day CANS material), the proportion of postnominal possessives is already remarkably high (87\% postnominal vs. 13\% prenominal) in the earliest recordings, and remains similar in recordings from the 1980, 1990s and 2010s. Many of the speakers in Coon Valley and Westby have ancestors from the valley of Gudbrandsdalen, and \citet{eidehjelde2023input} conducted a preliminary investigation of older Norwegian dialect recordings from this area, based on the Language Infrastructure made Accessible (LIA) corpus.\footnote{Corpus URL: \url{https://tekstlab.uio.no/glossa3/lia\_norsk}}  The sample is small; however, Eide \& Hjelde report a proportion of around 90\% postnominal possessives in this homeland dialect. They argue that taken together,  the results from the older NAmNo and older homeland recordings could suggest that the high proportion of postnominal possessives observed in many of the present-day NAmNo speakers are an inherited dialect feature.\footnote{Another point emphasized by \citet{eidehjelde2023input}, and also mentioned by \citet{anderssenwestergaard2012tospraklighet} and \citet{westergaardanderssen2015word}, concerns the minority of speakers who overuse \emph{prenominal} possessive determiners. Prenominal possessive determiners converge with English -- however, they are also used at high proportions in \emph{written} Norwegian (Bokmål), which is strongly influenced by Danish. 
Thus, input from the written language is an alternative source for prenominal possessive determiners. \citet{eidehjelde2023input} argue that especially in the early NAmNo period, the written language is a very relevant factor -- however, input from the written language dramatically decreased over time, and the present-day speakers of NAmNo do not generally read or write Norwegian.}






Regardless of whether the high proportion of postnominal possessive determiners is inherited or a result of CLO, there is consensus that for most NAmNo speakers, CLI from the majority language seems to play a very minor role  in the choice between prenominal and postnominal possessive determiners.  This is an important insight, and it corroborates the notion that syntax is generally stable in heritage languages (\citealt{benmamounetal2013hls, lohndaletal2019hlacquisition}). 






\subsection{Reflexive possessive determiners}\label{subsec:reflexive}
\largerpage



 The Norwegian paradigm of possessive determiners includes separate reflexive forms in the 3rd person sg. and pl. (see Table \ref{table:possdet}). 
The reflexive possessive determiner is \emph{sin}; it agrees in number and gender with the possessee. The reflexive forms are anaphors in the sense that they generally obey Condition A of  binding theory (e.g., \citealt[95]{chomsky1995minimalist}). A reflexive possessive can be bound in the contexts that allow binding of a regular non-possessive reflexive determiner (see e.g. \citealt[chap. 9]{faarlund2019syntax} for an overview); the main rule in EurNo is that they are bound by (i.e., their antecedent is) the subject of the (minimal) clause.\footnote{Exceptions are found; see e.g. Lødrup (\citeyear{lodrup2008objectsbinding}, \citeyear{lodrup2009animacyandlongdistancebinding}) and \citet{julien2020langdistansebinding} for further discussion.} This is illustrated in (\ref{ex:refl-vs.nonrefl}):


\ea \label{ex:refl-vs.nonrefl}
\ea \gll Ola så en mann på stasjonen. Mannen\textsubscript{i} tok kofferten \textbf{sin\textsubscript{i}} og gikk. \\
Ola saw a man at station.\textsc{def} man.\textsc{def} took suitcase.\textsc{def.m} \textsc{poss.refl.m} and left 
\\
\glt `Ola saw a man at the station. The man took his (own) suitcase and left.' (EurNo)\\
\ex \gll Ola\textsubscript{i} så en mann på stasjonen. Mannen tok kofferten \textbf{hans\textsubscript{i}} og gikk.\\
Ola saw a man at station.\textsc{def} man.\textsc{def} took suitcase.\textsc{def} his and went\\
\glt `Ola saw a man at the station. The man took his (Ola's) suitcase and left.' (EurNo) \\
\z 
\z 


\noindent The use of reflexive possessive determiners in NAmNo is still an understudied area, and it is an interesting one, as the majority language English does not have an equivalent distinction between reflexive and non-reflexive possessives. Examples in which reflexive possessive determiners are used in a baseline-like manner are easily found in CANS (see (\ref{ex:reflexive-baseline-like})); impressionistically, this seems to be the most common pattern. 

\ea \label{ex:reflexive-baseline-like}
\ea \gll  han\textsubscript{i} mista sønn-en \textbf{sin\textsubscript{i}} \\
he lost son-\textsc{def.m} \textsc{poss.refl.m}\\
\glt `he lost his son' (fargo\_ND\_01gm) \\
\ex \gll hun\textsubscript{i} hadde klipt hår-et \textbf{sitt\textsubscript{i}} \\
she had cut hair-\textsc{def.n} \textsc{poss.refl.\textsc{n}}\\
\glt `she had had her hair cut'  (sunburg\_MN\_03gm)\\
\ex \gll så han\textsubscript{i} tok ikke care av ung-ene \textbf{sine}\textsubscript{i} \\
so he took not care of kid-\textsc{def.pl} \textsc{poss.refl.pl}\\
\glt `so he didn't take care of his kids' (sunburg\_MN\_15gm)\\
\z 
\z 



\noindent There are, however, also examples of non-reflexive possessives bound by a local subject, as shown in (\ref{ex:non-refl}):

\ea \label{ex:non-refl}
\ea \label{ex:deresdrakt} \gll og de\textsubscript{i} hadde \textbf{deres}\textsubscript{i} drakt på \\
 and they had their costume on \\
\glt `and they had their costumes on' (spokane\_WA\_02gk) \\
\ex \gll så han\textsubscript{i} hadde all famili-en \textbf{hans\textsubscript{i}} over der \\
so he had all family-\textsc{def} his over there \\
\glt `so he had all his family over there' (saskatoon\_SK\_11gk)\\
\z 
\z 

\noindent Examples such as (\ref{ex:non-refl}) could be interpreted as CLI from English; it would be an interesting task for future research to see how frequent the use of non-reflexive determiners is, if it occurs systematically in individual speakers, and if it correlates with other signs of CLI from English (see \citealt{anderssenetal2018cross-linguistic} and \citealt{lundquistetal2020}).\footnote{The reviewers point out that there is some variation also among homeland Norwegian speakers with respect to the use of reflexive and non-reflexive possessives. This highlights the importance of establishing a sound baseline for comparison prior to concluding that CLI has taken place, an issue discussed in more depth in other chapters in this volume; see, e.g., \citetv{chapters/eide}, \citetv{chapters/vanbaal} and \citetv{chapters/larsson}.} Note that in (\ref{ex:deresdrakt}), the non-reflexive determiner occurs prenominally, although it has no focal reading; i.e., its position might be seen as another indication of influence from English (cf. \sectref{subsec:prenompostnomdet}).

While some examples of what could be interpreted as CLI are found, it is important to note that the ``opposite'' pattern is also attested; i.e., reflexives used with a non-local antecedent, as in the second coordinate clause in (\ref{ex:overgen-refl}): 


\ea \label{ex:overgen-refl} \gll han\textsubscript{i} brukte å si ``ikke'' \# og så bror \textbf{sin}\textsubscript{i} brukte å si ``inntje'' \\
he used to say ``not'' {} and then brother \textsc{poss.refl.m} used to say ``not'' \\
\glt `he used to say ``ikke'' (`not') and his brother used to say ``inntje''' (`not')' (fargo\_ND\_01gm) \\ 
\z 

\noindent In (\ref{ex:overgen-refl}), \emph{sin} is a part of the subject of the second coordinate clause; thus, it is not bound by it. The antecedent is the subject of the first coordinate clause. In this case, it appears that the use of the reflexive form has been extended. 


\subsection{Possessive determiners with proper names}\label{subsec:poss+prop}
In many spoken dialects of EurNo, the possessor can be expressed by a postnominal possessive determiner directly followed a proper name or a kinship noun. Like with simple, postnominal possessive determiners, the possessee carries a definite suffix (except certain kinship nouns) (see \sectref{subsec:prenompostnomdet}). The construction is well attested in NAmNo; some examples are given in (\ref{ex:poss+prop}) (\emph{M14} and \emph{F1} represent personal names):

\ea \label{ex:poss+prop}
\ea \label{ex:poss+prop-a} \gll innkjøring-a \textbf{hans} \textbf{M14} \\
driveway-\textsc{def} his M14 \\
\glt `M14's driveway' (westby\_WI\_01gm)\\
\ex \label{ex:poss+prop-b} \gll birthday-en \textbf{hennes} \textbf{F1} \\
bithday-\textsc{def} her F1 \\
\glt `F1's birthday' (coon\_valley\_WI\_06gm) \\
\ex \label{ex:poss+prop-c} \gll og bestefar \textbf{hans} \textbf{far} var \# sjømann \\
and grandfather his far was {} sailor\\
\glt `and my father's grandfather was a sailor' (vancouver\_VA\_03uk)\\
\z 
\z 

\noindent In (\ref{ex:poss+prop-a}) and (\ref{ex:poss+prop-b}), the possessive determiners are followed by proper names. In (\ref{ex:poss+prop-c}), it is followed by the kinship noun \emph{far} `father'.


Structurally, the construction with a possessive determiner + a proper name can be analyzed along the same lines as the construction with a simple, postnominal possessive determiner discussed above (see the tree structure in \figref{tree:postnom}). The difference is that the possessive determiner, which is base-generated in Spec-NP, takes a complement in the form of a proper name \citep[172]{julien2005nominals}. The construction has a non-possessive parallel in preproprial articles, which are often found in the same dialects of EurNo \citep[172]{julien2005nominals}. Preproprial articles are pronominal elements that function as determiners with proper names as complements; Johannessen \& Laake (\citeyear[221]{johannessenlaake2012tomyter}, \citeyear[318]{johannessenlaake2015myths}) note that preproprial articles are found in NAmNo too. Their distribution is yet to be investigated in 
detail, but an example from CANS is given in (\ref{ex:preprop}):

\ea \label{ex:preprop} \gll treffer du han M3?\\
see you he M3\\
\glt `Do you see M3?' (coon\_valley\_WI\_03gm)\\
\z 

\noindent We now turn to another possessive construction, namely the possessive \emph{-s}.

\section{The possessive \emph{-s}}\label{sec:s-gen}
The possessive -\emph{s} is sometimes referred to as the ``\emph{-s} genitive'', and it bears strong resemblance to so-called ``Saxon genitive'' in English. Similar to the Saxon genitive, the present-day possessive \emph{-s} in Norwegian is not a morphological case marker (as the label ``genitive'' might be taken to suggest); it has the distribution of a clitic and attaches to phrases. Some EurNo examples are given in (\ref{ex:sgen}); note that the \emph{-s} in (\ref{ex:naboenshund}) appears directly after a preposition that modifies the noun \emph{naboen} `the neighbor'; this clearly shows that the \emph{-s} is not a marker of nominal inflection. 

\ea \label{ex:sgen} 
\ea \label{ex:jentaskatt} \gll \textbf{jenta-s} katt \\
girl.\textsc{def}-\textsc{poss} cat\\
\glt `the girl's cat' (EurNo) \\
\ex \label{ex:naboenshund} \gll \textbf{naboen} \textbf{under-s} hund \\
neighbor.\textsc{def} below-\textsc{poss} dog \\
\glt `the neighbor below's dog' (EurNo) \\
\z 
\z 


\noindent In written, present-day EurNo, the possessive \emph{-s} is more common in the Bokmål standard than in Nynorsk \citep[259]{faarlundetal1997nrg}. In the spoken language, it is typically associated with more formal registers than, e.g., possessive constructions involving a PP (see \sectref{sec:PP-poss}). 

In NAmNo, the possessive \emph{-s} is  attested. Although extensive quantitative investigations must be left for the future, it seems that  it occurs at rates fairly similar to EurNo.\footnote{A query in the Norwegian part of CANS, v. 3.1, restricted to speakers  recorded in 2010 or later, for nouns ending in -s and tagged as genitive, directly followed by another noun, gave 65 relevant hits. (This query yielded many irrelevant hits, which were excluded manually.) The number of word tokens in this part of CANS is 614,613. For comparison, the same query in the Norwegian part of the Nordic Dialect Corpus (NDC, \citealt{Johannessen2009}) to speakers aged 40 or older) gave 126 relevant hits; the size of this part of NDC is 1,102,568 tokens.} Thus, there are no clear indications that the convergence with the ``Saxon genitive'' in English has led to a preference for this possessive construction among the options available to NAmNo speakers.  



Three examples of possessive -\emph{s} in NAmNo are given in (\ref{ex:s-genitive}): 

\ea \label{ex:s-genitive}
\ea \label{ex:minfarsgård} \gll \textbf{min}  \textbf{far-s} gård \\
my father-\textsc{poss} farm \\
\glt `my father's farm' (outlook\_SK\_04gm)\\
\ex \label{ex:morsfolk} \gll \textbf{mor-s} folk \\ 
mother-\textsc{poss} people \\
\glt `my mother's family' (fargo\_ND\_02gm)\\
\ex \label{ex:bestemormins} \gll  \textbf{bestemor} \textbf{min-s} mor \\
grandmother my-\textsc{poss} mother\\
\glt `my grandmother's mother' (blair\_WI\_04gk) \\
\z 
\z 



\noindent The example in (\ref{ex:bestemormins}) stands out somewhat from a present-day EurNo perspective: As mentioned, the possessive \emph{-s} is typically associated with more formal registers. In (\ref{ex:bestemormins}), however, the \emph{-s} is attached to a phrase which is itself a possessive construction (\emph{bestemor min} `my grandmother') which has an informal, dialectal flavor: it is postnominal, and it also follows one of the special patterns for kinship nouns which is typical for many spoken Norwegian dialects (see further in \sectref{sec:special}). The combination of two possessive constructions which, from a present-day EurNo perspective, have different stylistic connotations, is presumably related to the present-day heritage speakers' low amount of exposure to  written language and normative rules that apply in the homeland. 

The syntactic analysis that has been proposed for the possessive \emph{-s} in EurNo can be straightforwardly extended to NAmNo. The syntactic structure of (\ref{ex:jentaskatt}) (\emph{mors folk} `my mother's family') is shown in \figref{ex:jentaskatt-tree} (based on \citealt[225]{julien2005nominals}, details omitted). 

\begin{figure} 
\Tree [.DP [.{\emph{mor}} ] [.D\1 [.D ] [.PossP [.\sout{\emph{mor}} ] [.Poss\1 [.Poss\\{\emph{s}} ] [.ArtP [.\sout{\emph{mor}} ] [.Art\1 [.Art\\\emph{folk} ] [.NP [.DP\\\emph{\sout{mor}} ] [.N\1 [.N\\\sout{\emph{folk}} ] ] ] ] ] ] ] ] ] 
\caption{Possessive construction with possessive \emph{-s}, following \citet{julien2005nominals}.} 
\label{ex:jentaskatt-tree} 
\end{figure}
 
As the tree shows, the possessor \emph{mor} `mother' is base-generated in Spec-NP and moves up to Spec-DP (via Spec-PossP). The possessive -\emph{s} is located in the Poss head below D. 

\section{Possessive PPs}\label{sec:PP-poss}
When the possessor role is expressed by a noun or a modified DP (i. e., not just a simple pronoun/determiner, \citealt[70]{johannessenjulienlødrup2014menneskesentrert}), EurNo possessive constructions may involve a PP. The possessor is realized as the complement of a preposition (\emph{til} or \emph{åt} `to'), the PP invariably follows the noun expressing the possessee, and nothing can intervene between the two. Possessive PPs are found in NAmNo too, as shown in  (\ref{ex:NAmNo-possessivePP}):

\ea \label{ex:NAmNo-possessivePP}
\ea \gll navn-et \textbf{til} \textbf{hotell-et} \\
name-\textsc{def} to hotel-\textsc{def}\\
\glt `the name of the hotel' (stillwater\_MN\_01gm) \\
\ex \gll bak-en \textbf{til} \textbf{car-en} \textbf{min} \\
back-\textsc{def} to car-\textsc{def} my \\
\glt `the back of my car' (westby\_WI\_07gk) \\
\ex \gll sid-a \textbf{åt} \textbf{mor} \textbf{hennes} \\
side-\textsc{def} to mother her \\
\glt `her mother's side' (coon\_valley\_WI\_06gm) \\
\z
\z

\noindent In possessive constructions with a PP in EurNo, the possessee must generally carry a definite suffix, similar to the possessive constructions with a postnominal determiner described in \sectref{sec:prenom_postnom} (certain kinship nouns are an exception, see discussion in \sectref{sec:special}). 
The requirement that the possessee must carry a definite suffix generally seems to hold in NAmNo, like in EurNo \citep[203]{kinn2021split}.\footnote{However, occasional exceptions to the the requirement that the possessor cannot be a simple pronoun are found \citep[196]{kinn2021split}. More research is required to establish how systematically this occurs.} In terms of syntactic  structure, the same analysis, which is in essence similar to the analysis of postnominal possessive determiners (\sectref{sec:prenom_postnom}), can be applied to both EurNo and NAmNo: the PP introduced by (\emph{til/åt} `to') is base-generated in the specifier of NP, and the possessed noun moves past it, yielding the postnominal order. 






In some EurNo dialects, possessive PPs are only used with common nouns as possessors; if the possessor is a proper name, the construction with a possessive determiner and proper name is used (\sectref{subsec:poss+prop} and \citealt[141]{julien2005nominals}). Although it is not categorical, and more systematic research must be left for the future, NAmNo appears to tend towards this distribution, with possessive PPs mainly used with common nouns as complements.


\section{The possessor doubling construction}\label{sec:poss_doubling}
The so-called possessor doubling construction \citep[214]{julien2005nominals} consists of a possessor, which in most EurNo dialects must be a noun, a possessive marker \emph{sin}, and the possessee, which is typically a noun in the indefinite form. The possessive marker  agrees in number (and gender, if sg.) with the possessee. The possessor doubling construction is found in NAmNo; some examples are given in (\ref{ex:garpegen}) (as before, codes such as M21 represent personal names).\footnote{I queried for nouns directly followed by the lemma \emph{sin} and a noun in the indefinite form.}

\ea \label{ex:garpegen}
\ea \gll \textbf{mor-a} \textbf{sitt} folk \\
mother-\textsc{def} \textsc{poss.\textsc{n}} family \\
\glt `the mother's family' (webster\_SD\_02gm)\\
\ex \gll \textbf{alle} [...] \textbf{M21} \textbf{sine} unger \\
all ... M21 \textsc{poss.pl} kids \\
\glt `all M21's kids' (hatton\_ND\_04gk)\\
\ex \gll \textbf{M5} \textbf{sin} bror \\
M5 \textsc{poss.\textsc{m}} brother \\
\glt `M5's brother' (saskatoon\_SK\_01gk)\\
\z 
\z 



\noindent Although the possessive marker \emph{sin} that appears in this construction is homophonous with the 3rd person reflexive possessive determiner (see Table \ref{table:possdet} and \sectref{subsec:reflexive}), its syntactic behavior is different (see \citealt[214ff]{julien2005nominals} for discussion). The syntax of the possessor doubling construction can be analyzed along the same lines as the possessive \emph{-s} (cf. \sectref{sec:s-gen}): Like the possessive \emph{-s}, \emph{sin} is generated in the Poss head below D, while the possessor phrase originates in Spec-NP and moves up to Spec-DP to make the D-layer visible. This yields strings in which the possessor is directly followed  (``doubled'') by the possessive \emph{sin}. 


The possessor doubling construction has traditionally been  considered a Western Norwegian feature, possibly of Low German origin (see \citealt{norde2012possessordoubling} and references therein for further discussion), although it is now common in other parts of the country as well. 
What seems like a considerable share of the attestations of the possessor doubling construction in NAmNo comes from speakers in Canada, who are a minority compared to NAmNo speakers in the US. Some of the speakers in Canada report the county of Rogaland in Western Norway as one of the places of origin of their Norwegian ancestors. The construction is, however, not restricted to speakers of Western Norwegian heritage. More systematic studies of the distribution of the possessor doubling construction must be left for the future. 



\section{Special patterns with relational nouns}\label{sec:special} 
Relational nouns are nouns whose meaning is defined by their relationship to another entity  \citep{matthews2014dictionaryoflinguistics}, e.g., `friend' or `sister'. EurNo (like many other languages, see, e.g., \citealt{stolzetal2008splitpossession} and \citealt[chap. 10]{dixon2010basic}) has some possessessive constructions in which relational nouns display special patterns. In particular, this concerns body parts and kinship terms, which express inalienable possession (\citealt{lødrup2009external, lødrup2014kinship, lodrup2018prominent, johannessenjulienlødrup2014menneskesentrert}). In this section, I  briefly look at some of these patterns and the extent to which they are also found in NAmNo.\footnote{Constructions not discussed here include kinship nouns with prenominal, unfocused possessive determiners and indefinite kinship nouns with a pragmatically anchored implicit possessor (\emph{John bodde hos far} `John lived with father', 
where `father' can be interpreted as, e.g., John's father or the speaker's father, depending on the context) \citep{lødrup2014kinship}. Both of these topics would require extensive prosodic and/or contextual analysis.} \sectref{subsec:bare} discusses postnominal possessives combined with a bare kinship noun; \sectref{sec:bodypart-på} discusses body part nouns and possessive PPs with \emph{på}, and \sectref{subsec:implicit} discusses definite kinship nouns with an implicit possessor.


\subsection{Bare kinship nouns with postnominal possessives}\label{subsec:bare}
As mentioned in Sections \ref{sec:prenom_postnom} and \ref{sec:PP-poss},  constructions with postnominal possessesive determiners and postnominal possessive PPs generally require the possessed noun to be definite. However, in spoken EurNo dialects, certain kinship nouns, typically the most frequent ones denoting close family relations, are exceptions from this rule and appear in their bare form with no suffix. Some EurNo examples from the Nordic Dialect Corpus are given in (\ref{ex:eurnobare}) (from \citealt{kinn2021split}):

\ea \label{ex:eurnobare}
\ea \gll \textbf{bror} min eide gard-en\\
brother my owned farm-\textsc{def}\\
\glt `my brother owned the farm' (EurNo, alvdal\_04gk) \\
\ex \gll for \textbf{mor} di hun er jo oppvokst her\\
because mother your she is \textsc{mod.part} grown.up here\\
\glt `because your mother grew up here' (EurNo, ballangen\_03gm)\\
\ex \gll og \textbf{far} til venninn-a mi jobba jo på Namnå\\
and father to friend-\textsc{def} my worked \textsc{mod.part} at Namnå\\
\glt `and my friend's father worked at Namnå (EurNo, kirkenaer\_08gk)
\z
\z



\noindent \citet{kinn2021split} shows that this pattern is generally retained in NAmNo, and that it does not appear to be in decline (abstracting away from some inter-speaker variation). In fact, compared to a baseline of older, homeland dialect speakers, some NAmNo speakers seem to be using bare nouns even more extensively, with more kinship terms, suggesting an extension of the  special pattern. The use of bare nouns in possessive constructions does not seem to extend beyond kinship terms \citep[203]{kinn2021split}, suggesting that the overuse of bare forms is a result of systematic overgeneralization of the special rule for kinship nouns rather than, e.g., more general instability of the definite suffix. This is in line with findings from other studies focusing on definiteness marking in NAmNo (e.g., \citealt{anderssenetal2018cross-linguistic} and \citealt{vanbaal2020comp}, see discussion in \citetv{chapters/vanbaal}).  

The examples in (\ref{exs:bare}) illustrate bare nouns in the relevant possessive constructions in NAmNo (more examples are given in \citealt{kinn2021split}). 

\ea \label{exs:bare}
\ea  \label{ex:søsterhans} \gll \textbf{søster} hans var fire år\\
sister his was four years \\
\glt `his sister was four years old' (sunburg\_MN\_03gm)\\
\ex \label{ex:bestefartilmor} \gll \textbf{bestefar} til mor mi  [...] kom i attenåtteogsytti \\
grandfather to mother my {} came in 1878 \\
\glt `my mother's grandfather came in 1878' (fargo\_ND\_01gm)\\
\ex \label{ex:tremenning} \gll \textbf{tremenning} til kon-a var i sykehjem \\
second.cousin to wife-\textsc{def} was in nursing.home \\
\glt `my wife's second cousin was in a nursing home' (fargo\_ND\_10gm)\\
\z
\z 

\noindent \emph{Søster} `sister' in  (\ref{ex:søsterhans}) (with a postnominal possessive determiner) and  \emph{bestefar} `grandfather' in (\ref{ex:bestefartilmor}) (with a possessive PP) are kinship nouns that are attested in their bare form in these constructions in EurNo too. \emph{Tremenning} `second cousin' in (\ref{ex:tremenning}), on the other hand, along with a handful of other, more peripheral kinship relations, was not found in its bare form in the EurNo sample investigated by \citet{kinn2021split}, or in a small sample from 1st generation speakers (emigrants to North America) that was also a part of that study. The current version of CANS (v. 3.1) includes 45 more recordings of Norwegian emigrants (primarily recordings from the 1940s); additional searches for possessive constructions with the more peripheral kinship nouns in these data also gave no examples of bare forms. This corroborates the argument that the use bare forms with certain peripheral kinship nouns is a fairly  recent development that has happened in North American heritage language context.

Another observation which is compatible with the idea of productivity of the pattern with bare kinship nouns, is that it is occasionally found with English loans, such as \emph{auntie}. This is shown in (\ref{ex:auntie}) (from \citealt[195]{kinn2021split}):

\ea 
\label{ex:auntie} \gll ... var gift med \textbf{auntie} mi\\
... was married to auntie my\\
\glt `... was married to my auntie' (westby\_WI\_01gm, CANS) \\
\z 

\noindent Here, the loanword \emph{auntie} appears in a postnominal possessive construction without a definite suffix, analogous to Norwegian-origin kinship nouns. For a more detailed formal analysis of the use of bare kinship nouns, see \citet{kinn2021split} and references there.

\subsection{Body part nouns and possessive PPs with \emph{på} `on'}\label{sec:bodypart-på}
Another construction in EurNo involves body part nouns in the definite form and a PP introduced by \emph{på} `on' (\citealt{lødrup2009external, lodrup2018prominent, johannessenjulienlødrup2014menneskesentrert}). \citet{lodrup2018prominent} refers to the possessor as a `prominent internal possessor', implying that the possessor in some respects behaves like an argument of the clause, even if it is not a separate constituent.\footnote{Constituency tests reveal that the possessive PP can be either DP-internal and DP-external, see Lødrup (\citeyear{lødrup2009external}, \citeyear{lodrup2018prominent}), and it can sometimes be difficult to decide which analysis is correct for a given example. I leave this issue aside here.}  Cf. (\ref{ex:ryggenpåhan}):

\ea \label{ex:ryggenpåhan} \gll De skar dypt i \textbf{rygg-en} \textbf{på} \textbf{ham}.\\
they cut deep in back-\textsc{def} on him\\
\glt `They cut deep in his back.' (EurNo, \citealt[238]{lodrup2018prominent}) 
\z 


\noindent Only body part nouns (such as \emph{ryggen} `the back' in (\ref{ex:ryggenpåhan})), and garments worn by the possessor, can occur as possessees in this construction. The construction has a number of additional characteristics and restrictions \citep[235--238]{lodrup2018prominent}: For example, the body part noun cannot be modified by a non-restrictive adjective (see (\ref{ex:på-modified})); if the body part noun denotes something that the body only has one of, it occurs the singular, with a distributive reading if the possessor is plural (see (\ref{ex:kakerimunnene})), and the possessor must be affected by the verbal action (see (\ref{ex:diskuterteryggen})).\footnote{The preposition \emph{på} is  used DP-internally beyond this construction, to express, e.g., part-whole relationships and locations, but without the restrictions and special properties described here.}   

\ea 
\ea \label{ex:på-modified} \gll *Hun vasket grundig (*den skitne) ryggen på ham.\\
she washed thoroughly the dirty back.\textsc{def} on him\\
\glt `She washed his (dirty) back thoroughly.' (EurNo; \citealt[236]{lodrup2018prominent})\\
\ex \label{ex:kakerimunnene} \gll Hun stappet kaker i munnen / *munnene på dem. \\
she popped cakes in mouth.\textsc{def} / *mouths.\textsc{def} on them.\\
\glt `She popped cakes into their mouths' (EurNo; \citealt[237]{lodrup2018prominent})\\
\ex \label{ex:diskuterteryggen} \gll *Legene diskuterte ryggen på dem.\\
doctors.\textsc{def} discussed back.\textsc{def} on them\\
\glt (Intended:) `The doctors discussed their backs.' (EurNo; \citealt[237]{lodrup2018prominent})\\
\z 
\z 


\noindent Lødrup proposes an LFG-based account of the construction in terms of backward possessor raising -- see \citet{lodrup2018prominent} for details of the analysis. Turning now to NAmNo,  queries in CANS only yielded clear examples of this construction  in the older recordings  (1942), even though the recent recordings (2010--2016) constitute the biggest part of the corpus by far (roughly 10 times as many word tokens).\footnote{I queried for definite nouns directly followed by \emph{på} and then a pronoun, determiner or another defininte noun. There is  one more recent example from the speaker westby\_WI\_01gm transcribed as \emph{hodet på han}; however, this  is very difficult to hear.} Cf. (\ref{ex:cans-ryggenpå}):

\ea \label{ex:cans-ryggenpå}
\ea \gll  også \# dyppa hun \# bomull nedi oil of cloves også hadde inni \textbf{kjeft-en} \textbf{på} \textbf{han}\\
and.then {} dipped she {} cotton down.in oil of cloves and.then had inside mouth-\textsc{def} on him\\
\glt `and then she dipped cotton in oil of cloves and put it in his mouth' (blair\_WI\_23um, recorded 1942)\\
\ex \label{ex:manenpåhestene} \gll så veit du mange tider \textbf{man-en} \textbf{på} \textbf{hest-ene} var nesten isammenvottet\\
so know you many times mane-\textsc{def} on horse-\textsc{pl.def} was almost together.tangled \\
\glt `so, you know, many times, the manes of the horses were all tangled'
(spring\_grove\_MN\_32gm, recorded 1942)\\
\z 
\z 

\noindent Note that in (\ref{ex:manenpåhestene}), the body part noun \emph{man} `mane' is in the singular, with a distributive reading; the possessor \emph{hestene} `the horses' is plural. The construction with body part nouns and \emph{på} thus seems to follow the same restriction with respect to number as in homeland Norwegian. 

It is not clear if the lack of attestations of the construction in the most recent recordings in CANS is a coincidence, or perhaps an artefact of different recording situations, or if the construction has lost ground to other possessive constructions in NAmNo over time. If the latter is the case, this would be  an exception to the general picture described in the previous sections, whereby many of the EurNo ways of expressing possession are retained, including constructions that have no clear English counterpart.  




\subsection{Definite nouns with implicit possessors}\label{subsec:implicit}
In all the possessive constructions discussed so far in this chapter, the possessor is overtly realized, either prenominally or postnominally. However, another special pattern for some relational nouns involves the absence of an  overt possessor; instead, a plain, definite noun referring to the possessee is used. Both (certain) kinship nouns and nouns denoting body parts can be used in this way (and also some nouns that can be seen as extensions of kinship nouns and body part nouns, e.g. \emph{nabo} `neighbor' and \emph{lomme} `pocket'; see Lødrup \citeyear{lodrup2010implicit}, \citeyear{lødrup2014kinship}). The implicit possessor may be the subject of the clause, another referent which is salient in the discourse, or the speaker. Some EurNo examples are given in (\ref{ex:eurnoimplicit}) (based on \citealt{lødrup2014kinship}):\footnote{For ease of exposition, both examples have been translated with the subject of the clause as the possessor. However, depending on the context, the possessor could also be someone else.}

\ea \label{ex:eurnoimplicit}
\ea \gll John vasker \textbf{ansikt-et}\\
John washes face-\textsc{def}\\
\glt `John washes his face' (EurNo) \\
\ex \gll John snakket med \textbf{far-en}\\
John spoke with father-\textsc{def}\\
\glt `John spoke to his father' (EurNo) \\
\z 
\z 

\noindent Definite nouns with an implicit possessor are found in NAmNo, both with kinship nouns and body parts; some examples are given in (\ref{exs:implicitpossessor}).


\ea \label{exs:implicitpossessor}
\ea \label{ex:hodet} \gll hun sitter og rister på \textbf{hod-et} \\
she sits and shakes on head-\textsc{def} \\
\glt `she is shaking her head' (coon\_valley\_WI\_06gm)\\
\ex \label{ex:håret} \gll jeg må vaske \textbf{hår-et} for [...] jeg har vært \# i [...] barn\\
I must wash hair-\textsc{def} for {} I have been {} in {} barn\\
\glt `I have to wash my hair because I have been in the barn' (sunburg\_MN\_15gm) \\
\ex \label{ex:faren} \gll han var norsk og dansk \#	\textbf{far-en} kom	her fra Denmark \\
he was Norwegian and Danish {} father-\textsc{def} came here from Denmark\\
\glt `He was Norwegian and Danish. His father came here from Denmark.' (fargo\_ND\_05gk) \\
\z 
\z 

\noindent In (\ref{ex:hodet}) and (\ref{ex:håret}), the implicit possessor of \emph{hodet} `the head' and \emph{håret} `the hair' is the subject of the clause; in (\ref{ex:faren}), the possessor of \emph{faren} `the father' is a salient referent in the discourse (the subject of the preceding clause; this  example differs from (\ref{ex:hodet}) and (\ref{ex:håret}) in that the possessed noun \emph{faren} is not in a configuration where it can be syntactically bound, see \citealt{lødrup2014kinship}).\footnote{Another example is \emph{kona} `the wife' in (\ref{ex:tremenning}) above; this is a complement of a possessive PP which is the possessor of \emph{tremenning} `second cousin', but at the same time it has its own implicit possessor, who is the speaker of the utterance (i.e., the husband).} 
Although the use of definite nouns with an implicit possessor is found in NAmNo, it is a task for future research to investigate the distribution of these constructions compared to constructions with an overt possessive determiner, which are more similar to English, and which are also an option in both EurNo and NAmNo.

\section{Discussion and theoretical relevance of the observations}\label{sec:theoreticalrelevance}
A recurring observation throughout this chapter has been that the various possessive constructions of EurNo are also found in NAmNo, and that the analyses that have been proposed for EurNo can be extended to the heritage variety. This adds to the body of evidence suggesting a high degree of stability of syntactic representations in heritage languages (e.g., \citealt{benmamounetal2013hls, polinsky2018heritage, lohndaletal2019hlacquisition} and references therein). Some differences with respect to the \emph{distribution} of the different options have been noted. Importantly, however, NAmNo speakers do not generally display a preference for options that converge with the majority language English in their choice of possessive constructions. This is shown perhaps most clearly by \citet{anderssenetal2018cross-linguistic} and \citet{eidehjelde2023input}, who found that a majority of NAmNo speakers prefer postnominal possessive determiners rather than prenominal possessive determiners. Relatedly, the possessive \emph{-s}, which strongly resembles the Saxon genitive in English, was not shown to be more frequent in  NAmNo than in a sample from EurNo dialect speakers. These observations show that direct CLI from the majority language English overall plays a very limited role in the syntax of possessive constructions in NAmNo (although there is some interesting variation, particularly on the inter-individual level, as shown by \citealt{anderssenetal2018cross-linguistic} and \citealt{eidehjelde2023input}).

The findings from possessive constructions with bare kinship nouns (\sectref{sec:special}) are slightly different in the sense that there is evidence for a qualitative difference between NAmNo and EurNo, at least for some speakers: Some present-day speakers of NAmNo use the special patterns (i.e., no definite suffix) with more kinship terms than what has been attested  in EurNo or data from emigrant speakers. However, as \citet[212]{kinn2021split} points out, the change is systematic and restricted -- the use of bare forms does not seem to extend beyond kinship nouns. This lends support to the (by now well-established) notion that change in heritage languages does not necessarily involve loss or incompleteness (\citealt{hoppputnam2015restructuring, kupishrothman2016terminology, bayrametal2019terminology}), despite the reduced input that heritage languages receive. Heritage languages develop on their own terms, and although the outcomes are not always the same as in the homeland variety, they can be analyzed with the same analytical tools (in this case, an over-generalisation of a pattern already present in the language) (see also \citealt{kupischpolinsky2021fastforward}).


\section{Conclusion and outlook}\label{sec:conclusion}
This chapter has discussed DP-internal possessive constructions in NAmNo. The homeland variety EurNo has a wide range of possessive constructions; the chapter has shown that the various ways of expressing possession are generally found in NAmNo, too. The majority of the speakers do not display any clear preference for options that converge with English, and  the syntactic analyses proposed for EurNo can be extended to NAmNo. 

Relevant topics for future research on NAmNo possessive constructions include (but are of course not restricted to) the use of reflexive determiners, for which only some preliminary observations were presented in this chapter, and the pragmatic and information-structural conditions on prenominal and postnominal possessive determiners. Investigations into the latter topic would enhance our understanding of the observed preference for postnominal possessive determiners (generally used when the possessor is topical in present-day \mbox{EurNo}), possibly at the expense of prenominal ones (generally used to focus the possessor in EurNo). Moreover, it could shed light on the more general question of the vulnerability of phenomena at the syntax-pragmatic interface and the difference between core and periphery in grammar (e.g., \citealt{sorace2011pinning, lohndaletal2019hlacquisition} and references therein). In terms of methodologies, it might be beneficial to supplement corpus queries in CANS with experimental data in order to target low-frequency phenomena and aspects of the linguistic competence that relate to comprehension rather than production. However, experimental methods must be chosen with care, especially with elderly heritage speakers such as the speakers of NAmNo (\citealt{montrul2015acquisition, polinsky2018heritage}). Also, as NAmNo is a moribund language, time would be of the essence.


\section*{Abbreviations}
\begin{tabbing}
MMMMM \= Courpus\kill
CANS \> Corpus of American Nordic Speech\\
\textsc{def} \> Definite\\
CLI \> Cross-linguistic Influence \\
CLO \> Cross-linguistic Overcorrection\\
DP \> Determiner Phrase \\
EurNo \> European (= homeland) Norwegian\\
\textsc{f} \> Feminine \\
\textsc{m} \> Masculine \\
\textsc{mod.part} \> Modal Particle\\
\textsc{n} \> Neuter\\
NAmNo \> North American Norwegian \\
\textsc{pl} \> Plural\\
\textsc{poss} \> Possessive \\
PP \> Preposition Phrase\\
\textsc{refl} \> Reflexive\\
\textsc{sg} \> Singular
\end{tabbing}

\section*{Acknowledgements}
I would like to express my gratitude to two reviewers and Mike Putnam for their constructive feedback.  Any remaining errors are my own.
 The research reported in this chapter was supported by the Research Council of Norway, project 301114.  
\printbibliography[heading=subbibliography,notkeyword=this]


\end{document}
