\documentclass[output=paper]{langscibook}
\ChapterDOI{10.5281/zenodo.15274574}
\author{Ida Larsson\orcid{}\affiliation{Østfold University College} and Kari Kinn\affiliation{University of Bergen}}
\title[Argument placement in North American Norwegian]
	  {Argument placement in North American Norwegian: Subject shift, object shift and verb particles}
\abstract{This chapter discusses argument placement, in particular the position of subjects and objects relative to negation and sentence adverbials, and the position of objects relative to verb particles. These are areas in which European Norwegian (EurNo) displays complex variation, involving factors such as information structure, semantics and prosody in addition to syntactic conditions in the stricter sense. The chapter argues that North American Norwegian (NAmNo) overall displays a remarkable stability. There are, however, some patterns that may seem surprising from a present-day EurNo perspective. The chapter argues that cross-linguistic influence (CLI) plays a certain, but limited, role in explaining these patterns, and that some of the properties of NAmNo argument placement can be traced back to a baseline of older, rural Norwegian dialects, which is not necessarily identical to EurNo as spoken today. It thus highlights the importance of carefully establishing a baseline in heritage language studies, and of studying the developments of heritage languages over time when this is possible.}

\IfFileExists{../localcommands.tex}{
  \addbibresource{../localbibliography.bib}
  \usepackage{langsci-optional}
\usepackage{langsci-gb4e}
\usepackage{langsci-lgr}

\usepackage{listings}
\lstset{basicstyle=\ttfamily,tabsize=2,breaklines=true}

%added by author
% \usepackage{tipa}
\usepackage{multirow}
\graphicspath{{figures/}}
\usepackage{langsci-branding}

  
\newcommand{\sent}{\enumsentence}
\newcommand{\sents}{\eenumsentence}
\let\citeasnoun\citet

\renewcommand{\lsCoverTitleFont}[1]{\sffamily\addfontfeatures{Scale=MatchUppercase}\fontsize{44pt}{16mm}\selectfont #1}
   
  %% hyphenation points for line breaks
%% Normally, automatic hyphenation in LaTeX is very good
%% If a word is mis-hyphenated, add it to this file
%%
%% add information to TeX file before \begin{document} with:
%% %% hyphenation points for line breaks
%% Normally, automatic hyphenation in LaTeX is very good
%% If a word is mis-hyphenated, add it to this file
%%
%% add information to TeX file before \begin{document} with:
%% %% hyphenation points for line breaks
%% Normally, automatic hyphenation in LaTeX is very good
%% If a word is mis-hyphenated, add it to this file
%%
%% add information to TeX file before \begin{document} with:
%% \include{localhyphenation}
\hyphenation{
affri-ca-te
affri-ca-tes
an-no-tated
com-ple-ments
com-po-si-tio-na-li-ty
non-com-po-si-tio-na-li-ty
Gon-zá-lez
out-side
Ri-chárd
se-man-tics
STREU-SLE
Tie-de-mann
}
\hyphenation{
affri-ca-te
affri-ca-tes
an-no-tated
com-ple-ments
com-po-si-tio-na-li-ty
non-com-po-si-tio-na-li-ty
Gon-zá-lez
out-side
Ri-chárd
se-man-tics
STREU-SLE
Tie-de-mann
}
\hyphenation{
affri-ca-te
affri-ca-tes
an-no-tated
com-ple-ments
com-po-si-tio-na-li-ty
non-com-po-si-tio-na-li-ty
Gon-zá-lez
out-side
Ri-chárd
se-man-tics
STREU-SLE
Tie-de-mann
} 
  \togglepaper[1]%%chapternumber
}{}

\begin{document}
\maketitle 


\section{Introduction}\label{sec:larsson:1}

European Norwegian (EurNo) is a V2-language, allowing (and requiring) one constituent before the finite verb in declarative main clauses. As is shown in \textcitetv{chapters/anderssen} and references therein, the V2-property is generally robust in North American Norwegian (NAmNo) as well, although some interesting variation is found. The preverbal position is often occupied by the subject, but it can also be filled by objects or other constituents. The preverbal element is fronted to Spec-CP (see \citetv{chapters/introduction});\footnote{Throughout this chapter we assume a symmetric analysis of V2 whereby the verb moves to C both in subject-initial and non-subject-initial clauses. The preverbal element (subject or non-subject) is thus in Spec-CP; see, e.g., \citet[362ff]{Holmberg2015} and \textcitetv{chapters/anderssen} for discussion of this and alternative proposals. We do not reject more fine-grained approaches to the C-domain in Scandinavian (see \citealt{LarssonForthcoming}); however, a simple CP is sufficient for the purposes of this chapter.}  EurNo examples with a subject and an object in Spec-CP are given in \REF{ex:larsson:1}.

\ea \label{ex:larsson:1}
\ea (Subject in Spec-CP)\label{ex:larsson:1a}\\
\gll \textbf{Gutten} leste den boka {i går}.\\
      boy.\textsc{def} read that book.\textsc{def} yesterday\\
\glt ‘The boy read the book yesterday.’

\ex (Object in Spec-CP)\label{ex:larsson:1b}\\
\gll \textbf{Den} \textbf{boka} leste gutten {i går}.\\
     that book.\textsc{def} read boy.\textsc{def} yesterday\\
\glt ‘The boy read the book yesterday.’
\z
\z

Fronting vs. non-fronting does not exhaust the options for subject and object placement; non-fronted arguments display more fine-grained variation. Non-fronted subjects can either precede or follow negation and other sentence adverbials, as shown in \REF{ex:larsson:2}:

\ea \label{ex:larsson:2}
\ea (S-Neg)\label{ex:larsson:2a}\\
\gll Den boka har \textbf{gutten} \textbf{ikke} lest.\\
     that book.\textsc{def} has boy.\textsc{def} not read\\
\glt ‘The boy hasn’t read that book.’

\ex (Neg-S)\label{ex:larsson:2b}\\
\gll Den boka har \textbf{ikke} \textbf{gutten} lest.\\
	 that book.\textsc{def} has not boy.\textsc{def} read\\
\glt ‘The boy hasn’t read that book.’
\z
\z

The pattern in \REF{ex:larsson:2a} (S-Neg) is referred to as \textit{subject shift}, implying that the subject has shifted past negation. Subjects following negation (like in \REF{ex:larsson:2b}) are referred to as \textit{unshifted} (see, e.g., \citealt{LundquistTengesdal2022} and references therein). 

Non-fronted objects can also either precede or follow negation and other sentence adverbials, as shown in \REF{ex:larsson:3}: 

\ea\label{ex:larsson:3}
\ea (O-Neg)\label{ex:larsson:3a}\\
\gll Han leste \textbf{den} \textbf{ikke}.\\
     he read it not\\
\glt ‘He didn’t read it.’
\ex (Neg-O)\label{ex:larsson:3b}\\
\gll Han leste \textbf{ikke} \textbf{boka}.\\
    he read not book.\textsc{def}\\
\glt  ‘He didn’t read the book.’ 

\ex (O-Neg)\label{ex:larsson:3c}\\
\gll * Han leste \textbf{boka} \textbf{ikke}.\\
     {} he read book.\textsc{def} not \\
\glt (Intended:) ‘He didn’t read the book’
\ex (Neg-NonFinV-O)\label{ex:larsson:3d}\\
\gll Han har \textbf{ikke} lest \textbf{den}.\\
    he has not read it\\
\glt ‘He hasn’t read it.’
\ex (O-Neg-NonFinV)\label{ex:larsson:3e}\\
\gll * Han har \textbf{den} \textbf{ikke} lest\\
     {} he has it not read\\
\glt   (Intended:) ‘He hasn’t read it.’
\z
\z

The pattern in \REF{ex:larsson:3a} is referred to as \textit{object shift} \citep{Holmberg1986}. As illustrated in (\ref{ex:larsson:3}b–c), object shift across negation only applies to unstressed (weak) pronouns. It does not apply to non-pronominal objects (\ref{ex:larsson:3}c) or stressed pronouns in EurNo (see further \sectref{sec:larsson:4.1}); these objects must follow negation. Furthermore, object shift only applies when the main verb has moved out of the vP (Holmberg’s generalization, e.g., \citealt{Holmberg1986,Holmberg1999}). Thus, in practice, object shift is limited to main clauses with a single, finite main verb and no auxiliaries.\footnote{Recall that the verb generally stays in situ in EurNo embedded clauses, see \textcitetv{chapters/introduction} and \textcitetv{chapters/anderssen}; thus, object shift does not apply there.} In \REF{ex:larsson:3d}, object shift does not apply, as the main verb is non-finite and remains in situ. Shifting the object past a non-finite verb and negation, as in \REF{ex:larsson:3e}, is unacceptable. 

Non-fronted objects display variation along one further axis: EurNo is a satellite-framed language \citep{Talmy2000}, making wide use of verb particles to express path or result. A non-fronted object can either precede or follow a verb particle, as shown in \REF{ex:larsson:4}:\footnote{Curly brackets are used when examples show two (or more) alternative word orders. The percent sign \% means that the acceptability of the indicated position varies between speakers.} 

\ea \label{ex:larsson:4}
\ea \label{ex:larsson:4a}
\gll Vi kastet \textbf{\{søppelet\}} \textbf{ut} \textbf{\{søppelet\}}.\\
     we threw rubbish.\textsc{def} out rubbish.\textsc{def}\\
\glt ‘We threw out the rubbish.’
\ex \label{ex:larsson:4b}
\gll Vi kastet \textbf{\{det\}} \textbf{ut} \textbf{\{\%det\}}\\
     we threw it out it\\
\glt ‘We threw it out.’
\z
\z

Non-pronominal objects can either precede or follow particles; see \REF{ex:larsson:4a}. Pronominal objects generally precede particles, but not categorically in all dialects (\citealt{LarssonLundquist2014}); some further details are given in \sectref{sec:larsson:5.1}; see also \citet{Aa2020} and references therein. 

As the data above suggests, the variation in argument placement in EurNo is complex (this also goes for the other homeland Scandinavian languages). Further conditions involving, e.g., information structure, semantics and prosody will be discussed in the sections to come. Moreover, the patterns are often not categorical, and there is variation between dialects and speakers (see, e.g., \citealt{LarssonLundquist2014, LundquistTengesdal2022}). The complexity of argument placement in EurNo provides an interesting starting point for studies of NAmNo~– what happens to this subtle variation in a context of reduced input and use of Norwegian?

This chapter builds on the research on argument placement in NAmNo that has been done to date, which mainly focuses on subject shift and object shift (\citealt{AnderssenWestergaard2020, LarssonKinn2021, LarssonForthcoming}). Additionally, new data and observations on particle constructions are offered (but see also \citealt{LarssonKinn2021}).

In the following, we take an approach to subject shift and object shift whereby interaction between \textit{all three positions} for subjects and objects (the preverbal position in addition to the shifted and unshifted positions) is taken into account, i.e., the shifted and unshifted positions are not studied in isolation (see \citealt{Andréasson2010, LindahlEngdahl2022}). Previous research has observed an increase of subject-initial clauses in NAmNo (e.g., \citealt{WestergaardEtAl2021}; see also \citetv{chapters/anderssen}); the present chapter argues, following \textcite{LarssonKinn2021, LarssonForthcoming}, that this has some knock-on effects on the positions further down in the clause. In addition, we observe certain limited effects of cross-linguistic influence (CLI) from English, consistent with the findings of \citet{AnderssenWestergaard2020}. Apart from this, however, it is argued that the complex variation of argument placement remains remarkably stable in NAmNo. 

Since argument placement is an area of intra- and interspeaker variation in EurNo, we need to be particularly careful when establishing a baseline to which NAmNo can be compared (e.g., \citealt{Polinsky2018}; see also discussions in \citetv{chapters/vanbaal} and \citetv{chapters/eide}). In our treatment of argument placement, the baseline will be established using older, rural Norwegian dialect recordings which can approximate the language of the emigrants, which is not necessarily identical to EurNo as spoken today.\textsuperscript{} 

The chapter has the following structure: In \sectref{sec:larsson:2}, we discuss the question of the baseline. \sectref{sec:larsson:3} is concerned with subject shift. In \sectref{sec:larsson:4}, we discuss object shift, and in \sectref{sec:larsson:5} we turn to particle constructions. \sectref{sec:larsson:6} considers the theoretical relevance of the findings, and \sectref{sec:larsson:7} concludes the chapter. 

\section{Comparing NAmNo to baseline data}\label{sec:larsson:2}

In many previous studies of NAmNo, NAmNo is compared to present-day \mbox{EurNo} dialects or urban present-day varieties; for example, \citet{LarssonJohannessen2015Incomplete}, \citet{vanBaal2020} and \citet{Lykke2020} use the Nordic Dialect Corpus \citep{JohannessenEtAl2009}, while \citet{AnderssenWestergaard2020} use corpora from the cities of Oslo and Tromsø. As is acknowledged by several of these authors, differences between present-day homeland varieties and the heritage language can have two main causes: they can either depend on changes in the homeland after the period of emigration, or changes in the heritage variety. Without historical data, it can be hard (if not impossible) to determine which is the most likely cause. However, historical studies of heritage languages are rarely possible, due to limitations in the available data. For Norwegian, though, we are now lucky to have precisely the corpora needed, CANS \citep{Johannessen2015CANS} and LIA (see further below). 

The appropriate baseline in heritage language studies depends on the research questions (e.g., \citealt{Polinsky2018}: 11). As NAmNo is a heritage language with a long history (see, e.g., \citetv{chapters/hjelde}), and as we are interested in stability and change \textit{over time} in NAmNo, the optimal baseline for comparison is, in our case, the language in relevant dialect areas in the homeland at the time of emigration. In this respect, we use the term “baseline” slightly differently from some other authors, who place more emphasis on the linguistic input of present-day heritage speakers (e.g., \citetv{chapters/vanbaal} and \citetv{chapters/riksem}). The input of present-day speakers of NAmNo will typically be the language of speakers who themselves grew up as heritage speakers in North America. We are instead concerned with the language that the first emigrants brought with them (typically in the late 1800s\slash early 1900s); this is the diachronic “starting point” of NAmNo, in the sense that it served as input for the first generation of heritage speakers. These speakers, in turn, provided input for the next generations, including present-day NAmNo speakers. Of course, since we do not have direct access to the input of the first heritage speakers, our notion of baseline will still be an approximation – although we actually do have access to recordings of some of the emigrants who left Norway during the mass emigration (see further below).\footnote{The questions of which groups of speakers (or texts) to compare is known also in other types of diachronic studies, where the available data is often more limited. However, whereas we tend to expect a higher degree of stability across generations in most types of historical studies, we might expect rapid change in heritage languages (see \citealt{KupischPolinsky2022}). This might make the baseline question more urgent in heritage language studies, although not different in principle.} 

In the case of argument placement in Norwegian, it is important to consider which dialects should be included in the baseline. In present-day urban Eastern Norwegian, pronominal subject shift tends to be categorical, object shift of weak pronouns with nominal antecedents is also nearly categorical, and pronominal objects almost always precede particles (see \citealt{LundquistTengesdal2022}). However, other dialects show more variation: unshifted pronouns are particularly common in Trøndersk (spoken in central Norway, also known as \textit{Trøndsk}), and also Western Norwegian shows more variability in argument placement (e.g., \citealt{OestboeMunch2013, Bentzen2014a,Bentzen2014b, LarssonLundquist2014}). In short, there is both inter- and intraspeaker variation in argument placement in present-day homeland Norwegian. With respect to Norwegian dialects at the time of emigration, much less is known, and the study of argument placement the history of NAmNo should therefore ideally involve new investigations of the older Norwegian dialects. The studies presented in this chapter use the recently developed LIA corpus for this purpose. The LIA corpus (\textit{Language Infrastructure made Accessible}) is a collection of old dialect recordings which have been digitized and tagged; the corpus includes speakers born as early as the late 1800s (see \citealt{HagenEtAl2021} and further works referenced there, as well as \citealt{hagenvangsnes2023}).  

The data sets used for the studies in this chapter are summarized in \tabref{tab:larsson:1}. In the sections on subject shift and object shift (Sections 3 and 4), the data and results are the same as in \textcite{LarssonForthcoming}. In the discussion of particle placement in \sectref{sec:larsson:5}, we carry out a new investigation using the same datasets. The two first rows (LIA and Emigrant Norwegian) constitute the baseline. In the rest of this section, we discuss the properties of the data sets, and how they were selected and delimited, in further detail. 

\begin{table}
\small
\caption{Data sets (\citealt{LarssonForthcoming}, their \tabref{tab:larsson:1}).}
\label{tab:larsson:1}
\begin{tabularx}{\textwidth}{lQ >{\centering}p{\widthof{speakers}} >{\raggedright}p{\widthof{1889–1926,}}  >{\raggedleft\arraybackslash}p{\widthof{Corpus size}}}
\lsptoprule
   & Dialect area & Num. of speakers & Year of birth & Corpus size (tokens) \\\midrule
LIA & Eastern Norwegian (Oppland) & 27 & 1889–1926, 1955 & 57,900 \\
Emigrant No & Eastern, Western and mixed backgrounds & 13 & 1849–1903 & 17,400\\
Older NAmNo & Mostly Eastern Norwegian & 75 & 1862–1927 & 61,100 \\
Modern NAmNo & Eastern Norwegian (Oppland) & 11 & 1918–1946 & 73,100 \\
\lspbottomrule
\end{tabularx}
\end{table}


Not all dialects are equally relevant for the development of NAmNo; it is known from previous studies that dialects in the interior of Eastern Norway (and in particular the Gudbrandsdalen valley) have a special position; many of the present-day NAmNo speakers exhibit traits from this area (see, e.g., \citetv{chapters/hjelde}). In the studies below, the investigations of the LIA corpus have been restricted to speakers in the former county of Oppland in Eastern Norway (which includes Gudbrandsdalen), and data for present-day NAmNo only includes speakers with a dialect background from this county. This limitation is important, as the aim is to uncover whether there is change in NAmNo that has not been inherited from the baseline, or if NAmNo shows stability with regard to subject shift, object shift and particle placement. 

The studies to be presented in this chapter also include some of the language of the first emigrants in the baseline (\textit{Emigrant No} in \tabref{tab:larsson:1}). As pointed out by \citet[14]{Polinsky2018}, the emigrant language might already differ from the homeland variety in some respects (as the emigrant speakers are bilingual, and no longer part of the same speech community as the homeland speakers). Therefore, the language of the emigrants is the best approximation we have of the input of the first heritage NAmNo speakers. The emigrant data (recorded by Einar Haugen, Didrik A. Seip and Ernst W. Selmer in the 1930s and 1940s, and available in CANS) is more mixed with respect to dialect background than the data from LIA. Although most of the speakers have a background in Eastern Norway, some have a Western Norwegian background; we might therefore expect more variability in argument placement. We have not attempted to distinguish speakers of different dialect backgrounds in the emigrant data set, in part because the available data is quite limited, and in part because, despite a tendency for clustering of dialects (\citetv{chapters/hjelde}), speakers of other dialect backgrounds could conceivably represent part of the linguistic input of speakers born of Eastern Norwegian parents in the Norwegian settlements. The emigrant data set used in the studies includes all speakers currently available in CANS (version 3.1) who are 1\textsuperscript{st} generation, according to the corpus metadata.

\textit{Older NAmNo} in \tabref{tab:larsson:1} consists of data from heritage NAmNo speakers recorded in the 1930s and 40s (by Einar Haugen, Didrik A. Seip and Ernst W. Selmer). In the studies below, all available heritage speakers from this period have been included (more precisely, all speakers in CANS, v. 3.1, recorded in the 1930s/1940s who are \textit{not} 1\textsuperscript{st} generation speakers according to the corpus metadata). The majority of these speakers have an Eastern Norwegian dialect background (see further in \citealt{LarssonForthcoming}). Finally, the data set labelled as \textit{Modern NAmNo} includes heritage speakers (2\textsuperscript{nd}–4\textsuperscript{th} generation) with a dialect background in Oppland. To make the investigation manageable, the study of Modern NAmNo has been restricted to recordings made in 2010; although the speakers are fewer, the Modern NAmNo data set is still larger than the other three. 

We follow \citet{LarssonForthcoming} in dividing the heritage speakers according to time of recording (``older'' vs. ``modern'' NAmNo), rather than generation since emigration (there are, for example, 3\textsuperscript{rd} generation speakers in both groups). This choice is motivated by the characteristics of the Norwegian communities, in particular regarding the status of the heritage language at different points in time. Many of the older NAmNo speakers are expected to have lived at least part of their lives in communities where Norwegian was more widely used than today, e.g., in church and other local institutions (see, e.g., \citealt{Natvig2022, EideHjelde2023}, \citetv{chapters/hjelde}). For present-day speakers, Norwegian is largely restricted to the family, and speakers might go years (or even decades!) without speaking Norwegian. It is therefore expected that heritage speakers in the Modern NAmNo data set are more dominant in English than the older NAmNo group, independently of generation since emigration, and the present-day speakers most likely received less Norwegian input when they were children, than the older NAmNo speakers did (see \citealt{LarssonForthcoming} for further discussion). 

\section{Subject shift}\label{sec:larsson:3}

This section is concerned with subject placement in NAmNo, compared to EurNo. Although the focus is on subject shift, i.e., the ordering of subjects and negation/sentence adverbials, we also include the clause-initial position. As we will argue in \sectref{sec:larsson:3.2}, this can help us understand changes in the incidence of shifted subjects in NAmNo. 

\subsection{Subject shift in EurNo}\label{sec:larsson:3.1}

As mentioned in \sectref{sec:larsson:1}, subjects are often placed in the preverbal position in Norwegian; this applies to around 70\% of all declarative main clauses in EurNo (\citealt{Olsen2019}; see also \citealt{WestergaardLohndal2019}: 96, \citealt{WestergaardEtAl2021}: 3 and references therein).\footnote{As argued in detail by \citet{LindahlEngdahl2022}, the clause-initial position serves a double function. It links the utterance to the previous discourse, either by maintaining the same topic (topic-chaining) or by introducing a switch topic (focus-chaining); see further \sectref{sec:larsson:4} below. In addition, it serves as a starting point for the utterance, often by introducing an aboutness topic (frequently the subject).}  An example was given in \REF{ex:larsson:1a}, repeated in \REF{ex:larsson:5} for convenience:

\ea \label{ex:larsson:5}
\gll \textbf{Gutten} leste den boka {i går}.\\
	 boy.\textsc{def} read that book.\textsc{def} yesterday\\
\glt ‘The boy read that book yesterday.’
\z

Postverbal subjects can be either shifted or unshifted; factors influencing the distribution include the form of the subject (pronominal vs. non-pronominal), information structure, and clause type (see, e.g., \citealt{Svenonius2002, Westergaard2011, Bentzen2014a} and references therein). Unstressed pronominal subjects tend to be shifted, while non-pronominal subjects display more variation, as illustrated in \REF{ex:larsson:6}:

\ea \label{ex:larsson:6}
\ea \label{ex:larsson:6a}
\gll Derfor leste \textbf{han} \textbf{ikke} boka.\\
	 therefore read he not book.\textsc{def}\\
\glt ‘Therefore he didn’t read the book.’
\ex \label{ex:larsson:6b}
\gll Derfor leste \textbf{\{gutten\}} \textbf{ikke} \textbf{\{gutten\}} boka.\\
     therefore read boy.\textsc{def} not boy.\textsc{def} book.\textsc{def}\\
\glt ‘Therefore the boy didn’t read the book.’
\z
\z

\citet{Westergaard2011} observes that in spontaneous speech, non-pronominal subjects in the shifted position \REF{ex:larsson:6b} are very rare, although they are not perceived as ungrammatical. 

Non-initial pronominal subjects that are topics are generally shifted; thus, \REF{ex:larsson:6a} above, with a shifted pronoun (\textit{han} ‘he’), would be natural in a context where the antecedent of the pronoun was the topic of the utterance (for a more detailed discussion of the notion of topic, see \sectref{sec:larsson:4}). Pronominal subjects that are focused and/or contrastive, on the other hand, are often found in the unshifted position. This is illustrated in \REF{ex:larsson:7}, where the subject pronoun is focused and follows negation:

\ea (Unshifted subj.)\label{ex:larsson:7}\\
\gll Derfor leste \textbf{ikke} \textbf{HAN} boka, men de andre gjorde det.\\
	 therefore read not HE book.\textsc{def} but the others did it\\
\glt ‘Therefore HE didn’t read the book, but the others did.’
\z

Clause type also plays a certain role for subject placement. In most types of embedded clauses, only the postverbal subject positions are available; the C-position is filled by a complementizer and fronting to Spec-CP is unavailable (with certain exceptions, e.g., \citealt{Julien2015,Julien2020} and references therein; \citealt{Ringstad2019}). \citet{Westergaard2011} notes, based on spontaneous speech data, that there is a clear preference for the shifted position in embedded clauses, both for pronominal and non-pronominal subjects; see \REF{ex:larsson:8}. This is different from main clauses, in which non-pronominal subjects are rarely shifted. 

\ea \label{ex:larsson:8}
\gll Hun visste at \textbf{gutten/han} \textbf{ikke} leste boka.\\
     she knew that boy.\textsc{def}/he not read book.\textsc{def}\\
\glt ‘She knew that the boy/he didn’t read the book.’
\z

In the basic clausal structure sketched for Norwegian in \textcitetv{chapters/introduction}, negation and sentence adverbials mark the left border of the verb phrase (vP). The position of the verb relative to negation can therefore serve as a diagnostic for verb placement; in \REF{ex:larsson:8}, the verb remains in situ in the verb phrase. The position of negation/sentence adverbials can, however, not be used to diagnose unshifted subjects in Norwegian as vP-internal, even though unshifted subjects follow these elements. In \REF{ex:larsson:9} below, the subject is unshifted, and follows negation, but it is still clearly vP-external, as it precedes the sentence adverbial \textit{allerede} ‘already’ and the (vP-internal) non-finite verbs. 

\ea \label{ex:larsson:9}
\gll  Derfor må ikke \textbf{elevene} \textbf{allerede} ha lest boka før timen.\\
therefore must not pupil.\textsc{pl.def} already have read book.\textsc{def}     before class.\textsc{def}\\
\glt ‘Therefore the students are not obliged to have read the book already before class.’  
\z

It appears, therefore, that there is more than one vP-external subject position in Norwegian, and that a more elaborate structure than the basic architecture given in \textcitetv{chapters/introduction} is required. Different analyses have been proposed in the literature (see, e.g., \citealt{Westergaard2011}: 302 for an overview); while some authors have proposed that all subject positions are located in a (split) C-domain, we assume for simplicity that the two postverbal positions are in the T-domain, intercepted by a position for negation/sentence adverbials.\footnote{This leaves some important questions open, for which a full account is still needed. There are differences between negation and other sentence adverbials with regard to subject placement: non-pronominal subjects shift across sentence adverbials like \textit{alltid} ‘always’ more often than across negation (\citealt{LundquistTengesdal2022}). Subjects can also occupy different positions in between a combination of adverbials (cf. \REF{ex:larsson:9}). \citet{BörjarsEtAl2003} argue for a flat structure of the middle field, and that the linear variation in subject placement is determined by OT constraints.}  The distribution of subjects described above suggests that the unshifted position might be associated with focus, but as argued by \citet{LundquistTengesdal2022}, information structure cannot fully account for the variation in subject placement. 

Before we move on to subject placement and subject shift in NAmNo, we should highlight that the patterns for subject placement in EurNo are not entirely categorical, even considering the placement of unstressed pronouns. As mentioned in Sections \ref{sec:larsson:1} and \ref{sec:larsson:2}, there is some dialectal variation (see, e.g., \citealt{OestboeMunch2013, Bentzen2014a}), as well as individual variation that in some cases crosscuts the tendencies described above. As discussed above, this makes it important to be particularly aware of the baseline question, and to take older dialect data/emigrant data into account as a part of the investigation.\largerpage

\subsection{Subject placement and subject shift in NAmNo}\label{sec:larsson:3.2}

As mentioned above, previous studies have observed an increase of subject-initial clauses in NAmNo (see also \citetv{chapters/anderssen}). \citet[14]{WestergaardEtAl2021} investigate 50 speakers of modern (i.e., present-day) NAmNo and find that 18.5\% of all declaratives are non-subject-initial; in EurNo, the corresponding number is around 30\% (see above). Moreover, \citet{AnderssenWestergaard2020} observe a change in subject shift: considering the placement of non-initial subjects, there is a higher proportion of unshifted pronominal subjects (i.e., negation-subject) in present-day NAmNo than in present-day EurNo. \textcite{LarssonKinn2021, LarssonForthcoming} suggest that these two changes – the change in the proportion of SV, and the change in the proportion of subject shift – are connected.{\interfootnotelinepenalty=10000\footnote{\citet{AnderssenWestergaard2020} also note that the increase in SV (subject-initial clauses) has consequences for the contexts for subject shift; however, they do not link the two phenomena directly to each other in the way that we will argue for below.}}  In the rest of this section, we look closer at their results.

Larsson \& Kinn observe that the frequency of SV order increases already in older NAmNo, at the expense of the shifted subject position; see \tabref{tab:larsson:2}. In declarative main clauses with negation, the frequency of SV order is around 75–80\% in the baseline (LIA and emigrant No),\footnote{This might appear slightly higher than expected, but see \citet{LarssonForthcoming} for further discussion.}  whereas in older NAmNo, it has increased to 88.3\%. In modern NAmNo, 93.8\% of the declaratives have SV order.

\begin{table}
\caption{Three subject positions in declarative main clauses with negation; pronominal subjects. (\citealt{LarssonForthcoming}, their Table 6).}
\label{tab:larsson:2}
\begin{tabular}{l *3{r@{~}r} c}
\lsptoprule
             & \multicolumn{2}{c}{Spec-CP} & \multicolumn{2}{c}{Shifted} & \multicolumn{2}{c}{Unshifted} & Total\\\midrule
LIA          & 435 & (76.9\%) & 118 & (20.8\%) & 13 & (2.3\%) & 566\\
Emigrant No  & 92  & (82.1\%) &  15 & (13.4\%) &  5 & (4.5\%) & 112\\
Older NAmNo  & 257 & (88.3\%) &  27 &  (9.3\%) &  7 & (2.4\%) & 291\\
Modern NAmNo & 783 & (93.8\%) &  32 &  (3.8\%) & 20 & (2.4\%) & 835\\
\lspbottomrule
\end{tabular}
\end{table}


Considering all three subject positions, there is a significant increase in initial subjects over time, and a corresponding, significant decrease in the proportion of shifted subjects. There is, however, no statistically significant change in the frequency of \textit{unshifted} subjects in declarative main clauses (see \citealt{LarssonForthcoming} for details and discussion). Larsson \& Kinn conclude that the growing preference for fronting subjects to Spec-CP particularly has consequences for the shifted position. This is not completely surprising, since the initial position and the shifted position have in common that they generally contain topics; as we have seen, the unshifted position is largely dedicated to focused subject pronouns. Compare the emigrant example in \REF{ex:larsson:10a} to modern NAmNo in \REF{ex:larsson:10b}. In \REF{ex:larsson:10a} an object pronoun is fronted to Spec-CP,\footnote{Or more precisely, a predicate pronoun, but this distinction is not of crucial importance to the point we are making here. The pronoun is extracted from an embedded clause; it is a question for future research whether NAmNo in any way differs from the baseline with respect to extraction.} and the postverbal topical subject pronoun precedes negation. In the heritage language, the topical subject is instead fronted to Spec-CP, as in \REF{ex:larsson:10b}.\footnote{The transcriptions are given with the standardized orthography provided in the corpora. Note that while CANS uses the \textit{Bokmål} standard, examples from LIA are in \textit{Nynorsk}.}

\ea\label{ex:larsson:10}
\ea 
\gll {å}   {nei} \textbf{{det}} {trur}   {jeg}   {ikke}   {han}   {var}\\
     oh   no that believe I   not   he   was\\
\glt ‘oh no, I don’t believe that he was’ (emigrant,      blair\_WI\_19um)\label{ex:larsson:10a}\\
\ex 
\gll {nei}   \textbf{jeg}  {trur}   {ikke}   {det}\\
      no I   believe not   that\\
\glt ‘no, I don’t think so’ (modern NAmNo, westby\_WI\_03gk; \citealt{LarssonForthcoming}, ex. 14–15)\label{ex:larsson:10b}\\
\z
\z

\citet{LarssonForthcoming} argue that the change in the proportions of shifted and unshifted subjects noted by \citet{AnderssenWestergaard2020} can be explained when all three subject positions are taken into account: It is a direct consequence of the stronger preference of SV order (subject-initial clauses) in NAmNo – the increase in SV “steals” subjects from the shifted position, but not from the unshifted position, which results in a higher proportion of unshifted subjects when the two postverbal positions are compared. The stronger preference for SV, in turn, has been attributed to CLI from English in the previous literature \citep{WestergaardEtAl2021}. However, it might rather be a consequence of a general pressure to ease processing in heritage languages. We return to the question of CLI and ease of processing in the discussion in \sectref{sec:larsson:5.2}.\footnote{\citet{LarssonForthcoming} also note a drop in the proportion of shifted subjects in embedded clauses in NAmNo, where fronting to Spec-CP is generally not an option. They conclude that subject placement in different types of embedded clauses still needs to be investigated in more detail, both in NAmNo and EurNo.}\largerpage 

There is one case where it seems clear that CLI must be involved, namely in a certain type of V1-question, tag questions. As noted by \citet{AnderssenWestergaard2020}, NAmNo has more cases of unshifted subjects in V1-questions than EurNo. Similarly, \citet{LarssonForthcoming} observe that there is 93.3\% subject shift in V1-questions in LIA, but only 23.9\% in present-day NAmNo. Unlike the increase in SV order in declaratives, this change can only be seen in the last generation of speakers, and it involves tag questions, but not other types of interrogative clauses. Tag questions are not very widespread in EurNo, and we have not found them in our homeland (LIA) or emigrant data, but they are found in modern NAmNo, just like English. Compare the example in \REF{ex:larsson:11} to the English translation – both have the pronominal subject after negation. Tag questions are, as mentioned, not very common in EurNo and would generally (at least in Eastern Norwegian) have a shifted pronominal subject.

\ea 
\gll du   kan   klare     det,   kan   \textbf{ikke}  \textbf{du}?\\
	 you   can   manage   it   can   not   you\\
\glt ‘you can manage it, can’t you?’ (modern NAmNo, coon\_valley\_WI\_07gk; \citealt{LarssonForthcoming}, ex. 13a)\label{ex:larsson:11}\\
\z

Importantly, none of the changes discussed thus far introduces completely new syntactic patterns in NAmNo. Rather, the changes involve shifts in the preferences for different available options (with different pragmatic functions). Although tag questions are rare in EurNo, questions with unshifted pronominal subjects do occur, as in, e.g., \textit{Kan ikke du komme?} ‘Can’t you come?’ (see further \sectref{sec:larsson:6.2} below). In declarative main clauses, the increase in SV order in NAmNo lies within what the baseline grammar allows (disregarding sporadic examples involving V2-violations; see \citetv{chapters/anderssen}). 

Crucially, subject placement is not categorical in the baseline. Most obviously, there is generally a choice in declarative main clauses to front either the subject or a non-subject. In addition, \citet{LarssonForthcoming} note that there is variation in the placement of postverbal unstressed pronouns in the baseline. Although unstressed pronouns generally shift across negation in Eastern Norwegian, they are on occasion unshifted in the LIA data, as in \REF{ex:larsson:12}; shifting appears to be less categorical in the older dialect recordings than in present-day urban Eastern Norwegian. Similar examples can be found in NAmNo; see \REF{ex:larsson:13}. Although examples like \REF{ex:larsson:13} might sound odd to modern urban Eastern Norwegian speakers, they appear to be inherited from the baseline, as shown by \REF{ex:larsson:12}, rather than introduced in North America. Recall that the data from LIA here only include speakers from Oppland in Eastern Norway.

\ea 
\gll så var  eg nå i mine beste   år eg   au men e da   tente \textbf{ikkje}   \textbf{eg} meir enn   to kroner dagen.\\ 
    so was I now in my best years I   too but eh then earned not   I more than   two   crowns day.\textsc{def}\\
\glt ‘I was now in my best years, me too, but then I didn’t earn more than two crowns a day’ (LIA, oeyer\_uio\_0201, \citealt{LarssonForthcoming}, ex. 11a)\label{ex:larsson:12}\\

\ex  
\gll ja de kommer ifra alle steder e siste \#siste   åra      har \textbf{ikke} \textbf{det}  vært så mye   men…\\
yes they come from all places eh last last year.\textsc{pl.def} have not it been so much but\\
\glt ‘Yes, they come from all places. During the last years, it has not been so much, but…’ (modern NAmNo, coon\_valley\_WI\_06gm; \citealt{LarssonForthcoming}, ex. 12c)\label{ex:larsson:13}\\
\z

To sum up this section, we have seen that there is a change in the proportion of shifted postverbal subjects in NAmNo compared to the baseline, which, on our account, can be explained as a knock-on effect of the strong preference to front subjects to Spec-CP in NAmNo. We have argued that the preference for SV affects the shifted position, since both this position and Spec-CP typically contain topics; the unshifted position is not affected. Whereas the growing preference for SV order arguably is due to pressure to ease processing (perhaps rather than CLI, see further discussion in \sectref{sec:larsson:6.2}), there is evidence for CLI in a pragmatically well-defined context, namely tag questions. Finally, we have observed that subject shift of weak subject pronouns is not completely categorical in NAmNo, nor, in fact, in the baseline.

\section{Object shift}\label{sec:larsson:4}

In this section, we turn to the placement of pronominal objects relative to negation/sentence adverbials. \sectref{sec:larsson:4.1} gives an overview of object shift in EurNo, and in \sectref{sec:larsson:4.2} we turn to object shift in NAmNo, compared to the baseline. As with subject shift, we will argue that there is variation in the baseline, even when only speakers from Oppland are considered. This variation is maintained in NAmNo, but, again, the changed preferences for the clause\hyp initial position can affect the placement of arguments further down in the clause.

\subsection{Object shift in EurNo}\label{sec:larsson:4.1}

As in the previous section on subjects, we start this section on object shift with a brief discussion of the preverbal position. We noted above that subjects in the preverbal position are often topical; this is true also of objects, but some further details should be mentioned. \citet{LindahlEngdahl2022} show that one of the typical functions of (pronominal) objects in the preverbal position in Swedish is that of \textit{switch topic}, i.e., a topic which refers back to the \textit{focus} of the preceding utterance (see also \citealt{BentzenAnderssen2019}). A switch topic is not new information, but its status as a topic is new; this sets switch topics apart from \textit{continued} topics, which refer back to the topic of the preceding utterance; see also \citet{FrascarelliHinterhölzl2007}. Lindahl \& Engdahl’s description of Swedish can essentially be extended to EurNo; thus, an object-initial sentence with preceding context could typically be something like \REF{ex:larsson:14}:

\ea \gll Jeg vil ikke ha pizza til middag. Det spiste jeg {i går}.\\
         I want not have pizza for dinner it ate I yesterday\\
	\glt ‘I don’t want pizza for dinner. I had that yesterday.’
	\label{ex:larsson:14}
\z

In the first sentence in \REF{ex:larsson:14}, \textit{pizza} is the focus. In the next sentence, it is a switch topic, expressed by the pronoun \textit{det} ‘it’, which is fronted to the preverbal position (Spec-CP). Note, however, that this fronting is not obligatory; objects that are switch topics may also occur further down in the clause. Lindahl \& Engdahl distinguish between two pragmatic strategies: focus-chaining (fronting of a switch topic), and topic-chaining (fronting of a continued topic); cf. fn. 5 above.

Turning now to the postverbal positions, the basic rules for object shift in EurNo were given in \sectref{sec:larsson:1}: pronominal objects shift past negation when the verb has left the vP; non-pronominal objects do not shift (see \REF{ex:larsson:3}). However, there is further and rather systematic variation (see \citealt{BentzenAnderssen2019}). A well-known observation is that object pronouns with certain types of antecedents often remain unshifted. This applies to object pronouns that refer back to \textit{types} of referents rather than individuated tokens, and also to objects with non-nominal antecedents, i.e., a vP or a whole clause (CP). Examples are given in \REF{ex:larsson:15}:

\ea%15
	\label{ex:larsson:15}
	\ea
    \label{ex:larsson:15a}
    \gll Nesten alle hadde blå sykkel, men Lisa hadde \textbf{ikke} \textbf{det}.\\
		 almost all had blue bike but Lisa had not it\\
	\glt ‘Almost everyone had a blue bike, but Lisa didn’t.’ 
	\ex \label{ex:larsson:15b}
		\gll Nesten alle syklet til jobb, men Petter gjorde \textbf{ikke} \textbf{det.}\\
		almost all cycled to work but Petter did not it.\\
	\glt ‘Almost everyone cycled to work, but Petter didn’t.’ 
    \ex \label{ex:larsson:15c}
	    \gll Værmeldingen sier at det blir snø, men jeg tror \textbf{ikke} \textbf{det.}\\
			weather.forecast.\textsc{def} says that there will.be snow but I think not it\\
	\glt ‘The weather forecast says that there will be snow, but I don’t think so.’
    \z
\z


In \REF{ex:larsson:15a}, the object pronoun \textit{det} refers back to \textit{blå sykkel} ‘blue bike’, which is a bare noun \citep{Borthen2003} denoting a type of referent rather than a specific, individuated token.\footnote{Note that the object pronoun \textit{det,} formally a neuter form, does not agree with \textit{blå sykkel} ‘blue bike’, which is masculine. This is typical of bare nouns with type reference.} In (\ref{ex:larsson:15}b–c), the antecedents are a vP (\textit{syklet til jobb} ‘cycled to work’) and a whole clause (\textit{at det blir snø} ‘that there will be snow’). In all three cases, the object pronoun \textit{det} follows negation. 

Another recurring observation is that information structure affects the position of object pronouns. Object pronouns that are focused and/or stressed do not undergo object shift. This is illustrated in \REF{ex:larsson:16}:

\ea%16
    \label{ex:larsson:16}
    \ea (O-Neg)\label{ex:larsson:16a}\\
	\gll Han leste \textbf{den} \textbf{ikke}.\\
		 he read it not\\
	\glt ‘He didn’t read it.’
	\ex (Neg-O)\label{ex:larsson:16b}\\
	\gll Han leste ikke \textbf{DEN}, men han leste de andre bøkene.\\
		he read not that but he read the other book.\textsc{pl.def}\\
	\glt ‘He didn’t read that, but he read the other books.’
    \z % you might need an extra \z if this is the last of several subexamples
\z

In \REF{ex:larsson:16a}, the object pronoun is unstressed and shifted; in \REF{ex:larsson:16b}, it carries contrastive focus and is stressed, and therefore appears in the unshifted position; object shift would be unacceptable.

Moreover, different types of topics behave differently. Also in this case, the distinction between switch topics and continued topics is relevant: while continued topics are generally shifted, switch topics often remain unshifted (if they are not fronted to the preverbal position) (e.g., \citealt{BentzenAnderssen2019, LindahlEngdahl2022}).  Consider the example in \REF{ex:larsson:17}.

\ea%17
    \label{ex:larsson:17}
    \gll Skal du se den nye Marvel-filmen {i kveld} igjen? Så du ikke den {i går?}\\
         shall you see the new Marvel-movie.\textsc{def} tonight again saw you not it yesterday?\\
    \glt ‘Are you watching the new Marvel movie again tonight? Didn’t you see it yesterday?’ 
    \z % you might need an extra \z if this is the last of several subexamples

Here, the first question introduces the new Marvel movie as focus. In the second question, \textit{den} (‘it’, referring to the movie) is a switch topic, and it is unshifted, although it is not focused. Spec-CP is not available in the question in \REF{ex:larsson:17}. In a declarative clause, the switch topic can be fronted to Spec-CP, as in \REF{ex:larsson:18a}, or remain unshifted, as in \REF{ex:larsson:18b}, where it follows the modal adverb \textit{jo}.

\ea\label{ex:larsson:18}%18
    \ea \label{ex:larsson:18a}
    \gll Skal du se den nye Marvel-filmen {i kveld} igjen? Den så du jo {i går}!\\
         shall you see the new Marvel-movie.\textsc{def} tonight again it saw you \textsc{mod} yesterday\\
    \glt ‘Are you watching the new Marvel movie again tonight? You saw it yesterday!’
    \ex \label{ex:larsson:18b}
    \gll Skal du se den nye Marvel-filmen {i kveld} igjen? Du så jo den {i går}!\\
         shall you see the new Marvel-movie.\textsc{def} tonight again you saw \textsc{mod} it yesterday \\
    \glt ‘Are you watching the new Marvel movie again tonight? You saw it yesterday!’
    \z % you might need an extra \z if this is the last of several subexamples
\z

Information structure interacts with antecedent type. As mentioned above, object pronouns with vP or CP antecedents tend to remain unshifted – these pronouns also commonly function as switch topics (\citealt{BentzenAnderssen2019, LindahlEngdahl2022}). Notably, if an object pronoun with a vP or CP antecedent is used multiple times, the first instance is often unshifted, while following instances tend to be shifted (\citealt{BentzenAnderssen2019}). This pattern can be straightforwardly explained by information structure: when the pronominal object is used the second time, it has been established as a continued topic; cf. \REF{ex:larsson:19}:

\ea%19
    \label{ex:larsson:19}
    \gll Nesten alle syklet til jobb, men Petter gjorde \textbf{ikke} \textbf{det.} Og Anne gjorde \textbf{det}   \textbf{ikke,} hun heller. \\
         almost all cycled to work but Petter did not it and Anne did it not either\\
    \glt ‘Almost everyone cycled to work, but Petter didn’t. And Anne didn’t either.’
\z % you might need an extra \z if this is the last of several subexamples

The formal analysis of object shift has been a topic of debate for a long time, with accounts ranging from purely syntactic ones to analyses placing object shift in prosody/phonology; see e.g. \citet{Erteschik-ShirEtAl2021} for a prosodic analysis. There is still no consensus on the issue (see, e.g., \citealt{LyskawaEtAl2022} for arguments against the phonology-based view). As with subject shift, it should be noted that there is some dialectal and individual variation concerning shifted and unshifted objects (e.g., \citealt{LundquistTengesdal2022}). Much like subject shift, object shift of pronouns with nominal antecedents appears to be more categorical in present-day Eastern Norwegian than in Western Norwegian dialects (e.g., \citealt{OestboeMunch2013, Bentzen2014b}). Again, this is important to keep in mind when studying NAmNo, which we turn to in the next section. 

\subsection{Object placement and object shift in NAmNo}\label{sec:larsson:4.2}

Contexts for object shift are very rare in spontaneous speech (recall that object shift is restricted to pronominal objects in clauses with one single, finite verb which has moved out of vP), and it can therefore be difficult to investigate the phenomenon in corpora. Still, corpus data shows that NAmNo exhibits object shift of weak pronominal objects across negation and other sentence adverbials, just like EurNo; see the examples in \REF{ex:larsson:20}. In \REF{ex:larsson:20a}, an object pronoun (with a nominal antecedent) precedes negation, and in \REF{ex:larsson:20b}, an object pronoun (with non-nominal reference) precedes the modal adverb \textit{jo} ‘of course’.\footnote{The examples in (\ref{ex:larsson:20}--\ref{ex:larsson:22}) come from a search for negation and sentence adverbials in the entire Norwegian part of CANS, v. 3.1. The data from \citet{LarssonForthcoming} in \tabref{tab:larsson:3} is more restricted (see \sectref{sec:larsson:2} above) and only includes potential object shift across negation.  The examples cited in (\ref{ex:larsson:20}--\ref{ex:larsson:22}) were all recorded in the 2010s. }  

\ea%20
    \label{ex:larsson:20}
    \ea 
    \gll men jeg kjenner \textbf{henne} \textbf{ikke}  \\
         but I know her not \\
    \glt ‘but I don't know her’ (sunburg\_MN\_12gk)\label{ex:larsson:20a}\\
    \ex 
    \gll ja jeg forstår \textbf{det} \textbf{jo}\\
         yes I understand it \textsc{mod}\\
    \glt ‘yes, I do understand it’ (harmony\_MN\_02gk)\label{ex:larsson:20b}\\
    \z % you might need an extra \z if this is the last of several subexamples
\z

Only weak object pronouns shift in NAmNo, as in EurNo. Other types of objects therefore follow negation and other clause-medial sentence adverbials, unless they have been fronted to the clause-initial position; examples with nouns and demonstratives are given in \REF{ex:larsson:21}.

\ea%21
    \label{ex:larsson:21}
    \ea 
    \gll da snakker vi jo \textbf{ikke} \textbf{norsk}\\
         then speak we \textsc{mod} not Norwegian\\
    \glt ‘then we don’t speak Norwegian’ (wanamingo\_MN\_04gk)\label{ex:larsson:21a}\\
    \ex 
    \gll jeg forstår \textbf{ikke} \textbf{dette}\\
         I understand not this\\
    \glt `I don’t understand this’ (blair\_WI\_04gk)\label{ex:larsson:21b}\\
    \z % you might need an extra \z if this is the last of several subexamples
\z

Moreover, as in EurNo, object shift does not seem to be possible across an intervening, vP-internal verb (Holmberg’s generalization); object pronouns therefore follow negation in sentences with an auxiliary, as in \REF{ex:larsson:22}.\footnote{NAmNo sometimes has verb movement (V-to-T) in embedded clauses where verb movement is not licit in the homeland variety (see \citetv{chapters/anderssen} and references therein). Object shift is therefore expected to be possible in embedded contexts, as long as the main verb has moved out of the vP. Since object shift is very infrequent in spontaneous speech, this is, however, difficult to investigate.}

\ea%22
    \label{ex:larsson:22}
    \ea  
    \gll  han far \# ville	\textbf{ikke} svare \textbf{oss} \\
         he father {} would not answer us \\
    \glt ‘father would not answer us’ (coon\_valley\_WI\_07gk)\label{ex:larsson:22a}\\
    \ex 
    \gll jeg kan \textbf{ikke} huske \textbf{{henne}}\\
         I can not remember her\\
    \glt ‘I can’t remember her’ (hatton\_ND\_03gm)\label{ex:larsson:22b}\\
    \z % you might need an extra \z if this is the last of several subexamples
\z


\citet{AnderssenWestergaard2020} investigate the placement of object pronouns relative to negation in clauses where object shift would be possible in the baseline. They show that object shift is less common in present-day NAmNo than in present-day EurNo: 61\% (25/41) of the pronouns with nominal reference shift in their study, whereas studies of present-day EurNo report higher proportions (e.g., 87.6\% in Bentzen \citeyear{Bentzen2014a}, based on the Nordic Dialect Corpus; 95\% in Lundquist \& Tengesdal's \citeyear{LundquistTengesdal2022} experimental study). 

NAmNo object shift across negation is also investigated by \citet{LarssonKinn2021, LarssonForthcoming}, who compare present-day NAmNo to older NAmNo and a baseline of old dialect recordings (cf. the study on subject shift discussed above). Results from \citet{LarssonForthcoming} are given in \tabref{tab:larsson:3} below; only clauses where object shift would be possible in the baseline are included, and pronouns with different kinds of reference are distinguished. The data sets in \tabref{tab:larsson:3} are identical to those in the subject shift study discussed above; see \sectref{sec:larsson:2} for an overview. This means that only speakers from Oppland are included in the LIA data, and only speakers with a family background in Oppland are included in modern NAmNo. 

\begin{table}
\caption{The proportion of object shift across negation in relevant clauses, including cases with shifted or unshifted object pronouns (but not topicalized objects) (based on \citealt{LarssonForthcoming}, their Table~7).}
\label{tab:larsson:3}
\begin{tabular}{l ccccc}
\lsptoprule
 & Nominal & Reflexive & Non-nominal & Total \\\midrule
LIA         & 4/6   & 4/4 & 3/14 & 11/24 & (45.8\%)\\
Emigrant No & 1/3   & {}- & 0/1  & 1/4   & (25.0\%) \\
Older NAmNo  & 4/7   & 3/3 & 3/19 & 10/29 & (34.5\%)\\
Modern NAmNo & 11/17 & 1/1 & 2/33 & 14/51 & (27.5\%)\\\addlinespace
Total       & 20/33 & 8/8 & 8/67 & 36/108&  \\
            & (60.6\%) & (100\%) & (11.9\%) & (33.3\%)\\
\lspbottomrule
\end{tabular}
\end{table}

Despite the restriction to speakers from Oppland, the studies by \citet{LarssonForthcoming} show that there is more variability in the baseline than in present-day (urban) Eastern Norwegian. Whereas reflexives always seem to shift in all speaker groups, there is variation in the placement of pronouns with nominal reference; examples of unshifted weak pronouns with nominal reference can be found in the different speaker groups; examples from LIA and older NAmNo are given in \REF{ex:larsson:23}. In both \REF{ex:larsson:23a} and \REF{ex:larsson:23b}, the object pronoun \textit{meg} ‘me’ is a continued topic, and it is unstressed (as can be heard from the sound files accessible through the corpus interface). 


\ea \label{ex:larsson:23}
\ea 
\gll eg kom der heilt uforløyves så han såg   \textbf{{ikke}}  \textbf{{meg}}  da\\
     I came there completely without.permission so he saw not me then\\
\glt ‘I came there completely without permission so he didn’t see me then’ (LIA, biri\_uio\_0202; \citealt{LarssonForthcoming}, ex. 31)\label{ex:larsson:23a}\\
\ex  
\gll … han \#   kom på meg   uvitende og \#   så   \textbf{ikke} \textbf{meg}\\
     {} he {} came at me unknowingly and {}  saw   not   me\\
\glt ‘he came at me unknowingly, and didn’t see me’ (older NAmNo, blair\_WI\_23um; \citealt{LarssonForthcoming}, ex. 32a)\label{ex:larsson:23b}\\
\z
\z

The results show that object shift is not categorical either in the older EurNo dialects or in NAmNo. In fact, given the variability in the baseline, the stability of NAmNo object shift is quite remarkable. 

As in the baseline, object pronouns with non-nominal reference are often unshifted – particularly when they are switch topics. An example is given in \REF{ex:larsson:24}, where \textit{det} refers back to a CP introducing the focus in the previous utterance. 

\ea%24
    \label{ex:larsson:24} 
    \gll ja vi   visste   \textbf{{ikke}}  \textbf{{det}} \\
         yes   we knew   not   it \\
    \glt ‘yes, we didn’t know it’ (modern NAmNo, coon\_valley\_WI\_07gk; \citealt{LarssonForthcoming}, ex. 34b)\\
\z % you might need an extra \z if this is the last of several subexamples

As noted above, switch topics are generally unshifted in EurNo – when they are not fronted to Spec-CP (see \REF{ex:larsson:17} and \REF{ex:larsson:18} above). This is true also in NAmNo~– with the difference that objects are much less often clause\hyp initial. \citet{LarssonForthcoming} show that while almost 10\% of all declarative main clauses (with negation) are object-initial in the baseline, there is less than 1\% object-initial clauses in the present-day NAmNo sample; as we saw above, NAmNo has a strong preference for SV order. While not all these declaratives provide a context for object shift (many include auxiliaries), it seems quite clear that the preference for SV order affects the proportion of unshifted objects. Consider again the examples in \REF{ex:larsson:11}, repeated below. In the baseline, the switch topic pronoun is fronted to Spec-CP, whereas in NAmNo, where SV order is preferred, the object is instead in the unshifted position. This is precisely what is expected from the patterns for object shift in the baseline. 

\ea\label{ex:larsson:25}
\ea 
\gll å   {nei} \textbf{{det}} {trur}   {jeg}   {ikke}   {han}   {var}\\
	oh   no that believe I   not   he   was\\
\glt ‘oh no, I don’t believe that he was’ (emigrant, blair\_WI\_19um; \citealt{LarssonForthcoming}, ex. 14a)\label{ex:larsson:25a}\\
\ex 
\gll nei jeg trur \textbf{ikke} \textbf{det}\\
     no I believe not that\\
\glt ‘no, I don’t think so’ (modern NAmNo, westby\_WI\_03gk; \citealt{LarssonForthcoming}, ex. 15a)\label{ex:larsson:25bs}\\
\z
\z

Summing up, we have seen, first of all, that NAmNo has object shift, and that it follows the same principles as in EurNo. However, compared to present-day urban, Eastern EurNo, the proportion of shifted objects in NAmNo is somewhat lower, i.e., there is more variation. We have observed that this variation can at least in part be traced back to variation that was already present in the baseline. We have also argued that it is to some extent a by-product of the increased preference for subject-initial clauses in NAmNo, and the concomitant decline of object-initial clauses: the objects that are typically fronted to the initial position in the baseline are the same type of objects that remain unshifted when not fronted (notably switch topics).

\section{Objects and verb particles}\label{sec:larsson:5}

In this section, we present a new study of the order between verb particles and objects in NAmNo, using the same data set as in the studies of subject shift and object shift above. \sectref{sec:larsson:5.1} gives an overview of verb particle placement in EurNo, and \sectref{sec:larsson:5.2} is concerned with NAmNo as compared to the baseline. 

\subsection{Objects in verb particle constructions in EurNo}\label{sec:larsson:5.1}

Verb particles (as in \textit{sign} \textbf{\textit{up,}} \textit{give} \textbf{\textit{in}}) have sometimes been analyzed as intransitive prepositions (see, e.g., \citealt{Faarlund2019}: 137). In the typical case, a preposition locates one entity (the Figure) in relation to another (the Ground): In \textit{put the food in the fridge, the food} is Figure, and related to the Ground, \textit{the fridge}. In a verb particle construction, the Ground is often implicit, as in \textit{take the dog out}. While the verb and the particle syntactically behave like two independent words, the combination often has idiomatic, non-compositional meaning; it is then generally less clear what is Figure and Ground (consider for instance \textit{she gave up}, \textit{I found it out}). The particle often, but not always, introduces a result (as in \textit{go out}) or modifies a result introduced by the verb (\textit{open up}). Particles also frequently change the argument structure of the verb (cf. \textit{give somebody something} and \textit{give up}). Verb particles have been much discussed in the literature; see \citet{Åfarli1985}, \citet{Svenonius1996}, \citet{RamchandSvenonius2002}, \citet{Toivonen2003}, \citet{LarssonLundquist2014, LarssonLundquist2021}, \citet{Aa2020} and many others. 

EurNo verb particles have many properties in common with their English counterparts. In EurNo, as in English, transitive particle combinations can be distinguished from combinations of verb + PP through constituency tests: unlike a preposition + complement, the particle and the object do not form a constituent. Consider \REF{ex:larsson:26} below, which shows that a particle + object cannot be topicalized together.

\ea \label{ex:larsson:26}
\ea[*]{\label{ex:larsson:26a}
  \gll Ut hunden slapp jeg.\\
      out dog.\textsc{def} let I\\
  \glt (Intended:) ‘I let the dog out.’}
\ex[]{\label{ex:larsson:26b}
	\gll Hunden slapp jeg ut.\\
    dog.\textsc{def} let I out\\
	\glt ‘I let the dog out.’}
\z
\z

One important characteristic of transitive verb particle constructions in both English and EurNo is word order: an object can either precede or follow the particle; consider the Norwegian examples in \REF{ex:larsson:27} and their English translations, which show the same word order variability.\footnote{Norwegian particle constructions are also characterized by specific prosodic properties. The verb and the particle are often realized with so called compound accent; compound accent is otherwise not found outside words (see \citealt{TengesdalForthcoming} for discussion).} 

\ea\label{ex:larsson:27}
\ea\label{ex:larsson:27a}
\gll Hun slapp hunden ut (i skogen).\\
she let dog.\textsc{def} out in forest.\textsc{def}\\
\glt ‘She let the dog out in the forest.’

\ex\label{ex:larsson:27b}
\gll Hun slapp ut hunden.\\
	she let out dog.\textsc{def}\\
\glt ‘She let out the dog.’
\z
\z

In EurNo, the order particle-object in \REF{ex:larsson:27b} is much preferred when the object is non-pronominal; the order NP-particle in \REF{ex:larsson:27a} is more frequent when the particle is followed by a directional PP (see \citealt{LundquistTengesdal2022}). Pronominal objects and reflexives, on the other hand, generally precede the particle, as in \REF{ex:larsson:28-firsta}; the order particle-pronoun is not accepted by all speakers, as indicated by the \% sign in \REF{ex:larsson:28-firstb}. 

\ea\label{ex:larsson:28-first}
\judgewidth{\%}
\ea[]{\label{ex:larsson:28-firsta} 
\gll Hun tok den med.\\
	 she took it with\\
\glt ‘She brought it with her.’}
\ex[\%]{\label{ex:larsson:28-firstb}
	\gll Hun tok med den.\\
         she took with it\\
	 \glt ‘She brought it with her.’
}
\z
\z

Again, EurNo resembles English, where the order particle-pronoun is generally unacceptable; see \REF{ex:larsson:29-firstb}.\footnote{However, as mentioned by \citet{Toivonen2003}, particle-pronoun order can on occasion be found also in English.} Unlike in EurNo, though, the distribution of non-pronominal objects is more even; in present-day American English, there is only a slight preference for the particle-object order (\citealt{HaddicanEtAl2020}: 215 report a proportion of 53\% particle-object order in corpus data from US Twitter users).

\ea\label{ex:larsson:29-first}
\ea She let the dog/him out.\label{ex:larsson:29-firsta}
\ex She let out the dog/*him.\label{ex:larsson:29-firstb}
\z
\z

The variation in object placement in particle constructions resembles subject shift and object shift, as there is a clear difference between pronominal and non-pronominal arguments. Also, EurNo reflexives precede both particles and negation more consistently than pronouns (\citealt{LundquistTengesdal2022}). However, unlike in particle constructions, the difference between the two types of arguments is categorical in object shift – non-pronominal objects cannot shift. In the case of subject shift, there is variation with non-pronominal subjects, but a preference for the position following negation. Another difference between subject placement and object placement in particle constructions is that the former depends partly on clause type (see \sectref{sec:larsson:3} above). 

There is some dialectal variation in EurNo with regard to the frequencies of the two word orders (see \citealt{LarssonLundquist2014}, \citealt{LundquistTengesdal2022}). All dialects seem to distinguish between pronouns and non-pronouns, but Trøndersk (spoken in central Norway) has a higher frequency of particle-pronoun order than other areas (and also less pronominal subject shift and object shift; see \citealt{OestboeMunch2013}). Outside of Trøndersk, the order particle-pronoun and particle-reflexive is very infrequent in the present-day dialects; in a study based on the Nordic Dialect Corpus \citep{JohannessenEtAl2009}, \citet{LarssonLundquist2014} find a single example from Northern Norwegian, and only a handful from the south. Larsson \& Lundquist also observe that the variation in the placement of non-pronominal objects is restricted; younger speakers in Finnmark (Northern Norway) have a stronger preference for the order particle-NP (94\%) than older speakers (74\%). The younger Finnmark speakers resemble speakers (of all ages) in Buskerud (Eastern Norway), who produce the order particle-NP in around 90\% of the cases.

In the next section, we investigate object placement in particle constructions in NAmNo, as compared to the same baseline that was used for subject and object shift, i.e., older homeland dialect speakers (LIA) and emigrant speakers. Again, we will see that there is some variability already in the baseline. 

\subsection{Objects in particle constructions in NAmNo}\label{sec:larsson:5.2}

To investigate object placement in particle constructions in NAmNo, we have, as mentioned, used the same data set that was used for the study of subject and object placement in \citet{LarssonForthcoming}; cf. \tabref{tab:larsson:1} above for details. Recall, again, that this data set includes four groups of speakers: homeland Norwegian dialect speakers born in Oppland around the time of mass emigration (the LIA corpus), emigrants (1\textsuperscript{st} generation immigrants recorded in the 1930s/1940s), older NAmNo (heritage speakers recorded in the 1930s/1940s) and modern NAmNo speakers with ancestral ties to the county of Oppland (recorded in 2010). 

The same type of verb particle combinations can be found in NAmNo as in EurNo (e.g., \textit{vokse opp} ‘grow up’, \textit{gå ut} ‘go out’, \textit{rente ut} ‘rent out’, \textit{lese opp} ‘read out’). We have searched for the particles \textit{opp} ‘up’, \textit{ned} ‘down’, \textit{ut} ‘out’, and \textit{inn} ‘in’. All hits were manually tagged and irrelevant hits disregarded; we only include transitive verb particle combinations where there is a possibility for word order variation in EurNo. Cases where the object is fronted (topicalized, relativized or questioned) were excluded, as were a couple of unclear cases. Relevant hits were annotated for word order (particle-object or object-particle), type of object (pronoun, reflexive or NP), and the presence of a directional PP. We also included a rough semantic annotation, distinguishing between directional and non-directional particles (cf. also \citealt{LarssonKinn2021}). 

The results are presented in \tabref{tab:larsson:4}. As expected, there is a clear difference between lexical NPs and pronouns – the former are with few exceptions placed after the particle, whereas pronouns tend to precede particles. A couple of examples are given in \REF{ex:larsson:28}.

\ea%28
    \label{ex:larsson:28}
    \ea 
    \gll du renter \textbf{han} \textbf{ut}, eller?\\
         you rent it out or\\
    \glt ‘you rent it out, right?’ (modern NAmNo, blair\_WI\_01gm)\label{ex:larsson:28a}\\ 
    \ex 
    \gll jeg har renta \textbf{ut} \textbf{åtti} \textbf{acre}\\
         I have rented out eighty acres\\
    \glt ‘I have rented out eighty acres’ (modern NAmNo, blair\_WI\_01gm)\label{ex:larsson:28b}\\
    \z % you might need an extra \z if this is the last of several subexamples
\z

\begin{table}[t]
\caption{Word order in transitive verb particle constructions with the particles \textit{opp} ‘up’\textit{, ned} ‘down’\textit{, ut} ‘out’\textit{, inn} ‘in’.}
\label{tab:larsson:4}
\begin{tabular}{l *4{c@{~}>{(}r<{)}} }
\lsptoprule
            & \multicolumn{2}{c}{Particle–} & \multicolumn{2}{c}{Particle–} & \multicolumn{2}{c}{Particle–} & \\
            & \multicolumn{2}{c}{pronoun}   & \multicolumn{2}{c}{reflexive} & \multicolumn{2}{c}{NP}        & \multicolumn{2}{c}{Total}\\\midrule
LIA         & 7/40 & 17.5\%    & 0/11 & 0\%         & 75/78 & 96.2\% & 82/129 & 63.6\%\\
Emigrant No & 2/20 & 10.0\%    & 0/6  & 0\%         & 23/26 & 88.5\% & 25/52 & 48.1\%\\
Older NAmNo  & 3/33 & 9.1\%    & 0/8  & 0\%         & 52/57 & 91.2\% & 55/98 & 56.1\%\\
Modern NAmNo & 5/42 & 11.9\%   & 0/2  & 0\%         & 40/41 & 97.6\% & 45/85 & 52.9\%\\
\lspbottomrule
\end{tabular}
\end{table}

The number of reflexives is small overall, but reflexives always precede the particle in our data; this is expected from what we know about reflexive placement in present-day EurNo. Examples with reflexives from the baseline and older NAmNo are given in \REF{ex:larsson:29}. 

\ea%29
    \label{ex:larsson:29}
    \ea  
    \gll før enn han begynner i heile tatt å bikke \textbf{seg} \textbf{ned}\\
         before that he begins in {the whole} taken to tip \textsc{refl} down \\
    \glt ‘before he starts to tip down at all’ (LIA, sel\_uio\_0301)\label{ex:larsson:29a}\\
    \ex  
    \gll de kledde \textbf{seg} \textbf{ut}\\
         they dressed \textsc{refl} out\\
    \glt ‘they were dressing up in costumes’ (older NAmNo, springdale\_WI\_01gm)\label{ex:larsson:29b}\\
    \z % you might need an extra \z if this is the last of several subexamples
\z

Furthermore, all speaker groups prefer the order object-particle with pronominal objects and the order particle-object with non-pronominal objects. The word order preference is clearly independent of the semantics of the verb particle combination: pronouns generally precede both directional and non-directional particles, as in \REF{ex:larsson:30}, and non-pronominal objects follow both types of particles, as in \REF{ex:larsson:31}.

\ea%30
    \label{ex:larsson:30}
    \ea
    \gll {henge} \textbf{den} \textbf{opp} og plukke av løva\\
            hang it up and pick off leaves.\textsc{def}\\
    \glt    ‘hang it up and pick off the leaves’  (Modern NAmNo, coon\_valley\_WI\_02gm)\label{ex:larsson:30a}\\
    \ex 
    \gll ja en kan ikke figure \textbf{det} \textbf{ut}\\
         yes one can not figure it out \\
    \glt ‘yes, one can’t figure it out’ (Modern NAmNo, blair\_WI\_01gm)\label{ex:larsson:30b}\\
    \z % you might need an extra \z if this is the last of several subexamples
\ex%31
    \label{ex:larsson:31}
    \ea  
    \gll kråka eter \textbf{opp} \textbf{alt} \textbf{sammen}\\
         crow.\textsc{def} eats up all together\\
    \glt ‘a crow will eat up everything’ (Modern NAmNo, coon\_valley\_WI\_02gm\label{ex:larsson:31a})\\
    \ex  
    \gll så djup at en kunne grave \textbf{{ned}} \# \textbf{en} \textbf{hest} \textbf{eller} \textbf{ei} \textbf{ku}\\
         so deep that one could bury down {} a horse or a cow\\
    \glt ‘so deep that you could bury a horse or a cow’ (Modern NAmNo, coon\_valley\_WI\_03gm)\label{ex:larsson:31b}\\ 
    \z % you might need an extra \z if this is the last of several subexamples
\z

However, there is also some variability with both pronouns and NPs. Pronouns on occasion follow particles, as in the examples in \REF{ex:larsson:32}.

\ea%32
    \label{ex:larsson:32}
    \ea 
    \gll så \# tok eg utor den kjuka og sila mysa ned i att og \# og kokte \textbf{opp} \textbf{henne} og så slo ned fløyten\\
         then {} took I out that curd and filtered  whey.\textsc{def} down in again and {} and boiled up it.\textsc{fem} and then poured down cream.\textsc{def}\\
    \glt ‘then I took the curd out and filtered the whey and poured it back down and boiled it poured down the cream’ (LIA, lesja\_ma\_01)\label{ex:larsson:32a}\\
    \ex 
    \gll du plukka \textbf{opp} \textbf{det} fra dem\\
         you picked up it from them\\
	\glt ‘you picked it up from them’ (Modern NAmNo, coon\_valley\_WI\_06gm)\label{ex:larsson:32b}\\
    \z % you might need an extra \z if this is the last of several subexamples
\z

Interestingly, the incidence of particle–pronoun order is higher in LIA than in the other three samples, and considerably higher than expected from studies of present-day EurNo dialects (cf. above). However, rather than representing change (towards less word order variability), the frequency differences might reflect intra\hyp individual variation. Four different speakers produce the order particle\hyp pronoun in LIA. Three of these speakers also produce the order pronoun-particle; the fourth has only one example with a particle and a pronominal object. The examples from modern NAmNo are produced by two different speakers; both speakers produce both word orders. Notably, there is no statistically significant difference between the four speaker groups with respect to the ordering of pronouns and particles.\footnote{Like in \citet{LarssonForthcoming}, significance was tested using the function \texttt{prop.test()} in R  \citep{rcoreteam}.}

In all four speaker groups, there is variability also in the placement of non-pronominal objects, and there are cases where an NP precedes the particle, as in \REF{ex:larsson:33}.\footnote{\textit{Att} (as in (\ref{ex:larsson:33a})) is an element that occurs in some of our examples. We analyze it as an adverb; its meaning can be paraphrased as `again’ or `back’ (\textit{Nynorskordboka}, \citealt{Nynorskordboka}). We treat \textit{att} as an independent word, consistently with the LIA transcriptions. An alternative analysis whereby \textit{att} forms a compound with certain particles (e.g., \textit{oppatt}) is mentioned in \textit{Norsk ordbok} \parencite{Groenvik2009}. Importantly for our purposes, the presence of \textit{att} in \REF{ex:larsson:33a} cannot fully explain the word order NP-particle. The order NP-particle is attested without \textit{att} (cf. \REF{ex:larsson:33b}); moreover, \textit{att} occurs with various word orders in LIA in addition to NP-particle-\textit{att} (as in \REF{ex:larsson:33a}): We have found particle–\textit{att–}NP (\textit{bar ned att NP} ’carried down again NP’) and particle–NP–\textit{att (bygge ut NP att} ’build out NP again’).}

\ea%33
    \label{ex:larsson:33}
    \ea  
    \gll ja det var å gå og bere \textbf{ei} \textbf{svær} \textbf{bør} \textbf{opp} {att}\\
         yes it was to go and carry a big load opp again\\
    \glt ‘Yes, one would have to carry a big load back up again’ (LIA, brandbu\_ma\_01)\label{ex:larsson:33a}\\
    \ex  
    \gll får \# \textbf{plantene} \textbf{opp} {til å} plante ut på fielda \\
         gets \# plants.\textsc{def} up to plant out on field.\textsc{def}\\
	\glt ‘gets the plants up so they can be planted out on the field’ (Modern NAmNo, coon\_valley\_WI02gm)\label{ex:larsson:33b}\\
    \z % you might need an extra \z if this is the last of several subexamples
\z

There is perhaps more variability with directional particles than with non-directional verb-particle combinations; there is only one example of a non-pronominal object preceding a clearly non-directional particle in our sample; see \REF{ex:larsson:34}.

\ea%34
    \label{ex:larsson:34}
  
    \gll og de fant \textbf{dette} \textbf{ut} på omtrent en to vinter etter at…\\
          and they found this out on approximately on to winter after that \\
    \glt ‘and they found this out in about a winter or two after …’   (Older NAmNo blair\_WI\_16um)\\
\z % you might need an extra \z if this is the last of several subexamples
        

However, it is even more striking how limited the variation is in all speaker groups (even considering directional particles). Recall that \citet{LarssonLundquist2014} found that present-day dialect speakers produced around 90\% particle-NP order, and that older speakers in some areas had more variation. Our samples suggest that the strong preference for particle-NP order is not a recent phenomenon; LIA has 96.2\% particle-NP order. Again, there is no statistically significant difference between the speaker groups – the restricted variability seems to be retained over time in NAmNo. 

The investigation of argument placement in particle constructions shows stable variability both with respect to pronouns and NPs. As was the case with subject shift and object shift, there is more variability with respect to pronoun placement relative to particles in these corpus samples than in most present-day EurNo varieties. In the case of NP placement the variation is, on the other hand, more limited in our study, if anything. However, compared to a baseline of older dialect speakers and emigrants, there is no evidence of word order change in NAmNo. As mentioned, English has a strict order pronoun-particle (\textit{take it out,} not \textit{*take out it}) but a more even distribution in non-pronominal objects (cf. \textit{take the garbage out, take out the garbage}). If the word order in NAmNo had been affected by CLI from English, we would have expected less variability in the placement of pronouns and rather more variation in the placement of non-pronominal objects.

\section{Theoretical relevance}\label{sec:larsson:6}

The investigations of argument placement in NAmNo as compared to baseline homeland Eastern Norwegian dialects and emigrant Norwegian reveal systematic variation, which is largely stable across generations in the heritage variety. One important observation is that there is variation already in the homeland (even when we only consider speakers from Oppland), both with respect to the ordering of subject pronouns and negation, and concerning the order of objects and negation/particle. In all three environments, we can observe more variation in the old dialect recordings from Oppland than in present-day (urban) Eastern Norwegian. Thus, the study of argument placement highlights the importance of establishing an appropriate baseline for comparison with the heritage variety (see, e.g., \citealt{Polinsky2018}: 33). 

Despite variation in the baseline, we have observed that the general patterns of subject shift, object shift, and particle placement remain remarkably stable across time. There are some shifts in frequencies, which we largely analyze as a byproduct of the stronger preference for SV order in NAmNo, and to some extent perhaps also a reflection of inter- and intra-individual variation. We see some effects of cross-linguistic influence, but this is limited to particular contexts and does not seem to affect the underlying syntax of NAmNo (see further below in \sectref{sec:larsson:6.2}). 

In the remainder of this section, we discuss stability, variability, and change, and reflect on some of the theoretical consequences of our findings. \sectref{sec:larsson:6.1} is concerned with stability and variability, and \sectref{sec:larsson:6.2} discusses factors that might lead to change in the heritage language. 

\subsection{Stability and variability}\label{sec:larsson:6.1}

In the discussion of subject shift and object shift above, we made the case that the postverbal subject and object positions (shifted and unshifted) should not be discussed in isolation; the clause-initial position must also be considered (as pointed out for homeland Scandinavian by \citealt{Andréasson2010}). We argued that the growing preference for SV order in NAmNo affects the proportion of postverbal subjects in the \textit{shifted} position (which decreases, while unshifted subjects are largely unaffected), as well as the proportion of \textit{unshifted} object pronouns (which increases, while shifted pronouns remain stable). The data from NAmNo in this respect provides additional information about the pragmatic functions of the different argument positions; for instance, it is clear that switch topic object pronouns are either clause-initial or unshifted both in EurNo and NAmNo (cf. \citealt{BentzenAnderssen2019}). Moreover, the initial position can be used for either topic-chaining or focus-chaining. The difference between the two pragmatic strategies does not appear to be syntactically encoded, either in EurNo or NAmNo (see \citealt{LindahlEngdahl2022} for further discussion). 

There are some categorical patterns for argument placement in both EurNo and NAmNo; in particular, non-pronominal objects cannot shift across negation, and reflexives shift across both negation and particles. However, the effects of factors like information structure and prosody are generally non-categorical in both the baseline and the heritage variety. There are, for instance, examples of the order negation-pronoun and particle-pronoun in all investigated speaker groups. The variation is retained across generations in the heritage context. This is particularly remarkable considering that we are dealing with low-frequency phenomena such as object shift. It seems clear, then, that the variable patterns can be learnt and maintained even with the restricted input and use in the heritage language context. 

Previous work on heritage languages has shown that core syntax is generally stable (see, e.g., \citealt{LohndalEtAl2019} and references cited there). Prosody has not been investigated to the same extent; however, given that it is a domain to which children are sensitive at a very early stage of L1 acquisition (\citealt{deCarvalhoEtAl2018}) it seems likely that general prosodic patterns are also among the more stable domains (though \citealt{Lleó2018} points out cases of interaction between the different languages in simultaneously bilingual L1 aquisition; see also \citealt{Polinsky2018}: 147ff. for some discussion).

For object shift, it seems quite clear that syntax (as we generally understand it) is involved – there is a categorical difference between pronominal and non-pronominal objects, and there is also a clear difference between objects and subjects (see \citealt{LyskawaEtAl2022} for further arguments). Prosody can also to some degree be involved, since only weak pronouns shift. However, the patterns of variation between shifted and non-shifted pronominal objects cannot, as far as we can see, be understood fully in terms of syntactic or prosodic principles. For subject shift and particle placement, it is even less clear that syntax determines word order (but see \citealt{Westergaard2011} and references therein for suggestions along those lines); the variation we observe is hardly syntactically determined. Precisely what the structural analysis should be is still an open question (and the clausal structure presented in \citetv{chapters/introduction} cannot account for the variation in argument placement). 

As we have seen in previous sections, argument placement correlates with information structure (although the patterns are not categorical): subject placement depends on whether the pronoun is topic or focus, and object placement depends on whether the pronoun is a switch topic or a continued topic. It has sometimes been argued that phenomena at the interface between two linguistic domains (e.g., syntax and information structure) are particularly susceptible to change in bilingual settings; this has been referred to as the Interface Hypothesis (e.g., \citealt{SoraceFiliaci2006, Sorace2011}). From the point of view of the Interface Hypothesis, the stability observed here is even more striking. We return to the Interface Hypothesis in the next section, where we consider the growing preference for SV order in NAmNo.

Previous studies of verb particles in heritage languages have suggested that these constructions can be vulnerable (see \citealt{Polinsky2018}: 52ff and references therein). Idiosyncratic properties of verb-particle combinations might be affected in the heritage language contexts. It has also been observed that the order particle-pronoun can appear in the heritage language when it is unacceptable in the baseline; this has been observed for instance in heritage English. \citet[56]{Polinsky2018} suggests that the particle is analyzed as an adverb by the heritage speaker, leading to more flexibility in word order. \citet{LarssonLundquist2021} suggest that particles in older Swedish and present-day Norwegian are light phrasal modifiers which can branch either to the right or to the left; that pronouns tend to precede particles can be analyzed along the same lines as object shift. The stable (though not categorical) preference for particle-NP order in EurNo and NAmNo should then be considered in connection to the preferences for linearization of modifiers in Norwegian more generally. If the account proposed by \citet{LarssonLundquist2021} is on the right track, particles are adverbs even in the baseline, and the word order flexibility is maintained in the heritage language.

\subsection{Change: CLI and processing}\label{sec:larsson:6.2}

As we have argued, argument placement in NAmNo is, overall, characterized by remarkable stability. There are, however, some examples of change, which we turn to in this section.

First, there is a growing preference for SV order (subject-initial clauses) in NAmNo (compare \citetv{chapters/anderssen}), which, on our analysis, has consequences for the placement of arguments further down in the clause. It should be noted that a preference for SVO has been observed in a number of heritage languages (see e.g. \citealt{Laleko2021} and references therein), so this development is not unique. As discussed by \citet{LarssonForthcoming}, there are two possible (and possibly interacting) explanations for the preference of SV in NAmNo. The first is processing: Heritage speakers are known to prefer structures that are easier to process (e.g., \citealt{Polinsky2018}: 36), and this would plausibly favor SV order over XVS order (see, e.g., \citealt{BickelEtAl2015, Hörberg2016}, and references therein). In syntactic terms, ease of processing can be related to the length, and nature, of dependencies within the clause: fronting of subjects to Spec-CP involves shorter syntactic dependencies than fronting of, e.g., objects, and heritage speakers are known to prefer shorter dependencies (e.g. \citealt{BenmamounEtAl2013, HoppEtAl2019}). Relatedly, fronting to Spec-CP is A´- movement, a type of movement that often displays restrictions, typically favoring the highest structural constituent (i.e., the subject) in heritage languages (see e.g. \citealt{Polinsky2018}: 241ff). Another possibility is cross-linguistic influence from English; as English has SV in the vast majority of declarative clauses, CLI would also lead to more SV in NAmNo (see \citealt{WestergaardEtAl2021}). With English as the dominant language, it is hard to evaluate the two options (see also discussion in \citealt{Polinsky2018}: 273ff). However, an exploratory study by \citet{Melvær2023} suggests that the preference for SV increases also in a different heritage language context where Norwegian is involved: Latin American Norwegian, where Spanish is the dominant language. SV order is not as frequent in Spanish as in English (e.g., \citealt{Zagona2002, Arús2010, Lavid2010}), and CLI is therefore perhaps a less likely explanation. This invites us to consider factors other than (or interacting with) CLI. 

Regardless of the explanation of the increase in SV order in NAmNo, it should be noted that there is no reason to assume that the frequency change corresponds to a change in underlying clausal structure. Underlying syntactic change would be unexpected in the context of previous studies indicating that syntax is a stable domain in heritage languages (see \sectref{sec:larsson:6.1} above); also, it would falsely predict additional effects which we do not see, e.g., a higher amount of V2 violations. The increase of SV is better explained as a change in the speakers’ choice between different options already available in the baseline grammar. This choice lies at the interface between syntax and pragmatics, and given the Interface Hypothesis (see above), it is perhaps not surprising that we see (frequency) change in the heritage language. However, we noted above that the information-structural patterns in subject shift and object shift remain stable, and it might seem surprising that we do not see any clear effects here. Plausibly, this is related to the fact that fronting to Spec-CP is different from subject shift and object shift in several respects. First, it has a double pragmatic function: it connects the utterance to the previous discourse (through focus-chaining or topic-chaining) and it also (often) introduces the aboutness topic of the utterance (e.g., \citealt{LindahlEngdahl2022} and references therein). Moreover, the choice between focus-chaining and topic-chaining is not only a question of individual preferences on a general level – it depends on individual choices in particular utterances in specific contexts. The placement of postverbal subjects and objects is considerably less free: there is a strong preference to shift topical subjects and leave switch topic objects unshifted, across utterances and individual speakers. The fact that the choice in fronting to Spec-CP is pragmatically complex but free A´-movement, and the fact that fronting of non-subjects leads to longer syntactic dependencies, might conspire to make the fronting patterns to Spec-CP more susceptible to change. Importantly, the frequency of the different patterns seems to have very little to say; fronting of objects to Spec-CP is for instance considerably more frequent in the baseline, than the contexts for object shift.  

The second change we have noted is that tag questions are introduced as a common discourse strategy in present-day NAmNo, most likely due to influence from English (see also \citealt{AnderssenWestergaard2020}). As with SV order, the introduction of tags does not necessarily involve change the underlying structure. Tag questions are not found in our baseline data, but questions with the order negation-subject do occur in EurNo (\textit{kan ikke du…?} ‘can’t you…?’; see, e.g., \citealt{UrbanikSvennevig2019}). In NAmNo, but not in our EurNo baseline data, tags are used with a specific discourse function, which seems to correspond to the English function: they make the addressee participate in the interaction or have a confirmatory or attitudinal function (\citealt{TottieHoffmann2006}, see also \citealt{LarssonForthcoming}). Thus, CLI from English affects the discourse function, but not the underlying syntax. Discourse markers have been shown to be sensitive to CLI (see \citealt{MoquinSalmons2020} and \citealt{SøftelandHjelde2021} on NAmNo), and this seems to be the case also for tags. However, as noted by Larsson \& Kinn, the effect seems to be restricted to modern NAmNo; there are no examples of tags in the older NAmNo data. 

The studies of argument placement overall show limited effects of CLI. For instance, we do not see a growing acceptance of the order non-pronominal object-particle (as in English \textit{throw the garbage out}), or a lower tolerance for particle-pronoun order (English *\textit{throw out it}), as would be expected if CLI from English was involved. 

Importantly, as mentioned above, none of the changes observed here seem to affect the underlying syntax of NAmNo. Instead, we see evidence of variable patterns that are maintained across generations. The effects of CLI are very limited, and the increase in SV order may plausably be a consequence of a more general pressure to ease processing.

We take the position that any theory of cross\hyp linguistic influence must be restrictive enough, as not to over-generate. The stability observed in argument placement lends little support to direct syntactic influence from English. Beyond argument placement we also do not see clear evidence of syntactic borrowing, where structures from English (which are not identical to structures already present in the baseline), are incorporated into the grammar of NAmNo.\footnote{\citet{Riksem2017} and \textcitetv{chapters/riksem} discuss the use of English functional items in NAmNo, in particular the plural marker \textit{{}-s,} which could possibly be interpreted as syntactic borrowing. However, the plural \textit{{}-s} seems to be largely restricted to language mixing contexts (combined with an English noun); with Norwegian noun stems, it only occurs sporadically (\citetv{chapters/riksem} report to have found this in ``a handful of examples” in CANS).} 

\section{Concluding remarks}\label{sec:larsson:7}

This chapter has discussed argument placement in NAmNo as compared to a baseline of old EurNo dialect recordings and recordings of the first-generation emigrants. We have built on previous studies of subject placement and object placement relative to negation/sentence adverbs (i.e., subject shift and object shift), and we also presented a new study of object placement relative to verb particles. The investigations of subject shift and object shift included three positions: one clause-initial and two postverbal. We have argued that including all three positions is important: the shifting preferences for the initial position have knock-on effects on the postverbal positions, but in different ways for subjects and objects. Fronting of subjects bleeds the shifted subject position and also leads to more unshifted object pronouns. The investigations also revealed that there is more variability in pronoun placement in the baseline than might be expected from studies of present-day (urban) Norwegian; this applies to both subject shift, object shift and particle placement. The variability in the placement of non-pronominal objects relative to particles is, on the other hand, quite restricted in all speaker groups. Overall, the variation remains remarkably stable across generations, and the general patterns of argument placement remain the same. This is an important result, both with respect to our understanding of heritage languages (considering e.g. the limited effects of cross-linguistic influence), and for the analysis of subject shift, object shift and particle placement. 

\section*{Abbreviations}
\begin{tabbing} 
MMMM   \= Corpus\kill
CANS   \> Corpus of American Nordic Speech\\
CLI    \> cross-linguistic influence\\
\textsc{def}    \> definite\\
\textsc{fem}    \> feminine \\
EurNo  \> European (=homeland) Norwegian\\
LIA    \> Language Infrastructure made Accessible\\
\textsc{mod}    \> modal\\
NAmNo  \> North American Norwegian\\
\textsc{pl}     \> plural  
\end{tabbing}

\section*{Acknowledgements}

The research reported in this chapter was supported by the Research Council of Norway, projects 301114 and 250755.  We would like to thank two reviewers and Mike Putnam for their helpful comments, and Inger Frantzen for checking the references in our manuscript. Finally, we are grateful to Janne Bondi Johannessen, whose memory remains an inspiration. Any remaining errors are our own. 

\printbibliography[heading=subbibliography]
\end{document}
