\documentclass[output=paper]{langscibook}
\ChapterDOI{10.5281/zenodo.15274568}
\author{Kristin Melum Eide\orcid{}\affiliation{NTNU – Norwegian University of Science and Technology \& University of Bergen}}
\title{Tense, modality, and aspect in North American Norwegian}
\abstract{This chapter offers an overview over some patterns of TMA-markers (tense, modality, aspect) attested in a corpus of North American Norwegian (NAmNo); a resource spanning 80 years of recordings. The inventories of markers and the contextual restrictions on their use resemble those of the European Norwegian dialects and variants serving here as the primary baseline for comparison (EurNo). However, there are also interesting differences. First, new markers are borrowed from American English (AmE); second, certain constructions and patterns occur with changed frequency over time; third, the inventory of productive distinctions and oppositions tends to reduce for several phenomena. Overall, the more recent TMA-systems are inclined towards simplification as compared with the early recordings, but simultaneously displaying an interesting mix of innovations (e.g. borrowings) and archaisms (old dialectal traits) so often observed in Heritage Language research.}
\IfFileExists{../localcommands.tex}{
  \addbibresource{../localbibliography.bib}
  \usepackage{langsci-optional}
\usepackage{langsci-gb4e}
\usepackage{langsci-lgr}

\usepackage{listings}
\lstset{basicstyle=\ttfamily,tabsize=2,breaklines=true}

%added by author
% \usepackage{tipa}
\usepackage{multirow}
\graphicspath{{figures/}}
\usepackage{langsci-branding}

  
\newcommand{\sent}{\enumsentence}
\newcommand{\sents}{\eenumsentence}
\let\citeasnoun\citet

\renewcommand{\lsCoverTitleFont}[1]{\sffamily\addfontfeatures{Scale=MatchUppercase}\fontsize{44pt}{16mm}\selectfont #1}
   
  %% hyphenation points for line breaks
%% Normally, automatic hyphenation in LaTeX is very good
%% If a word is mis-hyphenated, add it to this file
%%
%% add information to TeX file before \begin{document} with:
%% %% hyphenation points for line breaks
%% Normally, automatic hyphenation in LaTeX is very good
%% If a word is mis-hyphenated, add it to this file
%%
%% add information to TeX file before \begin{document} with:
%% %% hyphenation points for line breaks
%% Normally, automatic hyphenation in LaTeX is very good
%% If a word is mis-hyphenated, add it to this file
%%
%% add information to TeX file before \begin{document} with:
%% \include{localhyphenation}
\hyphenation{
affri-ca-te
affri-ca-tes
an-no-tated
com-ple-ments
com-po-si-tio-na-li-ty
non-com-po-si-tio-na-li-ty
Gon-zá-lez
out-side
Ri-chárd
se-man-tics
STREU-SLE
Tie-de-mann
}
\hyphenation{
affri-ca-te
affri-ca-tes
an-no-tated
com-ple-ments
com-po-si-tio-na-li-ty
non-com-po-si-tio-na-li-ty
Gon-zá-lez
out-side
Ri-chárd
se-man-tics
STREU-SLE
Tie-de-mann
}
\hyphenation{
affri-ca-te
affri-ca-tes
an-no-tated
com-ple-ments
com-po-si-tio-na-li-ty
non-com-po-si-tio-na-li-ty
Gon-zá-lez
out-side
Ri-chárd
se-man-tics
STREU-SLE
Tie-de-mann
} 
  \togglepaper[1]%%chapternumber
}{}

\begin{document}
\maketitle 
%\shorttitlerunninghead{}%%use this for an abridged title in the page headers

\section{Introduction: Tense, modality and aspect in heritage language research}\label{sec:eide:1}

Einar Haugen’s studies of speakers of Heritage Norwegian in the American Midwest (North American~Norwegian; NAmNo) are often considered one of the main starting points of language contact research; hence any endeavor to conduct investigations on his home turf, including Haugen’s own recordings, fills one with awe. On the other hand, the topic of the present chapter, TMA-markers – markers of tense, modality, and aspect – started to attract substantial interest within language contact research only recently, creating a strong imperative to undertake such investigations even for NAmNo. TMA-markers typically appear as verbal suffixes or function words like particles and auxiliaries, and the traditional view on the borrowability of such markers was as expressed by Haugen himself (\citeyear[67]{Haugen1956}); “function words, which only occur as parts of utterances, are seldom borrowed”.

However, as is often the case, reports were lacking because thorough investigations had not yet taken place, and later studies, like Matras and Sakel’s study of  more than 30 language contact situations, demonstrate that exponents of TMA, like other function words, are “borrowed easily and relatively early on in contact situations”; (\citealt[24]{Sakel2007}; cf. also \citealt{Oestman1981}, \citealt{Salmons1990}, \citealt{BoasWeilbacher2007}, \citealt{Matras2009}, \citealt{Matras2011}, a.o.). Still, it seems that function words are less prone to be borrowed and affected by language contact than content words, and among the content words, verbs (typical exponents of TMA, or hosting such exponents) are less affected by contact\footnote{“Affected by contact” may imply a range of different features, i.e. borrowing of an element or the emergence of a different, usually reduced or in some sense simplified form or system as compared to the \textit{baseline}, i.e. our set of data selected for comparison. Affectedness typically involves omission of distinctions, cf. \sectref{sec:eide:2}.} than nouns and adjectives, according to \citegen{TadmorHaspelmathTaylor2010} study on 41 sample languages. 

Within the formalist framework adopted here “[t]ense, aspect, and modality are grammatical categories that occur as functional heads in clause structure. They are traditionally grouped together by virtue of their semantic cohesion and their frequent morphological clustering or fusion” \citep[746]{Zagona2013}. These functional heads tend to appear in a certain relative order, where mood scopes over tense, in turn scoping over aspect. As free\hyp standing preverbal auxiliaries or particles, TMA-markers often appear in this order (MTA), while the order is reversed for bound markers due to \textit{The Mirror Principle} \citep{Baker1985}. The generalization across free-standing and bound morphemes can be expressed as “relative closeness to the verbal stem”. This generalization, expressed hierarchically as in \REF{ex:eide:1}, is attested across approaches and frameworks, e.g. \citegen{Bybee1985} “principle of relevance”; \citegen{VanValinLaPolla1977} “principle of scope assignment”, and others, cf. \citet{Cinque2014}. 

\ea\label{ex:eide:1} Mood/modality > Tense > Aspect\z

Formalist approaches to HL-linguistics are still a young field of linguistic research (cf. \citetv{chapters/introduction}), and many empirical and theoretical discoveries have yet to be made. With certain exceptions (e.g. \citealt{EideHjelde2012, EideHjelde2015Borrowing, EideHjelde2015Verb, EideHjelde2018, Eide2019, EikLaanemets2021} and \citealt{Lykke2020,Lykke2022}), the grammar of NAmNo TMA-markers specifically has undergone modest systematic scrutiny thus far. Still, the recent upsurge of HL-research provides an empirical fundament for drawing generalizations at a more comfortable level of certainty. Findings from studies on TMA in other HLs (e.g.  \citealt{Montrul2009, Montrul2016, RodriguezBrandl2019, CorbetDominguez2020, Laleko2008, Laleko2010}) provide a point of departure for our study and our expectations on certain trends to emerge in the TMA-systems of NAmNo. 

First, as is attested across all languages and frameworks, TMA-systems interact in intricate ways, with markers sharing domains (e.g. possible worlds) and grammatical tasks (e.g. expressing counterfactuality). A TMA-marker may for instance serve to express either tense or modality, and future tense may be expressed by a modal (like English \textit{will}) while past tenses are expressed morphologically (-\textit{ed}). TMA-markers are restricted at the syntax-pragmatic interface by complex pragmatic restrictions (discourse features) for their felicitous use, and such phenomena are especially prone to cross-linguistic influence (CLI) (\citealt{CasalicchioMoroni2023, Sorace2004}, \citealt{TsimpliSorace2006}, \citealt{SoraceSerratrice2009}, \citealt{Sorace2011}).  

CLI is expected whenever two languages are in contact within a society, i.e. in the minds of language users. Even genetically unrelated languages come to share grammatical features after prolonged contact (\textit{Sprachbunds})\footnote{\citet[2]{Aikhenvald2006}, “If two or more languages are in contact, with speakers of one language having some knowledge of the other, they come to borrow, or copy […] linguistic features and forms of all kinds”.} and as NAmNo and AmE have been in close contact for two centuries, we expect to find linguistic change induced by language contact in the NAmNo TMA-grammar. CLI comes in different guises; besides borrowing, mentioned above, convergence may occur, yielding hybrid systems with features from both languages, and different types of affectedness. Affectedness is often described as “simplification”,\footnote{Especially in contexts of (short-term) extensive adult second language acquisition \citep{Trudgill2011}, due to “the lousy language-learning abilities of the human adult” \citep[372]{Trudgill2009}, compared to the amazing language-learning ability of the human child. Added complexity is a rare result of language contact, but exceptions exist; cf. \citet{Thomason2008} on \textit{Michif,} “the language of the children of French trappers and their Indian wives”, \citep[16]{Bossong2009}. The verbs of Michif are Cree morphosyntactically, the morphosyntax of the noun is French, resulting in an “intertwined” language \citep{ParkvallJacobs2023}, more complex than either of the parent languages.}  a notion less straightforward than it seems,\footnote{\citet{Trudgill2011}, pace \citet{Mühlhäusler1977}: Simplification includes 1) regularization of irregularities; 2) increase in lexical\slash morphological transparency; 3) redundancy reduction (agreement); 4) loss of morphological categories.}  but we will assume here that simplification involves omission of distinctions on one level or another. Based on these observations in other language contact scenarios, our second expectation is that NAmNo TMA-systems will show various CLI-effects, as borrowing, convergence, and simplification. 

TMA-markers are typically expressed via verbal suffixes, i.e. morphology, or free-standing particles and auxiliaries, i.e. syntactically. Earlier studies (e.g. \citealt{BenmamounEtAl2013, ScontrasEtAl2015, Lohndal2021, PutnamEtAl2021}, a.o.) suggest that morphology is more easily affected by language contact than (most) syntactic phenomena, as “inflectional morphology is the linguistic domain most noticeably affected in heritage language grammars”  \citep[54]{Montrul2016}; including loss of morphological oppositions and categories. Hence, we expect NAmNo TMA-morphological markers to be more affected than syntactic TMA-markers, i.e. verbal suffixes to be more easily affected than e.g. auxiliaries. 

Syntactic and morphological exponents of tense, modality, or aspect often occur simultaneously, and TMA-marking can be quite complex. Tenses can be simple (\textit{went}) or complex (\textit{has gone}), the bounded lexical aspect of \textit{eat the lefse} can be cancelled by adding an unbounded progressive grammatical aspect, as in \textit{She was eating the lefse}. Likewise, in EurNo dialects subjunctive mood is still expressed as inflection on the verb, deontic and dynamic modalities are expressed via modal auxiliaries.\footnote{\citet{Roberts1985, Roberts1993}, and \citet{Åfarli1996}, suggest that erosion of mood in Old English and Old Norse is due to increased use of modals (cf. also \citealt{vanKemenadeEtAl2023}). However, modals do not entirely replace mood inflections in Old Norse; instead these categories coexist and interact. Modals rather tend to host and preserve mood markings, cf. EurNo data in \citet[3]{EideDalenJenstad2017}. Contracted English modals can also be seen as emerging mood markers: \textit{coulda, shoulda, woulda, supposta} (the \textit{a} in \textit{supposta} goes back on \textit{to}, not \textit{have}).} In Norwegian dialects these modals may appear with subjunctive inflections, cf. (\ref{ex:eide:2}a--b) where the modals express ability and obligation. In (2cd) these modalities are counterfactual\slash hypothetical; cf. \citet{EideDalenJenstad2017}.

\ea Norwegian dialect Fosen\label{ex:eide:2}
\ea \gll Æ konnj hjølp dæm.\\
    I could.\textsc{pret} help.\textsc{inf} them\\
\glt ‘I was able to help them.’

\ex \gll Æ mått hjølp dæm.\\
		 I must.\textsc{pret} help.\textsc{inf} them\\
\glt ‘I had to help them.’

\ex \gll Æ konnja hjørt dæm.\\
		 I could.\textsc{subj} help.\textsc{part} them\\
\glt ‘I could’ve helped them.’

\ex \gll Æ måtta hjørt dæm.\\
		 I must.\textsc{subj} help.\textsc{part} them\\
\glt     ‘I would have to have helped them.’
\z
\z

HL-speakers tend to replace inflectional morphology with analytic forms (\citealt{Polinsky2018}: 183, \citealt{PutnamEtAl2021}: 618), favoring the introduction and preservation of function words, like auxiliaries, and reducing morphological exponents, e.g. of mood (imperatives, subjunctives); see \citet{Polinsky2018, BenmamounEtAl2013, ScontrasEtAl2015}. This is often discussed in the context of language users’ preference for transparency (e.g \citealt{Polinsky2018, PutnamEtAl2021}). We add this as a fourth point emerging from our investigations: NAmNo TMA-systems include analytical and morphological forms, both types may occur simultaneously, but transparent systems where form and meaning mappings are one-to-one will gain momentum. 

Finally, degrees of borrowability and affectedness are also attested \textit{within} systems of TMA, in that some syntactic-semantic categories of TMA markers are more easily borrowed or otherwise affected by language contact. \citet[71]{Montrul2016} summarizes a range of relevant studies on HL-grammars in a hierarchy given here as \REF{ex:eide:2}; a hierarchy also supported by other large language contact studies such as \citet{MatrasSakel2007}, \citet[45]{Matras2007}. 

\ea\label{ex:eide:3} Mood/modality > Aspect > Tense\z

The hierarchy claims that mood and modality markers are more prone to be borrowed and affected than markers of aspect, which in turn are more easily affected and borrowed than tense markers. These trends cut across and may level out the effects of the morphology-syntax axis, such that tense – though it is expressed primarily via verbal suffixes in NAmNo – is expected to be less affected by language contact than modality, often appearing as auxiliaries, i.e. syntactic exponents. Interestingly, the hierarchy in \REF{ex:eide:3} does not mirror the one given in \REF{ex:eide:1}, and both the hypothesized universal ordering of TMA-markers in \REF{ex:eide:1} and the affectedness hierarchy in \REF{ex:eide:3} face many exceptions and irregularities suggesting that the hierarchies amount to trends more than absolute universals. Adding to the potential confusion, some authors use the term \textit{mood} where others use \textit{modality}; the latter a broader term subsuming morphological mood and analytical exponents like modal auxiliaries \citep[78]{Cinque1999}. Both hierarchies have been modified and refined to distinguish between different types of modalities (e.g. speech act modality, subjunctive vs. indicative moods, epistemic vs. deontic modality), and between different types of tenses (e.g. future tense vs. other tenses). Aspects likewise can be split into several types, e.g. lexical vs. grammatical aspect, occupying different slots in the hierarchies. 

Summing up our first expectations, 1) as “interface” phenomena, we predict that the TMA-systems of NAmNo at large will show CLI affectedness over time; 2) we expect to find several different CLI-effects; borrowing, convergence, and a loss of oppositions and distinctions in certain TMA-domains; 3) we expect morphology (e.g. verbal suffixes) to be more affected than syntax (free-standing auxiliaries); 4) TMA-exponents may be complex and express several TMA-categories simultaneously, e.g. one via its stem (e.g. modal ability) and a different one via its verbal suffix (tense or subjunctive mood), but the drive will be towards one-to-one mappings; 5) we expect affectedness to be high in the domain of mood\slash modality, low in the tense domain, with aspectual domains in the middle.

By the wording “high and low affectedness” I make no attempt at explicitly quantifying over occurrences of a marker or certain phenomenon over time. The data in this chapter are taken from many different speakers, generations, settlements, and different times, with much intra- and interindividual variation. The examples attest to linguistic change but come with no claim of statistically significant values for the tokens, types, or categories shown. Instead, the data serve to illustrate an overview over a wide range of phenomena, and our expectations will guide the search for trends and fluctuations in the material, still enlightening when paired with knowledge of what sociologically characterizes the same time spans. 

In the next section I discuss some fundamental issues regarding the speakers, the input and the baselines relevant to this investigation. In the remainder of the paper \sectref{sec:eide:3} discusses mood and modality in NAmNo, \sectref{sec:eide:4} investigates the aspectual domain, and \sectref{sec:eide:5} is concerned with tense and finiteness in NAmNo. \sectref{sec:eide:6} sums up and concludes the chapter. 

\section{Speakers of NAmNo, affectedness and input, and baselines}\label{sec:eide:2}

A typical Heritage Language (HL) speaker by today’s consensus is a bilingual who shifted from their (minority) heritage language to the dominant language of the society when attending school, and many of the speakers providing data to the current investigation, especially from Haugen’s own recordings, fall under this category. However, since Haugen’s recordings over the years have been complemented by those of a range of successors, resulting in a corpus spanning 80 years of recordings (the CANS corpus; cf. \citetv{chapters/introduction}), the HL-speakers of NAmNo discussed in this chapter spread along a much broader continuum. Many different definitions of heritage speakers exist; cf. for instance \citet[260]{BenmamounEtAl2013}:

\begin{quote}
Defined broadly […] a heritage speaker is anyone who has an ethnic, cultural or other connection with a language, regardless of whether that person learned the heritage language as a child. Defined narrowly, a person is a heritage speaker if and only if he or she grew up learning the heritage language and has some proficiency in it.
\end{quote}

Although all speakers included in this investigation fall under the narrow definition quoted here, huge differences may exist between the language and the linguistic situation of Haugen’s speakers – many still conducting their daily affairs in Norwegian, listening to Norwegian services in church, reading Norwegian newspapers, books and letters from the homeland, some selecting Norwegian as a foreign language in high school – and the HL-speakers recorded in 2010 and after. The latter typically learned Norwegian from their grandparents, and after these grandparents were gone, the contexts in which Norwegian was spoken were usually few and far apart, which clearly affects the speaker’s fluency of speech and processing of NAmNo. 

The recordings discussed in this chapter span almost a century and mirror the contemporary sociolinguistic history of a society where potential contexts for natural use of the heritage language occur more rarely for each generation, gradually limited to conversations with close family members, confining the HL to a “home language”. This again affects the potential quantity, quality, and diversity of the HL-input that the speaker is likely to receive, in turn affecting the properties of their NAmNo, including their TMA-systems. The residing sources of HL-input provide the heritage speaker with fewer contexts to acquire fine-grained nuances and contextual restrictions of particular linguistic forms, affecting the vocabulary size – the HL-speakers of recent recordings usually possess a smaller and less specialized (productive) vocabulary than earlier generations, leading to the register reduction attested in many HLs. Moreover, the production of later generations of HL-speakers usually inclines toward structural simplification, visible through a reduction or omission of certain distinctions available to previous generations of NAmNo speakers and old and new speakers of contemporary EurNo. This loss of (mostly formal) distinctions seems to be a hallmark of bilingualism and language contact, cf. e.g. this early statement from \citet[39]{Vogt1948}. 

\begin{quote}
 \relax
 [O]n observe souvent qu’une langue ….. perd des distinctions formelles, dans des circonstances qui rendent l’hypothese d’influence étrangère assez naturelle.\footnote{Quoted here from \citet{Trudgill2011}. Translation: “It can often be observed that a language loses formal distinctions in circumstances in which the hypothesis of a foreign influence is very natural.”}
\end{quote}

Loss of oppositions can thus be seen mainly as a result of less input from each language, characteristic of any bilingual situation.\footnote{But cf. \citet[15]{CorbetDominguez2020}, following \citet{CuzaMiller2015}, \citet{PolinskyScontras2020}: Simplification is often due to divergence in the input, but also to innovation in HL-speaking children’s grammars.} Consider the potential relation between HL input and the maintenance of a certain productive opposition in EurNo. Whenever the EurNo variety contains two or more exponents of a closely related morphosyntactic feature, e.g. two different TMA-markers, sophisticated contextual restrictions may regulate the selection between the two. It does not suffice that both variants are present in the input to some extent; to learn the felicitous uses of both, the learner must first discover the two different forms available, then proceed to identify the differences between them. However, to observe the different forms in sufficiently abundant different contexts to learn the pragmatic restrictions for each form requires substantial contextually diverse input. In HL acquisition diversity of contexts tends to reside \citep{GollanEtAl2015}, and contexts ultimately may be too few to learn the original context\hyp dependent restrictions on the use of each of the two markers. As the rules for the opposition is not acquired, it is in effect omitted, and the language learner might as well settle for acquiring only one of the variants to cover the domain previously split between two markers.

Contextual restrictions on morphosyntactic markers operate on the interface between syntax and pragmatics, a well-known challenge for  bilinguals (\citealt{Sorace2004, TsimpliSorace2006, SoraceSerratrice2009, Sorace2011}), and omission of a distinction is especially likely to occur where several domains of grammar interact. Attested cases in NAmNo HL include the presence vs. absence of resumptive pronouns in topicalizations \citep{BousquetteEtAl2021}, the choice of topicalization structures vs. SVO-structures (\citealt{EideHjelde2015Verb, WestergaardLohndal2019, BousquetteEtAl2021}), and prenominal vs. post-nominal possessives (\citealt{WestergaardAnderssen2015}). These studies on NAmNo attest how one of the variants prevails over the other variant, which eventually occurs less frequently or not at all. Though fostering ambiguity, omission of oppositions yield simplification as two variants reduce to one. 

The same type of mechanism can be argued to apply to borrowing or transfer. The NAmNo modal marker \textit{spost ‘}supposed to’ (\citealt{EideHjelde2012}) is absent from all variants of contemporary and current EurNo, hence likely borrowed from the dominant language AmE. At the outset, \textit{be supposed to} comes with the contextual restriction “English linguistic context,” in opposition to the Norwegian modal auxiliary \textit{skulle}, in turn contextually restricted to all-Norwegian contexts. Semantically, in EurNo \textit{skulle} conveys the same two modal meanings as \textit{be supposed to}, social obligation (deontic reading) and hear-say (evidential reading), cf. \citet{Eide2005}. For the first speakers to introduce \textit{spost} into NAmNo, restriction “English context” is relaxed, and for the next generation there is already conflicting evidence from the input as some speakers use \textit{spost} even in Norwegian contexts. Eventually there is too little evidence from input to uphold the distinction between \textit{spost} and \textit{skulle}, and both are used in NAmNo.\footnote{The data in CANS suggest that \textit{skulle} and \textit{spost} come to share the domain between them, as no instances of \textit{skulle} are unambiguously evidential. Instead, \textit{skulle} acts as a future marker (corresponding to \textit{will}), cf. sect. \ref{sec:eide:5.2}} 

The acquisition and maintenance\footnote{For most of our informants we have no principled way of knowing whether a certain opposition was present in their production at an earlier stage, or never aquired in the first place. \citet[18]{Polinsky2018} lists 1) transfer from the dominant language, 2) attrition across the lifespan, and 3) divergent attainment as possible explanations for deviation from a baseline. Cf. \citet{Kinn2020} for enlightening discussions on these notions specifically for NAmNo.} of two or more similar variants in an opposition is thus likely to depend on whether the input is “rich” enough to provide the relevant contextual cues. An ongoing debate within HL-research discusses whether quantity, quality or diversity of input is more important. According to \citet{Meisel2011}, frequency remains a poor predictor for language change, as we have no good estimate on what constitutes the threshold frequency for acquiring a feature from HL-input. The issue likely has idiosyncratic aspects as certain speakers will rely on diversity more than quantity, some will need more input than others to transform HL-input data to useful intake material. Moreover, even though we have much data on spoken and written varieties, available in principle in the society as input to the different generations of NAmNo speakers, in practice there is no way to determine to what extent the individual subject made use of all possible available opportunities to practice their NAmNo, how much of the available input was \textit{actual} input, intake and eventually internalized as a grammar of each of the individual speakers that happened to be recorded e.g. for the CANS corpus. Even if we subject speakers to elaborate interviews, linguistic self-reporting is unreliable, an observation going back to \citet{Labov1972} and reconfirmed many times since.

HL-studies observe omission of oppositions of many types, but authors disagree on whether the different cases are best described as contact-induced simplification, convergence, or transfer/borrowing. Most studies feature English as the dominant language surrounding the immigrant society (\citealt{Montrul2016}, \citealt{BenmamounEtAl2013}), and as English has few formal grammatical oppositions compared to most HL-languages investigated, transfer from English may be hard to separate from the simplification otherwise attested in language contact (and even in monolingual, language-internal linguistic change, e.g. grammaticalization). 

Borrowing or transfer implies a direction from a source language (here AmE) to a recipient language (here NAmNo). No directionality is implied in the notion \textit{convergence}, implying instead an expectation of the emergence of hybrid phenomena, incorporating features from both languages. \citet{Matras2009} discusses how any bilingual will look for overlap between their two linguistic systems, as maintaining a strict separation requires cognitively costly selection procedures to match the language to the context. Lowering the bar between the two languages and “allowing patterns to converge” thus maximizes “the efficiency of speech production in a bilingual situation” (\citealt{Matras2009}: 151, 237; cf. also \citealt{Wald1987}, an early proponent of these ideas). Again \textit{spost} is a convenient example. Though \textit{spost} has been subject to some legitimate linguistic interest since first discussed in \citet{EideHjelde2012,EideHjelde2015Borrowing}, no investigation has determined whether its syntactic features are maintained from EurNo or borrowed from AmE. Instead, the syntactic features of \textit{spost} seem to converge on both EurNo and AmE patterns \citep[329]{PutnamSøfteland2021}; cf. also \sectref{sec:eide:5.2}.\footnote{Convergence also captures the principle of L1/L2 Non-interference in \citet[3]{Roeper2016}: “Rules from L1 may apply in L2 as long as no obligatory module from either language is violated or ignored.”}\footnote{In one example the selectional requirements of an infinitival complement are suspended, cf. (i) from speaker mabel\_MN\_01gk. Here the complement is finite, with a visible nominative subject and a finite present verb. 

\ea
\gll Va re {i går}  du va spåos te du jør de?\\
     was it yesterday you were spos to you do it\\
\glt ‘Was it yesterday you were supposed to do it?
\z}

In the previous discussion EurNo is the explicit comparison for NAmNo, i.e. the \textit{baseline}. Using monolingual homeland speakers,\footnote{To the extent that monolingual speakers exist, which they do not, according to theories of “universal bilingualism” \citet{Roeper1999}, multiple grammar theory (\citealt{AmaralRoeper2014}), and parallel grammars (\citealt{EideSollid2011, EideÅfarli2020}). Cf. also the discussions in \citet{WieseEtAl2022}, see \citet{EideHjelde2023} for details. Recent studies conclude that earlier bilingual studies underestimate the variation found in monolingual native speaker control groups, i.e. the baseline \citep{DabrowskaEtAl2020}, who show “a degree of lexical and morphosyntactic variation that [is] sometimes higher than that of bilinguals” \citep{WieseEtAl2022}. To present this variation as non-existing or theoretically unimportant is neither “true, defensible, nor helpful” \citep{RothmanEtAl2023}. \citet{DabrowskaEtAl2020} notes that L2 learners often score as high as, sometimes higher than native controls. E.g. \citet{Wakabayashi1996} requires a score at 100\% from L2 subjects to for their production to be “nativelike” (though not all natives score this high). \citet{Dabrowska2012} shows how “nativelike behavior” of actual native speakers hinges on high education. This suggests a biased perspective on below-ceiling scores in the two populations. For L2 and HL speakers, lower scores suggest inadequate attainment or language attrition (i.e. a deficiency). For native controls variation is ascribed to meaningful selection of items (i.e. a resource).}   present or past, as a default baseline for HL-grammar production may seem straightforward, but recently the notion of \textit{baseline} has been questioned and refined within HL-linguistics, e.g. \citet{Polinsky2018}, \citet{AalberseEtAl2019}. 

Something like a “methodological control” is a “well-known, essential part of the standard scientific method” employed in human social sciences; hence is inevitable also in heritage language studies, \citet[317]{RothmanEtAl2023}. \citet{Serratrice2020} discusses recent seminal HL-studies and notes that the term “baseline” is used in at least three different ways: 1) The language that served as the input for acquisition; 2) The diasporic variety spoken by first-generation immigrants; and 3), Homeland monolingual subjects. \citet{RothmanEtAl2023} object to using the latter as a default baseline. First, it sends the message that monolingualism is the “norm” state of linguistic knowledge (but cf. e.g. \citet{DeHouwer2021} for the observation that globally, multilingualism is the norm). Second, explicitly stated or implicitly inferred, the homeland monolingual variety already has some “privileged status” in the HL community, making it “the ubiquitous target and/or benchmark”. Third, as stated by \citet{WieseEtAl2022}:

\begin{quote}
A[nother] – implicit or explicit – requirement, often found in heritage language research, is that in order to be fully recognized as such, a native speaker has to master a repertoire that also involves standard or formal registers of their heritage language. [...] Since such registers are learned primarily in the context of formal education, heritage speakers often only acquire them for the majority language, which would then exclude them from the group of proficient native speakers of their heritage language. 
\end{quote}

This requirement would likewise efficiently exclude present-day speakers of NAmNo from scoring at “native levels”, since they too received their formal education in English, the dominant language of their society, not in Norwegian, although written registers and school-like contexts were available to NAmNo speakers in the earliest time periods we are studying (cf. below). \citet[174ff.]{Montrul2016} acknowledges that using the monolingually raised native speakers as baseline promotes the monolingual biases mentioned above. 

\begin{quote}\relax
[However, this] comparison cannot be avoided entirely, especially when it is necessary to answer specific and important research questions. [Without it,] we would not be able to understand how one language influences the other during development, or to tease apart developmental versus transfer errors in second language and bilingual acquisition. We would not be able to understand diachronic change in a language that develops independently versus change in the same language in contact with another language. 
\end{quote}

In the case of NAmNo and the trends reported here, there are ideally several relevant baselines; the time span of 80 years of recordings and a fairly substantial conception of the selected society during this period allows us, at least in principle, to stipulate different baselines for different times; cf. \citet{LarssonJohannessen2015Incomplete}, \citet{Lykke2020}, \citet{vanBaal2022New}, \citet{EikLaanemets2021}, \citet{KinnLarsson2022}, and \citet{EideHjelde2023,EideHjeldeForthcoming}. To fully understand the NAmNo data, especially when studying the several different development paths of tense exponents and their relatives in the TMA-domain, ideally several intermediate steps in the development should be identified, representing different idealized generations, \textit{cohorts}, in the diachronic development (adapted from \citealt{EideHjelde2023}; see also Table 4 in \sectref{sec:eide:5}).

\begin{table}[h]
\caption{Breakdown of the cohorts of NAmNo speakers \citep{EideHjelde2023}}
\label{tab:eide:cohorts_short}
\begin{tabularx}{\textwidth}{l *5{Q}}
\lsptoprule
Cohort & I & II & III & IV & V\\\midrule
Born & Around 1870 & 1900--1920 & 1920--1930 & 1940--1950 & after 1950\\
%Data sets & Haugen 1940s & Haugen 1940s & Hjelde 1980s\&1990s & CANS Eide\slash Hjelde 2010 & CANS Hjelde 2010--2018\\
\lspbottomrule
\end{tabularx}
\end{table}

%\begin{quote}
%Cohorts: I (born around 1870), II (1900--1920), III %(1920--1930), IV (1940--1950), V (after 1950). 
%\end{quote}

For some cohorts, e.g. the years 1945--1985, we have little speech data. Instead, we make qualified conjectures about the input of these cohorts based on the documented written input available, i.e. the number and volume of Norwegian newspapers published in the period, the number of parochial schools taught in local Norwegian Lutheran churches, the number of church services and confirmations conducted in Norwegian, and the status of Norwegian as the only foreign language offered in local schools. These extrapolated data allow for assumptions about the written input and the spoken formal “church language”, whereas the more home-based spoken registers from this period (1945--1985) are less well documented, cf. \citet{EideHjelde2023}.  

\section{Mood and modality in NAmNo}\label{sec:eide:3}

According to our expectations listed in \sectref{sec:eide:1}, modality is easily affected in language contact, strongly related to contextual-pragmatic restrictions on the famously vulnerable pragmatic-syntactic interface. We expect mood, i.e. modality expressed via morphology, to be more affected than any other TMA-category. \citet[54]{Montrul2016} duly notes: 

\begin{quote}
Perhaps the verbal category that is most affected in heritage languages is mood in languages that express it morphologically. That is, not all languages express modality overtly in the grammar, but those which do present a challenge to heritage speakers when it comes to the expression of mood.
\end{quote}

For NAmNo it is relevant to point out that non-indicative moods, especially the Old Norse subjunctives, were clearly moribund already in EurNo when the immigrants left the “old country” (cf. \citealt{Eide2010}, \citealt{Hjelde1992}). In some settlements and certain frequently appearing contexts, the subjunctive was however present in NAmNo speakers’ input for a longer time. Finite mood markers in NAmNo constitute the topic of \sectref{bkm:Ref159263343}. In Sections \ref{subsect_irrrealis} and \ref{subsect_modalauxs} we look at non-finite mood morphology and modal auxiliaries, respectively. 

\subsection{Finite moods: Subjunctive 1 and 2, imperatives and replacement forms}
\label{bkm:Ref159263343}
Contemporary Norwegian has almost no morphologically encoded finite mood distinctions, apart from the split between indicative and imperative, but in Old Norse the subjunctive was quite productive and occurred frequently in the old saga texts. The two morphologically distinct subjunctives split the domain of subjunctive meanings between them; the “present subjunctive” or subjunctive 1 expressed optative meanings like desires, wants, wishes, demands, and the “past subjunctive” or subjunctive 2 expressed hypothetical and counterfactual meanings (cf. \citealt{Eide2010} for details).\footnote{The difference between the “present” and “past” subjunctive is clearly not one of temporality, but of distance to reality, cf. \citet[142]{Iversen1990} on Old Norse. \citet[319]{Thieroff2004}: 

\begin{quote}
The subjunctive 1 and the subjunctive 2 do not differ regarding time reference. For example, both \textit{er singe} and \textit{er sänge} have non-past time reference and differ only with regard to their modal meaning. In contrast, in the indicative the present form \textit{er singt} has non-past reference, whereas the preterit form \textit{er sang} has past time reference. The same behavior holds for the subjunctive forms in all German (and in most other European) languages.
\end{quote}
}  
The latter tasks came to be expressed by the plusperfect in Mainland Scandinavian; but the subjunctive 1 continued to be present in the input of NAmNo speakers for many generations. One text NAmNo speakers likely heard and recited on a regular basis is \textit{The Lord’s Prayer}, possessing a central position in the Liturgy of Norwegian Lutheran Churches. The 1930 edition of The Lord’s Prayer features many present subjunctive verb forms.

\ea%4
    Fader vår  (The Lord’s Prayer)\label{ex:eide:4}\\
	\gll Fader vår, du som er i himmelen!\\
		 Father \textsc{poss}, you who are in heaven!\\
	\gll Helliget vorde ditt navn;    komme ditt rike;\\
	Hallowed be.\textsc{subj1} your name  come.\textsc{subj1} your kingdom\\
	\gll Skje din vilje,         som i himmelen, så og på jorden\\
	Happen.\textsc{subj1} your will,     as in heaven so also on earth\\
	\glt ‘Our Father, who art in heaven, hallowed be thy name; thy kingdom come; thy will be done; on earth as it is in heaven.’
\z



Also, in many hymns the optative was very much in use, like hymn number 620, \textit{Synge vi av hjertens grund} ‘Let us sing from the bottom of our hearts’ from \textit{Landstad’s Book of Hymns for Lutheran Christians} (1909). In this case the prevalent reading seems to be hortative (‘let us’).



\ea \gll  Synge vi av hjertens Grund,\\
		  Sing.\textsc{subj1} we of heart.\textsc{def.poss} ground\\
	\gll Love Gud med Maal og Mund\\
		Praise.\textsc{subj1} God with voice and mouth\\
	\glt  ‘Let us sing from the bottom of our hearts, let us praise God with our voices.’
\z

%\begin{figure}
%\includegraphics[width=\textwidth]{figures/eide-figure1.png}
%\caption{Passage from \textit{Landstad's Book of Hymns for Lutheran Christians (1909)}}
%\end{figure}


In Old Norse the subjunctive occurred most frequently with copulas. \citet{Hjelde1992} notes that some NAmNo speakers still use the subjunctive with copulas \textit{være} `be', \textit{bli, verte} ‘become’. 

\begin{quote}
With many of my America-trønder informants I found that the subjunctive of the verb \textit{å vera} ‘to be’ still was in use. For this verb they maintained the distinction between /va:/ in indicative and /vo:/ in the subjunctive. The same is true for the verb \textit{å verte} where the indicative preterit is der /vart:/ and the subjunctive form is /vort:/. It is true that even American English may use the subjunctive with the verb \textit{be}, like in the sentence \textit{If I were you.} But even if American English has some relics of the subjunctive, I see no reason to assume that the NAmNo subjunctive forms are a result of transfer since both / vo:/ and /vort:/ still exist in the old dialect of Inntrøndsk.\footnote{“Hos mange av dei Amerika-trøndske informantane mine fann [eg] at konjunktiv av verbet å \textit{vere} enno vart nytta. For dette verbet skilde dei mellom /va:/ i indikativ og /vo:/ i konjunktiv. Det same gjeld også for verbet å \textit{verte}, der indikativforma i preteritum er /vart:/ og konjunktivforma /vort:/. Rett nok kan også amerikansk ha konjunktiv av verbet \textit{to} \textit{be}, som t.d. i setningen \textit{If I were you}, Men sjølv om også amerikansk har nokre restar av konjunktiv, ser eg ingen grunn til å rekne dei norske konjunktivformene som eit resultat av interferens av di bade /vo:/ og /vort:/ enno finst i gammalt inntrøndsk mål.”}  (My translation).
\end{quote}


In contrast, the CANS corpus contains no subjunctives in the production of NAmNo speakers; the attribute \textit{mood}  has only has ‘imperative’ not ‘subjunctive’, as a potential value.\footnote{Hjelde’s recordings are also incorporated into CANS, with certain exceptions. Some recordings were left out of CANS; e.g. because of privacy agreements, others because of the topics discussed require discretion.}\footnote{The CANS data are rendered here with a unique informant code. To give the reader an impression of whether the informant belongs to the very early or more recent generations of speakers, I have chosen to indicate the year of recording for each example taken from the CANS corpus. Also, the rendition of the examples in this chapter deviates from that of all the other chapters in this volume in that in this chapter the CANS examples are rendered in a semi-phonetic transcription. As stated in the introductory chapter of the book, the editors have opted for orthographic transcriptions as the default, and they give good reasons for this choice. However, in the present chapter the phonetic forms seem rather important, especially those which are clearly fused or hybrid forms containing features from AmE as well as EurNo, e.g. examples in \REF{ex:eide:16}, e.g. \REF{ex:eide:16a} given here as (i). The othographic transcription “far could not e kunne ikke få em petroleum” suggests that there is code-swiching between AmE and NAmNo in this case, but the semi-phonetic transcription shows instead that the first modal is “kudd”, with a Norwegian pronounciation, not could, with an AmE pronounciation. Since I will make a point of these fused or hybrid forms, I have two options; either opt for coherence with the rest of the chapters in this volume or else opt for coherence across the present chapter. I chose the latter option, which makes the data rendition in this chapter stand out from that of the other chapters, although I try to minimize the semi-phonetic transcription to the relevant cases. I apologize to readers confused by this choice.   
\ea 
	\gll Far kuddn’t {\#ee\#} kunna ikke få petroleum.\\
	     Dad couldn’t   {}    could not get petroleum\\
    \glt ‘Dad couldn’t get hold of gas.’ (harmony\_MN\_01gk, rec. 2010)\\
\z}\footnote{A verb often tagged as imperative is \textit{se} ‘see', cf. (6abc). Formally an imperative, its function in NAmNo is often as a discourse particle, likely transfer from AmE ‘you see,’ as only a few dialects of EurNo feature this particle.}


\ea%6
    \label{ex:eide:6}
    \ea 
	    \gll \textbf{Se} der, Nystuggu.\\
		look there, new.house.\textsc{def}\\  
	    \glt ‘Look, the new house.' ({{albert\_lea\_MN\_01gk}}, rec. 2010)\\
    \ex    
	    \gll Det var verdt turen å få \textbf{se} det.          \\
             it was worth trip.\textsc{def} to get see.\textsc{inf} it\\
        \glt ‘It was worth the trip to get to see it.’ ({{chicago\_IL\_01gk}}, rec. 2010)\\ 
    \ex 
        \gll \textbf{Se,} det var ikke land å ta i Nord-Dakota mere.      \\
              see there was not land to take in {North Dakota} anymore    \\
        \glt  ‘See, there was no more land to take in North Dakota.’ ({stillwater\_MN\_01gm}, rec. 2010)\\
    \ex 
        \gll Å, unnskyld, ja.\\
             oh excuse yes           \\
        \glt ‘Oh, excuse me!’    ({{chicago\_IL\_01gk}}, rec. 2010)\\    
    \z 
\z

Although the imperative of the verb \textit{se} ‘see’ is homophonous with the infinitive (6ab), in NAmNo as well as in EurNo, the imperative in EurNo usually differs from the infinitive and the present as the imperative form is the verb stem (cf. 6d), the infinitive has an -\textit{e} suffix. A phonological rule sometimes adds a final -\textit{e} to the imperative to ease pronunciation \citep{Rice2003}. Infinitives and imperatives may be homophonous, especially in dialects where apocope deletes a final unstressed -\textit{e} even with infinitives. However, syncretism occurring between the present tense and an imperative ending is quite rare in EurNo. This happens in NAmNo, though. The example in \REF{ex:eide:7} is from informant “Lena”, who clearly uses the present tense verb form as an imperative (cf. \citealt{EideHjelde2015Verb} for details).  


\ea \label{ex:eide:7} 
	\gll Får sjå nå….\\
         let.\textsc{pres} see.\textsc{inf} now\\
	\glt ‘Let’s see now.’     (blair\_WI\_04gk, rec. 2010.)\\
    \textit{Baseline: Få sjå nå}
\z


As mentioned above, the Old Norse Subjunctive 2 expresses counterfactuality and hypothetical situations, in contemporary Norwegian and Swedish replaced by the plusperfect. The plusperfect itself is not very frequent in NAmNo, and among the examples found, few are counterfactuals. Montrul (2016: 62, 70) discusses several HL-studies where first generations of HL-speakers use counterfactuals, conditionals, and subjunctive moods to express different degrees of hypothetical reasoning, whereas already the second generation tends to replace the different exponents of degrees of possibility and certainty by indicative morphology, and compound constructions by simpler ones (e.g. \citealt{SilvaCorvalan1994}). This simplified TMA-system does not differentiate morphosyntactically between more or less possible situations. 

The plusperfect consists of a preterit auxiliary \textit{ha(dde)} ‘had’ and a past participle main verb. A feature of many of the Norwegian dialects brought over by the immigrants is the invariant auxiliary (\textit{ha}) (cf. \citealt{Eide2021} for details on the extension of this auxiliary); i.e. the auxiliary has the form \textit{ha} in the infinitive, present and preterit. Invariant \textit{ha} in EurNo occurs only as an auxiliary, never as a lexical verb; the latter has ordinary present and past forms (e.g. \textit{har, hadde}) (cf. also \sectref{sec:eide:5.1}). In standard EurNo these are the forms also for the auxiliary. CANS contains many plusperfects with invariant \textit{ha,} but also many examples with the more “written standard-like” \textit{hadde,} and sometimes both occur in the same utterance, cf. \REF{ex:eide:8d}. 

\ea%8
    \label{ex:eide:8}
    \ea \label{ex:eide:8a} 
    \gll E ønsker at e hadde lerrd {te å} læsa det og skrive det.   \\
         I wish that I had learned to read it and write it       \\
    \glt ‘I wish I had learned to read and write it.’ ({{billings\_MT\_01gm}}, rec. 2012)\\
    \ex \label{ex:eide:8b} 
    \gll Ha e vårr kjuge år yngre så kannskji.      \\
         have I been.\textsc{part} twenty years younger then maybe   \\
    \glt ‘Had I been twenty years younger, then maybe.’ ({{coon\_valley\_WI\_02gm}}, rec. 2010)\\
    \ex \label{ex:eide:8c} 
    \gll Vi ha ællrin kommi {ti å} fårstin fårsto nå nåssjt hell laga hell. \\
         we have never come to understood.\textsc{part} understood.\textsc{pret} any Norwegian or speak or…\\
    \glt ‘We would never have understood or spoken any Norwegian.’ ({{coon\_valley\_WI\_07gk}}, rec. 2010)\\
    \ex \label{ex:eide:8d} 
    \gll Kanskje ha vøri annsjsless viss kona hadde vør norsk veit du. \\
         maybe had been different if wife.\textsc{def} had been Norwegian know you\\
    \glt ‘Maybe it would have been different if the wife had been Norwegian, you know.’ 
  ({{ulen\_MN\_03gm}}, rec. 2014)\\
    \z % you might need an extra \z if this is the last of several subexamples
\z

The contracted AmE modal forms \textit{coulda, shoulda, woulda}, and also \textit{s(up)posta} can quite easily be analyzed as (emerging) mood markers (cf. note 5), coding counterfactuality or irrealis. We also find examples of this kind in NAmNo, cf. (9ab) from CANS and \REF{ex:eide:9c} from  \citet[258]{EideHjelde2015Verb}. Note that the English modals \textit{should} and \textit{could,} and also the semi-modal \textit{supposed to} are pronounced with NAmNo phonology and occurred in all-Norwegian sentences in these examples. These are evidently borrowings from AmE, what \citet{MatrasSakel2007} would refer to as \textit{matter replication}, where a form (\textit{sjudda}) and its meaning (counterfactual) is copied from the source langauge, although pronounced with NAmNo phonology, but also converge on EurNo dialectal patterns. For \textit{vudda} (\emph{woulda}), I found no examples with NAmNo phonology or outside an English context. 

\ea%9
    \label{ex:eide:9}
    \ea \label{ex:eide:9a} 
    \gll [E] sjudd a spørt mye.          \\
         I shoulda {} asked {a lot}.          \\
    \glt ‘I should have asked a lot.’ ({{hatton\_ND\_04gk}}, rec. 2010)\\
    \ex \label{ex:eide:9b} 
    \gll Ja, romm kudd hævv skup upp, ja.       \\
         yes they could have scooped up yes       \\
    \glt ‘Yes, they could have scooped it up, right.’ ({{glasgow\_MT\_01gm}}, rec. 2012)\\
    \ex \label{ex:eide:9c} 
    \gll [D]u {va itj} spost {te å} ji dæm nå mat.       \\
         you weren’t sposta to give them any food    \\
    \glt ‘You were not supposed to give them any food.’ ({{appleton\_MN\_01gm}}, rec. 1987)\\ 
    \z % you might need an extra \z if this is the last of several subexamples
\z

These examples attest to the high borrowability of modality markers, as predicted, both syntactic (\textit{sjudd}) and morphological (-\textit{a}), and simultaneously demonstrate that morphological mood markers – usually seen as prone to erosion and “perhaps the verbal category that is most affected in heritage languages”, according to \citet{Montrul2016} – are still present even in later generations. The preserverence of old dialectal modal traits like invariant \textit{ha} also attests to the fact that certain archaic NAmNo modal markers are capable of surviving a diasphora lasting for almost two centuries (although our recordings only span half of that time). 

\subsection{Irrealis infinitives: Non-finite mood morphology} \label{subsect_irrrealis}
\largerpage[2]

Non-finite mood markings also appear on non-finite forms inn most spoken varieties of Norwegian, though never recognized in any of the two written standards, seemingly due to an historical fluke.\footnote{Cf. \citet{Aasen1864}. Aasen laid the groundwork for \textit{Nynorsk} and decided (rather arbitrarily) that these constructions not qualify as infinitives, which made norm. Cf. \citet{Eide2011Ghost, Eide2021}, \citet{Aa2022}, and \citet{Sandøy1991}.} I refer to this construction as \textit{have-less perfects} \citep{Eide2021}, cf. (10ab); traditionally analyzed as perfects with an omitted auxiliary \textit{ha} ‘have’. In EurNo they appear after counterfactual (formally preterit) modals, cf. \REF{ex:eide:10a}, and as infinitival complements, cf. \REF{ex:eide:10b}. The written norm disallows the infinitival marker \textit{å} `to' to appear with what is seemingly a past participle, and writers resort to the homophonous coordinator \textit{og} ‘and’ to substitute for \textit{å,} cf. \REF{ex:eide:10b}. If the analysis of ‘have-deletion’ had merit and \textit{og} was the right marker, one would expect (10b’), with \textit{ha} “retained”, to be grammatical with \textit{og}, but it is not – only with \textit{å}. 

\ea%10
    \label{ex:eide:10}
    \ea \label{ex:eide:10a} EurNo\\
    \gll Vi skulle reist {i gå}r.\\
         we should left.\textsc{part} yesterday\\
    \glt ‘We should have left yesterday.’
    \ex \label{ex:eide:10b} EurNo\footnote{\url{https://www.nrk.no/telemark/tidenes-bryllupsbilde-\_-men-aner-ikke-hvem-brudeparet-er 1.13615318}}\\
    \gll Det hadde vært så hyggelig og visst hvem de var.    \\
         it had been so nice and known who they were \\
    \glt ‘It would have been so nice to know who they were.’
    \exi{b.'} \label{ex:eide:10b'} EurNo\\
    \gll Det hadde vært så hyggelig å/*og ha visst hvem de var.\\
         it had been so nice to/and have known who they were\\
    \glt ‘It would have been so nice to know who they were.’
    \ex \label{ex:eide:10c} Hallingdalen dialect\\
    \gll Fått’n se kji kvæmmfølk, så døytt’n.         \\
         {gotten he} himself not womanfolk then {died.\textsc{part} he} \\
    \glt ‘If he doesn’t get himself a woman, he might die.’   
    \z % you might need an extra \z if this is the last of several subexamples
\z
          
\citet{Sandoey1991, Sandoey2008}, \citet{Julien2003}, and \citet{Eide2011Ghost, Eide2021}, analyze these “participles” as irrealis infinitives. Diachronically deriving from subjunctives, they appear even in finite functions and positions in many Norwegian dialects (cf. 10c); importantly, in dialects spoken in areas of Norway where many of the immigrants had their origin.\largerpage 

The finite “participle” (cf. 10c) is unattested in NAmNo, but the “irrealis infinitive” is frequent in CANS, after counterfactual modals (11abc) and as infinitival complements (11de). These forms rarely appear in edited writing; hence they are commonly transmitted via spoken NAmNo. Like in EurNo written texts, they often appear in CANS with \textit{og} ‘and’, not \textit{å} ‘to’.{\interfootnotelinepenalty=10000\footnote{Signe Laake (p.c): The annotators recognized the participle\hyp looking form as an infinitive, but because of Norwegian \textit{pseudo\hyp coordination} with e.g. posture verbs, e.g. \textit{sitter og spiser}, lit. \textit{sits and eats}, ‘is eating’ (cf. \sectref{bkm:Ref159266133}), a normative tradition exists for \textit{og} in functions besides coordination, and the annotators were advised to use \textit{og}.}} 

\ea%11
    \label{ex:eide:11}
    \ea  
    \gll E kunna jorrt det bære.           \\
         I could done.\textsc{part} it better         \\
    \glt ‘I could have done better.’ ({{billings\_MT\_01gm}}, rec. 2012)\\
    \ex  
    \gll Det måtte vøri nokså tåfft.         \\
         it must been.\textsc{part} quite tough        \\
    \glt ‘That must have been quite hard.’ ({{coon\_valley\_WI\_06gm}}, rec. 2010)\\
    \ex  
    \gll Mæn jæ sjudd nå jort dæ.         \\
         but I should now done.\textsc{part} it        \\
    \glt ‘But I still should have done it.’ ({{portland\_ND\_01gm}}, rec. 2012)\\
    \ex  
    \gll Je sku likt å hatt en akksjon je å, men.      \\
         I should liked.\textsc{part} to had and auction I too but  \\
    \glt ‘I too should have liked to have had an auction.’ ({blair\_WI\_01gm}, rec. 2010)\\
    \ex  
    \gll Dåm ha villa vørri falig  å vårri rundt veit du.          \\
         they had will.\textsc{part} been dangerous   to be.\textsc{part} around know you   \\
    \glt ‘They would have been dangerous to be around, you know.’ ({{coon\_valley\_WI\_20gm}}, rec. 1992)\footnote{For this particular recording the recording year is not entered, but it is likely to be 1992.} \\
    \z % you might need an extra \z if this is the last of several subexamples
\z

In some EurNo dialects the “infinitive guised as participle” has meanings other than counterfactual \citep{Sandøy1991}, but only the counterfactual type, attested in spoken varieties all over Norway, is attested in CANS, i.e. the “irrealis infinitive”. I replace the coordinator conjunction \textit{og} appearing in these examples in CANS with \textit{å} in my rendering of the data; cf. \citet{Eide2021} and \citet{Aa2022} for a more detailed justification of this choice. 

\subsection{Modal auxiliaries: Inventory and syntax} \label{subsect_modalauxs}

In the previous examples we see that NAmNo employs modal forms clearly borrowed from the dominant language AmE; e.g. \textit{kudd} ‘could’, \textit{sjudd} ‘should’, and \textit{spost} ‘supposed to’. It is also evident that many speakers have both the borrowed form and the EurNo form in their vocabulary. There are many instances where speakers correct themselves, replacing the borrowed AmE-form with the more EurNo-like one, cf. examples in \REF{ex:eide:12}. 

\ea%12
    \label{ex:eide:12}
    \ea  
    \gll Far kuddn’t \#ee\# kunna ikke få petroleum.                  \\
         Dad couldn’t  {}       could not get petroleum       \\
    \glt ‘Dad couldn’t get hold of gas.’ (harmony\_MN\_01gk, rec. 2010)\\
    \ex  
    \gll Den kjerringa sa kanhende du sjudd \#du sku\# sjå på Heddal.      \\
         that woman said maybe you should you should look at Heddal   \\
    \glt ‘That woman said, maybe you should look at Heddal.’ (harmony\_MN\_01gk, rec. 2010)\\
    \ex  
    \gll Vi hædd ski, vi kudd \#vi kunne ski e skol.              \\
         we had skies, we could we could ski to school              \\
    \glt ‘We had skies, we could ski to school.’ (coon\_valley\_WI\_12gm, rec. 2012)\\
    \z % you might need an extra \z if this is the last of several subexamples
\z

Certain differences between AmE and EurNo modals make them an interesting testing ground with respect to potential syntactic transfer (cf. \citealt{Eide2005}, chapter 2). AmE-modals occur in non-finite forms only, thus do not stack, unlike EurNo-modals, which appear as both finite and non-finite. Also, as most Germanic modals, except English and Icelandic ones, EurNo-modals feature verbless directional complements (cf. \ref{ex:eide:14} below).

\ea%13
    \label{ex:eide:13}
    \ea \label{ex:eide:13a}  
    \gll      E driver tenke på å skø reise over igjen.       \\
              I continue think on to shall travel over again       \\
    \glt     ‘I am thinking about traveling over there again.’ (harmony\_MN\_02gk, rec. 2010)\\
    \ex \label{ex:eide:13b} 
    \gll     For å kunne overleve di måtte ta inn forskjellige.    \\
             for to can survive they must take in different.\textsc{pl}    \\
    \glt     ‘To be able to survive they had to take different people in.’ (seattle\_WA\_03gm, rec. 2012)\\
    \ex \label{ex:eide:13c} 
    \gll     Vi kvinnfolka ha måtta liggje på på hand og kne      og skura golv.           \\
             we womenfolk have must.\textsc{part} lie on {} hand and knee   and scrub.\textsc{part} floors  \\
    \glt 	‘We have had to lie on our hands and knees to scrub the floors.’ (coon\_valley\_WI\_45gk, rec. 1942)\\
    \ex \label{ex:eide:13d} 
    \gll     E veit itte hå årstal dømm a kunna kommi, men.      \\
             I know not what year they have could come but      \\
    \glt     ‘I don’t know what year they may have come.’ (westby\_WI\_11gm, rec. 2011)\\
    \ex \label{ex:eide:13e} 
    \gll     Det a måtta vøre strevsamt.          \\
             that have must been arduous          \\
    \glt     ‘That had to have been arduous.’ (madison\_MN\_07gm, rec. 2016)\\
    \ex \label{ex:eide:13f} 
    \gll     Dåm ha villa vørri falig å vårri rundt                                 \\
         	they had will.\textsc{part} been dangerous to be.\textsc{part} around.       \\
    \glt 	‘They would have been dangerous to be around.’ ({{coon\_valley\_WI\_20gm}}, rec. year missing)\\
    \ex \label{ex:eide:13g} 
    \gll Men det e spås te regne {i mårrå} sie rømm.          \\
         but it is supposed to rain tomorrow say them       \\
    \glt ‘But it is supposed to rain tomorrow, they say.’ (coon\_valley\_WI\_06gm, rec. 2010)\\
    \z % you might need an extra \z if this is the last of several subexamples
\z

Finite and non-finite modals are attested in CANS and occur with both deontic (or root) and epistemic (and evidential) readings, where the deontic reading is e.g. social obligation or permission (\textit{John must/may v}), while the epistemic reading concerns the speaker’s evaluation of the truth of a proposition (\textit{It must/may be the case that John is v-ing}). Non-finite modals favor root readings, and by traditional assumption non-finite modals never denote epistemic readings, but cf. \citet{Eide2011Modals} for massive counterevidence from both Germanic and Romance.

Against this background, it is quite intriguing that although examples (\ref{ex:eide:13}a–c) are clearly deontic, examples (\ref{ex:eide:13}d--f) are all more natural on an epistemic reading. In Norwegian epistemic readings of modals in the present perfect are non-standard, i.e. rarely appear in edited writing. Thus, the pattern attested in (\ref{ex:eide:13}d–f) with epistemic modals in the present perfect, is clearly transmitted orally. Finally, \REF{ex:eide:13g} features an evidential ‘hear-say’ reading of the borrowed modal \textit{spost}; this is also the non-root reading it has in AmE. 

Modals are often ambiguous between a root and a non-root reading. The root reading of \textit{spost}, ‘social obligation’ is however not possible in \REF{ex:eide:13g}, as you cannot place a social obligation on the weather. This root reading is however illustrated in \REF{ex:eide:14e}; modals with directionals never allow epistemic readings, cf.  \citet{Eide2005}. NAmNo features stacked modals (\ref{ex:eide:14}a--b) and non-verbal directionals (\ref{ex:eide:14}c--e), both clearly transmitted from EurNo. Though such examples exist, even with the borrowed modal \textit{spost} (\ref{ex:eide:14}e), the constructions in \REF{ex:eide:14} are very infrequent in the CANS corpus.

\ea%14
    \label{ex:eide:14}
    \ea \label{ex:eide:14a} 
    \gll Å ja, det skulle e nå kunna […]      \\
         oh yes that should I now can      \\
    \glt ‘Oh yes, I think I should be able to do that.’ (chetek\_WI\_01gk, rec. 1942)\\
    \ex \label{ex:eide:14b} 
    \gll …hvor villt dei kunn være.      \\
         …where would they can be      \\
    \glt ‘…where they might be.’ (fargo\_ND\_06gm, rec. 2014) \\
    \ex \label{ex:eide:14c} 
    \gll Han skulle tell Kannada, Sæsskætun    \\
         he should to Canada, Saskatoon      \\
    \glt ‘He was going to Canada, Saskatoon.’ (coon\_valley\_WI\_03gm, rec. 2010)\\
    \ex \label{ex:eide:14d} 
    \gll Da måtte han hem att.        \\
         then must he home again       \\
    \glt ‘Then he had to go back home.’ (coon\_valley\_WI\_02gm, rec. 2010)\\
    \ex\label{ex:eide:14e} 
    \gll Menn e æ spåost to æ tjærrke no om synndan men.     \\
         but I am supposed to a church now on Sunday but    \\
    \glt ‘But I am supposed to go to a church this Sunday.’  (hatton\_ND\_04gk, rec. 2010)\\
    \z % you might need an extra \z if this is the last of several subexamples
\z

Epistemic readings clearly occur less often in the corpus than deontic and dynamic readings, in line with all other languages investigated (\citealt{BiberEtAl1999, HacquardWellwood2012}), and the difference is even more pronounced in spoken corpora.\footnote{\citet[182]{Cornillie2007} found that one and the same modal (Spanish \textit{poder}) occurred with an epistemic reading in 30\% of the cases in writing, but with 10\% in the spoken corpus. The numbers vary, however, for each modal.} Epistemic readings also occur later in acquisition both for monolinguals and bilinguals, possibly related to both input and processing, since root modals are less abstract and more frequent than epistemic ones. 

‘Can’ in its different appearances, as \textit{kunne, kan}, or \textit{kudd,} frequently receives the dynamic reading ‘know how to’. This is not very surprising in this context, as speakers often talk about their Norwegian skills, and (15abc) are representative examples. Note especially the presence of the infinitival marker \textit{å} after \textit{kudd nt} in \REF{ex:eide:15c}. In many EurNo dialects an infinitival marker heading the complement of \textit{kunne} disambiguates the modal towards a dynamic reading only: ‘know how to’. Again, this non-standard feature attests to the oral transmission of NAmNo.

\ea\label{ex:eide:15}
	\ea \label{ex:eide:15a} 
	\gll Eldre broren min han kudd læsa nårst.    \\
	     older brother mine he could read Norwegian  \\
	\glt ‘My older brother, he could read Norwegian.’ (blair\_WI\_15gm, rec. 2012)\\
	\ex \label{ex:eide:15b}   
	\gll  Han kudd skriva nårskt.        \\
	      he could write Norwegian      \\
	\glt ‘He could write Norwegian.’ (blair\_WI\_15gm, rec. 2012) \\  
	\ex \label{ex:eide:15c} 
	\gll   Ja, det var ittje ein såmm kudd nt å læse.  \\
	       yes there was not one who could not to read  \\
	\glt  ‘Not a single one was unable to read.’ (gary\_MN\_01gm, rec. 2010)\\
	\ex \label{ex:eide:15d} 
	\gll Je ska bli ni og åtti ti såmmern.       \\
	     I will become nine and eichty to summer  \\
	\glt ‘I will be eighty-nine this summer.’ ({{gary\_MN\_02gk}}, rec. 2010)\\
	\ex \label{ex:eide:15e} 
	\gll De yngre vil gå mæire og mæire åver te ænngels.     \\
	     the younger will go more and more over to English  \\
	\glt ‘The younger will go over to English more and more.’ (spring\_grove\_MN\_19gm, rec. 1942)\\
	\z
\z  
      

Comparing the inventory of modals in NAmNo to EurNo, the modals \textit{burde} ‘should’ and \textit{trenge} ‘need’ are infrequent. The borrowed newcomer \textit{spost} ‘supposed to’ is only a little more frequent than e.g. \textit{burde}. To a modern EurNo speaker \textit{burde} sounds formal, and if this carries over to NAmNo, \textit{burde} is likely to be part of the written, but not spoken input. Raw numbers show that the most frequent modals in NAmNo across the CANS corpus are \textit{kunne, måtte, skulle,} and \textit{ville.} The latter two sometimes occur as future markers, likely as transfer from AmE, since modals in EurNo are never pure future markers (cf. \sectref{sec:eide:5.2}). EurNo root modals are future-projecting, i.e. point to future situations, but always come with some extra modal reading, e.g. as intention, cf. \REF{ex:eide:15d}, volition or prediction, as in \REF{ex:eide:15e}.

According to our expectations, morphological markers of modality, i.e. mood, should be very vulnerable to erosion in NAmNo. This is borne out in the case of finite mood markers, as the subjunctive is as good as extinct in NAmNo, but this was already the case in EurNo long before the immigrants left for the US. Interestingly, the non-finite morphological marker on “irrealis infinitives” is very much intact across the corpus, and the invariant auxiliary \textit{ha}, a feature of many dialects in the areas where the immigrants originated, is also intact (more on this auxiliary in \sectref{sec:eide:4.3} below). Comparing NAmNo modality markers with modern EurNo we find a range of differences, especially in relative frequency of particular markers and readings. It is tempting in many cases to ascribe these differences to the input. Features characteristic of EurNo written corpora are infrequent in NAmNo (\textit{burde}, epistemic readings of present tense \textit{skulle} (cf. \sectref{sec:eide:5.1} below), auxiliary \textit{hadde} in counterfactual pluperfects). Modal markers maintained in NAmNo attest to the oral transmission of this HL, as invariant auxiliary \textit{ha}, irrealis infinitives, and epistemic readings of non-finite modals are all features which hardy occur in EurNo edited texts but exist in abundance in spoken corpora.\footnote{A possible CLI effect is the fact noted in \citet{Lykke2022} that NAmNo modal \textit{må} ‘must’ tends to show up in the present tense in preterit contexts. As AmE \textit{must} has no specific past form, this may be CLI transfer from AmE.}  

\section{Aspect in NAmNo}
\label{sec:eide:4}

Aspect occupies an intermediate position of the hierarchies of borrowability and affectedness outlined as \REF{ex:eide:3} above, and whereas all types of aspect are conflated to one in \REF{ex:eide:3}, aspect can be seen to operate on three different levels \citep{Laleko2010}; lexical aspect (also called \textit{Aktionsart}), grammatical aspect (also called viewpoint aspect; \citealt{Smith1991}), and pragmatic aspect. Lexical aspect is expressed “by the inherent lexical semantics of the verb and its interaction with direct and indirect arguments and adjuncts" (\citealt{Dowty1986}, \citealt{Verkuyl1994}); cf. also \citet[63]{Montrul2016}. Grammatical aspect is expressed wither via inflectional morphology~– one rather well-studied distinction in HLs is the perfective-imperfective distinction~– or syntactically, e.g. as posture verbs. The pragmatic level of aspect concerns when one form should be chosen over another, based on contextual cues. 

\citet{Laleko2010} compared HL-speakers of Russian to a baseline of monolinguals raised in Russia. The monolinguals mastered all three levels of aspect and their complex interactions, but each of the three different levels were seemingly affected~-- potentially separately~-- in HL. \citet[66]{Montrul2016} concludes that “the degree of erosion of aspect and the structural level affected seem to be related to the level of proficiency of the HL-speakers in their language”. \citet{Laleko2010} suggests an implication between the levels in HL: Speakers with low proficiency will ignore pragmatically restricted aspectual cues and eventually start losing the morphological opposition between imperfective and perfective (cf. also \citealt{Polinsky2006,Polinsky2008}). \citet{Polinsky2006} demonstrates that heritage speakers with low proficiency in their heritage language tend to conflate the three levels and typically start to lexicalize certain verbs as perfective and others as imperfective, mostly based on their lexical aspect (\textit{Aktionsart}), a trend also observed for HL Spanish in the US (e.g. \citealt{SilvaCorvalan1994}, \citealt{Montrul2002,Montrul2009}). 

Modern EurNo has no aspectual inflections but expresses grammatical aspect via periphrastic constructions. Pragmatic-contextual cues are often important to determine the choice between stative and dynamic aspect even at the lexical level (stative \textit{være} `be' and \textit{ha} ‘have’ vs. dynamic \textit{bli} ‘become’ and \textit{få} ‘get’), and to illustrate conditions restricting aspects on the syntactic level, we investigate the present perfect in NAmNo. The present perfect is often categorized as a type of aspect (but cf. below for a different view), and the felicitous choice between the preterit and the present perfect is governed by pragmatic-contextual restrictions. We look at lexical aspect in \sectref{sec:eide:4.1}, periphrastic constructions are the topic of \sectref{sec:eide:4.2}, and we discuss the NAmNo present perfect in \sectref{sec:eide:4.3}. 

\subsection{Lexical aspect}
\label{sec:eide:4.1}

The distinction between dynamic and stative aspect (cf. Figures~\ref{fig:eide:1}, \ref{fig:eide:2}) plays quite an important part in EurNo lexical aspects, and native speakers are quite sensitive to it. AmE does not maintain this distinction to the same extent, and a potential conflation of these two aspects in NaAmNo could be due to a simplification or to transfer from AmE. 

Recent studies of HL Danish (\citealt{Kühl2018}, \citealt{KuehlPetersen2017}) and HL Swedish in the US (\citealt{LarssonTingsellAndréasson2015}, \citealt{Hasselmo2005}) suggest that such conflation is indeed taking place, and that the stative \textit{være} ‘be’ appears in many instances where the homeland variety would prefer the dynamic \textit{bli(ve)} ‘become’, although to different degrees with different tenses and predicates. HL-speakers, also those recorded in CANS, often talk about when they were born, confirmed members of the congregation, and married, all important transitions naturally expressed with dynamic \textit{bli} in the homeland varieties, and several of the previous studies investigate exactly these predicates.

\begin{figure}
\begin{tikzpicture}
	\draw (0,0) 
	      -- (2,0) node [midway, below] {¬S\textsubscript{1}} 
	      |- (4,0.5) node [pos=0.75, above=0.75cm] {S} 
	      |- (6,0) node [pos=0.5,below] {¬S\textsubscript{2}};
	\draw[thick,-{Triangle[]}] (6,0.25) -- +(2cm,0pt);
	\draw (2,0.5) circle [radius=0.75cm];
\end{tikzpicture}
\caption{Dynamic aspect}
\label{fig:eide:1}
\end{figure}

\begin{figure}
\begin{tikzpicture}
	\draw (0,0) 
	      -- (2,0) node [midway, below] {¬S\textsubscript{1}} 
	      |- (4,0.5) node [pos=0.75, above=0.75cm] {S} 
	      |- (6,0) node [pos=0.5,below] {¬S\textsubscript{2}};
	\draw[thick,-{Triangle[]}] (6,0.25) -- +(2cm,0pt);
	\draw (3,0.5) circle [radius=0.75cm];
\end{tikzpicture}
\caption{Stative aspect}
\label{fig:eide:2}
\end{figure}

CANS has many examples with dynamic \textit{bli} ‘become’, but also data seemingly confirming the transfer analysis of the studies above, e.g. \citet{Kühl2018}, who ascribes this to CLI from AmE. 

\ea%16
    \label{ex:eide:16}
    \ea  \label{ex:eide:16a}
		  
    \gll     Jeg ble fødd i nittnhunndreeinåtjuge.      \\
             I became born in 1921          \\
    \glt     ‘I was born in 1921.’   (coon\_valley\_WI\_03gm, rec. 2010)\\
    \ex \label{ex:eide:16b} 
    \gll     See jeg var født i nittenhundreogfemogtjuge.    \\
             see I was born in 1925          \\
    \glt     ‘You see, I was born in 1925.’ ({{billings\_MT\_01gm}}, rec. 2012)\\
    \ex \label{ex:eide:16c} 
    \gll      Mor vadd gifte dær i Oslo.        \\
              Mother became married there in Oslo       \\
    \glt     ‘My mother got married there, in Oslo.’ (billings\_MT\_01gm, rec. 2012)\\
    \ex \label{ex:eide:16d} 
    \gll Kussn min var gift i Hamar i domkirka.      \\
         cousin my was married in Hamar in cathedral.\textsc{def}\\
    \glt ‘My cousin was married in Hamar cathedral.’ (blair\_WI\_04gk, rec. 2010)\\
    \z % you might need an extra \z if this is the last of several subexamples
\z
 

However, \citet{EikLaanemets2021}, following up on the discussion in \citet[100--102]{Lykke2020} on what constitutes a relevant baseline for comparison with HL-speaker production, decide to compare the data in CANS, not with contemporary EurNo, but instead with a corpus of older Norwegian dialects (the \textit{LIA} corpus). Their results do show differences in use of stative vs. dynamic exponents in NAmNo with respect to the old EurNo dialects (cf. \tabref{tab:eide:from-figure:1}), but smaller than suspected. In comparing the corpora, the authors conclude that the stative-dynamic is not very much affected by CLI, at least not for \textit{være} and \textit{bli} with the relevant predicates. 

  
%%please move the includegraphics inside the {figure} environment
%%\includegraphics[width=\textwidth]{figures/Chapter6EideHeritageVolumeTMAFinalversionSeptember9KK2024-img002.png}
\begin{table}
\caption{Corpus comparison (from \citealt{EikLaanemets2021})}
\label{tab:eide:from-figure:1}
\begin{tabular}{ll *2{rr}}
\lsptoprule
 & & \multicolumn{2}{c}{CANS} & \multicolumn{2}{c}{LIA}\\\cmidrule(lr){3-4}\cmidrule(lr){5-6}
 & & \% & $N$ & \% & $N$\\\midrule
 VÆRE & \textit{født}       & 96 & 609 & 89 & 512\\
 BLI  & \textit{født}       & 4  &  27 & 11 &  63\\
 VÆRE & \textit{konfirmert} & 70 &  48 & 58 &  74\\
 BLI  & \textit{konfirmert} & 30 &  21 & 42 &  54\\
 VÆRE & \textit{gift}       & 74 & 236 & 67 & 436\\
 BLI  & \textit{gift}       & 26 &  81 & 33 & 214\\
\lspbottomrule
\end{tabular}
\end{table}

The results from \citet{EikLaanemets2021} suggest that the stative-dynamic distinction may be more pronounced in modern EurNo for the predicates in their investigation. This means that we need to proceed with caution in claiming that this distinction is eroded or omitted in NAmNo, even with other verbs and predicates, like stative \textit{ha} ‘have’ vs. dynamic \textit{få} ‘get’.  In CANS we find examples with these verbs attesting to the distinction being maintained, but also some where the use of stative aspect seems to replace the dynamic variant. 

\ea%17
    \label{ex:eide:17}
    \ea \label{ex:eide:17a} 
    \gll         Dåtter mi gifte seg og hadde fire onger.         \\
                 daughter my married \textsc{refl} and had four kids    \\
    \glt        ‘My daughter got married and had four kids.’ (blair\_WI\_01gm, rec. 2010)\\
    \ex \label{ex:eide:17b} 
    \gll        Å så da hadd i enådher \#hadd i {enn aan} ee dåtter.\\
                and so then had I another \#had I another  ee daughter\\
    \glt        ‘And then I had another daughter.’ ({{harmony\_MN\_01gk}}, rec. 2010)\\
    \ex  \label{ex:eide:17c} 
    \gll        Da fekk ho tvilling, det trengte ho itte.        \\
                then got she twin, that needed she not        \\
    \glt        ‘Then she had twins, which was the last thing she needed.’ (billings\_MT\_01gm, rec. 2012)\\
    \ex \label{ex:eide:17d} 
    \gll       Når e va små gutt I nittenhundreogtræddve vi hadde bil.    \\
         	  when I was small boy in 1930 we had car      \\
    \glt       ‘When I was a little boy, in 1930, we had a car.’ (billings\_MT\_01gm, rec. 2010)\\
    \ex \label{ex:eide:17e} 
    \gll E va kjue år før vi fækk bil.           \\
         I was 20 years before we got car        \\
    \glt ‘I was twenty years old before we had a car.’ (glasgow\_MT\_01gm, rec. 2012)\\
    \z % you might need an extra \z if this is the last of several subexamples
\z

To a modern EurNo speaker, \REF{ex:eide:17a} carries the implication that the daughter no longer has four kids (but either more or fewer than four), since the stative situation of the daughter having four kids placed in the preterit implies that the situation described no longer holds. \REF{ex:eide:17b} has the same implication, that the speaker used to have one more daughter. Employing \textit{fekk ‘got’}, \REF{ex:eide:17c} simply describes a change of state in the past, and there is no implication as to whether or not the resulting situation still holds. The same contrast is seen with the data in (17de); in \REF{ex:eide:17d} the implication is that the family owned a car only for the year 1930, whereas the most likely interpretation pragmatically is the same as in \REF{ex:eide:17e}, that the family got a car without any implication about for how long they kept it. 

\subsection{Periphrastic expressions of aspect}
\label{bkm:Ref159266133}\label{sec:eide:4.2}
EurNo employs a number of periphrastic expressions of aspect; here we will mention posture verbs (\textit{sitte og}, ‘sit and’ \textit{stå og} ‘stand and’), \textit{drive og, ‘}be v-ing’, \textit{ta og} ‘take and’, and \textit{bruke på} ‘use to’, cf. e.g. \citet{Tonne2007}. EurNo also features the construction \textit{være i ferd med å} ‘be in the process of’, seemingly belonging to a formal written register, and not found in CANS. 

  Posture verbs are not uncommon cross-linguistically and exist in several Germanic languages, e.g. Mainland Scandinavian and Dutch. Posture verbs are natural parts of the NAmNo grammar, and examples are abundant. The reading resembles the English progressive; cf. \REF{ex:eide:18}. In addition to \textit{sit} and \textit{stand}, \textit{ligge} `lie' is also a posture verb, but infrequent in CANS. 

\ea%18
    \label{ex:eide:18}
    \ea 
    \gll     Hun hadde pipa som hun satt og røyka med.    \\
             she had pipe.\textsc{def} that she sat and smoked with    \\
    \glt     ‘She had the pipe that she was smoking.’  (harmony\_MN\_01gk, rec. 2010)\\
    \ex   
    \gll {I dag} tili e satt og leste ei stund.     \\
         today early I sat and read a while      \\
    \glt ‘This morning I was reading for a while.’ (westby\_WI\_01gm, rec. 2010)\\
    \ex  
    \gll E satt dærr og stirrde på.        \\
          I sat there and stared on         \\
    \glt  ‘I was stearing at her.’ (albert\_lea\_MN\_01gk, rec. 2010)\\
    \ex 
    \gll Bussen sto og venta på åss.   \\
         bus.\textsc{def} stood and waited on us\\
    \glt ‘The bus was waiting for us.’ (gary\_MN\_02gk, rec. 2010)\\
    \ex  
    \gll Han sto og {såg på} {se sjøl} i spilin veit du.    \\
          he stood and watched himself in mirror.\textsc{def} know you \\
    \glt  ‘He was watching himself in the mirror, you know.’  (coon\_valley\_WI\_20gm, rec. year missing)\\
    \ex   
    \gll   Je stod og gråt mæssta tå tia.   \\
          I stood and cried most of time.\textsc{def}\\
    \glt  ‘I was crying most of the time.’ (dorchester\_IA\_01um, rec. 1942)\\
    \z % you might need an extra \z if this is the last of several subexamples
\z

Note that there is no requirement for the felicitous use of these constructions that the subject in fact is sitting or standing for the intended period of time, the verbs are purely ornamental in this respect, resembling other grammaticalization outcomes (cf. also \citealt{Lødrup2019}). 

Another aspectual periphrastic construction is the \textit{drive (med) og v,} also a construction reading as a pure progressive. The construction is quite frequent in NAmNo, seemingly more so than in modern EurNo, but the LIA corpus reveals that it is surprisingly frequent in the old EurNo dialects. Of the very many examples in CANS, some appear with agentive verbs, like \REF{ex:eide:19d} where the EurNo correspondent might occur in the same construction. However, the construction seems to have a wider application semantically in NAmNo than in EurNo and some of the examples, like (\ref{ex:eide:19}a--c), sound outlandish in an EurNo context. The reading of \textit{drive og v} seems too active and dynamic to match the more perceived or experienced situation with the rather non-agentive verbs and predicates in (19abc).  

\ea%19
    \label{ex:eide:19}
    \ea \label{ex:eide:19a} 
    \gll     Kanskje ho driv å så såvå.        \\
             maybe she be.at and to sleep        \\
    \glt     ‘Maybe she is sleeping.’ (blair\_WI\_07gm, rec. 2010)\\
    \ex \label{ex:eide:19b}
    \gll Å romm driv og ser på oss òg.     \\
         and they be.at and look at us too \\
    \glt ‘And they are looking at us too.’ (harmony\_MN\_02gk, rec. 2010)\\
    \ex \label{ex:eide:19c}  
    \gll Håper han driv og tenke om det da veit du.    \\
         hopes he be.at and think about it then know you\\
    \glt ‘I hope he is thinking about it, you know.’ (harmony\_MN\_02gk, rec. 2010)\\
    \ex \label{ex:eide:19d}  
    \gll  Han drev og bader mæi så itt je fekk såva. \\
          he be.at and bother me so not I got sleep  \\
    \glt  ‘He was bothering me so I could not sleep.’ (blair\_WI\_23um, rec. 1942)\\
    \z % you might need an extra \z if this is the last of several subexamples
\z

The \textit{take and v} construction with its agentive, perfective meaning seems incompatible with passives or psych-verbs. In CANS NAmNo examples exist, but not very many. 

\ea%20
    \label{ex:eide:20}
    \ea   
    \gll      Da tok han og kasta [n] ut.        \\
             Then took he and threw him out        \\
    \glt     ‘Then he just threw him out.’ (hatton\_ND\_04gk, rec. 2010)\\
    \ex    
    \gll   Åsså tåk domm og kjørdes vit veit du.      \\
           and took they and drove wheat know you    \\
    \glt   ‘And then they just transported the wheat, you know.’ (westby\_WI\_24gm, rec. 1942)\\
    \ex  
    \gll Så tok jæi og kopiærte noen a di dær replikkene. \\
         then took I and copied some of those there lines \\
    \glt ‘So I just copied some of the lines.’ (seattle\_WA\_03gm, rec. 2012)\\
    \z % you might need an extra \z if this is the last of several subexamples
\z

Finally, the construction \textit{bruke (på) å} is not very frequent in the LIA corpus in this exact shape. It clearly existed in EurNo, though exclusively in certain dialects, but crucially in those dialects spoken by the immigrants. However, its NAmNo use seems to converge on the AmE \textit{used to v} construction; both occur mostly in the preterit, and \textit{bruke (på) å} is translatable into \textit{used to v.} 

\ea%21
    \label{ex:eide:21}
    \ea \label{ex:eide:21a} 
    \gll     Du brukte på å bruke heit damp på å stime.     \\
             you used {} to  use hot steam {} to  steam (it)      \\
    \glt     ‘You used to use hot steam to steam it.’ (coon\_valley\_WI\_03gm, rec. 2010)\\
    \ex  \label{ex:eide:21b} 
    \gll  Vi brukte på å reise på mange danser. \\
          we used {} to travel to many dances      \\
    \glt  ‘We used to go to a lot of dances.’ (coon\_valley\_WI\_07gk, rec. 2010)\\
    \ex \label{ex:eide:21c}  
    \gll  Vi brukte gå på søttendemaiparade.      \\
          we used go to {17th of May parade}      \\
    \glt  ‘We used to go to a 17th of May parade.’ (chicago\_IL\_01gk, rec. 2010)\\
    \ex \label{ex:eide:21d} 
    \gll Nårth ævvenu og Kælefornja brukte være nårskt.  \\
         North avenue and California {used to} be Norwegian  \\
    \glt ‘North avenue and California used to be Norwegian.’ (chicago\_IL\_01gk, rec. 2010)\\
    \z % you might need an extra \z if this is the last of several subexamples
\z    

This construction seemingly has a wider application than in modern EurNo, especially (21ad) sound coerced in EurNo. The AmE \textit{used to v}{}-construction has as its core meaning that the situation described no longer holds, whereas the EurNo construction focuses on the habituality of the situation. This explains why (21ad) sound strange to an EurNo ear. In \REF{ex:eide:21a} the two instances of \textit{bruke} sound superfluous, whereas replacing the first instance \textit{brukte på} with \textit{pleide} ‘used to’ sounds more idiomatic. In contrast, \textit{They used to use steam} sounds less superfluous; the first instance of \textit{use} denotes ‘detached past’ (\textit{v} no longer holds), the second is the lexical verb (‘make use of’). In \REF{ex:eide:21d} a direct translation into AmE seems idiomatic, as suggested by the gloss and translation, but in EurNo the habituality of \textit{bruke} makes it incompatible with the stative situation of being Norwegian. This habituality is evidently not a necessary part of the reading in NAmNo, which suggests CLI from English – transfer of the semantics or convergence on the common parts of meaning. 

\subsection{The present perfect: Remote and immediate past}
\label{sec:eide:4.3}
There is a longstanding debate in the linguistics literature as to whether the present perfect is in fact a type of aspect or a type of tense. Here \citet{GrønnvonStechow2020}:

\begin{quote}
The perfect is one of the most complex constructions in temporal semantics. It shares properties with both tense and aspect, which is a source of constant confusion. It is therefore not easy to characterize its meaning in a few words.
\end{quote}

This echoes \citegen[269]{Jespersen1924} words: “The perfect cannot be fitted into the simple series, because besides the purely temporal element it contains the element of result”, cf. \citet{Comrie2020}, \citet{Dahl2000} and \citet{EideFryd2021} for similar descriptions. \citeauthor{Eide2002} (\citeyear{Eide2002} and subseq.) classifies the present perfect in Mainland Scandinavian as a compositional tense consisting in a stative auxiliary (\textit{har} ‘have’) encoding a present or future state and a supine (past participle) encoding an event in the past, i.e. in the past with respect to the stative auxiliary. The conceptual relation between the past event and the stative auxiliary is perceived as causation, the event caused the subject’s present state. We see very clearly in examples like \textit{Har du mistet brillene dine? ‘}Have you lost your glasses?’ An affirmative answer strongly implies that you are still in the state (e.g. of poor eyesight) caused by the past event (losing your glasses). Uncancelled, the present perfect implies that the state still holds. These aspectual features often ascribed to the present perfect are due to the stative auxiliary (cf. \citealt{Eide2020Modality, EideFryd2021}). 

Whether we classify the present perfect as an aspect or a tense, it competes with the preterit for functional domains. Relevant HL-studies (e.g. \citealt{Montrul2016, Fenyvesi2000}) suggest that the challenge with tense in HL production is the mapping of the correct tense forms to the right context; i.e. once again contextual restrictions on members in an opposition pair. 

In EurNo the present perfect is felicitous when it is used to refer to the “immediate past”, i.e. within the time span of “the current cycle” (cf. \citealt{Eide2020Modality}, \citealt{EideFryd2021}). The current cycle is very often salient and not explicit, like in \textit{Har du spist}? ‘Have you eaten?’ where the interesting point is whether or not you are hungry, that is, have you had a meal within the relevant cycle (here, the cycle consisting of usual mealtimes). Salient cycles are days, weeks, months, decades, centuries; and the present perfect is felicitous within the current cycle \REF{ex:eide:22a}, but situations temporally linked to the previous cycle as in \REF{ex:eide:22b} require the simple past, i.e. the preterit, referred to as “the remote past” in \citet{Eide2020Modality} and \citet{EideFryd2021}. 

\ea%22
    \label{ex:eide:22}
    \ea[]{\label{ex:eide:22a}
    \gll Jeg har sett ham idag / denne uka / denne måneden.\\
          I have seen him today {} this week {} this month\\
    \glt  ‘I have seen him today/this week/this month.’ }
    \ex[*]{\label{ex:eide:22b}
    \gll  Jeg har sett ham igår / forrige uke / forrige måned. \\
              I have seen him yesterday {} last week {} last month\\
    \glt      ‘I have seen him yesterday/last week/last month.’}
    \z % you might need an extra \z if this is the last of several subexamples
\z

The present perfect is quite frequent in CANS, and the examples are mostly similar in their use to EurNo. The quite complex pragmatic-syntactic restrictions discussed above lead us to expect CLI-effects, according to current consensus. But it seems that the restrictions for AmE present perfects are quite similar to the EurNo ones, since discrepancies are few – or the construction is still robust enough in the input and the grammar of NAmNo to adhere to the restrictions relevant also in the earliest variants of NAmNo varieties. The examples in \REF{ex:eide:23} mirror the use of present perfects in present-day EurNo (with certain irrelevant deviations).

\ea%23
    \label{ex:eide:23}
    \ea \label{ex:eide:23a} 
    \gll     Hun har snakka mer nårskt {i dag}    {ell   o a} gjort i mange år.     \\
             she has spoken more Norwegian today  {than she has} done   in many years\\
    \glt     ‘She has spoken more Norwegian today than she has done for many years.’ (coon\_valley\_WI\_06gm, rec. 2010)\\
    \ex \label{ex:eide:23b} 
    \gll Je har levd hær siden nittenhundreogsækksogsæksti. \\
         I have lived here since 1966          \\
    \glt ‘I have lived here since 1966.’ (sunburg\_MN\_03gm, rec. 2011) \\
    \ex \label{ex:eide:23c}  
    \gll  Men je tru je a lært i skolen at det var mange.      \\
         but I think I have learned in school.\textsc{def} that there were many \\
    \glt ‘I think I learned in school that there were many.’ (kalispell\_MT\_02uk, rec. 2012)\\
    \ex \label{ex:eide:23d}  
    \gll Det er noen ongdommer som har reister til Nårrge.    \\
         there are some youngsters who have travel to Norway    \\
    \glt ‘There were some young people who travelled to Norway.’ (outlook\_SK\_08uk, rec. 2013)\\
    \ex \label{ex:eide:23e} 
    \gll Je ha gjort høgskolen tolv år.        \\
         I have done {high school} twelve years. \\
    \glt ‘I went to high school, twelwe years.’ (blair\_WI\_01gm, rec. 2010)\\
    \ex \label{ex:eide:23f}
    \gll Je trur je ha vøri dær ei vikus anndre gånnga.   \\
         I think I have ben there one week second time    \\
    \glt ‘I think I was there for a week the second time.’  (blair\_WI\_01gm, rec. 2010)\\
    \z % you might need an extra \z if this is the last of several subexamples
\z

In EurNo, examples (\ref{ex:eide:23}c--f) seem equally natural with the preterit, in line with the AmE translations. However, though the felicitous use of the present perfect is restricted by the \textit{cycles} and \textit{present relevance} effect, the language user very often has a choice between the preterit, the present, and the present perfect, depending on contextually given information or \textit{common ground}, and there are surprisingly few examples in running text where a speaker of EurNo could say with absolute confidence that this is not a possible context for the present perfect in our variety. In the CANS examples in (\ref{ex:eide:24}a--b), however in EurNo the present is more felicitous than the present perfect (\ref{ex:eide:24}c--d). Furthermore, speakers of EurNo would replace preterit with the present perfect.

\ea%24
    \label{ex:eide:24}
    \ea  
    \gll Men det har vært lenge siden   jæ har lesst bøker på nårsjk.      \\
         but is  has been {long ago} since I have read books in Norwegian  \\
    \glt ‘It has been a long time since I have read books in Norwegian.’ (outlook\_SK\_08uk, rec. 2013)\\
    \ex  
    \gll Nå har det vært tolv år sidn jæ bynnte dær.        \\
         now it has been twelve years since I started there \\
    \glt ‘It was twelve years ago that I started there.’ (minneapolis\_MN\_01uk, rec. 2012)\\
    \ex 
    \gll Ja, mange […] ting som kamm framm ibei.   \\
         yes many  {}  things that came from ebay       \\
    \glt ‘Yes, lots of things that came from ebay.’  (albert\_lea\_MN\_01gk, rec. 2010)\\
    \ex  
    \gll Jei våkkst åpp her så var femeljen alltid hær.     \\
         I grew up here so was family.\textsc{def} always here      \\
    \glt ‘I grew up here, so my family was always here.’ (spokane\_WA\_04uk, rec. 2012)\\
    \ex  
    \gll Hå mange gonger var du te Nårje sier du?        \\
         how many times were you to Norway say you      \\
    \glt ‘How many times did you go to Norway, did you say?’  ({blair\_WI\_02gm}, rec. 2010)\\   
    \z % you might need an extra \z if this is the last of several subexamples
\z

The last example especially reveals the contextual restrictions. The implication of the preterit here is that the subject is now incapable of taking more trips to Norway, that your Norway-visiting days are over.  In contrast, \textit{Hå mange gonger har du vært te Nårje?} asks for the present state of the interlocutor, the experience of having been to Norway a specific number of times. 

As mentioned in the context of counterfactual plusperfects in \sectref{bkm:Ref159263343}. above, many EurNo dialects spoken in the areas where immigrants had their origin share a particular quirk relevant also for the present perfect. As opposed to written standards, where auxiliary \textit{ha} and main verb \textit{ha} share their inflection, in these dialects the auxiliary occurs in a non-variant form (\textit{ha/he}) which serves as the infinitive, present and preterit (i.e. there is systematic ambiguity between the present perfect and the pluperfect). The lexical verb \textit{ha,} in contrast, occurs with an -\textit{r} affix in the present tense (and as \textit{hadde} in the preterit). Studying specific speakers, we find that this split is still observable and, in some sense, robust in NAmNo. Speaker albert\_lea\_MN\_01gk, recorded in 2010, has rather consistently (and almost categorically) lexical \textit{ha} with present tense affix -\textit{r} (\ref{ex:eide:25}a--b), but nonvariant auxiliary \textit{ha} (\ref{ex:eide:25}c--d).\footnote{This example is annotated as ‘har’ in the corpus, but to my ear there is no \textit{-r} affix.} 

\ea%25
    \label{ex:eide:25}
    \ea  
    \gll   E har hunndre å sekksti aker åg land i NorDekota\\
           I have hundred and sixty acers of land in {North Dakota}      \\
    \glt  ‘I have one hundred and sixty acers of land in North Dakota.’ 
    \ex  
    \gll  De har kvæite sannflaors ee såi bins\\
          they have whete sunflowers and soy beans\\
    \glt ‘They have whete, sunflowers and soy beans.’
    \ex  
    \gll  Ho ha tjøpi bok fårr me te læsa. \\
          she has bought book for me to read\\
    \glt  ‘She has bought a book for me to read.’
    \ex 
    \gll  Ha du høyrt om de golden pærasjut?\\
          have you heard of the golden parachute\\
    \glt  ‘Have you heard of the golden parachute?’
    \z % you might need an extra \z if this is the last of several subexamples
\z

The opposition between lexical \textit{ha} and invariant auxiliary \textit{ha} is again an opposition not receiving any support from written input or standard variants. Hence, the opposition must be transmitted orally, and after several generations of AmNo (the speaker is fourth generation HL speaker) the distinction is still robustly in place with certain speakers. 

Summing up some findings in the aspectual domain, lexical aspect may be affected, but maybe less than previous studies have suggested (e.g. \citealt{Kühl2018}), as the tendency to conflate stative and dynamic aspect is present even in older EurNo dialects (cf. \citealt{EikLaanemets2021}). Periphrastic constructions are abundant in CANS, especially (certain) posture verbs, the \textit{brukte på} ‘used to’ construction seems to have extended its use, which seems like a CLI-effect, and the present perfect is used in a manner very much resembling EurNo use. Even for rather quirky constructions, oppositions are maintained (e.g. invariant auxiliary \textit{ha} ‘have’ vs. lexical inflected \textit{ha} ‘have’), and as in the rest of CANS, we find overwhelming evidence that NAmNo is orally transmitted, featuring dialectal traits quite different from any written EurNo standard.  

\section{Tense and finiteness in NAmNo}
\label{sec:eide:5}

\citet[62]{Montrul2016} notes that in HL, “tense and agreement morphology tend to be better preserved than aspect and mood” and that “In general, there are few if any reports of errors with tense in heritage grammars \citep{Fenyvesi2000}”. Also, Polinsky (2018, chapter 5) discusses that although morphology is seen as vulnerable to erosion in language contact, tense as a morphological subdomain has proved to be highly resistant to change. 

\citet{Lykke2020, Lykke2022} and \citet{NatvigEtAl2023} argue that this is also true for NAmNo – for the most part of its history. \citet[78]{Lykke2022} states that “the change found in present-day AmNo has arisen in the present generation”.\footnote{\citet[78]{Lykke2022}: This was also argued in \citet{Lykke2020}, but “with a minimum of empirical substantiation”.}  In contrast, the recent investigation \citet{EideHjelde2023} concludes that the verbal paradigms in NAmNo in two selected settlements (Coon Valley and Blair) change throughout different times, affected by the amount of written input available at different times. The discrepancy between the two views is less that it may seem. Firstly, Eide and Hjelde’s study (2023) addresses whole paradigms, where even non-finite forms count as tensed, focusing on the most productive class at intermediate stages and periods. \citegen{Lykke2022} focus is the felicitous use of finite forms in specific contexts, with a special emphasis on modal \textit{må} ‘must’ and lexical verb \textit{gå} ‘walk’ complemented by other verbs of all classes. Thus, the two views can be reconciled to complement rather than contradict each other.\largerpage

Tense obviously belongs in an investigation of TMA-systems in HL, but it may seem more controversial to include finiteness. The finiteness feature rarely figures in HL studies; instead subject-verb agreement is frequently used as the visible counterpart of finiteness in natural languages, including HLs.\footnote{\citet{Eide2016} shows why neither tense nor agreement, by themselves or in combination, equal finiteness, though these three often travel together. Instead, finiteness is a primitive feature associated with the speech act.} It is also not uncommon in the traditional literature to treat agreement as a visible marker for indicative mood (cf. e.g. \citealt{Amritavalli2014}: 293). Moreover, as subject-verb agreement very often pairs up with the tense feature, the two are very often treated together in the literature on the topic. On wide-spread assumption the mere occurrence of visible subjects, especially when showing target-like case markings, is taken to signal the presence of tense features in the clause, even where no tense feature is explicitly expressed.\footnote{\citet[1]{Lardiere1998} shows how an adult Chinese learner of English, despite consistently supplying tense markings at a low rate (34\%), shows “perfect distribution of pronominal case (100\%) in all contexts, suggesting the presence of a TP bearing a fully specified [± finite] feature”.}  Subject-verb agreement has been at the focus of attention in TMA-investigations on language contact generally and HL-linguistics specifically, presumably because of its salience in overt morphology \citep[62]{Montrul2016}. However, subject-verb agreement is of little relevance to an investigation of NAmNo, as no NAmNo baseline features this trait. Still, Mainland Scandinavian languages clearly employ finiteness distinctions; finite verbs are systematically different from non-finite ones morphologically, in the total absence of subject-verb agreement.

\subsection{Tense paradigms and the finiteness distinction in NAmNo}
\label{sec:eide:5.1}

\citet{Eide2002, Eide2005, Eide2009Finiteness, Eide2009Tense} and \citet{EideHjelde2015Verb, EideHjelde2023} argue that EurNo verb forms adhere to the productive paradigm in \tabref{tab:eide:1}, consisting in a four-way taxonomy of tensed verb forms where two tenses are finite and two are non-finite. Furthermore, each verb form is specified as past or non-past, thus each verb form in the paradigm has a different feature matrix ±finite and ±past. Committing to this paradigm, it makes little sense to treat tense morphology with no reference to the finiteness feature.

\begin{table}[h]
\begin{subtable}{.5\textwidth}
\centering
\caption{}\label{tab:eide:1a}
\begin{tabular}{lll}
\lsptoprule
& +finite & −finite\\
\midrule
+past & Preterit:        & Participle: \\
      & \textit{kleima}  & \textit{kleima}\\
−past & Present:         & Infinitive: \\
      & \textit{kleimar} & \textit{kleima}\\
\lspbottomrule
\end{tabular}
\end{subtable}%
\begin{subtable}{.5\textwidth}
\centering
\caption{}\label{tab:eide:1b}
\begin{tabular}{lll}
\lsptoprule
& +finite & −finite\\
\midrule
+past & Preterit:         & Participle: \\
      & \textit{kleimet}  & \textit{kleimet}\\
−past & Present:          & Infinitive:\\
      &  \textit{kleimer} & \textit{kleime}\\
\lspbottomrule
\end{tabular}
\end{subtable}
\caption{Verb forms in Nynorsk and Bokmål (1907), NAmNo in the 1940s.}
\label{tab:eide:1}
\end{table}

The paradigm is exemplified here by the borrowed verb \textit{claim}, adapted as \textit{kleima} in NAmNo, a deliberate choice in that borrowed verbs show the contemporary productive inflections, in the paradigm of the \textit{kasta}{}-class, by far the most productive class for loan verbs in NAmNo.\footnote{\citet[82]{EideHjelde2015Verb} state that 94\% of loan verbs investigated for this study are assigned to this class, remarkably stable as compared to Haugen’s count of 93\%.}  The contemporary \textit{Nynorsk} paradigm in \tabref{tab:eide:1a} is provided as it is the explicit baseline for \citegen{Haugen1953} investigations of productive verb conjugations in NAmNo, but we also provide the \textit{Bokmål} paradigm from the 1907 standard in \tabref{tab:eide:1b}. \textit{Nynorsk} and \textit{Bokmål} were both present in the written input of the NAmNo speakers, but the latter occurred in much larger quantities, serving as the written norm for most NAmNo speakers at the time. 

According to Haugen’s discussion of NAmNo verb forms in the 1940s, the distinction ±finite is retained in the non-past forms (infinitive \textit{kleima} versus present \textit{kleimar}), but absent from the +past forms: the preterit and the past participle share the form \textit{kleima} (cf. \tabref{tab:eide:1a}). Haugen’s paradigm for \textit{Nynorsk} shows syncretism between all forms except the present. \textit{Bokmål} in the 1907 standard shows distinct forms for infinitive and present, but non-distinct forms for the +past cells; cf. \tabref{tab:eide:1b}. 

Bilingual production is characterized by variation. NAmNo is no exception, not surprisingly, as there is also a lot of variation in the input. Although each settlement tends to be dominated by a particular dialect, there are always other dialects in the area to provide ample “conflicting” input, in addition to the archaic standard used in church texts and the Bible, the different standards occurring in the Norwegian newspapers of the area, and the constant pressure from the dominant language delivering opportunities for cross-linguistic influence in constant supply. Thus, just looking at the infinitive of \textit{fortelle} ‘tell’, we find many different variants even within the small settlement of Blair in the 1942 recordings (\textit{fårtella, fåtella, fåtelja, fortæl, værtellja}). Other variants of this infinitive in NAmNo are \textit{tella, tælja, tæl, tel.} What is interesting here is the infinitival inflection, which varies between null, \textit{-a}, and -\textit{e} (\ref{ex:eide:26}a--c), not only between speakers, but also with one and the same speaker (cf. \ref{ex:eide:26}c--d)

\ea%26
    \label{ex:eide:26}
    \ea  
    \gll Åh, je ska fortæl di jeg var n liten smågutt      \\
         oh I will tell you I was a little toddler\\
    \glt ‘Oh, I will tell you, I was a little toddler.’ (blair\_WI\_24um, rec. 1942)\\
    \ex  
    \gll Det va hardt det, ska jeg fårtella re.  \\
         that was hard {} shall I tell you      \\
    \glt ‘That was hard, I can tell you.’ (blair\_WI\_34gm, rec. 1942)\\
    \ex  
    \gll Væll, e kann da føtællje        \\
         well I can \textsc{ptl} tell          \\
    \glt ‘Well I guess I could tell you…’ (blair\_WI\_17gm, rec. 1942)\\
    \ex  
    \gll De æ ein tinng såmm je må fårtælja  \\
         there is one thing that I must tell \\
    \glt ‘There’s one thing I have to tell’ (blair\_WI\_17gm, rec. 1942)\\
    \z % you might need an extra \z if this is the last of several subexamples
\z

\citet{EideHjelde2023} show how the verbal paradigms for the \textit{kasta}{}-class change in NAmNo in two selected settlements (specifically Coon Valley and Blair) throughout different times, on our analysis affected by the amount of written input from Norwegian newspapers, parochial schools and confirmation preparation education. In keeping with the discussion on baselines in \sectref{sec:eide:2}, we chose to separate the speakers in the available recordings into five cohorts or idealized generations to detect diachronic changes.\footnote{In HL-research the more common way to differentiate between groups of speakers is to refer to the generation they belong to e.g. typically 3\textsuperscript{rd} generation for this investigation. In placing such a heavy burden on input, it is more interesting to us to know how much and what types of written Norwegian they were exposed to. That means birth year is at least as interesting as how many generations have passed to produce this specific speaker.}   Note that the cohorts do not cover time spans of equal length, anticipating more changes for the more recent cohorts. For the first, mostly stable decades, we allowed for wider time spans for each cohort. 

\begin{table}[h]
\caption{Breakdown of the cohorts of NAmNo speakers and datasets \citep{EideHjelde2023}}
\label{tab:eide:cohorts}
\begin{tabularx}{\textwidth}{l *5{Q}}
\lsptoprule
Cohort & I & II & III & IV & V\\\midrule
Born & Around 1870 & 1900--1920 & 1920--1930 & 1940--1950 & after 1950\\
Data sets & Haugen 1940s & Haugen 1940s & Hjelde 1980s \& 1990s & CANS Eide\slash Hjelde 2010 & CANS Hjelde 2010--2018\\
\lspbottomrule
\end{tabularx}
\end{table}

Especially speakers belonging to Cohorts II and III will be subjected to much more written NAmNo input than later generations, and the written standards clearly have a normative effect. We see this in particular with the occurrences of the present tense -\textit{r,} since unlike the two written standards, the paradigms of the two dialects spoken in Blair at the time do not feature an -\textit{r} inflection in the present tense. However, for a period of time, the present tense -\textit{r} of the written standard(s) occurs also in the spoken production of the relevant speakers. Selecting e.g. a speaker from cohort III, billings\_MT\_01gm (born 1925, 2nd generation immigrant), we observe variation, but the speaker clearly uses a lot of -\textit{r} suffixes in the present tense. 

\ea%27
    \label{ex:eide:27}
    \ea  
    \gll  Nå {e re} itte nånn te snakke nårsjk med så glømmer det bort.  \\
          now there is nobody to talk Norwegian to so forget it (away)   \\
    \glt ‘Now there’s nobody to talk Norwegian to, so I forget it.’
    \ex  
    \gll Du {kan itte} bruke æit ord ti seie på enngelsk tia du snækker på nåsjk\\
          you cannot use a word to say in English, time you speak in Norwegian\\
    \glt ‘You cannot speak a word in English when you are speaking Norwegian.’
    \ex  
    \gll  Det er sju som lever enda. \\
          there are seven who live still\\
    \glt ‘There are seven who are still alive.’
    \ex  
    \gll Men e ønnsker att jeg hadde lerrd {te å} læsa det. \\
         but I wish that I had learned to read it\\
    \glt ‘But I wish that I had learned to read it.’
    \z % you might need an extra \z if this is the last of several subexamples
\z

After a period where the written paradigms were on the rise, sources providing written input subsided, and the spoken variants once again took foothold over the NAmNo grammar (cf. \citealt{EideHjelde2023}). There were several competing systems, but the paradigm eventually prevailing in speaker production is the system in \tabref{tab:eide:3a}, converging on the English paradigm with respect to the distinctions made \tabref{tab:eide:3b}. The distinction ±past is retained, ±finite is lost, like the productive paradigm for English (cf. \citealt{Eide2016} for discussion). 

\begin{table}[h]
\begin{subtable}{.5\textwidth}\centering
	\caption{NAmNo}\label{tab:eide:3a}
	\begin{tabular}{lll}
	\lsptoprule
	      &	+finite         &  −finite \\\midrule
	+past & Preterit        & Participle	  \\
	      & \textit{kleima} & \textit{kleima} \\
	−past & Present         & Infinitive      \\
	      & \textit{kleime} & \textit{kleime}\\
	\lspbottomrule
	\end{tabular}
\end{subtable}\begin{subtable}{.5\textwidth}\centering
	\caption{English}\label{tab:eide:3b}
	\begin{tabular}{lll}
	\lsptoprule
	      &	+finite         &  −finite \\\midrule
	+past & Preterit        & Participle	  \\
	      & \textit{claimed} & \textit{claimed} \\
	−past & Present         & Infinitive      \\
	      & \textit{claim} & \textit{claim}\\
	\lspbottomrule
	\end{tabular}
\end{subtable}
\caption{Prevailing productive paradigms for NAmNo and English}
\end{table}

This does not necessarily mean that the present tense \textit{-r} is suddenly completely absent from the data, but it no longer necessarily serves to code the present tense in opposition to the infinitive. Take speaker albert\_lea\_MN\_01gk, born in 1925 and 3\textsuperscript{rd} /4\textsuperscript{th} generation NAmNo speaker, belonging to the same cohort as the speaker in \REF{ex:eide:27}, cohort III. Collecting all forms of the verb \textit{snakke} ‘talk’ in her CANS data we can compile the paradigm in \tabref{tab:eide:4a} and compare this to the two dominating dialects in the area in the 1940s (5b: Southern system; 5c: Northern system; cf. \citealt{EideHjelde2023}). Note that this speaker uses the \textit{{}-r} to code the participle, but not the present, example given in \REF{ex:eide:28a}, another example from a different speaker is given in \REF{ex:eide:28b}. 


\begin{table}[h]
\begin{subtable}{.5\textwidth}\centering
\caption{}\label{tab:eide:4a}
\begin{tabular}{lll}
\lsptoprule
      & +finite & −finite\\\midrule    
+past & Preterit    & Participle \\
      & \textit{snakka (4x)} & \textit{snakkar}    \\
      & \textit{snakk (2x)}  &            \\
−past & Present     & Infinitive \\
      & \textit{snakka (2x)} & \textit{snakke (4x)} \\
      & \textit{snakke}      & \textit{snakk}      \\
\lspbottomrule
\end{tabular}
\end{subtable}\medskip\\
\begin{subtable}{.5\textwidth}\centering
\caption{}\label{tab:eide:4b}
\begin{tabular}{lll}
\lsptoprule
      & +finite & −finite\\\midrule   
+past & Preterit    & Participle \\
      & \textit{snakke}      & \textit{snakke} \\
−past & Present     & Infinitive\\
      & \textit{snakker}     & \textit{snakke} \\
\lspbottomrule
\end{tabular}
\end{subtable}\begin{subtable}{.5\textwidth}\centering
\caption{}\label{tab:eide:4c}
\begin{tabular}{lll}
\lsptoprule
      & +finite & −finite\\\midrule   
+past & Preterit    & Participle \\
      & \textit{snakka}      & \textit{snakka} \\
−past & Present     & Infinitive\\
      & \textit{snakke}      & \textit{snakke} \\
\lspbottomrule
\end{tabular}
\end{subtable}
\caption{Paradigm of \textit{snakke} ‘talk’ in the CANS production of speaker albert\_lea\_MN\_01gk, compared to two dialects spoken in the area in the 1940s.}
\label{tab:eide:4}
\end{table}

\ea%28
    \label{ex:eide:28}
    \ea \label{ex:eide:28a}  
    \gll E he kji snakkar nårsjk for menni år.        \\
         I have not speak.\textsc{pres} Norwegian for many years    \\
    \glt ‘I haven’t spoken Norwegian for many years.’ (albert\_lea\_MN\_01gk, rec. 2010)\\
    \ex \label{ex:eide:28b} 
    \gll Det er noen ongdommer som har reister til Nårrge.    \\
         There are some youngsters who have travel to Norway    \\
    \glt ‘There were some young people who travelled to Norway.’ (outlook\_SK\_08uk, rec. 2013)\\  
    \z % you might need an extra \z if this is the last of several subexamples
\z

\begin{sloppypar}
In addition to the significant inter-speaker and intra-speaker variation, the paradigms of the local dialects also dictate syncretism between forms. The Northern system has homophony between present and infinitive forms, cf. (\ref{ex:eide:29}a--b) and between the participle and the preterit, cf. (\ref{ex:eide:29}c--d).  
\end{sloppypar}

\ea%29
    \label{ex:eide:29}
    \ea  
    \gll Nå ræise rømm ått Nåri.\\
         now travel they to Norway          \\
    \glt ‘Now they travel to Norway’ ({{westby\_WI\_03gk}}, rec. 2010)\\
    \ex  
    \gll Du må ræise ned bLåffa førr å komma åt Konn Valle     \\
         you must travel down bluff.\textsc{def} {in order} to come to Coon Valley   \\
    \glt ‘You have to go down the bluff to get to Coon Valley.’ ({{westby\_WI\_03gk}}, rec. 2010)\\
    \ex  
    \gll De gammleste romm hadde ræist borrt allteræie       \\
         the oldest they had traveled away already      \\
    \glt ‘The oldest ones had left already.’ ({{westby\_WI\_03gk}}, rec. 2010)\\
    \ex  
    \gll Å summe ræist åt Læ Krås.                \\
         and some travelled to La Crosse\\
    \glt ‘And some travelled to La Crosse.’ ({{westby\_WI\_01gm}}, rec. 2010)\\
    \z % you might need an extra \z if this is the last of several subexamples
\z
        


In the more recent recordings, we also find verb forms showing up as homophonous where the finite and non-finite forms used to be different; infinitives in present contexts, e.g. in (\ref{ex:eide:30}a--b, expected forms: \textit{søv}, \textit{veks}), and preterits as participles (\ref{ex:eide:30}c--d, expected forms \textit{veksi, fått}).   

\ea%30
    \label{ex:eide:30}
    \ea  
    \gll Kanskje ho driv å så såvå.          \\
         maybe she be.at and to sleep          \\
    \glt ‘Maybe she is sleeping.’ (blair\_WI\_07gm, rec. 2010)\\
    \ex  
    \gll Mi vekksa grønnsake i garden.            \\
         we grow.\textsc{inf} vegetables in yeard.\textsc{def}        \\
    \glt ‘We grow/grew vegetables in the yard.’ (albert\_lea\_MN\_01gk, rec. 2010)\\
    \ex  
    \gll Je hadde vaks opp med rømmegrøt.\\
         I had grew up with {sour cream porridge}\\
    \glt ‘I grew up with \textit{rømmegrøt}.’  ({{blair\_WI\_04gk}}, rec. 2010)\\
    \ex  
    \gll Han laga renn je hadde fekk ja.                     \\
         he made that I had got.\textsc{pret} yes        \\
    \glt ‘He made the one I had got.’ (harmony\_MN\_02gk, rec. 2010)\\
    \z % you might need an extra \z if this is the last of several subexamples
\z

We also find more unexpected mix-ups, e.g. infinitive forms where we expect participles (\ref{ex:eide:31}a--b), or preterits where we expect infinitives, after modals and infinitival markers, cf. (\ref{ex:eide:31}c--d).\footnote{This could also be a preterit form, as is the case in many EurNo dialects.}\footnote{This is likely to be cross-linguistic influence from German, infintive \textit{verstehen.}}          

\ea\label{ex:eide:31}
	\ea 
	\gll Men domm ha dø  nå.\\
		 but hey have die.\textsc{inf} now\\
   \glt ‘But they are dead now.’ (blair\_WI\_12gm, rec. 2010)\\
	\ex 
	\gll Jei ville ha aldri møte dæi.\\
		 I would have never meet you\\
	\glt ‘I would have never met you.’  (chicago\_IL\_01gk, rec. 2010)\\
	\ex 
	\gll Så vi ska bisøkkte hann.\\
        so we shall visited him \\    
	\glt ‘So we are going to visit him.’ (coon\_valley\_WI\_12gm, rec. 2012)\\
	\ex 
	\gll Vi ha ællrin kommi ti å {fårstin– fårsto} nå nåssjt.\\
		 we have never come {} to understand.\textsc{part/pret} any Norwegian\\
    \glt ‘We would never have understood or spoken any Norwegian.’ ({{coon\_valley\_WI\_07gk}}, rec. 2010)\\
	\z
\z

Despite the massive syncretism and a lot of unexpected forms in the more recent recordings, it still seems that the finiteness distinction is less present in NAmNo than the tense distinction. \citet{EideHjelde2015Verb} explain this as a possible CLI-effect (although it also receives support from the most dominant dialect), since English has no finiteness distinction within the paradigm for main verbs (cf. \tabref{tab:eide:3b}). However, there is a lot of variation, not only across the finiteness distinction, but also across the tense distinction, and there is a lot of inter- and intra-individual variation in the realizations of the different forms in the paradigm (cf. \tabref{tab:eide:4a}). 

\subsection{Striving for transparency}\label{sec:eide:5.2}

Future tense in EurNo expresses future via the present form of the main verb as a default. Second language learners of EurNo have a tendency to express future with a modal instead, e.g. \textit{skulle} ‘shall’ or \textit{ville} ‘want to’. This is also the case in NAmNo, and it is not very farfetched to see this as a CLI-effect from AmE. Moreover, recall that there is a tendency for HL-speakers to replace morphology with analytic forms and to strive for transparency, and the temporal ambiguity residing in the present form denoting either future or simultaneity, may drive this simplification. Reserving the present form for simultaneity and using a modal for future achieves both to reduce ambiguity, to increase transparency, and allows for two patterns to converge (adding a future auxiliary as in AmE, but using \textit{skulle}, not \textit{ville}, as the chosen modal). 

\ea%32
    \label{ex:eide:32}
    \ea \label{ex:eide:32a} 
    \gll O ska være hunndre år gammal omm onnsdan.    \\
         she shall be hundred years old on Wednesday    \\
    \glt ‘She will be 100 years on Wednesday.’ 
    ({{coon\_valley\_WI\_08gm}}, rec. 2011)\\
         \textit{Bokmål: Hun blir hundre år gammel på onsdag.} 
    \ex \label{ex:eide:32b} 
    \gll Onngeste o ska bLi førti hun nå om onsdagen\\
         youngest she shall become forty she now on Wednesday   \\
    \glt ‘The youngest will be forty years old on Wednesday.’ ({{coon\_valley\_WI\_07gk}}, rec. 2010)\\
         \textit{Bokmål: Den yngste blir førti år, hun, nå på onsdag.} 
    \ex \label{ex:eide:32c} 
    \gll Je ska bli ni og åtti ti såmmern.                            \\
         I shall become nine and eighty to summer.\textsc{def}     \\
    \glt ‘I will be eighty-nine this summer.’ ({{gary\_MN\_02gk}}, rec. 2010)\\
         \textit{Bokmål: Jeg blir niogåtti til sommeren.} 
    \ex \label{ex:eide:32d} 
    \gll I neste år det skal være førti år.        \\
         in next year it shall be forty years      \\
    \glt ‘Next year it will be forty years.’ ({{saskatoon\_SK\_14gk}}, rec. 2013)\\
         \textit{Bokmål: Til neste år blir det førti år.} 
    \z % you might need an extra \z if this is the last of several subexamples
\z

In all these cases, the example sounds more idiomatic in EurNo replacing the modal + infinitive with the present form, which in EurNo is underspecified, spanning simultaneity and  future. In EurNo, aspectual features of the predicate as stative or dynamic helps to specify the temporal relation in that a dynamic predicate typically gives a future reading, a stative reads as simultaneity (cf. \citealt{Eide2002,Eide2005, Eide2012}). We see this very clearly in the \textit{Bokmål} renderings of (\ref{ex:eide:32}b--d) above, but even a stative predicate \REF{ex:eide:32a}, can be coerced by the future adverbial. 

We mentioned in the introduction that TMA-systems interact in intricate ways. This is easy to illustrate through examples in which e.g. aspect, temporality, and modality depend on each other. Cf. (\ref{ex:eide:33}a--b), from EurNo, where a stative predicate \textit{være} ‘be’ yields a reading as present, which in turn gives the modal an evidential reading as ‘hear-say’. A dynamic predicate \textit{bli} ‘become’ gives the reading of future, which in turn yields the root reading ‘intention’ of the modal. 

\ea%33
    \label{ex:eide:33}
    \ea\label{ex:eide:33a}
    \gll Jon skal [være\textsubscript{stative} arkitekt ] → present reading → evidential \\
         John shall be architect\\
    \glt ‘John is supposed to be an architect, allegedly.’
    \ex\label{ex:eide:33b}
    \gll Jon skal [bli\textsubscript{dynamic} arkitekt] → future reading → root\\
         John shall become architect\\
    \glt ‘John intends to become an architect.’
    \ex \label{ex:eide:33c} 
    \gll Nå ska vi ræse te å besikk dæm e Nårrge, så de ska være moLo tru e\\
         now shall we travel to \textsc{inf} visit them in Norway so that shall be fun think I\\
    \glt ‘Now we are going to visit them in Norway, and that will/should be fun, I think’ ({{coon\_valley\_WI\_12gm}}, rec. 2012)\\
    \z % you might need an extra \z if this is the last of several subexamples
\z


In contrast, in NAmNo there is a tendency that the stative \textit{være} spreads to describe even dynamic future situations. Example \REF{ex:eide:33c} would easily have a ‘hear-say’ reading in EurNo (‘that is supposed to be fun’), but this is clearly not the intended reading in \REF{ex:eide:33c}. Using the predicate \textit{bli} instead of \textit{være} would instantily give the future-projecting root reading of the modal, and a possible next step is to remove the modal altogether, although that would give the reading of pure future instead of evaluative ‘should’. 

When analyzing the data in CANS, it seems that not a single instance of \textit{skulle} in the present tense is unambigously evidential. It is always possible to interpret \textit{skulle} as intention, or future tense. An infinitive perfect complement is usually a very efficient to tease out non-root readings of a modal (e.g. \textit{John must have left}). Interestingly, although there are instances in CANS of the construction \textit{skulle+ha}+participle where the construction is counterfactual, hence modal, I cannot find a single instance of \textit{skal+ha}+participle with \textit{skulle} in the present. In contrast, the study of \textit{spost} reported in \citet{EideHjelde2015Borrowing} finds that spost has an evidential reading in half of the examples, cf. (\ref{ex:eide:34}a--b).\footnote{I cannot find these data in CANS, which is why I refer to the information provided in \citet{EideHjelde2015Borrowing}.} 

\ea%34
    \label{ex:eide:34}
    \ea \label{ex:eide:34a} 
    \gll Derre e spousa å vårra eit tre som va plantja [der].  \\
         That.\textsc{def} is spos to be a tree that was planted there   \\
    \glt ‘That is supposed to be a tree that was plantet there.’ (Lac Qui Parle, MN, rec. 1987)\\
    \ex \label{ex:eide:34b} 
    \gll  Han e spost te å vara riktig god, han.    \\
          He is spost to \textsc{inf} be right good he      \\
    \glt ‘He is supposed to be quite good, he is.’ (Coon Valley, WI, rec. 1992)
    \z % you might need an extra \z if this is the last of several subexamples
\z

This suggests that \textit{skal} and \textit{spost} have taken on different domains in NAmNo; \textit{skal} as a future marker, having lost its ‘hear-say’, i.e., evidential, reading, a function taken over in NamNo by the modal-newcomer \textit{spost}.  

\section{Summing up}
\label{sec:eide:6}

We started out in this chapter by introducing five expectations; notably 

\begin{enumerate}
\item TMA-systems of NAmNo, as “interface” phenomena will show CLI affectedness over time; 
\item Different CLI effects; borrowing, convergence, and a loss of oppositions; 
\item Morphology (e.g. verbal suffixes) will be more affected than free-standing auxiliaries; 
\item TMA-systems will strive towards transparency and one-to-one mappings; 
\item Affectedness will be high for mood/modality, low for tense, with aspect in the middle.
\end{enumerate}

Prediction 1 is fulfilled in that we can see trends of simplification and convergence. For instance, non-finite modal forms are quite frequent in NAmNo recordings from the 40s (i.e. those collected by Einar Haugen), but less frequent in later generations. However, the inventory of productive modals is reduced, even though \textit{spost} (and other English-sounding versions of many modals like \textit{sjudd} and \textit{kudd}) adds to the pool. The most frequent modals are becoming relatively more frequent (\textit{kunne, måtte, skulle, ville}), at the expense of less frequent modals like \textit{burde} ‘should’ and \textit{trenge} ‘need’. We also see clear CLI-effects in the aspectual domain, e.g. the construction \textit{brukte (på) å} has extended its domain in NAmNo as compared to EurNo. Tense is seemingly less affected, but this is as expected on prediction 5. Moreover, tense markers fluctuate over time, at some stages of development the written norms seem to have an impact e.g. in the temporarily increased occurrence of present tense -\textit{r}. 

Prediction 2 is borne out in that we have attested borrowing, e.g. of modals, but also convergence, e.g. in the case of periphrastic aspect marker \textit{brukte (på) å}, incorporating contextual restrictions from AmE (\textit{used to}) whereas the construction itself stems from EurNo. We find loss of oppositions, e.g. in the tense domain, in that the finiteness restriction which had its fifteen minutes of fame in NAmNo before WWII, was lost again – however, this distinction was also lacking from the productive class of one of the dominating dialects (cf. \tabref{tab:eide:4}). Otherwise, a lot of oppositions are maintained, e.g. the distinction between invariant auxiliary \textit{ha} ‘have’, and lexical \textit{ha} ‘have’, traditionally inflected totally differently. It also seems that new oppositions have emerged, e.g. \textit{skulle} takes on a role as future marker, in opposition to \textit{spost}, which monopolizes the evidential reading ‘hear say’. Non-verbal directional complements are also part of the NAmNo grammar, although lacking in English.

Prediction 3 asserts that morphology ought to be more affected than auxiliaries, which is a bit hard to determine. One modal domain expressed in NAmNo as morphology, the irrealis infinitive, is very much alive and not at all moribund, even though it seems to tick all the boxes: Morphological mood markings are more vulnerable than any other domain in HL, according to wide-spread consensus, but seemingly not in NAmNo. For the other domains this is hard to determine conclusively, since aspect is not expressed morphologically and tense belongs to a stable domain; again according to consensus. 

Prediction 4 is borne out at least in the functions taken on by modal \textit{skulle} and \textit{spost}, where \textit{skulle} acts as a future tense marker in a more general an extended way as compared to EurNo, where the present tense form acts as a future marker, hence \textit{skulle} resolves ambiguity in this respect. It appears that \textit{skulle} has lost its function as marker of ‘hear-say’ evidentiality, a task taken over by \textit{spost}. 

Whether our findings comply with the affectedness hierarchies in \REF{ex:eide:3}, where modality is more affected than aspect, with tense as the most stable domain, is also very difficult to judge based on the material we have studied. First, the phenomena studied here are picked out because of saliency more than theoretically defined feature grids. Second, as promised, in very few instances have I referred to statistics, percentages, and absolute numbers of tokens and types in this investigation. Also, to answer these questions it would be necessary to study all modal markers, all tense markers, and all aspectual markers occurring in the production of specific HL-speakers, and even then we would have to bear in mind that we do not know how representative our speakers in CANS are for the entire population.  

Overall, our observations in this overview attest to an impressive level of maintenance in a diasporic variety of Norwegian that has endured for almost two centuries, where the data leave no doubt that the transmission has taken place in oral contexts. There are so many traits preserved in this variety which do not comply to EurNo written norms, regarding all domains of NAmNo TMA, and across morphology and syntax. This pays tribute to the conservativism often observed for HLs, but there is also ample evidence of innovations, both in the forms we observe and also in the underlying patterns they reflect. 

\section*{Abbreviations}
\begin{multicols}{2}
\begin{tabbing}
MMMM \= American\kill
AmE   \>  American English\\
CLI   \>  Crosslinguistic Influence\\
\textsc{def}   \>  Definite\\
EurNo \>  European Norwegian\\
\textsc{inf}   \>  Infinitive\\
NAmNo \>  North American Norwegian\\
HL    \>  Heritage Language\\
\textsc{part}  \>  Participle\\
\textsc{pres}  \>  Present\\
\textsc{pret}  \>  Preterite\\
\textsc{refl} \> Reflexive\\
\textsc{subj}  \>  Subjunctive\\
TMA   \>  Tense, Mood, Aspect
\end{tabbing}
\end{multicols}

\section*{Acknowledgements}

I would like to thank the two reviewers for their helpful suggestions and comments that made me rewrite substantial parts of the chapter. I am also very grateful to the late Janne Bondi Johannessen, who introduced me to the wonderful field of heritage language research by inviting me to partake in her fieldwork in 2010. My deepfelt thanks also to the two editors Kari and Mike for their fruitful feedback, for their patience, support, and encouragement.

\printbibliography[heading=subbibliography,notkeyword=this]
\end{document} 
