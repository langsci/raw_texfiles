\documentclass[output=paper]{langscibook}
\ChapterDOI{10.5281/zenodo.15274562}
\author{Brita Ramsevik Riksem\affiliation{Norwegian University of Science and Technology, Trondheim} and Mari Nygård\affiliation{Norwegian University of Science and Technology, Trondheim}}
\title{Agreement in North American Norwegian determiner phrases}
\abstract{In this chapter, we discuss agreement in determiner phrases (DPs) in North American Norwegian (NAmNo), and we focus on DP-internal agreement, i.e., agreement within the nominal phrase, realized as definite suffixes and determiners. As a backdrop, the structure and nature of European Norwegian DPs are presented, as well as previous analyses of NAmNo DPs, concerning the features definiteness, number, and gender. The bulk of the chapter is devoted to language mixing. First, theoretical implications of such mixing are explored, and we demonstrate the analytical benefits of a late-insertion syntactic model. Moreover, we employ language mixing data to unfold the structural architecture of the DP, in particular the locus of formal gender. We also discuss a mechanism for gender assignment as well as an analysis of agreement involving attributive adjectives. Last, we show through empirical data that agreement in NAmNo DPs seems to undergo diachronic changes, and we discuss whether such data can be analyzed as signs of changes in the functional exponents or in the syntactic structure itself.}


\IfFileExists{../localcommands.tex}{
  \addbibresource{../localbibliography.bib}
  \usepackage{langsci-optional}
\usepackage{langsci-gb4e}
\usepackage{langsci-lgr}

\usepackage{listings}
\lstset{basicstyle=\ttfamily,tabsize=2,breaklines=true}

%added by author
% \usepackage{tipa}
\usepackage{multirow}
\graphicspath{{figures/}}
\usepackage{langsci-branding}

  
\newcommand{\sent}{\enumsentence}
\newcommand{\sents}{\eenumsentence}
\let\citeasnoun\citet

\renewcommand{\lsCoverTitleFont}[1]{\sffamily\addfontfeatures{Scale=MatchUppercase}\fontsize{44pt}{16mm}\selectfont #1}
   
  %% hyphenation points for line breaks
%% Normally, automatic hyphenation in LaTeX is very good
%% If a word is mis-hyphenated, add it to this file
%%
%% add information to TeX file before \begin{document} with:
%% %% hyphenation points for line breaks
%% Normally, automatic hyphenation in LaTeX is very good
%% If a word is mis-hyphenated, add it to this file
%%
%% add information to TeX file before \begin{document} with:
%% %% hyphenation points for line breaks
%% Normally, automatic hyphenation in LaTeX is very good
%% If a word is mis-hyphenated, add it to this file
%%
%% add information to TeX file before \begin{document} with:
%% \include{localhyphenation}
\hyphenation{
affri-ca-te
affri-ca-tes
an-no-tated
com-ple-ments
com-po-si-tio-na-li-ty
non-com-po-si-tio-na-li-ty
Gon-zá-lez
out-side
Ri-chárd
se-man-tics
STREU-SLE
Tie-de-mann
}
\hyphenation{
affri-ca-te
affri-ca-tes
an-no-tated
com-ple-ments
com-po-si-tio-na-li-ty
non-com-po-si-tio-na-li-ty
Gon-zá-lez
out-side
Ri-chárd
se-man-tics
STREU-SLE
Tie-de-mann
}
\hyphenation{
affri-ca-te
affri-ca-tes
an-no-tated
com-ple-ments
com-po-si-tio-na-li-ty
non-com-po-si-tio-na-li-ty
Gon-zá-lez
out-side
Ri-chárd
se-man-tics
STREU-SLE
Tie-de-mann
} 
  \togglepaper[1]%%chapternumber
}{}

\begin{document}
\maketitle 
%\shorttitlerunninghead{}%%use this for an abridged title in the page headers


\section{Introduction}
The topic of the present chapter is agreement within determiner phrases (DPs) in North American Norwegian (NAmNo). DPs in Norwegian can be quite complex: The noun itself is categorized for grammatical gender and is inflected for number and definiteness, and associated words such as determiners, adjectives and possessive pronouns agree with these functional features, see \REF{ex:riksem:1a}. Moreover, Norwegian has DP-external agreement, as exemplified in \REF{ex:riksem:1b}, where the properties of the DP determine the shape of associated personal pronouns. In many Norwegian varieties this pronominal reference is primarily grammatical and does not distinguish between an animate or inanimate antecedent regarding gender (\citealt{FaarlundEtAl1997, JohannessenLarsson2018}). 


\ea \label{ex:riksem:1}
\ea \label{ex:riksem:1a}
\gll det     gul-e           hus-et           mitt\\
	 the.\textsc{n} yellow-\textsc{df} house-\textsc{df.sg.n} my.\textsc{df.sg.n}\\
\glt ‘my yellow house’

\ex \label{ex:riksem:1b}
\gll Eg kjøpte   ei    ny    jakke. Ho er brun.\\
	 I     bought a.\textsc{f} new coat.  She is brown.\\
\glt ‘I bought a new coat. It’s brown.’
\z
\z


In this chapter, we focus on DP-internal agreement, more specifically agreement realized as definite suffixes on the noun stem (and potential attributive adjectives) and determiners within the same nominal phrase.\footnote{DP-external agreement in NAmNo has also been investigated in recent works: \citet{Rødvand2017} and \citet{JohannessenLarsson2018} on gender in pronouns, and \citet{Kinn2020} (briefly) on number agreement in predicate nouns.} 



The corresponding agreement pattern in English DPs is not as complex, and due to the long-lasting contact between Norwegian and English in the heritage language NAmNo, the agreement pattern in DPs thus constitutes an interesting domain for studying resilience or vulnerability in grammar. In this chapter, we will address this issue, and we discuss how language mixing data in particular may shed new light on the structural architecture of the DP and its agreement patterns. 



We assume at the outset that agreement encompasses both the overt morphological exponents in a DP and the underlying agreement between morphosyntactic features in an abstract syntactic structure. Hence, agreement concerns both a syntactic operation between features and its morphosyntactic realization. Our main questions of interest are then the following: How is agreement (overtly) realized in NAmNo DPs, and how can data from agreement in language mixing contribute to the general description of the structural architecture of the DP?



The distinction between phonological exponents and abstract syntactic structures mirrors our general, theoretical foundation. We follow a late-insertion approach to grammar like the one discussed in the Introduction chapter by \textcitetv{chapters/introduction}, where a core assumption is the separation of an abstract syntactic structure and its phonological realization. More specifically, the model we utilize is anchored in an exoskeletal approach (\citealt{Borer2005a,Borer2005b,Borer2013}), and builds on key components from Distributed Morphology (DM) (e.g., \citealt{HarleyNoyer1999, EmbickNoyer2007, Embick2015}). In DM, abstract syntactic structures are generated based on two types of syntactic terminals: roots and functional features. We assume roots to be bare, i.e., devoid of any functional features (\citealt{Arad2005, Borer2014, Marantz1997}), and in the syntactic structure, roots are combined with a designated category-defining head, a categorizer, forming what we (informally) refer to as a stem (see, e.g., \citealt{Embick2015}). Functional features are either valued or unvalued, and the operation Agree ensures a matching relation between a probe and a goal in the syntactic structure (cf. \citetv{chapters/introduction}). At Spell-Out, phonological exponents are inserted into the syntactic terminals. We assume that insertion into stem positions is relatively free, whereas spell-out of functional terminals is more restricted. We argue that a late-insertion model is fruitful to account for agreement in NAmNo DPs, and in Sections 4 and 5 we will show how such a framework allows us to analyze cases of language mixing and at the same time explore the details of the DP structure.\footnote{Dealing with DPs, our chapter has some thematical overlap with van Baal’s chapter \parencitetv{chapters/vanbaal}. However, whereas van Baal focuses on definiteness marking, our goal is to discuss phrase-internal agreement and the morphophonological shape of functional exponents.} 



The bulk of our chapter will be dedicated to discussing the theoretical implications of language mixing. The term language mixing refers to cases where functional or lexical items from two or more languages appear together (see, e.g., \citealt{Muysken2000, Lohndal2013}). These data are especially interesting as they may function as windows into underlying syntactic structures and help carve out the models we employ to understand grammar. Here, we focus mainly on two issues: First, we inquire about the structural architecture of the DP and the locus of functional features ensuring agreement throughout the phrase. We discuss how gender can be used to illuminate the structural architecture of the DP, and more specifically the locus of the gender feature itself. Second, we turn to the morphophonological realizations of the agreement patterns. Here we investigate data where the expected realizations are missing or where English functional exponents are used instead of Norwegian ones. These data, we argue, are tokens of (diachronic) changes in the underlying agreement patterns of the NAmNo DP resulting in a different morphophonological realization. We discuss what such changes may be in \sectref{sec:riksem:5.3}.  



The reasoning in this chapter is primarily based on empirical data and analyses found in previous studies in the field, some of which are our own. Common for most of these is that they generally apply data from the \textit{Corpus of American Nordic Speech} (CANS, \citealt{Johannessen2015CANS}) in the discussion of present-day NAmNo. For purposes of comparison with respect to diachronic changes, we look primarily to \citet{Haugen1953}, and supplement the discussion with some examples from Haugen’s recordings from 1942 (made available in the most recent version of CANS (v. 3.1)). Importantly, the main purpose of the current chapter is not to present an overview of the frequency of different types of DPs and agreement patterns. Our goal is thus not to reveal how common a certain phenomenon is, or how a phenomenon varies between speakers. Rather, we inquire strategically selected examples of DP agreement with an aim to discuss how these can best be analyzed in a grammatical model. 



The chapter is structured as follows: First, in \sectref{sec:riksem:2}, we briefly present and discuss the Norwegian DP and propose a syntactic structure. In \sectref{sec:riksem:3}, we present previous studies of agreement patterns in NAmNo DPs. Thereafter, we turn to language mixing. In \sectref{sec:riksem:4}, we explore the theoretical implications of language mixing, discussing first how the typical pattern of language mixing in NAmNo may be neatly analyzed in a late-insertion model. In \sectref{sec:riksem:4.2}, we employ language mixing data to inquire about the structural architecture of the DP and specifically the locus of formal gender, and in \sectref{sec:riksem:4.3}, we discuss agreement involving attributive adjectives, also from a language mixing perspective. Then, in \sectref{sec:riksem:5}, we present empirical data showing a potential (diachronic) change in the agreement patterns in NAmNo DPs. Based on these data we discuss potential analyses, and how such data can be signs of changes either in the functional exponents or in the syntactic structure itself. Hence, these data may shed light on both our questions of interest. \sectref{sec:riksem:6} concludes the chapter. 


\section{The Norwegian DP}\label{sec:riksem:2}

In this section, we give a brief overview of the morphosyntax of noun phrases in (European) Norwegian, and their agreement relations. This will serve as a baseline for the subsequent discussion. Establishing a baseline for a heritage language is, however, a much-debated issue (see, e.g., \citealt{Montrul2016, Polinsky2018, Lohndal2021, D'AlessandroEtAl2021}). Ideally, the baseline should be the input that each speaker had as a child acquiring the language. This kind of data is, nevertheless, not always available in a heritage language, and the homeland variety is often used instead. In the case of NAmNo, present-day speakers (in CANS) are all second to fifth generation immigrants. Using European Norwegian as a baseline is thus problematic as this was never a part of their input in the first place. In some studies of NAmNo (e.g., \citealt{Riksem2017}), the data collected by Einar Haugen in the 1930s and 1940s serve as a baseline instead as this presumably is more resemblant to the present-day speakers’ input. Nevertheless, in this chapter we take European Norwegian DPs as our starting point in order to present the basic agreement patterns in a Norwegian DP as well as a structural analysis. See also the discussion of a proper baseline for NAmNo in \textcitetv{chapters/vanbaal}, \textcitetv{chapters/eide} and \textcitetv{chapters/larsson}.  

\subsection{Agreement in Norwegian DPs}\label{sec:riksem:2.1}
As mentioned in the Introduction, Norwegian nouns are inflected for definiteness (definite and indefinite) and number (singular and plural). Moreover, Norwegian nouns are categorized for gender, and most Norwegian varieties have three genders (masculine, feminine, and neuter). Gender assignment is primarily non-transparent, meaning that in most cases it is impossible to infer the grammatical gender of nouns based on phonological, morphological, or semantic features. DP agreement is visible on functional suffixes and associated words such as determiners, adjectives, and possessive pronouns, as shown in \REF{ex:riksem:1a}.\footnote{ \textrm{The DP can also contain weak quantifiers and PPs, but these are not affected by agreement.}} In this section, we give a short overview of DP agreement in European Norwegian, focusing on agreement on definite suffixes, determiners, and adjectives.\footnote{\textrm{Possessive pronouns are also influenced by agreement in Norwegian DPs: Singular pronouns vary according to the gender of the noun (e.g.,} \textrm{\textit{min} }\textrm{‘my.}\textrm{\textsc{m}}\textrm{’,} \textrm{\textit{mi}} \textrm{‘my.}\textrm{\textsc{f}}\textrm{’,} \textrm{\textit{mitt} }\textrm{‘my.}\textrm{\textsc{n}}\textrm{’), and there is a separate plural form (}\textrm{\textit{mine}} \textrm{‘my.}\textrm{\textsc{pl}}\textrm{’). The possessive pronoun can occur either prenominally or postnominally, the latter being the more frequent (see, e.g., \citealt{WestergaardAnderssen2015, AnderssenEtAl2018} for more details). \citet{WestergaardAnderssen2015} and \citet{AnderssenEtAl2018} investigate the structural position of the possessive pronoun in NAmNo. They conclude that the postnominal possessive is robust and productive. This is not the expected result given that postnominal possessives are structurally more complex and in addition more vulnerable in a bilingual context. However, as the postnominal possessive is the more frequent pattern, they conclude that high frequency may protect against language attrition. For a more general discussion of possessive constructions in NAmNo, see \textcitetv{chapters/kinn}.}}


Indefinite singular phrases are realized with a prenominal determiner, an indefinite article, varying according to gender: \textit{ein} (\textsc{m})\textit{, ei} (\textsc{f}), and \textit{eit} (\textsc{n}).\footnote{ \textrm{The examples of European Norwegian used throughout the chapter (unless stated otherwise) are variants from} \textrm{\textit{Nynorsk}}\textrm{, one of Norwegians two written standards. See, e.g., \citet{Vikør2001} for more about the Norwegian written standards.} } Definite singular phrases and indefinite and definite plural phrases are realized with a functional suffix also varying according to gender (though see discussion in footnote 9). The available suffixes are presented in the table below (Table 1). Note, however, that there is dialect variation in the phonological shape of the suffixes. 


\begin{table}
\begin{tabular}{l *3{l@{~}l} }
\lsptoprule
 & \multicolumn{2}{c}{\textsc{m}} & \multicolumn{2}{c}{\textsc{f}} & \multicolumn{2}{c}{\textsc{n}}\\\midrule
\textsc{df.} \textsc{sg.}  & \textit{båt-en} & ‘the boat’ & \textit{skei-a} & ‘the spoon’ & \textit{tre-et} & ‘the tree’\\
\textsc{indf.} \textsc{pl.} & \textit{båt-ar} & ‘boats’ & \textit{skei-er} & ‘spoons’ & \textit{tre-} & ‘trees’\\
\textsc{df.} \textsc{pl.} & \textit{båt-ane} & ‘the boats’ & \textit{skei-ene} & ‘the spoons’ & \textit{tre-a} & ‘the trees’\\
\lspbottomrule
\end{tabular}
\caption{Functional suffixes in Norwegian DPs}
\label{tab:riksem:1}
\end{table}



Definite phrases may also contain a prenominal determiner in addition to the definite suffix. This is the phenomenon referred to as double definiteness in Norwegian (see, e.g., \citealt{Julien2005}: 26ff.). The shape of the determiner is then dependent on gender, \textit{den} (\textsc{m/f}), \textit{det} (\textsc{n}), and number, \textit{dei} (\textsc{pl).} Using the nouns from \tabref{tab:riksem:1} as our examples, the corresponding phrases with double definiteness would be \textit{den båt-en} (the.\textsc{m/f} boat-\textsc{df.sg.m}), \textit{den skei-a} (the.\textsc{m/f} spoon-\textsc{df.sg.f}) and \textit{det tre-et} (the.\textsc{n} tree-\textsc{df.sg.n}).



The agreement of the DP will also affect potential attributive adjectives in the phrase. Norwegian adjectival inflection can be divided into a strong and a weak inflection. The weak inflection is used in definite phrases, and it has only one possible suffix (\textit{{}-e}). The strong inflection, on the other hand, is present in indefinite phrases, and is realized as different suffixes according to gender and number. The table below shows the inflection of the adjective \textit{gul} ‘yellow’. Note also the syncretism between masculine and feminine in the strong inflection. 


\begin{table}
\begin{tabular}{l llll}
\lsptoprule
 & \textsc{m} & \textsc{f} & \textsc{n} & \textsc{pl}\\\midrule
Strong inflection (\textsc{indf}) & \textit{gul-} & \textit{gul-} & \textit{gul-t} & \textit{gul-e}\\
Weak inflection (\textsc{df}) & \textit{gul-e} & \textit{gul-e} & \textit{gul-e} & \textit{gul-e}\\
\lspbottomrule
\end{tabular}
\caption{Adjectival inflection in Norwegian}
\label{tab:riksem:2}
\end{table}


\subsection{The basic DP structure}\label{sec:riksem:2.2}

We take the structural analysis in \citet{Julien2005} as our starting point for the analysis of the Norwegian DP structure (see also \citetv{chapters/introduction}), however with some small alterations. Since we in this chapter focus primarily on agreement on the functional suffixes for number and definiteness, the determiner and occasionally also an attributive adjective, these are the only items accounted for in the applied structure, \figref{fig:riksem:fromex:2a}. The example with glosses and translation is given in \REF{ex:riksem:2}. See \citet[11]{Julien2005} for a complete structure of the full DP in Norwegian. 


\begin{figure} 
\caption{The basic Norwegian DP structure}
\label{fig:riksem:fromex:2a}
\begin{forest}
[DP
  [D\\\textit{dei}]
  [αP
    [AP\\\textit{gamle}]
    [α'
      [α]
      [ArtP
        [Art\\\textit{-ne}]
        [NumP
          [Num\\\textit{-a}]
          [n
            [n\\\textit{-ing}]
            [√TEIKN\\\textit{teikn}]
          ]
        ]
      ]
    ]
  ]
]
\end{forest}
\end{figure}

\ea \label{ex:riksem:2}
\gll dei       gaml-e  teikning-ane\\
	 the.\textsc{pl}  old-\textsc{pl}    drawing-\textsc{df.pl.f}\\
\glt ‘the old drawings’
\z



Starting at the bottom, we assume, like \citet{Julien2005}, that the noun stem consists of a root and a categorizer, n. Note that \citet{Julien2005} uses the label N for the categorizer and the stem, whereas we use n in line with, e.g., \citet[44–45]{Embick2015}. When we have phrases with a derived noun stem, like in \figref{fig:riksem:fromex:2a}, derivational affixes are considered a realization of the categorizer itself. In phrases with a simple noun stem, the categorizer is not overtly realized, but the stem is nevertheless considered structurally complex (\citealt{Embick2015}: 44–45).



Above the stem follow the functional heads of the DP. The categories number and definiteness are assumed to constitute individual heads, Num and Art.\footnote{\textrm{\citet{Julien2002Determiners} proposes the projection ArtP as the locus for the definite suffix in Norwegian and other Scandinavian varieties with double definiteness. This projection is labelled} \textrm{\textit{n}}\textrm{P in \citet{Julien2005} and DefP in, e.g., \citet{ÅfarliEtAl2021}. In line with the Introduction chapter as well as van Baal's chapter in the present volume, we use the label ArtP.}} A functional head above Num, but below D, is motivated by double definiteness. Such phrases will have both a determiner and a definite suffix. Following \citet{Julien2002Determiners}, Art is postulated as the locus for the definite suffix (the determiner being a realization of D). Furthermore, exponents of Num and Art will be suffixed to the noun stem as it obligatorily moves to these positions. In the higher positions, we find adjectives in the specifier of a functional projection, αP, and determiners in D. Gender is not accounted for in \figref{fig:riksem:fromex:2a}. There have been several discussions and proposals concerning the structural position of a formal gender feature, which we will discuss in more detail in \sectref{sec:riksem:4.2}. 



Since we will later discuss language mixing between Norwegian and English, we briefly also consider a simple structure for English DPs. A significant difference between Norwegian and English DPs is gender; whereas Norwegian has three genders affecting the shape of functional exponents in the DP, grammatical gender is an alien category to English nouns. Moreover, definiteness in English is expressed by a prenominal determiner (\textit{the}), hence, we assume that an English DP does not have Art. A potential representation of the English DP, parallel to the Norwegian one in \figref{fig:riksem:fromex:2a}, is presented in \figref{fig:riksem:fromex3}. 


\begin{figure}
\begin{forest}
[DP
  [D\\\textit{the}]
  [αP
    [AP\\\textit{old}]
    [α'
      [α]
      [NumP
        [Num\\\textit{-s}]
        [n
          [n\\\textit{-ing}]
          [√DRAW\\\textit{draw}]
        ]
      ]
    ]
  ]
]
\end{forest}
\caption{The basic English DP structure}
\label{fig:riksem:fromex3}
\end{figure}
%%please move the includegraphics inside the {figure} environment
%%\includegraphics[width=\textwidth]{figures/Chapter3AgreementinNAmNoDPs-img002.png}
 



When we now turn to NAmNo DPs, we take the discussion above as our point of departure. As we will see later, in particular in \sectref{sec:riksem:5}, however, NAmNo DPs show some differences compared to such a baseline, either in the phonological realizations of agreement or potentially the underlying syntactic structure. Gender will turn out to be a crucial feature in exploring and analyzing such patterns. 


\section{The North American Norwegian DP}\label{sec:riksem:3} 

\sectref{sec:riksem:2} outlined the patterns of the European Norwegian DP. However, it is well\hyp documented that, in general, heritage speakers have certain difficulties with morphology, and inflectional morphology in particular is often affected (\citealt{vanBaal2020, Montrul2016, ScontrasEtAl2015}). For instance, the speakers tend to simplify the inflectional system by omitting otherwise obligatory forms. In the current section we turn to the NAmNo DPs and some recent studies of agreement. 



In a cursory study of old recordings of NAmNo (and North American Swedish), \citet{JohannessenLarsson2015} find that overall, the agreement system in the heritage language is identical to the baseline language, and that, if anything, the older heritage language had more morphological distinctions (e.g., case distinctions) within the noun phrase, not fewer. In a study of diachronic change in the nominal morphology, focusing on cases of language mixing, \citet{Riksem2017}, on the other hand, finds omission of functional suffixes both in plural and/or definite noun phrases, as well as an increased use of English functional exponents among present-day speakers. In the following, we will address definiteness, number and gender successively.



Concerning definiteness in NAmNo DPs, there are few studies, compared to, e.g., studies on gender.\footnote{Investigating young heritage Swedish speakers, \citet{Håkansson1995} finds that they struggle with definiteness agreement as well as gender and number agreement; they omit functional suffixes both on attributive adjectives and the noun itself.} In a comprehensive study of compositional definiteness (what we call double definiteness above), \citet{vanBaal2020} finds that NAmNo nominal phrases are very similar to homeland Norwegian, for instance with respect to word order and to the marking of definiteness. She states that, in DPs without compositional definiteness, the speakers typically use the indefinite article and the suffixed definite article in a stable, baseline-like manner. 



Yet, compositional definiteness appears to be vulnerable, and \citet{vanBaal2020} points out that the prenominal determiner is the most vulnerable to omission: All the speakers were found to omit the determiner in some modified definite phrases, and for most, this was the most frequent type of non\hyp baseline\hyp like modified definite phrase. In fact, this pattern of modified definite phrases realized without the determiner was more frequent than modified definite phrases with compositional definiteness, and this pattern was also supported by acceptability judgment tasks. Thus, \citet[218]{vanBaal2020} concludes that whereas the typical homeland modified definite phrase contains compositional definiteness, as in \REF{ex:riksem:4a}, the typical NAmNo modified definite phrase lacks the determiner, as in \REF{ex:riksem:4b}. \Citet{vanBaal2020} argues that this pattern can be explained as an example of incomplete acquisition. 


\ea \label{ex:riksem:4}
\ea \label{ex:riksem:4a}
	\gll den stor-e bil-en\\
		 the\textsc{.sg.m} large-\textsc{df} car-\textsc{df.sg.m}\\
	\glt ‘the large car’
\ex \label{ex:riksem:4b}
	\gll stor-e bil-en\\
		 large-\textsc{df} car-\textsc{df.sg.m}\\
	\glt ‘the large car’
\z
\z



Moreover, \citet{vanBaal2020} observes that for some speakers, the definiteness distinction is disappearing in plural phrases. In her data, some plural nouns are realized with the indefinite plural suffix even though they appear in a definite context. For example, the phrase \textit{svarte hester} ‘black horses’, with no determiner and the indefinite plural suffix \textit{{}-er}, is used both in indefinite and definite contexts. See also \textcitetv{chapters/vanbaal} for a more elaborate discussion of definiteness marking in NAmNo. 



Apart from \citegen{vanBaal2020} finding concerning the diminishing distinction between definite and indefinite plural phrases, and the general comments on morphology in the NAmNo DPs, the literature is rather sparse on the issue of number agreement. One might suspect that number is less vulnerable than for instance gender, given that, as a feature, number is more indexical in nature. A few targeted searches in CANS give support to this assumption, showing that the weak quantifiers \textit{flere} ‘several’, \textit{mange} ‘many’ and \textit{disse} ‘these’ in nearly all observed cases appear with plural morphology on associated adjectives and nouns.\footnote{\textrm{We found one exception, namely} \textrm{\textit{mange slektning} }\textrm{‘many relative’ (CANS;} %\href{https://tekstlab.uio.no/glossa2/cans3}
%{\textrm{
albert\_lea\_MN\_01gk).}
%}}). } 
A couple of examples are \textit{mange hest-er} ‘many horses\textsc{’} and \textit{flere forskjellig-e skol-er} ‘several different schools’. However, all in all, number morphology in NAmNo DPs has not been thoroughly investigated and should be subject to future research. 



For heritage speakers, a feature that has been found to be particularly difficult is gender. \citet{JohannessenLarsson2015} investigate NAmNo nominal phrases with the combination of (a determiner) an adjective, and a noun stem. Among the 34 speakers included in the study, they find that the majority (20 speakers) produce target-like DPs. Examples of non-target-like cases are \textit{en fin-t maskin} ‘a.\textsc{m} nice-\textsc{n} machine’ or \textit{denna andre skolehus-et} ‘this\textsc{.m} other school house-\textsc{n’,} where the determiner does not agree with the gender of the inflectional suffix of the adjective or the noun, respectively. Going further into the details, \citet{JohannessenLarsson2015} report that the deviations occur primarily in what they label complex noun phrases, i.e., phrases involving a determiner, an adjective, and a noun stem. Hence, they argue that the complexity of the phrase is decisive of how target-like the agreement patterns may be. 



Concerning gender, \citet{JohannessenLarsson2015} conclude that gender in NAmNo is stable for the majority of the speakers, although there is a tendency of overgeneralization to the masculine (the most frequent gender in Norwegian). Also \citet{Rødvand2017} reaches a similar conclusion. Through two elicitation tasks she investigates the gender system of 25 NAmNo speakers. She uses agreement with personal pronouns, indefinite articles as well as the definite suffix to determine whether the speakers have retained a three-gender system and concludes that all speakers show relics of such a system. Despite great inter-individual variations, she finds no sign of a general breakdown in the gender system. More specifically, the definite suffix is produced target-like to a great extent, while the observed variation is connected to gender on pronouns and the indefinite article. 



In a different study of gender agreement among 50 NAmNo speakers, \citet{LohndalWestergaard2016} conclude, on the contrary, that gender in NAmNo is vulnerable. They observe what they call an erosion of the three-gender system and argue that this is due to the lack of transparency in the Norwegian gender system. The different conclusions in these studies are striking. One possible explanation for this is the different position they take regarding the definite suffix as a marker for gender. Whereas \citet{JohannessenLarsson2015} and \citet{Rødvand2017} assume the suffix to be a marker for gender, \citet{LohndalWestergaard2016} argue that the suffix is a marker for declension class. Hence, \citet{LohndalWestergaard2016} hold a more restrictive view in the study of gender, and the status of the definite suffix as a gender marker might be decisive for the conclusion regarding a stable or eroding gender system in NAmNo.\footnote{ \textrm{The interplay and differences between gender and declension class are a recurring discussion (see, e.g., \citealt{Enger2004}). According to the Norwegian reference grammar \citep{FaarlundEtAl1997}, the definite suffix is considered a marker for gender. This is also what we assume in this chapter going forward.} }



Studies of gender (in NAmNo), like the ones discussed above, all touch upon two interrelated issues: gender assignment and gender agreement; the assigned gender of a given noun will affect the expected agreement in the DP as a whole. Subsequently, one may discuss whether or not a DP should be considered target-like in cases where the assigned gender is unexpected (compared to a baseline), even if the agreement pattern matches the assigned gender. For instance, \citet{JohannessenLarsson2015} use the phrase \textit{ei stor famili} ‘a.\textsc{f} big family’ as an example of a \textit{non}{}-target-like phrase (the target being \textit{en stor famili} ‘a.\textsc{m} big family’).\footnote{Note that \emph{famili} is the phonological transcription, whereas the orthographic spelling is \emph{familie}.} However, if the noun \textit{familie} is in fact assigned feminine gender in this specific case, the phrase should be considered target-like.\footnote{In fact, there are several examples of NAmNo DPs where the gender assignment might be judged non-target-like compared to European Norwegian, but where the subsequent agreement pattern matches the assigned gender:
\ea \textit{en liten rom} ({\textsc{m}}{) ‘a small room’ European Norwegian target:} {\textsc{n}} {(CANS; decorah\_IA\_01gm)}

\ex \textit{ei tynn papir} ({\textsc{f}}{) ‘a thin paper’}{ }{European Norwegian target:} {\textsc{n}} {(CANS; gary\_MN\_02gk)}

\ex \textit{et lite øy} ({\textsc{n}}{) ‘a small island’}{ }{European Norwegian target:} {\textsc{f}} {(CANS; kalispell\_MT\_02uk)}

\ex \textit{et vakkert tur} ({\textsc{n}}){ }{‘a nice trip’   European Norwegian target:} {\textsc{m}} {(CANS; willmar\_MN\_01gm)}

\ex \textit{ei lita skule} ({\textsc{f}}{) ‘a small school’}{ }{European Norwegian target:} {\textsc{m}} {(CANS; westby\_WI\_12gm)}\z} 
In other words, even though a functional exponent has a non-target-like shape (compared to, e.g., European Norwegian), it does not necessarily indicate differences or attrition in the agreement patterns as it could also mean that the specific noun has been assigned a different gender.\footnote{ \textrm{Note also that many nouns are assigned different genders across various dialects in European Norwegian. For instance, the word} \textrm{\textit{eple} }\textrm{‘apple’ is feminine in some dialects, and neuter in others. Judging how target-like a DP is based on gender assignment in European Norwegian is therefore not always unproblematic.} } The relation between assignment and agreement is therefore essential when discussing gender in NAmNo.\largerpage



In this chapter, we are mainly concerned with agreement in NAmNo, hence we will primarily discuss how a gender feature affects the phonological realization of functional items in the DP. However, gender assignment will become relevant when dealing with English nouns being mixed into Norwegian. As we will see, also English nouns are assigned gender when they occur in a Norwegian structure, and the subsequent question is how a noun from a non-gender language like English can be assigned gender in a mixing context? We will argue that formal gender is a feature of the syntactic structure, and that gender assignment takes place through a gender translation mechanism (\citealt{ÅfarliEtAl2021}, see \sectref{sec:riksem:4.2}. for discussion). Consequently, a single noun may be assigned different genders by different speakers. We will return to the discussion of this question and possible hypotheses for analyses in the next section, where we investigate language mixing and theoretical perspectives on this issue more broadly.


\section{The theoretical implications of language mixing}\label{sec:riksem:4}
\begin{sloppypar}
Several recent studies of NAmNo have focused on language mixing (e.g., \citealt{GrimstadEtAl2014, Åfarli2015, AlexiadouEtAl2015, Grimstad2018, Riksem2018Thesis, RiksemEtAl2019, RiksemEtAl2021}). Still, language mixing in NAmNo is not a recent phenomenon; the usage of English items is observed in many early studies on NAmNo (\citealt{Flaten1900, Flom1900, Flom1903, Flom1926, Hjelde1992, Haugen1953}). The pattern of mixing has been described like this: 
\end{sloppypar}

\begin{quote}
Some words are, indeed, used without any appreciable difference in pronunciation, but more generally the root, or stem, is taken and Norse inflections are added as required by the rules of the language. \citep[115]{Flaten1900}
\end{quote}


\begin{quote}
A single form is usually imported and is then given whatever endings the language requires to make it feel like a proper word and to express the categories which this particular language requires its words to express. \citep[440]{Haugen1953}
\end{quote}



These two quotes highlight the typical pattern for language mixing in NAmNo: English lexical items are used in Norwegian contexts where they appear together with Norwegian inflectional suffixes. In other words, English lexical items are integrated into a Norwegian structure.\footnote{Note that by referring to something as a Norwegian structure or an English item, we do so based on functional features and lexical items typically associated with those languages.}\footnote{This implies that functional features are not associated directly with lexical items but are properties of the syntactic structure. \citet{GrimstadEtAl2018} discuss how a lexicalist model will fall short in analyzing such data, and how a late-insertion exoskeletal model fares better.}



\citet{Riksem2018Language} investigates language mixing in present-day NAmNo DPs, and some examples of the typical pattern are given in \REF{ex:riksem:5}. English items are boldfaced, and the code in the parenthesis identifies the speaker in CANS. 


\ea \label{ex:riksem:5}
\ea \label{ex:riksem:5a}
	\gll ei \textbf{nurse}\\
	a.\textsc{f} nurse (CANS; %\href{https://tekstlab.uio.no/glossa2/cans3}{
    coon\_valley\_WI\_02gm)\\
\ex \label{ex:riksem:5b}
	\gll en \textbf{chainsaw}\\
		 a\textsc{.m} chainsaw (CANS; %\href{https://tekstlab.uio.no/glossa2/cans3}{
         blair\_WI\_07gm)\\
\ex \label{ex:riksem:5c}
	\gll \textbf{chopper}{}-en\\
		 chopper-\textsc{df.sg.m} (CANS; %\href{https://tekstlab.uio.no/glossa2/cans3}{
         blair\_WI\_01gm)\\
\ex \label{ex:riksem:5d}
	\gll den \textbf{field}{}-a\\
		 that field-\textsc{df.sg.f} (CANS; %\href{https://tekstlab.uio.no/glossa2/cans3}{
         coon\_valley\_WI\_02gm)\\
\ex \label{ex:riksem:5e}
	\gll det \textbf{stuff}{}-et\\
		 the stuff-\textsc{df.sg.n} (CANS; %\href{https://tekstlab.uio.no/glossa2/cans3}{
         starbuck\_MN\_01gk)\\
\ex \label{ex:riksem:5f}
	\gll \textbf{birthday}{}-en hennes\\
		 birthday-\textsc{df.sg.m} hers (CANS; %\href{https://tekstlab.uio.no/glossa2/cans3}{
         coon\_valley\_WI\_06gm)\\
\ex          \label{ex:riksem:5g}
	\gll \textbf{deck}{}-en hans\\
		 deck-\textsc{df.sg.m} his (CANS; %\href{https://tekstlab.uio.no/glossa2/cans3}{
         westby\_WI\_01gm)\\
\z
\z



In these examples, an English noun is used, but it occurs together with a Norwegian indefinite article (\ref{ex:riksem:5}a–b), a Norwegian determiner (\ref{ex:riksem:5}d–e), a Norwegian definite suffix (\ref{ex:riksem:5}c–g) and/or a Norwegian possessive pronoun (\ref{ex:riksem:5}f–g). These data clearly demonstrate how the English noun stems may be described as integrated into a Norwegian functional structure, where they appear with Norwegian functional exponents and in a Norwegian word order, cf. the postnominal possessive pronoun. 



In the following, we discuss how language mixing data like those in \REF{ex:riksem:5} can shed light on agreement in NAmNo DPs. First, in \sectref{sec:riksem:4.1}, we show an analysis of the typical pattern and how the agreement in NAmNo DPs is realized in phrases with an English nominal stem. Then, in \sectref{sec:riksem:4.2}, we discuss gender in particular. Following \citet{ÅfarliEtAl2021} we use grammatical gender in language mixing data to inquire about the locus of the formal gender feature in the DP structure. By extension, gender may also illuminate the structural architecture of the DP in general. Lastly, in \sectref{sec:riksem:4.3}, we briefly address agreement in phrases with an attributive adjective based on the work by \citet{RiksemEtAl2021}. Here, we retain the focus on language mixing and will show how agreement unexpectedly is \textit{not} overtly realized in cases where English adjectival stems are mixed into NAmNo DPs. Also in this case language mixing data may be a key to unfolding the details of the DP structure. 


\subsection{Analyses of the typical mixing pattern}\label{sec:riksem:4.1}
\largerpage

In general, there are two main approaches to analyzing language mixing data: One is assuming that such data are peculiar and restricted by some special constraints in grammar, whereas the other approach assumes that language mixing is restricted by the same principles as unmixed data. The latter approach is referred to as a Null Theory, or a constraint-free approach to language mixing (\citealt{Mahootian1993, MacSwan1999}). Following a Null Theory approach entails that mixing data might inform us not only about the patterns and restrictions of language mixing itself, but about grammar in general. Mixed data would then potentially serve as entries into the underlying grammatical structures enabling us to tease apart and identify details in a syntactic structure. 



\textcite{Riksem2018Language, Riksem2018Thesis}, among others, argues that a late-insertion exoskeletal model is well-suited to account for language mixing data without resorting to special constraints, hence based on a Null Theory view. She employs the structure in \figref{fig:riksem:fromex:6} as the %structural 
basis for analyzing data such as those presented in \REF{ex:riksem:5}, and in the following we will briefly present the main points of this analysis.\largerpage


\begin{figure}[h]
\begin{forest}
[DP
  [D\\
   {$\left[\begin{tabular}{@{} l@{~}l @{}}
   	\textsc{def}: & \textsc{u}\\
   	\textsc{num}: & \textsc{u}\\
   	\textsc{gen}: & \textsc{u}\\
   \end{tabular}
   \right]$} 
  ]
  [FP
    [F\\
    {$\left[\begin{tabular}{@{} l@{~}l @{}}
    		\textsc{def}: & \textsc{x}\\
    		\textsc{num}: & \textsc{y}\\
    		\textsc{gen}: & \textsc{z}\\
    	\end{tabular}
    	\right]$} 
    ]
    [n
      [n]
      [√ROOT]
    ]
  ]
]
\end{forest}
\caption{Simple DP structure prior to vocabulary insertion}
\label{fig:riksem:fromex:6}
\end{figure}
%%please move the includegraphics inside the {figure} environment
%%\includegraphics[width=\textwidth]{figures/Chapter3AgreementinNAmNoDPs-img003.png}
 



One obvious difference between this structure and the one proposed in \figref{fig:riksem:fromex:2a} above concerns the functional heads between the nominal stem and the DP layer. Instead of using separate heads for definiteness (Art) and number (Num), a common functional head, (F), holding a feature bundle including definiteness, number, and gender, is employed.\footnote{In \textcite[95–96]{Riksem2018Thesis} the functional head F is motivated partly for its ease of exposition in the analysis. Moreover, it proves to be sufficient in analyzing the typical cases of language mixing where English nouns appear with Norwegian functional suffixes. Nevertheless, such a feature bundle will not enable fine-grained investigations of the complexity of Norwegian functional suffixes. In \sectref{sec:riksem:4.2} we discuss how separate functional heads are necessary to reveal details of the syntactic structure and the locus of gender.} The features in F are assumed to be valued, whereas D holds a feature bundle of corresponding unvalued features. The features in D will then be valued by a probe-goal relation with F during the derivation. Feature valuation thus ensures agreement between D and F. 



The structure in \figref{fig:riksem:fromex:7} demonstrates how \REF{ex:riksem:5e} can be analyzed. At Spell-Out, an English noun stem, \textit{stuff}, is inserted as a realization of the stem complex (root\,+ categorizer) in the structure. Insertion into the terminals holding functional feature bundles, on the other hand, is more restricted. \citet{Riksem2018Language} utilizes the Subset Principle (see \citealt{Halle1997}: 428) in her analyses.\footnote{ \textrm{The Subset Principle is one possible tool among others determining exponency in a late-insertion model. Due to the scope of the current chapter, we do not go further into the details of the theoretical framework.}} To put it briefly, this principle ensures the insertion of the most suitable functional exponent, which is the exponent matching \textit{the most} of the functional features specified in the syntactic terminal while not containing features not present in the syntactic terminal. In the case of language mixing in NAmNo, Norwegian functional exponents will be specified for all three of these features, and hence the relevant Norwegian exponent will be preferred over an English functional exponent. Phonological exponents are given in italics, and in course of the derivation the noun stem will move to F where it combines with the suffix \textit{{}-et.}  


\begin{figure}
\begin{forest}
[DP
  [D\\
   {$\left[\begin{tabular}{@{} l@{~}l @{}}
   	\textsc{def}: & \textsc{df}\\
   	\textsc{num}: & \textsc{sg}\\
   	\textsc{gen}: & \textsc{n}\\
   \end{tabular}
   \right]$}\\
   \textit{det} 
  ]
  [FP
    [F\\
    {$\left[\begin{tabular}{@{} l@{~}l @{}}
    		\textsc{def}: & \textsc{df}\\
    		\textsc{num}: & \textsc{sg}\\
    		\textsc{gen}: & \textsc{n}\\
    	\end{tabular}
    	\right]$}\\
    	\textit{-et}
    ]
    [n
      [n,name=stuff-1]
      [√STUFF,name=stuff-2]
    ]
  ]
]
\path let \p1=(stuff-1), 
		  \p2=(stuff-2),
		  \p{center} = ($ (\p1) !.5! (\p2) $) 
      in coordinate (stuff-node) at (\p{center});
\node[below=.5\baselineskip of stuff-node] {\textit{stuff}};
\end{forest}
\caption{Structural representation of (4e) \emph{det stuffet}}
\label{fig:riksem:fromex:7}
\end{figure}
%%please move the includegraphics inside the {figure} environment
%%\includegraphics[width=\textwidth]{figures/Chapter3AgreementinNAmNoDPs-img004.png}
 



This structure demonstrates how Norwegian agreement, including gender, is preserved even in cases where the noun stem is English. Moreover, it does not utilize any special mechanism to account for the fact that the phrase is a case of language mixing, i.e., it forms a Null Theory. We therefore assume that the derivation of all-Norwegian phrases is parallel to \figref{fig:riksem:fromex:7}, the only difference being the inserted stem. 


\subsection{What gender can tell us about the architecture of the DP structure}\label{sec:riksem:4.2}

Comparing Norwegian and English DPs, grammatical gender constitutes a significant difference. This means that in a language mixing context involving English and Norwegian, gender is particularly interesting to investigate. As mentioned in \sectref{sec:riksem:3}, we are primarily concerned with gender agreement, and in the current section, we use language mixing data to inquire about the position of gender in a syntactic structure and at the same time show how gender may function as a tool in unfolding the architecture of the DP in general. However, first we will briefly address a couple of questions concerning gender assignment, namely how gender can be assigned a nominal stem from a non-gender language like English, and how gender assignment can take place at all when it is assumed to be a formal feature of the abstract syntactic structure and not an inherent feature of the noun.  



Regarding the assignment of gender to an English noun, there are two common hypotheses. Following the first hypothesis, one assumes that nouns from a non-gender language will be assigned one specific gender, most likely the most frequent gender, when mixed into a gender language. In the case of NAmNo, this would imply that English nouns are assigned masculine gender when used in Norwegian utterances. Even though signs of overgeneralization to the masculine have been observed in NAmNo (e.g., \citealt{LohndalWestergaard2016}), this hypothesis is dubious as English nouns are distributed across all three genders in NAmNo, which also emerges from the examples in \REF{ex:riksem:5}. The second hypothesis is that an English noun would be assigned the gender of the Norwegian equivalent (see, e.g., \citealt{JakeEtAl2002, LicerasEtAl2008, ParafitaCouto2015} for discussion). \textcite[93–95]{Riksem2018Thesis} problematizes this hypothesis. First, a single English noun may potentially have many different Norwegian equivalents with different genders. An example is \textit{field}, a noun which is often assigned feminine gender in NAmNo, and may correspond to \textit{åker} (\textsc{m}), \textit{eng} (\textsc{f}), \textit{mark} (\textsc{f}) or \textit{jorde} (\textsc{n}). Consequently, it is difficult to settle on which Norwegian noun was the basis for gender assignment. Moreover, \citet[94]{Riksem2018Thesis} compares the gender assigned to the singular English nouns in her material to a potential Norwegian equivalent and finds that almost 40\% of the nouns in question are assigned a different gender than their most obvious Norwegian counterparts (see also \citealt{Tjugum2016}).



It thus seems that neither of the two hypotheses are suited to account for gender assignment for English nouns mixed into Norwegian, and subsequently a more general question arises: How is gender assigned in the first place? The issue of gender assignment is discussed in \citet{ÅfarliEtAl2021}, building on a distinction between semantic and formal gender. More specifically, the claim is that even though there are semantic gender distinctions in all languages, only some languages – so called gender languages – grammaticalize these gender distinctions, with the consequence that these languages display visible gender morphology through affixes and functional words. This implies that even though the biological difference between \textit{a girl} and \textit{a boy} is manifest in the semantics of English, the determiner is the same since English is not a gender language, as opposed to Norwegian.\footnote{ \textrm{Note, however, that even if a gender language has a semantic core, it is not necessarily the case that semantics will dominate the assignment of grammatical gender. For instance, the noun} \textrm{\textit{mamma} }\textrm{‘mom’ is masculine in many Norwegian dialects. See \citet{Enger2009} for a discussion of gender assignment rules.} } In general, there are in all languages many semantic properties that are not grammaticalized as formal features (\citealt{NarrogHeine2011}), as is the case for English gender.



To account for these patterns, \citet{ÅfarliEtAl2021} suggest that in languages without grammatical gender, such as English, there are no gender features in the DP structure, whereas gender languages, such as Norwegian, have unvalued gender features (\textsc{ugen}) somewhere in their syntactic structure, requiring valuation in course of the syntactic derivation. Moreover, the formal gender features in gender languages need first to be assigned a value, and subsequently gender agreement will be visible in the language. The lowest unvalued gender feature in the DP structure is subject to the assignment process, and gender agreement higher in the structure is then accomplished through a probe–goal relation. 



The question of the nature of gender assignment can thus be reformulated as: How does the initial valuation of the lowest \textsc{ugen} feature happen? \citet{ÅfarliEtAl2021} propose that this happens through the process of gender ``translation'', where a non-linguistic semantic gender property is transformed or fixed into a formal gender feature. They thus assume from the outset that gender has a conceptual core (see also \citealt{Kramer2015, Corbett2013}). This core can be based on different semantic categorizations such as biology, but also morphology or phonology, and the term conceptual covers all these nuances. We leave aside here the complex issue of which nouns get which gender in a language. Gender translation finds its parallel in the valuing of number features. It is even more obvious that number morphology mirrors semantic properties, and gender translation is thus only a variant of a more general mechanism which transforms semantic properties into formal features in the grammar, in the languages where these features are part of the grammatical inventory and therefore expressed in the formal inflection system. 



Some could argue that such a translation process would imply an undesirable mixing of domains: the formal and the conceptual ones. We argue, however, that these processes seem very common. Åfarli et al.’s (2021) analysis resembles the one proposed by \citet{Picallo2008}, namely that gender is the formal exponent of the uninterpretable feature CLASS, which “serves to translate to the grammatical system non-linguistic processes of entity categorization” \citep[50]{Picallo2008}. It also reflects the overall view in \citet{Broschart2000}, where it is argued that classifiers define units of different kinds and that there is no difference of principle between the function of a linguistic classifier and a concrete perceptual contour. They are both needed for the discrimination of units and for the possibility of recognizing them as a certain kind \citep[264]{Broschart2000}.   



\citet{ÅfarliEtAl2021} thus argue that the lowest \textsc{ugen} feature in a DP structure is subject to gender translation, where it is assigned a specific value, and that other unvalued gender features are valued subsequently through probe-goal agreement. The next task then is to identify the locus of this lowest gender feature in the DP. In the following, we will use language mixing data to inquire about the possible structural loci of the gender feature. In other words, we explore the projections in a structure like \figref{fig:riksem:fromex:8} to determine the position of formal gender (see \citealt{ÅfarliEtAl2021, Riksem2018Thesis}: 89ff. for more discussion). 


\begin{figure}
\begin{forest}
[DP
  [D]
  [ArtP
    [Art]
    [NumP
      [Num]
      [n
        [n]
        [√ROOT]
      ]
    ]
  ]
]
\end{forest}
\caption{Structural basis for exploring the locus of formal gender}
\label{fig:riksem:fromex:8}
\end{figure}  
%%please move the includegraphics inside the {figure} environment
%%\includegraphics[width=\textwidth]{figures/Chapter3AgreementinNAmNoDPs-img005.png}
 



Starting at the bottom of the structure there is both theoretical and empirical evidence against assuming a gender feature on the root. Having a gender feature on the root is theoretically dismissed as the root is, by definition, an entity devoid of any grammatical features (cf. \citealt{Borer2005a, Borer2005b, Borer2014, Arad2005, Marantz1997}.). Moreover, as one root may surface in different word classes (for instance the verb \textit{braid} and the noun \textit{braid}), this would either mean a substantial expansion of the vocabulary assuming an individual entry for all possible realizations, or it would mean that also the verb \textit{braid} would carry a gender feature. Both alternatives appear theoretically implausible. At the empirical level, language mixing also constitutes a counterargument against this position, as it would entail that English nouns also carry a formal gender feature enabling gender agreement in mixed DPs like those presented in \REF{ex:riksem:5}.\largerpage 



The next possible position for a gender feature is then the nominalizer \textit{n} (see, e.g., \citealt{Alexiadou2004, Julien2005, Kramer2014}). Arguments in favor of this analysis are the fact that gender is an essential feature of the noun, and that gender often comes across as an inherent property of the noun (stem). However, this analysis is also contested by language mixing, more specifically cases where English derived nouns, (\ref{ex:riksem:9}a–b), and compounds, (\ref{ex:riksem:9}c–d), appear with Norwegian determiners or functional suffixes specified for gender. 


\ea \label{ex:riksem:9}
\ea \label{ex:riksem:9a}
	\gll \textbf{basement}{}-en\\
		 basement-\textsc{df.sg.m} (CANS; %\href{https://tekstlab.uio.no/glossa2/cans3}{
         westby\_WI\_02gm)\\
\ex \label{ex:riksem:9b}
	\gll et \textbf{township}\\
		 a.\textsc{n} township (CANS; %\href{https://tekstlab.uio.no/glossa2/cans3}{
         flom\_MN\_01gm)\\
\ex \label{ex:riksem:9c}
	\gll ei \textbf{comic} \textbf{book}\\
		a.\textsc{f}  comic book (CANS; %\href{https://tekstlab.uio.no/glossa2/cans3}{
        coon\_valley\_WI\_10gm)\\
\ex \label{ex:riksem:9d}
	\gll \textbf{graveyard}-en\\
		{grave yard}-\textsc{df.sg.m} (CANS; %\href{https://tekstlab.uio.no/glossa2/cans3}{
        blair\_WI\_07gm)\\
\z
\z



Let us consider derived nouns first. As discussed in \sectref{sec:riksem:2}, derivational affixes are considered realizations of the categorizer, in our case the nominalizer. In data like (\ref{ex:riksem:9}a–b), both the root and the nominalizer are then presumably drawn from English. The example in \REF{ex:riksem:9b} is the most obvious one in this regard as the derivational affix -\textit{ship} does not exist in Norwegian. Hence, it is the complete stem (root + n) that is mixed from English into NAmNo. \figref{fig:riksem:fromex:10} shows an analysis of the stem in \REF{ex:riksem:9b}. 


\begin{figure}
\begin{forest}
[
  [,phantom]
  [n
   [n {[-ship]},name=township-1]
   [√TOWN,name=township-2]
  ]
]
\path let \p1=(township-1), 
		  \p2=(township-2),
		  \p{center} = ($ (\p1) !.5! (\p2) $) 
	  in coordinate (township-node) at (\p{center});
\node[below=.5\baselineskip of township-node] {\textit{township}};
\end{forest}
\caption{Structural representation of (5b) \emph{township}}
\label{fig:riksem:fromex:10}
\end{figure}
%%please move the includegraphics inside the {figure} environment
%%\includegraphics[width=\textwidth]{figures/Chapter3AgreementinNAmNoDPs-img006.png}
 



Still, these derived nouns appear with Norwegian functional determiners or suffixes in parallel to the examples in \REF{ex:riksem:5}. Now, if the gender feature is assumed to be positioned on the nominalizer, this brings us back to the question raised above: How can an English noun stem provide a gender feature ensuring agreement with other parts of the DP when English nouns do not have grammatical gender? The fact that \REF{ex:riksem:9a} and \REF{ex:riksem:9b} are assigned different genders, masculine and neuter respectively, reinforces this argument, as this indicates the presence of a gender feature available for gender assignment.\largerpage



Compounds also support an analysis where stems are the items being mixed. These are, by definition, complex stems, and finding cases like (\ref{ex:riksem:9}c–d), where all-English compounds are mixed into NAmNo, provides extra evidence that stems are the items being mixed. Neither the lefthand nor the righthand member of these compounds can be expected to carry a gender feature, but nevertheless, the compounds are assigned gender visible in determiners and functional suffixes (see \citealt{EikRiksem2022} for more discussion on compounds and language mixing). Based on cases like those in \REF{ex:riksem:9}, we argue that the stem must be the mixed item, and the gender feature must be located above stem level. However, strictly speaking we cannot be certain whether it is always the stem that is mixed, or if the root is available for mixing in different contexts. For the purpose of this chapter, we acknowledge that stems are available for mixing, but we remain agnostic as to whether roots alone can be mixed.



The data we have considered so far, provide arguments in favor of an analysis where gender is positioned above the nominal stem, i.e., in the higher functional structure of the DP. As discussed in \sectref{sec:riksem:4.1}. above, \citet{Riksem2018Language} does not go further into the question of the exact locus of the gender feature, and instead assumes a functional feature bundle with definiteness, number, and gender combined in a functional head, F. This analysis is sufficient to account for of the bulk of the language mixing data. However, more detailed analyses require splitting the feature bundle into separate heads, at least for number and definiteness. The question is then if grammatical gender is a feature of one of these heads. \citet{ÅfarliEtAl2021} discuss the locus of the gender feature in these heads, and propose, based on data from language mixing, that gender is distributed, i.e., that gender is in fact a formal feature present in both these heads. In the following, we present the main outline of this analysis.\footnote{\citet{ÅfarliEtAl2021} also argue against an analysis in which gender is assumed to be head of a separate functional projection, as in \citet{Picallo1991} (see also \citealt{Ritter1993, Kramer2015}).}



First, let us consider Num. In the case of number, the examples below demonstrate how gender affects the shape of plural suffixes in European Norwegian. This suggests that gender is a feature of Num.\footnote{Notice that one’s definition of gender will determine whether functional suffixes like these are considered markers for gender or declension class (see, e.g., \citealt{Enger2004} for more discussion). In parallel with the singular, definite suffix (see footnote 9), we consider the plural suffixes markers for gender.}


\begin{table}[H]
\begin{tabular}{l lll}
\lsptoprule
                 & \textsc{m} & \textsc{f} & \textsc{n}\\\midrule
\textsc{indf.pl} & \textit{båt-ar} & \textit{skei-er} & \textit{tre-$\emptyset$}\\
& boat-\textsc{indf.pl.m} & spoon-\textsc{indf.pl.f} & tree-\textsc{indf.pl.n}\\
& ‘boats’ & ‘spoons’ & ‘trees’\\
\lspbottomrule
\end{tabular}
\caption{Gender variation in indefinite plural suffixes}
\label{tab:riksem:3}
\end{table}



However, gender may also be considered a feature of Art. The next table shows how definite suffixes also vary according to gender in Norwegian.  


\begin{table}[h]
\begin{tabular}{l lll}
\lsptoprule
 & \textsc{m} & \textsc{f} & \textsc{n}\\\midrule
\textsc{df.sg} & \textit{båt-en} & \textit{skei-a} & \textit{tre-et}\\
& boat-\textsc{df.sg.m} & spoon-\textsc{df.sg.f} & tree-\textsc{df.sg.n}\\
& ‘the boat’ & ‘the spoon’ & ‘the tree’\\
\lspbottomrule
\end{tabular}
\caption{Gender variation in definite singular suffixes}
\label{tab:riksem:4}
\end{table}


A typical analysis of the pattern presented in the two tables above is that gender is a feature of Num and that Art holds an unvalued gender feature that receives valuation through a probe-goal relation with Num. In other words, the lowest locus of gender would be Num, and this feature is subject to gender translation, as discussed above. However, in NAmNo (as well as occasionally in present-day European Norwegian) we see examples like those in \REF{ex:riksem:11}:\footnote{ \textrm{The examples in \REF{ex:riksem:11a} and \REF{ex:riksem:11b} are both transcribed with -}\textrm{\textit{ene} }\textrm{in the orthographic transcription in CANS, which could entail both masculine and feminine gender in some Norwegian varieties and in the written standard} \textrm{\textit{Bokmål.} }\textrm{However, according to the phonological transcription, the speakers use the suffix} \textrm{\textit{{}-an(e)}}\textrm{, which is more clearly masculine.}}


\ea \label{ex:riksem:11}
\ea \label{ex:riksem:11a}
	\gll grade-s-ene\\
		grade-\textsc{pl-df.pl.m} (CANS; %\href{https://tekstlab.uio.no/glossa2/cans3}{
        coon\_valley\_WI\_08gm)\\
\ex \label{ex:riksem:11b}
	\gll brewerie-s-ene\\
		 brewerie-\textsc{pl-df.pl.m} (CANS; %\href{https://tekstlab.uio.no/glossa2/cans3}{
         coon\_valley\_WI\_14gm)\\
\ex \label{ex:riksem:11c}
	\gll pit-s-a\\
		 pit-\textsc{pl-df.pl.n} (CANS; %\href{https://tekstlab.uio.no/glossa2/cans3}{
         sunburg\_MN\_07gm)\\
\z
\z

In these examples, the nominal stem is combined with both an English plural suffix and a Norwegian definite suffix. Due to the plural suffix being English, we can assume that Num does not hold a gender feature here. If it did, a Norwegian exponent would be preferred. Nevertheless, the Norwegian functional suffixes, spelling out Art, show gender variation. Based on data like these, \citet{ÅfarliEtAl2021} argue that not only the nominal stem is mixed from English but the stem and Num together, whereas the rest of the DP is considered Norwegian.\footnote{ \textrm{Admittedly, gender variation is notoriously hard to confirm in cases like \REF{ex:riksem:11}, especially due to the infrequency of such examples. For instance, \REF{ex:riksem:11c} is realized with a suffix typically representing neuter (cf. the overview in \sectref{sec:riksem:2.1}), but there are no other examples of} \textrm{\textit{pit}} \textrm{used with neuter, and one example where} \textrm{\textit{pit}} \textrm{is used with the masculine definite suffix} \textrm{\textit{{}-en} }\textrm{in the corpus. Moreover, we have yet to find clear examples of feminine gender used in such phrases. One potential example is} \textrm{\textit{queen-s-a} }\textrm{‘queen-}\textrm{\textsc{pl-df.pl.n’} }\textrm{(CANS; sunburg\_MN\_12gk) which could be considered feminine based on semantic properties, but the shape of the suffix is nevertheless more compatible with neuter. Finally, we cannot exclude potential dialectal variations in the phonological realization of the suffix. Future studies of this phenomenon should therefore pursue a larger pool of examples and also account for dialect variations.} } A structural representation is provided in \figref{fig:riksem:fromex:12}, where the dotted line marks the border between the Norwegian DP and the mixed items from English. As the noun stem moves to Num and Art, we get the form \textit{gradesene.} 


\begin{figure}
\begin{forest}
[DP
  [D\\
  {$\left[\begin{tabular}{@{} l@{~}l @{}}
  		\textsc{def}: & \textsc{df}\\
  		\textsc{num}: & \textsc{pl}\\
  		\textsc{gen}: & \textsc{m}\\
  	\end{tabular}
  	\right]$}
  	]
  [ArtP,name=artp
    [Art\\
    {$\left[\begin{tabular}{@{} l@{~}l @{}}
    		\textsc{def}: & \textsc{df}\\
    	\end{tabular}
    	\right]$}\\
    \textit{-ene},name=art]
    [NumP,name=nump
      [Num\\
       {$\left[\begin{tabular}{@{} l@{~}l @{}}
       		\textsc{num}: & \textsc{pl}\\
       	\end{tabular}
       	\right]$}\\
       \textit{-s}]
      [n
        [n,name=grade-1]
        [√GRADE,name=grade-2]
      ]
    ]
  ]
]
\path let \p1=(grade-1), 
		  \p2=(grade-2),
		  \p{center} = ($ (\p1) !.5! (\p2) $) 
	  in coordinate (grade-node) at (\p{center});
\node[below=.5\baselineskip of grade-node] {\textit{grade}};
\path let \p1=(artp.north),
		  \p2=(nump)
	  in coordinate (topright) at (\x2,\y1);
\path let \p1=(art),
	      \p2=(grade-node.south)
	  in coordinate (bottomleft) at (\x1,\y2);
\draw [dashed] (bottomleft) -- (topright);
\end{forest}
\caption{Structural representation of (6a) \emph{gradesene}. The dotted line marks the border between the mixed chunck from English and the otherwise Norwegian structure.}
\label{fig:riksem:fromex:12}
\end{figure}  
%%please move the includegraphics inside the {figure} environment
%%\includegraphics[width=\textwidth]{figures/Chapter3AgreementinNAmNoDPs-img007.png}
 



Now, if NumP is drawn from English, but the definite suffix still shows gender variation, there must be a gender feature available in Art. In this case, then, the lowest gender feature is in Art and this is the feature subject to gender translation. We return to data showing English functional exponents, like the plural suffix, in \sectref{sec:riksem:5}.



Then, \citet{ÅfarliEtAl2021} investigate whether an even larger English piece may be mixed into Norwegian and still be assigned gender. In other words, can the lowest gender feature be positioned also in D? The examples presented in \REF{ex:riksem:13} indicate that this is the case, as both the attributive adjectives and the noun stems are English, but the determiners nevertheless show gender variation.


\ea \label{ex:riksem:13}
\ea \label{ex:riksem:13a}
	\gll det \textbf{third} \textbf{grade}\\
		 the.\textsc{n} third grade (CANS; %\href{https://tekstlab.uio.no/glossa2/cans3}{
         westby\_WI\_10gk)\\
\ex \label{ex:riksem:13b}
	\gll en \textbf{covered} \textbf{sled}\\
		 a.\textsc{m} covered sled (CANS; %\href{https://tekstlab.uio.no/glossa2/cans3}{
         gary\_MN\_01gm)\\
\ex \label{ex:riksem:13c}
	\gll en \textbf{hard} \textbf{dialect}\\
		 a.\textsc{m} hard dialect (CANS; %\href{https://tekstlab.uio.no/glossa2/cans3}{
         flom\_MN\_01gm)\\
\z
\z



The structure in \figref{fig:riksem:fromex:14} shows an analysis of \REF{ex:riksem:13b}.


\begin{figure}
\begin{forest}
[DP, name=dp, s sep=1cm
  [D\\
  {$\left[\begin{tabular}{@{} l@{~}l @{}}
  		\textsc{def}: & \textsc{indf}\\
  		\textsc{num}: & \textsc{sg}\\
  		\textsc{gen}: & \textsc{m}\\
  	\end{tabular}
  	\right]$}\\
  	\textit{en}
  	]
  [αP,name=alphaP
    [α\\
    \textit{covered},name=alpha]
    [NumP
      [Num\\
       {$\left[\begin{tabular}{@{} l@{~}l @{}}
       		\textsc{num}: & \textsc{sg}\\
       	\end{tabular}
       	\right]$}
      ]
      [n
        [n,name=sled-1]
        [√SLED,name=sled-2]
      ]
    ]
  ]
]
\path let \p1=(sled-1), 
		  \p2=(sled-2),
		  \p{center} = ($ (\p1) !.5! (\p2) $) 
	  in coordinate (sled-node) at (\p{center});
\node[below=.5\baselineskip of sled-node] {\textit{sled}};
\path let \p1=(alphaP),
		  \p2=(dp)
	  in coordinate (topright) at (\x1,\y2);
\path let \p1=(dp),
	      \p2=(sled-node)
	  in coordinate (bottomleft) at (\x1,\y2);
\draw [dashed] (bottomleft.north) ++(-2cm,0pt) to (topright);
\end{forest}
\caption{Structural representation of (7b) \emph{en covered sled}}
\label{fig:riksem:fromex:14}
\end{figure}
%%please move the includegraphics inside the {figure} environment
%%\includegraphics[width=\textwidth]{figures/Chapter3AgreementinNAmNoDPs-img008.png}
 



\citet{ÅfarliEtAl2021} use these language mixing data to argue that the lowest locus of the formal gender feature is flexible; gender does not occupy one specific position in the structure, but each functional projection in the Norwegian DP spine is argued to hold a gender feature. Moreover, all gender features are assumed to be unvalued at the start of the derivation, and the lowest gender feature is valued through the gender translation mechanism (cf. the discussion above), and thereafter the other gender features (higher in the structure) are valued through the probe-goal mechanism. A general, structural representation is given in \figref{fig:riksem:fromex:15}.


\begin{figure}
\begin{forest}
[DP
  [D\\{[\textsc{ugen}]}]
  [ArtP
    [Art\\{[\textsc{ugen}]}]
    [NumP
      [Num\\{[\textsc{ugen}]}]
      [n
        [n\\{[\textsc{ugen}]}]
        [√ROOT]
      ]
    ]
  ]
]
\end{forest}
\caption{General, structural representation of distributed gender, assuming that all functional projections hold an unvalued gender feature at the start of the derivation}
\label{fig:riksem:fromex:15}
\end{figure}
%%please move the includegraphics inside the {figure} environment
%%\includegraphics[width=\textwidth]{figures/Chapter3AgreementinNAmNoDPs-img009.png}
 



\citet{ÅfarliEtAl2021} argue further that in Norwegian DP structures, the most common locus for the gender feature is the nominalizer, n (see also \citealt{Julien2005}: 3). The pattern where the position of the lowest gender feature is in a higher structural position only occurs in situations where it is forced, so to speak, upwards, for instance in cases where nominal stems or larger chunks are mixed from a non-gender language like English. 


\subsection{A brief note on agreement on attributive adjectives}\label{sec:riksem:4.3}

Before we conclude this section, we will briefly discuss how agreement is realized on attributive adjectives in a language mixing context in NAmNo DPs. These are also subject to agreement in (European) Norwegian DPs (cf. \sectref{sec:riksem:2.1}), but also in these cases language mixing may shed light on the structural architecture as adjectives seem to behave differently than nouns in mixing. 



As discussed in \sectref{sec:riksem:2.1} above, attributive adjectives in Norwegian are realized with a functional suffix corresponding with the definiteness, number, and gender of the DP. Based on a late\hyp insertion model, one would expect that mixed adjectives would also get the appropriate Norwegian suffix, in parallel with the typical pattern for nouns, \REF{ex:riksem:5}. On the contrary, English adjectives are rarely accompanied by Norwegian suffixes when they are mixed into NAmNo. \citet{RiksemEtAl2021} investigate adjectival agreement among 163 present-day speakers in CANS (v. 2) examining both the realization of agreement in present-day NAmNo and potential changes compared to historical NAmNo. 



A first discovery in this search is that there are in general few English adjectives appearing together with a Norwegian noun; only 12 relevant cases combining an English adjective and a Norwegian noun were identified in the corpus. For many of these, no functional suffix is expected in the specific context (cf. the discussion in \sectref{sec:riksem:2.1}). In the cases where a functional suffix is expected, this is only realized in two examples~– involving the same adjective and noun combination, uttered by the same speaker. One example is given in \REF{ex:riksem:16a}. The examples in (\ref{ex:riksem:16}b–c), on the other hand, show phrases where a functional suffix would be expected from a Norwegian point of view: \textit{land} in \REF{ex:riksem:16b} is typically neuter in Norwegian, meaning that the adjective should be realized with the suffix -\textit{t}, and \REF{ex:riksem:16c} is a definite phrase entailing a suffixed \textit{{}-e} to the adjective.\footnote{The pronunciation of \emph{land} in (8b) is clearly Norwegian [lan].}         


\ea \label{ex:riksem:16}
\ea \label{ex:riksem:16a}
\gll Det er ikke noen \textbf{small}{}-e farm-er\\
	 It     is   not  some small-\textsc{w} farm-\textsc{indf.sg}\\
\glt ‘There are no small farms’  (CANS; westby\_WI\_06gm)
\ex \label{ex:riksem:16b}
\gll Det var \textbf{cheap} land her\\
	 It     was cheap land here\\
\glt ‘There was cheap land here’ (CANS; glasgow\_MT\_01gm)
\ex \label{ex:riksem:16c}
\gll my \textbf{eld-est} søster\\
	 my old-\textsc{sup} sister\\
\glt ‘my oldest sister’ (CANS; vancouver\_WA\_01gm)
\z
\z

From a diachronic perspective, \citet{RiksemEtAl2021} find that this is not a recent phenomenon in present-day NAmNo as both \citet{Haugen1953} and \citet{Hjelde1992} describe a pattern where only a handful of English adjectives receive a Norwegian functional suffix. Hence, it is not the result of a diachronic change in NAmNo grammar. Instead, \citet{RiksemEtAl2021} propose an analysis whereby structural differences can account for this pattern. They argue that the agreement relation between, e.g., the functional features in D and in the lower functional heads in the DP is a direct one; they are all part of the extended projection of the DP. However, when the DP includes an attributive adjective, this is generated in a separate functional head in the specifier position of αP, potentially like the structure in \figref{fig:riksem:fromex:17}. Notice that, for ease of exposition, the functional features in this structure are gathered in feature bundles in designated functional heads.


\begin{figure}
\begin{forest}
[
[,phantom]
[DP
  [D\\
  {$\left[\begin{tabular}{@{} l@{~}l @{}}
  		\textsc{def}: & \textsc{indf}\\
  		\textsc{num}: & \textsc{sg}\\
  		\textsc{gen}: & \textsc{n}\\
  	\end{tabular}
  	\right]$}  
  ]
  [αP
    [FP
      [F\\
      {$\left[\begin{tabular}{@{} l@{~}l @{}}
      		\textsc{def}: & \textsc{indf}\\
      		\textsc{num}: & \textsc{sg}\\
      		\textsc{gen}: & \textsc{n}\\
      	\end{tabular}
      	\right]$}]
      [a
        [a,name=cheap-1]
        [√CHEAP,name=cheap-2]
      ]
    ]
    [α'
      [α]
      [FP
        [F\\
        {$\left[\begin{tabular}{@{} l@{~}l @{}}
        		\textsc{def}: & \textsc{indf}\\
        		\textsc{num}: & \textsc{sg}\\
        		\textsc{gen}: & \textsc{n}\\
        	\end{tabular}
        	\right]$}]
        [n
          [n,name=land-1]
          [√LAND,name=land-2]
        ]
      ]
    ]
  ]
]
]
\foreach \i in {cheap,land}
  {
	\path let \p1=(\i-1), 
			  \p2=(\i-2),
			  \p{center} = ($ (\p1) !.5! (\p2) $) 
		  in coordinate (\i-node) at (\p{center});
	\node[below=.5\baselineskip of \i-node] {\textit{\i}};
  }
\end{forest}
\caption{Potential structural representation of (8b) \emph{cheap land}}
\label{fig:riksem:fromex:17}
\end{figure}
%%please move the includegraphics inside the {figure} environment
%%\includegraphics[width=\textwidth]{figures/Chapter3AgreementinNAmNoDPs-img010.png}
 

The functional features associated with the adjective do then not form a direct agreement relation with the functional features associated with the nominal. \citet{RiksemEtAl2021} instead argue that this agreement relation is an indirect one, which could potentially be more vulnerable, and that this might explain why so few adjectives occur with Norwegian functional suffixes. 



We leave it up to future research to investigate the structure of adjectives in more detail. In the rest of the chapter, we turn our attention back to determiners and the functional suffix on the noun stem, and potential diachronic changes in the agreement patterns of NAmNo DPs.


\section{(Diachronic) change in the morphophonological realization of functional exponents}\label{sec:riksem:5}

In this section, we will look at some different language mixing data from present-day NAmNo, more specifically cases where the functional exponent is either English or missing. In other words, they show patterns that diverge from the analysis of the typical mixing pattern discussed in \sectref{sec:riksem:4.1} above. Based on these data, and by comparison with historical NAmNo (primarily \citealt{Haugen1953}), we will discuss how agreement patterns and the morphological shape of inflectional suffixes may change over time. An interesting question is consequently the nature of such a change: Is it due to change in the exponents themselves, or the outcome of a structural reanalysis in the heritage language? In other words, this section deals with both our main interests in this chapter, as they concern both surface changes in the realization of agreement and a discussion of what this implies for the underlying structure of the DP. 



The following discussion is largely based on the work by \citet{Riksem2017}. She investigates diachronic changes in agreement patterns in mixed DPs in NAmNo comparing historical NAmNo (primarily \citealt{Haugen1953}) and present-day NAmNo in CANS, and presents potential analyses. Concerning gender, she also consults \citet{Flom1900, Flom1903} and \citet{Hjelde1992} and finds that the distribution of gender assigned to English nouns has been relatively stable over time. About 70\% of the English stems are assigned masculine gender, 10\% or less are assigned feminine gender and assignment to neuter gender varies from 6 to 16\%. Some English stems also appear with different genders. As there is no clear pattern of diachronic change in this category, \citet{Riksem2017} does not discuss gender further, and in the following we will also continue focusing on number and definiteness.



The examples we discuss in the following are mostly taken from \citet{Riksem2017}. However, in certain cases we have supplemented the data with a few targeted searches in CANS. The most obvious supplements are examples from Haugen’s recordings, which were made available in CANS (v. 3.1) in 2021. These examples are highlighted by including the year of the recording (1942) along with the speaker code in the following. 


\begin{sloppypar}
The DPs we discuss in the following subsections are either realized without an overt realization of functional features, or they are realized with English functional exponents. These data are especially interesting because they show a change in the phonological realization of agreement in NAmNo DPs, and they may also be tokens of changes in the abstract syntactic representations. As we will see, there might be various analyses of such data, and we will explore some potential analyses in \sectref{sec:riksem:5.3}. First, we discuss the observed diachronic changes concerning number in \sectref{sec:riksem:5.1} and definiteness in \sectref{sec:riksem:5.2}. 
\end{sloppypar}

\subsection{Number}\label{sec:riksem:5.1}

Among the present-day speakers in CANS, many of the plural DPs with an English noun are realized with an English functional suffix (\textit{{}-s}) rather than a Norwegian suffix. According to \citet[440]{Haugen1953}, when English forms were used in NAmNo, they were \textit{usually} provided with the appropriate Norwegian functional items, which we find in (\ref{ex:riksem:18}a–c).\footnote{Assuming that \textit{jar}, in \REF{ex:riksem:18c}, is assigned neuter gender, a null ending (${\emptyset}$) is expected in a Norwegian DP.} Still, occasional usage of the English plural \textit{{}-s} is observed by \citet[450]{Haugen1953}, as shown in \REF{ex:riksem:18d}.\largerpage


\ea \label{ex:riksem:18}
\ea \label{ex:riksem:18a}
	\gll femti \textbf{bush}{}-er\\
		 fifty bush-\textsc{indf.pl.m/f} (CANS-1942; %\href{https://tekstlab.uio.no/glossa2/cans3}{
         blair\_WI\_22um)\\
\ex \label{ex:riksem:18b}
	\gll noen \textbf{crop}{}-er\\
		some crop-\textsc{indf.pl.m/f} (CANS-1942; %\href{https://tekstlab.uio.no/glossa2/cans3}{
        ferryville\_WI\_04gm)\\
\ex \label{ex:riksem:18c}
	\gll store \textbf{jar}{}-\textsc{${\emptyset}$}\\
		 big jar-\textsc{indf.pl.n} (CANS-1942; %\href{https://tekstlab.uio.no/glossa2/cans3}{
         blair\_WI\_34gm)\\
\ex \label{ex:riksem:18d}
	\gll gamle \textbf{antique-s}\\
		 old antique-\textsc{pl} (CANS-1942; %\href{https://tekstlab.uio.no/glossa2/cans3}{
         beaver\_creek\_WI\_01gk)\\
\z
\z

To account for this observation, \citet{Haugen1953} separates the speakers into pre-bilingual borrowers and childhood bilinguals, and he argues that the former group of speakers take the \textit{{}-s} to be part of the nominal stem, whereas the latter group of speakers may use the \textit{{}-s} as a realization of plurality. All the 50 speakers investigated by \citet{Riksem2017} are childhood bilinguals, and presumably they use the \textit{{}-s} as an actual exponent for plurality. In fact, \citet{Riksem2017} finds that the majority of indefinite plural phrases involving an English noun has the plural \textit{{}-s.} Hence, we may assume that these speakers use the plural \textit{{}-s} as a replacement for the Norwegian plural suffix. Among these data, most cases are indefinite, but there are also some definite phrases. Some examples are provided in \REF{ex:riksem:19}. 


\ea \label{ex:riksem:19}
\ea \label{ex:riksem:19a}
	\gll mange \textbf{lawyer-s}\\
		 many lawyer-\textsc{pl} (CANS; %\href{https://tekstlab.uio.no/glossa2/cans3}{
         sunburg\_MN\_03gm)\\
\ex \label{ex:riksem:19b}
	\gll andre \textbf{tool-s}\\
		 other tool-\textsc{pl} (CANS; sunburg\_MN\_03gm)\\
\ex \label{ex:riksem:19c}
	\gll fem \textbf{dialect-s}\\
		 five dialect-\textsc{pl} (CANS; portland\_ND\_01gm)\\
\ex \label{ex:riksem:19bd}
	\gll alle disse \textbf{minute-s}\\
		 all these minute-\textsc{pl} (CANS; %\href{https://tekstlab.uio.no/glossa2/cans3}{
         stillwater\_MN\_01gm)\\
\ex \label{ex:riksem:19be}
	 \gll de samme \textbf{gene-s}\\
		  the same gene-\textsc{pl} (CANS; %\href{https://tekstlab.uio.no/glossa2/cans3}{
          flom\_MN\_02gm)\\
\z
\z



In a handful of examples, the plural \textit{{}-s} is also used together with a Norwegian noun stem, \REF{ex:riksem:20}. This amplifies the impression that the suffix is active, i.e., it is the realization of a plural feature, and not interpreted as part of the nominal stem.\footnote{The stem \emph{mil} in (11b) is pronounced in a distinct Norwegian way, [mi:l].}


\ea\label{ex:riksem:20}
\ea 
	\gll innvandrer-\textbf{s}\\
		 immigrant-\textsc{pl} (CANS; %\href{https://tekstlab.uio.no/glossa2/cans3}{
         flom\_MN\_01gm)\\
\ex 
	\gll mil-\textbf{s}\\            
		 mile-\textsc{pl} (CANS; sunburg\_MN\_03gm)\\
\ex 
	\gll spise-plass-\textbf{es}\\
		 eat-place-\textsc{pl}\\
	 \glt ‘places to eat’ (CANS; coon\_valley\_WI\_01gk) 
\z
\z



One speaker, in \REF{ex:riksem:21}, clearly demonstrates how the plural \textit{{}-s} should not be considered part of the stem, but acts as a functional exponent of plurality, as she uses the stem \textit{bluebird} both with and without the \textit{{}-s} in the same conversation: 


\ea\label{ex:riksem:21}
\ea 
\gll Jeg trur  det er en   \textbf{bluebird} som sitter der     nå\\
	 I     think it    is  a.\textsc{m} bluebird that  sits    there now\\
    
\ex
\gll Vi    hadde familier av \textbf{bluebirds} dette året \\
	 We had      families of bluebirds this    year   \\
     \glt (CANS; %\href{https://tekstlab.uio.no/glossa2/cans3}{
     coon\_valley\_WI\_01gk)
\z
\z



A different pattern found in the realization of plurality in NAmNo is plural DPs without a functional suffix at all. Some examples are given in \REF{ex:riksem:22}. 


\ea\label{ex:riksem:22}
\ea 
\gll fem seks hour\_\\
	 five six hour (CANS; %\href{https://tekstlab.uio.no/glossa2/cans3}{
     spring\_grove\_MN\_05gm)\\
\ex 
\gll flere store\_\\
	 several store (CANS; %\href{https://tekstlab.uio.no/glossa2/cans3}{
     westby\_WI\_03gk)\\
\ex 
\gll mange memorial\_\\
	many memorial (CANS; %\href{https://tekstlab.uio.no/glossa2/cans3}{
    billings\_MT\_01gm)\\
\z
\z



Accompanying words like weak quantifiers tell us that these are plural phrases, but the lack of a functional suffix realizing plurality is an unexpected pattern both in Norwegian and English structures. This pattern is not discussed in \citet{Haugen1953} and thus appears to be a change in the realization of agreement in present-day NAmNo.\footnote{ \textrm{A few targeted searches in CANS using a weak quantifier (}\textrm{\textit{mange} }\textrm{‘many’}\textrm{\textit{, flere} }\textrm{‘several’}\textrm{\textit{, noen} }\textrm{‘some’) in combination with an English noun support this claim as the pattern in \REF{ex:riksem:22} is not found in Haugen’s data.} }


\subsection{Definiteness}\label{sec:riksem:5.2}

Also concerning definiteness some diachronic changes have been observed when comparing present-day NAmNo to \citet{Haugen1953}. According to \citet[451]{Haugen1953}, definiteness behaves as expected from a Norwegian perspective: The appropriate definite suffix is added when relevant, and in cases of double definiteness, a determiner is present as well. This is also the case in the typical mixing pattern in CANS as discussed in \sectref{sec:riksem:4.1}. above (see \citealt{Riksem2018Language}). Some examples of double definiteness in mixed phrases both from Haugen’s 1942-recordings and present-day NAmNo are given in \REF{ex:riksem:23}, and these phrases show the expected agreement between the determiner, the attributive adjective, and the functional suffix.\footnote{ \textrm{According to the phonological transcription, the speaker in (14d) uses the suffix} \textrm{\textit{{}-ane}}\textrm{, which is typically masculine, even though the orthographic transcription is} \textrm{\textit{{}-ene.} }}

\ea\label{ex:riksem:23}
\ea 
\gll den andre \textbf{crop}{}-en\\
	the second crop-\textsc{df.sg.m} (CANS-1942; %\href{https://tekstlab.uio.no/glossa2/cans3}{
    blair\_WI\_22um)\\
\ex 
\gll denne \textbf{claim}{}-en\\
	 this claim-\textsc{df.sg.m} (CANS-1942; %\href{https://tekstlab.uio.no/glossa2/cans3}{
     westby\_WI\_24gm)\\
\ex 
\gll det \textbf{shanty}{}-et\\
	 that shanty-\textsc{df.sg.n} (CANS-1942; %\href{https://tekstlab.uio.no/glossa2/cans3}{
     new\_auburn\_WI\_01um)\\
\ex 
\gll disse \textbf{log}{}-ene\\
	these log-\textsc{df.pl.m} (CANS-1942; %\href{https://tekstlab.uio.no/glossa2/cans3}{
    viroqua\_WI\_04gm)\\
\ex 
\gll den beste aure\textbf{{}-creek-}en\\
	 the best  {brown trout creek}-\textsc{df.sg.m}\\
\glt `the best creek to fish brown trout’ (CANS; %\href{https://tekstlab.uio.no/glossa2/cans3}{
coon\_valley\_WI\_06gm)
\ex 
\gll dette gamle \textbf{stuff-}et\\
	 this      old     stuff-\textsc{df.sg.n} (CANS; %\href{https://tekstlab.uio.no/glossa2/cans3}{
     blair\_WI\_07gm)\\
\z
\z
\largerpage[2]



In parallel with the discussion of Number above, we find definite DPs where an expected Norwegian functional exponent is either lacking or replaced by an English one. In \REF{ex:riksem:24} we present some data where the expected definite suffix is missing.\footnote{Haugen’s data typically include the definite suffix as described by \citet[451]{Haugen1953}. A couple of exceptions are nevertheless \textit{det første} \textrm{\textbf{\textit{year\_} }}\textrm{‘the first year’ (CANS-1942;} {\textrm{chetek\_WI\_01gk}}\textrm{) and} \textrm{\textit{det norske} }\textrm{\textbf{\textit{course}}}\textrm{\textit{\_} }\textrm{‘the Norwegian course’ (CANS-1942;} {\textrm{iola\_WI\_09gm}}\textrm{), in which a neuter definite suffix (}\textrm{\textit{{}-et}}) \textrm{would be expected in both.}}


\ea\label{ex:riksem:24}
\ea\label{ex:riksem:24a}
	 
	\gll den \textbf{birdhouse}\_\\
		 the.\textsc{m/f} birdhouse (CANS; %\href{https://tekstlab.uio.no/glossa2/cans3}{
         coon\_valley\_WI\_12gm)\\
\ex\label{ex:riksem:24b} 
	
	\gll denne \textbf{cheese}\_\\
		 this.\textsc{m/f} cheese (CANS; %\href{https://tekstlab.uio.no/glossa2/cans3}{
         blair\_WI\_04gk)\\
\ex\label{ex:riksem:24c} 
	
	\gll det \textbf{candy}\_\\
		 the.\textsc{n} candy (CANS; %\href{https://tekstlab.uio.no/glossa2/cans3}{
         sunburg\_MN\_13gk)\\
\ex\label{ex:riksem:24d} 
	
	\gll den andre \textbf{sister}\_\\
		 the.\textsc{m/f} other sister (CANS; %\href{https://tekstlab.uio.no/glossa2/cans3}{
         saskatoon\_SK\_14gk)\\
\ex\label{ex:riksem:24e} 
	
	\gll den store \textbf{building}\_\\
		 the.\textsc{m/f} big building (CANS; %\href{https://tekstlab.uio.no/glossa2/cans3}{
         chicago\_IL\_01gk)\\
\ex\label{ex:riksem:24f} 
	
	\gll det gamle \textbf{stuff}\_\\
		 the.\textsc{n} old stuff (CANS; %\href{https://tekstlab.uio.no/glossa2/cans3}{
         chicago\_IL\_01gk)\\
\z
\z



As mentioned in \sectref{sec:riksem:3}, \citeauthor{vanBaal2020} (\citeyear{vanBaal2020}, \citeyear{chapters/vanbaal} [this volume]) discusses definiteness and double definiteness in detail and finds that even though double definiteness is still used in present-day NAmNo, it is used less than in historical NAmNo and in European Norwegian. Furthermore, she finds that many phrases where double definiteness is expected are either missing the prenominal determiner, the definite suffix or both. In most cases, the phrase contains the suffix but not the prenominal determiner, and \citet[218]{vanBaal2020} argues that this is the typical pattern for NAmNo modified definite phrases.\footnote{It is interesting that the speakers do not use the most English-like pattern with only a prenominal determiner in these phrases, due to the influence from English. Investigating the position of possessive pronouns, \citet{AnderssenEtAl2018} also find that most speakers use the pattern most different from English, in that case a postnominal possessive. They also find a correlation between the two patterns, roughly forming two groups: Speakers utilizing English-like patterns (no definite suffix in double definite phrases and a prenominal possessive) and speakers using the pattern most deviant from English (functional suffix in double definite phrases and a postnominal possessive).} Among our mixed phrases, we also find cases where the functional suffix is realized, but no determiner, even though this would be expected from the context, \REF{ex:riksem:25}. In this respect, the data presented in \REF{ex:riksem:24} are similar to the more English-like pattern having a prenominal determiner but no definite suffix.


\ea \label{ex:riksem:25}
\ea 
\gll beste \textbf{plac[e]}{}-en\\
	 best place-\textsc{df.sg.m} (CANS; spring\_grove\_MN\_05gm)\\
\ex 
\gll norske \textbf{family}{}-en\\
	 Norwegian family-\textsc{df.sg.m} (CANS; %\href{https://tekstlab.uio.no/glossa2/cans3}{
     flom\_MN\_02gm)\\
\ex 
\gll første \textbf{shower}{}-en\\
	 first shower-\textsc{df.sg.m} (CANS; %\href{https://tekstlab.uio.no/glossa2/cans3}{
     coon\_valley\_WI\_20gm)\\
\z
\z



However, we have not explored the frequency of these patterns and can thus not say whether one is more or less common than the other in mixed DPs. For our purposes, what is interesting about data like those in \REF{ex:riksem:24} and \REF{ex:riksem:25} is the missing overt realization of a suffix or a determiner that is expected if the underlying structure and its features are Norwegian.  



Moreover, we also find mixed DPs where the Norwegian determiner has been replaced by the English \textit{the}. This pattern is described by \citet[451]{Haugen1953} as unacceptable in NAmNo, which suggests that this is a diachronic change in heritage Norwegian grammars.\footnote{ \textrm{Also in this case, we do find a few exceptions in Haugen’s recordings:} \textrm{\textit{the gråsteinskirke} }\textrm{‘the grey stone church’ (CANS-1942; spring\_grove\_MN\_19gm) and} \textrm{\textit{the natt}} \textrm{‘the night’ (CANS-1942; spring\_grove\_MN\_18um).}} At the same time it is important to note that there is substantial individual variation among the speakers in CANS, and these observed changes do not necessarily represent general and uniform diachronic changes (see also \citealt{AnderssenEtAl2018}).\footnote{\textrm{The pronunciation of \emph{land} in (17d) is [lan] which is clearly Norwegian.} } 


\ea\label{ex:riksem:26}
\ea \label{ex:riksem:26a}
\gll \textbf{the} by\\
	 the city (CANS; %\href{https://tekstlab.uio.no/glossa2/cans3}{
     chicago\_IL\_01gk)\\
\ex \label{ex:riksem:26b}
\gll \textbf{the} ungdom\\
	 the youth (CANS; harmony\_IL\_01gk)\\
\ex \label{ex:riksem:26c}
\gll \textbf{the} gammalost\\
	 the old-cheese\\
\glt ‘the mature cheese’ (CANS; %\href{https://tekstlab.uio.no/glossa2/cans3}{
gary\_MN\_01gm)
\ex \label{ex:riksem:26d}
\gll \textbf{the} land\\
	 the land (CANS; %\href{https://tekstlab.uio.no/glossa2/cans3}{
     gary\_MN\_01gm)\\
\z
\z

Notice that the examples in \REF{ex:riksem:26} also lack the Norwegian definite suffix, displaying a DP structure quite similar to English DPs. This could suggest that instead of an English item being mixed into a Norwegian structure, these cases are examples of Norwegian noun stems being mixed into an English structure. This could be a reasonable analysis. However, many of the DPs in question appear in utterances that are otherwise primarily Norwegian, as shown for some cases in \REF{ex:riksem:27}. 


\ea\label{ex:riksem:27}
\ea
\gll Jeg husker ikke \textbf{the} by der vi stoppet\\
	 I remember not the city there we stopped\\
\glt ‘I don’t remember the city where we stopped’

\ex
\gll They bor er \# \textbf{the} land som \textbf{the} indianer har\\
	 They live are \# the land that the Indians have\\
\glt ‘They live on the land that the Native Americans own’
\z
\z



Instead of an analysis where the structure of a sentence switches, so to say, from Norwegian to English and back to Norwegian for the sake of a single DP, we suggest that these unexpected patterns are indications of a change in the NAmNo grammar allowing an English functional exponent to be inserted. 



Amplifying this suggestion are DPs exhibiting both an English determiner and a Norwegian definite suffix, \REF{ex:riksem:28}. In these cases, double definiteness is retained, but the determiner is realized by an English exponent. We argue that these structures still have Art, where the definite suffix is generated, but the functional features of the determiner allow the insertion of an English functional exponent.  


\ea\label{ex:riksem:28}
\ea \label{ex:riksem:28a}
\gll \textbf{the} gård-en\\
	 the farm-\textsc{df.sg.m} (CANS; gary\_MN\_01gm)\\
\ex \label{ex:riksem:28b}
\gll \textbf{the} topp-en\\
	 the top-\textsc{df.sg.m} (CANS; gary\_MN\_01gm)\\
\ex \label{ex:riksem:28c}
\gll \textbf{the} rest-en\\
	 \textbf{the} rest-\textsc{df.sg.m} (CANS; %%\href{https://tekstlab.uio.no/glossa2/cans3}{
     vancouver\_WA\_03uk)\\
\ex \label{ex:riksem:28d}
\gll \textbf{the} katt-a\\
	 the cat-\textsc{df.sg.f}  (CANS; %\href{https://tekstlab.uio.no/glossa2/cans3}{
     saskatoon\_SK\_14gk)\\
\ex \label{ex:riksem:28e} 
\gll \textbf{the} samme tid-a\\
	 the same time-\textsc{df.sg.f} (CANS; %\href{https://tekstlab.uio.no/glossa2/cans3}{
     westby\_WI\_10gk)\\
\z
\z

In the next section we discuss possible analyses of these changes. 


\subsection{Analyzing the diachronic changes}\label{sec:riksem:5.3}

In this section we consider some possible analyses of the data presented above that show signs of diachronic changes in the (overt) realizations of agreement. Subsequently, we also consider whether these data could be the manifestation of a more fundamental change in syntactic representation of agreement. The NAmNo data we have considered thus far may not be sufficient to draw strong conclusions, but they are nevertheless examples that may help illuminate potential changes in a heritage grammar. See also \citet{Lohndal2021} for more discussion.



First, we will consider how the distributed gender analysis (cf. \citealt{ÅfarliEtAl2021}) could be utilized in explaining some of the observed changes. In this analysis, chunks larger than the noun stem alone may be mixed into a Norwegian structure without compromising the realization of gender in higher positions. Since all functional heads in the DP are assumed to hold an unvalued gender feature, and that gender is assigned through gender translation, the value of the gender feature will be assigned in the lowest available position in a given phrase. Let us consider an example where the expected Norwegian functional suffix is not realized, like \REF{ex:riksem:24c} \textit{det candy} ‘the.\textsc{n} candy’ (see Figure \ref{fig:riksem:fromex:29}). 

\begin{figure}
\begin{forest}
	[DP, s sep=1cm, name=dp
		[D\\
		{$\left[\begin{tabular}{@{} l@{~}l @{}}
				\textsc{def}: & \textsc{df}\\
				\textsc{num}: & \textsc{sg}\\
				\textsc{gen}: & \textsc{n}\\
			\end{tabular}
			\right]$}\\
		\textit{det},name=det
		]
		[NumP,name=nump
			[Num\\
			{$\left[\begin{tabular}{@{} l@{~}l @{}}
					\textsc{num}: & \textsc{sg}\\
				\end{tabular}
				\right]$}
			]
			[n
				[n,name=candy-1]
				[√CANDY,name=candy-2]
			]
		]
	]
	\path let 	\p1=(candy-1), 
				\p2=(candy-2),
				\p{center} = ($ (\p1) !.5! (\p2) $) 
		  in coordinate (candy-node) at (\p{center});
	\node[below=.5\baselineskip of candy-node] {\textit{candy}};
	\path let 	\p1=(nump.north),
				\p2=(dp.north)
	    	in coordinate (topright) at (\x1,\y2);
	\path let \p1=(dp.west),
			  \p2=(candy-node.south)
	in coordinate (bottomleft) at (\x1,\y2);
	\draw [dashed] (topright) ++(-1cm,0pt) -- (bottomleft);
\end{forest}
\caption{Potential structural representation of (15c) \textit{det candy}}
\label{fig:riksem:fromex:29}
\end{figure}
%%please move the includegraphics inside the {figure} environment
%%\includegraphics[width=\textwidth]{figures/Chapter3AgreementinNAmNoDPs-img011.png}
 



In this analysis, we would assume that not only the noun stem is mixed from English, but also Num, and since English does not have a functional exponent for singular, the phonological realization is \textit{candy}. The D projection, on the other hand, may still be Norwegian, holding an unvalued gender feature which is valued based on the gender translation mechanism. In this case it is assigned neuter gender and spelled out by the determiner \textit{det}. 



Also the data in \REF{ex:riksem:19}, where the English plural \textit{{}-s} is used instead of a Norwegian exponent, could be analysed in this model. Then the mixed chunk would be parallel to the one in \figref{fig:riksem:fromex:29}, but as the number feature is plural, the exponent \textit{{}-s} is inserted. A potential gender feature in the higher structure in such cases is nevertheless more difficult to validate due to syncretism in plural determiners and weak quantifiers. 



However, encountering other types of diachronic change, like the usage of the plural \textit{{}-s} with Norwegian noun stems, \REF{ex:riksem:20}, and the English determiner \textit{the}, \REF{ex:riksem:26} and \REF{ex:riksem:28}, the distributed gender hypothesis alone is not able to account for the unexpected patterns. Instead, we might be facing some more overreaching questions concerning syntactic structures and their phonological realizations. In analyzing the observed diachronic changes, \citet{Riksem2017} proposes two potential explanations: either the structure is intact, and the observed change is due to changes in the exponents, or the syntactic structure itself is undergoing change. In the final part of this chapter, we will present these two approaches and how they may account for the changes in the agreement patterns of the NAmNo DP.



Support for the first scenario, that the observed change is due to changes in the phonological exponents, is found in the Missing Surface Inflection Hypothesis (\citealt{Lardiere2000, PrévostWhite2000}). This hypothesis is motivated by research on second language acquisition, and put briefly it implies that the syntactic terminals and functional features may be retained even though overt morphology is missing. Due to factors like limited input, the speaker might not hold a full repertoire of functional exponents matching the possible conditions in the syntactic structure. Therefore, the speaker will rather avoid inserting a form than using the wrong form. 



Such an analysis may explain some of the data discussed earlier in this section, in particular the data where the expected functional exponent is not realized; the speakers might be avoiding using a functional exponent considering no insertion to be a better alternative than a wrongful insertion. By extension, also the usage of the English plural \textit{{}-s} could be the outcome of such an avoidance strategy, where this plural suffix is considered an all-round alternative, avoiding the risk of inserting the wrong Norwegian suffix with respect to gender. However, \citet{Riksem2017} goes on to problematize such an analysis, arguing that anything potentially could be explained as an avoidance strategy and that the analysis therefore lacks clear predictions as to what can and cannot be avoided. 



The second analysis is that the abstract syntactic structure itself is, or has been, undergoing changes. This has been discussed in relation to other heritage grammars, e.g., heritage Russian (\citealt{Polinsky2011,Polinsky2016}), heritage Spanish \citep{ScontrasEtAl2015} and heritage German \citep{YagerEtAl2015}. Due to factors like absence of consistent input, the availability of functional features in the syntactic structures may be changed, resulting in a structural reanalysis of the heritage grammar (\citealt{Polinsky2011, PutnamSánchez2013}). This will consequently complicate the insertion of phonological exponents potentially blocking functional exponents expected in the baseline or enabling the insertion of different functional exponents. 



In the discussion of diachronic changes in NAmNo, \citet{Riksem2017} argues in favor of such a structural reanalysis of the DP. More specifically she uses gender as a potential domain for such a change. If the gender feature in one or more syntactic terminal is diminishing, this would affect the exponents available for insertion, and an English exponent could be preferred over the Norwegian one. Take for instance the examples with the English determiner \textit{the.} For this exponent to be inserted, at the expense of a Norwegian one, we assume that the gender feature in D must be weakened or erased from this terminal. Otherwise, the Norwegian alternatives would be more specified and therefore preferred for insertion. Thus, changes in the syntactic architecture may have clear consequences for the phonological realizations. 



Importantly, the data from NAmNo do not show an abrupt change in the mixing patterns, but rather indication of an ongoing structural reanalysis. To illustrate this, we can consider the examples in \REF{ex:riksem:26} and \REF{ex:riksem:28} above with the English determiner \textit{the}. \figref{fig:reksim:fromex:30} shows a potential structural representation of \REF{ex:riksem:28a} \textit{the gården} ‘the farm-\textsc{df.sg.m}’, and \figref{fig:reksim:fromex:31} shows a potential representation of \REF{ex:riksem:26a} \textit{the by} ‘the city’. 


\begin{figure}
\begin{forest}
  [DP
    [D\\
    {$\left[\begin{tabular}{@{} l@{~}l @{}}
    		\textsc{def}: & \textsc{df}\\
    	\end{tabular}
    	\right]$}\\
    \textit{the}]
    [ArtP
      [Art\\
      {$\left[\begin{tabular}{@{} l@{~}l @{}}
      		\textsc{def}: & \textsc{df}\\
      		\textsc{gen}: & \textsc{m}\\
      	\end{tabular}
      	\right]$}\\
      \textit{-en}]
      [NumP
        [Num\\
        {$\left[\begin{tabular}{@{} l@{~}l @{}}
        		\textsc{num}: & \textsc{sg}\\
        		\textsc{gen}: & \textsc{m}\\
        	\end{tabular}
        	\right]$}]
        [n
          [n\\
          {$\left[\begin{tabular}{@{} l@{~}l @{}}
          		\textsc{gen}: & \textsc{m}\\
          	\end{tabular}
          	\right]$},name=gard-1]
          [√GÅRD,name=gard-2]
        ]
      ]
    ]
  ]
\path let \p1=(gard-1), 
		  \p2=(gard-2),
		  \p{center} = ($ (\p1) !.5! (\p2) $) 
	  in coordinate (gard-node) at (\p{center});
\node[below=1.5\baselineskip of gard-node] {\textit{gård}};
\end{forest}
\caption{Potential structural representation of (19a) \textit{the gården}}
\label{fig:reksim:fromex:30}
\end{figure}
%%please move the includegraphics inside the {figure} environment
%%\includegraphics[width=\textwidth]{figures/Chapter3AgreementinNAmNoDPs-img012.png}
 

\begin{figure}
\begin{forest}
[
[,phantom]
[DP
    [D\\
    {$\left[\begin{tabular}{@{} l@{~}l @{}}
    		\textsc{def}: & \textsc{df}\\
    	\end{tabular}
    	\right]$}\\
    \textit{the}]
	[NumP
		[Num\\
		 {$\left[\begin{tabular}{@{} l@{~}l @{}}
	      		\textsc{num}: & \textsc{sg}\\
	      	\end{tabular}\right]$}]
		[n
		  	[n,name=by-1]
		  	[√BY,name=by-2]
		]	
	]
]
]
\path let \p1=(by-1), 
		  \p2=(by-2),
		  \p{center} = ($ (\p1) !.5! (\p2) $) 
	  in coordinate (by-node) at (\p{center});
\node[below=.5\baselineskip of by-node] {\textit{by}};
\end{forest}
\caption{Potential structural representation of (17a) \textit{the by}}
\label{fig:reksim:fromex:31}
\end{figure}
%%please move the includegraphics inside the {figure} environment
%%\includegraphics[width=\textwidth]{figures/Chapter3AgreementinNAmNoDPs-img013.png}
 



Common for these structures is the D head, which only holds a definiteness feature, thus allowing the insertion of the determiner \textit{the}. Considering the rest of the structures, \figref{fig:reksim:fromex:30} may be considered an in-between stage where the gender feature is still present in the syntactic heads below D, and the presence of Art also ensures the realization of a definite suffix. The structure in \figref{fig:reksim:fromex:31}, would then represent a progression of this reanalysis, where the gender feature is not present at all in the DP, and the lack of the definite suffix may also indicate the absence of Art. If this is the case, \figref{fig:reksim:fromex:31} also indicates a change that has progressed so far that the structural architecture now resembles the English DP structure, and consequently the Norwegian agreement patterns are no longer expected.   



Before we conclude this chapter, it is nevertheless important to point out that the data discussed in this section are not very frequent. More data is therefore needed in order to further develop the analyses of such patterns. 


\section{Conclusion}\label{sec:riksem:6}

In this chapter, we have reviewed some of the main findings concerning agreement in NAmNo DPs. A large part of the chapter has been dedicated to language mixing as these data provide valuable windows into the details of syntactic structures. We set out with a twofold goal, namely to describe the attested patterns of agreement in the DPs, and also to describe the underlying syntactic structure of the DP. In this respect, studying agreement patterns in mixed NAmNo nominal structures is fruitful as it involves the contact between two grammars, Norwegian and English, with some important differences in the DP structure. 



Analyses of the typical pattern of language mixing shows that English nouns may be inserted into Norwegian DP structures, where they are accompanied by Norwegian functional items varying according to gender, number, and definiteness. A late-insertion exoskeletal model is well-suited to account for these data by using the same constraints as for monolingual DPs. Hence, this theoretical model constitutes a Null Theory of language mixing. Studying language mixing data will thus not only give us insights into this specific outcome of bilingualism, but they may also inform our understanding of grammatical representations in general. 



We have employed language mixing data to highlight two issues in particular: First, we use gender as a key feature in exploring the syntactic architecture of the DP. Showing how English items of varying size may be mixed into a Norwegian structure – and this structure is still marked for gender -- is an argument in favor of gender being a feature of the abstract syntactic structure, and moreover that gender may be flexible as to where its lowest locus in the DP may be. In other words, it is distributed across the functional heads in the DP \citep{ÅfarliEtAl2021}. Second, we have considered mixing data displaying a change in the morphophonological realizations as compared to the data presented and described in \citet{Haugen1953}. We take these to be examples of an underlying (and ongoing) change in the agreement patterns of NAmNo DPs, resulting in data where the functional exponents are either missing or replaced by English alternatives. Also these data may be analysed in a late-insertion exoskeletal framework. 

We have also pointed out some data for which the analysis is not conclusive. Some of these are possible objects for future research. However, since new empirical data from this group of speakers is becoming increasingly less available, some of the uncertainties may also remain empirical mysteries.

\section*{Abbreviations}
\begin{multicols}{2}
\begin{tabbing}
MMMM \= Phrase\kill
AP \> Adjective Phrase\\
ArtP \> Article Phrase\\
αP \> α Phrase (AP-related \\ \> functional phrase)\\
CANS \> Corpus of American \\ \> Nordic Speech\\
\textsc{def} \> Definiteness\\
\textsc{df} \> Definite\\
DM \> Distributed Morphology\\
DP \> Determiner Phrase\\
\textsc{f} \> Feminine\\
FP \> Functional Phrase\\
\textsc{gen} \> Gender\\
\textsc{indf} \> Indefinite\\
\textsc{m} \> Masculine\\
\textsc{n} \> Neuter\\
NAmNo \> North American Norwegian\\
\textsc{num} \> Number\\
NumP \> Number Phrase\\
\textsc{pl} \> Plural\\
PP \> Preposition Phrase\\
\textsc{sg} \> Singular\\
\textsc{sup} \> Superlative\\
\textsc{ugen} \> Unvalued gender feature\\
\textsc{w} \> Weak adjectival inflection
\end{tabbing}
\end{multicols}


\section*{Acknowledgements}

First and foremost, we would like to express our gratitude to Professor Tor Anders Åfarli, whose conceptual ideas and theoretical work have inspired and pervaded the content of this chapter. Moreover, we sincerely appreciate the advice and comments provided by the editors of this volume, Professor Kari Kinn and Professor Michael Putnam. It has without doubt contributed to improving the quality of the chapter. Finally, many thanks to an anonymous reviewer for very helpful suggestions.  

\printbibliography[heading=subbibliography,notkeyword=this]
\end{document} 
