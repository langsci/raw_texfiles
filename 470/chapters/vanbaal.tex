\documentclass[output=paper]{langscibook}
\ChapterDOI{10.5281/zenodo.15274564}
\author{Yvonne van Baal\orcid{0000-0002-1766-2911}\affiliation{University of Stavanger}}
\title{Definiteness in determiner phrases in North American Norwegian}
\abstract{This chapter provides an overview of definiteness marking in North American Norwegian (NAmNo), and compares present-day NAmNo to historical NAmNo and homeland Norwegian. Three types of definiteness marking are discussed: (i) indefinite determiners in singular indefinite phrases; (ii) the definite suffix in unmodified definite phrases; and (iii) double definiteness in modified definite phrases. Review of the available data and studies shows that present-day NAmNo is largely characterized by stability, although some aspects of the grammar are found to be more vulnerable to change. The first main conclusion of this chapter is that the syntax of definiteness is largely stable, while the morpho-phonological realization is less robust. This is in line with other findings in heritage languages. The second conclusion is that while indefinite determiners and definite suffixes are stable in present-day NAmNo, there has been an innovation in double definite phrases, where the prenominal determiner is often omitted. The chapter discusses theoretical implications of these findings, and discusses the factors that have made double definiteness especially vulnerable in NAmNo. At the end of the chapter, some venues for future research are identified.}

\IfFileExists{../localcommands.tex}{
   \addbibresource{../localbibliography.bib}
   \usepackage{langsci-optional}
\usepackage{langsci-gb4e}
\usepackage{langsci-lgr}

\usepackage{listings}
\lstset{basicstyle=\ttfamily,tabsize=2,breaklines=true}

%added by author
% \usepackage{tipa}
\usepackage{multirow}
\graphicspath{{figures/}}
\usepackage{langsci-branding}

   
\newcommand{\sent}{\enumsentence}
\newcommand{\sents}{\eenumsentence}
\let\citeasnoun\citet

\renewcommand{\lsCoverTitleFont}[1]{\sffamily\addfontfeatures{Scale=MatchUppercase}\fontsize{44pt}{16mm}\selectfont #1}
  
   %% hyphenation points for line breaks
%% Normally, automatic hyphenation in LaTeX is very good
%% If a word is mis-hyphenated, add it to this file
%%
%% add information to TeX file before \begin{document} with:
%% %% hyphenation points for line breaks
%% Normally, automatic hyphenation in LaTeX is very good
%% If a word is mis-hyphenated, add it to this file
%%
%% add information to TeX file before \begin{document} with:
%% %% hyphenation points for line breaks
%% Normally, automatic hyphenation in LaTeX is very good
%% If a word is mis-hyphenated, add it to this file
%%
%% add information to TeX file before \begin{document} with:
%% \include{localhyphenation}
\hyphenation{
affri-ca-te
affri-ca-tes
an-no-tated
com-ple-ments
com-po-si-tio-na-li-ty
non-com-po-si-tio-na-li-ty
Gon-zá-lez
out-side
Ri-chárd
se-man-tics
STREU-SLE
Tie-de-mann
}
\hyphenation{
affri-ca-te
affri-ca-tes
an-no-tated
com-ple-ments
com-po-si-tio-na-li-ty
non-com-po-si-tio-na-li-ty
Gon-zá-lez
out-side
Ri-chárd
se-man-tics
STREU-SLE
Tie-de-mann
}
\hyphenation{
affri-ca-te
affri-ca-tes
an-no-tated
com-ple-ments
com-po-si-tio-na-li-ty
non-com-po-si-tio-na-li-ty
Gon-zá-lez
out-side
Ri-chárd
se-man-tics
STREU-SLE
Tie-de-mann
}
   \boolfalse{bookcompile}
   \togglepaper[23]%%chapternumber
}{}

\begin{document}
\maketitle

\section{Introduction} \label{Sect-Introduction}
This chapter discusses definiteness marking in determiner phrases (DPs) in North American Norwegian (NAmNo). Definiteness is a semantic and pragmatic feature that is expressed morphosyntactically in Norwegian through the use of (in)definite morphemes. It is quite difficult to define ``definiteness'' precisely in terms of semantics and pragmatics \citep[253]{Lyons1999}. I therefore use the following, somewhat simplified, definitions. Indefinite phrases typically refer to an entity that is either not familiar to the listener or not previously mentioned in the discourse. Definite phrases, on the other hand, refer to entities that the listener is familiar with or can identify, either through context or because they have been mentioned in the discourse. In this chapter, I focus on the morphosyntactic expression of definiteness rather than its semantics or pragmatics. Three types of phrases are discussed: indefinite singular phrases \REF{ex:vanbaal:1a}, unmodified definite phrases \REF{ex:vanbaal:1b}, and modified definite phrases that contain double definiteness \REF{ex:vanbaal:1c}. The chapter is generally restricted to referential nominal phrases containing common nouns.\footnote{Proper nouns are not discussed in this chapter, but they are an interesting topic for future research. Certain dialects in Norwegian use so-called proprial articles, i.e., articles that combine with a proper name \citep[see][]{JohannessenGarbacz2014}. Given the dialect background of most NAmNo speakers, we may expect to find proprial articles, such as \textit{kjente du \textbf{han} M5?} `did you know M5?' (coon\_valley\_WI\_03gm, where M5 represents the anonymized proper name). However, this has not been investigated systematically. See also \citet{KinnLarsson2022} for a study of pronominal demonstratives in homeland and North American Scandinavian and its differences from proprial articles.} 

\ea\label{ex:vanbaal:1}
\ea\label{ex:vanbaal:1a}
\gll vi var på \textbf{en} kirke \\
we were at \textsc{indf.sg} church \\
\glt `we were at a church' \\
(blair\_WI\_04gk, CANS)
\ex\label{ex:vanbaal:1b}
\gll så de gikk på ski gjennom skog-\textbf{en} \\
so they went on ski through forest-\textsc{def.sg.m} \\
\glt `so they skied through the forest' \\
(coon\_valley\_WI\_04gm, CANS)
\ex \label{ex:vanbaal:1c}
\gll i \textbf{det} norsk-e språk-\textbf{et} \\
in \textsc{def.sg.n} Norwegian-\textsc{def} language-\textsc{def.sg.n} \\
\glt `in the Norwegian language' \\
(westby\_WI\_09gm, CANS)
\z
\z

The goal of this chapter is to describe the morpho-syntax of definiteness in NAmNo, and to identify innovations from the baseline, if any. There have been several studies on definiteness marking, and especially double definiteness, in NAmNo and these will serve as the basis for this chapter. In addition, some new data from previous generations of speakers is presented. This chapter is based on studies that use corpus data from the Corpus of American Nordic Speech (CANS, \citealt{Johannessen2015CANS}) as well as studies that use elicited production data. Despite differences in methodology, these types of data are equally useful in studying definiteness: while corpus data come from a more naturalistic context, elicitation data allows for controlled semantic-pragmatic contexts and, in some cases, a larger data set. The type of data available depends on the type of noun phrase studied, but most phenomena in this chapter are illustrated with examples from the corpus and from elicited data. In addition to a speaker code, all NAmNo examples in this chapter are accompanied by the code ``CANS-year'' (for corpus data, where the year refers to the year of recording) or ``elicitation'' (for elicited speech).
 
In describing the definiteness marking in NAmNo, I also point out differences and similarities to homeland Norwegian and historical NAmNo. It is important to point out that this does not mean that homeland Norwegian is considered the target for NAmNo or that I use a deficiency-oriented approach. Instead, the comparison is purely descriptive. Heritage language grammars are consistent and complete grammars \citep[25, 350]{Polinsky2018}, and comparing heritage languages to their homeland varieties allows us to investigate which aspects can change or vary in natural human language.
 
In this chapter, I follow the same core theoretical principles and concepts as discussed in the Introduction chapter by \textcitetv{chapters/introduction}. I adopt an exoskeletal (non\hyp lexical) approach to syntax (\citealt{Borer2005a,Borer2005b,Lohndal2014}), compatible with Distributed Morphology, in which morphological processes apply after syntactic structures have been generated. The framework also maintains a separation between syntactic structures and their morpho\hyp phonological realization. This means, for example, that I distinguish between the syntactic determiner projection (DP) and its possible realizations in Norwegian (\textit{den, det, de}). In other words, I assume that syntax generates structures, which are then “sent off” to be realized phonologically through a process called Spell-Out. In this process, a syntactic structure is mapped onto its morphophonological realization, and this is argued to be processing or performance related (see \cites[394]{Benmamoun2021}[647]{Lohndal2021}{PutnamSanchez2013}). A difference between NAmNo and homeland Norwegian may either be located in Syntax, or at Spell-Out (the mapping). The main criteria I use to distinguish between them is systematicity. I take changes in Syntax to be systematic, whereas differences at Spell-Out are related to performance\hyp factors and display much inter- and intra\hyp speaker variation. When other performance-factors (such as length of the phrase, lexical access) play a role, I also consider this to be located at Spell-Out rather than at Syntax.
 
The chapter is structured as follows. In \sectref{Sect-Baseline}, I describe the baseline that present-day NAmNo is compared to. In this section, definiteness marking in homeland Norwegian and in historical NAmNo is discussed and the three phrase types illustrated in \REF{ex:vanbaal:1} are described in more detail. In \sectref{Sect-PresentdayAmNo}, I describe present-day NAmNo and the syntax of indefinite phrases (\sectref{Sect-IndefArt}), definite phrases (\sectref{Sect-DefiniteSuffix}), and double definiteness (\sectref{Sect-DoubleDef}). A brief summary of the differences and similarities between homeland Norwegian, historical NAmNo and present-day NAmNo is provided in \sectref{Sect-SummaryFindings}. The theoretical implications of the findings are presented in \sectref{Sect-TheoreticalImpl}. Finally, \sectref{Sect-SummaryConclusion} concludes the chapter.
 

\section{The baseline} \label{Sect-Baseline}
In describing a heritage language, we typically compare it to a baseline. There have been several approaches to the baseline, and it has often been argued that monolingual homeland speakers should not be the baseline for heritage speakers \citep[e.g.,][11--16]{Polinsky2018}. Instead, the baseline should consist of the input to heritage speakers, which is typically defined as the language of immigrant speakers. However, for NAmNo, with its long migration history, this approach is difficult: The present-day speakers are several generations removed from the homeland, all elderly, and their input has come from people who were heritage speakers themselves. On the one hand, this makes it all the more important to establish a ``proper'' baseline; yet on the other hand, this can only be done if data from previous generations are available.\footnote{This situation is not unique for NAmNo, as the same difficulties with establishing the baseline apply to other moribund heritage languages (\citealt{Putnametal2018, DAlessandroetal2021}).} For NAmNo, these data are available, as we will see in the next section, which makes it possible to use them in establishment of the baseline.

Still, in the present chapter, I start out by describing definiteness marking in homeland Norwegian, for several reasons. Firstly, homeland Norwegian has been studied in much more detail. In addition, homeland Norwegian has often been used as a baseline in previous studies on NAmNo. Generally, measures have been taken to optimize the baseline, e.g., by including (older) dialect data and not only the present-day standard forms of the homeland variety \citep[for example,][]{JohannessenLarsson2015}. Finally, including homeland Norwegian allows for a three-way comparison, between the homeland, previous generations of NAmNo speakers, and today’s NAmNo speakers. As pointed out in \sectref{Sect-Introduction}, homeland Norwegian serves as a neutral point of comparison and not as a target for NAmNo.

In this section, I first describe definiteness marking in homeland Norwegian, with a focus on the three types of phrases central to this chapter: indefinite, definite, and double definite. Next, I discuss how these types of phrases are expressed in the data available from previous generations of NAmNo speakers. As we will see, historical NAmNo is similar to homeland Norwegian for the relevant aspects.


\subsection{Homeland Norwegian} \label{Sect-Homeland}
The description presented in this section is brief; for more details on Norwegian nominal phrases, see \citet{FaarlundEtAl1997} and \citet{Julien2002Determiners,Julien2005}. Here, I focus on three types of phrases: indefinite, definite, and double definite. After a brief description of definiteness marking in these phrases, I discuss their syntactic structure (expanding on the introduction in \citetv{chapters/introduction}).

Singular indefinite phrases in Norwegian contain an indefinite determiner, while plural indefinites do not (they may be preceded by the plural determiner \textit{noen} `some'). The indefinite determiner agrees with the noun for grammatical gender, illustrated in \REF{ex:vanbaal:2}. Mass nouns like \textit{melk} `melk' and \textit{kjøtt} `meat' appear without the indefinite determiner. In addition, Norwegian has bare singulars: countable, singular, and indefinite phrases that occur without the determiner \citep[see][]{Borthen2003}. One context where bare singulars are found, are so-called classifying predicates which express that the subject belongs to a certain class or set. An example is given in \REF{ex:vanbaal:3a}, while \REF{ex:vanbaal:3b} by contrast illustrates the use of an indefinite determiner in descriptive predicates.
 

\ea \label{ex:vanbaal:2}
\gll en bil, ei bok, et hus \\
    \textsc{indf.sg.m} car \textsc{indf.sg.f} book \textsc{indf.sg.n} house \\
\glt `a car, a house, a book' (M, F, N)
\ex \label{ex:vanbaal:3}
\ea \label{ex:vanbaal:3a}
\gll Ola er lærer \\
    Ola is teacher \\
\glt `Ola is a teacher'
\ex \label{ex:vanbaal:3b}
\gll Ola er en god lærer \\
    Ola is \textsc{indf.sg.m} good teacher \\
\glt `Ola is a good teacher' \\ 
\citep[6]{Kinn2020}
\z
\z

Like the other Scandinavian languages, Norwegian uses a definite suffix in definite phrases. These definite suffixes also inflect for grammatical gender, illustrated in \REF{ex:vanbaal:4}.\footnote{It has been debated whether the definite suffix expresses gender, or rather declension class \citep[see the overview in][]{LohndalWestergaard2021}. For ease of exposition, I gloss the suffixes as inflected for masculine, feminine, or neuter gender.} Norwegian has both pre- and postnominal possessives; definite nouns only occur in the latter. In some dialects of Norwegian, however, certain kinship terms appear in their bare form even with a postnominal possessive (see \citealt{Kinn2021} and references therein). Examples of possessive phrases with and without the definite suffix are given in \REF{ex:vanbaal:5}.
 

\ea \label{ex:vanbaal:4}
\gll bil-en, bok-a, hus-et \\
    car-\textsc{def.sg.m} book-\textsc{def.sg.f} house-\textsc{def.sg.n} \\
\glt `the car, the book, the house' (M, F, N)
\ex \label{ex:vanbaal:5}
\ea  \label{ex:vanbaal:5a}
\gll sykkel-en min \\
    bike-\textsc{def.sg.m} my.\textsc{m} \\
\glt `my bike' (suffix obligatory)
\ex \label{ex:vanbaal:5b}
\gll mor mi \\
    mother my.\textsc{f} \\
\glt `my mother' (kinship term, no suffix) \\
\citep[2--3]{Kinn2021}
\z
\z

In definite phrases modified by an adjective or numeral, Norwegian displays double definiteness, sometimes also called compositional definiteness \citep{Anderssen2012}. In these phrases, the definite suffix on the noun is accompanied by a prenominal (and pre-adjectival) definite determiner, as illustrated in \REF{ex:vanbaal:6}.\footnote{In definite phrases like the one in \REF{ex:vanbaal:6}, adjectives receive the so-called ``weak'' inflection (schwa, e.g., \textit{stor-e}), irrespective of the gender and number of the noun. See \textcitetv{chapters/riksem} for more on adjectival agreement in NAmNo.} The co-occurrence of these two definite morphemes has been central to many descriptions of Norwegian nominal syntax (\citealt{Taraldsen1990, Delsing1993, Santelmann1993, Kester1993,Kester1996, Vangsnes1999, Julien2002Determiners, Julien2005, Anderssen2006,Anderssen2012}), and most current analyses take them to be located at different places in the syntactic tree. Recently, it has been argued that the determiner and the suffixed article each contribute to the definite meaning of modified definite phrases (\citealt{Julien2002Determiners,Julien2005, Anderssen2006,Anderssen2012}).


\ea \label{ex:vanbaal:6}
\gll den stor-e bil-en \\
    \textsc{def.sg} large-\textsc{def} car-\textsc{def.sg.m} \\
\glt `the large car' 
\z  

\begin{sloppypar}
Although double definiteness is generally obligatory, there are several contexts in which either the prenominal determiner or the definite suffix may be omitted. An alternative to double definiteness is adjective incorporation, in which the adjective is incorporated into the definite noun and no prenominal determiner is present. Adjective incorporation is productive in some dialects, most notably those spoken in Trøndelag (Central Norway), and it is also found in NAmNo, as we will see in \sectref{Sect-DoubleDef}. In addition, there is a set of adjectives which can appear without the prenominal determiner in definite phrases, exemplified in \REF{ex:vanbaal:7}, but this is subject to much variation (see \citealt{vanBaal2024}; see \citealt[124--125]{Dahl2015} for a discussion of these exceptions in Swedish).
\end{sloppypar}

\ea  \label{ex:vanbaal:7}
\ea  \label{ex:vanbaal:7a}
\gll (den) best-e plass-en \\
    \textsc{def.sg} best-\textsc{def} place-\textsc{def.sg.m} \\
\glt `the best place' 
\ex \label{ex:vanbaal:7b}
\gll (den) andre sid-a \\
    \textsc{def.sg} other side-\textsc{def.sg.f} \\
\glt `the other side' 
\z 
\z

\begin{sloppypar}
The brief discussion above showed how definiteness in Norwegian is expressed with indefinite determiners, definite suffixes, and/or definite determiners. These different morphemes inflect for grammatical gender, and the definite morphemes also inflect for number. This is illustrated across the examples used above, and visible in the glosses. The focus of this chapter is definiteness marking rather than phrase\hyp internal agreement; the latter is discussed in \textcitetv{chapters/riksem}. In discussing NAmNo, my focus is therefore on the presence versus absence of the different definiteness morphemes, and less on the morphophonological shape they take as a result of inflection.
\end{sloppypar}
 
The Scandinavian languages display quite some variation in their nominal phrases, which has led to much work on the syntax of Scandinavian and Norwegian nominals (see references above). One of the most comprehensive and recent works is \citet{Julien2005}, and the syntactic structure she argues for is presented in \figref{fig:vanbaal:fromex:8} (repeated from \citetv{chapters/introduction} for convenience). Here, we can see the two projections for the definite suffix and the prenominal determiner, ArtP and DP, respectively. ArtP is the location where the definiteness feature (i.e., indefinite or definite) is merged. Between the noun and ArtP, there is NumP, which is the place where number features (i.e., singular or plural) are merged. In addition, there are αP, which has adjectives in its specifier, and CardP, which has cardinal numbers and other weak quantifiers in its specifier. These two positions are argued to be present only when the phrase contains adjectives or numerals.


\begin{figure}
\begin{floatrow}
\ffigbox{%
\begin{forest}
[DP [D] [CardP [Card] [$\alpha$P [AP ] [$\alpha$P [$\alpha$] [ArtP [Art] [NumP [Num] [NP ]]] ]]]]
\end{forest}}
{\caption{Nominal phrase, based on \citet{Julien2002Determiners,Julien2005}}\label{fig:vanbaal:fromex:8}}

\ffigbox{%
\begin{forest}
    [DP [D [\textit{en}]] [ArtP [Art [Num [N [\textit{bil}]] [Num [∅]]] [Art [∅]] ] [NumP [\sout{Num}] [NP [\sout{N}]]]]]
\end{forest}}
{\caption{Syntactic structure of an indefinite singular phrase}\label{fig:vanbaal:fromex:9}}

\end{floatrow}
\end{figure}

A central property of homeland Norwegian is that the DP-layer has to be phonologically realized, which means that either a determiner has to be located in D, or other material has to be moved to the Spec-DP position \citep{Julien2002Determiners,Julien2005}.{\interfootnotelinepenalty=10000\footnote{The phonological realization of the D-projection is not necessary in phrases that are not referential or phrases that are inherently referential (i.e., contain a proper name). See \citet{Julien2002Determiners,Julien2005} for details. Proper names may be preceded by a proprial article in D, cf. footnote 1.}} This phonological realization happens differently in different phrase types. In all phrases, the noun moves up to merge with the Num-head, and then this complex head moves up to merge with the Art-head. The noun now contains features for grammatical gender, number, and definiteness. In indefinite phrases, the Art-head is not realized overtly (i.e., there is no morpheme spelling out an indefinite Art), and the Num-head is only realized if it contains a plural feature. In singular indefinite phrases, the indefinite determiner is located in D (after being moved there from Card), see the structure in \figref{fig:vanbaal:fromex:9}.\largerpage[2]

In definite phrases, the Art-head is realized as the definite suffix. In unmodified phrases, the whole ArtP is moved to Spec-DP (by phrasal movement), but in modified phrases, the presence of the adjective or cardinal number blocks this movement. Instead, ArtP stays \textit{in situ} in these phrases, and D is spelled out as a prenominal determiner to fulfill the requirement that the DP-layer needs to be phonologically realized. Examples of unmodified and modified definite phrases are given in Figures~\ref{fig:vanbaal:fromex:10} and~\ref{fig:vanbaal:fromex:11}, respectively. For ease of exposition, the internal structure of ArtP in \figref{fig:vanbaal:fromex:11} is not provided, but it is identical to the one in \figref{fig:vanbaal:fromex:10}.


\begin{figure}
\begin{floatrow}
\ffigbox[.6\textwidth]
    {\begin{forest}
        [DP [ArtP [Art [Num [N [\textit{bil}]] [Num]] [Art [\textit{-en}]] ] [NumP [\sout{Num}] [NP [\sout{N}]]]] [D´ [D ] [\sout{ArtP}]]]
    \end{forest}}
    {%
        \caption{Syntactic structure of an unmodified definite phrase}
        \label{fig:vanbaal:fromex:10}
    }

\ffigbox[.35\textwidth]
    {\begin{forest}
        [DP [D [\textit{den}]] [$\alpha$P [AP [\textit{store}, roof] ] [$\alpha$P [$\alpha$] [ArtP [\textit{bilen}, roof]]]]]
    \end{forest}}
    {%
        \caption{Syntactic structure of a modified definite phrase, with double definiteness}
        \label{fig:vanbaal:fromex:11}
    }
\end{floatrow}
\end{figure}

\subsection{Historical North American Norwegian} \label{Sect-HistoricalAmNo}
It has been argued that the ideal baseline for heritage language speakers consists of their input \citep[11--16]{Polinsky2018},\footnote{The choice for the optimal baseline also depends on the research question. See also \citet[chapter 6]{Aalberseetal2020} on the use of a combination of multiple baselines.} and this is also the approach taken in this chapter. In line with this view, definiteness marking in previous generations of NAmNo speakers is a better baseline than the homeland variety discussed above, but it is also much less studied. NAmNo has a long migration history, and present-day speakers are third- to fifth-generation immigrants. Because of their old age, it is not possible to study the language of their parents directly. However, there are recordings available of previous generations Norwegian Americans, and they can be used to establish a baseline. In addition, investigating historical NAmNo allows us to study trajectories of language change. If any differences between homeland Norwegian and all NAmNo (historical and present-day) are found, we know that the change happened early in NAmNo history, potentially shortly after arrival in the US.\footnote{In this scenario, another possibility would be that homeland Norwegian has changed. For the cases at hand, however, there are no indications for this.} If, on the other hand, we would find that present-day NAmNo differs from historical NAmNo, we know that this is a change in the current generation of speakers.
 
As described in \textcitetv{chapters/introduction}, there are recordings in the Corpus of American Nordic Speech (CANS) from several points in time: the 1930s and 1940s, the 1980s--1990s, and the period 2010--2016. The latter are the present-day recordings and make up the largest portion of CANS. From the historical material, the recordings by Einar Haugen form the largest part (ca. 78,000 tokens of the 729,000 NAmNo tokens in CANS). These recordings resulted in the seminal two-volume work \textit{The Norwegian Language in America} \citep{Haugen1953}. The historical data in CANS became available relatively recently, and as a result, there are many formal aspects of historical NAmNo that have not been studied yet. Some exceptions are \citet{KinnLarsson2022} on pronominal demonstratives, \citet{vanBaal2022New} on double definiteness, and \citet{Lykke2022} on verbal tense inflection. With respect to definiteness marking, only the use of double definiteness in historical NAmNo has been investigated. The data presented in this section on indefinite phrases and unmodified definite phrases are new and specifically collected for the present volume.
 
Since I focus on the presence versus absence of definiteness morphemes, the investigation of historical NAmNo in CANS involves quite some manual work to collect and analyse data. It is not possible to simply search for a morpheme (e.g., the indefinite determiner), because that would not provide cases where that morpheme is required but absent. In addition, the context is important to determine whether a phrase is (in)definite and whether or not a certain morpheme is expected. I took two measures in order to reduce the data set to a manageable size. First, I restricted the searches to the recordings from 1942 by Haugen (81 speakers and 76,147 tokens). In addition, I did not simply search for all nouns but restricted my searches to three contexts: (i) right after subjunctions,\footnote{In case of the subjunction \textit{som} `who, that', nominal phrases also occur right before the subjunction, viz. the correlates of the relative clause. These are also included in the data, although these are not restricted to subjects or direct objects (cf. footnote 9).} (ii) the phrase before finite verbs, and (iii) the place after finite verbs. These are typical surface positions of subjects and objects, and hence places where we expect to find nominal phrases with overt (in)definite morphemes \citep[e.g.,][]{Longobardi2010}.\footnote{Nominal phrases do not only function as subjects or objects. They can also be found in prepositional phrases, but those contexts are not included in the searches for the present investigation.} After manual checking and exclusion of irrelevant cases, these searches resulted in a total of 1310 nominal phrases. Here, I focus on the singular indefinite phrases and unmodified definite phrases in this set.
 
The data set contains 105 singular indefinite phrases and the vast majority of them contains the indefinite determiner as expected, as in (\ref{ex:vanbaal:12}a,b). However, there are 6 cases (5.71\%) that do not contain a determiner while this would be expected. One example is given in (\ref{ex:vanbaal:12}c). There are many more bare phrases in the examined data (a total of 267), but these are all contexts where a bare noun would be expected in homeland Norwegian, for example because they are mass nouns (\ref{ex:vanbaal:12}d). Given the fact that most bare phrases are homeland\hyp like and that omission of the indefinite determiner is very uncommon, I conclude that historical NAmNo is the same as homeland Norwegian with respect to indefinite singular phrases: the determiner is obligatory, and is located in D.
 

\ea \label{ex:vanbaal:12}
\ea \label{ex:vanbaal:12a}
\gll han hadde en bror som fikk gård-en \\
    he had \textsc{indf.sg.m} brother who got farm-\textsc{def.sg.m} \\
\glt `he had a brother who got the farm' (viroqua\_WI\_04gm, CANS-1942)
\ex \label{ex:vanbaal:12b}
\gll så organiserte de en luthersk menighet \\
    then organized they \textsc{indf.sg.m} Lutheran congregation \\
\glt `then they organized a Lutheran congregation' (gays\_mills\_WI\_01gm, CANS-1942)
\ex \label{ex:vanbaal:12c}
\gll og gammel lampe \# har jeg \\
    and old lamp \# have I \\
\glt `and I have an old lamp' (beaver\_creek\_WI\_01gk, CANS-1942)\\
homeland: \textit{ei gammel lampe}
\ex\label{ex:vanbaal:12d}
\gll jeg vil ha brød \\
    I want have bread \\
\glt `I want bread' (blair\_WI\_28um, CANS-1942)
\z
\z 

In the indefinite phrases, there are three cases where the indefinite determiner is present, but would not have been used in homeland Norwegian, as in (\ref{ex:vanbaal:13}a). These are classifying predicates, which are generally bare in Norwegian \citep[see][]{Kinn2020}. It has to be noted that  there are also examples where the determiner is absent (as would be expected) in historical NAmNo, such as (\ref{ex:vanbaal:13}b).\footnote{\citet{MorlandBA} specifically investigated predicative constructions in historical NAmNo (recordings from 1935, 1936 and 1942). Although the data set is relatively small, she also finds some cases of indefinite determiners where bare nouns would be expected (including (\ref{ex:vanbaal:13}a)). However, use of the homeland-like bare predicative (as in (\ref{ex:vanbaal:13}b)) is more frequent.} \citet{Kinn2020} notes that some present-day speakers of NAmNo sometimes ``overuse'' the determiner in these contexts, but this is not frequent and most speakers use bare phrases in classifying predicates. The fact that there are only three examples in the data set I examined suggests that this is not frequent, and since present-day NAmNo typically still uses bare phrases, I conclude that there is no systematic change in historical NAmNo in this respect.
 

\ea\label{ex:vanbaal:13}
\ea\label{ex:vanbaal:13a}
\gll far min var en snekker \\
    father my.\textsc{m} was \textsc{indef.sg.m} carpenter \\
\glt `my father was a carpenter' (spring\_grove\_MN\_22um, CANS-1942) \\
homeland: \textit{var snekker}
\ex \label{ex:vanbaal:13b}
\gll M2 er prest \\
    M2 is priest \\
\glt `M2 is a priest' (blair\_WI\_19um, CANS-1942)
\z
\z

As for definite phrases, these do not differ from homeland Norwegian described in \sectref{Sect-Homeland} above. In all cases, singular and plural definite phrases contain the definite suffix. Some examples are given in \REF{ex:vanbaal:14}. These are all pragmatically or semantically definite. There are, in other words, no cases of ``overuse" of the definite suffix in non-definite contexts. I found only 1 potential case of a definite phrase that lacks the suffix (out of 346 phrases, 0.29\%). This phrase is given in (\ref{ex:vanbaal:15}a). It contains a post-nominal possessive and should therefore have the definite suffix. When listening to this phrase, however, it is difficult to tell if the suffix /e/ is present, so it is unclear if this actually is a bare phrase. There is, in other words, no evidence that the definite suffix is vulnerable in historical NAmNo. Only kinship terms can occur in their bare form with postnominal possessives, and they frequently do in historical NAmNo, as in (\ref{ex:vanbaal:15}b) (and also in (\ref{ex:vanbaal:13}a) above).
 

\ea \label{ex:vanbaal:14}
\ea\label{ex:vanbaal:14a}
\gll skulle gi hest-ene e mat \\
    should give horse-\textsc{def.pl} e food \\
\glt `(we) should give the horses food' \\
(spring\_grove\_MN\_32gm, CANS-1942)
\ex \label{ex:vanbaal:14b}
\gll når tobakk-en er blitt større \\
    when tobacco-\textsc{def.sg.m} is become bigger \\
\glt `when the tobacco has grown bigger' \\
(ferryville\_WI\_02gm, CANS-1942)
\ex \label{ex:vanbaal:14cs}
\gll bygd-a ho var pen\footnotemark \\
    village-\textsc{def.sg.f} she was nice \\
\glt `the village was nice' \\
(coon\_valley\_WI\_45gk, CANS-1942)
\z 
\ex \label{ex:vanbaal:15}
\ea \label{ex:vanbaal:15a}
\gll så var navn hans M2 \\
    so was name his M2 \\
\glt `so his name was M2' (blair\_WI\_31gm, CANS-1942) \\
homeland: \textit{navn-et hans} 
\ex \label{ex:vanbaal:15b}
\gll far min var født i Norge \\
    father my.\textsc{m} was born in Norway \\
\glt `my father was born in Norway' (chetek\_WI\_01gk, CANS-1942)
\z 
\z 

\footnotetext{A literal translation of the example would be ``the village she was nice". The anaphoric pronoun \textit{ho} `she' agrees with the noun in grammatical gender (feminine) in Norwegian, unlike in English, where the pronoun \textit{she} almost exclusively refers to female (human) referents. See \citet{RodvandMA} and \citet{JohannessenLarsson2018} for studies on the grammatical gender on anaphoric pronouns in present-day NAmNo.}

The analyzed data from 1942 in CANS contains two cases where indefinite and definite morphemes co-occur, as in \REF{ex:vanbaal:16}. In both cases, the semantics seem to indicate an indefinite phrase, but I chose to translate the phrase as `a/the' to indicate the conflicting definiteness marking. This is not possible in homeland Norwegian and therefore surprising. It is however also very infrequent (2 times in 1310 analyzed nominal phrases, or 0.15\%), in fact, almost non-existent. The discussion above also indicates that the NAmNo speakers recorded in 1942 only use indefinite and definite morphemes in pragmatically acceptable contexts. I therefore consider these two instances production errors, rather than that they indicate that historical NAmNo allowed for phrases with both an indefinite determiner and a suffixed article.
 

\ea \label{ex:vanbaal:16}
\gll en jule-dag-en \\
    \textsc{indf.sg.m} Christmas-day-\textsc{def.sg.m} \\
\glt `a/the Christmas day' (spring\_grove\_MN\_19gm, CANS-1942) \\
homeland: \textit{en juledag} or \textit{juledag-en}
\z 

Modified definite phrases have been investigated in historical NAmNo before: \citet{vanBaal2022New} examined the use of double definiteness in version 3.0 of CANS.\footnote{The most recent version of CANS is v3.1, which was used for the investigation of historical NAmNo in the present chapter. The version that \citet{vanBaal2022New} used contains fewer recordings from 1942 and none from 1935 and 1936.} The study included data from 1942 (Haugen's recordings) and 1987--1992 (Hjelde's recordings). The data set contained a total of 139 modified definite phrases and the results indicate that double definiteness was used in an almost-homeland-like manner. A large majority of the modified definite phrases contains double definiteness, as in \REF{ex:vanbaal:17}. However, there is also a small number of phrases without double definiteness, which could be indicative of the start of a language change. In the phrases without double definiteness, there are phrases with only the definite suffix, phrases with only the prenominal determiner, and phrases with neither definite morpheme. These non-homeland-like patterns are roughly equally (in)frequent, and the typical modified definite phrase still contains double definiteness. The data from 1942 and 1987--1992 also contain many modified definite phrases that do not require double definiteness (similar to the examples in \REF{ex:vanbaal:7} above), and \citet{vanBaal2022New} argues that this may make the determiner vulnerable for loss. I come back to this point in \sectref{Sect-TheoreticalImpl}.
 

\ea \label{ex:vanbaal:17}
\ea \label{ex:vanbaal:17a}
\gll den stor-e båt-en skulle ta oss over Atlanterhavet \\
\textsc{def.sg} large-\textsc{def} boat-\textsc{def.sg.m} would take us over Atlantic.Ocean \\
\glt `the large boat was going to take us over the Atlantic Ocean' \\
(blair\_WI\_34gm, CANS-1942)
\ex\label{ex:vanbaal:17b}
\gll på den engelsk-e skol-en \\
on \textsc{def.sg} English-\textsc{def} school-\textsc{def.sg.m} \\
\glt `at the English school' \\
(westby\_WI\_22gm, CANS-1942)
\z
\z

Taken together, the data presented in this section show that historical NAmNo is highly similar to homeland Norwegian as discussed in \sectref{Sect-Homeland} above. Indefinite singular phrases contain an indefinite determiner, and unmodified definite phrases contain the definite suffix. For both types of phrases, there are exceptions where the noun is typically bare, but these exceptions exist in both homeland Norwegian and historical NAmNo. Modified definite phrases in historical NAmNo have co-occurrence of the prenominal determiner and definite suffix, as in homeland Norwegian, but there is a small number of phrases without double definiteness that may be indicative of the start of a change. There are no systematic differences in the use of indefinite determiners, definite suffixes, and double definiteness, and I therefore assume that nominal phrases in homeland Norwegian and historical NAmNo have the same syntactic structure. In the next section, we turn to definiteness in present-day NAmNo and compare this to the baseline we have just established.
 

\section{Definiteness in present-day North American Norwegian} \label{Sect-PresentdayAmNo}
This section presents the use of definiteness marking in present-day NAmNo and the syntax of indefinite phrases, definite phrases, and double definite phrases. The description is based partially on elicited production data collected by myself, and partially on corpus data described by others. The elicitation data has been collected from 20 present-day NAmNo speakers in 2016, and consisted of translation and picture-aided elicitation. Details about the used methods can be found in \citet[chapter 4]{vanBaal2020}.
  

\subsection{The indefinite determiner} \label{Sect-IndefArt}
In the elicited data from present-day NAmNo, singular indefinite phrases contain an indefinite determiner, just like in the baseline described above. Some examples are given in \REF{ex:vanbaal:18}. Indefinite phrases have not been studied systematically with corpus data, but some examples from CANS are provided in \REF{ex:vanbaal:19}. 
 

\ea\label{ex:vanbaal:18}
\ea \label{ex:vanbaal:18a}
\gll en fisk \\
    \textsc{indf.sg.m} fish \\
\glt `a fish' (flom\_MN\_01gm, elicitation)
\ex \label{ex:vanbaal:18b}
\gll et tre \\
    \textsc{indf.sg.n} tree \\
\glt `a tree' (sunburg\_MN\_12gk, elicitation)
\ex \label{ex:vanbaal:18c}
\gll en grønn bil \\
    \textsc{indf.sg.m} green car \\
\glt `a green car' (fargo\_ND\_01gm, elicitation)
\ex \label{ex:vanbaal:18d}
\gll ei gul klokke \\
    \textsc{indf.sg.f} yellow clock \\
\glt `a yellow clock' (westby\_WI\_06gm, elicitation)
\z
\ex \label{ex:vanbaal:19}
\ea \label{ex:vanbaal:19a}
\gll et ord \\
    \textsc{indf.sg.n} word \\
\glt `a word' (billings\_MN\_01gm, CANS-2012)
\ex \label{ex:vanbaal:19b}
\gll ei stor pakke \\
    \textsc{indf.sg.f} large package \\
\glt `a large package' (albert\_lea\_MN\_01gk, CANS-2010)
\ex \label{ex:vanbaal:19c}
\gll en vakker tur \\ 
    \textsc{indf.sg.m} beautiful trip \\
\glt `a beautiful trip' (willmar\_MN\_01gm, CANS-2015)
\z
\z

The examples illustrate two relevant observations from the data on indefinite phrases: the indefinite determiner is present in both unmodified and modified phrases, and it occurs on the left-edge of the phrase. The patterns in the elicited data indicate no change from historical NAmNo. \citet[57]{Johannessen2015Germanic} investigates the spontaneous speech of a single NAmNo speaker, ``Daisy'', and observes that she occasionally omits the indefinite determiner. However, this is not frequent, and it is also argued that Daisy is an atypical NAmNo speaker whose language is affected by attrition. It is therefore not possible to generalize Johannessen's findings on Daisy to all present-day speakers. There are no other descriptions of the presence of the indefinite determiner based on corpus data that indicate a change from the baseline. I therefore conclude that indefinite determiners are used in a stable way in present-day NAmNo, and that there is no need to assume a different syntactic structure. In other words, the indefinite determiner in NAmNo occurs in the D-head and there is no overt realization of the indefinite Art-head, as in \REF{ex:vanbaal:20} (cf. \figref{fig:vanbaal:fromex:9}).
 

\ea \label{ex:vanbaal:20}
\begin{forest}
    [DP [D [\textit{en}]] [ArtP [\textit{fisk}, roof] ]]
\end{forest}
\z

Although the indefinite determiner is typically present in the NAmNo data, there are two contexts where \citet{vanBaal2020} finds phrases that lack the determiner unexpectedly \citep[see also][]{vanBaal2024Definiteness}. First, one of the elicitation tasks sometimes led speakers to produce noun phrases in isolation and in those cases, the determiner was more frequently omitted than in cases where the phrase was part of a sentence. Potentially, there is no full DP present when nouns are produced in isolation. Additionally, phrases without the determiner often occurred if the speakers paused or hesitated while trying to find the noun. In such instances, difficulty with lexical retrieval likely caused the determiner omission, therefore I do not analyze it as a change in syntactic structure.
 
In the baseline, D agrees with the noun for definiteness, gender, and number, and indefinite determiners have separate forms for masculine, feminine, and neuter gender. Although the current chapter is not primarily concerned with agreement within the DP, it is worth noticing that the indefinite determiner sometimes has a non-baseline-like morpho-phonological form. This is described in detail in \citet{JohannessenLarsson2015,LohndalWestergaard2016} and \citet{RodvandMA} and also present in my data. Typically, masculine indefinite determiners occur with feminine or neuter nouns in these cases, as in \REF{ex:vanbaal:21}.
 

\ea \label{ex:vanbaal:21}
\ea \label{ex:vanbaal:21a}
\gll en blå glass \\
    \textsc{indf.sg.m} blue glass \\
\glt `a blue glass' (fargo\_ND\_01gm, elicitation) \\
baseline: \textit{et blå-tt glass}
\ex\label{ex:vanbaal:21b}
\gll en hvit høne \\
    \textsc{indf.sg.m} white chicken \\
\glt `a white chicken' (sunburg\_MN\_11gk, elicitation) \\
baseline: \textit{ei hvit høne}\footnotemark
\z
\z
\footnotetext{In homeland Norwegian, the feminine gender is unstable. In written Bokmål, the (traditionally) masculine determiner can function as a common gender determiner, and be combined with masculine and feminine nouns. The same is true in several spoken dialects at present. However, the Eastern Norwegian dialects spoken by those who immigrated to the US (i.e., the ancestors of the present-day speakers) had a three-gender system, which is therefore part of the baseline \citep[see][]{RodvandMA}.}

As noted at the start of this chapter, I adhere to a framework where syntactic structures are separated from their morpho-phonological realization. The fact that the indefinite determiners are not always spelled out in the same form as in the baseline is therefore no reason to assume a different syntactic structure. In addition, there is a lot of individual variation with respect to the forms of the indefinite determiners, and examples like \REF{ex:vanbaal:21} are more frequent in modified, longer phrases than in simplex phrases with only a noun \citep{vanBaal2024Definiteness}. I therefore treat these instances as the result of a process at Spell-Out (or the interface between Syntax and Spell-Out), not as the result of a change in syntax.
 
In homeland Norwegian, the indefinite determiner does not occur in classifying predicates (see \sectref{Sect-Homeland}). \citet{Kinn2020} has investigated predicate constructions in present-day NAmNo and finds that bare nouns are used in a stable way in these constructions, as in (\ref{ex:vanbaal:22}a). This is despite the fact that English uses an indefinite determiner in this context (as in the translations of the examples). However, \citet{Kinn2020} also finds that a small subgroup of the speakers occasionally uses a determiner in a classifying predicate, as in (\ref{ex:vanbaal:22}b). She argues however that the syntax of classifying predicates is unchanged in present-day NAmNo. Kinn did not investigate the older NAmNo recordings (these were not available in CANS yet). In \sectref{Sect-HistoricalAmNo}, we saw an example of a predicate construction with an indefinite determiner, similar to (\ref{ex:vanbaal:22}b), but I did not search systematically for classifying predicates. The results from \citet{MorlandBA} suggest that such phrases are not frequent in historical NAmNo. It seems most likely that the occasional use of ``English\hyp like'' constructions with a determiner is caused by performance related factors, as argued by \citet{Kinn2020}.
 

\ea \label{ex:vanbaal:22}
\ea \label{ex:vanbaal:22a}
\gll hun var sykepleierske \\
she was nurse \\
\glt `she was a (female) nurse' (flom\_MN\_02gm) \\
\citep[9]{Kinn2020}
\ex  \label{ex:vanbaal:22b}
\gll han var en sjømann \\
he was \textsc{indf.sg.m} sailor \\
\glt `he was a sailor' (stillwater\_MN\_01gm) \\
\citep[9]{Kinn2020}
\z
\z

Summarizing, the available data on present-day NAmNo shows that singular indefinite phrases contain the indefinite determiner, identical to homeland Norwegian and historical NAmNo. Present-day NAmNo also uses bare phrases in classifying predicates in a stable way. Therefore, we can conclude that the three varieties have the same syntax, where the indefinite determiner occurs in D. In some cases, the morpho-phonological realization of the determiner is different, and the determiner is occasionally present in predicates, but this can be considered to be related to syntax-external processes.
 

\subsection{The definite suffix} \label{Sect-DefiniteSuffix}
In this section, we take a look at unmodified definite phrases. In the elicited production data from \citet{vanBaal2020}, the presence of the definite suffix is very stable in these phrases. It is used even more consistently than the indefinite determiner. Some examples of phrases with the definite suffix are given in \REF{ex:vanbaal:23}, and some examples from CANS are provided in \REF{ex:vanbaal:24}. Although there is no systematic study of the definite suffix in CANS, there is no reason to assume it would differ substantially from the elicited production data.
 

\ea \label{ex:vanbaal:23}
\ea \label{ex:vanbaal:23a}
\gll hjul-et \\
    wheel-\textsc{def.sg.n} \\
\glt `the wheel' (fargo\_ND\_08gm, elicitation)
\ex\label{ex:vanbaal:23b}
\gll bok-a \\
    book-\textsc{def.sg.f} \\
\glt `the book' (ulen\_MN\_01gm, elicitation)
\ex \label{ex:vanbaal:23c}
\gll hest-ene \\
    horse-\textsc{def.pl} \\
\glt `the horses' (coon\_valley\_WI\_06gm, elicitation)
\ex \label{ex:vanbaal:23d}
\gll sleigh-en \\
    sleigh-\textsc{def.sg.m} \\
\glt `the sleigh' (fargo\_ND\_08gm, elicitation)
\z
\ex \label{ex:vanbaal:24}
\ea \label{ex:vanbaal:24a}
\gll stol-en \\
    chair-\textsc{def.sg.m} \\
\glt `the chair' (hatton\_ND\_03gm, CANS-2010)
\ex \label{ex:vanbaal:24b}
\gll land-et \\
    land-\textsc{def.sg.n} \\
\glt `the land' (westby\_WI\_02gm, CANS-2010)
\z
\z

The suffix is used in a stable manner and it is productive. This can be seen in the fact that the suffix is only used in definite phrases,\footnote{There are a few scattered occurrences of definite phrases in indefinite contexts in the picture-aided elicitation task, but only when nouns were mentioned in isolation and not when they were part of sentences \citep[see][119]{vanBaal2020}. They are so infrequent that I think it is justified to argue that NAmNo uses the definite suffix only in definite phrases.} and that it combines with words from English origin. One example is given in (\ref{ex:vanbaal:23}d) above, but see \textcitetv{chapters/riksem} and references therein for more on language mixing in NAmNo. There are no differences in the use of the definite suffix in present-day NAmNo and historical NAmNo or homeland Norwegian. Therefore, I assume the same syntax for definite phrases in these varieties: the definite suffix is located in the Art-head, and the whole ArtP moves to Spec-DP, as in \REF{ex:vanbaal:25} (cf. \figref{fig:vanbaal:fromex:10} above).
 

\ea  \label{ex:vanbaal:25}
\begin{forest}
    [DP [ArtP [\textit{hjul-et}, roof ]] [D' [D ] [\sout{ArtP}]]]
\end{forest}
\z

The definite suffix is inflected for gender and number. With respect to gender, the data contain sporadic cases where the definite suffix occurs in an unexpected form. This is however very infrequent, and also less frequent than for indefinite determiners, as \citet{JohannessenLarsson2015} and \citet{LohndalWestergaard2016} also note based on corpus data. For number marking, there are some cases where the noun occurs with an indefinite plural suffix while the context is clearly definite. This leads to a unified plural (as in English and other North Sea Germanic varieties) where there is no distinction between indefinite and definite plural forms. An example is given in (\ref{ex:vanbaal:26}a), where the speaker describes which of the previously mentioned pictures disappears. In other words, the context is definite, while the suffix is indefinite plural. One speaker tends to realize the definite feature with an English determiner, illustrated in (\ref{ex:vanbaal:26}b), but this is highly atypical for the group of present-day NAmNo speakers. There is considerable variation in the use of the definite plural suffix, and the majority of phrases contains the definite plural suffix. In modified phrases (see \sectref{Sect-DoubleDef} below), the unified plural is more frequent, which suggests that performance-related factors play a role. In isolation, the features $[$definite$]$ and $[$plural$]$ are very stable in NAmNo, and I therefore argue that their syntactic projections (Art and Num, respectively) are also stable. However, the feature bundle seems more vulnerable for changes at Spell-Out, and as a result, the bundle is sometimes realized as an indefinite plural suffix. It is worth noting that such phrases are occasionally found in historical NAmNo as well, though in demonstrative or double definite phrases, as in \REF{ex:vanbaal:27}. 
 

\ea \label{ex:vanbaal:26}
\ea\label{ex:vanbaal:26a}
\gll sau-er \\
    sheep-\textsc{indef.pl} \\
\glt intended: `the sheep' (fargo\_ND\_09gm, elicitation) \\
baseline: \textit{sau-ene} 
\ex \label{ex:vanbaal:26b}
\gll the gutt-er \\
    the boy-\textsc{indef.pl} \\
\glt `the boys' (hendricks\_MN\_07gk, elicitation) \\
baseline: \textit{gutt-ene}
\z
\ex \label{ex:vanbaal:27}
\gll disse amerikanske gutt-er \\
    \textsc{dem.pl} American boy-\textsc{indef.pl} \\
\glt `these American boys' (viroqua\_WI\_04gm, CANS-1942) \\
baseline: \textit{disse amerikanske gutt-ene}
\z 

A particular context where definite suffixes are used in Norwegian is with post\hyp nominal possessives, while prenominal possessives occur with bare nouns. However, certain kinship nouns can occur in their bare form even with post\hyp nominal possessives. In present-day homeland Norwegian, this is becoming less frequent, especially in urban areas \citep[26--27]{Kinn2021}. However, in present-day NAmNo, the bare kinship nouns are still frequently used. See \citet{Kinn2021} and \textcitetv{chapters/kinn} for examples and discussion. It is important to note, however, that non-kinship nouns occur with the definite suffix as in the baseline. This adds to the observed stability of the definite suffix in present-day NAmNo \citep{vanBaal2020,vanBaal2024Definiteness}.
 

\subsection{Double definiteness in modified phrases} \label{Sect-DoubleDef}
The previous section discussed unmodified phrases, and we saw that the definite suffix is very stable in NAmNo. In this section, we take a look at modified definite phrases, which require double definiteness (the co-occurrence of the definite suffix and a prenominal determiner). Double definiteness has been studied in corpus data \citep{ALW2018} as well as with elicitation data \citep{vanBaal2020,vanBaal2024Definiteness}. In both types of data, phrases with double definiteness can be found. Some examples are given in \REF{ex:vanbaal:28}. In addition, there are phrases with adjective incorporation without a prenominal determiner. In these phrases, which are also found in certain dialects of homeland Norwegian, the adjective is incorporated into the definite noun. An example is given in \REF{ex:vanbaal:29}.
 

\ea \label{ex:vanbaal:28}
\ea \label{ex:vanbaal:28a}
\gll det stor-e hus-et \\
    \textsc{def.sg.n} big-\textsc{def} house-\textsc{def.sg.n} \\
\glt `the big house' (blair\_WI\_04gk) \\
\citep[755]{ALW2018}
\ex  \label{ex:vanbaal:28b}
\gll den gaml-e maskin-en \\
    \textsc{def.sg} old-\textsc{def} machine-\textsc{def.sg.m} \\
\glt `the old machine' (fargo\_ND\_01gm, CANS-2012)
\ex\label{ex:vanbaal:28c}
\gll den hvite hest-en \\
    \textsc{def.sg} white horse-\textsc{def.sg.m} \\
\glt `the white horse' (westby\_WI\_01gm, elicitation)
\ex \label{ex:vanbaal:28d}
\gll de grønne epl-ene \\
    \textsc{def.pl} green apple-\textsc{def.pl} \\
\glt `the green apples' (iola\_WI\_05gm, elicitation)
\z
\ex \label{ex:vanbaal:29}
\gll grønn-epl-et \\
    green-apple-\textsc{def.sg.n} \\
\glt `the green apple' (flom\_MN\_01gm, elicitation)
\z

Although double definiteness is still used in present-day NAmNo, it is used much less frequently than in historical NAmNo and homeland Norwegian. In the investigated corpus data and the elicited data, many modified definite phrases that require double definiteness do not have it.\footnote{There is a set of adjectives that do not require double definiteness, see \sectref{Sect-Homeland}. The elicited data do not contain phrases like this. In the data from CANS \citep{ALW2018} they are however rather frequent.} There are three types of phrases without double definiteness: phrases without the prenominal determiner \REF{ex:vanbaal:30}, phrases without the definite suffix \REF{ex:vanbaal:31}, and phrases with neither suffix or determiner \REF{ex:vanbaal:32}. The final type is not reported by \citet{ALW2018}, but that could be the result of how the corpus was searched (\citealt{vanBaal2022New} finds bare definite phrases in the historical NAmNo data).
 

\ea \label{ex:vanbaal:30}
\ea  \label{ex:vanbaal:30a}
\gll varm-e kjøkken-et \\
    warm-\textsc{def} kitchen-\textsc{def.sg.n} \\
\glt `the warm kitchen' (iola\_WI\_05gm, elicitation) \\
baseline: \textit{det varme kjøkkenet} 
\ex\label{ex:vanbaal:30b}
\gll gul-e konvolutt-en \\
    yellow-\textsc{def} envelope-\textsc{def.sg.m} \\
\glt `the yellow envelope' (coon\_valley\_WI\_10gm, elicitation) \\
baseline: \textit{den gule konvolutten}
\ex  \label{ex:vanbaal:30c}
\gll stor-e skei-a \\
    big-\textsc{def} spoon-\textsc{def.sg.f} \\
\glt `the big spoon' (sunburg\_MN\_09gm, elicitation) \\
baseline: \textit{den store skeia}
\z
\ex \label{ex:vanbaal:31}
\ea \label{ex:vanbaal:31a}
\gll den stor-e wheel \\
    \textsc{def.sg} large-\textsc{def} wheel \\
\glt `the large wheel' (sunburg\_MN\_11gk, elicitation) \\
baseline: \textit{den store wheel-en} \footnotemark
\ex \label{ex:vanbaal:31b}
\gll den andre bror \\
    \textsc{def.sg} other brother \\
\glt `the other brother' (harmony\_MN\_01gk) \\
baseline: \textit{den andre bror-en} \citep[755]{ALW2018}
\ex \label{ex:vanbaal:31c}
\gll den hvit-e e \# geit \\
    \textsc{def.sg} white-\textsc{def} eh \# goat \\
\glt `the white eh goat' (sunburg\_MN\_12gk, elicitation) \\
baseline: \textit{den hvite geit-a}
\z
\ex \label{ex:vanbaal:32}
\gll grønn-e bil \\
    green-\textsc{def} car \\
\glt intended: `the green car' (sunburg\_MN\_06gm, elicitation) \\
baseline: \textit{den grønne bil-en}
\z\largerpage

{\interfootnotelinepenalty=10000\footnotetext{The homeland Norwegian word for `wheel' is \textit{hjul} and one might argue that the use of the English noun has led to omission of the suffix. However, it is actually very common in NAmNo to combine English nouns with Norwegian grammatical morphemes, including the definite suffix. An example was given in (\ref{ex:vanbaal:23}d) above, and see \citet{Riksem2018Thesis} and \textcitetv{chapters/riksem}.}}

\Citet{vanBaal2020,vanBaal2024Definiteness} argues that the typical modified definite phrase in present-day NAmNo contains only the suffixed article and not the prenominal determiner, as in \REF{ex:vanbaal:30}. There are several arguments for this: first, phrases without the determiner are very frequent, on the group-level as well as the individual level. All speakers in \citet{vanBaal2020} produce these phrases, and most speakers in \citet{ALW2018} do so as well.\footnote{Since modified definite phrases are not frequent in spontaneous speech \citep[121]{Dahl2015}, there are speakers in \citet{ALW2018} who produce very few phrases in  general and some of them produce no phrases without the determiner. An advantage of elicitation production data is, in this case, that we can investigate a higher number of phrases for each individual.} In addition, the omission of the determiner is more systematic than omission of the suffix. The suffix is thus not only stable in unmodified phrases (\sectref{Sect-DefiniteSuffix}), but also in modified phrases. There are speakers who only omit the determiner, but there are no speakers in \citet{vanBaal2020} who only omit the suffix. Most speakers never omit the suffix or do so less than that they omit the determiner. The omission of the suffix is more frequent in plural phrases, where sometimes the indefinite plural is used. We have already seen in \sectref{Sect-DefiniteSuffix} that this can be analyzed as a vulnerability of the feature bundle $[$\textsc{def, pl}$]$ at Spell-Out. The fact that this is more common in modified (hence longer and more complex) phrases indicates that this is performance-related, and that the suffix is otherwise very stable.
 
Finally, omission of the suffix has been found to correlate with the general proficiency of the speaker. \citet{ALW2018} use marking of grammatical gender as a cue for proficiency, while \citet{vanBaal2020} uses speech rate and a vocabulary test to measure proficiency. Both find that there is a correlation between suffix omission and proficiency: speakers with a lower proficiency produce more modified definite phrases without the suffix. This is another indication that suffix omission has not become a standard part of NAmNo, but is instead restricted to performance-related issues.
 
The typical present-day NAmNo modified definite phrase thus only contains the definite suffix, but no prenominal determiner. This is an innovation from the baseline: in homeland Norwegian, and in historical NAmNo, double definiteness is highly stable. As was mentioned in \sectref{Sect-HistoricalAmNo}, the start of a change may be observed in historical NAmNo, but double definiteness is generally used with high frequency in these speakers \citep{vanBaal2022New}. This contrasts with the present-day speakers. It is interesting to note that this innovative pattern is not like English, the dominant language of the speakers. This is also noted by \citet{ALW2018}, who show that there is a correlation between double definiteness and possessives: speakers who tend to omit the determiner use many postnominal possessives, while speakers who more frequently omit the suffix use many prenominal possessives (also in contexts where this is pragmatically strange in homeland Norwegian). In other words, there is a sub-group of speakers who use more English-like structures (suffix omission and prenominal possessives), and they are also less proficient in Norwegian. Most speakers, however, frequently use Norwegian structures that are unlike English (determiner omission and postnominal possessives).

When it comes to the syntax of modified definite phrases in NAmNo, \citet{vanBaal2020} and \citet{vanBaal2024Definiteness} argue that the structure is as in \figref{fig:vanbaal:fromex:33}. Here, the realization of the D-head (\textit{det}) is placed between brackets, to indicate that it is not obligatory. When the D-head is overtly realized, this leads to double definiteness, which is still observed in NAmNo (e.g., the examples in \REF{ex:vanbaal:28}). Typically, however, the D-head is not overtly realized in present-day NAmNo, as in the examples in \REF{ex:vanbaal:30}.
 

\begin{figure}
\caption{Syntactic structure of modified definite phrases in North American Norwegian, with optional realization of D}
\label{fig:vanbaal:fromex:33}
\begin{forest}
    [DP [D [\textit{(det)}]] [$\alpha$P [AP [\textit{varme}, roof] ] [$\alpha$P [$\alpha$] [ArtP [\textit{kjøkken-et},roof]]]]]
\end{forest}
\end{figure}

The syntactic structure in \figref{fig:vanbaal:fromex:33} is an innovation, not only in Norwegian, but among the Scandinavian languages in general. Icelandic and the Northern Swedish vernaculars also have modified definite phrases without the determiners, but their syntax does not allow an empty D-head. Instead, \citet{Julien2002Determiners,Julien2005} argues that the whole $\alpha$P, which includes the adjective and the definite noun, is moved to Spec-DP in these languages. However, there are restrictions to this movement that NAmNo does not comply with (see \citealt{vanBaal2020,vanBaal2024Definiteness} for details). In the available data on NAmNo, there are no clear restrictions on the omission of the determiner. It is not restricted to certain adjectives (as in homeland Norwegian, see \citealt{vanBaal2024}), it occurs when there are cardinal numbers and in phrases with ellipsis (where it cannot occur in Icelandic and Northern Swedish), and there do not seem to be pragmatic factors involved either. \Citet{vanBaal2024Definiteness} discusses several potential syntactic structures to capture determiner-less phrases in NAmNo, and concludes that ``None of the syntactic analyses proposed for other (Scandinavian) languages is easily extended to AmNo, as clear counter evidence for all of them exist. AmNo is better understood as a language where D can be empty in modified definite phrases". In other words, the syntactic requirement of overt realization of the D-head is no longer present in NAmNo.

\subsection{Summary} \label{Sect-SummaryFindings}
In this description of definiteness marking in NAmNo, three types of phrases are discussed: indefinites, definites, and modified definites. In comparing homeland Norwegian, historical NAmNo, and present-day NAmNo, it has become clear that there is a large amount of stability, by which I mean that no change is observed. Indefinite singular phrases contain the indefinite determiner, and definite phrases contain the definite suffix~-- in all three varieties discussed. There are specific contexts where a bare noun would be used in the baseline (classifying predicates and kinship possessives), and some new patterns can be observed in a subset of the speakers. In addition, there are some phrases without the indefinite determiner in present-day NAmNo, and with indefinite determiners that have an unexpected morpho-phonological realization. However, I have argued that these can be accounted for by processes related to (the interface with) Spell-Out. In the literature at present, no cases of syntactic change have been documented when it comes to indefinite singular phrases and definite phrases.
 
Modified definite phrases, on the other hand, are subject to change. While homeland Norwegian and historical NAmNo use double definiteness (a prenominal determiner and a definite suffix) in these phrases, present-day NAmNo typically only uses the suffix. Double definiteness is not used completely consistently in historical NAmNo either, which can be indicative of the start of a language change. In present-day NAmNo, however, the determiner is omitted with high frequency and by all speakers. Given this systematicity within and across speakers, I have argued that this is a case of change in which the overt realization of the prenominal determiner is optional. In the next section, I discuss some theoretical implications of these findings.
 

\section{Theoretical implications} \label{Sect-TheoreticalImpl}
\subsection{Stability and change in moribund languages}
The available data and studies on NAmNo show a large amount of similarity with homeland Norwegian and previous generations of heritage speakers. This is not only the case for definiteness marking discussed in this chapter, but also for other aspects of the language. The language these heritage speakers produce is unmistakably Norwegian, and in many respects identical to homeland Norwegian and the language of their \mbox{(grand-)} parents. Although differences or ``unexpected'' phrases of some sort may attract much attention at first sight, it is actually important to underline that present-day NAmNo is identical to the baseline to a high degree. The present-day speakers are the final generation of NAmNo speakers, and the language is therefore classified as moribund (like many other Germanic heritage languages in the US, cf. \citealt{PutnamEtAl2018}). The term “moribund” refers to the status of use in a group of speakers, and reflects that the language is not passed on to next generations anymore. Importantly, the term does not in any way expresses a judgement on the ``quality'' of the language as spoken by these speakers. We classify the language as moribund based on societal factors, not based on structural linguistic factors. As we have seen in the sections above, there are in fact aspects where NAmNo is syntactically non-distinguishable from homeland Norwegian or historical NAmNo, i.e., from communities where the language is or was actively used and passed on to the next generations.
 
In general, heritage languages can provide important insights to formal linguistics, as they can inform which aspects of language can be acquired and maintained in the specific circumstances of heritage speakers, who typically receive reduced (and often different) input compared to monolinguals, and additionally often use the heritage language less over their lifespan (cf. \citealt{Benmamoun2021, Lohndal2021}). We have seen in this chapter that indefinite singular phrases and unmodified definite phrases are stable across time in NAmNo. In other words, the use of the indefinite determiner and the definite suffix are unchanged, and we can conclude that these syntactic elements can be acquired in a minority language context, and are unaffected by the reduced use of Norwegian across the speakers' lives. The case study in \citet{Johannessen2015Germanic} demonstrates how a severe lack of use of Norwegian may impact the use of the indefinite determiner, which may suggest that the indefinite determiner is somewhat more vulnerable than the definite suffix. In monolingual acquisition, the definite suffix is acquired earlier than the indefinite determiner \citep[229--230]{KupischEtAl2009}, but the data from NAmNo indicate that both can be acquired by heritage speakers.
 
Contrary to the stability of indefinite and unmodified definite phrases, the available data from NAmNo also showed that double definiteness in modified definite phrases is more vulnerable. Typically, the present-day speakers omit the prenominal determiner and only use the definite suffix in these phrases. From this, we can conclude that double definiteness is a phenomenon that is difficult to acquire and/or difficult to maintain in a heritage language context. I come back to the point of acquisition versus attrition in \sectref{Factors} below, but first it is important to note that the change in NAmNo is systematic. There is no complete breakdown of definiteness marking in NAmNo; rather, there is a specific aspect of definiteness marking that has changed. All speakers frequently produce modified definite phrases with only the definite suffix. This is a pattern which exists in the baseline, with a subgroup of adjectives, but that is now extended to all adjectives in present-day NAmNo. This systematicity in change is also seen in other aspects of change. In postnominal possessives, NAmNo has a larger set of kinship terms that occur in their bare form (see \citealt{Kinn2021, chapters/kinn}), but this is not extended to all nouns. On a more general level, this shows that innovation and change is possible in moribund languages as in any natural language, and furthermore that changes in moribund heritage languages are systematic and not a matter of “anything goes”. Again, this is true for what we know from change in non-heritage languages, too \citep[cf.][]{KupischPolinsky2022}.
 

\subsection{Syntax and Spell-Out} 
\largerpage

As pointed out several times in this chapter, the syntax of definiteness marking in NAmNo is highly stable. Yet, this is not always the case of the morpho-phonological realization of definiteness markers. The indefinite determiner sometimes has an unexpected form in terms of gender (e.g., \textit{en glass} rather than \textit{et glass} `a glass' in (\ref{ex:vanbaal:21}a) above). This is also occasionally found for the definite suffix. In addition, the definite plural suffix sometimes appears as an indefinite plural suffix (e.g., \textit{sau-er} instead of \textit{sau-ene} `the sheep' in (\ref{ex:vanbaal:26}a)). Because such forms are not systematic across participants, and are more frequent in longer (i.e., more complex) phrases, I have argued that they are processing-related and not the result of a change in Syntax.
 
In other words, I argue that there is syntactic agreement between the noun and the definite morphemes in terms of gender and number features. This is the case even in instances where we do not see this agreement in the surface forms. In these instances, something happens in the (interface with) Spell-Out, such that the morpho-phonological form that realizes the syntactic feature bundle is different from the one in the baseline. This indicates that there is more vulnerability at Spell-Out than at Syntax with respect to definiteness marking.
 
Vulnerability of inflectional morphology has been observed in many other heritage languages. In fact, this is one of the most well-documented type of differences between baseline or homeland languages and heritage languages. Detailed discussions of this can be found in \citet[54--71]{Montrul2016}, \citet[chapter 5]{Polinsky2018}, and \citet{PutnamSchwarzHoffman2021}. In the nominal domain, vulnerability of inflectional morphology involving grammatical gender or number is often observed (e.g., \citealt{AlbiriniEtAl2011, BenmamounEtAl2014} on heritage Arabic, \citealt{Bolonyai2007} on heritage Hungarian, and \citealt{Haakansson1995} on heritage Swedish). As \citet[614]{PutnamSchwarzHoffman2021} note, the vulnerability of morphology contrasts strongly with the observed stability of syntax and phonology. \citet{Benmamoun2021} discusses several examples which indicate that core Syntax is generally stable in heritage languages, while the morpho-phonological realization of elements is not always like the baseline. He concludes that ``The difference [between heritage speakers and monolingual speakers, YvB] has more to do with the interface between syntax and PF, where, as expected, heritage grammar shows vulnerabilities" \citep[393]{Benmamoun2021}. The data presented in this chapter on definiteness marking in NAmNo converge on this observation. In this way, the NAmNo data are yet another example of the stability of Syntax versus the vulnerability of the (interface with) Spell-Out or morphological realization. This, in turn, is an argument for formal models that separate syntactic structures from morphological exponents (\citealt{Lohndal2021, PutnamSchwarzHoffman2021}), such as the theoretical model that this volume and chapter adhere to.

\subsection{Phrases without an obligatory determiner} \label{PhrasesNoDet}
In this chapter, I have argued that the only syntactic change in present-day NAmNo is that the prenominal determiner is no longer obligatory in modified definite phrases. At face value, this analysis seem to go against two claims that have been made in the literature: (i) that DPs are generally stable in heritage languages, and (ii) that heritage speakers avoid null\hyp elements and prefer overt realizations instead. Here, I briefly discuss these claims in light of NAmNo.

\citet{Polinsky2018} argues that high syntactic projections are resilient to change, and that heritage speakers tend to maintain ``forms and structures that have strong perceptual salience or rely on salient conceptual categories" (ibid.: 64). She argues that this can be shown across heritage languages for determiners and the DP in nominal domain, and for tense at the clausal level. This claim seems to be at odds with the data from NAmNo. \citet{LohndalVanBaal2025} discuss this puzzle, and they argue that the general claim by \citet{Polinsky2018} can be nuanced in order to account for the NAmNo data. They bring forward two important points. Firstly, a linguistic element can be salient in different ways: it can be phonologically salient, and/or syntactically salient. The prenominal determiner in Norwegian modified definite phrases lacks prosodic salience (i.e., it is unstressed), but has syntactic salience as it appears at the top of the syntactic phrase. The fact that the determiner is vulnerable for omission is then not surprising: apparently, the lack of prosodic salience “overrides” the syntactic salience for NAmNo, and non-salient elements are more susceptible to change or restructuring, in line with what Polinsky argues for. As a second point, \citet{LohndalVanBaal2025} argue that stability within the nominal domain need not be restricted to the D-level. Instead, a lower functional projection within the phrase may be stable, if the language has such a position. NAmNo in fact has the lower position ArtP (see \sectref{Sect-Homeland} above), which is realized by the definite suffix in all definite phrases. As \citet{LohndalVanBaal2025} point out, and as we have seen in \sectref{Sect-DefiniteSuffix}, this position is highly stable in NAmNo. In other words, NAmNo has a stable element within the DP, even though it is not the D-position itself as in other heritage languages with a less detailed carthography of the nominal phrase.
 
The NAmNo data presented here may also seem at odds with claims in the literature that heritage speakers generally avoid null\hyp elements. They also differ from monolingual speakers in ``their ability to identify and evaluate missing elements with discourse antecedents" \citep[136]{LalekoPolinsky2017}. This has been termed the ``Silent Problem'' by \citet{LalekoPolinsky2017}. However, I would like to argue that this Silent Problem is something else than the use of phrases without a prenominal determiner in NAmNo. The cases discussed in \citet{LalekoPolinsky2017} and \citet{Polinsky2018} as examples of the Silent Problem are cases of discourse-licensed silent elements. For example, heritage speakers of pro-drop languages have been found to overuse overt pronouns in pragmatic\hyp semantic contexts where monolingual speakers would use a null pronoun. However, in these instances, the null\hyp element has to be “recovered” and interpreted as a referent in the discourse. This is not the case for modified definite phrases in NAmNo, where not the whole phrase is null, but rather one element of the phrase (the determiner) is omitted. In these phrases, the null\hyp element is not necessary to establish or interpret the referent of the phrase, and the null\hyp element is not licensed by discourse\hyp pragmatic principles either. In my understanding, the findings on NAmNo described in this chapter are thus not at odds with the Silent Problem, as they are simply a different type of phenomenon.
 
In addition, other studies have found the introduction of null\hyp elements in other heritage languages. For Wisconsin heritage German, \citet{BousquetteEtAl2016} describe the use of parasitic gaps that are not acceptable in the baseline, and \citet{Bousquette2018} finds the use of preposition stranding, which again is not possible in the baseline. Innovative introduction of preposition stranding is also found in heritage Spanish in the US \citep{PascualyCabo2015}. These instances are not equivalent to the determiner omission in NAmNo that I discussed, but they make clear that the introduction of null\hyp elements (or traces) in heritage languages is possible, and that the observed Silent Problem may not apply across the board for all silent or null elements. The data on NAmNo discussed here are thus not automatically at odds with the Silent Problem defined by \citet{LalekoPolinsky2017}.
 

\subsection{Acquisition, attrition, and language shift} \label{Factors}
A central finding in the work on definiteness marking in NAmNo is that most aspects are unchanged compared to the baseline, while there is change in double definiteness in modified definite phrases. As we have seen, the prenominal determiner has become optional in these phrases in present-day NAmNo. This raises the question \textit{why} double definiteness is changed, and also raises questions to the \textit{causes} of this change. In work on differences between heritage languages and the baseline varieties, causes that are often discussed are acquisition-related (a different acquisitional outcome), attrition (loss of a language phenomenon over the course of the lifespan), and transfer from the dominant language. A full overview of these processes and how they have been found to shape heritage languages is beyond the scope of this chapter (but see, e.g., \citealt{BenmamoumEtAl2013, Polinsky2018, ScontrasEtAl2015}), but I want to point out some relevant observations that can be made about NAmNo.
 
First, it is important to note that the change in modified definite phrases has not resulted in more similarity with English. Rather the opposite: the element that is different between the two languages~-- the definite suffix~-- is stable in NAmNo, while the English-like element~-- the prenominal determiner~-- is vulnerable to omission. This is also observed by \citet{ALW2018}, who furthermore observe that the speakers who frequently omit the determiner in double definites also very often use (and sometimes overuse) postnominal possessives. In other words: these NAmNo speakers rely more on non-English-like structures than on English-like structures. \citet{ALW2018} describe this as cross-linguistic overcorrection, a process to maximize the differences between two languages by bilingual speakers. However, it has to be noted that the definite suffix and the postnominal possessive are also very frequent in the baseline, and this may also play a role in addition to overcorrection. But importantly, present-day NAmNo is not shaped by transfer from English when it comes to definiteness marking.

\Citet[chapter 6]{vanBaal2020} delves deeper into the role that acquisition or attrition may play in the innovation of omitting the prenominal determiner, and concludes that this is caused during the acquisition of Norwegian as a heritage language and not by attrition later in life. There are several arguments for this. First, as we also have seen above, the omission of the determiner is highly systematic across and within speakers. It has been argued that attrition is a less systematic process that leads to high levels of inter- and intra\hyp speaker variation (see, e.g., \citealt{LohndalWestergaard2016, Kinn2020, PerezCortesEtAl2019, PutnamSanchez2013}), and that is more related to processing.
 
Most importantly, however, \citet{vanBaal2020} describes how the patterns in NAmNo reflect those in monolingual acquisition of Norwegian (and Swedish). Monolingual children acquire the definite suffix very early on (around age 2;3, \citealt{Anderssen2006,Anderssen2012}), while they acquire double definiteness much later. In Anderssen's longitudinal data (ending when the children are 3;3), double definiteness is not completely acquired. Based on data from \citet{BusterudEtAl2019}, \citet[163--164]{vanBaal2020} established an age of acquisition of 6--7 years, i.e., quite late. In addition, the prenominal determiner is often omitted during the acquisitional period of double definiteness \citep{Anderssen2006, Anderssen2012}. In other words, the typical NAmNo modified definite phrase that only contains the definite suffix is \textit{also} the typical phrase during monolingual acquisition.
 
The late acquisition of double definiteness, and the determiner in particular, may make it vulnerable to change in a heritage language context. It is clearly a phenomenon that needs much input over a long time before it can be acquired. \Citet{vanBaal2020} argues that it is likely that the present-day NAmNo speakers did not receive enough input to acquire double definiteness completely, and that it is more plausible that the omission of the determiner is the result of this acquisition than of attrition later in life. However, as she also points out, there is no a priori reason that elements vulnerable in acquisition could not become even weaker as the result of lack of use.

There are several factors that make the determiner more vulnerable in acquisition compared to the definite suffix. The definite suffix is much more frequent in the input than the prenominal determiner (\citealt[751]{ALW2018}, see also \citealt[168--169]{vanBaal2020}), and contexts for double definiteness are infrequent in general \citep[121]{Dahl2015}. There are also many exceptions to double definiteness: a subgroup of adjectives does not require the prenominal determiner. These exceptions are very frequent, and they are especially frequent in spoken Norwegian compared to written (Bokmål) Norwegian \citep{vanBaal2020,vanBaal2024}. Present-day NAmNo speakers have typically not had any written Norwegian input, and no formal schooling in Norwegian either. In their solely spoken input, the determiner is thus even less frequent than in the (spoken and written) input to monolingual children. In addition to its low frequency, the determiner is also prosodically weak and not salient (see \sectref{PhrasesNoDet} above), and it falls outside the typical trochaic metrical template of Norwegian. This has been argued to play an important role in the late acquisition of the determiner (\citealt{Anderssen2006, Bohnacker2003, KupischEtAl2009}). Although some acquisitional works focus on why one of the factors is more important in acquisition, I follow \citet[236]{Bohnacker2003} and assume that both frequency and prosody play a role in the late acquisition of double definiteness. In other words, I assume that the combination of several factors makes the determiner more vulnerable in acquisition, and hence (even) more vulnerable in the heritage language context as well.
 
Previous work has found that double definiteness (and the determiner) were used in a stable way in historical NAmNo, and that the change of determiner omission is an innovation in the present-day speakers (\citealt{vanBaal2022New}, and see \sectref{Sect-DoubleDef} above). If double definiteness is vulnerable to change in the heritage language context, why did it then not change earlier in the NAmNo history? \Citet{vanBaal2022New} argues that the status of the language in the communities is important: the historical recordings are made in pre\hyp language\hyp shift communities, where Norwegian was used actively and on daily basis by the whole community. Present-day speakers, on the other hand, live and partly grew up in post-shift communities where the status of Norwegian was much weaker. This has likely had consequences for amount and type of input, as well as possibilities for use of Norwegian, which has led to a language change \citep[cf.][]{EideHjelde2023}. In this way, the innovation of determiner omission is an example of language change that happens faster (on “fast forward”) in heritage languages \citep{KupischPolinsky2022}.
 

\section{Concluding remarks} \label{Sect-SummaryConclusion}
In this chapter, marking of definiteness in NAmNo DPs has been described. Based on previous studies and some new data, three types of phrases were discussed: (i) indefinite singular phrases, (ii) definite phrases, and (iii) double definite phrases. A comparison between homeland Norwegian, historical NAmNo, and present-day NAmNo shows that the three varieties have the same syntax for (i) and (ii). However, when it comes to (iii), there has been an innovation in present-day NAmNo: Rather than double definiteness, the current speakers of NAmNo tend to omit the prenominal determiner from modified definite phrases. I have argued that this is a change in Syntax, such that the D-head is no longer obligatorily realized with a determiner.
 
The findings on definiteness marking have several theoretical implications, discussed in \sectref{Sect-TheoreticalImpl}. For example, the data presented here add to the existing body of research across heritage languages that report the stability of Syntax versus the vulnerability of morpho-phonological Spell-Out. In \sectref{Factors}, I discussed several factors that likely have contributed to the fact that double definiteness has changed in NAmNo. These factors include late acquisition, low frequency, a high amount of exceptions, and lack of prosodic salience. Based on this, we can make predictions about other aspects of a language that may be vulnerable to change in a heritage language context. On the other hand, the stability of the indefinite determiner and definite suffix across time and heritage language context also lead to predictions about elements that are likely to be stable in heritage languages more general.
 
I pointed out in \sectref{PhrasesNoDet} that the determiner omission in NAmNo is not governed by discourse pragmatics. However, this may be a simplified statement: at present, very little work has been done on heritage language pragmatics in general and pragmatics in NAmNo in particular. In homeland Norwegian, there are several exceptions to double definiteness. With a selected group of adjectives, the prenominal determiner may be omitted, but it is yet unclear which factors govern \textit{whether} the determiner is omitted with these adjectives. It is likely that pragmatic-semantic factors play a role here \citep[see][]{vanBaal2024}, but the exact role of pragmatics in double definiteness is an open question. Future research could establish in which pragmatic contexts the determiner can be omitted, and compare homeland and American Norwegian in this respect.
 
Another aspect of definiteness marking in NAmNo that warrants further investigation is the use of demonstratives. The elicitation data includes occasional demonstrative phrases, while a deictic demonstrative reading is not necessary \citep{vanBaal2020,vanBaal2024Definiteness}. See \REF{ex:vanbaal:34} for an example. These are not frequent, and they include the definite suffix as in homeland Norwegian demonstratives, so they are syntactically baseline-like. In terms of semantics and pragmatics, however, the status of these phrases is less clear. As \citet{vanBaal2024Definiteness} points out, these phrases may have a demonstrative intention. On the other hand, they could be a case of “overusing” the demonstrative. In similar elicitation tasks with homeland speakers \citep[e.g.,][]{BusterudEtAl2019}, no such demonstratives are found, which may suggest an innovation in NAmNo. However, it is unclear how the NAmNo speakers (historical and present-day) use demonstratives in spontaneous speech, and there are relatively few demonstratives in the elicitation data. \citet[7]{KupischPolinsky2022} suggest that the use of the demonstrative ``could indicate renewal in the grammaticalization cycle", in which determiners typically develop from demonstratives. Future research could shed light on the use of demonstratives in NAmNo generally, the pragmatic contexts in which they are used, and test the suggestion by Kupisch \& Polinsky.
 

\ea \label{ex:vanbaal:34}
\gll denne hvit-e hest-en \\
    \textsc{dem.sg} white-\textsc{def} horse-\textsc{def.sg.m} \\
\glt `this/the white horse' (fargo\_ND\_01gm, elicitation) \\
\z 

The finding that double definiteness has changed in present-day NAmNo also raises questions about the vulnerability of double definiteness, and the determiner in particular, in other Scandinavian heritage languages. Very recently, research on Latin American Norwegian has started, and data on (double) definiteness are being collected. It will be very interesting to see whether these speakers, who are dominant in Spanish rather than English, use double definiteness similarly to homeland Norwegian or similar to NAmNo (or different from both). Both English and Spanish do not have double definiteness, but the NAmNo and LatAmNo populations differ from each other in time of migration, size of the heritage community, contact with the homeland, and contact with the written language. Comparisons of the two varieties may therefore shed light on the role that these factors play in shaping double definiteness in the heritage language.

Comparisons with other Scandinavian heritage languages are another interesting avenue for future research. Swedish also has double definiteness, and furthermore has a number of exceptions where the prenominal determiner may be omitted. The status of double definiteness in American Swedish is not studied much yet, but preliminary findings suggest that the determiner is omitted in similar contexts as in homeland Swedish, but not more \citep{vanBaalLarsson2022}. Although Danish does not have double definiteness, it has different definiteness marking in unmodified phrases (only the definite suffix) and modified phrases (only the prenominal determiner). This means that comparisons with heritage Danish, both in North and Latin America, may provide additional insights on the acquisition and maintenance of definiteness morphemes in heritage language contexts.
 
There are several non-Scandinavian languages that have nominal phrases with more than one (in)definite determiner (or determiner-like element). \citet{Alexiadou2014} describes multiple determiner constructions in Greek, Romanian, Scandinavian, French, and Hebrew, among others. She argues that there is no unified account for the different types of multiple determiner constructions, and that ``multiple marking does not realize the same property/structure across languages" \citep[7]{Alexiadou2014}. Phrases with multiple determiners in other languages are thus not structurally equivalent to double definiteness in Norwegian and Swedish. Still, I believe that investigating such constructions in heritage languages cross-linguistically can provide valuable insights on whether phrases with multiple determiners are always vulnerable to change in heritage language contexts, or whether only certain types of determiner doubling are vulnerable. Given that the underlying syntax is different in the various languages, cross-linguistic heritage language studies may shed light on which of these syntactic structures is more likely to be acquired and/or maintained in heritage speakers' grammar.
 
Studies comparing several heritage languages are still relatively rare, but I hope they will become more common in the future. However, studies on single heritage languages are a necessary step towards comparative work. As I have shown in the present chapter, the definiteness marking in NAmNo has been studied with various types of data and in both historical and present-day NAmNo. This has lead to clear findings, most notably (i) a contrast in vulnerability between Syntax and morpho-phonological realization or Spell-Out; and (ii) a contrast between vulnerable double definiteness on the one hand and stable indefinite determiners and definite suffixes on the other. With this established, I pointed out two possible directions of future research that I believe to be of interest: the relation between double definiteness and pragmatics, and cross-linguistic comparisons with other heritage languages that have multiple determiner constructions. Insights from such future work will be valuable for our general understanding of heritage languages, and ultimately, the human language capacity.
 

\section*{Abbreviations}
\begin{tabbing}
MMMM \= Corpus\kill
CANS \> Corpus of American Nordic Speech \\
DEF \> definite \\
DEM \> demonstrative \\
F \> feminine gender \\
INDF \> indefinite \\
M \> masculine gender \\
N \> neuter gender \\
NAmNo \> North American Norwegian \\
PL \> plural \\
SG \> singular \\
\end{tabbing}

\section*{Acknowledgements}
I would like to thank the editors for inviting me to contribute to this volume, and an anonymous reviewer and Kari Kinn for their feedback on this chapter. Many thanks to all North American Norwegian speakers who have participated in the recording of their language, and to all fieldworkers who collected these recordings. Finally, I would like to express my gratitude to Janne Bondi Johannessen (1960--2020) for her enormous contribution to the field and to my own research. 

\printbibliography[heading=subbibliography,notkeyword=this]

%% everything below is from the template from LSP and kept in case necessary later.
%\section{Where we came from}
%Phasellus maximus erat ligula, accumsan rutrum augue facilisis in. Proin sit amet pharetra nunc, sed maximus erat. Duis egestas mi eget purus venenatis vulputate vel quis nunc. Nullam volutpat facilisis tortor, vitae semper ligula dapibus sit amet. Suspendisse fringilla, quam sed laoreet maximus, ex ex placerat ipsum, porta ultrices mi risus et lectus. Maecenas vitae mauris condimentum justo fringilla sollicitudin. Fusce nec interdum ante. Curabitur tempus dui et orci convallis molestie \citep{Chomsky1957}.

%\begin{table}
%\caption{Frequencies of word classes}
%\label{tab:myname:frequencies}
 %\begin{tabularx}{.8\textwidth}{X rrrr}
 % \lsptoprule
  %          & nouns & verbs  & adjectives & adverbs\\
  %\midrule
  %absolute  &   12  &    34  &    23      & 13\\
  %relative  &   3.1 &   8.9  &    5.7     & 3.2\\
  %\lspbottomrule
 %\end{tabularx}
%\end{table}

%Sed nisi urna, dignissim sit amet posuere ut, luctus ac lectus. Fusce vel ornare nibh. Nullam non sapien in tortor hendrerit suscipit. Etiam sollicitudin nibh ligula. Praesent dictum gravida est eget maximus. Integer in felis id diam sodales accumsan at at turpis. Maecenas dignissim purus non libero scelerisque porttitor. Integer porttitor mauris ac nisi iaculis molestie. Sed nec imperdiet orci. Suspendisse sed fringilla elit, non varius elit. Sed varius nisi magna, at efficitur orci consectetur a. Cras consequat mi dui, et cursus lacus vehicula vitae. Pellentesque sit amet justo sed lectus luctus vehicula. Suspendisse placerat augue eget felis sagittis placerat.

%\is{Cognition} %add "Cogntion" to subject index for this page

%\ea
%\gll cogito                           ergo      sum\\
%     think.\textsc{1sg}.\textsc{pres} therefore \textsc{cop}.\textsc{1sg}.\textsc{pres}\\
%\glt `I think therefore I am.'
%\z
%\il{Latin} %add "Latin" to language index for this page

%Sed cursus eros condimentum mi consectetur, ac consectetur sapien pulvinar. Sed consequat, magna eu scelerisque laoreet, ante erat tristique justo, nec cursus eros diam eu nisl. Vestibulum non arcu tellus. Nunc dignissim tristique massa ut gravida. Nullam auctor orci gravida tellus egestas, vitae pharetra nisl porttitor. Pellentesque turpis nulla, venenatis id porttitor non, volutpat ut leo. Etiam hendrerit scelerisque luctus. Nam sed egestas est. Suspendisse potenti. Nunc vestibulum nec odio non laoreet. Proin lacinia nulla lectus, eu vehicula erat vehicula sed.





%\section*{Contributions}
%John Doe contributed to conceptualization, methodology, and validation. 
%Jane Doe contributed to writing of the original draft, review, and editing.

%\sloppy
%\printbibliography[heading=subbibliography,notkeyword=this]
\end{document}
