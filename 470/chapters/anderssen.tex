\documentclass[output=paper]{langscibook}
\ChapterDOI{10.5281/zenodo.15274572}
\author{Merete Anderssen\affiliation{UiT The Arctic University of Norway} and 
		Helene R. Jensberg\affiliation{UiT The Arctic University of Norway} and 
		Terje Lohndal\affiliation{UiT The Arctic University of Norway; NTNU – Norwegian University of Science and Technology} and 
		Björn Lundquist\affiliation{UiT The Arctic University of Norway} and 
		Marit Westergaard\affiliation{UiT The Arctic University of Norway; NTNU – Norwegian University of Science and Technology}
	}
\title[Verb second word order and finite verb placement in NAmNo]
	  {Verb second word order and finite verb placement in North American Norwegian}
\abstract{This chapter discusses verb second word order and finite verb placement in North American Norwegian. As in other studies of this phenomenon in heritage languages, verb second word order (V2) in main clauses is robust in this population. Embedded structures, however, are found to be considerably more vulnerable, with a strong tendency for the finite verb to move above negation and adverbials. The chapter discusses possible explanations for these behaviours, suggesting that the high activation of English causes occasional crosslinguistic influence in the production of V2 in Norwegian, while several factors conspire to make the heritage speakers overuse verb movement in embedded structures.}
\IfFileExists{../localcommands.tex}{
  \addbibresource{../localbibliography.bib}
  % add all extra packages you need to load to this file

\usepackage{tabularx,multicol}
\usepackage{url}
\urlstyle{same}

\usepackage{listings}
\lstset{basicstyle=\ttfamily,tabsize=2,breaklines=true}

\usepackage{langsci-basic}
\usepackage{langsci-optional}
\usepackage{langsci-lgr}
\usepackage{langsci-osl}
% \usepackage{./langsci/styles/langsci-lgr}
% \usepackage{./langsci/styles/langsci-osl}
% \usepackage{langsci-gb4e}

\usepackage{tikz}
\usetikzlibrary{patterns,calc}
\pgfdeclarepatternformonly{south east lines}{\pgfqpoint{-0pt}{-0pt}}{\pgfqpoint{3pt}{3pt}}{\pgfqpoint{3pt}{3pt}}{
    \pgfsetlinewidth{0.6pt}
    \pgfpathmoveto{\pgfqpoint{0pt}{3pt}}
    \pgfpathlineto{\pgfqpoint{3pt}{0pt}}
    \pgfpathmoveto{\pgfqpoint{.2pt}{-.2pt}}
    \pgfpathlineto{\pgfqpoint{-.2pt}{.2pt}}
    \pgfpathmoveto{\pgfqpoint{3.2pt}{2.8pt}}
    \pgfpathlineto{\pgfqpoint{2.8pt}{3.2pt}}
    \pgfusepath{stroke}}
    
\usepackage{stmaryrd}
\usepackage{wasysym}
\usepackage{multirow}
\usepackage{caption}
\usepackage{subcaption}
\usepackage{mathrsfs}
\usepackage{qtree}

\usepackage{linguex}


  %pminos do not split footnotes
% \interfootnotelinepenalty=10000 %Footnote in Laporte chapters has to be split SN


%\DeclareIndexNameFormat{default}{%
%\nameparts{#1}%
%\usebibmacro{index:name}%
%{\index[names]}%
%{\namepartfamily}%
%{\namepartgiveni}%
% {}% L1
% {}% L2
%{\namepartprefix}% generates spurious space L3
%{\namepartsuffix}% generates spurious space L4
%}

%  {\DeclareIndexNameFormat{default}{%
%     \usebibmacro{index:name}{\index[names]}{#1}{#3}{#5}{#7}}}

%\DeclareIndexNameFormat{default}{%
%  \usebibmacro{index:name}{\sindex[nom]}{#1}{#3}{#5}{#7}}

%\DeclareIndexNameFormat{default}{%
%  \usebibmacro{index:name}{\sindex[person]}{#1}{#3}{#5}{#7}}
%\DeclareIndexNameFormat{default}{%
%\nameparts{#1} \usebibmacro{index:name}{\sindex[person]]}{\namepartfamily}{‌​\namepartgiven}{\nam‌​epartprefix}{\namepa‌​rtsuffix}}

%\newcommand{\smiley}{:)}

%\renewbibmacro*{index:name}[5]{%
%\usebibmacro{index:entry}{#1}%
%{\iffieldundef{usera}{}{\thefield{usera}\actualoperator}\mkbibindexname{#2}{#3}{#4}{#5}}}

% \newcommand{\noop}[1]{}

%remove for final
%\overfullrule=1mm

\newcommand{\tobi}[2]}}
\renewcommand{\S}[1]{\tobi{#1}{\textsc{*}}}

% this volume references
% puts: [this volume]
% already defined: \citetv
%\newcommand{\citepv}[1]{(\citeauthor{#1} \citeyear*{#1} [this volume])}
\newcommand{\citealtv}[1]{\citeauthor{#1} \citeyear*{#1} [this volume]}

%parentheses around example number
\newcommand{\pref}[1]{(\ref{#1})}

% in-text examples

\newcommand{\lnex}[1]{\textit{#1}} %target lang word
\newcommand{\lnlit}[1]{(lit.: `#1')} %literal reading
\newcommand{\lnlat}[1]{(#1)} % latinization
\newcommand{\lntrans}[1]{`#1'} %translation
\newcommand{\lnexl}[2]%
{\lnex{#1}{} \lnlat{#2}} % ex with latinization
\newcommand{\lnexlat}[3]{\lnex{#1}{} \lnlat{#2}{} \lntrans{#3}} % ex with latinization and tranl.

%ch01
\newcommand{\co}[1]{\mbox{\textbf{#1}}}

%ch09

\newcommand{\cyrbulg}[1]{\begin{otherlanguage*}{bulgarian}#1\end{otherlanguage*}}


%ch10
\newcommand{\nlp}{{\small NLP}}
\newcommand{\mwe}{{\small MWE}}
\newcommand{\rae}{{\small RAE}}
\newcommand{\lvc}{{\small LVC}}
\newcommand{\pos}{{\small P}o{\small S}}
%\newcommand{\todo}[1]{ \textcolor{red}{#1} }

%\renewcommand{\labelenumi}{\theenumi}
%\ainamefmt{{vv}{ll}{, ff}{, jj}} % fullname

\newcommand{\biberror}[1]{{\color{red}#1}}

\newcommand{\osenovaitem}{--~} 
  %% hyphenation points for line breaks
%% Normally, automatic hyphenation in LaTeX is very good
%% If a word is mis-hyphenated, add it to this file
%%
%% add information to TeX file before \begin{document} with:
%% %% hyphenation points for line breaks
%% Normally, automatic hyphenation in LaTeX is very good
%% If a word is mis-hyphenated, add it to this file
%%
%% add information to TeX file before \begin{document} with:
%% %% hyphenation points for line breaks
%% Normally, automatic hyphenation in LaTeX is very good
%% If a word is mis-hyphenated, add it to this file
%%
%% add information to TeX file before \begin{document} with:
%% \include{localhyphenation}
\hyphenation{
    Beck-man
    Ngu-yen
    back-chan-nel
    back-chan-nels
    mo-not-o-nous
    ste-reo-typ-i-cal
}

\hyphenation{
    Beck-man
    Ngu-yen
    back-chan-nel
    back-chan-nels
    mo-not-o-nous
    ste-reo-typ-i-cal
}

\hyphenation{
    Beck-man
    Ngu-yen
    back-chan-nel
    back-chan-nels
    mo-not-o-nous
    ste-reo-typ-i-cal
}
 
  \togglepaper[1]%%chapternumber
}{}

\begin{document}
\AffiliationsWithIndexing{}
\lehead{Merete Anderssen et al.}
\maketitle 

\section{Introduction}
\label{sec:anderssen:1}

Verb second (V2) word order is a central property of Germanic languages, including Norwegian. It refers to the finite verb appearing in the second position in main clauses, as illustrated in \REF{ex:anderssen:1}.

\ea%1
    \label{ex:anderssen:1}
    \gll På mandager \textbf{spiser} Hedda grøt.\\
         on Mondays  eats     Hedda porridge\\
    \glt ‘On Mondays, Hedda eats porridge.’
    \z 

Indeed, it is difficult to imagine a book on North American Norwegian (NAmNo) that does not address the question of how verb placement in main and embedded clauses is affected by the heritage situation. As discussed in \citetv{chapters/introduction}, Norwegian is an asymmetric V2 language, with verb movement in main clauses but generally not in embedded structures. Consequently, the order finite verb\hyp negation\slash adverbial is expected in main clauses, while the opposite order, negation\slash adverbial\hyp finite verb, is expected in embedded structures, even though main clause word order is found in European Norwegian (EurNo) in a limited number of embedded contexts (see, e.g., \citealt{Ringstad2019} and the discussion below). Several studies of heritage languages have shown that V2 remains quite robust in situations of diminished language exposure and use (e.g., \citealt{Håkansson1994}, \citealt{Schmid2002}), and as we will see in this chapter, this is the case also for North American Norwegian. Violations of verb second are observed only occasionally, and following \citet{WestergaardLohndalLundquist2021}, we suggest that the low activation of Norwegian causes crosslinguistic influence (CLI) from English in production, initially resulting in a decrease in the proportion of non-subject initial structures and a subsequent weakening of the V2 grammar (i.e., making it more susceptible to influence from English). Embedded clauses are found to be considerably more vulnerable, and these structures are characterized by the overapplication of verb movement. This overuse of verb movement may seem surprising, given that embedded clauses with negation or an adverbial are very infrequent structures, and a more likely outcome of the heritage situation might be a complete loss of the few contexts in which the finite verb can precede negation/adverbials in embedded clauses. We argue that a conspiracy of factors can explain this behavior. Based on findings in \citet{HoppPutnam2015} and \citet{JensbergEtAl2024}, we suggest that the development of embedded clauses can be explained by a combination of what we will call \textit{language-internal drift} and influence from English, pulling in the direction of word order symmetry between main and embedded clauses. However, we also argue that there is more direct structural crosslinguistic influence from English, specifically in embedded structures which do not allow verb movement in EurNo, because in these structures, there is a much stronger tendency for auxiliaries to precede negation and adverbials than thematic verbs, making NAmNo structurally similar to English. We also briefly discuss to what extent verb placement in main and embedded structures in this variety of Norwegian can contribute to the ongoing discussion about whether verb movement in non\hyp subject\hyp initial and subject\hyp initial main clauses should be analysed the same way.

The chapter starts with a background (\sectref{sec:anderssen:2}) that introduces the syntax of V2, as well as previous studies of verb placement in main and embedded structures in other heritage populations. Then we provide an overview of studies on V2 in main clauses and show how they account for the characteristics observed in the heritage variety (\sectref{sec:anderssen:3}), before we do the same for verb placement in embedded clauses (\sectref{sec:anderssen:4}). The chapter finishes with a brief discussion (\sectref{sec:anderssen:5}) and conclusion (\sectref{sec:anderssen:6}).

\section{Background}
\label{sec:anderssen:2}
In this section, we introduce some relevant background regarding the syntax of V2, with a particular focus on the syntax of V2 in heritage languages.

\subsection{The syntax of verb second word order}
\label{sec:anderssen:2.1}
Verb second (V2) word order is typically defined as a requirement that the finite verb appear in the second position in main clauses, generally assumed to be the result of syntactic movement to the left periphery (e.g., \citealt{DenBesten1983, HolmbergPlatzack1995, Vikner1995}). The traditional analysis of subject\hyp initial and non\hyp subject\hyp initial declaratives such as (2a, b) is thus that the finite verb moves to C and the initial element appears in SpecCP (\figref{fig:anderssen:fromex:2c}).

\ea%2
    \label{ex:anderssen:2}
	\ea \label{ex:anderssen:2a}
	\gll Studenter \textbf{sover} \textbf{aldri}.\\
	students sleep never\\
	\glt `Students never sleep.’

	\ex \label{ex:anderssen:2b}
	 \gll {I går}  \textbf{sov} \textbf{studentene}.\\
		  yesterday slept students.\textsc{def}\\
	\glt  ‘Yesterday the students slept.’
	\z
\z


\begin{figure}[h]
\caption{Structural representation of non-subject-initial declaratives}
\label{fig:anderssen:fromex:2c}
\begin{forest}
[CP
    [DP\\\textit{I går}]    
    [C’
		[C\\\textit{sov}]    
		[TP
			[DP\\\textit{studentene}]
			[T'
				[T\\\sout{\textit{sov}}]
				[\textit{v}P 
                  [\sout{\textit{studentene sov i går}},roof] 
				]
			]
		]
	]
]
\end{forest}
\end{figure}


More recent cartographic approaches to V2 make it harder to state the V2 requirement relative to the C head, which is why \citet[375]{Holmberg2015} suggests a definition of V2 that is unspecified with respect to the actual landing site of the verb and the fronted element‚ as in \REF{ex:anderssen:3}.

\ea%3
    \label{ex:anderssen:3}
    \ea A functional head in the left periphery attracts the finite verb.\\
  \ex  This functional head wants a constituent moved to its specifier position.
\z
\z

Both requirements must be met. An intensely debated issue in the analysis of V2 is whether subject\hyp initial and non\hyp subject\hyp initial declaratives are derived in the same way; see \citet{Holmberg2015} and \citet{WestergaardLohndalAlexiadou2019} for references to the relevant literature. Two different syntactic analyses have been developed based on their approach to this issue, known as the symmetric and the asymmetric analyses, respectively. If we take the structure in \figref{fig:anderssen:fromex:2c} as our point of departure, the symmetric analysis assumes that both subjects and non-subjects move to SpecCP. The main argument for this analysis is the observation that V2 is a main clause phenomenon in most Germanic languages; it is in complimentary distribution with complementizers, which are argued to occupy the same position as the finite verb. The asymmetric analysis, on the other hand, argues that subjects and non-subjects do not move to the same position, and instead, subjects are argued to occupy a lower position. The main argument for this is that this verb movement as well as the corresponding subject movement to C would be vacuous, and it is questionable whether such movement is learnable. In support of the asymmetric analysis, \citet{WestergaardLohndalAlexiadou2019} review a range of acquisition/attrition evidence from different populations and languages, showing that the two constructions do not behave the same, in that they are not necessarily learned (or lost) at the same time.

The V2 requirement is found in all Germanic languages, to a greater or lesser extent (see e.g., \citealt{Westergaard2007,Westergaard2008,Westergaard2009Microvariation}), and it is typically a requirement of main clauses. With the exception of Icelandic and Yiddish, the V2 requirement usually does not apply in embedded clauses, making V2 asymmetric in most Germanic languages. Thus, in embedded clauses, subjects typically appear in SpecTP, and the verb either stays in situ or moves to T. An example of a typical embedded clause in Norwegian with the verb in situ is given in \REF{ex:anderssen:4}.

\ea%4
    \label{ex:anderssen:4}
    \gll Jeg vet    [at    Emilie ikke \textit{spiser} kjøtt]\\
  I     know that Emilie not   eats    meat\\
\glt  ‘I know that Emilie doesn’t eat meat.’
\z

For Norwegian in particular, there is considerable variation and deviation from the main V2 pattern in main clauses across many dialects (see, among others, \citealt{Westergaard2009Acquisition, WestergaardVangsnesLohndal2017, LohndalWestergaardVangsnes2020, Westendorp2021} for recent overviews). The variation is most salient in different types of main clause questions, where many dialects optionally have non-V2 word order in all or a subset of question types. In \REF{ex:anderssen:5}, this is illustrated for the Tromsø dialect (from \citealt{Westergaard2009Acquisition}: 23).

\ea%5
    \label{ex:anderssen:5}
\ea \gll Ka slags bil  kjøpte du? / *Ka slags bil du kjøpte?\\
which {kind of}   car bought you\\
\glt ‘Which car did you buy?’

\ex  \gll Korfor gikk  ho? / *Korfor ho gikk?\\
why     went she\\
\glt ‘Why did she go?’

\ex  \gll Korsn har   ungan         det? / *Korsn ungan har det?\\
		  How    have kids.\textsc{def}  it\\
\glt ‘How are the kids doing?’

\ex  \gll Ka    legen         sa? / Ka sa legen?\\
what doctor.\textsc{def} said\\
\glt ‘What did the doctor say?’
\z
\z

We briefly discuss some of this variation in \sectref{sec:anderssen:3}, but other types of V2 variation in main clauses in Norwegian will not be discussed in the present chapter.

In embedded clauses, there are environments which are known to facilitate additional verb movement, for instance verb movement that looks like V2 in main clauses. Bridge verbs such as \textit{say} or \textit{tell} are typically known to do this \citep{Vikner1995}, though later work has also shown that verb movement can take place independently of this property (e.g., \citealt{Bentzen2007}, \citealt{WiklundEtAl2009}). Generally, the occurrence of V2 in embedded clauses has been found to be dependent on many different factors, including the semantic status of the matrix verb, the type of embedded clause, and discourse status (\citealt{Heycock2006, WiklundEtAl2009, Julien2007,Julien2015, Bentzen2014, Ringstad2019}). An example is provided in \REF{ex:anderssen:6} (cf. \REF{ex:anderssen:4}).

\ea%6
    \label{ex:anderssen:6}
\gll Jeg vet    [at    Emilie \textit{spiser} ikke kjøtt]\\
      I     know that Emilie eats     not  meat\\
\glt ‘I know that Emilie doesn’t eat meat.’
\z

\citet{Ringstad2019} investigates five different corpora of spoken Norwegian to examine to what extent the semantics of the matrix verb and the type of embedded clause influence the use of verb movement in embedded clauses involving negation. Overall, she finds that the verb precedes the negation in 33\% (377/1145) of these structures, varying from 21\% to 43\% across corpora. \citet{Ringstad2019} also explores to what extent different types of clauses exhibit verb movement and finds no examples in relative clauses and embedded questions, supporting previous research. However, when it comes to adjunct clauses, the picture is a bit more varied. While temporal and conditional clauses generally are not attested with verb movement, as expected, other types of adjunct clauses, such as causal and consequential clauses (especially \textit{fordi} ‘because’ and \textit{slik at} ‘so that’ clauses) are attested with verb movement relatively frequently. 

\citet{Ringstad2019} also investigates whether auxiliaries are more likely to precede negation than thematic verbs in embedded clauses and finds that, while there is a difference between the two, it is quite marginal, with auxiliaries preceding the negation at 36\%, lexical verbs at 33\%, and the copula at 30\%. 

\subsection{Verb second in heritage languages}%2.2
\label{sec:anderssen:2.2}
In recent years, several scholars have investigated V2 in different Germanic heritage varieties. Heritage Norwegian has been particularly well studied, e.g., by \citet{Strømsvåg2013}, \citet{EideHjelde2015Verb, EideHjelde2018}, \citet{Johannessen2015Germanic}, \citet{Khayitova2016}, \citet{AlexiadouLohndal2018}, \citet{WestergaardLohndal2019}, \citet{LundquistEtAl2020} and \citet{WestergaardLohndalLundquist2021}, but there are also studies on Heritage Danish by \citet{KühlHeegård2018}, Heritage Swedish by \citet{LarssonJohannessen2015Embedded,LarssonJohannessen2015Incomplete} and \citet{LarssonKinn2022}, and Heritage Icelandic by \citet{ArnbjoernsdottirEtAl2018} (see also \citealt{Angantysson2023}). Here we review only a couple of earlier studies that all illustrate the resilience of V2 in Germanic heritage languages.

The investigation of V2 in heritage languages started with \citet{Håkansson1995}, who studied five bilingual expatriate (heritage) speakers with slightly different backgrounds. One grew up in France learning Swedish and French, another grew up in Sweden and France and acquired both Swedish and French, whereas the last three speakers grew up in the US where they used English at school and Norwegian or Swedish at home. At the time of testing, the five speakers were also studying Swedish as a second language in Sweden. Håkansson’s main finding was that V2 word order is resilient to attrition. She also compared the five heritage speakers to L2 learners of Swedish and found that the latter group was significantly different: they frequently made V2 mistakes, whereas only one of the heritage speakers as a group made \textit{one} mistake in total. \citet[160]{Håkansson1995} concluded that “the V2 rule resists attrition”. 

\citet{Schmid2002} provides evidence that goes in the same direction. In Schmid’s study, she considered data for 54 German Jews who had emigrated to England and the US during the Nazi regime. In her corpus, Schmid found 5,050 sentences that require V2 word order, and only 2\% (102/5,050) had non-target word order. This suggests that V2 is resilient also in this population. However, a couple of caveats are in order: Schmid does not provide the total number of subject\hyp initial clauses in the corpus, and furthermore, we do not know if some of the speakers were influenced by English and displayed a word order more like English.

\citet{HoppPutnam2015} collected production and acceptability judgment data from speakers of Moundridge Schweitzer German, a variety of heritage German spoken in the USA (Kansas). Their general finding is that V2 is retained in main clauses, again demonstrating that this feature of the grammar is quite stable.

Unlike main clause V2, embedded verb placement has not been extensively investigated in heritage varieties. To our knowledge, there are only three studies: \citet{LarssonJohannessen2015Embedded} for heritage Norwegian and Swedish, \citet{Hartling2016} for heritage Danish, and \citet{HoppPutnam2015} for heritage German. \citet{HoppPutnam2015} show a higher use and acceptance of what they refer to as embedded V2 (movement to T past objects and negation/adverbials) in Moundridge Schweizer German, notably in clauses introduced by the complementizers \textit{dass} ‘that’ \REF{ex:anderssen:7} and \textit{weil} ‘because’ \REF{ex:anderssen:8}.

\ea%7
    \label{ex:anderssen:7}
    \gll … dass da  Lieber Gott \textbf{hot}  uns auch \textbf{net} alles          genomm wie dat    in Oklahoma\\
	   {}  that the dear    God  has           us also  not         everything     taken  like there in O.\\
\glt  ‘that the dear God hasn’t taken everything away from us like in Oklahoma.’

(Participant 102; \citealt{HoppPutnam2015}: 195)
\z

\ea%8
    \label{ex:anderssen:8}
    \gll ... weil        ich \textbf{duh}      \textbf{net} Hochdeutsch  redde\\
         {}   because    I     {do/can}          not         High.German talk\\
\glt  ‘because I can’t speak standard German’

(Participant 103; \citealt{HoppPutnam2015}: 195)
\z

The word order is mainly V2 in embedded clauses with these complementizers. Thus, \citet{HoppPutnam2015} argue that there ‘is little to no evidence in the production data that English SVO word order has affected [Moundridge Schweitzer German]’ (\citealt{HoppPutnam2015}: 203). Instead, they claim that ‘the combination of lesser use or activation of [Moundridge Schweitzer German] and crosslinguistic influence from English which does not instantiate asymmetric word order in main and subordinate clause contexts leads to a particular type of levelling of word order distinctions across clause types within the constraints afforded by German syntax’ (\citealt{HoppPutnam2015}: 206). Furthermore, since Modern German is currently developing options that license V2 in certain embedded clauses, Hopp \& Putnam speculate that these changes may be instances of what they call ‘typological drift’.

\section{Main clauses}
%3
\label{sec:anderssen:3}
As mentioned in the previous section, V2 in main clauses in North American Norwegian heritage language has been investigated in numerous studies. Common for these articles is that they all build on the spoken material from the CANS corpus \citep{Johannessen2015CANS}. In this overview, we mainly summarize the findings in \citet{WestergaardEtAl2021}, as it constitutes the most systematic investigation of V2 and V2 violations in main clauses, but we also point out similarities and differences between the findings of this study and other studies (e.g., \citealt{Strømsvåg2013, EideHjelde2018, Johannessen2015Germanic, Khayitova2016}). The study in \citet{WestergaardEtAl2021} covers all the declarative main clauses from the 50 speakers that were included in the corpus at the time when the study was carried out. In the summary below, we also briefly mention the result from a recent corpus study that targets the word order in main and embedded questions.\largerpage

The study in \citet{WestergaardEtAl2021} covers in total 10,609 declarative clauses.{\interfootnotelinepenalty=10000\footnote{The study focuses on a subset of the declarative utterances that either have or could potentially have non-V2. That is, simple SV(O) clauses without adverbs are excluded, as well as 16 sentences with non-V2 containing sentence adverbs such as \textit{kanskje} \textrm{‘maybe’ and} \textrm{\textit{bare}} ‘only’, which are grammatical in homeland Norwegian. Furthermore, topic drop and instances of left dislocation have also been excluded; see \citet{BousquetteEtAl2021}. In this reduced dataset, V2 violations make up 6.5\% (230/3,534).}} 
Out of these, only 2.2\% (230 clauses) contain illicit non-V2 structures. Examples of licit structures with V2 are given in \REF{ex:anderssen:9} and non-V2 structures which are illicit in EurNo see examples \REF{ex:anderssen:10}.

\ea%9
    \label{ex:anderssen:9}
    \ea  \gll Da \textbf{reiste} hun\\
    then travelled she\\\jambox*{(albert\_lea\_MN\_01gk)}
	\glt    ‘Then she travelled.’

	  \ex \gll Søskena          \textbf{snakka} bare engelsk\\
	    siblings.\textsc{def} spoke    only English\\\jambox*{(stillwater\_MN\_01gm)}
	\glt    ‘The siblings only spoke English.’
	\z
\ex%10
    \label{ex:anderssen:10}
    \ea \gll Quicktrip de    \textbf{kaller} det\\
    quicktrip  they call     it\\\jambox*{(coon\_valley\_WI\_02gm)}
    \glt ‘Quicktrip they call it.’

	\ex  \gll Han aldri   \textbf{kom}\\
	    he    never came\\\jambox*{(flom\_MN\_02gm)}
	\glt    ‘He never came.’
	\z
\z

This small number of V2 errors is in line with some previous studies of heritage V2 languages (e.g., \citealt{Strømsvåg2013, LarssonJohannessen2015Incomplete}), but it is still different from the almost complete absence of V2 violations reported in studies of older recordings or from smaller sets of speakers (\citealt{Haugen1953, Hjelde1992}). However, as \citet{WestergaardEtAl2021} emphasize, the low number of V2 errors is partly due to the fact that many of the speakers in the corpus produce mainly subject\hyp initial sentences (see below), in which it is not possible to distinguish SVO from V2 word order. Out of 10,609 relevant sentences, only 3,534 provide contexts where V2 violations could be detected. As described in the background section, there are two types of declarative clauses in which it is possible to tease apart SVO from V2: (i) non\hyp subject\hyp initial clauses and (ii) subject\hyp initial clauses with sentence adverbials. In the non\hyp subject\hyp initial clauses ($n=1{,}961$), there were 9.6\% ($n=188$) V2 errors. In the subject\hyp initial sentences with sentence adverbials, there were much fewer errors, only 58 out of 1,810 clauses (3.2\%). In most of the clauses with sentence adverbials, the adverb is negation, and in these clauses the proportion of V2 errors is extremely low, only 2\% (32/1,564). In the clauses with other types of sentence adverbs, however, the proportion of V2 errors is similar to that of the non\hyp subject\hyp initial clauses (26/246, 10.6\%). In short, V2 is generally target\hyp like in subject\hyp initial sentences with negation, while around 10\% of V2 errors are observed both in non\hyp subject\hyp initial clauses and subject\hyp initial clauses with other sentence adverbs than negation. This pattern could in principle be explained as a result of influence from English: English exhibits non-V2 in non\hyp subject\hyp initial declaratives and declaratives that contain sentence adverbs, while main clauses with negation always have an auxiliary verb preceding negation (\textit{do}). Note, however, that there are many instances of Verb-Negation order in CANS where the finite verb is a lexical verb, see \REF{ex:anderssen:11}.

\ea%11
    \label{ex:anderssen:11}
    \gll  Jeg \textbf{gikk} \textbf{ikke} på skolen der\\
          I     went        not           on  school.\textsc{def} there\\\jambox*{(wanamingo\_MN\_04gk)}
\glt   ‘I didn’t go to school there.’
\z

\citet{WestergaardEtAl2021} further investigate which linguistic factors influence the likelihood of V2 violations in clauses where potential V2 violations may be spotted (cf. footnote 1). They look more closely into the properties of (i) the subject, (ii) the finite verb, and (iii) the fronted element in non\hyp subject\hyp initial clauses and find effects of all three factors. First, they show that both verb type and subject type matter: Non-V2 is more common with lexical verbs (8.4\%) than auxiliaries and copula verbs (3.6\%), and more common with DP subjects (9.7\%) than pronominal subjects (6.2\%). Sentences with a DP subject and a finite lexical verb have the highest percentage of V2 violations (14\%). In sentences with a pleonastic subject (\textit{det} `it') and an auxiliary or copula verb, non-V2 is virtually non-existent (fewer than 2\%). 

As for the initial element, \citet{WestergaardEtAl2021} divide the fronted non-subject elements into eight distinct categories based on their form and function: \REF{ex:anderssen:1} embedded clauses (Emb); \REF{ex:anderssen:2} prepositional phrases (PP); \REF{ex:anderssen:3} adverb phrases (AdvP, e.g. ‘sometimes’, ‘this day’); \REF{ex:anderssen:4} direct objects (Object); \REF{ex:anderssen:5} simple adverbs (\textit{da/nå}, ‘then’ and ‘now’); \REF{ex:anderssen:6} a phrase followed by the connector \textit{så} (Ph-så ‘so’); \REF{ex:anderssen:7} quotes, and \REF{ex:anderssen:8} the short adverbial connector \textit{så}. In the examples (\ref{ex:anderssen:12}a--g), we provide licit (\ref{ex:anderssen:12}a, e, f) as well as illicit (\ref{ex:anderssen:12}b, c, d, g) non\hyp subject\hyp initial declaratives with the eight categories of initial element.

\ea%12
    \label{ex:anderssen:12}
    \ea \gll Når   vi   kom   til Oslo \textbf{viste}     hun  …\\
             when we came to Oslo showed she\\\jambox*{(wanamingo\_MN\_04gk)}
\glt    ‘When we came to Oslo she showed …’

  \ex \gll Opp på fjellet              de    \textbf{har}   {seter.\hspace*{1.1115cm}(coon\_valley\_WI\_12gm)}\\
    up   on  mountain.\textsc{def} they have summer.mountain.farm\\
\glt    ‘On the mountain they have a summer mountain farm.’

  \ex  \gll En  gang vi  \textbf{kunne} ikke finne ham.\\
    one time we could   not  find   him\\\jambox*{(gary\_MN\_01gk)}
\glt    ‘One time we couldn’t find him.’

  \ex \gll Han jeg \textbf{kunne} ikke forstå.\\
    he    I    could   not   understand\\\jambox*{(portland\_ND\_01gm)}
\glt   ‘Him I couldn’t understand.’

  \ex \gll Nå \textbf{kommer} naboene.\\
    now come neighbors.\textsc{def}\\\jambox*{(sunburg\_MN\_03gm)}
\glt    ‘Now the neighbors are coming.’

  \ex \gll Når    jeg var  hjemme så \textbf{gikk}  jeg ...\\
    when I     was home     so went I\\\jambox*{(wanamingo\_MN\_04gk)}
\glt    ‘When I was at home, I went …’

  \ex  \gll «Nei takk» jeg \textbf{sa}.\\
    no thanks I said\\\jambox*{(harmony\_MN\_02gk)}
\glt    ‘“No thanks”, I said.’
\z
\z

In Figures~\ref{fig:anderssen:1} and \ref{fig:anderssen:2} (from \citealt[19]{WestergaardEtAl2021}), we show the proportion of V2 and non-V2 clauses for the different types of initial element, both in actual numbers and proportions.
    
\begin{figure}
\includegraphics[width=.8\textwidth]{figures/Fig2_Handbook.png}
\caption{V2 with different initial elements (numbers)}     
\label{fig:anderssen:1} 
\end{figure}

\begin{figure}
\includegraphics[width=.8\textwidth]{figures/Fig3_Handbook.png}
\caption{Proportion of V2 with different initial elements}
\label{fig:anderssen:2}
%%\includegraphics[width=\textwidth]{figures/Chapter8AmNoV2Second-img001.emf}
\end{figure}

As is evident from the graphs, the type of initial element has a strong influence on word order. Clauses with initial short adverbs like \textit{da} (‘then’)\textit{, nå} (‘now’) or \textit{så} (‘so’), as well as sentences that contain \textit{så} in addition to a longer clause\hyp initial element, very rarely give rise to V2 violations. This pattern has also been reported by \citet{EideHjelde2015Verb} and \citet{Johannessen2015Germanic}. The elements that are most likely to appear with V2 violations are those that are rarely fronted, namely embedded clauses, PPs, and adverbial phrases. As is shown in \citet{WestergaardEtAl2021}, the patterns cannot straightforwardly be explained in terms of the length of initial element: for the fronted phrasal categories that can be longer than one word, length does not affect the likelihood of V2 violations; see also section 4.3 in \citet{Khayitova2016} for problems with direct length/weight-based explanations of V2 violations.

The patterns above do not necessarily support any specific theory of V2: errors are found both in subject\hyp initial and non\hyp subject\hyp initial clauses, and both with lexical verbs and auxiliary verbs. Nor do the patterns suggest that the heritage speakers have developed a new grammar that is qualitatively different from the grammar of non-heritage speakers in Norway; see also \citet{LarssonJohannessen2015Incomplete} for the same conclusion. Rather, it seems like V2 violations are likely to occur in contexts where there is a lot of new or heavy information in the left periphery: the proportion of V2 errors are higher in sentences with full phrasal subjects compared to pronominal subjects, with lexical verbs compared to auxiliaries, and with phrasal adverbials compared to light adverbial sentence connectors. V2 errors are nearly absent in clauses with commonly used combinations of auxiliaries, subjects, and initial elements, e.g., \textit{nå er det…} (‘now is it…’). Overall, the findings suggest that the V2 violations are related to processing rather than grammar: lexically heavy material as well as rarely used structures increase the chances of V2 errors.

As a last step, \citet{WestergaardEtAl2021} investigate whether V2 errors are equally distributed among the speakers in the CANS corpus. They find that all the 50 speakers produce V2 sentences in contexts where English would not allow V2, i.e., they all have a V2 grammar. However, they differ in the proportion of V2 violations they make. Nine of the 50 speakers make no errors at all (although 4 of them produce fewer than 15 unambiguous V2 contexts), and most of the participants make fewer than 10\% errors. Yet, there are speakers who make 20\% V2 errors, or as many as 60\%; see also \citet{Khayitova2016}, who finds similar individual variation in her analysis of a smaller set of speakers from the CANS corpus. \citet{WestergaardEtAl2021} carefully investigate the relationship between the proportion of V2 errors and the proportion of non\hyp subject\hyp initial clauses on an individual level. European Norwegian, just like other V2 languages, has quite a high proportion of non\hyp subject\hyp initial sentences, about 30--40\% (depending on genre; see \citealt{Olsen2019}). In English, non\hyp subject\hyp initial structures make up less than 10\% of structures (see \citealt{Yang2001}). \citet{WestergaardEtAl2021} find that speakers who have a low proportion of non\hyp subject\hyp initial declaratives produce a higher proportion of V2 violations (see also \citealt{WestergaardLohndal2019} for the initial observation). Speakers who have a more “native-like” fronting pattern (20--40\% non\hyp subject\hyp initial declaratives) make few or no V2 errors. This observation raises the question of whether the low proportion of non\hyp subject\hyp initial clauses is causing the weakening of the V2 grammar, or whether it is an effect of this. The authors suggest that the fronting pattern is affected first in this variety of heritage Norwegian: because of their predominant use of English, the speakers first change the syntactic encoding of information structure by producing more subject\hyp initial sentences in contexts where European Norwegian would use non\hyp subject\hyp initial structures. 

The hypothesis that this tendency to avoid non\hyp subject\hyp initial clauses precedes the weakening of V2 is supported by a cross\hyp generational study reported in \citet{LarssonToappear}. They show that the proportion of subject\hyp initial clauses gradually increases across generations, and that the start of this development predates the first V2 violations. The increase in subject\hyp initial clauses does not in itself result in a non-V2 grammar, as these structures will have a V2-compatible SVO order. However, as the speakers produce more subject\hyp initial sentences, the “V2-muscle” also gets less “practice”, and a V2 grammar out of practice is more likely to produce errors, especially in contexts that are lexically and syntactically demanding from a processing perspective (e.g., lexically “heavy” left-periphery elements and syntactically rare fronting patterns). This suggests that the V2 violations are caused by the low activation of Norwegian (\citealt{PutnamSánchez2013}), which is likely to have persisted for many years in this (elderly) population, causing the structure of English to occasionally affect the production of Norwegian V2 structures, especially in lexically and syntactically demanding contexts.

Finally, we would like to add a note about the word order in different types of questions. As was illustrated in \sectref{sec:anderssen:2}, many Norwegian dialects have variable V2 in main clause questions. Most dialects with variable V2 also have restrictions on the form of the \textit{wh}{}-word (monosyllabic interrogative pronouns are more likely to trigger non-V2), see e.g., \citet{Westergaard2009Microvariation,Westergaard2009Acquisition}. Furthermore, in subject questions, the non-V2 properties are expressed through the presence of the relative marker/complementizer \textit{som} (‘that’) which surfaces directly after the subject \textit{wh}{}-phrase, just like in embedded questions. This variable V2 is also present in heritage Norwegian, although we are not aware of any published work that systematically studies this. In the preparation of this chapter, we investigated word order in main clause questions in the 50 speakers that were analyzed in \citet{WestergaardEtAl2021}. We found 333 questions, of which 79 had non-V2 word order. The non-V2 examples all appear to be correct dialectal Norwegian. We give examples of a subject question \REF{ex:anderssen:13a}, an object question \REF{ex:anderssen:13b}, and an adverbial question \REF{ex:anderssen:13c} (note that we give the phonetic form of the question word in the examples, and not the standardized written form):   

\ea%13
    \ea \label{ex:anderssen:13a}
    \gll Håkke  \textbf{som} \textbf{var}   president da?\\
         who     that  was  president  then\\\jambox*{(westby\_WI\_03gk)}
	\glt     ‘Who was president then?’

	\ex \label{ex:anderssen:13b}
	\gll Hå     \textbf{dere}     \textbf{skal} gjøre {i morgen}?\\
	         what you.\textsc{pl} will do     tomorrow.\\\jambox*{(coon\_valley\_WI\_07gk)}
	\glt     ‘What are you going to do tomorrow?’

	\ex \label{ex:anderssen:13c}
	\gll Hå   \textbf{du}   \textbf{var}     hen da?\\
	where  you were  at    then?\\\jambox*{(coon\_valley\_WI\_03gm)}
	\glt ‘Where were you then?’
	\z
\z

Observe that (\ref{ex:anderssen:13a}--\ref{ex:anderssen:13c}) are all examples of word orders that exist neither in English nor in standard EurNo. Interestingly, the English translational equivalents of both \REF{ex:anderssen:13b} and \REF{ex:anderssen:13c} would require subject-verb inversion, i.e., residual V2, which shows that the heritage word order has remained surprisingly stable. It is also worth noting that some of the speakers in the corpus only produce V2 questions, but at the time of writing, we are not able to tell if these speakers have their origin in areas where the dialect lacked non-V2 questions.

As a final note on word order in questions, we note that many of the V2 questions indeed have a copula verb or an auxiliary as the finite verb,\footnote{This is the case also in European Norwegian; see e.g., \citet{Westergaard2016}.} i.e., they make up structures that would have V2 in English as well. However, we also find many structures where the subject inverts with a finite main verb, as in the following examples:

\ea%14
    \label{ex:anderssen:14}
    \ea \label{ex:anderssen:14a}
	    \gll Hå     \textbf{lever} \textbf{du}   nå    da?\\
			where   live   you now then\\\jambox*{(blair\_WI\_04gk)}
	\glt ‘So where do you live now?’

	\ex \label{ex:anderssen:14b} 
		\gll Hva    \textbf{gjør} \textbf{man}         da?\\
			what  do     one/man  then\\\jambox*{(kalispell\_MT\_02uk)}
	\glt ‘What do you do then?’
	\z
\z

Example \REF{ex:anderssen:14a} is especially interesting, as it contains a verb whose meaning is borrowed from English. Although the verb \textit{leve} exists in Norwegian, it does not have a locative meaning, but only the reading ‘being alive’. Nevertheless, even with the use of this inappropriate lexical item, V2 is adhered to, demonstrating the resilience of this grammatical property. 

To sum up, the word order patterns found in both declarative and interrogative clauses suggest that the speakers in the CANS corpus have a V2 grammar in main clauses (cf. \citealt{BousquetteEtAl2021, LarssonKinn2022}), although some errors occur in a subset of the contexts. 

\section{Embedded clauses}
%4
\label{sec:anderssen:4}

As mentioned above, Norwegian can be characterized as exhibiting V2 in main clauses but generally not in embedded structures. While V2 word order in Norwegian main clauses is relatively consistent, there is somewhat more variation when it comes to verb placement in embedded clauses, where V2 is attested quite frequently in certain types of embedded clauses, but not in others (cf. \sectref{sec:anderssen:2.1}).  In the types of embedded clauses where V2~is licit, it occurs both in subject\hyp initial embedded clauses, where the finite verb precedes negation or other adverbials \REF{ex:anderssen:16}, and in embedded clauses that are non\hyp subject\hyp initial \REF{ex:anderssen:15}, where inversion is in fact obligatory (\citealt{HolmbergPlatzack1995, Vikner1995, Heycock2006, WiklundEtAl2009, Julien2007, Julien2015, Bentzen2014, Ringstad2019, Westendorp2021}). 

\ea%15
    \label{ex:anderssen:15}
    \gll Per sa     at    den  boka       \textbf{hadde} \textbf{han} ikke lest\\
Per said that that book.\textsc{def} had      he   not read\\
\glt ‘Per said that he had not read that book.’

\ex%16
    \label{ex:anderssen:16}
    \gll Per sa     at    han \textbf{hadde} \textbf{ikke} lest  den boka\\
Per said that he   had      not read that book.\textsc{def}\\
\glt ‘Per said that he had not read that book.’
\z

In the current chapter, we do not consider embedded structures with non-subjects in initial position, that is, embedded V2 clauses where both the subject and the verb move to the C-domain. Rather, we focus on \textit{that}{}-clauses, relative clauses, embedded questions, and various types of adverbial clauses in which the finite verb may potentially precede the negation or an adverb, but not the subject. Thus, verb movement here refers to V-to-T movement.

To our knowledge, there are four studies that investigate verb placement in embedded clauses in NAmNo. These are \citet{Taranrød2011}, \citet{LarssonJohannessen2015Embedded,LarssonJohannessen2015Incomplete} and \citet{JensbergEtAl2024}. \citet{LarssonJohannessen2015Embedded,LarssonJohannessen2015Incomplete} discuss increased use of V-Adv in embedded structures in the heritage variety, and they show that this also occurs in structures where it is not permitted in European Norwegian, such as relative clauses (see also \citealt{Taranrød2011}). \citet{LarssonJohannessen2015Embedded} compare a subset of the Nordic Dialect Corpus (old informants from Oppland County) to the CANS Corpus, which at the time consisted of considerably less data than it does now. The results reveal a clear difference between the two corpora. While the Nordic Dialect Corpus exhibits a high proportion of V-Adv in \textit{that-}clauses (13/32) but hardly any examples in relative clauses (1/17), the CANS speakers use Verb-Negation order in approximately 50\% of relative clauses (6/13) and almost all the time in \textit{that-}clauses (15/16) (\citealt{LarssonJohannessen2015Embedded}). \citet{LarssonJohannessen2015Embedded} also show instances of verb movement in relative clauses in Haugen’s recordings sixty years ago \REF{ex:anderssen:17} (\citealt{LarssonJohannessen2015Embedded}: 248), but not in first-generation speakers. The latter group exhibits a behaviour very similar to European Norwegian today, with no verb movement in most adjunct clauses, such as conditional clauses \REF{ex:anderssen:18}, but quite frequently in consequence of degree clauses \REF{ex:anderssen:19} (examples from \citealt{LarssonJohannessen2015Embedded}: 253):

\ea%17
    \label{ex:anderssen:17}
    \gll Då     di     kåm   ti detti landi            då    settla   dæ   på en homstedde som \textbf{e}  \textbf{no}    Taon åv Farmington\\
		when they came to this  country.\textsc{def} then settled they on a   homestead that  is now town of  Farmington\\
\glt ‘When they came to this country, they settled in a homestead that is now the town of Farmington.’
          (Winfield Krostu, Waupacs county, born in Wisconsin (1884), recorded 1942) 

\ex%18
    \label{ex:anderssen:18}
    \gll viss du   \textbf{inkje} \textbf{har}   riktig  goe   sement så     dett an sund.\\
		 if     you not     have really good cement then falls he apart\\
 \glt ‘If you don’t have really good cement, it falls apart’\\
        (Jacob Seljestad, born in Hardanger, Norway, 1866, emigrated in 1887, recorded by Einar Haugen in 1942) 

\ex%19
    \label{ex:anderssen:19}
    \gll snøen       va   så dyp   at    me \textbf{kunne} \textbf{ikkje} gå\\
		 snow.\textsc{def} was so deep that we  could  not     walk\\
	 \glt ‘The snow was so deep that we could not walk.’\\
(Jacob Seljestad, born in Hardanger, Norway, 1866, emigrated in 1887, recorded by Einar Haugen in 1942) 
\z

\citet{LarssonJohannessen2015Embedded,LarssonJohannessen2015Incomplete} consider different explanations for the observed increase of verb movement past negation/adverbials and argue that the behaviour can best be explained as incomplete acquisition (e.g., \citealt{Montrul2002,Montrul2008, Polinsky2018}). This is based on the observation that overgeneralization of verb movement to embedded clauses in Norwegian and Swedish is attested in both L1 (\citealt{WestergaardBentzen2007, Waldmann2008}) and L2 learners (\citealt{PienemannHåkansson1999}). It is thus a characteristic of learner language. This proposal is inspired by \citet{WestergaardBentzen2007}, who propose that Norwegian children initially assume V-to-T movement in subject\hyp initial structures (\ref{ex:anderssen:2a} above) for economy reasons. Through exposure to various non\hyp subject\hyp initial structures (e.g., \ref{ex:anderssen:2b} above), the children become aware that Norwegian has V-to-C movement, resulting in a period of co-existence of both types of structures in the learner grammars. It is the absence of verb movement in embedded structures that is the cue for V-to-C movement in all main clauses. However, as mentioned in \sectref{sec:anderssen:2.1}, adult asymmetric V2 grammars have also been argued to have V-to-T in subject\hyp initial structures, see \citet{Holmberg2015} and \citet{WestergaardLohndalAlexiadou2019} for an overview of the discussion. \citet{LarssonJohannessen2015Embedded,LarssonJohannessen2015Incomplete} reject influence from English as an explanation for the overuse of V-Adv in embedded clauses, because English does not allow verb movement, at least not of thematic verbs. They also argue against attrition as the cause of this development, both because verb movement past negation in embedded clauses is more of an innovation than the loss of a feature, and because it seems to affect both fluent and less fluent speakers (\citealt{LarssonJohannessen2015Embedded,LarssonJohannessen2015Incomplete}).

\citet{JensbergEtAl2024} investigate the same 50 speakers in the CANS corpus that were the object of study in \citet{WestergaardEtAl2021} and \citet{LundquistEtAl2020}.\footnote{The search criteria in CANS were subordinating conjunctions and \textit{wh}-words followed by an adverb or the negation \textit{ikke} ‘not’. This identified 264 embedded clauses, which in turn were manually categorized in Excel based on the variables word order (V-Adv or Adv-V), verb type (main verb, copula, auxiliary) and clause type (the typology outlined at the end of this paragraph in the main text).} Of these, 37 speakers produce embedded structures with negation or an adverbial, making it possible to distinguish between clauses with and without verb movement to T. The speakers produce a total of 264 relevant sentences, 60 of which (22.7\%) involve an adverbial, while the remaining involve the negation (\textit{ikke} ‘not’). On average, each speaker produces 7.14 examples, but the standard deviation is 5.87, which means that there is a great deal of individual variation. Based on the overview in \citet{Ringstad2019}, \citet{JensbergEtAl2024} do not take the semantics of the matrix verb into account, but rather consider \textit{that-}clauses, causal clauses (\textit{fordi} ‘because’), and consequence of degree-clauses (\textit{så stor at} ‘so big that’) as clause types that permit verb movement, and temporal-clauses (\textit{når/da} ‘when’), relative clauses (relativizer \textit{som} ‘who/that’), embedded questions, consequential clauses with \textit{så} ‘so that’, and conditional clauses (\textit{hvis/om} ‘if’) as clause types that do not permit verb movement in European Norwegian.

What is clear from the results is that the heritage speakers produce considerably more verb movement in embedded structures than both Norwegian speakers in Norway today and the first-generation immigrants. Like in \citeauthor{LarssonJohannessen2015Incomplete}’s (\citeyear{LarssonJohannessen2015Embedded,LarssonJohannessen2015Incomplete}) studies, this increase includes both a higher proportion of embedded V-Adv word order in structures that exhibit variability in Norwegian and an extension of the phenomenon to structures where it is not acceptable in the homeland variety. Recall that \citet{Ringstad2019} found 33\% verb movement in embedded clauses in the corpora of European Norwegian that she investigated. This proportion is higher in the heritage speakers, who produce V-Adv in 57.4\% (152/264) of embedded structures, 80.6\% (58/72) in clauses where variable verb placement is found in European Norwegian and 47\% (94/192) in clauses where it does not occur.\footnote{In the elicited production experiments of \citet{RingstadKush2021} and \citet{Westendorp2021}, the proportion of V-Neg is a lot lower than in \citegen{Ringstad2019} study, 12.3\% and 11.3\%. However, as \citet{Ringstad2019} is a corpus study, like \citet{JensbergEtAl2024}, it is a better point of comparison than the experimental studies.} \figref{fig:anderssen:3} \citep{JensbergEtAl2024} shows the distribution between verb movement (blue) and no verb movement (red) in different types of embedded clauses, of which causal, consequence of degree and \textit{that-}clauses (on the left) permit it, while the rest do not. \figref{fig:anderssen:3} confirms that the heritage speakers produce V-Adv word order in structures where it is not acceptable in European Norwegian and a higher proportion of verb movement in structures that allow this.

\begin{figure}
\includegraphics[width=.8\textwidth]{figures/Fig4HandbookNew.png}
\caption{Distribution of verb movement (blue) and no verb movement (red) in different clause types.}
\label{fig:anderssen:3}
\end{figure}


Unlike \citeauthor{LarssonJohannessen2015Incomplete}, \citet{JensbergEtAl2024} suggest that the activation of Norwegian influences the extent to which speakers produce V-Adv in embedded clauses. \figref{fig:anderssen:4} \citep{JensbergEtAl2024} provides an overview of the total distribution of embedded clauses with verb movement (blue) and those without verb movement (red) in the 37 speakers who produce relevant structures. There is considerable inter-speaker variation in both the total number of embedded clauses and proportion of verb movement in these structures. Much of the variation is due to the fact that it varies a great deal how many sentences the speakers produce in the course of a recording, ranging from 35 to 812 declarative sentences per speaker. However, some of the variation is also due to differences in the production patterns of the speakers, as some of the heritage speakers produce fewer complex structures, sticking to structurally simple main clauses and avoiding subordination, while others have production patterns more on a par with speakers of European Norwegian. On the assumption that speakers who produce fewer complex structures, such as embedded clauses, also are more likely to make errors when producing these structures (c.f. the discussion on V2 errors in main clauses above), \citet{JensbergEtAl2024} investigate whether there is a relationship between the number of illicit V-Adv structures in embedded clauses and the total number of embedded clauses produced by the speakers relative to their production of main clauses. Indeed, \citet{JensbergEtAl2024} find a significant effect of the main-to-embedded clause ratio on non-target-like verb placement: speakers who frequently produce embedded clauses are less likely to use the V-Adv word order in embedded clauses that do not allow verb movement ($\beta = 0.8736$, $\text{SE} = 0.3495$, $p =0.0124$, mixed effects logistic regression), see \figref{fig:anderssen:5} for an illustration of the effect. 

\begin{figure}
\includegraphics[width=\textwidth]{figures/Fig05_AllPartPlot.png}
\caption{The distribution of verb movement and no verb movement in all embedded clauses, 37 speakers.}
\label{fig:anderssen:4}
\end{figure}

\begin{figure}
\includegraphics[width=.9\textwidth]{figures/Fig06_Ratio.png}
\caption{Predicted effect of main-to-embedded clause ratio on illicit V-adv, 37 speakers.}
\label{fig:anderssen:5}
\end{figure}


The effect shown in \figref{fig:anderssen:5} is reminiscent of what we saw in the section on V2 in main clauses above, which revealed that those speakers who produce few non\hyp subject\hyp initial main clauses also make more errors with V2 \citep{WestergaardEtAl2021}. \citet{JensbergEtAl2024} also find a negative correlation between the proportion of non\hyp subject\hyp initial declaratives and the proportion of V-Adv in clauses that do not permit it ($r = -0.41$, $p = 0.017$*), meaning that heritage speakers who produce fewer non\hyp subject\hyp initial clauses are also less target\hyp like with verb placement in embedded clauses. Furthermore, there is also a correlation between V2 violations in non\hyp subject\hyp initial clauses and the proportion of non-target V-Adv in embedded clauses ($r = 0.44$, $p = 0.009$**). All these analyses made \citet{JensbergEtAl2024} conclude that the increase in the occurrence of verb movement past the negation and adverbs in embedded clauses is the result of lower activation of Norwegian in the heritage speakers (see \citealt{PutnamSanchez2013}).

While \citet{LarssonJohannessen2015Embedded,LarssonJohannessen2015Incomplete} reject crosslinguistic influence as an explanation for the increase of V-Adv in embedded clauses in NAmNo, \citet{JensbergEtAl2024} argue that it looks like it might be a contributing factor. One of \citeauthor{LarssonJohannessen2015Incomplete}’s main arguments against CLI is the fact that English in general does not have verb movement, in neither main \REF{ex:anderssen:20a} nor embedded clauses \REF{ex:anderssen:20b}. However, as \citet{JensbergEtAl2024} point out, auxiliaries and the copula do appear in T in English, both in main and embedded clauses \REF{ex:anderssen:21}, and as a result, they will always precede negation and usually precede adverbials in both structures.

\ea%20
    \label{ex:anderssen:20}
    \ea \label{ex:anderssen:20a} The man (*works) actually (works) up there.
	\ex \label{ex:anderssen:20b} The man who (*works) actually (works) up there.
	\z
\ex%21
    \label{ex:anderssen:21}
	\ea \label{ex:anderssen:21a} The man \textbf{is} \textbf{not/probably} working up there.
	\ex \label{ex:anderssen:21b} The man who \textbf{is} \textbf{not/probably} working up there. 
	\z
\z

Thus, crosslinguistic influence from the position of auxiliaries in embedded clauses in English might also have contributed to the high proportion of V-Adv in embedded clauses in NAmNo. \citet{JensbergEtAl2024} reason that if this were the case, auxiliary verbs and the copula should be more likely to appear in front of negation or adverbs than thematic verbs in NAmNo embedded clauses. Recall that in \citegen{Ringstad2019} study, the proportion of verb movement is very similar for auxiliaries, thematic verbs, and copula (36\%, 33\% and 30\%). In CANS, however, there is a clear difference between the distribution of V-Adv in embedded clauses with auxiliaries and \textit{be} on the one hand and thematic verbs on the other in contexts that permit and contexts that do not permit it in European Norwegian. In the 190 contexts where the V-Adv word order is not permitted in the homeland variety, auxiliaries and the copula exhibit verb movement at 67.9\% (36/53) and 61.8\% (34/55), while thematic verbs involve V-Adv only 29.3\% (24/82) of the time. In contexts that allow V-Adv, there are no differences between the verb types. These results are summarized in \tabref{tab:anderssen:1}.

\begin{table}
\begin{tabularx}{\textwidth}{Qccc}
\lsptoprule
 & Auxiliaries & Be & Lexical verbs\\\midrule
EurNo – contexts that allow V-Adv \citep{Ringstad2019} & 36\% (120/334) & 30\% (73/244) & 33\% (164/493)\\\addlinespace
NAmNo – contexts that allow V-Adv & 88.9\% (24/27) & 83.5\% (14/17) & 85.2\% (14/17)\\\addlinespace
NAmNo – contexts that do not allow V-Adv & 67.9\% (36/53) & 61.8\% (34/55) & 29.3\% (24/82)\\
\lspbottomrule
\end{tabularx}
\caption{\label{tab:anderssen:1}Proportion of V-Adv word order in EurNo and NAmNo with different verb types.}
\end{table}

Finally, \citet{JensbergEtAl2024} suggest that the fact that English has symmetric word order in main and embedded clauses might also potentially have influenced the direction of this development in NAmNo, both as a catalyst for the kind of typological drift discussed in \citet{HoppPutnam2015} (which \citeauthor{JensbergEtAl2024} refer to as language\hyp internal drift) and more directly as CLI.

To sum up, studies on verb placement in embedded clauses reveal that the asymmetry regarding verb movement in main and embedded clauses may be vulnerable in North American Norwegian, as a large increase in verb movement past negation/adverbials has been observed in this variety, both in contexts that allow this word order and in contexts that do not allow it in the homeland variety (cf. \citealt{Taranrød2011, LarssonJohannessen2015Incomplete, LarssonJohannessen2015Embedded}, and \citealt{JensbergEtAl2024}).

\section{Discussion}%5
\label{sec:anderssen:5}

As the survey of the studies in this chapter reveals, there are some interesting differences between verb placement in main and embedded clauses in North American Norwegian. While V2 has been found to be remarkably robust in main clauses in the vast majority of speakers, the lack of verb movement typically associated with embedded clauses in asymmetric V2 languages such as Norwegian is found to be vulnerable. This makes for an interesting starting point for a discussion of what this means for the structural analysis of verb placement in Norwegian and how these data can contribute to our understanding of the factors that determine the course of development in heritage languages in general. The second point is mostly relevant to embedded structures, as these are the ones that appear to be most affected by the language context.  

Regarding the analysis of V2, it is not completely clear how the results should be interpreted. On the one hand, the fact that non\hyp subject\hyp initial clauses and subject\hyp initial structures with an adverbial exhibit similar proportions of non-V2 appears to go against an asymmetric analysis of these two types of structures, that is, an analysis in which the two structures target different landing sites for the finite verb (CP versus TP; see \citealt{Holmberg2015, WestergaardLohndalAlexiadou2019}). On the other hand, if negative structures are included in the latter category, there is a clear difference between the two clause types, with V2 being more vulnerable in non\hyp subject\hyp initial clauses than subject\hyp initial ones. However, given the obligatory presence of an auxiliary in negative structures in English, treating subject\hyp initial structures with negation and adverbials the same way might mask the fact that this V2-like characteristic of English provides some evidence for V2 in negative structures for the heritage speakers (for a more detailed discussion of this and the other issues raised here, see \citealt{WestergaardEtAl2021}). Another question is how the high proportion of non\hyp target\hyp like verb movement in embedded clauses should be interpreted relative to the two analyses of subject\hyp initial main clauses. If we assume that this feature of the heritage grammar is to some extent driven by both typological\slash language\hyp internal drift and an external drive towards symmetry between main and embedded clauses, the latter due to influence from English, this could be interpreted as support for a TP analysis of verb movement in subject\hyp initial structures.

When it comes to the question of which mechanisms can explain the development of verb placement in North American Norwegian, there are several factors that might be at play. First, if we take into consideration the frequency of the different structures discussed in this chapter, we have already seen that more frequent initial elements (see Figures~\ref{fig:anderssen:1} and~\ref{fig:anderssen:2} in \sectref{sec:anderssen:3}) are less likely to trigger V2 violations in non\hyp subject\hyp initial clauses. Second, the frequency of unambiguous V2 structures in main clauses is considerably higher than the frequency of contexts in which verb movement in embedded structures can be observed, with 1,961 examples of non\hyp subject\hyp initial clauses (9.6\% non-target-like) and 1,810 subject\hyp initial structures with negation or an adverbial (3.2\% non-target-like) in main clauses, and only 264 (57.4\% V2) unambiguous contexts in embedded structures. Given the low activation of Norwegian, it is perhaps not surprising that infrequent embedded structures are more vulnerable, especially as they can be subdivided into structures that do (72) and do not (192) allow verb movement. However, low frequency cannot explain the direction of the development, that is, towards more verb movement. Also, given the considerable difference in frequency between non\hyp subject\hyp initial clauses (1,961) and subject\hyp initial structures with an adverbial (246), it is surprising that they appear to be equally target\hyp like in the heritage population. 

\citet{LarssonJohannessen2015Embedded,LarssonJohannessen2015Incomplete} propose incomplete acquisition as an explanation for the overuse of V-Adv in embedded clauses in the heritage population. One problematic aspect of this account is that the stage during which verb movement is overgeneralized to embedded structures in child language ends quite early. \citet{RingstadKush2021} show that verb movement is quite infrequent in relative clauses (less than 20\%) and is only found in children under 5, that is, before the heritage speakers started school and typically became English\hyp dominant. However, it is possible that the overgeneralization of verb movement observed in Norwegian child language may play a role in the development of the heritage language, as the general input situation may have resulted in \textit{differential} acquisition (\citealt{RothmanKupisch2018}) in the course of several generations, making verb movement in embedded clauses ever more frequent. 

One thing that is puzzling about the increase in verb movement in embedded clauses in heritage varieties is that it embodies a process where a structure that is extremely infrequent spreads under conditions of diminished and potentially differential input, that is, conditions under which infrequent linguistic phenomena normally would be expected to disappear. This makes the notion of typological or language-internal drift attractive (\citealt{HoppPutnam2015}, \citealt{JensbergEtAl2024}), that is, the assumption that there is a “drift” within languages towards structural symmetry between embedded and main clauses. \citet{HoppPutnam2015} suggest this as a possible account for the overuse and acceptance of V-Adv in embedded clauses in Moundridge Schweitzer German. While many properties of verb movement in main and embedded clauses are intact in this variety, embedded clauses introduced by the complementizers \textit{dass} `that' and \textit{weil} `because' almost consistently exhibit verb movement past not only objects, but frequently also negation and adverbs, in spontaneous speech. Interestingly, while \textit{weil} ‘because’ is the only complementizer that can traditionally occur with V-Adv word order in modern German (depending on semantic\hyp pragmatic distinctions), \citet{Freywald2008} shows that \textit{dass}{}-clauses with this word order are attested in a corpus of modern spoken German at 0.34\% (in similar pragmatic contexts to \textit{weil}). Despite its low frequency, this word order is steadily increasing in Modern German, and this makes language-internal drift, accelerated by the symmetry between main and embedded clauses in English, an interesting explanation for the unexpected direction of the development in Moundridge Schweitzer German: the occurrence of verb movement increases in a context where a complete loss of these structures would seem a more likely outcome, given their low frequency.

As pointed out by \citet{JensbergEtAl2024}, Norwegian is different from German in many respects. For example, as we have seen, verb movement in embedded clauses is considerably more frequent (33\% in total, \citealt{Ringstad2019}) and is found in many more contexts in Norwegian than in German, which makes the spread of the phenomenon more likely. Furthermore, \citet{JensbergEtAl2024} have data that suggest that it is also a feature of the language of the first generation of immigrants. An example is provided in \REF{ex:anderssen:22}.

\ea%22
    \label{ex:anderssen:22}
\gll men så   spekulerte  han på det og når  de\\
     but then speculated  he  on it and when they \\\jambox*{(viroqua\_WI\_04gm)}
\gll skulle    til Amerika om han \textbf{kunne} ikke likså   godt reise\\
     were.going to America if    he   could   not  just.as well travel\\
\glt ‘But then he speculated on it and when they were going to America, if he could just as well leave.’
\z

This suggests that the language-internal drift where optional verb movement has been expanded beyond the constraints offered by European Norwegian syntax in NAmNo is now developing towards word order symmetry between main and embedded clauses. The fast progression in NAmNo contra Heritage German is not surprising, given that the change from no verb movement to verb movement in embedded clauses in Norwegian involves a much smaller step than in German. In German it involves a change from verb-final SOV to V-Adv in embedded structures \REF{ex:anderssen:23a}, while in Norwegian, it is a change from SVO to a structure with verb movement \REF{ex:anderssen:23b}. The latter can only be distinguished in the presence of an adverbial or negation, making most embedded clauses ambiguous with respect to whether they are SVO or involve verb movement. Thus, if there is a drift towards verb movement in embedded clauses in V2 languages, it is not surprising that it is more advanced in Norwegian, as the evidence that embedded clauses are different from main clauses is much more unambiguous and frequent in German. 

\ea%23
    \label{ex:anderssen:23}
\ea \label{ex:anderssen:23a}
	\gll … dass   er (\textbf{nicht}) Angst \textbf{hat}  \textbf{/}   \textbf{er} \textbf{hat}  (\textbf{nicht}) Angst \\
         … that    he  not     fear    has {}   he has   not      fear \\
\ex  \label{ex:anderssen:23b}
	\gll … at      han (\textbf{ikke}) \textbf{er} redd  /  han \textbf{er} (\textbf{ikke}) redd\\
        ...  that   he    not     is afraid {}   he   is   not     afraid\\
\z
\z

Along the same lines, another way in which Norwegian and German are different is related to precisely how similar the two languages are to English, and especially in embedded clauses.

As European Norwegian does allow verb movement in some embedded clauses, it is not possible to unequivocally conclude exactly what role language internal-drift towards symmetry has played in this development. However, it is likely to be a contributing factor, especially as the same development, albeit less pervasive, is found in Moundridge Schweitzer German. In a low input situation, it is not unlikely that the analysis that corresponds to that of main clauses is preferred.

\citet{WestergaardEtAl2021} suggest that the V2 violations observed in NAmNo main clauses are the result of crosslinguistic influence from English in production\slash processing, caused by the low activation of Norwegian, which initially affects how frequently non\hyp subject\hyp initial declaratives are used and subsequently weakens the V2 grammar (see also \citealt{LarssonKinn2022} for the same argument for American Swedish). \citet{HoppPutnam2015} also appeal to CLI in an overarching way to account for verb movement in embedded clauses, by arguing that language-internal drift is exacerbated by the symmetry in the word order of main and embedded clauses in English. As we have seen, the same argument is made by \citet{JensbergEtAl2024} for embedded clauses in NAmNo. However, Jensberg et al. also argue that there is likely to be some more direct structural CLI from English involved because of the higher proportion of V-Adv with auxiliaries in contexts where verb movement is not permitted (see \sectref{sec:anderssen:4}). Naturally, this does not unequivocally prove that crosslinguistic influence plays a role here. Other studies have found a greater tendency for verb movement in embedded clauses with auxiliary verbs in the homeland variety as well. For example, the experimental study of \citet{RingstadKush2021} shows a distribution of 20\% V-Adv with auxiliaries and 12.3\% with thematic verbs in adult Norwegian 
 speakers. A similar preference has been found for children in several studies, such as \citet{RingstadKush2021} for Norwegian, \citet{Håkansson1994} for Swedish, and \citet{HeycockEtAl2013} for Faroese. Thus, it is possible that there are also language-internal preferences for auxiliaries to precede negation/adverbs more than thematic verbs. It is striking, however, that in the most comparable (corpus-based) study of the homeland variety \citep{Ringstad2019}, the proportion of V-Adv in embedded clauses with auxiliaries and copula is vastly different from that of the heritage speakers, while with thematic verbs, it is quite similar.

A final possible explanation that should be discussed is whether the development observed for verb placement in NAmNo can be attributed to a general tendency towards simplification of the grammar (\citealt{ScontrasEtAl2015}: 3). On the one hand, such a general tendency could clearly explain the decrease in the proportion of non\hyp subject\hyp initial structures in main clauses and the V2 violations in subject\hyp initial clauses with adverbials as avoidance of verb movement. However, it could not really explain the overuse of verb movement in embedded structures, unless we assume that symmetry between main and embedded clauses represents a type of simplification. 

\section{Conclusion}
%6
\label{sec:anderssen:6}

We have seen in this chapter that in North American Norwegian, V2 is relatively robust in main clauses. However, the high activation of English has caused the use of non\hyp subject\hyp initial clauses to go down, which in turn has precipitated a weakening in the V2 grammar which is manifested by occasional V2 violations. Ironically, while the heritage speakers sometimes fail to apply verb movement in main clauses, they overapply it in embedded clauses, both in structures where it is accepted in European Norwegian and in structures where it is not. There is also a clear connection between main and embedded clauses in the sense that an analysis of individual variation reveals that speakers who fail to apply V2 consistently in main clauses have a stronger tendency to overapply verb movement in embedded structures and are also less likely to produce non\hyp subject\hyp initial declaratives \citep{JensbergEtAl2024}. While the slight weakening of the V2 grammar in main clauses can be accounted for quite straightforwardly as a result of the occasional inability to inhibit the highly activated majority language, English, it is less clear why embedded clauses develop in the direction that they do. What is clear, however, is that they are more affected by the lack of language use than main clauses. Given how infrequent embedded clauses are, one might have expected verb movement in these structures to disappear completely in a heritage language situation. Instead, what we see is clear development in the opposite direction. 

Several factors seem to conspire to make this happen. One such factor is the fact that most embedded clauses are ambiguous with regard to whether they involve verb movement or not (i.e., clauses with no negation\slash adverbial), making most embedded structures compatible with both a grammar with and one without verb movement. In a situation of impoverished input and limited use, it might be that the analysis that corresponds to that of the main clause would be preferred, especially as the overapplication of verb movement is a characteristic of Norwegian learner language as well. This situation would be further accelerated and exacerbated by language-internal drift and language-external influence from the majority language driving the heritage language towards symmetry between main and embedded clauses. In addition, more direct structural CLI seems to be at play, causing auxiliaries to precede negation and other adverbials more often than thematic verbs in contexts that do not permit V-Adv. In such a situation, the output of each generation is likely to be more and more skewed towards embedded verb movement to T and symmetry between main and embedded clauses, causing the next generation to apply it even more.

\section*{Abbreviations}
\begin{multicols}{2}
\begin{tabbing}
MMMMI \= Adverbial\kill
AdvP   \>  Adverbial Phrase\\
Adv-V  \>  Adverbial-Verb word order\\
CANS   \>  Corpus of American Nordic \\ \> Speech\\
CLI    \>  Crosslinguistic Influence\\
C(P)   \>  Complementizer (Phrase)\\
DP     \>  Determiner Phrase\\
Emb    \>  Embedded\\
EurNo  \>  European Norwegian\\
NAmNo  \>  North American European\\
PP     \>  Prepositional Phrase\\
SOV    \>  Subject Object Verb\\
SpecCP \>  Specifier of CP\\
SpecTP \>  Specifier of TP\\
SVO    \>  Subject Verb Object\\
T(P)   \>  Tense (Phrase)\\
V-Adv  \>  Verb-Adverbial word order\\
V-to-C \>  Verb movement to C\\
V-to-T \>  Verb movement to T\\
V2     \>  Verb second
\end{tabbing}
\end{multicols}

\section*{Acknowledgements}

We are grateful to two anonymous reviewers for constructive comments on a previous version, as well as to the editors for their feedback and support.

\printbibliography[heading=subbibliography,notkeyword=this]
\end{document} 
