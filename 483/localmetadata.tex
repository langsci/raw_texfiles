\title{Foundational approaches to Celtic linguistics}
% \subtitle{Add subtitle here if it exists}
\BackBody{This book showcases the latest research from the world’s leading experts on Celtic linguistics. The 15 chapters span a variety of linguistic subdisciplines as well as theoretical and methodological perspectives. Together, these articles highlight critical aspects of contemporary inquiry into the linguistic systems of Breton, Cornish, Irish, Manx, Scottish Gaelic, Welsh and their ancestor languages. The volume is organized around four key sub-areas: (1) syntax and semantics, (2) phonology and phonetics, (3) language Change, historical linguistics, and grammaticalization, and (4) sociolinguistics and language documentation. The volume's papers offer detailed investigations of current theoretical issues in Celtic syntax, semantics, phonology, and phonetics, as well as of language policy and ideology, language weaponization, and diachronic and synchronic language change. These state-of-the-art contributions represent the impressive diversity of the field of Celtic linguistics and emphasize the wide body of work being conducted in the language communities of the six Celtic nations.}

\author{Andrew Carnie and Diane Ohala and Dee Hunter and Samantha Prins and Michael Hammond and Luis Irizarry} 
 
\renewcommand{\lsISBNdigital}{978-3-96110-515-1}
\renewcommand{\lsISBNhardcover}{978-3-98554-144-7}
\BookDOI{10.5281/zenodo.15532992}
\typesetter{Andrew Carnie, Diane Ohala, Dee Hunter, Samantha Prins, Michael Hammond, Luis Irizarry}
\proofreader{David Carrasco Coquillat,
Dmitry Nikolaev,
Elen Le Foll,
Elisa Roma,
Elliott Pearl,
Eva Schultze Berndt,
Jean Nitzke,
Jeroen van de Weijer,
Katja Politt,
Lachlan Mackenzie,
Mary Ann Walter,
Matthew Korte,
Nicoletta Romeo,
Rainer Schulze,
Rebecca Madlener,
Sebastian Nordhoff,
Silvie Strauß
}
% \lsCoverTitleSizes{51.5pt}{17pt}% Font setting for the title page


\renewcommand{\lsSeries}{cicl} 
\renewcommand{\lsSeriesNumber}{1} 
\renewcommand{\lsID}{483}
