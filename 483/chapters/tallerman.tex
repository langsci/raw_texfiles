\documentclass[output=paper,colorlinks,citecolor=brown]{langscibook}
\ChapterDOI{10.5281/zenodo.15654865}
\author{Maggie Tallerman\orcid{}\affiliation{Newcastle University}}
\title{Some complexities and idiosyncrasies of Welsh consonantal mutation} 
\abstract{This chapter examines some of the peculiarities and idiosyncrasies of Welsh initial consonantal mutation and is especially targeted at those unfamiliar with the phenomena of mutation in general. I first provide an outline of the main contexts in which mutations occur in Welsh, which include lexical mutation triggers, morphosyntactic triggers and syntactic triggers. I go on to examine some of the complexities of initial mutation, including the fact that not all mutations have a trigger at all, and not all triggers are overtly realized. Mutations of \isi{loanwords}, both established loans and nonce loans, are also discussed and illustrated.}

\IfFileExists{../localcommands.tex}{
  % add all extra packages you need to load to this file

\usepackage{tabularx,multicol}
\usepackage{url}
\urlstyle{same}

\usepackage{listings}
\lstset{basicstyle=\ttfamily,tabsize=2,breaklines=true}

\usepackage{langsci-basic}
\usepackage{langsci-optional}
\usepackage{langsci-lgr}
\usepackage{langsci-osl}
% \usepackage{./langsci/styles/langsci-lgr}
% \usepackage{./langsci/styles/langsci-osl}
% \usepackage{langsci-gb4e}

\usepackage{tikz}
\usetikzlibrary{patterns,calc}
\pgfdeclarepatternformonly{south east lines}{\pgfqpoint{-0pt}{-0pt}}{\pgfqpoint{3pt}{3pt}}{\pgfqpoint{3pt}{3pt}}{
    \pgfsetlinewidth{0.6pt}
    \pgfpathmoveto{\pgfqpoint{0pt}{3pt}}
    \pgfpathlineto{\pgfqpoint{3pt}{0pt}}
    \pgfpathmoveto{\pgfqpoint{.2pt}{-.2pt}}
    \pgfpathlineto{\pgfqpoint{-.2pt}{.2pt}}
    \pgfpathmoveto{\pgfqpoint{3.2pt}{2.8pt}}
    \pgfpathlineto{\pgfqpoint{2.8pt}{3.2pt}}
    \pgfusepath{stroke}}
    
\usepackage{stmaryrd}
\usepackage{wasysym}
\usepackage{multirow}
\usepackage{caption}
\usepackage{subcaption}
\usepackage{mathrsfs}
\usepackage{qtree}

\usepackage{linguex}


  %pminos do not split footnotes
% \interfootnotelinepenalty=10000 %Footnote in Laporte chapters has to be split SN


%\DeclareIndexNameFormat{default}{%
%\nameparts{#1}%
%\usebibmacro{index:name}%
%{\index[names]}%
%{\namepartfamily}%
%{\namepartgiveni}%
% {}% L1
% {}% L2
%{\namepartprefix}% generates spurious space L3
%{\namepartsuffix}% generates spurious space L4
%}

%  {\DeclareIndexNameFormat{default}{%
%     \usebibmacro{index:name}{\index[names]}{#1}{#3}{#5}{#7}}}

%\DeclareIndexNameFormat{default}{%
%  \usebibmacro{index:name}{\sindex[nom]}{#1}{#3}{#5}{#7}}

%\DeclareIndexNameFormat{default}{%
%  \usebibmacro{index:name}{\sindex[person]}{#1}{#3}{#5}{#7}}
%\DeclareIndexNameFormat{default}{%
%\nameparts{#1} \usebibmacro{index:name}{\sindex[person]]}{\namepartfamily}{‌​\namepartgiven}{\nam‌​epartprefix}{\namepa‌​rtsuffix}}

%\newcommand{\smiley}{:)}

%\renewbibmacro*{index:name}[5]{%
%\usebibmacro{index:entry}{#1}%
%{\iffieldundef{usera}{}{\thefield{usera}\actualoperator}\mkbibindexname{#2}{#3}{#4}{#5}}}

% \newcommand{\noop}[1]{}

%remove for final
%\overfullrule=1mm

\newcommand{\tobi}[2]}}
\renewcommand{\S}[1]{\tobi{#1}{\textsc{*}}}

% this volume references
% puts: [this volume]
% already defined: \citetv
%\newcommand{\citepv}[1]{(\citeauthor{#1} \citeyear*{#1} [this volume])}
\newcommand{\citealtv}[1]{\citeauthor{#1} \citeyear*{#1} [this volume]}

%parentheses around example number
\newcommand{\pref}[1]{(\ref{#1})}

% in-text examples

\newcommand{\lnex}[1]{\textit{#1}} %target lang word
\newcommand{\lnlit}[1]{(lit.: `#1')} %literal reading
\newcommand{\lnlat}[1]{(#1)} % latinization
\newcommand{\lntrans}[1]{`#1'} %translation
\newcommand{\lnexl}[2]%
{\lnex{#1}{} \lnlat{#2}} % ex with latinization
\newcommand{\lnexlat}[3]{\lnex{#1}{} \lnlat{#2}{} \lntrans{#3}} % ex with latinization and tranl.

%ch01
\newcommand{\co}[1]{\mbox{\textbf{#1}}}

%ch09

\newcommand{\cyrbulg}[1]{\begin{otherlanguage*}{bulgarian}#1\end{otherlanguage*}}


%ch10
\newcommand{\nlp}{{\small NLP}}
\newcommand{\mwe}{{\small MWE}}
\newcommand{\rae}{{\small RAE}}
\newcommand{\lvc}{{\small LVC}}
\newcommand{\pos}{{\small P}o{\small S}}
%\newcommand{\todo}[1]{ \textcolor{red}{#1} }

%\renewcommand{\labelenumi}{\theenumi}
%\ainamefmt{{vv}{ll}{, ff}{, jj}} % fullname

\newcommand{\biberror}[1]{{\color{red}#1}}

\newcommand{\osenovaitem}{--~} 
  %% hyphenation points for line breaks
%% Normally, automatic hyphenation in LaTeX is very good
%% If a word is mis-hyphenated, add it to this file
%%
%% add information to TeX file before \begin{document} with:
%% %% hyphenation points for line breaks
%% Normally, automatic hyphenation in LaTeX is very good
%% If a word is mis-hyphenated, add it to this file
%%
%% add information to TeX file before \begin{document} with:
%% %% hyphenation points for line breaks
%% Normally, automatic hyphenation in LaTeX is very good
%% If a word is mis-hyphenated, add it to this file
%%
%% add information to TeX file before \begin{document} with:
%% \include{localhyphenation}
\hyphenation{
    Beck-man
    Ngu-yen
    back-chan-nel
    back-chan-nels
    mo-not-o-nous
    ste-reo-typ-i-cal
}

\hyphenation{
    Beck-man
    Ngu-yen
    back-chan-nel
    back-chan-nels
    mo-not-o-nous
    ste-reo-typ-i-cal
}

\hyphenation{
    Beck-man
    Ngu-yen
    back-chan-nel
    back-chan-nels
    mo-not-o-nous
    ste-reo-typ-i-cal
}
 
   \boolfalse{bookcompile}
  \togglepaper[5]%%chapternumber
}{}

\AffiliationsWithoutIndexing

\begin{document}
\maketitle

\il{Welsh (Modern)|(}
\is{Consonant Mutation|see {Initial Consonant Mutation}}
\is{Initial Consonant Mutation|(}



\section{Introduction}\label{sec:tallerman:1}

This chapter examines some of the complexities and idiosyncrasies found in Welsh initial consonantal mutation. A central goal is to showcase the specific features of Welsh mutation in the spoken language, including more recent developments in colloquial Welsh and in the mutation of loanwords\is{loanwords}, both established loans and \isi{nonce loans}. We see that not all mutation in Welsh is triggered by a preceding lexical item: some mutations are syntactically or morphosyntactically triggered; some are not triggered at all; some mutation triggers are not overtly realized; and some lexical items fail to undergo mutation. We examine variation in mutation, including idiolectal and dialectal\is{dialectal variation} variation. More formal varieties of Welsh will be referred to when relevant.

The term “consonantal mutation” refers to sets of morphophonological alternations in the initial segments of words or morphemes. In modern Welsh, these alternations are conditioned by factors other than a purely phonological context.  Mutations are realized phonologically, but not triggered phonologically. I use the term “trigger” to refer to the lexical item, phrase or (morpho)syntactic context that conditions the mutation, and “target” to refer to the element that undergoes the mutation. In Welsh, mutation is strictly local and only immediately preceding items are possible triggers. As we will see, not all mutation is triggered, and various factors determine whether a predicted mutation occurs in each context. \sectref{sec:tallerman:2} briefly outlines the basic facts of Welsh and illustrates some of the main triggering contexts for mutation; it is not intended to be a comprehensive survey of the mutation system. \sectref{sec:tallerman:3} examines some further complexities of Welsh mutation, showing that not all mutation triggers are overt, and that not all mutations have a trigger. \sectref{sec:tallerman:4} turns to some idiosyncrasies of mutation, showing speaker variation and the effects of mutation in loanwords\is{loanwords}. \sectref{sec:tallerman:5} examines a mutation paradox, and proposes a solution based on the notion of a non-overt mutation trigger. 

\section{Overview of Welsh mutations, their triggering contexts and targets}\label{sec:tallerman:2}
\subsection{The basic facts of mutation}

Welsh has three series of mutations, traditionally referred to as “soft”, “nasal” and “aspirate”, with the term “radical” referring to the basic lexical form of the word, which is the citation form. In this paper, I focus on soft mutation (SM), by far the most frequent mutation~– both in terms of triggering contexts and occurrence in all kinds of data, but also by far the most interesting and diverse of the different types. 

\begin{table}
\centering
\caption{Welsh consonantal mutations: Orthographic [phonetic]}
\label{tab:tallerman1}
\begin{tabular}{*8{l}}
\lsptoprule
\multicolumn{2}{c}{{Radical}} & \multicolumn{2}{c}{{Soft}} & \multicolumn{2}{c}{{Nasal}} & \multicolumn{2}{c}{{Aspirate}}\\
\cmidrule(lr){1-2}\cmidrule(lr){3-4}\cmidrule(lr){5-6}\cmidrule(lr){7-8}
p  & [p]  & b  & [b]  & mh  & [mh] & ph & [f]\\
t  & [t]  & d  & [d]  & nh  & [nh] & th & [θ]\\
c  & [k]  & g  & [g]  & ngh & [ŋh] & ch & [x]\\
b  & [b]  & f  & [v]  & m   & [m]  &    & \\
d  & [d]  & dd & [ð]  & n   & [n]  &    & \\
g  & [g]  & –  & zero & ng  & [ŋ]  &    & \\
m  & [m]  & f  & [v]  &     &      &    & \\
ll & [ɬ]  & l  & [l]  &     &      &    & \\
rh & [rʰ] & r  & [r]  &     &      &    & \\
\lspbottomrule
\end{tabular}
\end{table}

As \tabref{tab:tallerman1} shows, only a word with an initial voiceless stop can undergo each of the three possible mutations; nasal mutation applies to all the six stops, but aspirate mutation only applies to the voiceless stops. I follow \citet{Hammond2019} in assuming that phonologically the voiceless nasals are sequences of a nasal followed by [h]: [mh, nh, ŋh]. My examples here are generally given in Welsh orthography, which, as is clear from \tabref{tab:tallerman1}, represents phonetic values somewhat differently than in English. \REF{ex:mt:tallerman1} illustrates, with the superscript indicating the item that is the trigger for the specific mutation, and denoting which of the three mutation series (soft, nasal or aspirate) is triggered. Here, the triggers are agreement proclitics that occur in appropriate agreement contexts before nouns and non-finite verbs; these small functional elements are very typical of the types of trigger for Welsh mutations:

\ea\label{ex:mt:tallerman1}
Radical: \textit{tad} ‘father’\\
Nasal mutation: \textit{fy/yn\textsuperscript{N}} \textit{nhad} (\textsc{1s} father) ‘my father’\\
Soft mutation: \textit{dy\textsuperscript{S}} \textit{dad} (\textsc{2s} father) ‘your father’\\
Soft mutation: \textit{ei\textsuperscript{S}} \textit{dad} (\textsc{3ms} father) ‘his father’\\
Aspirate mutation: \textit{ei\textsuperscript{A}} \textit{thad} (\textsc{3fs} father) ‘her father’\\
Radical form: \textit{ein tad} (1\textsc{pl} father) ‘our father’
\z

\noindent {As these examples show, all the singular proclitics trigger one of the three different mutation series. None of the plural proclitics trigger any mutation, so are followed by the radical form of the following item.}


\subsection{Simple mutation triggers}


Most contexts for triggered mutation in Welsh fall into one of two categories: first, mutations which are simply triggered by a preceding item with no restrictions, and second, mutations triggered in specific morphosyntactic or syntactic contexts, in which either trigger or target may exhibit various restrictions. In this section, I illustrate the first category, where the mutation is triggered by an immediately preceding, c-commanding lexical item. Generally, these triggers are functional or semi-functional elements such as those in (\ref{ex:mt:tallerman1}), including various agreement proclitics, prepositions, numerals, conjunctions, determiners, complementizers and particles. Example (\ref{ex:mt:tallerman2}) illustrates some typical examples (triggers in bold; radical form shown in parentheses), all showing soft mutation triggered by a specific preceding element. The relevant c-command domain is shown in square brackets: 

\ea\label{ex:mt:tallerman2}
\ea \label{ex:mt:tallerman2a}
\gll \textbf{gan} [flodyn] (\textit{blodyn})\\
with flower\\
\glt ‘with a flower’
\ex\label{ex:mt:tallerman2b}
\gll \textbf{gan} [bedwar blodyn tlws] (\textit{pedwar})\\
with four flower pretty\\
\glt ‘with four pretty flowers’
\ex\label{ex:mt:tallerman2c}
\gll \textbf{dau} [fis] (\textit{mis})\\
two month\\
\glt ‘two months’
\ex\label{ex:mt:tallerman2d}
\gll \textbf{pa} [ddiwrnod] (\textit{diwrnod})\\
which day\\
\glt ‘which day’
\ex\label{ex:mt:tallerman2e}
\gll  te \textbf{neu} [goffi] (\textit{coffi})\\
tea or coffee\\
\glt ‘tea or coffee’
\ex\label{ex:mt:tallerman2f}
\gll  \textbf{Mi} [brynais i docyn] (\textit{prynais})\\
\textsc{prt} buy.\textsc{past.1s} \textsc{I} ticket\\
\glt ‘I bought a ticket.’
\ex\label{ex:mt:tallerman2g}
\gll \textbf{yn} [ofalus] (\textit{gofalus})\\
\textsc{pred} careful\\
\glt ‘carefully’
\z
\z
    
\noindent {The contrast between (\ref{ex:mt:tallerman2a}) and (\ref{ex:mt:tallerman2b}) shows that mutation always targets the initial consonant of the first item in the following phrase. So, in both cases the mutation is realized on the initial consonant of the NP: in (\ref{ex:mt:tallerman2a}) the NP consists only of its head N, while in (\ref{ex:mt:tallerman2b}) the NP is \textit{pedwar blodyn tlws}  ‘four pretty flowers’.}

Mutation does not target the head of the following phrase, which may or may not be the initial element. This illustrates the strict locality of all mutation in Welsh. So, if the initial element of a target phrase is a vowel or does not have a mutable initial consonant – which happens often, since there are only nine mutable consonants and the total consonant inventory is larger – then the phrase gives no indication at all of the triggered mutation. For instance (cf., \ref{ex:mt:tallerman2b}), in \textit{gan saith blodyn} ‘with seven flowers’, we have a soft mutation trigger, the preposition \textit{gan}, which has as its target the phrase \textit{saith blodyn}. But the numeral \textit{saith} ‘seven’ does not have a mutable initial, so there is nothing to indicate here that \textit{gan} is a trigger for soft mutation. The mutation crucially does not skip over the numeral to target the head noun: \textit{*gan saith flodyn}. If, on the other hand, the phrase is \textit{gan dri chi}, ‘with three dogs’, then \textit{gan} does c-command a following phrase with a mutable initial, [\textit{tri chi}], and triggers soft mutation as predicted: \textit{tri > dri}. 

\subsection{Morphosyntax and syntax}

In the second major category for triggered mutation, either target or trigger display some specific morphosyntactic or syntactic restrictions. 


\subsubsection{Morphosyntactic restrictions}

First, we examine morphosyntactic restrictions on the target. The definite article in Welsh triggers soft mutation, but only feminine singular nouns are a target for this mutation (\ref{ex:mt:tallerman3a}), and never plural nouns (\ref{ex:mt:tallerman3b}), or masculine nouns (\ref{ex:mt:tallerman3c}, \ref{ex:mt:tallerman3d}). (Welsh has two genders, traditionally termed “masculine” and “feminine”):

\ea\label{ex:mt:tallerman3}
\ea\label{ex:mt:tallerman3a}
\gll \textbf{y} gath    (\textit{cath})\\
the cat.\textsc{f}\\
\glt ‘the cat’
\ex\label{ex:mt:tallerman3b}
\gll y cathod\\
the cat.\textsc{pl}\\
\glt ‘the cats’
\ex\label{ex:mt:tallerman3c}
\gll y carlwm\\
the stoat.\textsc{m}\\
\glt ‘the stoat’   
\ex\label{ex:mt:tallerman3d}
\gll y carlymod\\
the stoat.\textsc{m.pl}\\
\glt ‘the stoats’
\z
\z

A second instance of a morphosyntactic restriction is shown in (\ref{ex:mt:tallerman4}), this time involving a restriction on the trigger. Feminine singular nouns trigger soft mutation on each following adjective (\ref{ex:mt:tallerman:4a}), but masculine and plural nouns do not trigger mutation on following adjectives  (\ref{ex:mt:tallerman:4b}):

\ea\label{ex:mt:tallerman4}
\ea\label{ex:mt:tallerman:4a}
\gll y gath fawr ddu    (\textit{mawr, du})\\
the cat.\textsc{f} big black\\
\glt ‘the big black cat’ 
\ex\label{ex:mt:tallerman:4b}
\gll y cathod mawr\\
the cats big\\
\glt ‘the big cats’
\z
\z

The mutation triggered by feminine singular nouns also applies if one of a string of adjectives has a non-mutable initial consonant, as in \textit{y gath ifanc ddu} ‘the young black cat’, where the adjective \textit{ifanc} ‘young’ does not have a mutable initial but the following adjective does, so bears soft mutation (\textit{du} > \textit{ddu}). Various approaches to handling adjective mutation are discussed in \citet{BorsleyEtAl2007}. As with all other mutations in Welsh, the mutation is strictly local, and the trigger must immediately precede and c-command the target. A plausible structure for the phrase in (\ref{ex:mt:tallerman:5}) is shown in \figref{ex:mt:tallerman:6}:

\ea\label{ex:mt:tallerman:5}
\gll gardd fawr, breifat          (\textit{mawr}{,} \textit{preifat})\\
garden.{\textsc{f}} big private\\
\glt ‘a large, private garden’  
\z

\noindent In \figref{ex:mt:tallerman:6}, the assumption is that a feminine singular NP triggers soft mutation on its sister, and the adjectives are each adjoined to form a new NP which in turn is a trigger for the mutation. 

\begin{figure}[!ht]
\noindent\begin{forest}
[NP$_1$ [NP$_2$ [NP$_3$ [gardd\\{[+f,sg]}]] [AP$_1$ [fawr]] ] [AP$_2$ [breifat]] ]
\end{forest}
\caption{Tree for (\ref{ex:mt:tallerman:5})}
\label{ex:mt:tallerman:6}
\end{figure}

\noindent {So, the trigger for \textit{mawr} > \textit{fawr} is the feminine singular NP$_3$ \textit{gardd}, and the trigger for \textit{preifat} > \textit{breifat} is the feminine singular NP$_2$ \textit{gardd fawr}, which has inherited its number and gender features from \textit{gardd}.}

Note that the mutation in examples like (\ref{ex:mt:tallerman:5}) is not plausibly treated as morphological agreement of an adjective with a feminine singular head noun. In (\ref{ex:mt:tallerman:7}), the AP undergoes soft mutation as expected following the feminine singular noun, with the mutation realized as usual on its initial segment, giving \textit{tra} > \textit{dra}. But the head adjective, \textit{campus} ‘splendid’, does not agree with the feminine singular head N to give soft mutation; instead, it undergoes aspirate mutation (AM), triggered by the modifier \textit{tra}:

\ea\label{ex:mt:tallerman:7}
\gll gardd [\textsubscript{AP} dra\textsuperscript{A} champus] (\textit{tra, campus})\\
garden.\textsc{f} {} quite splendid(+\textsc{am})\\
\glt ‘a quite splendid garden’ 
\z

\noindent{Example (\ref{ex:mt:tallerman:7}) also illustrates nicely the strict locality of mutation in Welsh: as indicated, mutation is always triggered by an immediately preceding c-commanding item, and the most local mutation trigger always takes precedence.}

\subsubsection{Marked word orders}

Adjectives in Welsh are generally post-nominal, but some adjectives occur, or may occur, pre-nominally. The following element, generally the head noun, then undergoes soft mutation irrespective of number and gender:

\ea\label{ex:mt:tallerman8}
\ea\label{ex:mt:tallerman8:a}
\gll hen gathod    (\textit{cathod})\\
old cat.\textsc{pl}\\
\glt ‘old cats’
\ex\label{ex:mt:tallerman8:b}
\gll hen garlymod  (\textit{carlymod})\\
old stoat.\textsc{pl}\\
\glt ‘old stoats’
\z
\z

\noindent{Again, the triggering adjective immediately precedes and, under reasonable assumptions, c-commands the target (see also \citealt{Willis2006}). \citet{Tallerman1999} treats this as one instance of a more general triggering environment for soft mutation in which a marked word order is signaled by mutation occurring on the head: in Welsh, head-initial word order is unmarked, while head-final order is marked}\footnote{{There is no support in Welsh for the proposal that NPs should be analysed as Determiner Phrases (see, for instance, \citet{BorsleyEtAl2007}: ch. 3.1; 5.6). The head of phrases consisting of D+N and Q+N will therefore be assumed to be N in what follows.}}{:} 

\ea\label{ex:mt:tallerman9}
The markedness hypothesis:\\
In any phrase XP with an unmarked word order [X{\textsuperscript{0}}{ α], X}{\textsuperscript{0}}{ bears soft mutation if it is} {\textit{preceded} }{by α.}\\
\z

While Adj-N marked word order is relatively easy to find, we also see the effects of this mutation trigger in far less common environments, especially in Literary Welsh. For instance, numerals in Welsh are standardly pre-nominal, as in (\ref{ex:mt:tallerman10:a}), but in the marked N-Num word order in (\ref{ex:mt:tallerman10:b}), soft mutation is triggered on the numeral:

\ea\label{ex:mt:tallerman10}
\ea\label{ex:mt:tallerman10:a}
\gll tri dyn \\
three man\\ \jambox{(Num-N unmarked)}
\glt ‘three men’
\ex\label{ex:mt:tallerman10:b}
\gll dynion dri    (\textit{tri}) \\
men three\\ \jambox{(N-Num marked, SM on Num)}
\glt‘three men’
\z
\z

Similarly, quantifiers are standardly pre-nominal, as in (\ref{ex:mt:tallerman11a}); unlike adjectives, these do not trigger soft mutation on the following noun, so for instance \textit{llawer} ‘many’, \textit{pob} ‘every’ and \textit{peth} ‘a bit of’ are all followed by the radical initial consonant. However, in the marked word order, (\ref{ex:mt:tallerman11b}), the post-nominal quantifier undergoes soft mutation:

\ea\label{ex:mt:tallerman11}
\ea\label{ex:mt:tallerman11a}
\gll llawer dyn\\
many man\\ \jambox{(Q-N unmarked)}
\glt ‘many men’
\ex\label{ex:mt:tallerman11b}
\gll dynion lawer     (\textit{llawer}) \\
men many\\ \jambox{(N-Q marked, SM on Q)}
\glt ‘many a man’
\z
\z

\noindent The assumption here is that the noun is the head of its phrase, rather than the numeral or the quantifier. 

However, it is not so clear whether the markedness hypothesis handles the situation in (\ref{ex:mt:tallerman:12}), which has a string of pre-nominal adjectives. Here we see that the head noun undergoes soft mutation, as predicted by the hypothesis, but each adjective following the first in the series is also a target for soft mutation, triggered by the preceding adjective:

\ea\label{ex:mt:tallerman:12}
\gll yr hen brif breswylfa    (\textit{prif,} {\textit{preswylfa}})\\
the old main residence\\
\glt ‘the former main residence’
\z

\noindent{In this example, the order of the adjectives (}{\textit{hen} }{>} {\textit{prif}}{) gives rise to the meaning ‘former’ for} {\textit{hen}}{, whereas if the meaning is the standard ‘old’, the order is reversed:}

\ea\label{ex:mt:tallerman:13}
\gll y prif hen gymeriadau    (\textit{cymeriadau})\\
the main old character.\textsc{pl}\\
\glt `the main old characters'
\z

\noindent{A reasonable assumption is that (\ref{ex:mt:tallerman:12}) does in fact have a marked word order, since} {\textit{hen}}{ has the atypical ‘former’ meaning here, and therefore appears in an atypical position compared to examples like (\ref{ex:mt:tallerman:13}). Note also that equative, comparative and superlative adjectives, which at least in standard Welsh have an unmarked pre-nominal position, do not generally trigger soft mutation.} 

{However, the mutation of the adjective} {\textit{prif > brif} }(in \ref{ex:mt:tallerman:12}) remains unaccounted for by the formulation of the markedness hypothesis in (\ref{ex:mt:tallerman9}), since this only applies to heads and their modifiers, and not to the case in (\ref{ex:mt:tallerman:12}), where a modifier triggers soft mutation on another modifier. 


\subsubsection{Syntactic soft mutation}


Finally in this section I briefly outline what we can call syntactic soft mutation~– mutation which is triggered in a specific syntactic context in Welsh, unique amongst the Celtic languages. We will see that phrases (XPs) are triggers for soft mutation and trigger the mutation on a following complement. The “XP trigger hypothesis” was proposed by \citet{BorsleyTallerman1996} and further defended by \citet{Borsley1999}, \citet{Tallerman2006}, \citet{Tallerman2009}, and \citet{BorsleyEtAl2007}:

\ea\label{ex:mt:tallerman:14}
XP trigger hypothesis \citep{Borsley1999}:\\
A complement bears soft mutation if it is immediately preceded by a  phrasal sister. 
\z

\noindent Thus, the trigger for soft mutation both c-commands and immediately precedes the target. The formulation of the XP trigger hypothesis in (\ref{ex:mt:tallerman:14}) accounts for numerous contexts for soft mutation in Welsh which have traditionally been regarded as distinct environments. In (\ref{ex:mt:tallerman:15a}) we see the context sometimes known as “direct object mutation”, illustrating the mutation of the object of a finite verb. Here, the XP trigger is the subject NP (in bold), \textit{y ddynes,} and the direct object which is the target for the mutation is a complement of the finite verb: \textit{beic > feic}: 

\ea\label{ex:mt:tallerman:15}
\ea\label{ex:mt:tallerman:15a}  
\gll Prynodd \textbf{y} \textbf{ddynes} feic. (\textit{beic})\\
buy.\textsc{past.3s} the woman bike\\
\glt ‘The woman bought a bike.’
\ex\label{ex:mt:tallerman:15b}
\gll Beic / *feic brynodd y ddynes \_\_\_\_\_.   \\
bike / bike(\textsc{+sm}) buy.{\textsc{past.3s} } the woman \\
\glt ‘The woman bought {\textit{a bike}}{.’}
\z
\z
The XP trigger hypothesis correctly predicts the mutation in (\ref{ex:mt:tallerman:15a}) under the assumptions of a surface-orientated phrase structure in which the subject is the sister of the direct object; see for instance \citet{Borsley1999}.  However, in (\ref{ex:mt:tallerman:15b}) where the direct object is fronted, it is nonetheless a complement of the verb but does not follow an XP trigger and therefore does not mutate; nor does a sentence fragment object mutate, for the same reason. Such data show that the idea that ‘direct object mutation’ reflects case-marking, as proposed by \citet{Roberts2005}, is untenable. There is also no soft mutation on the direct object of a non-finite verb, which is again predictable under the XP trigger hypothesis since there is no preceding XP trigger: 

\ea\label{ex:mt:tallerman:16}
\gll  Mae ’r ddynes wedi prynu beic /*feic.\\
be.\textsc{pres.3s} the woman \textsc{perf} buy.\textsc{inf} bike /bike(\textsc{+sm}) \\
\glt ‘The woman has bought a bike.’
\z

{There is, though, soft mutation on the VP in (\ref{ex:mt:tallerman:17}), realized as usual on the initial element of the phrase, in this case the non-finite verb:} {\textit{gwerthu > werthu}}{. Here, the trigger is again the subject NP. The XP trigger hypothesis predicts this mutation on the reasonable assumption that the non-finite VP is a complement of the finite auxiliary} {\textit{gwnaeth}}{ ‘did’:}

\ea\label{ex:mt:tallerman:17}
\gll Gwnaeth \textbf{y} \textbf{ddynes} [\textsubscript{VP} werthu beic]. (\textit{gwerthu})\\
do.\textsc{past.3s} the woman {} sell.\textsc{inf} bike\\
\glt ‘The woman sold a bike.’
\z

\noindent{Note that again, there is no soft mutation on the direct object (}{\textit{beic}}{) in (\ref{ex:mt:tallerman:17}), since no XP (or other trigger) precedes it. Further arguments against the “mutation as case” hypothesis are outlined in \citet{Tallerman2006}.} 

The XP trigger hypothesis correctly predicts the occurrence of soft mutation in various other contexts. Here, I illustrate just two. The first is traditionally known as ‘soft mutation following a parenthesis’; compare (\ref{ex:mt:tallerman:18a}) and (\ref{ex:mt:tallerman:18b}): 

\ea
\ea \label{ex:mt:tallerman:18a}
\gll Mae ci mawr yn yr ardd.\\
be.\textsc{pres.3s} dog big in the garden \\
\glt ‘There’s a big dog in the garden.’
\ex \label{ex:mt:tallerman:18b}
\gll Mae \textbf{[\textsubscript{PP}} \textbf{yn} \textbf{yr} \textbf{ardd]} gi mawr. (\textit{ci}) \\
be.\textsc{pres.3s} {} in the garden dog big \\
\glt ‘There’s a big dog in the garden.’
\z
\z

The usual constituent order is shown in (\ref{ex:mt:tallerman:18a}), while (\ref{ex:mt:tallerman:18b}) illustrates a less common word order, with the PP occurring in a focalized position before the subject. In (\ref{ex:mt:tallerman:18a}) there is no XP preceding the subject {\textit{ci mawr}}{ ‘a big dog’, while in (\ref{ex:mt:tallerman:18b}), the PP precedes the subject, triggering soft mutation, which is realized as usual on the initial element of the target:} {\textit{ci > gi}}{. On the assumption that subjects are also complements of the finite verb or auxiliary \citep{Borsley1999}, the PP is correctly predicted by the XP trigger hypothesis to be the phrasal trigger for the mutation here. Such ‘parentheses’ can be PPs or some other intervening constituent, such as an Adverb Phrase, and these occur in various syntactic contexts. For instance, in (\ref{ex:mt:tallerman:19}), where the non-finite VP} {\textit{dysgu’r gelfyddyd} }{‘to learn the art’ is a complement of the verb} {\textit{ffaelio}}{ ‘fail’, an intervening adverbial occurs before the VP, triggering soft mutation on its initial element, just as predicted by the XP trigger hypothesis \citep{Tallerman2006}:} 

\ea\label{ex:mt:tallerman:19}
\gll  ... yn ffaelio [\textsubscript{AdvP} \textbf{’n} \textbf{glir lân]} [\textsubscript{VP} ddysgu ’r gelfyddyd] \\
{} \textsc{prog} fail.\textsc{inf} {} \textsc{pred} completely {} learn.\textsc{inf} the art\\
\glt ‘completely failing to learn the art’ (adapted from \citealt{Morgan1952}: 432) 
\z

\noindent {Without the Adverb Phrase trigger, the VP would not bear soft mutation.}


A further illustration of the power of the XP trigger hypothesis is given in (\ref{ex:mt:tallerman:20}):

\ea\label{ex:mt:tallerman:20}
\gll Dw \textbf{i} [\textsubscript{AP} lawn mor grac â chi]. (\textit{llawn})\\
be.\textsc{pres.1s}  I {} full as angry as you\\
\glt ‘I’m just as angry as you.’
\z

\noindent{Here, an AP predicate complement bears soft mutation (}{\textit{llawn} }{>} {\textit{lawn}}{), triggered by the subject NP} {\textit{i}}{. Other non-nominal constituents, such as PPs, can also occur as predicate complements, and as predicted these also bear soft mutation \citep{Tallerman2006}.} 

Finally, note that it is crucial in (\ref{ex:mt:tallerman:14}) that the XP trigger hypothesis refers to complements, and not just any constituent, as the target for soft mutation triggered by a preceding XP. In (\ref{ex:mt:tallerman:21}), we have a direct object which consists of a set of conjoined NPs:


\ea\label{ex:mt:tallerman:21}
\gll Prynon \textbf{nhw} [fara, menyn, a chaws].   (\textit{bara})\\
buy.\textsc{past.3p} they bread butter and cheese\\
\glt  ‘They bought bread, butter and cheese.’
\z

\noindent{Here, it is the whole direct object} \textit{bara, menyn, a chaws} that forms the complement of the verb. This follows a trigger for soft mutation, the subject {\textit{nhw}}{, and the direct object therefore bears soft mutation on its initial element as usual:} {\textit{bara > fara}}{. Crucially, each individual conjunct is not a complement, so no mutation is triggered, for instance, by the first conjunct onto the second:}{ \textit{*fenyn.} }{And the final conjunct takes aspirate mutation, triggered by the immediately preceding trigger} {\textit{a}}{ ‘and’. As usual, if the initial element of the first conjunct happened not to have a mutable initial consonant, then the direct object would show no other signs of soft mutation.} 

Thus, we see that the XP trigger hypothesis correctly predicts the occurrence of soft mutation in a wide variety of syntactic contexts, more of which are discussed and illustrated in the references shown. We have also seen that targets for syntactic soft mutation include objects, subjects, verb phrases headed by non-finite verbs, and predicates in copular clauses. 


\section{Further complexities of Welsh mutation}\label{sec:tallerman:3}

\subsection{Not all mutations have a trigger}


Though most mutations are triggered, some are not. The most common context involves adverbials of time, which at least in standard Welsh bear soft mutation in any sentence position, realized as usual on the initial element wherever this has a mutable initial:

\ea\label{ex:mt:tallerman:22}
\gll Ddwy flynedd {yn ôl}, diflannodd hi.     (\textit{dwy})\\
two year ago disappear.\textsc{past.3s} she\\
\glt ‘Two years ago, she disappeared.’
\z

\ea\label{ex:mt:tallerman:23}
\gll Fis {yn ôl} dechreuodd y broblem. \textit{(mis)}\\
month ago begin.\textsc{past.3s} the problem\\
\glt ‘A month ago the problem began.’
\z

{Probably the most frequently occurring example is} {\textit{ddoe,} }{‘yesterday’, which is never} {\textit{*doe.}}\footnote{{There is generally no soft mutation on} {\textit{bore} }{‘morning’ or} {\textit{prynhawn} }{‘evening’, whatever the position. Idiosyncratic exceptions to mutation in Welsh are not at all uncommon.}}{ }

Other adjuncts also often bear soft mutation optionally, such as the measure adverbial here:

\ea\label{ex:mt:tallerman:24}
\gll ’Dan ni ’n byw dair milltir o ’r dre’. \textit{(tair)}\\
be.\textsc{pres.1p} we \textsc{prog} live.\textsc{inf} three mile from the town\\
\glt ‘We live three miles from town.’
\z

Since adjuncts are not complements, such examples do not fall under the XP trigger hypothesis. The mutation of adjuncts is subject to dialectal\is{dialectal variation} and idiolectal variation, but where it does occur, it consistently lacks any trigger. 

{Various other items always appear in their soft mutation form, such as the negative} {\textit{fawr} }{(<} {\textit{mawr}}{, ‘big’), which means ‘not much’:} 

\ea\label{ex:mt:tallerman:25}
\gll Does fawr yn newid. \\
\textsc{neg.}be.\textsc{pres.3s} not.much \textsc{prog} change.\textsc{inf} \\
\glt ‘Not much changes.’
\z

{\textit{Fawr} }{also occurs in expressions like} {\textit{fawr o neb}}{ ‘not many people’ (}{\textit{neb} }{‘no one’), and} {\textit{fawr o ddim}}{ ‘not much’ (literally, ‘not much of nothing’):}

\ea\label{ex:mt:tallerman:26}
\gll Be’ wnest ti? Fawr o ddim. \\
what do.\textsc{past.2s} you not.much of nothing\\
\glt ‘What did you do? Not much.’
\z

{Note that the occurrence of soft mutation in (\ref{ex:mt:tallerman:26}) contrasts with the situation seen in sentence fragment objects, which do not bear soft mutation, since they do not have an appropriate XP trigger. It is perhaps the case that the negative} {\textit{fawr} }{is more properly regarded as a lexicalized form, as its mutation is not triggered. Other lexicalized soft mutation forms occur, such as} {\textit{gartre(f)}}{ ‘at home, homewards’ (cf.} {\textit{cartref} }{‘home’) and} {\textit{bellach} }{‘any more’ (cf.} {\textit{pell} }{‘far’); these have fixed soft mutation and only occur in that form (\citealt[257ff]{Morgan1952}).} 


\subsection{Not all mutation triggers are overtly realized}

We showed above that the singular proclitics which appear on nouns and non-finite verbs in agreement contexts are triggers for various mutations; see (\ref{ex:mt:tallerman1}). However, in the colloquial language the proclitics are often omitted. In this case, the mutation that the proclitic would have triggered may remain, as these examples show (NM is nasal mutation). The relevant proclitics are shown in parentheses:

\ea\label{ex:mt:tallerman:27}
\gll Dw i wedi (\textbf{ei}) weld    o {o’r blaen}.  (\textit{gweld})\\
be.\textsc{pres.1s} I \textsc{perf} \textsc{3ms} see.{\textsc{inf(+sm)}} him before\\
\glt ‘I’ve seen him before.’
\z

\ea\label{ex:mt:tallerman:28}
\gll Hwyrach dy fod ti wedi (\textbf{yn/fy}) ngweld i.   (\textit{gweld})\\
perhaps \textsc{2s} be.\textsc{inf} you \textsc{perf} \textsc{1s} see.\textsc{inf(+nm)} me\\
\glt  ‘Perhaps you’ve seen me.’
\z

{In the most colloquial register (cf. \citealt{BorsleyEtAl2007}: Sections 4.1.5; 4.2.2), both the proclitic and the mutation are absent, but the forms shown above are also acceptable in spoken Welsh. Crucially, even if the proclitic that would be a mutation trigger is phonetically unrealized, it continues to block any mutation from an immediately preceding overt trigger. For instance, (\ref{ex:mt:tallerman:29}) contains the aspect marker} {\textit{heb} }{(lit. ‘without’)}{,}{} {a trigger for  soft mutation:} 

\ea\label{ex:mt:tallerman:29}
\gll Dw i heb (ei) gweld/*weld hi ers misoedd\\
be.\textsc{pres.1s} I without (\textsc{3fs}) see.\textsc{inf}/*see(+\textsc{sm}) her since months\\
\glt ‘I haven’t seen her in months.’
\z

\noindent {Here, more formal Welsh would have an overt agreement proclitic,} {\textit{ei}}{ (}{\textsc{3fs}}{), which triggers aspirate mutation. Since} {\textit{gweld}}{ ‘see’ does not have a voiceless initial stop it cannot take this mutation, so a formal variant of (\ref{ex:mt:tallerman:29}) would have} {\textit{ei gweld hi.} }{If} {\textit{ei}}{ is unrealized,} {\textit{gweld} }{appears to immediately follow the trigger for  soft mutation,} {\textit{heb}}{. But the unrealized proclitic blocks the  soft mutation, as (\ref{ex:mt:tallerman:29}) shows: *}{\textit{weld}}{; even if phonetically unrealized, the proclitic continues to occupy a syntactic position as the closest mutation trigger.} 

{In fact, numerous  small functional elements are phonetically unrealized in the spoken language, yet the mutations they trigger can remain. The clause-initial complementizers (often termed particles) are typical examples. For instance, affirmative root clauses can have an optional complementizer} {\textit{mi}}{ or} {\textit{fe}}{, both triggers for  soft mutation, and the mutations may remain even when the triggers are not overtly present. Similarly, many place names start, at least historically, with the definite article} {\textit{y(r),}}{ which is a trigger for  soft mutation on a following feminine singular noun, as in (\ref{ex:mt:tallerman3}) above. The article is often unrealized in modern contexts, but the mutation remains on the initial element of feminine singular placenames:} {\textit{Drenewydd < Y Drenewydd} }{(}{\textit{tre’ > dre’} }{‘town’}{),}{ ‘Newtown’;} {\textit{Gaerwen < Y Gaerwen} }{(}{\textit{caer > gaer ‘}}{fort’).} 

{Finally, I turn to non-overt mutation triggers under the XP trigger hypothesis. Within the generative tradition, the two ‘Case-marked’ empty category NPs} {\textit{pro}}{ and} {\textit{wh}}{{}-trace are recognized as placeholders for the overt phrases that they replace. As predicted by the XP trigger hypothesis, these \isi{empty categories} continue to trigger mutation, and indeed to block mutation, just as if they were overt. The first example shows the empty category} {\textit{pro,} }{the null subject of a finite clause which occurs in more formal varieties of Welsh:}

\ea\label{ex:mt:tallerman:30}
\gll Prynodd \textit{pro} delyn.       (\textit{telyn}) \\
buy.\textsc{past.3s} {} harp \\
\glt ‘He/she bought a harp.’
\z

The direct object, \textit{telyn}{, bears  soft mutation under the XP trigger hypothesis just as it would with an overt subject pronoun or full NP trigger. And in (\ref{ex:mt:tallerman:31}) we see that} {\textit{wh}}{{}-trace in the subject position of a finite clause also triggers  soft mutation under the XP trigger hypothesis:} 

\ea\label{ex:mt:tallerman:31}
\gll y ddynes brynodd \textit{wh-t} delyn    ({\textit{telyn}})\\
the woman buy.\textsc{past.3s} {} harp\\
\glt ‘the woman who bought a harp’
\z

See \citet{Tallerman2006} and \citet{Tallerman2009} for a full account of the role played by \isi{empty categories} in mutation; further discussion is found in \citet{BorsleyEtAl2007}: ch. 7. 


\subsection{Not all lexical items bear mutation}

{It is worth briefly mentioning that not all words undergo mutation, even if they are found in a triggering environment. This set of non-mutating words is relatively small, and generally consists of functional or semi-functional elements: some never mutate, such as} {\textit{dy}}{, the second person singular proclitic (i.e., ‘your’), and} {\textit{tra}}{ ‘while’; other items do not standardly mutate or may vary between retaining the radical and bearing mutation. The adjectival modifiers} {\textit{mor} }{and} {\textit{cyn} }{‘so, as’, for instance, do not mutate (nor does the preposition} {\textit{cyn}}{ ‘before’), though for some speakers the pre-adjectival} {\textit{mwy}}{ ‘more’ does take  soft mutation (>} {\textit{fwy}}{). A few content words fail to mutate; for instance, the very old loanword} {\textit{braf}} {‘fine, splendid’ generally does not undergo  soft mutation, with the apparent exception of some southern dialects\is{dialectal variation}. Various prepositions typically occur in a lexicalized  soft mutation form and cannot take further mutation, e.g.} {\textit{dan} }{‘under’,} {\textit{dros} }{‘across, over’ and} {\textit{drwy}}{ ‘through’. Etymologically these derive from forms with radical initial} {\textit{t-}}{, and this still occurs in certain set expressions, such as} {\textit{tros y pen a’r clustiau}}{ ‘head over heels’ (lit. ‘over the head and the ears’).} 


This section has briefly presented a few of the further complexities of Welsh mutation, though many more exist. More details concerning the variation in mutations, the inability of various items to mutate and the occurrence of lexicalized mutations can be found in \citet{Morgan1952}. 

\section{Some idiosyncrasies of mutation}\label{sec:tallerman:4}

In this section, I explore some of the less obvious instances of mutation in Welsh. We will see that there are variations in the perceived mutability of forms, and in the effects of mutation. Unless otherwise stated, the examples of mutation in this section are soft mutation. 


\subsection{Mutation and variation in the radical initial segment}

{First, though we saw in \sectref{sec:tallerman:1} that the standard consonantal inventory for mutation consists of only nine consonants, in fact a tenth consonant is typically added, especially by northern speakers, in the form of a loan consonant \citep{Griffen1974}. The voiceless affricate [ʧ] does not occur in the standard set of mutable consonants}\footnote{{The affricates, both voiceless and voiced, are not part of the traditional consonant inventory at all \citep{Hannahs2013} but have developed due to palatalization in native words such as} {\textit{diawl,} }{‘devil’, with initial [ʤ].}}{, but appears in initial position in a few well-established \isi{loanwords}, such as} {\textit{tsiaen}}{ ‘chain’ and} {\textit{tsiocled}}{ ‘chocolate’. In a mutation context such as (\ref{ex:mt:tallerman:32}), where the conjunction} {\textit{neu}}{ ‘or’ is a soft mutation trigger, the noun undergoes mutation to give an initial voiced affricate [ʤ]:}

\ea\label{ex:mt:tallerman:32}
\gll {(Peidiwch byth â chysylltu)} rhaff \textbf{neu} dsiaen...  (\textit{tsiaen)}\\
{(Never connect)} rope or chain\\
\glt ‘Never connect a rope or a chain...’
\z

{Similarly, the loanword} {\textit{tsips}}{ ‘chips’ with initial [ʧ] is common in spoken Welsh and mutates to} {\textit{jips}}{ with initial voiced [ʤ] in  soft mutation contexts (here, an XP trigger hypothesis context):} 

\ea\label{ex:mt:tallerman:33}
\gll Gymi \textbf{di} jips?    ({\textit{tsips}})\\
take.\textsc{fut.2s} you chips\\
\glt ‘Do you want some chips?’
\z

{The addition of the affricate [ʧ] to the set of mutable consonants was noted as far back as the start of the last century by \citet{Fynes-Clinton1913}, who cites examples such as (\ref{ex:mt:tallerman:34}), where loaned English ‘chap’,} {\textit{tshiap}}{, with a voiceless initial affricate, mutates in a soft mutation context (following a pre-nominal adjective) to give a voiced initial affricate. Note that there is no standard way to spell words with initial affricates, so a variety of solutions exist, as these examples show:}

\ea\label{ex:mt:tallerman:34}
\gll yr hen [ʤap] \\
the old chap \\
\glt `the old chap'
\z

{For some speakers, as a reviewer notes, [ʧ] can also undergo nasal mutation following a trigger such as} {\textit{fy/yn} }{‘my’, giving examples such as} {\textit{tsiocled} }{>} {\textit{fy/yn nhiocled} }{‘my chocolate’.} 

Second, we turn to instances of speaker variation over the choice of ‘correct’ initial radical consonant, apparently at least partly motivated by the  soft mutation form of the word. In fact, even in the standard language doublets occur, so that alternative citation forms of some common words exist:

\ea
\ea
benthyg / menthyg ‘borrow’ (V)
\ex
benyw / menyw ‘woman’ (N)
\ex
bigwrn / migwrn ‘ankle, knuckle’ (N)
\z
\z

{It seems reasonable to suggest that such doublets occur with initial} {\textit{b-}}{ or} {\textit{m-}}{ because both have the  soft mutation form [v], leading to two possible choices of radical form. Since} {\textit{benyw/menyw} }{is also a feminine noun, it frequently occurs in a soft mutation context following the definite article:} {\textit{y fenyw}}{;} {\textit{bigwrn/migwrn}}{, however, is masculine.} 

Loanwords \is{loanwords} from English are especially susceptible to reanalysis of the initial consonant. If the loanword has initial [v] in English, there is a tendency for this to be perceived as the  soft mutation form of the word: only a handful of native Welsh words have [v]-initial citation forms, and these comprise either  small function words such as {\textit{fy}}{ ‘my’ or grammaticalized items such as} {\textit{faint}}{ (<} {\textit{pa faint}}{) ‘how much, how many’ (Welsh <f> = [v]). So, in southern dialects\is{dialectal variation}, instead of standard} {\textit{fecso, fecsio}}{ (< ‘vex’) we find the form} {\textit{becso}}{, where a perceived  soft mutation form is reanalysed to have an initial consonant expected in native words, [b]:} 

\ea\label{ex:mt:tallerman:36}
fecso / becso ‘worry’ (V) 
\z

\noindent Other doublets or triplets are frequently attested in English \isi{loanwords} with initial [v]: 

\ea\label{ex:mt:tallerman:37}
\ea
{  ‘(ad)vantage’ >} {\textit{mantais}}{, but also} {\textit{fantais}}
\ex
{  ‘venture’ >} {\textit{mentro}}{, but also} {\textit{fentro}}
\ex
{  ‘varnish’ >} {\textit{barnais}}{, but also} {\textit{marnais,}}{} {\textit{farnais}}
\ex
{  ‘vicar’ >} {\textit{bicar}}{, but also} {\textit{ficer}}
\z
\z

\noindent In all these examples, initial [v] is apparently perceived to represent a soft mutation form, just as it would in native content words; it is therefore reanalysed to have the ‘correct’ initial consonant in the radical form. 

{Speaker variation concerning the radical form of a word also occurs with words that could be perceived as either [g]-initial or vowel-initial in the radical form. Since [g] deletes under soft mutation, speakers may reanalyse vowel-initial words with a perceived ‘missing’ radical initial consonant. Again, some of this variation occurs in standard Welsh, giving rise to doublets such as} {\textit{allt/gallt}}{ ‘slope, hill’;} {\textit{ordd/gordd} }{‘hammer, mallet’,} {\textit{ogof/gogof}}{ ‘cave’. These nouns are feminine, and clearly often occur following the definite article, which, as we saw in \sectref{sec:tallerman:1}, is a trigger for soft mutation; this might account for some variation. But} {\textit{wyneb/gwyneb}}{ ‘face’ is masculine, so would not lose the initial [g] following the article. In all such cases, the vowel-initial form of the noun seems standardly to be the etymon, while the [g]-initial form is the reanalysed form. Conversely, the standard citation form of the loanword from English ‘honest’ is} {\textit{gonest}}{, though} {\textit{onest} }{also occurs as an alternative citation form. Again, motivations for this second reanalysis are not hard to find, since adjectives often occur in a triggering context for soft mutation. For instance, adjectives in the predicative position follow the predicate marker} {\textit{yn}}{, a trigger for soft mutation:} 

\ea\label{ex:mt:tallerman:38}
\gll Mae ’r dyn \textbf{yn} onest.     (\textit{gonest}) \\
be.\textsc{pres.3s} the man \textsc{pred} honest\\
\glt ‘The man is honest.’
\z

In spoken Welsh, the verb {\textit{addo}} ‘promise’ is reanalysed as (non-standard) \textit{gaddo} by some speakers, reconstructing a ‘missing’ radical initial [g]. This interspeaker variation occurs quite naturally in exchanges such as (\ref{ex:mt:tallerman:39}):

\ea\label{ex:mt:tallerman:39}
\ea
\gll Speaker A:   Addo? \\
{} {} promise.{\textsc{inf}}\\
\glt Speaker A: ‘(Do you) promise?’
\ex
\gll Speaker B:   Gaddo!\\
{} {} promise.{\textsc{inf}}\\
\glt Speaker B: ‘(I) promise!’    
\z
\z

Finally in this section, we also find dialectal variation\is{dialectal variation} over the choice of radical initial consonant, especially in \isi{loanwords} (\citealt{Morgan1952}:455f): 

\ea
\ea
{\textit{tamp}}{ > SM form} {\textit{damp} }{(northern form)   ‘damp’}
\ex
{\textit{damp} }{> SM form} {\textit{ddamp} }{(southern form)   ‘damp’}
\z
\z

\noindent {Here we see that if} {\textit{damp}}{ is perceived as the soft mutation form, then it must be reconstructed to have the radical form} {\textit{tamp}}{.}\footnote{{A reviewer for this volume suggests that the variation may be due to historical differences in voice onset time between English and different dialects of Welsh\is{dialectal variation}, with north Walians hearing English /d/ as voiceless.}}


\subsection{Extending mutation to nonce loans}

\is{nonce loans|(}
{We have already seen some of the effects of mutation in \isi{loanwords} in previous sections, and mutation may either recreate the phonology of the loaning language or destroy it. Examples of the former include} {\textit{mantais > fantais}}{ and} {\textit{gonest~> onest}}{; the latter includes examples such as} {\textit{tsips > jips}}{.} 


Instances of mutation in established \isi{loanwords} are commonplace; Welsh has thousands of such loans, and many of course begin with a mutable initial consonant. Some typical examples of established loans are shown here, all following triggers for soft mutation:

\ea\label{ex:mt:tallerman:41}
\ea
\settowidth\jamwidth{(marsipán)}
\textit{cwpl o fois} ‘couple of boys’ \jambox{(\textit{bois})} 
\ex
\textit{bocs llawn o d\^wls}  ‘a box full of tools’ \jambox{(\textit{t\^wls})}
\ex 
\textit{prif falerina} ‘principal ballerina’  \jambox{(\textit{balerina})}
\ex
\gll gofyn i gwsmeriaid i foicotio {}'r archfarchnad \\
ask the customers to boycott the supermarket\\ \jambox{(\textit{boicotio})}
\glt ‘Ask the customers to boycott the supermarket.’
\ex
\textit{dy basport di} ‘your passport’ \jambox{(\textit{pasport})}
\ex
\textit{yn y farced} ‘in the market’ \jambox{(\textit{marced})}
\ex
\textit{bach o farsipán} ‘a bit of marzipan’ \jambox{(\textit{marsipán})}
\ex
\textit{bach o dreiffl} ‘a bit of trifle’ \jambox{(\textit{treiffl})}
\z
\z

{Loans are also incorporated into the standard system for the other mutation series, as (\ref{ex:mt:tallerman:42}) shows with aspirate mutation following the feminine singular proclitic} {\textit{ei}}{:}

\ea\label{ex:mt:tallerman:42}
\gll ei thedis    hi    (\textit{tedis})\\
\textsc{3fs} teddies(+AM) her\\
\glt ‘her teddies’
\z

Not only well-established loans, but also \isi{nonce loans} are subject to mutation, entirely regularly. Some typical examples with soft mutation are shown in (\ref{ex:mt:tallerman:43}); single quotes indicate a clearly English word and pronunciation:

\ea\label{ex:mt:tallerman:43}                        
\ea\label{ex:mt:tallerman:43a}
\textit{dy [ð]êt di}    ‘your date’  (\textit{dêt}{ – cf. the established loan} {\textit{dât}}{, meaning}   a date on the calendar)
\ex\label{ex:mt:tallerman:43b}
\textit{ym [mh]rofile i} ‘my profile’ (English pronunciation; cf. the standard loan {\textit{proffil}}{, with a distinct pronunciation)}
\ex\label{ex:mt:tallerman:43c}
\textit{ti’n [v]echanig} ‘you’re a mechanic’ (\textit{mechanig}{, not the standard loan} \textit{mecanic})
\ex\label{ex:mt:tallerman:43d}
\textit{rhyw [ð]ampener} ‘kind of dampener’ (\textit{dampener)}
\ex\label{ex:mt:tallerman:43e}
\textit{cwpl o [ð]rincs} ‘couple of drinks’ (\textit{drincs}) 
\ex\label{ex:mt:tallerman:43f}
\textit{siarad efo ei [g]ontacts} ‘speak to his contacts’ (\textit{contacts})
\ex\label{ex:mt:tallerman:43g}
\textit{mwy o [v]uscles} ‘more muscles’
\ex\label{ex:mt:tallerman:43h}
\textit{dy [v]acon butty} ‘your bacon butty’
\ex\label{ex:mt:tallerman:43i}
\textit{unrhyw}\textit{ ‘[v]oiler’} ‘any boiler’
\z
\z

{In (\ref{ex:mt:tallerman:43a}), the English word for a date with a boyfriend or girlfriend is loaned, and bears soft mutation following the second person proclitic} {\textit{dy}}. In (\ref{ex:mt:tallerman:43b}), there exists a standard loanword {\textit{proffil}}{, with standard Welsh phonology, but in this (oral) example the English word} {\textit{profile}}{ was used, and the nonce loan bears nasal mutation following the first person proclitic} {\textit{yn}} here assimilated to {\textit{ym}}{. In (\ref{ex:mt:tallerman:43h}), it is entirely clear that} {\textit{bacon butty} }{is a nonce loan, since it does not follow Welsh head-initial word order, cf.} {\textit{brechdan cig moch}}{ (}{\textit{brechdan ‘}}{sandwich’,} {\textit{cig moch}}{ ‘bacon’) ‘bacon sandwich’.} 

Clearly, then, mutation is entirely productive in Welsh, and both \isi{nonce loans} and established loans are integrated fully into the mutation system. 
\is{nonce loans|)}
\subsection{Mutation of non-citation and novel citation forms of words}

Given the fluidity of mutation illustrated in previous sections, it is not too surprising that a variety of non-standard instances of mutation occur quite commonly. These include what are sometimes probably just nonce slips, though these are often not merely speech errors, but also occur in written Welsh, even of a fairly formal variety. This section illustrates with various instances of soft mutation.

{First, the deletion of initial syllables in Welsh is quite common, even if these are penultimate syllables and therefore stressed, the regular stress being on the penult. In southern varieties, for instance, we find} {\textit{hefyd} }{> ’}{\textit{fyd}}{ ‘also’ and} {\textit{eto} }{> ’}{\textit{to}}{ ‘again, yet’. In some cases, the reduction gives rise to a new or alternative citation form, which may then produce a mutable initial consonant distinct from that of the full form:}

\ea\label{ex:mt:tallerman:44}
\textit{baned} (< \textit{paned}) {from \textit{cwpanaid} ‘cupful, cuppa’}
\z

{In this case, the standard citation form for ‘cupful, cuppa’ is} {\textit{cwpanaid}}{, a noun which has a mutable initial consonant, [k], but which is masculine, so does not bear soft mutation following the definite article:} {\textit{y cwpanaid}}{. However, the reduced form} {\textit{paned, panad}}{ is the normal form in spoken Welsh, and somewhat surprisingly, this is a feminine noun,} so soft mutation is triggered after the definite article.\footnote{{The root noun} {\textit{cwpan}}{ ‘cup’ can be masculine or feminine, and a reviewer suggests that} \textit{cwpanaid} {may have originated in a dialect\is{dialectal variation} in which it is feminine.}}

Similarly, the full form of the noun meaning ‘bird’ is {\textit{aderyn}}{, but the common spoken form is} {\textit{deryn}}. This noun is masculine, so does not bear soft mutation following the definite article. However, in other mutation environments this reduced form is always treated as a normal citation form, and so mutates in appropriate contexts. Here the soft mutation occurs following an XP trigger:

\ea
\ea\label{ex:mt:tallerman:45a}
\gll Mi welais \textbf{i} dderyn bach.   \textit{(deryn)} \\
\textsc{prt} see.\textsc{past.1s} I bird small \\
\glt ‘I saw a small bird.’
\ex\label{ex:mt:tallerman:45b}
\gll Mae \textbf{yna} dderyn bach.   (\textit{deryn}) \\
be.\textsc{pres.3s} there bird small \\
\glt ‘There’s a small bird.’
\z
\z

\noindent {Note that the plural form of the noun is vowel-initial,} {\textit{adar}}{ ‘birds’, so the formal citation form of the singular noun is presumably quite salient.}

The full form of the noun meaning ‘Christmas’ is \textit{Nadolig}, often contracted to \textit{Dolig,} {which then undergoes soft mutation following a trigger such as predicative} {\textit{yn}}{:}

\ea\label{ex:mt:tallerman:46}
\gll Mae ’n Ddolig.    (\textit{Dolig}) \\
be.\textsc{pres.3s} \textsc{pred} Christmas\\
\glt ‘It’s Christmas.’
\z

{Second, double mutations are also not uncommon in speech and even in writing: a form of a word that is the soft mutation form is treated as the radical, and then undergoes further mutation. Here, the trigger for soft mutation is the predicate marker} {\textit{yn} }{(reduced to} {\textit{’n}}{ after a vowel). The citation form is} {\textit{parod,} }{soft mutation form} {\textit{barod}}{; if} {\textit{barod} }{is then perceived as the radical form, it mutates to become *}{\textit{farod}}{:}

\ea\label{ex:mt:tallerman:47}
\gll Dw i ’n farod i helpu. ({\langscicheckmark}\textit{parod}{ >} {\langscicheckmark}\textit{barod} > {*\textit{farod}})\\
be.\textsc{pres.1s} I \textsc{pred} ready to help radical (+SM) {} {(SM of SM form)}   \\
\glt ‘I’m prepared to help.’
\z

\ea\label{ex:mt:tallerman:48}
\gll ...pan fydd o ’n farod i {}'r twrnamaint i ddechrau\\
when be.\textsc{fut.3s} he  \textsc{pred} ready for the tournament to begin\\
\glt ‘...when he is ready for the tournament to begin’
\z

{Similarly, the citation form of the verb for ‘vote’ is} {\textit{pleidleisio}}{, SM form} {\textit{bleidleisio}}{, but the doubly mutated form *}{\textit{fleidleisio,}}{ following a soft mutation trigger, is not hard to find in internet searches:}

\ea\label{ex:mt:tallerman:49}
\gll {Ar ôl} y rhyfel gafodd fenywod dros 30 yr hawl i fleidleisio.\\
after the war get.\textsc{past.3s} women over 30 the right to vote.\textsc{inf}\\
\glt ‘After the war women over 30 got the right to vote.’
\z

\ea
\gll y rhesymau dros fleidleisio cynnar\\
the reasons for vote.\textsc{inf} early \\
\glt ‘the reasons for early voting’
\z

We thus see the productivity of the Welsh mutation system, its extensions to novel and \isi{nonce loans}, and the existence of both idiolectal and dialectal variation\is{dialectal variation} in mutation and in the citation forms of words.


\section{A final conundrum}\label{sec:tallerman:5}

I now briefly present an interesting conundrum involving the definite article and the soft mutation which it triggers on a following feminine singular noun. The definite article has three surface forms, {\textit{’r}} [r], \textit{yr} [ər] and \textit{y}{ [ə]. The enclitic form} {\textit{’r}}{ occurs after a vowel-final word and takes precedence over the other two forms; it will not be further discussed here. The variant} {\textit{y} }{occurs before a consonant, as in} {\textit{y llyfr}}{ ‘the book’ and} {\textit{yr}}{ occurs elsewhere, i.e., before vowels, e.g.} {\textit{yr afon} }{‘the river’, and before glides and [h]. Now consider the following data:}

\ea
\ea\label{ex:mt:tallerman:51a}
gardd \textsc{[fem.} \textsc{sg.]}
\glt ‘garden’
\ex \label{ex:mt:tallerman:51b}
\gll yr ardd \\
the garden(+SM) \\ \jambox{[g] > ${\emptyset}$}
\glt ‘the garden’
\z
\z

{In (\ref{ex:mt:tallerman:51b}) we see mutation on the noun following the article, resulting in the soft mutation form:} {\textit{gardd} }{>} {\textit{ardd}}{. Since the noun is now vowel-initial, the article must be} {\textit{yr.} }{So} {\textit{yr} }{occurs not only before nouns which are canonically vowel-initial (}{\textit{yr afon}}{), but also before nouns in which the initial vowel results from mutation:} {\textit{yr ardd}}{. The soft mutation form of the noun determines the form of the article, yet the article triggers the mutation. Here, then, we have a paradox: the mutation (}{\textit{gardd} }{>} {\textit{ardd}}{) is triggered by the preceding article, but the correct form of the article (}{\textit{y / yr}}{) cannot be chosen unless the initial segment of the noun is known~– which in turn depends on whether or not mutation has occurred.}


The solution presented by \citet{HannahsTallerman2006} relies on the notion of staggered lexical insertion (\citealt{Emonds2000,Emonds2002}), in which content words are inserted into a derivation before functional elements.

\ea
{  [\_\_]}{\textsubscript{Def Art}}{ [gardd]}{\textsubscript{N}}{} $\Rightarrow${ [\_\_]}{\textsubscript{Def Art}}{ [ ardd]}{\textsubscript{N}}{}
\z

\noindent{Here, the soft mutation on the noun is triggered by the as-yet-unfilled slot for the definite article; the article itself is inserted later in the derivation, in its correct form} {\textit{yr}}{, since it can now} {‘see’ the initial segment of the noun.}\footnote{{As a variant of this analysis, a reviewer suggests that the input to the definite article is} {\textit{y(r)} }{in all cases, where the [r] is present, but has no phonological slot. If the following word begins with a vowel, the [r] is saved. If not, it is deleted.}}

{Independent support for this analysis comes from the fact that various small functional elements are often unrealized; see (\ref{ex:mt:tallerman:27}) and (\ref{ex:mt:tallerman:28}). But the mutations triggered by these phonetically null items remain, at least in some registers; therefore, unfilled slots that trigger various mutations are independently needed. Moreover, these unfilled slots have a syntactic presence in that they block an overt preceding item from triggering some mutation, as we also saw in (\ref{ex:mt:tallerman:29}). Example (\ref{ex:mt:tallerman:53}) shows the full form of a PP with an overt agreement proclitic} {\textit{ein}}{:}

\ea\label{ex:mt:tallerman:53}
\gll  yn ein tŷ ni \\
in \textsc{1p} house us\\ \jambox{(\textit{ein} is not a mutation trigger) }
\glt ‘in our house’
\z

{It is common in speech for the proclitic to be unrealized, as in (\ref{ex:mt:tallerman:54a}). The noun} {\textit{tŷ}}{ then appears to follow a trigger for nasal mutation,} {\textit{yn}}{, but this does not trigger any mutation here, as shown in (\ref{ex:mt:tallerman:54b}), because the unfilled slot for the proclitic blocks the mutation:}

\ea
\ea[]{\label{ex:mt:tallerman:54a}
\gll yn \sout{ein} tŷ ni \\
in \sout{\textsc{1p}} house us \\
\glt ‘in our house’
}
\ex[*]{\label{ex:mt:tallerman:54b}
\gll yn nhŷ ni\\
in house(\textsc{+nm}) us\\
\glt 'in our house'}
\ex[]{\label{ex:mt:tallerman:54c}
\gll yn nhŷ Alun \\
in house(\textsc{+nm}) Alun \\
\glt ‘in Alun’s house’
}
\z
\z

\noindent {By contrast, in (\ref{ex:mt:tallerman:54c}) where there is no proclitic (as these only occur with pronominal possessors), the preposition \textit{yn} does precede the noun, and this takes NM: \textit{tŷ}~> \textit{nhŷ.}}


\section{Future directions}\label{sec:tallerman:6}

The mutation system is clearly still active in the modern Welsh language, as previous sections have shown. The vast majority of environments for mutation are contexts for soft mutation, and the other two series are very limited, each having only a few triggers in the form of small functional elements. However, one relatively recent development not yet mentioned is the strong tendency for soft mutation to occur in certain non-traditional contexts, spreading to environments which are formally contexts for AM or NM. For instance, \textit{yn} ‘in’ is a trigger for nasal mutation in standard Welsh, but is often followed instead by soft mutation. Thus, we might find \textit{yn Fangor} ‘in Bangor’ instead of the standard \textit{ym Mangor}. More details concerning sociolinguistic aspects of the modern mutation system can be found in \citet{Webb-DaviesToAppear}. The full picture of dialectal\is{dialectal variation} and idiolectal variation in the mutation system remains to be investigated, however. In addition, changes are in progress owing to the numbers of new speakers, typically those who do not come from a Welsh-speaking household but have attended a Welsh-medium school.

This paper has outlined some of the more interesting features of Welsh mutation, yet numerous aspects of the mutation system are only touched on here. The ‘bible’ of Welsh mutations is \citet{Morgan1952}, though the fact that this is in Welsh precludes a wide readership. A reasonably comprehensive list of mutation triggers can be found in many pedagogical grammars; perhaps the most accessible of these is \citet{King2015}, which also contains useful details of dialectal variation\is{dialectal variation}. Discussion and formal analysis of various aspects of mutation is found in \citet{BorsleyEtAl2007}, and discussion of the phonological aspects of mutation is found in \citet{Hannahs2013}.
\il{Welsh (Modern)|)}
\is{Initial Consonant Mutation|)}

\section*{Acknowledgements}
I am grateful to two anonymous reviewers for suggesting very helpful changes and corrections, and to Bob Borsley and S.J. Hannahs for their comments on an earlier version of this paper. All remaining errors are my own.

\sloppy
\printbibliography[heading=subbibliography,notkeyword=this]
\end{document}
