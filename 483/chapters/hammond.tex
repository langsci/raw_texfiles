\documentclass[output=paper,colorlinks,citecolor=brown]{langscibook}
\ChapterDOI{10.5281/zenodo.15654867}
\author{Michael Hammond\orcid{}\affiliation{University of Arizona}}
\title{Phonology of the \ce\ languages}

%keywords
%DONE
%Celtic
%Welsh,breton,cornish,irish,manx
%Scottish Gaelic
%phonology
%inventory
%mutation,mutate
%palatalization
%lenition
%eclipsis
%vowel reduction
%epenthesis
%new speakers
%prestopping
%velarization
%preaspiration

\IfFileExists{../localcommands.tex}{
   \usepackage{langsci-optional}
\usepackage{langsci-gb4e}
\usepackage{langsci-lgr}

\usepackage{listings}
\lstset{basicstyle=\ttfamily,tabsize=2,breaklines=true}

%added by author
% \usepackage{tipa}
\usepackage{multirow}
\graphicspath{{figures/}}
\usepackage{langsci-branding}

   
\newcommand{\sent}{\enumsentence}
\newcommand{\sents}{\eenumsentence}
\let\citeasnoun\citet

\renewcommand{\lsCoverTitleFont}[1]{\sffamily\addfontfeatures{Scale=MatchUppercase}\fontsize{44pt}{16mm}\selectfont #1}
  
   %% hyphenation points for line breaks
%% Normally, automatic hyphenation in LaTeX is very good
%% If a word is mis-hyphenated, add it to this file
%%
%% add information to TeX file before \begin{document} with:
%% %% hyphenation points for line breaks
%% Normally, automatic hyphenation in LaTeX is very good
%% If a word is mis-hyphenated, add it to this file
%%
%% add information to TeX file before \begin{document} with:
%% %% hyphenation points for line breaks
%% Normally, automatic hyphenation in LaTeX is very good
%% If a word is mis-hyphenated, add it to this file
%%
%% add information to TeX file before \begin{document} with:
%% \include{localhyphenation}
\hyphenation{
affri-ca-te
affri-ca-tes
an-no-tated
com-ple-ments
com-po-si-tio-na-li-ty
non-com-po-si-tio-na-li-ty
Gon-zá-lez
out-side
Ri-chárd
se-man-tics
STREU-SLE
Tie-de-mann
}
\hyphenation{
affri-ca-te
affri-ca-tes
an-no-tated
com-ple-ments
com-po-si-tio-na-li-ty
non-com-po-si-tio-na-li-ty
Gon-zá-lez
out-side
Ri-chárd
se-man-tics
STREU-SLE
Tie-de-mann
}
\hyphenation{
affri-ca-te
affri-ca-tes
an-no-tated
com-ple-ments
com-po-si-tio-na-li-ty
non-com-po-si-tio-na-li-ty
Gon-zá-lez
out-side
Ri-chárd
se-man-tics
STREU-SLE
Tie-de-mann
}
   \boolfalse{bookcompile}
   \togglepaper[6]%%chapternumber
}{}

\AffiliationsWithoutIndexing

\abstract{\mhhl{This chapter offers a review of the phonological systems of the extant \ce\ languages, including their consonant and vowel inventories\is{inventory} as well as their salient phonological properties. The chapter focuses especially on how these phonologies differ from each other and on the similar developments that have occurred among them.}}

\begin{document}

\maketitle

\is{phonology|(}

\section{Introduction}

In this chapter, I offer an overview of the synchronic phonology of the modern \ce\ languages. My goal is to see what is common to these languages and to take seriously the proposition that there might be a more general \ce\ phonological system that characterizes all of these languages together. The individual languages vary, of course, but I proceed on the hypothesis that the variations are in balance and that the differences between the languages aren't random but are tempered by other things.

The organization of this chapter is as follows. First, I lay out the general historical and geographic context for the modern \ce\ languages. I then turn to a \emph{very} brief overview of the phonological systems of each of the \ce\ languages, focusing on points of similarity and comparison. I then try to synthesize these different systems and discuss the common threads that run through the phonologies of all the languages.

There are many many issues I simply cannot cover here given the breadth of the topic. Thus, my goal is to give enough of a sense of these individual systems to address the larger question, so that interested readers can pursue specific topics further on their own.

\section{Overview and context}

\ce\ is a branch of the Indo-European family. The modern representatives of this group include Breton\il{Breton (Modern)}, Cornish\il{Cornish (Modern)}, Irish\il{Irish (Modern)}, Manx\il{Manx Gaelic (Modern)}, Scottish Gaelic\il{Scottish Gaelic (Modern)}, and Welsh\il{Welsh (Modern)}. \mhhl{Breton\il{Breton (Modern)}, Cornish\il{Cornish (Modern)}, and Welsh\il{Welsh (Modern)} comprise the Brythonic subgroup and Irish\il{Irish (Modern)}, Manx\il{Manx Gaelic (Modern)}, and Scottish Gaelic\il{Scottish Gaelic (Modern)} the Goidelic subgroup \citep{mm:fife_typological_2010}.}

\ce\ languages used to be spoken across much of Europe but now are spoken only in Ireland, Britain, and France \citep{emergence}.\footnote{There are, of course, expatriate communities in other places. The most significant ones are: Scottish Gaelic\il{Scottish Gaelic (Modern)} in Nova Scotia, Canada and Welsh\il{Welsh (Modern)} in Patagonia, Argentina.} It's generally thought that \ce\ peoples migrated from the European continent to the British islands and their languages then died out on the continent. The Breton\il{Breton (Modern)} language in France resulted from \ce\ migration \emph{back} to the continent.

Subsequent migrations or invasions by various Germanic peoples, Romans, and Norman French forced the various \ce\ groups to the peripheries of Britain, Ireland, and France. In all cases, the languages are in significant and increasing competition with large national languages.

The speaker populations are all fairly small and difficult to quantify: Do we include all speakers or restrict ourselves to first language speakers? Do we only include speakers who use the language regularly? Do we expect speakers to be literate in their language? Table~\ref{speakers.tab} gives \emph{very} rough estimates of speaker populations for each language, based on a fairly generous interpretation of native speaker.\footnote{More official, though perhaps less accurate, estimates are available at \url{https://www.ethnologue.com}.}

\begin{table}
\caption{Estimated speakers}
\label{speakers.tab}
\begin{tabular}{lr}
\lsptoprule
Language & Speakers \\
\midrule
Breton\il{Breton (Modern)} & 200,000 \\
Cornish\il{Cornish (Modern)} & 0 \\
Irish\il{Irish (Modern)} & 40,000 \\
Manx\il{Manx Gaelic (Modern)} & 0 \\
Scottish Gaelic\il{Scottish Gaelic (Modern)} & 50,000 \\
Welsh\il{Welsh (Modern)}  & 500,000 \\
\lspbottomrule
\end{tabular}
\end{table}

The numbers for Cornish\il{Cornish (Modern)} and Manx\il{Manx Gaelic (Modern)} require some context. Both of them are formerly dormant languages that are currently undergoing active revitalization efforts \citep{lowecornish,lewin2022continuity}. The last first-language native speaker of Manx\il{Manx Gaelic (Modern)}, Ned Maddrell, died in 1974. The last first-language native speaker of Cornish\il{Cornish (Modern)}, Dolly Pentreath, died in 1777. Both languages have a number of second language speakers, probably numbering near 1000 in each case. In the case of Manx\il{Manx Gaelic (Modern)}, there are reportedly new first-language native speakers. The first-language situation with Cornish\il{Cornish (Modern)} is unknown at this time.

The \ce\ languages exhibit a number of frequently cited typological features that are fairly rare \citep{mm:fife_typological_2010}. These include the following:

\begin{enumerate}
\item consonant \m\is{mutation}
\item inflected prepositions
\item VSO word order
\end{enumerate}

\noindent Other properties are occasionally cited as well, but these \mhhl{three} are generally considered to be core \ce\ properties. In order to get a sense of the larger grammatical setting within which each phonology is seated, each of these properties are investigated in turn.

The consonant \m s\is{mutation} involve systematic changes to the initial consonant of a word in specific morphological or syntactic contexts. For example, in all the \ce\ languages, the initial consonant of a feminine singular noun will change after the definite article.\footnote{The Goidelic languages distinguish grammatical case using the \m s\is{mutation}; in those languages, the generalization applies only in the nominative.} This consonant change is overtly encoded in the orthographies of these languages, as seen in Table~\ref{fem.muts.tab}.

\begin{table}
\caption{Consonant \m\ with feminine nouns after the definite article}
\label{fem.muts.tab}
\begin{tabular}{llll}
\lsptoprule
Language & Noun  & Gloss     & With article \\
\midrule
Breton      & \itshape bro   & `country' & \itshape ar vro \\
Cornish      & \itshape tus   & `people'  & \itshape an dus \\
Irish      & \itshape bean  & `woman'   & \itshape an bhean \\
Manx      & \itshape ben   & `woman'   & \itshape yn ven \\
Scottish Gaelic      & \itshape caora & `sheep'   & \itshape a' chaora \\
Welsh       & \itshape cath  & `cat'     & \itshape y gath \\
\lspbottomrule
\end{tabular}
\end{table}

The key fact is that words that begin with the same consonant mutate\is{mutation} the same way in the same environment. Thus, for example, any feminine singular noun that begins with [b] in Breton\il{Breton (Modern)} mutates\is{mutation} to [v] after the definite article. There are additional niceties in each language, which are discussed below.

All the modern \ce\ languages exhibit inflected prepositions, where some prepositions are marked for the person and number of their objects. In Table~\ref{preps1.tab} and Table~\ref{preps2.tab} we see the relevant orthographic forms for the cognate preposition `on'. These may or may not co-occur with the relevant overt pronoun. For example, in Scottish Gaelic\il{Scottish Gaelic (Modern)}, the overt pronoun does not co-occur with an inflected preposition; in Welsh\il{Welsh (Modern)}, an overt pronoun does not occur in formal registers but typically does occur in the spoken language, e.g.\ \emph{arna i} `on me'.

\begin{table}
\tabcolsep=.75\tabcolsep
\begin{floatrow}
\ttabbox[.525\textwidth]
{\begin{tabular}{>{\scshape}llll}
\lsptoprule
      & Breton\il{Breton (Modern)}      & Cornish\il{Cornish (Modern)}        & Irish\il{Irish (Modern)} \\
\midrule
1sg   & \itshape warnon   & \itshape warnav/m   & \itshape orm \\
2sg   & \itshape warnout  & \itshape warnas     & \itshape ort \\
3sgm  & \itshape warna\~n & \itshape warnadho   & \itshape air \\
3sgf  & \itshape warni    & \itshape warnedhi   & \itshape uirthi \\
1pl   & \itshape warnomp  & \itshape warnan     & \itshape orainn \\
2pl   & \itshape warnoc'h & \itshape warowgh    & \itshape oraibh \\
3pl   & \itshape warno    & \itshape warnedhans & \itshape orthu \\
\lspbottomrule
\end{tabular}}
{\caption{Inflected preposition `on' in Breton, Cornish, and Irish}
\label{preps1.tab}}
\ttabbox[.475\textwidth]
{\begin{tabular}{>{\scshape}llll}
\lsptoprule
      & Manx\il{Manx Gaelic (Modern)} & SG & Welsh\il{Welsh (Modern)} \\
\midrule
1sg   & \itshape orrym & \itshape orm   & \itshape arna \\
2sg   & \itshape ort   & \itshape ort   & \itshape arnot \\
3sgm  & \itshape er    & \itshape air   & \itshape arno \\
3sgf  & \itshape urree & \itshape oirre & \itshape arni \\
1pl   & \itshape orrin & \itshape oirnn & \itshape arnon \\
2pl   & \itshape erriu & \itshape oirbh & \itshape arnoch \\
3pl   & \itshape orroo & \itshape orra  & \itshape arnyn \\
\lspbottomrule
\end{tabular}}
{\caption{Inflected preposition `on' in Manx, Scottish Gaelic, and Welsh}
\label{preps2.tab}}
\end{floatrow}
\end{table}

The \ce\ languages are also noted for VSO (verb--subject--object) word order. In all of them, this occurs \mhhl{when the verb is directly inflected (e.g., in the simple past tense}). Following are examples in Welsh\il{Welsh (Modern)} (\ref{vso.w.ex}) and in Scottish Gaelic\il{Scottish Gaelic (Modern)} (\ref{vso.sg.ex}).

\ea\label{vso.w.ex}
\gll Gwelodd y   dyn  y  gath \\
     saw     the man the cat \\
\glt `The man saw the cat.'
\ex\label{vso.sg.ex}
\gll Chunnaic an  duine an  cat \\
     saw      the man   the cat \\
\glt `The man saw the cat.'
\z

As previously noted, Cornish\il{Cornish (Modern)} and Manx\il{Manx Gaelic (Modern)} are actively undergoing revitalization, and this surely has consequences for how researchers might characterize the structures of these languages. However, it's fair to say that the other languages are also strongly affected by their contact with neighboring majority languages. To put it bluntly, natural conversations in these languages almost always involve code-switching \mhhl{or nonce borrowing} with the neighboring dominant language.

For example, in speaking Welsh\il{Welsh (Modern)} casually and talking about `voting in the election', one is far more likely to hear borrowed or code-switched \emph{fotio yn y lecsiwn} [vɔtjo ən ə lɛkʃʊn], rather than native \emph{pleidleisio yn yr etholiad} [pʰlejdlejʃo ən ər ɛθɔljad].

This general situation has obvious consequences for the phonologies of these languages. The phonologies don't exist autonomously but in a, perhaps sometimes conflicting, relationship with other phonologies. \mhhl{For example, \citet{bennett2018} discuss how the \isi{palatalization} contrast is being lost in various Irish\il{Irish (Modern)} dialects, presumably as an effect of contact with English. \citet{morris2021} discusses a very interesting example where some Welsh\il{Welsh (Modern)} varieties have acquired English [ɹ], but some English varieties have acquired Welsh\il{Welsh (Modern)} [r, ɾ].}

Keeping this in mind, I now turn to a very brief overview of the specific phonologies of each of the \ce\ languages, focusing on \isi{inventory}, \m\is{mutation}, and any salient phonological processes. The systems are treated in two groups. The Brythonic or P-\ce\ languages can be opposed to the Goidelic or Q-\ce\ languages. \mhhl{As noted above,} the Brythonic languages are comprised of Breton\il{Breton (Modern)}, Cornish\il{Cornish (Modern)}, and Welsh\il{Welsh (Modern)}; the Goidelic languages include Irish\il{Irish (Modern)}, Manx\il{Manx Gaelic (Modern)}, and Scottish Gaelic\il{Scottish Gaelic (Modern)}. The P- and Q- nomenclature reflects the fact that Indo-European *kʷ generally corresponds to Brythonic [p] and Goidelic [k] (written as either 〈c〉 or 〈q〉 in the orthography). Table~\ref{pq.tab} provides some examples of these correspondences in word-initial position.

\begin{table}
\caption{Reflexes of Indo-European *kʷ}
\label{pq.tab}
\begin{tabular}{l@{\quad}lllll}
\lsptoprule
&          & `children' & `five'    & `what' & `four' \\
\midrule
\multicolumn{5}{l}{P-Celtic}\\
& Breton\il{Breton (Modern)}      & \itshape (bugale)   & \itshape pemp      & \itshape pe     & \itshape pevar \\
& Cornish\il{Cornish (Modern)}      & \itshape (flehes)   & \itshape pemp/pymp & \itshape pe/py  & \itshape peswar \\
& Welsh\il{Welsh (Modern)}      & \itshape plant      & \itshape pump      & \itshape pa     & \itshape pedwar \\
\multicolumn{5}{l}{Q-Celtic}\\
& Irish\il{Irish (Modern)}      & \itshape clann      & \itshape c\'uig    & \itshape c\'en  & \itshape ceithre \\
& Manx\il{Manx Gaelic (Modern)}      & \itshape cloan      & \itshape queig     & \itshape cre    & \itshape kiare \\
& Scottish Gaelic\il{Scottish Gaelic (Modern)}      & \itshape clann      & \itshape c\`oig    & \itshape (d\`e) & \itshape ceithir \\
\lspbottomrule
\end{tabular}
\end{table}

\section{Goidelic}

The phonologies of the Goidelic languages exhibit some commonalities. First, all the languages exhibit a \isi{palatalization} contrast among the consonants, \mhhl{an innovation from Proto-\ce}. Second, the \m\is{mutation} systems are somewhat simpler than in the Brythonic languages and include at most two \m s\is{mutation}, traditionally termed \emph{lenition}\is{lenition} and \emph{eclipsis}\is{eclipsis}.\footnote{This is the traditional analysis. Some analysts posit many more, e.g.\ \citet{carnie} for Irish\il{Irish (Modern)}.} On the other hand, the \m s\is{mutation} affect more consonants and the mappings are somewhat more dramatic. Third, all the languages exhibit \isi{vowel reduction} of stressless short vowels. Finally, the languages all include a marked \isi{epenthesis} process involving clusters of a sonorant and an obstruent.\footnote{As I show later, the process appears to have occurred in Manx\il{Manx Gaelic (Modern)}, but does not seem to be active in the synchronic language.}

\il{Irish (Modern)|(}

\subsection{\ir}

\ir\ is spoken in Ireland, principally the Republic of Ireland, but to a lesser extent in Northern Ireland as well. The densest speaker populations are in the \emph{Gaeltacht} or rural areas predominantly along the west coast of Ireland in the counties of Donegal, Galway, Kerry, and Mayo. In the south and east, Cork, Waterford, and Meath are official Gaeltacht areas as well. There are substantial numbers of second language speakers in other parts of the country, especially Dublin. There are also what we might term ``new'' speakers\is{new speakers}, speakers whose proficiency comes from immersion at school or in communities, but whose parents are not speakers. This is an important category for the other \ce\ languages as well.

The language exhibits substantial dialect variation. Traditionally, three main dialect areas are distinguished.

\begin{itemize}
\item Munster (Cork, Kerry, Waterford)
\item Connacht (Galway, Mayo)
\item Ulster (Donegal)
\end{itemize}

\noindent There are interesting phonological differences between the dialects, but I will \mhhl{largely} set these aside given the scope of the chapter.

There are excellent comprehensive phonological treatments of the language, including \citet{ni.chiosain.diss}, \citet{green.diss}, \citet{hickey}, and \citet{o.siadhail}. Generalizing across the dialects, the consonants are given in Table~\ref{ir.consonants.tab}. Notice that there is a contrast between plain \mhhl{(properly, velarized\is{velarization})} and palatalized\is{palatalization} consonants for virtually all consonants. As well, in some dialects the consonants include velarized\is{velarization} versions of some or all of [l, r, n] that, together with \isi{palatalization}, makes for a four-way contrast among these segments.

\begin{table}
\caption{\ir\ consonants}
\label{ir.consonants.tab}
\begin{tabular}{lccccccc}
\lsptoprule
        &  &  & \multicolumn{3}{c}{Palatalized}  &   &  \\\cmidrule(lr){4-6}
        & Labial & Coronal & Labial  & Coronal & Dorsal  & Dorsal & Glottal\\
\midrule
Stop    & p, b & t, d & pʲ, bʲ  & tʲ, dʲ & kʲ, gʲ   & k, g       & \\
Nasal   & m    & n    & mʲ      & nʲ     & ŋʲ       & ŋ          & \\
Fric.   & f, v & s    & fʲ, vʲ  & sʲ     & xʲ, j    & x, ɣ       & h \\
Approx. &      & l, r &         & lʲ, rʲ &          &            & \\
\lspbottomrule
\end{tabular}
\end{table}

The \ir\ vowels appear in Table~\ref{ir.vowels.tab}. Note that there is a contrast for length at all heights with some qualitative effects, though it's unclear whether this contrast extends to the low vowel(s). \citet{ni.chiosain.diss} proposes a much sparer input \isi{inventory} for the short vowels.

\begin{table}
\caption{\ir\ vowels}
\label{ir.vowels.tab}
\begin{tabular}{lccc}
\lsptoprule
     & Front & Central & Back \\
\midrule
High & i, iː &         & ʊ, uː \\
Mid  & e, eː & ə       & ɔ, oː \\
Low  &       & a, aː   & \\
\lspbottomrule
\end{tabular}
\end{table}

Historically, the \isi{palatalization} contrast for the consonants can be connected to the vowels, but the vowel system has undergone such radical changes that this is no longer the case for long vowels, and the contrast is phonemic synchronically. Short non-low vowels do show a relationship: they are front or back depending on the \isi{palatalization} value of the following segment. \citet[141]{ni.chiosain.diss} cites the examples in Table~\ref{ir.shortvowels.tab}.

\begin{table}
\caption{Backness of a non-low vowel determined by the following consonant in \ir}
\label{ir.shortvowels.tab}
\begin{tabular}{>{\itshape}lccc}
\lsptoprule
cur   & [kur]  & `put'   & nonfin. \\
cuir  & [kirʲ] & `put'   & imper. \\
muc   & [muk]  & `a pig' & nom.\ sg. \\
mhuic & [wikʲ] & `a pig' & dat.\ sg. \\
\lspbottomrule
\end{tabular}
\end{table}

There are a number of alternations between palatalized\is{palatalization} and plain consonants, but these are all morphologized. The only non-morphologized part of these alternations is that they are transitive. When a consonant becomes palatalized\is{palatalization} or non-palatalized\is{palatalization}, adjacent consonants to the left follow suit. In other words, sequences of adjacent consonants agree in \isi{palatalization}.

Like all the other \ce\ languages, \ir\ has initial consonant \m\is{mutation}. As previously noted, the initial consonant of a word undergoes systematic changes depending on its morphosyntactic environment. Setting aside the nature of the environment, traditionally there are two basic \m\is{mutation} patterns in \ir.\footnote{Though see \citet{carnie} for a more complex picture.} Following traditional terminology, these are referred to as \emph{lenition}\is{lenition} and \emph{eclipsis}\is{eclipsis}. Both are laid out in Table~\ref{ir.mutations.tab}. Basically, \isi{lenition} converts stops to fricatives; compare examples (\ref{Ir.nl.ex1}--\ref{Ir.l.ex1}) and (\ref{Ir.nl.ex2}--\ref{Ir.l.ex2}). In addition, voiceless fricatives either delete or debuccalize. Coronal consonants also debuccalize. Finally, [m] and [mʲ] become fricatives as well. Eclipsis\is{eclipsis} makes voiceless stops voiced and makes voiced stops into nasals. In addition, the fricatives [f] and [fʲ] become voiced. These changes are overtly expressed in the orthography as well.\footnote{Note that \isi{palatalization} is expressed with vowel symbols in the spelling.}

\begin{table}
\caption{\ir\ consonant \m}
\label{ir.mutations.tab}
\begin{tabular}{cccccc}
\lsptoprule
\multicolumn{3}{c}{Phonology} & \multicolumn{3}{c}{Orthography} \\\cmidrule(lr){1-3}\cmidrule(lr){4-6}
Base & Lenition\is{lenition}    & Eclipsis\is{eclipsis} & Base & Lenition\is{lenition} & Eclipsis\is{eclipsis} \\\midrule
p    & f           & b        & p    & ph       & bp \\
t    & h           & d        & t    & th       & dt \\
tʲ   & hʲ/xʲ       & dʲ       & t    & th       & dt \\
k    & x           & g        & c    & ch       & gc \\
kʲ   & xʲ          & gʲ       & c    & ch       & gc \\
b    & v           & m        & b    & bh       & mb \\
bʲ   & vʲ          & mʲ       & b    & bh       & mb \\
d    & ɣ           & n        & d    & dh       & nd \\
dʲ   & j           & nʲ       & d    & dh       & nd \\
g    & ɣ           & ŋ        & g    & gh       & ng \\
gʲ   & j           & ŋʲ       & g    & gh       & ng \\
f    & $\emptyset$ & v        & f    & fh       & bhf \\
fʲ   & $\emptyset$ & vʲ       & f    & fh       & bhf \\
s    & h           & s        & s    & sh       & s \\
sʲ   & hʲ/h/xʲ     & sʲ       & s    & sh       & s \\
m    & v           & m        & m    & mh       & m \\
mʲ   & vʲ          & m        & m    & mh       & m \\
\lspbottomrule
\end{tabular}
\end{table}

As already noted, the \m s\is{mutation} are triggered by morphosyntactic environments so they are curiously morphosyntactic and phonological at the same time. On the one hand, the morphosyntactic conditioning suggests that they should be viewed as something morphological. On the other hand, the fact that the consequences of \m\is{mutation} are phonological suggests that it should be viewed as a phonological process.

There is a virtual industry in debating which of these is correct for different \m s\is{mutation} in the different \ce\ languages from myriad theoretical perspectives. See, for example:
\citet{hamp,
rogers,
fula,
lieber,
ballmuller,
kibre,
pyatt,
stewart,
green,
hannahs.mut,
hannahs.book}.
I will not address this question here and instead focus on the phonological \emph{consequences} of this process.

The most obvious consequence of these is that they alter the set of possible word-initial consonants and clusters (\cite{grijzenhout, irish.syl}). Thus, for example, [x] is not generally allowed word-initially in content words but is permitted word-initially in \isi{lenition} contexts, as shown in the past tense forms (\ref{Ir.l.ex1}) and (\ref{Ir.l.ex2}).

%\dko{Table 9 - until I added this comment - interrupts example 3 and is placed well after its first citation in text. Is there a way to keep the examples together?}

\ea\label{Ir.nl.ex1}
\gll caithim   \'e \\
     throw-1sg  it \\
\glt [kaim eː] \\ `I throw it.'
\ex\label{Ir.l.ex1}
\gll chaith  m\'e  \'e \\
     threw   I     it \\
\glt [xa(j) meː eː] \\
     `I threw it.'
\ex\label{Ir.nl.ex2}
\gll caillim   \'e \\
     lose-1sg it \\
\glt [kalʲim eː] \\
     `I lose it.'
\ex\label{Ir.l.ex2}
\gll chaill m\'e  \'e \\
     lost   I     it \\
\glt [xalʲ meː eː] \\
     `I lost it.'
\z

Similarly, the \m s\is{mutation} create word-initial clusters not otherwise allowed outside of \m\is{mutation} contexts. For example, while [sn] and [sʲlʲ] are \mhhl{generally} legal onsets, [hn] and [hʲlʲ] are only legal in \isi{lenition} contexts; \mhhl{they are not otherwise acceptable}. Compare (\ref{Ir.sn.ex}) and (\ref{Ir.sjlj.ex}) with (\ref{Ir.hn.ex}) and (\ref{Ir.hjlj.ex}); again, all in the past tense:

\ea\label{Ir.sn.ex}
\gll sn\'amhaim \\
     swim-1sg \\
\glt [snaːwim] \\
     `I swim.'
\ex\label{Ir.hn.ex}
\gll shn\'amh m\'e \\
     swam     I \\
\glt [hnɔv meː] \\
     `I swam.'
\ex\label{Ir.sjlj.ex}
\gll sleamhna\'im \\
     slip-1sg \\
\glt [sʲlʲawniːm] \\
     `I slip.'
\ex\label{Ir.hjlj.ex}
\gll shleamhnaigh m\'e \\
     slipped      I \\
\glt [hʲlʲawnə meː] \\
     `I slipped.'
\z

\noindent These sorts of effects on the \isi{inventory} of (word-initial) segments and the \isi{inventory} of word-initial consonant clusters run through all the \ce\ languages.

Another very salient aspect of \ir\ phonology that has received a lot of attention in the literature is \isi{epenthesis} into rising sonority clusters.\footnote{\mhhl{In the literature, this is sometimes referred to as \emph{svarabhakti}.}} The phenomenon is curious in a number of respects. First, the clusters that this applies to are composed of a sonorant followed by a voiced consonant or a voiceless fricative; \isi{epenthesis} is blocked when the second consonant is a voiceless stop or if the consonants are homorganic. The context following the cluster is irrelevant. In Table~\ref{ir.epenth.tab}, I give examples where \isi{epenthesis} occurs \citep[from][560]{irish.syl}. In Table~\ref{ir.epenth.vcl.stops.tab}, I give examples where \isi{epenthesis} does not occur because the second consonant is a voiceless stop \citep[from][561]{irish.syl}. And in Table~\ref{ir.epenth.homorganic.tab}, I give cases where \isi{epenthesis} is blocked because the two consonants are homorganic.

\begin{table}
\caption{Examples of \ir\ epenthesis}
\label{ir.epenth.tab}
\begin{tabularx}{.6\textwidth}{lllQ}
\lsptoprule
lb & \itshape bolb       & [boləb]    & `caterpillar' \\
lw & \itshape gealbhan   & [gʲaləwən] & `sparrow' \\
lx & \itshape tulchach   & [tuləxəx]  & `hilly' \\
rb & \itshape borb       & [borəb]    & `abrupt' \\
rm & \itshape gorm       & [gorəm]    & `blue' \\
rx & \itshape dorcha     & [dorəxə]   & `dark' \\
nb & \itshape binb       & [bʲinʲibʲ] & `venom' \\
nm & \itshape ainm       & [anʲimʲ]   & `name' \\
nv & \itshape leanbh     & [lʲanəv]   & `child' \\
\lspbottomrule
\end{tabularx}
\end{table}

\begin{table}
\caption{Voiceless stop examples (no \ir\ epenthesis)}
\label{ir.epenth.vcl.stops.tab}
\begin{tabularx}{.6\textwidth}{lllQ}
\lsptoprule
rp & \itshape corp     & [korp]   & `body' \\
rt & \itshape gort     & [gort]   & `field' \\
rk & \itshape cearc    & [kʲark]  & `hen' \\
lp & \itshape spalp    & [spalp]  & `burst forth' \\
lt & \itshape alt      & [alt]   & `joint' \\
lk & \itshape olc      & [olk]    & `evil' \\
nt & \itshape cantal   & [kantəl] & `irritation' \\
\lspbottomrule
\end{tabularx}
\end{table}

\begin{table}
\caption{Homorganic consonant examples (no \ir\ epenthesis)}
\label{ir.epenth.homorganic.tab}
\begin{tabularx}{.6\textwidth}{lllQ}
\lsptoprule
lr & \itshape iolra      & [olrə]       & `plural' \\
ld & \itshape gallda     & [galda]      & `foreign' \\
ls & \itshape foilsigh   & [folʲsʲi]    & `reveal' \\
mb & \itshape siombalach & [sʲombələx]  & `symbolic' \\
nd & \itshape gandal     & [gandal]     & `gander' \\
ng & \itshape teanga     & [tʲaŋgə]     & `language' \\
\lspbottomrule
\end{tabularx}
\end{table}

Finally, the process \mhhl{also cannot occur} when the preceding vowel is long. \citet{ni.chiosain.diss} cites these examples: \emph{t\'earma} [tʲeːrmə] `term' and \emph{l\'eargas} [lʲeːrgəs] `insight'.

This is a devilishly tricky phenomenon to treat. First, the process exhibits no alternations. The first consonant must be a coda, but \isi{epenthesis} doesn't care whether the second consonant is tautosyllabic or heterosyllabic. There are no processes that bear on the consonantal restrictions. Finally, the length of the preceding vowel doesn't alternate. The main evidence for the process is the surface absence of clusters where \isi{epenthesis} must apply (but does not).

\mhhl{An additional argument can be made from the morphology \citep{irish.syl}. When a final consonant is palatalized\is{palatalization} for morphological reasons, this \isi{palatalization} also occurs on the consonant preceding the epenthetic\is{epenthesis} vowel, as if the consonants were adjacent when \isi{palatalization} occurs, e.g.\ in \emph{colm} `dove' [koləm], but \emph{coilm} `doves' [kelʲimʲ].}

A second complication is that syllable structure doesn't matter, as shown by the irrelevance of what follows the cluster. Yet moraic status \emph{does} matter as far as the preceding vowel (since it cannot be long). Finally, the odd restriction on voiceless stops presents an additional complication \mhhl{\citep{fullwood2}}. I return to the process of \isi{epenthesis} further on because very similar facts obtain in Scottish Gaelic\il{Scottish Gaelic (Modern)}.

\mhhl{Stress in most dialects of \ir\ falls on the initial syllable, except for a few borrowings. However, stress in Munster \ir\ is quite striking and has received a fair amount of attention in the theoretical literature, e.g. \citet{iosadmunster} and \citet{outopt}.}

\mhhl{In Munster, stress falls on the first syllable, e.g.\ in \emph{clagarnach} [ˈklagərnəx] `clattering'. If one of the first three syllables contains a long vowel or diphthong, then it gets stress, e.g.\ in \emph{cail\'in} [kaˈlʲiːnʲ] `girl', \emph{amad\'anta\'iocht} [əməˈdɑːntiːxt] `foolishness'.}

\mhhl{There are two interesting additional complications. First, if the first two syllables are long, then stress falls on the second one, e.g.\ \emph{d\'iomhaoin} [dʲiːˈviːnʲ] `idle'. Otherwise stress falls on the first long vowel in the first three syllables, e.g.\ \emph{ampar\'ana\'iocht} ['oumpərɑːniːxt] `ungainliness'.} 

\mhhl{Second, if there are no long vowels or diphthongs in the first three syllables and the second vowel is [a] followed by [x], then it gets stressed, e.g.\ in \emph{bacach} [bəˈkax] `beggar', \emph{bacacha} [bəˈkaxə] `beggars', \emph{casachtach} [kəˈsaxtəx] `coughing'.}

\mhhl{Finally, a comment on syllable structure. It's been observed that a single intervocalic consonant affiliates to the syllable to the left in \ir, rather than to the right. \citet{nichiosain2012} investigate this claim experimentally and find that, in fact, affiliation to the left is preferred when the preceding vowel is short, and that the consonant typically affiliates in \emph{both} directions. Strikingly, very similar facts have been cited for English (\cite{mhlangeng}; \citeyear{mybook}). Notice that the stress facts just cited for Munster [ax] sequences lend support to such an account, as [ax] would constitute a syllable rhyme on this view.}

\il{Irish (Modern)|)}

\il{Manx Gaelic (Modern)|(}

\subsection{\ma}

The \ma\ language is another Goidelic language very closely related to Irish\il{Irish (Modern)}. As already noted above, \ma\ is spoken on the Isle of Man and went dormant upon the death of the last native speaker (at that time) in 1974. However, also as previously noted, \ma\ is currently the object of active revitalization efforts and there are now over a thousand speakers, including children.

As with Irish\il{Irish (Modern)}, I outline the segmental \isi{inventory}, the \m s\is{mutation}, and salient aspects of the phonology. Where appropriate, I compare these with the Irish\il{Irish (Modern)} system above. It's important to keep in mind, though, that \ma\ is in great flux due to the recent revitalization efforts. My description is based mostly on \citet{broderick}.

The \ma\ consonants appear in Table~\ref{ma.consonants.tab}. The system is quite similar to that of Irish\il{Irish (Modern)}, though note the absence of palatalized\is{palatalization} labials. \cut{and} \mhhl{Note also} that the coronal stops are dental and occasionally affricated.

\begin{table}
\caption{\ma\ consonants}
\label{ma.consonants.tab}
\begin{tabular}[t]{lcccccc}
\lsptoprule
        &  &  & \multicolumn{2}{c}{Palatalized\is{palatalization}} &  &  \\\cmidrule(lr){4-5}
        & Labial & Coronal & Coronal  & Dorsal  & Dorsal & Glottal\\
\midrule
Stop    & p, b & t͡θ, d͡ð & tʲ, dʲ     & kʲ, gʲ        & k, g & \\
Fricative & f, v & s        & sʲ         & xʲ            & x    & h \\
Nasal   & m    & n        & nʲ         & ŋʲ            & ŋ    & \\
Approximant & w    & l, r     & lʲ, j      &               &      & \\
\lspbottomrule
\end{tabular}
\end{table}

\ma\ has a simple five-vowel system with a length contrast and schwa. The vowels appear in Table~\ref{ma.vowels.tab}.

\begin{table}
\caption{\ma\ vowels}
\label{ma.vowels.tab}
\begin{tabular}[t]{lccc}
\lsptoprule
         & Front & Central & Back \\
\midrule
High     & i, iː &         & u, uː \\
Mid      & e, eː & ə       & o, oː \\
Low      &       & a, aː   & \\
\lspbottomrule
\end{tabular}
\end{table}

There is a sporadic process of intervocalic weakening of stops, but it's not clear under what conditions this occurs and what the precise targets are. For example, a word like \emph{scapail} `escaping' can be pronounced [skəpʰeːlʲ], [skəβeːlʲ], or [skəveːlʲ]. Similarly, \emph{brattag} `flag' can be pronounced [bratag], [bradag], or [braðag]. Voiced stops exhibit a similar pattern. For example: \emph{\c chibbyrt} `well' as [tʲubərt], [tʲuβərt], or [tʲuvərt]. \cut{Historically,} The \isi{lenition} \m\is{mutation} occurred \mhhl{historically} within words and it's reasonable to suppose this weakening process is a similar reassertion of that pattern.

Another salient process is \isi{prestopping} where nasals and the liquid [l] can acquire a preceding voiced stop when the preceding vowel is stressed, as exemplified in Table~\ref{ma.prestopping.tab}. Interestingly, as we'll see later, a similar process occurs in Cornish\il{Cornish (Modern)}.

\begin{table}
\caption{\ma\ prestopping}
\label{ma.prestopping.tab}
\begin{tabular}[t]{llll}
\lsptoprule
\itshape cam    & `crooked'  & [kam]    & [kabm] \\
\itshape eeym   & `butter'   &          & [iːbm] \\
\itshape bane   & `white'    & [beːn]   & [be(ː)dn] \\
\itshape keayn  & `sea'      &          & [kidn] \\
\itshape lhong  & `ship'     & [loŋ]    & [lo/ugŋ] \\
\itshape shooyil & `walking' & [sʲuːl]  & [sʲuːdl] \\
\itshape keeill & `church'   & [kʲiːlʲ] & [kʲidlʲ] \\
\lspbottomrule
\end{tabular}
\end{table}

Unlike the other Goidelic languages, \ma\ does not seem to exhibit \isi{epenthesis}. There are many examples of forms that would undergo \isi{epenthesis} in the other Goidelic languages that do not do so in \ma. See Table~\ref{ma.noepenthesis.tab} for examples where no \isi{epenthesis} appears in \ma\ but does appear in the corresponding Irish\il{Irish (Modern)} form. On the other hand, there are a number of cases where the cognate form in \ma\ has apparently undergone \isi{epenthesis} (see Table~\ref{ma.epenthesis.tab}). It would seem reasonable to suppose that \mhhl{\ma} once had \isi{epenthesis}, but it has since been lost \mhhl{leaving now lexicalized remnants}.

%\dko{the IPA transcriptions of the Irish forms would make Table 16-17 easier to understand}

\begin{table}
\caption{No epenthesis in \ma}
\label{ma.noepenthesis.tab}
\begin{tabular}[t]{lllll}
\lsptoprule
\multicolumn{3}{l}{\ma}   & \multicolumn{2}{l}{Irish}   \\
\midrule
\itshape jiarg        & [dʲarg]    & `red'     & \itshape dearg & [dʲarəg] \\
\itshape pershagh     & [pɛrʒəx]   & `peach'    \\
\itshape argid        & [ərgəd]    & `money'   & \itshape airgead & [ærʲɪgʲəd] \\
\itshape lorg         & [lɔrg]     & `pole'     \\
\itshape margher      & [margɛ]    & `market'   \\
%emshyr       & [ɛmʃər]    & `weather'  & aimsir & [amʃər] \\
\lspbottomrule
\end{tabular}
\end{table}

\begin{table}
\caption{Lexicalized epenthesis in \ma}
\label{ma.epenthesis.tab}
\begin{tabular}[t]{lllll}
\lsptoprule
\multicolumn{3}{l}{\ma}   & \multicolumn{2}{l}{Irish} \\
\midrule
\itshape gorrym        & [gɔrəm]   & `blue'     & \itshape gorm     & [gɔrəm]  \\
\itshape dorraghey     & [dɔrəxə]  & `dark'     & \itshape dorcha   & [dɔrəxə] \\
\itshape ennym         & [ɛnəm]    & `name'     & \itshape ainm     & [ænʲɪm] \\
\itshape (thie) fillym & [fɪləm]   & `cinema'   & \itshape          &  \\
\lspbottomrule
\end{tabular}
\end{table}

\ma\ \m s\is{mutation} are given in Table~\ref{ma.mutations.tab}. It's important to keep in mind that \isi{eclipsis} is virtually absent from the modern language.\footnote{\mhhl{Eclipsis\is{eclipsis} occurs only after the article in the genitive plural and after the plural possessive pronoun \emph{nyn}.}}

\begin{table}
\caption{Manx \m s}
\label{ma.mutations.tab}
\begin{tabular}[t]{ccc}
\lsptoprule
Base & Lenition\is{lenition}    & Eclipsis\is{eclipsis} \\
\midrule
p    & f           & b \\
t    & h           & d \\
tʲ   & h/xʲ        & dʲ \\
k    & x           & g \\
kʲ   & xʲ          & gʲ \\
b    & v           & m \\
bw   & w           & mw \\
m    & v \\
mw   & w \\
d    & ɣ           & n \\
dʲ   & j           & nʲ \\
g    & ɣ           & ŋ \\
gʲ   & j           & ŋʲ \\
f    & $\emptyset$ & v \\
fw   & hw          & w \\
s    & h \\
sʲ   & h/xʲ  \\
\lspbottomrule
\end{tabular}
\end{table}

\il{Manx Gaelic (Modern)|)}

\il{Scottish Gaelic (Modern)|(}

\subsection{\sg}

\sg\ is the third Goidelic language.\footnote{The language is occasionally referred to as ``Scots Gaelic'' but this term is dispreferred by speakers of the language.} The language formerly extended over much of Scotland, but is now fairly restricted to the extreme northern and northwestern peripheries and islands. There was substantial emigration during the highland clearances \mhhl{in the 18th and 19th centuries}, but, except perhaps for Nova Scotia and Prince Edward Island in Canada, there are no current substantial speaker communities outside of Scotland.

The language is the object of active revitalization efforts and there are now Gaelic medium schools in Edinburgh and Glasgow, as well as in the north. Historically, there is a great deal of dialect variation, but this is greatly reduced as a function of language attrition and media in English. There are a number of useful discussions of the phonology and phonetics of \sg\ in the literature and my discussion draws on a number of these, for example:
\citet{blas},
\citet{bosch.syl},
\citet{bosch},
\citet{bosch.dejong},
\citet{barra},
\citet{green.diss},
\citet{sg.structures},
\citet{sginsertion},
\citet{gaelic.len},
\citet{nance},
\citet{smith}.

The \sg\ consonants appear in Table~\ref{sg.consonants.tab}. Note a \isi{palatalization} contrast as in Irish\il{Irish (Modern)} and Manx\il{Manx Gaelic (Modern)}. Note, too, the presence of a velarized\is{velarization} series for sonorant coronals. These are not all present in all dialects and the contrast is occasionally realized instead in terms of length and/or diphthongization of the preceding vowel. The voicing contrast for stops is different from the other Goidelic languages. The contrast in initial position is between aspirated and unaspirated voiceless stops. Medially, the unaspirated series is typically voiced, and the aspirated series is typically \emph{preaspirated}\is{preaspiration}\cut{medially} \mhhl{in non-initial position}, e.g.\ in \emph{cat} [kʰaʰt] `cat'.\footnote{The word-initial version of this second series is what appears in Table~\ref{sg.consonants.tab}.}

\begin{table}
\caption{\sg\ consonants}
\label{sg.consonants.tab}
\begin{tabular}[t]{lccccccc}
\lsptoprule
        &  &   &  & \multicolumn{2}{c}{Palatalized\is{palatalization}} &   & \\\cmidrule(lr){5-6}
          & Labial & Dental & Alveolar & Coronal    & Dorsal  &   Dorsal    & Glottal \\
\midrule
Stop      & pʰ, p & t̪ʰ, t̪ &      & tʲʰ, tʲ & kʲʰ, kʲ           & kʰ, k & \\
Nasal     & m     &       & n    & ɲ       &                  & nˠ    & \\
Fricative & f, v  &       & s    & ʃ       & ç, ʝ             & x, ɣ  & h \\
Lateral   &       &       & l    & lʲ      &                  & l̪ˠ    & \\
Rhotic    &       &       & ɾ    & ɾʲ      &                  & rˠ    & \\
Approx    &     &       &      & \multicolumn{2}{c}{j}      &       & \\
\lspbottomrule
\end{tabular}
\end{table}

The \sg\ vowels appear in Table~\ref{sg.vowels.tab}. Note the presence of length and nasality contrasts, though the nasality contrast is not extensive and may be being lost among younger speakers. Note, too, that there are non-low back unrounded vowels in the system.\footnote{\mhhl{\citet{nichasaide} reports that [ɤ] occurs in some varieties of Irish\il{Irish (Modern)}.}}

\begin{table}
\caption{\sg\ vowels}
\label{sg.vowels.tab}
\begin{tabular}[t]{lcccc}
\lsptoprule
         & Front        & Central      & \multicolumn{2}{c}{Back} \\\cmidrule(lr){4-5}
         &              &              & Round      & Unround \\
\midrule
High     & i, iː, ĩ, ĩː &              & u, uː, ũ, ũː & ɯ, ɯː, ɯ̃, ɯ̃ː \\
Mid High & e, eː        &              & o, oː        & ɤ, ɤ: \\
Mid Low  & ɛ, ɛː, ɛ̃, ɛ̃ː &              & ɔ, ɔː, ɔ̃, ɔ̃ː & \\
High     &              & a, aː, ã, ãː &              &  \\
\lspbottomrule
\end{tabular}
\end{table}

The \sg\ \isi{lenition} \m\is{mutation} is given in Table~\ref{sg.lenition.tab}. The effects are essentially the same as in Irish\il{Irish (Modern)} and Manx\il{Manx Gaelic (Modern)}. Note that in dialects with the \isi{velarization} contrast, \isi{lenition} may also affect these consonants such that sonorant coronals lose \isi{velarization} and \isi{palatalization} in the relevant contexts. When this occurs, it is not represented in the orthography.

\begin{table}
\caption{\sg\ lenition \m}
\label{sg.lenition.tab}
\begin{tabular}[t]{cccccc}
\lsptoprule
\multicolumn{2}{c}{Orthographic} & \multicolumn{2}{c}{Plain} & \multicolumn{2}{c}{Palatalized\is{palatalization}} \\\cmidrule(lr){1-2}\cmidrule(lr){3-4}\cmidrule(lr){5-6}
Unlenited & Lenited & Unlenited & Lenited     & Unlenited & Lenited \\
\midrule
p         & ph      & pʰ        & f           & ---       & --- \\
b         & bh      & p         & v           & ---       & --- \\
t         & th      & tʰ        & h           & tʲʰ       & ç \\
d         & dh      & t         & ɣ           & tʲ        & ʝ \\
c         & ch      & kʰ        & x           & kʲʰ       & ç \\
g         & gh      & k         & ɣ           & kʲ        & ʝ \\
f         & fh      & f         & $\emptyset$ & ---       & --- \\
s         & sh      & s         & h           & ʃ         & ç \\
m         & mh      & m         & v           & ---       & ---  \\
\lspbottomrule
\end{tabular}
\end{table}

Eclipsis\is{eclipsis} \m\is{mutation} is absent in many dialects and not reflected in the orthography. If it does occur, it has the effects in Table~\ref{sg.eclipsis.tab}. The expected pattern is shown in the Eclipsis\is{eclipsis} column. However, \citet{eclipsis} notes additional patterns, which appear in different dialects; these are shown in the right three columns.\footnote{\citeauthor{eclipsis} describes the patterns in terms of a voicing contrast; I've translated it into the aspiration contrast used here. It's a little unclear what \citeauthor{eclipsis} intends by the voiced aspirated symbols. The key point about the B group is that a voicing distinction of some sort is maintained under \isi{eclipsis}.}

\begin{table}
\caption{\sg\  \m}
\label{sg.eclipsis.tab}
\begin{tabular}[t]{ccccc}
\lsptoprule
Consonant & Eclipsis\is{eclipsis} & A & B  & C \\
\midrule
pʰ        & p        & p & bʰ & mʰ \\
tʰ        & t        & t & dʰ & nʰ \\
kʰ        & k        & k & gʰ & ŋʰ \\
p         & m        & p & p  & m \\
t         & n        & t & t  & n \\
k         & ŋ        & k & k  & ŋ \\
f         & v        & f & f  &  \\
\lspbottomrule
\end{tabular}
\end{table}

Another \cut{extremely} interesting difference between \sg\ and the other Goidelic languages is that the trigger for \isi{eclipsis} \cut{when it occurs in \sg} always ends in a nasal consonant. This may be evidence that \isi{eclipsis} is \mhhl{in the process of becoming phonological again} in the language.

Finally, \sg\ exhibits essentially the same \isi{epenthesis} process as Irish\il{Irish (Modern)}. See Table~\ref{sg.epenthesis.tab} for a few examples. Note that the inserted vowel (bolded) is a copy of the preceding vowel in \sg. In addition, there are interesting prosodic properties associated with the inserted vowel in \sg\ (\cite{bosch.dejong, bosch.syl}). \mhhl{For example, \citet{bosch.dejong} report that the epenthetic\is{epenthesis} vowel in Barra \sg\ is acoustically stressed.}

%\dko{In Table 23, what are the epenthesis sites? Unable to tell in some forms and it would be good to either show the cluster as the similar Irish table does or otherwise highlight the epenthetic vowel; same issue in Table 24}

\begin{table}
\caption{Epenthesis (bolded) in \sg}
\label{sg.epenthesis.tab}
\begin{tabular}[t]{>{\itshape}llll}
\lsptoprule
seanmhair & [ʃɛn\textbf{ɛ}vɛr]  & `grandmother'   \\
arm       & [ar\textbf{a}m]     & `army'          \\
dorcha    & [dɔr\textbf{ɔ}xə]   & `dark'          \\
tilg      & [t͡ʃɪl\textbf{ɪ}k]   & `to toss away'  \\
urchair   & [ʊr\textbf{ʊ}xʊr]   & `shot'          \\
\lspbottomrule
\end{tabular}
\end{table}

The process is blocked in the same contexts as in Irish\il{Irish (Modern)} as shown in Table~\ref{sg.blocked.tab}: when the first consonant is not a sonorant, when the preceding vowel is long or a diphthong, when the consonants are homorganic, or when the second consonant is a voiceless stop.

\begin{table}
\caption{\sg\ examples where epenthesis is blocked (marked with underline)}
\label{sg.blocked.tab}
\begin{tabular}[t]{l>{\itshape}lll}
\lsptoprule
Nonsonorant C1    & smachd    & [smax\_k]      & `stifle'     \\
                  & sgriobhte & [skriv\_t͡ʃə]   & `written'    \\
Long vowel        & dualchas  & [duəl\_xəs]    & `tradition'  \\
                  & iarmailt  & [iːrmɛl\_t͡ʃ]   & `firmament'  \\
Homorganic CC     & \`Olaind  & [oːlan\_d͡ʒ]    & `Holland'    \\
                  & mandrag   & [man\_drak]    & `mandrake'   \\
C2 Voiceless Stop & olc     & [ɔl̥\_k]        & `evil'       \\
                  & corp      & [kɔr̥\_p]       & `body'        \\
\lspbottomrule
\end{tabular}
\end{table}

Note how the restriction against the second consonant being a voiceless stop makes a bit more sense in \sg\ than in Irish\il{Irish (Modern)}. In \sg, medial voiceless stops are preaspirated\is{preaspiration}. When there is a preceding sonorant, that sonorant is thus voiceless. We may then subsume the voiceless stop and homorganicity restriction under the same rubric of shared features blocking an intervening inserted element.

\il{Scottish Gaelic (Modern)|)}

\section{Brythonic}

The Brythonic languages (Breton\il{Breton (Modern)}, Cornish\il{Cornish (Modern)}, and Welsh\il{Welsh (Modern)}) differ, broadly speaking, from the Goidelic languages in several respects.

\begin{enumerate}

\item None of the Brythonic languages make a distinction between broad and slender contrasts (a \isi{palatalization} contrast). Concomitantly perhaps, none of the Brythonic languages distinguish overt grammatical case, since this distinction is one of the markers.

\item The \m\is{mutation} systems in the Brythonic languages are richer and typically more direct in the phonological relationships they exhibit.

\item Except perhaps for late Cornish\il{Cornish (Modern)}, none of the Brythonic languages have the kind of extensive \isi{vowel reduction} evident in the Goidelic languages. Stress in Goidelic is either on the first syllable or may, in some cases and dialects, shift to a heavy syllable to the right. In the Brythonic languages, stress is typically penultimate.

\item The Brythonic languages do not exhibit the vowel \isi{epenthesis} pattern we saw in Irish\il{Irish (Modern)} and Scottish Gaelic\il{Scottish Gaelic (Modern)} (though a different \isi{epenthesis} pattern is evident in Welsh\il{Welsh (Modern)}).

\end{enumerate}

\il{Breton (Modern)|(}

\subsection{\br}

\br\ is spoken in Brittany, France. It is unique among the \ce\ languages in that it is in contact with French rather than English. However, the situation could be viewed as somewhat more complex. First, \br\ speakers are thought to have migrated from Cornwall (and elsewhere) in the \mhhl{M}iddle \mhhl{A}ges and thus, to that point, the language was under the influence of English (and its forebearers) and Latin. Second, another factor that sets it apart is that in its present setting in France, it is in some contact with the Gallo language, a Romance language closely related to French and spoken in upper Brittany.\footnote{Interestingly, Cornish\il{Cornish (Modern)} shows evidence of French influence as well, at the very least because of the Norman conquest, but also because of continuing contact with \br\ \citep{cornish.history}. Thus, one would not want to say that French influence on \br\ began with its transposition to France.}

\br\ exhibits considerable dialect variation and my characterization here attempts to generalize beyond these distinctions. Fuller descriptions of the general system can be found in \citet{hemon} and \citet{press}.

The \br\ consonant \isi{inventory} appears in Table~\ref{breton.consonants.tab}. Note the absence of a plain versus palatalized\is{palatalization} contrast and the presence of labio-velar stops.

\begin{table}
\caption{\br\ consonants}
\label{breton.consonants.tab}
\begin{tabular}[t]{lcccccc}
\lsptoprule
       & Labial & Coronal & Pal-Alv & Dorsal & Lab-Dors & Uvular/Glottal \\
\midrule
Stop   & p, b    & t, d     &         & k, g    & kʷ, gʷ     & \\
Fricative   & f, v    & s, z     & ʃ, ʒ     & x, ɣ    &          & h \\
Nasal  & m      & n       & ɲ       &        &          & \\
Approx &        & l, r, ɹ   & ʎ, j, ɥ   &        & w        & ʁ \\
\lspbottomrule
\end{tabular}
\end{table}

It's worth noting that the rhotics [r, ɹ, ʁ] are extremely variable with some speakers producing taps and other assimilating fully to a French uvular. An unresolved issue is whether some obstruents contrast in fortis-lenis and/or length in at least some dialects.\footnote{See \citet{press} for discussion.}

A very salient consonant process in the language is final devoicing and voicing assimilation. Obstruents are voiceless in absolute phrase-final position but assimilate in voicing to a following segment. Compare (\ref{Br.fd1.ex}) with (\ref{Br.va1.ex}--\ref{Br.va4.ex}) from \citet[79]{hemon}.

\ea\label{Br.fd1.ex}
\gll sellit ouzh ar  beleg {[sɛlid uz aʁ bɛːlɛk]} \\
     look   at   the priest \\
\glt `Look at the priest.'
\ex\label{Br.va1.ex}
\gll setu ama\~n ar beleg kozh {[sɛty am\~a aʁ bɛːlɛk koːs]} \\
     voila here   the priest old \\
\glt `Here is the old priest.'
\ex\label{Br.va2.ex}
\gll beleg  ar  barrez {[bɛːlɛg aʁ baʁes]} \\
     priest the parish \\
\glt `the priest of the parish'
\ex\label{Br.va3.ex}
\gll ur beleg  mat {[œʁ bɛːlɛg maːt]} \\
     a  priest good \\
\glt `a good priest'
\ex\label{Br.va4.ex}
\gll ur beleg  mat  eo {[œʁ bɛːlɛg maːd eɔ]} \\
     a  priest good is \\
\glt `He is a good priest.'
\z

Strikingly, if two identical obstruents come together, they surface as voiceless (as shown in \ref{Br.doo1.ex} and \ref{Br.doo2.ex}; again, from \cite{hemon}). This happens whether they initially agree in voicing or not.

\ea\label{Br.doo1.ex}
\gll bloas zo {[blwa(s) so]} \\
     year  ago \\
\glt `a year ago'
\ex\label{Br.doo2.ex}
\gll dek gwele {[deːk kweːle]} \\
     10  beds \\
\glt `10 beds'
\z

\noindent \mhhl{However, the outcome does vary by dialect when both consonants are initially voiceless \citep{ledu}.} Interestingly, this same restriction is \mhhl{also} part of Welsh\il{Welsh (Modern)} classical meter, though it is \emph{not} a part of the synchronic language (\cite{morris, clywed, anghenion}).

The \br\ vowel \isi{inventory} appears in Table~\ref{breton.vowels.tab}. Note the presence of front round vowels and a contrast between open and close among mid vowels. Note too the robust nasalization contrast.

\begin{table}
\caption{\br\ vowels}
\label{breton.vowels.tab}
\begin{tabular}[t]{lcccc}
\lsptoprule
          & \multicolumn{2}{c}{Front}    & \multicolumn{2}{c}{Back} \\\cmidrule(lr){2-3}\cmidrule(lr){4-5}
          & Unround     & Round           & Unround     & Round \\
\midrule
High      & i, ĩ         & y, ỹ             &             & u, ũ \\
Close-mid & e, ẽ         & ø, ø̃             &             & o, õ \\
Open-mid  & ɛ, ɛ̃         & œ, œ̃             &             & ɔ, ɔ̃ \\
Low       & a, ã         &                 & ɑ, ɑ̃         &  \\
\lspbottomrule
\end{tabular}
\end{table}

The \br\ \m s\is{mutation} appear in Table~\ref{breton.mutations.tab}. The pattern is similar across the Brythonic languages: more \m s\is{mutation} and a more transparent phonological relationship between each sound and what it mutates\is{mutation} to.

\begin{table}
\caption{\br\ \m s}
\label{breton.mutations.tab}
\begin{tabular}[t]{ccccc}
\lsptoprule
     & \multicolumn{4}{c}{Mutation\is{mutation} type}\\\cmidrule(lr){2-5}
C    & Soft & Aspirate & Hard & Mixed \\
\midrule
p   & b    & f        &      & \\
t   & d    & z        &      & \\
k   & g    & x        &      & \\
b   & v    &          & p    & v \\
d   & z    &          & t    & t \\
g   & ɣ    &          & k    & ɣ \\
gw  & w    &          & kw   & w \\
m   & v    &          &      & v  \\
\lspbottomrule
\end{tabular}
\end{table}

There are a number of interesting restrictions on the \br\ \m\is{mutation} system. Some are reminiscent of those seen in the Goidelic systems, but some are unique to \br. For example, \emph{va} [ɛm] `my' generally triggers the aspirate \m\is{mutation}; compare (\ref{Br.reg1.ex}--\ref{Br.myasp1.ex}) and (\ref{Br.reg2.ex}--\ref{Br.myasp2.ex}). However, when it contracts to \emph{m} with a previous vowel-final particle, it only triggers that \m\is{mutation} on non-labials; compare (\ref{Br.reglab.ex}) with (\ref{Br.labnoasp.ex}).

\ea\label{Br.reg1.ex}
\gll kambr \\
     room \\
\glt [kambʁ] \\
     `room'
\newpage
\ex\label{Br.myasp1.ex}
\gll em    c'hambr \\
     to-my room-asp \\
\glt [ɛm xambʁ] \\
     `to my room'
\ex\label{Br.reg2.ex}
\gll ti \\
     house \\
\glt [ti] \\
     `house'
\ex\label{Br.myasp2.ex}
\gll em    zi \\
     to-my house-asp \\
\glt [ɛm zi] \\
     `to my house'
\ex\label{Br.reglab.ex}
\gll paner \\
     basket  \\
\glt [panɛr] \\
     `basket'
\ex\label{Br.labnoasp.ex}
\gll 'm paner \\
     my basket \\
\glt [m panɛr] \\
     `my basket'
\z

\noindent This is similar to the blocking of \isi{lenition} that occurs with coronals in Irish\il{Irish (Modern)} and Scottish Gaelic\il{Scottish Gaelic (Modern)} (Goidelic languages). Table~\ref{ir.sg.blocking.tab} gives examples of this \isi{lenition} process following a definite article that precedes a feminine singular noun. If the noun begins with a coronal stop, it does not mutate\is{mutation}.\footnote{It is unclear whether this occurs in Manx\il{Manx Gaelic (Modern)} \citep{teare.wheeler}.}

\begin{table}
\caption{Lenition \m\ in Irish and Scottish Gaelic. Bolded examples are cases where lenition is blocked before coronals.}
\label{ir.sg.blocking.tab}
\begin{tabular}[t]{l>{\itshape}lll}
\lsptoprule
Irish\il{Irish (Modern)} & an bhean     & [a vaːn]           & `the woman'   \\
    & an ghal      & [a ɣal]            & `the vapor'   \\
    & an deoch     & \textbf{[a dʲɔx]}  & `the drink'   \\
\midrule
Scottish Gaelic\il{Scottish Gaelic (Modern)} & a' bhean     & [ə vɛn]            & `the woman'   \\
    & a' chaileag  & [ə χalak]          & `the girl'    \\
    & an d\`anachd & \textbf{[ən daːnaxk]}& `the poetry'  \\
\lspbottomrule
\end{tabular}
\end{table}

\nocite{iosad.diss,kraemer}

\il{Breton (Modern)|)}

\il{Cornish (Modern)|(}

\subsection{\co}

As already noted, \co\ became dormant upon the death of the last (at that time) native speaker in 1777. Nonetheless, it has been the subject of almost constant revitalization attempts since then. Interestingly, several different versions of modern \co\ exist, based either on middle or late \co. I view these simply as dialects, or better yet \emph{chronolects}, of the same language. My main sources on the language are \citet{jenner} and \citet{brown.cornish}.

The \co\ consonant \isi{inventory} appears in Table~\ref{cornish.consonants.tab}. Note the absence of a plain versus palatalized\is{palatalization} contrast just as\cut{it's missing} in Breton\il{Breton (Modern)}.

\begin{table}
\caption{\co\ consonants}
\label{cornish.consonants.tab}
\begin{tabular}[t]{lcccccc}
\lsptoprule
         & Labial & Dental & Alveolar & Pal-Alv & Dorsal & Glottal \\
\midrule
Stop/Affricate  & p,b    & t,d    &          & t͡ʃ,d͡ʒ & k,g    & \\
Fricative     & f,v    & θ,ð    & s,z      & ʃ,ʒ     & x      & \\
Nasal    & m      & n      &          &         & ŋ      & \\
Approx   & w,ʍ    &        & l,ɾ/ɹ    & j       &        & h \\
\lspbottomrule
\end{tabular}
\end{table}

Late \co\ undergoes an interesting pre-stopping process that is reminiscent of that seen in Manx\il{Manx Gaelic (Modern)}, except that it is reflected in the orthography. Pre-stopping in \co\ occurs after a stressed vowel and before a nasal consonant, e.g. \emph{kabm} [kabm] `step' and \emph{pedn} [pɛdn] `head'. If the nasal occurs in a stressless syllable, pre-stopping does not occur, e.g. \emph{perghen} [ˈpɛrhɛn] `owner' and \emph{pedrevan} [pəˈdrɛvən] `lizard, newt'.

Since stress shifts under suffixation, this creates alternations in the plural, e.g.\ in \emph{kabmas} [ˈkabmas] `bay, bend' vs.\ \emph{kamajow} [kaˈmad͡ʒow] `bays, bends' and \emph{pednyn} [ˈpɛdnɪn] `tadpole' vs.\ \emph{penydnow} [pɛˈnɪdnow] `tadpoles'.

There is evidence that the language used to undergo voicing assimilation and final devoicing as in Breton\il{Breton (Modern)}, but it's unclear whether this occurs in the synchronic language.

The language has a number of sonority-reversed consonant clusters. These undergo \isi{epenthesis} in final position, which is transparent in the Cornish orthography, but alternations occur in suffixed forms (for example, in the plural; see Table~\ref{co.epenthesis.tab}).\footnote{\mhhl{\citet{bennettdiss} argues for a similar process in Irish\il{Irish (Modern)}. See \citet{smith} for Scottish Gaelic\il{Scottish Gaelic (Modern)}.}} 

\begin{table}
\caption{Epenthesis alternations in singular vs.\ plural forms in \co}
\label{co.epenthesis.tab}
\begin{tabular}[t]{l>{\itshape}lll}
\lsptoprule
vr   & lever     & [lɛvɐr]    & `book'   \\
     & levrow    & [lɛvrow]   & `books' \\\addlinespace
tr   & ownter    & [ɔʊntɐr]   & `uncle'  \\
     & owntres   & [ɔʊntrəs]  & `uncles'  \\\addlinespace
dhl  & kenedhel  & [kɛnəðɐl]  & `nation'  \\
     & kenedhlow & [kənɛðlow] & `nations'  \\\addlinespace
ml   & amal      & [æmɐl]     & `border, side' \\
     & emlow     & [ɛmlow]    & `borders, sides' \\
\lspbottomrule
\end{tabular}
\end{table}

Orthographically, the epenthetic\is{epenthesis} vowel is a copy of the preceding vowel, but this does not seem to be the case in the spoken form. That this is truly \isi{epenthesis}, rather than deletion, is evidenced by forms that don't alternate; e.g. \emph{lether} [lɛθɐr] `letter' vs.\ \emph{letherow} [ləθɛrow] `letters'.

The \co\ vowel \isi{inventory} appears in Table~\ref{cornish.vowels.tab}. Note the length contrast and the presence of front rounded vowels. The latter only occur in middle \co\ and the corresponding chronolect.

\begin{table}
\caption{\co\ vowels}
\label{cornish.vowels.tab}
\begin{tabular}[t]{lcccc}
\lsptoprule
         & \multicolumn{2}{c}{Front} & Central & Back \\\cmidrule(lr){2-3}
         & Unround & Round \\
\midrule
High     & iː, ɪ    & (yː, ʏ)           &         & uː, ʊ \\
Mid      & eː, ɛ    & (øː, œ)           & ə       & oː, ɔ/ɤ \\
Low      & æː, æ/a  &                  &         & ɑː, ɑ  \\
\lspbottomrule
\end{tabular}
\end{table}

Finally, the \co\ \m s\is{mutation} appear in Table~\ref{cornish.mutations.tab}. Note that they are quite similar to the \m s\is{mutation} in Breton\il{Breton (Modern)}.

\begin{table}
\caption{\co\ \m s}
\label{cornish.mutations.tab}
\begin{tabular}[t]{ccccc}
\lsptoprule
     & \multicolumn{4}{c}{Mutation\is{mutation} Type} \\\cmidrule(lr){2-5}
 C   & Soft          & Aspirate & Hard & Mixed \\
\midrule
p   & b             & f        &      & \\
t   & d             & θ        &      & \\
k   & g             & h        &      & \\
b   & v             &          & p    & f \\
d   & ð             &          & t    & t \\
g   & $\emptyset$,w &          & k    & h \\
gw  & w             &          & kw   & hw \\
m   & v             &          &      & f \\
t͡ʃ & d͡ʒ           &          &      & \\
\lspbottomrule
\end{tabular}
\end{table}

\il{Cornish (Modern)|)}

\il{Welsh (Modern)|(}

\subsection{\w}

Unsurprisingly, \w\ is spoken primarily in Wales, but there are also \w-speaking communities abroad, especially in England. Notably, there is a \w-speaking community in Argentina, where the \w\ established a colony to escape the effects of contact with the English \citep{bell}. Like the other Celtic languages, there are a number of dialects (\cite{jones98, thomas, pembrokeshire}), but there is a significant divide between dialects in the north and south, which have interesting phonological differences.

There is a fairly significant literature on \w\ grammar and phonology. This ranges from traditional grammars like \citet{morris.welsh} and \citet{king} to more modern treatments like \citet{hannahs.book}. There are more specific discussions of the \m\is{mutation} system that include \citet{awbery.mut} and \citet{hannahs.mut} as well as a wonderful set of volumes that includes papers on a number of aspects of the phonology, phonetics, and \m\is{mutation} system: \citet{ballwilliams}, \citet{ballmuller}, and \citet{balljones}. Additional papers of note on other aspects of the sound system of \w\ include
%\citet{unity.hannahs};
%\citet{hannahs};
%\citet{welshsvara};
\citet{hannahs, welshsvara, unity.hannahs, hammond.h, iosad.svara, jones.dist}, and \citet{awbery.tactics}, among others.

The surface consonant \isi{inventory} for \w\ is given in Table~\ref{welsh.consonants.tab}. Again, note the absence of a \isi{palatalization} contrast. Unlike Breton\il{Breton (Modern)} and Cornish\il{Cornish (Modern)}, there are no labio-velars, but it's not clear that this is a substantive difference. Cognate elements exist in \w, but are typically analyzed as a sequence of a velar followed by [w], e.g.\ \emph{cwyno} [kʰwɨno] `complain', \emph{gwin} [gwin] `wine', and \emph{chwarel} [χwarɛl] `quarry'.\footnote{\mhhl{Evidence against this comes from distributional observations. For example, in unmutated\is{mutation} onset position, [χ] is almost always followed by [w]. The only exception is \emph{chi} [χi] `you', historically \emph{chwi} [χwi]. Similarly, while velars can be followed by [w], other segment classes cannot be.}} Some researchers recognize a series of voiceless nasals as well, but I do not represent those here \citep{hammond.h}. Finally, note that the back unrounded glide [ɰ] only occurs in northern dialects.

\begin{table}
\caption{\w\ consonants}
\label{welsh.consonants.tab}
\begin{tabular}[t]{lcccccc}
\lsptoprule
             & Labial & Dental & Alveolar & Pal-Alv  & Dorsal & Uvular/Glottal \\
\midrule
Stop/Affric  & pʰ, b   & tʰ, d   &          & t͡ʃʰ, d͡ʒ & kʰ, g \\
Fricative    & f, v    & θ, ð    & s, ɬ      & ʃ        &        & χ \\
Nasal        & m      & n      &          &          & ŋ  \\
Approx       & w      &        & l, r, r̥    & j        & ɰ      & h \\
\lspbottomrule
\end{tabular}
\end{table}

I note two interesting phonological processes. First, as in Cornish\il{Cornish (Modern)}, there are final sonority-reversed clusters that are resolved by \isi{epenthesis}. Unlike in Cornish\il{Cornish (Modern)}, however, there are two additional restrictions. First, per Hannahs (\citeyear{welshsvara}, \citeyear{unity.hannahs}), \isi{epenthesis} is at least partially conditioned by stress. Sonority-reversed clusters in words of one syllable undergo \isi{epenthesis}, e.g.\ \emph{llyfr} [ˈɬɨvɨr] `book' and \emph{pobl} [ˈpʰɔbɔl] `people'. However, in longer words where the cluster appears in a stressless syllable, deletion of the final consonant is a more typical strategy, e.g.\ \emph{ffenestr} [ˈfɛnɛst] `window' and \emph{posibl} [ˈpʰɔsib] `possible'. Furthermore, the final clusters [dl] and [vn] do not seem to undergo either \isi{epenthesis} or deletion, e.g.\ \emph{cenedl} [ˈkʰɛnɛdl] `nation' and \emph{ofn} [ɔvn] `fear', as noted in \citet{hannahs.book} and \citet{iosad.svara}.

It's interesting to compare this pattern with Cornish\il{Cornish (Modern)}. There, the cognate for \w\ \emph{cenedl} does undergo \isi{epenthesis}: \emph{cenedhel}. However, the cognate form for \w\ \emph{ofn} is Cornish\il{Cornish (Modern)} \mhhl{\emph{own}} [ɔʊn], where the consonant (pronounced in \w\ as [v]) is lost. The consonant [v] is also lost in \w, but at the ends of words, as in \emph{haf} [ha(v)] `summer', \emph{cyntaf} [kʰəntʰa(v)] `first', etc. Thus, the absence of \isi{epenthesis} \cut{is} \mhhl{in} \w\ forms like \emph{ofn} may be due to the precarious status of [v].

Another prominent process in \w\ involves the deletion of a medial [h] when it precedes a stressless vowel \citep{hammond.h}. This process also affects [r̥], with its voiced counterpart [r] occurring where [h] would be deleted. Since, as in Cornish\il{Cornish (Modern)}, stress moves under suffixation, this again gives rise to alternations, e.g.\ \emph{dihareb} [diˈharɛb] `proverb' vs.\ \emph{diarhebion} [diaˈr̥ɛbjɔn] `proverbs', \emph{eang} [ˈeaŋ] `wide' vs.\ \emph{ehangach} [eˈhaŋaχ] `wider'. The \emph{dihareb} form is particularly nice as it shows both the h/$\emptyset$ and the r/r̥ alternations.

Historically, medial nasal--voiceless stop sequences were replaced with nasal--[h] sequences; these exhibit alternations as well:
\emph{cenedl} [ˈkʰɛnɛdl] `nation' vs.\ \emph{cenhedloedd} [kʰɛnˈhɛdlɔ(ɰ)ð] `nations',
\emph{angen} [ˈaŋɛn] `need' vs.\ \emph{anghenion} [aŋˈhɛnjɔn] `needs',
\emph{tymor} [ˈtʰəmɔr] `season' vs.\ \emph{tymhorau} [tʰəmˈhɔra(ɰ)] `seasons'.

The surface \isi{inventory} of \w\ vowels is given in Table~\ref{welsh.vowels.tab}. Note that the high central unrounded vowels only occur in the north.

\begin{table}
\caption{\w\ vowels}
\label{welsh.vowels.tab}
\begin{tabular}[t]{lccc}
\lsptoprule
         & Front & Central & Back \\
\midrule
High     & iː, ɪ  & ɨː, ɨ    & uː, ʊ \\
Mid      & eː, ɛ  & ə       & oː, ɔ \\
Low      &                 & aː, a  \\
\lspbottomrule
\end{tabular}
\end{table}

Vowel length is only contrastive in monosyllables and then only before certain consonants. Historically, this only included [l, r, n], but this has been extended in recent borrowings where long vowels may now occur where previously only short ones did, e.g.\ \emph{str\^oc} [stroːk] `stroke, shock', \emph{c\^ot} [kʰoːt] `coat', \emph{gr\^wp} [gruːp] `group', \emph{cl\`os} [kʰloːs] `close', etc.\footnote{The distribution of vowel length in \w\ is discussed in \citet{awbery.tactics}, \citet{hannahs.book}, \citet{iosad.vowels}, and \citet{bell}.}

The \w\ \m s\is{mutation} appear in Table~\ref{welsh.mutations.tab}. Note that the system is not as complex as in Breton\il{Breton (Modern)} and Cornish\il{Cornish (Modern)}, but still more complex than in any of the Goidelic languages. As in the other Brythonic languages, the relationships are fairly transparent phonologically. Interestingly, similar to \isi{eclipsis} in the Scottish Gaelic\il{Scottish Gaelic (Modern)} dialects cited above, the nasal \m\is{mutation} can be treated phonologically, triggered only by elements that end in an overt nasal \citep{hammond.h}.

\begin{table}
\caption{\w\ \m s}
\label{welsh.mutations.tab}
\begin{tabular}[t]{cccc}
\lsptoprule
       & \multicolumn{3}{c}{Mutation\is{mutation} Type} \\\cmidrule(lr){2-4}
C      & Soft        & Nasal & Aspirate \\
\midrule
p      & b           & mh    & f \\
t      & d           & nh    & θ \\
k      & g           & ŋh    & χ \\
b      & v           & m \\
d      & ð           & n \\
g      & $\emptyset$ & ŋ \\
ɬ      & l \\
\mhhl{r̥} & r \\
m      & v \\
\lspbottomrule
\end{tabular}
\end{table}

\mhhl{Finally, the stress system of \w\ deserves mention. Stress falls generally on the penult. There are a fair number of words with exceptional final stress and a very few items with antepenultimate stress. What's striking about the \w\ system is that English speakers routinely intuit stress as falling on the final syllable. There are two reasons for this. First, \w\ does not have stress-related \isi{vowel reduction} so final stressless syllables are routinely not reduced\is{vowel reduction}. Second, as noted by \citet{williams} and \citet{prosody}, the intonational system can, in some cases, give conflicting cues so that the final syllable surfaces with a fairly high pitch even when it isn't stressed.}

\il{Welsh (Modern)|)}

\section{General discussion}

%Let's now try to synthesize the systems into a coherent whole. 
Table~\ref{summary.tab} provides a broad summary of interesting differences that emerge across the Celtic languages with respect to the phenomena discussed in this chapter.

\begin{table}
\caption{Summary of differences}
\label{summary.tab}
\small
\begin{tabular}[t]{rcccccc}
\lsptoprule
            & \multicolumn{3}{c}{Goidelic}  &  \multicolumn{3}{c}{Brythonic} \\\cmidrule(lr){2-4}\cmidrule(lr){5-7}
             & Irish\il{Irish (Modern)}           & Manx\il{Manx Gaelic (Modern)}     & SG         & Breton\il{Breton (Modern)}    & Cornish\il{Cornish (Modern)}    & Welsh\il{Welsh (Modern)} \\
\midrule
Number of Mutations\is{mutation}   & 2             & 1--2     & 2          & 4      & 4      & 3 \\
Transparent Mutation\is{mutation}  & no            & no      & no         & yes    & yes    & yes \\
\midrule
V Reduction\is{vowel reduction}   & yes           & yes     & yes        & no     & no     & no \\
Stress Placement     & initial${}^*$ & initial & initial    & penult & penult & penult \\
\midrule
Grammatical Case  & yes           & ?       & yes        & no     & no     & no \\
Palatalization\is{palatalization} Contrast         & yes           & yes     & yes        & no     & no     & no \\
\lspbottomrule
\end{tabular}
\end{table}

The first row of Table~\ref{summary.tab} gives the number of \m s\is{mutation}, showing that generally there are more of them in Brythonic than in Goidelic languages. The second row gives a crude characterization of the transparency of the \m s\is{mutation}, where the \m s\is{mutation} of the Goidelic languages can be seen as generally less transparent than those of the Brythonic languages. And, though it's not shown in Table~\ref{summary.tab}, the Goidelic \m s\is{mutation} also affect more consonants than those in the Brythonic languages. The next two rows show that the Goidelic languages generally exhibit \isi{vowel reduction} and initial stress, whereas the Brythonic languages exhibit penult stress with no or less \isi{vowel reduction}. Finally, the last two rows show that \mhhl{overt} grammatical case occurs in the Goidelic languages along with a \isi{palatalization} contrast, while neither occur in the Brythonic languages.

Additionally, there are other commonalities described in this chapter that do not follow the Goidelic/Brythonic divide. For example, in all the languages, \m\is{mutation} is triggered by morphosyntactic environments. However, I also noted several cases of `phonological \m\is{mutation}', namely the \isi{eclipsis} \m\is{mutation} in Scottish Gaelic\il{Scottish Gaelic (Modern)} and the nasal \m\is{mutation} in Welsh\il{Welsh (Modern)}. Finally, there is pre-stopping in virtually the same environment in two of the languages, Manx\il{Manx Gaelic (Modern)} and Cornish\il{Cornish (Modern)}.

Taking a closer look at the differences summarized in Table~\ref{summary.tab}, one can observe that the phonology of the \m s\is{mutation} is somewhat simpler in the Brythonic languages (as measured by transparency) than in the Goidelic languages~-- both in terms of the number of consonants affected and in terms of the mappings that occur. It would seem reasonable to suppose that this is balanced against the number of \m s\is{mutation} in the Brythonic languages.

Another observation to ponder is that all the Goidelic languages exhibit a \isi{palatalization} contrast whereas the Brythonic languages do not. One could ask if the absence of overt case in the Brythonic languages is connected to this, because case in the Goidelic languages is at least partially marked by \isi{palatalization}.

Another potential point of covariance is \isi{vowel reduction} and \isi{palatalization}. The Goidelic languages exhibit both a \isi{palatalization} contrast and \isi{vowel reduction} whereas the Brythonic languages exhibit neither. Perhaps the way to look at these is a shifting of contrast from vowels to consonants in the Goidelic languages.

Of course, these observations are just hypotheses at this stage. To truly test them,  these properties could be examined in the context of historical change and language acquisition. For example, if these variables are connected as speculated above, I would expect them to arise diachronically in close succession. For example, is there evidence that \isi{vowel reduction} arises when the stress systems separate? Similarly, I would expect to see these connections recapitulated in the timing of acquisition. For example, does the acquisition of the case system in the Goidelic languages correlate with the acquisition of the \isi{palatalization} contrast?

In sum, these languages exhibit a range of interesting phonological properties. It is reasonable to see the variation across the systems as different ways of balancing phonological and morphological requirements and options. I've laid out several such possible connections in this chapter and propose that they can be tested with further study in the historical and language acquisition domains.

\section*{Acknowledgments}

Thanks to 
Elise Bell,
Ryan Bennett,
Andrew Carnie,
S.J.\ Hannahs,
Diane Ohala,
and two anonymous reviewers
for useful discussion.
All errors are my own.

\is{phonology|)}

\printbibliography[heading=subbibliography,notkeyword=this]

\end{document}
