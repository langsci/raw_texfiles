\documentclass[output=paper,colorlinks,citecolor=brown]{langscibook}
\ChapterDOI{10.5281/zenodo.15654871}
\author{Ryan Bennett\affiliation{University of California, Santa Cruz} and
        Jaye Padgett\affiliation{University of California, Santa Cruz} and
        Grant McGuire\affiliation{University of California, Santa Cruz} and
        Máire Ní Chiosáin\affiliation{University College Dublin} and 
        Jennifer Bellik\affiliation{University of California, Santa Cruz} 
        }
\title{Gestural timing and contrast: An Irish case study}
\abstract{Though a palatalized vs. velarized (\emph{slender} vs. \emph{broad}) contrast is fundamental to Goidelic languages of the Celtic family, contrasts involving either palatalization or velarization are typologically uncommon. Across languages, such contrasts are often dispreferred in onset compared to coda position, and in labial compared to coronal consonants. Previous phonetic research on both Russian and Irish has suggested that these asymmetries in contrast loss might be explained by, or at least correlated with, asymmetries in articulatory reduction and timing. We explore the timing of palatalization and velarization gestures in Irish stop consonants, and find differences between onset vs. coda position consistent with this general claim. However, we also find that gestural timing is structured in ways that allow for perceptual cues to the contrast to be recoverable, including in ways that might facilitate perception of the contrast in codas and (possibly) in labial consonants. Put simply, gestural timing in Irish may be optimized to support vulnerable contrasts.}


\IfFileExists{../localcommands.tex}{
   % add all extra packages you need to load to this file

\usepackage{tabularx,multicol}
\usepackage{url}
\urlstyle{same}

\usepackage{listings}
\lstset{basicstyle=\ttfamily,tabsize=2,breaklines=true}

\usepackage{langsci-basic}
\usepackage{langsci-optional}
\usepackage{langsci-lgr}
\usepackage{langsci-osl}
% \usepackage{./langsci/styles/langsci-lgr}
% \usepackage{./langsci/styles/langsci-osl}
% \usepackage{langsci-gb4e}

\usepackage{tikz}
\usetikzlibrary{patterns,calc}
\pgfdeclarepatternformonly{south east lines}{\pgfqpoint{-0pt}{-0pt}}{\pgfqpoint{3pt}{3pt}}{\pgfqpoint{3pt}{3pt}}{
    \pgfsetlinewidth{0.6pt}
    \pgfpathmoveto{\pgfqpoint{0pt}{3pt}}
    \pgfpathlineto{\pgfqpoint{3pt}{0pt}}
    \pgfpathmoveto{\pgfqpoint{.2pt}{-.2pt}}
    \pgfpathlineto{\pgfqpoint{-.2pt}{.2pt}}
    \pgfpathmoveto{\pgfqpoint{3.2pt}{2.8pt}}
    \pgfpathlineto{\pgfqpoint{2.8pt}{3.2pt}}
    \pgfusepath{stroke}}
    
\usepackage{stmaryrd}
\usepackage{wasysym}
\usepackage{multirow}
\usepackage{caption}
\usepackage{subcaption}
\usepackage{mathrsfs}
\usepackage{qtree}

\usepackage{linguex}


   %pminos do not split footnotes
% \interfootnotelinepenalty=10000 %Footnote in Laporte chapters has to be split SN


%\DeclareIndexNameFormat{default}{%
%\nameparts{#1}%
%\usebibmacro{index:name}%
%{\index[names]}%
%{\namepartfamily}%
%{\namepartgiveni}%
% {}% L1
% {}% L2
%{\namepartprefix}% generates spurious space L3
%{\namepartsuffix}% generates spurious space L4
%}

%  {\DeclareIndexNameFormat{default}{%
%     \usebibmacro{index:name}{\index[names]}{#1}{#3}{#5}{#7}}}

%\DeclareIndexNameFormat{default}{%
%  \usebibmacro{index:name}{\sindex[nom]}{#1}{#3}{#5}{#7}}

%\DeclareIndexNameFormat{default}{%
%  \usebibmacro{index:name}{\sindex[person]}{#1}{#3}{#5}{#7}}
%\DeclareIndexNameFormat{default}{%
%\nameparts{#1} \usebibmacro{index:name}{\sindex[person]]}{\namepartfamily}{‌​\namepartgiven}{\nam‌​epartprefix}{\namepa‌​rtsuffix}}

%\newcommand{\smiley}{:)}

%\renewbibmacro*{index:name}[5]{%
%\usebibmacro{index:entry}{#1}%
%{\iffieldundef{usera}{}{\thefield{usera}\actualoperator}\mkbibindexname{#2}{#3}{#4}{#5}}}

% \newcommand{\noop}[1]{}

%remove for final
%\overfullrule=1mm

\newcommand{\tobi}[2]}}
\renewcommand{\S}[1]{\tobi{#1}{\textsc{*}}}

% this volume references
% puts: [this volume]
% already defined: \citetv
%\newcommand{\citepv}[1]{(\citeauthor{#1} \citeyear*{#1} [this volume])}
\newcommand{\citealtv}[1]{\citeauthor{#1} \citeyear*{#1} [this volume]}

%parentheses around example number
\newcommand{\pref}[1]{(\ref{#1})}

% in-text examples

\newcommand{\lnex}[1]{\textit{#1}} %target lang word
\newcommand{\lnlit}[1]{(lit.: `#1')} %literal reading
\newcommand{\lnlat}[1]{(#1)} % latinization
\newcommand{\lntrans}[1]{`#1'} %translation
\newcommand{\lnexl}[2]%
{\lnex{#1}{} \lnlat{#2}} % ex with latinization
\newcommand{\lnexlat}[3]{\lnex{#1}{} \lnlat{#2}{} \lntrans{#3}} % ex with latinization and tranl.

%ch01
\newcommand{\co}[1]{\mbox{\textbf{#1}}}

%ch09

\newcommand{\cyrbulg}[1]{\begin{otherlanguage*}{bulgarian}#1\end{otherlanguage*}}


%ch10
\newcommand{\nlp}{{\small NLP}}
\newcommand{\mwe}{{\small MWE}}
\newcommand{\rae}{{\small RAE}}
\newcommand{\lvc}{{\small LVC}}
\newcommand{\pos}{{\small P}o{\small S}}
%\newcommand{\todo}[1]{ \textcolor{red}{#1} }

%\renewcommand{\labelenumi}{\theenumi}
%\ainamefmt{{vv}{ll}{, ff}{, jj}} % fullname

\newcommand{\biberror}[1]{{\color{red}#1}}

\newcommand{\osenovaitem}{--~}
   %% hyphenation points for line breaks
%% Normally, automatic hyphenation in LaTeX is very good
%% If a word is mis-hyphenated, add it to this file
%%
%% add information to TeX file before \begin{document} with:
%% %% hyphenation points for line breaks
%% Normally, automatic hyphenation in LaTeX is very good
%% If a word is mis-hyphenated, add it to this file
%%
%% add information to TeX file before \begin{document} with:
%% %% hyphenation points for line breaks
%% Normally, automatic hyphenation in LaTeX is very good
%% If a word is mis-hyphenated, add it to this file
%%
%% add information to TeX file before \begin{document} with:
%% \include{localhyphenation}
\hyphenation{
    Beck-man
    Ngu-yen
    back-chan-nel
    back-chan-nels
    mo-not-o-nous
    ste-reo-typ-i-cal
}

\hyphenation{
    Beck-man
    Ngu-yen
    back-chan-nel
    back-chan-nels
    mo-not-o-nous
    ste-reo-typ-i-cal
}

\hyphenation{
    Beck-man
    Ngu-yen
    back-chan-nel
    back-chan-nels
    mo-not-o-nous
    ste-reo-typ-i-cal
}

   \boolfalse{bookcompile}
   \togglepaper[8]%%chapternumber
}{}

\newfontfamily\ipafont{Charis SIL}
%\newcommand{\ipa}[1]{{\ipafont #1}}
\newcommand{\pal}{\ipa{ʲ}}
\newcommand{\vel}{\ipa{ˠ}}
\newcommand{\velop}{\textsuperscript{(}\vel\textsuperscript{)}}
\newcommand{\palcongen}{/C\velop\ C\pal/}
\newcommand{\palvelcon}{/C\vel\ C\pal/}

% Formatting commands
% \newcommand{\flag}[1]{{\textsc{\color{lsMidDarkBlue}{#1}}}}
\newcommand{\flag}[1]{{\textsc{\large\color{red}{#1}}}}

\begin{document}
\maketitle

\is{broad consonant|see {velarization}}
\is{slender consonant|see {palatalization}}
\is{palatalization|(}
\is{velarization|(}

\section{Introduction}
The living Celtic languages are divided into two groups, a Goidelic branch (including Irish\il{Irish (Modern)} and Scottish Gaelic\il{Scottish Gaelic (Modern)}) and a Brittonic branch (including Welsh\il{Welsh (Modern)} and Breton\il{Breton (Modern)}).\footnote{For a recent discussion of the classification of Celtic languages, see \citet{Eska2017_Celtic_phono}.} In terms of phonological properties, an important feature distinguishing Goidelic languages from Brittonic ones is a phonemic contrast\is{contrast} between pala\-talized and non-palatalized consonants, traditionally called \textit{slender\is{release}} and \textit{broad} respectively. For example, compare the Irish\il{Irish (Modern)} minimal pair \textit{brád} \ipa{/bˠrˠɔːdˠ/} 'drizzle' and \textit{bráid} \ipa{/bˠrˠɔːdʲ/} 'neck, throat'; the two words differ only in whether the final consonant is palatalized or velarized. (These terms are explained below.) This palatalization contrast\is{contrast} developed in Proto/Old Irish\il{Old Irish} (\cite{Green1973_Old_Irish_pal}; \cite{mccone:1996}; \cite{Stifter2017_Celtic_phono}), and it occurs also in Scottish Gaelic\il{Scottish Gaelic (Modern)} (\cite{ternes:2006}; \cite{Nance_OMaolalaigh2021_Sottish_Gaelic_JIPA}) and Manx\il{Manx Gaelic (Modern)} \citep{Lewin:2019}.

Many phonemic contrasts\is{contrast} are allowed only in certain phonological environments. For example, though English distinguishes voiced obstruents from voiceless ones both word\slash syllable\hyp initially and -finally (e.g., \textit{back}
vs. \textit{pack} and \textit{cab} vs. \textit{cap}, Russian prohibits the contrast\is{contrast} finally, neutralizing\is{neutralization (positional)} in favor of the voiceless series: compare \ipa{[ˈlˠugˠə]} `meadow (\textsc{gen.})' and \ipa{[ˈlˠukˠə]} `onion (\textsc{gen.})' (from \ipa{/lˠugˠ-a/} and \ipa{/lˠukˠ-a/}, respectively) to \ipa{[lˠukˠ]} `meadow' and \ipa{[lˠukˠ]} `onion' (from \ipa{/lˠugˠ/} and \ipa{/lˠukˠ/}, respectively).  

It has long been understood that phonological alternations, including those leading to the positional neutralization\is{neutralization (positional)} of contrast\is{contrast}, might have historical origins in phonetic processes, a hypothesis pursued most notably by Ohala (e.g. \cite{Ohala1981_listener_sound_change}; \citeyear{Ohala1993_phonetics_sound_change}), but since explored by many. One way to test a hypothesis about the phonetic origins of positional contrast\is{contrast} and neutralization\is{neutralization (positional)} is to examine a language that does \textit{not} neutralize\is{neutralization (positional)} the relevant contrast\is{contrast}, and look for potential phonetic seeds of neutralization\is{neutralization (positional)} there. Indeed, we have done this for Irish\il{Irish (Modern)} (see discussion further on). However, there is also much work showing that languages can be structured in ways that militate \emph{against} contrast\is{contrast} loss, a view most notably pursued by Lindblom (\citeyear{Lindblom1986_V_system_universals}; \citeyear{Lindblom1990_HH_theory}), but again explored since then by many. 

In this paper, we explore details of the timing of palatalization and velarization\is{velarization} gestures\is{gesture} in Irish\il{Irish (Modern)} stop consonants as a function of prosodic position (word\slash syllable\hyp initial vs. -final) and place of articulation (labial, coronal, dorsal). We find that timing patterns vary more in word\slash syllable\hyp final position, a fact that may help explain the vulnerability of this contrast\is{contrast} to loss in that position. However, we also find evidence that timing in that position can be structured in ways that might \textit{facilitate} perception of the palatalization contrast\is{contrast}. As for place of articulation, we find that patterns of timing in labial consonants, which are also vulnerable to loss of a palatalization contrast\is{contrast}, are more consistent than for other places of articulation. This may likewise facilitate perception of the palatalization contrast\is{contrast}, though here the facts are less conclusive. Overall, our results suggest that gestural timing\is{gestural timing} in Irish\il{Irish (Modern)} may be optimized to support contrasts\is{contrast} that are vulnerable. 

%%%%%%%%%%%%%%%%%%%%%%%%%%
\section{Irish and the typology of contrastive palatalization}

%%%%%%%%%%%%%
\subsection{Palatalization and velarization}
\is{velarization|(}
Palatalized (slender\is{release} consonants [C\pal] are produced with a secondary high-front tongue constriction, akin to the constriction for a palatal vowel \ipa{[i]} or glide \ipa{[j]}, which is roughly simultaneous with the primary constriction for that consonant.\footnote{Since not all languages with a palatalization contrast\is{contrast} are reported to oppose velarization to palatalization, we use `\palcongen' to refer to palatalization contrasts\is{contrast} generally. We use capital `C' to refer to any consonant, and capital `P', `T', `K' to refer generically to labial, coronal, and dorsal consonants, respectively. Slash brackets `/ /' indicate abstract phonemes, while square brackets `[ ]' indicate surface phonetic forms. Transcriptions are broad: for example, coronal \ipa{[d\vel\ d\pal]} are phonetically closer to dentalized \ipa{[d̪\vel]} and retracted \ipa{[d̠\pal]} in e.g. Gaoth Dobhair Irish\il{Irish (Modern)} \citep{NiChasaide1995_Irish}.} Velarized (broad) consonants [C\vel] are produced with a secondary tongue body constriction that is high and back in the mouth, as for \ipa{[ɯ u]} or \ipa{[ɰ w]}, or sometimes in the mid-back or uvular region, as for \ipa{[ɤ o]} (\cite{IPA_handbook1999}; \cite{Kochetov2002_diss}; \cite{Bennett_etal2018_Conamara_ultrasound}; \cite{Shaw_etal2021_complex_segments}). In this article, the term `secondary articulation' encompasses palatalization and velarization as defined above.
\is{velarization|)}
%%%%%%%%%%%%%
\subsection{The palatalization contrast in Irish}
All consonants in Irish\il{Irish (Modern)} are contrastively palatalized or velarized. \tabref{tab:Irish-C} illustrates these phonemic contrasts\is{contrast} with the basic inventory of consonants in Connacht Irish\il{Irish (Modern)} (or more specifically Connemara Irish\il{Irish (Modern)}), which here represents the western dialect group (\cite{DeBhaldraithe1975_Cois_Fhairrge}; \cite{ModernIrish}; \cite{NiChiosain_1991_diss}; \citeyear{NiChiosain1994_Irish_place_features}; \citeyear{NiChiosain1999_Irish_phonotactics}; \cite{NiChiosain_Padgett2012_Irish_pal_acous_percep}). The specific phonetic form of palatalized and velarized consonants varies somewhat across dialects (e.g. \cite{NiChasaide1995_Irish}; \cite{Hickey2011_Irish_dialects}), but all varieties of Irish\il{Irish (Modern)} share fundamentally the same inventory of phonological contrasts\is{contrast} illustrated in \tabref{tab:Irish-C}.

\begin{table}
\caption{Phonemic consonant inventory of Connemara Irish\il{Irish (Modern)}. Sounds in parentheses have highly restricted distributions.}
\label{tab:Irish-C}
\begin{tabularx}{.65\textwidth}{lcccc}
\lsptoprule
& Labial & Coronal & Dorsal & Glottal\\\midrule
Stop & p\vel\ p\pal & t\vel\ t\pal & k\vel\ k\pal\\
&b\vel\ b\pal&d\vel\ d\pal&\ipa{g}\vel\ \ipa{g}\pal\\
Fricative & f\vel\ f\pal & s\vel\ s\pal & x\vel\ x\pal&h\vel\ (h\pal)\\
&v\vel\ v\pal& & (\ipa{ɣ}\vel) (\ipa{ɣ}\pal)&\\
Nasal & m\vel\ m\pal & n\vel\ n\pal & \ipa{ŋ}\vel\ \ipa{ŋ}\pal\\
Liquid & & l\vel\ l\pal & & \\
& & r\vel\ r\pal& &\\ 
\lspbottomrule
\end{tabularx}
\end{table}

As illustrated in \tabref{tab:Irish-C}, \palvelcon\ contrasts\is{contrast} are maintained across all places of articulation in Irish\il{Irish (Modern)}. In addition, /C\pal/\ and /C\vel/\ can appear across all vowel contexts as well as word\slash syllable\hyp initially and -finally (some examples given in \tabref{tab:Irish-Words}, left). In the data analyzed in this paper, word-initial sounds are also syllable-initial (and vice-versa), and word\hyp final sounds are also syllable\hyp final (and vice versa). For ease of reference we will often use `onset' to refer to the former and `coda' to refer to the latter.\footnote{Consonant clusters typically agree in their secondary articulations in Irish\il{Irish (Modern)} (\cite{NiChiosain_1991_diss}; \citeyear{NiChiosain1999_Irish_phonotactics}). Such clusters may be palatalized (e.g. \emph{coirpeach} \ipa{[k\vel ir\pal p\pal əx\vel]} `criminal') or velarized (e.g. \emph{l\'eargas} \ipa{[l\pal eːr\vel g\vel əs]} `insight'). So, although word-internal codas may not have \emph{independent} \palvelcon\ specifications, given that secondary articulations must assimilate in clusters, word-internal codas may be either palatalized or velarized just like word\hyp final codas. (Compounds and prefixes allow disagreeing clusters, e.g. \emph{inphosta} \ipa{[in\pal-f\vel oːs\vel t\vel ə]} `marriable', \emph{fadt\'earma} \ipa{[f\vel ad\vel-t\pal eːr\vel m\vel ə]} `longterm'; \citealt[Ch.2]{NiChiosain_1991_diss}.\label{fn:codas})

The backness of short vowels is predictably determined by the secondary articulations of neighboring consonants, which is why we focus on long vowels in our discussion of vocalic contexts. See \sectref{sec:items}, \citet{NiChiosain_1991_diss}, and \citet{ModernIrish} for details.}

\begin{table}
\caption{Words showing distribution of \palvelcon\ (left) and minimal pairs (right).}
\label{tab:Irish-Words}
\begin{tabularx}{.85\textwidth}{llllll}
\lsptoprule
\emph{caoin}     & \ipa{[k\vel iːn\pal]} & `lament'                    & \emph{p\'an} & \ipa{[p\vel ɔːn\vel]} & `pawnshop' \\
\emph{ti\'us}    & \ipa{[t\pal uːs\vel]} & `thickness'                 & \emph{peann} & \ipa{[p\pal ɔːn\vel]} & `pen' \\
\emph{l\'aib}    & \ipa{[l\vel ɔːb\pal]} & `mud, muck'                 & \emph{cat}   & \ipa{[k\vel at\vel]}  & `cat (\textsc{sg.})' \\
\emph{p\'{\i}ob} & \ipa{[p\pal iːb\vel]} & `pipe ({\textsc{gen.pl.}})' & \emph{cait}  & \ipa{[k\vel at\pal]}  & `cat (\textsc{pl.})' \\
\lspbottomrule
\end{tabularx}
\end{table}

%\begin{exe}
%\ex\label{secart-ex}
%\begin{xlist}
%\ex\emph{caoin} \ipa{[k\vel iːn\pal]} `lament'
%\ex\emph{ti\'us} \ipa{[t\pal uːs\vel]} `thickness'
%\ex\emph{l\'aib} \ipa{[l\vel ɔːb\pal]}  `mud, muck'
%\ex\emph{p\'{\i}ob} \ipa{[p\pal iːb\vel]} `pipe ({\textsc{gen.pl.})'
%\end{xlist}
%\end{exe}

The wide distribution of the \palvelcon\ contrast\is{contrast} across all of these contexts allows for many minimal pairs (Table~\ref{tab:Irish-Words}, right). It is also what makes the Irish\il{Irish (Modern)} language a useful and important testing ground for phonetically-oriented explanations of recurrent distributional restrictions on \palcongen\ contrasts\is{contrast} across languages.


\subsection{Typological asymmetries in palatalization contrasts}
Though allophonic secondary palatalization is found in many languages \citep{Bateman2011_pal_typology}, its \textit{contrastive} use, as in Irish \ipa{/bˠrˠɔːdˠ/} vs. \ipa{/bˠrˠɔːdʲ/}, is relatively uncommon. Furthermore, among languages that do distinguish \palcongen, the contrast\is{contrast} is not equally likely in all phonological environments (\cite{Takatori1997_Slavic_phono_PhD}; \cite{Kochetov2002_diss}; \cite{Iskarous_Kavitskaya2018_Slavic_palatalization}). First, such contrasts\is{contrast} tend to be 
neutralized\is{neutralization (positional)} or lost diachronically in codas, as in Bulgarian\il{Bulgarian} (e.g. \cite{Scatton1993_Bulgarian_phono}; \cite{Iskarous_Kavitskaya2018_Slavic_palatalization}) or Nenets \citep[e.g.][]{Kochetov2002_diss}. As a result, the presence of a \palcongen contrast\is{contrast} in coda position normally entails a corresponding contrast\is{contrast} in onset position (but not vice-versa).\footnote{Marginal exceptions to this typological generalization include Estonian and Romanian, in which phonological palatalization only occurs word\hyp finally or before a consonant (i.e. only in coda position). These unusual distributions owe to contextual palatalization before \ipa{/i/}, rendered opaque by historical or synchronic vowel deletion (\cite{Operstein2010_C_prevocalization}; \cite{Chitoran2013_Romanian_phono}; \cite{Malmi_etal2023_Estonian_pal}).} Second, \palcongen contrasts\is{contrast} tend to be neutralized\is{neutralization (positional)} or lost diachronically in labial consonants compared to coronals. For example, this pattern occurs in Czech (\cite{Short1993_Czech_phono}; \cite{Iskarous_Kavitskaya2018_Slavic_palatalization}). These two restrictions may also interact, as in Belarusian, where palatalization is contrastive for both labials and coronals in syllable onsets, but palatalized labials do not occur in syllable codas (e.g. \citealt{Bird_Litvin2021_Belarusian_JIPA}).

Some signs of these asymmetries are evident within Celtic. Scottish Gaelic\il{Scottish Gaelic (Modern)} may have lost, or be in the process of losing, its palatalization contrast\is{contrast} among labial consonants, with remnants in the form of /Pj/ (labial consonant~-- glide) sequences (\cite{Jackson1967_Gaelic_pal_lab}; \cite{deBurca1977_syllabicity_pal}; \cite{ternes:2006}; cf. \cite{Nance_OMaolalaigh2021_Sottish_Gaelic_JIPA}), as is true in some Polish dialects (\cite{Czaplicki2010_Polish_pal_labial}; \cite{Iskarous_Kavitskaya2018_Slavic_palatalization}). The same may have been true of Classical Manx\il{Manx Gaelic (Modern)} \citep{Lewin:2019}. In Irish\il{Irish (Modern)}, there are many fewer words ending in palatalized labial stops compared to palatalized coronal stops, and the proportion of word\hyp final labial stops that are palatalized is low compared to the proportion of word\hyp final coronal stops that are palatalized, based on either a dictionary or corpus search of word types \citep{NiChiosain_Padgett2012_Irish_pal_acous_percep}. As \citet{Kochetov2002_diss} notes, discussing similar facts for Russian, such asymmetries in frequency may reflect lexical attrition of contrasts\is{contrast} in more vulnerable positions. They might, in turn, also enable further attrition of contrasts\is{contrast} and possibly lead to neutralization\is{neutralization (positional)}, because language learners will have to learn the contrast\is{contrast} based on fewer relevant lexical items.

In spite of such asymmetries in segment frequency, Irish\il{Irish (Modern)} still maintains the palatalization contrast\is{contrast} in codas and in labials. For this reason, Irish\il{Irish (Modern)} is an ideal testing ground for hypotheses about the phonetic bases of contrast\is{contrast} and contrast\is{contrast} neutralization\is{neutralization (positional)}. For example, if contrasts\is{contrast} are lost in certain contexts due to systematic phonetic asymmetries, then we might expect to find such phonetic asymmetries in Irish\il{Irish (Modern)}, even though it maintains the contrast\is{contrast}. The hunt for phonetic precursors to neutralization\is{neutralization (positional)} of a \palcongen\ contrast\is{contrast} has precedents, including work of our own (\cite{Kochetov2002_diss}; \citeyear{Kochetov:2005}; \citeyear{Kochetov2006_syll_poss}; \cite{NiChiosain_Padgett2012_Irish_pal_acous_percep}; \cite{Stoll:2017}; \cite{Padgett_NiChiosain2018_Russian_Irish_pal_percep}; \cite{Iskarous_Kavitskaya2018_Slavic_palatalization}; \cite{Kirkham_Nance2022_diachronic_contrast_Gaelic_pal_son}; \cite{Padgett_etal2023_Irish_pal_syllpos}; \cite{Bennett_etal2023_jphon_submission}). On the other hand, if languages are structured in ways that \emph{support} contrast\is{contrast}, then we might expect to find ways in which this is true for the Irish\il{Irish (Modern)} palatalization contrast\is{contrast}. The work presented here explores both of these possibilities, focusing on the timing of palatalization and velarization with respect to a primary place constriction. It develops ideas first presented by \citet{Padgett_etal2023_Irish_pal_syllpos}. Here we bring new analyses to bear, including analyses of the gestural trajectories of individual tokens and an exploration of individual subject behavior. We develop the idea of perceptual optimization, and we consider and reject an alternative understanding of gestural timing\is{gestural timing} that relies on articulatory principles along the lines of \citet{Sproat_Fujimura1993_l-allophony} and \citet{Krakow1999_physiological_syllables}.


%%%%%%%%%%%%%%%%%%%%%%%%%%
\is{gestural timing|(}
\section{Gestural timing and contrast}
\subsection{Gestural organization and perception}\label{sec:timing&contrast}
Perceptual studies on Irish\il{Irish (Modern)} and Russian have found that listeners have more difficulty identifying or distinguishing \palvelcon\ in coda position compared to onset position, and for Russian, in labials compared to coronals. \citeauthor{Kochetov2002_diss}  (\citeyear{Kochetov2002_diss}; \citeyear{Kochetov2004_perception_place}) found that Russian listeners misidentified \ipa{/pˠ pʲ tˠ tʲ/} more often in word\slash syllable\hyp final position than in initial position, even though the relevant consonant was intervocalic due to a following vowel-initial word.\footnote{There is no evidence that consonants resyllabify across words in Russian. For example, word\hyp final consonants devoice even when vowel-initial words follow \citep[see discussion in][]{Padgett_Myers2014_domain_generalization}. See also the discussion in \sectref{sec:items}.} In addition, listeners misidentified \ipa{/pʲ/} more often than the other three consonants in coda position, most often confusing it with \ipa{/pˠ/}. \citet{NiChiosain_Padgett2012_Irish_pal_acous_percep},  \citet{Padgett_NiChiosain2018_Russian_Irish_pal_percep} found that Irish\il{Irish (Modern)} listeners discriminate \palvelcon\ more poorly in coda position compared to onset position, for a number of stop and fricative consonants. (The codas in these stimuli were phrase-final.) As can be seen, these perceptual asymmetries mirror the typological ones discussed earlier, raising the possibility that the perceptual asymmetries \emph{explain} the typology. In contexts where listeners cannot perceive the \palvelcon\ contrast\is{contrast} well, we might expect attrition and eventual loss of that contrast\is{contrast}. We might also expect such perceptual asymmetries to have some basis in productions that are weaker, more variable, etc.

However, it's possible that languages are structured to be perceptually optimizing, in which case we might have a countervailing expectation: production of a \palvelcon\ contrast\is{contrast} might vary by position (initial or final) or by place (labial or coronal) in ways that \emph{support} the contrast\is{contrast}, specifically in vulnerable contexts. This idea and the previous are not contradictory; rather, there is strong reason to believe both. Which factor prevails in a given case, leading to loss or preservation of a contrast\is{contrast}, depends on many factors that are generally beyond our power to predict.

The idea that languages might be structured in ways that facilitate the maintenance of phonological contrast\is{contrast} goes back decades (e.g. \cite{Martinet1952_functional_load}; \citeyear{Martinet:1964}; \cite{Liljencrants_Lindblom1972_V_inventories}; \cite{Lindblom1986_V_system_universals}; \citeyear{Lindblom1990_HH_theory}). Most relevant here is work more specifically suggesting that \emph{gestural timing} can be organized in ways that are perceptually optimizing. An important precedent is \citet{Kingston1990_artic_binding}, who argues that across languages glottal articulations are `bound' in time most often to the release\is{release} of stop consonants, because the burst characteristic of stops renders the glottal gesture\is{gesture} most perceptually effective. Sonorant and fricative consonants lack bursts and so do not coordinate glottal articulations in the same way. \citet{Silverman1997_laryng_compl} argues in a similar way that laryngeal states creating phonation and tone contrasts\is{contrast} in vowels are timed in nuanced ways that facilitate their perceptibility, and \citet{Wright:1996_cue_preserv} argues that place gestures\is{gesture} in consonant clusters are timed so as to ensure recoverability of burst cues in consonants that lack internal place cues or formant transitions. More generally, \citet{Kingston_Diehl1994_phonetic_knowledge} argue that speakers systematically control articulations in ways that support the perceptual distinctiveness of contrast\is{contrast}, and there is a great deal of other evidence that phonetic targets are ultimately auditory, not gestural. (For an overview and arguments, see \citealt{Kingston:2019_interface}.)

The above observations do not necessarily imply that speakers \emph{intentionally} structure their articulations to make their speech clear or intelligible for listeners. While that \emph{could} be the case, it is also plausible that articulatory patterns which are easier to hear are also easier to learn, and therefore more likely to persist over time (e.g. \citealt{Ohala1993_coarticulation}). Articulatory patterns which produce less robust acoustic outcomes are more likely to be misheard by listeners: gradually, such misperceptions can lead to sound changes involving the collapse of contrasts\is{contrast} (e.g. \citealt{Ohala1981_listener_sound_change} and many others). In this way, perceptually-supportive articulatory strategies can emerge organically, without any intent on the part of speakers. That said, we remain open to the possibility that speakers do in fact strive to produce clear, intelligible speech in at least some circumstances. In either case, articulatory timing clearly plays an important role in the perceptibility of contrasts\is{contrast}.


%%%%%%%%%%%%%%%%%%%%%


\subsection{Gestural timing in palatalization contrasts}\label{sec:timing}
Previous work on the timing of secondary articulations in both Irish\il{Irish (Modern)} and Russian suggests that the tongue body gesture\is{gesture} associated with palatalization reaches its target (or peaks) around the release\is{release} of the primary place constriction for onset consonants (e.g. \citealt[Ch.10]{Ladefoged_Maddieson1996_SOWL}; \citealt{Kochetov2002_diss}; \citeyear{Kochetov2006_syll_poss}; \cite{Iskarous_Kavitskaya2010_phonetic_variability}; \cite{Bennett_etal2018_Conamara_ultrasound}; \cite{Padgett_etal2023_Irish_pal_syllpos}). More specifically for Irish\il{Irish (Modern)}, \citet{Bennett_etal2018_Conamara_ultrasound} find that tongue body fronting for palatalized /C\pal/ tends to peak at C release\is{release} in both stops and fricatives. In contrast\is{contrast}, velarized /C\vel/ does not show any clear, consistent asymmetries across time points: the magnitude of dorsal backing tends to be comparable at the beginning of C constriction, midpoint, and release\is{release}. 

\citet{Bennett_etal2018_Conamara_ultrasound} only examine consonants that are onsets (and prevocalic), and so do not address the timing of secondary dorsal constrictions in codas (or postvocalic position). However, based on an electromagnetic midsagittal articulometry (EMMA) study of four speakers, \citet{Kochetov2006_syll_poss} found that Russian /p\pal/ was produced differently in onset position compared to coda position. Kochetov measured the lag between the achievement of the secondary and primary articulations (defined in terms of velocity minima) of /p\pal/, as well as the lag between the release\is{release} of the two gestures\is{gesture}. For all speakers, in onset position both the achievement and release\is{release} of the palatalization gesture\is{gesture} followed that of the labial constriction by 5--55\,ms on average (depending on the speaker). This is consistent with our findings for Irish mentioned above. In coda position, the relative timing of the palatalization gesture\is{gesture} with respect to the labial gesture\is{gesture} was found to be more variable by speaker, but it generally occurred earlier than in onset position. For example, the achievement of the palatalization gesture\is{gesture} preceded that of the labial gesture\is{gesture} by 5--25\,ms for three of the speakers, but followed it for the fourth speaker. \citet[][159--160]{Biteeva2021_Russian_palatals_PhD} reports very similar findings for /t\pal/, based on another EMMA study of nine speakers. Both studies characterize the relationship between primary and secondary articulations as sequential for onsets and more simultaneous for codas.
% \footnote{Foreshadowing the discussion in \sectref{sec:codaphasing}, we note that sequential organization in onsets and simultaneous organization in codas is exactly \emph{opposite} what might be expected if secondary articulations were coordinated with their primary articulations like consonant clusters rather than complex segments; see e.g. \citet{Browman_Goldstein1988_syll_struc}.}

\subsection{Gestural timing and perceptual consequences}\label{sec:timing_percep}
It is worth considering the timing facts above in light of their effects on the production of cues to palatalization. As \citet{Kochetov2006_syll_poss} points out, it is perceptually adaptive for the palatalization gesture\is{gesture} of a stop to lag its primary place gesture\is{gesture} in onset consonants. There are two major cues to the palatalization status of a stop: one is in the formant transitions between the consonant and a neighboring vowel (primarily the second formant), and the other is in the quality of the consonant burst, the spectral properties of which are shaped by the presence of palatalization or velarization. In onset consonants, these distinct cues more or less coincide in time at the release\is{release} of the primary place gesture\is{gesture}, indicated by the leftmost oval in \figref{fig:spectrogram}. Following the reasoning of \citet{Kingston1990_artic_binding} for the timing of glottal articulations with respect to stop consonants (discussed earlier), coordinating palatalization to peak around this location maximizes its contrast\is{contrast} potential. This fits well with the general finding that onset consonant-vowel transitions are perceptually robust (\cite{Fujimura_etal1978_stop_transition}; \cite{Ohala1990_assimilation}; \cite{Steriade2001}; \cite{Wright2004_cue_robustness}; \cite{Kochetov2006_syll_poss}).

\begin{figure}[!htb]
    \centering
    \includegraphics[width=0.65\linewidth]{figures/Bennett-img001.png}
    \caption{Waveform and spectrogram of the word \emph{p\'{\i}ob} \ipa{/pʲiːbˠ/} (Connacht Speaker 1 repetition 1), showing location of major cues to Irish\il{Irish (Modern)} secondary articulations in onset (left oval) and coda (right two ovals) stops.}
    \label{fig:spectrogram}
\end{figure}

Things are different for coda stops, where the two major cues to palatalization are split in time, as shown on the right of \figref{fig:spectrogram}. The formant transition cues to the contrast\is{contrast} precede the stop closure (note the salient drop in the second formant for /b\vel/), while the burst cues lie at the release\is{release} of the stop constriction (rightmost oval in the figure). If secondary articulations are timed to be perceptually optimizing, what should that entail for stops in final position? We do not know enough about the perception of palatalization cues to answer this question definitively. However, a reasonable first thought would be that gestures\is{gesture} should be coordinated in a way that facilitates the perception of \emph{both} formant transitions and burst properties. This would be true if the secondary articulation were to be achieved at, or slightly before, the onset of the primary articulation and be released\is{release} at, or slightly after, the release\is{release} of the primary articulation.\footnote{The discussion here is oversimplified, since it ignores the larger phrasal context of the word in \figref{fig:spectrogram}. We return to this issue in the paper's discussion and conclusion.}

As discussed in the last section, previous work on Russian has suggested that the primary and secondary articulations of coda palatalized stops can be roughly simultaneous, consistent with the above prediction, or that the palatalization gesture\is{gesture} slightly precedes the primary one, perhaps favoring the formant transition cues. Roughly simultaneous coordination has also been found for Russian /l\pal/ \citep{Kochetov:2005} and /s\pal/ \citep{Kochetov2009_Russian_C_variation}. In these studies, the sounds occurred word\slash syllable\hyp finally before a vowel-initial word. As \citet{Kochetov:2005} notes for /l\pal/, these facts also seem consistent with the hypothesis about timing and perception: since these sounds have internal cues to palatalization (that is, the palatalized state of the consonant is audible during the steady state portion of its production), have no bursts, and would have formant transitions on both sides, we have no reason to expect timing of the palatalization gesture\is{gesture} to be coordinated with either the beginning or end of the consonant in particular, regardless of syllable position (see also \citealt{Gick_etal:2006_liquids}).\footnote{In this paper we may use the terms `start' and `beginning' interchangeably, both referring to the point in time when a stop closure is first achieved. The same is true of `end' and `release\is{release}', but we often use `release\is{release}' when the focus is on the acoustic and perceptual consequences of the end of consonant closure.} 

However, in the same phonetic context (coda before a vowel-initial word), \citet{Kochetov2009_Russian_C_variation} found that the palatalization gesture\is{gesture} of /t\pal/ was timed to coincide more with the \emph{release\is{release}} of the primary coronal gesture\is{gesture}. Assuming such a difference turned out to be systematic, it might suggest that gestural timing\is{gestural timing} is sensitive not just to the presence, but to the relative salience, of phonetic cues. \citet{NiChiosain_Padgett2012_Irish_pal_acous_percep} report that Irish\il{Irish (Modern)} /t\pal\ d\pal/ differ more from /t\vel\ d\vel/, respectively, in burst duration, intensity and center of gravity than /p\pal\ b\pal/ differ from /p\vel\ b\vel/. Indeed, in both Irish\il{Irish (Modern)} and Russian the release\is{release} of /t\pal\ d\pal/ is very salient, even sounding somewhat affricated. In contrast\is{contrast}, /p\pal\ b\pal/ differ \emph{more} from /p\vel\ b\vel/, respectively, in second formant values than /t\pal\ d\pal/ differ from /t\vel\ d\vel/, in both Irish\il{Irish (Modern)} and Russian (\cite{Purcell:1979}; \cite{NiChiosain_Padgett2012_Irish_pal_acous_percep}). These considerations might suggest that, if the timing of secondary articulation gestures\is{gesture} in coda consonants is to favor either the beginning or end of the primary gesture\is{gesture}, it would be more adaptive to favor the beginning in the case of labial stops and the end in the case of coronal stops. However, it should be noted that \citet{Kochetov2009_Russian_C_variation} found coordination with the end also in the case of Russian coda /n\pal/. This obviously does not sit well with the above reasoning, and overall this remains fairly speculative given the amount of available data.\footnote{As noted above, \citet{Biteeva2021_Russian_palatals_PhD} found roughly simultaneous coordination of primary and secondary articulation in /t\pal/, unlike \citet{Kochetov2009_Russian_C_variation}. In Biteeva's stimuli, the target coda preceded a word-initial coronal consonant, in which case burst cues might have been less reliable. However, in Kochetov's stimuli, the target coda was prevocalic. An interesting question for future work is whether such differences are systematic.}
 
%%%%%%%%%%%%%%%%%%
\subsection{Hypotheses}\label{sec:hypotheses}
Though there is not a great deal of previous evidence to go on, what evidence there is suggests that the relative timing of primary and secondary articulation gestures\is{gesture} in palatalization contrasts\is{contrast} is more variable in the coda than in the onset, across consonant types and speakers (and possibly within them as well). We would therefore expect to find such an onset-coda asymmetry in our Irish\il{Irish (Modern)} data.

If the guiding principle is that gestural timing\is{gestural timing} is perceptually adaptive, then we expect both palatalization and velarization gestures\is{gesture} to be aligned roughly with the release\is{release} of onset stops in Irish\il{Irish (Modern)}, because that is where all cues to the contrast\is{contrast} can be maximized. As noted above, previous work on Irish\il{Irish (Modern)} and Russian already leads us to expect this pattern for palatalized consonants. As for velarization, past work has \emph{not} found any consistent tendency for \ipa{/Cˠ/} gestures\is{gesture} to align with consonant release\is{release} in onset position, as observed for \ipa{/Cʲ/}. Regardless of syllabic position, velarization seems fairly static over the duration of the consonant. However, relevant data are limited.

In the case of coda stops, where cues to the palatalization contrast\is{contrast} are split between the beginning and end of the primary place gesture\is{gesture}, we have no general reason to expect alignment specifically with release\is{release}. Assuming both formant transition and burst cues are important, we would expect primary and secondary articulation gestures\is{gesture} to be roughly simultaneous. Put differently, we might expect secondary articulation gestures\is{gesture} to be achieved by the beginning of the consonant and maintained through to the release\is{release}. 

However, as discussed earlier, perhaps not all cues are equal in every context. If formant transitions are more effective cues than burst properties are for labial stops, then we might expect alignment of secondary articulations with the beginning of coda labials, at the vowel-consonant transition. If the reverse is true for coronal stops, we might expect alignment with the coda consonant release\is{release} instead. However, these latter predictions are somewhat speculative.

\is{gestural timing|)}
\is{ultrasound|(}
\section{Ultrasound study}

%%%%%%%%%%%
\subsection{Methods}
\subsection{Participants}
We recorded 7 native Irish\il{Irish (Modern)} speakers, whose language background corresponded to each of the three major Irish\il{Irish (Modern)} dialect groups (\figref{fig:map}). The shaded portions of \figref{fig:map} show the Gaeltachta\'{\i}~- areas where Irish\il{Irish (Modern)} is spoken as a community language. It is conventional to distinguish three major dialect areas for Irish\il{Irish (Modern)}, that of Ulster in the northwest, Connacht in the west, and Munster in the southwest. Our study includes speakers from all three dialect areas, and we use these terms, but it should be noted that there is dialectal diversity within each dialect area.

\begin{figure}
    \includegraphics[width=.6\linewidth]{figures/Bennett-img002.png}
    \caption{Officially recognized \emph{gaeltachta\'{\i}} (Irish\il{Irish (Modern)}-speaking areas) in the Republic of Ireland, shaded in green (\emph{gaeltachta\'{\i}} map data provided by \texttt{\url{https://data-osi.opendata.arcgis.com/}}, under CC by 4.0 license, \texttt{\url{https://creativecommons.org/licenses/by/4.0/}}).}
    \label{fig:map}
\end{figure}

We recorded two speakers of Ulster Irish\il{Irish (Modern)} (U1, 24, M; U2, 40, M), three of Connacht Irish\il{Irish (Modern)} (C1, 42, F; C2, 50, F; C3, 43, F), and two of Munster Irish\il{Irish (Modern)} (M1, 34, M; M2, 56, M). (Mean age = 41, median = 43.) Recordings took place in Gaoth Dobhair, Casla, and Baile na nGall, indicated on the map.

All of our speakers were professional Irish\il{Irish (Modern)}-language broadcasters, working for RT\'E Raidi\'o na Gaeltachta, the Irish\il{Irish (Modern)}-language radio network of Raidi\'o Teilif\'{\i}s \'Eireann, Ireland's national public service media system. Their jobs required them to speak and read Irish\il{Irish (Modern)} throughout the workday. All were native speakers of their respective dialects who were raised, and currently live, in the relevant dialect area. Their parents were native speakers who spoke Irish\il{Irish (Modern)} to them, except that one speaker's father (C3) was a native English\il{English (Modern)} speaker who spoke both English\il{English (Modern)} and Irish\il{Irish (Modern)} to her, and another's (M1) mother was a native English\il{English (Modern)} speaker who spoke Irish\il{Irish (Modern)} to him. They attended primary and secondary school through Irish\il{Irish (Modern)} in their respective dialect areas, though C1-3 and M1 used both English\il{English (Modern)} and Irish\il{Irish (Modern)} in secondary school. All speakers attended third level colleges. All reported reading Irish\il{Irish (Modern)} on a daily basis. None reported any difficulties in hearing, speaking, or reading.

As in \citet{Bennett_etal2018_Conamara_ultrasound}, we chose to work with Irish\il{Irish (Modern)}-language broadcasters for several reasons. First, even in traditional Irish\il{Irish (Modern)}-speaking communities (\emph{gael\-tachta\'{\i}}), English\il{English (Modern)} has grown increasingly dominant, particularly among younger speakers, as noted earlier. In this fragile sociolinguistic context, native speakers may avoid interactions~-- such as experimental phonetic studies~-- in which they perceive that their Irish\il{Irish (Modern)} is being evaluated. The speakers for this study represent relatively traditional varieties of Irish\il{Irish (Modern)}, are comfortable with audio recording technology, and are accustomed to speaking while being recorded.

Given the number of speakers of each dialect area, we cannot infer anything firm about dialect differences based on our study, though the speakers clearly showed expected properties of their dialects. (For example, the Ulster speakers pronounced /t\pal\ d\pal/ as \ipa{[t͡ɕ d͡ʑ]} respectively.) A tentative investigation of inter-speaker differences is provided in \sectref{sec:individ-var}.

%%%%%%%%%%%
\subsection{Items}\label{sec:items}
The materials were (mostly) monosyllabic words with a target stop consonant controlled for secondary articulation (/C\pal/ or /C\vel/), place of articulation (labial, coronal, dorsal), position (word-initial or -final) and adjacent vowel \ipa{/iː uː ɔː/} (= 2 $\times$ 3 $\times$ 2 $\times$ 3 = 36 total conditions). There was one word per condition. The words used are given in Tables~\ref{tab:OnsetWords} and~\ref{tab:CodaWords} in Appendix~\ref{appendix:bennett:A}.

We used long \ipa{/iː uː ɔː/} for the vowel contexts because these vowels vary in both backness~-- which is key to \palvelcon\ contrasts\is{contrast} in Irish\il{Irish (Modern)}~-- as well as height. Short vowels cannot be used for this purpose, because non-low short vowels take on the backness of a following consonant (\citealt{NiChiosain_1991_diss}; \cite{ModernIrish}): they are predictably front before palatalized consonants (e.g. \emph{min} \ipa{[m\pal \uline{\textbf i}n\pal]} `grain') and back before velarized consonants (e.g. \ipa{[m\pal \uline{\textbf u}n\vel]} `small').

In using all three of \ipa{/iː uː ɔː/}, this study expands on \citet{Bennett_etal2018_Conamara_ultrasound}, who only considered onset consonants preceding \ipa{/iː uː/}. Having said this, we found substantial variation in the pronunciation of \ipa{/ɔː/} depending on dialect, speaker, context and word, particularly for the Ulster speakers, who had qualities including \ipa{[ɔ a ɑ ʌ ɛ]}. The high vowels were more consistent, though we saw some variation in the backness of \ipa{/uː/} depending on the speaker and word. Due to these and other unanticipated variations in vowel production, the conclusions we can draw about vowel context effects are limited.\footnote{In two forms the vowel context differed from the expected because of diphthongized pronunciations: \ipa{[iːəd\vel]} for \textit{iad} `them' and \ipa{[b\vel uːɨk\pal]} or \ipa{[b\vel uːɪk\pal]} for \textit{buaic} `pinnacle'.}

Since perfect control of materials would have made it impossible to use actual words, the target stops in our word list were either voiceless or voiced as needed (e.g. \emph{tu\'\i} \ipa{/\uline{\textbf t\vel}iː/} `straw' vs. \emph{d\'\i on} \ipa{/\uline{\textbf d\pal}iːnˠ/} `roof'). For similar reasons, three of our forms are disyllabic instead of monosyllabic, though in all cases the target consonant was an onset/word-initial one in the stressed (initial) syllable (these are \emph{p\'\i osa} \ipa{/\uline{\textbf pʲ}iːsˠə/} `piece', \emph{p\'uca} \ipa{/\uline{\textbf pˠ}uːkˠə/} `ghost', and \emph{ci\'unas} \ipa{/\uline{\textbf kʲ}u:nˠəsˠ/} `quiet (noun)'). For one word, \ipa{/b\pal eːk\pal/} `shout' with a final /k\pal/ target consonant, the vowel context was \ipa{/eː/} instead of \ipa{/iː/}. In the case of onset/word-initial consonants, the nearest consonant (after the following vowel), if there was one, was a velarized coronal (e.g. \ipa{/d\pal iːnˠ/} `roof'), except in the case of \ipa{/p\vel uːk\vel ə/} `ghost'. In the case of coda/word-final target consonants, the nearest consonant (before the preceding vowel) was more varied (see \tabref{tab:CodaWords} in Appendix~\ref{appendix:bennett:A}), again to ensure that actual words were used.

Items were recorded in the context of the carrier sentence  \mbox{\ipa{[ˈd\vel uːr\pal t\pal\ ˈiːf\vel ə \_\_\_\_}} \\
\mbox{\ipa{əˈn\vel ʊr\vel ə]}} `Aoife said \_\_\_\_ last year'. This means that word-final target consonants were positioned before a vowel-initial word, something that made segmentation and labeling based on the acoustic signal easier. However, it does raise the question whether the relevant target consonants are actually syllable-final in these sentences, as we assume they are. We do not believe that Irish\il{Irish (Modern)} has resyllabification between lexical words: we refer the reader to arguments in \citet{Dubach_Green:2001_clitics} that resyllabification in Irish\il{Irish (Modern)} is possible only in the case of certain proclitics.\footnote{For discussion of Irish\il{Irish (Modern)} syllabification in general, see \citet{nichiosain_inpress_irishphon}.} Note also that \citet{Kochetov2006_syll_poss}'s finding of different timing for word-final labials compared to word-initial ones, discussed in \sectref{sec:timing}, was also based on items placed in a carrier sentence before a vowel. As we will see, our own timing results also show that final stops do not behave like initial ones.


% Since this question matters to our conclusions, we briefly address it here. There is indeed evidence that final consonants resyllabify into the onset position of a following vowel-initial word in Irish---but only in the case of proclitics, as argued by \citet{Dubach_Green:2001_clitics}. Dubach Green notes that vowel-initial words in Irish trigger palatalization or velarization of preceding consonants that are final in certain proclitics. For example, the final \ipa{/nˠ/} of the definite article \ipa{/ənˠ/} is palatalized when procliticized to \ipa{/ʲiːrʲənʲə/} `truth', giving \ipa{[ə.nʲiːrʲənʲə]} `the truth' (with the assumed syllabification shown), while the final \ipa{/gˠ/} of the progressive marker \ipa{/əgˠ/} is velarized when procliticized to \ipa{/ˠosˠkˠəlʲtʲ/} `open', giving \ipa{/ə.gˠosˠkˠəlʲtʲ/} `opening'.\footnote{Surprisingly, whether the consonant is palatalized or velarized does not depend on the backness of the vowel but on the word. This is why Dubach Green assumes, following  that word-initial secondary articulations are specified in the input, as seen in the example forms given here.} The final consonants of open-class lexical items never behave in this way, even in compounds. Given such facts, and the lack of any work arguing for general resyllabification in Irish, we are reasonably confident that our syllable-final target consonants are actually syllable-final. 


%%%%%%%%%%%
\subsection{Recording procedure}
Recordings were done in RT\'E Raidi\'o na Gaeltachta studios in Gaoth Dobhair (Gweedore, Ulster), Casla (Costelloe, Connacht) and Baile na nGall (unofficially Ballydavid, Munster). Speakers repeated the 36 experimental items five times each in the frame sentence described above. All speakers read the words in the same pseudo-random order. 

%%%%%%%%%%%
\subsubsection{Ultrasound and audio recording}
We recorded midsagittal images of the tongue surface using a portable Terason T3000 ultrasound\is{ultrasound} system and model 8MC3 3--8\,MHz probe with a 90$^{\circ}$ field of view (using a depth setting of 10 and focal setting of 8, giving $\approx$ 46 fps, or 1 frame every 22\,ms), using the Ultraspeech software package \citep{Hueber_etal:08}, which automatically aligned the audio and video recordings. The probe was stabilized and held in place with an Articulate Instruments ultrasound\is{ultrasound} stabilization headset (\cite{Wrench2008_stabilization_headset_manual}; \cite{Scobbie_etal_2008_head_movt}). Simultaneous audio was recorded using a Shure WH20 dynamic cardioid microphone attached to the headset, recording directly to the  ultrasound\is{ultrasound} system (which included a laptop computer) at a 44.1\,kHz sampling rate.\footnote{The microphone failed for the Connacht recordings, and the laptop's built-in microphone had to be used instead.}

%%%%%%%%%%%
\subsection{Data annotation}

%%%%%%%%%%%
\subsubsection{Landmark identification and frame}\label{sec:landmarks}
Temporal landmarks for each consonant were annotated in Praat \citep{Praat}, on the basis of acoustic information present in the waveform and spectrogram of the audio signal (see \citealt{Turk_etal2006_acoustic_prosody_research}). The beginning of C closure was primarily identified by a drop in amplitude from the preceding vowel. For onset stops, this vowel belonged to the preceding word in the frame sentence (the proper name \emph{Aoife} \ipa{[ˈiːf\vel ə]}). For coda stops, the preceding vowel belonged to the same target word. Stop offsets were marked at the beginning of the stop release\is{release} burst. In cases of uncertainty, the choice was generally made to err in the direction of smaller consonant duration, so as to minimize the possibility of including vocalic information as part of the consonant.

We extracted ultrasound\is{ultrasound} frames for analysis at three different landmarks: stop onset, stop release\is{release}, and the midpoint between them. (These correspond to the terms `C start', `C end', and `C midpoint' used below.)
% depending on syllable position. In onset position, we extracted frames corresponding to stop release. In coda position, we extracted frames at the end of the vowel at the VC transition. 
These landmarks were determined as described above (see also \figref{fig:spectrogram}). Analyzing secondary articulations at each of these three landmarks should provide us with at least a rough measure of how those secondary articulations are timed relative to major articulatory events in the corresponding primary articulations of the same consonant.


%%%%%%%%%%%
\subsubsection{Contour tracing}\label{sec:contours}
Mid-sagittal ultrasound\is{ultrasound} images were traced using EdgeTrak software \citep{EdgeTrak_Li_etal2005}. EdgeTrak produces contours consisting of 100 points per traced tongue surface (\figref{fig:EdgeTrak}), each with (X,Y) coordinates in Cartesian space.

\begin{figure}[!ht]
    \centering
    \textbf{Connacht Speaker 1 onset \ipa{/Pʲiː/}}\\
    \includegraphics[width=\linewidth]{figures/Bennett-img003.png}
    \caption{Left: sample EdgeTrack tracings for 5 tokens of onset /P\pal\ipa{iː}/ at C end, Connacht Speaker 1. Right: example identification of dorsal peak
    for one EdgeTrak tracing in this set (repetition 1).}
    \label{fig:EdgeTrak}
\end{figure}

Some mid-sagittal images could not be traced due to issues with image quality. This led to imbalances in the number of tokens across speakers and places of articulation. For some speakers, all tokens at a particular place of articulation could not be traced reliably, usually because the tongue surface for that consonant type consistently moved outside of the region in which the tongue body could be confidently identified (i.e. outside of the field of view and/or depth of the image produced by our ultrasound\is{ultrasound} probe). This most often occurred for dorsal stops, and sometimes for coronal stops. More detailed information about the number of tokens traced per speaker is provided in \tabref{tab:token-counts} in Appendix~\ref{appendix:bennett:B}.



%%%%%%%%%%%%%%%%%%%%%%%%%%%%%%%%%%%%%%%%%%%%%%%%%%%%%%%%
\subsection{Quantitative analysis}

%%%%%%%%%%%
\subsubsection{Normalization}
In order to ensure that contour tracings were comparable across speakers, all tracings were range-normalized on a by-speaker basis. The X-coordinates were rescaled to the interval $[0,1]$, and Y-coordinates were rescaled by the same proportion and similarly shifted to have a minimum value of 0 (see \figref{fig:EdgeTrak}).

%%%%%%%%%%%
\subsubsection{Dorsal peak backness}
To assess the degree of velarization (= tongue body backing) and palatalization (= tongue body fronting) across contexts, we identified the X coordinate of the highest point of the tongue body in each contour (\figref{fig:EdgeTrak}). The position of this X coordinate can then be used as a measure of the degree of fronting or backing of the tongue body in a given context.

In some tokens, there was a plateau rather than a unique highest point for the traced contour. In such cases, the center of the plateau was selected as the `peak' of the tongue body and the X coordinate of that center point used as our measure of tongue body fronting/backing.

We visually inspected each raw tracing to ensure that the highest point of the tongue contour was in a region that plausibly corresponded to the tongue body. For one speaker's coronal data (M2), the highest point of the tongue surface fell on the tongue blade or tip, and not the tongue body; this data has been excluded from the analysis of dorsal backness.

Overall, we examined the X position of the dorsal peak for 782 tokens in our data. The data and R scripts used to carry out this analysis are available on GitHub.\footnote{\url{https://github.com/rbennett24/articles/tree/master/Irish_pal_timing_syll_FACL}}


\subsubsection{Dependent measure: Change in backness over time}
Our primary research questions concern the timing of secondary articulations in Irish\il{Irish (Modern)}. In our statistical analysis, we investigated differences in dorsal backness at C end vs. C start, for each token in our data. To do this, for each token we subtracted the dorsal backness value at C start from the corresponding value at C end. This difference~-- dorsal backness at C end relative to C start in each token~-- is the dependent variable for our analysis. Values that are different from zero indicate a difference in the magnitude of dorsal backing/fronting across time points; if consistent across tokens, this would suggest that the secondary articulation gestures\is{gesture} for \palvelcon\ peak at either C end or at C start in a given condition.

Such skews can be taken as evidence for a particular pattern of gestural alignment between primary and secondary articulations. For example, if dorsal front\-ing peaks at C end (= release\is{release}) for \ipa{/Cʲ/}, we expect values of C end~-- C start to be positive for tokens of \ipa{/Cʲ/}, because the tongue should move forward over time. If dorsal backing for \ipa{/Cˠ/} also peaks at C release\is{release}, we expect values of C end~-- C start to be \textit{negative} for \ipa{/Cˠ/}, since velarization involves backing the tongue body rather than fronting it.

Somewhat more succinctly: positive numbers mean the tongue body moves forward over time during the consonant, whereas negative numbers mean the tongue body moves backward. Consistent patterns of either type can be used as evidence for timing relationships between primary and secondary articulations.

\is{ultrasound|)}
%%%%%%%%%%%%%%%%%%%%%%%%%%%%%%%%%
\section{Results}\label{sec:results}
\figref{fig:trajectories} shows dorsal backness trajectories, from C start to C midpoint to C end, for each token in our data. Certain tendencies are already apparent. For example, the position of the tongue dorsum appears to be further front at C release\is{release}, compared to C start, for onset /P\pal\ T\pal\ K\pal/, suggesting gestural alignment between the secondary /C\pal/ articulation and stop release\is{release}. Similarly, the tongue dorsum is further \emph{back} at C release\is{release} compared to C start for onset \ipa{/Pˠ/}. We also observe a certain amount of variability in timing patterns across tokens in the same condition. Some of this variability may owe to the influence of vowel context (e.g. \citealt{Bennett_etal2023_jphon_submission}). Below we focus on other sources of variability.\footnote{A point of clarification: for each token, the value at C end is relative to C start. As a consequence, the values at C end should \emph{not} be interpreted as reflecting the overall degree of palatalization (fronting) or velarization (backing) in a given token. Instead, values at C end should be interpreted as reflecting \emph{change in backness over time}. (The same goes for values at C midpoint, of course, though we do not analyze that data here.) See \citet{Bennett_etal2023_jphon_submission} for an analysis of the strength of palatalization/velarization in this data.}

\begin{figure}[!ht]
    \centering
    \includegraphics[width=\linewidth]{figures/Bennett-img004.png}
    \caption{Trajectories for backness of dorsal peak over C start, midpoint, and end time points, relative to values at C start. Grey lines represent trajectories for individual tokens and black lines represent loess-smoothed regressions over pooled values, with an overlaid ribbon in blue showing confidence intervals around estimates of the mean.}
    \label{fig:trajectories}
\end{figure}

To help interpret the values presented in \figref{fig:trajectories} and subsequent figures, Tables \ref{tab:sum-stats-alldata} and \ref{tab:sum-stats-palvel} provide some summary statistics about the distribution of dorsal backness values in our data. For example, a difference of 0.11 between start and end position in \figref{fig:trajectories} would correspond to a difference of about one standard deviation for the full set of backness values in our data (across all three of C start, midpoint and end), or about 1/6 of the overall range of these values (\tabref{tab:sum-stats-alldata}). It would also be about the same size as the difference in mean dorsal backness between velar \ipa{/K\pal/} vs. \ipa{/K\vel/} in onset position at C end in our data (\tabref{tab:sum-stats-palvel}). Figures~\ref{fig:EdgeTrak} and \ref{fig:loess} are also useful for visualizing these differences; see \tabref{tab:lmer} for related statistical analysis.

\begin{figure}
    \centering
    \textbf{Connacht Speaker 1 onset \ipa{/Pʲiː/} vs. \ipa{/Pˠiː/} (left) and coda \ipa{/iːPʲ/} vs. \ipa{/iːPˠ/} (right)}\\
    \includegraphics[width=\linewidth]{figures/Bennett-img005.png}
    \caption{Loess-smoothed curves for \ipa{/P\pal/} vs. \ipa{/P\vel/} in the context of \ipa{/iː/}, in both onset (left) and coda (right) position. Connacht Speaker 1 (see \figref{fig:EdgeTrak}).}
    \label{fig:loess}
\end{figure}

\begin{table}[ht]
\resizebox{\textwidth}{!}{%
\begin{tabular}{llllllll}
\lsptoprule
 {Mean} & {Median} & {Range} & {$\Delta$(Range)} & SD & \textsc{IQ range} & \textsc{$\Delta$(IQ range)} \\\midrule
 0.5020 & 0.5080 & [0.1356, 0.7789] & 0.643 & 0.108 & [0.4266, 0.5952] & 0.173\\
\lspbottomrule
\end{tabular}}
\caption{Summary statistics for distributions of dorsal backness values in our data. \textsc{sd} = standard deviation, \textsc{IQ range} = inter-quartile range, $\Delta$\textsc{(X)} = size of range for \textsc{X}.}\label{tab:sum-stats-alldata}
\end{table}



\begin{table}[ht]
\begin{tabular}{ccccc}
\lsptoprule
  \textsc{Syllable position}& /P\pal/ - /P\vel/ & /T\pal/ - /T\vel/& /K\pal/ - /K\vel/ & \textsc{Overall}\\ 
  \midrule
  Onset & 0.166 & 0.206 & 0.100 & 0.163\\
  Coda & 0.132 & 0.190 & 0.079 & 0.134\\
\lspbottomrule
\end{tabular}
\caption{Differences between mean backness values of /C\pal/ - mean backness values of /C\vel/ at C end, across places of articulation and syllable positions}\label{tab:sum-stats-palvel}
\end{table}

We statistically analyzed values of tongue body backness at C end (relative to C start) using linear mixed-effects modeling in \textsc{R} \citep{RStats}  with the \textsc{lmerTest} package (\citealt{lmerTest}; see also e.g. \citealt{Pinheiro_Bates2000_mixed_effects_book}; \cite{Gelman_Hill2006_regression_BOOK}; \cite{Baayen_etal2008_random_effects}; \cite{Bolker_etal2009_glms_prac_guide}; \cite{Barr_etal2013_max_rand_effects}; \cite{Matuschek_etal2017_error_power_mixed_models}; \cite{Tomaschek_etal2018_collinearity}; \cite{Oberpriller_etal2022_fixed_or_random}). The predictors \textsc{Consonant Place}, \textsc{Secondary Articulation}, \textsc{Syllable Position}, and \textsc{Vowel Context} were sum-coded in order to reduce collinearity between simple predictors and higher-order interactions, and to help with model interpretability. All two-way interactions between these predictors were included in the initial model; we also included a three-way interaction between \textsc{Consonant Place, Secondary Articulation}, and \textsc{Syllable Position}. The reference level for \textsc{Consonant Place} was \textsc{Labial}; for \textsc{Vowel Context} it was \ipa{/iː/}; for \textsc{Syllable Position} it was \textsc{Onset}; and for \textsc{Secondary Articulation} it was /C\pal/.

A random effect for \textsc{Speaker} was included, along with by-speaker random slopes for \textsc{Syllable Position}. Models with additional by-speaker random slopes failed to converge, in part because of the unbalanced nature of our data (see Appendix~\ref{appendix:bennett:B}). We did not include a random effect of \textsc{Word} because we only had one word per condition of interest. The specifications for the model are provided in (\ref{ex:model-specs}), where \textsc{Consonant Place}=\textsc{C Place}, \textsc{Vowel Context}=\textsc{V Con}, \textsc{Syllable Position}=\textsc{Syll Pos}, and \textsc{Secondary Articulation}=\textsc{Sec Art}.

\begin{exe}
\ex Linear mixed-effects model specifications:\\ 
\emph{Dependent variable}:
\textsc{Backness of dorsal peak at C end - C start $\sim$\\
}
\emph{Three-way} (1): \textsc{C Place $\times$ Syll Pos $\times$ Sec Art} +\\
\emph{Two-way} (6): \textsc{
C Place $\times$ Syll Pos + 
C Place $\times$ V Con +
C Place $\times$ Sec Art + 
V Context $\times$ Syll Pos +
V Context $\times$ Sec Art +  
Sec Art $\times$ Syll Pos +\\
}
\emph{Simple} (4): \textsc{C Place + Syll Pos + V Con + Sec Art} +\\
\emph{Random effects}: \textsc{(1+Syll Pos|Speaker)}
\label{ex:model-specs}
\end{exe}

This model was fit with the parameter \textsc{REML = False} so that we could use the log-likelihood test to carry out model criticism and simplification. However, model criticism did not lead to the exclusion of any fixed-effect predictors (with $\alpha = 0.05$), starting with the highest-order (three-way) interaction term, and retaining all simple predictors participating in interactions. As such, our final model was the same as our initial model (\ref{ex:model-specs}). The residuals appeared approximately normally-distributed in the model. This model has only limited collinearity: the maximum variance inflation factor (VIF) was 1.34, as reported by the \texttt{check\_collinearity()} function in the \textsc{performance} package \citep{performancePackage}. VIF values below 5 are generally accepted to signal fairly low collinearity (e.g. \citealt{Tomaschek_etal2018_collinearity}).

The model results are provided in Tables \ref{tab:lmer} and \ref{tab:lmer2}. The \emph{p}-values we report were estimated by the \textsc{lmerTest} package. Results shaded in gray indicate certain significant (or near-significant) predictors for the hypotheses we consider here: namely, that the timing of secondary palatalization and/or velarization\is{velarization} gestures\is{gesture} may be less stable in codas than in onsets, and less stable in labials than in consonants at other places of articulation (especially coronals).
%\footnote{For arguments against treating $p < 0.05$ as a rigid cutoff for interpreting possible effects in statistical models, see e.g. \citet[68--71]{Baayen2008_stats}.}

%
%\clearpage
%
%

%\begin{tabular}{ll}
%\cellcolor{red!25}bwrdd & coch \\
%cath & \cellcolor{blue!25}melyn
%\end{tabular}
%black, blue, brown, cyan, darkgray, gray, green, lightgray, lime, magenta, olive, orange, pink, purple, red, teal, violet, white, yellow.

\begin{table}
\textsc{%
\begin{tabular}{ll S[table-format=-1.4] S[table-format=1.4] S[table-format=<1.3{*}]}
\lsptoprule
\multicolumn{2}{l}{Predictor} & {Estimate} & {SE} & {$p$} \\\midrule
\multicolumn{2}{l}{(Intercept)} & 0.0054 & 0.0049 & .31 \\ 
\multicolumn{4}{l}{\textbf{C place}} & <.001* \\
  & Coronal & 0.0099 & 0.0031 & <.01* \\ 
  & Dorsal & 0.0144 & 0.0034 & <.001* \\ 
\multicolumn{4}{l}{\textbf{Syllable position}} & < .46 \\
  & Coda & -0.0051 & 0.0065 & .46 \\ 
\multicolumn{4}{l}{\cellcolor{gray!25}\textbf{Secondary articulation}} & < .001* \\
  & \ipa{/Cˠ/} & -0.0207 & 0.0020 & < .001* \\ 
\multicolumn{4}{l}{\textbf{V context}}  & < .001* \\
  & \ipa{/uː/} & 0.0023 & 0.0028 & .43 \\ 
  & \ipa{/ɔː/} & 0.0150 & 0.0028 & < .001* \\ 
\lspbottomrule
\multicolumn{4}{l}{Range of values: [-0.2870, 0.2858] ($\Delta$ = 0.573)}\\
\multicolumn{4}{l}{Mean = 0.0003, median = 0, sd = 0.0716}\\
\multicolumn{4}{l}{Reference levels: \textsc{labial, \ipa{/Cʲ/}, onset, {\normalfont\ipa{/iː/}}}}
\end{tabular}}
\caption{Main effects of linear mixed-effects model parameters for analysis of dorsal backness trajectories (values at C end, relative to C start).}\label{tab:lmer}
\end{table}
  
\begin{table}
\centering
\textsc{%
\begin{tabular}{lccr}
  \lsptoprule
Predictor & Estimate & SE & {\normalfont\itshape p} \\ 
 \hline
  \multicolumn{1}{l}{\cellcolor{gray!25}\footnotesize\textbf{C place $\times$ Syllable position}} & & & $<$  .075 \\
  Coronal:Coda & 0.0072 & 0.0032 & $<$ .05* \\ 
  Dorsal:Coda & -0.0058 & 0.0034 & .09 \\ 
  %
  \hline
  \multicolumn{1}{l}{\cellcolor{gray!25}\footnotesize\textbf{Secondary articulation $\times$ C place}} & & & $<$ .01* \\
  \textnormal{\ipa{/Cˠ/}}:Coronal & -0.0067 & 0.0029 & $<$ .05* \\ 
  \textnormal{\ipa{/Cˠ/}}:Dorsal & 0.0099 & 0.0031 & $<$ .01* \\ 
  %
  \hline
  \multicolumn{2}{l}{\cellcolor{gray!25}\footnotesize\textbf{Secondary articulation $\times$ Syllable position}} & & $<$ .001* \\
  \textnormal{\ipa{/Cˠ/}}:Coda & 0.0141 & 0.0020 & $<$ .001* \\ 
  %
  \hline
  \multicolumn{3}{l}{\footnotesize\textbf{C place $\times$ V context}} & $<$ .001* \\
  Coronal:\textnormal{\ipa{/uː/}} & -0.0138 & 0.0041 & $<$ .001* \\ 
  Dorsal:\textnormal{\ipa{/uː/}} & -0.0018 & 0.0044 & .69 \\ 
  Coronal:\textnormal{\ipa{/ɔː/}} & 0.0078 & 0.0040 & .05 \\ 
  Dorsal:\textnormal{\ipa{/ɔː/}} & 0.0062 & 0.0043 & .15 \\ 
  %
  \hline
  \multicolumn{3}{l}{\footnotesize\textbf{V context $\times$ Syllable position}} & $<$ .001* \\
  \textnormal{\ipa{/uː/}}:Coda & 0.0012 & 0.0027 & .64 \\ 
  \textnormal{\ipa{/ɔː/}}:Coda & 0.0234 & 0.0026 & $<$ .001* \\ 
  %
  \hline
  \multicolumn{3}{l}{\footnotesize\textbf{Secondary Articulation $\times$ V context}} & $<$ .01*\\
  \textnormal{\ipa{/Cˠ/}}:\textnormal{\ipa{/uː/}} & 0.0009 & 0.0026 & .74 \\ 
  \textnormal{\ipa{/Cˠ/}}:\textnormal{\ipa{/ɔː/}} & 0.0067 & 0.0026 & $<$ .05* \\ 
  %
  \hline
 \multicolumn{3}{l}{\footnotesize\textbf{Secondary articulation $\times$ C place  $\times$ Syllable Position}} & $<$ .01* \\
  \textnormal{\ipa{/Cˠ/}}:Coda:Coronal & -0.0049 & 0.0029 & .09 \\ 
  \textnormal{\ipa{/Cˠ/}}:Coda:Dorsal & -0.0041 & 0.0031 & .18 \\ 
   \lspbottomrule
\multicolumn{4}{l}{Range of values: [-0.2870, 0.2858] ($\Delta$ = 0.573)}\\
\multicolumn{4}{l}{Mean = 0.0003, median = 0, sd = 0.0716}\\
\multicolumn{4}{l}{Reference levels: \textsc{labial, \ipa{/Cʲ/}, onset, {\normalfont\ipa{/iː/}}}}
\end{tabular}
}

\caption{Interactions of linear mixed-effects model parameters for analysis of dorsal backness trajectories (values at C end, relative to C start).}\label{tab:lmer2}
\end{table}

%
%\clearpage
%

While the results in Tables \ref{tab:lmer} and \ref{tab:lmer2} imply that gestural alignment may be influenced by a number of factors, the precise patterns of alignment indicated by these results are a bit difficult to interpret, given the various overlapping interactions and main effects. Still, certain patterns do seem apparent. For example, while velarized /C\vel/ shows evidence of backing at C end, relative to C start (negative estimate for /C\vel/), this tendency is much weaker in coda position (roughly comparable positive effect for \textsc{Coda $\times$ /C\vel/}). This suggests that velarized consonants may tend toward release\is{release} alignment in onsets, but not in codas (or at least, not as strongly in codas).

However, the significant effects of \textsc{Dorsal $\times$ /C\vel/}, \textsc{Coronal $\times$ /C\vel/}, and \textsc{Coronal $\times$ Coda} suggest that these tendencies may vary depending on place of articulation~-- in particular, labial \ipa{/P\vel/} may be driving most of the onset vs. coda asymmetries for velarized \ipa{/C\vel/}, as implied by \figref{fig:trajectories}. At a minimum, it seems that \textsc{consonant place, secondary articulation}, and \textsc{syllable position} play \emph{some} role in determining patterns of gestural alignment in our data.

To clarify the structure of our results, \figref{fig:violins} plots the distributions of backness values, relative to C start, for all combinations of \textsc{consonant place, secondary articulation}, and \textsc{syllable position} in our data. (We abstract away from vowel context because it is not related to the specific hypotheses we focus on here.) For each cell, we assessed whether the distribution of dorsal backness at C end was significantly different from zero (i.e. different from C start, since C start was subtracted from all time points for each token). This was determined by \emph{t}-tests with conservative Bonferroni-corrected significance values of $\alpha = 0.05/12 = 0.0041$. Distributions significantly different from zero in \figref{fig:violins} indicate patterns of gestural timing\is{gestural timing} in which dorsal backness for secondary articulations peaks at either C end or C start (note that we do not analyze midpoint values at all here).\footnote{The significant results labeled in \figref{fig:violins} remain the same if significance is assessed using the bootstrap estimation method described in \sectref{sec:individ-var} rather than \emph{t}-tests.} We label such distributions with \textsc{Rel} for `alignment with C release\is{release}' (= C end), and \textsc{VC} for `alignment with VC transition in coda position' (= C start in coda position). These are the only two patterns of alignment that our statistical analysis uncovered.

\begin{figure}
    \centering
    \includegraphics[width=\linewidth]{figures/Bennett-img006.png}
    \caption{Values for backness of dorsal peak over C start, midpoint, end time points, relative to values at C start. Black lines connect mean values of distributions at each time point. Boxed text labels indicate statistically significant patterns of alignment in a given condition: `\textsc{Rel}' indicates that alignment of the secondary articulation peaks at C release\is{release}, and `\textsc{VC}' indicates that alignment of the secondary articulation peaks at the VC transition in coda position.}
    \label{fig:violins}
\end{figure}

\figref{fig:violins} shows that there are clear asymmetries across conditions with respect to the timing of dorsal gestures\is{gesture} for \ipa{/C\vel\ C\pal/}. In onset position, the dorsal fronting gesture\is{gesture} for \ipa{/Cʲ/} peaks at C release\is{release}; this is consistent across labial, coronal, and dorsal places of articulation (see \figref{fig:loess-PJu-all} below for a visual illustration with onset \ipa{/Pʲuː/}). Additionally, velarized labial \ipa{/Pˠ/} also shows release\is{release} alignment in onset position. Velarized \ipa{/T\vel\ K\vel/} show no particular pattern of alignment in onset position in this data.

\largerpage
In coda position, timing patterns are more varied. Coda /P\pal/ shows VC alignment, while coda \ipa{/P\vel/} shows release\is{release} alignment.\footnote{It's important to remember that negative values at C end mean different things for /C\pal/ vs. /C\vel/. Negative values at C end indicate dorsal backing relative to C start: this implies stronger velarization\is{velarization} over time, but \emph{weaker} palatalization. So while the distributions of coda /P\pal/ and /P\vel/ are similar in \figref{fig:violins}, they reflect different alignment patterns: release\is{release} alignment for /Pˠ/ (stronger backing over time), and VC alignment for /Pʲ/ (weaker fronting over time).} We illustrate this difference with a pair of examples from Ulster Speaker 1's data in \figref{fig:p-backing}.

\begin{figure}
    \centering
    \begin{tabular}{cc}
    \multicolumn{1}{r}{\textbf{Ulster Speaker 1 onset \ipa{/ɔːPˠ/}}} & \multicolumn{1}{l}{\textbf{\qquad Ulster Speaker 1 coda \ipa{/ɔːPʲ/}}}\\
    \includegraphics[width=0.4\linewidth]{figures/Bennett-img007.png}&
    \includegraphics[width=0.58\linewidth]{figures/Bennett-img008.png}\\
    \end{tabular}
    \caption{Loess curves for Ulster Speaker 1's coda \ipa{/ɔːPˠ/} and \ipa{/ɔːPʲ/} data showing change over time. The solid black line corresponds to C start, the dashed yellow line to C midpoint, and the dashed blue line to C end.}
    \label{fig:p-backing}
\end{figure}

Both \ipa{/Pˠ/} and \ipa{/Pʲ/} show backing over time in coda position after \ipa{/ɔ:/} in \figref{fig:p-backing}, indicating an \emph{increase} in velarization\is{velarization} over time, but a \emph{decrease} in palatalization. This reflects different timing patterns for coda \ipa{/Pˠ Pʲ/}: release\is{release} alignment for coda \ipa{/Pˠ/}, and VC alignment for coda \ipa{/Pʲ/}.
 
Coda /T\pal/ also shows release\is{release} alignment in \figref{fig:violins}, while the dorsal gesture\is{gesture} for coda /T\vel\ K\pal\ K\vel/ do not appear to show any particular alignment pattern at all.

Some of these results are also evident in the model output in \tabref{tab:lmer2}. For example, the positive estimate for \textsc{{\normalfont\ipa{/C\vel/}} $\times$ Dorsal} is consistent with the observation that coda labial \ipa{/P\vel/} (the reference level for \textsc{C place}) shows release\is{release} alignment (negative values) across syllable positions, while dorsal \ipa{/K\vel/} shows values centered around zero for both onset and coda contexts.

The significant trends in \figref{fig:violins} are readily apparent from visual inspection of our raw tracings and corresponding loess-smoothed curves like \figref{fig:p-backing}.\footnote{The full set of raw tracings and loess curves discussed in this section is available at \url{https://github.com/rbennett24/articles/tree/master/Irish_pal_timing_syll_FACL}.} For example, \figref{fig:loess-PJu-all} confirms that the tongue body in onset \ipa{/Pʲuː/} tends to be raised and/or fronted at C release\is{release} (C end) relative to C start for most speakers. The degree of fronting involved varies~-- it is clearest for Connacht Speaker 2, and the Ulster speakers~-- but the overall pattern appears to hold for 5/7 participants (Connacht Speakers 1 and 3 being apparent exceptions).

\begin{figure}
    \centering
    \textbf{Onset \ipa{/Pʲuː/}, all speakers}\\
    \includegraphics[width=\linewidth]{figures/Bennett-img009.png}
    \caption{Loess-smoothed ultrasound\is{ultrasound} tracings for every speaker's \ipa{/Pʲuː/} tokens at C start, midpoint, and end.}
    \label{fig:loess-PJu-all}
\end{figure}


\figref{fig:raw-M1-Pju} provides a side-by-side comparison of raw tracings for Munster Speaker 1's onset \ipa{/Pʲuː/} productions at C start and C release\is{release} (C end). We note that (i) this speaker's productions are fairly consistent from token-to-token, and (ii) there is clear raising and fronting of the tongue body at C start relative to C end, despite the fact that the following vowel is back \ipa{/uː/} (CV coarticulation for backness is limited in Irish\il{Irish (Modern)}; \cite{NiChasaide_Fealy1991_Irish_EPG}; \cite{Bennett_etal2018_Conamara_ultrasound}; \citeyear{Bennett_etal2023_jphon_submission}).

\begin{figure}
    \centering
    \begin{tabular}{cc}
    \multicolumn{2}{c}{\textbf{Munster Speaker 1 onset \ipa{/Pʲuː/}}}\\
    \textsc{\textbf{C start}}&\textsc{\textbf{C end}}\\
    \includegraphics[width=0.4\linewidth]{figures/Bennett-img010.png}&
    \includegraphics[width=0.4\linewidth]{figures/Bennett-img011.png}\\
    \end{tabular}
    \caption{Raw ultrasound tracings for Munster Speaker 1's onset \ipa{/Pʲuː/} tokens at C start (left) and end (right).}
    \label{fig:raw-M1-Pju}
\end{figure}

We conclude that the significant results in in \figref{fig:violins} are reflected in qualitative patterns as well as quantitative ones.

\largerpage
The \emph{non}-significant results in \figref{fig:violins} must be interpreted with caution, for two reasons. First, in frequentist statistics it is not generally valid to draw conclusions from the \emph{lack} of a statistically-significant effect (this is sometimes called “accepting the null hypothesis”; \citealt{Frick1995_accepting_the_null} and many others).\footnote{In principle, Bayesian modeling can be used to justify acceptance of a null hypothesis (e.g. \citealt{Kruschke2013_Bayesian_test}). However, in the specific case at hand, nothing is gained on this front by using Bayesian rather than frequentist statistics. Bayesian \emph{t}-tests, implemented using the \textsc{ttestBF()} function in the \textsc{BayesFactor} package and the \textsc{describe\_posterior()} function in the \textsc{bayestestR} package in \textsc{R} (\cite{BayesFactor}; \cite{bayestestR}), suggest that there is indeed sufficient evidence to accept the significant effects labeled in \figref{fig:violins}. For the remaining conditions, the Bayesian \emph{t}-tests suggest that the evidence is insufficient to either accept or reject the null hypothesis of no difference. When it comes to drawing inferences about the distributions in \figref{fig:violins}, Bayesian methods leave us in the same place as regular frequentist ones.\label{fn:bayes}
% This was very helpful: https://easystats.github.io/bayestestR/articles/region_of_practical_equivalence.html
}

The second issue of interpretation surrounding \figref{fig:violins} is that a non\hyp significant effect could correspond to two different scenarios regarding articulatory timing. First, it could be that backness values at C start are comparable to those at C end because the secondary articulation reaches its peak at C start, and is then maintained more or less constant until C end. In this scenario, we expect a fairly tight cluster of points around zero at C end in \figref{fig:violins}, since the position of the tongue body should be more-or-less the same, consistently, at C start vs. C end.

Alternatively, it could be that articulatory timing is more noisy and variable: sometimes the secondary articulation peaks at C start, sometimes at C end, and sometimes in-between. In this scenario, we might expect a looser grouping of points around zero at C end. This is because differences in dorsal position at C end, relative to C start, would be less stable than in a scenario in which the secondary articulation peaks at C start and is held till C end.\footnote{A third scenario, which we reject, is that the peak of the secondary articulation varies categorically between C start and C end for those conditions that lack a significant skew away from zero. In this scenario, we would expect to find a \emph{bimodal} distribution of points on either side of zero with one peak for tokens showing clear C start alignment and another peak, on the opposite side of zero, for tokens showing clear C end alignment. However, the distributions in \figref{fig:violins} appear to be consistently unimodal across conditions and time points, and so we reject this possibility. See also \citet{Padgett_etal2023_Irish_pal_syllpos}, who use Hartigan's dip test to argue that all of the distributions in \figref{fig:violins} are unimodal.}

To try to disentangle these two interpretations, we can first consider some qualitative patterns in our data. \figref{fig:C2-coda_iT} provides loess-smoothed ultrasound\is{ultrasound} tracings for coda \ipa{/Tˠ/} following \ipa{/iː/} for one speaker (U2), at C start, midpoint, and end. We've chosen this context as an example because (i) coda \ipa{/VTˠ/} is one of the non-significant conditions in \figref{fig:violins}, and (ii) some of our other work on Irish\il{Irish (Modern)} suggests that velarized \ipa{/Cˠ/} is often significantly fronted after \ipa{/iː/} in coda position \citep{Bennett_etal2023_jphon_submission}. Thus, coda \ipa{/iːTˠ/} is a condition in which we might expect to find greater backing at C end than C start, where influence from the vowel \ipa{/iː/} could pull the consonant articulation for \ipa{/Tˠ/} forward at the VC transition.

\begin{figure}
    \centering
    \textbf{Connacht Speaker 2 coda \ipa{/iːTˠ/}}\\
    \includegraphics[width=0.8\linewidth]{figures/Bennett-img012.png}
    \caption{Loess-smoothed ultrasound\is{ultrasound} tracings for Connacht Speaker 2's \ipa{/iːTˠ/} tokens at C start, midpoint, and end.}
    \label{fig:C2-coda_iT}
\end{figure}

As is evident from \figref{fig:C2-coda_iT}, tongue body position and posture are very similar across time points. There is perhaps a slight lowering of the tongue body at C end and midpoint relative to C start, but otherwise, there are very few differences when comparing these three landmarks. This implies that the constriction for \ipa{/Tˠ/} is already achieved by the beginning of the consonant in \ipa{/iːT\vel/} for this speaker.

Loess-smoothed curves are a kind of average, which could potentially hide some informative variability across tokens. However, we can also see from \figref{fig:raw-C2-iT} that there is little variability in tongue body position across tokens at either C start or C end. This, too, is consistent with the claim that the constriction for \ipa{/Tˠ/} is achieved at the VC transition (C start) and maintained basically constant until C release\is{release} (C end).

\begin{figure}
    \centering
    \begin{tabular}{cc}
    \multicolumn{2}{c}{\textbf{Connacht Speaker 2 coda \ipa{/iːTˠ/}}}\\
    \textsc{\textbf{C start}}&\textsc{\textbf{C end}}\\
    \includegraphics[width=0.4\linewidth]{figures/Bennett-img013.png}&
    \includegraphics[width=0.4\linewidth]{figures/Bennett-img014.png}\\
    \end{tabular}
    \caption{Raw ultrasound tracings for Connacht Speaker 2's coda \ipa{/iːTˠ/} tokens at C start (left) and end (right).}
    \label{fig:raw-C2-iT}
\end{figure}

Generally speaking, visual inspection of raw tracings and loess curves suggests that the non-significant conditions in \figref{fig:violins} resemble Figures~\ref{fig:C2-coda_iT} and \ref{fig:raw-C2-iT}: tongue body position is more or less the same at C start as at C end. Still, there are two interesting exceptions to this generalization. First, coda \ipa{/Cˠ/} is sometimes realized with a fronter (less velarized) dorsal constriction at C start than at C end when following \ipa{/i:/} (\figref{fig:i-fronting}). This reflects coarticulation for backness between \ipa{/i:/} and a following \ipa{/Cˠ/}.

\begin{figure}
    \centering
    \textbf{Connacht Speaker 2 coda \ipa{/iːKˠ/}}\\
    \includegraphics[width=0.8\linewidth]{figures/Bennett-img015.png}
    \caption{Loess-smoothed ultrasound tracings for Connacht Speaker 2's \ipa{/iːKˠ/} tokens at C start, midpoint, and end.}
    \label{fig:i-fronting}
\end{figure}

At the VC transition, \ipa{/Cˠ/} is fronter than at \ipa{/Cˠ/} release\is{release}, where the influence of the \ipa{/i:/} is more removed. This is consistent with \citeauthor{Bennett_etal2023_jphon_submission}'s (\citeyear{Bennett_etal2023_jphon_submission}) finding that \ipa{/iːCˠ/} is the context in which consonant backness is most affected by coarticulation with a neighboring vowel in this dataset. 

Second, some speakers appear to transition towards a more neutral tongue body position at the end of coda C in some conditions. For example, \figref{fig:isp} shows that Ulster Speaker 2 has a fronter tongue body at C end than at C start in coda \ipa{/uːKˠ/}. Both the vowel and consonant are back in this context, so fronting is not motivated by the segments themselves.

\begin{figure}
    \centering
    \textbf{Ulster Speaker 2 coda \ipa{/uːKˠ/}}\\
    \includegraphics[width=0.8\linewidth]{figures/Bennett-img016.png}
    \caption{Loess-smoothed ultrasound tracings for Ulster Speaker 2's \ipa{/uːKˠ/} tokens at C start, midpoint, and end.}
    \label{fig:isp}
\end{figure}

Recall that our target words were produced in a frame sentence (\sectref{sec:items}), which means that each target word was followed by the word \ipa{[əˈn\vel ʊr\vel ə]} `last year'. The fronting observed in \ref{fig:isp} could reflect anticipatory coarticulation with \ipa{[ə]} in \ipa{[əˈn\vel ʊr\vel ə]}. Alternatively, it could correspond to a phrase-final tongue posture of some kind, since our target words were generally produced with focus prosody, and so may have been prosodically phrase-final. (See e.g., \cite{Gick_etal2004_rest_position}; \cite{Katsika2016_boundary_lengthening} for related discussion.)

Taking these observations together, our overall conclusion is that the non-significant conditions in \figref{fig:violins} generally represent a production pattern in which the secondary articulation is achieved at the VC transition (C start) and held more or less consonant through C release\is{release} (C end). That said, we recognize that other factors (e.g. coarticulation) may play a role as well.

Further support for this interpretation comes from a more quantitative interpretation of the patterns in \figref{fig:violins}. Our statistical question concerns the spread of the distributions in \figref{fig:violins}: in cases where backness values at C end overlap with those at C start, are the distributions at C end tightly clustered ($\approx$ alignment at C start, continued to C end) or more widely spread out ($\approx$ variable alignment)?

To answer this question, we can compare the variability in our non-significant conditions to the variability in conditions where we believe there \emph{is} a specific, consistent pattern of gestural alignment. \tabref{tab:sum-stats} reports summary statistics, including standard deviations and inter-quartile ranges, at C endpoint for each of the 12 conditions shown in \figref{fig:violins}.

\newcommand{\rbalign}[2]{\multicolumn{1}{#1}{#2}}

\begin{table}[ht]
\centering
\resizebox{\textwidth}{!}{
\begin{tabular}{rrrrrrrrr}
  \lsptoprule
 & \rbalign{c}{\textsc{Mean}}
 & \rbalign{c}{\textsc{Range}}
 & \rbalign{c}{\textsc{$\Delta$(Range)}}
 & \rbalign{c}{\textsc{sd}}
 & \rbalign{c}{\textsc{IQ range}}
 & \rbalign{c}{\textsc{$\Delta$(IQ range)}} \\
  \hline
  *Onset Pʲ & 0.017 & [-0.194,0.272] & 0.466 & 0.049 & [0,0.036] & 0.036 \\ 
  \cellcolor{gray!25}{\bfseries *Onset Pˠ} & \textbf{-0.031} & \textbf{[-0.287,0.061]} & \textbf{0.348} & \textbf{0.050} & \textbf{[-0.056,0]} & \textbf{0.056} \\ 
  \cellcolor{gray!25}{\bfseries *Onset Tʲ} & \textbf{0.038} & \textbf{[-0.045,0.286]} & \textbf{0.331} & \textbf{0.058} & \textbf{[0,0.062]} & \textbf{0.062} \\ 
  Onset Tˠ & -0.004 & [-0.204,0.095] & 0.299 & 0.041 & [-0.011,0.004] & 0.015 \\ 
  \cellcolor{gray!25}{\bfseries *Onset Kʲ} & \textbf{0.027} & \textbf{[-0.159,0.204]} & \textbf{0.364} & \textbf{0.049} & \textbf{[0,0.052]} & \textbf{0.052} \\ 
  Onset Kˠ & 0.010 & [-0.082,0.132] & 0.214 & 0.032 & [0,0.024] & 0.024 \\ 
  *Coda Pʲ & -0.011 & [-0.153,0.133] & 0.286 & 0.042 & [-0.027,0.004] & 0.031 \\ 
  *Coda Pˠ & -0.011 & [-0.228,0.2] & 0.428 & 0.051 & [-0.027,0] & 0.027 \\ 
  *Coda Tʲ & 0.019 & [-0.07,0.183] & 0.253 & 0.040 & [0,0.033] & 0.033 \\ 
  Coda Tˠ & -0.002 & [-0.138,0.154] & 0.292 & 0.044 & [-0.016,0.003] & 0.019 \\ 
  Coda Kʲ & 0.005 & [-0.09,0.131] & 0.221 & 0.040 & [-0.012,0.026] & 0.038 \\ 
  Coda Kˠ & 0.002 & [-0.092,0.181] & 0.273 & 0.048 & [-0.021,0.002] & 0.022 \\ 
   \lspbottomrule
\end{tabular}
}
\caption{Summary statistics for distributions shown at C endpoint in \figref{fig:violins}. Starred rows indicate conditions with statistically significant differences from zero in \figref{fig:violins}. \textsc{sd} = standard deviation, \textsc{IQ range} = inter-quartile range, $\Delta$\textsc{(X)} = size of range for \textsc{X}.}\label{tab:sum-stats}
\end{table}

As benchmarks, we can consider the three conditions with the largest deviations from zero, as determined by the absolute value of the mean. These are onset \ipa{/P\vel\ T\pal\ K\pal/}, which are highlighted in gray and in bold in \tabref{tab:sum-stats}. These three conditions are the clearest examples of consistent articulatory alignment in our dataset, with secondary articulations seeming to peak at C end (= release\is{release}) for all three conditions.

If we compare the measures of variance for the \emph{non-}significant conditions in \figref{fig:violins} and \tabref{tab:sum-stats}, we see that those measures are similar in size to the measures of variance in our three significant benchmark conditions. For example, the mean of onset \ipa{/T\vel/} (-0.004) is not significantly different from zero. The measures of dispersion for this condition~-- \textsc{sd} = 0.041, $\Delta\textsc{Range}$ = 0.299, and $\Delta\textsc{IQ Range}$ = 0.015~-- are in fact \emph{smaller} than the same measures for all three of our benchmark significant conditions (onset \ipa{/P\vel\ T\pal\ K\pal/}). This suggests that the values for onset \ipa{/T\vel/} are at least as tightly clustered as the values for our significant benchmarks. The same general point can be made for \emph{all} the non-significant conditions in \figref{fig:violins} and \tabref{tab:sum-stats}: the dispersion around the mean appears to be basically the same, and sometimes smaller, than found for the significant conditions demonstrating specific patterns of articulatory alignment. This is also consistent with simple visual inspection of the distributions in \figref{fig:violins}.\footnote{\citet{Padgett_etal2023_Irish_pal_syllpos} report that the distribution for non-significant onset \ipa{/T\vel/} is more spread out than expected for a normal distribution according to a Shapiro-Wilks test with Bonferroni-corrected $\alpha = 0.015/12 = 0.0042$. However, this is also true of significant onset \ipa{/P\pal\ P\vel\ T\pal/} and coda \ipa{/T\pal/}, and so this measure of dispersion also fails to distinguish our significant and non-significant conditions in \figref{fig:violins}.}

We conclude that the distributions for the non-significant conditions in \figref{fig:violins} and \tabref{tab:sum-stats} are not particularly loose or spread-out around zero. Instead, they appear to be about as tightly clustered as the distributions in our significant conditions. This is consistent with the claim that the secondary articulations in our non-significant conditions typically reach a peak at C start, and are then maintained more-or-less unchanged until C end. In the discussion below, we sometimes use the term `held alignment' to refer to this scenario, in which there is no significant change in dorsal backness from C start to C end.

% Though speculative, we interpret this as providing support for the claim that the secondary articulations in our non-significant conditions typically reach a peak at C start, and are then maintained more-or-less unchanged until C end. That said, we are basing this conclusion on a series of null results, and so it must be considered a tentative hypothesis in need of more systematic testing in the future.



%%%%%%%%%%%%%%%%%%%%%%%%%%%%%%%%
\section{Interim discussion}\label{sec:interim}
Recall our hypotheses from \sectref{sec:hypotheses}:

\begin{exe}
\exi{(1)}
The relative timing of primary and secondary articulations should be more variable in the coda than in the onset, across consonant types and speakers (and possibly within them as well). 
\end{exe}

Our measures of dispersion in \tabref{tab:sum-stats} do not support this prediction. (Nor do they support an analogous hypothesis concerning labials compared to other places of articulation.) This hypothesis would lead us to expect more spread-out distributions for codas, relative to onsets, at C end in \figref{fig:violins}. This is not the case, either visually or numerically (\tabref{tab:sum-stats}). However, we find some \emph{tentative} support for this hypothesis below when we examine the results by individual participants.

\begin{exe}
\exi{(2)} In onset stops, both palatalization and velarization\is{velarization} gestures\is{gesture} should be aligned roughly with the release\is{release} of the consonant closure. 
\end{exe}

As expected based on previous findings, for palatalization this prediction holds across all three places of articulation (Figures~\ref{fig:trajectories}, \ref{fig:violins}): values at C end for onset \ipa{/P\pal\ T\pal\ K\pal/} skew positive, and are significantly different from zero, indicating greater fronting at C end relative to C start.

As we noted earlier, previous work has not found such a clear pattern for onset velarization\is{velarization}, and we do not find compelling evidence \emph{overall} for release\is{release} alignment of velarization\is{velarization} (= significant negative skew at C end) in this study either (but see the discussion below).

\begin{exe}
\exi{(3)} In coda stops, both palatalization and velarization\is{velarization} gestures\is{gesture} should be roughly simultaneous with the primary articulation, achieved around the beginning of the consonant and maintained through to the release\is{release}. 
\end{exe}

Here we must be more cautious, given limitations of our analysis discussed in the previous section. Still, our results support such an interpretation for \ipa{/T\vel\ K\pal\ K\vel/} (Figures~\ref{fig:trajectories}, \ref{fig:violins}, discussion around \tabref{tab:sum-stats}): values at C end seem to be centered around zero for coda \ipa{/T\vel\ K\pal\ K\vel/}.

Also in a more speculative vein, we hypothesized that secondary articulations in the coda might be more aligned with the \emph{beginning} of labial closures but with the \emph{end} of coronal closures. This is based on the relative strength of formant cues for labial stops vs. release\is{release} cues for coronal stops, and it has been observed for limited Russian data (\sectref{sec:timing_percep}). Our findings depart from predictions 1 and 2 above in ways that are suggestive in this regard.

First, in our data neither secondary articulations in the coda nor velarization\is{velarization} anywhere generally favors alignment to a particular landmark~-- C beginning or release\is{release}~-- \ipa{/P/} contradicts these trends in having alignment with a landmark in all four conditions (syllable position x secondary articulation). In three out of these four conditions alignment favors vowel formant transitions (release\is{release} alignment for onsets, beginning or VC alignment in codas). The exception is coda /P\vel/, which favors release\is{release} alignment. Second, coda /T\pal/ also contradicts the above trends, in favoring release\is{release} alignment. All of these facts except for that of coda /P\vel/ are consistent with the hypothesis that alignment of secondary articulations is fine-tuned not just to the location but to the relative \emph{importance} of cues in labials vs. coronals. Assuming there is validity to these considerations, it's not clear why /P\vel/ does not follow the expectation.

There is a separate, articulatory, consideration relevant to the facts about labials. In the case of \ipa{/Tʲ Tˠ/} or \ipa{/Kʲ Kˠ/}, there is tongue coupling, and possibly direct articulatory competition, between the primary articulator and a secondary dorsal one, while this is not true of \ipa{/Pʲ Pˠ/}. Perhaps this allows other factors~-- such as constraints on perceptibility like those discussed here~-- to more freely determine timing in the latter case. However, we \emph{do} find alignment with C release\is{release} in onset \ipa{/T\pal\ K\pal/} and coda \ipa{/T\pal/}, casting some doubt on the importance of such an articulatory effect. (See \citealt{Bennett_etal2018_Conamara_ultrasound} for related discussion.)

%%%%%%%%%%%%%%%%%%%%%%%%%%%%%%%%
\section{Inter-speaker variation}\label{sec:individ-var}
In this section we provide a preliminary investigation of inter-speaker variation in timing patterns in our data. The goal is to see how well the generalizations above hold up by speaker. In addition, previous research has reported inter-speaker variation in the production of /C\vel\ C\pal/ contrasts\is{contrast} in Scottish Gaelic\il{Scottish Gaelic (Modern)} (\citealt{Sung_etal:2018}) and in Russian (\cite{Kochetov2002_diss}; \citeyear{Kochetov2009_Russian_C_variation}). As for Irish\il{Irish (Modern)}, drawing on the same dataset analyzed here, \citet{Bennett_etal2023_jphon_submission} find that speakers differ in how robustly /C\vel\ C\pal/ contrasts\is{contrast} are distinguished by dorsal position across contexts. For example, some speakers seem to completely merge coda labial /P\vel\ P\pal/ following \ipa{/iː/}, at least in terms of dorsal shape and position, while other speakers maintain distinct dorsal postures for these consonants in this context.\footnote{\citet{Bennett_etal2023_jphon_submission} report measures taken at C release\is{release} for onset consonants, and at C start in the VC transition for coda consonants; they do not investigate possible changes in articulatory posture across time points within a given consonant, as we do here.} This pattern of inter-speaker variation, along with others observed by \citet{Bennett_etal2023_jphon_submission}, raises the question of whether speakers might differ with respect to the patterns of gestural timing\is{gestural timing} reported in aggregate in \sectref{sec:results}.

\figref{fig:trajectories-SPK} shows loess-smoothed regressions for the trajectories of dorsal backness values across consonant start, midpoint, and end, for all combinations of syllable position, consonant place, and secondary articulation. This is akin to \figref{fig:trajectories}, but here separate regressions are provided for each speaker. For the sake of readability, we omit lines corresponding to the trajectories of specific tokens, as well as confidence intervals around the loess-smoothed estimates.

\begin{figure}
    \centering
    \includegraphics[width=\linewidth]{figures/Bennett-img017.png}
    \caption{By-speaker trajectories for backness of dorsal peak over C start, midpoint, end time points, relative to values at C start. Colored lines represent loess-smoothed regressions over each speaker's values at each time point.}
    \label{fig:trajectories-SPK}
\end{figure}

In general, the gestural timing\is{gestural timing} patterns we observed in \sectref{sec:results} are broadly reflected in the patterns of timing observed for each individual's data. For example, the secondary articulation for onset \ipa{/K\pal/} seems to peak at C release\is{release} for all three speakers whose data is included in this condition. Similarly, for both onset and coda \ipa{/T\pal/}, speakers mostly show a consistent pattern of aligning peak dorsal fronting with C release\is{release}. However, still focusing on /T\pal/, speakers also seem to differ in how distinct dorsal position is at C end vs. C start for this sound, and two speakers seem to show no significant fronting over time for onset \ipa{/T\pal}/.

To quantify these patterns, we computed confidence intervals around the mean of each speaker's dorsal backness values at consonant end, relative to consonant start, using a non-parametric bootstrapping resampling method, using the \textsc{smean.cl.boot} function in the \textsc{Hmisc} package in \textsc{R} (\cite{Hmisc}; \cite{RStats}). If the confidence interval around each speaker's mean at C end is significantly different from zero, this would indicate that the dorsal position of the secondary articulation peaks at release\is{release}, assuming that the mean values are fronter for \ipa{/C\pal/} and backer for \ipa{/C\vel/}. If instead the mean values are \emph{backer} for \ipa{/C\pal/} and \emph{fronter} for \ipa{/C\vel/}, this would indicate that secondary articulations peak at C start.

In order to reduce the number of statistical comparisons involved, we limited this analysis to backness values at C end. We chose non-parametric bootstrapping to estimate the means and confidence intervals, rather than a parametric method like a \emph{t}-test, because (i) bootstrapping methods do not assume that the underlying data is normally distributed, and (ii) bootstrapping is more robust for smaller sample sizes. Sample size is a potential concern here, since splitting our data up by speaker substantially reduces the number of observations per comparison: for most combinations of speaker and condition, we have just 10--15 observations at C end (minimum = 8; 58/60 comparisons have at least 10 observations). In applying non-parametric bootstrapping to this data, we sampled with replacement 10,000 times for each combination of speaker and condition. This resulted in 60 total bootstrapped estimates (\figref{fig:trajectories-SPK}).

We reiterate that caution must be taken in interpreting these results given the small sample sizes involved (see e.g. \citealt[Ch.9]{Chernick2011_bootstrap_methods} with respect to bootstrap methods, specifically). Nonetheless, we think it is useful to consider, in a broad and exploratory sense, the extent to which individual speakers might differ in their patterns of gestural timing\is{gestural timing} for secondary articulations across contexts. \tabref{tab:sig-align} reports the significant results produced by this method with a significance threshold of $\alpha = 0.05$.\footnote{Given the fact that \tabref{tab:sig-align} reports multiple comparisons, we should arguably use a Bonferroni-corrected significance threshold of $\alpha = 0.05/60 \approx 0.0008$. Because this is an extremely conservative threshold for significance, and because our aims here are mostly exploratory, we opted for the standard $\alpha = 0.05$ instead. Using the conservative threshold of $\alpha = 0.0008$ shrinks the number of significant results in \tabref{tab:sig-align} from 33 to 21, but does not affect the broad qualitative conclusions we would draw from this data.
% Re-running these \emph{t}-tests as Bayesian \emph{t}-tests (footnote \ref{fn:bayes})~-- another strategy for dealing with small sample sizes and multiple comparisons \citep{Kruschke_etal2012_Bayes_time_has_come}~-- also yields essentially the same results as the bootstrap method with $\alpha$ = .05.
}


\begin{table}
\begin{tabular}{llll}
\lsptoprule
      & P\pal  & T\pal  & K\pal \\\midrule
Onset & Release: 5/7 & Release: 3/4   & Release: 3/3  \\
Coda  & VC: 3/7  &  Release: 3/4  &  0/3 significant\\
\midrule
      & P\vel  & T\vel  & K\vel  \\\midrule
Onset &  Release: 6/7 &  Release: 2/4, C start: 1/4  &  C start: 1/3  \\
Coda  &  Release: 3/7 &  Release: 1/4  &  VC: 1/3\\
\lspbottomrule
\end{tabular}
\caption{Significant differences between distributions of backness values, relative to C start, at C end, by condition. Counts refer to number of speakers showing each pattern.}\label{tab:sig-align}
\end{table}

Again, the results in \tabref{tab:sig-align} broadly reflect the findings of \sectref{sec:results}. First, palatalized /C\pal/ consistently shows release\is{release} alignment in onset position. For each place of articulation, the majority of speakers represented in that condition show release\is{release} alignment, though not everyone. Two speakers do not align the peak of onset palatalization with the release\is{release} of /P\pal/, and likewise for /T\pal/. In most other conditions, few or no speakers show alignment with either consonant landmark. The exceptions are familiar from the discussion in \sectref{sec:interim}: several speakers \emph{do} align secondary articulations for labials and for /T\pal/.

\figref{fig:trajectories-SPK-noplace} plots the same data as in \figref{fig:trajectories-SPK}, but pooled across places of articulation, so that each estimated trajectory is based on a larger number of observations. One can see in this figure our overall findings that onset palatalization aligns with C release\is{release} while coda secondary articulations remain relatively unchanged throughout. There is release\is{release} alignment for onset velarization\is{velarization} as well, but this is mostly attributable to onset /P\vel/ as noted previously.

\begin{figure}
    \centering
    \includegraphics[width=0.65\linewidth]{figures/Bennett-img018.png}
    \caption{By-speaker trajectories for backness of dorsal peak over C start, midpoint, end time points, relative to values at C start. Colored lines represent loess-smoothed regressions over each speaker's values at each time point, collapsing across place of articulation.}
    \label{fig:trajectories-SPK-noplace}
\end{figure}

On the whole, we can draw two tentative conclusions from the above. First, alignment patterns for individual speakers more or less replicate the overall population trends reported in \sectref{sec:results}. However, second, there are signs of the variability predicted by our first hypothesis in \sectref{sec:interim}. Though we did not find greater variability for codas (or for labials) in our pooled data, here more \emph{patterns} of alignment occur in codas than in onsets. Specifically, while \emph{release\is{release}} alignment is the only pattern found for onset consonants (apart from limited cases of C start alignment for velarized /T\vel\ P\vel/), there is release\is{release}, VC transition, and held alignment in the case of codas. \footnote{When alignment is not aligned uniquely with the beginning or end of the consonant (= `held alignment'), this might be understood as alignment with both, i.e. simultaneous production, as discussed earlier.} There is likewise some suggestive evidence here that speakers may differ, to some extent, in their preferred timing patterns across different contexts.

To be sure, more work needs to be done to verify that these patterns of variability reflect bonafide inter-speaker differences, rather than accidental facts about our particular sample of speakers and recordings. Confirming these patterns of variability in a larger dataset and exploring potential underlying causes (e.g. dialect variation) are  clear topics for future research.

%%%%%%%%%%%%%%%%%%%%%%%%%%%%%%%%%%
% \section{An alternative account of the onset-coda asymmetry in palatalization?}\label{sec:codaphasing}
\section{Against anti-phase timing as the source of onset-coda asymmetries in palatalization}\label{sec:codaphasing}
Our discussion has assumed that secondary dorsal articulations in Irish\il{Irish (Modern)} are coordinated with consonantal landmarks, as defined by the production of the consonant's primary articulation. This strikes us as a reasonable assumption: after all, palatalization and velarization\is{velarization} are properties of consonants, at least phonologically.

Beyond alignment with consonant landmarks, we have suggested (\sectref{sec:timing_percep}) that timing of secondary articulations might be constrained in perceptually adaptive ways. For example, in coda position, speakers must `choose' whether to align the peak of a secondary articulation near the V-C boundary, where formant transition cues reside, or at C release\is{release}, where there are cues from burst quality, or with both landmarks. Both of these landmarks are important, perceptually speaking, for expressing secondary articulation contrasts\is{contrast}. In coda position, unlike in onset position, these landmarks do not coincide, and so speakers can align peak /C\pal\ C\vel/ gestures\is{gesture} at the V-C transition, or at the C release\is{release}; or they can try to maintain the peak /C\pal\ C\vel/ gesture\is{gesture} throughout the entire consonant, so that the gesture\is{gesture} occurs at \emph{both} the V-C transition and C release\is{release}. Evidently, speakers adopt different timing strategies in coda position depending on the place and secondary articulation of the consonant, in ways that seem largely consistent with the hypothesis of perceptual optimization (\sectref{sec:results}).

However, given the broad asymmetry we find between patterns of gestural alignment in the onset vs. coda for palatalization, we might look elsewhere for an understanding of what is happening. Studies of articulatory coordination have reported asymmetries in how segment-internal gestures\is{gesture} are timed in onset vs. coda position. To illustrate, consider the American English\il{English (Modern)} lateral \ipa{/l/}, which has both a primary tongue tip constriction, and a secondary, vowel-like dorsal constriction (e.g. \cite{Krakow1999_physiological_syllables}; \cite{Sproat_Fujimura1993_l-allophony}; \cite{Proctor2009_liquids}; \cite{Lee_etal2013_Eng_l_darkening_morphology}; \cite{Turton2017_l_darkening}). In onset position (e.g. \emph{\uline{l}ab}), the coronal and dorsal gestures\is{gesture} for \ipa{/l/} are largely synchronous. However, in coda position (e.g. \emph{ba\uline{ll}}), they appear to be produced sequentially, with the dorsal constriction preceding the coronal constriction in time. In codas, the overlap of the dorsal constriction for \ipa{/l/} with the preceding vowel leads to a distinct and highly audible coloration of the vowel itself.

Similar observations have been made for the relative timing of oral closure and velum lowering in the American English\il{English (Modern)} nasals \ipa{/m n/} (\cite{Krakow1999_physiological_syllables}; \cite{Byrd_etal2009_nasal_timing}). In this case, velar lowering is largely synchronous with oral closure formation in onsets, but precedes closure formation in codas, producing extensive coarticulatory nasalization on the preceding vowel. Generalizing over these results, we can perhaps say that more open constrictions tend to precede more narrow constrictions for articulatorily complex consonants in coda position, at least in American English\il{English (Modern)} (\cite{Krakow1999_physiological_syllables}; \cite{Sproat_Fujimura1993_l-allophony}; \cite{Iskarous_Kavitskaya2010_phonetic_variability}).

These asymmetries in articulatory timing have been interpreted as a reflection of different underlying patterns of gestural coordination\is{gestural coordination} in onsets vs. codas. In onsets, gestures\is{gesture} tend to be coordinated \emph{in-phase}, or synchronously. In codas, gestures\is{gesture} tend to be coordinated \emph{anti-phase}, or sequentially. It has been argued that in-phase coordination is inherently more stable than anti-phase coordination, not just in speech but in motor planning and execution more generally (e.g. \cite{Goldstein_etal2006_gesture_phono_evolution}; \cite{Nam_etal2009_syll_coupled_oscil}; \cite{Parrell2012_gestures_Spanish_aspiration} and references there). We might thus expect gestural alignment between primary and secondary constrictions to be more variable in codas for \ipa{/C\pal\ C\vel/}, exactly as we find (at least for palatalization), if those constrictions are timed in an anti-phase relationship.

There are at least two reasons to doubt this alternative analysis. First, Irish\il{Irish (Modern)} palatalization timing is more variable in the coda only in the sense of allowing more \emph{patterns} of alignment depending on place of articulation (and to some extent speaker), as just discussed: alignment tends toward C start for /P\pal/, C release\is{release} for /T\pal/, and neither for /K\pal/. In other words, as shown in \figref{fig:violins} and \tabref{tab:sum-stats}, differences between onset and coda position in our study mostly boil down to with which landmark the secondary articulation is coordinated. Measures of variance for the timing of primary and secondary articulations actually imply similar variance in onset vs. coda position, \emph{not} increased variance in codas.

Second, palatalization is \emph{not} simultaneous with the primary constriction in onset position; rather, this study and previous ones have found a very consistent lag in the palatalization gesture\is{gesture}, in both Irish\il{Irish (Modern)} and Russian, such that the peak occurs around the release\is{release} of the primary constriction (see also \citealt{Shaw_etal2021_complex_segments}). As for coda position, though secondary articulations in our data peak around the onset of the primary constriction in the case of /P\pal/, this is not true at all for other conditions. Furthermore, as we saw in \sectref{sec:timing_percep}, for Russian consonants other than stops, the facts suggest secondary articulations that are co-extensive with the primary ones, as we have found for some conditions here.

In the current study, we can investigate this question empirically, building on the discussion of Russian in \citet{Iskarous_Kavitskaya2010_phonetic_variability}. If secondary \ipa{/C\pal\ C\vel/} are timed anti-phase (sequentially), then the gestural peak for \ipa{/C\pal\ C\vel/} should occur during the vowel, well prior to the achievement of the primary constriction. This would mirror findings for American English\il{English (Modern)} \ipa{/l/} and nasals in coda position, in which the dorsal constriction for \ipa{/l/} and the velar lowering for \ipa{/m n/} are timed to reach a peak during the vowel itself.

A major acoustic correlate for the \ipa{/C\pal\ C\vel/} contrast\is{contrast} in Irish\il{Irish (Modern)} (and other languages) is the F2 transition on neighboring vowels. F2 largely reflects dorsal backness in this context; it is also affected by lip rounding (e.g. \citealt{Bennett_etal2019_Irish_contrast_enhance}). If \ipa{/C\pal\ C\vel/} gestures\is{gesture} peak during the vowel in VC\# coda sequences, as predicted by anti-phase alignment, F2 values should show evidence of peaking during the vowel as well.

To assess this prediction, we measured F2 during the vowel in /VC\#/ sequences, where our target consonant was a word-final coda consonant. The vowel was segmented from the preceding consonant at the beginning of voicing: since our focus is on VC transitions, not CV transitions, the criterion used to identify the beginning of the vowel is not especially important. The vowel was segmented from the following target stop consonant at the point where spectral energy in the vowel abruptly dropped off, particularly in the region of F2 
(see e.g. \citealt{Turk_etal2006_acoustic_prosody_research}). Any preaspiration on the following final consonant was segmented as part of the vowel itself, because significant formant transitions may occur during periods of aspiration (e.g., \citealt{Stevens_Klatt1974_formants_stop_voicing}; see also \citealt{Clayton2010_preaspiration_DISS}). Formant tracking may be less accurate during preaspiration than during vowels, given the presence of aperiodic noise and the relative weakness of higher formants during aspiration. That said, visual inspection of a number of preaspirated tokens suggests that F2 was still tracked fairly well during preaspirated transitions. 630 vowels were segmented in this analysis.\footnote{F1 tracking was noisy, particularly for our Ulster speakers. We assume this is due to preaspiration, which appears to have less of an effect on F2 tracking (the formant we're mostly interested in here).}
% E.g. CM-subject1_list1_sent27, or Ulster 1


F1, F2, and F3 were extracted from each vowel using the Fast Track Praat plugin (\cite{Praat}; \cite{Barreda2021_FastTrack_formant_tracking}).\footnote{\url{https://github.com/santiagobarreda/FastTrack}} The first three formants were measured every 2ms during the vowel, and the resulting formant trajectories were then time-normalized before plotting (\figref{fig:lognormf2}). Formants were extracted using the default settings of the Fast Track plugin, except that the number of analysis steps was set to 24.

In order to pool F2 measurements across speakers, each speaker's formant values were normalized using Barreda-Neary log-additive regression normalization \citep{Barreda_Neary2018_regression_vnorm}. These normalized values were highly correlated ($r = 0.96$) with normalized values produced by a different method, namely dividing F1 and F2 by F3 at the same measurement point (e.g. \citealt{Monahan_Idsardi2010_formant_ratios}). Outliers were excluded from analysis by removing values for normalized F2 that were greater than 2 standard deviations away from the mean of the normalized F2 values. This resulted in the elimination of 460 F2 measurements, about 0.7\% of the original F2 data. This left 66,677 F2 measurements for analysis (630 vowels, with a mean of 106 F2 measurements per vowel).

\figref{fig:lognormf2} provides loess-smoothed F2 trajectories during the vowel in /VC\#/ sequences, comparing coda \palvelcon\ at each place of articulation in combination with the vowels \ipa{/iː ɔː uː/}.

\begin{figure}
    \centering
    \includegraphics[width=\linewidth]{figures/Bennett-img019.png}
    \caption{Pooled, speaker-normalized F2 trajectories across time-normalized steps, grouped by primary place of articulation and vowel context, for coda /VC\#/. Lines represent loess-smoothed regressions over F2 values preceding \palvelcon\ in each condition: a solid black line for \ipa{/C\vel/}, a dashed red line for \ipa{/C\pal/}. A gray vertical line marks 50\% point of the vowel.}
    \label{fig:lognormf2}
\end{figure}

\newpage
Our empirical question is when F2 peaks for /C\pal/, or reaches a minimum for /C\vel/, in /VC\#/ sequences. Since F2 is our acoustic surrogate for dorsal fronting and backing, extreme values for F2 should correspond to the time points at which the tongue body is furthest front for /C\pal/, or back for /C\vel/.

Answering this question is easiest for /VC\#/ sequences in which the vowel and following consonant have different backness specifications. This is because we expect to find very salient F2 movements in cases of mismatched vowel and consonant backness \citep{Bennett_etal2018_Conamara_ultrasound}, which should make it easy to see where F2 trajectories reach their most extreme values. In our data, this corresponds to \ipa{/iːC\vel/} sequences for velarized coda consonants, and \ipa{/uːC\pal/} and \ipa{/ɔːC\pal/} sequences for palatalized coda consonants.\footnote{It should be acknowledged that our use of the symbols \ipa{/iː uː ɔː/} to describe vowel quality obscures some variation in the quality of these vowels between speakers. Most notably, our Ulster speakers tend to front \ipa{/uː ɔː/}, often producing them as something like \ipa{[ʌː ɛː]}.
% Ulster speaker \#2 produces something like \ipa{[ɡʲlʲʌkˠ]} rather than \ipa{[ɡʲlʲuːkˠ]} `peep'.
Note, though, that the fronting of \ipa{/uː ɔː/} should actually make palatalized \ipa{/Cʲ/} seem \emph{less} front, by comparison, in this context. Despite that fact, we still see clear fronting at C start relative to the preceding vowel for \ipa{/Cʲ/} after \ipa{/uː ɔː/} in \figref{fig:lognormf2}.}

We begin with \ipa{/uːC\pal/}, corresponding to dashed red lines in the bottom row of \figref{fig:lognormf2}. Here, it is clear that F2 begins to rise around V midpoint, or even earlier, and does not reach its peak until the beginning of the following consonant. The same pattern holds, perhaps even more clearly, for \ipa{/Cʲ/} after \ipa{/ɔ:/} (dashed red line, middle row).

For \ipa{/iːC\vel/} sequences (solid black line, top row) it is again clear that F2 reaches an extreme value at C start, rather than any earlier time point. A potential exception is \ipa{/iːk\vel/}, which shows a relatively flat F2 trajectory throughout the vowel; but even here we can see a slight lowering of F2 at the VC transition, into the beginning of the consonant.

The same patterns are visible if we plot F2 trajectories in physical rather than normalized time. \figref{fig:lognormf2-ms} shows vowel F2 for up to 200ms preceding the following coda consonant. Vertical arrows mark when F2 reaches its expected extreme value in each loess curve for \ipa{/iːCˠ uːCʲ ɔːCʲ/}, the contexts in which there is a mismatch in backness between the vowel and following coda consonant.

\begin{figure}
    \centering
    \includegraphics[width=\linewidth]{figures/Bennett-img020.png}
    \caption{Pooled, speaker-normalized F2 trajectories across time, grouped by primary place of articulation and vowel context, for coda /VC\#/. X-axis shows ms preceding closure of following coda consonant (up to 200ms). Arrows mark F2 extrema of loess curves for \ipa{/iːCˠ uːCʲ ɔːCʲ/} (peak for \ipa{/Cʲ/}, trough for \ipa{/Cˠ/}).}
    \label{fig:lognormf2-ms}
\end{figure}

We can compare the timecourse of F2 movement in \figref{fig:lognormf2-ms} to the timing lags for the tongue tip and body gestures\is{gesture} of non-contrastive dark \ipa{[ɫ]} reported by e.g. \citet{Sproat_Fujimura1993_l-allophony} and \citet{Gick_etal:2006_liquids}. \citet[299, 303]{Sproat_Fujimura1993_l-allophony} report lags of 30--110ms between an earlier dorsal constriction and a later tongue tip constriction for coda \ipa{[ɫ]} in American English\il{English (Modern)}. \citet[Table 2]{Gick_etal:2006_liquids} report smaller lags ($\approx$ 20--25ms) for Western Canadian English\il{English (Modern)}, Quebec French\il{French (Modern)}, and Squamish Salish.\footnote{The difference in lag magnitudes reported by \citet{Sproat_Fujimura1993_l-allophony} vs. \citet{Gick_etal:2006_liquids} likely reflects the fact that \citeauthor{Sproat_Fujimura1993_l-allophony} investigated \ipa{/l/} darkening across a wider range of contexts, leading to greater variability in segment duration. Inter-speaker variation may also play a role, as \citet[65]{Gick_etal:2006_liquids} note. \citet{Gick_etal:2006_liquids} is also an ultrasound\is{ultrasound} study with a relatively low framerate (30 fps $\approx$ 1 frame every 33ms), so their reported lag values are not particularly precise.} \citet{Browman_Goldstein1995_gestural_syll_position} present 4 X-ray microbeam tokens of `peel' \ipa{[pʰiɫ]} from one speaker, and find that tongue body retraction for \ipa{[ɫ]} peaks approximately 20--50ms before tongue tip closure for \ipa{[ɫ]} is completed, i.e. during the preceding vowel.

In contrast\is{contrast}, \figref{fig:lognormf2-ms} shows that F2 extrema for these loess curves generally occur at the beginning of closure for the coda consonant, at least in those conditions where F2 transitions are most visible and interpretable due to a mismatch in backness between the vowel and following coda consonant. The only exception is \ipa{/ɔːPʲ/}, which has just a 14ms lag between the F2 peak during the vowel and the following coda \ipa{/Pʲ/} in our data. There is clearly a contrast\is{contrast} between these timing patterns, which imply (near) simultaneity between primary and secondary articulations in \ipa{/VC/}, and the more sequential timing reported for non-contrastive dark \ipa{[ɫ]} in \citet{Sproat_Fujimura1993_l-allophony}, and \citet{Gick_etal:2006_liquids} (see also \cite{Browman_Goldstein1995_gestural_syll_position}; \cite{Krakow1999_physiological_syllables}).\footnote{\citet{Gick_etal:2006_liquids} similarly find simultaneous alignment of primary and secondary articulation in Serbo-Croatian /l\vel/, rather than anti-phase alignment as in English\il{English (Modern)}. They relate this finding to the fact of a palatalization contrast\is{contrast} for laterals in the language (/l\vel/ vs. /l\pal/).\label{fn:serb}}

These patterns are incompatible with the claim that dorsal gestures\is{gesture} for secondary \palvelcon\ articulations in coda position peak during the preceding vowel. Consequently, they are also incompatible with the claim that secondary \palvelcon\ articulations are timed to precede primary articulations in coda position. That is, the available evidence argues \emph{against} the hypothesis that primary and secondary articulations are timed in a sequential, anti-phase relation in coda position.

The alternative hypothesis, which we adopt, is that secondary \palvelcon\ articulations are timed \emph{in}-phase with either C start or C release\is{release}, or with both, in coda position. (See also \citealt{Shaw_etal2021_complex_segments} on the timing of palatalization gestures\is{gesture} with primary articulations in onset position.) We propose that this variability in timing reflects the fact that there is no single perceptually-optimal timing pattern in coda position, unlike onset position. In the absence of a unique, optimal timing pattern, speakers may be expected to show variation between different alternatives, giving rise to the gestural timing\is{gestural timing} patterns we report here.



%%%%%%%%%%%%%%%%%%%%%%%%%%%%%
\section{Discussion and conclusion}
We began with the observation that secondary \palcongen\ contrasts\is{contrast} are often eliminated in coda position and among labial consonants. This typological fact likely has a phonetic source in the relative difficulty of perceiving \palcongen\ contrasts\is{contrast} in these contexts (e.g. \cite{Kochetov2002_diss}; \citeyear{Kochetov2004_perception_place}; \cite{NiChiosain_Padgett2012_Irish_pal_acous_percep}; \cite{Padgett_NiChiosain2018_Russian_Irish_pal_percep}), and the tendency to reduce articulatory gestures\is{gesture}, including palatalization gestures\is{gesture}, in codas (e.g. \cite{Kochetov2002_diss}; \citeyear{Kochetov2006_syll_poss}; \citeyear{Kochetov2009_Russian_C_variation}; \cite{Bennett_etal2023_jphon_submission}).

Our results may extend this broad line of reasoning, to the extent that they suggest greater variability in timing strategies in coda position compared to onset position. Timing patterns which are more variable across contexts (such as place and syllable position) might be harder for listeners to attend to. If timing patterns also differ by speaker, this would be even more clearly true, because listeners cannot know ahead of time when the peak dorsal gestures\is{gesture} for \palcongen\ contrasts\is{contrast} will occur.

However, we have also argued that productions may be organized in ways that make life easier for listeners (whether this is intentional or not, see discussion in \sectref{sec:timing&contrast}). In our data, secondary articulations in onset position are either timed to peak at C release\is{release}, or appear to be held constant from C beginning to C release\is{release}. In either case, gestural timing\is{gestural timing} makes key acoustic information about \palcongen\ contrasts\is{contrast} available at crucial perceptual landmarks, namely C release\is{release} itself, and the following CV transition for onsets.

In coda position, consonants at different places of articulation show different timing patterns. The palatalization gesture\is{gesture} for coda labial \ipa{/P\pal/} peaks at the VC transition. We have suggested that this reflects the perceptual importance of formant transitions for signaling \palcongen\ contrasts\is{contrast} in labials, given the weak and relatively uninformative release\is{release} bursts of labial stops. For coda coronal \ipa{/T\pal/}, we instead find alignment of secondary articulations to C release\is{release}; we have suggested that this reflects the acoustic and perceptual salience of coronal stop releases\is{release}, which are very informative as to \palcongen\ contrast\is{contrast} in Irish\il{Irish (Modern)} (e.g. \citealt{NiChiosain_Padgett2012_Irish_pal_acous_percep}). If the secondary articulations for coda \ipa{/T\vel\ K\pal\ K\vel/} are held roughly constant until C release\is{release}, as we've argued, that too may reflect the perceptual importance of release\is{release} cues for coronal and dorsal stops.

We have seen that timing patterns for velarized \ipa{/C\vel/} seem to be more ambiguous, or perhaps more varied, than the timing patterns found for palatalized \ipa{/C\pal/}, particularly in onset position (as also found by \citealt{Bennett_etal2018_Conamara_ultrasound}). Why should this be? One idea pursued in \citet{Bennett_etal2023_jphon_submission} connects this finding to an independently known fact about palatal constrictions: they are more resistant to coarticulation (and more likely to induce coarticulation in other sounds) than many other kinds of sounds. Put differently, palatal constrictions exhibit a high degree of articulatory constraint (\cite{Recasens_etal1997_lingual_coarticulation}; \cite{Recasens1999_lingual_coarticulation}; \cite{Recasens_Espinosa2009_coarticulatory_resistance}; \cite{Farnetani_Recasens2010_coarticulation}; \cite{Recasens_Rodriguez2016_coarticulation_ultrasound}). A reviewer suggests a second (possibly related) idea: that tongue body fronting may involve less complex muscle engagement than tongue body raising and backing (see e.g. \citealt[152--158]{Gick_etal2012_articulatory_phonetics_textbook}). Finally, we know that lip rounding plays a role in enhancing velarization\is{velarization} in Irish\il{Irish (Modern)} \citep{Bennett_etal2019_Irish_contrast_enhance}. A fuller understanding of gestural timing\is{gestural timing} in \palvelcon\ contrasts\is{contrast} may therefore require attention to the movement of the lips, as well as the tongue. We leave all of these ideas for further research.

It should be borne in mind that our results here pertain only to stop consonants. As discussed in \sectref{sec:timing_percep}, previous findings on Russian suggest that timing facts may depend on a consonant's manner of articulation.\footnote{\citet{Bennett_etal2018_Conamara_ultrasound} do not find a difference in timing between Irish\il{Irish (Modern)} stops and fricatives, but they only examine consonants in word-/syllable-initial position.} Unlike stops, other consonants such as fricatives, nasals and liquids do not have release\is{release} bursts, and they may carry internal cues to the palatalization contrast\is{contrast}. Whether a lateral is palatalized or velarized, for example, is easily audible during closure. Hence, we do not readily expect our findings to extend to other manners of articulation and further research on this topic is needed.

Our characterization of the issues has been oversimplified in an important way: we have discussed cues to palatalization without considering the larger phonetic context in which a word occurs. If a word like \emph{p\'{\i}ob} \ipa{/pʲiːbˠ/} occurs in isolation, then indeed the cues to palatalization of the initial consonant lie only at the consonant's release\is{release}, and those of the final consonant lie at both the VC transition and the consonant's release\is{release} (assuming the consonant is released\is{release}, as it generally would be in Irish\il{Irish (Modern)}). However, our target words were in a different context, since we used the carrier sentence \ipa{[ˈd\vel uːr\pal t\pal\ ˈiːf\vel ə \_\_\_ əˈn\vel ʊr\vel ə]} `Aoife said \_\_\_ last year'. They were preceded by a vowel-final word and followed by a vowel-initial one. In such a context, a listener could, in principle, have access to VC formant transitions, a burst, and CV formant transitions for \emph{both} initial and final consonants. Why, then, did we not find identical patterns of timing for initial and final consonants? 

Had we manipulated the phrasal context, independently varying the immediate segmental context of initial versus final consonants, and had this had effects on timing, we might have concluded that speakers can adjust the relative timing of primary and secondary articulations `online'; that is, speaker adjustments can be made in a way that adapts to phrasal context. We did not manipulate phrasal context. However, since we in fact found \emph{different} patterns of timing for initial and final consonants, even though the segmental context was the same (V\uline{~~}V), our conclusion might rather be that speakers \emph{cannot} adjust the relative timing of articulations in this way. Though somewhat speculative, this conclusion seems well in line with the hypothesis that word-internal gestural timing\is{gestural timing} is lexically-specified, and so less variable than the relative timing of gestures\is{gesture} across a word boundary (\cite{Browman_Goldstein1988_syll_struc}; \cite{Byrd1996_phase_windows}; \cite{Browman_Goldstein2000_competing_coordination}), an idea supported by the results of \citet{Cho2001_boundaries_overlap} (see \cite{Strycharczuk2019_phonetics_morphology}; \cite{Mousikou2021_coarticulation_morphemes} and references therein for further evidence and discussion). The question then becomes: what patterns of timing, fixed in the lexicon, would be most perceptually adaptive? Such patterns of timing would have to facilitate the perception of cues to palatalization not only in the most accommodating of environments (such as that of our carrier sentence); they would have to succeed in \emph{all} environments, including ones where a consonant is preceded or followed by silence or another consonant (see \citealt[30--31]{Flemming2001_unified_model} for a similar point). It is possible that lexically-specified gestural timing\is{gestural timing} `gravitates' to settings that work best for listeners across all environments.

Overall, our data suggests that relatively low-level timing patterns in the production of Irish stop consonants may be structured in such a way as to maximize the perceptual salience and recoverability of cues to \palcongen\ contrasts\is{contrast}.





\section*{Abbreviations}
\begin{tabularx}{.5\textwidth}{@{}lQ@{}}
\textsc{sg.} & singular\\
\textsc{pl.} & plural\\
\textsc{gen.} & genitive\\
fps & frames per second\\
\end{tabularx}%
\begin{tabularx}{.5\textwidth}{@{}lQ@{}}
P & labial stop\\
T & coronal stop\\
K & dorsal stop\\
\end{tabularx}

\section*{Acknowledgements}
{Thanks to audiences at HISPhonCog and ICPhS in 2023, as well as Washington University in St. Louis and the University of Connecticut, for feedback on earlier versions of this work. We are also grateful to Nick Van Handel, and a number of undergraduate research assistants, for helping us process the data presented on here. This paper was improved by helpful comments from two anonymous reviewers, who we thank as well.}

\section*{Contributions}
Ryan Bennett and Jaye Padgett contributed to conceptualization of the project, design of the methodology and items, data collection, data analysis, and writing.
Grant McGuire and M\'aire N\'{\i} Chios\'ain contributed to conceptualization of the project, design of the methodology and items, data collection, reviewing of the manuscript, and editing.
Jennifer Bellik contributed to data analysis.


\appendixsection{Words used in the study}
\label{appendix:bennett:A}

\begin{table}[H]
\begin{tabular}{llll}
\lsptoprule
&{p/b}&{t/d}&{k}\\\midrule
{\ipa{iː}} & \ipa{b\vel iː} `yellow' & \ipa{t\vel iː} `straw' & \ipa{k\vel iː} `way' \\
& \ipa{p\pal iːs\vel ə} `piece' & \ipa{d\pal iːn\vel}  `roof' & \ipa{k\pal iːr\vel} `a comb'\\\addlinespace
{\ipa{uː}} & \ipa{p\vel uːk\vel ə} `ghost' & \ipa{t\vel uːs\vel} `a start' & \ipa{k\vel uː} `a hound'\\
& \ipa{b\pal uː} `would be worth' & \ipa{t\pal uːs\vel} `thickness' & \ipa{k\pal u:n\vel əs\vel} `quiet (noun)'\\\addlinespace
{\ipa{ɔː}} & p\vel\ipa{ɔː} `pay, wages' & t\vel\ipa{ɔː} `is (be)' & \ipa{k\vel ɔːr\vel} `car'\\
& \ipa{p\pal ɔːn\vel} `pen'  & \ipa{t\pal ɔːn\vel}  `tight' & \ipa{k\pal ɔːn\vel} `one/head'\\
\lspbottomrule
\end{tabular}
\caption{Onset/word-initial position}
\label{tab:OnsetWords}
\end{table}

\begin{table}[H]
\begin{tabular}{llll}
\lsptoprule
&{p/b}&{t/d}&{k/\ipa{g}}\\\midrule
{\ipa{iː}} & p\pal\ipa{iː}b\vel\ `pipes (\textsc{gen.pl})' & \ipa{i:əd\vel} `them'& \ipa{i:ək\vel} `pay' \\
& \ipa{p\pal iːb\pal} `pipes' & \ipa{tr\pal iːd\pal} `through' & b\pal\ipa{e:}k\pal `shout'\\\addlinespace
{\ipa{uː}} & \ipa{l\vel uːb\vel} `stitch' & \ipa{uːd\vel} `that' & \ipa{ɡl\pal uːk\vel} `peep'\\
& \ipa{l\vel uːb\pal} `stitch (\textsc{dat.sg.}/variant)' & \ipa{kl\vel uːd\pal} `covering' & \ipa{b\vel uːk\pal} `pinnacle' \\\addlinespace
{\ipa{ɔː}} & \ipa{l\vel ɔːb\vel} `mud (variant)' & \ipa{st\vel ɔːt\vel} `state' & \ipa{f\vel ɔːɡ\vel} `leave'\\
& \ipa{l\vel ɔːb\pal} `mud' & \ipa{ɔːt\pal} `place' & \ipa{r\vel ɔːɡ\pal} `sudden rush'\\
\lspbottomrule
\end{tabular}
\caption{Coda/word-final position}
\label{tab:CodaWords}
\end{table}

\appendixsection{Token counts}
\label{appendix:bennett:B}

\begin{longtable}{lclcr}
\caption{Token counts by speaker, place, and secondary articulation}\label{tab:token-counts}\\
\lsptoprule
Dialect & Speaker & C place & /C\vel\ C\pal/ & $n$ \\\midrule
\endfirsthead
\midrule
Dialect & Speaker & C place & /C\vel\ C\pal/ & $n$ \\\midrule
\endhead
\endfoot
\lspbottomrule
\endlastfoot
Connacht& 1 & Labial & C\vel\ &  89 \\ 
  Connacht& 1 & Labial & C\pal\ &  90 \\ 
  \midrule
  Connacht& 2 & Labial & C\vel\ &  72 \\ 
  Connacht& 2 & Labial & C\pal\ &  72 \\ 
  Connacht& 2 & Coronal & C\vel\ &  71 \\ 
  Connacht& 2 & Coronal & C\pal\ &  66 \\ 
  Connacht& 2 & Dorsal & C\vel\ &  72 \\ 
  Connacht& 2 & Dorsal & C\pal\ &  71 \\ 
  \midrule
  Connacht& 3 & Labial & C\vel\ &  90 \\ 
  Connacht& 3 & Labial & C\pal\ &  89 \\ 
  Connacht& 3 & Dorsal & C\vel\ &  84 \\ 
  Connacht& 3 & Dorsal & C\pal\ &  87 \\ 
  \midrule
  Munster& 1 & Labial & C\vel\ &  90 \\ 
  Munster& 1 & Labial & C\pal\ &  90 \\ 
  Munster& 1 & Coronal & C\vel\ &  85 \\ 
  Munster& 1 & Coronal & C\pal\ &  83 \\ 
  \midrule
  Munster& 2 & Labial & C\vel\ &  86 \\ 
  Munster& 2 & Labial & C\pal\ &  88 \\ 
  Munster& 2 & Coronal & C\vel\ &  79 \\ 
  Munster& 2 & Coronal & C\pal\ &  85 \\ 
  \midrule
  Ulster& 1 & Labial & C\vel\ &  90 \\ 
  Ulster& 1 & Labial & C\pal\ &  90 \\ 
  Ulster& 1 & Coronal & C\vel\ &  90 \\ 
  Ulster& 1 & Coronal & C\pal\ &  86 \\ 
  \midrule
  Ulster& 2 & Labial & C\vel\ &  90 \\ 
  Ulster& 2 & Labial & C\pal\ &  90 \\ 
  Ulster& 2 & Coronal & C\vel\ &  90 \\ 
  Ulster& 2 & Coronal & C\pal\ &  84 \\ 
  Ulster& 2 & Dorsal & C\vel\ &  90 \\ 
  Ulster& 2 & Dorsal & C\pal\ &  90 \\ 
\end{longtable}

\is{velarization|)}
\is{palatalization|)}

\printbibliography[heading=subbibliography,notkeyword=this]
\end{document}
