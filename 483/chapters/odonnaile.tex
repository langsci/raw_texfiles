\documentclass[output=paper,colorlinks,citecolor=brown]{langscibook}
\ChapterDOI{10.5281/zenodo.15654875}
\title{Bunadas: A network database of cognate words, with emphasis on Celtic}
\author{Caoimhín P. Ó Donnaíle\affiliation{Sabhal Mòr Ostaig}}

\IfFileExists{../localcommands.tex}{
   \usepackage{langsci-optional}
\usepackage{langsci-gb4e}
\usepackage{langsci-lgr}

\usepackage{listings}
\lstset{basicstyle=\ttfamily,tabsize=2,breaklines=true}

%added by author
% \usepackage{tipa}
\usepackage{multirow}
\graphicspath{{figures/}}
\usepackage{langsci-branding}

   
\newcommand{\sent}{\enumsentence}
\newcommand{\sents}{\eenumsentence}
\let\citeasnoun\citet

\renewcommand{\lsCoverTitleFont}[1]{\sffamily\addfontfeatures{Scale=MatchUppercase}\fontsize{44pt}{16mm}\selectfont #1}
  
   %% hyphenation points for line breaks
%% Normally, automatic hyphenation in LaTeX is very good
%% If a word is mis-hyphenated, add it to this file
%%
%% add information to TeX file before \begin{document} with:
%% %% hyphenation points for line breaks
%% Normally, automatic hyphenation in LaTeX is very good
%% If a word is mis-hyphenated, add it to this file
%%
%% add information to TeX file before \begin{document} with:
%% %% hyphenation points for line breaks
%% Normally, automatic hyphenation in LaTeX is very good
%% If a word is mis-hyphenated, add it to this file
%%
%% add information to TeX file before \begin{document} with:
%% \include{localhyphenation}
\hyphenation{
affri-ca-te
affri-ca-tes
an-no-tated
com-ple-ments
com-po-si-tio-na-li-ty
non-com-po-si-tio-na-li-ty
Gon-zá-lez
out-side
Ri-chárd
se-man-tics
STREU-SLE
Tie-de-mann
}
\hyphenation{
affri-ca-te
affri-ca-tes
an-no-tated
com-ple-ments
com-po-si-tio-na-li-ty
non-com-po-si-tio-na-li-ty
Gon-zá-lez
out-side
Ri-chárd
se-man-tics
STREU-SLE
Tie-de-mann
}
\hyphenation{
affri-ca-te
affri-ca-tes
an-no-tated
com-ple-ments
com-po-si-tio-na-li-ty
non-com-po-si-tio-na-li-ty
Gon-zá-lez
out-side
Ri-chárd
se-man-tics
STREU-SLE
Tie-de-mann
}
   \boolfalse{bookcompile}
   \togglepaper[10]%%chapternumber
}{}

\AffiliationsWithoutIndexing


\abstract{Bunadas is a \isi{network database} of \isi{cognate} words, freely available as a WWW resource. It contains so far about 100,000 words in about 40 Indo-European languages ancient and modern, going back to PIE itself, and including 70,000 in the Celtic languages. \is{cognate}Cognate words are useful for language learning, especially vocabulary. They provide insight into the basic underlying meanings of words, and into the history of languages, peoples and technologies. \is{etymology}Etymological \is{dictionary}dictionaries typically provide long, but rather random and very incomplete lists of \is{cognate}cognates, with lots of inefficient duplication between \is{dictionary}dictionaries, and small languages often lack any good \is{etymology}etymological \isi{dictionary} at all. It is more efficient to assemble together the information for many languages at once. However, the tabular, rectangular structure typically used for short illustrative lists of \is{cognate}cognates simply does not scale up. Bunadas uses instead a network structure, which enables “genealogical \is{tree}trees” of \is{cognate}cognates to be generated on the fly. This chapter describes some vital aspects of its design and gives some thoughts on its future.}


\begin{document}

\maketitle

\section{Introduction}

\is{Bunadas|(}\is{Wiktionary|(}\is{cognate}Cognate words provide insight into the basic underlying meanings of words, and into the history of languages and peoples, thoughts, and technologies. \is{etymology}Etymological \is{dictionary}dictionaries typically provide long, but rather random and very incomplete lists of \is{cognate}cognates, with lots of inefficient duplication between \is{dictionary}dictionaries. This chapter describes a new online tool for researchers developed by the author, that allows for comparative analysis of lexemes across Indo-European languages. I explore the structure and functioning of the database and search tool. Using Celtic as a test case, I demonstrate how an output based on \is{tree}trees that can be generated ``on the fly” provides more insightful \is{etymology}etymological results than traditional tabular comparison tools.\footnote{This chapter is best read while simultaneously trying out the online resource itself at \url{https://www.smo.uhi.ac.uk/teanga/bunadas}.}

\figref{fig:odonnaile:1} shows an example of the kind of \isi{cognate} \isi{tree} that Bunadas can generate on the fly: the \isi{tree} starting from the \ili{Proto-Indo-European} word \textit{ḱwṓ} ‘dog’ and showing descendants (\is{reflex}reflexes) such as \ili{Proto-Celtic} celt:\textit{kū}, Old Irish\il{Old Irish} sga:\textit{cú}, \il{Scottish Gaelic (Modern)}Scottish Gaelic gd:\textit{cù}, \ili{Proto-Germanic} germ:\textit{hundaz}, \ili{Old English} ang:\textit{hund}\textsuperscript{dog}, and English\il{English (Modern)} en:\textit{hound}. Each word in Bunadas has two, or occasionally three, components. First comes a standard\footnote{ISO 639-1 where available, or ISO 639-3, with occasional ad hoc additions for ancient and proto languages.} language tag such as sga (Old Irish) or gd (Scottish Gaelic). Important for usability, this is color\hyp coded to reflect the language family: green for Celtic, black for Germanic, red for Romance, blue for Slavic\il{Slavic language family}, bronze for \ili{Greek}. Next comes the actual spelling of the word, and again this is color-coded to reflect finer distinctions. The general progression is from white or gray colors for the proto languages, progressing to bolder colors for the modern languages.

% %%please move the includegraphics inside the {figure} environment
% %%\includegraphics[width=\textwidth]{figures/ODonnaileNEWACedit20240715-img001.png}

\begin{figure} 
\includegraphics[height=.5\textheight]{figures/ODonnaile-img001.jpg} \caption{A tree starting from the \ili{Proto-Indo-European} word \textit{ḱwṓ}} \label{fig:odonnaile:1} 
\end{figure}

%{\figref{fig:key:1}: 


Finally, there can be a third component, a disambiguator to distinguish \is{homograph}homographs. We see this in the Old English word ang:\textit{hund}\textit{\textsuperscript{dog}} for example, which has been given the \isi{disambiguator} ‘dog’, because there is another word ang:\textit{hund}\textit{\textsuperscript{100}} meaning ‘hundred’. For languages which have a single very authoritative \isi{dictionary}, such as \il{Welsh (Modern)}Welsh (\isi{GPC}\footnote{ \url{https://www.geiriadur.ac.uk/} }) \citep{cod:Thomas2022} or Old Irish (DIL\footnote{ \url{https://dil.ie/} }) 
\citep{cod:TonerEtAL2019}, Bunadas uses the \isi{disambiguator} from this \isi{dictionary}, but in other cases it can use any arbitrary text string which will serve the purpose. This disambiguation of \is{homograph}homographs is vital to make Bunadas function. Internally, Bunadas works using numeric IDs for words and is therefore not confused by \is{homograph}homographs, but these ID tags are not seen by users so these \isi{disambiguator}disambiguators are critical for users.


\section{Words and clusters}

The \is{tree}trees that Bunadas displays are perhaps its most impressive feature, but Bunadas does not in fact store anything in \isi{tree} form. Its basic building blocks are words and clusters of words. Using the word-cluster-word connections, it connects the words into a network, a network which is not a \isi{tree}, and which can and does contain lots of cycles.\footnote{A ``cycle” just means that you can follow the network round in a circle, such as germ:hundaz $\rightarrow$  de:\textit{Hund} $\rightarrow$ en:\textit{hound} $\rightarrow$ germ:\textit{hundaz} \il{Proto-Germanic}(Proto-Germanic to \ili{German}, to English\il{English (Modern)}, back to Proto-Germanic).} When the user requests, though, to see a \isi{tree} starting from say ieur:\textit{ḱwṓ}, Bunadas walks the network and produces a minimal \is{tree}spanning-tree of words \isi{cognate} to \textit{ḱwṓ}.

A cluster in Bunadas is just a particular group of words which are all related to one another \is{etymology}etymologically. This includes not only groups of words from different languages, but also groups of morphological derivatives within a language. Clusters can overlap, and they can be used in whatever way the compiler(s) and editor(s) of Bunadas find convenient. 

If one clicks on the word ang:\textit{hund}\textit{\textsuperscript{dog}}, this produces the webpage shown in \figref{fig:odonnaile:2}. At the very top of the page, there is a fuzzy M on a yellow background. This connects to a \isi{dictionary} lookup of the word via \is{multidict}Multidict.\footnote{ \url{https://multidict.net} } \is{mutlidict}Multidict (\cite{cod:ÓDonnaíle2014}) contains a database of information on lots of online \is{dictionary}dictionaries, including \is{etymology}etymological \is{dictionary}dictionaries, in lots of languages, together with the parameters which enable it to link through to them transparently. To the right of the yellow M, there is a [W] favicon which connects to a Wiktionary\footnote{ \url{https://en.wiktionary.org/} } lookup of the word. Thus, we see that Bunadas affords excellent connections to other resources. Following that at the top of the page are the properties stored for the word ang:\textit{hund}\textit{\textsuperscript{dog}} in the Bunadas database. At the very bottom of the page is a grey section ``Lexically similar words”, showing \is{hypergraph}hypergraphs, \is{homograph}homographs and \is{hypograph}hypographs\footnote{ \is{homograph}Homographs are words which are identical in the way they are written, such as ``invalid” (disabled person) and ``invalid” (not valid), while ``invalidity” would be a \isi{hypergraph} and ``valid” a \isi{hypograph}.} (if any), but this is just additional information which might be of use to users and editors and is nothing to do with Bunadas itself.

The section in the middle, showing the clusters, is what is important for the structure of Bunadas. We see that our word ang:\textit{hund}\textit{\textsuperscript{dog}} belongs to three clusters. The first connects it with its modern English \isi{reflex} en:\textit{hound}. The second connects it with another \ili{Old English} word ang:\textit{grī\.ghund}. The third puts it in a cluster of words descended from \ili{Proto-Germanic} germ:\textit{hundaz}.

The numbers shown after the words in the clusters, are a metric indicating the distance of the word from the center of the cluster. Normally, we try to pick some word to be at the center of the cluster, at distance 0. This would normally be the word in the proto language if all the other words are descendants of it; or the basic stem word if the others are morphologically related. It is not necessary to have a center word. A cluster of \isi{cognate} words with no known antecedent could show them all as distance 1 from the cluster, and they would then each be distance 1 + 1 = 2 from each other. The metric is not restricted to values 0 or 1. It can be any number at all, such as 2, or 4.3, and this can be useful when we know that two words are related, but probably rather distantly. Bunadas is not normally very sensitive to the distance metric because the important thing is the connectivity, so for the most part the metric is 1. However, tweaking this metric a bit, from 1 to 1.1 or to 0.9, can be useful when we want to refine the \is{tree}tree to show word derivation via a particular path, such as sga:\textit{cú} $\rightarrow$ sga:\textit{cúallacht} $\rightarrow$ ga:\textit{cuallacht} (\isi{synchronic} in Old Irish, followed by \isi{diachronic} to Modern Irish), in preference to sga:\textit{cú} $\rightarrow$ ga:\textit{cú} $\rightarrow$ ga:\textit{cuallacht} (\isi{diachronic} to \il{Irish (Modern)}Irish, followed by \isi{synchronic} in \il{Irish (Modern)}Modern Irish).

% %%please move the includegraphics inside the {figure} environment
% %%\includegraphics[width=\textwidth]{figures/ODonnaileNEWACedit20240715-img002.png}
 
\begin{figure}
\includegraphics[height=.6\textheight]{figures/ODonnaile-img002.jpg}
\caption{The word ang:\textup{hund}\textup{\textsuperscript{dog}} showing the clusters it belongs to.}
\label{fig:odonnaile:2}
\end{figure}

Note that the way the \is{tree}trees are constructed, as minimal spanning \is{tree}trees calculated by walking through the network and minimizing the total metric distance, means that if a cluster was duplicated, there would be no difference whatsoever in the \is{tree}trees. This was a design decision early on in the design of Bunadas. An alternative approach would be to strengthen connection among words if they are related in two different clusters, perhaps derived from two different information sources. Rejecting this alternative approach and choosing to use minimal spanning \is{tree}trees has proven in practice to be the right decision.

In the second cluster shown in \figref{fig:odonnaile:2}, ang:\textit{hund}\textit{\textsuperscript{dog}} is linked to ang:\textit{grī\.ghund}. We might ask why ang:\textit{grī\.ghund} does not appear in the \isi{tree} shown in \figref{fig:odonnaile:1}. The solution lies in the ${\succ}$ symbol shown after \textit{grī\.ghund} in the cluster. This symbol means that \textit{grī\.ghund} is made up of two (or more) main components, and \textit{hund} is only one of them. In order to see ang:\textit{grī\.ghund} in the \isi{tree}, it is necessary to switch on the option ``Show super-elements”. The result is shown in \figref{fig:odonnaile:3}. 

% %%please move the includegraphics inside the {figure} environment
% %%\includegraphics[width=\textwidth]{figures/ODonnaileNEWACedit20240715-img003.png}

\begin{figure}
\includegraphics[height=.6\textheight]{figures/ODonnaile-img003.jpg}
\caption{Tree with “Show super-elements” turned on.}
\label{fig:odonnaile:3}
\end{figure}

\figref{fig:odonnaile:3} also illustrates some other features of Bunadas. Hovering over a word causes the part of speech and gloss for the word to pop up, in this case ``[nm] greyhound”.

The word cy:\textit{trychchwn} is preceded by two red question marks, and hovering over them will display ``\isi{probability} 0.6”. It is sometimes uncertain as to whether words are related, and Bunadas can take this uncertainty into account by attaching a \isi{probability} to whether a word rightly belongs to a cluster. When generating \is{tree}trees, Bunadas naively multiplies any probabilities it encounters, and if the resulting \isi{probability} drops to 0.5 or lower, it stops there and goes no further along that particular path. So, for example if there is a \isi{probability} 0.6 that word A is an ancestor of word B, and a \isi{probability} 0.7 that word B is an ancestor of word C, then Bunadas naively assumes that there is a \isi{probability} 0.6 × 0.7 = 0.42 that word A is related to word C, and therefore it does not (unless there is other evidence) show word C or its descendants in the \isi{tree} it generates starting from word A, because the connection is too dubious.

The word celt:\textit{Kunowalos} has a (* after it, as do four other words. This is because I have ``pruned” the \isi{tree} at these points, compressing branches to reduce clutter and fit the \isi{tree} into a reasonably sized figure. Any branch can be pruned by clicking the language tag of the headword and can be ``unpruned” by clicking again or by clicking the * symbol. The \isi{pruning} is “remembered” in the URL, which is very useful when passing Bunadas \isi{tree} URLs to other people as illustrations of particular points.

As well as the option ``Show super-elements” in Bunadas \is{tree}trees, there is an option ``Show sub-elements”. But here Bunadas has to be very careful. If both options are switched on, it will happily go down to sub-elements when constructing \is{tree}trees and then back up again: from cy:\textit{corgi} to cy:\textit{ci} to cy:\textit{milgi} for example, which is fine because the words are all related. However, the Bunadas algorithm will not ever go up and then back down. Otherwise, it would go from de:\textit{Hund} to de:\textit{Dachshund} to de:\textit{Dachs} and claim wrongly that de:\textit{Hund} and de:\textit{Dachs} were related.

In addition to the ${\succ}$ symbol, Bunadas has two other symbols which are important for relating words. The ${\gg}$ symbol shows that one ``word” (or \isi{prefix} or \isi{suffix}) is only a very minor component of another word. Conversely the ${\succcurlyeq}$ symbol is for showing that two words are substantially the same; that one is a very major component of the other. Thus the ${\succcurlyeq}$ symbol is typically used when linking a word with its root. The ${\gg}$ symbol is used when linking words to minor morphemes such as \is{prefix}prefixes and \is{suffix}suffixes. In analyzing the English\il{English (Modern)} word en:\textit{only}, for example, it could be linked to the suffix en:\textit{{}-ly} using the ${\gg}$ symbol and linked to the English word en:\textit{one} using the ${\succcurlyeq}$ symbol. When constructing \is{tree}trees, the rule ``no going up then back down”, which applies to the ${\succ}$ symbol, does \textit{not} apply to the ${\succcurlyeq}$ symbol. When constructing \is{tree}trees, Bunadas does not normally follow ${\gg}$ links at all – otherwise it would construct many false links via trivial \is{prefix}prefixes and \is{suffix}suffixes. The ${\gg}$ symbol can also be used to record that a word is a calque of a word in another language. Thus, the Old Irish \textit{día lúain} ‘Monday’, for example, is derived from the native words \textit{día} and \textit{lúan}, and the relationship is shown with the ${\succ}$ symbol. It is a calque of Latin \textit{diēs Lūnae}, and this connection can be shown with the ${\gg}$ symbol, ensuring that it is not followed when constructing \is{tree}trees.

The ${\succ}$ symbol is most often used when words have combined into one by compounding, as in English \textit{seafarer}, \ili{German} \textit{Seefahrer}, or \ili{Proto-Celtic} \textit{tegoslougom}. However, it can also be used where two words which were near homonyms have merged into one. Another situation in which the ${\succ}$ symbol may be used is where a word may have come partially via one route, partially via another – such as where an Old Irish word may have come partially from an \ili{Old Norse} word but partially from a similar \ili{Old English} word, and both of these have in any case come from the same \ili{Proto-Germanic} word. In this case, it is possible to get the best of both worlds by linking the \ili{Old Irish} word using ${\succ}$ to the \ili{Old Norse} and \ili{Old English} words, but also linking it directly to the \ili{Proto-Germanic} ancestor, maybe setting the distance metric here as 2. Alternately, we might want to give it \isi{probability} 0.5 as having come from \ili{Old Norse}, 0.5 from \ili{Old English}, while linking it with certainty to the \ili{Proto-Germanic} ancestor.

Most of the time it is obvious which symbol is appropriate when linking words together. Sometimes, though, it is not certain whether it is best to break a word down into two equal-status components using ${\succ}$ for each; or whether to apply ${\succcurlyeq}$ to one and ${\gg}$ to the other, and if so which of them should be considered the major component (which will therefore appear in \is{tree}trees) and which the minor. It is important for anyone editing Bunadas to maintain consistency, and if say, an \ili{Old French} word is broken down in a particular way, then the \il{French (Modern)}Modern French word should be broken down in the same way.

The idea of a \isi{tree} of descendants can go ``wrong”, or at least exhibit somewhat pathological behavior. Take, for example, the \isi{tree} of descendants for PIE \textit{tḱey{}-}. In all the descendants, we can at least see some \isi{reflex} of the \textit{tḱ} from the PIE root – except in the branch derived from Latin \textit{sinō} and \textit{pōnō}:

\ea
\textit{tḱey} $\rightarrow$ ieur:\textit{tḱi{}-né-ti} $\rightarrow$ la:\textit{sinō} $\rightarrow$ la:\textit{pōnō} $\rightarrow$ la:\textit{conpōnō} $\rightarrow$ en:\textit{compose}
\z

The \textit{s} of \ili{Latin} \textit{sinō} comes from the \textit{tḱ} of \textit{tḱey-}, but this has disappeared in \textit{pōnō} after \is{prefix}prefixing by \textit{po}{}-, and \textit{pōnō} has absolutely nothing left of the original \textit{tḱey-}; everything it contains comes from \is{prefix}prefixes and \is{suffix}suffixes added later. This is even more extreme by the time we reach English \textit{compose}. On each step of the way from \textit{tḱey-} to \textit{compose}, each word has clearly been derived from the word immediately before it, and yet no smidgeon of \textit{tḱey-} can be found in the word \textit{compose}. Other examples can be found, such as PIE \textit{gʰed}{}- $\rightarrow$ English\il{English (Modern)} \textit{prison} (as well as the more transparent English\il{English (Modern)} \textit{get}). Fortunately, this situation does not occur often enough to be a big problem, and the \is{tree}trees generated by Bunadas are nearly always informative and meaningful.



There is a question as to whether we need the concept of ``cluster” at all, or whether it would be better to relate words directly to one another, going word-word rather than word-cluster-word. There are advantages and disadvantages to each approach. From a purely theoretical point of view, both approaches are equivalent and can be converted into one another. Word-word connections are obviously a special case of small clusters. And clusters can be broken down automatically into word-word connections, albeit rather a lot of them in the case of a large cluster having no ``center” word. Having direct word-word connections would simplify the programming a bit, perhaps simplify the logic a bit, and would speed up the computations slightly. The clusters, though, are very useful to human editors, and facilitate meaningful groupings which are easier to assess and to manage. The Bunadas editing facilities allow words to be drag-and-dropped from cluster to cluster, which is very useful.


The Bunadas editing tools can be tried out by anyone who is interested, simply by swapping to the bunTest database in the dropdown on the main Bunadas page. The Bunadas programs are all publicly available on \isi{GitHub}\footnote{ \url{https://github.com/caoimhinsmo/bunadas} } to anyone who is interested. The /cloning webpage\footnote{\url{https://www.smo.uhi.ac.uk/teanga/bunadas/cloning/} } gives details, and even includes a moderately up-to-date dump of the entire Bunadas database. The Bunadas user interface is available in a dozen languages so far, and there is a good translation interface so it would be easy to add still more interface languages. As can be seen from \isi{GitHub}, the programs are written in \isi{PHP}, with \isi{MariaDB} (or \isi{MySQL}) as the backend relational database, and with \is{JavaScript}JavaScript/AJAX \is{AJAX} used to help provide a responsive user interface.

\section{Orthography, and which wordform to use for Bunadas words} 

It is important for Bunadas to be consistent about orthography\is{orthography|see {spelling}} so that the same word is not added to Bunadas multiple times, to the confusion of users. Bunadas nearly always uses the orthographic guidelines defined by Wiktionary for the various languages.\footnote{ \url{https://en.wiktionary.org/wiki/Category:Wiktionary_language_considerations} } This is particularly important for the reconstructed proto languages, where no historic spelling system exists, and diverse systems have been used by different authorities and authors. Wiktionary has created detailed orthographic guidelines for \ili{Proto-Indo-European}, \ili{Proto-Celtic}, \il{Proto-British}Proto-Brythonic and \il{Proto-Germanic}, among others. For languages where Wiktionary sometimes uses more detailed \is{spelling}orthography, sometimes less detailed, Bunadas takes care to use the more detailed \is{spelling}orthography. This is the case for Ancient Greek, for example, where the particular entries have detailed diacritics marking breathings, short vowels, etc., but the internal links in Wiktionary may use less detail.

At present \ili{Old Irish} is an exception to the rule that Bunadas follows Wiktionary spelling guidelines. Here Bunadas generally follows the spelling in \isi{DIL}, the \is{dictionary}Dictionary of the Irish Language. This is because \isi{DIL} is so authoritative and so readily available online, but most of all because of the huge contribution to Bunadas derived from the \textit{Innéacs Nua-Ghaeilge don \isi{DIL}} (\cite{cod:DeBaldraithe1981}) about 13,000 Old-Irish–Modern-Irish pairs. In particular, this means that Bunadas currently uses the hyphen (-) rather than the raised dot (·) to show the “joint” of the verbal complex. It is expected that Bunadas will move at some stage to using the Wiktionary \is{spelling}orthography for \ili{Old Irish}. This will not, though, cause the link with \isi{DIL} to be lost, because each Old Irish word in Bunadas is already linked to its numeric entry number in \isi{DIL}.

Just as important as consistency in \is{spelling}orthography, is consistency in which word\-form to use for Bunadas words – whether, for example, to use the first-person-singular for verbs, or else the third person singular, or singular imperative, or infinitive. Here again, Bunadas follows the Wiktionary preference for each particular language: the first singular for Latin verbs, for example, but the infinitive for \il{French (Modern)} French and \ili{Old French} verbs. Wiktionary does in fact also try to include brief entries for the infinitive of Latin verbs, first singular of \il{French (Modern)} French verbs, etc., and indeed for nearly all other grammatical forms, but its preferred form is clear for each language. Bunadas goes for much less morphological granularity than Wiktionary does, so as to avoid clutter which would be unhelpful to the user. To illustrate from English, Bunadas includes both the verb \textit{state} and the noun \textit{statement}, but it does not include verbforms \textit{stating}, \textit{stated}, and \textit{states}, whereas Wiktionary includes all of these. This difference in morphological granularity means that while Wiktionary can link a \il{French (Modern)}French verb (in the infinitive) to the \ili{Latin} infinitive from which it is derived, Bunadas links it to the Latin first person singular; and while Wiktionary can link a \il{French (Modern)} French noun to the Latin accusative from which it is derived, Bunadas links it to the Latin noun in the nominative.

In the case of \il{Irish (Modern)}Modern Irish, the need to standardize to the singular-imperative of verbs required a major conversion exercise. The singular-imperative is used by Wikipedia and by most \is{dictionary}dictionaries, but unfortunately the third singular is the form which was used by the \textit{Innéacs Nua-Ghaeilge}, so all of these had to be converted, and \is{disambiguator}disambiguators added where required.

In the case of \isi{suppletive} verbs or \isi{suppletive} nouns, the idea of sticking to a standard word form for each language breaks down. The English past tense \textit{went}, for example, cannot be simply covered by the verb \textit{go} as in many \is{dictionary}dictionaries – it requires a separate entry in Bunadas. The choice of which word form to use to represent a group of \isi{suppletive} forms often has to be made on an ad hoc basis.

\section{What Bunadas does and does not do}

What Bunadas primarily does is exploit connectivity. If word A is \is{etymology}etymologically related to word B, and word B is \is{etymology}etymologically related to word C, then word A is \is{etymology}etymologically related to word C, and so on. There is no need to store separately the information that A is \is{etymology}etymologically related to C. This avoids duplication and a lot of the gaps in conventional \is{etymology}etymological \is{dictionary}dictionaries.

What Bunadas does not do is concern itself with how or why or even in what direction A is \is{etymology}etymologically related to B: whether A is \isi{inherited} from B, or is a modern \isi{borrowing} of B, or vice versa, or whether they are both derived separately from a common ancestor. Bunadas can link A with B in all these different circumstances. Of course, Bunadas can give clues about how A might be related to B. If A is a \ili{Latin} word at the center (metric distance 0) of a cluster which also contains B and other modern Romance words, this is a very strong clue that B is derived from A rather than the other way round. If a French word B is linked straight to a Latin word A, while another \il{French (Modern)}French word D is linked via \ili{Middle French} and \ili{Old French} words to A, then this indicates strongly that \il{French (Modern)} French word D is \isi{inherited}, while B may or may not be a modern \isi{borrowing}. As more words in intermediate languages are added to Bunadas, these clues become stronger, but they are still just clues. Bunadas has a free-text comments field for each word where information about derivation can be added. All these things, though, are peripheral to Bunadas. They are not part of its core functionality, which is about connectivity. Bunadas contains no information at all about sound laws: about how an \isi{inherited} \il {French (Modern)}Modern French word D, for example, might have been shaped by going through the mangle of \ili{Middle French} and \ili{Old French} and \il{Latin}Vulgar Latin sound changes. It is conceivable that sound law information might somehow someday be added to Bunadas, but this is not planned or envisaged in the near future.

\section{The future}

Where does Bunadas go from here? Obviously adding more words in more languages would increase its functionality. Adding more words would make the \is{tree}trees more cluttered, but this could be alleviated by improving the \is{filtering}filtering/prun\-ing facilities: by introducing \isi{filtering} by language family for example.

Bunadas was initially populated by about 5000 words from the \textit{Stòr-fhaclan Co-dhàimheil Ceilteach} (Celtic Cognates Database), a project which had become moribund because its rectangular structure, languages × word groups, was too inflexible and made adding data too difficult. To this were added about 12,000 \il{Old Irish} \il{Irish  (Modern)}Old-Irish–Modern-Irish word pairs from the \textit{Innéacs Nua-Ghaeilge don \isi{DIL}} which Kevin Scannell has put online as \textit{Droichead \isi{DIL}}.\footnote{ \url{https://cadhan.com/droichead/} } Bunadas also contains about 21,000 Manx words and word forms and \is{cognate}cognates (mostly Irish), a donation from Kevin Scannell, although these have not yet been fully integrated into the system. All the other thousands of words have been added laboriously by hand, using resources such as \isi{GPC} for \il{Welsh (Modern)}Welsh \citep{cod:Thomas2022} and \isi{DIL} for \il{Irish (Modern)}Irish \citep{cod:TonerEtAL2019}, but most of them, especially for the non-Celtic languages, are copied from information in Wiktionary. This work continues, even though Bunadas has never been a funded project.



The sources of the \is{etymology}etymological information in Wiktionary are documented in the entries in Wiktionary itself. There are hundreds of them, both books and papers, far too many to list here, but two which could be mentioned for the Celtic languages are \citet{cod:Matasovic2009} and \citet{cod:Zair2012}. Apart from Wiktionary, the Celtic language information in Bunadas is gleaned from dozens of sources, such as \isi{GPC} and \isi{DIL} and all the usual Celtic language \is{etymology}etymological \is{dictionary}dictionaries, including \citet{cod:Deshayes2003}, \citet{cod:Henry1900}, and \citet{cod:MacBain1911}.

100,000 words is a good start but is a drop in the ocean compared to all the words in all the languages of the world. By adding more words to it, Bunadas could be transformed into a useful resource for not just the Celtic languages, but lots of other languages too. An obvious question is, if much of the data is being copied over from Wiktionary, would it not be better to scrape data automatically from Wiktionary, since this is all in the public domain? This thought has already occurred to various very knowledgeable data scientists, who have constructed facilities based on etymological\is{etymology} data scraped from Wiktionary.\footnote{Lexvo: \url{http://www.lexvo.com/} \citep{cod:DeMelo2015}; Rabbitique: \url{https://www.rabbitique.com}; Etymologeek $\rightarrow$ Cooljugator: 
\url{https://cooljugator.com/etymology}} The results so far are useful, have more words and more languages than Bunadas has, but they have nothing to compare with the impressive, broad \is{tree}trees constructed by Bunadas. The problem is that even though the \is{etymology}etymological information contained in Wiktionary is generally very good and is improving by the week, the way it is structured is dire. Wiktionary does not give unique IDs to words, so links between words are based on spelling and can be broken if the spelling is changed. Wiktionary does not distinguish properly between \is{homograph}homographs. It piles all the \is{homograph}homographs, both between languages and within languages, onto the one webpage. Wiktionary currently says, for example, that English \textit{bole} ‘tree trunk’ is from \ili{Norse} \textit{bolr}, and that \textit{bolr} is from PIE \textit{bʰel-}, but it does not clarify which of the four different PIE roots \textit{bʰel-} it means: the link just leads to the four of them.

I believe that the long-term solution lies in \isi{Wikidata} Lexical Data.\footnote{\url{https://ordia.toolforge.org}} The various Wikipedias, especially for smaller languages, are increasingly starting to automatically incorporate data from \isi{Wikidata}, which is a common database of information of all kinds, and \isi{Wikidata} Lexical Data is its extension into lexicography. The words in \isi{Wikidata} Lexical Data do have proper unique identifiers, ‘L’ numbers (such as L6419 for en:\textit{hound}), corresponding to \isi{Wikidata}’s ‘Q’ numbers. The long term aim, I believe, should be for \is{etymology}etymological data to be shared in \isi{Wikidata} Lexical Data, from where it can be accessed and used by all the various Wiktionaries, \url{http://ru.wiktionary.org}, \url{http://fr.wiktionary.org}, etc., as well as \url{http://en.wiktionary.org}, and used also by Bunadas and other such facilities. \isi{Wikidata} Lexical Data, as it stands at present, has a way to go before it is ready for this. It has not yet fully embraced the essential idea of separating out \is{etymology}etymologically distinct \is{homograph}homographs as distinct words. It lacks the categorization of types of word components which Bunadas expresses using the ${\succ}$, ${\succcurlyeq}$, and ${\gg}$ symbols. It is likely that it will be several years before \isi{Wikidata} Lexical Data is ready to provide a foundation for the \is{etymology}etymological information expressed in Bunadas and in the various Wiktionaries. So, in the meantime Bunadas can continue on its present course, providing useful facilities especially for the Celtic languages, and exploring what will be possible and useful in an \is{etymology}etymological facility for all languages once the data foundation is there. \is{Wiktionary|)} \is{Bunadas|)}

\printbibliography[heading=subbibliography,notkeyword=this]
\end{document}
