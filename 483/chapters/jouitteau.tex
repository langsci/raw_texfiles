\documentclass[output=paper,colorlinks,citecolor=brown]{langscibook}
\ChapterDOI{10.5281/zenodo.15654859}
\author{Melanie Jouitteau\affiliation{Centre National de la Recherche Scientifique (CNRS)}}
\title{When linearization triggers embedded V2: Evidence from Breton}


\IfFileExists{../localcommands.tex}{
  \usepackage{langsci-optional}
\usepackage{langsci-gb4e}
\usepackage{langsci-lgr}

\usepackage{listings}
\lstset{basicstyle=\ttfamily,tabsize=2,breaklines=true}

%added by author
% \usepackage{tipa}
\usepackage{multirow}
\graphicspath{{figures/}}
\usepackage{langsci-branding}

  
\newcommand{\sent}{\enumsentence}
\newcommand{\sents}{\eenumsentence}
\let\citeasnoun\citet

\renewcommand{\lsCoverTitleFont}[1]{\sffamily\addfontfeatures{Scale=MatchUppercase}\fontsize{44pt}{16mm}\selectfont #1}
   
  %% hyphenation points for line breaks
%% Normally, automatic hyphenation in LaTeX is very good
%% If a word is mis-hyphenated, add it to this file
%%
%% add information to TeX file before \begin{document} with:
%% %% hyphenation points for line breaks
%% Normally, automatic hyphenation in LaTeX is very good
%% If a word is mis-hyphenated, add it to this file
%%
%% add information to TeX file before \begin{document} with:
%% %% hyphenation points for line breaks
%% Normally, automatic hyphenation in LaTeX is very good
%% If a word is mis-hyphenated, add it to this file
%%
%% add information to TeX file before \begin{document} with:
%% \include{localhyphenation}
\hyphenation{
affri-ca-te
affri-ca-tes
an-no-tated
com-ple-ments
com-po-si-tio-na-li-ty
non-com-po-si-tio-na-li-ty
Gon-zá-lez
out-side
Ri-chárd
se-man-tics
STREU-SLE
Tie-de-mann
}
\hyphenation{
affri-ca-te
affri-ca-tes
an-no-tated
com-ple-ments
com-po-si-tio-na-li-ty
non-com-po-si-tio-na-li-ty
Gon-zá-lez
out-side
Ri-chárd
se-man-tics
STREU-SLE
Tie-de-mann
}
\hyphenation{
affri-ca-te
affri-ca-tes
an-no-tated
com-ple-ments
com-po-si-tio-na-li-ty
non-com-po-si-tio-na-li-ty
Gon-zá-lez
out-side
Ri-chárd
se-man-tics
STREU-SLE
Tie-de-mann
} 
   \boolfalse{bookcompile}
  \togglepaper[2]%%chapternumber
}{}

\AffiliationsWithoutIndexing

\abstract{In this chapter, I present a subset of Breton embedded verb-second word orders. These data are intriguing because they implicate operations in embedded domains which appear to be optional. By contrast, we know that in matrix clauses the same operations are last resort, non-optional ones. I show that a proper ordering of operations at PF can ascribe optionality to a dialect parameter that allows for the exclusion of high functional heads from the start of a linearization domain. If and only if a functional head is excluded, would we observe last resort operations for T2. This ordering of operations is significant for our study of the modularity of grammar, because linearization of the syntactic structure can be shown to precede a set of morphological operations which operate in a linear domain.}


\begin{document}
\maketitle 

\section{Introduction}\label{sec:jouitteau:1}

\il{Breton (Modern)|(}
\is{tense second|(}
\is{second placement phenomena|see {tense second}}

Unlike the restrictions against verb second (V2\is{verb second}) in languages like German, Breton allows embedded verb second orders in some circumstances. The matrix word order in Breton is consistently verb-second: the tensed element typically follows another constituent. I will refer to the phenomena of ordering in Breton as Tense second (T2) rather than V2\is{verb second} because in compound tenses, the second position is defined relative to Finiteness and Tense. Consider the data in (\ref{ex:jouitteau:1}).\footnote{In this chapter, the finite element subject to second position placement is highlighted in {\textit{italic}}, while the element fulfilling the first position is in \textbf{bold}. Trace copies are indicated with \sout{strikethrough}.} In the matrix clause of (\ref{ex:jouitteau:1a}), the subject \textit{tout an dud} ‘everybody’ precedes a tensed lexical verb. This subject can be replaced by the object \textit{keuz} ‘regret’, an adverb, or even a syntactic head. In the embedded domain of (\ref{ex:jouitteau:1a}), the tensed element is the second constituent following a complementizer, which lacks a phonological matrix. This is not surprising~– it aligns with extensive evidence that the T2 requirement may be satisfied by a syntactic head, such as negation, or by a phonologically empty element such as an omitted topic. The alternate word order in (\ref{ex:jouitteau:1b}) is particularly intriguing. It involves an additional derivational step through stylistic fronting. The puzzle lies in the apparent optional nature of participle head movement in the embedded clause. Normally, a defining characteristic of \isi{stylistic fronting} in Breton is its role as a last resort mechanism to ensure the second placement of the tensed element.


\ea\label{ex:jouitteau:1}
\ea \label{ex:jouitteau:1a}
\gll \FirstPosition{Tout} \FirstPosition{an} \FirstPosition{dud} {\SecondPosition{o}} {\SecondPosition{d-eus}} keuz \FirstPosition{\BoldNull} {\SecondPosition{e}} {\SecondPosition{n-eus}} lennet Alain an dra-se. \\ 
     all the people \textsc{fin}.\textsc{3pl} \textsc{3}-\textsc{aux} regret C \textsc{fin} \textsc{\textbf{}3}-\textsc{aux}.\textsc{pres} read.\textsc{ptcp} Alain the thing-there \\
\glt ‘Everyone regrets that Alain read this thing.’
\ex \label{ex:jouitteau:1b}
\gll \FirstPosition{Tout} \FirstPosition{an} \FirstPosition{dud} {\SecondPosition{o}} {\SecondPosition{d-eus}} keuz {∅}   {\FirstPosition{lennet}}     {\SecondPosition{e}}    {\SecondPosition{n-eus}}   \sout{lennet}  Alain an dra-se. \\ 
All the people \textsc{fin.3pl} 3\textsc{-aux.pres} regret C read\textsc{.ptcp} \textsc{fin} \textsc{{3}-aux.pres} \sout{read\textsc{.ptcp}} Alain the thing-there\\
\glt `Everyone regrets that Alain read this thing.'
\z
\z

\noindent The data from Breton were gathered through a series of 14 elicitation sessions conducted between 2015 and 2018, involving three traditional native speakers from distinct Breton dialects. These speakers, all over 70 years of age, were raised primarily speaking Breton with French as their early second language. They are referred to by their chosen names or acronyms. The first speaker, Huguette Gaudart from Skaer and Bannaleg, is proficient in both the East Kerne dialect and Standard Breton. The other two speakers, MLB from Plougerne and AM from Lesneven/Kerlouan, each represent a point of variation in the Leon dialect. They both have minimal influence from other dialects, including Standard Breton. At the time of the elicitation, the speakers were already familiar with the author from previous sessions and were accustomed to providing grammaticality judgments. The elicitation protocol involved translation tasks from French. The responses were transcribed in front of the speakers using the orthography of their preference. Following this, the speakers were asked to evaluate the grammaticality of various word permutations that altered the sentence they had originally provided. Comprehensive documentation of the raw elicitation data, along with a detailed description of the elicitation processes, is available online at the Elicitation Center of the wikigrammar ARBRES (\citealt{mj:Jouitteau2009}).


The structure of this chapter is as follows. First, I establish a theoretical framework for the analysis of T2 in Breton. This includes a detailed description of the architecture of the left periphery in Breton clauses, along with a summary of relevant research on the phenomena of second position placement within the matrix domain. \sectref{sec:jouitteau:3} introduces the pivotal dataset. I present novel data from the Breton Kerne and Leon dialects, documenting a range of unexpected embedded V2\is{verb second} orders. A complementizer is always acceptable as the first element of the embedded (C-VSO), but sometimes in Leon we also find intervening heads (C-X-VSO). The paradigms they come from are last resort for T2 in matrix domains (expletive insertion, \isi{stylistic fronting}, analytic tenses). I show the limitations on these rare data before turning to their analysis. \sectref{sec:jouitteau:4} presents, and subsequently refutes, the hypothesis that embedded T2 orders can be fundamentally considered as root clauses with a relaxed syntactic integration. This section demonstrates that the embedded T2 domain exhibits weak island effects. Our arguments are supported by evidence from the C-condition, quantifier binding, and the deictic interpretation of temporal nouns. On these grounds, I reject the hypothesis of \citeauthor{mj:Wurmbrand2012} (\citedate{mj:Wurmbrand2012}, \citedate{mj:Wurmbrand2014}): embedded T2 domains are fully integrated into the syntactic structure. \sectref{sec:jouitteau:5} examines matrix and embedded evidence for the linearization of high complementizers and clause external heads. Remarkably, even Breton clause external heads can be mobilized to pass the T2 filter. 

\section{Breton in the typology of V2 languages}\label{sec:jouitteau:2}

The sentences in example (\ref{ex:jouitteau:2}) illustrate a classic paradigm for second-position phenomena. A constituent occupies the initial position of the clause, followed by the inflected element, with the remaining arguments succeeding these. These sentences are semantically equivalent, differing only in their discourse effects. Breton hence qualifies as a V2\is{verb second} or T2 language \citep{mj:Stephens1982}, a class of languages that consistently show the same paradigms (for a detailed typology of those languages, see 
\citealt{mj:Jouitteau2010}). 




\ea \label{ex:jouitteau:2}
\ea \label{ex:jouitteau:2a}
\gll \FirstPosition{Warc'hoazh} {\SecondPosition{e}} {\SecondPosition{preno}} Yann ul levr d'am c’hoar \sout{Warc'hoazh}. \\  
 tomorrow \textsc{fin} buy.\textsc{fut}.\textsc{3s} Yann a book {to my} sister \sout{tomorrow} \\ 
\glt `Yann will buy a book for my sister TOMORROW.’ \source{}{Adv}
\ex \label{ex:jouitteau:2b}
\gll \FirstPosition{Ul levr} {\SecondPosition{a}} {\SecondPosition{breno}} Yann \sout{ul levr} d'am c’hoar warc'hoazh. \\ 
{a book} \textsc{fin} buy.\textsc{fut}.\textsc{3s} Yann {\sout{a book}} {to my} sister tomorrow \\
\glt `Yann will buy A BOOK for my sister tomorrow.’ \source{}{O}
\ex \label{ex:jouitteau:2c}
\gll \FirstPosition{Yann} {\SecondPosition{a}} {\SecondPosition{breno}} \sout{Yann} ul levr d'am c’hoar warc'hoazh.\\
Yann \textsc{fin} buy.\textsc{fut}.\textsc{3s} \sout{Yann} a book {to my} sister tomorrow\\
\glt `YANN will buy a book for my sister tomorrow.’\source{}{S}
\ex \label{ex:jouitteau:2d}
\gll Warc’hoazh, \FirstPosition{Yann} {\SecondPosition{a}} {\SecondPosition{breno}} \sout{Yann} ul levr d'am c’hoar \sout{Warc’hoazh}.\\
tomorrow Yann \textsc{fin} buy.\textsc{\textsc{fut}}.\textsc{3s} \sout{Yann} a book {to my} sister \sout{tomorrow}\\
\glt ‘Yesterday, YANN will buy a book for my sister tomorrow.’\source{}{Adv, S}
\z
\z

Within the typology of T2 languages, the Breton data reveal certain specific features such as a wide left periphery (\ref{ex:jouitteau:2d}), and some Celtic-specific features such as argument ordering (in the middle field, subjects precede objects, VSO), and a particle, either \textit{a} or \textit{e}, that consistently precedes the tensed element. It is glossed here as \textsc{fin}, the head of the Finite projection. Unlike Romance or Germanic languages then, the lowest head in the articulated CP periphery is morphologically realized in Breton. This facilitates the structural placement of the finite verb inside the syntactic structure, in both matrix and embedded sentences.

\subsection{The structure of the left periphery}

The \textsc{fin} particle invariably precedes only the tensed element, and is absent before infinitives, participles, or in small clauses. It manifests solely as \textit{a} /a/ or \textit{e} /e/. However, higher complementizers, such as negation, may fuse with it. In Standard Breton, as illustrated above, these particles alternate based on the \pm{nominal} character of the preceding element \citep{mj:Rezac2004}. For example, in (\ref{ex:jouitteau:2b},\ref{ex:jouitteau:2c},\ref{ex:jouitteau:2d}), the particle \textit{a} follows a nominal element, while in (\ref{ex:jouitteau:2a}), the particle \textit{e} follows a non-nominal element. These particles are typically unstressed and often omitted in normal speech tempo, though speakers are aware of the unpronounced particle. Each particle, whether pronounced or not, induces a specific mutation in the subsequent word. For example, the particle \textit{a} in (\ref{ex:jouitteau:2b},\ref{ex:jouitteau:2c},\ref{ex:jouitteau:2d}) consistently triggers lenition, changing the initial consonant of the inflected verb \textit{prenañ} ‘to buy’ to a /b-/. Conversely, the particle \textit{e} in (\ref{ex:jouitteau:2a}) triggers a set of mixed mutations, leaving the initial /p-/ in this case unaffected.

Another contrast with Germanic T2 languages is that the Breton verb consistently raises in all tensed sentences, resulting in embedded VSO in the simplest cases. Importantly, the phonological realization of a C head above the verb does not influence verb raising. Verb raising occurs not only in T2 constructions but also in the (comparatively rare) Tense initial sentences in Breton, as well as in embedded clauses. The analysis of the trigger for T2 requirement is delineated in \citet{mj:Jouitteau2020}, under the term of Left Edge Filling Trigger. Here, I present only the properties that are relevant for the study of embedded domains. The filter applying at linearization is in (\ref{ex:jouitteau:3}). 

\ea \label{ex:jouitteau:3}
A \textsc{fin} head can be linearized if and only if it is not initial.
\z 

The domain to be linearized is maximally structured as illustrated in \figref{ex:jouitteau:4} (page~\pageref{ex:jouitteau:4}). In matrix sentences, only a coordination marker can be merged above the C line in \figref{ex:jouitteau:4}. The region between the C and B lines represents the higher projections of the CP area. This area can be empty or heavily populated. It is recursive. It may include a hanging topic, often followed by a prosodic break. It can also encompass adverbs and clauses that serve as scene-setting material as in (\ref{ex:jouitteau:2d}). The T2 requirement is entirely blind to this high left periphery located between the C and B lines. Empirically, hanging topics are never the only element preceding the tensed one. Scene-setting adverbs can, but only when they are focalized, and thus positioned below the B line. Placement below the B line obtains a discourse difference and restricts scene-setting adverbs to a focus reading. The expansive space between the B and C lines in Breton permits variations up to T6 orders. This makes the Breton T2 distinctly different from the strict T2 order found in Norwegian. Consequently, Breton is categorized as an “at least T2” type, similar to the patterns observed in Old Romance, Rhaeto-Romance, Karitiana (Arikém), and the Germanic languages Mòcheno or Cimbrian. This categorization is supported by the work of \citet{mj:BidesEtAl2014}, as well as \citeauthor{mj:Jouitteau2010} (\citeyear{mj:Jouitteau2010}; \citeyear{mj:Jouitteau2009}), who discuss the concept of Breton T3 orders and provide relevant references. In embedded domains, this area is expected to be reduced. Every element situated below this, between the B line and the A line, engages with the T2 requirement and can be the only pronounced element before a tensed head, as illustrated in the previous examples. Notably, in Breton, the syntactic heads located between the B line and the A line can also function as the sole initial element of a sentence. This property is crucial for the inquiry of embedded domains.

\begin{figure} 
\begin{forest}
for tree={s sep=1mm, inner sep=1mm,}
[ForceP [{Hanging\\topic},name=A ] [Force$'$ [{Scene setting\\adverbs}] [Force$'$ [C$^{\circ}$,name=B][TopP [topic] [FocP [focus] [Mode$'$ [C\\negation\\\textit{ne}][FinP [~] [Fin$'$ [\textit{a}/\textit{e}\\[\textbf{[Fin-(object clitic)-V]},roof,name=C]][IP [~][{FP${\varphi}$=[3{\footnotesize SG}]} [subject] [ \textit{v}P [\sout{subject}][v$'$ [\textit{\sout{v}}][VP [\sout{V}] [object] ] ]]] ]] ]] ] ] ] ] ]
\draw 
  ([xshift=-24pt]A) arc[start angle=160,end angle=60,radius=3cm] node[below] {C};
\draw 
  ([xshift=-24pt]B) arc[start angle=180,end angle=60,radius=2cm] node[below] {B};
\draw 
  ([xshift=-70pt]C) arc[start angle=150,end angle=60,radius=5cm] node[below] {A};
\end{forest}
\caption{Structure of the Breton left periphery}
\label{ex:jouitteau:4}
\end{figure}

\subsection{Breton T2 is linear}

Breton instantiates \textit{linear} \textit{verb second}, as described by \citet{mj:BorsleyKathol2000}, \citet{mj:Jouitteau2020} and references therein. This terminology underlines the fact that the T2 requirement is blind to the [\pm{head}] distinction. Initial functional heads can be Q particles (\ref{ex:jouitteau:5}), preverbal negation (\ref{ex:jouitteau:6}), and even some C particles which are integrated into the verbal morphology (\ref{ex:jouitteau:7}). The latter case is predominantly observed with the verb \textit{emañ}, which raises beyond the Fin head, reaching up to the C head of the ForceP projection. Additionally, a rare instance of verb-first word order is given in (\ref{ex:jouitteau:8}). In that case, an epenthetic \textit{h-} is present at the beginning of the verbal root. This is likely the morphological residue of an ancient epenthetic consonant, induced by a complementizer that ends in a vowel, as illustrated in the phrase (\textit{Ec'h an da …} `I'm going to …').\footnote{See also the matrix verb of (\ref{ex:jouitteau:18}) below.}


\ea \label{ex:jouitteau:5}
\gll \FirstPosition{Hag} {\SecondPosition{ez}} {\SecondPosition{eo}} gwir an dra-se? \\
Q \textsc{fin} \textsc{aux.3s}   true   the  thing-here \\
\glt ‘Is this true?’
\z 


\ea \label{ex:jouitteau:6}
\gll Yann  ha   Lisa  \FirstPosition{ne}    {\SecondPosition{brenint}}      ket   al    levr  d'am   c’hoar   warc'hoazh.\\
Yann   and Lisa   \textsc{neg}  buy\textsc{.fut.3pl} \textsc{neg}  the  book {to my} sister tomorrow \\
\glt  ‘Yann and Lisa will not buy the book for my sister tomorrow.’ 
\z 

\ea%8
    \label{ex:jouitteau:7}
    \gll \FirstPosition{E}{\SecondPosition{mañ}} Yann  ha  Lisa   o  prenañ  ul levr  d'am   c’hoar.\\
        C.be\textsc{.3s} Yann and Lisa  at buy\textsc{.inf}   a  book {to my} sister  \\
    \glt ‘Yann and Lisa are buying a book for my sister.’
    \z

\ea
    \label{ex:jouitteau:8}
    \gll \FirstPosition{‘H}{\SecondPosition{an}}      da brenañ  ul levr   d'am   c’hoar. \\
     C.go\textsc{.1s}  to buy\textsc{.inf}   a  book {to my} sister \\
    \glt ‘I will buy a book for my sister.’
    \z 


Thus far, we have seen only examples where the element before Tense was either introduced from the numeration (e.g., a Q head, a Neg head, or a Force head when it is morphologically integrated into a verb), or by an obligatory movement due to the information structure (e.g., preverbal movement of a \textit{wh-} phrase or another focalized element, or generation of a topic). The most striking property of the Breton T2 requirement is observed when none of those elements appear in the numeration. This context reveals a wide array of last resort strategies obtaining T2.

% examples from this point forward are in the latex code at 1 number more i.e jumping from 8 to 10

\subsection{Markers of last resort operations}

Positive, affirmative, matrix wide focus sentences (\ref{ex:jouitteau:10}--\ref{ex:jouitteau:13}) provide the proper context for observing the last resort operations. The sentence in example ({\ref{ex:jouitteau:10}}) demonstrates expletive insertion, where the expletive \textit{Bez’} manifests as a morphological abbreviation of the verb ‘to be’. In (\ref{ex:jouitteau:10}), none of the verb’s arguments are omitted, a sign that we are observing a transitive expletive. Example (\ref{ex:jouitteau:11}) shows the productive paradigm of \isi{stylistic fronting} (also referred to in the literature on Slavic languages as \isi{long head movement}) characterized by a fronted participle head crossing over a finite auxiliary in the derivation (\cite{mj:BRS1996} and references therein). The structure in example (\ref{ex:jouitteau:12}) represents an analytical tense using dummy ‘do’ insertion, while the structure in example (\ref{ex:jouitteau:13}) is essentially analogous to (\ref{ex:jouitteau:12}) but featuring verb-doubling.\footnote{\citet{mj:Jouitteau2012} derives (\ref{ex:jouitteau:12}) and (\ref{ex:jouitteau:13}) by excorporation of the lexical verb out of the Fin complex, followed by erasure (\ref{ex:jouitteau:12}) or pronunciation (\ref{ex:jouitteau:13}) of the lowest copy. The lexical verb in these analytical tenses in never found lower in the structure, after the tensed element, which is support for the last resort excorporation analysis.}

\ea \label{ex:jouitteau:10}
    \gll \FirstPosition{Bez’}  {\SecondPosition{en}}            {\SecondPosition{deus}}      prenet    Yann    ul levr  d'am   c’hoar.\\
       \textsc{expl}  \textsc{fin.3sm} aux\textsc{.3s} buy\textsc{.ptcp} Yann   a  book  {to my} sister   \\
    \glt `Yann has bought a book for my sister.'
\z 


\ea \label{ex:jouitteau:11}
    \gll \FirstPosition{Prenet} {\SecondPosition{en}}      {\SecondPosition{deus}}     \sout{prenet}             Yann    ul levr   d'am   c’hoar. \\
        buy\textsc{.ptcp} \textsc{fin.3sm} \textsc{aux.3s}  \sout{buy\textsc{.ptcp}} Yann    a  book  {to my} sister  \\
    \glt `Yann has bought a book for my brother.'
    \z
          
\ea \label{ex:jouitteau:12}
    \gll \FirstPosition{Prenañ} {\SecondPosition{a}}     {\SecondPosition{ra}} Yann ul levr  d'am   c’hoar.\\
        buy\textsc{.inf}   \textsc{fin} do\textsc{.pres.3s}  Yann   a  book {to my} sister \\
    \glt `Yann buys a book for my brother.' 

    \z 
     
\ea
 \label{ex:jouitteau:13}
\gll \FirstPosition{Gouzout} {\SecondPosition{a}}     {\SecondPosition{ouzon}} ar wirionez.\\
know\textsc{.inf}  \textsc{fin} know\textsc{.pres.1s}  the truth \\
\glt `I know the truth.'
\z 

\noindent All the structures in (\ref{ex:jouitteau:10}) to (\ref{ex:jouitteau:13}) appear in pragmatic contexts lacking focus on an individual constituent.\footnote{More examples are provided in \citet{mj:Jouitteau2009} on the pages on ‘Bez’ and ‘verb-doubling’.} Each serves as a valid response to questions of the \textit{What happened?} variety. However, in terms of information structure, the sentences under consideration are not strictly equivalent. The structures observed in examples (\ref{ex:jouitteau:11}) and (\ref{ex:jouitteau:12}) are obligatorily backgrounded. In contrast, the structures in examples (\ref{ex:jouitteau:10}) and (\ref{ex:jouitteau:13}) typically exhibit a \textit{verum focus}. This focus emphasizes the truthfulness of the entire sentence. The scope of this focus encompasses the entire sentence. 

The sentences presented in examples (\ref{ex:jouitteau:10}) to (\ref{ex:jouitteau:13}) are all derived from mechanisms employed as a last resort to satisfy the T2 requirement. This requirement is characterized as a prohibition against placing T or Fin at the beginning of a linear sentence, as represented in (\ref{ex:jouitteau:14a}). The hypothesis that the mechanism filling the empty position in (\ref{ex:jouitteau:14a}) is a last resort strategy whose various results are observed in (\ref{ex:jouitteau:10}) to (\ref{ex:jouitteau:13}), leads to several accurate predictions. The structures observed in (\ref{ex:jouitteau:10}) to (\ref{ex:jouitteau:13}) are incompatible with focus of any argument, with topic, or with negation, banning of course \textit{wh-} questions. Furthermore, each structure in (\ref{ex:jouitteau:10}) to (\ref{ex:jouitteau:13}) is mutually exclusive with the others. 

\begin{multicols}{2}
\ea \label{ex:jouitteau:14}
\ea \label{ex:jouitteau:14a} matrix clause \\
\begin{forest}
for tree={s sep=10mm, inner sep=0, l=0, after packing node={s+=0.1pt}}
[,s sep=10mm,nice empty nodes [~][ [Fin-V][[S][...]]]]
\end{forest} \\ $\fbox{\phantom{C}}$-Fin-VSO ...
\ex \label{ex:jouitteau:14b} embedded clause\\
\begin{forest}
for tree={s sep=10mm, inner sep=0, l=0, after packing node={s+=0.1pt}}
[,s sep=10mm,nice empty nodes [C][ [Fin-V][[S][...]]]]
\end{forest} \\ $\fbox{C}$-Fin-VSO
\z
\z
\end{multicols}

\noindent Under the assumption that embedded domains begin with a complementizer, this complementizer should uniformly fulfill the Tense-second (T2) linear requirement, as shown in the tree in ({\ref{ex:jouitteau:14b}}). Empirically speaking, documenting embedded domains with any structures akin to those in examples (\ref{ex:jouitteau:10}) through (\ref{ex:jouitteau:13}) would imply that the complementizer, for some reason, failed to linearize in the expected position, which in turn triggered last resort T2 operations. Understanding the underlying reasons for this phenomenon could markedly advance our comprehension of the \isi{embedding} process in grammatical theory. The subsequent section will summarize prior findings on the grammar modules at play.

\subsection{Post-syntactic morphology and linearization}

We have evidence suggesting that the operation responsible for positioning the initial element in its place in examples (\ref{ex:jouitteau:10}) to (\ref{ex:jouitteau:13}) occurs in a morphological module whose input is the output of syntactic operations. The last resort effects are invisible for interpretation. They violate syntactic rules. Specifically, the initial element in example (\ref{ex:jouitteau:11}) contravenes the head-movement constraint \citep{mj:Travis1984} by allowing a head to cross over another head in the derivation process. In example (\ref{ex:jouitteau:10}), merge of the transitive expletive has no impact on interpretation, not even indirectly such as a restriction on the definiteness of an argument. In (\ref{ex:jouitteau:12}), pronunciation of the infinitive verb before Tense makes no difference with a synthetic verb on the semantic interface. Auxiliary insertion is equally imperceptible to semantic interpretation, and independently suggests a morphological module. Lastly, verb-doubling, as observed in example (\ref{ex:jouitteau:13}), is confined to a specific set of verbs that do not constitute a syntactic class, and show morphological idiosyncrasies, as argued by \citet{mj:Jouitteau2020}.\footnote{Verb-doubling is also idiosyncratic. Speakers differ in the verbs they double. Not all speakers are doublers. The Leon speaker MLB for example never does.} \citet{mj:Rivero1999, mj:Rivero2000} first proposed that Breton Tense-second results from a filter operating after syntax. She identified this module as the phonological form (PF). This hypothesis is on the right track but is empirically incorrect: the last resort operations for T2 are followed by morphological operations. The mechanism responsible for the initial element is located in a grammatical module capable of processing categorical distinctions. The preverbal particle, glossed as \textsc{fin} in our examples, is responsive to the [± nominal] category of the initial element.\footnote{This phenomenon is not uniformly evident across all dialects. Only in the dialects of Leon, Standard, and Gwened do the Fin heads consistently alternate based on the [± nominal] category of the preceding constituent. In this article, embedded T2 orders are from Leon and show robust alternations depending on categorical distinctions.} 

In prior research (\cite{mj:Jouitteau2011, mj:Jouitteau2012, mj:Jouitteau2020}), I have suggested that the T2 requirement amounts to morphological exponence operating at the clause level. This requirement operates after syntax, in a module where non-syntactic operations can be dealt with, such as excorporation, inversion, idiosyncrasy, auxiliary insertion, copy deletion and non-deletion. These operations are extremely local (inversion, excorporation), and linear in nature. 

The derivation of Breton clauses is recapitulated in (\ref{ex:jouitteau:15}). Breton is typologically consistent with the other Celtic languages in its syntax: at the level of syntax, Breton is VSO like any other Celtic language. A language-specific morphological requirement operating at the level of the clause, historically gained by contact with Romance T2, imposes that Fin is not initial. 

\ea
    \label{ex:jouitteau:15}
    \ea \label{ex:jouitteau:15a} Breton has an obligatory morphological exponence operating at the level of clauses.
    \ex \label{ex:jouitteau:15b} It applies to both matrix and embedded clauses. 
    \ex \label{ex:jouitteau:15c} This filter operates after syntax and before PF, in a module that can manipulate [\pm{nominal}] categories, proceed to linear inversion of elements, apply excorporation and handle morphological support like external merge of dummy auxiliaries or expletives. 
    \z
    \z
    
\noindent The generalization in (\ref{ex:jouitteau:15c}) leads to the speculation that the last resort operations for T2 happen during the linearization process of the \textsc{fin} phrase. The invisibility of hanging topics and scene-setting adverbs between the B and C lines in \figref{ex:jouitteau:4} is tentatively correlated with the sensitivity of this linearization process to prosody. If the linearization process is prosody sensitive, then we could correctly obtain the desired prediction that the obligatory exponence is blind to the domains that are prosodically isolated. I leave this for further research. We now are equipped to step into the embedded domains.

\section{The trouble with linear T2 embeddings}\label{sec:jouitteau:3}

\is{embedding|(}
Embedded domains are initiated by a C head. Our observations indicate that Breton functions as a linear T2 language wherein functional heads can fulfill T2 requirements. Consequently, it is predicted that the linearization of wide focus embedded sentences should consistently follow a C-Fin-VSO pattern.\footnote{Some embedded domains are known to show an exceptionally wide left periphery, like those headed by \textit{rak} ‘because’, and those which tolerate focus or even hanging topics hosted in their wide left periphery. Their word order is basically that of root clauses. These complementizers never lead to C-Fin-VSO orders. We will not consider these structures here.}  Consequently, employing last resort T2 strategies should be uniformly ungrammatical across these embedded domains. 

This prediction generally holds true, and definitely does in standard Breton. Our Kerne speaker Huguette Gaudart strongly opposes any form of embedded \isi{stylistic fronting}, as exemplified in example (\ref{ex:jouitteau:16})~– here with a phonologically null Fin head. Her judgments apply both to her native Skaer dialect of Kerne and to her proficient standard dialect of the language. This generalization has been found consistently valid for her across various types of complementizers. Her C-\textsc{fin}-VSO orders never accommodate optional alternations with structures such as presented in examples (\ref{ex:jouitteau:10}) to (\ref{ex:jouitteau:13}).

\ea \label{ex:jouitteau:16} \langinfo{Kerne (Skaer/Banaleg)}{}{Gaudart 04/2016b}\\
\ea[]{ \label{ex:jouitteau:16a}
    \gll  N’ uion ket \FirstPosition{ma} {\SecondPosition{‘}}     {\SecondPosition{n-eus}} lennet al levr. \\
    C.\textsc{neg} know\textsc{.pres.1s} \textsc{neg}  if     \textsc{fin} 3\textsc{{}-aux} read\textsc{.ptcp} the book \\
    \glt `I don’t know if she has read the book.’
    }
\ex[*]{ \label{ex:jouitteau:16b}
    \gll N’       uion                 ket  ma \FirstPosition{lennet}        {\SecondPosition{‘}}     {\SecondPosition{n-eus}}  \sout{lennet}               al  levr. \\
     C.\textsc{neg} know\textsc{.pres.1s} \textsc{neg}  if   read\textsc{.ptcp}   \textsc{fin} 3.\textsc{aux} \sout{read\textsc{.ptcp}} the book \\
    \glt `I don’t know if she has read the book.’\\
    }
    \z
    \z
     
\noindent Surprisingly, the speakers of the Leon dialect exhibit sporadic exceptions and allow for structures equivalent to (\ref{ex:jouitteau:16b}). In the remainder of this section, we will first examine these exceptions. We then address the limitations of this evidence. 

\subsection{Evidence for embedded expletive strategies}

In example (\ref{ex:jouitteau:17a}), the word order resembles that in (\ref{ex:jouitteau:16}), where the \textit{ma} complementizer leads a conventional linear T2 order. In contrast, example (\ref{ex:jouitteau:17b}) exhibits \isi{stylistic fronting}, also grammatical for the speaker. Some caution is in order in this specific example because my conclusion is  dependent on the generalization that the speaker has Aux-V-S orders (\textit{Ma ‘n-eo renket an amezog…}) before inversion of the auxiliary and the verb. The grammaticality of this specific structure however could not be empirically verified during the research protocol: the speaker gave Aux-S-V in (\ref{ex:jouitteau:17a}). The analysis that (\ref{ex:jouitteau:17b}) involves \isi{stylistic fronting} proper and not remnant is predicated on the assumption that the past participle has undergone derivation at an intermediary stage, positioned immediately after the Tense head. This assumption however appears here plausible, given that this speaker consistently permits participles to precede a lexical subject in the middle field in other instances. 



\ea \label{ex:jouitteau:17} \langinfo{Leon (Plougerne)}{}{MLB 05/2018}\\
\ea \label{ex:jouitteau:17a}
    \gll \FirstPosition{Ma}   {\SecondPosition{‘}}     {\SecondPosition{n-eo}}            an  amezog     renket      e    wele  ne   hellan   ket  lavaret dit!\\
     if     \textsc{fin} \textsc{3s-aux.fut}  the neighbor tidy\textsc{.ptcp}  his bed    \textsc{neg} can\textsc{.1s} \textsc{neg} tell\textsc{.inf}  you    \\
    \glt `Whether the neighbor will have made his bed, I can’t tell you!’ 
    \ex \label{ex:jouitteau:17b}
    \gll Ma  \FirstPosition{renket}    {\SecondPosition{‘}}    {\SecondPosition{n-eo}}          \sout{renket}     an  amezog       \sout{renket}      e    wele  ne   hellan     ket lavaret dit! \\
    if tidy\textsc{.ptcp} \textsc{fin} \textsc{3s-aux.fut}  \sout{tidy\textsc{.ptcp}} the neighbor \sout{tidy\textsc{.ptcp}} his bed \textsc{neg} can\textsc{.1s} \textsc{neg} tell\textsc{.inf} you \\
    \glt `Whether the neighbor will have made his bed, I cannot tell you!' \\
    \z
    \z 

\noindent The sentences in (\ref{ex:jouitteau:18}) replicate the findings with the tense complementizer \textit{pa} ‘when’. 

\newpage
\ea \label{ex:jouitteau:18} \langinfo{Leon (Plougerne)}{}{MLB 05/2018}\\
    \ea \label{ex:jouitteau:18a}
    \gll \FirstPosition{Pa}       {\SecondPosition{m-eus}}    kleet         ar janson-se     evit  ar  wech kentañ, eved-on         dougeres. \\
     When  \textsc{1s-aux} hear\textsc{.ptcp}  the song-there  for   the time  first      C.\textsc{aux.past}{}-\textsc{1s} pregnant   \\
    \glt `When I first heard this song, I was pregnant.’
    \ex \label{ex:jouitteau:18b}
    \gll Pa     \FirstPosition{kleet}         {\SecondPosition{m-eus}}  \sout{kleet}       ar janson-se    evit ar wech kentañ, eved-on    dougeres. \\
     when hear\textsc{.ptcp} 1\textsc{s-aux} \sout{hear\textsc{.ptcp}} the song-there  for the time  first     C.aux\textsc{.past}{}-\textsc{1s} pregnant \\
    \glt `When I first heard this song, I was pregnant.’ \\
    \z 
    \z  

The sentences in (\ref{ex:jouitteau:19}) replicate the findings with another Leon speaker and another last resort strategy: an analytic tense (\ref{ex:jouitteau:11}).

\ea \label{ex:jouitteau:19} \langinfo{Leon (Lesneven/Kerlouan)}{}{AM 05/2016}\\
\ea \label{ex:jouitteau:19a}
    \gll  Bevañ    a    r-eomp        un amzer \FirstPosition{hag}  {\SecondPosition{e}}     {\SecondPosition{hoar}}         ar  vugale  diouzh  an   ordinatourien  muioc'h  eget  o      zud. \\
       live\textsc{.inf}  \textsc{fin} do\textsc{-pres.1pl}  a  time    that  \textsc{fin}  know\textsc{.3s} the child\textsc{.pl}
  of         the  computers       more       than their parents   \\
    \glt   ‘We live a time where children know more about computers than their parents.’
    \ex \label{ex:jouitteau:19b}
    \gll Bevañ    a    r-eomp       un amzer ha    \FirstPosition{gouzout}    {\SecondPosition{a}}     {\SecondPosition{r-a}}     ar  vugale diouzh  an   ordinatourien  muioc'h  eget  o      zud. \\
    live\textsc{.inf}  \textsc{fin} do\textsc{{}-pres.1pl} a  time     that  know\textsc{.inf}  \textsc{fin} do-3\textsc{s} the child\textsc{.pl}   of         the  computers       more       than their parents \\
    \glt ‘We live a time where children know more about computers than their parents.’ 
    \z 
    \z
       
The following examples replicate the findings with a declarative C head, likely the head of ForceP. This element is phonologically null (∅). This silent complementizer typically selects clausal complements in most Breton dialects, creating an apparent radical Tense-first order on the surface. The sentence in (\ref{ex:jouitteau:20a}) demonstrates that this C head alone can fulfill the T2 requirement. Embedded \isi{stylistic fronting} is also a possibility, as shown in (\ref{ex:jouitteau:20b}). Within the clausal complement of a noun, as seen in (\ref{ex:jouitteau:21}), the C can either precede the verb directly or allow the insertion of an expletive, as in (\ref{ex:jouitteau:21a}, where the \textit{bed} morpheme is a dialectal variation of the transitive expletive \textit{bez’} (previously mentioned in (\ref{ex:jouitteau:10})). An analytic tense form, similar to what was illustrated in (\ref{ex:jouitteau:12}), represents yet another option, as seen in (\ref{ex:jouitteau:21b}) and in (\ref{ex:jouitteau:22}).

\ea \label{ex:jouitteau:20}
\ea \label{ex:jouitteau:20a}
    \gll Amelia  e    n-eus      zouchet      \FirstPosition{\BoldNull}    {\SecondPosition{en}}         {\SecondPosition{n-eus}}      kemeret     an  alc’hwez fall. \\
        Amelia \textsc{fin} \textsc{3s-aux}  think\textsc{.ptcp}   C  \textsc{fin.3s} \textsc{3s-aux}   take\textsc{.ptcp}   the key wrong \\
    \glt ‘Amelia thought she had taken the wrong key.’
    \ex \label{ex:jouitteau:20b}
    \gll Amelia  e    n-eus       zouchet    ∅   \FirstPosition{kemeret}   {\SecondPosition{en}}         {\SecondPosition{n-eus}}      \sout{kemeret}      an  alc’hwez fall. \\ 
    Amelia \textsc{fin} \textsc{3s-aux}  think\textsc{.ptcp} C  take\textsc{.ptcp} \textsc{fin.3s} \textsc{3s-aux} \sout{take\textsc{.ptcp}}  the key         wrong  \\
    \glt ‘Amelia imagined/thought she had taken the wrong key.’ 
    \z 
    \z

\ea \label{ex:jouitteau:21}
\ea \label{ex:jouitteau:21a}
    \gll Ar  brud          vat     e-n        d-oa           ∅  (\FirstPosition{bed})  {\SecondPosition{e}}    {\SecondPosition{houie}}               dresañ  pep    tra. \\
     the reputation good  \textsc{fin}-\textsc{3s} \textsc{3-aux.past}  C  expletive \textsc{fin} know\textsc{.past.3s} fix\textsc{.inf}    every thing
    \\
    \glt ‘He had a good reputation for knowing how to fix anything.’ 
    \ex \label{ex:jouitteau:21b}
    \gll  Ar  brud          vat    e-n        d-oa          ∅    \FirstPosition{gouzout}   {\SecondPosition{a}}    {\SecondPosition{r-ae}}            dresañ  pep    tra. \\
    the reputation good \textsc{fin}{}-\textsc{3s} 3\textsc{{}-aux.past} C   know\textsc{.inf} \textsc{fin} do\textsc{{}-past.3s} fix\textsc{.inf}    every thing \\
    \glt ‘He had a good reputation for knowing how to fix anything.’ 
    \z 
    \z

\ea
    \label{ex:jouitteau:22} \langinfo{Leon (Plougerne)}{}{MLB 05/2018}\\
    \gll Simone a    jouch  ∅   \FirstPosition{gouzout}   {\SecondPosition{a}}    {\SecondPosition{r-aio}}         ar sodenn-se  troc'hañ uheloc'h ar harzenn. \\
      Simone Fin think  C  know\textsc{.inf} \textsc{fin} do\textsc{{}-fut.3s} the idiot-here  cut\textsc{.inf}   higher    the hedge \\
    \glt  ‘Simone thinks that this idiot will know to cut the hedge higher.’ 
    \z 

The other Leon speaker AM also permits the optional inclusion of embedded \isi{stylistic fronting}, as demonstrated in example (\ref{ex:jouitteau:23}). This is illustrated with a past participle in (\ref{ex:jouitteau:23b}). Linear inversion can target any element immediately following the tensed element, as shown with an extra aspectual marker in (\ref{ex:jouitteau:23c}). 


\protectedex{
\ea \label{ex:jouitteau:23} \langinfo{Leon (Lesneven/Kerlouan)}{}{AM 05/2016}\\
\ea \label{ex:jouitteau:23a}
\gll   Mimi  a     joñch           \FirstPosition{\BoldNull}    {\SecondPosition{e}}     {\SecondPosition{trouc'ho}}     ar  sodenn-se uheloc'h ar  harzenn e-giz ar   bloavezh paseet.  
 \\
 Mimi \textsc{fin} think\textsc{.pres.sg}  C    \textsc{fin} cut\textsc{.fut.3s} the idiot-here   higher    the hedge    like   the  year        pass\textsc{.ptcp} \\
\glt ‘Mimi thinks that this idiot will cut the hedge higher than last year.’
\ex \label{ex:jouitteau:23b}
\gll Mimi  a     joñch             ∅   \FirstPosition{troc'het}  {\SecondPosition{n-eus}}  \sout{troc'het}   ar  sodenn-se   uheloc'h ar  harzenn e-giz ar   bloavezh paseet. \\
Mimi \textsc{fin} think\textsc{.pres.sg} C   cut\textsc{.ptcp}  \textsc{3s-aux}  \sout{cut}  the idiot-here   higher    the hedge    like   the  year      pass\textsc{.ptcp} \\
\glt ‘Mimi thinks that this idiot has cut the hedge higher than last year.’  
\ex \label{ex:jouitteau:23c}   
\gll Mimi  a     joñch              ∅    \FirstPosition{bet}    {\SecondPosition{n-eus}}      \sout{bet}        troc'het  ar  sodenn-se uheloc'h ar  harzenn e-giz ar   bloavezh paseet. \\ 
Mimi  \textsc{fin} think\textsc{.pres.sg}   C    be\textsc{.ptcp} \textsc{3s-aux}  \sout{be\textsc{.ptcp}} cut\textsc{.ptcp} the idiot-here higher    the hedge    like   the  year        pass\textsc{.ptcp} \\
\glt  ‘Mimi thinks that this idiot has cut the hedge higher than last year.’ 
\z
\z 
}

This study is not the first to report \isi{stylistic fronting} as permissible within embedded domains in Breton. \citet{mj:Rivero1999} presented the critical minimal pair in (\ref{ex:jouitteau:24}), based on the grammatical judgment of Janig Stephens, a native speaker of the Treger dialect. The \textit{-eñ} morpheme is homophone with the third-person singular masculine independent pronoun and forms a unique compound \textit{hag-eñ}, which is exclusive to the Treger dialect. It occurs solely within embedded domains. The prevailing hypothesis, put forth in \citet[318]{mj:Lambert1998} and \citet[81--82]{mj:Rivero1999}, posits that \textit{-eñ} in example (\ref{ex:jouitteau:24a}) functions as an expletive. Notably, the embedded expletive \textit{-eñ} and the \isi{stylistic fronting} operation illustrated in example (\ref{ex:jouitteau:24b}) are observed to be in complementary distribution. 


\ea \label{ex:jouitteau:24} \langinfo{Treger}{}{\cite[81--82]{mj:Rivero1999}}\\
\ea \label{ex:jouitteau:24a}
    \gll  N'   ouz-on              ket  hag-\FirstPosition{eñ}  {\SecondPosition{e-n}}        {\SecondPosition{d-eus}}         lennet     al   levr.  \\
     \textsc{neg} know\textsc{.pres}{}-\textsc{1s} \textsc{neg} if-\textsc{expl}   \textsc{fin}{}-\textsc{3sg} 3\textsc{{}-aux.pres} read\textsc{.ptcp} the book      \\
    \glt ‘I don't know if he has read the book.’
    \ex \label{ex:jouitteau:24b}
    \gll N'   ouz-on              ket   ha \FirstPosition{lennet}       {\SecondPosition{e-n}}   {\SecondPosition{d-eus}}             \sout{lennet}         al   levr. \\
    \textsc{neg} know\textsc{.pres}{}-\textsc{1s} \textsc{neg}  if    read\textsc{.ptcp} \textsc{fin}{}-\textsc{3s} 3\textsc{{}-aux.pres} \sout{read\textsc{.ptcp}}  the book   \\
    \glt ‘I don't know if he has read the book.’
    \z
    \z 

The empirical generalization that in the northern dialects of Leon and Treger, the initial presence of a C head in embedded domains does not ban post-syntactic word order re-arrangements captures these examples. Operations that we know to be last resort for T2 in matrix sentences here seem optional. Such options are limited, however.

\subsection{Limits to embedded last resort T2}

There are significant limitations to embedded T2. First, and this is very important, it is absent from corpora.\footnote{Breton corpora have reasonable size and quality, but the lack of a richly annotated corpus and parser impacts its searchability. The wikigrammar ARBRES (\citealt{mj:Jouitteau2009}) describes Breton syntactic variation with hand-picked examples coming from about 500 published references. No embedded verb-second has been reported in syntactic environment where it is last resort, despite a long-term focus on the matter (\citealt{mj:Jouitteau2009}: ‘Embedded V2\is{verb second}’).} Following the descriptive literature so far, it does not exist. Prescriptive or pedagogical texts in standard Breton would unambiguously mark it as ungrammatical. Embedded last resort T2 orders are obtained only in an elicitation context. However, its grammaticality among the speakers interviewed is strong. Prior to participating in this elicitation, these individuals underwent testing to ascertain their ability to evaluate grammatical structures accurately. I verified that their judgments remain consistent over time. When queried about the potential distinction between the two options, the speakers reported no perceivable difference. Nevertheless, when asked to express a preference between the two, there was a clear and consistent inclination towards the regular C-Fin-VSO word orders. 
\is{V2|see {verb second}}
\is{T2|see {tense second}}

Subtle variations in information structure significantly affect the grammaticality of embedded expletive strategies. Consider example (\ref{ex:jouitteau:25}), where MLB does not permit the \isi{stylistic fronting} of the participle (\ref{ex:jouitteau:25b}). This phenomenon unlikely results from an intervention effect related to inversion, as this speaker typically allows for the positioning of past participles before lexical subjects. In contrast, subject fronting is deemed grammatical in (\ref{ex:jouitteau:25c}). The analysis in this case must consider the possibility that the speaker is invoking a context that induces a contrastive focus on the entity ‘Julie’. Replication of this inquiry necessitates a rigorously controlled protocol attentive to information structure nuances and well-informed about the specific potential prosodies in both matrix and embedded sentences. For the moment, we are here facing a gap in the availability of prosodic models for Breton that would enable reliable predictions. 


\ea \label{ex:jouitteau:25}  \langinfo{Leon (Plougerne)}{}{MLB (05/2018}\\
\ea[]{ \label{ex:jouitteau:25a}
    \gll Me  chouj               d'ar   poent-se    e     houie \FirstPosition{\BoldNull}  {\SecondPosition{e}}     {\SecondPosition{n-oa}}   Julie   kemeret va oto. \\
     I think\textsc{.pres.1s} {at the} time-there \textsc{fin} know\textsc{.past.3s} C  \textsc{fin} 3\textsc{{}-aux.past}   Julie   take\textsc{.ptcp}      my car    \\
     }
    \ex[*]{ \label{ex:jouitteau:25b}
    \gll Me  chouj               d'ar   poent-se    e     houie ∅  \FirstPosition{kemeret} {\SecondPosition{e}}     {\SecondPosition{n-oa}}  Julie  \sout{kemeret}       va oto.\\
    I      think\textsc{.pres.1s} {at the} time-there \textsc{fin} know\textsc{.past.3s} C  take\textsc{.ptcp}     \textsc{fin} 3\textsc{{}-aux.past}   Julie \sout{take\textsc{.ptcp}} my car \\
    }
    \ex[]{ \label{ex:jouitteau:25c}
    \gll Me  chouj              d'ar   poent-se    e     houie ∅  \FirstPosition{Julie}  {\SecondPosition{e}}     {\SecondPosition{n-oa}}  \sout{Julie}      kemeret va oto. \\
    I      think\textsc{.pres.1s} {at the} time-there \textsc{fin} know\textsc{.past.3s} C  Julie  \textsc{fin} 3\textsc{{}-aux.past}  \sout{Julie} take\textsc{.ptcp} my car \\
    \glt  ‘I think from that moment he knew that Julie had taken my car.' \\
    }
    \z
    \z

            
At this stage, it is important to observe that alterations in prosody directly influence the interpretability of embedded clauses as independent root clauses. The introduction of prosodic breaks results in variations in grammaticality judgments. For instance, the speaker AM in (\ref{ex:jouitteau:26a}) deemed the embedded \isi{stylistic fronting} as ungrammatical. However, the inclusion of a prosodic break, as shown in example (\ref{ex:jouitteau:26b}), rendered it grammatically acceptable. It should be noted, however, that this rectification through prosodic intervention is not universally applicable (\ref{ex:jouitteau:27}). The specific conditions where a prosodic break will make a sentence grammatical or not should be seriously investigated, with an informed modeling of Breton prosody.


\ea \label{ex:jouitteau:26} \langinfo{Leon (Lesneven/Kerlouan)}{}{AM 05/2016}\\
\ea[*]{ \label{ex:jouitteau:26a}
    \gll Me a gred ∅   \FirstPosition{plantet}  {\SecondPosition{‘}}     {\SecondPosition{n-eus}} Simone patatez   er bloavezh-mañ egist he  amezog. \\
    I \textsc{fin} thinks   C  planted \textsc{fin} 3.\textsc{aux.pres}   Simone potato\textsc{.pl} in.the year-here   like  her neighbor   \\
    \glt ‘I think that Simone has planted potatoes like her neighbor this year.’
    }
    \ex[]{ \label{ex:jouitteau:26b}
    \gll Me a gred   ∅  //\FirstPosition{plantet}  {\SecondPosition{‘}}     {\SecondPosition{n-eus}} Simone patatez   er bloavezh-mañ egist he  amezog. \\ 
    I \textsc{fin} thinks C      planted \textsc{fin} 3.\textsc{aux.pres}   Simone potato\textsc{.pl} in.the year-here   like  her neighbor \\ 
    \glt ‘I think that Simone has planted potatoes like her neighbor this year.’  
    }
    \z
    \z 


\ea \label{ex:jouitteau:27} \langinfo{Leon (Lesneven/Kerlouan)}{}{AM 05/2016}\\
    \gll *Mimi  a     joñch  ∅  //\FirstPosition{gouzout}  {\SecondPosition{a}}     {\SecondPosition{raio}}        ar  sodenn-se  troc’hañ berroc'h  ar  harz      e-giz ar   bloavezh paseet. \\
     Mimi \textsc{fin} thinks C     know.\textsc{inf} \textsc{fin} do\textsc{.fut.3s}   the idiot-here  cut.\textsc{inf} shorter    the hedge like the year passed \\
    \glt ‘Mimi thinks that this idiot will know to cut the hedge shorter than last year.’ \\
    \z 

\noindent With this in mind, we will now check that in the gathered evidence for embedded T2\is{tense second} so far, the clauses were fully integrated into the syntactic structure, and not merely pronounced after another root clause. Our conclusion will be that embedded T2 sentences under investigation here are indeed syntactically embedded. 

\section{The syntactic integration of embedded T2}\label{sec:jouitteau:4}

Wurmbrand's (\citeyear{mj:Wurmbrand2012}; \citeyear{mj:Wurmbrand2014}) work proposes an interface mechanism that accounts for variations in the integration of embedded T2 sentences. According to her model, if two clauses merge in syntax, they achieve full integration. Alternatively, they may merge in a module termed “Transfer” before being processed by Phonological Form (PF) and Logical Form (LF). The level of integration will next depend on Spell-out considerations. When realized as separate Spell-out domains, the two domains maintain complete independence. Conversely, if they are realized within a single Spell-out domain, their integration is partial. In such a case, phenomena such as quantifier binding are observable, yet the two structures retain syntactic independence. This is summarized in (\ref{ex:jouitteau:28c}).

\ea \label{ex:jouitteau:28c}
\begin{itemize}
    \item {\textbf{Fully integrated clauses}: Merge in syntax}
    \item {\textbf{Partially integrated clauses}: Merge at Transfer, single Spell-out domain}
    \item {\textbf{Non-integrated clauses}: Merge at Transfer, separate Spell-out domains}
\end{itemize}
\z

\citet{mj:Wurmbrand2014} relies on the root generalization, by which a TP or CP is a root clause if and only if it has interpretable tense in the top projection. This is meant to explain why in German the clauses that do not instantiate verb raising will never count as roots: German verb-final sentences will always be fully integrated embedded clauses. A stipulation bans Tense marked clauses Merging with a higher verb in syntax, and this leads to Merge outside of syntax, at \textit{Transfer} before spell-out.  

It remains unclear why in this scenario syntactic root clauses persistently start with a complementizer. However, the prediction is clear. In clauses that are tense-second on the surface but are only partially integrated into syntax or not integrated at all, we should document no extraction, no movement, and only root phenomena. I now show that Breton disproves those predictions. I propose that, in Breton and in probably all linear T2 languages, the word orders that surface as last resort T2 clauses are syntactically integrated. At the syntactic stage of the derivation they might still not be T2, and this order will be linearly enforced later in the linearization process. 

\subsection{Weak island effects}

The control sentence in (\ref{ex:jouitteau:29a}) shows gram\-ma\-ti\-cal\-ity of embedded stylistic front\-ing. The speaker showed some hesitation, and finally accepted the sentence. Extraction out of the clausal complements is grammatical for the adjunct \textit{betek peleah} in (\ref{ex:jouitteau:29b}). Stylistic fronting\is{stylistic fronting} is not compatible with this extraction in (\ref{ex:jouitteau:29c}).  

\ea \label{ex:jouitteau:29} \langinfo{Leon (Plougerne)}{}{MLB 05/2018}\\
\ea[?]{ \label{ex:jouitteau:29a}
    \gll Jack  e   n-eus chouch  ∅   \FirstPosition{baleet}   {\SecondPosition{e}}    {\SecondPosition{n-eus}} gad   e   jochonoù beteg an daol  hep    kouezhañ.\\
    Jack \textsc{fin} \textsc{3s-aux.pres} think\textsc{.ptcp} C walk\textsc{.ptcp} \textsc{fin} \textsc{3s-aux.pres}    with his slipper\textsc{.pl}    to    the table without fall.\textsc{inf} \\
    \glt ‘Jack remembers that with the slippers on, he walked to the table without falling.’
    }
    \ex[]{ \label{ex:jouitteau:29b}
    \gll \FirstPosition{Betek} \FirstPosition{peleah}  {\SecondPosition{e}} {\SecondPosition{n-eus}} Jack kredet     \FirstPosition{\BoldNull} [{\SecondPosition{e}}   {\SecondPosition{n-eche}} baleet   hep        kouezhañ  \sout{betek peleah}]? \\
    until  where    \textsc{fin} \textsc{3s-aux.pres} Jack believe\textsc{.ptcp} C \textsc{fin} \textsc{3s-aux}.\textsc{cond.past} walk\textsc{.ptcp} without fall.\textsc{inf}  \sout{until  where} \\
    \glt  ‘Until where did Jack think he would walk without falling?’
    }
    \ex[*]{ \label{ex:jouitteau:29c}
    \gll \FirstPosition{Betek} \FirstPosition{peleah}  {\SecondPosition{e}} {\SecondPosition{n-eus}} Jack kredet     ∅ [\FirstPosition{baleet}   {\SecondPosition{e}}   {\SecondPosition{n-eche}}  \sout{baleet}             hep       kouezhañ  \sout{betek peleah}]? \\
    until  where    \textsc{fin} \textsc{3s-aux.pres} Jack believe\textsc{.ptcp} C walk\textsc{.ptcp} \textsc{fin} \textsc{3s-aux}.\textsc{cond.past} \sout{walk\textsc{.ptcp}} without fall.\textsc{inf}  \sout{until  where} \\
    \glt ‘Until where did Jack think he would walk without falling?’ \\
    }
    \z
    \z 

                
Object extraction does not show the same restriction. Extraction out of the clausal complements is grammatical for the object \textit{pese parti} in (\ref{ex:jouitteau:30a}). Stylistic fronting\is{stylistic fronting} is compatible with this extraction in (\ref{ex:jouitteau:30b}). The grammaticality of (\ref{ex:jouitteau:30b}) shows this embedded T2 domain is syntactically integrated. Under the assumption that T2 was realized here by inversion of a postverbal subject into the pre-Tense position (\isi{stylistic fronting}), (\ref{ex:jouitteau:30b}) shows that the embedded domains where we observe last resort operations for T2 are syntactically integrated. (\ref{ex:jouitteau:31}) replicates (\ref{ex:jouitteau:30b}) with a context that lowers the plausibility of the embedded subject being fronted by focus. The contrast between object extraction (\ref{ex:jouitteau:30b}, \ref{ex:jouitteau:31}) and (\ref{ex:jouitteau:29c}) suggests that embedded T2 structures are weak islands. 


\ea \label{ex:jouitteau:30}
\ea \label{ex:jouitteau:30a}
    \gll Pese parti   ar  journal      a    lavar            \FirstPosition{\BoldNull}  [\textbf{e}  \textbf{n-eus}  an amezog      soutenet \sout{pese parti}]? \\
    what party the newspaper \textsc{fin} say\textsc{.pres.3s} C    \textsc{fin} \textsc{3s-aux.pres}   the neighbor support\textsc{.ptcp} \sout{what party}    \\
    \glt   ‘What party did the journal say that the neighbor has supported?’
    \ex \label{ex:jouitteau:30b}
    \gll Pese parti   ar  journal       a    lavar ∅   [\FirstPosition{an amezog}       \textbf{e}  \textbf{n-eus}    \sout{an amezog}                soutenet \sout{pese parti}]? \\
    what party the newspaper \textsc{fin} say\textsc{.pres.3s}    C     {the neighbor} \textsc{fin} \textsc{3s-aux.pres} \sout{the neighbor} support\textsc{.ptcp} \sout{what party}\\
    \glt ‘What party did the journal say that the neighbor has supported?’
    \z 
    \z 

    
\ea \label{ex:jouitteau:31} \langinfo{Leon (Plougerne)}{}{MLB (05/2018}\\
    \gll Petra  an archerien    e   n-eus soñjet   ∅  [\FirstPosition{ar  seurezed} {\SecondPosition{a}}   {\SecondPosition{houie}}   \sout{ar  seurezed}]?  \\
    what  the policeman\textsc{.pl} \textsc{fin} \textsc{3s-aux.pres}  think\textsc{.ptcp} C     {the sister\textsc{.pl}}     \textsc{fin} know\textsc{.past.3s}  \sout{the sister\textsc{.pl}}\\
    \glt ‘What did the policemen think that the sisters knew (all along)?’
    \z 

\noindent Similar data are available in Germanic languages, but their analysis is less straightforward because of parentheticals. It has been proposed for German that argument extraction from an embedded T2 clause results from the use of a parenthetical in a monoclausal \textit{wh-}sentence (\cite{mj:Reis1997}; \cite{mj:Reis2002}; see \cite{mj:Kiziak2010} for a summary of this debate for Germanic languages). According to this perspective, the embedding in the example (\ref{ex:jouitteau:32}) should be analyzed more accurately as \textit{Welchen film} [hat sie gesagt] \textit{haben die Kinder gesehen.} 

\ea \label{ex:jouitteau:32}
    \gll  Welchen film [hat  sie  gesagt [haben die  Kinder  gesehen]]?   \\
     what       film   \textsc{aux.3s}  she say\textsc{.ptcp}   have   the  child\textsc{.pl}         see\textsc{.ptcp}    \\
    \glt `What film did she say that the children have seen?’
    \z 
             
\noindent This counterargument does not apply to Breton. Breton lexical subjects in parentheticals are uniformly postverbal, as opposed to the lexical subjects illustrated above in (\ref{ex:jouitteau:30b}) or (\ref{ex:jouitteau:31}).

\subsection{Freezing of clausal complements}

Another syntactic piece of evidence comes from movement of clausal complements. In most dialects, including the standard, fronting of a clausal complement is not allowed. This is shown in (\ref{ex:jouitteau:33}). In (\ref{ex:jouitteau:33a}), the clausal complement is in-situ, and the initial position of the matrix clause is filled in by a last resort excorporation of the lexical material of the inflected verb, leaving a semantically empty \textit{do} auxiliary in Tense position. In (\ref{ex:jouitteau:33b}), the clausal complement fills in the initial matrix position, letting the lexical verbal material in the Tense position. The pattern illustrated in (\ref{ex:jouitteau:33}) is consistent across dialects.

\ea  \label{ex:jouitteau:33} \langinfo{Standard}{}{\cite{mj:Kervella1947}}
\ea[]{  \label{ex:jouitteau:33a}
\gll  Karout a    rafen           [∅   e    tiskouezfec’h         an dra-se        dezhañ]. \\
love.\textsc{inf} \textsc{fin} do.\textsc{cond.pres.1s}    C  \textsc{fin} show.\textsc{cond.pres.2pl} the thing-here to.\textsc{3sm} \\
\glt ‘I would like you to show this to him.’}
\ex[*]{ \label{ex:jouitteau:33b}
\gll [∅  E   tiskouezfec’h an dra-se        dezhañ] a garfen \\
C  \textsc{fin} how.\textsc{cond.pres}.\textsc{2pl} the thing-here to.him     \textsc{fin} like.\textsc{cond}.\textsc{pres.1s} \\}
\z
\z

However, in (\ref{ex:jouitteau:34}), MLB in Plougerneau (Leon) does allow for movement of a clausal complement (with an empty C)\footnote{I leave unexplained why her grammar allows for this movement. Fronting of elements starting with a phonologically empty head is not usual.} in (\ref{ex:jouitteau:34b}). This clausal complement is compatible with \isi{stylistic fronting} (c), but movement is ungrammatical if the clausal complement is T2. The relevant minimal pair is between (\ref{ex:jouitteau:34b}) and (\ref{ex:jouitteau:34d}). Presumably, movement of the clausal complement must involve the complementizer, which forces it to linearize within the same domain as Fin prior to movement. This forced inclusion of C in turn prevents last resort operations for T2.  


\ea \label{ex:jouitteau:34}\langinfo{Leon (Plougerne)}{}{MLB 05/2018}\\
\ea[]{ \label{ex:jouitteau:34a}
\gll Tout an dud      o           d-eus         keuz  [\FirstPosition{\BoldNull}  {\SecondPosition{e}} {\SecondPosition{n-eus}}   lennet  Alain   an dra-se].\\
all    the people \textsc{fin.3pl} 3\textsc{{}-aux.pres} regret C  \textsc{fin} \textsc{3s{}-aux.pres}  read\textsc{.ptcp} Alain the thing-here    \\
\glt ‘Everyone regrets that Alain has read this.’
}
\ex[]{ \label{ex:jouitteau:34b}
\gll [\FirstPosition{\BoldNull}    {\SecondPosition{E}}  {\SecondPosition{n-eus}} lennet  Alain an dra-se],  tout  an dud        {}     n-eus keuz.  \\
C \textsc{fin} 3\textsc{s{}-aux.pres}   read\textsc{.ptcp}   Alain the thing-here all the people (\textsc{fin}) \textsc{3s-aux.pres}   regret \\
\glt ‘Everyone regrets that Alain has read this.’
}
\ex[]{  \label{ex:jouitteau:34c}
\gll Tout an dud      o           d-eus         keuz  [∅  \FirstPosition{lennet}       {\SecondPosition{e}} {\SecondPosition{n-eus}}          \sout{lennet}   Alain an dra-se]. \\
     all    the people \textsc{fin.3pl}  \textsc{3-aux.pres} regret  C  read\textsc{.ptcp} \textsc{fin} \textsc{3s-aux.pres} \sout{read\textsc{.ptcp}} Alain the thing-here  \\
     \glt ‘Everyone regrets that Alain has read this.’
     }
    \ex[*]{ \label{ex:jouitteau:34d}
    \gll [\FirstPosition{Lennet}     {\SecondPosition{e}}    {\SecondPosition{n-eus}}     \sout{lennet}    Alain  an dra-se],  tout  an   dud    o    d-eus  keuz      ∅. \\
    read\textsc{.ptcp} \textsc{fin} \textsc{3s-aux.pres} \sout{read\textsc{.ptcp}} Alain the thing-here    all   the people \textsc{fin.3pl} 3\textsc{-aux.pres} regret C \\
    \glt ‘Everyone regrets that Alain has read this.’ \\
    }
    \z
    \z

\subsection{Binding inside a T2 complement clauses}

In the example (\ref{ex:jouitteau:35}), the quantifier ‘nobody’ is licensed by clausal negation. This quantifier binds the subject of the subordinate clause (\ref{ex:jouitteau:35a}). It does so equally when the past participle is positioned after the inflected auxiliary in a C-Fin-VSO structure, or before it, as is the case with \isi{stylistic fronting} (\ref{ex:jouitteau:35b}). 

\newpage
\ea \label{ex:jouitteau:35} \langinfo{Leon (Plougerne)}{}{MLB 05/2018}\\
\ea \label{ex:jouitteau:35a}
\gll Den     ne   lavar james  [\FirstPosition{\BoldNull}   {\SecondPosition{e}}    {\SecondPosition{n-eus}}  laeret   an  asiedoù kaer]. \\
nobody \textsc{neg} say\textsc{.pres.3s}   never C \textsc{fin} \textsc{3s-aux.pres}  steal\textsc{.ptcp}  the  plate\textsc{.pl}  beautiful \\
\glt ‘Nobody ever tells he stole the beautiful plates.’ 
\ex \label{ex:jouitteau:35b}
\gll Den     ne   lavar james  [∅    \FirstPosition{laeret} {\SecondPosition{e}}    {\SecondPosition{n-eus}}   \sout{laeret}         an  asiedoù kaer]. \\
nobody \textsc{neg} say\textsc{.pres.3s}   never C steal\textsc{.ptcp} \textsc{fin} \textsc{3s-aux.pres}  \sout{steal\textsc{.ptcp}}  the plate\textsc{.pl} beautiful \\
\glt ‘Nobody ever tells he stole the beautiful plates.’ \\
\z
\z


In example (\ref{ex:jouitteau:36}), the subject of the upper clause, \textit{Amelia}, c-commands the subject of the lower clause, \textit{ar c’hruadur} ‘the child’. The condition C prohibits the co-referencing of these two subjects, as evidenced by the impossible interpretation marked with an asterisk. Again, the element that underwent inversion through \isi{stylistic fronting} is not the participle, but rather the subject. Consequently, the evidence is somewhat less robust, as it is challenging to identify contexts unequivocally precluding any form of focus fronting for this subject. Nevertheless, in each sentence, no explicit focus, such as contrastive focus, was discerned. Additionally, no distinct prosody was observed. For this neutral subject, its canonical position is immediately adjacent to the right of the inflected element, thereby rendering it a likely candidate for inversion. 

\ea \label{ex:jouitteau:36} \langinfo{Leon (Plougerne)}{}{MLB 05/2018}\\
\gll Amelia  e     n-eus rentet kount [\FirstPosition{ar c’huadur}  {\SecondPosition{e-n}}    {\SecondPosition{n-eus}} kemeret     an  alc’hwezh fall]. \\
Amelia \textsc{fin} \textsc{3s-aux.pres} give\textsc{.ptcp} report {the child} \textsc{fin}{}-\textsc{3s} \textsc{3s-aux.pres}  take\textsc{.ptcp} the  key wrong    \\
\glt ‘Amelia realized that the child had taken the wrong key.’ \\ 
\glt *‘Amelia (=the child) realized that she had taken the wrong key.’ \\
\z 


\subsection{Non-deictic readings in temporal nominals}

The final piece of evidence comes from the analysis of temporal nominal phrases. Phra\-ses such as 'the current president' are typically constrained to deictic interpretations when used in root clauses. Consider, for instance, sentence (\ref{ex:jouitteau:37a}); the only plausible interpretation is that the clause's subject expresses dislike towards someone who would subsequently become president, within a timeframe that encompasses the moment of utterance. Conversely, in clausal complement structures, temporal nominal phrases are not limited to deictic interpretations. Sentence (\ref{ex:jouitteau:37b}), a corpus example from \citet{mj:Norland2020}, illustrates this point. Here, the interpretation is that the subject of the embedded clause has a dislike for the president who was in office at the time of this sentiment. 

\protectedex{
\ea \label{ex:jouitteau:37} 
\ea \label{ex:jouitteau:37a}  He didn’t like the current president.
\ex \label{ex:jouitteau:37b}  He wanted to make a revolution for his country because he didn’t like the current president.
\z
\z 
}

In Breton, the temporal nominal phrase \textit{va gwaz em beche} in sentence (\ref{ex:jouitteau:38}), which translates to ‘my future husband’, is not necessarily constrained to future time relative to the moment of utterance. This exemplifies that the second clause does not function as an independent root clause; instead, it is genuinely embedded within the syntactical structure, and interpreted as such. 


\ea \label{ex:jouitteau:38} \langinfo{Leon (Plougerne)}{}{MLB 05/2018}\\
\gll Pa vedoun bihan, me soñje ∅ \FirstPosition{va} \FirstPosition{gwaz} \FirstPosition{e-m} \FirstPosition{b-eche} {\SecondPosition{e}} {\SecondPosition{veche}} brasoc’h  ‘it     ma  ’z   eo.\\
when  \textsc{aux.past.1s} small I     think\textsc{.past.3s} C my husband \textsc{fin-1s} \textsc{1.aux.cond.pres} \textsc{fin} \textsc{3.aux}.\textsc{cond.pres} bigger than that \textsc{fin} \textsc{aux.pres.3s}\\
\glt ‘When I was a child, I thought my future husband would be bigger than he is.’
\z 
\is{embedding|)}

\section{Linearization of high functional heads}\label{sec:jouitteau:5}

The generalization presented in (\ref{ex:jouitteau:39}) captures the optional nature of T2 within embedded domains, and accounts for the observed compatibility of this optionality with last resort operations. Optionality is a result of the linearization of an initial functional head. When this initial head is not included in the linearization domain, a last resort operation becomes obligatory.  

\ea \label{ex:jouitteau:39} Linearization can include or exclude an initial functional head iff it heads ForceP or a higher   functional projection. 
\z 

I leave open whether (\ref{ex:jouitteau:39}) can be derived by language-dependent prosodic rules. The dialect variation in the sporadic formation of T2 last resort embedded order arises from the dialectal parameterization of (\ref{ex:jouitteau:39}). All Breton speakers show a preference for incorporating the complementizer within the spell-out domain of the embedded clause. In the Leon dialect, speakers like MLB and AM also accept configurations where this domain does not include the complementizer, thereby initiating last resort strategies. Conversely, Huguette Gaudart, representing the East Kerne dialect, consistently includes the complementizer in her linearization process, which means that for her, C-Fin-VSO orderings have no T2 alternative. 

In the remainder of this section, I check (\ref{ex:jouitteau:39}) against functional heads other than ForceP. The sentence in (\ref{ex:jouitteau:40}) represents a polar question, commencing with an initial Q head that is followed by a complementizer, morphologically indistinguishable from a coordination marker. This complementizer may either be included into the linearization domain as in (\ref{ex:jouitteau:40a}), or omitted during linearization, a process which precipitates \isi{stylistic fronting} as observed in (\ref{ex:jouitteau:40b}). 

\ea \label{ex:jouitteau:40} \langinfo{Leon (Lesneven/Kerlouan)}{}{AM 05/2016}\\
\ea \label{ex:jouitteau:40a}
\gll Daouste \FirstPosition{hag}  {\SecondPosition{e-n}} {\SecondPosition{n-eus}} diskouezet Anna laouen he loa, pe inventet e-m eus?  \\
Q if  \textsc{fin-3sm} 3\textsc{{}-aux.pres}    show\textsc{.ptcp} Anna happy  her spoon or invent\textsc{.ptcp}  \textsc{fin-1s} \textsc{aux.pres}  \\
\glt ‘Did Anna proudly hold up the spoon or did I make it up?’
\ex \label{ex:jouitteau:40b}
\gll  Daouste hag \FirstPosition{diskouezet} {\SecondPosition{e-n}} {\SecondPosition{n-eus}} \sout{diskouezet} Anna laouen he loa, pe inventet e-m eus? \\
Q if show\textsc{.ptcp} Fin-\textsc{3sm} 3\textsc{{}-aux.pres} \sout{show\textsc{.ptcp}} Anna happy her spoon or invent\textsc{.ptcp} \textsc{fin}-\textsc{1s} aux\textsc{.pres} \\
\glt ‘Did Anna proudly hold up the spoon or did I make it up?’\\
\z
\z 

\noindent This effect can also be demonstrated using an actual coordination marker. Referring back to the structure of the left periphery as illustrated in (\ref{ex:jouitteau:40}), this marker is external to the clause, positioned above the C line. Demonstrating this effect with a coordination marker in a declarative clause proves somewhat more challenging. The issue does not lie in identifying the surface alternations schematized in (\ref{ex:jouitteau:41}). These alternations are fully productive. The coordination marker \textit{ha}, which is pronounced \textit{hag} before a vowel, can precede a regular T2 matrix sentence. This leads to \&-T2 orders, as depicted in (\ref{ex:jouitteau:41a}), prior to a last resort operation for T2. An alternative, \&-VSO orders, is also invariably feasible (\ref{ex:jouitteau:41b}), without triggering any last resort operation. In these instances, it is the coordination marker that occupies the initial position. 

\newpage
\begin{multicols}{2} 
\ea \label{ex:jouitteau:41} 
\ea \label{ex:jouitteau:41a} \& T2\\
\begin{forest}
for tree={s sep=10mm, inner sep=0, l=0, after packing node={s+=0.1pt}}
[,s sep=10mm,nice empty nodes [\& ...][ [Fin-V][[S][...]]]]
\end{forest} \\ $\fbox{\&}$-\textcolor{blue}{X}-\textbf{\textit{Fin-V}} S O ...
\ex \label{ex:jouitteau:41b} \&-VSO \\
\begin{forest}
for tree={s sep=10mm, inner sep=0, l=0, after packing node={s+=0.1pt}}
[,s sep=10mm,nice empty nodes [\& ...][ [Fin-V][[S][...]]]]
\end{forest} \\ $\fbox{\textcolor{blue}{\&}}$-\textbf{\textit{Fin-V}} S O ...
\z
\z
\end{multicols}


The challenging aspect of our analysis involves demonstrating that no phonologically null element intervenes between the coordination marker and the Fin head in example (\ref{ex:jouitteau:41b}), and that indeed the initial element saturating the T2 requirement is the coordination marker and not a topic-drop or a null adverb. 

Consider the sentence presented in (\ref{ex:jouitteau:42}). This sentence comes from a corpus.\footnote{Specifically the transcription of an oral interview with a native Leon speaker from Plougerne, conducted in the 1980s.} The sentence preceding (\ref{ex:jouitteau:42}) in the context is unrelated, stating, “That was quite expensive at the time!”. Hence, all information in (\ref{ex:jouitteau:42}) is contextually novel. This sentence likely exhibits a flat information structure with all-focus. The coordination marker introduces a temporal, consecutive interpretation. This interpretation complicates the task of conclusively ruling out the presence of a phonologically null adverb, such as ‘then’, which may have been omitted due to semantic redundancy with the coordination marker.\footnote{See \citet{mj:Willis1997} for the analysis of similar \&-Fin-VSO orders in Middle Welsh, at the T2 stage of Welsh diachronic development. He discusses the possibility of an empty operator.}


\ea \label{ex:jouitteau:42} \langinfo{Leon (Plougerne)}{}{\cite[16]{mj:Elégoët1982}}\\
\gll \FirstPosition{Hag} {\SecondPosition{e}}    {\SecondPosition{koumañsemp}} ar  sezon   da     viz     C'hwevrer betek  fin  miz    Gwengolo. \\ 
and   \textsc{fin} started\textsc{.past.1pl} the season at month February   until  end month September \\ 
\glt `And (then?) we used to start the work season in February until the end of September.' 
\z 

However, certain instances exist where the coordination marker predominantly conveys an additive meaning~– devoid of any temporal consecutive implications. In the preceding context for (\ref{ex:jouitteau:43}), the speaker stated “We had a long dock that we constructed ourselves, extending from the sea at its lowest point to the top. Imagine large rocks! Occasionally, they were as sizable as the table here…”.

\ea \label{ex:jouitteau:43} \langinfo{Leon (Plougerne)}{}{\cite[39]{mj:Elégoët1982}}\\
    
    \gll \FirstPosition{Hag} {\SecondPosition{e}}    {\SecondPosition{vije}}                  m’en dare     pet,           gant  limieri  o   klask    ober   ar c’hal. \\
     and  \textsc{fin} \textsc{aux.cond.3s}  1s-3s know.not how.many with lever\textsc{.pl}  at  search  do\textsc{.inf}  the dock    \\
    \glt `And there was I don’t know how many of them, with levers building the dock.’ 
    \z 


I summarize in (\ref{ex:jouitteau:44}) the linearization patterns we have examined, for both matrix and embedded domains. The lowermost elements on the right side of the structure~-- focus, topic, or the functional negation head~-- mandatorily fulfill the T2 requirement, precluding any subsequent operations of a last resort nature. On the other hand, the uppermost elements on the left side, specifically the coordination marker, the Q head of polar questions and the C head of ForceP, demonstrate variability in their linearization with the verb. They obligatorily linearize with Fin in the Kerne dialect, and they preferably do so in Leon, with some sporadic exceptions that have no impact on semantics and show non-syntactic reflexes. 


\ea \label{ex:jouitteau:44} \& Q [hanging topics, scene-setting adverbs] C focus topic neg  \textsc{fin}-T …
\z 

\noindent Hanging topics and scene-setting adverbs, which are positioned between these functional high heads in the cartographic representation, never satisfy the T2 requirement. This characteristic correlates with their unique prosodic isolation, further supporting the argument that linearization plays a crucial role in this linguistic phenomenon. 

Under the generalization that a coordination marker can be the first element in a T2 matrix sentence, Breton root-embedded asymmetries in word order is only epiphenomenal. Embedded domains are selected by complementizers, which are normally linearized in the initial position of the domain. Consequently, they are usually C-Fin-VSO. The same word order is observed in matrix sentences preceded by a coordination marker, \& -Fin-VSO, adding to the list of matrix linear T2 orders illustrated from (\ref{ex:jouitteau:5}) to (\ref{ex:jouitteau:8}). 

\section{Conclusion}\label{sec:jouitteau:6}

Building on previous research, I have proposed that the T2 requirement in Breton is a post-syntactic effect, an obligatory morphological exponence filtering word order at the scale of the sentence. I have rejected the hypothesis of \citeauthor{mj:Wurmbrand2012} (\citeyear{mj:Wurmbrand2012}; \citeyear{mj:Wurmbrand2014}) for Breton, and demonstrated that embedded T2 domains are fully integrated into the syntactic structure. The embedded T2 clauses under investigation are not syntactically root clauses. They are fully integrated at the syntactic level (cf. movement, binding), and are calculated as such by semantics (cf. readings of temporal nominal phrases).

The key element is that in the syntax of Breton, these embedded domains are not even T2. Like in any Celtic language, they are VSO. We saw that when an embedded clause is moved in syntax inside the domain of its matrix sentence, it is frozen to C-Fin-VSO order prior to this movement. Only in very special circumstances can the high complementizer be excluded from the linearization domain, calling for last resort strategies to ensure morphological exponence at the scale of the clause. The T2 filter is contained in another module operating after syntax, on the output of linearization. This is consistent with the diachronic evidence we have for T2 languages in Europe. The T2 property has a diachronic propensity for contagion by contact. The southern branch of Celtic languages has developed for centuries in contact with Old and Middle French T2. This contagion is efficient because it targets a morphological component (and probably prosody). Syntax is less prone to contact changes. After centuries of heavy Romance influence, Modern Breton for example still has its agreement system untouched, is still \textit{pro}-drop, and has its subjects licensed post-verbally.

Linearization is subject, maybe through prosody, to dialectal parameterization. Breton dialects differ empirically in their tolerance for a leftmost functional head to be linearized separately. Only the Leon speakers allow it, even though they tend to dislike it. When they do so, they consequently deploy the last resort strategies available to them for Fin not to be linearized first. The impact of the interplay of linearization and prosody on Breton word order calls for further research. We need a proper modeling of the prosody of the sentence for each type of information structure that we detect in syntax and semantics, and especially the prosody of word orders that we know to be last resort solution for T2.


\appendixsection{High adjuncts and T2} 

Across T2 languages, including the Germanic domain, it has been noted that adjuncts vary with respect to the word orders they tolerate. Central adverbial clauses resist T2, whereas peripheral adverbials do not (\citealt{mj:Haegeman2012} and references therein). 

\appendixsubsection{Adjunct temporal clauses}

In temporal clauses, we observe low adjuncts to be more restricted than high adjuncts. With the Leon speakers, diverse T2 orders are obtained with \textit{pa} ‘when’, as we saw in (\ref{ex:jouitteau:18}), repeated here as (\ref{ex:jouitteau:45}) for convenience. \isi{stylistic fronting} in (\ref{ex:jouitteau:45b}) is grammatical inside a temporal clause located in the recursive high left periphery, at the initial of the matrix clause, inside a prosodically isolated domain.  In our analysis, \isi{stylistic fronting} is a sign that the clause has been generated in this high position, as syntactic clausal movement would have frozen the structure to C-Fin-VSO orders. 

\ea \label{ex:jouitteau:45} \langinfo{Leon (Plougerne)}{}{MLB (05/2018}\\
\ea \label{ex:jouitteau:45a}
\gll \FirstPosition{Pa}       {\SecondPosition{m-eus}}    kleet         ar janson-se     evit  ar  wech kentañ, eved-on         dougeres. \\
when  1\textsc{s{}-aux} hear\textsc{.ptcp}  the song-there  for   the time  first      C.\textsc{aux.past}{}-\textsc{1s} pregnant \\
\glt `When I first heard this song, I was pregnant.’
\ex \label{ex:jouitteau:45b}
\gll Pa \FirstPosition{kleet} {\SecondPosition{m-eus}} \sout{kleet} ar janson-se    evit ar wech kentañ, eved-on    dougeres.  \\
when hear\textsc{.ptcp} 1\textsc{s-aux} \sout{hear\textsc{.ptcp}} the song-there  for the time  first     C.\textsc{aux.past}{}-\textsc{1s} pregnant \\ 
\glt `When I first heard this song, I was pregnant.’ \\
\z 
\z 


In (\ref{ex:jouitteau:46}) and (\ref{ex:jouitteau:47}), low adverbials starting with the same complementizer \textit{pa} are incompatible with \isi{stylistic fronting}.

\ea \label{ex:jouitteau:46}
\ea[]{ \label{ex:jouitteau:46a}
\gll Plijet e-n   n-eus  din \FirstPosition{pa}   {\SecondPosition{m-eus}} klevet va mab e-n  d-oa plantet  va  patatez.  \\
please\textsc{.ptcp} \textsc{fin}{}-\textsc{3s} 3\textsc{{}-aux.pres.}3s to.\textsc{1s} when \textsc{1s{}-aux.pres} hear\textsc{.ptcp} my son \textsc{fin}{}-\textsc{3s} 3\textsc{{}-aux.past} plant\textsc{.ptcp} my potatoes     \\
\glt `I liked when I heard that my son had planted my potatoes.’
}
\ex[*]{ \label{ex:jouitteau:46b}
\gll Plijet          e-n   n-eus                     din  pa \FirstPosition{klevet} {\SecondPosition{m-eus}} \sout{klevet} va mab e-n  d-oa plantet  va  patatez. \\
please\textsc{.ptcp} \textsc{fin}{}-3\textsc{s} 3\textsc{{}-aux.pres.3s} to.\textsc{1s} when hear\textsc{.ptcp} 1\textsc{s{}-aux.pres} \sout{hear\textsc{.ptcp}} my son \textsc{fin}{}-3\textsc{s} 3\textsc{{}-aux.past} plant\textsc{.ptcp} my potatoes \\
\glt `I liked when I heard that my son had planted my potatoes.’ 
}
\z 
\z

%%There is an unglossed morpheme in both of these sentences. I've written to Melanie for clarification.

\protectedex{%
\ea \label{ex:jouitteau:47} \langinfo{Leon (Plougerne)}{}{MLB (05/2018}\\
\ea[]{ \label{ex:jouitteau:47a}
\gll Amelia e n-eus c’hoarzet \FirstPosition{pa}  {\SecondPosition{n-eus}}     komprenet           e n-oa   {en em} dromplet.  \\
Alemia \textsc{fin} \textsc{3s-aux.pres} laugh\textsc{.ptcp} when \textsc{3s-aux.pres} understand\textsc{.ptcp} \textsc{fin} \textsc{3s-aux.past}  reflex mistake\textsc{.ptcp}   \\
\glt `Amelia laughed when she understood she had been mistaken.’
}
\ex[*]{ \label{ex:jouitteau:47b}
\gll Amelia e n-eus c’hoarzet pa \FirstPosition{komprenet} \textbf{n-eus} \sout{komprenet} e n-oa {en em} dromplet.  \\
Amelia \textsc{fin} \textsc{3s-aux.pres} laugh\textsc{.ptcp} when understand\textsc{.ptcp}  \textsc{3s-aux.pres}   \sout{understand\textsc{.ptcp}} \textsc{fin} \textsc{3s-aux.past}  reflex mistake\textsc{.ptcp} \\
\glt `Amelia laughed when she understood she had been mistaken.’ \\
}
\z
\z 
}
       

\appendixsubsection{Adjunct causal clauses}

In causal clauses also, we observe low adjuncts to be more restricted than high adjuncts. The empirical observation is that negation in a matrix clause cannot scope over a T2 cause clause (see \citealt{mj:deHaan2001, mj:Heycock2006, mj:Haegeman2012} and references therein; \citealt{mj:Holmberg2015}). This also holds true in Breton (\ref{ex:jouitteau:48}). The given context for (\ref{ex:jouitteau:48a}), ‘This was a long trip’, aims at suggesting the interpretation [¬ \textit{he came}] \textit{because he is lazy}. Structurally, it forces negation to restrict its scope to the matrix domain. When negation does not scope over the causal clause, but C-Fin-VSO orders (\ref{ex:jouitteau:48a}) and C-T2 are tolerated (\ref{ex:jouitteau:48b}).

\ea \label{ex:jouitteau:48}
\ea \label{ex:jouitteau:48a}
\gll N’   eo          ket  deuet  \FirstPosition{peogwir} {\SecondPosition{eo}} lezireg, ha kavet        e-n d-oa  hir an hent.   \\
\textsc{neg} \textsc{aux.3s} \textsc{neg} come\textsc{.ptcp} because \textsc{aux.3s}  lazy     and find\textsc{.ptcp} \textsc{fin}-\textsc{3s} 3\textsc{-aux.past} long the road \\
\glt `He didn’t come because he is lazy and he found it was too far away.’
\ex \label{ex:jouitteau:48b}
\gll N’   eo ket deuet  peogwir \FirstPosition{lezireg} {\SecondPosition{eo}},  ha   kavet e-n d-oa hir an hent. \\ 
\textsc{neg} \textsc{aux.3s} \textsc{neg} come\textsc{.ptcp} because lazy     aux\textsc{.3s} and find\textsc{.ptcp} \textsc{fin}-3\textsc{s} 3\textsc{-aux.past} long the road \\
\glt `He didn’t come because he is lazy, and he found it was too far away.’ 
\z
\z 

In (\ref{ex:jouitteau:49}), the given context is ‘Don’t be nasty! He didn’t come with me only because I have a car, and he didn’t want to walk…’. This is meant to force integration of the cause clause inside the scope of negation (¬ [\textit{he came because he is lazy}] ). The embedded central adverbial clause is C-Fin-VSO in (\ref{ex:jouitteau:49a}) and excludes a T2 alternative (\ref{ex:jouitteau:49b}).

\ea \label{ex:jouitteau:49}
\ea[]{ \label{ex:jouitteau:49a}
\gll N’   eo          ket  deuet        \FirstPosition{peogwir} {\SecondPosition{eo}}        lezireg, met evit kaozeal      samples. \\
\textsc{neg} \textsc{aux.3s} \textsc{neg} come\textsc{.ptcp} because \textsc{aux.3s} lazy      but for   discuss\textsc{.inf} together     \\
\glt `He didn’t come because he is lazy but for us to talk.’
}
\ex[*]{ \label{ex:jouitteau:49b}
\gll N’   eo         ket   deuet         peogwir \FirstPosition{lezireg} \SecondPosition{eo},        met evit kaozeal      samples. \\
\textsc{neg} \textsc{aux.3s} \textsc{neg} come\textsc{.ptcp} because  lazy     \textsc{aux.3s} but for   discuss\textsc{.inf} together \\
\glt `He didn’t come because he is lazy, but for us to talk.’  
}
\z\z 

The Breton facts align with the generalizations developed for Germanic languages. Peripheral adjuncts are attached high, outside of the scope of the negation in the matrix sentence. The high clause can be T2. Central adverbials however resist T2. The generalization is that the adjuncts which linearize in prosodically autonomous domains have the freedom to linearize with or without the initial complementizer. For some other reason, central adverbial domains resist this optionality. My proposal provides no new insight concerning this paradigm but suggests we should enrich the typological debate with prosodic information.\il{Breton (Modern)|)}\is{tense second|)}

\printbibliography[heading=subbibliography,notkeyword=this]

\end{document} 
