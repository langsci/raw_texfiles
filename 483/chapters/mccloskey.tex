\documentclass[output=paper,colorlinks,citecolor=brown]{langscibook}
\ChapterDOI{10.5281/zenodo.15654857}
\title{Polarity sensitivity and fragments in Irish}
\author{James McCloskey\affiliation{University of California, Santa Cruz}}

%\bibliography{mccloskey.bib}
\IfFileExists{../localcommands.tex}{
   \usepackage{langsci-optional}
\usepackage{langsci-gb4e}
\usepackage{langsci-lgr}

\usepackage{listings}
\lstset{basicstyle=\ttfamily,tabsize=2,breaklines=true}

%added by author
% \usepackage{tipa}
\usepackage{multirow}
\graphicspath{{figures/}}
\usepackage{langsci-branding}

   
\newcommand{\sent}{\enumsentence}
\newcommand{\sents}{\eenumsentence}
\let\citeasnoun\citet

\renewcommand{\lsCoverTitleFont}[1]{\sffamily\addfontfeatures{Scale=MatchUppercase}\fontsize{44pt}{16mm}\selectfont #1}
  
   %% hyphenation points for line breaks
%% Normally, automatic hyphenation in LaTeX is very good
%% If a word is mis-hyphenated, add it to this file
%%
%% add information to TeX file before \begin{document} with:
%% %% hyphenation points for line breaks
%% Normally, automatic hyphenation in LaTeX is very good
%% If a word is mis-hyphenated, add it to this file
%%
%% add information to TeX file before \begin{document} with:
%% %% hyphenation points for line breaks
%% Normally, automatic hyphenation in LaTeX is very good
%% If a word is mis-hyphenated, add it to this file
%%
%% add information to TeX file before \begin{document} with:
%% \include{localhyphenation}
\hyphenation{
affri-ca-te
affri-ca-tes
an-no-tated
com-ple-ments
com-po-si-tio-na-li-ty
non-com-po-si-tio-na-li-ty
Gon-zá-lez
out-side
Ri-chárd
se-man-tics
STREU-SLE
Tie-de-mann
}
\hyphenation{
affri-ca-te
affri-ca-tes
an-no-tated
com-ple-ments
com-po-si-tio-na-li-ty
non-com-po-si-tio-na-li-ty
Gon-zá-lez
out-side
Ri-chárd
se-man-tics
STREU-SLE
Tie-de-mann
}
\hyphenation{
affri-ca-te
affri-ca-tes
an-no-tated
com-ple-ments
com-po-si-tio-na-li-ty
non-com-po-si-tio-na-li-ty
Gon-zá-lez
out-side
Ri-chárd
se-man-tics
STREU-SLE
Tie-de-mann
}
   \boolfalse{bookcompile}
   \togglepaper[1]%%chapternumber
}{}

\AffiliationsWithoutIndexing


\abstract{In Irish, negative polarity items\is{negative polarity item} may serve, isolated, as fragment answers, something which seems initially to be both theoretically and typologically anomalous. This chapter explores some theoretical consequences of this observation. It begins by arguing that the apparent anomaly is real; that is, that the relevant fragments are in fact negative polarity items\is{negative polarity item}. It then argues that the anomaly can be resolved. The possibility of fragment negative polarity items\is{negative polarity item} is parasitic on an independent property of the language, namely that it possesses a movement operation (so-called Narrative Fronting) by which a narrow scope\is{scope} indefinite may be raised to a position to the left of the verb in a finite negative clause. It is argued that this option feeds a sluicing-like process of clausal ellipsis, and that this operation may strand licensed negative polarity items\is{negative polarity item}. In the context of `move and delete' analyses of fragments, as developed especially by \citealt{merchant:04}, the apparently anomalous possibility is routine. A by-product of the discussion is a more complete mapping of the landscape of polarity sensitive items in Irish than has so far been provided.}

% % move the following commands to the "local..." files of the master project when integrating this chapter
% %
% %%%%%%%%%%%%%%%%%%%%%%%%%%%%%%%%%%%%%%%%%%%%%%%%%%%%%%%%%%%%%%%%%%%%%%%%%%%%%%%%%%%%%%%%%%%%%%%%%%
% %
% %       JIM'S LOCAL STUFF 
% %
% %       Packages
% %
% %
% \usepackage{paralist}                    %  Provides compactenum etc
% \usepackage{url}
% \urlstyle{same}
% %


% \alignSubExtrue                          %  linguex needs this to get alignments right
% \renewcommand{\firstrefdash}{}           %  (15a) rather than (15-a)
% %
% %       For Trees
% %
% \usepackage{pstricks}
% \usepackage{pst-xkey}
% \usepackage{pst-jtree}
% %
% %      My Local Commands
% %
% \newcommand*{\orcid}{}
% %
% %
% %     Vertical Spacers
% %
% \newcommand*{\breathe}{\par\vspace*{\baselineskip}\noindent}
% \newcommand*{\sigh}{\par\vspace*{0.5\baselineskip}\noindent}
% %
% %    Convenient Abbreviations
% %
% \newcommand*{\wh}{\WH}
% \newcommand*{\thi}{{\scshape tm1}}
% \newcommand*{\tlo}{{\scshape tm2}}

% %
% %
% %   Category Abbreviations
% %

% \newcommand*{\F}{\textsc{f}}
% \newcommand*{\littlev}{$\mathit{v}$}
% \newcommand*{\pol}{\textsc{pol}}
% \newcommand*{\PP}{\textsc{pp}}
% \newcommand*{\DP}{\textsc{dp}}
% %
% %  For Glosses:

% \newcommand*{\past}{.\textsc{\footnotesize past}}
% \newcommand*{\presaut}{.\textsc{\footnotesize pres.impers}}
% \newcommand*{\futaut}{.\textsc{\footnotesize fut.impers}}
% \newcommand*{\condaut}{.\textsc{\footnotesize cond.impers}}
% \newcommand*{\pasthabitaut}{.\textsc{\footnotesize past.habit.impers}}
% \newcommand*{\preshabit}{.\textsc{\footnotesize pres.habit}}

% \newcommand*{\preshabitrel}{.\textsc{\footnotesize pres-habit-wh}}
% \newcommand*{\habit}{\pasthabit}
% \newcommand*{\futrel}{.\textsc{\footnotesize fut.wh}}
% \newcommand*{\fin}{.\textsc{\footnotesize fin}}
% \newcommand*{\pass}{.\textsc{\footnotesize pass}}
% \newcommand*{\perf}{.\textsc{\footnotesize perf}}
% \newcommand*{\perfpass}{.\textsc{\footnotesize perf.pass}}
% %
%
% \newcommand*{\comp}{\textsc{\footnotesize c}}
% \newcommand*{\aNgloss}{\textsc{\footnotesize c.pro}}

% \newcommand*{\go}{\textsc{\footnotesize c}}

% \newcommand*{\ar}{\textsc{\footnotesize c.q-past}}
% \newcommand*{\ma}{\textsc{\footnotesize c.cond}}
% \newcommand*{\da}{\textsc{\footnotesize c.cond.irr}}
% %

% \newcommand*{\no}{\textsc{\footnotesize c.neg.fin}}
% \newcommand*{\nachq}{\textsc{\footnotesize c.neg.q}}
% \newcommand*{\nach}{\textsc{\footnotesize c.neg.fin}}
% \newcommand*{\nar}{\textsc{\footnotesize c.neg-past}}
% \newcommand*{\na}{\textsc{\footnotesize neg-impv}}
% %

% \newcommand*{\bac}{\textsc{\footnotesize cop.cond}}
% %
% \newcommand{\voc}{.\textsc{\footnotesize voc}}

% \newcommand*{\nom}{.\textsc{\footnotesize nom}}
% \newcommand*{\acc}{.\textsc{\footnotesize acc}}
% \newcommand*{\dat}{.\textsc{\footnotesize dat}}
% \newcommand*{\masc}{.\textsc{\footnotesize masc}}
% \newcommand*{\fem}{.\textsc{\footnotesize fem}}
% \newcommand*{\neut}{.\textsc{\footnotesize neut}}
% \newcommand*{\sg}{.\textsc{\footnotesize sg}}
% \newcommand*{\his}{\textsc{\footnotesize ms3}}
% \newcommand*{\her}{\textsc{\footnotesize fs3}}
% \newcommand*{\yall}{\textsc{\footnotesize p2}}
% \newcommand*{\agrmy}{-\textsc{\footnotesize s1}}
% \newcommand*{\agryour}{-\textsc{\footnotesize s2}}
% \newcommand*{\agrhis}{-\textsc{\footnotesize ms3}}
% \newcommand*{\agrher}{-\textsc{\footnotesize fs3}}
% \newcommand*{\agrour}{-\textsc{\footnotesize p1}}
% \newcommand*{\agryall}{-\textsc{\footnotesize p2}}
% \newcommand*{\agrtheir}{-\textsc{\footnotesize p3}}
% %
% %
% %                **FOR LABELLED BRACKETS**>

% \newcommand*{\lbTP}{$_{\hbox{\scriptsize TP}}$}
% %
% %	         **MISCELLANEOUS**
% %
% %\newcommand*{\define}{$\buildrel \rm def \over =$}
% %
% %  EXAMPLES CITATION
% %

% %
% %  ONE ARGUMENT OVER THE OTHER
% %
% %
% %%
% %
% %%%%%%%%%%%%%%%%%%%%%%%%%%%%%%%%%


\begin{document}

\maketitle

\section{Background and goals}\label{sec:intro}

\il{Irish (Modern)|(}



The English question-answer pair in (\ref{ex:badfrag}) is ill-formed:

\ea\label{ex:badfrag}
\textsc a:  What did the priest say?
\quad
\textsc b: *Anything.
\z

%\ex. \label{ex:badfrag}
%\A:  What did the priest say?
%{\quad}{\scshape b}: *Anything.

\noindent Its apparent Irish\il{{Irish (Modern)}} counterpart in (\ref{ex:goodfrag}), however, is routine and well-formed:

\ea\label{ex:goodfrag}
\gll {\textsc a:} Cad a dúirt an sagart? {}{\textsc b:} Aon rud. \\
{}  what {\go} {say\past} the priest {}  any thing \\
\glt {\textsc a:} `What did the priest say?' {\textsc b}: `Anything'.\source{chd}{57}
\z

%\exg. \A: Cad a dúirt an sagart?
%{\qquad} {\scshape b}: Aon rud.\\
%{}  what {\go} {say\past} the priest {} {} any thing\\ \label{ex:goodfrag}
%\A: `What did the priest say?' {\ \ \ \qquad}%{\scshape b}: `Anything'.\source{chd}{57}

\noindent To express what Irish expresses by way of (\ref{ex:goodfrag}), English must use (\ref{ex:goodfrag.eng}), exploiting the presence in its lexicon of the \isi{inherently negative expression} {\itshape  nothing}, used here as a fragment answer\is{fragment}:

\ea\label{ex:goodfrag.eng}
\textsc{a}: What did the priest say?
\quad
\textsc{b}: Nothing.
\z

%\ex. \A: What did the priest say?
%{\qquad}{\scshape b}: Nothing.

\noindent This chapter is concerned with the theoretical issues raised by the contrasts in (\ref{ex:goodfrag}--\ref{ex:goodfrag.eng}).  Those issues are important in part because the Irish possibility shown in (\ref{ex:goodfrag}) seems to represent a theoretical and typological anomaly. Negative polarity items\is{negative polarity item} are not supposed to be able to function, in isolation, as fragment answers\is{fragment}.\footnote{The star on English (\ref{ex:badfrag}) is well-deserved, but there are conditions under which NPIs may appear as fragment answers\is{fragment} in English. This issue is taken up in \sectref{sec:english-redux} below.}

The first goal of this chapter is to establish that this interesting and unexpected possibility does in fact exist; the answer in (\ref{ex:goodfrag}) really does consist solely (in its overt form) of a negative polarity item. Its second goal is to develop a way of understanding that possibility which resolves the apparent anomaly. Its third goal is to consider some larger implications which flow from that account for the general theory of subsentential fragments\is{fragment} and for the theory of ellipsis.

A by-product of the discussion will be a more complete map of the landscape of polarity sensitive items in Irish than has so far been available.  Most discussions of polarity sensitivity currently available deal with languages that have among their lexical resources expressions which are inherently negative\is{inherently negative expression}, such as English {\itshape nobody}, {\itshape nothing} or {\itshape never}.  I will argue here, though, that \citet{jm:acquaviva:96} was right to claim that Irish has no such expressions. We are thus presented with an opportunity to explore what patterns emerge in their absence.

Two sources of data form the basis for that exploration here: work with six native speaker consultants over a period of several years, along with a collection of 1800 naturally occurring examples of polarity sensitive items of various kinds and in various contexts of use.\footnote{Examples from this corpus are indicated by a tag consisting of an abbreviation of the title (in the case of published texts) followed by a number indicating the page from which the example was extracted. The abbreviations used in these tags are explained in Appendix Two.} 

Some terminological preliminaries to begin with. I will use the term `polarity sensitive expression' (PSE) for such elements as {\itshape aon rud} in \ref{ex:goodfrag} or \textit{never} in English. These are expressions whose interpretation and well-formedness depend on the polarity of the larger environment in which they find themselves.  The set of PSEs is the superset which includes as subsets both negative polarity items (NPIs\is{negative polarity item}) and `inherently negative expressions' (INEs)\is{inherently negative expression}, which is the term I will use here for the elements called variously `{\itshape n}-words' or `negative indefinites' or `negative quantifiers' in other discussions (English {\itshape nothing} or French {\itshape personne}). In some languages and varieties, such expressions may stand alone to express sentential negation\is{negation}; in others, they must or may enter into negative concord dependencies to serve that function.

\section{The landscape of polarity sensitivity in Irish}
\label{sec:npi}

In presenting the contrast between Irish (\ref{ex:goodfrag}) and English (\ref{ex:badfrag}), I glossed the Irish{ \itshape aon rud} as English `any thing'. If I had instead chosen to use `no thing' as the gloss, I could have created the impression that there was no anomaly to be concerned about, since many languages allow inherently negative expressions\is{inherently negative expression} to function as fragment answers\is{fragment}. The glossing decision, then, mirrors an important theoretical issue in this case. Consider the pair of Irish examples in (\ref{ex:pair})\footnote{The glossing conventions used are, for the most part, familiar and transparent but a few notes are in order. Because the inflected prepositions are not a matter of concern here, they are glossed for readability using English pronouns. So Irish \textit{liom} is glossed as `with.me'. The `direct relative' complementizer is glossed as {\scshape c.wh} and the marker of polar interrogatives\is{polar question} as \textsc{c.q}. All other complementizers, including the conditional, are simply glossed as {\scshape c}. Past and conditional forms of the complementizers are segmentalized, so that, for instance, \textit{gur} is spelled \textit{gu-r} and glossed as {\scshape c-past}.  `Autonomous' forms of verbs are glossed as \textsc{impers} and imperative forms as \textsc{impv}. The particle \textit{a}, which precedes verbal nouns, is glossed as \textsc{vce} (suggesting `voice') and verbal nouns themselves are glossed with \textsc{.vn}. The progressive particle \textit{ag} is glossed as \textsc{prog}.}:

\ea\label{ex:pair}
\ea[]{
\gll  Ní-or iarr duine {ar bith} orm é. \\
      {\nior} {ask\past} {person} {any} on.me it \\
      \label{ex:goodbody}
\glt  `Nobody asked me for it.'}
\ex[*]{
\gll D' iarr duine {ar bith} orm é. \\
     {\textsc{past}} {ask\past} {person} {any} on.me it \\}
\z
\z

% \ex. \ag. Ní-or iarr duine {ar bith} orm é.\\
%           {\nior} {ask\past} {person} {any} on.me it\\\label{ex:goodbody}
%           `Nobody asked me for it.'
%      \bg. *D' iarr duine {ar bith} orm é.\\
%           {\did} {ask\past} {person} {any} on.me it\\

Since (\ref{ex:pair}b) is impossible, we know that {\itshape duine ar bith} cannot express negation\is{negation} (and is, in fact, ill-formed) when not within an appropriate licensing environment. In (\ref{ex:pair}a) it appears in just such an environment -- that determined by the clause-initial sentential negation\is{negation}. It is therefore well-formed and the meaning it expresses is that of the English translation.  {\itshape Duine ar bith} in (\ref{ex:pair}a) must then be either a negative polarity item or an INE\is{inherently negative expression} in a concord dependency with the clause-initial negation\is{negation} (e.g., \cite{labov:72}; \cite{laka:90}; \cite{haegeman-zanuttini:91}; \cite{ladusaw:92}; \cite{zanuttini:97}; \cite{penka-zeijlstra:10}; Zeijlstra (\citeyear{hedde:04}; \citeyear{hedde:08}; \citeyear{hedde:16}); \cite{giannakidou-zeijlstra:16}; \cite{deal:22}).  But, given the possibility of (\ref{ex:goodfrag}), to maintain consistency with the received wisdom concerning fragment answers\is{fragment}, we must assume that {\itshape aon rud} (and many similar expressions we will encounter shortly) are not NPIs\is{negative polarity item} but are rather INEs. However, \citet{jm:acquaviva:96} has argued that Irish (like M\={a}ori\footnote{See \cite[p. 298--289, \S 19.5]{bauer:97}: `Negative quantifiers do not exist in Maori. Sentence negation\is{negation} or other lexical means are used.'  Mandarin, Korean and Bengali are also reported to lack such expressions (see \citealt[15]{hedde:08}).})  possesses no INEs. If he is right, it follows {\itshape a fortiori} that Irish cannot be a negative concord language. And it follows in turn that \textit{duine ar bith} in (\ref{ex:pair}) and \textit{aon rud} in the fragment answer\is{fragment} of (\ref{ex:goodfrag}) must be NPIs\is{negative polarity item} and that the apparent conflict with typological and theoretical expectation is real.

It is important, then, to determine whether the pattern in (\ref{ex:pair}) reflects a negative concord system or an NPI-centered system\is{negative polarity item}, and the first business of the present chapter will be to address that question.  Making such a distinction, though, is not as straightforward as it once seemed to be; the empirical and theoretical landscapes seem more intricate than they once did (see \cite{laka:93} and, especially, \cite{herburger:01}).  I will argue, though, that Acquaviva was correct about the absence of INEs in Irish, that all of its polarity sensitive expressions are in fact NPIs\is{negative polarity item}, and that the theoretical questions raised by (\ref{ex:goodfrag}) are therefore real.

Two convictions drive the discussion. The first is that the distribution of NPIs\is{negative polarity item} is determined by fundamentally semantic and pragmatic relations (Fauconnier (\citeyear{fauconnier:75}; \citeyear{fauconnier:79}); \cite{ladusaw:79}; \cite{heim:84}; \cite{kadmon-landman:93}; \cite{krifka:95}; \cite{zwarts:96}; van der Wouden (\citeyear{van-der-wouden:94}; \citeyear{van-der-wouden:97}); \cite{giannakidou:98}; \cite{lahiri:98}; \cite{von-fintel:99}; \cite{hoeksema:00}; Gajewski (\citeyear{gajewski:05}; \citeyear{gajewski:11}); \cite{guerzoni-sharvit:07}; \cite[Chap. 2]{homer:11}; \cite{chierchia:13}; \cite{crnic:14}; \cite{gajewski-hsieh:14}; \cite{barker:18}; \cite{crnic:18}; \cite{homer:21a}; \cite{jeong-roelofsen:23}).  The second is that INEs, though they may well be strong NPIs\is{negative polarity item} in their semantics, are licensed in negative concord dependencies by featural interactions in the syntax, subject to characteristically syntactic requirements of locality and relative prominence. This is why analyses in terms of the operation {\scshape agree} have been so influential and seemed so persuasive in recent years (see \cite{hedde:08} and recent exchanges among \cite{zeijlstra:12}, \cite{preminger-polinsky:15}, \cite{bjorkman-zeijlstra:19} and \cite{deal:22}).

\subsection{The inventory of polarity sensitive expressions}

I begin by trying to establish a more complete inventory of polarity sensitive expressions in Irish than has so far been attempted.  The element {\itshape tada}, illustrated in (\ref{ex:tada}), is one such.

\ea\label{ex:tada}
\ea
\gll Ní-or ith mé tada ar maidin. \\
    {\nior} {eat\past} {I} {anything/nothing} on morning \\
\glt `I didn't eat anything this morning.'\quad `I ate nothing this morning.'
\ex
\gll *D' ith mé tada ar maidin.\label{ex:bad-tada} \\
    {\textsc{past}} {eat\past} {I} {anything/nothing} on morning \\
\z
\z

% \ex. \ag. Ní-or ith mé tada ar maidin.\\
%           {\nior} {eat\past} {I} {anything/nothing} on morning
%           \\ `I didn't eat anything this morning.'\quad
%           `I ate nothing this morning.'
%      \bg. *D' ith mé tada ar maidin.\label{ex:bad-tada}\\
%           {\did} {eat\past} {I} {anything/nothing} on morning \\

\noindent {\itshape Tada} (and its dialectal variants {\itshape dada} and {\itshape dadaidh}) is one member of a large class of elements united by a shared distribution and a shared interpretive profile. Like {\itshape tada}, they express existential quantification in the scope\is{scope} of sentential negation\is{negation} and are ill-formed when not so licensed.  The illustrative examples below have the candidate elements in the scope\is{scope} of sentential negation\is{negation}, but readers should assume that the corresponding example without negation\is{negation} is ungrammatical. This is true for almost every case considered. Elements for which it is not (entirely) true will be discussed in \sectref{sec:free} below. The larger class of environments in which PSEs appear (which do not all involve the explicit appearance of negation\is{negation}) will be considered in \sectref{sec:beyond}.

The class of PSEs of this type includes nominals, along with some adverbs of temporal perspective\is{temporal adverbial}. The nominal group in turn includes two subtypes: a class of monomorphemic lexical items that stand alone (as in \ref{ex:tada}) and more complex nominal expressions that include one of a set of functional elements that define the containing expression as a PSE. One element of the latter type is the prepositional phrase \textit{ar bith} (diachronically or literally `in the world' or `on earth'), which we have already encountered in (\ref{ex:goodbody}), and which is exemplified further in (\ref{ex:goodbody.again}). This is a post-nominal modifier which attaches to indefinites in certain environments and converts its host into a PSE.  Note that since negation\is{negation} is high in Irish (marked on \C\ in finite clauses), elements in subject position will always be in the semantic scope\is{scope} and in the syntactic domain of negation\is{negation} when it is present (\cite{jm:acquaviva:96}; \cite{duffield:95}; \cite{mccloskey:96a}; \cite{mccloskey:01b}; \cite{mccloskey:17}).\footnote{An anonymous reviewer raises the\label{fn:amhain} interesting question of whether or not the alternative word in Irish for the numeral `one' (\textit{amháin}) may also determine a PSE. The answer is that it does not. The two elements differ in a number of ways from one another: \textit{aon} is pre-nominal, \textit{amháin} is post-nominal. \textit{Amháin} has a use as a focus particle meaning `only', while \textit{aon} does not. The question of why and how the two elements differ in their ability to determine a PSE clearly deserves a fuller investigation than is possible here and almost certainly has implications for the fundamental question of what makes a PSE. I am grateful to the reviewer for raising this important issue, even if these remarks in response must remain unsatisfying.}

\ea\label{ex:goodbody.again}
\ea
\gll Ní-or thóg bean {ar bith} de na mná seo riamh an fiabhras. \\
    {\nior} {take\past} woman any of the women {\seo} ever the fever \\
\glt `None of these women ever contracted fever.'\source{gog}{132}\label{ex:ariamh1}
\ex
\gll Níl taibhsí {ar bith} ann agus ní raibh ariamh. \\
    {is.not} ghosts any in.it and {\no} {be\past} (n)ever \\
\glt `There are no ghosts and there never were.'\source{coc}{156}\label{ex:ariamh2}
\z
\z

% \ex. \ag. Ní-or thóg bean {ar bith} de na mná seo riamh an fiabhras.\\
%           {\nior} {take\past} woman any of the women {\seo} ever the fever\\
%           `None of these women ever contracted fever.'\source{gog}{132}\label{ex:ariamh1}
%      \bg. Níl taibhsí {ar bith} ann agus ní raibh ariamh.\\
%           {is.not} ghosts any in.it and {\no} {be\past} (n)ever\\
%           `There are no ghosts and there never were.'\source{coc}{156}\label{ex:ariamh2}

The numeral \textit{aon} `one', in addition, can be used to determine polarity sensitive nominals and in this use (and only in this use) it may compose with mass nouns (see \ref{ex:aon}b-c) and, in some dialects at least, with plural count nouns (see \ref{ex:aon}d). 

\ea\label{ex:aon}
\ea
\gll Ní-or dúradh aon chuid de seo riamh {go hoscailte}. \\
    {\nior} {say\pastaut} one part of this (n)ever openly \\
\glt `No part of this was ever said openly.'\source{png}{187} \label{ex:ariamh3}
\ex
\gll Ní-or thugais aon ghrá ceart riamh dom. \\
    {\nior} {give\past.\your} one love proper (n)ever to.me \\
\glt `You never gave me any proper love.'\source{annf}{59} \label{ex:ariamh4}
\ex
\gll Ní raibh aon eolas eile ar a mhalairt. \\
     {\no} {be\past} one knowledge other on its alternative \\
\glt `Nobody knew any different.'\source{lan}{26}
\ex
\gll Ní raibh aon bháid ar an Oileán ag na chéad daoine. \\
     {\no} {be\past} one {boat\pl} on the Island at the first people \\
\glt `The first people on the Island hadn't any boats.'\source{lan}{27}
\z
\z

% \ex. \ag. Ní-or dúradh aon chuid de seo riamh {go hoscailte}\\
%          {\nior} {say\pastaut} one part of this (n)ever openly\\ \label{ex:ariamh3}
%          `No part of this was ever said openly'\source{png}{187}
%      \bg. Ní-or thugais aon ghrá ceart riamh dom\\
%           {\nior} {give\past.\your} one love proper (n)ever to.me\\\label{ex:ariamh4}
%          `You never gave me any proper love.'\source{annf}{59}
%      \cg. Ní raibh aon eolas eile ar a mhalairt\\
%           {\no} {be\past} one knowledge other on its alternative\\
%          `Nobody knew any different.'\source{lan}{26}
%      \dg. Ní raibh aon bháid ar an Oileán ag na chéad daoine\\
%          {\no} {be\past} one {boat\pl} on the Island at the first people\\
%          `The first people on the Island hadn't any boats.'\source{lan}{27}

\textit{Éinne} (in western and southern varieties) is a fusion of \textit{aon} with the noun \textit{duine} `person':

\ea
\gll Ní-or labhair éinne ariamh liom. \\          
     {\nior} {speak\past} anyone (n)ever {with.me} \\
\glt `Nobody ever spoke to me.'
\z

% \exg. Ní-or labhair éinne ariamh liom.\\          
%      {\nior} {speak\past} anyone (n)ever {with.me}\\
%     `Nobody ever spoke to me.'

Among the elements which seem to lack any internal structure are \textit{tada}/{\itshape dada}, \textit{faic}, and \textit{a dhath}, all of which correspond to English `anything' or `nothing'.\footnote{Such elements, while themselves monomorphemic, can be “strengthened” by addition of \textit{ar bith} or minimizers\is{minimizer} of the type to be considered in the subsection which follows:

\ea
\gll {Ní-or} ghá dóibh tada {ar bith} beo eile {a dhéanamh} aríst {go deo}. \\
     {\nior} need  {to.them} nothing any alive other {do.\textsc{nfin}} again ever \\
\glt `They never had to do a single solitary thing ever
     again.'\source{afap}{33}
\z
}

% \exg. {Ní-or} ghá dóibh tada {ar bith} beo eile {a dhéanamh} aríst {go deo}.\\ {\nior} need
%        {to.them} nothing any alive other {do.\textsc{nfin}} again ever\\ `They never had to do a single solitary
%        thing ever again.'\source{afap}{33}

\ea
\ea
\gll Ní dúrt faic.\\  
     {\no} {say\past.\my} anything\\ 
\glt `I didn't say anything.'\source{agfc}{181}
\ex
\gll Ní léann siad tada. \\
     {\no} {read\pres} they anything\\
\glt `They don't read anything.' \source{aced}{267}
\ex
\gll Ní ba léir dó {a dhath} ariamh. \\
     {\no} {\ba} clear to.him anything (n)ever \\\label{ex:ariamh5}
\glt `Nothing was ever clear to him.'\source{gfh}{102}
\z
\z

% \ex. \ag. ní dúrt faic\\  
%          {\no} {say\past.\my} anything\\ 
%          `I didn't say anything.'\source{agfc}{181}
%      \bg. ní léann siad tada.\\
%          {\no} {read\pres} they anything\\
%         `they don't read anything' \source{aced}{267}
%      \cg. ní ba léir dó {a dhath} ariamh\\
%          {\no} {\ba} clear to.him anything (n)ever\\\label{ex:ariamh5}
%         `Nothing was ever clear to him.'\source{gfh}{102}

In Munster dialects, the element {\itshape puinn} is also available. It can appear pre-nominally as in (\ref{ex:puinn}a), or alone, as in (\ref{ex:puinn}b), with the meaning `at all' or `much':

\ea\label{ex:puinn}
\ea
\gll Ní-or fhág puinn bád riamh an t-oileán seo. \\
     {\nior} {leave\past} {any} boat ever the island {\seo} \\
     \label{ex:ariamh6}
\glt `No boat ever left this island.'\source{cfc}{129}
\ex
\gll Ní fhaca puinn {ina dhiaidh} sin é.\\
     {\no} {see\past.\my} any after that him \\
\glt `I didn't see him much at all after that.'\source{lgl}{421}
\z
\z

% \ex. \ag. Ní-or fhág puinn bád riamh an t-oileán seo.\\
%           {\nior} {leave\past} {any} boat ever the island {\seo}\\\label{ex:ariamh6}
%          `No boat ever left this island'\source{cfc}{129}
%      \bg. ní fhaca puinn {ina dhiaidh} sin é\\
%           {\no} {see\past.\my} {} after that him\\
%          `I didn't see him much at all after that.'\source{lgl}{421}

The polarity sensitive adverbs are adverbs of temporal perspective\is{temporal adverbial} like \textit{(a)riamh} `ever, still', \textit{choíche}, \textit{go brách}, \textit{go deo} `ever, forever'. The first (\textit{ariamh}) is seen in (\ref{ex:ariamh1}, \ref{ex:ariamh2}, \ref{ex:ariamh3}, \ref{ex:ariamh4}, \ref{ex:ariamh5}) and (\ref{ex:ariamh6}). The others are exemplified in (\ref{ex:ariamh.more}).   

\ea\label{ex:ariamh.more}
\ea
\gll Cuirimse de gheasa ort gan aon fhear {a choíche} a  {phósadh}. \\
     {put\pres.\my} of injunctions on.you {\gan} any man ever {\vce} {marry.\textsc{nfin}} \\
\glt `I put you under an injunction not to ever marry any man.'\source{srnf}{49}
\ex
\gll Ní-or     mhaith leis {a bheith} {go deo} ag caint air féin. \\
     {\nior} good with.him    {be.\textsc{nfin}}  {ever} {\prog} talk {on.him} {\fein} \\     
\glt `He didn't like to ever be talking about himself.'\source{lgl}{109}
\z
\z

% \ex. \ag. cuirimse de gheasa ort gan aon fhear {a choíche} a  {phósadh}\\
%           {put\pres.\my} of injunctions on.you {\gan} any man ever {\vce} {marry.\textsc{nfin}}\\
%          `I put you under an injunction not to ever marry any man.'\source{srnf}{49}
%      \bg. ní-or     mhaith leis {a bheith} {go deo} ag caint air féin\\
%           {\nior} good with.him    {be.\textsc{nfin}}  {ever} {\prog} talk {on.him} {\fein}\\     
%          `He didn't like to ever be talking about himself'\source{lgl}{109}

Within this group there is also an extensive catalog of expressions conventionally taken to denote the lowest possible point on some scale, so-called `minimizers'\is{minimizer}.  \textit{Smid} `breath' refers to the tiniest sound audible, \textit{deor} `drop' to the smallest imaginable quantity of a liquid, \textit{ceo} `mist' to the most insubstantial thing, while \textit{pioc} `a pick' refers to the smallest measure imaginable.

\ea
\ea
\gll Ach ní-or labhair sé smid leofa. \\
     but {\nior} {speak\past} he breath with.them \\
\glt `But he didn't breathe a word to them.'\source{sif}{33}
\ex
\gll Ní-or chaoin sé deoir ariamh ina shaol. \\
     {\nior} {cry\past} he drop ever in.his life \\
\glt `He never shed a tear in his life.'\source{gddr}{166}
\ex
\gll Ach ní-or fhéad sé ceo {a dhéanamh.} \\
     but {\nior} {can\past} he mist {do.\textsc{nfin}} \\
\glt `But he couldn't do a thing.'\source{cc}{79}
\ex
\gll Níl pioc fírinne i {n-a} cheann. \\
     {is.not} pick {truth\gen} in his head \\
\glt `He is incapable of telling the truth.'\source{btfs}{124}
\z
\z

% \ex. \ag. ach ní-or labhair sé smid leofa\\
%           but {\nior} {speak\past} he breath with-them\\
%          `But he didn't breathe a word to them.'\source{sif}{33}
%      \bg. Ní-or chaoin sé deoir ariamh ina shaol.\\
%           {\nior} {cry\past} he drop ever in-his life\\
%          `He never shed a tear in his life.'\source{gddr}{166}
%      \cg. Ach ní-or fhéad sé ceo {a dhéanamh}.\\
%           but {\nior} {can\past} he mist {do.\textsc{nfin}}\\
%          `But he couldn't do a thing.'\source{cc}{79}
%      \dg. Níl pioc fírinne i {n-a} cheann.\\
%          {is.not} pick {truth\gen} in his head\\
%          `He is incapable of telling the truth.'\source{btfs}{124}

An Irish particularity (apparently) is that many minimizers\is{minimizer} are conventionalized disjunctions with the disjunction necessarily interpreted within the scope\is{scope} of negation\is{negation}:

\ea
\ea
\gll Ní-or ghéill sí  ionga ná orlach don chiúnas. \\
     %a bhí dlite do theach an phobail
     {\nior} {yield\past} she fingernail or inch to.the silence \\
\glt `She didn't yield an inch to the silence.'           % <INIT065> <G> 
\ex
\gll Ní thabharfadh  duine ná deoraí freagra air. \\
     {\no}  {give\cond} person or exile/stranger answer on.him \\
\glt `Not a soul would answer him.' %  <INIT125> 
\ex
\gll Ní raibh  tásc ná tuairisc ar an chainteoir. \\
     {\nior} {be\past} sign or report on the speaker \\
\glt `There wasn't a sign or a trace of the speaker.' % <SGC075>
\ex
\gll Ní-or chorruigheadar  lámh ná cos ariamh {ar a son}. \\
    {\nior} {move\past\their} hand or foot ever on-their-behalf \\
\glt `They didn't ever lift a finger on their behalf.'  %   <IAE092-1> 
\ex
\gll Ní-or fhág sé bonn bán ná pingin rua ag a clann sise. \\
     {\nior} {leave\past} he coin white or penny red at her family she \\
\glt `But he didn't leave a penny to her family.'\source{ctp}{4}
\z
\z

% \ex. \ag. ní-or ghéill sí  ionga ná orlach don chiúnas\\ %a bhí dlite do theach an phobail
%           {\nior} {yield\past} she finger-nail or inch to-the silence\\
%          `She didn't yield an inch to the silence.'           % <INIT065> <G> 
%      \bg. ní thabharfadh  duine ná deoraí freagra air.\\
%           {\no}  {give\cond} person or exile/stranger answer on.him\\
%          `Not a soul would answer him.' %  <INIT125> 
%      \cg. ní raibh  tásc ná tuairisc ar an chainteoir\\
%          {\nior} {be\past} sign or report on the speaker\\
%          `There wasn't a sign or a trace of the speaker.' % <SGC075>
%      \dg. ní-or chorruigheadar  lámh ná cos ariamh {ar a son}\\
%           {\nior} {move\past\their} hand or foot ever on-their-behalf\\
%          `They didn't ever lift a finger on their behalf.'  %   <IAE092-1> 
%      \eg. ní-or fhág sé bonn bán ná pingin rua ag a clann sise\\
%           {\nior} {leave\past} he coin white or penny red at her family she\\
%          `But he didn't leave a penny to her family.'\source{ctp}{4}

There are certain other minimizers\is{minimizer} which should be mentioned as well. The prepositional phrase {\itshape dá laghad} `of the least' acting as a post-nominal modifier may create one such:

\ea
\gll Níl aird dá laghad aici ar an mbeirt. \\
     {is.not} attention {of.the} least at.her on the two.people \\
\glt `She doesn't pay the slightest attention to the two of them.'
\z

% \exg. níl aird dá laghad aici ar an mbeirt\\
%      {is.not} attention {of-the} least at.her on the two-people\\
%      `She doesn't pay the slightest attention to the two of them.'

And in a slightly more colorful turn of phrase, we have {\itshape faic na fríde}. {\itshape Faic} is the monomorphemic element already discussed in this section, meaning `nothing' or `anything'. When modified by the possessor {\itshape na fríde} `of the mite' it signifies `the slightest/tiniest thing':

\ea
\gll Níl {faic na fríde} le déanamh acu. \\
     {is.not} {} to {do\vn} at.them \\
\glt `They have nothing whatever to do.'
\z

% \exg. Níl {faic na fríde} le déanamh acu.\\
%       {is.not} {} to {do\vn} at.them\\
%      `They have nothing whatever to do.'

Finally, some minimizers\is{minimizer} are based on now opaque metaphors. In the Irish of Conamara, for example, the expression {\itshape mac an éin bheo} `the son of the living bird' refers to the smallest possible set of people:

\ea
\gll Ní raibh mac an éin bheo le feiceáil an tráth sin de mhaidin. \\
     {\no} {be\past} son the {bird\gen} {living\gen} {\asp} {see\vn} the time {\seo} of morning \\
\glt `There wasn't a single solitary person to be seen at that time of the morning.'
\z

% \exg. ní raibh mac an éin bheo le feiceáil an tráth sin de mhaidin\\
%       {\no} {be\past} son the {bird\gen} {living\gen} {\asp} {see\vn} the time {\seo} of morning\\
%      `There wasn't a single solitary person to be seen at that time of the morning.'

There will be little or nothing in this inventory to surprise those who have closely studied the inventory of negative polarity systems in other languages.

\subsection{Beyond negation: The licensing environments}
\label{sec:beyond}
\is{negation}
We have so far considered just one environment in which the PSEs described in the previous section may appear: in the scope\is{scope} of sentential negation\is{negation}. But their distribution is, in fact, much broader.\footnote{Before extending the investigation to licensing contexts beyond that of negation\is{negation}, we should note that PSEs are also licensed in the scope\is{scope} of the emphatic or `demonic' negation\is{negation} studied by \citet[326--331]{o-siadhail-book} and especially by \citet{dantuono:23}. In this construction a phrase is   fronted to a position immediately to the right of the emphatic negators {\itshape diabhal} `devil' or {\itshape dheamhan} `demon' and the clause out of which the phrase is extracted is headed by the \WH-complementizer\is{wh-question}. PSEs may be fronted (as in \ref{ex:fronted}) or appear in a clause-internal position (as in \ref{ex:internal}):

\ea
\label{ex:fronted}
\gll Dheamhan freagra {ar bith} a thug sí orm. \\
     demon answer any {\aL} {give\past} she on.me \\
\glt `Not an answer did she give me.'\source{aa}{240}
\z

\ea
\label{ex:internal}
\gll Diabhal duine a thug aon aird orm. \\       
     devil person {\aL} {give\past} any attention on.me\\
\glt `Not a person paid any attention to me.'\source{dgd}{79}
\z
}

%     \exg. Dheamhan freagra {ar bith} a thug sí orm.\\
%           demon answer any {\aL} {give\past} she on.me\\
%    `Not an answer did she give me.'\source{aa}{240}

%    \exg. Diabhal duine a thug aon aird orm.\\       
%          devil person {\aL} {give\past} any attention on.me\\
%         `Not a person paid any attention to me.'\source{dgd}{79}

All of them, for instance, also appear in the scope\is{scope} of semi-negative expressions such as `rarely' or `hardly' -- which in Irish are predicates that select clausal complements (nonfinite or finite):

\ea
\gll Is rí-annamh anois éinne acu {a theacht} abhaile. \\
     {\cop} {very-rare} now anyone {of.them} {come.\textsc{nfin}} home \\
\glt `It's very rare now for any of them to come home.'\source{caa}{265}
\z

% \exg.  Is rí-annamh anois éinne acu {a theacht} abhaile.\\
%        {\cop} {very-rare} now anyone {of.them} {come.\textsc{nfin}} home\\
%       `It's very rare now for any of them to come home.'\source{caa}{265}

\ea
\gll {Ar éigean} a bheas tada le déanamh agat.\\
     hardly  {\C} {be\fut} anything {\scshape ptc} {do.\textsc{nfin}} at.you \\
\glt `You'll have hardly anything to do.'\source{aa}{215}
\z

% \exg. {Ar éigean} a bheas tada le déanamh agat\\
%        hardly  {\C} {be\fut} anything {\scshape ptc} {do.\textsc{nfin}} at.you\\
%       `You'll have hardly anything to do.'\source{aa}{215}

The PSEs we are concerned with appear in fact in a broad range of environments, all of them characteristic of those in which NPIs\is{negative polarity item} have been shown to appear in other languages. Documenting this pattern and its breadth is important work, but it makes for tedious reading. I have therefore gathered the relevant data in Appendix One. What is shown there is that PSEs in Irish, with their characteristic existential interpretations, are licensed in the following range of environments:

%\sigh
%\begin{compactenum}[\quad$\circ$]

%\begin{minipage}{\textwidth}
\begin{enumerate}
\item In  polar questions \is{polar question}
\item In \WH-questions (when rhetorical or when demanding exhaustive, rather than partial, answers) \is{wh-question}
\item In conditional clauses (realis and irrealis)\is{conditional clause}
\item In equative clauses\is{equative clause}
\item In comparative and superlative clauses \is{conditional clause} \is{superlative clause}
\item In phrases and clauses introduced by the degree particle \textit{ró-} `too'
\item In certain temporal clauses introduced by {\itshape sul} `before' or {\itshape nuair} `when'
\item In the complement of adversative (and some implicative) predicates\is{implicative predicate} \is{adversative predicate}
\item In the restrictive clauses of universal quantification structures
\end{enumerate}
%\end{minipage}

%\end{compactenum}
%\sigh

This is a list which is familiar from decades of research on NPIs\is{negative polarity item}. Appendix One documents this distribution and also considers some particularities of the Irish patterns, having to do especially with the licensing potential of adversative and negative implicative predicates.\is{adversative predicate} \is{implicative predicate}

Irish PSEs may in fact appear in an additional environment which, as far as I have been able to tell, has not so far been identified for NPIs\is{negative polarity item} in other languages~-- in clauses introduced by (the equivalent of) {\itshape to the extent that}. This is illustrated in (\ref{ex:additional}):

\ea\label{ex:additional}
      To the extent that anyone ever believed this \ldots \\ 
\gll  sa mhéid is gu-r chuir mé aithne {ar bith} air \\
     in.the extent as {\gur} {put\past} I acquaintance any on.him \\
\glt `to the extent that I got to know him at all'\source{lsc}{130}
\z

% \ex. \a.  To the extent that anyone ever believed this \ldots\ 
%      \bg. sa mhéid is gu-r chuir mé aithne {ar bith} air\\
%           in-the extent as {\gur} {put\past} I acquaintance any on.him\\
%          `to the extent that I got to know him at all'\source{lsc}{130}

The licensing potential of such a context presumably arises from the implication of doubt or of disbelief that it conveys concerning the content of the complement clause.

In sum: the distribution of PSEs in Irish and the distribution of NPIs\is{negative polarity item} in other well-studied languages are parallel in strikingly complete and exact ways.  And although real progress has been made in recent years on the question of what this broad range of environments might have in common in terms of their semantics (for overviews, see \cite{giannakidou:11}, \cite{chierchia:13}, or \cite{homer:20}), the task of identifying any plausible syntactic commonality, one that might provide the basis for an agreement or concord relation, seems very challenging.

\subsection{Available readings}
\label{sec:free}

It is a well-known, if not well-understood, property of NPI systems\is{negative polarity item} that a subset of the NPIs of a language may appear outside the licensing environments just listed, but with quasi-universal (or perhaps generic) rather than existential force. These are the so-called “free choice” readings\is{free choice reading} of certain NPIs (\cite{bolinger:72}; \cite[400 ff]{horn:89}; \cite{horn:00}; \cite[Chap. 6]{chierchia:13}). Roughly half of the 115 languages in the \citet{haspelmath:97} sample allow this option for some of their negative polarity items\is{negative polarity item}. Irish can be added to that subgroup. The ill-formed (\ref{ex:bad-tada}) above, for example, is well-formed if the main verb is in conditional mood:
\is{NPI|see {negative polarity item}}

\ea\label{ex:conditional}
\gll D' íosfadh sé tada. \\
     {\textsc{past}} {eat\cond} he anything \\
\glt `He'd eat anything.'
\z

% \exg. D' íosfadh sé tada.\\
%      {\did} {eat\cond} he anything\\
%     `He'd eat anything.'

Among the PSEs, all but the minimizers\is{minimizer} allow such readings.  The facilitating environments are the familiar ones -- modal contexts (as in \ref{ex:conditional}) or the presence of a restrictive modifier such as a relative clause modifying the PSE (`subtrigging' in the sense of \cite{legrand:75}).  This possibility is exemplified for the nominal PSEs in (\ref{ex:free}) and in (\ref{ex:subtrig2}). (\ref{ex:free}) illustrates the modal environment (generic in \ref{ex:free}b); those in (\ref{ex:subtrig2}) illustrate the `subtrigging' effect.

\ea\label{ex:free}
\ea
\gll Ceann de na hoícheanta sin go dtarlódh faic\\
     one  of the nights {\seo} {\C} {happen\cond} anything\\
\glt `one of those nights when anything could happen'\source{lgl}{250}
\ex
\gll {maidir le} daoine bochta, tá rud {ar bith} sách maith dóibh.\\
     {as-for} people poor {be\pres} thing any enough good {for.them}\\
\glt `As for poor people -- anything is good enough for them.'\source{cg}{59}
\ex
\gll Dhéanfainn rud {ar bith} ach tusa a fháil domh féin. \\
     {do\cond.\my} thing any  but you {\vce} {get\vn} to.me {\fein} \\
\glt `I'd do anything to get you for myself.'\source{ssotc}{222}
\z
\z

% \ex.\label{ex:free}\ag. Ceann de na hoícheanta sin go dtarlódh faic\\
%           one  of the nights {\seo} {\C} {happen\cond} anything\\
%          `one of those nights when anything could happen'\source{lgl}{250}
%      \bg. {maidir le} daoine bochta, tá rud {ar bith} sách maith dóibh\\
%           {as-for} people poor {be\pres} thing any enough good {for-them}\\
%           `As for poor people -- anything is good enough for them.'\source{cg}{59}
%      \cg.  Dhéanfainn rud {ar bith} ach tusa a fháil domh féin\\
%            {do\cond.\my} thing any  but you {\vce} {get\vn} to.me {\fein}\\
%           `I'd do anything to get you for myself.'\source{ssotc}{222}

\ea\label{ex:subtrig2}
\ea
\gll Rud {ar bith} a tugadh ar iasacht domsa ariamh    thug mé {ar ais} é. \\
    thing any {\C} {give\past\imp} on loan to.me ever {give\past} I back it \\
\glt `Anything that I was ever lent, I gave it back.'\source {aa}{90}
\ex
\gll Aon áit a chuais, ní raibh aon ní á labhairt ach Gaolainn. \\
     {one/any} place {\C} {go\past.\your} {\nior} {be\past} any thing {\progpass} {speak} but Irish \\
\glt `Any place you went, there wasn't anything being spoken but Irish.'\\\hfill\source{tmgb}{39}
\z
\z

% \ex. \ag. Rud {ar bith} a tugadh ar iasacht domsa ariamh    thug mé {ar ais} é.\\
%           thing any {\C} {give\past\imp} on loan to.me ever {give\past} I back it\\
%          `Anything that I was ever lent, I gave it back.'\source {aa}{90}
%      \bg. Aon áit a chuais, ní raibh aon ní á labhairt ach Gaolainn.\\
%           {one/any} place {\C} {go\past.\your} {\nior} {be\past} any thing {\progpass} {speak} but Irish\\
%          `Any place you went, there wasn't anything being spoken but Irish.'\\\hfill\source{tmgb}{39}

The temporal adverbials\is{temporal adverbial} \textit{(a)riamh}, {\itshape go brách}, {\itshape go deo} and {\itshape choíche} `ever' may, in addition, appear outside the licensing environments just discussed and in that context they have universal rather than existential force and translate as English `always' or `forever': 

\ea\label{ex:ever}
\ea
\gll Bhí sé ariamh ann. \\
     {be\past} it ever in.it \\
\glt `It has always existed.'\source{ff}{167}
\ex
\gll An síleann tú go bhfuil tú ag imeacht {go brách} uainn? \\
     {\interr} {think\pres} you {\go} {be\pres} you {\prog} {leave\vn} forever {from.us}\\
\glt `Do you think you are leaving us for ever?'\source{atfs}{207}
\ex
\gll Bhíodh clocha i nGleann Easa riamh agus beidh {go deo}. \\
     {be\pasthabit} stones in {} {} ever and {be\fut} always\\
\glt `There have always been stones in Gleann Easa and there always will be.'\source{dead}{96}
\ex
\gll Beidh cuimhne choíche agam air. \\
    {be\fut} memory ever at.me on.it\\
\glt `I will always remember it.'\source{omgs}{290}
\z
\z

% \ex.\label{ex:ever}\ag. Bhí sé ariamh ann.\\
%           {be\past} it ever in.it\\
%          `It has always existed.'\source{ff}{167}
%      \bg. An síleann tú go bhfuil tú ag imeacht {go brách} uainn?\\
%          {\interr} {think\pres} you {\go} {be\pres} you {\prog} {leave\vn} forever {from.us}\\
%          `Do you think you are leaving us for ever?'\source{atfs}{207}
%      \cg. Bhíodh clocha i nGleann Easa riamh agus beidh {go deo}.\\
%           {be\pasthabit} stones in {} {} ever and {be\fut} always\\
%          `There have always been stones in Gleann Easa and there always will be.'\source{dead}{96}
%      \dg. Beidh cuimhne choíche agam air.\\
%          {be\fut} memory ever at.me on.it\\
%          `I will always remember it.'\source{omgs}{290}

The two meanings expressed by these adverbs are at least close to those expressed by NPIs\is{negative polarity item} and their “free choice” counterparts\is{free choice reading}. In addition, both interpretations (existential in NPI-licensing contexts\is{negative polarity item}, universal otherwise) are available across the class, suggesting that something more systematic than lexical polysemy is at work. It may be, then, that the possibilities seen in (\ref{ex:ever}) reflect a “free choice”\is{free choice reading} option for certain NPIs. These universal readings, however, are not subject to the licensing restrictions observed for the determiner NPIs such as English {\itshape any} (\cite{legrand:75}; \cite{kadmon-landman:93}; \cite{dayal:98}; \cite{dayal:13}; \cite{giannakidou:01}; \cite[Chap. 6]{chierchia:13}), for which modality or genericity seems to be crucial, neither of which is relevant for (\ref{ex:ever}). 

It might, alternatively, be more profitable to think about cases such as (\ref{ex:ever}) as being parallel, in relevant respects, to cases such as English \textit{until} and related items in other European languages. Temporal clauses introduced by \textit{until} are strong NPIs\is{negative polarity item} when interpreted as punctual\is{temporal adverbial}:

\ea
\ea[]{He didn't finish the paper until July.}
\ex[*]{He finished the paper until July.}
\z
\z

% \ex. \a. He didn't finish the paper until July.
%      \b. *He finished the paper until July.

But when modifying atelic predications they are, as seen in (\ref{ex:dur}), durative in their interpretation and are positive polarity items expressing extension over relatively long intervals (\cite{karttunen:74}; \cite{mittwoch:77}; \cite{giannakidou:02a}; \cite{declerck:95}; \cite{deswart:96}; and especially \cite{condoravdi:08}): 

\ea\label{ex:dur}
We remained in Cambridge until the end of the year.
\z
% \ex. We remained in Cambridge until the end of the year.

The connection with our Irish cases is that the non-NPI readings\is{negative polarity item} of the temporal adverbs\is{temporal adverbial} in (\ref{ex:ever}) also appear only in the context of atelic predications.\footnote{The same restriction seems to hold for the more or less archaic use of English {\itshape {\itshape ever}} when universal in its force (see \cite{israel:98}; \cite[181--183]{horn:00}).} Needless to say, this discussion is little more than a marker laid down for a future research project.\footnote{I am grateful to Nicola D'Antuono for discussion of these matters.}

\subsection{Non-local licensing}
\label{sec:non-local}

One of the principal themes of research on negative concord systems has been that of locality.  To a first approximation, the concord relation may not cross a finite clause boundary unless it is subjunctive (\cite{haegeman-zanuttini:91}, \cite[43--45]{hedde:08}, \cite{deal:22}).\footnote{For complications, exceptions, and for approaches to those issues, see \citet{robinson-thoms:21}, among others.}  This is why analyses of the concord relation in terms of agreement or movement, with their associated locality requirements, have been persuasive and influential.  The licensing of PSEs in Irish, however, is subject to no such restriction, as shown by examples like those in (\ref{ex:long}--\ref{ex:island}), which are commonplace and frequent (304 in our data-set).

\ea\label{ex:long}
\ea
\gll Ní-or chualas gu-r mharaigh na tramanna duine {ar bith} ariamh. \\
     {\nior} {hear\past.\my} {\C\past} {kill\past} the trams person any ever \\
\glt `I haven't heard that the trams ever killed anyone.'\source{ctp}{49}
\ex
\gll má cheapann sibh go bhfuil mise ag déanamh aon fhocal bréige \\
     if {think\pres} {you\pl} {\C} {be\pres} I {\prog} {make\vn} any word {lie\gen} \\
\glt `if you think that I am telling any lies'\source{mabat}{56}
\ex
\gll An síleann tú go {dtiocfadh le} cailín {ar bith} grá mar sin a thabhairt uaithe? \\
    {\interr} {think\pres} you {\go} {could} girl any love like that {\vce} {give\vn} from.her \\
\glt `Do you think that any girl could give such love?'\source{atfs}{343}
\z
\z

% \ex. \label{ex:long}\ag. Ní-or chualas gu-r mharaigh na tramanna duine {ar bith} ariamh.\\
%          {\nior} {hear\past.\my} {\C\past} {kill\past} the trams person any ever\\
%          `I haven't heard that the trams ever killed anyone.'\source{ctp}{49}
%      \bg. má cheapann sibh go bhfuil mise ag déanamh aon fhocal bréige\\
%           if {think\pres} {you\pl} {\C} {be\pres} I {\prog} {make\vn} any word {lie\gen}\\
%          `if you think that I am telling any lies'\source{mabat}{56}
%      \cg. An síleann tú go {dtiocfadh le} cailín {ar bith} grá mar sin a thabhairt uaithe?\\
%           {\interr} {think\pres} you {\go} {could} girl any love like that {\vce} {give\vn} from.her\\
%          `Do you think that any girl could give such love?'\source{atfs}{343}

\noindent In each of the cases in (\ref{ex:long}), the licensing element (negation\is{negation}, the complementizer \textit{má} in (\ref{ex:long}b) or the polar interrogative particle\is{polar question} in (\ref{ex:long}c)) is separated from the PSE it licenses by at least one finite \CP-boundary. Longer dependencies are also possible, as in (\ref{ex:atleast}). In (\ref{ex:atleast}a), the licensed PSE is separated from its licensing negation\is{negation} by two finite clause boundaries, one of them the complement to the experiencer noun \textit{súil} `hope'; in (\ref{ex:atleast}b) the licensing environment is established by the noun \textit{eagla} `fear', which is adversative\is{adversative predicate} and licenses PSEs in its complement.

\ea\label{ex:atleast}
\ea
\gll Ní hamhlaidh {[\lbCP} a tá súil agam {[\lbCP} go dtiocfaidh éinne {]]}. \\
    {\negcop} so {} {\go} {be\pres} hope at.me {} {\go} {come\fut} anyone\\
\glt `It is not the case that I really expect that anyone will come.'\source{lgl}{60}
\ex
\gll ar eagla {[\lbCP} go gceapfadh sé {[\lbCP} go raibh duine {ar bith} díobh chomh díthcéillí {]]} \\
    on fear {} {\go} {think\cond} he {} {\go} {be\past} person any of.them so foolish \\
\glt `for fear that he would think that any of them were so foolish'\source{atim}{90}
\z
\z

% \ex. \ag. Ní hamhlaidh {[\lbCP} a tá súil agam {[\lbCP} go dtiocfaidh éinne {]]}.\\
%          {\negcop} so {} {\go} {be\pres} hope at.me {} {\go} {come\fut} anyone\\
%         `It is not the case that I really expect that anyone will come.'\source{lgl}{60}
%      \bg. ar eagla {[\lbCP} go gceapfadh sé {[\lbCP} go raibh duine {ar bith} díobh chomh díthcéillí {]]}\\
%           on fear {} {\go} {think\cond} he {} {\go} {be\past} person any of.them so foolish\\
%          `for fear that he would think that any of them were so foolish'\source{atim}{90}

The licensing relation can also span at least some island-boundaries, as seen in (\ref{ex:island}), a \WH-island\is{wh-island} in (\ref{ex:island}a)\footnote{For (28a) one might wonder whether the PSE is   licensed in the interrogative clause itself rather than by the matrix negation\is{negation}. This is not a   plausible interpretation, though, given that the \WH-interrogative\is{wh-question} is neither rhetorical nor exhaustive. In addition, the example is ill-formed when the matrix negation\is{negation} is removed.},  and in (\ref{ex:island}b) a finite \CP-argument of the noun {\itshape cuma} `appearance'.  Such structures are strong islands in Irish (\cite{mccloskey:85}; \cite{mccloskey:02}; \cite{maki-obaoill:11}.)

\ea\label{ex:island}
\ea 
\gll Cha-r fhoghlaim mé ariamh cén dóigh le rud {ar bith} a tharraingt. \\
    {\nior} {learn\past} I ever what way with thing any {\vce} {draw\vn}\label{ex:draw} \\
\glt `I didn't ever learn how to draw anything.'\source{apb}{12}
\ex
\gll Ní raibh cuma uirthi go raibh eagla {ar bith} roimh an astar uirthi. \\
     {\no} {be\past} appearance on.her {\go} {be\past} fear any before the journey on.her \\
\glt `It didn't look as if she had any fear of the journey.'\source{nlab}{54}
\z
\z

% \ex. \label{ex:island}\ag. cha-r fhoghlaim mé ariamh cén dóigh le rud {ar bith} a tharraingt.\\
%             {\nior} {learn\past} I ever what way with thing any {\vce} {draw\vn}\label{ex:draw}\\
%            `I didn't ever learn how to draw anything.'\source{apb}{12}
%      \bg. ní raibh cuma uirthi go raibh eagla {ar bith} roimh an astar uirthi\\
%          {\no} {be\past} appearance on.her {\go} {be\past} fear any before the journey on.her\\
%          `It didn't look as if she had any fear of the journey.'\source{nlab}{54}

\subsection{Modification by `almost'}
\label{sec:almost}

Finally, all PSEs in Irish strongly resist modification by {\itshape almost}, a property which in many languages distinguishes NPIs\is{negative polarity item} from inherently negative expressions\is{inherently negative expression}:

\ea
\ea
\gll *Ní raibh {comhair a bheith} duine {ar bith} {i láthair}. \\
     {\nior} {be\past} almost person any present \\
\glt `There was almost nobody present.'   
\ex
\gll *Ní dhéanainn freastal ar {chomhair a bheith} léacht {ar bith}.\\
    {\nior} {do\pasthabit.\my} attendance on almost lecture any\\
\glt `I attended almost no lectures.'
\z
\z

% \ex. \ag. *Ní raibh {comhair a bheith} duine {ar bith} {i láthair}.\\
%            {\nior} {be\past} almost person any present\\
%           `There was almost nobody present.'   
%      \bg. *Ní dhéanainn freastal ar {chomhair a bheith} léacht {ar bith}.\\
%            {\nior} {do\pasthabit.\my} attendance on almost lecture any\\
%           `I attended almost no lectures.'

\subsection{Interim conclusion}
\label{sec:sum}

The items surveyed in this section (all of the PSEs so far identified in Irish),  while being polarity sensitive: (i) are incapable of expressing negation\is{negation} outside an appropriate licensing context, (ii) appear in the range of environments typical of NPIs\is{negative polarity item} investigated in other languages, (iii) support quasi-universal readings outside those environments, (iv) can be licensed non-locally, even across certain island boundaries, and (v) are incompatible with modification by \textit{almost}.  All of this suggests that \citet{jm:acquaviva:96} was correct in arguing that Irish possesses a rich and familiar inventory of negative polarity items\is{negative polarity item} but has no plausible candidate for the role of \isi{inherently negative expression}, as far as is known at present. It therefore also lacks the mechanisms of negative concord and our PSEs are negative polarity items\is{negative polarity item}.\footnote{Elena Herburger (\citeyear{herburger:01}) develops an important analysis of the distribution of polarity sensitive expressions in Spanish, another case in which the distinction between NPIs\is{negative polarity item} and INEs seems less than clear. She shows that that complex of data can be accounted for on the assumption that the relevant PSEs in Spanish are systematically ambiguous between being NPIs and items lexically specified as being   `negative'. Her analysis is remarkably successful, but it cannot be applied to the problems we   deal with in the following section. In Spanish, the class of elements Herburger examines can always express negation\is{negation} on their own, so to speak (because they are inherently negative expressions\is{inherently negative expression}).  But that is not possible for the class of Irish elements we are concerned with here, as we have seen with examples like (\ref{ex:bad-tada}) above. Put differently, the set of contexts in which the `{\itshape n}-words' of Spanish may appear is the union of the distributions of NPIs\is{negative polarity item} and what I have called here INEs, a distribution much broader than that of the Irish PSEs we are concerned with.} From this point on, then, I will abandon the neutral term PSE and call all of the items discussed here negative polarity items (NPIs\is{negative polarity item}).

In light of that conclusion, however, the observations of the section which follows seem unpleasantly anomalous.

\section{The anomaly}
\label{sec:but}

Each of the polarity sensitive elements identified in the previous section may appear in apparent isolation as a subsentential fragment\is{fragment}, often in answer to a \WH-question\is{wh-question} (as in \ref{ex:npiwh} or \ref{ex:goodfrag} above) or to a polar question\is{polar question} (as in \ref{ex:npipolar}).\footnote{Gary Thoms reports that similar facts hold for Scottish Gaelic.}\footnote{Example (\ref{ex:petit}) is from a translation of {\itshape Le Petit Prince} by Antoine de Sainte-Exupéry. The French original has: \textit{Et que fais-tu de ces étoiles? Rien. Je les possède}.}

\ea\label{ex:npiwh}
{\scshape NPIs as fragment answers to wh-questions}:
\ea
\gll “Céard a tá uait?” “Tada, a Mháistir.” \\
     what {\aLgloss} {be\pres} from.you {\ anything}, {\voca} Master \label{ex:ll} \\
\glt `“What do you want?” {\quad}“Nothing, sir.”'\source{ll}{254}
\ex
\gll “Agus caidé a ghní tú leis na réaltógaí?” “Rud {ar bith}.” \\
     and  what {\aLgloss} do you with the stars {\ thing} {\ any}\\ \label{ex:petit}
\glt `“And what do you do with the stars?”  “Nothing.”'\source{apb}{47}
\ex
\gll “Cén chúis a gcuirfeá an cheist sin orm?” “Ó, cúis {ar bith}.” \\
     what reason {\C} {put\cond.\your} the question {\seo} {on.me} oh reason {any}\\
\glt `“Why would you ask me that question?”  “Oh, no reason.”'\source{lofrs}{241} \label{ex:lofrs}
\ex
\gll ach cé labhrann liom í? {\quad} Éinne ach do leithéidse. \\
     but who {speak\pres} with.me it {} anyone but your like \\ \label{ex:tmgb}
\glt `But who speaks it to me? \ Nobody except the likes of you.'\source{tmgb}{252}
\z
\z
\is{wh-question}
\is{fragment}
% \ex. {\scshape npi's as fragment answers to wh-questions}: \label{ex:npiwh}
%       \ag.  `Céard a tá uait?' `Tada, a Mháistir.'\\
%              what {\aLgloss} {be\pres} from.you {\ anything}, {\voca} Master\\ \label{ex:ll}
%             `What do you want?' {\quad}`Nothing, sir.'\source{ll}{254}
%       \bg.  `Agus caidé a ghní tú leis na réaltógaí?' `Rud {ar bith}.'\\
%              and  what {\aLgloss} do you with the stars {\ thing} {\ any}\\\label{ex:petit}
%             `And what do you do with the stars?'  `Nothing.'\source{apb}{47}\footnote{Example
%               \ref{ex:petit} is from a translation of {\itshape Le
%                 Petit Prince} by Antoine de Sainte-Exupéry. The French
%                 original has: \textit{Et que fais-tu de ces étoiles? Rien. Je les possède}.}
%       \cg. `Cén chúis a gcuirfeá an cheist sin orm?' `Ó, cúis {ar bith}.'\\
%             what reason {\C} {put\cond.\your} the question {\seo} {on.me} {} reason {\ \ any}\\
%            `Why would you ask me that question.'  `Oh, no reason.'\source{lofrs}{241} \label{ex:lofrs}
%       \dg.  ach cé labhrann liom í? {\quad} Éinne ach do leithéidse.\\
%             but who {speak\pres} with.me it {} anyone but your like\\ \label{ex:tmgb}
%            `but who speaks it to me? \ Nobody except the likes of you.'\source{tmgb}{252}

\ea\label{ex:npipolar}
{\scshape NPIs as fragment answers to polar questions}:
\ea
\gll “An ndéanfaidh aon duine m' áit-sa duit?” “{Go deo}.” \\
     {\interr} {make\fut} any person my place for.you {\ \ ever} \\ \label{ex:atfs}
\glt `“Will anyone ever take my place for you?” \quad “Never.”'\source{atfs}{488}
\ex
\gll “An bhfaigheadh na scéalaithe aon díolaíocht?”  “Aon rud {in aon chor}.”\\
     {\interr} {get\pasthabit} the storytellers any payment any thing {at.all}\\ \label{ex:al88}
\glt `“Would the storytellers get any payment?” “Nothing at all.”'\source{al}{88}
\ex
\gll “Agus ní raibh eagla ort roimhe?” {\ \ } “Eagla {ar bith}.”\\
     {\ and} {\no} {be\past} fear {on.you} {before.him} {} {\ fear} {\ any} \\
\glt `“And you weren't afraid of him?” “Not at all.”'\source{d}{18}
\z
\z
\is{polar question}
\is{negative polarity item}
% \ex.  {\scshape npi's as fragment answers to polar questions}: \label{ex:npipolar}
%       \ag. `An ndéanfaidh aon duine m' áit-sa duit?' `{Go deo}.'\\
%            {\interr} {make\fut} any person my place for.you {\ \ ever}\\ \label{ex:atfs}
%            `Will anyone ever take my place for you?'\ \  `Never.'\source{atfs}{488}
%       \bg. `An bhfaigheadh na scéalaithe aon díolaíocht?'  `Aon rud {in aon chor}.'\\
%            {\interr} {get\pasthabit} the storytellers any payment any thing {at-all}\\ \label{ex:al88}
%            `Would the storytellers get any payment?' `Nothing at all.'\source{al}{88}
%       \cg. `Agus ní raibh eagla ort roimhe?' {\ \ } `Eagla {ar bith}.'\\
%             {\ and} {\no} {be\past} fear {on.you} {before.him} {} {\ fear} {\ any}\\
%            `And you weren't afraid of him?' `Not at all.'\source{d}{18}

\noindent It is perhaps worth emphasizing that the answers in (\ref{ex:npiwh}) and (\ref{ex:npipolar}) are in no way strained. They require no particular contextualization, nor do they demand any special accommodation. They are routine, and as far as I am aware, there is no alternative way to express what they express (fully articulated clauses aside).

There are additional contexts in which such NPI fragments \is{fragment} appear\is{negative polarity item}, contexts that do not (in an obvious way at least) involve question-answer pairings. We will return to those cases and their implications, but the dilemma is already clear. If the arguments developed so far in this chapter are to be relied upon, bare NPIs\is{negative polarity item} may function as fragment\is{fragment} answers in Irish.  But there is very strong evidence from a range of languages already studied that NPIs\is{negative polarity item} cannot serve as fragment answers\is{fragment}. In fact, this has come to be recognized as one of the most reliable diagnostics for distinguishing between NPIs\is{negative polarity item} and inherently negative expressions\is{inherently negative expression}. As \citet[778]{penka-zeijlstra:10} put it:

\begin{quote}
The ability to contribute negation\is{negation} in fragmentary answers\is{fragment} can, thus, be regarded as a defining property of negative indefinites, distinguishing them from NPIs\is{negative polarity item} (cf. Bernini and Ramat 1996 and Haspelmath 1997).\nocite{bernini-ramat:96}
\end{quote}

\noindent We might, in the face of this dilemma, reject the arguments of the first half of this chapter and conclude that all Irish polarity sensitive expressions are actually negative indefinites. My own assessment (unsurprisingly) is that this would not be a wise move and in the remainder of the chapter, I propose an analysis which preserves those earlier results, while also preserving the basic integrity of the generalization articulated by Penka and Zeijlstra. The resolution proposed is well-supported by evidence internal to Irish, and it also has interesting theoretical implications.  The questions which need to be addressed in such a resolution are these:


\begin{itemize}
\item How can the fragments\is{fragment} in (\ref{ex:npiwh}--\ref{ex:npipolar}) be well-formed outside the licensing context that they otherwise require?
\item How can such examples have the interpretations that they do in the absence of that crucial licensing environment?
\item Why are the mechanisms, whatever they may be, that underlie the possibility in (\ref{ex:npiwh}--\ref{ex:npipolar}) unavailable in the other languages so far examined?
\end{itemize}

\noindent The answer to all of these questions, I believe, is that Irish has a movement rule whose very particular properties are crucial in permitting the possibilities on display in (\ref{ex:npiwh}--\ref{ex:npipolar}).

\section{The resolution}
\label{sec:solution}

\subsection{Narrative fronting}
\label{sec:nf}
\is{narrative fronting|(}
\citet[\S 9.2.2]{o-siadhail-book} and \citet{mccloskey:96a} discuss a process, named by Ó Siadhail {\itshape Narrative Fronting}, by way of which a phrase is moved leftward in a finite clause to a position immediately to the left of the negative complementizer and, therefore, also to the left of the inflected verb. The process is productive and of high frequency.\footnote{There is a similar but much less productive process that applies in clauses headed by the complementizer {\itshape go} and in which the inflected verb is in subjunctive mood (a form now archaic for almost all speakers). Such clauses express curses (as in \ref{ex:curses}a) or blessings (as in \ref{ex:curses}b):

\ea\label{ex:curses}
\ea
\gll Na seacht ndiabhal déag go dtuga {\gapline} leo sibh! \\
   the seven devil ten {\C} {take\subj} {} with.them {you\pl} \\
\glt `May the seventeen devils take you!'\source{cnf}{51}
\ex
\gll Ádh agus sonas go raibh {\gapline} ort.\\
    luck and happiness {\C} {be\subj} {} on.you \\
\glt `May you have good fortune and happiness.'\source{sk}{104}
\z
\z

\noindent Though superficially similar, it is not clear that the two processes have a common syntax. We will be concerned here only with the very productive negative case.}

%   \ex. \ag. Na seacht ndiabhal déag go dtuga {\gapline} leo sibh\\
%   the seven devil ten {\C} {take\subj} {} with.them {you\pl}\\
%   `May the seventeen devils take you!'\source{cnf}{51}
%   \bg. ádh agus sonas go raibh {\gapline} ort\\
%   luck and happiness {\C} {be\subj} {} on.you\\
%   `May you have good fortune and happiness.'\source{sk}{104}

A variety of phrase-types may be fronted, but the most frequently attested pattern is one in which the moved phrase is an indefinite nominal within the scope\is{scope} of the triggering negation\is{negation}.  The fronted phrase is nominal in (\ref{ex:moved.nom}), adverbial or prepositional in (\ref{ex:moved.prep}).

\ea\label{ex:moved.nom}
\textsc{narrative fronting in negative clauses}:
\ea
\gll Duine níba réasúnaí ní raibh ann. \\
     person more {reasonable.{\scshape compr}} {\scshape neg} was in.it \\
\glt `A more reasonable person was there none.'\source{ff}{107}
\ex
\gll Mo bhéal ní-or oscail mé {ar feadh} chúig lá. \\
     my mouth {\nior} {open\past} I during five day \\
\glt `I didn't as much as open my mouth for five days.'\source{tuair}{26-04-21}
\ex
\gll Leabhar gramadaí ní raibh ariamh agam. \\
     book {grammar\gen} {\nior} {be\past} ever {at.me} \\
\glt `I never had a grammar-book.'\source{abhm}{41}
\z
\z

% \ex. \textsc{narrative fronting in negative clauses}:
%     \ag. Duine níba réasúnaí ní raibh ann.\\
%          person more {reasonable.{\scshape compr}} {\scshape neg} was in.it\\
%          `A more reasonable person was there none.'\source{ff}{107}
%     \bg. Mo bhéal ní-or oscail mé {ar feadh} chúig lá.\\
%          my mouth {\nior} {open\past} I during five day\\
%         `I didn't as much as open my mouth for five days.'\source{tuair}{26-04-21}
%     \cg. Leabhar gramadaí ní raibh ariamh agam.\\
%          book {grammar\gen} {\nior} {be\past} ever {at.me}\\
%         `I never had a grammar-book.'\source{abhm}{41}
    
\ea\label{ex:moved.prep}
\ea
\gll Ach díreach ní-or bhreathnaigh sí air \\  %greim ní-or réitigh sí dó ná focal ní-or labhair sí leis.\\
     but straight {\nior} {look\past} she on.him \\  %bite {\nior} prepared she for-him or word {\nior} {speak\past} she with.him\\
\glt `But straight she didn't look at him.'\source{c}{24}
\ex
\gll isteach san fháinne ní thiocfaidh sí \\  
     into   {in.the} ring {\no} {come\fut} she \\
\glt `Into the ring she will not come.'\source {sgc}{112} 
\ex
\gll {Go deo} ná {go bráthach} ní scarfamaoid ón {a chéile} arís. \\
     ever or ever {\no} {separate\fut.\our}  from each-other again \\
\glt `Never again will we separate from one another.'\source{iae}{331}
\z
\z

% \ex. \ag. Ach díreach ní-or bhreathnaigh sí air\\  %greim ní-or réitigh sí dó ná focal ní-or labhair sí leis.\\
%           but straight {\nior} {look\past} she on.him\\  %bite {\nior} prepared she for-him or word {\nior} {speak\past} she with.him\\
%          `But straight she didn't look at him.'\source{c}{24}
%      \bg. isteach san fháinne ní thiocfaidh sí\\  
%           into   {in-the} ring {\no} {come\fut} she\\
%          `Into the ring she will not come.'\source {sgc}{112} 
%      \cg. {Go deo} ná {go bráthach} ní scarfamaoid ón {a chéile} arís\\
%           ever or ever {\no} {separate\fut.\our}  from each-other again\\
%          `Never again will we separate from one another.'\source{iae}{331}

\noindent Narrative Fronting is optional and the examples of (\ref{ex:moved.nom}) and (\ref{ex:moved.prep}) are equivalent in their truth-conditions to the corresponding examples in which it has not applied. (\ref{ex:moved.nom}) and (\ref{ex:moved.prep}), though, are felt to be “emphatic” in a way that their counterparts without fronting are not.  The general pattern, then, can be schematized as in (\ref{ex:nfa}):

\protectedex{
\ea\label{ex:nfa}\textsc{narrative fronting}:
\ea $[$ XP$_j$ \lsel{c}{neg} \lsel{v}{fin} \ldots\ \gapline$_j$ \ldots\ ]
\ex where XP can be any constituent type  but is frequently an indefinite nominal 
\ex and the interpretive effect is to express “emphatic” negation\is{negation}.
\z
\z
}

% \ex.  {\scshape narrative fronting}:\label{ex:nfa}
%       \a. [ XP$_j$  \lsel{c}{neg} \lsel{v}{fin} \ldots\ \gapline$_j$ \ldots\ ]
%       \b. where XP can be any constituent-type  but is frequently an indefinite nominal 
%       \c. and the interpretive effect is to express “emphatic” negation.

Narrative Fronting is relevant for us because all of the weak NPIs\is{negative polarity item} and all of the minimizers\is{minimizer} surveyed earlier appear freely and frequently in the XP-position of  (\ref{ex:nfa}a). The examples in (\ref{ex:npinarr1}) illustrate this fact for the weak NPIs; those in (\ref{ex:npinarr2}) for the minimizers\is{minimizer}.

\ea\textsc{weak npis in narrative fronting}:\label{ex:npinarr1}
\ea
\gll Cearta {ar bith} ní raibh {\gapline} ag gnáthdhaoine. \\
    rights any {\no} {be\past} {} at ordinary.people\\
\glt `Ordinary people had no rights.\source{abhm}{53}
\ex
\gll Aon mhoill ní-or dhein Cromaill. {\gapline} \\
     any delay {\nior} {make\past} Cromwell {}\\
\glt `Cromwell made no delay.'/ `No delay did Cromwell make.'\source{oogc}{199}
\ex
\gll Ach tada ní raibh sé {in ann} {\gapline} a chloisteáil. \\
     but anything {\no} {be\past} he able {} {\vce} {hear\vn} \\    
\glt `But nothing was he able to hear.'\source{sjsj}{55}
\ex
\gll {Go deo} arís ní dhéanfadh fear amadán {\gapline} dom\\
     ever again {\no} {make\cond} man fool {} {of.me} \\
\glt `Never again would a man make a fool of me.'\source{lgl}{158}
\ex
\gll Éinne de na comharsain ní cheannódh é. \\
     anyone of the neighbors {\no} {buy\cond} it \\
\glt `None of the neighbors would buy it.'\source{bm}{197}
\z
\z
\is{negative polarity item}
% \ex.  \textsc{weak npi's in narrative fronting}:\label{ex:npinarr1}
%       \ag. Cearta {ar bith} ní raibh {\gapline} ag gnáthdhaoine\\
%            rights any {\no} {be\past} {} at ordinary-people\\
%           `Ordinary people had no rights.\source{abhm}{53}
%       \bg. Aon mhoill ní-or dhein Cromaill {\gapline} \\
%            any delay {\nior} {make\past} Cromwell {}\\
%           `Cromwell made no delay.'/ `No delay did Cromwell make.'\source{oogc}{199}
%       \cg. Ach tada ní raibh sé {in ann} {\gapline} a chloisteáil\\
%            but anything {\no} {be\past} he able {} {\vce} {hear\vn}\\    
%           `But nothing was he able to hear.'\source{sjsj}{55}
%       \dg. {Go deo} arís ní dhéanfadh fear amadán {\gapline} dom\\
%            ever again {\no} {make\cond} man fool {} {of.me}\\
%           `Never again would a man make a fool of me.'\source{lgl}{158}
%       \eg. Éinne de na comharsain ní cheannódh é.\\
%            anyone of the neighbors {\no} {buy\cond} it\\
%           `None of the neighbors would buy it.'\source{bm}{197}

\ea {\scshape minimizers in narrative fronting}:\label{ex:npinarr2}
\ea
\gll {Faic na fríde} ní bhfuair mé {\gapline} mar fhreagra. \\
     {the-tiniest-thing} {\no} {get\past} I {} as answer \\
\glt `I didn't get the tiniest thing as an answer.'\source{paa}{24}
\ex
\gll Smid ní -l {\gapline} ann faoi Tone. \\
     breath {\no} {be\pres} {} in.it about {} \\
\glt `There's absolutely nothing in it about Tone.'\source{tii}{121}
\ex
\gll Pioc eagla ní raibh {\gapline} ar an tiománaí. \\
     {pick} fear {\no} {be\past} {} on the driver \\
\glt `The driver wasn't the tiniest bit afraid.'\source{dr}{15}
\ex
\gll Deoir ní-or chaoin sé. {\gapline} \\
     drop {\nior} {weep\past} he {} \\
\glt `He didn't cry a single tear.'\source{ll}{118}
\ex
\gll Le mac an éin bheo níor sceith ceachtar againn ár rún. \\
     with son the {bird\gen} living {\nior} {expose\past} either of.us our secret\\
\glt `Neither of us revealed our secret to a single living soul.'\source{ll}{437}
\z
\z
\is{minimizer}
% \ex.  {\scshape minimizers in narrative fronting}:\label{ex:npinarr2}
%       \ag. {Faic na fríde} ní bhfuair mé {\gapline} mar fhreagra.\\
%            {the-tiniest-thing} {\no} {get\past} I {} as answer\\
%           `I didn't get the tiniest thing as an answer.'\source{paa}{24}
%       \bg. Smid ní -l {\gapline} ann faoi Tone.\\
%            breath {\no} {be\pres} {} in.it about {}\\
%           `There's absolutely nothing in it about Tone.'\source{tii}{121}
%       \cg. pioc eagla ní raibh {\gapline} ar an tiománaí\\
%            {pick} fear {\no} {be\past} {} on the driver\\
%           `The driver wasn't the tiniest bit afraid.'\source{dr}{15}
%       \dg. deoir ní-or chaoin sé {\gapline} .\\
%            drop {\nior} {weep\past} he {}\\
%           `He didn't cry a single tear.'\source{ll}{118}
%       \eg. Le mac an éin bheo níor sceith ceachtar againn ár rún.\\
%            with son the {bird\gen} living {\nior} {expose\past} either of.us our secret\\
%           `Neither of us revealed our secret to a single living soul.'\source{ll}{437}

\noindent It is not just that weak NPIs\is{negative polarity item} {\itshape may} undergo Narrative Fronting; they clearly have a particular affinity for the environment created in (\ref{ex:nfa}), as is shown by the fact that in 43\% of the attested examples of Narrative Fronting in our database (196 of 486), the element fronted is an NPI\is{negative polarity item}. Among these, minimizers\is{minimizer} are particularly frequent; they represent 41\% (80 of 196) of all the examples in which NPIs\is{negative polarity item} are fronted under Narrative Fronting.

There is something, then, about the environment of Narrative Fronting that particularly favors NPIs\is{negative polarity item} and there is something about NPIs (and minimizers\is{minimizer}, in particular) which makes them especially susceptible to fronting in this context.  This is one of a number of observations suggesting that the two phenomena are deeply entangled.  What is the nature of that entanglement though?

The “emphatic” character of Narrative Fronting seems to have its source, at least in part, in that the structure in (\ref{ex:nfa}) expresses “scalar assertions” in the sense of \citet{krifka:95}.  They evoke scalar implicatures of a familiar kind in that the use of such a structure evokes alternatives to the proposition actually expressed~-- alternatives that are ranked on a scale of strength. The relevant notion of “strength” is in turn dependent on information shared among interlocutors, but also on the relative informational strength of those alternatives as measured by asymmetric entailment relations.  Such ranked alternatives, implicitly evoked, have been central to theoretical work in pragmatics and semantics for many years.  In that light, consider the examples of (\ref{ex:ranked}):

% Levinson 2000, Horn 1992, Chierchia 2013

\ea\label{ex:ranked}
\ea
\gll Acht sagart amháin ní tháinig {de chóir} fhéasta an Rí. \\
     but priest one {\no} {come\past} near  feast the king \\
\glt `But not a single priest came near the king's feast.'\source{umi}{13}\label{ex:ri}
\ex
\gll Míle fear ní bhainfeadh feacadh aisti as a háit. \\
     thousand men {\no} {take\cond} movement out.of.it from its place \\
\glt `A thousand men couldn't budge it from its place.' (a rock)\source{att}{34}
\z
\z

% \ex. \ag. Acht sagart amháin ní tháinig {de chóir} fhéasta an Rí\\
%           but priest one {\no} {come\past} near  feast the king\\
%          `But not a single priest came near the king's feast.'\source{umi}{13}\label{ex:ri}
%      \bg. míle fear ní bhainfeadh feacadh aisti as a háit\\
%           thousand men {\no} {take\cond} movement out-of.it from its place\\
%          `A thousand men couldn't budge it from its place' (a large rock)\source{att}{34}

\noindent In (\ref{ex:ranked}a) the alternatives evoked have to do with the number of priests who had attended the king's feast~-- a set of propositions of the form: `$n$ priests did not attend the king's feast' ordered by the value of $n$.  The asserted proposition is that the lowest number possible (namely none at all) attended. That is also the strongest proposition among the evoked alternatives because it entails all of the others (i.e., if it is not the case that one priest attended, then it is not the case that two attended, or that three attended, or four or \ldots{}).  The proposition expressed is therefore the logically strongest and the most informationally specific of the alternative-set.  In (\ref{ex:ranked}b), the alternative propositions evoked are of the form: `$n$ men could not move that rock', and those propositions are ranked by the value of $n$ from one thousand down as far as one. The strongest proposition on that scale (in the sense of entailing all of the others and of being, again, informationally more specific) is the one asserted to be true (i.e., if it is impossible for one thousand men to dislodge the rock, it is impossible for 900 to do so and also 800 and so on downwards).  The meaning ultimately conveyed, then, is that the rock is likely impossible to dislodge. The proposition actually expressed in both cases is presented as being at the extreme high-point of a scale of salient alternatives. This is the standard logic of scalar implicatures. It seems reasonable, then, to assume that Narrative Fronting is a syntactic operation which attracts to the XP-position of (\ref{ex:nfa}) constituents that evoke alternatives. (\ref{ex:nfb}) revises (\ref{ex:nfa})  accordingly: 

\ea\label{ex:nfb}\textsc{narrative fronting}:
\ea $[$ XP$_j$ \lsel{c}{neg} \lsel{v}{fin} \ldots\ \gapline$_j$ \ldots\ ]
\ex where \textsc{xp} is alternative-evoking\is{alternative-evoking}.
\z
\z

\noindent If the phrase fronted in (\ref{ex:nfb}) must be alternative-evoking\is{alternative-evoking}, an additional property of Narrative Fronting falls into place. That phrase is modified very frequently by the focus particle {\itshape féin}, a phrase-final focus marker whose meaning is close to that of English `even':

\ea
\gll Bhí an madra féin fachtha mífhoighneach. \\
     {be\past} the dog {\itshape féin} gotten impatient \\
\glt `Even the dog had become impatient.'\source{gsa}{19}
\z

% \exg. bhí an madra féin fachtha mífhoighneach.\\
%       {be\past} the dog {\itshape féin} gotten impatient\\
%      `Even the dog had become impatient.'\source{gsa}{19}

The particle \textit{féin} appears very frequently as a modifier of the phrase fronted under Narrative Fronting:

\ea\label{ex:fein.fronting}
\ea
\gll An toirneach féin ní dhúiseodh Johnny. \\
     the thunder {\itshape féin} {\no} {waken\cond} {} \\
\glt `Even thunder wouldn't waken Johnny.'\source{c}{17}
\ex
\gll An fhuiseog féin ní raibh {ina suí}. \\
     the lark {\itshape féin} {\no} {be\past} awake \\
\glt `Even the lark wasn't awake.'\source{an}{43}
\ex
\gll Feoirling féin ní thabharfadh sé dhó. \\
     farthing {\itshape féin} {\no} {give\cond} he to.him \\
\glt `Even a farthing he wouldn't give him.'\source{atim}{123}
\z
\z

% \ex. \ag. An toirneach féin ní dhúiseodh Johnny.\\
%           the thunder {\itshape féin} {\no} {waken\cond} {}\\
%          `Even thunder wouldn't waken Johnny.'\source{c}{17}
%      \bg. An fhuiseog féin ní raibh {ina suí}.\\
%           the lark {\itshape féin} {\no} {be\past} awake\\
%          `Even the lark wasn't awake.'\source{an}{43}
%      \cg. Feoirling féin ní thabharfadh sé dhó.\\
%           farthing {\itshape féin} {\no} {give\cond} he to.him\\
%          `Even a farthing he wouldn't give him.'\source{atim}{123}

This is an expected possibility given (\ref{ex:nfb}) because the effect of suffixing {\itshape féin} to some phrase XP is exactly to turn XP into an alternative-evoking expression\is{alternative-evoking}. In the case of (\ref{ex:fein.fronting}b), for instance, the alternative propositions evoked have to do with what creatures were up and about that morning. Given conventional ideas about bird life-styles, the lark will always be the earliest creature awake and the proposition that the lark was not awake therefore entails all of the alternative propositions evoked (i.e., the fox was not awake, the hare was not awake, the curlew was not awake \ldots{}). What is ultimately conveyed, then, is that no creature stirred and that was because it was really unusually early in the morning. 

Such assertions, then, convey that the proposition actually expressed is at the upper limit of some scale of imaginable alternatives and is therefore outside the range of conventional norms and expectations, thereby evoking in hearers a sense of surprise or unexpectedness. This seems to be the principal source of the intuitively “emphatic” character of Narrative Fronting examples; they are scalar assertions in Krifka's (\citeyear{krifka:95}) sense.

Negation\is{negation} plays a central role in these deductions. Its effect is to reverse the direction of entailment among the alternatives and therefore to reverse the ranking of those propositions on the scale of strength.  Returning to example (\ref{ex:ri}), for instance: the proposition {\itshape One   priest attended the king's feast} leaves open the possibility that two priests, or three, or four \ldots\ attended the feast, but it entails neither those propositions nor their negations. The presence of negation\is{negation} changes that calculation and rules out possibilities that would be allowed in its absence; it therefore `converts' what would be in its absence a logically weak and low-ranked proposition into a logically strong and high-ranked proposition.

This must be why the NPIs\is{negative polarity item} which are our central concern are so susceptible to Narrative Fronting.

If there were a class of expressions lexically specified to be alternative-evoking\is{alternative-evoking}, we would now expect those expressions to appear naturally and frequently in the XP-position of (\ref{ex:nfb}). But that is exactly the claim that is at the heart of one of the most important strands of current research on the nature and licensing of NPIs~-- from the domain-widening of \citet{kadmon-landman:93} to the explicit appeal to scalar implicatures in the work of \citet{krifka:95}, \citet{lee-horn:94}, \citet{israel:98}, \citet{lahiri:98}, \citet{horn:00}, \citet{condoravdi:08}, \citet{chierchia:13} and \citet{jeong-roelofsen:23}. The central commitment in this line of work is that the limited distribution of NPIs (that is, that they may appear only in downward-entailing environments) is to be attributed to the fact that they are required in their lexical semantics to be alternative-evoking\is{alternative-evoking} and further that they are, or at least that many of them are, lexically specified as representing minimal elements on the quantity\hyp scale implicitly defined by those alternatives. But by the logic we just reviewed those “minimal” elements will be strengthened exactly when they appear within the scope\is{scope} of negation\is{negation}.  Different theoreticians have different ways of working this reasoning into a formal theory of NPI-licensing\is{negative polarity item}, but the common thread, since \citet{krifka:95}, has been that unless such items appear within the scope\is{scope} of negation\is{negation} (or an element with similar logical properties) they run afoul of a version of Grice's (\citeyear{grice:75}) Quantity Maxim (namely, “be as informative as possible”) built into the compositional mechanisms.

But negation\is{negation} is, of course, also the syntactic driver of Narrative Fronting. Weak NPIs\is{negative polarity item} and minimizers\is{minimizer} will be ideal candidates for the role of XP in (\ref{ex:nfb}), then, because they are in their lexical definition alternative-evoking\is{alternative-evoking} and minimal.  In their interaction with negation\is{negation}, then, they will very naturally generate the logically strong propositions that are the hallmark of the construction.

Given this perspective, we understand why NPIs\is{negative polarity item} in Irish appear so frequently in the fronted position of Narrative Fronting structures. The two phenomena are “entangled”, as I put it above, because they exploit the same logical mechanism in evoking scalar implicatures -- namely, the strengthening effect of downward entailing contexts.  The difference between the two is that Narrative Fronting is syntacticized and so limited to a single downward-entailing environment (the domain of sentential negation\is{negation}), while weak NPIs\is{negative polarity item} and minimizers\is{minimizer} can exploit that logic in any environment that has the appropriate semantics (those described in \sectref{sec:beyond}). The descriptions given earlier now reduce to (\ref{ex:nfc}).

\ea {\scshape narrative fronting}:\label{ex:nfc} \\
The finite negation\is{negation} head may include a probe that attracts XPs that are alternative-evoking\is{alternative-evoking}.
\z

% \ex.  {\scshape narrative fronting}:\label{ex:nfc}\\
%       The finite negation\is{negation} head may include a probe which attracts
%       XPs which are alternative-evoking.

The description in (\ref{ex:nfc}), unlike our earlier formulations, makes no mention of the “emphatic” character of Narrative Fronting. That aspect of the construction, as we have seen, emerges organically from the interaction between the semantics of the negative head and the alternatives evoked by the attracted constituent.

A property of this account that is neither obviously correct nor obviously incorrect is that the fronting itself plays no role in establishing the “emphatic” character of Narrative Fronting. On this account, that aspect of the construction emerges from a semantic-pragmatic interaction between the attracting negation\is{negation} and the alternatives evoked by the fronted element; an interaction that would take place even when the alternative-evoking\is{alternative-evoking} phrase remains in its base position. The pairs of examples in (\ref{ex:min}) and (\ref{ex:weak}) should then be equally “emphatic”:

\ea\label{ex:min}
\ea
\gll Ach ní-or labhair sé smid leofa. \\
     but {\nior} {speak\past} he breath with.them \\
\glt `But he didn't breathe a word to them.'\source{sif}{33}
\ex Smid níor labhair sé {\gapline} leofa.
\z
\z

% \ex.\label{ex:min}\ag. ach ní-or labhair sé smid leofa\\
%           but {\nior} {speak\past} he breath with.them\\
%          `But he didn't breathe a word to them.'\source{sif}{33}
%      \b.  Smid níor labhair sé {\gapline} leofa.

\ea\label{ex:weak}
\ea
\gll Aon mhoill ní-or dhein Cromaill. {\gapline} \\
     any delay {\nior} {make\past} Cromwell {} \\
\glt `No delay did Cromwell make.'\source{oogc}{199}
\ex Níor dhein Cromaill aon mhoill.
\z
\z

% \ex.\label{ex:weak}\ag. Aon mhoill ní-or dhein Cromaill {\gapline} \\
%           any delay {\nior} {make\past} Cromwell {}\\
%           `No delay did Cromwell make.'\source{oogc}{199}
%      \b.  Níor dhein Cromaill aon mhoill.

Assessing whether or not this prediction is correct is a matter of such subtlety and vagueness that it will be next to impossible to investigate responsibly, I suspect. For what it may be worth (not much), my impression is that it is not obviously incorrect.\footnote{Note that to claim \label{fn:pain} that certain expressions are alternative-evoking\is{alternative-evoking} is not to claim that they are “\F-marked” in the sense familiar from work on the distribution of focal accents and contrastive focus (or to claim that Narrative Fronting is movement to a dedicated focal position). Rather I follow \citet{krifka:95}, \citet{jeong-roelofsen:23} and others in assuming that expressions in focus evoke alternatives but that that is just one of the contexts in which alternatives play a central role.  See \citet{krifka:95} and \citet{jeong-roelofsen:23} for important discussions of the issues that arise here. If we were to assume that the alternative-evoking character\is{alternative-evoking} of NPIs\is{negative polarity item} reflects a kind of inherent focus marking, we are left in a poor position to understand the differences between emphatic and non-emphatic uses of NPIs\is{negative polarity item} -- the central concern of the discussion in \citet{jeong-roelofsen:23}. They assume, with  \citet{krifka:95} and others, that all NPIs evoke alternatives, as a matter of lexical   specification, but that there is also a distinction between contingently\hyp emphatic and inherently emphatic members of the class. The latter are the minimizers\is{minimizer} and they are inherently focused; the former may or may not be focused. These commitments are entirely consistent with the discussion here. Chierchia (\citeyear{chierchia:13}) draws the same distinction in a different way. The minimizers\is{minimizer}, because they rely on an operator like {\itshape even} for the required exhaustification of their alternative-set, are always and strongly emphatic. For run of the mill weak NPIs\is{negative polarity item}, like {\itshape any}, however, the activation of the relevant alternatives is often undetectable.}

\subsection{Narrative fronting and scope}
\label{sec:nfscope}
\is{scope}
There is a final property of Narrative Fronting that we need to attend to, one that is implicit in the preceding discussion but which should be made explicit.

It was observed in \citet[76--86]{mccloskey:96a} that Narrative Fronting does not expand the scope\is{scope} of the raised item, at least with respect to negation\is{negation}. In (\ref{ex:temporal}), for instance, the temporal indefinite\is{temporal adverbial} {\itshape uair amháin} remains within the scope\is{scope} of negation\is{negation} despite apparently preceding it:

\ea\label{ex:temporal}
\gll uair amháin fiú, ní-or cheistigh sé conas a bhí an t-ullmhúchán {faoi bhráid} Mheiriceá ag dul. \\
    time one even {\nior} {question\past} he how {\C} {be\past} the preparation for America {\prog} {go\vn}\\ \label{ex:uair}
\glt `Not even once did he ask how the preparation for America was going.'\\ \hfill\source{pi}{67}
\z

% \exg.  uair amháin fiú, ní-or cheistigh sé conas a bhí an t-ullmhúchán {faoi bhráid} Mheiriceá ag dul. \\
%        time one even {\nior} {question\past} he how {\C} {be\past} the preparation for America
%        {\prog} {go\vn}\\ \label{ex:uair}
%       `Not even once did he ask how the preparation for America was going.'\\ \hfill\source{pi}{67}

\noindent (\ref{ex:ri}) above is similar, as are the examples in (\ref{ex:temporal2}):

\ea\label{ex:temporal2}
\ea
\gll Aon mhála amháin ní bhfaighidh tú. \\
     one bag one {\no} {get\fut} you\\
\glt `Not one bag will you get.'\source{ota}{194}
\ex
\gll duine amháin as ocht nduine dhéag i seomra na nuachtóirí ní-or labhair liom air \\
     person one out.of eight person ten in room the journalists {\nior} {speak\past} with.me on.it \\
\glt `Not one person out of eighteen in the newsroom spoke to me about it.' \source{aag}{027}
\ex
\gll Duine acu ní-or aithníos. \\
     person of.them {\nior} {recognize\past.\my}\\
\glt `I recognized none of them.'\source{btfs}{14}
\z
\z

% \ex. \ag. Aon mhála amháin ní bhfaighidh tú.\\
%           one bag one {\no} {get\fut} you\\
%          `Not one bag will you get.'\source{ota}{194}
%     \bg.  duine amháin as ocht nduine dhéag i seomra na nuachtóirí ní-or labhair liom air\\
%           person one out-of eight person ten in room the journalists {\nior} {speak\past} with.me on.it\\
%          `Not one person out of eighteen in the newsroom spoke to me about it.\\ \hfill\source{aag}{027}
%     \cg. duine acu ní-or aithníos\\
%          person of.them {\nior} {recognize\past.\my}\\
%         `I recognized none of them.'\source{btfs}{14}

\noindent (\ref{ex:temporal2}c), for example, in its actual context of use, does not convey that there was one person that I did not recognize -- a meaning naturally expressible in English by means of {\itshape One of them, I didn't recognize}. Rather, the intended interpretation has the indefinite interpreted within the scope\is{scope} of negation\is{negation}.  The examples of (\ref{ex:disjuncts}) show that fronted disjuncts also remain within the scope\is{scope} of the negation\is{negation} which triggers Narrative Fronting.

\ea\label{ex:disjuncts}
\ea
\gll Do chlann ná do chéile ní fheicfidh tú {go deo}. \\
     your family or your spouse {\no} {see\fut} you ever \\
\glt `You will never see either your spouse or your family.'\source{cdc}{230}
\ex
\gll Agus mionnán ná bainne ní bheadh {\ldots} an bhliain sin aici. \\
    and  kid.goat or milk {\no} {be\cond} {} the year {\seo} at.her \\
\glt `And she would have neither a goat nor (its) milk that year.'\source{cc}{16}
\z
\z

% \ex. \ag. do chlann ná do chéile ní fheicfidh tú {go deo}\\
%           your family or your spouse {\no} {see\fut} you ever\\
%          `You will never see either your spouse or your family.'\source{cdc}{230}
%      \bg. Agus mionnán ná bainne ní bheadh {\ldots} an bhliain sin aici.\\
%           and  kid-goat or milk {\no} {be\cond} {} the year {\seo} at.her\\
%          `And she would have neither a goat nor (its) milk that year.'\source{cc}{16}

\noindent For detailed discussion see \citet{mccloskey:96a}.\footnote{Note that the claim is not that wide-scope\is{scope} indefinites are never found in the fronted position of a Narrative Fronting structure. They are, though rarely. That is, they are found with the same degree of difficulty and at the same (low) level of frequency as is characteristic of indefinites in the base-position of the movement. That is, Narrative Fronting does not expand or change the scopal properties of fronted elements with respect to negation\is{negation}.}

The fact, then, that weak NPIs\is{negative polarity item} and minimizers\is{minimizer} may appear in clause-initial position under Narrative Fronting (as in \ref{ex:npinarr1} and \ref{ex:npinarr2}) is just one aspect of this larger generalization. In cases like (\ref{ex:npinarr1}) and (\ref{ex:npinarr2}), the fronted NPI\is{negative polarity item} remains within the scope\is{scope} of sentential negation\is{negation} just like the indefinites of (\ref{ex:temporal2}) and the disjuncts of (\ref{ex:disjuncts}), and this is why such examples are fully well-formed.

\subsection{Narrative fronting and ellipsis}

If the conclusions of the previous subsection are safe, the path is clear towards resolving the apparent anomaly I opened with: how there can be answers like (\ref{ex:goodfrag}), which consist only of a negative polarity item, in apparent violation of an otherwise valid crosslinguistic generalization. Following \citet[691]{merchant:04}, we can maintain that the possibility of fragment answers\is{fragment} consisting of, or containing, NPIs\is{negative polarity item} is parasitic on a prior application of Narrative Fronting.\footnote{The link between Narrative Fronting and the possibility of fragment\is{fragment} NPIs\is{negative polarity item} is made independently in \citet{dantuono-diss}.}  A routine application of Narrative Fronting raises the NPI\is{negative polarity item} to the left periphery, and the clausal remnant from which it has been raised is, if conditions warrant, elided by a sluicing-like operation that eliminates all but the fronted phrase.  The apparently isolated NPIs\is{negative polarity item} are well-formed because they are within the scope\is{scope} of negation\is{negation} in a pre-ellipsis representation, and they have the interpretations that they do because of the scope-preserving\is{scope} property of Narrative Fronting.  Many languages disallow the equivalents of (\ref{ex:npiwh}--\ref{ex:npipolar}) because they lack a movement operation with the particular set of properties that we have demonstrated for Narrative Fronting in Irish (though see \cite{laka:93} and \cite{giannakidou:00} for cases of the same general type).  In (\ref{ex:minimizer1}b), which involves a minimizer\is{minimizer}, {\scshape l}'s response will be derived roughly as in (\ref{ex:minimizer2}b), in which I use a greyed-out font to indicate elided material.

\ea\label{ex:minimizer1}
\ea
\begin{xlist}
\exi{J:}
\gll  Ná habair {a dhath} le n-ár gcairde. \\
     {\no} {say\impv} anything with our friends \\
\glt `Don't say anything to our friends.'
\end{xlist}
\ex
\begin{xlist}
\exi{L:}
\gll Smid. \\
     breath \\
\glt `Not a word.'\label{ex:smid}
\end{xlist}
\z
\z
\is{minimizer}
% \ex.  \ag. {J:\quad} Ná habair {a dhath} le n-ár gcairde.\\
%            {}   {\no} {say\impv} anything with our friends\\
%           `Don't say anything to our friends.'
%       \bg. {L:\quad} Smid.\\
%            {}   breath \\
%           `Not a word.'\label{ex:smid}

\ea\label{ex:minimizer2}
\ea
\gll Smid ní dhéarfaidh mé le n-ár gcairde. \\
     breath {\no} {say\fut} I with our friends \\
\glt `I won't breathe a word to our friends.'
\ex
\ea ní dhéarfaidh mé smid le n-ár gcairde.
\ex smid ní dhéarfaidh mé {\gapline} le n-ár gcairde.
\ex smid \elide{ní dhéarfaidh mé {\gapline} le n-ár gcairde}
\z
\z
\z
\is{minimizer}
% \ex. \ag. Smid ní dhéarfaidh mé le n-ár gcairde.\\
%           breath {\no} {say\fut} I with our friends\\
%          `I won't breathe a word to our friends.'
%      \b.  1. ní dhéarfaidh mé smid le n-ár gcairde.\\
%           2. smid ní dhéarfaidh mé {\gapline} le n-ár gcairde.\\
%           3. smid \elide{ní dhéarfaidh mé {\gapline} le n-ár gcairde}\ 

There is an additional set of observations that shows that the link between Narrative Fronting and legal fragment-types\is{fragment} is very tight.  There are items in Irish which resemble NPIs\is{negative polarity item} in many respects but to which Narrative Fronting may not apply. One of these is a focus\hyp exceptive construction, which is one of the ways of expressing {\itshape only}.  In these structures, which have much in common with the \textit{ne-que} construction of French, the exceptive particle {\itshape ach} attaches to a focused phrase (or to a larger phrase which includes a focused phrase) within the domain of sentential negation\is{negation}: 

\ea\label{ex:ach}
\ea
\gll Ní-or     labhair      ach seisear. \\
     {\nior}  {speak\past}  but {six.people} \\
\glt `Only six people spoke.'
\ex
\gll Ní ólaim ach tae. \\
     {\no} {drink\pres.\my} but tea \\
\glt `I drink only tea.'
\ex
\gll Ní     raibh    gluaisteán ach aige daoine saibhre an uair sin\\
    {\no} {be\past} car        but at people rich the time {\seo}\\
\glt `Only rich people had cars at that time.'\source{abfs}{26}
\z
\z

% \ex. \ag. Ní-or     labhair      ach seisear.\\
%           {\nior}  {speak\past}  but {six-people}\\
%          `Only six people spoke.'
%       \bg. Ní ólaim ach tae.\\
%           {\no} {drink\pres.\my} but tea\\
%           `I drink only tea.'
%       \cg. Ní     raibh    gluaisteán ach aige daoine saibhre an uair sin\\
%            {\no} {be\past} car        but at people rich the time {\seo}\\
%           `Only rich people had cars at that time.'\source{abfs}{26}

\noindent \textit{Ach}-phrases like those in (\ref{ex:ach}), interpreted like `only', are licensed in some, but not quite all, of the contexts in which NPIs\is{negative polarity item} are licensed. They appear, for instance, in the scope\is{scope} of negation\is{negation} or in polar questions\is{polar question}:

\ea
\ea
\gll nó an raibh ann acht rud a samhladh dó \\
     or {\interr} was in.it but thing {\go} {imagine\pastaut} to.him \\
\glt `Or was it only something he imagined?'\source{lcs}{110}
\ex
\gll An raibh ach an t-aon Naomh amháin {ina measc} go léir? \\
     {\interr} was but the one saint one {in-their-midst} all \\
\glt `Or was there only one Saint among them all?'\source{ag}{113}
\z
\z

% \ex.  \ag. nó an raibh ann acht rud a samhladh dó\\
%            or {\interr} was in.it but thing {\go} {imagine\pastaut} to.him\\
%           `Or was it only something he imagined?'\source{lcs}{110}
%       \bg. An raibh ach an t-aon Naomh amháin {ina measc} go léir?\\
%            {\interr} was but the one saint one {in-their-midst} all\\
%           `Or was there only one Saint among them all?'\source{ag}{113}

\noindent Despite their kinship with NPIs\is{negative polarity item} (discussed in \cite{mccloskey:13}) such exceptives are excluded from Narrative Fronting structures: 

\ea
\gll *Ach do dheartháir ní raibh ag an chruinniú aréir. \\
     but your brother {\no} {be\past} at the meeting {last.night} \\
\glt `Only your brother was at the meeting last night.'
\z

%  \exg. *Ach do dheartháir ní raibh ag an chruinniú aréir.\\
%         but your brother {\no} {be\past} at the meeting {last-night}\\
%        `Only your brother was at the meeting last night.'

\noindent They are also correspondingly impossible as fragments\is{fragment}, as shown in (\ref{ex:impossible}):

\ea\label{ex:impossible}
\ea
\gll Cé a bhí ag an chruinniú aréir? \\         
     who {\C} {be\past} at the meeting {last.night} \\
\glt `Who was at the meeting last night?'
\ex
\gll *Ach do dheartháir. \\
     but your brother \\
\glt `Only your brother.'
\z
\z

% \ex. \ag. Cé a bhí ag an chruinniú aréir?\\         
%            who {\C} {be\past} at the meeting {last-night}\\
%            `Who was at the meeting last night?'
%       \bg. *Ach do dheartháir.\\
%            but your brother\\
%            `Only your brother.'

\noindent Such close correlations are expected on the account developed here.\footnote{Speakers often use Narrative Fronting in providing paraphrases for fragment answers\is{fragment}.}

\subsection{English Redux}
\label{sec:english-redux}

Before we move on to larger questions, there is a final empirical issue that should be dealt with: this chapter opened by announcing the goal of better understanding a claimed contrast between English and Irish -- that Irish does, but English does not, allow fragment\is{fragment} NPIs\is{negative polarity item}. There is, though, a strand of research that questions the claim for English -- or at least suggests that the facts are more nuanced (\cite{marcel-et-al:00}; \cite{valmala:07}; \cite{weir-diss}; \cite{weir:15}). At issue is what we should conclude about the example in (\ref{ex:npi-frag-eng}):

\ea\label{ex:buy}
\textsc{q}: What \textsc{didn't} Owen buy?\quad\textsc{a}: Any wine.\label{ex:npi-frag-eng}
\z
\is{negative polarity item}
% \ex. \textsc{q}: What \textsc{didn't} Owen buy?\quad\textsc{a}: Any wine.\label{ex:npi-frag-eng}

\noindent As all investigators have been careful to note, (\ref{ex:npi-frag-eng}) is not accepted by all speakers and is, for many or most, of intermediate acceptability (I know of no quantitative study). The conditions that allow the fragment\is{fragment} NPI\is{negative polarity item} in (\ref{ex:npi-frag-eng}) seem also to be quite stringent; as noted by \citet[fn3, 44--45]{marcel-et-al:00} and \citet[167--171]{weir-diss}, what is required is a negative \WH-question\is{wh-question} with verum focus and a discourse context that includes, implicitly or explicitly, a set of propositions like that in (\ref{ex:pizza}): 

\ea\label{ex:pizza}
Owen bought pizza. \\
Owen bought bottled water. \\
Owen bought beer. \\
Owen bought chips.
\z

% \ex. Owen bought pizza.\\
%      Owen bought bottled water.\\
%      Owen bought beer.\\
%      Owen bought chips.

The possibility in (\ref{ex:buy}) also seems to be linked to the possibility of specificational pseudo-clefts in English like those in (\ref{ex:specificational}), in which the appearance of the NPI\is{negative polarity item} within the pivot position is also puzzling on most accounts.

\ea\label{ex:specificational}
\ea
What we didn't make was (we didn't make) any progress.
\ex
What Owen didn't buy was (he didn't buy) any wine.
\z
\z

% \ex. \a. What we didn't make was (we didn't make) any progress.
%      \b. What Owen didn't buy was (he didnt't buy) any wine.

It is exactly this connection that \citet{marcel-et-al:00} are centrally concerned with and in pursuing that connection they are brought to the conclusion that (\ref{ex:npi-frag-eng}) is an elided form of (\ref{ex:elided}): 

\ea\label{ex:elided}
\textsc{q}: What \textsc{didn't} Owen buy?\quad\textsc{a}: \elide{He didn't buy} any wine.
\z

% \ex. \textsc{q}: What \textsc{didn't} Owen buy?\quad\textsc{a}: \elide{He didn't buy} any wine.

The connection with the pseudoclefts in  (\ref{ex:specificational}) is then that the relation between the \WH-clause\is{wh-question} and the pivot in such constructions is similar in essential respects to the question-answer relation in (\ref{ex:npi-frag-eng}) and (\ref{ex:elided}). In both cases, the polarity item is licensed within a full finite clause that is subject to optional elision. Important support for this proposal derives from the generalization, which they establish, that bare NPI\is{negative polarity item} pivots are possible in pseudoclefts just in case the full clausal option is also possible (as illustrated in \ref{ex:specificational}).

Weir (\citeyear{weir-diss}; \citeyear{weir:15}) builds on these insights to develop an analysis that is consistent with the idea that fragment answers\is{fragment} are in general derived by movement followed by elision of all but the moved element. It also accounts for the marginal character of (\ref{ex:npi-frag-eng}) by assuming that the movement in question takes place in the derivation of phonological forms (and its effects are therefore invisible to semantic conditions such as those essential to the licensing of polarity items) and is a marked and `last resort' operation -- one that is specific to that context and one which applies only to head off the possibility that a focus marked expression might be elided.

These issues are difficult and important, but they clearly do not challenge the empirical claim that was the starting point for the present chapter: that Irish and English differ fundamentally with respect to the well-formedness of NPI fragments\is{negative polarity item}\is{fragment}. The Irish cases we have been concerned with are in no way marked or recondite; nor are they variably acceptable or restricted to very particular contexts, as is the English possibility in (\ref{ex:npi-frag-eng}). They are simply routine aspects of the grammar of the language: productive and unmarked.

The proposals developed here account for that property of Irish and are also compatible with the framework developed for the English cases by Weir and his predecessors. The difference between the two languages is that the grammar of Irish includes a syntactic operation that is routine and productive, and that renders appeal to marked or `last resort' operations unnecessary. The grammar of English, by contrast, includes no such operation and in providing a structural description for (\ref{ex:npi-frag-eng}) must rely on the logic of last resort.

\subsection{The Syntax of Negation and the Syntax of Narrative Fronting}
\label{sec:synneg}

For the announced purpose of this chapter (resolving the apparent anomaly of NPI fragments\is{negative polarity item})\is{fragment}, this is arguably as far as we need to go. It has been shown that the syntax of Irish must include a mechanism by which NPIs can be raised to a clause-peripheral position and that in that position raised expressions remain within the semantic scope\is{scope} of sentential negation\is{negation} and NPIs are therefore licensed.  That much is close to incontrovertible.  In addition, we need to appeal to an ellipsis process that elides the clause from which the NPI\is{negative polarity item} has been raised; but the process we must appeal to is of a kind that is familiar, well-studied, and well-attested.

There remain, though, obvious and large questions about how such a system can be integrated within some credible larger theory of Irish clausal syntax. I want to say something about those questions here, but the discussion will of necessity be somewhat unsatisfying because the issues that arise cannot be fully addressed within the scope of a chapter such as this.

Consider first the extended projection of the verb-initial finite clauses of the language. Sentential negation\is{negation} is expressed overtly in these clauses on \C; every candidate complementizer in the language (interrogative, \WH, conditional, root, default \ldots\ ) is paired with a negative counterpart with which it competes in the system of exponence (see \cite{mccloskey:01b} for data and discussion).

However, \citet{mccloskey:17} argues that the expression of negation\is{negation} in finite clauses in Irish is, in fact, distributed over two positions linked by an agreement relation: (i) on \C, the highest element of the extended clausal projection where its morphological exponence appears, and (ii) on a lower polarity head, one which appears in an arguably more expected position below the expression of clausal force and that is the position relevant for the semantics of polarity.

In finite clauses, the two polarity positions are separated by a head that has as its exponents the various preverbal tense markers of the language and that expresses a (limited) combination of tense and modality properties.  We will call that element here \textsc{tm1}.  The finite clause in (\ref{ex:tm1}), then, has the schematic structure in \figref{tree:vso}.

\ea \label{ex:tm1}
\gll Ní   -or    óladar             aon nimh     sa teach. \\
     {\C} {-\textsc{past}} {drink\past.\their} any poison in.the house {} \\
\glt `They did not drink any poison in the house.'
\z

\begin{figure}[!htb]  
\begin{forest}
             nice empty nodes,
             for tree={%
                     l sep=2em,
                     s sep=10mm,
                     inner sep=0,
                     calign=fixed edge angles,
                     calign angle=60,
                     l=0}
           [\CP [\lsel{c}{neg} [ {\itshape ní-}]]  [ \thip
                   [\thi [{\itshape -or}]] [ \polp
                         [ \lsel{pol}{neg} [\textit{ól-} \textit{-adar}, roof, name=pol]] [ \tlop
                           [\tlo,name=target]  
                           [\vP  [ \littlev, name=verb] [ 
                                     [{\itshape pro aon nimh sa teach}, roof]
                                        ]]]]]]
           {\draw[->]  (verb)   [out=south west] to [in=south] (target) ;}
           {\draw[->]  (target) [out=south] to [in=south] (pol) ;}
          \end{forest}
\caption{Tree for (\ref{ex:tm1})}
\label{tree:vso}
\end{figure}


\noindent The complement of that head is, in turn, projected by a lower tense-modality head, one which expresses a set of tense and modality distinctions that further refine those introduced by \textsc{tm1}.  This head (\textsc{tm2}) is realized morphologically in the post-verbal inflectional suffixes and hosts a raised subject in its specifier, yielding \vso\ order.

In nonfinite clauses, which are characterized by an absence of head-movement, the lower polarity head hosts the overt marker of negation\is{negation} (\textit{gan}), which clearly appears in a position lower than \C, and which precedes subject-position:

\ea
\gll Ba mhian leat gan mé creidbheáil ins an rud. \\
    {\ba} desire with.you {\gan} me {believe\vn} in the thing \\
\glt `You wanted me not to believe in the thing.'\source{umi}{167}
\z

% \exg. Ba mhian leat gan mé creidbheáil ins an rud.\\
%       {\ba} desire with.you {\gan} me {believe\vn} in the thing\\
%      `You wanted me not to believe in the thing.'\source{umi}{167}

How might Narrative Fronting be integrated into this framework? The point reached earlier is summarized in (\ref{ex:narr.front}), repeated from (\ref{ex:nfc}) above.

\ea\label{ex:narr.front}
{\scshape narrative fronting}: \\
The finite negation\is{negation} head may include a probe that attracts XPs that are alternative-evoking\is{alternative-evoking}.
\z

% \ex.  {\scshape narrative fronting}:\\
%       The finite negation head may include a probe which attracts
%       XPs which are alternative-evoking.

We can begin by assuming that there is a feature \textsc{alt} that identifies alternative-evoking\is{alternative-evoking} expressions to the syntax. This feature, I assume, is part of the lexical specification of NPIs, one of the ways in which the pure numeral use of \textit{aon}, for instance, is distinguished from its NPI\is{negative polarity item} use. Narrative Fronting, then, will be understood as a probe-goal interaction between a negative head and a phrase in its local domain which bears the \textsc{alt}-feature; in that interaction, the target phrase is raised to the specifier position of the polarity head. 

But given the structure in \figref{tree:vso}, we now have two candidate heads to consider when asking which head hosts the crucial probe: the higher negative \C\ or the lower polarity head. Given that in the overt structure, the raised XP appears to the left of the negative complementizer, the answer to that question would seem to be obvious. One might think that the \textsc{alt}-probe is an optional subpart of the negative complementizer and that Narrative Fronting is, therefore, movement to its specifier position.

Alternatively, it could be that the probe that attracts the alternative\hyp evoking\is{alternative-evoking} expression into its specifier position is hosted by the lower (negative) polarity head.  Following raising of the finite verb to the polarity position, the fronted XP will appear to the left of the inflected verb and niched between the two (linked) expressions of negation\is{negation}. (\ref{ex:two.neg}) would then have the syntax shown in \figref{tree:nf}.

\ea \label{ex:two.neg}
\gll Focal amháin féin ní-or labhair ceachtar acu.\\
 word one {even}  {\nior} {speak\past} either at.them\\
\glt `Not even a single word did either of them speak.'\source{init}{164}
\z

\begin{figure}[!htb] 
\begin{forest}
nice empty nodes,
            for tree={s sep=0.5mm, inner sep=0.5mm}
          [\CP
               [\lsel{c}{neg} [ {\itshape ní-}]] [\thip[\thi [{\itshape -or}]]
                      [\polp
                        [\lsel{dp}{alt}[{\itshape focal amháin féin}, roof, name=tgt]]
                          [[\lsel{pol}{\xover{neg}{agr:alt}},s sep=1mm 
                           [\lsel{v}{fin} [ {\itshape labhair}]]]
                             [\tlop
                               [ {\itshape ceachtar acu \\ \elide{\itshape focal amháin féin}}, roof,name=src]]
                                                  ]]]]]
          %\draw[->] (src) to[out=south:5em,in=south:10em,] (tgt);
          \draw[->, rounded corners] (src.south) |- ++(-2cm,-1.5cm) -| (tgt);
\end{forest}
\caption{Tree for (\ref{ex:two.neg})}
\label{tree:nf}
\end{figure}



\noindent The first option (raising to the specifier of negative C) accounts immediately for the observed order of constituents: the raised XP in Narrative Fronting always precedes the expression of finite negation\is{negation}. But it is problematic in important ways as well. In particular, this analysis defines Narrative Fronting as a movement to the clause-edge, thus risking the incorrect prediction that the operation would have the same locality-profile as the much-studied {\Abar}-movements of the language. It does not, though. As shown in \citet{mccloskey:96a}, Narrative Fronting is movement to a \TP-internal position and is strictly clause-bound.

In addition, the observed word-order in cases like (\ref{ex:two.neg}) is guaranteed by the proposal developed in \citet{mccloskey:96a} that finite complementizers in general in Irish attain their pronounced positions by way of a postsyntactic operation that lowers them, across intervening material if present, to the position of the inflected verb. On this view, \C-lowering is an exact analog of English \T-lowering (the `affix-hopping' of \cite{chomsky:57}) and is justified by exactly the same kind of argument. In the case of Irish, \C-lowering is one of a set of post-syntactic operations which jointly create the complex morphological word known in the Irish linguistic tradition as the `verbal complex', whose properties and intricate internal structure have been studied, for example, by Jason Ostrove (\citeyear{ostrove:18}). In the case of (\ref{ex:two.neg}), the effect is to create a complex morphological word of the form in (\ref{ex:complex.morph}) in the position of the polarity head:

\ea\label{ex:complex.morph}
\{\ {\scshape c-neg}$\frown$\textsc{tm1}$\frown$\textsc{root}$\frown${\scshape vce}$\frown${\scshape tm2}  \}
\z

% \ex. \{\ {\scshape c-neg}$\frown$\textsc{tm1}$\frown$\textsc{root}$\frown${\scshape vce}$\frown${\scshape tm2}  \}

In the context of the \C-lowering proposal, the analysis in \figref{tree:nf} becomes natural. On this account, the fronted XP is commanded by an expression of negation\is{negation} at every point in its derivational career and standard (and natural) mechanisms provide an understanding of the fact that Narrative Fronting applies only in negative clauses (an important advance over the account offered in \cite{mccloskey:96a}), while avoiding the risk of predicting that the operation would have the locality-profile of an \Abar-movement.

The proposal of \C-lowering has been controversial (see for instance \cite{maki-obaoill:17b}, who show that one of the strands of evidence offered in \cite{mccloskey:96a} was based on a misinterpretation of the relevant evidence), but I know of no competitor proposal that deals with the range of facts it accounts for.  In addition, there is no reason for embarrassment in a contemporary theoretical setting concerning appeal to a post-syntactic (as opposed to syntactic) lowering.
 
In the context of \figref{tree:nf}, the ellipsis process appealed to in accounting for the fragment answers\is{fragment} of, for instance, (\ref{ex:npiwh}), (\ref{ex:npipolar}) or (\ref{ex:smid}) -- one which will eliminate all but the phrase fronted by Narrative Fronting -- will emerge as a familiar kind of polarity ellipsis\is{polarity ellipsis} triggered by an ellipsis-licensing feature (Merchant's \textsc{e}) on the lower polarity head.\footnote{For an interesting comparison see especially \citet{gribanova:17} on polarity ellipses\is{polarity ellipsis} in Russian.}  Further,  following \citet{pesetsky-benjabi:22}, I will assume that in this case the \textsc{e}-feature extends to the immediate projection of the polarity head, ensuring elimination from the pronounced string of all but the specifier of the licensing head.

Much hard work clearly remains to be done on all of these questions. But I hope to have shown in this brief discussion that the project of integrating the proposals of this chapter into a credible larger theory of clause structure (in Irish) is far from being a hopeless one.
\is{narrative fronting|)}

\section{Implications}
\label{sec:implications}

The proposal outlined here (and anticipated in Merchant's paper) depends on a central element of the `move and delete' approach articulated by Merchant (\citeyear{merchant:04}) for subsentential fragments of propositional type\is{fragment}.  That approach links the ellipsis possibilities found in a given language with the inventory of movement-types available in that language. That inventory in turn reflects the inventory of probes in the language; so we have yet another case in which variation among languages has its roots in combinatorial properties of elements of the functional vocabulary. In the absence of such an approach, it is not clear how to make the required connection between the possibility of fragment\is{fragment} NPIs\is{negative polarity item} and the existence and properties of Narrative Fronting.

The analysis also preserves the essence of the generalization that NPIs\is{negative polarity item} may not function as fragment\is{fragment} answers. The possibility of NPI fragments will emerge in a language only if it has a movement operation that is not scope-enhancing\is{scope} and so does not raise the moved item out of the crucial licensing context.  Irish does not show that the initial generalization is wrong, then, but rather suggests a refinement.\footnote{The implication is, obviously, that in diagnosing some constituent as either an NPI\is{negative polarity item} or a negative concord item, one should not use the fragment answer\is{fragment} diagnostic in a simplistic way. Before concluding that an element is a negative concord item on the basis of the test, one needs to ask if the language has an independently available mechanism that could displace an NPI\is{negative polarity item} out of a potential ellipsis-site, yielding a pattern like the one discussed here for Irish.} This clarification too depends in a fundamental way on the move-and-delete approach to this kind of fragment response\is{fragment}.

Most important, perhaps, the arguments supporting an ellipsis-based analysis of at least this type of fragment response\is{fragment} are quite powerful.  Few expression-types are as subtly dependent on context as are NPIs\is{negative polarity item}; their interpretation and their well-formedness depend on very particular properties of the compositional settings in which they appear.  And even those who have pushed hardest to construct a fundamentally semantic or pragmatic understanding of how NPIs\is{negative polarity item} are licensed and interpreted recognize that the definition of that compositional setting has a syntactic component (see, for example \cite[206--207]{ladusaw:79}). \citet{gajewski:05}, \citet{homer:11}, and \citet{homer:21a}, in particular, have developed persuasive arguments that even the fundamental semantic constraint -- that NPIs\is{negative polarity item} must appear in (Strawson) downward-entailing environments -- must be stated in terms both syntactic and semantic, as in (\ref{ex:downward}):

\ea\label{ex:downward}
An NPI\is{negative polarity item} $\alpha$ is licensed in sentence $S$ only if there is a (syntactic) constituent $A$ of $S$ containing $\alpha$ such that $A$ is downward-entailing with respect to the position of $\alpha$. (\cite[5]{homer:21a})
\z

% \ex. An NPI $\alpha$ is licensed in sentence $S$ only if there is a (syntactic) constituent
%      $A$ of $S$ containing $\alpha$ such that $A$ is downward-entailing with respect to the position
%      of $\alpha$.\\\null\hfill\citet[\ 5]{homer:21a}

\noindent That is, syntactic constituents (rather than `operators') are downward or upward entailing (in virtue, of course, of their interpretive properties). In turn, what this means is that the minimizer\is{minimizer} fragment\is{fragment} {\itshape smid} in the exchange in (\ref{ex:smid}) above, if it is to be appropriately interpreted and licensed, must appear within a syntactic constituent whose semantic properties are such that it is (Strawson) downward entailing  with respect to the position of {\itshape smid}.  This is guaranteed in a straightforward way by the ellipsis analysis, because its principal commitment is exactly that the minimizer\is{minimizer} in (\ref{ex:smid}) is contained within an interpreted syntactic structure of the required kind -- though one that happens not to be pronounced. It is difficult to imagine how the required compositional scaffolding (semantic and syntactic) might be supplied in an approach to fragments\is{fragment} that eschewed such silent structure. 

But if appeal to ellipsis is correct for this material, there are also implications for what the theory of ellipsis must then look like. Consider again some examples of NPI fragment\is{fragment} responses\is{negative polarity item}:

\ea\label{ex:sod}
\ea
\gll “Beidh na girseachaí leat.” “Ó {go bráthach!}”\\
     {be\fut} the girls with.you {} ever \\
\glt `“The girls will be with you.”{\quad} “Oh, never!”'\source{atfs}{505}
\ex
\gll “Cá   mhéad   a     bhéas    ar sin?” “Pingin {ar bith},” arsa bean a' tsiopa. \\
     what amount {\go} {be\pres} on that  {\ \ penny} {\ any}  said woman the {shop\gen} \\
\glt `“How much will that be?”  “Not a penny,” said the shop-woman.' \source{brd}{81}
\ex
\gll Cé a d' inis duitse go bhfuil tú ag déanamh mar is ceart? Aon duine. \\
     who {\aLgloss}  {\did} {tell\past} to.you {\go} {be\pres} you {\prog} {do\vn} as is right one person\\
\glt `Who told you that you were doing the right thing?' `Nobody.' \source{isnb}{139}
\ex
\gll “Beidh dornán maith mónadh de dhíobháil ortha le teinte a choinneáil ann.” {\quad} “Fód {ar bith}.”\\
     {be\fut} quantity good {peat\gen} of need on.them to fires {\vce} {keep\vn} in.it {} {\ sod} any \\
\glt `“They'll need a lot of turf (peat) to keep the fires going in it.”  “Not a sod.”'\source{st}{206}
\z
\z

% \ex.  \ag.  `Beidh na girseachaí leat.' `Ó {go bráthach!}'\\
%              {be\fut} the girls with.you {} ever \\
%             `The girls will be with you.'{\quad} `Oh, never!'\source{atfs}{505}
%       \bg.  `Cá   mhéad   a     bhéas    ar sin?' `Pingin {ar bith},' arsa bean a' tsiopa.\\
%              what amount {\go} {be\pres} on that  {\ \ penny} {\ any}  said woman the {shop\gen}\\
%             `How much will that be?' {\quad} `Not a penny,' said the shop-woman.\\ \source{brd}{81}
%       \cg.   Cé a d' inis duitse go bhfuil tú ag déanamh mar is ceart? Aon duine.\\
%              who {\aLgloss}  {\did} {tell\past} to.you {\go} {be\pres} you {\prog} {do\vn} as is right
%              one person\\
%             `Who told you that you were doing the right thing?' `Nobody.'\\ \source{isnb}{139}
%       \dg.  `Beidh dornán maith mónadh de dhíobháil ortha le teinte a choinneáil ann.' {\quad} `Fód {ar bith}.'\\
%             {be\fut} quantity good {peat\gen} of need on.them to fires {\vce} {keep\vn} in.it {} {\ sod} any\\
%            `They'll need a lot of turf (peat) to keep the fires going in it.'\\ `Not a sod.'\source{st}{206}
            
\noindent For each example in (\ref{ex:sod}), there is a very natural paraphrase in terms of Narrative Fronting:

\ea
\ea
Go bráthach \elide{ní bheidh na girseachaí liom}. \\
`Never will the girls be with me.'
\ex
Pingin ar bith \elide{ní bheidh ar sin}. \\
`That won't cost a penny.' 
\ex
Aon duine \elide{níor inis domh go raibh mé ag déanamh mar is ceart}. \\
`Not one person told me that I was doing the right thing.'
\ex
Fód ar bith \elide{ní bheidh de dhiobháil orthu}. \\
`Not one sod will they need.'
\z
\z

% \ex.  \a. Go bráthach \elide{ní bheidh na girseachaí liom}.\\
%          `Never will the girls be with me.'
%       \b. Pingin ar bith \elide{ní bheidh ar sin}.\\
%          `That won't cost a penny.' 
%       \c. Aon duine \elide{níor inis domh go raibh mé ag déanamh mar is ceart}.\\
%          `Not one person told me that I was doing the right thing.'
%       \d. Fód ar bith \elide{ní bheidh de dhiobháil orthu}.\\
%          `Not one sod will they need.'

\noindent In each case, the elided clause must include sentential negation\is{negation} to license the stranded polarity item and to guarantee the right interpretation. However, no matching negation\is{negation} appears in the apparent antecedent.  Such examples are very frequent (see also \ref{ex:goodfrag}, all four in \ref{ex:npiwh}, and two in \ref{ex:npipolar} above) and they add to the steady accumulation of evidence in recent years that elided clauses are not required to match their antecedents in polarity (\cite{yoshida:10}; \cite{maziar:13}; \cite{deniz-margaret}; \cite{rudin:18}; \cite{kroll:18}; \cite{ahm:20}; \cite{ranero:21}). Further, these examples are of a particular type: negation\is{negation} appears in the elided clause but not in its antecedent. This is worth noting because in work on sluicing in English it has been observed (\cite{ahm:20}) that this pattern is rarer than its inverse (negation\is{negation} in the antecedent, no negation\is{negation} in the ellipsis-site).

The issues around antecedence do not end here though.  In question-response pairs, it is usually easy to identify an appropriate overt antecedent. However, NPI fragments\is{negative polarity item}\is{fragment} also occur in contexts in which there is no such antecedent:

\ea\label{ex:six}
\ea
\gll Bhain sé triail as an uimhir fóin. Freagra {ar bith}. \\
    {took\past} he try out.of the number {phone\gen} answer any \\
\glt `He tried the phone number. No answer.'\source{tair}{114}
\ex
\gll Chuaigh mé amach {a dh'éisteacht} le ceol na n-éan. Fuaim dá laghad. \\
     {go\past} I out  {to-listen} with music {the\gen} {birds\gen} sound {of.the} least \\
\glt `I went out to listen to the singing of the birds. Not a sound.' \source{rng}{20-10-18}
\ex
\gll Níl aon uaigh ann, a Dheaid. Chuartaigh muid an áit ó bhun go barr. Tada. \\
     {is.not} one grave in.it {\voca} Dad {search\past} we the place from bottom to top anything \\
\glt `There's no grave, Dad. We searched the place from top to bottom.\\ Nothing.'\source{a}{98}
\z
\z

% \ex.  \ag.  Bhain sé triail as an uimhir fóin. Freagra {ar bith}.\\
%             {took\past} he try out-of the number {phone\gen} answer any\\
%             `He tried the phone number. No answer.'\source{tair}{114}
%       \bg.  Chuaigh mé amach {a dh'éisteacht} le ceol na n-éan. Fuaim dá laghad.\\
%             {go\past} I out  {to-listen} with music {the\gen} {birds\gen} sound {of-the} least\\
%            `I went out to listen to the singing of the birds. Not a sound.'\\ \source{rng}{20-10-18}
%       \cg.  Níl aon uaigh ann, a Dheaid. Chuartaigh muid an áit ó bhun go barr. Tada.\\
%             {is.not} one grave in.it {\voca} Dad {search\past} we the place from bottom to top anything\\
%             `There's no grave, Dad. We searched the place from top to bottom.\\ Nothing.'\source{a}{98}

These are all attested examples. (\ref{ex:six}a) was also checked with six native speaker consultants, all of whom accepted it without hesitation as natural, well-formed, and clear. The crucial property of such cases is that, although the evidence for ellipsis is as clear and as strong as for the other instances of NPI fragments\is{negative polarity item}\is{fragment}, there is no overt antecedent in the discourse context. However, there is in each case a strongly salient but implicit question (a {\scshape qud} in the sense of \cite{roberts:12}) -- {\itshape Would he get an answer?} {\itshape What would he hear?} {\itshape What did they find?} -- in the local discourse context.

Given how strong the evidence for ellipsis is in such fragment answer cases\is{fragment}, it seems we must conclude (with \cite[Section 5]{merchant:04}) that, for certain kinds of ellipsis at least, antecedents with the necessary kind of syntactic and semantic properties may be found in discourse representations rather than in any overt linguistic signal. This conclusion runs against the grain of certain trends in the study of ellipsis, some of which tend to assume that if there is no overt antecedent there can be no “syntax in the silence” (to use Merchant's phrase). But the conclusion is, in a certain sense, profoundly unsurprising. The initiators of one of the most sophisticated and influential frameworks for the study of discourse dynamics \citet{farkas-bruce:10} take the following position:

\begin{quote}
We follow the literature \ldots\ in having a discourse component that records the questions under discussion \ldots\ and assume that the items on it are syntactic objects paired with their denotations. ({\cite[86]{farkas-bruce:10})}
\end{quote}  

\noindent The reason that they adopt this position is exactly that “the grammar of cross-turn conversation and ellipsis has to have access to the grammatical form (and not just the content) of immediately previous utterances” \citep[86]{farkas-bruce:10}.

\section{Conclusion}
\label{sec:finit}

Working on ellipsis is very difficult and can feel frustrating. The problems are unimaginably more challenging than they seemed to be fifty years ago and it is easy to yield sometimes to the feeling that little theoretical progress has been made in the interim, even though there are uncountably many more observations available to us now that might shape theory\hyp development.

Part of the difficulty is that it remains unclear for what phenomena the general theory of ellipsis is supposed to be responsible. The case of subsentential fragments\is{fragment} is a classic one of that type; the exchange between \citet{stainton:06} and \citet{merchant:10} in particular makes that very clear. I take the present chapter to be a modest contribution to that difficult and important area.

However, working on ellipsis also often brings useful spin-offs. In the present case, it has led to a more complete survey of the landscape of polarity\hyp sensitive items in Irish than has been available before. It has also led to a better understanding of Narrative Fronting~-- one of the more intriguing components of the syntax-pragmatics interface in Irish~-- than has been available before. 

And finally the chapter has also, I hope, provided a clearer view of what the extended projection of finite clauses in Irish looks like.  Research on extended projections and research on ellipsis are deeply entangled because the correct theory of the extended projections for a given language must provide a basis for understanding what the inventory of ellipsis processes in that language are.

% %
% %
% %%%%%%%%%%%%%%%%%%%%%%%%%%%%%%%%%%%%%%%%%%%%%%%%%%%%%%%%%%%%%%%%%%%%%%%%%%%%%%%%%%%%%%%%%%
% %%
% %% END OF MAIN TEXT
% %%
% %%%%%%%%%%%%%%%%%%%%%%%%%%%%%%%%%%%%%%%%%%%%%%%%%%%%%%%%%%%%%%%%%%%%%%%%%%%%%%%%%%%%%%%%%%
\section*{Acknowledgements}

This chapter has its origins in conversations with Jason Merchant some twenty years ago.  In bringing it finally to completion I owe an even larger debt of gratitude than usual to the friends and colleagues who have acted as my linguistic guides: Caitlín Nic Niallais, Lillis Ó Laoire, Máire Ní Neachtain, Pádhraic Ó Ciardha, Róise Ní Bhaoill and Seosaimhín Ní Bheaglaoich. I am also grateful for the advice, discussion, and suggestions of Paolo Acquaviva, Pranav Anand, Liam Breatnach, Lisa Cheng, Sandy Chung, Nicola D'Antuono, Vera Gribanova, Bill Ladusaw, Anikó Liptak, Gillian Ramchand, Ivy Sichel and Gary Thoms.  Presentations at UC Santa Cruz in November 2019 and at Leiden University in January of 2020 were very helpful in pushing the project along, as were comments and suggestions made by an anonymous reviewer. This research was supported by funding from the National Science Foundation via Award  Number 1451819:\textit{The Implicit Content of Sluicing} (principal investigators Pranav Anand, Daniel Hardt and James McCloskey) to the University of California Santa Cruz.



%%%%%%%%%%%%%%%%%%%%%%%%%%%%%%%%%%%%%%%%%%%%%%%%%%%%%%%%%%%%%%%%%%%%%%%%%%%%%%%%%%%%%%%%%%%%%%%%%%%%
% %%
% %% FIRST APPENDIX
% %%

\appendixsection{The licensing environments}% Appendix 1

This appendix presents data concerning the range of licensing environments in which the polarity\hyp sensitive expressions discussed in \sectref{sec:npi} may appear.

\appendixsubsection{In questions}

All of the PSEs of \sectref{sec:npi} appear in polar questions\is{polar question}, root and embedded:

\ea
\textsc{embedded polar questions}:
\ea
\gll féachaint an samhlódh faic dom \\
     {to-see-if} {\interr} {imagine\imp} anything/nothing {to.me} \\
\glt `to see if anything would spring to mind for me'\source{agmts}{3}
\ex
\gll go bhfeicfeadh muid an raibh tada beo {tar éis} na hóiche \\
     {\C} {see\cond} we  {\interr} {be\past} anything alive after the night \\
\glt `so that we could see if anything was alive after the night'\source{imsbrm}{55}
\z
\z
\is{polar question}

% \ex.  \textsc{embedded polar questions}:
%       \ag. féachaint an samhlódh faic dom\\
%           {to-see-if} {\interr} {imagine\imp} anything/nothing {to.me}\\
%          `to see if anything would spring to mind for me'\source{agmts}{3}
%      \bg. go bhfeicfeadh muid an raibh tada beo {tar éis} na hóiche\\
%           {\C} {see\cond} we  {\interr} {be\past} anything alive after the night\\
%          `so that we could see if anything was alive after the night'\source{imsbrm}{55}

\ea
\textsc{root polar questions}:
\ea
\gll An féidir liom tada {a fháil} duit? \\
     {\interr} possible with.me anything {get.\textsc{nfin}} {for.you} \\
\glt `Can I get anything for you?'\source{dead}{149}
\ex
\gll Agus a-r ghnóthaigh sí aon rás eile {i mbliana}? \\
     and  {\interr-\past}  {win\past} she one race other this-year \\
\glt `And did she win any other race this year?''\source{rng}{26-08-19}
\z
\z

% \ex. \textsc{root polar questions}:
%      \ag. An féidir liom tada {a fháil} duit?\\
%           {\interr} possible with.me anything {get.\textsc{nfin}} {for.you}\\
%          `Can I get anything for you?'\source{dead}{149}
%      \bg. Agus a-r ghnóthaigh sí aon rás eile {i mbliana}?\\
%           and  {\interr-\past}  {win\past} she one race other this-year\\
%          `And did she win any other race this year?''\source{rng}{26-08-19}

The issue of under what conditions NPIs\is{negative polarity item} are licensed in \Wh-questions\is{wh-question} has been important in debates about the nature of NPI-licensing\is{negative polarity item} \parencite{giannakidou:99, giannakidou:11, guerzoni-sharvit:07, mayr:13}. The PSEs we care about here appear in such questions, root and embedded: 


\ea\label{ex:wh.q}
\textsc{wh-questions}:
\ea
\gll Cé eile a bhfuil fhios aige tada faoi seo? \\
     who other {\go} {be\pres} knowledge at.him anything about this \\
\glt `Who else knows anything about this?'\source{cab}{89}
\ex
\gll {Goidé mar} a thiocfadh liom aon fhear a phósadh? \\
     how    {\C} {come\cond} with.me any man {\vce}  {marry\vn} \\
\glt `How could I marry any man?'\source{am}{162}
\ex
\gll Cén chaoi a mbeadh fhios acu tada? \\
     what way {\go} {be\cond} knowledge at.them anything \\
\glt `How would they know anything?'\source{smc}{304}
\ex
\gll Cé a chreidfeadh go raibh sé ariamh bródúil? \\
     who {\go} {believe\cond} {\go} {be\past} he ever proud \\
\glt `Who would believe that he was ever proud?'\source{atfs}{51}
\z
\z
\is{wh-question}

% \ex.   \textsc{wh-questions}:
%       \ag. cé eile a bhfuil fhios aaige tada faoi seo?\\
%            who other {\go} {be\pres} knowledge at.him anything about this\\
%           `Who else knows anything about this?'\source{cab}{89}
%       \bg. {goidé mar} a thiocfadh liom aon fhear a phósadh?\\
%             how    {\C} {come\cond} with.me any man {\vce}  {marry\vn}\\
%            `How could I marry any man?'\source{am}{162}
%       \cg. cén chaoi a mbeadh fhios acu tada?\\
%            what way {\go} {be\cond} knowledge at.them anything\\
%           `How would they know anything?'\source{smc}{304}
%       \dg. cé a chreidfeadh go raibh sé ariamh bródúil?\\
%            who {\go} {believe\cond} {\go} {be\past} he ever proud\\
%           `Who would believe that he was ever proud?'\source{atfs}{51}

\noindent Many of these cases involve rhetorical questions, e.g. (\ref{ex:wh.q}b--d). Others are infor\-mat\-ion-seeking questions which require exhaustive answers, as in (\ref{ex:wh.q}a), a pattern which is consistent in particular with the work of \citet{guerzoni-sharvit:07} and \citet{mayr:13}.  Whether or not this pattern holds consistently is something which remains to be investigated.

\appendixsubsection{In conditional clauses}
\is{conditional clause|(}

The PSEs of \sectref{sec:npi} may also appear, with their characteristic existential interpretation, in conditional clauses, both realis (\ref{ex:realis}a) and irrealis (\ref{ex:realis}b,c):

\ea\label{ex:realis}
\ea
\gll Má chaitheann tú choíchín tada {a rá}, abair {go cuí} é.\\ % if you have to say something
    if {must\pres} you ever anything {say.\textsc{nfin}} {say\impv} appropriately it\\
\glt `If you ever have to say anything, say it appropriately.'\source{ae}{62}
\ex
\gll Dá mbeadh baint {ar bith} leis na gnóithe agam. \\
    if {be\cond} connection any  with the business at.me \\
\glt `if I had any say in the matter'\source{clm}{38}
\ex
\gll dá dtarlaíodh tada dá hathair \\
     if {happen\cond} anything {to.her} father \\
\glt `if anything should happen to her father'\source{aa}{177}
\z
\z
\is{conditional clause|)}
% \ex. \ag. má chaitheann tú choíchín tada {a rá}, abair {go cuí} é\\ % if you have to say something
%           if {must\pres} you ever anything {say.\textsc{nfin}} {say\impv} appropriately it\\
%          `if you ever have to say anything, say it appropriately'\source{ae}{62}
%      \bg. dá mbeadh baint {ar bith} leis na gnóithe agam\\
%           if {be\cond} connection any  with the business at.me\\
%          `if I had any say in the matter'\source{clm}{38}
%      \cg. dá dtarlaíodh tada dá hathair\\
%           if {happen\cond} anything {to.her} father\\    
%          `if anything should happen to her father'\source{aa}{177}

\appendixsubsection{In excessive-degree clauses}

Phrases in construction with the degree-word {\itshape ró}- (`too') also freely host PSEs:

\ea
\ea
\gll Tá sé ró- mhall anois tada a dhéanamh. \\
     {be\pres} it too late now anything {\vce} {do\vn} \\
\glt `It's too late to do anything now.'\source{aa}{158}
\ex
\gll bhí sé ró- the chun éinne bheith ag spaisteoireacht \\ % ar na failltreacha\\
     {be\past} it too hot for anyone {be.\textsc{nfin}} {\prog} {stroll\vn} \\  %on the cliffs\\
\glt `It was too hot for anyone to be out walking.'\source{dpb}{102}
\ex
\gll duí gainmhe a bhí ró- aimhréidh d' aon ghalfchúrsa \\
     dune {sand\gen} {\go} {be\past} too uneven for any golf-course \\
\glt `a sand-dune that was too uneven for any golf-course'\source{lgl}{170}
\z
\z

% \ex. \ag. Tá sé ró- mhall anois tada a dhéanamh.\\
%           {be\pres} it too late now anything {\vce} {do\vn}\\
%          `It's too late to do anything now,'\source{aa}{158}
%      \bg. bhí sé ró- the chun éinne bheith ag spaisteoireacht\\ % ar na failltreacha\\
%           {be\past} it too hot for anyone {be.\textsc{nfin}} {\prog} {stroll\vn}\\  %on the cliffs\\
%          `It was too hot for anyone to be out walking.'\source{dpb}{102}
%      \cg. duí gainmhe a bhí ró- aimhréidh d' aon ghalfchúrsa\\
%           dune {sand\gen} {\go} {be\past} too uneven for any golf-course\\
%          `a sand-dune that was too uneven for any golf-course'\source{lgl}{170}

\appendixsubsection{In equative, comparative and superlative clauses}
\is{comparative clause|(}
\ea
\textsc{equative clauses}:
\ea
\gll Bhí cuma air comh folláin agus bhí ariamh air, \\
    {be\past} appearance on.him as healthy and {be\past} ever on.him \\
\glt `He looks as healthy as he ever did.'\source{atfs}{565}
\ex
\gll culaith éadaigh chomh deas agus a chuir aon fhear riamh ar a dhroim \\
     suit {clothes\gen} as nice as {\aLgloss} {put\past} any man ever on his back \\
\glt `as nice a suit of clothes as any man ever put on his back'\source{empp}{163}
\z
\z

% \ex. \textsc{equative clauses}:
%      \ag. Bhí cuma air comh folláin agus bhí ariamh air\\
%          {be\past} appearance on.him as healthy and {be\past} ever on.him\\
%          `He looks as healthy as he ever did.'\source{atfs}{565}
%      \bg. culaith éadaigh chomh deas agus a chuir aon fhear riamh ar a dhroim\\
%           suit {clothes\gen} as nice as {\aLgloss} {put\past} any man ever on his back\\
%          `as nice a suit of clothes as any man ever put on his back'\source{empp}{163}
\newpage
\ea
\textsc{comparative clauses}:
\ea
\gll Tá {níos mó} chéill i gcuid cainnte Dhomhnaill ná aidmhigheas aon duine. \\
    {be\pres} more {sense\gen} in share {talk\gen} {} than {admit\presrel} any person \\
\glt `There's more sense in Domhnall's talk than anyone admits.'\source{mo}{135}
\ex
\gll Bhí fhios aige-sean níos fearr ná bhí fhios ag duine {ar bith.} \ldots\ \\
     {be\past} knowledge {at.him-\contr} {more} better than {be\past} knowledge at person any {} \\
\glt `He knew better than anyone knew that \ldots\ '\source{am}{415}
\ex
\gll Is fusa éinne a smachtú ná do chuid féin. \\
     {\cop} {easy\er} anyone {\vce} {discipline\vn} than your portion {\fein} \\
\glt `It's easier to discipline anyone than your own (children).'\source{lgl}{122}
\z
\z

% \ex. \textsc{comparative clauses}:
%          \ag. Tá {níos mó} chéill i gcuid cainnte Dhomhnaill ná aidmhigheas aon duine.\\
%          {be\pres} more {sense\gen} in share {talk\gen} {} than {admit\presrel} any person\\
%          `There's more sense in Domhnall's talk than anyone admits.'\source{mo}{135}
%      \bg. Bhí fhios aige-sean níos fearr ná bhí fhios ag duine {ar bith} \ldots\ \\
%           {be\past} knowledge {at.him-\contr} {more} better than {be\past} knowledge at person any {}\\
%          `He knew better than anyone knew that \ldots\ '\source{am}{415}
%      \cg. is fusa éinne a smachtú ná do chuid féin\\
%           {\cop} {easy\er} anyone {\vce} {discipline\vn} than your portion {\fein}\\
%          `It's easier to discipline anyone than your own (children).'\source{lgl}{122}

\ea
\textsc{superlative clauses}:
\ea
\gll an obair thógála ba deacra dár déanadh {go hiomlán} as Gaeilge ariamh \\
     the work {building\gen} {\cop} {difficult\er} {\go} {do\pastaut} entirely out.of Irish ever \\
\glt `the most difficult building-work that was ever done entirely through Irish'\source{cct}{199}
\ex
\gll an mheancóg a ba mhó a rinne mé ariamh \\
     the mistake {\aLgloss} {\ba} {big\er} {\aLgloss} {make\past} I ever \\
\glt `the biggest mistake that I ever made'\source{atfs}{127}
\z
\z

% \ex. \textsc{superlative clauses}:
%          \ag. an obair thógála ba deacra dár déanadh {go hiomlán} as Gaeilge ariamh \\
%           the work {building\gen} {\cop} {difficult\er} {\go} {do\pastaut} entirely out-of Irish ever\\
%          `the most difficult building-work that was ever done entirely through Irish'\source{cct}{199}
%      \bg. an mheancóg a ba mhó a rinne mé ariamh\\
%           the mistake {\aLgloss} {\ba} {big\er} {\aLgloss} {make\past} I ever\\
%          `the biggest mistake that I ever made'\source{atfs}{127}

\noindent See \citet{hoeksema:83}, \citet{von-fintel:99}, \citet{gajewski:10}, \citet{sharvit-herdan:06}, \citet{bumford-sharvit:22}, and \citet{howard:14}. 

I will include in this group relative clauses attached to head nouns modified by the ordinal \textit{céad} (`first'), though the status of such phrases as superlatives remains controversial. PSEs are licensed here:

\ea
\ea
\gll Siod é an chéad am a labhair mé air le aon duine \\
     that it the first time {\C} {speak\past} I on.it with any person \\
\glt `That was the first time that I spoke about it with anyone.'\source{smc}{287}
\ex
\gll an chéad ál a bhí ariamh aici \\
     the first litter {\C} {be\past} ever at.her \\
\glt `the first litter she ever had'\source{ca}{26}
\z
\z

\is{comparative clause|)}
% \ex. \ag. Siod é an chéad am a labhair mé air le aon duine\\
%           that it the first time {\C} {speak\past} I on.it with any person\\
%          `That was the first time that I spoke about it with anyone.'\source{smc}{287}
%      \bg. an chéad ál a bhí ariamh aici\\
%           the first litter {\C} {be\past} ever at.her\\
%          `the first litter she ever had'\source{ca}{26}

\appendixsubsection{In certain temporal clauses}

As in many other languages, PSEs in Irish appear within certain temporal clauses\is{temporal adverbial}, notably those introduced by `before' or `when', but not those introduced by `after' (\cite{linebarger:87}; \cite{condoravdi:10}; \cite{krifka:10}).

\ea
\ea
\gll sul a raibh am ag an sáirsint aon cheo a rá leis \\
     before {\go} {be\past} time at the sergeant one mist {\vce} {say\vn} with.him \\
\glt `before the sergeant had time to say the slightest thing to him'\source{lsc}{202}
\ex
\gll sul má bheidh {a fhios} aige tada \\
     before {\C} {be\fut} knowledge at.him anything \\
\glt `before he knows anything'\source{cf}{129}
\ex
\gll sul má lonnaigh aon duine ariamh ann \\
     before {\go} {settle\past} any person ever there \\
\glt `before anyone ever settled there'\source{mabat}{74}
\z
\z

% \ex. \ag. sul a raibh am ag an sáirsint aon cheo a rá leis\\
%           before {\go} {be\past} time at the sergeant one mist {\vce} {say\vn} with.him\\
%          `before the sergeant had time to say the slightest thing to him'\source{lsc}{202}
%      \bg. sul má bheidh {a fhios} aige tada\\
%           before {\C} {be\fut} knowledge at.him anything\\
%          `before he knows anything'\source{cf}{129}
%      \cg. sul má lonnaigh aon duine ariamh ann\\
%           before {\go} {settle\past} any person ever there\\
%           `before anyone ever settled there'\source{mabat}{74}

\ea
\ea
\gll Bhí siad go han-mhaith dó nuair a bhí {a dhath} acu. \\
     {be\past} they {\scshape ptc} {very-good} to.him when {\C} {be\past} anything at.them \\ 
\glt `They were very good to him when they had anything.'\source{gog}{266}
\ex
\gll nuair a theastaíodh dada uaithi bhuaileadh sí cnag ar an urlár \\
     when {\go} {need\pasthabit} anything from.her {hit\pasthabit} she knock on the floor \\
\glt `When she needed anything, she would bang on the floor.'\source{cg}{20}
\ex
\gll Ní tostach dóibh nuair a thagas aon bhac in a mbealach. \\
     {\negcop} silent {to.them} when {\go} {come\pres} any obstacle in their way \\
\glt `They are not silent when any obstacle gets in their way.'\source{cg}{34}
\z
\z

% \ex. \ag. Bhí siad go han-mhaith dó nuair a bhí {a dhath} acu.\\
%           {be\past} they {\scshape ptc} {very-good} to.him when {\C} {be\past} anything at.them\\ 
%          `They were very good to him when they had anything.'\source{gog}{266}
%      \bg. nuair a theastaíodh dada uaithi bhuaileadh sí cnag ar an urlár\\
%           when {\go} {need\pasthabit} anything from.her {hit\pasthabit} she knock on the floor\\
%          `when she needed anything, she would bang on the floor'\source{cg}{20}
%      \cg. Ní tostach dóibh nuair a thagas aon bhac in a mbealach.\\
%           {\negcop} silent {to.them} when {\go} {come\pres} any obstacle in their way\\
%          `They are not silent when any obstacle gets in their way.'\source{cg}{34}

\appendixsubsection{In arguments of `negative' predicates}

The PSEs of \sectref{sec:npi} also appear in the complements of certain predicates whose meaning has a `negative' component (in a sense that remains to be clarified). For instance, as seen in (\ref{ex:neg-advers}), they appear in the complements of so-called adversative attitude predicates\is{adversative predicate} -- those which give rise to an implicature that the holder of the attitude has a negative view of the semantic content of that complement.

\ea\label{ex:neg-advers}
\ea
\gll Bhí imní orainn go dtarlódh aon cheo dhuit. \\
    {be\past} worry on.us {\C} {happen\cond} any mist to.you \\
\glt `We were worried that anything would happen to you.'\source{cgc}{68}
\ex
\gll Is deacair tada {a rá.} \\
     {\scshape cop} hard anything {say.\textsc{nfin}} \\
\glt `It's hard to say anything.'\source{fb}{146}
\ex
\gll ar eagla go dtarlódh {a dhath} dó \\
     on fear {\C} {happen\cond} {anything} to.him \\
\glt `for fear that anything might happen to him'\source{tair}{182}
\ex
\gll Is mór a' trua go gcaithfidh aon duine imeacht as. \\
     {\cop} great the pity {\go} must any person {leave.\textsc{nfin}} out-of.it \\
\glt `It's a great pity that anyone has to leave it.'\source{brd}{246}
\z
\z

% \ex. \ag. bhí imní orainn go dtarlódh aon cheo dhuit\\
%           {be\past} worry on.us {\C} {happen\cond} any mist to.you\\
%          `We were worried that anything would happen to you.'\source{cgc}{68}
%      \bg. Is deacair tada {a rá}\\
%           {\scshape cop} hard anything {say.\textsc{nfin}}\\
%          `It's hard to say anything.'\source{fb}{146}
%      \cg. ar eagla go dtarlódh {a dhath} dó\\
%           on fear {\C} {happen\cond} {anything} to.him\\
%          `for fear that anything might happen to him'\source{tair}{182}
%      \dg. Is mór a' trua go gcaithfidh aon duine imeacht as.\\
%           {\cop} great the pity {\go} must any person {leave.\textsc{nfin}} out-of.it\\
%          `It's a great pity that anyone has to leave it.'\source{brd}{246}

\noindent I include in this class some implicative predicates\is{implicative predicate} like {\itshape fail}:

\ea
\ea
\gll Chinn orm éinne a aimsiú. \\
     {fail\past} on.me anyone {\vce} {find\vn} \\
\glt `I failed to find anyone.'\source{tuair}{04-10-21}
\ex
\gll Chinn air tada in-ite a fháil. \\
     {fail\past} on.him anything edible {\vce} {find\vn} \\
\glt `He failed to find anything edible.'\source{att}{101}
\z
\z

% \ex. \ag. chinn orm éinne a aimsiú. \\
%           {fail\past} on.me anyone {\vce} {find\vn}\\
%          `I failed to find anyone.'\source{tuair}{04-10-21}
%      \bg. chinn air tada in-ite a fháil.\\
%           {fail\past} on.him anything edible {\vce} {find\vn}\\
%          `He failed to find anything edible.'\source{att}{101}

Here, though, there is an interesting contrast between Irish and English. In Irish (and also in Korean, judging by \cite{lee:95}), PSEs may appear as direct arguments of the relevant predicates (rather than appearing only within their complements):

\ea\label{ex:fail}
\gll Chinn sé ar aon dochtúir thall ann é a bhaint amach. \\
     {fail\past} it on any doctor over there it {\vce} {take\vn} out \\
\glt `No doctor over there succeeded in extracting it.'\source{sjccf}{275} \\
     ({\itshape lit.}: `Any doctor over there failed to extract it.')
\z

% \exg.\label{ex:fail}Chinn sé ar aon dochtúir thall ann é a bhaint amach.\\
%          {fail\past} it on any doctor over there it {\vce} {take\vn} out\\
%         `No doctor over there succeeded in extracting it.'\source{sjccf}{275}\\
%          ({\itshape lit.}: `Any doctor over there failed to extract it.')

An interesting subclass of such predicates are those which express reluctance on the part of an experiencer. These are among Karttunen's (\citeyear{lauri:71}) negative implicatives\is{implicative predicate}; “they seem to incorporate negation\is{negation}” (\cite[352]{lauri:71}) and license the inference that in the preference\hyp worlds of their experiencers the eventuality described in their complement is not instantiated.\footnote{For a sophisticated and relevant discussion, see \citet[115--121]{von-fintel:99}.} NPIs\is{negative polarity item} in English are licensed within the complements of such predicates, but not in the experiencer argument\hyp position. In Irish they are, in addition, licensed in the experiencer argument-position.  I emphasize the contrast between the two languages in this respect by offering for the examples in (\ref{ex:drogall}) literal English translations, which are ill-formed but, interestingly, understandable.\footnote{One might avoid this puzzle by taking the examples in (\ref{ex:fail}), (\ref{ex:drogall}) and (\ref{ex:cuma}) to involve “free choice” readings\is{free choice reading}. But this move is unlikely to be tenable given that the licensing conditions for such readings are not satisfied in such cases. In addition, the possibilities shown in (\ref{ex:fail}--\ref{ex:cuma}) hold only for   predicates whose meaning in some sense includes a negative component (\citealt{klima:64, lauri:71, kadmon-landman:93, von-fintel:99}).}

\ea\label{ex:drogall}
\ea
\gll Bhí drogall ar aon imreoir suí leis ag clár na himeartha. \\
     {be\past} reluctance on any player {sit.\textsc{nfin}} with.him at board {the\gen} {playing\gen} \\
\glt `Any player was reluctant to sit with him at the gaming-table.'\\ \hfill\source{sjccf}{337}
\ex
\gll Bhí leisce ar aon duine tada a rá. \\
     {be\past} reluctance on any person anything {\vce} {say\vn} \\
\glt `Anyone was reluctant to say anything.'\source{lgn}{96}
\z
\z

% \ex.\label{ex:drogall}
%     \ag. bhí drogall ar aon imreoir suí leis ag clár na himeartha\\
%           {be\past} reluctance on any player {sit.\textsc{nfin}} with.him at board {the\gen} {playing\gen}\\
%          `any player was reluctant to sit with him at the gaming-table'\\ \hfill\source{sjccf}{337}
%     \bg. Bhí leisce ar aon duine tada a rá.\\
%          {be\past} reluctance on any person anything {\vce} {say\vn}\\
%          `Anyone was reluctant to say anything.'\source{lgn}{96}

Similarly, the predicate {\itshape cuma} (`matterless, insignificant') allows NPIs\is{negative polarity item} in its two argument-positions: 

\ea\label{ex:cuma}
\ea
\gll Ba chuma liom faoi thada. \\
     {\ba} matterless with.me about anything \\
\glt `I didn't care about anything.'\source{rng}{31-01-20}
\ex
\gll Is cuma le ceachtar agaibh fá'n duine eile. \\
     {\cop} matterless with either of.you about.the person other \\
\glt `Neither of you cares about the other.'\source{atfs}{425} \\
({\itshape lit}. `Either of you doesn't care about the other.')
\z
\z

% \ex.\label{ex:cuma}\ag. Ba chuma liom faoi thada.\\
%           {\ba} matterless with.me about anything\\
%          `I didn't care about anything.'\source{rng}{31-01-20}
%      \bg. is cuma le ceachtar agaibh fá'n duine eile\\
%           {\cop} matterless with either of.you about.the person other\\
%          `Neither of you cares about the other.'\source{atfs}{425}\\
%          ({\itshape lit}. `Either of you doesn't care about the other.')

The English facts here may be complicated by a language\hyp particular oddity -- a requirement that the licensing element precede the licensed NPI\is{negative polarity item} \citep[206--207]{ladusaw:79}. It is also possible that the relevant arguments in Irish are internal and therefore within the scope\is{scope} of the (negative component of the) licensing predicate.

\appendixsubsection{In the restrictive clause of a universal quantification structure}

The final licensing environment to be documented is once again a familiar one: the restrictive clause of a universal quantification structure. 

The most frequent case of this type involves NPIs\is{negative polarity item} which have “free choice” or quasi-universal readings\is{free choice reading}. On this possibility (in English) see \citet[163]{horn:00}.  A relative clause attached to such an element acts as the restrictor for the universal quantification that it expresses and PSEs appear freely.

\ea
\ea
\gll éinne go bhfuil aon chiall aige \\
     anyone {\C} {be\pres} any sense at.him \\
\glt `anyone who has any sense'\source{g}{11}
\ex
\gll fear {ar bith} a bhfuil fhios aige tada \\
     man any {\C} {be\pres} knowledge at.him anything \\
\glt `any man who knows anything'\source{cnf}{25}
\ex
\gll am   {ar bith} a mbeadh {a dhath} le rádh aige liom \\
     time  any     {\C} {be\cond}  anything to {say\vn} at.him with.me \\
\glt `any time he had anything to say to me'\source{lcs}{201}
\z
\z

% \ex. \ag. éinne go bhfuil aon chiall aige\\
%           anyone {\C} {be\pres} any sense at.him\\
%          `anyone who has any sense'\source{g}{11}
%      \bg. Fear {ar bith} a bhfuil fhios aige tada\\
%           man any {\C} {be\pres} knowledge at.him anything\\
%          `any man who knows anything'\source{cnf}{25}
%      \cg. Am   {ar bith} a mbeadh {a dhath} le rádh aige liom \\
%           time  any     {\C} {be\cond}  anything to {say\vn} at.him with.me\\
%          `any time he had anything to say to me'\source{lcs}{201}

Unconditional clauses, which in Irish are marked by means of a distinctive \WH-determiner\is{wh-question} (\textit{pé}, or \textit{cibé}), similarly host PSEs:

\ea
\gll pé pingin a bhí riamh acu \\
     whatever penny {\C} {be\past} ever at.them \\
\glt `whatever pennies they ever had'\source{ai}{199}
\z

% \exg. pé pingin a bhí riamh acu\\
%       whatever penny {\C} {be\past} ever at.them\\
%      `whatever pennies they ever had'\source{ai}{199}

It is hardly a surprise, then, that PSEs are also licensed within relative clauses that restrict nominals headed by explicit universal quantifiers:

\ea
\ea
\gll timpeall ar chuile áit a gceapfaidís a mbeadh aon deis ag an ngail {a bheith} ag éalú \\
     around on every place {\go} {think\cond.\their} {\go} {be\cond} any opportunity at the steam {be.\textsc{nfin}} {\prog} {escape\vn} \\
\glt `around every place that they'd think there was any opportunity for the steam to escape'
\ex
\gll Gach neach a thug cath éagórach riamh d' Fhionn \\
     every being {\go} {give\past} battle unjust ever to {} \\
\glt `every being who ever joined battle unjustly with Fionn'\source{snaf}{226}
\z
\z

% \ex. \ag. timpeall ar chuile áit a gceapfaidís a mbeadh aon deis ag an ngail {a bheith} ag éalú\\
%           around on every place {\go} {think\cond.\their} {\go} {be\cond} any opportunity at the steam {be.\textsc{nfin}}
%                            {\prog} {escape\vn}\\
%          `around every place that they'd think there was any opportunity for the steam to escape'
%                            \source{sjccf}{266}
%      \bg. Gach neach a thug cath éagórach riamh d' Fhionn\\
%           every being {\go} {give\past} battle unjust ever to {}\\
%          `every being who ever joined battle unjustly with Fionn'\source{snaf}{226}

% %%%%%%%%%%%%%%%%%%%%%%%%%%%%%%%%%%%%%%%%%%%%%%%%%%%%%%%%%%%%%%
% %%
% %%    SECOND APPENDIX
% %%         


\appendixsection{Sources of examples} % Appendix 2

\begin{description}[leftmargin=!,labelwidth=\widthof{\textsc{agmts:}},font=\normalfont\scshape,noitemsep]
\sloppy
\item[a:]  \textit{Aileach}, Jackie Mac Donncha, Cló Iar-Chonnacht, 2010 
\item[aa:]  \textit{Athaoibhneas}, Pádhraic Óg Ó Conaire, Sáirséal agus Dill, 1959 
\item[aag:]  \textit{As an nGéibheann}, Máirtín Ó Cadhain, Sáirséal agus Dill, 1973 
\item[abfs:]  \textit{An Baile i bhFad Siar}, Domhnall Mac an tSíthigh, Coiscéim, 2000  
\item[abhm:]  \textit{Ábhar Machnaimh}, An tAthair Donncha Ó Corcora, Foilseacháin Náisiúnta 
                    Teoranta, 1985
\item[aced:]  \textit{An Chuid Eile Díom Féin -- Aistí le Máirtín Ó Direáin}, ed. Síobhra Aiken, Cló 
                    Iar-Chonnacht, 2018
\item[ae:]  \textit{An Eochair}, Máirtín Ó Cadhain, Dalkey Archive Press, 2015
\item[afap:]  \textit{An Fear a Phléasc}, Mícheál Ó Conghaile,  Cló Iar-Chonnachta, 1997
\item[ag:]  \textit{An Gabhar Sa Teampall},  Mícheál Ua Ciarmhaic, Coiscéim, 1986 
\item[agfc:]  \textit{An Grá Faoi Cheilt}, Pádraig Ó Cíobháin, Coiscéim, 1992  
\item[agmts:]  \textit{Ar Gach Maoilinn Tá Síocháin}, Pádraig Ó Cíobháin, Coiscéim, 1991
\item[ag:]  \textit{An Gabhar Sa Teampall},  Mícheál Ua Ciarmhaic, Coiscéim, 1986
\item[ai:]  \textit{Allagar na hInise},  Tomás Ó Criomhthain, Oifig an tSoláthair, 1977
\item[al:]  \textit{Abair Leat}, Joe Daly, ed. Pádraig Tyers, An Sagart, 1999
\item[am:]  \textit{An Mhiorbhailt},  C.B Kelland, trans. Niall Mac Suibhne,  Oifig Díolta
                    Foillseacháin Rialtais, 1936 
\item[an:]  \textit{Athnuachan}, Máirtín Ó Cadhain, Coiscéim, 1995
\item[annf:]  \textit{Ar Nós na bhFáinleog}, Siobhán Ní Shúilleabháin, Coiscéim, 2004
\item[apb:]  \textit{An Prionsa Beag}, Antoine de Saint-Exupéry, trans. Eoghan Mac Giolla Bhríde, 
                    Éabhlóid, 2015
\item[atfs:]  \textit{Ag Teacht Fríd an tSeagal}, Helen Mathers, trans. Seosamh Mac Grianna,
                   An Gúm, 1932
\item[atim:]  \textit{An tIolrach Mór, Díoghluim Gearr-Sgéal}, Pádhraic Ó Domhnalláin, Brún 
                   agus Ó Nualláin, 1941
\item[att:]  \textit{An tSraith Tógtha}, Máirtín Ó Cadhain, Sáirséal agus Dill, 1977
\item[bm:]  \textit{Bullaí Mhártain},  Síle Ní Chéileachair agus Donncha Ó Céilleachair, Sáirséal
                    agus Dill, 1969
\item[brd:]  \textit{Bean Ruadh de Dhálach},  Séamus Ó Grianna, Oifig Díolta Foillseacháin 
                    Rialtais, 1966                                                                        
\item[btfs:]  \textit{An Blascaod Trí Fhuinneog na Scoile}, Nóra Ní Shéaghdha, ed. Pádraig Ó hÉalaí,
                    An Sagart, 2015                                                                       
\item[c:]  \textit{Clochmhóin}, Joe Steve Ó Neachtain, Cló Iar-Chonnachta, 1998 
\item[ca:]  \textit{Cnuasacht Airneáin},  Colm Ó Ceallaigh, Coiscéim, 2006
\item[caa:]  \textit{Conamara agus Árainn 1880--1890: Gnéithe den Stair Shóisialta}, Micheál
                    Ó Conghaile, Cló Iar-Chonnachta, 1988  
\item[cab:]  \textit{Carraig an Bháis}, Colm Ó Ceallaigh, Coiscéim, 2007 
\item[cc:]  \textit{Cruithneacht agus Ceannabháin}, Tomás Bairéad, Comhlucht Oideachais
                    na hÉireann, 1940


\item[cct:]  \textit{Camchuairt Chonamara Theas}, Tim Robinson, trans. Liam Mac Con Iomaire, 
                     Coiscéim, 2002 
\item[cdc:]  \textit{Castar na Daoine ar a Chéile, Scríbhinní Mháire 1}, Séamus Ó Grianna, 
                      ed. Nollaig Mac Congail, Coiscéim, 2002                                      
\item[cf:]  \textit{Cois Fharraige Le Mo Linnse}, Seán Ó Conghaile, Clódhanna Teoranta, 1974 
\item[cfc:]  \textit{Céad Fáilte go Cléire}, ed. Marion Gunn, An Clóchomhar Tta, 1990      
\item[cg:]  \textit{Ceol na nGiolcach}, Padhraic Óg Ó Conaire, Oifig an tSoláthair, 1968  
\item[cgc:]  \textit{Caillte i gConamara, Scéalta Aniar}, ed. Brian Ó Conchubhair, Cló
                      Iar-Chonnacht, 2014                                                          
\item[chd:]  \textit{Chicago Driver},  Maidhc Dainín Ó Sé, Coiscéim, 1992 
\item[clm:]  \textit{Cúl le Muir agus Scéalta Eile},  Séamus Ó Grianna, Oifig an tSoláthair, 1961 
\item[cnf:]  \textit{Clann na Feannóige}, Colm Ó Ceallaigh, Coiscéim, 2004 
\item[coc:]  \textit{Cora Cinniúna},  Séamus Ó Grianna, ed. Niall Ó Domhnaill, An Gúm, 1993
\item[ctp:]  \textit{Cuimhne an tSeanpháiste},  Micheál Breatnach, Oifig an tSoláthair, 1966 
\item[d:]  \textit{An Draoidín}, Séamus Ó Grianna, Oifig Díolta Foillseacháin Rialtais, 1959 
\item[dead:]  \textit{Déirc an Díomhaointis}, Pádhraic Óg Ó Conaire, Sáirséal agus Dill, 1972 
\item[dgd:]  \textit{Deoir Ghoirt an Deoraí}, Colm Ó Ceallaigh, Cló Iar-Chonnachta, 1993 
\item[dpb:]  \textit{Dualgas Pheadair Bhig}, trans Séamus Ó Maolchathaigh, Oifig an tSoláthair, 1953
\item[dr:]  \textit{Dracula}, Bram Stoker, trans. Seán Ó Cuirrín, An Gúm, 1933/1997                
\item[empp:]  \textit{Eachtraí Mara Phaidí Pheadair as Toraigh}, Séamus Mac a' Bhaird, ed.
                     Aingeal Nic a' Bhaird, Caoimhín Mac a' Bhaird, Nollaig Mac Congail, 
                     Arlen House, 2019                                                                      
\item[fb:]  \textit{Feamain Bhealtaine}, Máirtín Ó Direáin, An Clóchomhar Tta., 1961               
\item[ff:]  \textit{Fonn na Fola},   Beairtle Ó Conaire, Cló Iar-Chonnacht, 2005                   
\item[g:]  \textit{Greenhorn},  Maidhc Dainín Ó Sé, Coiscéim, 1997 
\item[gddr:]  \textit{Go dTaga do Ríocht, Boicíní Bhóthar Kilburn, Cripil Inis Meáin}, Mícheál Ó Conghaile, 
                     Cló Iarr-Chonnachta, 2009                                                              
\item[gfh:]  \textit{An Ghlan-fhírinne}, Cóil Learaí Ó Finneadha, Cló Iarr-Chonnacht, 2014          
\item[gog:]  \textit{Glórtha ón Ghorta: Béaloideas na Gaeilge agus an Gorta Mór}, Cathal Póirtéir,
                     Coiscéim, 1996                                                                         
\item[gsa:]  \textit{An Giorria San Aer},  Ger Ó Cíobháin, Coiscéim, 1992                           
\item[iae:]  \textit{In Aimsir Emmet}, James Murphy trans. Colm Ó Gaora, Oifig Díolta Foillseacháin Rialtais,
                     Dublin, 1937         
\item[imsbrm:]  \textit{Idir Mná: Scríbhneoirí Ban Ros Muc}, ed. Máire Holmes, Pléar\-áca Chonamara, 1995  

\item[init:]  \textit{Idir Neamh is Talamh}, Joe Steve Ó Neachtain, Cló Iar-Chonnacht, 2014             
\item[isnb:]  \textit{Iad Seo Nach Bhfaca}, Beairtle Ó Conaire,  Cló Iar-Chonnacht, 2010               
\item[lan:]    \textit{Leoithne Aniar},  ed. Pádraig Tyers, Cló Dhuibhne, Baile an Fhéirtéaraigh, 1982  
\item[lcs:]  \textit{Le Clap-Sholus},  Séamas Ó Grianna, Oifig an tSoláthair, 1967                    
\item[lgl:]  \textit{Le Gealaigh}, Pádraig Ó Cíobháin, Coiscéim, 1991                                 
\item[lgn:]  \textit{Le Gean agus scéalta eile}, Mike P. Ó Conaola, Sián, 2020                        
\item[ll:]  \textit{Lámh Láidir}, Joe Steve Ó Neachtain, Cló Iar-Chonnachta,  2005                   
\item[lofrs:] \textit{Liam Ó Flaithearta - Rogha Scéalta}, trans. Micheál Ó Conghaile, Cló 
                    Iar-Chonnacht, 2020                                                                       
\item[lsc:] \textit{Lig Sinn i gCathú},  Breandán Ó hEithir, Sáirséal agus Dill, 1976                 
\item[mabat:]  \textit{Mar a Bhí Ar dTús: Cuimhne Seanghasúir}, Joe Steve Ó Neachtain, Cló 
                      Iarr-Chonnacht, 2018                                                                    
\item[mo:]  \textit{Muintir An Oileáin},  Peadar O'Donnell, trans. Seosamh Mac Grianna,  Oifig Díolta
                      Foillseacháin Rialtais, 1952                                                            
\item[nlab:]  \textit{Na Laetha a Bhí},  Eoghan Ó Domhnaill, Oifig an tSoláthair, 1968                 
\item[omgs:]  \textit{Ó Mhuir go Sliabh}, Séamus Ó Grianna, Oifig an tSoláthair, 1961                  
\item[oogc:]  \textit{Ó Oileán go Cuilleán}, Muiris Ó Súilleabháin, ed. Nuala Uí Aimhirgín, Coiscéim, 2000 
\item[ota:]  \textit{Ón tSeanam Anall, Scéalta Mhicí Bháin Uí Bheirn}, ed. Mícheál Mac Giolla Easbuic,
                      Cló Iar-Chonnachta, 2008                                                               
\item[paa:]  \textit{Peacaí Ár nAthaireacha}, Mícheál Ó Súilleabháin, Coiscéim, 1992                  
\item[pi:]  \textit{Punt Isló}, Maidhc Dainín Ó Sé, Coiscéim, 2013                                   
\item[png:]  \textit{Pobal na Gaeltachta}, ed. Gearóid Ó Tuathaigh, Liam Lillis Ó Laoire, Seán
                     Ua Súillebháin, Cló Iar-Chonnachta, 2000                                                
\item[rng:]  \textit{Raidió na Gaeltachta} (numerical index refers to date of broadcast)              
\item[sif:]  \textit{Seanchas Iascaireachta agus Farraige}, Seán Ó hEochaidh, \textit{Béaloideas} 33, 1965 
\item[sjccf:]  \textit{Seanchas Jimmí Chearra Chois Fharraige}, ed. Pádraic Ó Cearra, Coiscéim, 2010    
\item[sjsj:]  \textit{Seachrán Jeaic Sheáin Johnny}, Mícheál Ó Conghaile,  Cló Iarr-Chonnacht, 2002    
\item[sk:]  \textit{Sáile Chaomhánach}, C.J. Kickham, trans. Máirtín Ó Cadhain, An Gúm, 1932/1986    
\item[smc:]  \textit{Stairsheanchas Mhicil Chonraí -- Ón Máimín go Ráth Chairn}, ed. Conchúr
                      Ó Giollagáin, Cló Iar-Chonnachta, 1999                                                  
\item[snaf:]  \textit{Seanchas na Féinne}, Niall Ó Dónaill, An Gúm, 1943/1996                          
\item[srnf:]  \textit{Seanchas Rann na Feirste},  Maelsheachlainn Mac Cionaoith, 2006                  
\item[ssotc:]  \textit{Síscéalta ó Thír Chonaill}, ed. Seán Ó Heochaidh, Máire Ní Néill and Séamas 
                      Ó Catháin, Comhairle Bhéaloideas Éireann, 1977                                          
\item[st:]    \textit{An Sean-Teach},  Séamas Ó Grianna, Oifig an tSoláthair, 1968                     

\item[tair:]  \textit{Tairngreacht}, Proinsias Mac a' Bhaird, Leabhair Comhar, 2018                    
\item[tii:]  \textit{Tone Inné agus Inniu}, Máirtín Ó Cadhain, Coiscéim, 1992                         
\item[tmgb:]  \textit{Thiar sa Mhainistir atá an Ghaolainn Bhreá}, Brighid Ní Mhóráin, An Sagart, 1997 
\item[tuair:]  \textit{Tuairisc}: online newspaper: \url{https://tuairisc.ie} 
                      (numerical index refers to date of publication)                                         
\item[umi:]  \textit{Uaill-Mhian Iúdaigh}, Roy Bridges, trans. Tadhg Ó Rabhartaigh, Oifig Díolta 
                     Foillseacháin Rialtais, 1936
\end{description}

\il{Irish (Modern)|)}

\printbibliography[heading=subbibliography,notkeyword=this]

\end{document}
