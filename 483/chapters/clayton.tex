\documentclass[output=paper,colorlinks,citecolor=brown]{langscibook}
\ChapterDOI{10.5281/zenodo.15654869}
\title{Hebrides English in the 1980s and now} 
\author{Ian Clayton\affiliation{University of Nevada, Reno} and
    Cynthia Shuken\affiliation{Independent scholar}
    }

\IfFileExists{../localcommands.tex}{
   %\addbibresource{claytonbib.bib}
   % add all extra packages you need to load to this file

\usepackage{tabularx,multicol}
\usepackage{url}
\urlstyle{same}

\usepackage{listings}
\lstset{basicstyle=\ttfamily,tabsize=2,breaklines=true}

\usepackage{langsci-basic}
\usepackage{langsci-optional}
\usepackage{langsci-lgr}
\usepackage{langsci-osl}
% \usepackage{./langsci/styles/langsci-lgr}
% \usepackage{./langsci/styles/langsci-osl}
% \usepackage{langsci-gb4e}

\usepackage{tikz}
\usetikzlibrary{patterns,calc}
\pgfdeclarepatternformonly{south east lines}{\pgfqpoint{-0pt}{-0pt}}{\pgfqpoint{3pt}{3pt}}{\pgfqpoint{3pt}{3pt}}{
    \pgfsetlinewidth{0.6pt}
    \pgfpathmoveto{\pgfqpoint{0pt}{3pt}}
    \pgfpathlineto{\pgfqpoint{3pt}{0pt}}
    \pgfpathmoveto{\pgfqpoint{.2pt}{-.2pt}}
    \pgfpathlineto{\pgfqpoint{-.2pt}{.2pt}}
    \pgfpathmoveto{\pgfqpoint{3.2pt}{2.8pt}}
    \pgfpathlineto{\pgfqpoint{2.8pt}{3.2pt}}
    \pgfusepath{stroke}}
    
\usepackage{stmaryrd}
\usepackage{wasysym}
\usepackage{multirow}
\usepackage{caption}
\usepackage{subcaption}
\usepackage{mathrsfs}
\usepackage{qtree}

\usepackage{linguex}


   %pminos do not split footnotes
% \interfootnotelinepenalty=10000 %Footnote in Laporte chapters has to be split SN


%\DeclareIndexNameFormat{default}{%
%\nameparts{#1}%
%\usebibmacro{index:name}%
%{\index[names]}%
%{\namepartfamily}%
%{\namepartgiveni}%
% {}% L1
% {}% L2
%{\namepartprefix}% generates spurious space L3
%{\namepartsuffix}% generates spurious space L4
%}

%  {\DeclareIndexNameFormat{default}{%
%     \usebibmacro{index:name}{\index[names]}{#1}{#3}{#5}{#7}}}

%\DeclareIndexNameFormat{default}{%
%  \usebibmacro{index:name}{\sindex[nom]}{#1}{#3}{#5}{#7}}

%\DeclareIndexNameFormat{default}{%
%  \usebibmacro{index:name}{\sindex[person]}{#1}{#3}{#5}{#7}}
%\DeclareIndexNameFormat{default}{%
%\nameparts{#1} \usebibmacro{index:name}{\sindex[person]]}{\namepartfamily}{‌​\namepartgiven}{\nam‌​epartprefix}{\namepa‌​rtsuffix}}

%\newcommand{\smiley}{:)}

%\renewbibmacro*{index:name}[5]{%
%\usebibmacro{index:entry}{#1}%
%{\iffieldundef{usera}{}{\thefield{usera}\actualoperator}\mkbibindexname{#2}{#3}{#4}{#5}}}

% \newcommand{\noop}[1]{}

%remove for final
%\overfullrule=1mm

\newcommand{\tobi}[2]}}
\renewcommand{\S}[1]{\tobi{#1}{\textsc{*}}}

% this volume references
% puts: [this volume]
% already defined: \citetv
%\newcommand{\citepv}[1]{(\citeauthor{#1} \citeyear*{#1} [this volume])}
\newcommand{\citealtv}[1]{\citeauthor{#1} \citeyear*{#1} [this volume]}

%parentheses around example number
\newcommand{\pref}[1]{(\ref{#1})}

% in-text examples

\newcommand{\lnex}[1]{\textit{#1}} %target lang word
\newcommand{\lnlit}[1]{(lit.: `#1')} %literal reading
\newcommand{\lnlat}[1]{(#1)} % latinization
\newcommand{\lntrans}[1]{`#1'} %translation
\newcommand{\lnexl}[2]%
{\lnex{#1}{} \lnlat{#2}} % ex with latinization
\newcommand{\lnexlat}[3]{\lnex{#1}{} \lnlat{#2}{} \lntrans{#3}} % ex with latinization and tranl.

%ch01
\newcommand{\co}[1]{\mbox{\textbf{#1}}}

%ch09

\newcommand{\cyrbulg}[1]{\begin{otherlanguage*}{bulgarian}#1\end{otherlanguage*}}


%ch10
\newcommand{\nlp}{{\small NLP}}
\newcommand{\mwe}{{\small MWE}}
\newcommand{\rae}{{\small RAE}}
\newcommand{\lvc}{{\small LVC}}
\newcommand{\pos}{{\small P}o{\small S}}
%\newcommand{\todo}[1]{ \textcolor{red}{#1} }

%\renewcommand{\labelenumi}{\theenumi}
%\ainamefmt{{vv}{ll}{, ff}{, jj}} % fullname

\newcommand{\biberror}[1]{{\color{red}#1}}

\newcommand{\osenovaitem}{--~}
   %% hyphenation points for line breaks
%% Normally, automatic hyphenation in LaTeX is very good
%% If a word is mis-hyphenated, add it to this file
%%
%% add information to TeX file before \begin{document} with:
%% %% hyphenation points for line breaks
%% Normally, automatic hyphenation in LaTeX is very good
%% If a word is mis-hyphenated, add it to this file
%%
%% add information to TeX file before \begin{document} with:
%% %% hyphenation points for line breaks
%% Normally, automatic hyphenation in LaTeX is very good
%% If a word is mis-hyphenated, add it to this file
%%
%% add information to TeX file before \begin{document} with:
%% \include{localhyphenation}
\hyphenation{
    Beck-man
    Ngu-yen
    back-chan-nel
    back-chan-nels
    mo-not-o-nous
    ste-reo-typ-i-cal
}

\hyphenation{
    Beck-man
    Ngu-yen
    back-chan-nel
    back-chan-nels
    mo-not-o-nous
    ste-reo-typ-i-cal
}

\hyphenation{
    Beck-man
    Ngu-yen
    back-chan-nel
    back-chan-nels
    mo-not-o-nous
    ste-reo-typ-i-cal
}

   \boolfalse{bookcompile}
   \togglepaper[7]%%chapternumber
}{}


\abstract{Hebrides English is a distinct Scottish Gaelic\il{Scottish Gaelic (Modern)}-influenced variety of English spoken in the Hebridean\is{Hebrides} island chain. The variety is most likely of relatively recent origin: before the second half of the nineteenth century, English was not widely spoken in this part of Scotland, where until recently Gaelic \il{Scottish Gaelic (Modern)} has been the dominant language. Hebrides English exhibits a variety of distinctive linguistic features which may be due to widespread bilingualism among its speakers in both Gaelic\il{Scottish Gaelic (Modern)} and English, including \isi{preaspiration} and sibilant \isi{devoicing}. The present study examines these two features, as well as \isi{glottalization}, a possible incoming feature, in previously unanalyzed archival material from the early 1980s collected on the islands of \isi{Lewis}, \isi{Harris}, and \isi{Skye}. The chapter thus offers a glimpse of Hebrides English as it existed four decades ago, and taken in context with more recent research (e.g. \cite{Clayton:2017, Clayton:2018}) allows further insight into a little-studied variety of English evidently in considerable flux.}

\begin{document}

\maketitle

\il{Hebrides English|(}
\is{contact varieties|(}

\section{Introduction} 

\subsection{Outline of this study}

This chapter has two chief goals. The first is to offer some historical perspective on the phonological characteristics of Hebrides English, based on archival data collected in the early 1980s, and in so doing help to develop a picture of Hebrides English as it was spoken forty years ago. The second goal is to point to certain changes that have taken place in Hebrides English in the intervening decades. 

\newpage
The chapter makes use of two main sources of data. The first is the large body of sociolinguistic interviews conducted by Cynthia Shuken in the early 1980s.\footnote{While the first author is the writer of this chapter and conducted the bulk of the analysis it incorporates, the chapter depends crucially on Cynthia Shuken’s very substantial prior work, and indeed was possible only because she made her collected materials available to the first author. Upon request, Dr. Shuken agreed to be identified as a co-author, despite no longer working in academia. Any errors the chapter contains are solely those of the first author.} Much of the data from that collection of material has never previously been analyzed or published (\cite{Shuken:1984, Shuken:1985}, and \citeyear{Shuken:1986} offer analyses of subsets of the material). Shuken recorded interviews with nearly 100 individuals from \isi{Lewis}, \isi{Harris}, and \isi{Skye}. Upon her departure from academia in the late 1980s, she placed these recordings and associated questionnaires in an archive at the University of Edinburgh School of Scottish Studies, where they have remained effectively untouched since. This collection represents an invaluable and unique record of Hebrides English as it was used in that era and region, and contains ample material to support a much more wide-ranging analysis than current space permits.

\largerpage
The second body of data referred to in this chapter is that collected by the first author during fieldwork in 2013 (cf. \cite{Clayton:2015, Clayton:2017, Clayton:2018}). These studies included data from 24 speakers of Hebrides English, representing nine islands: \isi{Skye}, \isi{Harris}, \isi{Lewis}, North Uist, South Uist, Grimsay, Raasay, Tiree, and Vatersay. While the two collections are not perfectly parallel, differing in numbers of participants, geographic scope, and to some extent in methodology, there is sufficient overlap between them to permit a useful examination of the trajectory of two sociophonetic variables over the time period in question: \isi{preaspiration} and \isi{glottalization}. The present chapter also considers sibilant \isi{devoicing} as it occurs in Shuken’s data; the Clayton study did not address such \isi{devoicing} and so allows no comparative analysis of this feature. 

The balance of the chapter proceeds as follows: \sectref{backgroundandpriorresearch} describes a number of distinguishing characteristics of Hebrides English, and provides a review of the existing literature on the topic. \sectref{theshukenarchive} describes the material in the Shuken archive in detail. \sectref{analysis} presents the analyses of \isi{preaspiration}, \isi{glottalization}, and \isi{devoicing}, while \sectref{discussion} offers a summary of the results and a discussion. 


\section{Background and prior research}\label{backgroundandpriorresearch}

\subsection{What is Hebrides English?}

The remarkably diverse linguistic landscape found in Scotland is in large part due to the prolonged conditions of linguistic contact that have obtained in the region for centuries.  The island groups of Shetland and Orkney to the northeast of the Scottish mainland, for example, are home to a variety of insular Scots influenced by Norn, the North Germanic language once spoken in those islands (\citealt{Melchers:2008, McClure:1994, Barnes:1984, Wells:1982, Catford:1957a}). Meanwhile, in the Highland and Island regions of northwestern Scotland one encounters English varieties influenced to a greater or lesser degree by the lexicon, syntax, and phonology of Scottish Gaelic\il{Scottish Gaelic (Modern)}, the Celtic language that has been, until the last few decades, the dominant language in the region (\citealt{McClure:1994, Shuken:1984, Shuken:1985, Clement:1997, Stuart-Smith:2008, Clayton:2017, Clayton:2018}). 

While these Gaelic\il{Scottish Gaelic (Modern)}-influenced varieties are sometimes referred to collectively as \textit{Highland and Island English}, they are by no means uniform, and further subdivisions are possible. Frequently, for example, a distinction is made between Highland English on the one hand, and Hebrides (or Hebridean or Island) English on the other (\cite{Clement:1980, Clement:1997, Shuken:1985, Sabban:1985, Bird:1997}). The basis of this division is the degree and kind of presumed contact with Gaelic\il{Scottish Gaelic (Modern)}. Thus by \textit{Highland English} is usually meant the form spoken in the mainland Highland regions more or less adjacent to Scots-speaking areas. In this part of the Highlands, Gaelic\il{Scottish Gaelic (Modern)} has not been widely spoken for many decades if not longer, so linguistic contact between Gaelic\il{Scottish Gaelic (Modern)} and English is more a matter of historical record than contemporary fact. Hence, such characterizations of Highland English as “standard English on a Gaelic substratum” (\citealt{Mather:1986}), or as “post\hyp Gaelic English” \citep{Catford:1957b}  are reasonable ones. 

%\todo{Quotations in the pars above and below may need page numbers in their citations. Ask Clayton?}

By contrast, \textit{Hebrides English} refers self-evidently to the variety of Scottish English frequently heard in the Hebridean\is{Hebrides} island chain, especially those islands where Gaelic\il{Scottish Gaelic (Modern)} has at least until very recently remained a widely used language, and where bilingualism is commonplace if not universal. Though (\cite{Catford:1957b}'s) description of Hebrides English (HE hereafter) as “the English spoken, with varying degrees of imperfection, as a second language, by persons whose mother tongue is still Gaelic” is difficult to sustain six decades on as usage of Gaelic\il{Scottish Gaelic (Modern)} in the area continues to decline \citep{gw:ÓGiollagáin2020}, it is nevertheless true that the degree of contact between English and Gaelic\il{Scottish Gaelic (Modern)} is much closer in these islands than anywhere else in Scotland. 

Another factor likely to be relevant is the manner in which influences from English, or from Scots, have made their way into formerly Gaelic\il{Scottish Gaelic (Modern)}-dominant communities. \textcolor{brown}{\citeauthor{Clement:1980}'s (\citeyear{Clement:1980})} point that learning English at school is quite distinct from learning Scots from one’s neighbors remains a perspicuous one. English was largely introduced to the islands and more remote parts of the Highlands via instruction in the schools, upon English being made compulsory by the Education %(Scotland) 
Act of 1872 \citep{Withers:1984}. By the 1970s, bilingualism in English was effectively universal among Gaelic\il{Scottish Gaelic (Modern)} speakers, though even today one can readily find native Gaelic\il{Scottish Gaelic (Modern)} speakers who knew little or no English before beginning primary school. Thus, it is probably appropriate to regard Hebrides English as a variety of schoolroom Scottish Standard English that features significant Gaelic\il{Scottish Gaelic (Modern)} influences. 
%\slp{check ``Education (Scotland) Act" in par above} 

This description of HE, however, does not account for recent influences from urban varieties of Scots or Scottish English. The larger Island communities, such as Portree on \isi{Skye}, and Stornoway on \isi{Lewis} are prominent commercial centers with significant numbers of English- and Scots-speaking incomers with concomitant influences on local language use (\citealt{Shuken:1984, Speitel:1981, Clement:1980}). These influences seem to include the introduction of velarized [ɫ] \citep{Shuken:1985} and glottal reinforcement (\citealt{Clayton:2018, Wells:1982, Trudgill:1988, Stuart-Smith:1999a, Stuart-Smith:1999b}); there may well be others.  

Finally, it must be stressed that English as used in the \isi{Hebrides} is hardly uniform, as this chapter and other papers illustrate. What research has been published, largely focuses on the northern islands of the \isi{Hebrides}, especially \isi{Skye}, \isi{Harris}, North Uist, and \isi{Lewis}, and therefore presents a very incomplete picture. There remains a marked absence of research regarding English as it is spoken on the other Hebridean\is{Hebrides} islands. 

\subsection{Possible Gaelic influences in Hebrides English}\label{section 2.2}

Discussions of HE in the literature frequently mention certain features that may well be due to contact with Gaelic\il{Scottish Gaelic (Modern)}. They include:

\subsubsection{Morphosyntax}

\begin{enumerate}
    \item The “BE after V-ing” construction \citep{Sabban:1985}:
    \ea
    \gll Tha  mi  dìreach  air    mo  nigheadaireachd  a’  dhèanamh \\
    be.\textsc{prs}  I  just  \textsc{perf}  my  washing  \textsc{prog}  doing \\
    \glt ‘I'm just after finishing my washing.’
    \z

    \item Absence of the English present perfect construction \citep{Sabban:1985}: 
    \ea
    \gll Tha	mi	air a	bhith	nam	bhanntrach-fhir	a-nis	airson	sia	bliadhna \\
    be.\textsc{prs}  I  \textsc{perf}  \textsc{art}  be.\textsc{VN}  in-my   widower  now  for  six  years \\
    \glt ‘I’m a widower now for six years.’
    \z

    \item Cleft constructions \citep{Shuken:1984}:
    \ea
    \gll ‘S e   Gàidhlig    a   bhruidhinn  sinn    {an  còmhnaidh}   aig an  taigh \\
    cop.\textsc{prs}    it  Gaelic  \textsc{rel}    spoke   we  always  at  the home \\
    \glt ‘It was always Gaelic we spoke in the home.’
    \z
\end{enumerate}

%\todo{Unicode character check needed in section below}

\subsubsection{Phonology}

\begin{enumerate}
\item There is no voicing contrast among Gaelic\il{Scottish Gaelic (Modern)} sibilants. HE realizations like \textit{measure} as [mɛʃur], and \textit{just} as [tʃʌst] are thus presumably due to Gaelic\il{Scottish Gaelic (Modern)} influence (\citealt{Shuken:1985, Wells:1982}).

\item In the Hebridean\is{Hebrides} varieties of Gaelic\il{Scottish Gaelic (Modern)}, the unaspirated voiceless stops /p t k/ contrast with the voiceless aspirated series /pʰ tʰ kʰ/. There are no voiced stops. In like vein, many HE speakers omit voicing in English /b d ɡ/, e.g. \textit{glory} [klɔɾi], \textit{ragged} [rækɪt], \textit{building} [pɪltɪŋ] (\citealt{Grant:1913, Borgstrom:1940, Oftedal:1956, Wells:1982, Ladefoged:1998, ic:Gillies2009sg}).


\item In many varieties of Gaelic\il{Scottish Gaelic (Modern)}, the aspirated stop phonemes /pʰ tʰ kʰ/ are preaspirated\is{preaspiration} in non-initial positions, for instance as [ʰp ʰt ʰk] (the typical \isi{Lewis} pattern), or as [ʰp ʰt \textsuperscript{x}k] (as in \isi{Skye} and \isi{Harris}). In similar fashion, voiceless stops in HE may feature \isi{preaspiration}:  \textit{top} [tʰɔʰp], \textit{not} [nɔʰt], \textit{cat} [kʰæʰt], \textit{bake} [beʰk] (\citealt{Borgstrom:1940, Oftedal:1956, Shuken:1979, Ni-Chasaide:1985, Ladefoged:1998, Clayton:2010, Nance:2013}). \isi{Lewis} speakers have been particularly singled out in this regard, e.g. \citet{Borgstrom:1940} notes that “postvocalic  \textit{k}, \textit{t}, \textit{p} are usually rendered by Lewismen as \textit{\textsuperscript{h}}\textit{k}, \textit{\textsuperscript{h}}\textit{t}, \textit{\textsuperscript{h}}\textit{p}, and nobody seems to be aware that this is not the proper English pronunciation” (p. 21). Likewise, \citeauthor{Shuken:1984} (\citeyear{Shuken:1984, Shuken:1985}) found abundant \isi{preaspiration} in \isi{Lewis} English, but rather less among \isi{Skye} and \isi{Harris} speakers.
\item Epenthetic vowels: In Gaelic\il{Scottish Gaelic (Modern)}, epenthesis affects post-vocalic underlying VC\textsubscript{1}C\textsubscript{2} sequences: the vowel must be short, C\textsubscript{1} must be a nasal or liquid, and C\textsubscript{2} must be a non-homorganic consonant apart from /pʰ tʰ kʰ/. In such instances, an epenthetic vowel will be inserted between the two consonants and which matches the vowel preceding the CC cluster, e.g. \textit{arm}  /arm/ ‘army’ becomes [a.ram], \textit{airgead}  /ɛrʲkʲət/ ‘money’ becomes [ɛrʲɛkʲət]. In HE, a similar, though not identical, pattern obtains among some speakers, though it seems to apply in slightly different phonological conditions: an epenthetic vowel may appear in VC\textsubscript{1}C\textsubscript{2} sequences in which C\textsubscript{1} is a nasal or liquid, just as in Gaelic\il{Scottish Gaelic (Modern)}, though in HE the CC may be homorganic: \textit{milk} [mɪlɪk], \textit{farm} [faram], \textit{kiln} [kɪlɪn] (\citealt{Hammond:2014, Smith:1999, Bosch:1997}).

\item Dental stops: the coronals /t d/ may be dentalized as in Gaelic\il{Scottish Gaelic (Modern)}, e.g. \textit{tree} [t̪ʰɾi] (\citealt{Grant:1913, Holmer:1957, Dorian:1978, Holmer:1981, Shuken:1984}, \citealt{Ladefoged:1998}).

\item Retroflex /s/: HE frequently features retroflexion of /s/ after /r/, as in \textit{first} [fɹ̩ʂt], \textit{horse} [hɔɹ̩ʂ], \textit{parcel} [paɹ̩ʂəl], a pattern also found in Gaelic\il{Scottish Gaelic (Modern)}, e.g. \textit{spors} [spoːɹʂ] ‘sport’, \textit{dearrsach} [tʲaɹʂax] ‘bright’ (\citealt{Borgstrom:1940, Wells:1982, Shuken:1984, Stuart-Smith:2008}). 

\item Laterals: The lateral liquid common among HE speakers is a clear [l]. Elsewhere in Scotland, a velarized [ɫ] frequently predominates. While some varieties of Gaelic\il{Scottish Gaelic (Modern)} do have both velarized and non-velarized laterals, the velarized variant is rare in the northern \isi{Hebrides} (\cite{Holmer:1957, Holmer:1981, Ternes:1973, Dorian:1978, Wells:1982, Shuken:1984, Shuken:1985, O-Murchu:1989, Johnston:1997, Ladefoged:1998, Stuart-Smith:1999a, Nance:2014}). 
\end{enumerate}

\subsection{Prior research} 

\subsubsection{Published studies of Hebrides English}

Published in-depth studies of Hebrides English are notable for their scarcity. Those that are available point both to the distinctive character of the variety (in particular those features potentially of Gaelic\il{Scottish Gaelic (Modern)} origin, as described above), and to progressive convergence with mainland varieties through a process of dialect leveling, or what \citet{Hickey:2013} refers to supraregionalisation: a process in which a particular language variety loses those identifiably local features that have previously set it apart from a more broadly distributed (and perhaps less stigmatized) variety. Below I describe the work of \citet{Sabban:1985}, \citeauthor{Shuken:1984} (\citeyear{Shuken:1984, Shuken:1985, Shuken:1986}), and \citeauthor{Clayton:2015} (\citeyear{Clayton:2015, Clayton:2017, Clayton:2018}). I delve into the Clayton and Shuken studies in fair detail because they are the most relevant to the present chapter, particularly Shuken’s work. 

\subsubsection{\citet{Sabban:1985}}

\citet{Sabban:1985} is a morphosyntactic study of the English spoken in the Hebridean\is{Hebrides} islands of \isi{Skye} and North Uist. It is to date the only published study of HE morphosyntax, so far as I am aware. Sabban examined two patterns: the \textit{BE after V-ing} construction, and the absence of the present perfect (cf. \sectref{section 2.2}). Participants included 29 bilingual speakers, aged 6 to 89. Sabban found that these two patterns were overwhelmingly concentrated in speakers over 50, while speakers under 20 used them very rarely if at all. These results strongly suggest that these morphosyntactic features were obsolescent in these areas at the time of her study, a situation that is likely only to have advanced in the intervening four decades. 


%\dko{is the "cf. section 2.2 above" something that can be cross referenced in LaTex?}
\subsubsection{\citet{Shuken:1984, Shuken:1985, Shuken:1986}}

In the late 70s and early 80s, Cynthia Shuken conducted the first systematic sociophonetic study of Hebrides English (\citeyear{Shuken:1984, Shuken:1985, Shuken:1986}), with the aims of providing a thorough phonetic description of the variety, and identifying any key sociolinguistic variables that might be functioning as indices of identification with socioeconomic groups. She recorded nearly one hundred Hebridean English speakers, representing the islands of \isi{Harris}, \isi{Lewis}, and \isi{Skye}. Her material therefore is confined to the English spoken in these three islands. The participant pool was balanced between male and female speakers, and grouped into three age brackets (17--25, 35--45, and 55+). In addition, participants included rural speakers from all three islands, and urban speakers drawn from Stornoway, the chief town on \isi{Lewis}; and from Portree, the principal population center on \isi{Skye}. The \isi{Skye} and \isi{Lewis} urban speakers were further subdivided into English monolinguals and English-Gaelic\il{Scottish Gaelic (Modern)} bilinguals. Since no comparable urban center exists on \isi{Harris} (its largest community, Tarbert, had a population of 464 in the 1981 census), Shuken’s 14 \isi{Harris} speakers were all categorized as rural bilinguals. At the time of the study, the rural areas of all three islands were overwhelmingly bilingual, and so she made no effort to include rural monolinguals. Her collected material included recordings of wordlists, reading passages, and unstructured conversation. 

Three publications resulted from this study. The first, \citet{Shuken:1984}, provides a brief history of the variety as well as a profile of the phonology and morphosyntax of HE. No quantitative analysis is presented in this paper. 

The second publication, \citet{Shuken:1985}, was a sociophonetic analysis of HE based on data from the youngest age bracket in her participant group, i.e. 17--25 year olds, including rural bilinguals, urban bilinguals, and urban monolinguals. This paper compared the relative abundance of two HE variables of potential Gaelic\il{Scottish Gaelic (Modern)} provenance, the unvelarized or clear [l] and the \isi{preaspiration} of voiceless stops, with two variables of likely external origin, namely the velarized lateral [ɫ] and the \isi{glottalization} of voiceless stops (both in the form of glottal replacement or T-glottaling, and glottal reinforcement). 

Shuken found that the use of velarized [ɫ] rather than clear [l] was much more abundant among urban speakers compared to rural speakers in both \isi{Skye} and \isi{Lewis}, much more abundant on \isi{Skye} overall than \isi{Harris} or \isi{Lewis}, and most abundant among urban monolinguals from \isi{Skye}. Conversely, clear [l] dominated among rural speakers, especially those from \isi{Lewis}.

Shuken found that \isi{preaspiration} was more common among \isi{Lewis} speakers than those from \isi{Harris} or \isi{Skye}, that it was slightly more common among rural speakers than urban speakers in \isi{Lewis}, and that it was of similar frequency in both rural and urban speakers in \isi{Skye}. On the other hand, she found that \isi{glottalization} was much more abundant among \isi{Skye} speakers than those from \isi{Harris} or \isi{Lewis}, and more abundant among urban speakers than rural speakers in both \isi{Lewis} and \isi{Skye}. Finally, she found that glottal replacement was more common among \isi{Skye} speakers than those from \isi{Harris} or \isi{Lewis} and more common among urban speakers than rural speakers. 

Given the greater relative abundance of velarized [ɫ] and glottalized\is{glottalization} forms among young speakers in urban communities, and among those in greater proximity to the mainland, Shuken concluded that there was evidence for a significant and growing degree of mainland influence on HE at the time of this study. 

\citet{Shuken:1986} examined the phenomenon of vowel lengthening among HE speakers in relation to the patterns seen in Gaelic\il{Scottish Gaelic (Modern)}, in Scottish Standard English, and in Received Pronunciation. She found that a small number of word pairs like \textit{great {\textasciitilde} grate} and \textit{maid {\textasciitilde} made} were in contrast among some speakers, such that one member of the pair reliably differed from the other in vowel length. Shuken suggested that the feature is potentially due to influence from Gaelic\il{Scottish Gaelic (Modern)}, where contrastive vowel length is pervasive, e.g. \textit{srac} [st̪raxk] ‘rend!’ versus \textit{stràc} [st̪raːxk] ‘stroke’. However, in HE such unconditioned contrasts were seen to be sporadic, and with no consistent pattern between speakers. 

Apart from the few instances of contrastive lengthening, Shuken found two patterns of conditioned vowel length which varied in frequency by island. Speakers from \isi{Harris} and \isi{Lewis} were most likely to follow the pattern widely observed in Scottish Standard English, wherein (most) vowels are predictably lengthened before voiced fricatives, /r/, and morpheme boundaries \citep{Aitken.1981}. Meanwhile, \isi{Skye} speakers of HE were more likely to follow the Received Pronunciation pattern of lengthening vowels predictably before voiced consonants and in open syllables. In other words, the RP pattern had a greater degree of influence in \isi{Skye}, the island closest to the mainland, than in either of the two more remote islands. 

In general, then, Shuken’s published work on HE points to a linguistic situation in considerable flux, with evidence of expanding influence from mainland varieties of English. 


\subsubsection{\citet{Clayton:2015, Clayton:2017, Clayton:2018}}

The only work on HE to be published since the mid-1980s is found in \citet{Clayton:2015, Clayton:2017, Clayton:2018}. This study surveyed 24 bilingual speakers of HE and Gaelic\il{Scottish Gaelic (Modern)}, 10 male and 14 female. A range of islands within the Hebrides archipelago were represented, though unfortunately it was extremely difficult to identify more than one or two speakers from some of the islands. These islands included \isi{Skye} (8 speakers), \isi{Harris} (1), \isi{Lewis} (8), North Uist (2), South Uist (1), Grimsay (1), Raasay (1), Tiree (1), and Vatersay (1). The study addressed the incidence of \isi{preaspiration} and of \isi{glottalization} (both as glottal reinforcement and as T-glottaling). Preaspiration\is{preaspiration} was found to be abundant only among a subset of older female speakers from \isi{Lewis}, suggesting both that the feature is concentrated in the English of \isi{Lewis} (cf. \citealt{Borgstrom:1940, Shuken:1984, Sabban:1985}), and that \isi{preaspiration} is falling out of use among younger speakers. Glottal reinforcement was most abundant in younger speakers, particularly younger female speakers, though no clear pattern was found indicating that these features were comparatively more abundant in \isi{Skye} than in \isi{Harris} or \isi{Lewis} (cf. \citealt{Shuken:1985} with a similar result). Finally, T-glottaling was highly restricted in its occurrence: it only occurred in the word \textit{rotten}, and predominantly among \isi{Skye} speakers. Neither age nor sex appeared to affect the frequency of T-glottaling. The greater abundance among younger speakers suggests that glottal reinforcement, well-established in urban varieties of Scottish English \citep{Stuart-Smith:2008, Stuart-Smith:2003, Stuart-Smith:1999a, Stuart-Smith:1999b, Wells:1982}, are advancing in Hebrides English as well.

\section{The Shuken archive}\label{theshukenarchive}

\subsection{Overview}

Cynthia Shuken’s collected materials are archived at University of Edinburgh’s School of Scottish Studies. They include about 85 tape reels, of which 40 contain sociolinguistic interviews relevant to HE; detailed IPA transcriptions of some of the recorded interviews; 92 completed sociolinguistic questionnaires; and an assortment of fieldwork notes. 

The tapes in the archive seem to be copies in most instances rather than originals. Often, material from one speaker is scattered across several tape-reels. In a small number of cases, identical material occurs on more than one tape. While the tape-reels are in labeled boxes, in most cases the labels correspond only partially, or else not at all, with the material actually contained on the tape. These factors presented significant organizational challenges which had to be overcome before the present analysis could commence.

All tape-reels in the collection were digitized by the very helpful archivists at the first author’s request; these digitized versions were used for all the new analysis presented in this chapter.

\subsection{Recording procedures}

While a small number of Shuken’s participants were recorded in a laboratory setting at the University of Edinburgh, most were recorded in the field using a portable Uher reel-to-reel tape recorder and a table-top Sennheiser microphone. Recording locations varied widely, and included the participants’ homes, workplaces, and local primary schools among other settings. Most recordings were made in quiet conditions, but given the nature of fieldwork, some recordings inevitably feature a degree of ambient noise, such as doors slamming, background conversations, vehicle engines idling, and so forth. In no case, however, did such background noises significantly impede the type of analysis carried out for this chapter. The exact model and specifications of the recorder and microphone are unavailable at this point, but it seems safe to assume that they were industry standard for the era. The overall audio quality of the recorded material is generally quite high.

During a typical recording session, Shuken would first ask the participant to state their name and place of residence. She would then have them read two passages in English. Gaelic\il{Scottish Gaelic (Modern)}-speaking participants would then read a list of Gaelic\il{Scottish Gaelic (Modern)} words, and then move on to the English word list. All words were given to participants as printed lists without using a frame sentence; all participants encountered identical lists. Usually, Shuken would conclude each session by having participants complete a sociolinguistic questionnaire; frequently, the recording includes the participant’s commentary on the questionnaire, though not always.

\subsection{Questionnaires}

Shuken used a detailed 7-page sociolinguistic questionnaire to collect demographic information about her participants as well as their cultural and political attitudes. Two versions of the questionnaire were prepared: one for the use of Gaelic\il{Scottish Gaelic (Modern)}-speaking bilinguals, and one for English-speaking monolinguals. Among other things, the questionnaire asked about each participant’s place of birth and that of their parents; their residence history, education and employment; their degree of personal identification with the local community and region, with Scotland, and with the United Kingdom; their preferences and degree of knowledge of cultural matters such as traditional Highland music, literature, and history; the frequency with which they spoke Gaelic\il{Scottish Gaelic (Modern)} (if bilingual) and with whom; their perceptions of the relative attractiveness and correctness of various dialects of Gaelic\il{Scottish Gaelic (Modern)} and English; and their degree of pride in speaking a distinct variety of English. The first page of the bilingual version of the questionnaire is reproduced in \figref{fig:1} (page~\pageref{fig:1}). 

\begin{figure}[hp]
\includegraphics[width=\textwidth]{figures/Clayton-img001.jpg}
\caption{Page from Shuken's Gaelic-English bilingual questionnaire}
\label{fig:1}
\end{figure}

The archive contains completed sociolinguistic questionnaires for 92 individuals: 46 from \isi{Lewis}, 14 from \isi{Harris}, and 32 from \isi{Skye}. While the questionnaires are numbered consecutively (L1-L50 for \isi{Lewis}, H1-H14 for \isi{Harris}, and S1-S50 for \isi{Skye}), there are gaps for 4 questionnaires from the \isi{Lewis} group and 18 from the \isi{Skye} group, suggesting that the corresponding recordings were not made. Recorded interviews corresponding to almost all of the completed questionnaires could be located on the tape reels, with the exception of three participants from \isi{Lewis}, one from \isi{Harris}, and three from \isi{Skye}; however, Shuken’s notes include complete, highly detailed transcriptions of the English wordlist and the two reading passages for one of the three missing \isi{Skye} speakers (S12), and for the \isi{Harris} speaker (H6). Thus, there is audio available for 85 interviews plus detailed transcripts of two more interviews.

\subsection{Participants}

Shuken’s full participant pool includes 92 speakers of HE from the islands of \isi{Lewis}, \isi{Harris}, and \isi{Skye}, as represented in the archive (if we include the seven participants for whom we have questionnaires but no recordings). The pool was designed to be evenly balanced for sex and was stratified by age group (17--25, 35--45, 55+). In addition, Shuken included rural speakers from all three islands as well as urban speakers drawn from Stornoway, the chief town on \isi{Lewis}, and from Portree, the principal population center on \isi{Skye}. The urban speakers included both monolingual HE speakers and bilingual speakers of HE and Gaelic\il{Scottish Gaelic (Modern)}.

A bit more could be said at this point about the towns of Stornoway (on \isi{Lewis}) and Portree (on \isi{Skye}). At the time of Shuken’s study, the parish of Stornoway had a population of approximately 7900, according to the 1981 census \citep{Shuken:1985}. The town’s economy was, and still is, centered on retail, social services, tourism, textiles, aquaculture, and offshore gas and oil extraction (\citealt{HIE:2020}). Stornoway’s population has declined somewhat since 1981; the estimated population as of 2020 was 7280 (\citealt{NationalRecordsofScotland:2023}).

Like Stornoway, Portree is a regional commercial center, though more heavily dependent on tourism and arguably more subject to outside influence as a consequence. The completion of a road bridge connecting \isi{Skye} to the mainland in 1995 has only strengthened the commercial and social ties between \isi{Skye} and the mainland since Shuken conducted her fieldwork (\citealt{HIE:2007}) and thus magnified the potential for mainland linguistic influence. The 1981 census recorded a population of 1414 in Portree compared to an estimated 2,310 in 2020 (\citealt{NationalRecordsofScotland:2023}), indicating a significant pattern of growth in the community.

Shuken was able to record participants representing nearly every possible combination of demographic factors in her planned speaker population. For most combinations of the factors of island, age, rural\slash urban, and monolingual\slash bilingual, her recorded interviews include at least one male and one female speaker. The majority achieved her target of at least two male and two female speakers for each of these conditions. Where unfilled gaps do occur, these include female urban monolinguals from \isi{Lewis} in the 55--65 age group, male and female rural bilinguals from \isi{Skye} in the 17--25 age group, \isi{Skye} rural bilingual males in the 35--45 age group, and \isi{Skye} urban monolingual males in the 55--65 age group; no speakers representing these subgroups are present in the recorded material. Conversely, in a small number of conditions “surplus” speakers were recorded, i.e. more than the target of two women and two men per condition. For the present study, the data from only two male and two female speakers per condition were included in the analysis, in order to prevent overrepresenting any subpopulation. 

The final speaker pool for the present study contains 70 individuals, with the distribution shown in \tabref{tab:clayton:1}. 

\begin{table}
\small
\begin{tabular}{l *7{c}}
\lsptoprule
{Age/sex}
& {LRB} & {LUB} & {LUM} & {SRB} & {SUB} & {SUM} & {HRB}\\
\midrule
17--25 F & L1 L2 &  L13 L14 &  L25 L26 &  - &  S1 &  S25 S26 &  H1 H2\\
17--25 M & L3 L4 &  L15 &  L27 L28 &  - &  S3 S15 &  S27 S28 &  H3 H4\\\addlinespace
35--45 F & L5 L6 &  L17 &  L29 L30 &  S37 S38 &  S5 S6 &  S48 &  H5 H6\\
35--45 M & L7 L8 &  L19 L20 &  L31 L32 &  - &  S7 S19 &  S31 S49 &  H7 H8\\\addlinespace
55+ F & L9 L10 &  L21 L22 &  - &  S9 S10 &  S40 S42 &  S33 S34 &  H9 H10\\
55+ M & L11 L12 &  L23 L24 &  L35 L36 &  S11 S12 &  S41 S50 &  - &  H11 H12\\
\lspbottomrule
\end{tabular}
\caption{Distribution of participants according to island (\isi{Lewis}, \isi{Skye}, \isi{Harris}), age group, sex,  and urban/rural, bilingual/monolingual (e.g. LRB = \isi{Lewis}, rural, bilingual)}
\label{tab:clayton:1}
\end{table}

\subsection{Stimuli and reading passages}

Shuken’s full stimulus set consists of 319 English words.\footnote{The full set of stimuli is available upon request to the first author.} While the majority of the stimuli presented no difficulties for participants, the recordings do indicate that several of the words in the list were unfamiliar to many people, who struggled to read them accurately as a consequence. Problematic words included \textit{quoit, adroit, gesture, loathe,} and \textit{rouge.} These words were excluded from the present analysis. In addition, the word \textit{use} was excluded because it was orthographically ambiguous: participants were unsure whether the verb (with a final /z/) was meant, or the noun (with a final /s/), and so the word could not be reliably analyzed. 

In addition to the list of individual words, Shuken also presented participants with two short reading passages. Analysis of the data offered by these reading passages is ongoing, and is not included in this chapter. For the three variables considered in the present analysis, the relevant subsets of stimuli are illustrated in Tables~\ref{ex:clayton:stimuli-preaspiration} and~\ref{ex:clayton:stimuli-voicing}.


\begin{table}
\caption{Stimuli for preaspiration and glottalization of /p t k/}
\label{ex:clayton:stimuli-preaspiration}
\begin{tabular}{l llll ll}
\lsptoprule
\multicolumn{1}{c}{/p/}              &  \multicolumn{4}{c}{/t/}               &  \multicolumn{2}{c}{/k/}\\
\cmidrule(lr){1-1}\cmidrule(lr){2-5}\cmidrule(lr){6-7}
\itshape chip   & \itshape about   & \itshape boot     & \itshape grate      & \itshape soot  & \itshape actually  & \itshape plucking  \\
\itshape cup     & \itshape bait    & \itshape booty    & \itshape great      & \itshape suit  & \itshape choke     & \itshape spoke (n.)\\
\itshape dip     & \itshape beat    & \itshape caught   & \itshape idiot      & \itshape water & \itshape creak     & \itshape spoke (v.)\\
\itshape loop    & \itshape beauty  & \itshape coat     & \itshape illuminate &                & \itshape creek     & \itshape tack      \\
\itshape ship    & \itshape beet    & \itshape cot      & \itshape mate       &                & \itshape hawk      & \itshape taxes     \\
\itshape staple  & \itshape bet     & \itshape crate    & \itshape petal      &                & \itshape joke      & \itshape taxis     \\
\itshape stoop   & \itshape bit     & \itshape cut      & \itshape put        &                & \itshape leak      & \itshape weak      \\
\itshape stripes & \itshape bite    & \itshape feature  & \itshape quite      &                & \itshape leek      & \itshape week      \\
\itshape tip     & \itshape boat    & \itshape foot     & \itshape right      &                & \itshape lock      & \\
\lspbottomrule
\end{tabular}
\end{table}

\begin{table}
\caption{Stimuli used to evaluate voicing contrasts in sibilants}
\label{ex:clayton:stimuli-voicing}
\begin{tabular}{llllll}
\lsptoprule
\multicolumn{2}{c}{final /z/}                 & initial /z/    & final /dʒ/     & medial /dʒ/        & medial /ʒ/      \\
\midrule
\itshape boys      & \itshape rays            & \itshape zeal  & \itshape badge & \itshape dredges   & \itshape measure\\
\itshape cause     & \itshape resume          &                & \itshape cadge &                    &                 \\
\itshape caws      & \itshape rose            &              \\
\itshape days      & \itshape rouse           &              \\
\itshape daze      & \itshape rows (boat)     &              \\
\itshape enthuse   & \itshape rows (fights)   &              \\
\itshape ewes      & \itshape seas            &              \\
\itshape fez       & \itshape seize           &              \\
\itshape fuzz      & \itshape sighs           &              \\
\itshape halves    & \itshape size            &              \\
\itshape noise     & \itshape stretches       &              \\
\itshape ooze      & \itshape ways            &              \\
\itshape presume   & \itshape weighs          &              \\
\itshape raise     &                          &              \\
\lspbottomrule
\end{tabular}
\end{table}


\section{Analysis}\label{analysis}

\subsection{Analysis of preaspiration}

In Gaelic\il{Scottish Gaelic (Modern)}, the realization of \isi{preaspiration} tends to vary depending both on the dialect in question and the affected consonant. In many dialects, particularly those of the eastern mainland now moribund or extinct, \isi{preaspiration} does not occur at all. Among the varieties of Gaelic\il{Scottish Gaelic (Modern)} found in \isi{Lewis}, \isi{Harris}, and \isi{Skye}, there are two main \isi{preaspiration} patterns. On the one hand is the \isi{Lewis} pattern, in which \isi{preaspiration} is typically realized as a period of glottal frication or breathy voice [VhC {\textasciitilde} VɦC] occurring between a vowel and a following voiceless stop. On the other hand, we have the pattern found in \isi{Harris} and \isi{Skye}, where before labial and coronal stops \isi{preaspiration} appears much as in \isi{Lewis}, i.e. as glottal frication or breathy voice, but before dorsals it is typically realized as homorganic oral frication, i.e. [çc xk] (\citealt{Nance:2013, Clayton:2010, Ladefoged:1998, O-Dochartaigh:1994, Ni-Chasaide:1985, Borgstrom:1940}). The oral form [xp xt xk] alone is typical of certain dialects, particularly in Perthshire and Argyll (\citealt{Iosad:2015, O-Murchu:1989}). 

In HE, by contrast, \isi{preaspiration} seems to be rather less variable. There it typically takes the form of glottal frication or breathy voice (\figref{fig:2}). Homorganic oral frication does infrequently occur, particularly with dorsals (\citealt{Clayton:2018, Clayton:2017, Clayton:2015, Shuken:1985}).

\begin{figure}
\includegraphics[width=.8\textwidth]{figures/Clayton-img002.png}
\caption{Preaspiration in the word \textit{boot} as produced by L02, a female rural bilingual from \isi{Lewis} in the 17--25 age group.}
\label{fig:2}
\end{figure}

For the present study, tokens of \isi{preaspiration} were identified through inspection of the waveform and spectrogram in Praat (\cite{Boersma:2020}). Of the 319 stimuli produced per speaker, 57 contained the voiceless stops /p t k/ post-vocalically in medial or final position, and thus were potential contexts for \isi{preaspiration}. Of these, 9 stimuli contained /p/, 31 contained /t/, and 17 contained /k/ (\tabref{ex:clayton:stimuli-preaspiration}). Totaled across the 70 participants, there were thus 3990 relevant tokens. Of these, 658 tokens (16.5\%) were identified as featuring some form of \isi{preaspiration} (\tabref{tab:clayton:4}). Coronal tokens were by far the most numerous, numbering 452 (68.7\%) of the total, followed by dorsals (152 tokens, 23.1\%), then labials (54 tokens, 8.2\%). No labial tokens at all were noted among the \isi{Harris} and \isi{Skye} speakers, and very few dorsals (4 out of 29, 13.8\%) were present in those two groups. This distribution favoring coronals is similar to that found for HE in \citeauthor{Clayton:2017} (\citeyear{Clayton:2017, Clayton:2018}), but stands in sharp contrast to the pattern typically observed in Gaelic\il{Scottish Gaelic (Modern)}, where dorsals are favored in terms of frequency \citep{Clayton:2010}. 

\begin{table}
\begin{tabular}{lllllllr}
\lsptoprule
& \multicolumn{2}{c}{{\isi{Lewis}}} & \multicolumn{2}{c}{{\isi{Harris}}} & \multicolumn{2}{c}{{\isi{Skye}}} & \\
& {M} & {F} & {M} & {F} & {M} & {F} & {Total}\\
\midrule
 \multicolumn{8}{l}{Age 1 (17--25)}\\
 p &  7 &  13 &  0 &  0 &  0 &  0 &  20\\
 t &  50 &  79 &  1 &  5 &  0 &  0 &  135\\
 k &  15 &  27 &  0 &  0 &  0 &  0 &  42\\
 Total &  72 &  119 &  1 &  5 &  0 &  0 &  197\\
 \% &  25.26 &  34.80 &  0.88 &  4.39 &  0.00 &  0.00 & \\
 \multicolumn{8}{l}{Age 2 (35--45)}\\
 p &  6 &  10 &  0 &  0 &  0 &  0 &  16\\
 t &  57 &  81 &  1 &  4 &  2 &  1 &  146\\
 k &  23 &  26 &  0 &  0 &  1 &  0 &  50\\
 Total &  86 &  117 &  1 &  4 &  3 &  1 &  212\\
 \% &  25.15 &  41.05 &  0.88 &  3.51 &  1.32 &  0.44 & \\
 \multicolumn{8}{l}{Age 3 (55+)}\\
 p &  7 &  11 &  0 &  0 &  0 &  0 &  18\\
 t &  92 &  68 &  3 &  5 &  0 &  3 &  171\\
 k &  28 &  29 &  0 &  0 &  0 &  3 &  60\\
 Total &  127 &  108 &  3 &  5 &  0 &  6 &  249\\
 \% &  37.13 &  47.37 &  2.63 &  4.39 &  0.00 &  1.50 & \\
&  &  &  &  &  &  Total: &  658\\
\lspbottomrule
\end{tabular}
\caption{Distribution of preaspirated tokens according to island, sex, age group, and place of articulation.}
\label{tab:clayton:4}
\end{table}

\begin{figure}
\includegraphics[width=\textwidth]{figures/Clayton-img003.pdf}
\caption{Rates of preaspiration according to island (\isi{Lewis}, \isi{Harris}, \isi{Skye}), sex, and age group (1 = 17--25, 2 = 35--45, 3 = 55+)}
\label{fig:3}
\end{figure}

We see a very uneven distribution of \isi{preaspiration} according to social factors (Figures~\ref{fig:3} and \ref{fig:4}). Preaspiration\is{preaspiration} was overwhelmingly localized to \isi{Lewis} speakers, who produced 629 tokens of the total 658 observed (95.6\%). Each of the \isi{Lewis} speakers produced a minimum of 2 preaspirated\is{preaspiration} tokens, with the exception of L20, an urban bilingual male in the 35--45 age bracket, who produced none. 

\begin{figure}
\begin{subfigure}{.5\textwidth}\centering
\includegraphics[width=\linewidth]{figures/Clayton-img004anew.pdf}
\caption{}
\end{subfigure}%
\begin{subfigure}{.5\textwidth}\centering
\includegraphics[width=\linewidth]{figures/Clayton-img004b.pdf}
\caption{}
\end{subfigure}\medskip\\
\begin{subfigure}{.5\textwidth}\centering
\includegraphics[width=\linewidth]{figures/Clayton-img004c.pdf}
\caption{}
\end{subfigure}%
\begin{subfigure}{.5\textwidth}\centering
\includegraphics[width=\linewidth]{figures/Clayton-img004d.pdf}
\caption{}
\end{subfigure}
\caption{Comparisons of preaspiration rates by island (\isi{Lewis}, \isi{Harris}, \isi{Skye}), age group (1 = 17--25, 2 = 35--45, 3 = 55+), sex, and rural vs. urban.}
\label{fig:4}
\end{figure}

Statistical analysis of the effects of demographic variables on \isi{preaspiration} rate (\tabref{tab:clayton:5}) was conducted using a logistic mixed-effects model via the glmer function of the \texttt{lme4} package (\citealt{Bates:2015, Kuznetsova:2017}). Because of the very low rates of \isi{preaspiration} in \isi{Skye} and \isi{Harris} speakers, these islands were excluded from statistical analysis. The final model included \textit{word} as a random effect, and \textit{sex}, \textit{age}, and \textit{rural/urban} as fixed effects. Age was included as a 3-way variable instead of continuous variable, to correspond with the three age bins among Shuken’s participants; in addition, it was not always possible to identify the precise age of some speakers at the time of recording beyond the age bracket they occupied. While \textit{speaker} is commonly included as a random effect in mixed-effect models \citep{Baayen_etal2008_random_effects}, its inclusion in the models used here tended to obscure the very variability the study hopes to highlight, and so \textit{speaker} was not included as a factor in this and subsequent statistical models. 

\begin{table}
\begin{tabular}{l S[table-format=-1.3] S[table-format=1.3] S[table-format=-1.3] S[table-format=<1.3{***}]}
\lsptoprule
{Random effects} &  & {Variance} & {SD} & \\
\midrule
{word} &  & 4.349 & 2.085 & \\
\\
{Fixed effects} & {Estimate} & {SE} & {$z$} & {$p$}\\
\midrule
(Intercept) & -0.030 & 0.373 &  -0.081 & 0.936\\
sexM & -1.389 & 0.344 & -4.039 & <0.001{***} \\
rurbU & -1.869 & 0.305 & -6.122 & <0.001{***} \\
age2 & -0.591 & 0.340 & -1.738 & 0.082\\
age3 & -0.301 & 0.342 & -0.880 & 0.379\\
sexM:rurbU & 1.014 & 0.441 & 2.297 & 0.022{*} \\
sexM:age2 & 1.562 & 0.481 & 3.249 & <0.001{***} \\
sexM:age3 & 1.568 & 0.484 & 3.237 & <0.001{***} \\
rurbU:age2 & 1.471 & 0.430 & 3.422 & <0.001{***} \\
rurbU:age3 & 1.636 & 0.454 & 3.607 & <0.001{***} \\
sexM:rurbU:age2 & -3.013 & 0.627 & -4.808 & <0.001{***} \\
sexM:rurbU:age3 & -2.224 & 0.632 & -3.519 & <0.001{***} \\
\lspbottomrule
\end{tabular}
\caption{Mixed-effects model of preaspiration rates among \isi{Lewis} speakers. Intercept represents younger female rural speakers.}
\label{tab:clayton:5}
\end{table}

Among \isi{Lewis} speakers, the frequency of \isi{preaspiration} was skewed toward women overall (\EstimateZP{-1.389}{-4.039}{<0.001}) (\figref{fig:4}c), and toward older speakers (\figref{fig:4}b). Older male speakers in particular preaspirated\is{preaspiration} at significantly greater rates than the younger two groups of male speakers (\EstimateZP{1.568}{3.237}{<0.001}). Urban speakers also preaspirated\is{preaspiration} significantly less than rural speakers (\EstimateZP{-1.869}{-6.122}{<0.001}).

\subsection{Analysis of glottalization}

Just as for \isi{preaspiration}, \isi{glottalization} is frequently associated with voiceless stops in HE. Effectively then the number of possible sites for \isi{glottalization} in the current dataset is 3990, the same as for \isi{preaspiration}. Of these 3990 tokens, 336 featured some degree of \isi{glottalization} of the vowel just prior to the stop closure. Outright replacement of the consonant with a glottal stop (T-glottaling) was extremely rare in the Shuken data analyzed for this chapter and so not analyzed separately. Glottalization\is{glottalization} was identified in the data via irregularities in the timing of vocal fold pulsations (so-called “jitter”), or in the amplitude of the associated waveforms (“shimmer”). Fluctuations in the pitch track provided by Praat could also be observed in some glottalized\is{glottalization} tokens. \figref{fig:5} illustrates \isi{glottalization} in the word \textit{leak} as produced by S25, an urban monolingual female speaker from \isi{Skye} in the 17--25 age group. 

\begin{figure}[ht]
\includegraphics[height=.35\textheight]{figures/Clayton-img005.png}
\caption{Glottalization in the word \textit{leak} as produced by S25, a female monolingual from \isi{Skye} in the 17--25 age group.}
\label{fig:5}
\end{figure}

Statistical analysis of the \isi{glottalization} data (\tabref{tab:clayton:7}) was conducted by fitting a logistic mixed-effects model via the glmer function of the \texttt{lme4} package. The final model included \textit{word} as random effect; and \textit{age group}, \textit{sex}, and \textit{bilingual/monolingual} as fixed effects. Interactions between the fixed effects were also modeled.

In terms of place of articulation, the distribution of \isi{glottalization} is considerably more even than \isi{preaspiration}: 191 of 2170 coronal tokens displayed \isi{glottalization} (8.8\%), 90 of 1190 dorsal tokens (7.6\%), and 55 of 630 labial tokens (8.7\%). No significant difference by place of articulation was found. 

In terms of demographics, however, the distribution of \isi{glottalization} was in certain respects the inverse of \isi{preaspiration} (\figref{fig:6}). \isi{Skye} speakers glottalized\is{glottalization} at far greater rates than \isi{Harris} or \isi{Lewis} speakers (\tabref{tab:clayton:6}; \figref{fig:7}a), glottalizing 251 observed tokens (74.7\%), compared to 29 tokens (8.6\%) among the \isi{Harris} speakers and 56 (16.7\%) among speakers from \isi{Lewis}. Age was a significant factor (\figref{fig:7}b), where the older two brackets seem to have glottalized\is{glottalization} at proportionally higher rates, but the differences are very small (age 2:  \EstimateZP{1.270}{3.357}{<0.001}; age 3: \EstimateZP{0.922}{2.445}{= 0.014}). While women glottalized\is{glottalization} at slightly higher rates than men (\figref{fig:7}c), sex was not a statistically significant factor. \isi{Skye} women did lead \isi{Skye} men in the younger two age brackets, though \isi{glottalization} was more common among older \isi{Skye} men than women. Outside of \isi{Skye}, \isi{glottalization} could be described as abundant only among younger \isi{Lewis} women; notably, 31 of the 35 tokens attributed to that group were produced by one individual, L26, a monolingual urban speaker. The other apparent cluster is among the older \isi{Harris} women, though here all 11 tokens in that group were produced by one individual, H09. The other speaker in the older female \isi{Harris} group did not glottalize at all. Finally, monolinguals glottalized\is{glottalization} at significantly higher rates than bilinguals (\figref{fig:7}d; \EstimateZP{3.628}{9.975}{<0.001}) at all age groups and in both male and female speakers.

\begin{table}
\begin{tabular}{llllllll} 
\lsptoprule
& \multicolumn{2}{c}{{\isi{Lewis}}} & \multicolumn{2}{c}{{\isi{Harris}}} & \multicolumn{2}{c}{{\isi{Skye}}} & \\
& {M} & {F} & {M} & {F} & {M} & {F} & {Total}\\
\midrule
Age 1 (17--25) &  &  &  &  &  &  & \\
p &  2 &  6 &  0 &  1 &  3 &  7 &  19\\
t &  1 &  22 &  0 &  5 &  18 &  40 &  86\\
k &  1 &  7 &  0 &  1 &  9 &  17 &  35\\
Total &  4 &  35 &  0 &  7 &  30 &  64 &  140\\
\% &  1.40 &  10.23 &  0.00 &  6.14 &  13.16 &  37.43 & \\
Age 2 (35 -45) &  &  &  &  &  &  & \\
p &  1 &  3 &  1 &  2 &  5 &  1 &  13\\
t &  4 &  2 &  1 &  3 &  12 &  16 &  38\\
k &  2 &  1 &  0 &  2 &  6 &  15 &  26\\
Total &  7 &  6 &  2 &  7 &  23 &  32 &  77\\
\% &  25.15 &  2.11 &  0.88 &  6.14 &  1.32 &  0.44 & \\
Age 3 (55+) &  &  &  &  &  &  & \\
p &  0 &  0 &  0 &  2 &  12 &  9 &  23\\
t &  2 &  2 &  2 &  9 &  29 &  23 &  67\\
k &  0 &  0 &  0 &  0 &  17 &  12 &  29\\
Total &  2 &  2 &  2 &  11 &  58 &  44 &  119\\
\% &  0.01 &  0.88 &  1.75 &  9.65 &  25.44 &  11.03 & \\
&  &  &  &  &  & Total:  &  336\\
\lspbottomrule
\end{tabular}
\caption{Distribution of glottalized tokens according to island, sex, age group, and place of articulation.}
\label{tab:clayton:6}
\end{table}

\begin{figure}
\includegraphics[height=.35\textheight]{figures/Clayton-img006.pdf}
\caption{Comparison of glottalization rates by island (\isi{Lewis}, \isi{Harris}, \isi{Skye}), sex, and age group (1 = 17--25, 2 = 35--45, 3 = 55+)}
\label{fig:6}
\end{figure}

\begin{figure}
\begin{subfigure}{.5\textwidth}\centering
\includegraphics[height=.25\textheight]{figures/Clayton-img007a.pdf}
\caption{}
\end{subfigure}%
\begin{subfigure}{.5\textwidth}\centering
\includegraphics[height=.25\textheight]{figures/Clayton-img007b.pdf}
\caption{}
\end{subfigure}\medskip\\
\begin{subfigure}{.5\textwidth}\centering
\includegraphics[height=.25\textheight]{figures/Clayton-img007c.pdf}
\caption{}
\end{subfigure}%
\begin{subfigure}{.5\textwidth}\centering
\includegraphics[height=.25\textheight]{figures/Clayton-img007d.pdf}
\caption{}
\end{subfigure}
\caption{Glottalization rates by island (\isi{Lewis}, \isi{Harris}, \isi{Skye}), age group (1 = 17--25, 2 = 35--45, 3 = 55+), sex, and rural vs. urban}
\label{fig:7}
\end{figure}

\begin{table}
\begin{tabular}{l S[table-format=-2.3] S[table-format=2.3] S[table-format=-2.3] S[table-format=<1.3{***}]}
\lsptoprule
{Random effects:} &  & {Variance} & {SD} & \\\midrule
{word} &  & {0.474} & {0.689} & \\
\\
{Fixed effects:} & {Estimate} & {SE} & {$z$} & {$p$}\\
\midrule
(Intercept) & -3.971 & 0.349 & -11.370 & <0.001\\
sexM & -0.258 & 0.504 & -0.513 & 0.608\\
age2 & 1.270 & 0.378 & 3.357 & <0.001{***} \\
age3 & 0.922 & 0.377 & 2.445 & 0.014{*} \\
langM & 3.628 & 0.364 & 9.975 & <0.001{***} \\
sexM:age2 & -1.270 & 0.640 & -1.984 & 0.047{*} \\
sexM:age3 & 1.035 & 0.550 & 1.881 & 0.060 \\
sexM:langM & -1.572 & 0.563 & -2.794 & 0.005{**} \\
age2:langM & -3.791 & 0.511 & -7.421 & <0.001{***} \\
age3:langM & -2.088 & 0.468 & -4.459 & <0.001{***} \\
sexM:age2:langM & 3.653 & 0.786 & 4.650 & <0.001{***} \\
sexM:age3:langM & -16.378 & 33.615 & -0.487 & 0.626\\
\lspbottomrule
\end{tabular}
\caption{Mixed-effects model of glottalization data. Intercept represents younger female bilingual speakers.}
\label{tab:clayton:7}
\end{table}

\subsection{Analysis of devoicing}

A commonly noted feature of HE is the absence among some speakers of voicing in obstruents, producing (near) mergers of /p t k/ {\textasciitilde} /b d ɡ/, /tʃ/ {\textasciitilde} /dʒ/, and /s/ {\textasciitilde} /z/ (\citealt{Wells:1982, Shuken:1984, Jones:2002, Stuart-Smith:2008}). More rarely, a reversal of voicing is remarked; e.g., /s/ is realized as [z]. The usual assumption is that this trait has been acquired from the local varieties of Gaelic\il{Scottish Gaelic (Modern)}, which also lack a voicing contrast in such consonants, though contrast is maintained among stops via aspiration or \isi{preaspiration}. 

While this lack of distinctive voicing has been noted to affect oral stops as well as sibilants in HE, Shuken’s collected material was particularly well designed to illustrate the feature among sibilants, especially in word-final position. Her stimulus set includes 33 potential contexts for \isi{devoicing}, including 27 instances of word-final /z/, 1 example of /z/ in initial position; 1 medial and 2 final examples of /dʒ/; and 1 example of medial /ʒ/ (\tabref{ex:clayton:stimuli-voicing}). There were thus 2310 potential occurrences of sibilant \isi{devoicing} in the analyzed data. Of these, 639 tokens actually exhibited \isi{devoicing}, identified as the absence of visible pulses or a voice bar in the waveform and spectrogram (\figref{fig:8}). 

\begin{figure}
\includegraphics[height=.35\textheight]{figures/Clayton-img008.png}
\caption{The word \textit{zeal} as produced by H2, a rural bilingual female speaker from \isi{Harris} in the 17--25 age group showing devoicing of the initial /z/.}
\label{fig:8}
\end{figure}

Devoicing\is{devoicing} was somewhat more equitably distributed across the demographic groups than either \isi{preaspiration} or \isi{glottalization} (\tabref{tab:clayton:8}). That said, there are clear patterns all the same (\figref{fig:9}, \figref{fig:10}). First, \isi{devoicing} seems to be more frequent among \isi{Lewis} speakers than among those from \isi{Harris} or \isi{Skye} (\figref{fig:10}a); more common in the young and middle age groups (\figref{fig:10}b); and more common among female than male speakers overall (\figref{fig:10}c). 

\begin{table}[ht]
\begin{tabular}{lllllll} 
\lsptoprule
& \multicolumn{2}{c}{{\isi{Lewis}}} & \multicolumn{2}{c}{{\isi{Harris}}} & \multicolumn{2}{c}{{\isi{Skye}}}\\
& {M} & {F} & {M} & {F} & {M} & {F}\\
\midrule
{Age 1 (17--25)} &  &  &  &  &  & \\
{Rural} & { 39.4} & { 56.1} & { 36.4} & { 22.7} & { na} & { na}\\
{Urban} & { 29.3} & { 44.7} & { na} & { na} & { 15.2} & { 38.4}\\
{Age 2 (35--45)} &  &  &  &  &  & \\
{Rural} & { 45.5} & { 48.5} & { 15.2} & { 21.2} & { na} & { 18.2}\\
{Urban} & { 24.2} & { 53.5} & { na} & { na} & { 4.5} & { 39.4}\\
{Age 3 (55+)} &  &  &  &  &  & \\
{Rural} & { 59.1} & { 43.9} & { 16.7} & { 19.7} & { 1.5} & { 19.7}\\
{Urban} & { 15.9} & { 25.8} & { na} & { na} & { 1.5} & { 14.5}\\
\lspbottomrule
\end{tabular}
\caption{Distribution of devoiced tokens according to island, sex, age group.}
\label{tab:clayton:8}
\end{table}

\begin{figure}[ht]
\includegraphics[height=.35\textheight]{figures/Clayton-img009.pdf}
\caption{Rates of devoicing by according to island (\isi{Lewis}, \isi{Harris}, \isi{Skye}), sex, and age group (1 = 17--25, 2 = 35--45, 3 = 55+)}
\label{fig:9}
\end{figure}

\begin{figure}[ht]
\begin{subfigure}{.5\textwidth}\centering
\includegraphics[height=.25\textheight]{figures/Clayton-img10anew.pdf}
\caption{}
\end{subfigure}%
\begin{subfigure}{.5\textwidth}\centering
\includegraphics[height=.25\textheight]{figures/Clayton-img010b.pdf}
\caption{}
\end{subfigure}\medskip\\
\begin{subfigure}{.5\textwidth}\centering
\includegraphics[height=.25\textheight]{figures/Clayton-img010c.pdf}
\caption{}
\end{subfigure}%
\begin{subfigure}{.5\textwidth}\centering
\includegraphics[height=.25\textheight]{figures/Clayton-img010d.pdf}
\caption{}
\end{subfigure}
\caption{Devoicing rates by island, age group, sex, and rural vs. urban.}
\label{fig:10}
\end{figure}

Statistical analysis of the \isi{devoicing} data (\tabref{tab:clayton:9}) was conducted using a logistic mixed\hyp effects model via the glmer function of the lme4 package. The final model included \textit{word} as random effect; and \textit{island}, \textit{age group}, \textit{sex}, and \textit{rural/urban} as fixed effects. Interactions between the fixed effects were also modeled. Overall, the model indicated that the effect of \textit{island} was significant: the rate of \isi{devoicing} was significantly higher in \isi{Lewis} than in \isi{Harris} (\EstimateZP{-1.755}{4.195}{<0.001}), though not in \isi{Skye} (\EstimateZP{-0.874}{-1.360}{=0.174}). We also see a general pattern of male speakers \isi{devoicing} at significantly lower rates than female speakers in \isi{Lewis} and in \isi{Harris}, though not on \isi{Skye} except among the older urban groups. The \isi{Skye} result may be distorted by the absence of rural male \isi{Skye} speakers in the younger and middle age groups, however. 

\begin{table}[hp]
\begin{tabular}{l S[table-format=-1.3] S[table-format=1.3] S[table-format=-1.3] S[table-format=<1.3{***}]}
\lsptoprule
{Random effects} &  & {Variance} & {SD} & \\
word &  & 1.217 & 1.103 & \\
\\
{Fixed effects} & {Estimate} & {SE} & {$z$} & {$p$}\\
\midrule
(Intercept) & 0.247 & 0.337 & 0.733 & 0.464\\
islH & -1.755 & 0.418 & -4.195 & < 0.001{***}\\
islS & -0.874 & 0.643 & -1.360 & 0.174\\
age2 & -0.376 & 0.388 & -0.970 & 0.332\\
age3 & -0.600 & 0.388 & -1.544 & 0.123\\
rurbU & -0.562 & 0.337 & -1.669 & 0.095\\
sexM & -0.826 & 0.390 & -2.115 & 0.034{*}\\
islH:age2 & 0.276 & 0.592 & 0.466 & 0.641\\
islS:age2 & -0.821 & 0.845 & -0.972 & 0.331\\
islH:age3 & 0.394 & 0.596 & 0.661 & 0.509\\
islS:age3 & -0.486 & 0.482 & -1.010 & 0.312\\
islS:rurbU & 0.560 & 0.570 & 0.981 & 0.327\\
age2:rurbU & 0.812 & 0.488 & 1.664 & 0.096\\
age3:rurbU & -0.404 & 0.528 & -0.765 & 0.444\\
islH:sexM & 1.600 & 0.573 & 2.793 & 0.005{**}\\
islS:sexM & -2.485 & 1.650 & -1.506 & 0.132\\
age2:sexM & 0.677 & 0.548 & 1.235 & 0.217\\
age3:sexM & 1.580 & 0.553 & 2.858 & 0.004{**}\\
rurbU:sexM & 0.028 & 0.496 & 0.057 & 0.954\\
islS:age2:rurbU & 0.436 & 0.852 & 0.512 & 0.609\\
islH:age2:sexM & -1.910 & 0.841 & -2.271 & 0.023{*}\\
islS:age2:sexM & -0.724 & 0.732 & -0.988 & 0.323\\
islH:age3:sexM & -2.583 & 0.842 & -3.066 & 0.002{**}\\
islS:age3:sexM & -1.185 & 1.199 & -0.988 & 0.323\\
islS:rurbU:sexM & 1.845 & 1.585 & 1.164 & 0.244\\
age2:rurbU:sexM & -1.411 & 0.701 & -2.012 & 0.044{*}\\
age3:rurbU:sexM & -1.467 & 0.743 & -1.975 & 0.048{*}\\
\lspbottomrule
\end{tabular}
\caption{Mixed-effects model of complete devoicing results. Initial intercept represents younger female \isi{Lewis} speakers.}
\label{tab:clayton:9}
\end{table}

\section{Discussion}\label{discussion}

\subsection{Summary of results}

This analysis of a broad cross section of the recordings in the Shuken archive allows for a more detailed view of HE than was previously available. With respect to \isi{preaspiration} and \isi{glottalization}, this study both expands on \citeauthor{Shuken:1984} (\citeyear{Shuken:1984, Shuken:1985, Shuken:1986}) and provides a baseline for comparison with studies using data collected several decades later (\citeauthor{Clayton:2015} \citeyear{Clayton:2015, Clayton:2017, Clayton:2018}). For example, in Clayton’s later study, \isi{preaspiration} could be found at least sporadically in most HE speakers, including those from islands not included in Shuken’s sample. But of those speakers, only a small number of older rural women from \isi{Lewis} provided more than a handful of preaspirated\is{preaspiration} tokens. This could well be a development of the situation that \citet{Shuken:1985} discerned three decades previously, in which \isi{preaspiration} predominated among participants from \isi{Lewis}, particularly rural speakers, and particularly women. Though in that paper, Shuken did not consider the older two age groups from her full sample, the present analysis adds those two older age brackets, and finds that the same pattern holds: \isi{preaspiration} is more abundant in \isi{Lewis} than \isi{Harris} or \isi{Skye}, more abundant in rural areas compared to urban areas, and more abundant in women than men. The current study also finds that older speakers preaspirate more often than younger speakers. If we fast-forward three decades, so to speak, it is easy to see how the situation visible in Shuken’s earlier data could readily develop into the state of affairs Clayton found in speakers three decades later.

Taken as a whole, the pattern we see with \isi{preaspiration} is consistent with a conservative, local prestige feature, that is, one that speakers may exploit to express affiliation with the local island community (cf. \citealt{Labov:1963}). Given the salience of \isi{preaspiration} as a Gaelic\il{Scottish Gaelic (Modern)} feature, we may even suggest that the feature indexes a local Gaelic\il{Scottish Gaelic (Modern)}-speaking identity. However, with its evidently decreasing frequency, even among \isi{Lewis} speakers, it is possible that \isi{preaspiration} is becoming progressively less valuable to speakers in this regard.

On the other hand, the notably higher frequency of \isi{glottalization} among \isi{Skye} speakers, especially urban/monolingual speakers, instead bears the hallmarks of an incoming feature. Glottalization\is{glottalization} may well have been introduced, or reinforced, through contact with the mainland via newly arrived incomers from mainland urban communities like Glasgow or Aberdeen. Glottalization\is{glottalization}, particularly T-glottaling, is an expanding if stigmatized feature of working-class speech, in Scotland just as in England (\citealt{Wells:1982, Trudgill:1988, Stuart-Smith:1999a, Stuart-Smith:1999b}).
While we may expect a change favoring a non-standard form to be led by men (cf. \citealt{Trudgill:1972, Labov:1963}), especially one that confers working-class covert prestige, the lack of a significant difference by sex in the HE data is consistent with a similar lack of differentiation noted for T-glottaling in Glasgow English \citep{Stuart-Smith:1999a}, and indeed seems to presage a similar, later finding for HE in which \isi{glottalization} shows no significant gender distinction \citep{Clayton:2018}.

Devoicing\is{devoicing} in certain respects patterns like \isi{preaspiration} in the present study: it is more common among \isi{Lewis} speakers than those from the other two islands (though not exclusive to \isi{Lewis}), more common among female speakers than male speakers, and more abundant among rural speakers than urban speakers. On the other hand, \isi{devoicing} is distinct from \isi{preaspiration} in that it is noticeably more common among younger speakers than those in the middle or older age brackets (though statistically this age differentiation was only significant among men). The greater prevalence among women and among younger speakers is what we might associate with an incoming prestige feature (\citealt{Labov:1990, Eckert:1989}). But \isi{devoicing} is not an incoming feature: it has been ascribed to the English of the Highland and Island region in one form or another for at least a century~-- sometimes represented as an anomaly that ought to be corrected (\citealt{Grant:1913}\textcolor{brown}{: 42}), and sometimes caricaturized as in Compton Mackenzie's (\citeyear{Mackenzie:1947}) novel \textit{Whisky Galore.} Indeed, one might expect there to be a stigma associated with the feature, as there is with T-glottaling; in fact, during their interviews, at least two female speakers among Shuken’s participants singled out \isi{devoicing} as an incorrect or problematic aspect of the local English. Thus, \isi{devoicing} is not a prestige feature in the conventional sense. One wonders, though, whether \isi{devoicing} is being exploited by younger speakers in lieu of \isi{preaspiration}: as a local prestige feature indexing islander identity. 

\subsection{Further directions}

There are two broad areas of further analysis and fieldwork suggested by the analyses presented here. First, the findings discussed in this chapter prompt many questions about the current status of HE. The body of material collected for the most recent published study of HE (\citeauthor{Clayton:2015} \citeyear{Clayton:2015, Clayton:2017, Clayton:2018}) illustrated only \isi{preaspiration} and \isi{glottalization}, and so can tell us little about the third variable considered in this chapter, sibilant \isi{devoicing}. There is also the issue of geographic scope. While Clayton did consider material from a relatively broad array of islands within the Hebrides island group, apart from \isi{Skye} and \isi{Lewis} most of those islands were represented by only one or two speakers; no speakers were from the more southerly islands in the archipelago. Mull, Jura, Islay, Barra, and many other islands thus remain completely uninvestigated. 

There is also a substantial quantity of analysis remaining to be carried out within the Shuken collection, some of which is currently in progress. First, the large and varied stimulus set allows the consideration of additional phonological features beyond those already considered, for example retroflex /rs/ and vowel epenthesis (indeed, analyses of these two features are currently in progress). Second, the reading passage material should be added. As a stylistically distinct form of language use (cf. \citealt{Trudgill:1972}), this material could well bring additional focus to those variables considered in the present chapter and any future analyses. Finally, only a limited set of social factors were considered in this chapter, including island of origin, rural vs urban settings, sex, age, and language ability. While these factors have allowed a number of insights into the distribution of the linguistic features considered, in particular their geographic distribution and relative abundance within the relevant demographic groups, as yet we know little about how \isi{preaspiration}, \isi{glottalization}, and \isi{devoicing} may correlate with social factors such as education, economic class, or profession; or how speakers may exploit them to express group affiliation or construct identity. 

%%%% APPENDIX OMITTED PER CLAYTON EMAIL %%%%

%\section*{Appendix A \slp{need to fix these tables}}\label{appendix}
%
%\subsection*{Stimulus from the Shuken study}
%
%\begin{tabularx}{\textwidth}{XXXXXXXX}
%\lsptoprule
%{ beat} & { rowed} & { torn} & { peril} & { half} & { ooze} & { ship} & { coo}\\
%{ bit} & { allowed} & { pork} & { thirty} & { halves} & { illuminate} & { dip} & { William}\\
%{ bait} & { brewed} & { borne} & { far-sighted} & { pal} & { presume} & { cup} & { plugging}\\
%{ bet} & { rode} & { corn} & { Thursday} & { far} & { enthuse} & { chip} & { idiot}\\
%{ ant} & { sighs} & { bored} & { thirsty} & { psalm} & { beauty} & { which} & { seal}\\
%{ aunt} & { tied} & { accept} & { here's} & { bra} & { wounds} & { sinning} & { failure}\\
%{ cot} & { sighed} & { taxis} & { fur} & { dawn} & { new} & { since} & { cue}\\
%{ caught} & { rows (fights)} & { herd} & { furl} & { call} & { tomb} & { singing} & { argue}\\
%{ coat} & { thrown} & { hoarse} & { word} & { cross} & { jewel} & { prince} & { rose}\\
%{ cut} & { ewes} & { board} & { work} & { pour} & { noon} & { fingers} & { quite}\\
%{ put} & { stewed} & { taxes} & { burst} & { caw} & { feud} & { sprints} & {  milk}\\
%{ boot} & { seize} & { except} & { cursed} & { don} & { booty} & { safer} & { film}\\
%{ bite} & { raise} & { personality} & { carves} & { more} & { price} & { stable} & { kiln}\\
%{ about} & { weighs} & { cannon} & { league} & { loathe} & { awhile} & { crate} & { shrink}\\
%{ quoit} & { daze} & { milling} & { leaf} & { bowl} & { wine} & { staple} & { glimpse}\\
%{ leak} & { nod} & { cart} & { leave} & { both} & { byre} & { petal} & { felt}\\
%{ weak} & { throne} & { parcel} & { zeal} & { bow} & { try} & { saver} & { health}\\
%{ creak} & { use} & { canyon} & { bean} & { fuss} & { house} & { stretches} & { herb}\\
%{ agreed} & { stood} & { person} & { beard} & { cub} & { fowl} & { dredges} & { scalp}\\
%{ great} & { brood} & { million} & { tree} & { cull} & { down} & { scents} & { dwarf}\\
%{ made} & { tide} & { card} & { bib} & { sun} & { flour} & { pedal} & { slurp}\\
%{ joke} & { side} & { partial} & { gift} & { fuzz} & { row (fight)} & { talk} & { squelch}\\
%{ spoke v.} & { size} & { barley} & { give} & { foot} & { adroit} & { right} & { stench}\\
%{ week} & { loud} & { turned} & { bill} & { use} & { void} & { nature} & { twelfth}\\
%{ beet} & { rouse} & { carnage} & { wind n.} & { food} & { voice} & { tag} & { confess}\\
%{ leek} & { badge} & { cartoon} & { fir} & { tune} & { noise} & { tack} & { fez}\\
%{ creek} & { cause} & { curtail} & { girl} & { poor} & { boil} & { mew} & \\
%\lspbottomrule
%\label{appendix:1}
%\end{tabularx}
%
%\subsection*{Stimulus list, continued}
%
%\begin{tabularx}{\textwidth}{XXXXXXXX}
%\lsptoprule
%{ greed} & { gather} & { laird} & { race} & { do} & { coin} & { actually} & { clasp}\\
%{ mate} & { father} & { burden} & { fail} & { pure} & { lawyer} & { rouge} & { smash}\\
%{ grate} & { cadge} & { journey} & { train} & { dual} & { boy} & { lock} & { splashed}\\
%{ maid} & { caws} & { cared} & { wear} & { good} & { Mary} & { mixture} & { sprawled}\\
%{ boat} & { horse} & { theirs} & { grey} & { water} & { marry} & { loch} & { shone}\\
%{ choke} & { rather} & { world} & { bed} & { soot} & { warm} & { hawk} & { stoop}\\
%{ spoke n.} & { bother} & { worst} & { pressure} & { pool} & { arm} & { author} & { stripes}\\
%{ road} & { heard} & { earth} & { measure} & { suit} & { buyer} & { plucking} & { boys}\\
%{ seas} & { fork} & { first} & { bell} & { dew} & { flower} & { feature} & { rows (boat)}\\
%{ ways} & { pearl} & { there's} & { pen} & { pull} & { teeth} & { gesture} & { rude}\\
%{ days} & { farm} & { bird} & { were} & { hue} & { tip} & { wondered} & { rule}\\
%{ rays} & { born} & { birds} & { Sam} & { duel} & { teethe} & { tedious} & { loop}\\
%{ gnawed} & { bury} & { hears} & { batch} & { assume (resume)} & { three} & { undone} & { rued}\\
%\lspbottomrule
%\end{tabularx}

\section*{Acknowledgements}
I would like to thank the very helpful staff at the University of Edinburgh School of Scottish Studies archives, in particular Elliot Holmes and Kirsty Stewart, who made this work possible. I would also like to thank the two anonymous reviewers for their very helpful comments and suggestions on a previous draft of this chapter. 

\sloppy
\printbibliography[heading=subbibliography,notkeyword=this]

\il{Hebrides English|)}
\is{contact varieties|)}

\end{document}
