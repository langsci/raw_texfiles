\documentclass[output=paper,colorlinks,citecolor=brown]{langscibook}
\ChapterDOI{10.5281/zenodo.15654873}
\title{Comparing the syntactic complexity of Gaeltacht and urban Irish-Language broadcasters}
\shorttitlerunninghead{Comparing syntactic complexity}
\author{Brian Ó Broin\affiliation{William Paterson University, NJ}}

\IfFileExists{../localcommands.tex}{
   % add all extra packages you need to load to this file

\usepackage{tabularx,multicol}
\usepackage{url}
\urlstyle{same}

\usepackage{listings}
\lstset{basicstyle=\ttfamily,tabsize=2,breaklines=true}

\usepackage{langsci-basic}
\usepackage{langsci-optional}
\usepackage{langsci-lgr}
\usepackage{langsci-osl}
% \usepackage{./langsci/styles/langsci-lgr}
% \usepackage{./langsci/styles/langsci-osl}
% \usepackage{langsci-gb4e}

\usepackage{tikz}
\usetikzlibrary{patterns,calc}
\pgfdeclarepatternformonly{south east lines}{\pgfqpoint{-0pt}{-0pt}}{\pgfqpoint{3pt}{3pt}}{\pgfqpoint{3pt}{3pt}}{
    \pgfsetlinewidth{0.6pt}
    \pgfpathmoveto{\pgfqpoint{0pt}{3pt}}
    \pgfpathlineto{\pgfqpoint{3pt}{0pt}}
    \pgfpathmoveto{\pgfqpoint{.2pt}{-.2pt}}
    \pgfpathlineto{\pgfqpoint{-.2pt}{.2pt}}
    \pgfpathmoveto{\pgfqpoint{3.2pt}{2.8pt}}
    \pgfpathlineto{\pgfqpoint{2.8pt}{3.2pt}}
    \pgfusepath{stroke}}
    
\usepackage{stmaryrd}
\usepackage{wasysym}
\usepackage{multirow}
\usepackage{caption}
\usepackage{subcaption}
\usepackage{mathrsfs}
\usepackage{qtree}

\usepackage{linguex}


   %pminos do not split footnotes
% \interfootnotelinepenalty=10000 %Footnote in Laporte chapters has to be split SN


%\DeclareIndexNameFormat{default}{%
%\nameparts{#1}%
%\usebibmacro{index:name}%
%{\index[names]}%
%{\namepartfamily}%
%{\namepartgiveni}%
% {}% L1
% {}% L2
%{\namepartprefix}% generates spurious space L3
%{\namepartsuffix}% generates spurious space L4
%}

%  {\DeclareIndexNameFormat{default}{%
%     \usebibmacro{index:name}{\index[names]}{#1}{#3}{#5}{#7}}}

%\DeclareIndexNameFormat{default}{%
%  \usebibmacro{index:name}{\sindex[nom]}{#1}{#3}{#5}{#7}}

%\DeclareIndexNameFormat{default}{%
%  \usebibmacro{index:name}{\sindex[person]}{#1}{#3}{#5}{#7}}
%\DeclareIndexNameFormat{default}{%
%\nameparts{#1} \usebibmacro{index:name}{\sindex[person]]}{\namepartfamily}{‌​\namepartgiven}{\nam‌​epartprefix}{\namepa‌​rtsuffix}}

%\newcommand{\smiley}{:)}

%\renewbibmacro*{index:name}[5]{%
%\usebibmacro{index:entry}{#1}%
%{\iffieldundef{usera}{}{\thefield{usera}\actualoperator}\mkbibindexname{#2}{#3}{#4}{#5}}}

% \newcommand{\noop}[1]{}

%remove for final
%\overfullrule=1mm

\newcommand{\tobi}[2]}}
\renewcommand{\S}[1]{\tobi{#1}{\textsc{*}}}

% this volume references
% puts: [this volume]
% already defined: \citetv
%\newcommand{\citepv}[1]{(\citeauthor{#1} \citeyear*{#1} [this volume])}
\newcommand{\citealtv}[1]{\citeauthor{#1} \citeyear*{#1} [this volume]}

%parentheses around example number
\newcommand{\pref}[1]{(\ref{#1})}

% in-text examples

\newcommand{\lnex}[1]{\textit{#1}} %target lang word
\newcommand{\lnlit}[1]{(lit.: `#1')} %literal reading
\newcommand{\lnlat}[1]{(#1)} % latinization
\newcommand{\lntrans}[1]{`#1'} %translation
\newcommand{\lnexl}[2]%
{\lnex{#1}{} \lnlat{#2}} % ex with latinization
\newcommand{\lnexlat}[3]{\lnex{#1}{} \lnlat{#2}{} \lntrans{#3}} % ex with latinization and tranl.

%ch01
\newcommand{\co}[1]{\mbox{\textbf{#1}}}

%ch09

\newcommand{\cyrbulg}[1]{\begin{otherlanguage*}{bulgarian}#1\end{otherlanguage*}}


%ch10
\newcommand{\nlp}{{\small NLP}}
\newcommand{\mwe}{{\small MWE}}
\newcommand{\rae}{{\small RAE}}
\newcommand{\lvc}{{\small LVC}}
\newcommand{\pos}{{\small P}o{\small S}}
%\newcommand{\todo}[1]{ \textcolor{red}{#1} }

%\renewcommand{\labelenumi}{\theenumi}
%\ainamefmt{{vv}{ll}{, ff}{, jj}} % fullname

\newcommand{\biberror}[1]{{\color{red}#1}}

\newcommand{\osenovaitem}{--~}
   %% hyphenation points for line breaks
%% Normally, automatic hyphenation in LaTeX is very good
%% If a word is mis-hyphenated, add it to this file
%%
%% add information to TeX file before \begin{document} with:
%% %% hyphenation points for line breaks
%% Normally, automatic hyphenation in LaTeX is very good
%% If a word is mis-hyphenated, add it to this file
%%
%% add information to TeX file before \begin{document} with:
%% %% hyphenation points for line breaks
%% Normally, automatic hyphenation in LaTeX is very good
%% If a word is mis-hyphenated, add it to this file
%%
%% add information to TeX file before \begin{document} with:
%% \include{localhyphenation}
\hyphenation{
    Beck-man
    Ngu-yen
    back-chan-nel
    back-chan-nels
    mo-not-o-nous
    ste-reo-typ-i-cal
}

\hyphenation{
    Beck-man
    Ngu-yen
    back-chan-nel
    back-chan-nels
    mo-not-o-nous
    ste-reo-typ-i-cal
}

\hyphenation{
    Beck-man
    Ngu-yen
    back-chan-nel
    back-chan-nels
    mo-not-o-nous
    ste-reo-typ-i-cal
}

   \boolfalse{bookcompile}
   \togglepaper[9]%%chapternumber
}{}

\AffiliationsWithoutIndexing

\abstract{Irish, while nominally a national language, is primarily spoken as a native language in rural enclaves called Gaeltachtaí. The language has recently experienced an urban revival, however, with Irish now being spoken as a domestic language in towns and cities all over Ireland. Following on my previous work in phonetics and morphology, this paper analyzes the two varieties of Irish (rural and urban) with particular emphasis on syntax and lexicon, with a view to discovering whether they are distinguishable using those features. I undertake a syntactic and lexical analysis of Gaeltacht and Urban speakers of Modern Irish. In the case of syntax, I am applying the well-known D-Level complexity scale of Rosenberg and Abbeduto to a number of transcriptions from speakers on urban and Gaeltacht radio stations (and adapting the scale for Irish syntax), while in the case of lexicon I perform a simple lexical diversity analysis on the same transcriptions using TTR (Type-Token Ratio) calculation. A complexity scale for Irish has been a major desideratum for several decades, and this article intends to provide such a scale. Similarly, no TTR analysis of Irish adults has yet been done, and this analysis will be a first.}

\begin{document}

\maketitle

\il{Irish (Modern)|(}
\section{Introduction}

\is{Gaeltacht|(}
\is{syntactic complexity|(}
\citet{ob:RosenbergAbbeduto1987} developed a complexity scale for English sentences to analyze the language of adults with intellectual disabilities. Scholars quickly realized, however, that this complexity scale could be used for the linguistic analysis of any speakers, and the scale has been retained, with adaptations and revisions, as a useful yardstick for gauging  developmental level. This paper sets out to apply an adapted version of the Rosenberg and Abbeduto scale to urban and Gaeltacht broadcasters of Irish with the goal of determining if there are measurable differences between the syntactic complexity of broadcasters in the two communities.
\is{syntactic complexity|)}

There are two principal Irish language communities. The traditionally recognized group is that of the ``Gaeltacht," a government-recognized area where the number of Irish speakers exceeds a certain threshold.\footnote{Definitions and thresholds have changed and fluctuated since the foundation of the Irish state, but the constant is that these areas receive more government funding than other areas for Irish-language education and for projects that are seen to preserve the language community. See
\citet{ob:OGadhra} for a good introduction to the subject.} A second, larger, non-Gaeltacht community consists of mostly urban language activists and learners who frequently equate their language with identity in much the same way the Basque and Catalan communities do in France and Spain. This community, existing through multiple generations since even before the foundation of the Irish state  (\cite{ob:Edwards2016}),\footnote{Because most members of this community raise their children in English, however, the community's members come and go from generation to generation without transferring their Irish to the next generation.} has been much supported in recent decades by the growth of Irish-medium education under the Gaelscoileanna\is{Gaelscoil} (``Gaelic Schools") movement, which has now coalesced with secondary-level ``Gaelcholáistí" (``Gaelic Colleges") under the ``Gaeloideachas"\is{Gaeloideachas} (``Gaelic Education") banner. Somewhere in the region of 43,000 pupils attend such Irish-medium schools outside the Gaeltacht.\footnote{Another 14,000 attend school in the Gaeltacht itself, where Irish is the default language of schools \citep{ob:Gaeloideachas2023}.}

Neither community is monolithic. Some officially recognised Gaeltacht areas like Baile Cláir in Co. Galway now have certainly fewer than 5\% native speakers (\citealt{ob:OCinnseala2020}), while vigorous, but unrecognised, ``nua-Ghaeltacht" (``neo-Gaeltacht") areas like Carn Tóchair in Co. Derry may well have more.\footnote{See \citet{ob:Armstrong2012}. Localized language statistics for United Kingdom districts are unavailable.} The Gaeltacht community represents an unbroken language tradition that dates back scores of generations to the arrival of Celtic culture in Ireland. Its speakers, speaking what Feargal Ó Béarra termed ``Traditional Late Modern Irish" or ``Non-Trad\-ition\-al Late Modern Irish" \citep{ob:OBearra2007}, mostly take their language for granted and have little or no interest in language preservation or promotion. Gaeltacht activism exists, of course, and is the predominant reason for the existence of the national radio and television broadcasters \isi{RTÉ Raidió na Gaeltachta} and \isi{TG4} \citep[742--745]{ob:Watson2002}, but most native speakers in the Gaeltacht speak Irish simply because, as a mother tongue and community language for them, it is easier for them than English. The community has been much supported by the existence of the afore-mentioned Gaeltacht-focused radio station, which broadcasts almost entirely from its three studios in the Gaeltacht areas of Kerry, Galway, and Donegal. Although the station is available nationally and internationally on RTÉ's broadcast network,\footnote{RTÉ is Ireland's public-service radio and television network. It manages most of Ireland's radio and television transmitters. Raidió na Gaeltachta is considered one of RTÉ's four major national FM radio channels, and is also streamed live on RTÉ's website.} the six officially-recognized Gaeltacht areas around the country are considered its primary audience.\footnote{The Irish language television station, \isi{TG4}, serves a national audience (meaning mostly L2 speakers), and is mostly not associated with the Gaeltacht, except for being headquartered there.}

\is{Belfast|(}
\is{Dublin|(}
The non-Gaeltacht language community is more amorphous, and though this author refers to the community as urban, there are of course non-Gael\-tacht speakers who live in rural communities. That said, this study mostly confines itself to the non-Gaeltacht urban communities of Dublin and Belfast, cities where enough language activists and daily speakers exist to justify the existence of print and broadcast media to service them.\footnote{Belfast was home to a daily Irish language newspaper from the 1980s until 2008, when declining sales for print newspapers nationwide made the venture no longer viable. A popular monthly magazine, \textit{Nós}, still publishes from Belfast's Irish-language center \textit{An Cultúrlann}. An Irish-language insert named \textit{Seachtain} (`Week) still appears weekly in the Dublin-based \textit{Irish Independent}.} These speakers vary considerably in ability and interest, from native speakers raised through Irish in the city and educated in Irish-medium schools to adult learners who may never have learned a word until enrolling in an evening class. Both urban communities are serviced by local radio stations broadcasting around the clock: \isi{Raidió Fáilte} in Belfast and \isi{Raidió na Life} in Dublin. Dublin, Belfast, and Cork speakers are often easy to identify by their dialects.\footnote{There are nascent Irish language communities in Cork City, Galway City, and Derry City, but none has reached the critical mass of speakers that might justify a radio station.} Some might argue that this is merely the effect of strong English accents bleeding over into the second language, but these speakers are nevertheless recognizable. 
\is{Belfast|)}
\is{Dublin|)}

The urban community of Irish speakers is growing, with language-centered events taking place very regularly in many towns and cities,\footnote{The website \url{https://feilire.com/} lists near-daily language-centered events in and around Galway City, for example.} and there has been sociolinguistic analysis of those who participate in such events. \citet{ob:WalshORourke2015} and \cite{ob:ORourkeWalsh2020} refer to such participants as \textit{Nua\-chainteoirí} (`\isi{new speakers}), using the popular Catalonian \textit{\isi{linguistic mudes}} approach to language and identity. Their central argument is that a speaker's chosen language identity must be respected, regardless of background, and that the long-vaunted “authority of the ideology of the native speaker” is now to be questioned. 

Placing these ongoing sociolinguistic debates to one side, the very existence of locally directed radio stations implies the existence of two communities: that of the Gaeltacht, obviously represented in the broadcasters and audience of Raidió na Gaeltachta, and that of the urban areas, also obviously represented in the broadcasters and audiences of \isi{Raidió Fáilte} and \isi{Raidió na Life}.\footnote{We must, of course acknowledge that Gaeltacht speakers can be heard on the urban stations, as can urban speakers on the Gaeltacht station, but by and large the distinction holds. \isi{Raidió na Life} has been broadcasting to \isi{Dublin} since 1993, while \isi{Raidió Fáilte} has been broadcasting to \isi{Belfast} since at least 2002.} Up until now, however, no study has analyzed, compared, or contrasted the Irishes of these two communities. How could one choose informants and control for various factors? What features do the two varieties share? Where do they differ? And what measuring tools could be used to perform such a comparison?
\is{Gaeltacht|)}

\section{Methodology}

Gathering data was the first major challenge for this project. In an ideal world, one would find a sufficient number of native speakers from each community, ensuring their approximate comparability in things such as age, gender, socio-economic background, education, and the like. One would then interview them, under the same circumstances, on a subject that would ensure they would all speak unself-consciously and at length. Alas, in this case such an approach would be impossible without extraordinary resources.

The two communities' broadcasting fora offered an alternative solution. There were three radio stations, all streaming live over the internet, and all using articulate presenters presumably chosen for their comparatively good command of Irish. Studios provided relatively controlled environments, and the variety of programming from each station meant that similar types of programs could be found that allowed for meaningful comparison between speakers. So, \isi{Gaeltacht} news readers could be compared with urban news readers, and so on with chat show hosts, music presenters, etc.

One caveat in the choice of broadcasters is that those on Raidió na Gaeltachta are mostly professionals, while those on the urban stations are mostly volunteers. It is not clear, however, how or why that might affect the \isi{syntax} of the Irish that they speak, particularly in unscripted situations like chat show monologues. Newsreaders offer good baseline samples of carefully spoken, prepared speech. But news reports are rarely delivered extemporaneously, and it seemed politic to analyze impromptu speech also, also under reasonably controlled circumstances. In this case, I chose presenters of chat shows delivering off-the-cuff monologues at the beginning of their shows. In each case, I chose presenters from the \isi{Gaeltacht} (on Raidió na Gaeltachta), from \isi{Belfast} (on \isi{Raidió Fáilte}), and from \isi{Dublin} (on \isi{Raidió na Life}). Additionally, I selected two published texts as upper and lower baselines. These will be discussed further below.

Four speakers from each radio station were originally chosen for the \isi{phonetics} and \isi{morphology} study, each group divided between newsreaders and chat show hosts. The total eventually expanded to twenty speakers - ten from the \isi{Gaeltacht} station, and ten from the two urban stations.

Regarding the speakers, I did as much due diligence as I could to ensure that they were raised in the communities to which they were broadcasting (so, for example, a \isi{Dublin} speaker had to have been raised within the broadcasting footprint of \isi{Raidió na Life}; ditto for \isi{Belfast} speakers; and ditto for \isi{Gaeltacht} speakers).\footnote{This distinction is not without difficulties. Some broadcasters have been raised in multiple places.} As noted above, however, it was not possible to control for professional status.

In the case of each broadcaster, I took a 3--5 minute recording and transcribed it. Such recordings usually yielded between 400 and 700 words, or 25--35 sentences, which was more than enough to get a good representation of the speaker's language. My own early work on \isi{lexicon} suggested to me that c.500 words was enough to approximately calculate a speaker's lexical ability, and this figure has been partially confirmed by the analyses of \citet{ob:McCarthy2007}. Other studies suggested that this length of text was enough to also accurately provide sufficient data for syntactic analysis.\footnote{\citet{ob:Snowdon1996} got reliable results from texts that were mostly 200--300 words long.} For syntactic analysis, the total yield of sentences was approximately 660.\footnote{Twenty speakers with about thirty sentences per transcription, plus the two baseline texts.} I then analyzed the recording and the transcription using measuring tools for the categories already described above.

It is worth re-emphasizing that this is not a comparison of native speakers and learners. Using the \textit{\isi{linguistic mudes}} approach discussed above, I chose my speakers, not on what language they spoke at home, but on what language they were choosing to speak in front of the microphone.

\is{phonetics|(}
\is{morphology|(}
Most linguists analyze language using four distinct measures, because they can be systematically observed: Phonetics/phonology, morphology, \isi{syntax}, and \isi{lexicon}. I have already published my findings in the areas of and morphology  (\citealt{ob:OBroin2014b}), and I do not propose to do any more here than summarize my findings in these areas.  A complete phonetic analysis being out of the question, I elected to examine three representative phonetic markers: slender dentals, unvoiced velar fricatives, and unvoiced palatal fricatives.\footnote{e.g., the initial consonants of \textit{Dia} (``God") and \textit{teach} (``house"); the initial consonant of \textit{chú} (``dog" [GEN]); and the initial consonant of \textit{cheo} (``fog" [GEN]).} All of these are sounds that are not made in English but are nevertheless necessary for Irish grammar (a slender consonant at the end of a word can indicate plurality, for instance, while a lenited consonant at the beginning of a word can indicate gender).\footnote{An tendency not to mark plurals, gender or possession suggests that speakers will have great difficulty being understood unless alternatives are already in place.} The study revealed a noticeable phonetic variance between \isi{Gaeltacht} and urban broadcasters, reflected most obviously in expected palatals, of which Gaeltacht broadcasters missed an average of 2\%, while their urban counterparts missed an average of 64\%.  

This would mean little in languages such as English, where phonetics and morphology do not intersect quite so sharply, but the implications for Irish are clear. If speakers are not applying lenition or eclipsis, case forms such as the genitive plural and tense forms such as the past are liable to get lost. As with phonetics, a complete morphological analysis was impossible, and I elected to examine a single representative category - nouns declined in any form. So, for the lemma \textit{cainteoir} (``speaker"), a masculine noun of the third declension, I marked all occurrences of its declined forms: \textit{gcainteoir, cainteora, chainteora, cainteoirí, gcainteoirí,} and \textit{chainteoirí}. I counted as ``missed" any instances where the expected change did not occur.

\is{Gaeltacht|(}
As expected, given the above comments on Irish morphophonology, the phonetic observations in urban broadcast Irish noted above are similarly indexed in morphology. If speakers are not making phonetic changes that mark a genitive or a past tense, these are likely to drop out of the grammatical system. In this case, Gaeltacht broadcasters missed an average of 6\% of expected changes, while urban broadcasters missed an average of 45\%, rising to 63\% in relaxed speech. If city broadcasters are not making 63\% of expected morphological changes in relaxed speech, their language is very different from that of Gaeltacht broadcasters. While city broadcasts may sound odd to Gaeltacht speakers, however, Gaeltacht broadcasts must be even more radically different to city speakers, given that Gaeltacht broadcasters are using sounds not present in the urban phonetic system, as well as a different morphological system.
\is{phonetics|)}
\is{morphology|)}
\is{Gaeltacht|)}

\is{syntactic complexity|(}
Next, we turn to the focus of this chapter: Does \isi{syntax} also vary between these communities? Can qualitative comparison be done in this field? Some work has been done on the general \isi{syntax} of particular \isi{Gaeltacht} dialects in Irish (\cite{ob:OSiadhail1989, ob:OMuiri1982}) but no large-scale comparative studies have been done. While some other categories are worth analyzing here, particularly the presence or absence of Irish-specific features such as the copula or the autonomous form, my principal goal here was to analyze syntactic complexity, which \cite{ob:Snowdon1996} have used as an index of language development.\footnote{Snowdon et al. suggest that ``grammatical complexity is associated with working memory, performance on speeded tasks, and writing skill.” (\cite{ob:Snowdon1996}: 516)} Grammatical complexity\is{syntactic complexity}, using an approach developed by Sheldon Rosenberg and Leonard Abbeduto in 1987 and refined several times in the following decades, is measurable. Using a scale from 0 to 7, where zero designates a simple monoclausal sentence and seven designates a complex sentence containing embedding of phrases at multiple levels, a text can be analyzed sentence-by-sentence and an average complexity score calculated for the whole text.

Herein lies our first major problem. The Rosenberg and Abbeduto approach is specific to English. How might it be adapted to Irish? With regard to the indexing of Irish and English constructions in the seven-level structure, we can certainly keep very simple monoclausal sentences\footnote{For example, a single proposition containing a single transitive verb.} at level zero and complex ones\footnote{For example, a sentence containing a tensed or untensed verbal phrase that is nested inside another phrase.} at level seven, but what about the six levels between them? The revised scale for English awards one point to infinitives or participles, but Irish does not have these in the same manner that English does. Should we similarly score the Irish verbal noun, which is the closest structure in Irish to English non-finite verbal forms? Raising in English is awarded three points, but raising in Irish is very different (\cite{ob:McCloskey1989})~-- should it receive the same three points in Irish that it receives on the English scale? Verb nominalization receives three or six points in the English scale, depending on whether the nominalization occurs in object or subject position. Is there anything in Irish, whose basic sentence structure is very different (VSO, as opposed to English's SVO), that parallels this? Fortunately for this study, such questions, while important, had little effect on calculations, which ultimately depended very heavily on level-7 sentences.

As we now know, a certain amount of Rosenberg and Abbeduto's classification system initially involved guesswork, hence the several challenges and revisions since it came out, and some of this article's approach, for want of extensive data from L1 acquirers of Irish, is similarly open to revision. Some of it comes from the transcriptions that I have made myself of native-speaking children and adolescents (\cite{ob:OBroin2014}), but until we have a comprehensive searchable language database for Irish-speaking children, this adaptation will remain, as Rosenberg and Abbeduto's scale did for several decades, a work in progress.

Rosenberg and Abbeduto had been working on measurement tools for ascertaining complexity levels in cognitively disabled adults since 1974, but they only really codified a level-based system in 1987, when they established a seven-level scale (from earliest-to-appear to last-to-appear in normally developing children) for utterance complexity where, for example, a sentence containing an ``embedded infinitival complement with subject identical to that of the matrix clause" (e.g., \textit{I am going to meet John}) would receive a single point, while a sentence containing ``more than one use of sentence combining in a given sentence" (e.g., \textit{John decided to leave Mary when he heard that she was seeing Mark}) received seven points (\citealt{ob:RosenbergAbbeduto1987}: 26).

\largerpage
This scale was devised using data derived from ``literature on the development of complex sentences in normal children" \citep[26]{ob:RosenbergAbbeduto1987}. In 1994, Hintat Cheung and Susan Kemper slightly revised this “D-Level”\is{D-level complexity|see {syntactic complexity}} system by adding a zero-level for simple monoclausal sentences, and selected D-Level as one of three reliable measurement tools for the analysis of adult language, particularly in older speakers (\citealt{ob:CheungKemper1992}: 71). The D-Level scale received a significant boost when it was selected as the grammatical complexity measurement tool in the ``\isi{Nun Study}," (\citealt{ob:Snowdon1996}: 529) a well-known project involving the analysis of language examples provided by nuns in early life as predictors of mental health or senility in later life. A revision of the scale led by Michael Covington added some new sentence structures and (significantly) swapped levels 5 and 6  (\citealt{ob:CovingtonEtAL2006}: 10--11), and this revision appears to be the accepted version of the scale since then. It is Covington's revision that I adapt for this analysis (\ref{CovingtonScale}):

\ea\label{CovingtonScale}
\citet{ob:CovingtonEtAL2006}'s Revision of the Rosenberg and Abbeduto Scale 

Level 0  simple sentence\\
Level 1  non-finite clause as object without overt subject\\
Level 2  coordinated structure\\
Level 3  finite clause as object (and equivalents)\\
Level 4  non-finite clause as object with overt subject (and equivalents)\\
Level 5  finite or non-finite adjunct clause\\
Level 6  complex subject\\
Level 7  more than one structure of Level 1--6
\z

\noindent{As shown in examples in (\ref{ex.level0}) and (\ref{ex.level7}), defining levels 0 and 7 for Irish is mostly non-controversial. However, scoring sentences between levels 1 and 6 was difficult because Irish has several common grammatical features that are not present in English, as we see in the examples in (\ref{ex.level1}--\ref{ex.level6}). Be it understood, then, that this proposed adaptation of Rosenberg and Abbeduto's scale appears to be mostly reliable but may remain challengeable for specific linguistic features that are found only in Irish. Even so, it is likely that such features would only be promoted or demoted a single level. Be it also understood that this paper is not the appropriate location for an exhaustive description or defense of the adaptation. I intend to provide such a description separately at a later date but do provide some comments here in footnotes.}

\ea\label{ex.level0} {\textit{Level 0: Simple sentences, including questions}.}
\ea
\gll Bhí ionadaithe ón domhan spóirt i láthair ag an Aifreann.\\
be.\textsc{past} representatives from.\textsc{def} world sport.\textsc{gen} in presence at the mass\\
\glt ‘Representatives from the world of sport were present at the mass.’
\ex\
\gll Ba leo an lá faoi dheireadh.\\
\textsc{cop.\textsc{past}} to.\textsc{3pl} the day under end\\
\glt ‘The day was theirs in the end.’ (i.e., They won.)\footnote{There might some argument to be made for giving the copula more complexity, but I have elected to keep simple copular sentences in this category.}
\z
\z

\ea\label{ex.level1} {\textit{Level 1:  Simple non-finite verb forms in simple sentences}}\\
\gll Tá sé chun an bus a thógáil.\\
be.\textsc{prs} he to the bus to take.\textsc{vn}{\footnotemark}\\
\footnotetext{There is disagreement among linguists about the labeling of \textit{a} \citep[22--23]{ob:Lynn2016}, and I take Ó Baoill’s lead (\cite[290]{ob:OBaoill2010}) in simply glossing it as “to”. In my examples, I mostly stick to the standard Leipzig abbreviations, however a few Irish-specific abbreviations are also needed:
\begin{itemize}
    \item[] \textsc{aut} = autonomous form
    \item[] \textsc{va} = verbal adjective
    \item[] \textsc{vn} = verbal noun
\end{itemize}}
\glt ‘He is going to take the bus.’
\z

\protectedex{
\ea {\textit{Level 2: Conjoined constructions}}\\
\gll Cosnóidh an t-athnuachan seo ocht gcéad milliún Euro agus leanfaidh sé ar feadh trí bliana.\\
cost.\textsc{fut} the renovation this eight hundred million Euro and continue.{\textsc{fut}} it on duration three years.\\
\glt ‘The renovation will cost eight hundred million Euro and it will continue for three years.’
\z
}

\ea {\textit{ Level 3: Simple finite subordinate clauses (e.g., relative clauses and noun clauses; subject extraposition)}}\\
\ea
\gll Dúirt sé nár thaitin an t-amhrán leis.\\
say.\textsc{pst} he \textsc{comp.neg} shine.\textsc{pst} the song with.\textsc{3sm}\\
\glt ‘He said that he didn't like the song.’
\ex
\gll Tá sé tábhachtach go dtiocfaidh gach éinne in am.\\
be.\textsc{pres} it important \textsc{comp} come.\textsc{fut} every anyone in time.\\
\glt ‘It is important that everybody come on time.’
\z
\z

\ea{\textit{Level 4: Simple nonfinite verb clauses; autonomous form; stative passive}}
\ea
\gll Thug na húdaráis pleanála cead dóibh athnuachan a dhéanamh ar Bhaile Munna.\\
give.\textsc{past} the authorities planning.\textsc{vn.gen} permission to.\textsc{3pl} renovation to make.\textsc{vn} on Baile Munna.\\
\glt‘The planning authorities gave them permission to renovate Baile Munna.’
\ex
\gll Bualadh {go dona} iad.\\
beat.\textsc{past}.\textsc{aut}  badly them.\textsc{acc}\\
\glt ‘They were beaten badly.’
\ex
\gll Tá an áit tréigthe{\footnotemark} ag an teaghlach le fada.\\
be.\textsc{pres} the place abandoned.\textsc{va} by the family for long.\\
\footnotetext{It is an open question whether forms like this are verbal adjectives or past participles (\citet{ob:Lynn2016}: 62)}
\glt‘The place was abandoned by the family long ago.’
\z
\z

\ea{\textit{Level 5: Subordinate adverbial clauses}}\\	
\gll Imreoidh na foirne inniu muna gcuirfidh sé báisteach.\\
play.\textsc{fut} the teams today if.\textsc{neg} put.\textsc{fut} it rain\\
\glt ‘The teams will play today if it doesn't rain.’
\z
\begin{samepage}
\ea\label{ex.level6}{\textit{Level 6: Embedded finite or nonfinite clauses in subject position or modifying subject}}{\footnotemark}
\footnotetext{For copular constructions like this, I included clauses in either subject position or predicate NP position.}
\ea
\gll B' é Jeremiah Roche, a bhí ina bhall de shagairt Chill Téagáin, an fear a maraíodh.\\ 
\textsc{cop.\textsc{past}} \textsc{3sm} Jeremiah Roche \textsc{rel} be.\textsc{past} in.\textsc{3sm.gen} member of priests Chill Téagáin, the man \textsc{rel} kill.\textsc{pst.aut}\\
\glt ‘It was Jeremiah Roche, who was a member of the Kiltegan Fathers, who was killed.
\ex
\gll Tá an fear a gortaíodh sa tionósc ag teacht chuige féin san ospidéal.\\
be.\textsc{prs} the man \textsc{rel} injure.\textsc{past.aut} in.\textsc{def} accident at coming.\textsc{vn} to.\textsc{3sm} self in.\textsc{def} hospital\\
\glt ‘The man who was injured in the accident is recovering in the hospital.’
\z
\z
\end{samepage}


\ea\label{ex.level7}{\textit{Level 7: More than one level of embedding in a single sentence}.}\\
\gll [Deir grúpa caomhnaithe Ghaoth-Barra [go bhfuil imní mhór orthu faoi phleananna an rialtais [ceadúnas a eisiúint do chomhlacht mianadóireachta le [dhul {i mbun} tochailte i gceantair éagsúla i nDún na nGall.]]]\\
say.\textsc{pres} group preservation.\textsc{gen} Gaoth-Barra.\textsc{gen} that be.\textsc{prs} concern big on.\textsc{3pl} about plans the government.\textsc{gen} licence to issue.\textsc{vn} for company mining.\textsc{gen} to go.\textsc{vn} about digging.\textsc{gen} in areas particular in Dún na nGall\\
\glt ‘The Gaoth Barra Preservation Group says that they are very concerned about the government's plans to issue a licence to a mining company to begin digging in certain areas of County Donegal.’
\z

\noindent{Using this scale, I went through every recording in my samples for newsreaders and chat show hosts (plus two benchmark texts, discussed below), giving each of the approximately 660 sentences a D-Level score.}
\is{syntactic complexity|)}

\section{Analysis}

\is{syntactic complexity|(}
My earliest approaches to syntactic analysis of these recordings involved taking an entire transcription and calculating the average complexity level for all of the sentence-level utterances.\footnote{Coding such texts is very time-consuming, requiring exact transcription of relatively long texts, transfer to a spreadsheet, complexity-coding for every utterance, and calculation of averages for the entire text. For this reason, the sample-size in this study is small, but big enough to be provisionally representative, as the statistical analysis in this paper demonstrates.} This resulted in surprisingly low scores for all recordings, but a reanalysis of the texts demonstrated why: Simple filler sentences and common introductory or salutatory utterances were being included that did not necessarily reflect genuinely spontaneous (or planned) speech.\footnote{E.g. \textit{Tá sé cúig nóiméad tar éis a haon.} `It is five minutes past one. \textit{Bhur gcéad fáilte isteach inniu} `You're [all] very welcome to [the show] today'; \textit{Sin a bhfuil ón seomra nuachta} `That's it from the newsroom'; etc.} In order to avoid such utterances impacting complexity measurements for speakers, I discarded, after encoding all of the sentences in a transcription, the bottom 50\%, an action that predictably caused the complexity scores of all speakers to rise but also led to a more visible separation between speakers whose most complex utterances remained mostly paratactic and speakers whose most complex utterances were hypotactic (see Table \ref{tab:chapterhandle2:keytotable}).\footnote{I.e.\ utterances in which the clauses are presented in parallel, using coordinating conjunctions, versus utterances in which the clauses are presented as superordinate and subordinate by the use of subordinating conjunctions.}

\begin{table}[!htbp]
\small
\begin{tabularx}{\textwidth}{rXr}
\lsptoprule
\# & Sentence & Compl. \\
\midrule
1& Na cinnlínte nuachta (ar) raidió na life.& 0\\
2& Is mise XXX XXX& 0\\
3& D'fhreastail na céadta daoine ar shochraid an churaidh oilimpeach Darren Sutherland san Uaimh i gContae na Mí um thráthnóna.& 0\\
4& Bhí ionadaithe ón domhan spóirt i láthair ag an Aifreann.& 0\\
7& Thángadar ar chorp Darren ina árasán i Londain an Luain seo caite.& 0\\
15& Beidh an fear ag teacht os comhair cúirte i gClover Hill an ochtú lá is fiche den mhí seo.& 0\\
16& "Bhuaigh an t-aisteoir Brendan Gleeson Emmy aréir i Los Angeles ar son a ról mar Winston Churchill sa scannán ""Into the Storm""."& 0\\
18& Agus anois an aimsir.& 0\\
19& Lá réasúnta séimh inniu le seal grian anois is arís.& 0\\
20& Beidh beagáinín báisteach ag teacht ón iarthar i dtreo an oirthir níos déanaí inniu.& 0\\
21& An teocht is airde idir cúig déag agus ocht déag gcéim celsius.& 0\\
22& Agus sin deireadh na nuachta go fóill.& 0\\
10& Cosnóidh an t-athnuachan seo ocht gcéad milliún Euro agus leanfaidh sé ar feadh trí bliana, ag giniúint dhá mhíle post.& 2\\
12& Faoin scéim seo beidh trí chéad is a fiche árasán, leabharlann poiblí, pictiúrlann, oifigí, agus ionad siopadóireachta.& 2\\
17& Bhí sé i gcoinne aisteoirí den scoth ar nós Sir Ian McKellen agus Kiefer Sutherland.& 2\\
5& In áirithe bhí Kenny Egan, céile imeartha Darren, {a chaith a chulaith Éireannach in ómós do Darren}.& 3\\
11& Tá an ceadúnas tógála seo ar cheann de na ceadúnais is mó {a tugadh cead dóibh i mBaile Átha Cliath}.& 4\\
6& Cuimhníodh ar Darren mar laoch spóirtiúil ach freisin mar fhear mór muinteartha.& 4\\
8& Creidtear [gur chuir sé lámh ina bhás féin].& 4\\
9& Tá lánchead pleanála faighte ag Treasury Holdings inniu <chun athnuachan a dhéanamh ar  Bhaile Munna>.& 4\\
14& Fuair(eadh) corp Lisa Doyle, fiche ceathar, i gContae Ceatharlach maidin inné.& 4\\
13& [Tá fear tríocha haon bliain d'aois tar éis teacht os comhair cúirte inniu maidir le bás bean i gcontae ceatharlach.]& 4\\
& & 33\\
& & 3.0\\
\lspbottomrule
\end{tabularx}
\caption{Sample transcript with complexity score for top 50\% of sentences}
\label{tab:chapterhandle2:keytotable}
\end{table}

%\FloatBarrier

As mentioned above, in addition to the transcribed news reports and chat monologues that I recorded and transcribed, I added two items to establish upper and lower benchmarks. The text for the lower benchmark was a popular idiomatic version of Jack and the Beanstalk written by Mairéad Ní Ghráda for children (\cite{ob:NiGhrada2002}), which I was hoping would reflect the lowest realistic score that a coherent text might get, and the second a passage from Máirtín Ó Cadhain's short story \textit{Cé Acu} (\cite{ob:OCadhain2009}), which I think represents a very complex text.\footnote{One might be able to find more complex texts using, for example, legal documents or acts of the Oireachtas, but the language of such texts is highly artificial and unsuited for comparison with language intended for real-time communication.} Applying the same ``top 50\%" approach, the children's story scored 2.9\footnote{Including the bottom 50\% of utterances, this text had an average score of 0.9. One could make an argument for using this lower score (since it reflects a coherent text), but it seems only fair to treat this text in the same way that the recordings were treated.}, while the Ó Cadhain text scored 6.9.\footnote{Including all utterances, this fell to 4.35.}
\is{syntactic complexity|)}

\subsection{All recordings}

\is{syntactic complexity|(}
The D-Level scores for the top 50\% of sentences in all twenty recordings and the two benchmark texts are shown in Table \ref{tabDlevelscores}. 


\begin{table}
\small
\begin{center}
\begin{tabular}{llllr}
\lsptoprule
\#  &                   & Gaeltacht or City & Genre & Top 50\% Complexity \\
\midrule
9  & Gaeltacht Chat    & Gaeltacht         & Chat  & 2.3 \\
22 & SeanGasPonaire    & Yardstick         &       & 2.9 \\
5  & Urban RnL Nuacht  & City              & News  & 3 \\
4  & Urban RnL Chat    & City              & Chat  & 3.23 \\
34 & Urban RF Chat     & City              & Chat  & 3.69 \\
6  & Urban RF Chat     & City              & Chat  & 3.86 \\
28 & RnL Nuacht        & City              & News  & 4.07 \\
25 & Urban RF Nuacht   & City              & News  & 4.4 \\
1  & Urban RF Nuacht   & City              & News  & 4.45 \\
35 & Urban Nuacht RnL  & City              & News  & 5 \\
38 & Gaeltacht Chat  & Gaeltacht         & Chat  & 5.1 \\
10 & Urban RnL Chat    & City              & Chat  & 5.22 \\
27 & Urban RnL Chat    & City              & Chat  & 5.27 \\
\midrule
2  & Gaeltacht Chat& Gaeltacht         & Chat  & 5.66 \\
36 & Gaeltacht Chat  & Gaeltacht      & Chat  & 5.77 \\
41 & Gaeltacht Chat    & Gaeltacht         & Chat  & 5.78 \\
26 & Gaeltacht Nuacht  & Gaeltacht         & News  & 6 \\
33 & Gaeltacht Nuacht  & Gaeltacht         & News  & 6.27 \\
3  & Gaeltacht Nuacht  & Gaeltacht         & News  & 6.28 \\
31 & Gaeltacht Chat  & Gaeltacht        & Chat  & 6.83 \\
21 & Máirtín Ó Cadhain & Yardstick         &       & 6.9 \\
43 & Gaeltacht Nuacht  & Gaeltacht         & News  & 7 \\
\lspbottomrule
\end{tabular}
\caption{All D-level scores per recording}
\label{tabDlevelscores}
\end{center}
\end{table}

\is{Gaeltacht|(}
There is an obvious distinction to be seen immediately. Excluding the benchmark texts (labeled Yardstick), the eight highest-scoring items, those with complexity scores between 5.66 and 7.0, are Gaeltacht recordings. Conversely, ten of the twelve lowest-scoring items, those with scores between 3.0 and 5.27, are urban recordings. The urban average is 4.22, while the Gaeltacht average is 5.70. This suggests very strongly that there is a measurable dissimilarity between the syntactic complexity of Gaeltacht broadcast Irish and the syntactic complexity of urban broadcast Irish.
\is{Gaeltacht|)}
\is{syntactic complexity|)}

\subsection{Statistical analysis}

\is{syntactic complexity|)}
While Table \ref{tabDlevelscores} looks persuasive, it seemed politic to perform a statistical analysis on the data to confirm the significance of the findings. A power analysis, performed using SPSS, returned a score of 0.86 (that is, an 86\% chance that the design of the study was reliable), a very healthy score for a small-scale study like this. For significance, I applied an Independent Means t-test to the complexity scores of the broadcasters, treating urban broadcasters as sample 1 and treating \isi{Gaeltacht} broadcasters as sample 2. The null hypothesis was that there was no significant difference between sample 1 and sample 2, and the alternative hypothesis was that there was a statistically significant variance in syntactic complexity between the two groups. My calculations, performed both by hand and using SPSS, returned a p-value of 0.008 (equal variances not assumed), well below the standard significance threshold of 0.05, and arguing very strongly against the null hypothesis.
\is{syntactic complexity|)}

\subsection{Limitations}

There are three limitations to this study. The first is sample size (ten \isi{Gaeltacht} broadcasters and ten urban broadcasters), but both the power analysis and the t-test figures strongly suggest that the study is reliable and that there is a statistically significant variance between the two samples.

\is{syntactic complexity|)}
The second limitation is in the measurement of complexity scores. That is to say, observer bias in the scoring of the sentences cannot be discounted. A well-funded project involving a widely spoken language would be able to train and pay two individuals to independently assign complexity scores to the c.660 discrete sentences collected from all the recordings, thus ensuring a minimum of unconscious bias during the assignment of complexity scores. Alas, for a minority language like Irish, this was simply out of the question. Those individuals in the world who are comfortably conversant with syntactic theory, the Irish language, and Covington's adaptation of Rosenberg and Abbeduto's complexity score could be optimistically counted on a single hand.
\is{syntactic complexity|)}

The third limitation is the professional status of the broadcasters. It may well be that professional status affects the complexity score of a broadcaster, and for that reason, this study's findings remain localized to the broadcasters on the three stations.

The above analysis shows a clear distinction between the two samples, but it also treats all the recordings as equal. However, it is not a surprising admission to acknowledge that there is a difference between the prepared texts of news recordings and the ad-hoc nature of chat show monologues. This study, therefore, will analyze the two genres separately.

\subsection{Comparison of news broadcasts}

Since newsreaders read from prepared texts that describe often complex situations, one can assume that the language of their reports reflects careful thought and comparatively complex language. Material will be similar, whether the newsreader is working from an urban or \isi{Gaeltacht} newsroom; audiences will be similar (educated adults); the speakers' situations will be similar: they have prepared and printed out a news report beforehand and are reading it aloud live in a studio. 


\begin{table}
\begin{tabular}{llllr}
\lsptoprule
\#  &                   & Gaeltacht or City & Genre & Top 50\% Complexity \\
\midrule
22 & SeanGasPonaire    & Yardstick &      & 2.9 \\
5  & Urban RnL Nuacht  & City      & News & 3 \\
28 & RnL Nuacht        & City      & News & 4.07 \\
25 & Urban RF Nuacht   & City      & News & 4.4 \\
1  & Urban RF Nuacht   & City      & News & 4.45 \\
35 & Urban Nuacht RnL  & City      & News & 5 \\
\midrule
26 & Gaeltacht Nuacht  & Gaeltacht & News & 6 \\
33 & Gaeltacht Nuacht  & Gaeltacht & News & 6.27 \\
33 & Gaeltacht Nuacht  & Gaeltacht & News & 6.28 \\
21 & Máirtín Ó Cadhain & Yardstick &      & 6.9 \\
43 & Gaeltacht Nuacht  & Gaeltacht & News & 7 \\
\lspbottomrule
\end{tabular}
\caption{Low-high complexity scores for newsreaders}
\label{fig:OBroin:3}
\end{table}

\is{syntactic complexity|)}
The average complexity score for urban newsreaders is 4.18, while that of \isi{Gaeltacht} newsreaders is 6.39, and while the number of recordings for this study is comparatively small, there is a very clear pattern visible. As shown in Table \ref{fig:OBroin:3}, no \isi{Gaeltacht} newsreader scored less than 6.0, while no urban newsreader scored higher than 5.0.

The urban average may suggest that urban newsreaders are making sentences that contain basic adjectival and nominal clauses (level 4), but the real absence is of level-seven sentences in urban news reports, that is, sentences containing complex, nested clauses such as in (\ref{ex.complexnestedclauses}).


\ea\label{ex.complexnestedclauses}
\gll [Agus deir Cónaidhm Árachais na-hÉireann anois go mbeidh méadú ar tháillí árachais de-bharr an mhéid airgid (a íocadh le úinéirí tithe le ceithre mhí)].\\
and say.\textsc{prs} federation Insurance.\textsc{gen} of-Ireland now that be.\textsc{fut} increase on costs insurance.\textsc{gen} because the sum money.\textsc{gen} \textsc{rel} pay.\textsc{past.aut} to owners houses.\textsc{gen} for four months\\
\glt ‘And the Insurance Federation of Ireland now says that there will be an increase in insurance charges because of the amount of money that was paid to house owners in the last four months.’
\z

\noindent {The urban strategy for forming complex sentences seems to involve stringing together a series of utterances and connecting them with conjunctions, as shown in (\ref{ex.urbanstrings}). The strategy works, but makes for sentences that require patience from the listener to connect the various ideas together:}

\ea\label{ex.urbanstrings}
\gll Tá práinn tagtha ar an dóigh a bhfuil cúrsaí anois de-thairbhe neamhshocracht ins na margaí mar-gheall-ar an ngníomh seo, agus fosta mar-gheall-ar seo tá laghdú tagtha inniu ar chostas na n-iasachtaí fá-choinne gheilleagair na-hEorpa, agus mar-gheall-ar an ngéarchéim chéanna fosta, tá cruinniú socraithe ag an aire airgeadais Sammy Wilson anseo chun éifeacht na géarchéime ar an ghéilleagar seo a mheas.\\
be.\textsc{prs} urgency come.\textsc{va} on the way \textsc{rel} be.\textsc{prs} events now because instability in the markets because the action this and also because this be.\textsc{prs} fall come.\textsc{va} today on cost the loans.\textsc{gen} for economy Europe.\textsc{gen}, and because the crisis same also, be.\textsc{prs} meeting arrange.\textsc{va} by the minister finance.\textsc{gen} Sammy Wilson here for effect the crisis.\textsc{gen} on the economy this to measure.\textsc{vn} \\
\glt ‘Urgency has come to the way things are because of instability in the markets because of this action, and also because of this the cost of loans has fallen for the European economy, and because of the same crisis also, the finance minister Sammy Wilson has arranged a meeting to ascertain the effect of the crisis on the economy.' \footnote{I have attempted to retain the manner in which the original Irish's clauses \textit{squint}. That is, the listener is not certain what relationship they are in with the preceding and foregoing clauses.}
\z

Note the outlier in this group, who is an urban newsreader returning a D-Level score of 3.0. The speaker nevertheless does not seem to have any particularly idiosyncratic linguistic features that stand out. Their logical strategy seems to simply involve using pronouns as the principal aid in connecting one sentence to the next, as we see in (\ref{ex.urbanoutlier}):

\ea\label{ex.urbanoutlier}
\gll Bhuaigh an t-aisteoir Brendan Gleeson Emmy aréir i Los Angeles ar-son a ról mar Winston Churchill sa scannán “Into the Storm”. Bhí sé i-gcoinne aisteoirí den-scoth ar-nós Sir Ian McKellen agus Kiefer Sutherland.\\
win.\textsc{past} the actor Brendan Gleeson Emmy last-night in Los Angeles for his role as Winston Churchill in-the film Into the Storm. be.\textsc{past} he against actors excellent like Sir Ian McKellen and Kiefer Sutherland.\\
\glt ‘The actor Brendan Gleeson won an Emmy last night in Los Angeles for his role as Winston Churchill in the movie “Into the Storm.” He was [up] against excellent actors like Sir Ian McKellen and Kiefer Sutherland.’
\z

This material might possibly be compressed into a single, more-complex sentence by a speaker with higher D-Level averages.\footnote{For example, by subordinating the second sentence in the following manner: “Fighting off stiff competition from a star-studded list of candidates that included Sir Ian McKellen and Kiefer Sutherland, ...”.}

Equally interesting are the very stable figures returned by \isi{Gaeltacht} newsreaders, whose reports all achieved complexity scores between 6.0 and 7.0, suggesting both a very high frequency of level-seven sentences, and a very stable level of complexity. Urban syntactic complexity swung between 3.0 and 5.0. Much as the figures for \isi{phonetics} and \isi{morphology} suggested, the syntactic complexity of \isi{Gaeltacht} news texts is predictable and stable, while that of urban newsreaders varies widely.
\is{syntactic complexity|)}

\subsection{Comparison of chat show monologues}

\is{syntactic complexity|)}
Chat shows offered a better view of extempore language. The broadcasters were still speaking under controlled studio conditions, but apart from the possibility of having notes to remind them of what was coming up in their shows, it was virtually certain that their monologues were unscripted and that their language therefore reflected the genuine day-to-day patterns of the speakers.
 
\begin{table}
\small
\begin{tabular}{llllr}
\lsptoprule
\#  &                      & Gaeltacht or City & Genre & Top 50\% Complexity \\
\midrule
9  & Gaeltacht Chat   & Gaeltacht & Chat & 2.3 \\
22 & SeanGasPonaire       & Yardstick &      & 2.9 \\
4  & Urban RnL Chat       & City      & Chat & 3.23 \\
34 & Urban RF Chat        & City      & Chat & 3.69 \\
6  & Urban RF Chat        & City      & Chat & 3.86 \\
\midrule
38 & Gaeltacht Chat    & Gaeltacht & Chat & 5.1 \\
10 & Urban RnL Chat       & City      & Chat & 5.22 \\
27 & Urban RnL Chat       & City      & Chat & 5.27 \\
\midrule
2  & Gaeltacht Chat   & Gaeltacht & Chat & 5.66 \\
36 & Gaeltacht Chat  & Gaeltacht & Chat & 5.77 \\
41 & Gaeltacht Chat       & Gaeltacht & Chat & 5.78 \\
31 & Gaeltacht Chat    & Gaeltacht & Chat & 6.83 \\
21 & Máirtín Ó Cadhain    & Yardstick &      & 6.9 \\
\lspbottomrule
\end{tabular}
\caption{Complexity scores for chat show hosts}
\label{fig:OBroin:4}
\end{table}

A similar dichotomy exhibited itself with these speakers: the D-Level average for urban chat show hosts was 4.25, while that for \isi{Gaeltacht} chat show hosts was 5.24. As shown in Table \ref{fig:OBroin:4}, no urban broadcaster had a higher score than 5.27.

Level five of the Rosenberg and Abbeduto scale marks a level at which speakers are forming relatively complex adverbial clauses. That is to say, they are creating relationships between two separate events at main-clause and subordinate-clause level. Level four mostly concerns itself with structures that involve non-finite verbal structures, but not necessarily the establishment of relationships between events, as we find in the subordinate clauses of level 5.

The average for urban broadcasters falls well below level five, suggesting that adverbial clauses are uncommon in their speech. Three of the five urban broadcasters return scores below level four, suggesting that they rarely form sentences that include subordinate non-finite verb structures. Regarding the creation of relationships between utterances (which a complex speaker would achieve through the use of subordinating conjunctions and/or nested clauses), urban broadcasters appear to be expressing such relationships by using coordinating conjunctions and expecting that the listeners will intuit the relationships themselves, as in (\ref{ex.urbancoordconj}).

\ea\label{ex.urbancoordconj}
\gll So, [tháinig sé ar-ais óna laethanta-saoire] \textbf{agus} [bhí a dhoras chairr briste síos] \textbf{agus} [bhí, bhí a chlann tar-éis, sochraid \textbf{agus}, you know, {mar sin},[ a bheadh orthu]], \textbf{agus}, [bhí gach, gach éinne, like, “Tá tú marbh” [nuair a tháinig sé ar-ais]].\\
so come.\textsc{past} he back from-his holidays and be.\textsc{past} his door car.\textsc{gen} broken.\textsc{va} down and be.\textsc{past} be.\textsc{past} his family after funeral and you know therefore \textsc{rel} be.\textsc{cond} on-them and be.\textsc{past} every every person like be.\textsc{prs} you dead when \textsc{rel} come.\textsc{past} he back\\
\glt ‘So, he came back from his vacation and his car door was broken down and his family were after having have a funeral, you know, and, therefore, and everyone was like ``you're dead" when he came back.’
\z

\is{Gaeltacht|(}
There is nevertheless one obvious outlier among the chat show hosts, just as there was among the newsreaders: the lowest score of all was from a Gaeltacht broadcaster, who returned a score of 2.30. This presenter, a popular and well-known Gaeltacht personality, has a very recognizable telegrammatic style involving clipped single-phrase utterances that he frequently repeats. Despite his frequent use of conjunctions, it was usually clear when a new utterance was beginning, however, and even if more than two utterances could be conjoined, the resulting utterance would still only reach a D-level complexity\is{grammatical complexity} of 2. 

One would expect the D-level of news reports to be measurably higher than that of extemporaneous chat show monologues, and that is what happened in the Gaeltacht, where the newsreaders' average was 6.39 and the chat shows' average was 5.24. Extemporaneous speech does not lend itself so readily to complex, planned sentences containing deeply nested clauses and phrases. It is perhaps surprising, therefore, to find that the D-Level average for urban chat show hosts (4.25) is \textit{higher} than that of urban newsreaders (4.18). If we do allow the possibility that urban broadcasters, whether in prepared or extemporaneous speech situations, typically hover around the 4.2 mark, and if we recall that complex level-7 sentences appear to be making the most difference in these calculations, this suggests that urban broadcasters do not form such sentences, even when given ample time to prepare.
\is{syntactic complexity|)}

\section{Conclusions}

\is{syntactic complexity|(}
In prepared texts, there is a clear difference between the complexity scores for Gaeltacht news reports versus those of urban news reports. The Gaeltacht average (6.39) was more than two full points ahead of the urban average (4.18) on the eight-category D-Level scale adapted from Rosenberg and Abbeduto. This is a significant finding, and suggests that urban broadcast Irish, even in non-extemporaneous conditions where the speakers had ample time to prepare their texts, is being prepared on a complexity level significantly below that of the Gaeltacht and below that at which sentences with adverbial clauses are usually being made. Given that in previous research I had already found major differences in pronunciation and \isi{morphology}, this suggests that urban broadcast Irish is syntactically very different from Gaeltacht broadcast Irish.
In extemporaneous speaking situations, the average complexity level of Gaeltacht chat show hosts did fall one point (to 5.24), and this is what we would be expecting, given that complex, level-7 sentences often take preparation time that is not available to speakers who are planning sentences on the fly. This figure nevertheless remained nearly a point ahead of urban extemporaneous language. The most significant finding here, however, was that urban figures hardly moved. In fact, the urban average actually rose fractionally, from 4.18 to 4.25. This strongly suggests that urban broadcasters, whether given time to prepare their sentences or not, are mostly not forming complex sentences with nested phrases.

The figures for newsreaders provide a good illustration of how a D-Level of 5 seems to be the point at which the urban and Gaeltacht broadcasters separate. That is to say, level 5 sentences (e.g. subordinate adverbial clauses) are occasionally being formed by urban broadcasters, but level 7 sentences (complex multiclausal sentences with nested phrases) are mostly not. Gaeltacht broadcasters were much quicker to form complex sentences of this sort. Indeed, 72\% of Gaeltacht newsreader's sentences were scored at level 7, while only 15\% of urban sentences received similar scores.\footnote{These percentages are for each recording's top half of sentences, of course, and perhaps could be halved to represent the total of level-7 sentences for each speaker.} 

Staying with D-7 sentences, chat show hosts reflected the pattern that we saw above in our analysis of total D-Level figures. Gaeltacht extemporaneous speech D-7 figures fell from 72\% to 33\%, but urban figures hardly moved (rising from 12\% to 15\%). This suggests that urban broadcasters, whether or not they have time to prepare sentences, tend to produce only about 12\%-15\% of their most complex sentences at D-7 level.

One must note that languages encode complexity in different ways. An absence of embedded verbal clauses does not necessarily imply that a language or dialect is somehow less complicated. Nor does the frequent presence of embedded verbal clauses suggest that a language or dialect is necessarily more sophisticated. One would, however, expect equal levels of \isi{parataxis} and \isi{hypotaxis} between dialects of the same language, and it is therefore noteworthy that the syntactic complexity of urban and Gaeltacht broadcasters (as defined using the adapted Rosenberg and Abbeduto scale) is significantly different. That is, urban and Gaeltacht Broadcasters of Irish appear to have measurably different expectations of their listeners with regard to the manner in which they connect clauses.
\is{Gaeltacht|)}
\is{syntactic complexity|)}

\section{Implications and further work}

\is{syntactic complexity|(}
It remains an open question, and now one in urgent need of addressing, whether urban broadcast Irish is syntactically simpler because its speakers are L2 speakers. If measurable differences can be found between urban L1 and L2 speakers, this would bolster the suggestion that urban Irish is unlikely to develop in syntactic complexity unless it is adopted by parents as a home language and accrues more native speakers.

\is{Gaeltacht|(}
The sample for this study was relatively small: twenty speakers in total. No attempt was made to control for gender, age, education, professional status, or (crucially) status as an L1 or L2 speaker. The only control was for approximate geographical origin. So, due diligence was done in ensuring that a \isi{Dublin} speaker was, indeed, a Dublin speaker, a \isi{Belfast} speaker was a Belfast speaker, and a Gaeltacht speaker was, indeed, a Gaeltacht speaker. As noted above, two other limitations to this study must be acknowledged: that the professional/amateur status of the Gaeltacht/urban broadcasters may have influenced their \isi{syntax} scores; and that unconscious bias on the scorer's part may have influenced the D-level scoring of sentences. For this reason, it is best to view this paper's findings on the differences between urban and Gaeltacht Irish as preliminary and subject to confirmation or rejection in future studies.
\is{Gaeltacht|)}

As I noted in my previous work on \isi{phonetics} and \isi{morphology}, urban broadcasters tend to be discarding features of Irish that are not found in English. Velar and palatal fricatives are being dropped in favor of the nearest English sound, for example, while nouns are frequently no longer morphophonetically marked for case, with eclipsis and lenition becoming optional. In this paper we make a compelling case that urban broadcast Irish is also significantly different in \isi{syntax}, substituting subordinators with conjunctions that require the listener to intuit the relationships between clauses and rarely forming sentences that involve the nesting of embedded phrases and clauses.
\is{syntactic complexity|)}

\il{Irish (Modern)|)}

\section*{Acknowledgements}
I would like to extend my sincere thanks to several scholars: Andrew Carnie and the two anonymous readers of an early draft of this paper for suggesting a more formal statistical analysis; Michael Gordon of the Psychology Department, WPUNJ for guiding me in the direction of power analysis; and Nicole Vittoz of Douglas College, British Columbia, for providing procedural guidance in this particular analysis. Any errors, of course, are my own.



\sloppy
\printbibliography[heading=subbibliography,notkeyword=this]
\end{document}
