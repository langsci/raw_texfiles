\documentclass[output=paper,colorlinks,citecolor=brown]{langscibook}
\ChapterDOI{10.5281/zenodo.15654877}
\author{Pavel Iosad\affiliation{University of Edinburgh}}
\title{Laryngeal contrast in the Celtic languages: Variation, typology, and history}

\IfFileExists{../localcommands.tex}{
   % add all extra packages you need to load to this file

\usepackage{tabularx,multicol}
\usepackage{url}
\urlstyle{same}

\usepackage{listings}
\lstset{basicstyle=\ttfamily,tabsize=2,breaklines=true}

\usepackage{langsci-basic}
\usepackage{langsci-optional}
\usepackage{langsci-lgr}
\usepackage{langsci-osl}
% \usepackage{./langsci/styles/langsci-lgr}
% \usepackage{./langsci/styles/langsci-osl}
% \usepackage{langsci-gb4e}

\usepackage{tikz}
\usetikzlibrary{patterns,calc}
\pgfdeclarepatternformonly{south east lines}{\pgfqpoint{-0pt}{-0pt}}{\pgfqpoint{3pt}{3pt}}{\pgfqpoint{3pt}{3pt}}{
    \pgfsetlinewidth{0.6pt}
    \pgfpathmoveto{\pgfqpoint{0pt}{3pt}}
    \pgfpathlineto{\pgfqpoint{3pt}{0pt}}
    \pgfpathmoveto{\pgfqpoint{.2pt}{-.2pt}}
    \pgfpathlineto{\pgfqpoint{-.2pt}{.2pt}}
    \pgfpathmoveto{\pgfqpoint{3.2pt}{2.8pt}}
    \pgfpathlineto{\pgfqpoint{2.8pt}{3.2pt}}
    \pgfusepath{stroke}}
    
\usepackage{stmaryrd}
\usepackage{wasysym}
\usepackage{multirow}
\usepackage{caption}
\usepackage{subcaption}
\usepackage{mathrsfs}
\usepackage{qtree}

\usepackage{linguex}


   %pminos do not split footnotes
% \interfootnotelinepenalty=10000 %Footnote in Laporte chapters has to be split SN


%\DeclareIndexNameFormat{default}{%
%\nameparts{#1}%
%\usebibmacro{index:name}%
%{\index[names]}%
%{\namepartfamily}%
%{\namepartgiveni}%
% {}% L1
% {}% L2
%{\namepartprefix}% generates spurious space L3
%{\namepartsuffix}% generates spurious space L4
%}

%  {\DeclareIndexNameFormat{default}{%
%     \usebibmacro{index:name}{\index[names]}{#1}{#3}{#5}{#7}}}

%\DeclareIndexNameFormat{default}{%
%  \usebibmacro{index:name}{\sindex[nom]}{#1}{#3}{#5}{#7}}

%\DeclareIndexNameFormat{default}{%
%  \usebibmacro{index:name}{\sindex[person]}{#1}{#3}{#5}{#7}}
%\DeclareIndexNameFormat{default}{%
%\nameparts{#1} \usebibmacro{index:name}{\sindex[person]]}{\namepartfamily}{‌​\namepartgiven}{\nam‌​epartprefix}{\namepa‌​rtsuffix}}

%\newcommand{\smiley}{:)}

%\renewbibmacro*{index:name}[5]{%
%\usebibmacro{index:entry}{#1}%
%{\iffieldundef{usera}{}{\thefield{usera}\actualoperator}\mkbibindexname{#2}{#3}{#4}{#5}}}

% \newcommand{\noop}[1]{}

%remove for final
%\overfullrule=1mm

\newcommand{\tobi}[2]}}
\renewcommand{\S}[1]{\tobi{#1}{\textsc{*}}}

% this volume references
% puts: [this volume]
% already defined: \citetv
%\newcommand{\citepv}[1]{(\citeauthor{#1} \citeyear*{#1} [this volume])}
\newcommand{\citealtv}[1]{\citeauthor{#1} \citeyear*{#1} [this volume]}

%parentheses around example number
\newcommand{\pref}[1]{(\ref{#1})}

% in-text examples

\newcommand{\lnex}[1]{\textit{#1}} %target lang word
\newcommand{\lnlit}[1]{(lit.: `#1')} %literal reading
\newcommand{\lnlat}[1]{(#1)} % latinization
\newcommand{\lntrans}[1]{`#1'} %translation
\newcommand{\lnexl}[2]%
{\lnex{#1}{} \lnlat{#2}} % ex with latinization
\newcommand{\lnexlat}[3]{\lnex{#1}{} \lnlat{#2}{} \lntrans{#3}} % ex with latinization and tranl.

%ch01
\newcommand{\co}[1]{\mbox{\textbf{#1}}}

%ch09

\newcommand{\cyrbulg}[1]{\begin{otherlanguage*}{bulgarian}#1\end{otherlanguage*}}


%ch10
\newcommand{\nlp}{{\small NLP}}
\newcommand{\mwe}{{\small MWE}}
\newcommand{\rae}{{\small RAE}}
\newcommand{\lvc}{{\small LVC}}
\newcommand{\pos}{{\small P}o{\small S}}
%\newcommand{\todo}[1]{ \textcolor{red}{#1} }

%\renewcommand{\labelenumi}{\theenumi}
%\ainamefmt{{vv}{ll}{, ff}{, jj}} % fullname

\newcommand{\biberror}[1]{{\color{red}#1}}

\newcommand{\osenovaitem}{--~}
   %% hyphenation points for line breaks
%% Normally, automatic hyphenation in LaTeX is very good
%% If a word is mis-hyphenated, add it to this file
%%
%% add information to TeX file before \begin{document} with:
%% %% hyphenation points for line breaks
%% Normally, automatic hyphenation in LaTeX is very good
%% If a word is mis-hyphenated, add it to this file
%%
%% add information to TeX file before \begin{document} with:
%% %% hyphenation points for line breaks
%% Normally, automatic hyphenation in LaTeX is very good
%% If a word is mis-hyphenated, add it to this file
%%
%% add information to TeX file before \begin{document} with:
%% \include{localhyphenation}
\hyphenation{
    Beck-man
    Ngu-yen
    back-chan-nel
    back-chan-nels
    mo-not-o-nous
    ste-reo-typ-i-cal
}

\hyphenation{
    Beck-man
    Ngu-yen
    back-chan-nel
    back-chan-nels
    mo-not-o-nous
    ste-reo-typ-i-cal
}

\hyphenation{
    Beck-man
    Ngu-yen
    back-chan-nel
    back-chan-nels
    mo-not-o-nous
    ste-reo-typ-i-cal
}

   \boolfalse{bookcompile}
   \togglepaper[11]%%chapternumber
   \pgfplotsset{compat=1.18}
}{}

\AffiliationsWithoutIndexing

\abstract{This chapter addresses the nature of laryngeal contrast throughout the history of the Celtic languages. In the present day, there is significant phonetic and phonological variation across the languages in distinctions among two series of stops, while the historical data is ambiguous in multiple ways. The chapter re-evaluates the issues in light of some recent findings in phonetic and diachronic typology. I argue that the \enquote{aspirating} system of many present-day languages (contrasting aspirated \ipa{/pʰ~tʰ~kʰ/} stops with a variably voiced \ipa{/b~d~ɡ/} series) most likely arose around the mid 1st millennium CE \enquote{syncope period} in both Gaelic and Brythonic. I show there are significant difficulties in projecting it towards the earlier period, where a \enquote{voicing-based} system (unaspirated \ipa{/p~t~k/} vs. categorically voiced \ipa{/b~d~ɡ/}) is more compatible with the observed sound changes. This analysis implies that the rise of the aspirating system aligned with a suite of drastic phonological changes affecting the Insular Celtic languages in the sub-Roman period, and underscores the importance of synchronic variation for reconstructing the early history of the Celtic laryngeal contrast.}

\begin{document}

\maketitle

\is{aspirating languages|(}
\is{voicing languages|(}
\is{consonant strength|(}
\is{phonetic variation|(}

%\slp{Check/fix citations, e.g. "Iosad forthcoming[b]"}

\section{Introduction}
\label{sec:introduction}

The aim of this chapter is to consider the history of laryngeal contrast in the Celtic languages, especially the phonetics and phonology of stops. Most present\hyp day Celtic languages distinguish between two series of stops, spelled in most contexts as \orth{p~t~c/k} and \orth{b~d~g}.\footnote{I use \orth{angled brackets} to refer to orthographic representations, \cat{all caps} for abstract phonological categories, \featr{pipes} to signal phonological feature specifications, /slashes/ for phonological underlying specifications, and [square brackets] for phonological surface representations.} Their phonetic realisations vary non\hyp trivially across the family: to avoid ambiguity, I will use the terms \cat{fortis} and \cat{lenis} respectively to refer to the two series, eschewing the terminology of \enquote{voiced}, \enquote{voiceless}, or \enquote{(un)aspirated} to avoid confusion between phonological and phonetic description.\footnote{The terms \enquote{fortis} and \enquote{lenis} themselves, of course, do also have phonetic connotations and have been used to refer to aspects of phonetic realisations: for a review, see \textcite{honeybone08:_lenit}. It would have been preferable to use even less suggestive terminology: one possibility is calling the \orth{p~t~k} series \emph{tenues} and \orth{b~d~g} \emph{mediae}, a usage formerly common especially in Indo\hyp European (including Celtic) comparative philology. However, this terminology is now so unfamiliar that I have opted for \enquote{fortis} and \enquote{lenis}.} Synchronically, the present\hyp day Celtic languages are fairly similar in the (morpho)phonological patterning of stops, but are considerably more diverse in how the contrast is phonetically implemented. Diachronically, the trajectories of sound change that led to the present\hyp day states of the languages differ more significantly, and, as I will show in this chapter, this provides us with a wealth of information for phonetic analysis and typological comparison.

In order to consider the history of laryngeal contrast, in \cref{sec:laryng-phon-pres} I sketch the patterns observed in the present\hyp day languages. \Cref{sec:aspir-view-proto} lays out the arguments made in favour of a view of earlier stages of Celtic as an \enquote{aspirating} language characterized by aspiration of the \cat{fortis} series and variable voicing of the \cat{lenis} series of stops; in various guises, this analysis has long been commonly accepted as correct. In \cref{sec:exam-sound-chang} I lay out the relevant sound changes and consider whether they are phonetically and typologically plausible from an \enquote{aspirating} starting point. \Cref{sec:recons-evid} synthesizes the findings and lays out the key argument of this chapter that \enquote{aspirating} systems are a relatively recent innovation in the history of the attested Insular Celtic languages, possibly dating to no earlier than the mid 1st millennium CE. \Cref{sec:conclusion} concludes the chapter with a range of further considerations, with specific attention to reconstructing variation in the proto\hyp language and its consequences particularly in the case of Celtic.

\section{Laryngeal phonology in present day Celtic}
\label{sec:laryng-phon-pres}

\subsection{Theoretical preliminaries}
\label{sec:theor-prel}

The analysis of laryngeal stop contrasts in this chapter is informed by the influential framework of \isi{Laryngeal Realism} \parencite[a term due to][]{honeybone05}. It was formulated as an alternative to a view common in mainstream generative phonology, in which two\hyp term systems contrasting \orth{p~t~k} and \orth{b~d~g} series were generally analysed as specified \feature{-}{voice} and \feature{+}{voice}, respectively. Under this \enquote{broad interpretation of the feature \feature{\pm}{voice}} \parencite{hall2001introduction}, many cross\hyp linguistic differences in the phonetic correlates of the two specifications are considered irrelevant, as long as voicing participates in the realisation of the contrast in some way. Therefore, languages like French\il{French (Modern)} \glottocode{stan1290} and \ili{Ukrainian} \glottocode{ukra1253}, in which \cat{fortis} stops lack significant aspiration and \cat{lenis} stops have categorical voicing in all contexts could be analysed as identical to languages like English\il{English (Modern)} \glottocode{stan1293} and \ili{German} \glottocode{stan1295} with aspiration of the \cat{fortis} stops in many contexts and variable presence of voicing in the \cat{lenis} series: aspiration in the latter can be either ignored or considered redundant. It was only languages like \ili{Icelandic} \glottocode{icel1247}, which distinguish between \ipa{[pʰ~tʰ~kʰ]} and \ipa{[p~t~k]}, that needed a different analysis, since both series are \feature{-}{voice}: in this case, with a \feature{+}{spread glottis} specification for the \cat{fortis} stops.

For proponents of \isi{Laryngeal Realism}, the analysis of such systems should take into account both the phonetic properties of the contrast and the stops' morphophonological behaviour. Featural specifications are required to be supported by evidence of two kinds. First, the relevant feature's phonetic correlate (such as voicing or aspiration) should consistently participate in the realisation of the stop series. Second, one should look for (morpho)phonological evidence that the specification is active. For instance, if a stop bears the \featr{voice} specification, then this feature can trigger voicing assimilation, or be restricted from appearing in \enquote{weak} positions, such as word- or syllable\hyp finally, resulting in sound patterns such as word\hyp final devoicing. Widespread adoption of versions of \isi{Laryngeal Realism} has been encouraged by numerous cases where the two sets of criteria~-- phonetic realisation and phonological activity~-- point in the same direction. In this way, languages can be classified as belonging to one of a small number of \enquote{types}. Thus, English\il{English (Modern)} and \ili{German} have been shown to combine aspiration of \cat{fortis} stops with their phonological visibility on the one hand, and an absence of phonetic or phonological activity of voicing in the \cat{lenis} series; conversely, languages like \ili{Catalan} \glottocode{stan1289} or \ili{Bulgarian} \glottocode{bulg1262} combine categorical voicing of \cat{lenis} stops with evidence for phonological activity of this series, while \cat{fortis} stops are both unaspirated and relatively inert phonologically.

This happy confluence has led to the emergence of a particularly strong version of \isi{Laryngeal Realism}, in which phonetic and phonological criteria must always point in the same direction. Conceptually, this prediction arises because no mechanism within the theory allows for such conflict. Under this view, demonstrating that a particular phonetic cue is obligatorily associated with some category of segments is sufficient to infer the presence of the corresponding featural specification, even in the absence of clear (morpho)phonological evidence \parencite[e.\,g.][]{beckmanng:_empir}. The converse corollary also holds: (morpho)phonological evidence alone can be used to determine a language's place in the typological space. This is particularly useful for historical phonology. While phonetic detail is usually not directly recoverable, one can be on surer ground with respect to morphophonological patterns based on philological or dialectal evidence. If the corollary is correct, one is in a position to make inferences about phonetic detail in the past based on phonological reconstruction. For example, \textcite{spaargaren09:_chang_englis} argues that patterns of laryngeal assimilation in \ili{Old English} (as in \textit{cēp\mbox{-}\hili{te}} `{keep.\Pst.\Tsg}' vs. \textit{freme-\hili{de}} `{do.\Pst.\Tsg}', \textit{hīer-\hili{de}} `{hear.\Pst.\Tsg}') show greater phonological activity of the \featr{fortis} series of stops, which should lead us to reconstruct \ili{Old English} \featr{fortis} stops as being aspirated.

This very strong version of \isi{Laryngeal Realism} has been challenged on a number of grounds. \Textcite{salmons2017germanic} argues that a phonetic specification does not necessarily imply the presence of the relevant phonological feature, as in those Germanic languages like Dutch \glottocode{dutc1256} and perhaps Central Swedish \glottocode{stan1279} where \featr{lenis} stops are consistently voiced but remain relatively inert phonologically. \Textcite{kirby2019effects} question the privileged role of aspiration (voice onset time) in the typology, pointing to the important role of other cues. Most controversially, \textcite{cyran13:_polis} and \textcite{iosad16} develop case studies to show that phonetic and phonological criteria can be in outright conflict \parencite[for a rejoinder, however, see][]{raimy2021privativity}.

In this chapter, I will use \isi{Laryngeal Realism} terminology where appropriate to refer to covarying aspects of phonetic and phonological behaviour. With the exception of Breton\il{Breton (Modern)} and possibly several varieties of Scottish Gaelic\il{Scottish Gaelic (Modern)}, the present\hyp day Celtic languages belong to the \enquote{aspirating} type, as I will now discuss in more detail.

\subsection{The Gaelic languages}
\label{sec:gaelic-languages}

In \isi{Laryngeal Realism} terms, the Gaelic languages clearly belong to the \enquote{aspirating} type, with marked \featr{spread glottis} stops in the \orth{p~t~k} series and unspecified \orth{b~d~g} stops.

There is near\hyp universal agreement across auditory and instrumental descriptions that the \cat{fortis} stops are postaspirated in both Irish\il{Irish (Modern)} (e.g. \cite{nichasaide1986preaspiration,óraghallaigh2014fuaimeanna}) and the majority of Scottish Gaelic\il{Scottish Gaelic (Modern)} varieties (e.g. \cite{nichasaide1986preaspiration,Nance2019}) in the onset of a stressed syllable. Furthermore, \cat{fortis} stops are preaspirated\is{preaspiration} in postvocalic (or perhaps more precisely post\hyp sonorant) position in much of Scottish Gaelic\il{Scottish Gaelic (Modern)}, and likely also at least in some varieties of Irish\il{Irish (Modern)} \parencite{iosad2020phonological}.

Phonologically, \featr{spread glottis} is active in the \cat{fortis} series. The clearest evidence comes from Irish\il{Irish (Modern)}, which has a pattern of coalescence of a \cat{lenis} obstruent with a following \ipa{/h/} that results in a \cat{fortis} consonant. It is observed, for instance, with the future\fshyp conditional verbal suffix spelled \orth{-f}: Modern Irish\il{Irish (Modern)} (\textsc{Ir}\textit{lig} `{let.\Imp.\Ssg}' with \ipa{[ɡʲ]}, \textit{lig-f-idh} `{let-\Fut-\Tsg}' with \ipa{[kʲ]} (\cite{sommerfelt1960donegal,buachalla85:_moder_irish}). Similar behaviour is seen with the passive participle suffix \textit{-tha}: \textit{sníomh} `{spin.\Imp.\Ssg}' with [v]/[β], \textit{sníofa} `{spun}'.)

Evidence also comes from initial mutation: in \isi{lenition}, \cat{fortis} stops \ipa{/pʰ~tʰ~kʰ/} (and their slender counterparts) alternate with the voiceless fricatives \ipa{[f~h~x]}, and it is widely accepted that such fricatives are phonologically \featr{spread glottis} \parentext{\enquote{Vaux's Law}, after \textcite{vaux-fricatives}}. Assuming that \isi{lenition} alters only the manner specification of the stops (\emph{modulo} any necessary place changes), this can be taken as further evidence that the \cat{fortis} stops are \featr{spread glottis}.

The situation in Scottish Gaelic\il{Scottish Gaelic (Modern)} is somewhat less conclusive: due to morphological differences, it does not show as many clear examples of the assimilation pattern.\footnote{The future and conditional are formed without the \textit{-f} element in Scottish Gaelic\il{Scottish Gaelic (Modern)}, and the passive participle is not generally productive. The devoicing pattern can be observed in head\hyp final compounds, but it is not obvious that this is a productive rule, either \parencite[cf.][]{stifter2015from}.} It maintains the pattern of mutation, but without clear language\hyp internal evidence for \featr{spread glottis} activity in the fricatives, the argument is much weaker. Nevertheless, there is also no clear evidence against an \enquote{aspirating} analysis, and the phonetic data clearly points in the \enquote{aspirating} direction just like in Irish\il{Irish (Modern)}.

With regard to the \cat{lenis} stops, one is slightly hampered by the lack of detailed instrumental studies, especially of Irish\il{Irish (Modern)}; however, see \citet{nichasaide1986preaspiration} on Ulster (Gaoth Dobhair\il{Gaoth Dobhair Irish}). The consensus is that Irish\il{Irish (Modern)} \cat{lenis} stops are well\hyp behaved for an \enquote{aspirating} language, with variable voicing especially after sonorants \parencite[cf. again][]{óraghallaigh2014fuaimeanna}. Traditionally, the presence of voicing in the \cat{lenis} series is considered an isogloss separating Irish\il{Irish (Modern)} from Scottish Gaelic\il{Scottish Gaelic (Modern)}, which is generally described as having voiceless unaspirated (\enquote{short\hyp lag} VOT) \cat{lenis} stops \parencites[146 sqq.;]{orahilly}{omurchu85}{robinson2007}. This is largely confirmed by instrumental studies such as \textcite{ladefogedetal-scg}, although some voicing is possible even here \parencite{nance2013pre}; \textcite{nichasaide1986preaspiration} also reports (albeit with small sample sizes) some dialectal variation across different parts of the Western Isles.

The literature on \isi{Laryngeal Realism} does allow for \enquote{aspirating} languages to show either variably voiced or fully voiceless \cat{lenis} stops. Irish\il{Irish (Modern)} patterns with languages like English\il{English (Modern)} and \ili{German}, whilst Scottish Gaelic\il{Scottish Gaelic (Modern)} is similar to languages like \ili{Danish} \glottocode{dani1285} and \ili{Icelandic}, which are all considered to be \enquote{aspirating} by \textcite{beckmanng:_empir}. Both Irish\il{Irish (Modern)} and Scottish Gaelic\il{Scottish Gaelic (Modern)} also demonstrate a pattern common in Germanic \enquote{aspirating} languages whereby the contrast between the two series is neutralized after a tautosyllabic fricative (most notably in the clusters \textit{sp~st~sk}), with a voiceless unaspirated stop as the outcome. The usual interpretation of such structures \parencite[e.\,g.][]{salmons2017germanic} is that the stop and the fricative share a \featr{spread glottis} specification. Thus, from the perspective of \isi{Laryngeal Realism} both Irish\il{Irish (Modern)} and Scottish Gaelic\il{Scottish Gaelic (Modern)} present as \enquote{aspirating} languages.

\subsection{The Brythonic languages}
\label{sec:welsh-cornish}

The phonetics and phonology of Welsh\il{Welsh (Modern)} are likewise straightforwardly compatible with an \enquote{aspirating} analysis. The \cat{fortis} stops are postaspirated before a stressed vowel, and at least variably preaspirated\is{preaspiration} after a vowel (\cite{ball01:_welsh_phonet,jones,morris2017linguistic,iosad2020atr}). The \cat{lenis} stops are generally described as variably voiced, certainly in South Welsh\il{Welsh (Modern)} \parencite{jones}; there is a perhaps somewhat surprising dearth of instrumental investigations of North Welsh\il{Welsh (Modern)}, but \textcite{bell2021northern} report a Scottish Gaelic\il{Scottish Gaelic (Modern)}\hyp like system of categorically voiceless \cat{lenis} stops. As in the Gaelic languages, the contrast is neutralized to a voiceless unaspirated stop after a tautosyllabic \cat{fortis} fricative.

Phonologically, Welsh\il{Welsh (Modern)} is a very clear example of a language showing activity of \featr{spread glottis} in the \cat{fortis} series and inertness of the \cat{lenis} stops: for example, as in Irish\il{Irish (Modern)}, a suffix\hyp initial \ipa{/h/} coalesces with \cat{lenis} stops, as in [cy] \textit{gwag} `{empty}', \textit{gwacáu} `{to empty}'. This evidence is considered in detail by \textcite{iosad2020atr}, to which the reader is referred for further discussion. Thus, Welsh\il{Welsh (Modern)} is quite unproblematic as an \enquote{aspirating} language.

The phonetic detail of Cornish is not available, but the morphophonological patterns it shows are quite parallel to those seen in Welsh\il{Welsh (Modern)}, notably in devoicing (\enquote{provection}) patterns that are (residually) attested in the comparative forms of adjectives \parentext{[mco] \textit{teg} `{fair}', \textit{tecka} `{fair.\Cmp}', cf. \textit{gwan-ha} `{weak-\Cmp}'} and in the subjunctive \parentext{\textit{\textins{d}eppr-o} `{eat.\Sbjv-\Tsg}' to \textit{dybry} `{eat}'}; cf. \textcite{williams2011middle}. 

\subsection{Voicing in Breton and Scottish Gaelic}
\label{sec:voic-bret-scott}

So far, there has been no discussion of any Celtic variety that demonstrates categorically voiced \cat{lenis} stops. There are two groups of present\hyp day languages where such representation is found.

The first of these is Breton\il{Breton (Modern)}. As discussed in \textcite{iosad2022breton}, phonetically it is much more reminiscent of a \enquote{voicing} language similar to its close neighbour French\il{French (Modern)}, with fully voiced \cat{lenis} stops and a voiceless unaspirated \cat{fortis} series. However, its morphophonology is more like that of an \enquote{aspirating} language. In particular, just like Irish\il{Irish (Modern)}, Welsh\il{Welsh (Modern)}, and Cornish, Breton\il{Breton (Modern)} shows a pattern of \cat{lenis} stops coalescing with suffix\hyp initial \ipa{/h/} to their \cat{fortis} counterparts: [br]\textit{gleb} `{wet}',\footnote{The form \textit{gleb} is realised in isolation with a voiceless stop due to final devoicing, but cf. \textit{glebiañ} `{to wet}'.} \textit{glep-oc'h} `{wet-\Cmp}' patterns just like W \textit{gwlyb}, \textit{gwlypach}. Even patterns superficially resembling those common in \enquote{voicing} languages, such as final devoicing, turn out on closer inspection to involve greater markedness of the \cat{fortis} series, meaning that the phonology of Breton\il{Breton (Modern)} is more like that of an \enquote{aspirating} language just like its closest relatives. See \textcite{iosad16} and \textcite{iosad2022breton}.

Consistent closure voicing is also found in some Scottish Gaelic\il{Scottish Gaelic (Modern)} varieties. First, \textcite{henderson1903gaelic} reports that all \cat{lenis} stops are voiced in \enquote{peripheral} dialects, such as Sutherland \il{Sutherland Gaelic} and South Argyll\il{South Argyll Gaelic}. \Textcite{allen2022plosive} examined the published dialect survey \parencite{sgds} and found consistent voicing in the \cat{lenis} series in parts of eastern and south-eastern Sutherland (points 141--147), and variable presence of voicing in north and west Sutherland and Caithness (points 128--140).

Second, closure voicing in stops is associated with the process of post\hyp nasal sandhi (\enquote{eclipsis}) across many dialects of Scottish Gaelic\il{Scottish Gaelic (Modern)} (\cite{maolalaigh1996gaelic,bosch09:_fine_scott_gaelic}). Both \cat{fortis} and \cat{lenis} stops participate in various progressive assimilations following nasals, either word\hyp internally or (a common source of alternations) after nasal\hyp final clitics. There is significant variability in the precise outcomes of these, as \textcite{bosch09:_fine_scott_gaelic} document, but closure voicing of the stop is one possible result.

The two stop series tend to remain distinct (for instance, as breathy voiced vs. plain voiced stops) in these contexts: Gairloch\il{Gairloch Gaelic} \parencite{wentworth} \textit{an doras} `{the door}' [dɔrəs] vs. \textit{an toll} `{the hole}' [d̤o͡ul̪ˠ] (unmutated [tɔrəs], [tʰo͡ul̪ˠ]).\footnote{There is in many, probably most, dialects a strong tendency to show different developments in \enquote{close sandhi} (after proclitics and in close compounds like \textit{lon-dubh} `{blackbird}') and word-internally (as in \textit{cinnteach} `{certain}'): the series tend to remain distinct in the former contexts but often show a tendency towards merger in the latter, with most stops becoming \cat{lenis} and realised with voicing. The matter requires much further scrutiny.} Subsequent loss of the nasal may lead to outcomes that can be interpreted in terms of a third distinct phonemic series characterized by a \featr{voice} specification; this is proposed, for example, for Applecross\il{Applecross Gaelic} by \textcite{ternes06:_scott_gaelic_applec_ross} and Easter Ross\il{Easter Ross Gaelic} by \textcite{watson2022easter}.

Because of this, we can only rarely analyse postnasal voicing in terms of a \enquote{voicing} system: postnasal processes do not generally result in a system contrasting a voiceless (let alone unaspirated) \cat{fortis} series /p(ʰ)~t(ʰ)~k(ʰ)/ and a consistently voiced \cat{lenis} /b~d~ɡ/. The only exception here, as \textcite{bosch2009fine} note, are peripheral dialects of the Sutherland\il{Sutherland Gaelic} type, but within the broader Celtic context this phenomenon is quite marginal (in more than one sense).

To summarize this section, most present\hyp day Celtic languages are well\hyp behaved representatives of the \enquote{aspirating} type under \isi{Laryngeal Realism}. There is ample phonetic and phonological evidence in favour of associating the \cat{fortis} stops with a featural specification such as \featr{spread glottis}, and very little to justify assigning a feature such as \featr{voice} to \cat{lenis} stops. Even where consistent voicing does occur in the \cat{lenis} series, it does not clearly justify a reclassification of the variety in question as representing the \enquote{voicing} type. In the next section, the prevailing view of laryngeal contrast in an earlier era is considered.

\section{The aspirating view of Proto-Celtic}
\label{sec:aspir-view-proto}

Unsurprisingly given the situation discussed in \cref{sec:laryng-phon-pres}, Proto\hyp Celtic is usually reconstructed with two stop series: a \cat{fortis} category corresponding to \enquote{traditional} Proto\hyp Indo\hyp European \enquote{voiceless} \textit{*p~t~k~kʷ}, with a very early (but not Proto\hyp Celtic) change of \textit{*p} to zero, likely via a labial fricative and then \textit{*h} in most contexts. The \cat{lenis} series \textit{*b~d~ɡ~ɡʷ} represents traditional Indo\hyp European \enquote{voiced} and \enquote{voiced aspirate} stops. These stops could be either short or long (geminate). While \cat{fortis} geminates were not at all unknown, \cat{lenis} geminate stops were rather more marginal, occurring primarily as secondary developments of other clusters or across a morpheme boundary; for the details of gemination in Celtic, see \textcite{stifter2023celtic}.

A very simple interpretation is to treat the \cat{fortis} series as (plain) \enquote{voiceless} and the \cat{lenis} series as \enquote{voiced}. However, direct evidence in favour of this account is meagre. Much of the Continental Celtic inscriptional material is not probative at all: it is preserved using writing systems such as the Iberian and Lugano alphabets that generally do not distinguish laryngeal contrast in stops. Celtic material in the Greek and Latin scripts generally uses orthographic \orth{p~t~c} \emph{resp.} \orth{π~τ~κ} for the \cat{fortis} stops and \orth{b~d~g} \emph{resp.} \orth{β~δ~γ} for their \cat{lenis} counterparts. However, there is some potentially useful graphical hesitation. The \ili{Gaulish} material in the Latin alphabet presents non\hyp trivial amounts of vacillation: especially frequent is \orth{c} for expected \ipa{[ɡ]}, and we also find some confusion between \orth{t} and \orth{d}. Furthermore, a few Greek\hyp script inscriptions witness \orth{θ} and \orth{χ} for the \cat{fortis} stops. Following \citet{gray1944mutation} and \citet{Watkins1955}, this vacillation is often taken to indicate that singleton stops were pronounced differently in \ili{Gaulish} and in Latin. A common view is that \ili{Gaulish} \ipa{/b~d~ɡ/} possessed some \enquote{fortis} property that their Latin counterparts lacked, precluding orthographic \orth{b~d~g} from representing these \cat{lenis} stops: one possibility is that they would not be consistently voiced, being therefore liable to be identified with Latin or Greek voiceless stops.

Given the \enquote{aspirating} systems of the present\hyp day languages, it is also possible to simply interpret Proto\hyp Celtic as long\hyp lag VOT (\enquote{aspirated}) stops in the \cat{fortis} series. Already, \citet{pedersen1909vergleichende} and \citet{cccg} projected the present\hyp day aspiration into Proto\hyp Celtic. In a series of recent publications, \citeauthor{eska2017} (\citeyear{eska2017}; \citeyear{Eska2018}; \citeyear{eska2019laryngeal}; \citeyear{eska2020interarticulatory}) forcefully restated this position within the theoretical framework of \isi{Laryngeal Realism}, arguing that Proto\hyp Celtic is best analysed as an \enquote{aspirating} language, and thus that the present\hyp day systems continue the original pattern unchanged. He adduces evidence of two kinds for this proposition.

In terms of direct evidence, his proposal is supported by the interpretation of the \ili{Gaulish} material discussed earlier. In addition, \textcite{eska2017} discusses some Celtiberian inscriptions that make a distinction between two stop series,\footnote{The inscriptions in question are analysed by \textcite{jordán2007estudios}; for a recent overview of Palaeohispanic writing systems, see \textcite{ferrer2019palaeohispanic}.} giving special attention to examples where sequences of a (historically voiceless) fricative and a stop are spelled with the \cat{lenis} stop grapheme \orth{sd}. \Textcite{eska2017} suggests that this reflects the lack of aspiration after a fricative, much as in present\hyp day Welsh\il{Welsh (Modern)}, Irish\il{Irish (Modern)}, and Scottish Gaelic\il{Scottish Gaelic (Modern)} (where similar spellings of such clusters are, or have been, attested in the written tradition).

A second source of evidence is the typology of sound changes affecting Proto\hyp Celtic stops in the descendant languages. For instance, \textcite{Eska2018} considers the \enquote{loss} of Proto\hyp Indo\hyp European \textit{*p} in Celtic: as already noted, this process is normally considered to have proceeded via multiple stages involving voiceless fricatives (\textit{*p}~> \textit{*ɸ}~>\textit{*h}~> $\emptyset$ \emph{vel sim.}). \Textcite{Eska2018} argues that this is compatible with the presence of aspiration on the stop, with reference to \textcite[57]{kümmel2007konsonantenwandel}. Further arguments of this kind are given in \textcite{Eska2018} with respect to the Common Celtic period and by \citeauthor{eska2019laryngeal} (\citeyear{eska2019laryngeal}; \citeyear{eska2020interarticulatory}) for the medieval and modern attestations. It is to a re\hyp examination of this typological evidence that I now turn.

\section{Examining the sound changes}
\label{sec:exam-sound-chang}

In this section, I examine the typology of sound changes affecting Proto\hyp Celtic stops in the attested languages. After a brief overview of the overall pattern, I will consider the relevant changes by dividing them into four groups: changes that are more likely if Common Celtic belonged to the \enquote{voicing} type; changes more compatible with an \enquote{aspirating} system; changes where a reasonable, albeit non\hyp trivial case can be made for either possibility; and finally non\hyp probative changes that are easily compatible with either a \enquote{voicing} or an \enquote{aspirating} pattern in Common Celtic.

\subsection{Lenition of stops in Gaelic and Brythonic}
\label{sec:lenit-stops-gael}

Processes of \isi{lenition} affecting both \cat{fortis} and \cat{lenis} stops are a hallmark of Celtic historical phonology. They have been the subject of extensive scholarship over the years, some of which is of foundational importance even outwith the specific concerns of Celtic studies \parencite[such as][]{mart55}. In this chapter, however, I am mostly concerned with what these changes reveal about the nature of laryngeal contrast prior to their operation.

The fundamental facts are fairly simple. In intervocalic (or, rather, postvocalic) position, both series of stops in the Insular Celtic languages are reflected with outcomes that are advanced along well\hyp understood \isi{lenition} trajectories relative to their original state. The developments are exemplified in \cref{tab:stops-celtic}. The consonants of interest are bolded, and the transcription of the relevant consonant given when not trivial. As it shows, \cat{lenis} stops changed to \cat{lenis} fricatives \textit{*β~ð~ɣ} in both Brythonic and Gaelic languages. In the case of \cat{fortis} stops, there is a difference: in Gaelic, \isi{lenition} produced \cat{fortis} fricatives, whilst in Brythonic it resulted in \cat{lenis} stops. As for geminates, in the present\hyp day languages \cat{lenis} geminates \textit{*bb~dd~ɡɡ} are (mostly) reflected as singleton \cat{lenis} stops, whilst \cat{fortis} geminates again show a split: Brythonic reflects them as \cat{fortis} fricatives \textit{*f~θ~x}, whereas in Gaelic they are reflected as \cat{fortis} stops \textit{*p~t~k}.

\begin{table}
  \centering
  \begin{tabularx}{\textwidth}{lXllll}
    \lsptoprule
    && \multicolumn{2}{c}{Singletons} & \multicolumn{2}{c}{Geminates} \\
    \cmidrule(r){3-4}\cmidrule(l){5-6}
    \multicolumn{2}{l}{Language} & \cat{fortis} & \cat{lenis} & \cat{fortis} & \cat{lenis} \\
    \midrule
    \multicolumn{2}{l}{Proto-Celtic} & \textit{*katu-} & \textit{*budaro-} & \textit{*(ɸ)likkā-} & \textit{*kred-dī-} \\
                                     &            & `battle' & `deaf' & `flat stone' & `believe' \\
    \midrule
    Gaelic                           & \ili{Old Irish}    & \textit{ca\hili{th}} [θ] & \textit{bo\hili{d}ar} [ð] & \textit{le\hili{cc}} [k] & \textit{crei\hili{t}id} [dʲ] \\
                                     & Modern Irish\il{Irish (Modern)} & \textit{ca\hili{th}}     & \textit{bo\hili{dh}ar}    & \textit{lea\hili{c}}     & \textit{crei\hili{d}im} \\
    \midrule                         
    Brythonic                        & Welsh\il{Welsh (Modern)} & \textit{ca\hili{d}} & \textit{by\hili{dd}ar} [ð] & \textit{lle\hili{ch}} [χ] & \textit{cre\hili{d}u} \\
                                     & Breton\il{Breton (Modern)} & \textit{ka\hili{d}} & \textit{bou\hili{z}ar} & \textit{le\hili{c'h}} & \textit{kre\hili{d}iñ} \\
    \lspbottomrule
  \end{tabularx}
  \caption{Postvocalic stops in Celtic}
  \label{tab:stops-celtic}
\end{table}

Most of these processes are dated to some point in the sub\hyp Roman period of the mid 1st millennium CE (\cite{lheb,sims-williams1990dating,mccone}): they are generally not reflected (even incidentally) in the Continental material\footnote{But see \cref{sec:lenition} on the \cat{lenis} stops and \textcite{stifter2012lenition} on the possibly related \isi{lenition} of \textit{*s} to \textit{*h} in \ili{Gaulish}.} or in the very earliest Romano\hyp British elements borrowed into English\il{English (Modern)}. However, they do affect Latin borrowings in both Irish\il{Irish (Modern)} and Brythonic, as well as Brythonic and British Latin borrowings in Irish\il{Irish (Modern)} \parencite{mcmanus1983latin}, and were certainly complete by the time of the manuscript attestations. 

\subsection{Evidence for a “voicing” system}
\label{sec:evid-voic-syst}

This category includes three unrelated changes: early \enquote{devoicing} of geminates, the Brythonic \isi{lenition} of \cat{fortis} stops to \cat{lenis} stops characteristic of Brythonic, and the development of \cat{fortis} stops after nasals in Gaelic.

\subsubsection{Geminate devoicing}
\label{sec:geminate-devoicing}

The development of Common Celtic \cat{lenis} geminate stops \textit{*bb~dd~ɡɡ} presents difficult problems. They mostly arose from consonant clusters, usually at morpheme boundaries \parencite{stifter2023celtic}: Welsh\il{Welsh (Modern)} (W) \textit{credu} `believe', Modern Irish\il{Irish (Modern)} \textsc{(Ir)} \textit{creidim} from Proto-Celtic \textsc{(PCelt)} \textit{*kred-dī-} (cf. Sanskrit \textit{śraddhā́}); W \textit{aber} `estuary' from \textit{*ad-bero-}, \ili{Old Irish} (\textsc{OIr}) \textit{acaldam} `colloquy' (present\hyp day \textit{agallamh}) to the verb \textit{ad:gládathar} 'address.\Prs.\Tsg'. In general, they are reflected in the present\hyp day languages as short \cat{lenis} stops.

That said, there are least some examples of \cat{lenis} geminates, or \cat{lenis} consonant clusters, that have \cat{fortis} outcomes in Brythonic (\cite[159--161]{pedersen1909vergleichende}; \cite[145--146]{jørgensen2022celtic,stifter2023celtic}). Significantly, such examples are mostly root\hyp internal: W \textit{bychan}, Breton\il{Breton (Modern)} (\textsc{Br}) \textit{bihan} `small' (\textit{*bikko-}) compared to \textsc{OIr} \textit{bec}, \textsc{Ir} \textit{beag} (\textit{*biggo-}); \textsc{Br} \textit{bouc'h} `blunt' (\textit{*bukko-}) but \textsc{OIr} \textit{boc} `soft, tender', (\textsc{Ir, ScG}) \textit{bog} (\textit{*buggo-}), and a few others. Also likely related is the reflex of the \textit{*sd} (>~\textit{*zd}?) cluster as {θ} in cases like W \textit{nyth} `nest', (\ili{Middle Breton} (\textsc{MBret}) \textit{nez}, \textit{nyth}, from \textit{*nisdo-} (in contrast to \textsc{OIr}  \textit{net}, \textsc{Ir} \textit{nead}). The simplest explanation of this development is devoicing of a voiced geminate in Brythonic: as \textcite{stifter2023celtic} points out, it is strongly supported by the \textit{*-tt-} appearing in Gallo-Romance borrowings reflecting Celtic \textit{*-sd-}, such as \textit{pettia} `part, piece' (French\il{French (Modern)} \textit{pièce}) to (\textsc{PCelt}) \textit{*kʷesdi-} (\textit{OIr} \textit{cuit} `share, part', \textsc{Ir, ScG} \textit{cuid} but \textsc{W} \textit{peth} `thing', \textsc{Br} \textit{pezh}). The same outcome is visible in Scottish placenames reflecting Pictish \textit{*pett-} as \textit{Pet-}, \textit{Pit-}.

This development is not entirely regular, and not all relevant consonants and clusters undergo it. In many morpheme\hyp boundary cases, for instance after the prefix \textit{*ek(s)-}, such as W \textit{eglan} `seashore' (\textit{*ek-glannā-}), the context is highly vulnerable to cyclic\fshyp analogical effects (cf. unprefixed W \textit{glan} `shore'), and the data is notably difficult (\cite{russell1988celtic,stifter2023celtic}). I am persuaded, however, by \posscitet{jørgensen2022celtic} argument that devoicing in items like \textit{*biggo-} was early, and the irregularities arise because it did not operate after the stabilization of \cat{lenis} geminates across historical morpheme boundaries, as in \textit{*ad-bero-} `estuary' and \textit{*kred-dī-} `believe'.

Typologically, devoicing of voiced geminate obstruents and clusters is strongly motivated phonetically \parencite{ohala1997aerodynamics} and well known typologically: for example, phonetic devoicing of such geminates occurs in \ili{Japanese} \parencite{hirose07}; and a diachronic\fshyp morphophonological pattern of this kind is found in \ili{Chaha} \parencite[ch.~2]{petros2000sound}. Under the aspiration\hyp based account, the change from \textit{*bb~dd~ɡɡ} to \textit{*pp~tt~kk} must be reconstructed as *\ipa{[tt]} > *\ipa{[tʰ]} (or even \ipa{[ttʰ]}?), with subsequent spirantization. This, I suggest, is unlikely. Degemination of a plain stop to an aspirated one is not an impossible change: indeed, it is commonly cited as a possible source of \isi{preaspiration} \parencite{blevins1993ponapeic}. The rise of (post)aspirated geminates from plain ones is also attested, for example in Cypriot Greek (\cite{newton1972cypriot,armostis2009cypriot}). However, the ultimate outcome of this change in the Brythonic languages is a fricative: neither preaspirated\is{preaspiration} stops nor geminates of any kind are a plausible immediate source for a fricative.\footnote{In principle, one could imagine preaspirated\is{preaspiration} stops undergoing \isi{preaspiration} oralization \parencite[e.\,g.][]{silverman03} with subsequent loss of a stop (\textit{*tt}~> \textit{*ʰt}~> \textit{*θt}~> \textit{θ}), but this very long sequence of events stretches credulity.} I conclude that devoicing of voiced obstruent geminates\fshyp clusters is a far more preferable interpretation. (On the spirantization seen in Brythonic, see \cref{sec:bryth-second-lenit}.)

\subsubsection{Lenition of |fortis| stops in Brythonic}
\label{sec:lenit-fort-stops}

As we saw in \cref{sec:lenit-stops-gael}, Gaelic and Brythonic differ in how \isi{lenition} in (roughly) postvocalic position affects singleton \cat{fortis} stops. Here, we focus on their reflexes in Brythonic, returning to the Gaelic pattern in \cref{sec:spir-fort-stops}. The outcomes of original \cat{fortis} stops in roughly postvocalic position are \cat{lenis} stops:  W\textit{cad} `battle' < \textit{*katu-}. 

This Brythonic \enquote{first \isi{lenition}} is entirely unproblematic for a \enquote{voicing} language: intervocalic stop voicing is both phonetically and typologically trivial. It is more difficult to analyse it as involving loss of aspiration, as would be required under the alternative analysis. \Textcite{eska2020interarticulatory} proposes that this change resulted from earlier timing of the glottal spreading gesture in postvocalic position, leading to shorter VOT and reinterpretation of the relevant stops as \cat{lenis}. While not impossible, I suggest this explanation is problematic.

Voice onset time~-- the duration of postaspiration~-- is clearly an important cue to laryngeal contrast in word\hyp initial position, and the change indeed applies to word\hyp initial consonants in postvocalic sandhi (\textit{*esjō teɣos} `his house' > Welsh\il{Welsh (Modern)} \textit{ei dŷ}). However, it is less clear that VOT is particularly important word\hyp medially. To be sure, there are reports in the literature of languages with consistently postaspirated \cat{fortis} stops word\hyp medially: examples include \ili{German} \parencite{jessen1998phonetics}, Persian \parencite{bijankhan2009voice}, Northern Icelandic\il{Icelandic} \parencite{helgason}, and Georgian \parencite{Vicenik2010}. However, it is not clear what would motivate such a leftward shift in glottal opening after a vowel. Typologically, there are examples of across\hyp the\hyp board timing changes~-- consonant shifts~-- but it seems that the more usual case is an increase in VOT rather than a shortening. Grimm's Law in Germanic and the Armenian consonant shift \parencite[e.\,g.][]{sayeed2017armenian} are celebrated instances in Indo\hyp European, but we can find parallels elsewhere: for instance, a similar shift has occurred (apparently independently) in at least two branches of Austroasiatic: Angkuic and Mal-Pray \parencite{sidwell2021northern}. However, it is more difficult to understand such a change as a positionally conditioned pattern with relatively earlier glottal opening specifically in leniting positions.

According to \textcite[49]{kümmel2007konsonantenwandel}, intervocalic \isi{lenition} of \cat{fortis} stops that is realised as loss of aspiration rather as voicing is something of a northern European specialty: his examples are primarily Germanic varieties such as \ili{Icelandic}, \ili{Danish}, and Western Norwegian. I cannot discuss these parallels in detail here, and refer the reader to \textcite{iosad2020phonological}, where I suggest that this kind of \isi{lenition} involves not a shortening of postaspiration, but reduction of \isi{preaspiration} in \cat{fortis} stops. It is conceivable, but difficult to prove, that a similar scenario applied in Celtic, too. This would give some support to its analysis as an \enquote{aspirating} language, although not in the exact way envisaged by \textcite{eska2020interarticulatory}.

There is a second possibility. A distribution whereby original \cat{fortis} stops yield aspirated reflexes in \enquote{strong} positions, but unaspirated ones (ultimately merging with original \cat{lenis} ones) in \enquote{weak} positions could result not from \isi{lenition} targeting weak positions but from fortition in strong contexts. For example, \textcite{hill2007aspirated} argues that \ili{Tibetan} underwent aspiration of stops in initial position, which yields synchronic distribution analysable as intervocalic deaspiration in Old Tibetan\il{Old Tibetan} \parencite{hill2010old} or Amdo Tibetan\il{Amdo Tibetan} \parencite{green2012amdo}. This possibility cannot be easily rejected for Brythonic. Indeed, it is also consistent with the fact that stops escape (first) \isi{lenition} not only word-initially but also after a sonorant (W \textit {gwy\hili{nt}} `wind', \textit{se\hili{rch}} `love' with subsequent spirantization): it is widely recognized that the \enquote{coda mirror} position \{\#\_, C\_\} often counts as \enquote{strong} \parencite{segeral08:_posit}. However, even if feasible, this scenario presupposes an \emph{absence} of aspiration in \cat{fortis} stops in strong position prior to this change, and thus is not compatible with a view of earlier Celtic as an aspirating language.\footnote{A similar scenario for the origin of the \enquote{aspirating} systems in the present\hyp day Germanic languages is envisaged by \textcite{goblirsch2005lautverschiebungen}, building on \textcite{steblin}.}

To summarize, of the three possible scenarios that have been identified, two imply that Brythonic was a \enquote{voicing} language at least prior to the change, and of these one is phonetically trivial; only one is fully compatible with an aspirating system, but relies on rather tentative parallels. On balance, therefore, I suggest that the Brythonic first \isi{lenition} is much more compatible with a \enquote{voicing} system of laryngeal contrast.

\subsubsection{Nasal-stop clusters in Gaelic}
\label{sec:nasal-stop-clusters}

A third development that strongly supports reconstructing an earlier \enquote{voiced} pattern is exclusive to Gaelic, where clusters of a (lenis) nasal followed by a \cat{fortis} stop yield \cat{lenis} stops: \textit{*kantom} `hundred', (\textsc{OIr} \textit{cét}, \textsc{Ir} \textit{céad}, \textsc{ScG} \textit{ceud}, M \textit{keead} (but W \textit{cant} `hundred'). This change has been interpreted in multiple ways, but I suggest that all of them are more compatible with a voicing contrast.

Traditional accounts (\cite{thurneysen,martinet1952celtic}; see also \cite{stifter2023celtic}) consider the development to have gone through a \cat{lenis} geminate stop: \textit{*kanto-} > \textit{*kɛ̄dd}. This interpretation disaggregates the change into two components, gemination and voicing. I will analyse them separately.

There are two main approaches to the voicing of the stop in the literature. One, endorsed by \textcite{mccone} under the rubric of \enquote{second \isi{lenition}} \parencite[see also][]{kortlandt1982phonemicization}, considers the voicing to result from postvocalic \isi{lenition} that applied to newly postvocalic \cat{fortis} stops after the loss of a preceding consonant: \textit{*kant-} > \textit{*kɛ̄t-} > \textit{*kēd-}. This is essentially the same process as the Brythonic \enquote{first \isi{lenition}} (\cref{sec:lenit-fort-stops}), and the same considerations apply to it: it is easy to account for under a \enquote{voicing} view, but more difficult, albeit not impossible, under an \enquote{aspiration} account.

A second view, also discussed by \textcite[91]{mccone} building on \textcite{greene1960some}, is that the voicing was triggered by the nasal.\footnote{Such clusters do not merge with original \textit{*nd}-type sequences, because in the latter the nasal became \cat{fortis} and was preserved: \textit{*anɡu-} > \textit{*ɛNɡu-} > \textit{ing} `straight', in contrast to \textit{*anku-} > \textit{*ɛnɡu-} > \textit{éc} `death'.} This proposition is much more difficult to reconcile with an original \enquote{aspirating} system. Nasals can, of course, exert a voicing influence on following stops, but this does not seem to commonly result in a neutralizing loss of aspiration. In \cref{sec:voic-bret-scott}, I discussed the postnasal voicing processes of Scottish Gaelic\il{Scottish Gaelic (Modern)}: generally, the outcomes of nasal + \cat{fortis} stop sequences are distinct from underlying \cat{lenis} stops, and when they do result in a merger, it is precisely where the \cat{fortis} stops show little if any aspiration.

A close parallel is provided by (Amur) \ili{Nivkh} \glottocode{gily1242}. It contrasts aspirated \cat{fortis} stops \ipa{[pʰ~tʰ~cʰ~kʰ~qʰ]} with voiceless unaspirated \cat{lenis} stops \ipa{[p~t~c~k~q]}, and possesses a postnasal voicing process quite similar to the Scottish Gaelic\il{Scottish Gaelic (Modern)} \enquote{eclipsis}. Crucially for our purposes, it affects \cat{lenis} (unaspirated) stops, causing them to become voiced (\textit{təf-\hili{k}u} `house-\Pl' but \textit{qan-\hili{ɡ}u} `dog-\Pl', but fails to affect \cat{fortis} (aspirated) stops (\textit{alr̥-\hili{pʰ}e} `pick berries' but \textit{tʰoʁzaŋ-\hili{pʰ}e} `pick nuts').\footnote{Examples from \textcite[502]{gruzdeva2024amuric}.} More generally, \textcite[54]{kümmel2007konsonantenwandel} finds that postnasal voicing of aspirated stops usually produces aspirated or breathy\hyp voiced outcomes. If anything, postnasal position can be associated with plain stops becoming aspirated, as attested in numerous Bantu languages \parencite{downing2018nch}.\footnote{\Textcite{hyman-limits} notes a synchronic postnasal deaspiration process in the \ili{Nguni languages} \glottocode{ngun1276}, but diachronically it represents the blocking of aspiration of plain voiceless stops after nasals. If anything, this pattern \emph{supports} the proposition that \cat{fortis} stops were originally plain.} I conclude that a \textit{*nt} > \textit{*(n)d} scenario is much more strongly compatible with a view where the contrast is of the \enquote{voicing} type.

The second potential component of the change, gemination, is perhaps less relevant for our concerns (see \cref{sec:degemination} below). The literature widely accepts it to have occurred in these clusters, even though the evidence is rather meagre. If gemination did occur, it could be reconstructed as assimilation of \textit{*nd} clusters produced by postnasal voicing (this finds a good parallel, for instance, in a number of Semitic languages), or a direct \textit{*nt}~> \textit{*dd} change, with a parallel in the Western Sámi languages\il{Sámi languages} (\cite{sammallahti1998saami}; see further \cite[180--181]{kümmel2007konsonantenwandel}). As far as can be seen, the outcomes of this process merge with original degeminated \cat{lenis} stops, as in \textsc{OIr} \textit{bec} `small' (\textit{*biggo-}), but the dating of the degemination is unclear: as long as it follows the (likely very early) spirantization of original \textit{*b~d~ɡ} (see \cref{sec:lenition} below), it could be dated to almost any time in the history of the Gaelic languages. Whether or not the original nasal-stop clusters went through a geminate stage does not influence my conclusions as to the nature of laryngeal contrast.

%\begin{flushleft}\dko{There is something wrong with the sentence on the previous page that starts with 'As far as we can see', but I'm not sure what he meant to say, so I can't fix it.}\end{flushleft}

\subsection{Evidence for an “aspirating” system}
\label{sec:evid-an-enqu}

Two sound changes provide strong support for an \enquote{aspirating} system of laryngeal contrast. They are both late in the relative chronology, because they specifically affect \enquote{secondary} clusters that arose via a process of unstressed vowel syncope. This syncope can be dated to the early to mid 6th century CE (\cite{lheb,sims-williams1990dating}).

\subsubsection{Voicelessness assimilation}
\label{sec:post-syncope-changes}

In clusters created by syncope where the second member was a \cat{fortis} obstruent (not least \ipa{[h]}), one can observe regressive laryngeal assimilation. These processes occur in both Brythonic and Gaelic. For Brythonic examples, see W \textit{ateb} `answer' from \textit{*ad-heb-} < \textit{*ati-sekʷ-}, \textit{drycin} `storm' < \textit{*drug-hin-} < \textit{*druko-sīn-}. In Gaelic, there are examples like \textsc{OIr} \textit{de\hili{phth}igim} 'fight.\Prs.\Fsg', \textsc{Ir} \textit{dea\hili{f}ach} `contentious' (\textsc{OIr} \textit{de\hili{bth}ach}) by contrast with \textit{de\hili{b}uith} `discord', \textsc{Ir} \textit{dea\hili{bh}aidh} \parencite[128]{mccone}. For the behaviour of \textit{h} in Gaelic, cf. \textit{im\hili{b:s}oí} `turn.\Prs.\Tsg' but prototonic \textit{-im\hili{p}ai} from \textit{*ambi-sowet(i)}. As I discussed in \cref{sec:theor-prel}, this kind of pattern, which privileges \cat{fortis} phonological activity, is very strongly compatible with (if not diagnostic for) an aspirating system.

\subsubsection{Provection}
\label{sec:provection}

A second post\hyp syncope development, relevant only for Brythonic, also concerns secondary obstruent clusters: \cat{lenis} geminates arising from syncope yield \cat{fortis} stops in the present\hyp day languages, as in Welsh\il{Welsh (Modern)} \textit{llety} 'lodging' from \textit{*led-deɣ-}.\footnote{Cf. also the placenames Lichfield (Staffordshire) and Lytchett (Dorset) from Brythonic \textit{*lēd-ɡɛ̄d-} < \textit{*lēto-kaito-} `grey forest' \parencite[563]{lheb}.} Under an account like \posscitet{eska2020interarticulatory}, in which the \cat{lenis} stops were voiceless, the development, commonly known as \enquote{provection}, can be interpreted as a change from a voiceless geminate to a (preaspirated\is{preaspiration}) stop, which I already noted to be typologically well supported. Under an account based on voicing, the development needs to proceed via an intermediate stage of geminate devoicing: this is, as already noted, unproblematic phonetically, but makes the derivation more complex.  Overall, this process does not appear to be fully probative, but an aspirating analysis is more parsimonious in this case.

\subsection{Ambiguous changes: Spirantization of \texorpdfstring{\cat{fortis}}{FORTIS} stops}
\label{sec:ambiguous-changes}

One change that has a rather ambiguous status is the spirantization of \cat{fortis} \textit{*(p)~t~k} stops to the \cat{fortis} non\hyp strident fricatives \textit{*f~θ~x}. It occurs separately, with different conditioning, in the two branches.

\subsubsection{Goidelic first lenition}
\label{sec:first-lenit-goid}
\label{sec:spir-fort-stops}

As discussed in \cref{sec:lenit-stops-gael}, \cat{fortis} fricatives \textit{*θ~x~xʷ} are the usual outcome of original singleton \cat{fortis} stops in Goidelic in postvocalic position (including in postvocalic sandhi). The usual explanation is that the stops were originally aspirated and became fricatives likely via an affricate stage, a change with multiple typological parallels (not least the High \ili{German} Consonant Shift); see \textcite{koch90, isaac}. \textcite{Eska2018} further assimilates the spirantization of Proto\hyp Indo\hyp European \textit{*p} to \textit{*ɸ} in Early Celtic to this parallel, stretching the possible presence of aspiration even earlier.

In support of this proposition, \textcite{Eska2018} cites the findings of \textcite[57]{kümmel2007konsonantenwandel}. However, these results warrant further scrutiny: what \textcite{kümmel2007konsonantenwandel} finds is that aspiration is necessary for \cat{fortis} spirantization when it is an across\hyp the\hyp board consonant shift, but both changes of individual consonants (as with \textsc{PCelt} \textit{ɸ} < PIE \textit{*p}) and positionally restricted \isi{lenition} processes are attested as spirantizations of what are unaspirated \cat{fortis} stops. One particularly well\hyp known example of the latter is \textit{gorgia toscana}, the postvocalic \isi{lenition} of singleton \cat{fortis} stops found in Tuscan Italian\il{Italian} \glottocode{ital1282}. In a recent in\hyp depth discussion of this process, \textcite{marotta2008lenition} argues that it does not proceed via affrication of aspirated stops, although see \textcite{honeybone2023can} for some skepticism on this possibility. Another possible case is reported from Southern Jutland Danish\il{Danish} by \textcite{pi:puggaard-rode2024variation}.

My conclusion is that the spirantization of \cat{fortis} stops cannot clearly distinguish between the two approaches to laryngeal contrast. It is more easily compatible with an \enquote{aspirating} system, but a \enquote{voicing} pattern cannot be conclusively ruled out at this stage.

\subsubsection{Brythonic second lenition}
\label{sec:bryth-second-lenit}

In Brythonic, \cat{fortis} stops have fricative outcomes in two contexts. First, this is observed regularly with postvocalic \cat{fortis} geminates: W \textit{llech} `slate', \textsc{Br} \textit{lec'h}, Cornish \textit{lehan} < \textit{*(ɸ)likkā}. Second, the same outcome is found for \cat{fortis} singletons in two contexts: after a liquid (\textit{serch} < \textit{*sterkā} `love'; contrast \textsc{OIr} \textit{sercc}, \textsc{Ir} \textit{searc}) and in sandhi after the loss of certain word\hyp final consonants, especially \textit{*h} from earlier \textit{*s} (\textit{esjas teɣos} `her house' > \textit{*ehja θeɣoh} > W \textit{ei thŷ}; contrast \textsc{OIr} \textit{a (t)tech}, \textsc{Ir} \textit{a teach}).

A long line of work \parencite{lheb, jackson60, koch89, koch90, isaac, isaac08:_bryth} has taken this development to support the presence of aspiration on \cat{fortis} stops: the greater \enquote{force} of geminate stops is hypothesized to have facilitated aspiration, followed by affrication and subsequent spirantization (\textit{*ttʰ}~> \textit{*t(t)θ}~> \textit{θ}).

However, authors such as \textcite{thomasa, sims-williams1990dating, sims-williams08} and \textcite{mccone} have suggested this is unnecessary: instead, they analyse the Brythonic developments as a combination of \cat{fortis} degemination (non\hyp probative for our purposes, \cref{sec:degemination}) followed by spirantization of the type discussed in \cref{sec:spir-fort-stops} (so \textit{*likkā-}~> \textit{*likā-}~> \textit{llech}). I find the latter scenario more persuasive. In particular, the direct development from geminates to fricatives without degemination is typologically highly suspect \parencite[e.\,g.][]{kirchner}. This analysis also allows us to dispense with the otherwise highly odd gemination that the traditional scenario has to posit to account for spirantization of singletons in forms like W \textit{serch}, \textit{ei thŷ}. Instead, all of these developments can be subsumed under a \enquote{second \isi{lenition}}, where postvocalic singleton \cat{fortis} stops of secondary origin (i.\,e. produced by degemination or by loss of a preceding consonant), as well as post\hyp liquid stops, are reflected as \cat{fortis} fricatives, parallel to the \enquote{first \isi{lenition}} in Gaelic. If one accepts the degemination\hyp and\hyp \isi{lenition} scenario, the Brythonic development is now a combination of two non\hyp probative changes.

\subsection{Non-probative changes}
\label{sec:non-prob-chang}

Finally, I briefly consider two categories of changes that are not probative for the nature of the Early Celtic system, being easily compatible with either option.

\subsubsection{Spirantization of |lenis| stops}
\label{sec:lenition}

Original \cat{lenis} singleton stops yield \cat{lenis} fricatives, generally reconstructed as \textit{*β~ð~ɣ}, in both Gaelic and Brythonic. They are mostly reflected as fricatives, usually (but sometimes variably) voiced, in the present day \parencite{ball01:_welsh_phonet}. Intervocalically, they often undergo further \isi{lenition} processes such as debuccalization and ultimately deletion, especially in Irish\il{Irish (Modern)} and Breton\il{Breton (Modern)}, but the general pathway is widely agreed on.

Although traditionally this process is dated to the same mid 1st millennium CE stage as the other Insular lenitions\is{lenition}, more recent scholarship (e.g. \cite{sims-williams90:_datin_neo_britt}; \cite{villar93:_las}; \cite{mccone}; \cite{eska2017})  has been willing to date it to a much earlier period. The reason is that there are some traces of this process (such as loss of intervocalic \textit{*g} and fricative spellings of postvocalic \textit{*d}) in both \ili{Gaulish} and Celtiberian.

This change is not especially informative. Voiced fricatives are commonly produced by \isi{lenition} of voiced stops, but they can also arise from the spirantization of voiceless, contrastively unaspirated stops. \emph{Comparanda} for the \isi{lenition} of voiced stops to voiced fricatives are trivial to find: for instance, this process is pervasive in Romance languages, including Spanish \glottocode{stan1288}, \ili{Catalan}, and many languages of Italy. However, clearly \enquote{aspirating} languages can also show a pattern of \cat{lenis} voiceless unaspirated stops alternating with voiced continuants. I have already cited \ili{Nivkh}. The two series of stops, both aspirated /pʰ~tʰ~cʰ~kʰ~qʰ/ and voiceless unaspirated /p~t~c~k~q/, participate in a morphologically restricted process of morpheme\hyp initial consonant mutation. They alternate with voiceless and voiced continuants \ipa{[f~r̥~s~x~χ]} and \ipa{[v~r~z~ɣ~ʁ]} respectively, just like in Scottish Gaelic\il{Scottish Gaelic (Modern)}: \textit{təf-\hili{k}u} `house-\Pl' but \textit{mu-\hili{ɣ}u} `boat-\Pl'. I conclude that the \isi{lenition} of singleton \cat{lenis} stops to voiced fricatives, whether treated as part of the Insular \isi{lenition} processes or put further back in time, does not furnish probative arguments regarding the nature of laryngeal contrast at the time that it occurred.

\subsubsection{Degemination}
\label{sec:degemination}

The clearest case of degemination is found with \cat{fortis} geminates in the Gaelic languages, where they are reflected as \cat{fortis} stops (with, where appropriate, \isi{preaspiration}): \textit{*(ɸ)likkā-} `flat stone' > \textsc{Ir, ScG}, \textit{leac}, M \textit{lhiack}. There is no contrastive consonant (or in any case stop) length in the present\hyp day languages, so these stops are generally considered to be short (with a few exceptions). This implies a degemination at some point along the way. Unfortunately, there is no consensus as to when precisely it occurred: the possibilities range from the date prior to the manuscript record to, in some cases, incomplete degemination up to the present day (on issues of consonant quantity in Gaelic, see recently \cite{lewin2023preocclusion,stifter2023celtic,wheatley2022donegal}). Whatever the case may be, degemination itself is a trivial change that does not offer a criterion to distinguish between the possible systems of laryngeal contrast.

\section{Reconsidering the evidence}
\label{sec:recons-evid}

Having evaluated relevant changes individually, I can now synthesize the evidence for the two branches. In both cases, I will argue that the earlier changes are overall more compatible with a voicing\hyp based system, while syncope marks the advent of the present\hyp day \enquote{aspirating} patterns.

\subsection{Gaelic}
\label{sec:gaelic}

\Cref{tab:gaelic-lenitions} summarizes the developments in the Gaelic languages. To facilitate interpretation, cells corresponding to strongly diagnostic changes are shaded. The degemination of \cat{fortis} stops is difficult to place in the chronology, and it is in any case not probative, so it is excluded. As for \cat{lenis} geminates, one could perhaps place the degemination of the original ones relatively early, but since any geminates that arose due to syncope appear to have been shortened on par with the earlier ones, for the sake of concreteness I show a single post\hyp syncope round of degemination \parencite[in agreement with][1199]{stifter2017celtic}.

\begin{table}[htbp]
  \centering
  \begin{tabularx}{\linewidth}{Xll}
    \lsptoprule
    Early changes & Voicing scenario & Aspiration scenario \\
    \midrule
    (p) t k > (f) θ x / \phold V & \gc & \gc \\
    \textit{*katu-} > \textit{cath} &\multirow{-2}{*}{Compatible\gc} & \multirow{-2}{*}{Strongly compatible\gc}  \\
    \midrule
    (p) t k > (b) d ɡ / \phold N & \gc & \\ 
    \textit{*kant-} > \textit{cét} & \multirow{-2}{*}{Strongly compatible\gc} & \multirow{-2}{*}{Weakly or not compatible} \\
    \midrule
    Post\hyp syncope changes \\
    \midrule
    bb dd ɡɡ > b d ɡ & \multirow{3}{*}{Not probative} & \multirow{3}{*}{Not probative} \\
    \textit{*biggo-} > \textit{bec} \\
    \textit{*ag-gald-} > \textit{acaldam} \\
    \midrule
    Assimilation & \multirow{2}{*}{Unlikely} & \gc \\
    \textit{*-ámbi-sowet(i)}~> \textit{-impai} & & \multirow{-2}{*}{Strongly compatible\gc} \\
    \lspbottomrule
  \end{tabularx}
  \caption{Consonant developments in Gaelic}
  \label{tab:gaelic-lenitions}
\end{table}

My interpretation of \cref{tab:gaelic-lenitions} is that the earlier sound changes are very strongly compatible with a \enquote{voicing} system of laryngeal contrast, whereas an \enquote{aspirating} system is only very weakly, if it all, compatible with the data. The behaviour of nasal\hyp stop clusters in particular is very difficult to square  with an \enquote{aspirating} analysis. Admittedly, the evidence for spirantization of unaspirated voiceless stops is problematic in some ways, but the phenomenon appears to be not entirely unattested.

On the other hand, the later, post\hyp syncope pattern is essentially what is found in the present\hyp day languages, which show prototypical \enquote{aspirating} behaviour. Therefore, the prudent assumption is that secure evidence for the \enquote{aspirating} system does not date to earlier than around the time of syncope, late in the \ili{Primitive Irish} period, rather than to an earlier period in the history of Celtic, \textit{pace} \textcite{Eska2018}.

I suggest there is some circumstantial evidence that supports reconstructing the rise of aspiration in Gaelic \cat{fortis} stops to the late \ili{Primitive Irish} period. Almost all present\hyp day Gaelic languages show a process of vowel intrusion in heterorganic sequences of a sonorant and another consonant (as in Irish\il{Irish (Modern)} \ox{[dʲarəɡ]}{dearg}{red}, \textsc{OIr} \textit{derc}); see \textcite{morrison2018metrical, epenthesis-ms} for details. Today, this process is regularly blocked when the second consonant in such a cluster is a \cat{fortis} stop: such clusters lack intrusive vowels and in many dialects, especially in Scottsh Gaelic, voiceless sonorants are found instead (\ox{[ɔl̥k]}{olc}{evil}). \textcite{morrison2018metrical} persuasively argues that the intrusive vowel arises whenever the sonorant is moraic in the stem\hyp level phonology. If this is correct, the \cat{fortis} stop exception can be understood as failure of the sonorant to acquire a mora before such stops. \Textcite{morrison2018metrical} points to \ili{East Perthshire Gaelic} \parencite{ó1989east}, where the synchronic generalization is that coda sonorants are long (thus moraic) before historically \cat{lenis}, but not historically \cat{fortis} stops (\ox{[palːk]}{balg}{bag} but \ox{[ɔlk]}{olc}{evil}), a pattern \textcite{morrison2018metrical} considers archaic. One can hypothesize that sonorants before \cat{fortis} stops fail to acquire a mora precisely because they were devoiced  by the following stop \parencite[on the connection between voicing and moraicity, see][]{zec88:_sonor}, a phenomenon closely allied to \isi{preaspiration}. 

Now, although the exceptions to the generalization that \cat{fortis} stops block vowel intrusion are marginal in the present\hyp day languages, \textcite{eska2012remarks} identified a number of Ogam inscriptions that witness what he calls \enquote{non\hyp canonical} vowel intrusion, with examples such as \orth{ERACỊẠṢ} (cf. \orth{ERCIAS} and later \textit{Ercc}). He argues, convincingly, that, \textit{pace} \textcite{orahilly, hind-epenthesis}, such spellings do reflect the presence of intrusive vowels in these contexts in \ili{Primitive Irish}. If he is correct, then one can posit that intrusion was able to apply before \cat{fortis} stops precisely because such stops did not trigger sonorant devoicing. Once devoicing set in, it blocked sonorant moraicity, and the intrusive vowels were suppressed. Given that we observe the intrusive vowels in this context in the Ogam corpus, but not in the later Gaelic languages, I suggest that the rise of sonorant devoicing, associated with aspiration, may be tentatively dated to the late \ili{Primitive Irish} period~-- exactly as suggested by the analysis of \cref{tab:gaelic-lenitions}.

\subsection{Brythonic}
\label{sec:brythonic}

Turning now to Brythonic, \cref{tab:lenitions-brythonic} shows the sequence of changes and their evaluations. I place the degemination of stops early in the chronology, because in the case of \cat{fortis} stops it feeds second \isi{lenition}, whilst \cat{lenis} stops degeminated prior to syncope, since they escape the devoicing experienced by secondary \cat{lenis} clusters.

\begin{table}[htbp]
  \centering
  \begin{tabularx}{\linewidth}{Xll}
    \lsptoprule
    Early changes & Voicing scenario & Aspiration scenario \\
    \midrule
    p t k > b d ɡ / \phold V & \gc & \\
    \textit{*katu-} > \textit{cad} &\multirow{-2}{*}{Strongly compatible\gc} & \multirow{-2}{*}{Weakly compatible}  \\
    \midrule
    Sporadic geminate devoicing & \gc & \\
    \textit{*bigg-} > \textit{bikk-} &\multirow{-2}{*}{Strongly compatible\gc} & \multirow{-2}{*}{Highly unlikely} \\
    \midrule
    Degemination & \multirow{4}{*}{Not probative} & \multirow{4}{*}{Not probative} \\
    \textit{*bikk-an-} > \textit{*bikan-} > \textit{bychan} \\
    \textit{*likkā-} > \textit{*likā-} > \textit{llech} \\
    \textit{*ad-bero-} > \textit{*abbero-} > \textit{aber} \\
    \midrule
    p t k > f θ x / \phold V & \gc & \gc \\
    \textit{*broko-} > \textit{broch} & \gc & \gc \\
    \textit{*sterkā} > \textit{serch} & \multirow{-3}{*}{Compatible\gc} & \multirow{-3}{*}{Strongly compatible\gc} \\
    \midrule
    Post-syncope changes \\
    \midrule
    bb dd ɡɡ > p t k & \gc Strongly compatible & \gc \\
    \textit{*leto-teɣos} > \textit{*led-deɣ-} > \textit{llety} & but more complex \gc &\multirow{-2}{*}{Strongly compatible\gc} \\
    \midrule
    Assimilation & \multirow{2}{*}{Unlikely} & \gc\\
    \textit{*ati-sekʷ-} > \textit{*ad-heb} > \textit{ateb}&  &\multirow{-2}{*}{Strongly compatible\gc} \\
    \lspbottomrule
  \end{tabularx}
  \caption{Consonant developments in Brythonic}
  \label{tab:lenitions-brythonic}
\end{table}

As in Gaelic, later developments, like the present\hyp day patterns, agree strongly with the \enquote{aspirating} analysis. For the earlier stages, it would appear that both analyses are feasible. The balance might be tilted in favour of the voicing approach (as for Gaelic) if one considers that \isi{lenition} as deaspiration (or \isi{preaspiration} shortening) is somewhat rare typologically, and accepts the probative value of sporadic degemination despite its lack of regularity.

It is telling that the very earliest changes in Brythonic are also the ones most strongly pointing towards a \enquote{voicing} analysis. Just as with Gaelic, the closer one gets to the present day, the more the balance tips towards \enquote{aspirating} patterns, even though (with the problematic exception of geminate devoicing) a very strongly diagnostic pattern like the postnasal voicing of Gaelic cannot be found.

Overall, I conclude that while it is \emph{possible} that \cat{fortis} stops were aspirated at a very early stage of Celtic, as argued by \citeauthor{eska2017} (\citeyear{eska2017}; \citeyear{Eska2018}; \citeyear {eska2019laryngeal}; \citeyear{eska2020interarticulatory}), the evidence from the earliest sound changes is also compatible with a view of \cat{fortis} stops as voiceless (likely unaspirated) and \cat{lenis} stops as voiced. The change to the present\hyp day system can be tentatively dated to some point late in the era of considerable upheaval in the structure of the Insular Celtic languages in the mid 1st millennium CE.

\section{Conclusions}
\label{sec:conclusion}

Overall, I have argued that while most modern languages (other than Breton\il{Breton (Modern)}) are classic \enquote{aspirating} systems in the sense posited by \isi{Laryngeal Realism} \parencite[in agreement with][]{Eska2018}, solid evidence for these patterns does not stretch much further back than the sub\hyp Roman period in Britain and Ireland, when the Celtic languages underwent a series of drastic phonological and morphological changes that have largely defined their shape in the medieval and modern period. In this final section, I offer some possible perspectives on the wider relevance of this conclusion for Celtic historical linguistics.

In particular, I suggest that the proposed interpretation does not necessarily mean that Early Celtic was necessarily a \enquote{voicing} language \emph{tout court}. Rather, I suggest that we should take seriously the proposition that the Celtic languages, agreed to have been spoken over a large part of Western Europe throughout the 1st millennium BCE, demonstrated a degree of variability in their laryngeal contrast. Although the general slant of the reconstruction proposed here is that the development went from a \enquote{voicing} towards an \enquote{aspirating} system, the change need not have happened only once and/or in the same place.

Instead, I suggest that it is wise to envisage a pool of synchronic variation \parencite{Ohala1989} in the realisation of laryngeal contrasts across prehistorical Celtic varieties, with different dialects undergoing often similar shifts at different points in time. Such a régime would explain, for example, the evidence discussed in \cref{sec:aspir-view-proto}, which seems to suggest an \enquote{aspirating} system being attested on the Continent several centuries before my proposed change in the Insular varieties. If  \enquote{aspirating} patterns did indeed stabilize earlier in Continental Celtic, it may not be a coincidence that it was precisely the Continental varieties that were located near the spatial interface with Germanic \parencite[for up\hyp to\hyp date discussion of Celtic\hyp Germanic relationships, see][]{sluis2023european}, widely accepted to have been just such an \enquote{aspirating} language from a very early stage \parencite{salmons2017germanic}.

Even within Insular Celtic, it is not clear that the change happened at the same time. Even if a very late date for the split of Goidelic from Brythonic is accepted \parencite[perhaps along the lines of][]{schrijver2015pruners}, it would have already occurred by the time of syncope, a likely \emph{terminus post quem} at least for the Gaelic languages. The shift towards an \enquote{aspirating} system may thus have been at least partly independent in the two Insular groups. In this way, this change presents another example of Sapirean drift driven by the narrowing down of the range of inherited variation \parencite{joseph2006germanic, iosad2020phonological}.

A second corollary is that if some version of a \enquote{voicing} pattern was maintained at least in some part of the Celtic \emph{Sprachraum} as late as the first millennium CE, then its occurrence in Breton\il{Breton (Modern)} does not necessarily have to be fully ascribed to the influence of Romance. If some aspects of a \enquote{voicing} system were maintained during the spread from Britain, then perhaps the role of contact could have been more complex, which may have consequences for our understanding of the sociolinguistics of Romance\hyp Celtic interaction in Brittany.

More widely, I suggest that a model that takes synchronic variation in proto\hyp languages seriously \parencite[cf.][]{natvig2020fully} is potentially productive in tackling many classic issues in Celtic historical linguistics. I leave additional elaboration of such a model to further research.

\section*{Abbreviations}
\begin{multicols}{2}
\begin{tabbing}
MMMM \= Breton\kill
\textsc{Br} \> Breton\il{Breton (Modern)}\\
\textsc{Ir} \> Modern Irish\il{Irish (Modern)}\\
M \> Manx\il{Manx Gaelic (Modern)}\\
\textsc{MBret} \> \ili{Middle Breton}\\
\textsc{MCo} \> \ili{Middle Cornish}\\
\textsc{OIr} \> \ili{Old Irish}\\
 \textsc{PCelt} \> Proto-Celtic\\
PIE \> Proto-Indo-European\\
\textsc{ScG} \> Scottish Gaelic\il{Scottish Gaelic (Modern)}\\
SGDS \> Survey of the Gaelic Dialects \\ \> of Scotland\\
W \> Welsh\il{Welsh (Modern)}
\end{tabbing}
\end{multicols}


%\printglossaries

%\mh{Can't get the abbreviations to print here when I compile \texttt{iosad.tex}, but they do show up when I compile \texttt{main.tex}.}

\section*{Acknowledgments}

I have been thinking about the subject of this paper for a long time, all the way back to 2006 when I first presented some fledgling thoughts at the Second Colloquium of Societas Celto-Slavica (Institute of Linguistics, Russian Academy of Sciences). I am grateful to the organizers of the Foundational Approaches to Celtic Linguistics series at the University of Arizona for the opportunity to present this work in March 2023, and to the audience for stimulating questions. Thanks also to members of the Historical Phonology Reading Group at the University of Edinburgh for comments on a draft of this paper, and to two anonymous reviewers for numerous suggested improvements. All mistakes and shortcomings remain my own.

\is{aspirating languages|)}
\is{voicing languages|)}
\is{consonant strength|)}
\is{phonetic variation|)}

\printbibliography[heading=subbibliography,notkeyword=this]

\end{document}

% % Local Variables:
% % TeX-master: t
% % TeX-engine: xetex
% % End:

