\documentclass[output=paper,colorlinks,citecolor=brown]{langscibook}
\ChapterDOI{10.5281/zenodo.15654885}

\author{Stefan Moal\orcid{}\affiliation{Université Rennes 2}}
\title{France’s war on Breton diacritics: An incomprehensible obstinacy}
\abstract{The government of the French Republic has refused to sign the European Charter for Regional and Minority Languages. In this chapter, I review a series of legal decisions and official memos that exemplifies this bias against regional and minority languages as expressed through restrictions on the use of orthographic diacritics not found in standard French. Using non-French diacritics is considered to be an acknowledgment of the right to use a language other than French when dealing with the government.}
\IfFileExists{../localcommands.tex}{
  \usepackage{langsci-optional}
\usepackage{langsci-gb4e}
\usepackage{langsci-lgr}

\usepackage{listings}
\lstset{basicstyle=\ttfamily,tabsize=2,breaklines=true}

%added by author
% \usepackage{tipa}
\usepackage{multirow}
\graphicspath{{figures/}}
\usepackage{langsci-branding}

  
\newcommand{\sent}{\enumsentence}
\newcommand{\sents}{\eenumsentence}
\let\citeasnoun\citet

\renewcommand{\lsCoverTitleFont}[1]{\sffamily\addfontfeatures{Scale=MatchUppercase}\fontsize{44pt}{16mm}\selectfont #1}
   
  %% hyphenation points for line breaks
%% Normally, automatic hyphenation in LaTeX is very good
%% If a word is mis-hyphenated, add it to this file
%%
%% add information to TeX file before \begin{document} with:
%% %% hyphenation points for line breaks
%% Normally, automatic hyphenation in LaTeX is very good
%% If a word is mis-hyphenated, add it to this file
%%
%% add information to TeX file before \begin{document} with:
%% %% hyphenation points for line breaks
%% Normally, automatic hyphenation in LaTeX is very good
%% If a word is mis-hyphenated, add it to this file
%%
%% add information to TeX file before \begin{document} with:
%% \include{localhyphenation}
\hyphenation{
affri-ca-te
affri-ca-tes
an-no-tated
com-ple-ments
com-po-si-tio-na-li-ty
non-com-po-si-tio-na-li-ty
Gon-zá-lez
out-side
Ri-chárd
se-man-tics
STREU-SLE
Tie-de-mann
}
\hyphenation{
affri-ca-te
affri-ca-tes
an-no-tated
com-ple-ments
com-po-si-tio-na-li-ty
non-com-po-si-tio-na-li-ty
Gon-zá-lez
out-side
Ri-chárd
se-man-tics
STREU-SLE
Tie-de-mann
}
\hyphenation{
affri-ca-te
affri-ca-tes
an-no-tated
com-ple-ments
com-po-si-tio-na-li-ty
non-com-po-si-tio-na-li-ty
Gon-zá-lez
out-side
Ri-chárd
se-man-tics
STREU-SLE
Tie-de-mann
} 
   \boolfalse{bookcompile}
  \togglepaper[15]%%chapternumber
}{}

\AffiliationsWithoutIndexing

\begin{document}
\maketitle 

\il{Breton (Modern)|(}

\is{France|(}

\il{French (Modern)|(}



\shorttitlerunninghead{France’s war on Breton diacritics}%%use this for an abridged title in the page headers

\section{Introduction}

Those looking at France from the outside might not see it as a linguistically homogeneous state, but rather as a union of different communities and peoples. However, the very notion of ``minority" does not actually exist in France. While almost all other European states have ratified the \isi{Framework Convention for the Protection of Minorities}\footnote{\url{https://www.coe.int/en/web/minorities/at-a-glance}} – now virtually a precondition for entry into the European Union – France has not. The repeated rejection of the \isi{European Charter for Regional and Minority Languages}\footnote{\url{https://www.coe.int/en/web/european-charter-regional-or-minority-languages/text-of-the-charter}} (signed in 1999 but never ratified) even calls into question France's place in democratic Europe, since it remains one of the very few European Union member states to retain provisions privileging a single official language in its constitution. The only constitutional reference to other languages in its territory was, until recently, Article 75-1 (``Regional languages\is{regional languages} belong to the heritage of France") in keeping with historical and archival thinking. Of course, actual use of these languages in everyday life should be encouraged instead. 

On May 21, 2021, the Breton MP \name{Paul}{Molac} proposed the law ``on the protection of the heritage of \isi{regional languages} and their promotion”\footnote{\url{https://www.legifrance.gouv.fr/jorf/id/JORFTEXT000043524722}}, but the Constitutional Council ruled out two of its major provisions as contrary to Article 2 of the Constitution (``The language of the Republic is French"): (1) immersive teaching, carried out during a large part of school time in a language other than French, like in Diwan schools for Breton; (2) the use of \isi{diacritics} from \isi{regional languages} on civil status documents \citep{sm:ConseilConstitutionnel2021}. 

Using non-French \isi{diacritics} is considered to be an acknowledgment of the right to use a language other than French when dealing with the government. This chapter examines the details of the latter decision. After reviewing the definition of \isi{diacritics}, especially in the French context of the “Taubira" memo, I discuss the various court cases, from the initial \ili{Catalan} lawsuit to the particularly infamous \isi{Fañch} affair, currently ongoing in \isi{Brittany}. In passing, I point out a few positive outcomes, notably for placenames, and we will see how parents who plead for the correct spelling of their children’s names both in regional and in foreign languages have found a common cause to fight for. At the time of writing (summer, 2024), the situation remains deadlocked, and the current political turmoil in France is not conducive to a rapid resolution. Although the ban on \isi{diacritics} is not as serious as the ludicrous French transcriptions of Breton placenames in the past (e.g., the infamous translations of \textit{kroazhent} (crossroads) as \textit{croissant})\footnote{These names endured much longer, and still do in some cases, than those given during the short-lived anti-religion/anti-monarchy renaming craze of 1793--1794: Montagne-sur-Odet (instead of Kemper-Korantin/Quimper-Corentin), Pont Libre (Pontekroaz/Pont-Croix), Port de l’Égalité (Porzh-Loeiz/Port-Louis) or Port Nazaire (Sant-Nazer/Saint-Nazaire), to cite just a few examples in \isi{Brittany}.}, it can still be concluded that it is a symptom of a deeply rooted trend in France to demean so-called “\isi{regional languages}”.

\section{Definitions}

Let us start by reviewing the definition of ``diacritic\is{diacritics}", borrowed from the Greek διακριτικóς: ``which makes it possible to distinguish". A diacritic\is{diacritics} is a mark added to a letter of the alphabet to indicate a modification of its pronunciation or distinguish the word that includes it from another, homonymous word. It can be placed above, below, in front, behind, inside or across the letter to which it is attached. In French law, the administrative doctrine relating to the use of \isi{diacritics} in civil status records is based on several texts \citep{sm:DeBailliencourt2018}. The decree of Thermidor 2, Year II (July 20, 1794) requires that all public deeds be written in the French language throughout the territory of France. The decree of Prairial 24, Year XI (June 13, 1803) applies to all departments and all countries under Napoleon Bonaparte's Consulate and defines that ``where the practice of drawing up the said [public] acts in the language of these countries has been maintained, [they] must all be written in the French language", although the translation into ``the idiom of the country" may be written ``halfway down the French minute". Article 2, paragraph 1 of the Constitution of October 4, 1958, resulting from constitutional law no. 92-554 of June 25, 1992, states that “the language of the Republic is French". Finally, Article 1 of Law no. 94-665 of August 4, 1994, on the use of the French language, states that ``the language of the Republic under the Constitution, the French language is a fundamental element of the personality and heritage of France" and ``is the language of education, work, trade and public services".

Based on these texts, the State requires that civil status records, which hold authentic value, be written in French and obey French spelling rules. The memo dated July 23, 2014 \citep{sm:BOMJ2014} relating to civil status, known as the “Tau\-bira memo” states that “only the Roman alphabet may be used” and that “the only diacritic\is{diacritics} marks allowed are dots, umlauts, accents and cedillas as they appear attached to vowels and consonants authorized by the French language". It further states that: ``While Convention n°14 of the International Commission on Civil Status (ICCS) on the indication of surnames\is{surname} and first names\is{first name} in civil status registers recognizes foreign diacritic\is{diacritics} marks, it should be noted that this convention has not been ratified by France. The vowels and consonants accompanied by a diacritic\is{diacritics} used in French are: <à, â, ä, é, è, ê, ë, ï, ô, ö, ù, û, ü, ÿ, ç>. These \isi{diacritics} can be used on both upper- and lower-case letters. The ligatures <æ> (or <Æ>) and <œ> (or <Œ>), equivalent to <ae> (or <AE>) and <oe> (or <OE>), are accepted in the French language. Any other diacritic\is{diacritics} attached to a letter or ligature cannot be used to establish a civil status record”. 

These rules ignore the principle that parents should be free to choose their children's first names\is{first name}: parents can do so and may use non-traditional spellings, provided they only include the diacritic\is{diacritics} letters and ligatures of the French language mentioned above. This provision excludes: <ñ> in \ili{Basque} and Breton; <ò> in Creole\il{Guianese Creole} and \ili{Occitan}; <á, í, ó, ú>, in \ili{Occitan}, \ili{Catalan} and \ili{Corsican}; \textit{tārava} (literally: ``bar”) in \ili{Polynesian languages}. The simplification of the rules and procedures for changing first and last names, by the law of November 18, 2016, ``to modernize justice in the 21\textsuperscript{st} century" (Articles 60 to 61-4 of the Civil Code) have done nothing to change this state of affairs. 

Yet the presence or absence of these \isi{diacritics} are anything but trivial for a correct pronunciation in the above-mentioned languages. The <ñ> represents the palatal nasal consonant in \ili{Basque}, while in Breton it has been used since 1907 to nasalize the preceding vowel – without the <n> itself being pronounced. In \ili{Occitan} and \ili{Catalan}, the acute accent indicates the place of the tonic stress or indicates that a diphthong should be cut. In Marquesan, Tahitian, Wallisian and Futunian, all \ili{Polynesian languages}, the \textit{tārava} lengthens the vowels <ā, ē, ī, ō, ū>. \ili{Kanak languages} (in Kanaky/New-Caledonia), meanwhile, are spelled with the diacritic\is{diacritics} marks found in French.

In its decision no. 2021-818 DC of May 21, 2021, on the law on the protection of the heritage of \isi{regional languages} and their promotion, the \isi{French Constitutional Council} declared Article 9, which stipulates that the diacritic\is{diacritics} marks of \isi{regional languages} are authorized in civil status records, to be unconstitutional \citep{sm:ConseilConstitutionnel2021}. This decision brought to a halt a rather chaotic evolution in the authorization or use of these \isi{diacritics} to record names in French civil status registers. This process, which has been punctuated for a quarter of a century by a series of legal proceedings marked by numerous twists and turns, has mainly concerned users of certain \isi{regional languages} – namely \ili{Basque}, Breton, \ili{Catalan} and \ili{Occitan} – but now also concerns French citizens of Spanish- or Portuguese-speaking descent, who are also prohibited from registering the \isi{tilde} in their \isi{surname} or \isi{first name}. It is this process that will be discussed here, as well as the possible outcomes of the decision made by the so-called ``Sages” (``Wise Men/Persons").

\section{A memo (mis)named Taubira}

When \name{Christiane}{Taubira} was still a member of parliament, representing the South American French department of Guiana, situated between Suriname and Brazil, she published an article on her blog hosted on the \textit{Médiapart} information website, entitled ``Langues et Cultures 18 mars 2012" \citep{sm:Taubira2012} to commemorate the fifth anniversary of the entry into force of the UNESCO Convention on the Protection and Promotion of the Diversity of Cultural Expressions. It may therefore come as a surprise that \name{Christiane}{Taubira} never commented on the July 23, 2014 memo bearing her name. 

She is presumably much more honored to see her name associated with the May 21, 2001 law ``recognizing the slave trade and slavery as a crime against humanity", or the May 17, 2013 law ``opening marriage to same-sex couples", than with a text that bears her name only because she was then \textit{Garde des Sceaux} (Minister of Justice). Herself a fluent and proud speaker of Guianese (French-based) Creole\il{Guianese Creole}, she then reminded her readers that ``minority languages are often majority languages in their own right. [...] They can become official when the people decide to do so and history leads them to do so. This is the case of Haitian Creole\il{Haitian Creole} in Haiti and the Seychelles. She asserted that ``languages don't just convey words, they carry imaginations, universes", recalling that ``the original diversity of French cultures was sacrificed on the altar of a single, indivisible Republic, backed by a Nation that believed it could only prosper through uniformity. [...] The Nation waged war on \isi{regional languages} and cultures". 

The Guianese representative went on to describe as ``brutal" the term \textit{baragoin} (gobbledygook, gibberish) used by a President of the Roazhon/Rennes Court of Appeal who had adjourned a session on the grounds that two defendants were speaking Breton. As far as \isi{regional languages} are concerned, she considered that ``the \isi{Toubon Law} of 1994, blinded by its battle with English and anglicisms, had almost destroyed everything". Concluding that ``the issue is not linguistic, it's political". Insisting on the importance of transmission, she invited the readers of her blog to ask ``Paco Ibanez\footnote{Francisco Ibáñez Gorostidi (born 1934 to a Valencian father and a Basque mother) is an anarchist singer, fluent in \ili{Basque} and \ili{Catalan} in addition to French and Castilian. Despite \name{Christiane}{Taubira}'s words in strong favor of ``\isi{regional languages}”, which we have no reason to doubt are sincere, one may wonder whether the double absence of an acute accent on the <a> and \isi{tilde} on the <n> in the singer’s \isi{surname} wasn’t somehow foreshadowing the 2014 memo bearing her name, and the judicial and constitutional twists and turns that followed.} what he received from his grandmother in \ili{Basque}". 

\textit{Ibañez} happens to also be the name of the founder, in 2017, of a Facebook group called \textit{Touche pas à mon N tildé} (Don't touch my N \isi{tilde}), rebranded in 2024 \textit{Mon N tildé, ma bataille} (My N tildé, my battle) which as of July 2024 had 719 members and federated all initiatives aimed at the recognition by French civil status of all diacritic\is{diacritics} marks currently banned. The long legal battle over \isi{diacritics} goes back much further, however, since it began at the very start of the 21\textsuperscript{st} century.

\section{The \textit{Martí} case in North Catalonia, starting point of a long judicial process}

In the early 2000s, the first case concerning the \ili{Catalan} \isi{first name} \textit{Martí} gave rise to a lengthy and highly-substantiated dispute, which at the same time provided an opportunity to set out the facts of the legal problem \citep{sm:Esteban2016}. When his son was born in 1998, Alà Baylac-Ferrer never imagined that ten years later he would file, and lose, an appeal with the European Court in Strossburi/Strasbourg to register his son’s name, \textit{Martí}, correctly spelled. Ever since, the third child of the Baylac-Suarez family has been refused registration of the accent on the <í> in various courts, as this character does not belong to the French alphabet: in France, Martí is called Marti, therefore the stressed syllable is not the same and as a result the pronunciation is incorrect.

Baylac-Ferrer is a \ili{Catalan} language lecturer at the University of Perpinyà/Per\-pignan and one of the main promoters of the language in North Catalonia (department of Pyrénées-Orientales, South of France). The Baylac-Suarez couple have another son with a \isi{first name} also of \ili{Catalan} origin, Joan, which posed no problem as all its characters appear in the alphabet considered legally French. When Alà Baylac-Ferrer signed the papers at the registry office a few days after Martí's birth, he noticed that the accent on the newborn's name had disappeared. This policy of just removing the offending diacritic\is{diacritics} was plausibly meant to keep the parents from noticing and causing issues. Who knows how many busy and tired new mothers might have missed this and ended up with the incorrect name for their child as a result? The registrar's justification was that the character didn't appear on the computer keyboard, to which Alà Baylac-Ferrer replied that all he had to do was add the accent in pen to the paper. However, the civil servant refused, fearing that doing so would constitute a forgery.

It all appears to be a matter of conflicting norms: on the one hand, Article 57 of the French Civil Code guarantees parents' freedom to choose their child's \isi{first name}, stating that ``the child's first names\is{first name} are chosen by the father and mother"; on the other hand, the 1999 general instruction on civil status, based on the principle of mandatory use of French enshrined in Article 2 of the Constitution and the law of August 4, 1994, states that ``documents must be drawn up in French [...]. It follows in particular that the alphabet used must be [...] the only alphabet in use for writing the French language". This alphabet includes certain diacritic\is{diacritics} marks, but excludes others that are used to write some of the ``\isi{regional languages}” spoken in France, including \ili{Catalan}.

It was first the high court (\textit{Tribunal de Grande Instance,} Perpinyà\slash Perpignan, February 13, 2001) and then the court of appeal (\textit{Cour d'Appel}, Montpelhièr/Mont\-pell\-ier, November 26, 2001) that had the task of resolving the contradiction between the aforementioned provisions. In both cases, the judges ruled that the freedom to choose a \isi{first name} established by the French Civil Code could not override the obligation to use French and its alphabet in civil status documents. The registrar must agree to transcribe first names\is{first name} in ``regional” or ``foreign” languages, but these must be spelled \textit{à la française} which excludes diacritic\is{diacritics} marks not used in French. The \isi{first name} \textit{Martí} should therefore be spelled \textit{Marti}.

The next step was the last possible one within the French legal framework, namely the Paris Court of Cassation, but its decision was yet another setback (March 2, 2004). Martí's parents, believing that their fundamental rights had been violated and noting that their domestic remedies had been exhausted, took their case to the \isi{European Court of Human Rights} in Strossburi\slash Strasbourg. They invoked the principle of freedom to choose one's children's first names\is{first name}, guaranteed by the French Civil Code, as well as the right to respect for one’s private and family life, enshrined in Article 8 of the European Convention for the Protection of Human Rights \citep{sm:Ravasi2017}. 

However, the European Court in Strasbourg rejected the Baylac\hyp Suarez family's appeal in application of the principle of proportionality, stating that freedom of linguistic choice of a \isi{first name} ``does not form part of the matters governed by the Convention" and that member states thus enjoy a wide margin of appreciation as regards rules on surnames\is{surname} and first names\is{first name} \citep{sm:ECHR2012}. The harm would have to be particularly significant and undermine the child's personal identification to justify a violation of Article 8 of the Convention in relation to the linguistic regulation of first names\is{first name}.

Again, the ECHR validated the principle of acceptance of non-French first names\is{first name} by civil registries, provided they are spelled according to the French alphabet. It considered that the refusal to spell a \isi{first name} in \ili{Catalan} does not constitute an unjustified and disproportionate infringement of the parents' right to a private and family life (Article 8 of the European Convention for the Protection of Human Rights and Fundamental Freedoms (ECHR), nor discrimination on the grounds of belonging to a national minority (Article 14 ECHR), nor an infringement of the parents' right to a fair trial (Article 6 ECHR). It also added that ``the justification put forward by the Government, namely linguistic unity in relations with the administration and public services, was imperative [...] and proved to be objective and reasonable." 

\section{Fañch, ``the name that shakes the Republic" \citep{sm:Rouz2020}}

\is{Fañch|(}

The second major episode in the long-running controversy\footnote{See Appendix 1 for a timeline of the Fañch affair rulings.} surrounding \isi{diacritics} in names took place almost 20 years after the Martí affair began, far from the shores of the Mediterranean, in \isi{Brittany}. On May 11, 2017, a little boy was born in Kemper\slash Quimper and his parents, Lydia Fuzier and Jean-Christophe Bernard, living in the nearby town of Rosporden, chose to name him \textit{Fañch}, after one of his grandfathers, Fañch Cotten from Elliant. This grandfather was in fact registered as \textit{François}, historically one of the most popular male first names\is{first name} in Lower\hyp Brittany\is{Lower\hyp Brittany|see {Brittany}}, and which, ironically, originally meant `Frank' and by extension `French'. Men called \textit{François} in Lower\hyp Brittany\is{Brittany} are most often referred to by various Breton versions of the name: \textit{Frañsez, Frañsa, Saig, Soaig, Feñch}, and \textit{Fañch}. Occasionally, there would even be a \textit{Saig} and a \textit{Fañch} among the siblings of one same family, both officially called \textit{François}. The newborn's parents told the Kemper/Quimper registry office that \textit{Fañch} should be spelled with a \isi{tilde} over the <ñ>, but the officer initially refused to do so, transposing \textit{Fañch} into \textit{Fanch} on the grounds that the \isi{tilde} is not a diacritic\is{diacritics} used in the French language. The parents contacted the town hall of Kemper\slash Quimper, and the first deputy mayor decided to have the \isi{tilde} added to the birth certificate. She invoked Article 75-1 of the French Constitution, Article 57 paragraph 2 of the French Civil Code and a 1996 ruling by the \isi{European Court of Human Rights}, while appealing to the administrative authorities for understanding. Anticipating possible complications, Fañch's parents immediately had an identity card and passport issued in their son's name with the \isi{tilde} over the <ñ>, which didn’t pose any technical problems for the services concerned. The Roazhon\slash Rennes Public Prosecutor, who is responsible for ensuring that civil registrars comply with the law, challenged the decision in court and opened the legal proceedings. At the beginning of July 2017, the parents were summoned to appear before the Kemper\slash Quimper High Court with a view to a “request for rectification of a civil status record, or of declaratory or suppletive civil status judgments” \citep{sm:Feltin-Palas2017}.

The main arguments of the public prosecutor's office were as follows: The law of Thermidor 2, Year II, stipulates that public documents must be written in the French language. In accordance with Article 2 of the French Constitution, ``the language of the Republic is French", the Constitutional Council \citep{sm:ConseilConstitutionnel2021} ruled in 2001 that the use of French was mandatory for legal entities governed by public law and private law in the exercise of a public service mission. In 2011, the same Constitutional Council ruled that Article 75-1 of the Constitution, which since 2008 states that \isi{regional languages} are part of France's heritage, did not confer any rights or freedoms that could be enforced by individuals. In the Martí case, the ECHR ruled in favor of France in 2008, and even deemed ``objective and reasonable" the justification put forward by the French government for the need for linguistic unity in relations with the administration and public services \citep{sm:ECHR2012}. A 2014 memo lists exhaustively the diacritic\is{diacritics} marks admitted by civil status, exclusively those authorized by the French language, specifying that any other sign cannot be retained. Convention 14 of the International Commission on Civil Status (ICCS)\footnote{An intergovernmental organization whose aim is to promote international cooperation in civil status matters and to improve the functioning of national civil status services.} on the indication of surnames\is{surname} and first names\is{first name} in civil status registers recognizes foreign diacritic\is{diacritics} marks, but France has not ratified that convention.

Fañch's parents also put forward their own arguments. They were astonished that, in 2017, France continued to refer to the law of Thermidor 2, Year II, passed at the height of the \textit{Terreur} just a few days before the fall of Robespierre, which also stipulated that any civil servant who failed to write in French would be brought before the criminal court of his residence, sentenced to six months' imprisonment and dismissed. They pointed out that the \isi{Toubon Law} of 1994 on the use of the French language was intended to defend French against the hegemony of English, not against \isi{regional languages}, since Article 21 stipulates that it applies without prejudice to legislation relating to \isi{regional languages} and does not oppose their use. They were not asking for their son's birth certificate to be fully written in a regional language but simply for his \isi{first name} to be spelled in Breton. The \isi{tilde} has also existed in French, as demonstrated by Bernez Rouz, President of the Cultural Council of \isi{Brittany}. It is even found in the 1539 \isi{Villers-Cotterêts ordinance} by King François 1\textsuperscript{st}, which in its Article 111 proscribed the use of Latin in all judicial documents and imposed instead the use of the ``French mother tongue” – arguably not just French, but any of the languages spoken natively in the kingdom at the time. They challenged the Constitutional Council's interpretation of Article 75-1, according to which the introduction of \isi{regional languages} into the fundamental law does not give rise to any rights or freedoms enforceable by individuals: if these provisions have no normative scope, then they are purely declarative, even decorative. While the 1999 general instruction on civil status and the 2014 memo stipulate that only the Roman alphabet may be used, the <ñ> does belong to this so-called ``Roman" alphabet (in fact: \textit{Latin} alphabet) as demonstrated by its existence in several Romance languages. Accepting <ñ> poses no technical problem since the sign appears on all computer keyboards. According to Article 57 of the French Civil Code, ``the child's first names\is{first name} are chosen by his or her father and mother" and no mention is made of a ban on spelling first names\is{first name} in \isi{regional languages}. Article 8 of the European Convention on Human Rights states that everyone has the right to respect for his private and family life. The emotional choice of a \isi{first name} falls within the scope of the parents' private life. The same Article adds that there shall be no interference by a public authority with the exercise of this right except in the interests of national security or the protection of health, which are certainly not threatened by a \isi{tilde} in a \isi{first name}. Article 14 of the same Convention stipulates that the enjoyment of the rights and freedoms set forth shall be secured ``without discrimination on any ground such as sex, race, color, language, religion, political or other opinion, national or social origin, association with a national minority, property, birth or other status.” By refusing to accept the traditional spelling of one of the languages in its territory, the French state would be in contradiction with the European Convention on Human Rights.

In its ruling of September 13, 2017, the court annulled the correction made by the registrar, thus reinstating the spelling \textit{Fanch}, without the \isi{tilde}. The parents had not hired a lawyer to prepare their defense, as they probably didn't imagine such excesses on the part of the public authorities, especially as the Kemper/Quimper civil registry had already, very quickly, vindicated them \citep{sm:Blanchet2017}. Unhappy with this court injunction, but very determined, they took their case to the Roazhon/Rennes Court of Appeal. Local authorities in \isi{Brittany} supported them: after the town of Kemper/Quimper, the Regional Council of \isi{Brittany} unanimously adopted a vow on October 13, 2017 – the \textit{Front National} party not having taken part in the vote – just like the Penn-ar-Bed/Finistère Departmental Council the following week, also unanimously. Nathalie Appéré, mayor of Roazhon/Rennes, sent a letter to the Minister of Justice, Nicole Belloubet, invoking the French Constitution according to which ``\isi{regional languages} are part of France's heritage", as well as the freedom to choose a \isi{first name} as long as it is not contrary to the child's interests (Civil Code, Article 57).

Fourteen MPs from \isi{Brittany} belonging to \name{President}{Macron}’s majority, including the President of the National Assembly\is{National Assembly (France)}, sent a letter to the Minister of Justice, dated September 22, 2017, in which they asked her to kindly amend this memo so that spellings of first names\is{first name} in \isi{regional languages} can be authorized, even if they use particular \isi{diacritics}. They expressed their astonishment that a memo should prevail over the French Civil Code and Constitution, and that two very ancient texts were dug up: law no. 118 of Thermidor 2, Year II (July 20, 1794) and the decree of Prairial 24, Year XI (June 13, 1803). As of February 2024 though, nothing has changed in the ministerial memo. Seen from abroad, particularly from Spain but not exclusively, there was a great deal of incomprehension expressed: ``this information is so surreal that you'd think it was fake news" commented \textit{El Periódico de Catalunya}, for example, using dark humor in its headline alluding to the guillotine: ``France beheads the ñ". The Barcelona daily pointed out that, despite appearances, this was ``not an isolated case, nor a small unimportant detail”, since ``the stubbornness to impose French as the sole language is such that France entered a reservation when ratifying the Convention on the Rights of the Child [in 1990]”, stating that Article 30 of the Convention, on the cultural, religious and linguistic rights of minorities, did not apply to the territory of the Republic ``where there are no minorities, whether cultural or linguistic" \citep{sm:CourrierInternational2018}.

\section{Fañch wins on appeal then in cassation}

On October 19, 2018, the court in Roazhon/Rennes ruled against the Kemper/ Quimper court and decided to grant the petitioners' request, accepting the legality of registering the \isi{first name} \textit{Fañch} in its Breton spelling. Fañch's parents' lawyer, Mr Kerloc'h from the Naoned/Nantes bar, argued for their legal right to freely choose their child’s \isi{first name}, and warned against possible discrimination. The Advocate General, for his part, relied on the letter of the July 23, 2014 memo on civil status and pointed out that <{\textasciitilde}> is not included among the diacritic\is{diacritics} marks of French. Between a claimed right to difference and the strict observance of the law as currently established, the Roazhon/Rennes Court of Appeal's ruling therefore arbitrated this time for the former. The argument followed by the Court of Appeal reflects a conciliatory spirit and sets the debate on a linguistic level, since the Court proceeds to an \textit{in concreto} control of the use in the French language of the \isi{tilde} of the letter <ñ>. It notes that the <ñ> is used in French, as it ``appears several times in the \textit{Académie française} dictionary, in the Larousse dictionary and in the Petit Robert", and it is because it considers the <ñ> to be a French-language diacritic\is{diacritics} mark that the Court of Appeal authorizes the transcription of the \isi{first name} \textit{Fañch} in its Breton spelling \citep{sm:Gicquel2019}.

The court also referred to precedents where the use of <ñ> in official documents had been allowed by the administration, including cases in Roazhon/ Rennes (2002) and Paris (2009) where the same name \textit{Fañch} had been transcribed by officers of civil status in its proper Breton spelling. It also cited presidential decrees appointing senior civil servants whose \isi{surname} is spelled using the letter <ñ>: notably Laurent Nuñez, appointed Secretary of State to the Minister of the Interior precisely in the midst of the Fañch affair. The judge was therefore using linguistics to call into question the provisions of the memo of July 23, 2014, by reinstating the letter <ñ> in the corpus of signs used in French. At this time, the matter seemed settled, after eighteen months of proceedings for Fañch’s parents.

The president of the Cultural Council of \isi{Brittany}, Bernez Rouz, published a book about the Fañch affair \citep{sm:Rouz2020}, in which he recalled that the \isi{tilde} was used to mark nasals for centuries in French, even in ancient texts emblematic of the official status of French. The above-mentioned \isi{Villers-Cotterêts ordinance} contains several tildes\is{tilde}, for example in the word ``\textit{cõtenus}" (\textit{contenus}). The State recently relied on this pre-revolutionary ordinance to oppose any official use of “\isi{regional languages}”. In another official document from 1567, King Charles IX also used the word \textit{cõsiderations} (considerations). Bernez Rouz's efforts to demonstrate that <ñ> is indeed part of the French orthographic tradition are of course laudable, but the spelling of a native language like Breton could also simply be defended and illustrated for itself. 

As early as November 22, 2018, the Roazhon/Rennes public prosecutor's office lodged an appeal in cassation (abrogation), in a statement issued by Attorney General Jean-François Thony: ``as the texts stand, the \isi{tilde} is not recognized as a diacritic\is{diacritics} mark in the French language. It therefore seems necessary, in view of the possible national repercussions of the aforementioned ruling, to submit to the Court of Cassation the question of the use of the \isi{tilde} in a \isi{first name}". On October 17, 2019, the Court of Cassation rejected the appeal lodged by the Public Prosecutor of Roazhon/Rennes against the ruling of the Court of Appeal, and this time definitively closed the debate as far as little Fañch Bernard was concerned. His parents expressed their satisfaction that ``no one would be able to take it away from him" henceforth. Picking up on a dispatch from the Agence France Presse, a number of major media outlets reported on this latest episode, including Le Monde and The Guardian.

This was however a victory by default, since it was obtained only on the grounds of a formal legal error \citep{sm:LeMonde2019}. The Roazhon/Rennes public prosecutor's office lodged its appeal solely against the parents, forgetting to do so ``against in the capacity of legal representative of the child", which enabled the Court of Cassation, which does not judge on the merits, to declare the Rennes public prosecutor's appeal inadmissible. The July 23, 2014 memo on civil status records is certainly not abolished as a result of the Court of Cassation ruling. Although Mr Kerloc'h believed that it would be difficult to refuse the \isi{tilde} to another child, the Brest public prosecutor actually announced to all the mayors in his court's jurisdiction that the problem remained unresolved, precisely because the Court of Cassation had not ruled on the merits and summoned them to notify him of any new birth of a \textit{Fañch} with a \isi{tilde}.

\section{Four more Fañch affairs to date after the initial case}

As the Brest public prosecutor had suspected, a second Fañch case appeared in Montroulez/Morlaix on November 18, 2019, just one month after the Court of Cassation rejected the appeal in the first Fañch case. It is also with the correct Breton spelling that the parents of the new little Fañch had his \isi{first name} registered by the Montroulez/Morlaix civil registry, but the public prosecutor ordered the removal of the \isi{tilde} from this transcription in the birth certificate, again referring to the memo of July 23, 2014. This family also started a legal procedure but they did not wish to be exposed in the media and therefore did not communicate openly. They were represented and defended by the same lawyer, Mr Kerloc'h, and also supported by \textit{Skoazell Vreizh}\is{Skoazell Vreizh@\textit{Skoazell Vreizh}}, an NGO founded in 1969, originally to help the families of ``Breton political prisoners”.

Little Awen Fañch was born on April 15, 2020, at Gwengamp/Guingamp hospital. His parents chose the middle name of his paternal grandfather, \textit{Fañch} and filled in the civil registration form with a \isi{tilde} over the <ñ>. A week after the maternity hospital forwarded the file to the municipality of Pabu, the family's home town, the parents noticed that the \isi{tilde} did not appear on the birth certificate and first thought this was a mistake, like in the above-mentioned Baylac-Ferrer case. In fact, the town hall secretary had only done her job by complying with a memo from the Sant-Brieg/Saint-Brieuc public prosecutor. The family didn't hold this against her, but didn't intend to leave this “Fañch III” at that either, so they contacted \textit{Skoazell Vreizh}\is{Skoazell Vreizh@\textit{Skoazell Vreizh}}. ``While our initial approach was not at all militant, but rather a family affair", explains the mother, ``from now on, if we have to be militant to ensure that the Breton language and our traditions are respected, we're ready to be militant for ourselves and for anyone else it might concern in the future" \citep{sm:LeFur2020}. On May 4, the mayor of Pabu drew up a new birth certificate in which he had the \isi{tilde} reinstated, contrary to the memo from the Sant-Brieg/Saint-Brieuc public prosecutor, who to date has not commented on the matter. If he were to object, however, Awen Fañch's family would go to court to try and win the case, because, adds the mother with a certain good sense: ``the Breton language has precise rules, its own literature and a whole wealth of riches. I can't see why these rules should be ignored and considered a trifle, while one tries so hard to respect the rules of French".

In June 2023, the mayor of An Oriant/Lorient, Fabrice Loher, signed the birth certificate of yet another little Fañch, with a \isi{tilde} over the <ñ>, even though the registry office had initially said no. However, after the summer, the public prosecutor of An Oriant/Lorient, Stéphane Kellenberg, sent a letter to the parents requesting that the \isi{tilde} on the <ñ> be removed \citep{sm:James2023}. In this letter, he refers to the Constitutional Council's decision of May 21, 2021 “concluding that Article 9 of the law on the protection of \isi{regional languages} and their promotion is unconstitutional”. The letter states that “in its decision, the \isi{French Constitutional Council} recalls, in accordance with Article 2 of the French Constitution, that the use of French is mandatory for legal entities under public law and for persons under private law in the exercise of a public service mission. Private individuals may not claim, in their relations with administrations and public services, a right to use a language other than French, nor be compelled to do so.” In this decision, the \isi{French Constitutional Council} ruled that “by providing that entries in civil status records may be written using diacritic\is{diacritics} marks other than those used to write the French language, these provisions recognize the right of individuals to use a language other than French in their dealings with administrations and public services. They therefore fail to meet the aforementioned requirements of Article 2 of the Constitution.”

The parents, Etienne Pichancourt and Mélissa Yana, thought that since the mayor of An Oriant/Lorient had validated the name, and Fañch had received his identity card, his \isi{tilde} couldn't be taken away from him. The An Oriant/Lorient public prosecutor's office clearly didn't want to take this into account and decided to rely solely on the decision of the \isi{French Constitutional Council}. ``Insofar as this decision by the Constitutional Council, an obviously superior jurisdiction, is opposed to the case law of the Roazhon/Rennes Court of Appeal, I can only legally proceed with the administrative rectification of the purely material error vitiating the spelling of the \isi{first name} attributed to your child, under the terms of Article 99-1 of the Civil Code, by giving you the necessary instructions for the deletion of the \isi{tilde} appearing wrongly in this civil record, since this diacritic\is{diacritics} sign is non-existent, both in French and in positive law." Morbihan MP \name{Paul}{Molac}, the initiator of the law cited, declared: ``What absurdity! The only solution to this impasse? Amend the Constitution. Nothing else” \citep{sm:Etienne2023}. Fañch Pichancourt’s parents have started preparing their defense for an appeal with the support of \textit{Skoazell Vreizh}\is{Skoazell Vreizh@\textit{Skoazell Vreizh}} and regional councilor Stéphanie Stoll.

In January 2024, another couple, from the Maine-et-Loire (a French department bordering \isi{Brittany}) were summoned to appear before the family court for having chosen the name \textit{Fañch} with a \isi{tilde} for their son, born last summer to a Breton mother. Although the civil registrar at the maternity hospital had warned the parents that the spelling of \textit{Fañch} could pose a problem, they had chosen to keep the name, even if it meant having to fight for it if necessary. In this case, according to the Angers public prosecutor, not only is the \isi{tilde} not a diacritic\is{diacritics} mark retained by the French language, but moreover the name \textit{Fañch} is considered ``contrary to the child's best interests". Loïg Chesnais-Girard, the president of the Regional Council of \isi{Brittany}, spoke of an ``inadmissible social violence" and offered “his full support” to the Maine-et-Loire family, regretting that ``for a diacritic\is{diacritics} sign that is nonetheless not unknown to the French language, as the Roazhon/Rennes Court of Appeal recalled on November 19, 2018, it is time for the law to move forward" \citep{sm:Salliou2024}. Philippe Grosvalet, senator of the south \isi{Brittany} department of Loire-Atlantique, also intervened with the Minister of Justice, noting: ``it is common knowledge that our courts are clogged and that magistrates are unable to rule within a reasonable timeframe […]. In this context, initiating this procedure is tantamount to adding to the workload of judges who are already far too busy. Worse still, it diverts them from one of their essential missions: ensuring the well-being of the children." The senator concludes his letter with this request: ``Put an end to this sterile litigation, which shows little respect for children, parents and the work of family court judges" \citep{sm:Courrierdel’Ouest2024}.

\is{Fañch|)}

\section{A few positive rulings aside from the Fañch issue}

While the Martí affair was still underway – a procedural marathon that lasted almost a decade – the \textit{Tribunal de Grande Instance} (ordinary court of first instance) of Perpinyà/Perpignan did validate the civil registration of the \ili{Catalan} first names\is{first name} \textit{Lluís} (Montpelhièr/Montpellier TGI ruling of September 27, 2001), \textit{Joan-Lluís} (Perpinyà/Perpignan TGI ruling of January 13, 2004) and \textit{Alícia} (Perpinyà/Perpignan TGI ruling of February 16, 2006) in their \ili{Catalan} spelling, retaining their letter <í> with its acute accent. Conversely however, in 2011, the Tolosa/Toulouse registry office refused to register a little girl named \textit{Alaís} because the letter <í> does not exist in the French alphabet, even though the same office in the same city had previously accepted the registration of a little Anaís and a little Loís \citep{sm:Jornalet2016} in their correct \ili{Occitan} forms.

Meanwile, the issue had briefly shifted from anthroponymy to toponymy, as the \ili{Occitan} spelling of a placename on the Languedoc coast in the South of France had become the subject of headlines. In August 2009, the municipal council of Villeneuve-lès-Maguelone had its historically \ili{Occitan} name \textit{Vilanòva-de-Magalona} added alongside the existing sign bearing the official French name, marking the town’s entrance. In doing so, Vilanòva-de-Magalona was merely following the example of several hundred cities, towns and villages before it, in \isi{Brittany}, Corsica, the Northern Basque country, Northern Catalonia and other Occitan regions. At the request of the \textit{Mouvement Républicain de Salut Public}, whose name sounds like a deliberate reference to the \textit{Terreur} revolutionary period, the administrative court of Montpelhièr/Montpellier (October 12, 2010) ordered the mayor to remove the signs bearing the \ili{Occitan} name of his town. This was in ideological continuity with a former decision by Bernard Bonnet, \textit{préfet}\footnote{The State’s representative appointed by the central French government in a department or a region.} of the Pyrénées-Orientales, to bring the municipality of Perpinyà/Perpignan before the administrative court (January 27, 1997) because the elected city council had asked the local university to translate the city's toponyms into \ili{Catalan} for the land registry \citep{sm:Duran2000}. The court referred to both the Constitution and the Highway Code, concluding that, on the one hand, “the translation into \ili{Occitan} is devoid of any historical basis”, and that, on the other, the ``EB10 and EB20" urban entrance signs, as defined by Article 5 of the decree of November 24, 1967, as amended, include the accented letter <ò> in Vilanòva, whereas ``this diacritic\is{diacritics} sign is not included in any of the appendices to the aforementioned decree, as it does not exist in the French language". The President of the Court concluded that ``as a result, the signs in question are detrimental to the clarity of the information required by the obligation of caution and safety imposed on all road users entering a built-up area." 

However, the Marseille Administrative Court of Appeal (June 28, 2012) overturned the Montpelhièr/Montpellier ruling (October 12, 2010), rejecting the \textit{Mouvement Républicain de Salut Public}{}'s claim and ordering it to pay €2,000 under Article L.761-1 of the French Code of Administrative Justice. Indeed, according to the Court of Appeal, ``judicial jurisprudence to the effect that the registration of a \isi{first name} in the civil register must respect the spelling and toponymy of the French language is in no way transposable to spelling rules concerning regional language translations of town entrance signs indicating the French name of a commune". The court considered that the risk represented by the signs in question for the safety of road users has not been demonstrated and invoked the hierarchy of norms. “Articles 4 and 21 of the law of August 4, 1994, regulatory provisions that do not allow for the use of a grave accent on the letter o cannot be applied to the translation into the regional language of a French-language town entrance sign.” Thousands of signs in \ili{Basque}, Breton, \ili{Catalan}, \ili{Corsican}, \ili{Occitan}, etc., often hard-fought for, and obviously carrying the \isi{diacritics} of each language, are therefore safe, all the more so since Article 8 of the “Molac” law on the promotion of \isi{regional languages} legalized bilingual signage in 2021. Several municipalities of Lower-Brittany\is{Brittany} carry a <ñ> in the Breton spelling of their names: Hañveg (Hanvec), Lanleñv (Lanleff), Lañriware (Lanrivoaré), Lañveog (Lanvéoc), Gwiproñvel (Guipronvel), Plougoñ (Plogoff), Ploñger (Ploumoguer), Roskañvel (Roscanvel), Treglañviz (Tréglamus) not to mention dozens of townships, hamlets, localities and streets. Although the actual road signs are now protected, the only authoritative version to be used in any kind of official paperwork remains the French version.

Another ruling concerning the Breton \isi{first name} \textit{Derc’hen} eventually ended up with a favorable issue too. On January 23, 2018, the registry office of Roazhon/ Rennes, following the advice of the Public Prosecutor's Office, had refused to register a child, born in August 2017, whom his parents wished to name \textit{Derc'hen}. Although the apostrophe is one of the signs authorized in French, the trigram <c'h> – used in Breton ever since the spelling reform of Jesuit Father Julien Maunoir in 1659 – was criticized for its pronunciation ([x], [ɣ] or [h]), which does not exist in French. However, there have been Bretons officially named \textit{Goulc'han} or \textit{Goulc'hen} for a very long time, so the public prosecutor finally reversed his decision concerning \textit{Derc'hen}. Ironically, the names of at least two of the leading protagonists in the \isi{Fañch} affair, the public prosecutor of Kemper/Quimper, Thierry Lescouarc'h, and the parents' lawyer, Jean-René Kerloc'h, both include <c'h>, which clearly demonstrates the antiquity and banality of this trigram in surnames\is{surname} considered ``French". Three days after the local press had reported on the situation, and the mayor of Roazhon/Rennes, Nathalie Appéré, had announced that she was going to ask for the 2014 memo to be amended, the public prosecutor stated in a press release that the memo does not expressly rule on the use of the apostrophe, and since it is a commonly used spelling mark, it can be considered that its use is not formally prohibited. He announced that this issue would be re-examined with the Ministry of Justice in the name of the risk of breach of equality, as the first names\is{first name} Tu'iuvea, N'néné, D'Jessy or N'Gussan had indeed been accepted in Rennes even though the 2014 memo was already in force. 

\section{``Regional” and ``foreign” languages: {A} common cause}

While the interests of so-called ``regional" and ``foreign" languages may sometimes be pitted against each other in education or the media, a kind of sacred union prevails in the struggle to reclaim the diacritic\is{diacritics} marks of French citizens. One obvious illustration of this alliance is undoubtedly the story of João \isi{Fañch}, a baby born in Créteil on August 18, 2019 to a Breton mother and a Franco-Portuguese father. The Créteil town hall having refused him his two tildes\is{tilde}, his birth certificate therefore bears Joao Fanch, but his father made a point of expressing his disapproval by crossing out the “chosen first names\is{first name}” box to indicate “imposed first names\is{first name}” instead, and he started leading the legal battle over the \isi{tilde} with his partner \citep{sm:Delporte2019}. 

Alexandra Ibañez was not allowed to pass on to her son Ronin, born in March 2017, her \isi{surname} in its entirety, i.e. with a <ñ>, alongside Delaunay, the father's name \citep{sm:Serhani2017}. Despite the fact that her own childhood civil status booklet does indeed include the \isi{tilde} on her \isi{surname}, the civil registry of the Baiona/Bayonne town hall (department of Pyrénées-Atlantiques, French side of the Basque country), where her child's birth was declared, therefore imposed the spelling Delaunay-Ibanez on her. According to this great-granddaughter of a Spanish immigrant, this <ñ> is the story of her family, of her origins, and her son thus simply didn't have the same \isi{surname} as hers. She also thought it was an administrative error when she registered her son at the civil registry office, when the \isi{tilde} had disappeared from the birth certificate. The town hall in Donibane Lohizune/Saint-Jean-de-Luz, where the family lives, advised her to appeal to the Baiona/Bayonne public prosecutor, who finally signed the rectification on July 29, 2020. Alexandra Ibañez shared the file she compiled on her Facebook page \textit{Touche pas à mon N tildé} which enabled her to win her case after more than three years. Although her name is of Spanish origin, Alexandra Ibañez points out that many \ili{Basque} first names\is{first name}, such as \textit{Beñat}, \textit{Aña} or \textit{Iñaki}, are affected by the same problem. 

One of the active members of the Facebook group is Elodie Billaud Bidegain, who in 2011 faced the refusal of the Baiona\slash Bayonne town hall to spell her newborn daughter's name \textit{Aña}, a refusal confirmed by the Baiona\slash Bayonne public prosecutor. Although her daughter continued to be called \textit{Aña} by the family, her official \isi{first name} remained \textit{Ana}, without the <ñ>, for over a decade. When someone asked for “Ana”, she either believed they were addressing someone else, or thought they were mispronouncing it. In January 2022, following the sending of a new file to the Baiona/Bayonne public prosecutor at the end of November 2021, Elodie Billaud Bidegain was finally informed that the latter had granted the change of \isi{first name} to \textit{Aña} and requested the modification of the act at the town hall. Another young Basque woman called \textit{Aña} officially also got her \isi{tilde} back in 2023 \citep{sm:Etxezaharreta2023}, 28 years after her birth. There is something ironic in noting that this <ñ>, often brandished in Spain – including in anti-Catalanist\il{Catalan} demonstrations – as a symbol of \textit{hispanolidad}, has achieved the feat of bringing together under its small sinusoidal banner a large number of objective allies, supporters of the adoption of \isi{diacritics} present both in the languages of the descendants of immigration (Spanish and Portuguese in particular) and in territorial languages of the French Republic such as \ili{Basque} and Breton.

\section{National representation steps up}

The linguists' debate over the first \isi{Fañch} case, between the public prosecutor and the Court of Appeal in Roazhon\slash Rennes, confirmed the legal uncertainty faced by parents wishing to give their children names bearing these \isi{diacritics}. Some courts have ruled that such marks which appear in dictionaries, are not foreign to French and are therefore acceptable. At the time of writing in August 2024, the Facebook page \textit{Mignoned \isi{Fañch}} (Fañch’s friends) counted 29 officially accepted tildes\is{tilde} all over France and there appeared to be no problem any longer in the Basque country whereas elsewhere in France it still seemed to be a real lottery. Parents therefore have no way to be sure whether or not the transcription of a child’s \isi{first name} containing one of these \isi{diacritics} will be accepted by a registrar, public prosecutor or any other judicial authority. It was in this sense that the Roazhon/Rennes public prosecutor asserted, since the Court of Cassation had not ruled on the merits of the case in its ruling of October 17, 2019, that it was now up to elected representatives to introduce a bill to change the regulation \citep{sm:Zabaleta2020}.\footnote{See Appendix 2 for a timeline of legislative action on the issue.} 

After numerous attempts by various MPs in the past to introduce bills about \isi{regional languages}, the bill put forward by MP \name{Paul}{Molac}, entitled ``\textit{Loi relative à la protection patrimoniale des langues régionales et à leur promotion}”, was debated in both houses of parliament. It passed the first reading in the National Assembly\is{National Assembly (France)} but its main provisions were scaled back at that early stage except, ironically, the registration of names spelled in ``\isi{regional languages}” on birth certificates. The French Senate, in session on December 10, 2020, reinstated many of the proposals rejected by the National Assembly\is{National Assembly (France)}, a version which passed by 253 votes to 59. Senator Monique de Marco in her report on behalf of the Committee on Culture, Education and Communication, points out that the Senate adopted an Article in January 2020 to include <ñ> in the list, and stresses that the proposed wording for Article 9 is more protective in that it does not draw up a list but indicates that all regional language diacritic\is{diacritics} marks are admissible. At the second reading in the French National Assembly\is{National Assembly (France)}, on April 7 and 8, 2021, opponents of the Senate text and the Minister of Education put forward a series of amendments which, if adopted, would have sent the text back for a new ``shuttle" between the two Chambers, but their maneuver failed. 

On Thursday, April 8, 2021, during the parliamentary day reserved for \name{Paul}{Molac}'s small group, \textit{Libertés et Territoires}, bill n°2021-641\footnote{\url{https://www.legifrance.gouv.fr/jorf/id/JORFTEXT000043524722}} relative to ``\textit{la protection patrimoniale des langues régionales et à leur promotion}” was eventually adopted by a very large majority, including a hundred or so MPs of \name{President}{Macron}’s side, against the very advice of the government they supported. Under the Fifth Republic, this is the first law devoted to \isi{regional languages} to be definitively adopted, with a few specific measures to protect and promote these languages in three areas: heritage, education and public services. Its dual aim is to protect and promote intangible heritage and cultural diversity, of which \isi{regional languages} are one expression. The text recognizes the existence of a linguistic heritage made up of the French language and \isi{regional languages}, and specifies the role of the State and local authorities in their teaching, dissemination and promotion. It grants the status of ``national treasure" to assets of major interest to the knowledge of French and \isi{regional languages}, such as recordings or ancient manuscripts. Provisions of the ``\isi{Toubon Law}" (August 4, 1994) on the use of the French language now specify that they ``do not hinder the use of \isi{regional languages} and public and private actions in their favor". In the field of regional language teaching, communes with no bilingual schools are now obliged to contribute to the tuition fees of schools offering bilingual teaching in the area (such as Diwan schools in \isi{Brittany}). The teaching of \isi{regional languages} is supposed to be generalized as an optional subject within the normal teaching timetable, from kindergarten to high school, based on the model developed in Corsica since the turn of the century. Bilingual signage is now recognized in the law, which clearly authorizes public services to use regional language translations on public buildings, road signs and institutional communications, for example. 

This is where, ``after the legislative intoxication, the constitutional sobering up comes in” \citep{sm:Arlettaz2021}. On April 22, 2021, 61 petitioning deputies belonging to the parliamentary majority referred the law on the heritage protection of \isi{regional languages} and their promotion to the Constitutional Council. The article they challenged was Article 6, which the Councilors found nothing to object to, but they took up two other articles on their own initiative, including Article 4 (on immersive teaching) and Article 9 (on diacritic\is{diacritics} signs in civil status documents). It should be noted that ``these two declarations of unconstitutionality are intended to prevent the teaching of a subject which in fact already exists in the public service, and to prohibit the drafting of civil status documents, some of which have been deemed regular by the judicial judge" \citep{sm:Arlettaz2021}.

Where \isi{regional languages} are spoken, many local and national elected representatives, including from \name{President}{Macron}’s party, often voice their protests loudly in the media and on social networks. On May 29, 2021, large-scale demonstrations took place in Gwengamp/Guingamp and Baiona/Bayonne in particular, where in both cases organizers counted up to 10,000 participants. The NGO European Language Equality Network sent a communication to the United Nations Special Rapporteur on Minority Issues to remind the French state of its fundamental constitutional obligations to protect its own linguistic minorities. At the time of publication of the present article, however, no new measures concerning the authorization of diacritic\is{diacritics} marks have yet emerged. 

\section{Concluding remarks}

On November 1\textsuperscript{st}, 2023, a very strong windstorm called Ciarán severely affected \isi{Brittany} and other parts of France and Europe. Some Bretons didn't lose their sense of humor however, when they pointed out on social networks that the administrative authorities had widely used a banned diacritic\is{diacritics} mark on ``á” in their official communication about the storm: the Irish acute accent, also known as \textit{fada}\footnote{Which happens to coincidentally mean ``fool" in French.}. Although anecdotal, this story also shows that France's position is untenable in the long term. Such a typically French issue will certainly not be solved by retreating into strictly technical considerations (i.e., legal and legalistic considerations). In the Martí case, the \isi{European Court of Human Rights} ruled in favor of France, but its judgment was qualified by the following comments: on the one hand, ``given the total absence of consensus between the member states, they had a particularly wide margin of appreciation;" on the other, the government's justification was necessary ``for the time being". However, the 21\textsuperscript{st} century justice modernization law (November 18, 2016) contains amended Article 225 which now prohibits discrimination based on the ability to express oneself in a language other than French. The current regulations on diacritic\is{diacritics} signs are therefore untenable in the short term, as they create a breach of equality based on an illegitimate criterion, which defines discrimination \citep{sm:Blanchet2017}. Indeed, parents who choose first names\is{first name} using languages for which the currently restricted Latin alphabet of French is sufficient have the right to register them; on the other hand, those who choose first names\is{first name} for which this alphabet is not sufficient do not have the right to register the \isi{first name} of their choice. To take just one last example in French Polynesia~– which is on the official UN list of countries to be decolonized~– the head of civil status at Papeete town hall explains to Polynesian parents that Tahitian diacritic\is{diacritics} marks are not allowed \citep{sm:LaDépêchedeTahiti2015} and cites the apostrophe marking glottal occlusion among these signs, despite the fact that it is perfectly legal, as seen above, unlike the \textit{tārava}, a diacritic\is{diacritics} which, ironically, is known in French as a... ``\textit{macron}.” 

According to \name{Paul}{Molac} – re-elected MP by a large majority in July 2024 – several solutions remain possible. While amending Article 2 is not the easiest as it would challenge a sacred tenet, a but less ambitious supplement to Article 75-1 could be better adapted. Pragmatically, two sentences could be added about \isi{regional languages}: ``their teaching is determined by law, the immersion method is constitutional. Regional languages\is{regional languages} are used in public services and civil status" \citep{sm:Molac2022}. The Constitutional Council's decision has re-established the hierarchy of norms, since it takes precedence over a ministerial memo with no normative value. In the case of immersive teaching, the ruling of unconstitutionality was evaded – but with what sustainability? – by the publication of a mere memo \citep{sm:BOEMJS2021} as it was politically urgent to put out the fire ignited by the announcement that existing immersive teaching was unconstitutional. 

It remains to be seen whether the same strategy will apply to the issue of \isi{diacritics}, which is just one in a much larger issue of linguistic discrimination in France. In terms of language protection, France remains embarrassingly backward, in a position that is increasingly isolating it from democratically comparable states. The self-proclaimed \textit{Patrie des Droits de l'Homme} (Motherland of Human Rights) promotes its own particularism internationally, as well as multilingualism or linguistic diversity within the European Union, yet remains deaf to any similar recognition on its own territory. France strongly opposes the hegemony of English on a global scale, but does what it criticizes English for doing worldwide within its own borders. It is not credible to claim to be preserving cultural diversity in Europe without even subscribing to international commitments recognized by all. It is unseemly to want to protect minorities in other countries while asserting that there are none on one’s own territory. France’s credibility would be much stronger, and above all, much more coherent if the State were to commit itself to a genuine recognition of its own cultural and linguistic diversity. Let's go back to the original definition of the term ``diacritic\is{diacritics}": this chapter has dealt with it at length in the domain of orthography; but in the medical field, diacritic\is{diacritics} symptoms serve to characterize a disease, to distinguish it from all others. The single language ideology remains indeed a French disease, and \isi{diacritics} are but one of the many symptoms that still distinguish France from many other democratic countries in the world, starting with its closest neighbors. Treatments do exist, neighboring countries like Spain and the United Kingdom apply them. It is probably France's turn to take care of its languages, or rather the speakers thereof, and thus take better care of itself.

\appendixsection{Timeline of the Fañch affair rulings}

\is{Fañch|(}

\begin{itemize}
\item {\textit{May 11, 2017}: Fañch Bernard born in Kemper/Quimper. The officer initially spells the name Fanch but the first deputy mayor has the \isi{tilde} added to the birth certificate.}
\item {\textit{July 2017}: Fañch Bernards’s parents summoned to appear before the Kemper/Quimper High Court.}
\item {\textit{September 13, 2017}: The Kemper/Quimper High Court reinstates the spell\-ing \textit{Fanch} without the \isi{tilde}. The parents take their case to the Roazhon/ Rennes Court of Appeal, supported by local and regional authorities in \isi{Brittany}.}
\item {\textit{October 19, 2018}: Roazhon/Rennes Court of Appeal rules against the Kemper/Quimper court and accepts the legality of registering Fañch with its correct Breton spelling.}
\item {\textit{November 22, 2018}: The Roazhon/Rennes public prosecutor's office lodges an appeal in cassation against this ruling. Appeal rejcted on October 17, 2019, on the grounds of a mere formal defect. The Brest public prosecutor summons all the mayors in his jurisdiction to notify him of any new birth of a baby called Fañch.}
\item {\textit{August 18, 2019}: João Fañch is born in Créteil, near Paris, to a Breton mother and a Franco-Portuguese father. The Créteil town hall having refused both tildes\is{tilde}.}
\item {\textit{November 18, 2019}: Another baby Fañch born in Montroulez/Morlaix with his name correctly registered. The Brest public prosecutor orders the removal of the \isi{tilde}. The family starts a legal procedure.} 
\item {\textit{April 15, 2020}: Awen Fañch born in Gwengamp/Guingamp. His name is filled in the civil registration form with an <ñ> but the \isi{tilde} disappears on the birth certificate. The local mayor reinstates the \isi{tilde} on a new certificate. No reaction so far from the Sant-Brieg/Saint-Brieuc public prosecutor.}
\item \textit{June 17, 2023}: The mayor of An Oriant/Lorient, signs the birth certificate of yet another little Fañch with a \isi{tilde} over the <ñ> although the local registry office had initially opposed. In September, the public prosecutor of An Oriant/Lorient has the name modified: Fanch. The parents appeal.
\item{\textit{July 26, 2023}: Another baby Fañch born in Maine-et-Loire (bordering \isi{Brittany}). His parents are called to court in Angers on February 15, 2024. According to the president of the Regional Council of \isi{Brittany}, this is ``inadmissible social violence."}
\end{itemize}

\is{Fañch|}

\appendixsection{Timeline of the legislation on the use of diacritics in civil status} 

\begin{itemize}
\item {\textit{Thermidor 2, Year II (July 20, 1794)}: A law requires that all public deeds be written in the French language throughout the territory of France.}
\item {\textit{2014}: The Taubira memo states that the only \isi{diacritics} allowed in civil status records are those authorized by the French language.}
\item {\textit{2019}: The Roazhon/Rennes public prosecutor declares it is up to the elected national representation (National Assembly\is{National Assembly (France)} and Senate) to change the Tau\-bira memo.}
\item {\textit{February 13, 2020}: first reading in the National Assembly\is{National Assembly (France)} of MP Paul Mo\-lac’s bill on the heritage protection and promotion of \isi{regional languages}. The main provisions are scaled back, except the registration of names spell\-ed in \isi{regional languages} on birth certificates.} 
\item {\textit{December 10, 2020}: The French Senate reinstates many proposals rejected by the National Assembly\is{National Assembly (France)} and passes a version of the bill by 253 votes to 59. In this version, all regional language \isi{diacritics} are declared admissible.}
\item {\textit{April 8, 2021}: The Minister of Education and other opponents try to introduce amendments which, if adopted, would shuttle the bill back to the Senate, but their maneuver fails. At the second reading in the National Assembly\is{National Assembly (France)}, Molac's bill is adopted as law n°2021-641 by a large majority, including many MPs of \name{President}{Macron}’s side, against their own government’s advice.}
\item {\textit{April 22, 2021}: Instigated by the Minister of Education, 61 MPs of the pro-Macron parliamentary majority refer the Molac law to the Constitutional Council.} 
\item {\textit{May 21, 2021}: Decision n°2021-818 of the Constitutional Council declares unconstitutional two articles of the Molac law: immersion education in \isi{regional languages} and the use of regional language \isi{diacritics} in civil status records.}
\item {\textit{May 29, 2021}: Important language activist and civil rights marches in \isi{Brittany} and Northern Basque country against this appeal to the Constitutional Council, but to no avail.}
\end{itemize}

\il{Breton (Modern)|)}

\is{France|)}

\il{French (Modern)|)}

\sloppy
\printbibliography[heading=subbibliography,notkeyword=this]

\end{document} 
