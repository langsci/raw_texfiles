\documentclass[output=paper]{langsci/langscibook} 
\title{Preface} 
\author{Andrew Carnie and         Diane Ohala and         Dee Hunter and         Samantha Prins and          Michael Hammond and         Luis Irizarry\affiliation{University of Arizona}
        }
\abstract{}

\begin{document}
\lehead{Andrew Carnie et al.}
\maketitle

\noindent 
The Celtic Linguistics Group was founded at the University of Arizona in Spring 2006 and has had over 25 affiliated faculty, post-docs, graduate students, and undergraduates, as well as strong community partnerships with linguists and Celtic language speakers in Ireland, Scotland, Wales and Brittany. As such, it is a broad-based and interdisciplinary collective that has emphasized the importance of community-based linguistics from its inception. The group has sponsored nine US National Science Research Grants and Supplements, which have included work in instrumental phonetics, experimental and theoretical phonology, morphology, syntax, psycholinguistics, field and documentary linguistics, and language revitalization. It has also hosted two major conferences on the Celtic languages. Work completed by group members has resulted in several doctoral dissertations, numerous books and monographs, and countless articles and conference presentations. 

Like many others, much of our research was dragged to a screeching halt with the onset of the pandemic in 2020. As a consequence, we decided that the best course of action to keep our community vibrant was to host a series of 31 virtual talks, bringing in experts in Celtic linguistics from a wide range of areas. This talk series came to be known as Foundational Approaches to Celtic Linguistics (FACL). Many of these talks can still be seen at \url{https://www.youtube.com/@foundationalapproachestoce1421}. As life returned to normal, we invited the speakers of these talks to contribute to this volume. In a few cases, we allowed some of our speakers to contribute different papers than they had presented at the time, if they had published their original FACL talks elsewhere. The result is a collection of 15 influential articles by some of the most prominent researchers in the field of Celtic linguistics.

The first five chapters of the book are devoted to syntax. Jim McCloskey opens with an empirically rich and theoretically important work on the licensing of Negative Polarity items (NPIs) in Irish (\textit{Polarity sensitivity and fragments in Irish}). Irish has the typologically unique property that NPIs can serve as fragment answers to questions. McCloskey shows that this property falls out from the fact that Irish allows a narrative fronting operation, which in turn gives rise to a sluice-like construction that looks like a fragment. Melanie Jouitteau’s contribution (\textit{When linearization triggers embedded V2: Evidence from Breton}) also deals with stylistic fronting operations and their interaction with Tense-second (T2) orders in embedded contexts, this time in Breton. She compares word order effects in different dialects and shows that the relevant data emerge from the fact that T2 linearization is a post-syntactic effect dependent upon prosody. The derivation of Verb Initial order is the focus of Gillian Ramchand’s \textit{Deriving VSO in Scottish Gaelic}. On the basis of data from pronoun and adverbial placement, she proposes a novel direct linearization algorithm that is sensitive to prosodic patterning. Gary Thoms’ work (\textit{Reassessing Oehrle effects: Evidence from Scottish Gaelic}) uses the fact that Gaelic completely lacks Oehrle effects to show that this cannot be due to semantic incompatibility. He argues that their absence in Gaelic is due to the height of subjects and their interaction with shifted objects. Maggie Tallerman’s \textit{Some complexities and idiosyncrasies of Welsh consonant mutation} ably transitions us between the syntax and phonology sections of the book. She champions the idea that initial consonant mutation cannot be solely characterized by triggers and explores mutations without triggers, with unrealized triggers, and with exceptions in loan words.

The second part of the book highlights critical issues in the phonetics and phonology of the Celtic languages. \textit{Phonology of the Celtic languages} by Michael Hammond provides a comprehensive overview of the phonology of each of the Modern Celtic languages, comparing and synthesizing their similarities and differences. Ian Clayton and Cynthia Shuken’s \textit{Hebrides English in the 1980s and now} has a narrower focus on a variety of English that is heavily influenced by contact with Scottish Gaelic. Their chapter investigates the use of preaspiration, devoicing, and glottalization in Hebrides English and explores how these features have changed in the past 45 years. Of notable interest are the status of these features over time as either waxing or waning in prestige depending on the speaker population. The final contribution in this section, \textit{Gestural timing and contrast: An Irish case study} by Ryan Bennet, Jaye Padgett, Grant McGuire, Máire Ní Chiosáin, and Jennifer Bellik, examines in great detail the extensive secondary articulation contrasts of palatalization and velarization found in Irish stop consonants. Specifically, the chapter analyzes the timing of the gestural properties of these consonants as a function of syllable position and place of articulation and ably demonstrates that gestural timing is optimized to both maximize the contrasts and make them recoverable. 

Section 3 is focused on issues in language change, diachronic linguistics and grammaticalization. Brian Ó Broin’s work, \textit{Comparing the syntactic complexity of Gaeltacht and urban Irish-Language broadcasters}, uses the Rosenberg and Abbeduto complexity scale to examine the differences between traditional rural Gaeltacht communities and those of fluent speakers in urban areas. He shows that urban broadcasters are simplifying the grammar by moving from hypotactic to paratactic constructions. Caoimhín Ó Donnaíle’s chapter, \textit{Bunadas: A network database of cognate words with emphasis on Celtic}, focuses on the Bunadas tool that he has developed for investigating etymological links between words in the different Celtic languages. As such, Ó Donnaíle’s chapter provides a uniquely practical contribution to this volume. In \textit{Laryngeal contrast in the Celtic languages: Variation, typology and history}, Pavel Iosad examines distinctions among the stop series of consonants in the Celtic languages over time to evaluate the likeliest onset of the present day aspirating system. The chapter provides an extensive synchronic and diachronic analysis of the relevant data that ultimately argues for an initial voicing-based system among the stop series, which  gives way to the aspirating system at a point of key phonological change. Marieke Meelen’s \textit{Syntactic Reconstruction in Celtic} also focuses on reconstruction, but here takes on the notoriously tricky topic of syntactic reconstruction. She argues that recent generative approaches, such as the Parametric Comparison Method, shed light on the emergence of verb initial orders.

The final section of the book investigates sociolinguistic aspects of Celtic linguistics, with chapters on both language documentation efforts and critical issues in language use. Gordon Wells' chapter (\textit{Guth Thormoid: The “Island voice” of Norman MacLean}) provides a case study that highlights the exceptional value of working closely with an experienced native speaker to not only provide linguistic data but to document vanishing cultures and values also impacted with language loss. John Powell’s chapter (\textit{White supremacists’ weaponization of Celtic heritage languages}) provides an unsettling but necessary look into the minds of white supremacists who co-opt Celtic languages, language revitalization, and cultural documentation to further their own agendas. As such, the chapter is a clarion call to academic researchers about the need to pay careful attention to the gross misuse of the languages we treasure and the work we do. Stefan Moal’s \textit{France’s war on Breton diacritics: An incomprehensible obstinacy} closes the book with a thorough history of the politicization of diacritic use in minority language names in France. Moal's work underscores how the systematic suppression of Breton and other minority languages can result from unnecessary but deliberate governmental regulation of key orthographic features. 

As can be seen, the chapters in this volume vary greatly in terms of their theoretical orientation, methodologies, and scope. This aptly reflects our group's continued commitment to a multidimensional and interdisciplinary perspective on the study of the Celtic languages as a means of providing the greatest understanding, support, and impact. Thus, some chapters offer summaries or reviews in particular domains of Celtic linguistics, whereas others are technical descriptions of specific methodologies and developmental projects. Still others are in-depth reports on original research or analyses of critical issues in the sub-fields of Celtic linguistics. This variety in approach showcases the diversity of the field and mirrors that of the vibrant research being conducted on the Celtic languages.

\cleardoublepage
\end{document}
