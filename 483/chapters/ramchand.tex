\documentclass[output=paper,colorlinks,citecolor=brown]{langscibook}
\ChapterDOI{10.5281/zenodo.15654861}

\author{Gillian Catriona Ramchand\affiliation{UiT The Arctic University of Norway\\Oxford University}}
\title{Deriving VSO in Scottish Gaelic}
\abstract{In this chapter, I argue that standard approaches to word order typology in deriving the detailed properties of word order in Scottish Gaelic are too coarse and leave many properties unexplained. Data is drawn from adverb placement, \isi{copular clauses} and light pronouns to highlight the empirical issues. I argue that a direct linearization approach to the mapping between syntactic representations and serial order is well motivated both empirically and architecturally. \isi{direct linearization} contrasts with the family of approaches that places much of the typological burden on operations within the syntax itself in the form of “word order movements” (head movement, roll-up movements, EPP etc.). In the final sections of the chapter, I propose a novel \isi{direct linearization} algorithm for Scottish Gaelic which is intended to open a theoretical conversation about what such a theory might look like, and how such theories might be explored.}

\IfFileExists{../localcommands.tex}{
   \usepackage{langsci-optional}
\usepackage{langsci-gb4e}
\usepackage{langsci-lgr}

\usepackage{listings}
\lstset{basicstyle=\ttfamily,tabsize=2,breaklines=true}

%added by author
% \usepackage{tipa}
\usepackage{multirow}
\graphicspath{{figures/}}
\usepackage{langsci-branding}

   
\newcommand{\sent}{\enumsentence}
\newcommand{\sents}{\eenumsentence}
\let\citeasnoun\citet

\renewcommand{\lsCoverTitleFont}[1]{\sffamily\addfontfeatures{Scale=MatchUppercase}\fontsize{44pt}{16mm}\selectfont #1}
  
   %% hyphenation points for line breaks
%% Normally, automatic hyphenation in LaTeX is very good
%% If a word is mis-hyphenated, add it to this file
%%
%% add information to TeX file before \begin{document} with:
%% %% hyphenation points for line breaks
%% Normally, automatic hyphenation in LaTeX is very good
%% If a word is mis-hyphenated, add it to this file
%%
%% add information to TeX file before \begin{document} with:
%% %% hyphenation points for line breaks
%% Normally, automatic hyphenation in LaTeX is very good
%% If a word is mis-hyphenated, add it to this file
%%
%% add information to TeX file before \begin{document} with:
%% \include{localhyphenation}
\hyphenation{
affri-ca-te
affri-ca-tes
an-no-tated
com-ple-ments
com-po-si-tio-na-li-ty
non-com-po-si-tio-na-li-ty
Gon-zá-lez
out-side
Ri-chárd
se-man-tics
STREU-SLE
Tie-de-mann
}
\hyphenation{
affri-ca-te
affri-ca-tes
an-no-tated
com-ple-ments
com-po-si-tio-na-li-ty
non-com-po-si-tio-na-li-ty
Gon-zá-lez
out-side
Ri-chárd
se-man-tics
STREU-SLE
Tie-de-mann
}
\hyphenation{
affri-ca-te
affri-ca-tes
an-no-tated
com-ple-ments
com-po-si-tio-na-li-ty
non-com-po-si-tio-na-li-ty
Gon-zá-lez
out-side
Ri-chárd
se-man-tics
STREU-SLE
Tie-de-mann
}
   \boolfalse{bookcompile}
   \togglepaper[3]%%chapternumber
}{}


\begin{document}
\judgewidth{??}
\AffiliationsWithoutIndexing
\maketitle
\il{Scottish Gaelic (Modern)|(}


\section{Introduction}\label{sec:ramchand:1}
The leading intuition behind this chapter is that hierarchical representations are qualitatively different from the representations of linear order imposed by sensorimotor systems, and further, that syntactic implementations that build in a deterministic mapping between  hierarchical or compositional relationships and sequencing are both descriptively and explanatorily inadequate. While the relationship between hierarchy and order is systematic \textit{for a particular language}, and of necessity learnable, the relationship between the internal representations and their externalization is variable from language to language, and  not one that falls out automatically from the shapes of the graphs we draw to represent the former.  \citet{berwickchomsky11} define externalization as:


\begin{quotation}
“Externalization~-- the translation of the hierarchical and recursive representations characteristic of syntax and semantics into the kinds of serial representations that the sensorimotor systems can manipulate"  (\citealt{bennettetal16},  on the \citealt{berwickchomsky11} notion of Externalization) 
\end{quotation}

In this context, I take the order of words in a particular language to be a feature of externalization, not of hierarchical representation. Under this view, `syntax´  refers  to representations of hierarchy, recursion and categorization; correspondingly, word order facts are not to be generated by the syntactic component of a grammar, but are evidence for the nature of an interface (in this case, language-specific though within broad constraints), since word order is just the output of a mapping from syntax to linear representation. 

While most syntacticians would not reject the logical separation of hierarchy and word order, I would claim that working practice has not consistently operated with that intuition. We have been implicitly constrained by the graphical representations that we are accustomed to, and our analytic hypothesis space has been narrowed to favour solutions that embody universal interpretations of hierarchical representations in terms of ordering. In this exploratory and speculative chapter, I attempt to show that current approaches to word order generalizations in Scottish Gaelic (a \isi{VSO} language) are deeply unsatisfactory. I propose a different implementation of the word order patterns that takes seriously the generation of word order as an output of a linearization algorithm without the intermediate step of displacement implemented in the syntax. I take the non-standard flavour of these exploratory solutions as evidence that they do \textit{not} currently figure in our notional hypothesis space for these kinds of phenomena. 

Typological work has long noted the systematic ordering differences found among the world's languages, and the implicational patterns that seem to be found there, suggesting some underlying system of universal constraints (\citealt{greenberg66sug}). I begin by summarizing what I take to be the classical theoretical approaches to accounting for cross linguistic variation in base word order, as found in the  literature. 

Even with respect to the descriptive facts underlying something like Greenberg's Universal 20, one can broadly distinguish two  major approaches:

 
\begin{itemize}
\item[(i)] \textit{Head ordering parameters}: {Parametrize head-complement order directly (possibly on a category by category basis)} (\citealt{haider00ovmb, abelsneeleman12} on Universal 20)\footnote{Base generation accounts do not eschew movement operations in the syntax entirely, but operate with a more restricted and triggered set of movement operations than Kayneian accounts.}
\item[(ii)]\textit{LCA-style universal mapping}: {Assume a common phrase structural functional sequence, with a universal ordering (Kayne's LCA\is{Linear Correspondence Axiom}) and para\-met\-rize the choice of the kinds of movements that occur to distort base SVO phrase structures} (\citealt{kayne94as, koopmanszabolcsi00, cinque04} on Universal 20 ). 
\end{itemize}

To this, we can add a third option, which I call differences in \isi{direct linearization}.

\begin{itemize}
\item[(iii)]\textit{\isi{direct linearization}}: {Assume a common relative hierarchical system across languages, plus a learnable linearization algorithm which is language specific.}
\end{itemize}

The difference between \isi{direct linearization} and the head-order parameter approach is that it is  \textit{not}  assumed that the linearization algorithm is just confined to a parametric choice of head-complement order. The salient difference with LCA\is{Linear Correspondence Axiom}-style approaches is that it does not use any operations in the syntax designed purely to affect word order. This includes word order movements such as EPP movements, head movements, remnant movements, roll-up movements, etc. 

\section{Background on Scottish Gaelic word order}\label{sec:ramchand:2}
There is a respectable body of early (pre-LCA\is{Linear Correspondence Axiom}) work  on deriving word order for  \isi{VSO} languages, including Scottish Gaelic and Irish\il{Irish (Modern)}. After the extremely common SVO and SOV languages, \isi{VSO} is the next most common typological pattern with about 10 to 12 percent of attested world languages. When it comes to the head parameter however, \isi{VSO} (and the less common \isi{VOS}) are both considered subtypes of the head-initial setting for this parameter (\cite{cinque17}). Cinque  even considers \isi{VOS} to be the most pure, consistent word order type for the head-initial parameter setting, albeit noting that such `consistency' seems to be cross-linguistically rare. In order to get the subject after the verb to derive the \isi{VSO} preference from a base Spec-Head-Complement phrase structure, a number of different proposals have been made over the years, including a flat structure without a VP (\citealt{mccloskey:79, chung83}), lowering of the subject from a spec-IP position (\citealt{chung90, shlonsky97} for Arabic), and raising of the verb to C as in V2 (\citealt{emonds80, carnieharleypyatt00} for \ili{Old Irish}). 
The current consensus is that \isi{VSO} languages like Scottish Gaelic and Irish\il{Irish (Modern)} do indeed have a VP, the subject remains in situ in a vP-internal or AspP internal position, and that the verb raises to some head position at the top of the inflectional zone higher than the the subject (\citealt{mccloskey:90, mccloskey96svmi, guilfoyle90, noonan94}, inter alia). See \citealt{carnieguilfoyle00intro} for further discussion. 

If we take the  embedded clause in Scottish Gaelic in (\ref{ex:r1}) as an example, we see that the complementizer and the past tense particle precede the verb in the embedded clause, which is then immediately followed by the subject. 

\ea\label{ex:r1}
\gll Thuirt Calum [gun {do leugh} mi an leabhar].\\
say.\textsc{pst} Calum that read.\textsc{pst.dep}(V) I(S) the book(O) \\
\glt 'Calum said that I read the book.'
\z

\noindent This word order in the embedded clause is derived from a base generated SVO structure plus head movement of the verb to some position higher than the subject, as shown in the following tree. In this implementation, Fin, the lowest projection in the cartographic decomposition of the C layer is assumed to be the landing site for verb movement.  The precise label for the head that is the landing site of verb movement here is not crucial; it merely needs to be a functional projection that is above the final position of the subject (here assumed to be Spec, TP) (\figref{ex:r2}).

\begin{figure}[h]
\Tree [.{ForceP}  [.{Force}  {gun}  ] [   [.{Fin1}  {do} ]  [ [.{Fin2}  {leugh}  ] [.{TP}  [.{spec}  {mi}  ] [  [.{T}  ] [.{AspP/vP}   \edge[roof]; {\sout{leugh} an leabhar}  ] ] ] ] ] ]
\caption{Tree for the embedded clause in (\ref{ex:r1})}
\label{ex:r2}
\end{figure}

\noindent The relevant research here comes from Irish\il{Irish (Modern)} (\citealt{carnieharleypyatt00, carnie96, mccloskey91, mccloskey96svmi, mccloskey05, noonan94}).
 The standard account assumes further that there \textit{is} a VP constituent headed by the verbal noun, and that the genitive marked object also remains low within the VP.  In some accounts, a higher functional projection is argued to host the direct case marked object in non-finite and perfect aspectual constructions. In \citet{ramchand97} this is claimed to be AspP, while in \citet{adger94} it is AgrOP since it co-occurs with agreement on the verbal noun. However, these details will be irrelevant for the problems I raise for the standard account below.

In addition to the position of the subject, there are a number of typological properties of the \isi{VSO}/\isi{VOS} family of languages that have been noted in the literature which set them apart from the other dominant head\hyp initial type, SVO.   Apart from the Greenbergian properties of postnominal adjectives, and the existence of prepositions rather than postpositions which go straightforwardly together with head\hyp initiality, \isi{VSO} languages all tend to have 
\begin{enumerate}
\item[(i)] preverbal particles for higher functional elements in the verbal extended projection such as tense, mood, aspect, negation; 
\item[(ii)] inflected prepositions \citep{kayne94as}; 
\item[(iii)] left conjunct agreement \citep{Doron2000}; 
\item[(iv)] lack of a verb `have' \citep{freezegeorgopolous00}; 
\item[(v)] \isi{copular clauses} without verbs \citep{carnie96}; 
\item[(vi)] “verbal noun”  infinitives \citep{Myhill1985}, as listed in \citet{carnieguilfoyle00intro}. 
\end{enumerate}
In addition, most \isi{VSO} languages have \isi{VOS} as a possible word order alongside \isi{VSO} (\citealt{clemenspolinsky13}). At this point it is unclear whether, or how, these properties can be subsumed under a single property that unites \isi{VSO} languages.

By all of the above criteria, Scottish Gaelic is a well behaved and typical \isi{VSO} language, except for the possibility of \isi{VOS} as an alternate word order for \isi{VSO} order in regular clauses. I illustrate some of these basic word order properties below. 

In (\ref{ex:r3}) we see an example of a standard matrix clause with a transitive verb inflected for tense.  The word order here is obligatorily \isi{VSO}, with no \isi{VOS} word order option available. 

\ea\label{ex:r3}
\gll Leugh mi an leabhar./(*Leugh an leabhar mi).\\
Read.\textsc{pst} I the book\\
\glt `I read the book.'
\z

In addition, Scottish Gaelic possesses auxiliary constructions in which the verbal information is split across a main verb carrying the event-descriptive content, and an auxiliary which bears tense and agreement (when it exists) information. In this kind of construction, it is the tensed auxiliary that is in the `first' position, while the main verb (in the form a verbal noun~-- glossed as \textsc{vn}) stays low after the subject.
\ea\label{ex:r4}
\gll Tha  mi  a' leughadh  an leabhair. \\
be.\textsc{prs}  I  \textsc{Asp}$_{ag}$ read.\textsc{vn}  the book.\textsc{gen} \\
\glt 'I am reading the book.'
\z

While the position of subject and object always remains relatively fixed, objects  either precede or follow the verbal noun in periphrastic constructions depending on aspect and finiteness: when they follow the verbal noun in the progressive aspect, they immediately follow it and are in genitive case (as we see in fact in (\ref{ex:r4}). When they precede it as in infinitives or the perfect and prospective aspects, they take direct (Nom/Acc) case. An example of this for the perfect aspect is shown in (\ref{ex:r5}) and for an infinitival clause in (\ref{ex:r6}). 

\ea\label{ex:r5}
\gll Tha  mi  air  an leabhar seo  a  leughadh. \\
be.\textsc{prs}  I  \textsc{Asp}$_{air}$  the book this  \textsc{AgrO} read.\textsc{vn} \\
\glt `I have read this book.
\ex\label{ex:r6}
\gll Bu thoil  leam  an leabhar seo  a leughadh.\\
\textsc{cop.}\textsc{pst} pleasure  with.me  the book this  \textsc{AgrO} read.\textsc{vn} \\
\glt `I would like to read this book.'
\z

Congruent with head-initiality, Gaelic has a wealth of prepositions, but no postpositions. It also has inflected prepositions, which is one of the typological properties listed above for \isi{VSO} languages more generally. Given that this is rather unusual, in (\ref{ex:r7}) I show the paradigm for the “inflected” preposition \textit{le} `with' as an example of this phenomenon. 

\ea\label{ex:r7}
The preposition \textit{le} `with':\smallskip\\
\begin{tabular}{@{}ll@{}}
\textit{leam} & `with me' \\
\textit{leat} & `with you' \\
\textit{leis} & `with him' \\
\textit{leatha} & `with her' \\
\textit{leinn} & `with us' \\
\textit{leibh} & `with you.\textsc{pl}'\\
\textit{leotha} & `with them' \\
\end{tabular}
\z

With respect to the nominal extended projection, Scottish Gaelic is D/N initial, and has no pre-nominal modifiers apart from one or two special adjectives. It has neither pre-nominal relatives nor pre-nominal genitive phrases. The base order is therefore: Det N A Poss as shown in (\ref{ex:r8}). 

\ea\label{ex:r8}
\gll taigh  geal  Chaluim \\
house  white.\textsc{m}  Calum.\textsc{gen} \\
\glt `Calum's white house'
\z

The demonstrative consists of a post nominal phrase particle in combination with the initial determiner (\ref{ex:r9})

\ea\label{ex:r9}
\gll an taigh seo \\
\textsc{det} house \textsc{dem} \\
\glt `This house' 
\z

When it comes to \textit{wh}-question formation, Gaelic again behaves as expected of a head-initial language, in always having the \textit{wh}-pronoun (exemplified in (\ref{ex:r10}) by \textit{d\'e} `what' ) in initial position. 


\ea\label{ex:r10}
\gll D\`{e}  a leugh  thu  a-raoir? \\
What  \textsc{COMP$_{rel}$} read.\textsc{pst}  you  yesterday \\
\glt `What did you read yesterday?'
\z

In addition, conforming to the verb-initial property claimed by \citet{freezegeorgopolous00}, Scottish Gaelic has no lexical verb `have'. In (\ref{ex:r11}), we see a sentence expressing possession. In Scottish Gaelic, this looks like a locative sentence using `be', with the possessed element in subject position, and the possessor the complement of the preposition \textit{ag} `at'.

\ea\label{ex:r11}
\gll Tha  leabhar  agam. \\
be.\textsc{prs}  book  at.\textsc{1s} \\
\glt `I have a book (The book is at-me).'
\z

As already noted,  Scottish Gaelic has no \isi{VOS} alternative alongside \isi{VSO} for ordinary verbs,  but it is obligatorily subject-final for stative non-verbal predications involving the copula, as shown in (\ref{ex:r12}) and (\ref{ex:r13}). The correct phrase structure order for this construction  is non trivial, given the final position of the subject, and the correct analysis is still controversial, but see \citet{adgerramchand03} for a proposal. 

\ea\label{ex:r12}
\gll Is  tidsear  Calum. \\
\textsc{cop.}\textsc{prs}  teacher  Calum \\
\glt `Calum is a teacher.'
\ex\label{ex:r13}
\gll Is  leamsa  an c\`{o}ta seo. \\
\textsc{cop.}\textsc{prs}  with.1s.\textsc{emph}  the coat this \\
\glt `This coat is mine.'
\z

In short, Scottish Gaelic is a typologically “normal” member of the \isi{VSO} family of languages by most diagnostics. This makes it plausible that any issues that arise in the discussion of how word order is generated for Gaelic will generalize to other languages with this basic profile. Having said that, this is not intended to be a chapter that examines the properties of \isi{VSO} more generally, since it focuses empirically only on Gaelic (with some reference to established facts concerning the closely related Irish\il{Irish (Modern)} (also Goidelic) language. I argue that there are serious descriptive problems in characterizing the word order of Gaelic within a standard set of syntactic assumptions, even though these languages are classically considered to be one of the success stories of the head movement approach to verb placement. 

Summarizing the consensus briefly once again, it is often claimed for Scottish Gaelic that there is a VP that contains both the subject and the object, and that the tensed verb moves to a higher functional head in the inflectional domain to give verb-initial order. Higher functional positions host particles such as negation and complementizers which precede the lexical verb forming an integrated prosodic verbal complex. Auxiliaries base generated in the inflectional zone are also attracted to the high inflectional position, while the verbal noun spells out the verbal position within the VP.  Functional material in the middle inflectional zone such as Asp or Agr have  been proposed to host preposed objects in infinitival and perfect aspectual constructions.   


Within a framework that assumes the LCA\is{Linear Correspondence Axiom}, \isi{VOS}/\isi{VSO} is considered to be one of the most consistent versions of a descriptively head-initial parameter setting which regularly produces word orders that orders phrasal elements to the right of the head in reverse hierarchical order, and functional elements before the lexical head in straight LCA\is{Linear Correspondence Axiom} compliant order (\citealt{cinque17}). In the LCA\is{Linear Correspondence Axiom} compliant approach, this is delivered by roll-up movements of phrases including V involving pied piping of the \textit{whose}-pictures type (detailed derivations given in \citealt{cinque17}).  My position in this chapter is to reject a framework that involves word order movements and explore a \isi{direct linearization} approach, so I will not detail those derivations here. However, it is worth pointing out that such derivations, if implemented consistently for Scottish Gaelic would predict at least the possibility of ordering the direct object before the subject in a standard transitive clause. While, as we will see, adverbial modifiers do indeed stack up in reverse hierarchical order in Gaelic, the same is not true of arguments. This means that a principled distinction needs to be made between arguments and phrasal modifiers in the phrase structure in order to even state the pattern. In many cartographic approaches, null functional heads are assumed to host phrasal modifiers in their specifiers (\citealt{cinque99}), making them phrase structurally indistinguishable from A-positions which host arguments. An algorithm which generates movements based on a certain kind of pied-piping will affect both arguments and adjuncts equally unless a principled distinction is made between functional heads of a “core” projection that introduce arguments and distinct functional heads introducing  adjuncts. Something like this is what \citet{cinque17} suggests. At any rate, it is not the case that the problems for the head ordering parameter account can be solved by embracing the LCA\is{Linear Correspondence Axiom} and word order movements. 

While the assumptions (head-initiality, or its LCA\is{Linear Correspondence Axiom} equivalent, plus head movement) go a long way towards describing most of the major clause types of  Gaelic, there remain a few knotty problems and word order patterns that are more difficult to account for with this simple set of parametric choices. 

In the section that follows, I lay out some empirical challenges for the standard accounts. The first set of problems involves the placement and scope of adverbials, with some differences for simple vs. auxiliary verb constructions. The second problem involves \isi{copular clauses}. The third problem involves the placement of \isi{light object pronouns} (a problem already discussed by \citealt{bennettelfnermccloskey13} for Irish\il{Irish (Modern)}). The purpose of this section is to emphasize that word order has not yet been “solved” for any of the non Spec-Head-Complement languages, let alone for Gaelic. It is not simply a matter of the linguist being forced to choose between two different kinds of implementations with slightly different universalist implications; the patterns here are going to require extra or different mechanisms to state. They therefore invite us to rethink the architectural model and the relationship between syntax and linearization, and to experiment with different ways of stating the generalizations we find. 

\ea
Problems:
\ea Extra stipulations are required to ensure that none of the specifiers of those higher heads can be filled, plus ad hoc prohibitions on adjunction.
\ex Copula clause word order.
\ex Weak pronoun placement.
\z
\z

\section{Problems for the standard account}\label{sec:ramchand:3}

While the set of assumptions discussed in \sectref{sec:ramchand:2} correctly predicts the height of the verb with respect to subject and object in main and subordinate clauses, it has nothing to say about the positioning of adverbials and other adjoined phrases unless we subscribe to a view where adverbials are rigidly and templatically ordered.  Scottish Gaelic is discernibly different from both V2 languages and language such as English, when it comes to the possible  positions of adverbials. Let us take a broad look at the data to see whether the templatic account, combined with the  ``high'' position of the verb makes the right predictions. 



\subsection{Adverb placement}

For the purpose of illustration, I concentrate on the two broad classes of adverbials established in the literature VP-adverbials and sentential adverbials which have different syntactic position and semantic scope (\citealt{bellert77, jackendoff72}).\footnote{Of course, in the theoretical literature since the seventies many more types of adverbials have been proposed together with finer grained structural positioning and syntactic scope (\citealt{cinque99}). However, the more subtle the distinctions, the more difficult it is to establish clear ordering judgments. I concentrate on two broad syn/sem classes here because the problem can be more robustly demonstrated with them.}
Adverbials that describe the manner of the action of the verb can be adjoined to the VP in English, while evaluative or speaker oriented adverbs tend to take the whole sentence in their scope and semantically modify whole propositions. This gives rise to the asymmetry in ordering positions shown in (\ref{ex:r14}). 

\ea\label{ex:r14}
\ea[]{Fortunately, John  climbed the tree carefully.}
\ex[*]{Carefully, John climbed the tree fortunately.}
\z
\z

While a sentence adverbial can be left peripheral when the manner adverbial is right peripheral,  the opposite does not sound very good in English.   In addition, English seems to have a third possible position for adverbials in between the subject and the tensed verb. Subject-oriented adverbs such as \textit{cleverly} seem particularly felicitous in this position, although all types of adverbs seem to be able to appear linearly in here with the right intonation. In general, however, there seems to be a   hierarchical ordering of semantic adverb types  that finds cross\hyp linguistic support. In the following hierarchy, I confine myself just to the three basic old fashioned types, ignoring the more articulated proposals of e.g. \citet{cinque99}. 

\ea\label{ex:r15}
Speaker oriented adverbs $<$ Subject oriented adverbs $<$  Manner adverbs.
(\citealt{jackendoff72}, \citealt{alexiadou97},  \citealt{sportiche88}). 
\z

If we take a manner adverbial\footnote{The following adverbial data from Scottish Gaelic comes courtesy the University of Arizona archive of Scottish Gaelic data, and comes from the results of a number of targeted questionnaires on clausal ordering. Thanks to Andrew Carnie (PI) and Muriel Fisher (primary consultant) for this resource.} such as \textit{'s a'bhad}, `quickly' in Scottish Gaelic, we find that the most natural position for it is in fact clause finally, as shown in (\ref{ex:r16}).

\ea\label{ex:r16}
\gll Dh'ith	Iain  an	ubhal	{'s 		a'bhad.}  \\
eat.\textsc{pst}   Iain  the.\textsc{fsg}	 apple.\textsc{fsg}  quickly \\
\glt `Iain ate the apple quickly.'
\z

While it is possible to place a manner adverb in a clause initial position in English, with suitable comma intonation, as shown in (\ref{ex:r17}). 

\ea\label{ex:r17} Quickly, Iain ate the apple.
\z

\noindent The equivalent ordering is unacceptable in Scottish Gaelic.

\ea\label{ex:r18}
\ea[*]{'S  a'bhad dh'ith Iain an ubhal.} 
\ex[*]{Dh'ith  's a'bhad Iain an ubhal.}
\ex[*]{Dh'ith  Iain 's a'bhad  an ubhal.}
\z
\z

This is an instance in which Gaelic's preference for verb initiality seems to go beyond a simple requirement that the verb be ordered before the subject. However, if we change the adverb to a speaker\hyp oriented evaluative sentence adverb, then the sentence initial position does become felicitous in Scottish Gaelic, with what sounds like two separate phonological phrases being formed.  The sentence final position is also grammatical for subject-oriented adverbs. 
Finally, for speaker\hyp oriented (high) adverbials, we find that the initial position is possible, with appropriate intonation (\ref{ex:r19}), while the sentence final position is once again quite natural (\ref{ex:r20}). 

\ea\label{ex:r19}
\gll \textit{Gu	mi-fhortanach,} dh'ith	Iain	 an	ubhal. \\
unfortunately eat.\textsc{pst}	Iain	 the.\textsc{fsg}	apple.\textsc{fsg} \\
\glt `Unfortunately, John ate the apple.'
\z

\ea\label{ex:r20}
\gll Dh'ith	Iain	 an	ubhal \textit{gu	mi-fhortanach}. \\
eat.\textsc{pst}	Iain	 the.\textsc{fsg}	apple.\textsc{fsg}	 unfortunately \\
\glt `John ate the apple, unfortunately.'
\z

One very strong constraint is that in Scottish Gaelic, it is not possible to place an adverb in between the subject and the direct object, regardless of whether that adverb is a manner adverb as in (\ref{ex:r21}), or a speaker oriented adverb as in (\ref{ex:r22}). 

\ea[*]{\label{ex:r21}
\gll Dh'ith	Iain	{\textit{'s a bhad}}  	an	ubhal. \\
eat.\textsc{pst} Iain  quickly  the.\textsc{fsg} apple.\textsc{fsg}\\}
\ex[*]{\label{ex:r22}
\gll Dh'ith	Iain	{\textit{gu	mi-fhortanach}}  	an	ubhal.\\
eat.\textsc{pst} Iain  unfortunately  the.\textsc{fsg} apple.\textsc{fsg} \\}
\z

Interestingly, when the two types of adverbs combine we get an asymmetry. The “low” VP adverbial needs to be expressed before the the “high” one, in a left to right order that tracks the cross\hyp linguistically attested hierarchy, as shown in (\ref{ex:r23}). 

\ea\label{ex:r23}
\gll Dh'ith Iain an ubhal {\textit{'s a bhad}}, {\textit{gu mi-fhortanach}} \\
Eat.\textsc{pst} Iain the apple  quickly unfortunately \\
\glt `Iain ate the apple quickly, unfortunately.'
\z

\noindent As we have noted previously, the reverse hierarchical order of adjuncts is a common feature of many head initial languages and is usually handled via roll-up movements and \textit{whose}-pictures pied-piping.


So far, the adverbs have been sentence peripheral, but there is a class of construction types where we do find the possibility of a sentence-medial adverbial position. This occurs when you have an auxiliary construction, with the tensed verbal element being distinct from the contentful verb expressed as a non-finite verbal noun/participle. 

\ea
\label{ex:r24}
\gll Bhiodh bana-bhuidsich {\textit{gu tric/gu slaodach}} a'   briseadh nan sguaban aca. \\
be.\textsc{cond} witches often/slowly \textsc{prog}  break.\textsc{vn} the broomsticks at.\textsc{3pl} \\
\glt `Witches would often/slowly break their broomsticks.'
\z

\noindent The position in between the subject and the aspectual phrase is possible for these adverbials but no adverbials seem to be possible in the stretch between the aspectual particle and the verbal noun, or between a verbal noun and its direct object in the genitive. 

\ea\label{ex:r25}
\gll Tha oileanach {(\textit{gu tric})} air {(*\textit{gu tric})} na leabhraichean  sin {(*\textit{gu tric})} a reic {\textit{gu tric}} {ann a shin}.  \\
be.\textsc{prs} students (often) \textsc{perf} (often) det.\textsc{dir.pl}  book.\textsc{pl}  those (often) \textsc{prt} sell.\textsc{vn} (often) there \\
\glt `Students have often sold books there.' (from \citealt{adger10e})
\z


\noindent The position between the tensed verb and the subject is also quite restricted. There are however some constructions where it has been argued that the grammatical subject stays low within the VP (for example, in existential constructions), and in those cases, a suitably scoping adverb is possible before the subject (\ref{ex:r26}). 

\ea\label{ex:r26}
\gll Tha {\textit{gu tric}} oileanaich  {\textit{gu tric}} 'san leabharlann.\\
be.\textsc{prs} often students often in.the library \\
\glt `Students are often in the library.'
\z

In contrast, the classical subject position (the specifier of TP) cannot be separated from the tensed element by adverbs or by parentheticals.

\ea\label{ex:r27}
\gll Tha {(*\textit{gu tric})} Calum {\textit{gu tric}}  'san leabharlann \\
be.\textsc{prs} (often) Calum often in.the library \\
\glt `Calum is often in the library.'
\ex\label{ex:r28}
\gll Tha {(*\textit{tha mi cinnteach})} Calum {\textit{gu tric}}  'san leabharlann \\
be.\textsc{prs} ({I am sure}) Calum  often  in.the library \\
\glt `Calum is often in the library.'
\z

For Irish\il{Irish (Modern)}, \citet{Elfner2015}  shows  that in sentences such as (\ref{ex:r29}), there is little or no evidence for a prosodic boundary between the Verb and the Subject, while there is strong evidence for a prosodic boundary between the Subject and the Object.

\ea\label{ex:r29}
\gll Chonaic M\'{a}ire Se\'{a}n.\\
Saw Mary John\\
\glt`Mary saw John.' (Irish\il{Irish (Modern)})
\z

The facts about adverbial placement in Scottish Gaelic summarized here seem to give  evidence for hierarchy, since different scopal classes of adverbs  have different adjunction possibilities. Mid and high adverbs like  \textit{gu mi-fhortanach} `unfortunately'  can be both initial and final, while low (manner) adverbials can only be final in a simple clause. When these classes of adverbs co-occur in final position, they occur in reverse hierarchical order with manner adverbs preceding subject oriented, preceding speaker oriented adverbials. In medial position, which is only possible with periphrastic (auxiliary) constructions, both mid and low adverbials can occur between the subject and the aspectual particle.

Even though there are clear patterns here, it is not clear how to model them without stipulating in some way  \textit{which} projections with which directionality are allowed for adverbial adjunction/modification, or how to allow optionality in the absence of semantic effects.  The head-parameter (or its implemented equivalent in the type of pied piping required in roll-up movements) is also not well suited to making a difference between adjuncts and arguments if they are both phrases in specifier positions, or in preventing certain classes of adverbials from e.g., adjoining to the `left' of the VP they modify, as is the case here with manner adverbials in  Gaelic. 
Quite generally, it is not considered to be the job of word order typological theory to regulate the choices and preferences for elements like adjuncts that have a freer distribution. Yet, we do find language specific preferences and requirements in this domain as well. Going back to the difference between English and Scottish Gaelic, we find that Gaelic does not generally allow adverbial elements to stack up at the beginning of a sentence. In English, the following sentence in (\ref{ex:r30}) sounds quite natural.

\ea\label{ex:r30}
Yesterday, in the bathroom, with great reluctance, John shaved his mustache off.
\z

\noindent The equivalent in Scottish Gaelic would sound very awkward if not completely ungrammatical (\ref{ex:r31}).

\ea[??]{\label{ex:r31}
\gll {An d\`{e}}, anns an se\`{o}mar-ionnlaid, gu m\`{o}r leisg, chuir Iain  dheth an mustache aige. \\
Yesterday in the bathroom with great reluctance put.\textsc{pst} Iain of.him the mustache at.him \\}
\z

\noindent{The preferred order of adjuncts here, would be for all of them to sit in the sentence final position, in reverse order compared to the English one (\ref{ex:r32})}.

\ea\label{ex:r32}
\gll Chuir Iain dheth an mustache aige gu m\`{or} leisg, anns an se\`{o}mar-ionnlaid, {an d\`{e}. } \\
Put Iain off.him the mustache at.\textsc{3sm}  with great reluctance in the bathroom {yesterday} \\
\glt `Yesterday, in the bathroom, with great reluctance, Iain shaved his mustache off.'
\z

In some sense, we could see this preference in Scottish Gaelic as an extension of its verb initial property, since the dislike of sentence initial adverbial phrases is one of the things that delivers the overwhelming preponderance of pure verb initial sentences in the language. 

As mentioned earlier,  some recent cartographic accounts propose to directly explain \isi{adverb ordering} by interpreting adjunction as the filling of a specifier of particular functional heads. Under this view, the adjunct is semantically related to the content of the assumed functional head (which may be non-overt), and moreover, the relevant functional heads are hierarchically added to the core lexical phrase in a particular deterministic (often assumed to be universal) order (\citealt{cinque99}). This way of approaching the problem gives rise to the expectation that adverbs will appear in the same relative order in all languages, and that this will appear in lockstep with the heads that represent those particular abstract meanings. It also means that a parametric setting for head initiality (a certain kind of pied-piping during roll-up) should treat arguments and adjuncts alike within a particular language. Argument positioning in Scottish Gaelic is quite distinct from, and significantly more rigid with respect to head positions than adverbial positioning is, and the relative order of subject and object is always left to right respecting hierarchical order. Adverbials on the other hand stack up in \textit{reverse} hierarchical order when they follow the verb, and left `adjunction' seems rather restricted.   If we decide on the other hand that adverbs do attach in a syntactically distinct way to arguments in specifier positions via adjunction (\citealt{roberts96, kayne94as}), then word order patterns built around arranging the linear order of head-complement or specifier-head do not necessarily bear on adverbial placement and ordering at all. We have already seen that there \textit{are} generalizations about relative \isi{adverb ordering} across languages, so relating these adjunct positions to their scopal height, or height of certain functional material seems like a promising strategy (cf. \citealt{ramchandsvenonius14} for a zonal account) but doesn't solve the adverbial ordering problem directly or explain why it is not exactly the same as the ordering of heads and arguments in some languages.  

Moreover, within  Gaelic, we have at least one hard empirical problem that militates against the functional projections solution to \isi{adverb ordering}. This is the issue of medial adverbials. As we have seen, in sentences with a simple tensed main verb, it is not possible to insert adverbials in between the subject and the object. However, if the construction is periphrastic, this suddenly becomes a possibility. (The minimally contrasting sentences are shown below). 

\newpage
\ea\label{ex:r33}
\gll Bhiodh bana-bhuidsich {\textit{gu tric/gu slaodach}} a'   briseadh nan sguaban aca.\\
be.\textsc{cond} witches often/slowly \textsc{prog}  break.\textsc{vn} the broomsticks at.\textsc{3pl}\\
\glt `Witches would often/slowly break their broomsticks.'
\ex\label{ex:r34}
\gll Bhris a'bhana-buidseach \textit{*gu tric/*gu slaodach} nan sguaban aice.\\
be.\textsc{pst} {the witch} often/slowly the broomsticks at.\textsc{3sm}.\\
\glt `The witch often/slowly broke her broomsticks.'
\z
\noindent{The problem this raises for cartographic approaches to \isi{adverb ordering} is the following. We would like to assume that the functional sequence for periphrastic and synthetic verb phrases is exactly the same within a single language, since the actual spell out of the syntactic structure should not influence this (under standard assumptions of late insertion). This means that the two sentences above should have exactly the same abstract projections ordered in the same way. It is unclear, within this framework, how one would forbid the merge and left-linearization of an adverbial in the one and not the other.}

This long digression into adverbial placement preferences in  Gaelic has been to emphasize that accounting for language particular word order facts is not confined to regulating the order of head, complement and specifier. If adjunct placement is taken into account, a host of other generalizations emerge that have to be independently stated in some way. In the final section of this contribution where I introduce my own speculative algorithms, the goal will be to bring these sorts of facts into the fold as parts of the explanandum for linearization algorithms. 

To summarize this section, getting  V-initial order as a more general output crucially relies on preventing adverbial attachment to the left of the projection that the verb raises to. 
 The preference for right adjunction of adverbials in Scottish Gaelic needs to be stipulated and doesn't currently follow from anything, but it is arguably part of the conspiracy that delivers V1 in the first place.  Moreover, Scottish Gaelic seems to be showing us that a principled distinction needs to be made between the ordering of arguments (assumed to be filling specifier positions of heads) and the ordering of adjuncts. Finally,  under a universalist phrase structure template it is hard to see how to derive the different possibilities for medial adverbs just when the phrase structure is filled out in a periphrastic way as opposed to a “synthetic” way.

\subsection{Copular clauses}

In this subsection, I describe a word order pattern within  Gaelic that has proved recalcitrant over the years, if a general theory of phrase structure is to be proposed for the language.  The clause type I am referring to is the copular clause\is{copular clauses}, shown below in (\ref{ex:r35}). In (\ref{ex:r35}), we see the tensed copula \textit{Is}, followed by the adjectival predicate \textit{mor} `big', while the subject of the sentence \textit{an duine sin} `that person'  is sentence final.

\ea\label{ex:r35}
\gll Is m\'{o}r an duine sin.\\
\textsc{cop.}\textsc{prs} big the person that \\
\glt `That person is big.'
\z

This pattern for \isi{copular clauses} is quite general, regardless of the syntactic category of the predicate. The copula and predicate phrase form a unit to the exclusion of the grammatical subject, which is final. In (\ref{ex:r36}) , we see a prepositional phrase in predicate position, showing that head movement to the high copula position cannot be the derivation that delivers this word order pattern. 

\ea\label{ex:r36}
\gll Is le Calum an c\`{u}.\\
\textsc{cop.}\textsc{prs} with Calum the dog \\
\glt `The dog is Calum's.'
\z

In (\ref{ex:r37}), we see a noun phrase in predicate position, once again showing the categorial generality of the structure and the phrasal nature of the element that immediately follows the tensed copula. 

\ea\label{ex:r37}
\gll Is toil leam an c\`{u}. \\
\textsc{cop.}\textsc{prs} liking with.1\textsc{s} the dog \\
\glt `I like the dog.'
\z

\noindent{This has had many different treatments in the literature, but the movements that would derive this order from a sensible hierarchical base structure seem to violate certain basic laws of syntactic movement, as commonly construed. For example, in \citet{carnie96}, complex predicational phrases must be allowed to ``head move'' in these constructions. That the final DP is the subject of the copular predication can be shown by the following datum on reciprocal binding in (\ref{ex:r38}) from Irish\il{Irish (Modern)} (\citealt{doherty96}).}

\ea\label{ex:r38}
\gll Is cos\'{u}il lena ch\'{e}ile  iad.\\
\textsc{cop.}\textsc{prs} like with.\textsc{3pl.poss} each.other they \\
\glt `They are similar to each other.' (Irish\il{Irish (Modern)})
\z

The standard phrase structure for declarative clauses in Gaelic has a high T position which is usually filled by an auxiliary or by head movement of a main verb from lower in the structure. In the copular construction, we find the same tense-initial position, but the equivalent of the VP looks as if it has raised past the subject to be adjacent to this tensed position. Although these constructions do not involve transitive predicates with direct objects (the copula combines with APs, PPs and NPs), the order here is reminiscent of the \isi{VOS} orders often found in verb initial languages, in having the predicate phrase before the subject. 

Once again, this is not the sort of word order problem that traditionally falls under the remit of \isi{word order typology}, or word order parameters. And perhaps it is a mistake to consider this construction type at all when characterizing the Gaelic linearization algorithm, although \textit{some} way of representing it phrase structurally needs to be found that fits our expectations of hierarchy and ultimate word order.  If this order can be made to fall out of a more general way of thinking about how to characterize Scottish Gaelic word order,  it would be a case of solving a problem that does not currently figure in any of the other implementations for Scottish Gaelic syntax to word order mappings.  

\subsection{The light pronoun problem}

Finally, I turn to yet another word order issue that has not traditionally been considered part of the \isi{word order typology}. This is a phenomenon found in Scottish Gaelic and Irish\il{Irish (Modern)} completely independently of \isi{VSO}, that raises severe problems for the relationship between hierarchical representations and linear order. I include it here because it is a flagship case for more direct approaches to linearization (i.e. ones that do not avail themselves of an intermediate stage of syntactic displacement).  Since my own account of word order facts more generally is going to be in the \isi{direct linearization} family, it will be relevant to see this last problem too in the context of the more general proposal for delivering Verb initiality. 

The light pronoun problem concerns the linearization of direct object pronouns within the VP.   To see the nature of the pattern, we need to first look again at where object nominal phrases are linearized in a standard Scottish Gaelic sentence. As we can see in (\ref{ex:r39}) below, the direct object \textit{Iain} comes right after the grammatical subject, and before the two adjunct adverbials.

\ea\label{ex:r39}
\gll Chunnaic mi Iain 's gh\'{a}rradh {a-raoir} \\
saw I Iain in.\textsc{def} garden {last night} \\
\glt `I saw Iain in the garden last night.'
\z

\ea\label{ex:r40}
\gll Chunnaic mi Iain  {a-raoir} 's gh\'{a}rradh \\
saw I Iain {last night}  in.\textsc{def} garden  \\
\glt `I saw Iain in the garden last night.'
\z

As we saw in the section on adverbials, in a sentence such as this, the direct object always precedes both adverbials, and although there is some flexibility in the order of those adverbials, the preference is for the higher scoping adverb to be rightmost. 


The situation changes however, when the direct object is expressed as a light pronoun. While the order that replicates the order found with full nominal containing phrases is fully grammatical,  as shown in (\ref{ex:r41}). 

\ea\label{ex:r41}
\gll Chunnaic mi e 's gh\'{a}rradh {a-raoir} \\
saw I him in.\textsc{def} garden {last night} \\
\glt `I saw him in the garden last night.'
\z

Two additional word orders emerge as natural here, namely the positioning of the light object pronoun after adjunct 1, or adjunct 2 (\ref{ex:r42}). 

\ea\label{ex:r42}
\ea Chunnaic mi 's gh\'{a}rradh e a-raoir
\ex Chunnaic mi 's gh\'{a}rradh a-raoir e 
\z
\z

These linearization options are found only with `light' object pronouns and not for object phrases more generally. It can also be shown to carry no consequences for information structure (\citealt{bennettelfnermccloskey13}). 

One could imagine trying to use syntactic movement to account for pronoun placement (like e.g. object shift in Scandinavian), but the rightward nature of the movement and the restriction to light pronominal elements is a first  indication that the precise solution is likely to be  complex.  Movement accounts have indeed been explored, but they have been  shown to have deep, potentially fatal problems. In particular, 
\citet{bennettelfnermccloskey13} and \citet{adger07pronouns} show that if this variability in linearization possibilities were to be captured via a syntactic movement, that movement would have to have the following undesirable properties.

\ea
\ea{It would have to be able to lower into islands.}
\ex{It would not feed other syntactic movements.}
\ex{It would not feed ellipsis.}
\ex{It has no information structure import.}
\z
\z
Taking stock briefly of the problems for word order theory I have discussed in this section, I think that to achieve descriptive adequacy beyond the coarse-grained Greenbergian classes a different kind of theory is required. The empirical challenge includes  characterizing the adjunction possibilities of modifiers as distinct from arguments, as well as the positioning of rogue elements such as light pronouns and lexically driven constructional idiosyncrasies, such as we have seen in Scottish Gaelic. In short, the problem of how to characterize the linearization properties on a language specific basis of a given  hierarchical representation  has not really been solved. 

\section{Removing word order from the syntax}\label{sec:ramchand:4}

We have seen that one dominant approach to characterizing the mapping from structure to serialization is to stipulate that the  mapping is rigid and universal but then parametrize the different movements within the syntactic component that operate to deform the original hierarchy.  In all these cases, the original hierarchy is well motivated from a typological point of view,   reflecting the constituents and embedding relationships that also make sense from a syntactic diagnostic and semantic perspective. The final deformed structure  created by word order movements (especially remnant and snowballing movements) do not on the other hand retain any kind of internal logic or intuitive constituency but are there solely to deliver the correct serialization based on the LCA\is{Linear Correspondence Axiom}. Given the fact that these movements are also rather different from  those standardly allowed in the syntactic component, an obvious conclusion to draw is that implementing the mapping between hierarchy and serialization as a topological re-morphing within the syntax itself is a kind of category mistake.   We \textit{know} that language is hierarchically structured, and we have ample evidence for distinct parts of speech, selectional relationships, agreement relationships etc. both descriptively and within processing. The syntax component is rich and internally structured with its own well studied primitives and relations (including a class of well motivated `displacement' relationships). We also know that languages are communicated serially as part of the externalization process and that different languages serialize `equivalent' syn/sem representations differently. The minimal,  first-principles strategy should be to investigate directly how these two different independent and justifiable representations (the hierarchical and the serial) map on to each other and what patterns and generalizations concerning that mapping emerge across languages. The dominant, movement strategy turns that mapping into a part of syntax itself,  a kind of intricate distortion algorithm which, while in principle could lead to observationally accurate output, creates a kind of middle man which neither helps us understand the nature of syntax itself better nor is helpful in connecting to the interfaces with cognition more generally. 

In this section, I summarize some recent work to argue that we have good reasons to believe that linearization algorithms are part of a domain general translation between hierarchy and serialization, and not a parochial syntax-internal set of mechanisms. That some linearizations of a hierarchy are possible mappings while others are not is an important fact and one that we would like to understand better.   Some recent work that doesn't filter the generalization through the language of movements, offers an intriguing possibility for where the explanation might lie. 
\citet{medeiros24, medeiros18} makes the important observation that if you have a hierarchy (stated as a sequence 1,2,3,4 with 1 the most deeply embedded ), then the typologically attested orders are just those that can be sorted into that hierarchy by Knuth's simple stack-sorting algorithm (\cite{knuth1968}).  In the statement of the algorithm below, imagine an input sequence with first incoming element I, a separate stack with top element S  fed by the Push operation which places an element from the input on the top of the stack, and a final output that is fed by the Pop operation from the top of the stack to the output.

\ea\label{ex:stacksort}
Stack-sorting algorithm (from \citealt{medeiros24})\\
 \noindent While input is nonempty, \\
 \hspace*{1in}If S $<$ I, Pop \\
 \hspace*{1in}Else, Push. \\
 \noindent While stack is nonempty, \\
 \hspace*{1in}Pop
\z

Essentially, if the input element is lower in the hierarchy than the element on the top of the stack then it is pushed onto the stack, becoming the top element itself. If the input element is higher in the hierarchy than the top of the stack then it is sent immediately to the output.   For example, let us assume we are trying to output the base hierarchical order 123 (where 1 is the most deeply embedded) then if the \textit{input} is  123, identical to the desired output, then each element is pushed and immediately popped. On the other hand, if the input is the mirror image of the output, 321, the entire sequence is first pushed onto the stack, and then popped one by one, reversing its order.  The stack essentially acts as a buffer, keeping track of the order of incoming elements and keeping that in “memory” while/if the elements from the input are still decreasing in functional height. An increase in functional height from the input triggers a pop of the lower element from the stack into the output. In this way, this simple  algorithm takes a serialized input of elements and can convert it to a sequence that represents  hierarchy, in an order suitable to feed bottom up compositional interpretation.   However, not all possible orderings of a 1-2-3 base hierarchy can be stack-sorted back into 1-2-3 by this simple algorithm.  There are six possible permutations of 123, five of which will successfully return 1-2-3 ((123, 132, 213, 321, 312), and one of which will not (231). 

\ea
\begin{tabular}{ccc}
{\textbf Output} & Stack & {\textbf Input} \\
1-2-3 $\longleftarrow$ & $\bigcup$ & $\longleftarrow$ 1-2-3 \\
1-2-3 $\longleftarrow$ & $\bigcup$ & $\longleftarrow$ 3-2-1 \\
1-2-3 $\longleftarrow$ & $\bigcup$ & $\longleftarrow$ 1-3-2 \\
1-2-3 $\longleftarrow$ & $\bigcup$ & $\longleftarrow$ 2-1-3 \\
1-2-3 $\longleftarrow$ & $\bigcup$ & $\longleftarrow$ 3-1-2 \\
{*}2-1-3 $\longleftarrow$ & $\bigcup$ & $\longleftarrow$ 2-3-1 \\
\end{tabular}
\z

The typological generalizations underlying Greenberg's Universal 20 have recently proved a fertile testing ground for theories of word order parametrization and the balance between head ordering parameters and different types of movements (\citealt{cinque17, abelsneeleman09}). 
Strikingly, when it comes to Universal 20 in particular, \citet{medeiros24} demonstrates  how the orders that are “unsortable”  by stack sorting are precisely those which are claimed to be unattested in a typological survey of NP internal orderings of Dem, Num, Adj N. \citet{medeiros24} also argues that the Final\hyp Over\hyp Final Constraint (FOFC \citealt{biberaueretal07}) can also be derived from a prohibition against non stack-sortable orders. 

I follow the general assumption that some abstract hierarchical ordering of elements is linguistically universal and that this hierarchical arrangement of symbols is crucial to meaning composition in human languages (while remaining agnostic for the time being about whether languages all need to agree on the number of fine grained distinctions and functional elaborations in each domain they choose to grammaticalize).  If this is the source of the hierarchical representations in all languages, then these base representations should be the same for all languages,  modulo differences in fine-grainedness and in which of the hierarchical elements is grammaticalized. Different word orders for the same elements must therefore emerge from the language particular mapping between those structures and a linearized representation suitable for externalization.  

What does this mean? Knuth's stack sorting algorithm (and Medeiros' implementation of it) is not some sort of mystical incantation. It is an algorithm designed to do one functionally important thing~-- from an input \textit{sequence}, it transforms the material to output elements in the order of bottom to top hierarchical embedding. This simple algorithm can handle nearly any choice a language might come up with for how the linear order of symbols might proceed and fails on only a few (from the point of view of the algorithm) “pathological” choices. It turns out that human language ends up avoiding these pathological choices. This suggests  that something in the way that language is set up \textit{makes} those order choices pathological, suggesting in turn that the stack sorting algorithm shares logical properties with how human speakers actually parse incoming sequences as well.\footnote{I am assuming here that the constraint comes from the mechanisms of comprehension rather than production, since this algorithm sorts  left-right  sequences into bottom-up representations that would naturally feed a comprehension process.  \citet{abels16b} points out that if you wanted stack\hyp sorting to fail on the typologically bad orders when considering the \textit{production}\hyp motivated mapping from hierarchy into order, then the input would have to be top-down and the output right-to-left to get the correct typological pattern.}
In other words, the attested word orders are the ones that one can use a simple domain general “processing” algorithm for to recoup the underlying hierarchy.\footnote{Interpreting the \isi{stack-sorting} generalization in this (possibly simplistic) processing way, assumes that the hearer wants to convert the incoming stream of speech to an ordered output that will feed cumulative meaning composition. It also requires the parser to be able to track “height” in the functional sequence as an independent fact. This need not be problematic if all we assume the parser is sensitive to is whether the element on the stack “selects” the incoming element or not.}

What I take this to suggest for our discussion of ordering is that the cluster of constraints on movement necessary to derive the typological pattern of *2-3-1  are plausibly not a syntax-specific set of constraints on movement, but reflect a more general informatic principle concerning how one recoups hierarchical dependencies from a linear presentation.   Basically, it shows us that the typological generalizations underlying the modern understanding of Greenberg's Universal 20 should not be seen as  arguments for a parametrization for movement plus \textit{whose}-pictures type pied piping vs. pictures-of-\textit{whom} type pied piping in syntax.   Rather, they say something about the way working memory works to support the representational conversion from strings to underlying hierarchy.  Seeing the algorithm from a \isi{direct linearization} perspective allowed us to see it more abstractly in terms of working memory and tracking of selection. On the other hand, movement algorithms are so parochial to the vocabulary of syntax that they obscure this more abstract interpretation.  This is one example of how a difference in implementational choice is not mechanistically innocent, even when the two versions deliver the same results observationally. 

As fascinating as the \isi{stack-sorting} generalization is, it still does not actually answer our questions about the \textit{particular} \isi{stack-sorting} compliant word order a language ends up with, nor does the stack sorting have anything immediately to offer us when it comes to the relative placement of specifiers and adverbials within the same hierarchical projection. Relevantly for us, \citet{abels16b} explicitly considers the question of whether adverb orders, head orders, and argument orders can form a joint sequence that is ordered by the algorithm and concludes that it only works if \textit{each domain is sorted independently}~-- relative orderings within each class are predicted by the algorithm but contradictions arise if orderings across categories are included. This suggests that hierarchy is relevant to the embedding of complement projections inside the projection of the head, but that elements associated with the same projection must be internally differentiated in a different way. It also doesn't tell us how the grammar implements its own linearization rules formally and how they are cued for the learner and represented in the speaker's grammar. We will return to these questions for Scottish Gaelic in the final section of this chapter. 

We have seen how a perspective of \isi{direct linearization} uncovered a deep generalization about the mapping between hierarchy and word order in human languages, and how such a perspective can potentially lead to a simplification of the grammar in the form of the elimination of word order movements (phrasal roll up movements, in particular). 

In the next subsection we will see how the perspective of \isi{direct linearization} dissolves the problem of head movement.

\subsection{Head movement} 
 
Since the onset of the Minimalist Program, head movement has been largely deprecated because it fails to “extend the target” in the statement of a syntactic derivation. There are many ways to rethink head movement so as to get around this problem.

Within \isi{direct linearization} theories, there is a powerful early proposal for how to capture the descriptive effects of head movement, namely \posscitet{brody00} \isi{Mirror theory}.

Observationally speaking, functional material which is hierarchically higher than the root is often linearized to the right of it (as a suffix), while the opposite linearization is observed outside of the word with higher functional material to the left. This tendency in many head initial languages was what led to head-movement and left-adjunction analyses of complex word formation (\citealt{baker85mp}).  It is quite common for languages to have distinct linearization principles for word internal bound morphemes than for the ordering of syntactically movable units. In fact, if we look at linearization as a core property of phonologization, we can add the following 
criterion  in (\ref{ex:domain}) as a defining property of the phonological  word.

\eanoraggedright\label{ex:domain}
The phonological word as a linearization domain\smallskip\\
The phonological word is the unit of the external linearization algorithm; it is a product of the internal linearization algorithm. Internal and external linearization are different mapping systems.
\z

\noindent Note that the above statement does not attempt to relate the phonological word to atoms of the syntactic or semantic representation, but claims that the phenomenon of “word” in this sense  emerges whenever a language has two distinct linearization cycles which feed each other.

In general, we can follow \citet{brody00}  in distinguishing the linearization statements that relate \textit{one head to another directly subordinate one}  from the linearization of “phrasal”  elements.  For the former, \citet{brody00} gives the above \isi{direct linearization} statement in (\ref{ex:mirror}) essentially mimicking in certain respects and replacing the rule of “head\hyp movement”. 

\eanoraggedright\label{ex:mirror}
Word mirror\is{Mirror theory} (agglutinative head-head linearizations)\smallskip\\
The syntactic relation `X complement of Y'  is identical to an inverse-order morphological relation `X specifier of Y'. \\
(where the latter gives rise to the morphological structure [X [Y]] linearized from left to right). (after \citealt{brody00})
\z

Brody's proposal is in some ways simply a different but equivalent implementation of head movement. There is however, one crucial innovation/flexibility opened up by this \isi{direct linearization} approach. Under head movement, the pronounced word is spelled out at the landing site of movement, which is always the highest head position that is a part of the head word in question. The diacritic @ introduced by Brody (stated in (\ref{ex:height}) allows a headword formed by mirror\is{Mirror theory} (or \isi{spanning}) to be in principle spelled out at \textit{any} head position within the span, not just the highest position. This diacritic on the phrase structure can be used to regulate the height of Spell Out for a head within a language  as a whole. 

\eanoraggedright\label{ex:height}
Diacritic for height of Spell Out: (cf. \citealt{brody00})\smallskip\\
In a tree, @ indicates the position where the morphological head word spells out.
\z

Thus, in a tree such as \figref{ex:complex}, the complex head formed from the combination of X and Y would always raise to the X position in a head-movement implementation. But under the Brody proposal, the complex head so formed can be spelled out at Y as well, as a matter of parametric setting. This is indicated diacritically by the @ at Y, and indeed in cases of more extended sequences of heads forming head-words in this way, the @ can appear at the site of \textit{any} of the heads in that sequence.

\begin{figure} [H]
\Tree [.{XP}  [.{spec}  \edge[roof];{WP} ]  [ [.{X}  ] [.{YP} [.{spec} \edge[roof];{MP} ] [ [.{Y@} ] [.{ZP} ] ] ] ] ]
\caption{Height of Spell Out}
\label{ex:complex}
\end{figure}

Thus, if we think about the tree in \figref{ex:complex}, the word order of the elements will follow the general mapping from c-command, with the caveat that the complex word [[Y] X] is spelled out as if it were in the Y position of the tree. The word order for \figref{ex:complex} with its diacritic would therefore be   WP $<$ MP   $<$ [[Y] X]$_{\omega}$ $<$ ZP, whereas under head movement it would be WP $<$ [[Y] X]$_{\omega}$    $<$ MP $<$ ZP.

In the spirit of Brody, I further assume that this principle of head-word formation applies also when the morphology in question is not strictly agglutinative, or even when the inflected word is completely synthetic. The point is that the head-word formation is a combination of the syntactic and semantic features of contiguous heads even when those are not spelled out by separable pieces of the morphology. The extension of head-word to non-agglutinative configurations is expressed in (\ref{ex:spanning}). 

\eanoraggedright\label{ex:spanning}
Synthetic words, or adaptation to “\isi{spanning}”\smallskip\\
A morphological head span can also include spell-out of synthetic forms and not just agglutinative suffixation. (after \citealt{svenonius16, svenonius20})
\z

With these tools in place, let us seek to represent the ordering of heads in Scottish Gaelic at the start of a clause. The relevant sentence is shown in (\ref{ex:embedclause}) where I concentrate on the embedded clause `that he did not speak with him.'.

\ea\label{ex:embedclause}
\gll Thuirt Calum nach do bhruidhinn e  ris. \\
say.\textsc{pst} Calum that.\textsc{neg} \textsc{pst} speak.\textsc{pst} he with.3\textsc{sm} \\
\glt `Calum said that he did not speak with him.'
\z

In the embedded clause, C and Neg can be seen as being expressed by a synthetic head \isi{spanning} C and Neg to give \textit{nach}, T is separate here (\textit{do}), as is the verb \textit{bhruidhinn}.  In the main clause, V-v-Asp-T is also a synthetic form \textit{Thuirt}.  A diacritic for linearization can be placed at Fin to ensure the high placement of the inflected verb in such cases. In the embedded clause, we need to allow separate Force and Neg heads, and allow a complex head-word to be formed from them that is separate from the complex head word involving tense.  I will assume that the past tense particle \textit{do} is in Fin, while the inflected verb itself is a head word consisting of V-v-asp-T and sits in T.   In this system we see very clearly the ordering differences predicted for agglutinative morphology. If a tree like \figref{ex:cptree} linearizes according to syntactic rules the functional head positions will line up in order with the top most functional particle leftmost. If on the other hand, a head-word is built a la Brody, then these heads will linearize within the word in the opposite order, and the actual height of the word itself will depend on the @ diacritic. 

\begin{figure}[h]
\Tree  [.{ForceP}  [.{Force} ] [.NegP [.{Neg} ]  [.{FinP} [.{Fin} ]  [.{TP}    [.{T}  ] [.{AspP} [.{Asp}  ] [.{vP} [.{v} ] [.{VP} [.{V} ] ]]]]] ] ]
\caption{Tree with multiple CP level categories}
\label{ex:cptree}
\end{figure}

We therefore need a separate diacritic to delineate stretches of heads that do form their own head-word linearization domain in this sense.  In their discussion of \isi{spanning}, \citet{byesvenonius12} propose exactly such a diacritic to regulate the scope of Brody-type word formation in this sense (\ref{ex:wordform}).

\ea\label{ex:wordform}
Scope of word formation (after \citealt{byesvenonius12}) \\
A head marked with {*} indicates that the head so notated must form a Brody-type word with the closest head above it bearing an {@} diacritic in the functional sequence. 
\z

For Scottish Gaelic therefore, we would mark the finite verb word as creating a span incorporating the V right up to T, linearized high at T. On the other hand, the Fin head is lexicalized separately, and the Neg and Force heads also form a head word together (\figref{ex:forcetree}).

\begin{figure} [H]
\Tree  [.{ForceP}  [.{Force@} ] [.NegP [.{*Neg} ]  [.{FinP} [.{Fin} ]  [.{TP}    [.{T@}  ] [.{AspP} [.{Asp}  ] [.{vP} [.{v} ] [.{VP} [.{*V} ] ]]]]] ] ]
\caption{Linearization height for Scottish Gaelic}
\label{ex:forcetree}
\end{figure}

What the previous two subsections have shown us is that there are compelling \isi{direct linearization} accounts that mimic the standard proposed syntactic movement accounts when it comes to the ordering of phrases and heads. However, these still fall short of delivering \isi{word order typology} because they do not unify head, complement, specifier and adjunct ordering within a single system.  The stack sorting algorithms, as pointed out by \citet{abels16b}, only work to deliver the right typological patterns if heads, arguments and adjuncts are ordered separately with respect to each other and not combined within a single sequence. For Brody's mirror\is{Mirror theory}, we are only dealing with heads by definition, and we make use of a diacritic to represent where complex heads interleave linearly with the extended complement structure. We don't in this system currently regulate the linearization of adjuncts for those projections.   When it comes to  Gaelic, the contiguity of the higher functional elements of the clause is dependent on the impossibility of adjunction to TP, for example.  
  

\section{Back to Scottish Gaelic word order}\label{sec:ramchand:5}

The current drawback in \isi{direct linearization} accounts described so far is the failure to regulate the mutual linearization relationships between head, specifier, and adjunct within the same labeled projection. Indeed, it is not in the hierarchical ordering of functional projections per se that languages seem to differ (cartography has shown us remarkable similarity among languages with respect to functional sequence), but in the specific interleaving of heads and arguments, and the positioning of adverbials. It is here that languages make idiosyncratic, though systematic choices, and we need a way to describe these choices algorithmically  in pursuit of a deeper understanding of the way in which hierarchical patterning in language interacts with serial externalization.

There is nothing principled in the way of pursuing a \isi{direct linearization} approach to \isi{word order typology} and variation; it is just that the project has not been widely embraced thus far.  I hope that will soon change. 

To summarize, we have seen that in Gaelic, the heads all conform to top down hierarchical order, but although C and T are obligatorily adjacent, they do not form a mirror order and so are not standardly analysed as involving head movement from T to C.  There are no arguments or adjuncts that intervene within this stretch of heads. Adverbs are internally well behaved in that they line up in reverse scopal order when sentence final, but there is no statement in the grammar that explicitly prevents, for example, pre-adjunction to TP or to AspP in the absence of an overt head. With respect to arguments,  subjects are always before objects except in \isi{copular clauses}, something which echoes the typology of \isi{VSO} languages more generally.  Finally, \isi{light object pronouns} seem to enjoy some right-linearization options, interleaved with adjunct positions.  As I argued in \sectref{sec:ramchand:4}, it seems that the head-parameter approach combined with a list of language specific obligatory movements does not even sufficiently describe  these word order choices in Scottish Gaelic. 

How then does a language decide to order its arguments and adjuncts with respect to a given head, and what are the constraints and cues to this mapping?  In the last part of this chapter, I will speculatively pursue the idea that word order is a part of the syntax-phonology mapping and that phonologically motivated output constraints can be used to parametrize word order choices. The syntax-phonology interface is currently a theory of what the relationship is between the chunks, or constituents generated by the syntax, and the linearly sequenced chunks manipulated by the phonology. The basic question is therefore how syntactic phrases map onto prosodic units, given that the phonology seems to care about phonological words, phonological phrases (possibly recursive), and intonational phrases.\footnote{Disagreement exists with respect to whether the units of prosodic theory are related to the units of syntactic grouping by means of a \textit{matching}  algorithm between the units of syntax (XPs) and the units of prosody (phonological phrases $\phi$)  (\citealt{selkirk11}), or whether the syntactic constituents just determine the \textit{edges} of prosodic domains (\citealt{selkirk86} \citealt{nesporvogel82}), creating many cases of mismatch.}

While the syntax/phonology interface is a lively and productive subfield of linguistics, nobody seems to be actually trying to get word order to emerge from  a language particular mapping, the phonologists seem to be leaving the actual order of words to the syntacticians. Since sound representations are inherently sequenced in human language, it makes sense that the first place that linear information would be present in language is after the mapping to phonology. Phonemes (or possibly phonological features) are sequenced into syllables which are then sequenced into phonological words.  These latter units are now the units that are themselves linearized to give the neutral word order of the language.  \citet{bennettelfnermccloskey13} have argued that the placement of \isi{light object pronouns} in Irish\il{Irish (Modern)} is triggered by a phonological requirement and is a part of the externalization mechanism. The position I will explore next, in line with \isi{direct linearization} is that \textit{all}  word order arises from an externalization process and is therefore a property of the syntax-phonology mapping, not just this particular quirky word class. 

\eanoraggedright\label{ex:lindomain}
The phonological word as a linearization domain:\smallskip\\
The phonological word is the input unit of the external linearization algorithm; it is a \textit{product} of the internal linearization algorithm. Internal and external linearization are different mapping systems.
\z

Given the language-specific construction of words, what sorts of algorithms take those units and the information of their hierarchical relationships and maps them to a linear representation? 

In the next subsection, I will imagine the form of the mapping that would be needed to begin to make sense of some of the more problematic outstanding properties of Scottish Gaelic word order, including the placement of the subject immediately after the tensed verb, and the reduced availability of left adjoined positions for adverbials. 

\section{A Syn-Phon alternative to word order typology}\label{sec:ramchand:6}
\is{word order typology|(}
From the discussion so far, we have seen that the key element in establishing a viable expression of a language's word order behaviours is to regulate the mutual ordering of heads, specifiers and adjunctions against the background of an assumed common hierarchy of phrase structural zones across languages (cf. \citealt{ramchandsvenonius14}). 

To state the patterns, we must assume that the syntactic representation encodes hierarchical levels, and also encodes a difference between heads, specifiers and modifiers at each level. Given that this algorithm is a direct mapping between syntactic form and concrete serialization of phonologized elements of the structure, we must take note of whether a head of a particular phrase might be overtly filled or not, depending on independent language specific placements of the @ and {*} diacritics. For each phrase, there might be an argument in the specifier position or not, and there might be a modifier, or not. 

Given that the mapping is specific to each language, and given also that it essentially orders phonologically overt elements with respect to each other, the output patterns of this mapping will have higher order phonological properties such as rhythm and stress. It is possible that some of the constraints on the output of the mapping can be expressed therefore as standard prosodic or phonological constraints since the output representation is essentially a linearized phonological representation. So, perhaps the space of those possible algorithms is even parametrized via phonological or rhythmic parameters on the output rather than overtly syntactic ones. Rhythmic cues to word order do not seem like an outlandish possibility. We know that child directed speech is particularly rhythmic and intonationally careful, and that children attend to the rhythm of the speech they are acquiring already in the womb (\citealt{nesporvogel86}). 

Quite generally, Scottish Gaelic phrases seem to consist of lone (relatively light) heads followed by substantial phrasal complements with very little left adjunction. We take this seriously in what follows.

Let us proceed top-down reading off the syntactic representation. The top of a complex syntactic representation will encode elements that belong to the top tier of the hierarchy~-- the head, its related specifier, and any adjoined modifiers~-- as well as a phrase that contains all the embedded material, the complement. Following \citet{Chomsky2000, Chomsky2001}, I make use of the concept of the “edge”  of a phrase to divide each constituent into two linearizable pieces: the “edge”  and the complement, which I will call the “body” of the phrase.  The edge is chosen here because it figures in many accounts of syntactic locality as an escape hatch for the impenetrability of phases, see the definition of the phase impenetrability condition (PIC) definition in (\ref{ex:PIC}). The edge seems to be the part of a phrase\slash phase whose Spell Out is delayed compared to that of the complement. In the definition below, the edge is claimed to consist of both the head and the material in the specifier. In my own implementation, I will include adjuncts to the phrase as being part of the edge.

\ea\label{ex:PIC}
Phase Impenetrability Condition (PIC):\smallskip\\
If $\Psi$ is a phase with head H, then the complement of H (the
“domain”) is not accessible for operations involving a position
outside $\Psi$. Only H and its specifiers (the “edge”) are accessible for such operations.
\z
 
In my algorithm, I separate each phrase into edge and body, where linearization of the former will be stipulated  to precede the latter. While the edge is not itself a syntactic constituent, the “body” is, being the complement of the head. The edge however is well defined as the linear stretch bounded by the left boundary of the syntactic complement on its right, and the left boundary of the next constituent up.   The first rule of Edge $<$ Body linearization (Rule 1 below)  must then be combined with a rule that orders elements \textit{within} the edge if it happens to contain more than just a head (Rule 2). The body constituent is also complex of course, but then the whole cycle repeats recursively for body as the top node. The first rule of the cycle, Rule 1 is given in (\ref{ex:rule1}). 
 
\eanoraggedright\label{ex:rule1}
Rule 1 (Recursive linearization rule):\smallskip\\
For any syntactic phrase XP, with an overt edge and body portion, linearize the edge first followed by the body. \\
(Edge = Head and Phrase in specifier position, plus any adjoined phrases) \\ 
This is the part of the  algorithm that overlaps with the LCA\is{Linear Correspondence Axiom}. It enforces a general effect of the functional sequence with higher material tending to precede lower material. However, it is recursive with respect to the body.  
\z
The algorithm does not yet exhaustively linearize all of the phonologically independent elements because  edges themselves can be complex. In the maximal case, they can consist of a head and an argument phrase in specifier position, as well as an adjoined  modifier phrase.

Following the lessons from stack sorting, I will make the assumption that elements at the same height in this sense are not inherently ranked with respect to one another. The linearization algorithm is potentially sensitive to the phonological word vs. phonological phrase distinction, a phonological\slash prosodic distinction, but also potentially to any syntactic featural diacritic that is present on the syntactic representation it is interpreting. In particular, we should assume that the algorithm can see the distinction between a head and a phrase.   My proposal here is that Scottish Gaelic is characterized by an essentially \textit{iambic rhythmic choice} for complex edges. In other words, the head will always linearize before the phrase that is associated with it. Rule 2 is given in (\ref{ex:rule2}).  

\eanoraggedright
\label{ex:rule2}
Rule 2  (Complex edges):\smallskip\\ 
If the edge is complex, i.e., consisting of both a head (a phonological word) and a phrase (a phonological phrase), then a step of linearization needs to occur within it as well.
\z

A language can choose to solve that particular ordering statement either  \textit{iambically} or \textit{trochaically}:  Given that such a complex edge will consist of unequal prosodic constituents $\omega$ (phonological word) and $\phi$ (phonological phrase), either $\omega$ is ordered before $\phi$ (iamb), or vice versa (trochee). 

In this case, Scottish Gaelic orders the head before the specifier. I assume this is what gives rise to the [Tense Subject] linearization unit and the [Asp Object] linearization unit.  It is well known from prosodic work (see \citealt{elfner12}) that the tensed verb and the subject form a close prosodic unit, and I speculate that this is correlated with the fact that they form part of the same subdomain for the purposes of linearization. For the purposes of illustration, I show how the linearization algorithm I propose would work on a toy example. The tree in \figref{ex:leughtree} shows the embedded clause linearized as \textit{gun do leugh mi an leabhar}, where \textit{gun} is the embedding complementizer, \textit{do} is a past tense finiteness particle, \textit{leugh} is the past form of the verb `read', and \textit{an leabhar} is the direct object `the book'. 

\begin{figure}
\Tree [.{ForceP}  [.{Force}  gun  ]   [.{FinP}  [.{Fin}  {do}  ] [.{TP}  [.{spec}  {mi}  ] [  [.{T@}  {leugh} ] [.{AspP/vP}   \edge[roof];{an leabhar}   ] ] ] ] ] 
\caption{Linearization of an embedded clause}
\label{ex:leughtree}
\end{figure}

In the above tree, \textit{gun} linearizes before FinP, by Rule 1, and \textit{do} linearizes before TP by the application of Rule 1 on FinP.  \textit{Mi, leugh} together have to linearize before AspP, again by Rule 1, now applied to TP.  The relative linearization of \textit{mi} and \textit{leugh} is resolved by the application of Rule 2, which says that the head must precede the specifier, to give \textit{leugh-mi}. 

In the first obvious departure from standard accounts,  the \isi{direct linearization} theory does not need to assume that the tensed verb in Gaelic is sitting in a higher projection than the subject because of the \isi{VSO} order. The claim here is that the subject is in the specifier of TP syntactically,  as is assumed for most other tensed languages. VS order emerges because of the way that specifier head relation is \textit{linearized} in Gaelic, not because of the verb moving past the subject. 

Finally, we have seen multiple indications that modifiers are treated separately from arguments with respect to linearization, and so I will assume that specifiers (arguments) have a distinguished status here in having the option of being linearized within a complex edge  according to Rule 2. To make the difference between specifiers and adjoined phrases, I will assume  though  that \textit{non-selected}  phrasal modifiers such as adverbials do not linearize in the same domain as the head as part of a core “complex edge”. We will need a separate linearization rule for adjunct  phrases.

There seems to be a fair amount of evidence that adjuncts are attached to particular projections or zones of the clause and can attach at various heights, but they often seem to linearize independently of their hierarchically matched argument DPs. When they are alone in the edge position they can linearize preceding the body that they dominate (as follows from Rule 1), but we need another rule in our arsenal to deal with adjuncts in syntactically heavy edges. This is given in (\ref{ex:adjunctrule3}). 

\eanoraggedright\label{ex:adjunctrule3}
Rule 3 (Adjunct linearization):\smallskip\\
Adjuncts can escape linearization at their point of hierarchical attachment. They can be eliminated from an edge  and  saved in a separate \textsc{working memory buffer}.  Elements in the buffer are popped to the  linearization string starting from the top of the stack,  but only \textit{after} the recursive body computation it belongs to. 
\z

In general, it seems reasonable to assume that the use of this working memory stack  is parametrized across languages. Languages that do not tolerate heavy edges will place the offending adjunct in the \textsc{buffer}, when the edge contains other linearizable material.  Languages on the other hand, might choose to tolerate heavy edges and have a linearization rule for them (for example, edges that contain both an argument and an adjunct but no overt head,  could linearize trochaically as successive phrases as in English \textit{Mary suddenly saw a bear.} or \textit{Suddenly, Mary saw a bear}. Like stack sorting and roll up movements, buffering and popping at the end of a cycle of linearization, will produce an order  of adjuncts \textit{in reverse hierarchical order} at the end of the sentence. The notion of the buffer is thus a little bit like a special kind of stack, which preserves the order of the incoming input sequence (here driven by a top-down traveling through the syntactic tree) but then reverses it when translating into precedence.  Given that the necessity of buffering is dependent on the head of the edge being filled, we can use this to model the occasional sentence medial adverbials we find in Scottish Gaelic. While a detailed analysis of all the cases is beyond the remit of this chapter, the intuition would be that in auxiliary constructions, because of the extra functional structure,  the projection that the adverbial adjoins to has a phonologically null head  position in the inflectional domain, which would allow the adjunct to linearize leftwards as part of a non-heavy edge. Alternatively, the choice between linearizing as part of the edge or undergoing buffering could be part of a systematic optionality for Scottish Gaelic adjuncts, with the latter being “preferred”  under conditions that clearly require closer investigation. The point here is that a difference between the simple verb construction and the auxiliary verb construction with respect to adjunction positioning was ruled out in principle on a theory that simply reads off order from the abstract, non-phonologized phrase structure. 

If we consider the possibility of adding an adverbial \textit{gu slaodach} `slowly', to the TP, Rule 3 will fail to linearize it before the TP material \textit{an leabhar}, but will hold it in buffer until after that TP has been spelled out, and then order \textit{gu slaodach} after that.  If both a VP adverbial and a TP adverbial are present, then the VP adverbial will be popped out first, followed by the TP adverbial since the body linearization for VP completes before the full TP body linearization computation completes. 
 \is{word order typology|)}
 
\subsection{Characterising Scottish Gaelic word order}
 
Scottish Gaelic  in general imposes  an \textit{iambic} rhythmic pattern on its precedes relation for the phrasal algorithm.  So edge precedes body, and within the edge itself, the word element (the filled head if there is one), will precede the selected phrasal argument in specifier position. 

Adjuncts give rise to "heavy edge" with filled heads, and are placed in the buffer, and linearized after the end of that recursive cycle. The recursive nature of the linearization algorithm and the different syntactic height of adverbials, forces adjuncts therefore  to end up in reverse hierarchical order. 

Although the phrasal prosodic algorithm for Scottish Gaelic is claimed here to be “iambic”, it is worth pointing out that at the prosodic word level, Gaelic is generally considered to be trochaic (\citealt{morrison19}). This is not a contradiction because the domain of this parameter setting is phrasal linearization, not word internal linearization which is a completely separate cycle which leads to the definition of ``word''' in the first place as the elements that are then ordered by the later cycle. Scottish Gaelic is trochaic within the phonological word domain and iambic (if the word is even appropriate at these higher levels of abstraction) at the level of the phrasal linearization algorithm. It is possible in fact that the two domains will align oppositely with respect to this factor, as with ordering, and “mirror\is{Mirror theory}” (\citealt{brody00}). 

\subsection{VOS as a sub-pattern}

Given the algorithmic set up so far, we can make the following speculation:   \isi{VOS} is what happens in iambic languages when the Subject is linearized like adjuncts, via the buffer and so ends up last.  The difference between linearizing within the edge grouping and buffering may be freer than I have assumed, and dependent rather on something syntactic like, for example, whether an Agree relation between the head and phrase has been established or not, or on some aspect of the intonation.  The typological fact is that most \isi{VSO} languages do seem to have \isi{VOS} as a word order option.  While this does not seem to be true on the face of it for Scottish Gaelic, it might well be that what we are seeing in the copular construction is a grammaticalized vestige of precisely this broader typological possibility. 

One way of thinking about it, would be to say that the copula construction is different precisely in that the copula itself is base generated high and does not have an Agree dependency with the subject. This means that it will have to be buffered, just like an adjunct giving subject-last ordering for precisely this class of constructions. 

\ea
  \gll Is cìobar Calum.\\
  Cop shepherd Calum \\
  \glt `Calum is a shepherd.'
\z

If such a story could be made to work, it would also give possessor final order for possessed DPs, under the assumption that in Scottish Gaelic no formal Agree relation is established between the lexical noun  and the possessor.

\ea
  \gll car dearg a'  bhalaich \\
  car red the boy.\textsc{gen} \\
  \glt `The boy's red car'
\z

This strategy also allows for interleaved adverbs in the aspectual construction in Scottish Gaelic, where the verb and object similarly stay low, plausibly leaving the adverb alone in the edge of one of the intermediate functional projections. 
 
\subsection{The light pronoun problem}

A crucial component of the \citet{bennettetal16} analysis is the idea that object pronouns are too light to form their own phonological word.  The present theory would make use of this idea as well, and assert that it makes the pronoun ineligible for edge linearization. It would therefore have to be thrown into the buffer, where it would linearize interleaved with adverbials, as observed.
 
\subsection{What other linearization patterns are possible?}

The rhythmic theory, together with the edge body partition give rise to a number of different word order setting possibilities. I explore some of these potential settings here. 

For example, allowing trochaic ordering on the edge linearization rule, might systematically give rise to SVO. A relaxed tolerance for \textsc{heavy edges} might allow optionality on whether adverbials linearize via bufferization or not, but this would need more detailed examination of the patterns.

The general iambic rhythmic setting of Scottish Gaelic for edges is congruent with the properties of Rule 1, which orders head before complement in the simple case. If Rule 1 is specific to Gaelic, rather than being universal, it could be that head final languages would emerge from the opposite choice of linearization in Rule 1 making it the functional equivalent of the head parameter. However, Rule 1 is not a head ordering rule per se, since it orders the whole edge with respect to the body and not just the head. A reverse setting for Rule 1 would also give languages that order the arguments after the heads that introduce them, in addition to having the heads appear in reverse hierarchical order. The overwhelmingly most common pattern for head final languages is for the arguments and adjuncts to precede all the verbal heads stacked up at the end, as if they resided in ``leftward'' specifiers.  Another way to get head final languages from this parameter space would be to disallow prosodically unequal daughters in complex edges. This would force \textit{heads} into the buffer making them linearize at the end of the externalization algorithm and stack up at the end in reverse hierarchical order.  Then the edge would either be a single phrase, \textit{or}, if it were complex, would just consist of two phrases (argument, and adjunct, say) which would be perfectly linearizable as equal daughters. Thus, Rule 1 is really the moral equivalent of the LCA\is{Linear Correspondence Axiom}, building in an asymmetric preference for height to correspond to leftness, while head finality is a parameter setting that affects the rhythm of the edge material. The iambic setting orders elements of unequal weight, the other setting prohibits unequal daughters and  imposes an extrinsic order on things of equal weight (or maybe even allows both orders). The attractive thing about the approach being pursued here, to my mind, is the way that it utilizes factors that we already know the prosodic or rhythmic systems of language to be sensitive to. Bans on units that are too heavy, starting with things that are too light, unequal daughters etc. seem like reasonable constraints on a system regulating order via rhythm. 

It is unclear from this kind of rhythmic set of parameters what the relationship is to the stack sorting algorithms that might underpin the mapping to hierarchy. Stack sorting gave us a handle on the general principles for reconstituting hierarchy from a range of different linearization choices. The rhythmic parameters explored here regulate the specifiers, heads, and adjuncts with respect to one another (although the invocation of the buffer is probably the equivalent of having a separate stack sorting mechanism for adjuncts). Also, this investigation has barely scratched the surface in terms of accounting for word order patterns cross-linguistically, or even completely for Scottish Gaelic itself, something beyond the scope of this short chapter.  I hope instead to have achieved two modest goals. The first is to highlight the fact that the standard accounts currently fall short on accounting for some very general properties of Scottish Gaelic word order, including the positioning of adverbials (and the relative lack of left adjunction as a preferred option).  The second goal was to open up the potential space of hypothesis formation and theory building to include theories of \isi{direct linearization}. Direct linearization liberates the syntax from unmotivated movements and unwanted assumptions. It also allows for more phonological and prosodic considerations to make themselves felt if what we are dealing with in externalization really is a kind of mapping, or interface, between syn/sem hierarchical representations and sound-based sequencing.


 
\section*{Acknowledgments}
This chapter lived for a long time in basement of my brain before being finally written up. I am very grateful to Andrew Carnie for inviting me to present in his series at the University of Arizona, and thus giving me a chance to put my thoughts about \isi{word order typology} into words. I am also grateful to various people over the years for having conversations with me about word order and syntactic movements. I would particularly like to single out David Adger, Klaus Abels (who introduced me to the work of David Medeiros), Peter Svenonius and Björn Lundquist.  I would also like to thank an anonymous reviewer and Andrew Carnie for comments on an earlier draft, which greatly improved the chapter.  They are not responsible for any  remaining inaccuracies and infelicities.   A big thank you finally to Björn Lundquist and Mike Hammond for help with \LaTeX{} compatibility issues in the final stages.

\il{Scottish Gaelic (Modern)|)}

\printbibliography[heading=subbibliography,notkeyword=this]

\end{document}
