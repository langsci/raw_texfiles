\documentclass[output=paper,colorlinks,citecolor=brown]{langscibook}
\ChapterDOI{10.5281/zenodo.15654879}
\title{Syntactic reconstruction in Celtic}
\author{Marieke Meelen\affiliation{University of Cambridge}}

\IfFileExists{../localcommands.tex}{
   % add all extra packages you need to load to this file

\usepackage{tabularx,multicol}
\usepackage{url}
\urlstyle{same}

\usepackage{listings}
\lstset{basicstyle=\ttfamily,tabsize=2,breaklines=true}

\usepackage{langsci-basic}
\usepackage{langsci-optional}
\usepackage{langsci-lgr}
\usepackage{langsci-osl}
% \usepackage{./langsci/styles/langsci-lgr}
% \usepackage{./langsci/styles/langsci-osl}
% \usepackage{langsci-gb4e}

\usepackage{tikz}
\usetikzlibrary{patterns,calc}
\pgfdeclarepatternformonly{south east lines}{\pgfqpoint{-0pt}{-0pt}}{\pgfqpoint{3pt}{3pt}}{\pgfqpoint{3pt}{3pt}}{
    \pgfsetlinewidth{0.6pt}
    \pgfpathmoveto{\pgfqpoint{0pt}{3pt}}
    \pgfpathlineto{\pgfqpoint{3pt}{0pt}}
    \pgfpathmoveto{\pgfqpoint{.2pt}{-.2pt}}
    \pgfpathlineto{\pgfqpoint{-.2pt}{.2pt}}
    \pgfpathmoveto{\pgfqpoint{3.2pt}{2.8pt}}
    \pgfpathlineto{\pgfqpoint{2.8pt}{3.2pt}}
    \pgfusepath{stroke}}
    
\usepackage{stmaryrd}
\usepackage{wasysym}
\usepackage{multirow}
\usepackage{caption}
\usepackage{subcaption}
\usepackage{mathrsfs}
\usepackage{qtree}

\usepackage{linguex}


   %pminos do not split footnotes
% \interfootnotelinepenalty=10000 %Footnote in Laporte chapters has to be split SN


%\DeclareIndexNameFormat{default}{%
%\nameparts{#1}%
%\usebibmacro{index:name}%
%{\index[names]}%
%{\namepartfamily}%
%{\namepartgiveni}%
% {}% L1
% {}% L2
%{\namepartprefix}% generates spurious space L3
%{\namepartsuffix}% generates spurious space L4
%}

%  {\DeclareIndexNameFormat{default}{%
%     \usebibmacro{index:name}{\index[names]}{#1}{#3}{#5}{#7}}}

%\DeclareIndexNameFormat{default}{%
%  \usebibmacro{index:name}{\sindex[nom]}{#1}{#3}{#5}{#7}}

%\DeclareIndexNameFormat{default}{%
%  \usebibmacro{index:name}{\sindex[person]}{#1}{#3}{#5}{#7}}
%\DeclareIndexNameFormat{default}{%
%\nameparts{#1} \usebibmacro{index:name}{\sindex[person]]}{\namepartfamily}{‌​\namepartgiven}{\nam‌​epartprefix}{\namepa‌​rtsuffix}}

%\newcommand{\smiley}{:)}

%\renewbibmacro*{index:name}[5]{%
%\usebibmacro{index:entry}{#1}%
%{\iffieldundef{usera}{}{\thefield{usera}\actualoperator}\mkbibindexname{#2}{#3}{#4}{#5}}}

% \newcommand{\noop}[1]{}

%remove for final
%\overfullrule=1mm

\newcommand{\tobi}[2]}}
\renewcommand{\S}[1]{\tobi{#1}{\textsc{*}}}

% this volume references
% puts: [this volume]
% already defined: \citetv
%\newcommand{\citepv}[1]{(\citeauthor{#1} \citeyear*{#1} [this volume])}
\newcommand{\citealtv}[1]{\citeauthor{#1} \citeyear*{#1} [this volume]}

%parentheses around example number
\newcommand{\pref}[1]{(\ref{#1})}

% in-text examples

\newcommand{\lnex}[1]{\textit{#1}} %target lang word
\newcommand{\lnlit}[1]{(lit.: `#1')} %literal reading
\newcommand{\lnlat}[1]{(#1)} % latinization
\newcommand{\lntrans}[1]{`#1'} %translation
\newcommand{\lnexl}[2]%
{\lnex{#1}{} \lnlat{#2}} % ex with latinization
\newcommand{\lnexlat}[3]{\lnex{#1}{} \lnlat{#2}{} \lntrans{#3}} % ex with latinization and tranl.

%ch01
\newcommand{\co}[1]{\mbox{\textbf{#1}}}

%ch09

\newcommand{\cyrbulg}[1]{\begin{otherlanguage*}{bulgarian}#1\end{otherlanguage*}}


%ch10
\newcommand{\nlp}{{\small NLP}}
\newcommand{\mwe}{{\small MWE}}
\newcommand{\rae}{{\small RAE}}
\newcommand{\lvc}{{\small LVC}}
\newcommand{\pos}{{\small P}o{\small S}}
%\newcommand{\todo}[1]{ \textcolor{red}{#1} }

%\renewcommand{\labelenumi}{\theenumi}
%\ainamefmt{{vv}{ll}{, ff}{, jj}} % fullname

\newcommand{\biberror}[1]{{\color{red}#1}}

\newcommand{\osenovaitem}{--~}
   %% hyphenation points for line breaks
%% Normally, automatic hyphenation in LaTeX is very good
%% If a word is mis-hyphenated, add it to this file
%%
%% add information to TeX file before \begin{document} with:
%% %% hyphenation points for line breaks
%% Normally, automatic hyphenation in LaTeX is very good
%% If a word is mis-hyphenated, add it to this file
%%
%% add information to TeX file before \begin{document} with:
%% %% hyphenation points for line breaks
%% Normally, automatic hyphenation in LaTeX is very good
%% If a word is mis-hyphenated, add it to this file
%%
%% add information to TeX file before \begin{document} with:
%% \include{localhyphenation}
\hyphenation{
    Beck-man
    Ngu-yen
    back-chan-nel
    back-chan-nels
    mo-not-o-nous
    ste-reo-typ-i-cal
}

\hyphenation{
    Beck-man
    Ngu-yen
    back-chan-nel
    back-chan-nels
    mo-not-o-nous
    ste-reo-typ-i-cal
}

\hyphenation{
    Beck-man
    Ngu-yen
    back-chan-nel
    back-chan-nels
    mo-not-o-nous
    ste-reo-typ-i-cal
}

   \boolfalse{bookcompile}
   \togglepaper[12]%%chapternumber
}{}

\AffiliationsWithoutIndexing

\abstract{Reconstructing syntax has been the most challenging part of research into language history, phylogeny and proto-languages. Problems regarding directionality, transfer and in particular correspondence seemed impossible to address, until recent advances in syntactic theory. In this chapter, I discuss ways to address these issues, focusing on (generative) Minimalist approaches to syntactic reconstruction in particular, with concrete examples from Celtic languages. These methods combine the goal of reconstructing language history, with the analytical tools of formal grammar and cognitive science.}

\begin{document}

\maketitle

\is{syntactic reconstruction|(}

\section{Introduction}

The comparative historical study of languages, first seen in the work of Marcus Zuerius van Boxhorn\ia{van Boxhorn, Marcus Zuerius} and Sir William Jones\ia{Jones, William} in the 17th and 18th centuries, laid the foundation for breakthroughs in the reconstruction of a Indo-European proto-language\is{Proto-Indo-European}. Historical linguistics textbooks praise the \isi{comparative method} as the ``central” method of reconstruction (\citealt{mm:campbell_historical_2013}; \citealt{mm:TraskR.L1996Hl}; \citealt{mm:fox1995linguistic}; \citealt{mm:hock_principles_1991}).\footnote{The first visualizations of relations between languages go back to the 17th century with some Germanic varieties by Georg Stiernhielm in 1671 (\citealt{mm:sutrop2012estonian}). Other Indo-European languages followed much later (around 1800) in the work by Felix Gallet (\citealt{mm:auroux1990representation}). These early diagrams generally showed networks, however, rather than rooted trees. Trees were popularized in the second half of the 19th century by Auguste Schleicher's 1853 \textit{Stammbaum}, which had `Celten' as the first ones to branch off from `Slawogermanen', `Lateiner', `Griechen' and `Arier' (see \citealt{mm:schleicher_ersten_1853}, but also his main work from 1861–1862, as well as \citealt{mm:Morrison2016} and \citealt{mm:Pellardetal2024} for a full history of phylogenetic relations).} There are, however, few clear definitions or step-by-step execution guides for the method (\cite[57]{mm:fox1995linguistic}; \citealt{mm:durie_regularity_1996}; \citealt{mm:Walkden2014}), and it focused only on the reconstruction of phonology, morphology and lexical items. This is illustrated by the omission of syntax in Indo-European textbooks (e.g., \citealt{mm:beekes_comparative_1995}; \citealt{mm:szemerenyi_introduction_1996}; \citealt{mm:krasukhin_was_2017}). Traditionally, Indo-Europeanists have been skeptical about extending this methodology to syntax (\citealt{mm:meillet_methode_1954}; \citealt{mm:jeffers_syntactic_1976}). Because there was no comprehensive syntactic theory in the early heyday of the \isi{comparative method}, syntactic reconstruction was limited to the placement of individual grammatical items (\citealt{mm:wackernagel1892}), or mood and modal categories (\citealt{mm:thurneysen1885oskische} and \citealt{mm:delbruck_vergleichende_1893}). The fact that publications on PIE\il{Proto-Indo-European}\il{PIE|see {Proto-Indo-European}} alignment systems (e.g., \citealt{mm:uhlenbeck_agens_1901}; \citealt{mm:pedersen_neues_1907} and \citealt{mm:vaillant_lergatif_1936} vs \citealt{mm:villar_ergatividad_1983}; \citealt{mm:rumsey_chimera_1987}; \citealt{mm:lehmann_theoretical_1993}, etc.) and “basic word order” (\citealt{mm:dressler_textsyntaktische_1969}; \citealt{mm:lehmann_proto-indo-european_1974}; \citealt{mm:friedrich_proto-indo-european_1975-1}; \citealt{mm:miller_indo-european_1975}) reached different conclusions only reaffirmed the idea going back to Delbrück himself (\cite[v-vi]{mm:delbruck_vergleichende_1893}) that ``one can no more reconstruct the syntax of a proto-language than one can reconstruct last week's weather, and for the same reason: both reflect chaotic systems'' (\cite[135]{mm:lightfoot_myths_2002}). Proponents of more recent methods in grammaticalization and syntactic theory (see \sectref{sec:methods}), however, have had a more positive outlook on the possibility of unveiling the syntax of earlier languages. In addition to recovering the syntactic history of a proto-language, reconstruction methods can also potentially tell us more about whether and how languages are related and where related languages originate (i.e., the time-lined topology, subgrouping and root of a phylogenetic reconstruction). \citet{mm:hartmannwalkden2024strength} test a number of grammatical factors in a wide range of languages and conclude that syntactic properties are very weak when it comes to encoding phylogenetic information specifically. However, we might also use reconstruction to learn more about how syntax in general or the word order of one specific language can change over time and why it has changed in one particular way (\cite[9]{mm:ferraresi_principles_2008}).

In this chapter, I shed light on this last goal of syntactic reconstruction -- focusing on Celtic specifically. Despite the fact that the verb-initial word order in insular Celtic languages is an interesting outlier in the family, major publications on Indo-European syntax (e.g., \citealt{mm:delbruck_vergleichende_1893}; \citealt{mm:lehmann_proto-indo-european_1974}) do not discuss any examples from Welsh\il{Welsh (Modern)}, Cornish\il{Cornish (Modern)}, Breton\il{Breton (Modern)} or Irish\il{Irish (Modern)}. Important early comparative Celtic grammars like \citet{mm:lewispedersen1961concise} devote very few words to syntax and are often limited to the position of the verb only (\cite[301]{mm:thurneysen1927altirische}; \citealt{mm:lewispedersen1961concise}; \citealt{mm:watkins_preliminaries_1964}; \citealt{mm:Meid1968}). It is not until later that we start seeing comparative syntactic research on the insular Celtic languages. Where work on Celtic syntax initially focused on synchrony (e.g., \citealt{mm:harlow_government_1981}; \citealt{mm:hendrick_celtic_2000}; \citealt{mm:mccloskey_scope_1996}; \citealt{mm:carnie_vso_2000}), in the 1990s, we start to see the first attempts in syntactic reconstruction (e.g., \citealt{mm:koch_prehistory_1991}; \citealt{mm:maccana_further_1991}; \citealt{mm:eskaevans1993continental}; \citealt{mm:isaac_issues_1993}; \citealt{mm:eska_syntax_1996} for the verbal complex and basic word order; \cite{mm:willis_syntactic_1998}; \citeyear{mm:willis_reconstructing_2011}  for V2 and free relatives; \citealt{mm:newton_development_2006} for pre-Irish C-domains; \citealt{mm:meelen_why_2016} for British preverbal particles). These studies are extremely useful, but do not yet cover all aspects of Celtic syntactic reconstruction.

I start this chapter by outlining some issues with syntactic reconstruction in general, with specific reference to some examples from Celtic. I discuss how a range of more recent methods have tried to address those problems (\sectref{sec:methods}). I then zoom in on generative Minimalist approaches in particular (\sectref{sec:minapproaches}) in order to show that Celtic languages can teach us a lot about how to develop these methods further (see Sections~\ref{sec:celticrefiningPCM} on Celtic and \ref{sec:celticcontact} on Celtic in contact). Finally, in \sectref{sec:MinhelpCeltic}, I claim that the reverse is also true, namely that  Minimalist syntactic reconstruction can teach us more about the history of Celtic languages.

\section{Methods of syntactic reconstruction}
\label{sec:methods}

\subsection{Problems with syntactic reconstruction}
\label{subsec:submethods}

Some traditional methods for syntactic reconstruction exploit unidirectional tendencies of grammaticalization. \citeauthor{mm:givon_historical_1971}'s (\citeyear{mm:givon_historical_1971}) adage that “today's morphology is yesterday's syntax” as well as similar work by \citet{mm:gildea_reconstructing_2000} and \citet{mm:balles2008principles}, for example, rely on the assumption that morphology exhibits signs of fossilized syntactic items that used to be independent in earlier stages of the language. Similarly, fossilized lexical items whose derivation is transparent could be employed for reconstruction, e.g., present day Welsh \textit{sef} from \textit{ys} `(it) is' +\textit{ef} `it/this' (\citealt{mm:meelen_why_2016}) or \textit{efallai} `perhaps' from \textit{ef} `he' and \textit{(g)allai} `could' (\cite[14]{mm:willis_reconstructing_2011}). As \citet{mm:willis_reconstructing_2011} observes, in a VSO language examples like \textit{efallai} that point to SV word order are unmotivated unless in earlier stages of the language SV orders were possible. These studies focusing on individual grammatical items are limited in scope, however, because it is not clear which earlier stage can be reconstructed, nor can it be used for syntactic changes that do not involve grammaticalization.

While sound changes can be constrained by the limits of physical articulators, syntactic \emph{directionality} constraints are harder to discover or measure. A change from unvoiced to voiced consonants between vowels is much more likely than the reverse, for example, but directionality factors of these kind are harder to predict and/or explain in syntax. There are no articulatory restrictions on changes in word order from Object-Verb (OV) to Verb-Object (VO), for example. Typological approaches, such as those by \citet{mm:greenberg_universals_1963, mm:lehmann_proto-indo-european_1974, mm:friedrich_proto-indo-european_1975-1, mm:miller_indo-european_1975}, but also more recent ones like \citet{mm:von_mengden_reconstructing_2008} and \citet{mm:ferraresi_principles_2008}, use implicational relationships between grammatical phenomena to achieve a similar result. \citet{mm:walkden_comparative_2009} rightly points out, however, that in order to make any predictions about the syntax, this method requires historical texts in which at least one half of the relevant implicational universal is present. \citet[109]{mm:von_mengden_reconstructing_2008} acknowledges this drawback. For proto-languages, we have no texts by definition and hence no starting point for this method. \citet[65]{mm:Walkden2014} notes, however, that \citeauthor{mm:niyogi_proper_2009}'s (\citeyear[101--127]{mm:niyogi_proper_2009}) model can straightforwardly yield directionality in syntax if it is combined with a cue-based learning algorithm such as that of \citet{mm:lightfoot_development_1999}.  As I show later on in \sectref{sec:PCMforhistcelt}, recent discoveries in theoretical syntax -- such as parameter\is{parameters} hierarchies -- may be able to shed more light on this directionality issue as well.

In addition to this so-called “directionality problem”, \citet{mm:lightfoot1980reconstructing}  draws attention to the fact that grammars `must be created afresh by each new language learner' (\cite[37]{mm:lightfoot1980reconstructing}), which \citet[4]{mm:willis_reconstructing_2011} has called the “radical \isi{reanalysis} problem”: if children had been able to work out what the earlier structure was, they would not have introduced the new structure. Strictly speaking, this idea of change due to \isi{reanalysis} in first language acquisition is not unique to syntax. As \citet[372]{mm:harris_historical_1995} and \citet[23]{mm:walkden_comparative_2009} have pointed out, the same issue arises for phonological reconstruction, where sound change can occur by way of discrete reanalyses (\cite{mm:ohala_listener_1981}). Furthermore, \citet{mm:roberts_syntactic_2003}, among others, follow the traditional “poverty\hyp of\hyp stimulus” reasoning going back to \citet{mm:chomsky_binding_1980}, observing that most of the time, first-language acquisition is remarkably successful despite the lack of direct access to parental grammar. This should follow from \citet{mm:chomsky_three_2005}'s division of first, second and third factors: ``grammars actually acquired on the basis of similar primary linguistic data (PLD) do not vary substantially from one another, otherwise the acquisition task becomes intractable'' (\cite[49]{mm:Walkden2014}). This extends \citeauthor{mm:willis_reconstructing_2011}'s (\citeyear[2]{mm:willis_reconstructing_2011}) observation that innovative grammars constrain the range of possible hypotheses about earlier stages of the grammar, which thus aids syntactic reconstruction (see \sectref{subsec:submethods} below).

Furthermore, \citet{mm:willis_reconstructing_2011} reminds us that grammatical rules that are present in two or more languages could represent a continuous line of transmission \textit{or} they could be transferred via language contact (the “transfer problem”). In the correspondence sets for phonological reconstruction, borrowings are lexical and much easier to identify and eliminate. \citet{mm:clackson2017} illustrates this issue with \citeauthor{mm:bardal_syntax_2013}'s (\citeyear{mm:bardal_syntax_2013}) work on the reconstruction of the construction `woe is X.\textsc{dat}' found in a number of Germanic and other Indo-European languages. He argues that the Gothic\il{Gothic} and Armenian\il{Armenian} instances of these constructions could also simply have been transferred from \ili{Greek} through the common bible translations. In \sectref{sec:celticcontact} below, I demonstrate that although this is a serious issue, there are interesting opportunities related to syntactic transfer in reconstruction too.

Another issue that has been raised with syntactic reconstruction is the lack of arbitrariness. \citet{mm:harrison_limits_2003} argues that the real success of the \isi{comparative method} in terms of making genetic inferences relies on the fact that similarities in symbolic form-meaning pairings (i.e., lexical cognates) ``cannot be attributed to nature, and are unlikely to be the result of chance'' (\cite[223]{mm:harrison_limits_2003}). “Grammatical objects”, on the other hand, do not provide any such evidence for genetic relatedness, because form-meaning pairs in syntax are not arbitrary. Where lexical items represent individual linguistic signs that reside in a lexicon, which is “a repository of linguistic unpredictability” (\cite[223]{mm:harrison_limits_2003}), grammatical objects are “templates, diagrams, or rules” that are iconic, and not symbolic signs. Iconic signs are complex and do not reflect arbitrary mappings between linguistic categories and the world, which is why they cannot establish genetic relatedness (\cite[225]{mm:harrison_limits_2003}). Transferring the \isi{comparative method} is difficult due to the supposed lack of regularity in syntactic change compared to the famous \textit{Ausnahmslosigkeit} (`exceptionlessness') of sound change (\citealt{mm:osthoff_morphologische_1878}; \citealt{mm:labov_resolving_1981} for the regularity hypothesis in sound change; \citealt{mm:lightfoot_myths_2002}; \citealt{mm:pires_how_2008} for difficulties transferring this to syntax). All the above factors illustrate how difficult it is for syntactic patterns (or “templates, diagrams, or rules”) to meet the condition in (\ref{ex:DCC}): 

\ea Double Cognacy Condition (\cite[50]{mm:Walkden2014})\smallskip\\
In order to form a correspondence set, the contexts in which postulated cognate sounds occur must themselves be cognate.
\label{ex:DCC}
\z 

\noindent Since any potential higher units of syntax, such as phrases or sentences, are not transmitted from generation to generation in the same way lexical items are, there is no syntactic equivalent to the “cognate” we know from the \isi{comparative method} of phonological reconstruction (\cite[125--126]{mm:lightfoot_myths_2002}). The problem of what to compare instead of cognates, the so-called “correspondence problem” (\cite[312]{mm:watkins_towards_1976}), is the most difficult to address. Apart from perhaps some (poetic) stock phrases and idioms, such as PIE\il{Proto-Indo-European} *\textit{eg\textsuperscript{{wh}}ent} *\textit{og\textsuperscript{wh}im} `he killed (the) snake' (\cite[301]{mm:watkins1995kill}) and *\textit{ḱlewos} *\textit{n̥d\textsuperscript{h}g\textsuperscript{{wh}}itom} `unperishable glory' (\cite[467]{mm:kuhn1853alphaἰacuteomeganu}), there is no direct correspondence between sentences of one language and another. As syntactic theory has developed in recent years, however, there are now potential answers to search for appropriate alternative comparanda. 

In the next section, I discuss some more recent methods of syntactic reconstruction that address this issue. For more detailed exposés on addressing problems with syntactic reconstruction, see, among others, \citet{mm:bowern_syntactic_2008}, \citet{mm:willis_reconstructing_2011}, \citet{mm:Walkden2014}, \citet{mm:viti_historical_2015}, \citet{mm:clackson2017}, and \citet{mm:gildea_curious_2020}.

\subsection{Addressing the correspondence problem}

Alice Harris and Lyle Campbell suggested that syntactic “patterns” could solve the correspondence problem as patterns can be compared cross-linguistically (\cite{mm:campbell_syntactic_2002}). They illustrate this with the reconstruction of alignment systems in Kartvelian\il{Kartvelian} languages  (\cite{mm:harris_reconstruction_2008}), using language-internal facts. Similarly, \citet{mm:gildea_genesis_2000} reconstructed the Proto-Cariban\il{Proto-Cariban} verbal system and \citet{mm:kikusawa_proto_2002} focused on alignment systems in Proto-Central-Pacific\il{Proto-Central-Pacific}. While the results of these particular reconstruction efforts may be right, \citet{mm:willis_reconstructing_2011} and \citet{mm:Walkden2014} note a number of issues, such as the lack of psychological reality of “patterns”, the unrestricted nature of the method and that “patterns” do not appear to have a clear place in syntactic theory.

This specific issue could be resolved by an approach that uses syntactic patterns stored as units of grammar: Diachronic Construction Grammar\is{Diachronic Construction Grammar}\is{DCxG|see {Diachronic Construction Grammar}} (DCxG, see \citealt{mm:bardaleythorson2012};  \citealt{mm:viti_reconstructing_2015}; \citealt{mm:eythorsson_syntactic_2016}; \citealt{mm:gildea_curious_2020}). In DCxG\is{Diachronic Construction Grammar}, the “construction” is proposed as the equivalent to the cognate in the \isi{comparative method}, as they can be compared cross-linguistically (at least partially equivalent to Saussurean arbitrary form-meaning pairs \citealt{mm:goldberg_constructions_1995}, \cite[18]{mm:croft_explaining_2001}) and, they are hypothesized to be transmitted across generations (\citealt{mm:traugott_constructionalization_2013}; \citealt{mm:bardhdalgildea2015diachronic}). Crucially for syntactic reconstruction, constructions can be either lexical, e.g., \textit{cheeseburger} or schematic, e.g., ditransitive order [SVOO], which adds the semantics of transfer to a sentence like \textit{I'll bake you a pie} (\citealt{mm:goldberg_constructions_1995}). \citet{mm:giacalone_ramat_gradual_2013} employs DCxG and in particular the mechanism of “motivation”, which, according to him, explains the gradual rise in frequency of Absolute Initial Verb (AIV) order in Early Modern Welsh\il{Welsh (Modern)}. When it comes to reconstruction of syntax using DCxG\is{Diachronic Construction Grammar}, we find numerous recent attempts such as personal prefixes in \ili{Proto-Tupí-Guaraní} (\cite[322--323]{mm:gildea2002reconstructing}), dative experiencers in Germanic (\citealt{mm:bardaleythorson2012}), the subject possessor case marker in the \ili{Viceitic} (Chibchan) alienable possession construction (\citealt{mm:pacchiarotti_origins_2020}) and a number of other contributions in \citet{mm:barddalgildealujan2020}. \citet{mm:gildea_curious_2020} note that objections regarding constructions fail to meet the Double Cognacy Condition in (\ref{ex:DCC}) above (\cite[50]{mm:Walkden2014}) -- because sentences (i.e., the potential higher order units) they generate are not cognate -- are in fact based on a “domain error”. They argue that ``it is a logical error to require the equivalent of ``regular correspondences'' when seeking to identify cognates in other domains of language, such as syntax'' (\cite[18]{mm:gildea_curious_2020}). If this is indeed true we can still call constructions (or “patterns” for that matter) “cognate” simply because they can descend by inheritance from a common ancestor. 

Rejecting the need for the Double Cognacy Condition in syntax, however, also means that despite  suggested “safeguards” to identify patterns (\citealt{mm:harris_historical_1995}), similarities due to chance, language contact and/or typological factors are hard to identify. Some proposed methods to mitigate these issues, such as avoiding chance similarities (\cite[125--130]{mm:serzant_approach_2015} and \citealt{mm:gildea_curious_2020}) can only really work in cases of identity or strong similarity of the proposed cognates in the daughter languages. This, \citet{mm:willis_reconstructing_2011} observes, is a general weakness of the CxG\is{Diachronic Construction Grammar} approach to syntactic reconstruction: when there is structural \isi{reanalysis} (and thus a potential lack of identity in one or more daughter languages), this type of reconstruction can only work if it can rely on a straightforward account of how constructions are reanalyzed\is{reanalysis}, innovated or dissolved. \citet[35--38]{mm:traugott_constructionalization_2013}'s “neoanalysis” and “analogization” provide ideas in this direction. When it comes to identifying similarities due to language contact, \citet{mm:daniels_method_2017}  acknowledges this is an important issue and aims to provide a method for doing exactly this. His method exploits the possibility of a third type of construction, which is “somewhere between lexical and semantic” (\cite[382]{mm:daniels_method_2017}), e.g., as in [the X-er the Y-er] (\textit{The more you read, the less you understand}) (\cite[506--508]{mm:fillmore_regularity_1988}). \citet{mm:daniels_method_2017} builds on earlier work by \citet{mm:ross2015argument} and \citet{mm:serzant_approach_2015}, suggesting that in order to exclude contact, after hypothesizing correspondence sets, it is necessary to follow the next step in the traditional comparative method ensuring that the phonological material represented in the partially schematic constructions is in fact inherited. This then could work for those constructions that have at least some inherited lexical element, but it is hard to see how this method could be extended to purely schematic constructions. Even those with lexical elements cannot always provide a straightforward answer, as \citet{mm:clackson2017} has shown with his evaluation of the \textit{woe}-\textsc{dat} construction. Finally, the lack of reference to a theory of language acquisition makes it harder to extend the reconstruction of similar constructions to the reconstruction of a grammar that reaches ``the same level of analysis that one would expect for an attested language'' (\cite[10]{mm:willis_reconstructing_2011}). In the next section, I  discuss recent generative approaches to syntactic reconstruction that have aimed to do exactly that.

\subsection{Minimalist approaches to reconstruction}
\label{sec:minapproaches}

Using the Minimalist Program, \citet{mm:vangelderen2011linguistic} formalizes syntactic change through a feature cycle in which semantic features (e.g., a first-person pronoun) turn into interpretable formal features [iF:\textsc{1s}] and, finally, into uninterpretable features [uF] that can be probes. She connects common cases of syntactic reanalysis to labeling of phrases, which is seen as a driving factor of change (\citealt{mm:vangelderen2021third}). Using grammatical (formal) features in syntactic comparison and change goes back to the Borer-Chomsky Conjecture \citep[155]{mm:baker_macroparameter_2008}, which states the following shown in (\ref{ex:BCC}):

\is{Borer-Chomsky Conjecture|(}
\ea  Borer-Chomsky Conjecture (BCC)\smallskip\\
All parameters of variation are attributable to the features of the functional heads in the lexicon.
\label{ex:BCC}
\z 
\is{Borer-Chomsky Conjecture|)}
\is{BCC|see {Borer-Chomsky Conjecture}}

\noindent \citeauthor{mm:walkden_comparative_2009} (\citeyear{mm:walkden_comparative_2009}, \citeyear{mm:Walkden2014})  observes that this offers a solution to the correspondence problem in cases where functional items are overt. Cognacy of lexical items can be established in the usual way employing all the advantages of the traditional \isi{comparative method} for reconstruction, which means there is little room for chance or contact-induced similarities. 

\citet{mm:willis_reconstructing_2011} uses the example of British Celtic free relatives, i.e., the cognates of Modern Welsh\il{Welsh (Modern)} \textit{bynnag} `-ever', Middle Welsh\il{Middle Welsh} \textit{bynnac}, Middle Breton\il{Middle Breton} \textit{pennac} and Middle Cornish\il{Middle Cornish} \textit{penagh}. Just like in English\il{English (Modern)}, all these languages allowed for those forms to follow a \textit{wh}-word (cf.\ English\il{English (Modern)} \textit{whoever}, \textit{whatever}, etc.), but they differ in whether they allow for both strong and weak \textit{wh-}forms preceding reinforcing elements, e.g., Middle Cornish\il{Middle Cornish} \textit{penag-\textbf{el}} `who-ever-\textbf{all}', etc. He first describes the exact similarities and differences between the three languages and then asks what grammar we should reconstruct for the proto-language, before carefully laying out every stage of the abductive reanalyses and extensions. He starts from the output of Common British\il{Brittonic} \textit{py nag el} `whoever goes' with the pronoun \textit{py} with an  interpretable \textit{wh}-feature in SpecCP and the negator \textit{nag} + the verb \textit{el} incorporated into the C-head. At the next stage, the negator joins the \textit{wh}-pronoun in SpecCP opening up a slot in the C-head which can then, after extension, be filled with the regular pre-verbal particle \textit{a}, resulting in Middle Welsh\il{Middle Welsh} \textit{bynnag a el} `whoever PCL goes' (see \cite[25--36]{mm:willis_reconstructing_2011} for a full analysis). The strength of Willis's reconstruction lies in the meticulous attention to detail in justifying every reconstructed stage and form. He not only tests to what extent we can use traditional techniques for reconstruction such as majority and economy, but also refines the notion of directionality distinguishing “local directionality” from “universal directionality”, thus imposing important constraints on the possible reconstructions: the first testing plausible reanalyses that are consistent with detailed aspects of the known daughter language systems, the second through “broad-brush rules of thumb” akin to that in phonological reconstruction (\cite[36]{mm:willis_reconstructing_2011}). Finally, through describing and analyzing the microvariation between the daughter languages and carefully checking each stage of reanalyses and extensions, he tests which outputs could result from each stage of the reconstructed grammars, thus addressing the “radical reanalysis” problem. This method may be rather limited as it can generally only be applied to overt lexical items with functional features that have cognates in other languages, which means it also cannot work well for establishing reconstructed stages of languages that go very far back in time. However, the results are extremely reliable when each stage of the previous grammar and the various reanalyses and extensions are carefully checked. Further examples of exploiting true cognates with functional features can be found in \citet{mm:walkden_comparative_2009, mm:Walkden2014} for Germanic\il{Germanic language family} and \citet{mm:meelen_object-initial_2017} for pre-verbal particles in British Celtic.

Another recent Minimalist attempt at syntactic reconstruction uses the \isi{Parametric Comparison Method} (PCM) pioneered by \citet{mm:guardiano_parametric_2005} but extended significantly based on \citet{mm:roberts_parameter_2019}'s approach to microvariation and parameter\is{parameters} hierarchies. The \isi{Parametric Comparison Method} combines the goal of reconstructing language history with the analytical tools of formal grammar and cognitive science. This method ultimately goes back to the 1980s, with work on syntactic Principles and Parameters (P\&P) of Universal Grammar (UG), which concerns language acquisition, language change, markedness, implicational relations, learnability theory and opportunities for cross-linguistic comparison. \citet{mm:chomsky_binding_1980} draws on \citeauthor{mm:jacob_darwinism_1977}'s (\citeyear{mm:jacob_darwinism_1977}) ideas on biological diversification making a specific analogy to linguistics: ``the problem of accounting for the growth of different languages... is not unlike the general problem of growth, or for that matter, speciation'' (\cite[67]{mm:chomsky_binding_1980}). The P\&P approach offered solutions for Plato's Problem (i.e., the poverty of stimulus in language acquisition) and made excellent predictions for typological variation. However, within a Minimalist approach to cross\hyp linguistic variation, the idea that \isi{parameters} are part of the genetic endowment, the innate UG, is difficult to maintain, which is exactly what \citet{mm:roberts_parameter_2019} aims to remedy since proposals that abandon the notion of parameters altogether cannot offer viable alternatives (for a full discussion, see \citealt{mm:newmeyer_against_2005} and reviews and replies by \citealt{mm:dryer_review_2007} and \cites{mm:Walkden2014, mm:berwick_biolinguistic_2011}{mm:boeckx_thoughts_2015}[11--13]{mm:roberts_parameter_2019}). 

\citet{mm:ledgeway_cambridge_2017} follow \citet[6]{mm:chomsky_minimalist_1995} in the Borer-Chomsky Conjecture (BCC)\is{Borer-Chomsky Conjecture}: all microparameters\is{parameters} are  properties of individual heads, i.e., the formal features of functional heads in the syntactic structure. Cross-linguistic variation in syntax is variation in the values of these features (e.g., the C(omplementizer) head having C[+wh] vs C[−wh] to capture the distinction between \textit{wh}-movement and \textit{wh}-in-situ languages).  This aligns Roberts' earlier work (\citealt{mm:clark_computational_1993, mm:roberts_restructuring_1997, mm:roberts_syntactic_2003, mm:roberts_diachronic_2007}) with \citeauthor{mm:chomsky_three_2005}'s (\citeyear{mm:chomsky_three_2005}) proposal of the three factors of language design and makes it more concrete:

\ea Three factors relevant for parameters (\cite[7]{mm:roberts_parameter_2019})
\ea Factor 1: Underspecification of formal features in UG
\ex Factor 2: Trigger experience/what the child takes in (PLD)
\ex Factor 3: Feature economy (FE) and input generalization (IG)
\z
\z

\noindent Unlike earlier parametric theories, formal features are therefore not specified by UG, and parametric\is{parameters} variation arises from the interaction of the three factors. \citet{mm:biberauer_syntax_2015} and \citet{mm:biberauer_factors_2019} add that children postulate formal features on functional heads in cases where the PLD intake departs from the simplest Saussurean form\hyp meaning mapping such as cases of discontinuity and displacement (traditional “Move” or “Internal Merge”), multiple realization of elements (e.g., due to morphosyntactic agreement) and apparent non-realization of meaningful elements (i.e., silence). Similarly, we can recover a parameter setting for a particular language by looking at the overt realization of features, e.g., a participant, gender and/or number feature could be overtly realized through inflection or a clitic, as shown in the \ili{Fiorentino} inflectional paradigm in (\ref{ex:Fiorentino}):

\ea 
(e) parlo, tu parli, e/la parla, si parla, vu parlate, e/la parlano\\
`I/you/(s)he/we/you/they speak(s)' \jambox{(Fiorentino; \citealt{mm:brandi_two_1989})}
\label{ex:Fiorentino}
\z 


\largerpage
\noindent In addition to this straightforward spell-out (exponence, i.e., equivalent to F ↔︎ /p/ in Distributed Morphology), features could be “realized” in a more indirect way, e.g., when feature values are copied into functional heads via Agree with a defective goal leading to head-movement (\citealt{mm:roberts2010agreement}). When features are spelled out overtly, this approach comes close to the approach adopted by \citeauthor{mm:walkden_comparative_2009} (\citeyear{mm:walkden_comparative_2009}, \citeyear{mm:Walkden2014}) and \citet{mm:willis_reconstructing_2011} above, with the distinction that feature exponents in the microparametric\is{parameters} approach do not necessarily meet the Double Cognacy Condition, unless their exponents are in fact “true lexical cognates” that can be independently verified as “inherited” through their sound correspondences and the Regularity Hypothesis. The main advantage of reconstruction based on “true lexical cognates” resulting from the Double Cognacy Condition is that it allows us to eliminate resemblances due to chance, contact or typological factors. Examining cross-linguistic variation based on one single microparameter\is{parameters} would therefore yield rather limited results, which will need to be independently checked very carefully in order to establish whether they represent an inherited feature or not. Chance resemblance is furthermore problematic as all parameter\is{parameters} settings are binary and ``horizontal'' transfer due to contact may be harder to detect in syntax than it is for lexical items. The Parametric Comparison Method for examining cross-linguistic syntactic variation therefore relies on not one, but many \isi{parameters}, which effectively nullifies the opportunity to find chance resemblances of every parameter setting between any pair of languages and also reduces the potential impact of contact factors (see \sectref{sec:celticcontact} for how the PCM\is{PCM|see{Parametric Comparison Method}} could actually exploit potential lack of contact factors too). Typological factors, finally, are actually built into the method in the form of implicational factors and parameter hierarchies to which I turn now.

\citet{mm:baker_macroparameter_2008} observed that on the microparametric\is{parameters} view where variation lies in the feature make up of functional heads (i.e., the BCC) ``there should be many mixed languages of different kinds, and relatively few pure languages of one kind or the other'' (\cite[350]{mm:baker_macroparameter_2008}). Macroparameters\is{parameters} (\citealt{mm:baker_polysynthesis_1996}; \citeyear{mm:baker_macroparameter_2008}; \citealt{mm:boskovic_what_2008}; \citealt{mm:huang_syntactic_2015}), on the other hand, were traditionally proposed to be directly associated with UG principles, e.g., the Polysynthesis Parameter\is{parameters} (\citealt{mm:baker_polysynthesis_1996}), which is a general notion of “argument visibility” with two possible options: syntactic configurations vs formation of complex words. A macroparametric\is{parameters} view on cross-linguistic variation predicts that there is a rigid division of all languages into clear types (e.g., OV vs VO, polysynthesis or not, etc.). This means that every category in every language should pattern in one way or another. The empirical evidence, however, shows that instead we have a more bimodal distribution: when it comes to word order, for example, we find many languages clustering around one type or another, but there are also some exceptions to these principal patterns. This suggests that we need a combination of both micro- and macroparameters\is{parameters} to adequately account for cross-linguistic variation. Based on \citet[10]{mm:kayne2005movement} among others, \citet{mm:roberts_parameter_2019} argues that in addition to formal feature values of functional heads, there are Independent Factors (IF) ``which can cause groups of heads~-- over which we can generalize with an appropriately formulated feature system~-- to act in concert'' (\cite[5]{mm:roberts_parameter_2019}). This essentially turns macroparameters\is{parameters} into a surface effect of aggregates of microparameters\is{parameters} acting in unison, creating a strict hierarchy of \isi{parameters}, shown in \figref{ex:parahier}.

\begin{figure}
\small
% % % \includegraphics[width=12cm]{figures/meelen-LedgewayRoberts595.png}  
\begin{forest}
		for tree = {align=center, l sep+=\baselineskip}
		[Does p characterize L?
			[Macroparametric\\setting,edge label={node[midway, font=\footnotesize,sloped,above]{No}}]
			[All functional heads?,edge label={node[midway, font=\footnotesize,sloped,above]{Yes}}
				[Macroparametric\\setting,edge label={node[midway, font=\footnotesize,sloped,above]{Yes}}]
				[Extended to naturally\\ definable classes?,edge label={node[midway, font=\footnotesize,sloped,above]{No}}
					[Mesoparametric\\variation,edge label={node[midway, font=\footnotesize,sloped,above]{Yes}}]
					[Restricted to lexically\\ definable subclass?,edge label={node[midway, font=\footnotesize,sloped,above]{No}}
						[Microparametric\\variation,edge label={node[midway, font=\footnotesize,sloped,above]{Yes}}]
						[Limited to idiosyncratic\\ collection of individual\\ lexical items?,edge label={node[midway, font=\footnotesize,sloped,above]{No}}
							[Nanoparametric\\variation,edge label={node[midway, font=\footnotesize,right]{Yes}}]
						]
					]
				]
			]
		]
	\end{forest}
\caption{Parameter Hierarchies from \citet[595]{mm:ledgeway_cambridge_2017}}
\label{ex:parahier}    
\end{figure}

\citet{mm:biberauer_significance_2012, mm:biberauerroberts2017} thus propose a typology of \isi{parameters} where macroparameters\is{parameters} share a value of a variant feature on ALL heads; mesoparameters\is{parameters} share it on ALL heads \textit{of a given class} (e.g., [+V]), whereas microparameters\is{parameters} show that value on only a small, lexically-definable subclass of functional heads (e.g., subject clitics or modal auxiliaries) and, finally, nanoparameters\is{parameters} have this value specified to just one or more individual lexical items. This hierarchy makes specific predictions for syntactic typology, but also for language acquisition and change. Typologically, macroparameters\is{parameters} are expected to be pervasive or universal, affecting a large number of all relevant functional heads in the grammar of a language. Macroparameters\is{parameters} should be salient in the Primary Linguistic Data (PLD) and therefore easier to acquire and diachronically thus be more stable than microparameters\is{parameters}. Although this is still work-in-progress, a number of hierarchies have already emerged from the work by the Cambridge ReCoS group and others based on empirical evidence from a wide range of languages (see \citealt{mm:biberauerholmbergrobertssheehan2014, mm:sheehan2014, mm:vanderwal2020, mm:ledgeway_latin_2012, mm:ledgeway_cambridge_2017, mm:kinn2017, mm:douglas_unifying_2017} and mainly \citealt{mm:roberts_parameter_2019} and references therein).\footnote{See also \url{https://recos-dtal.mmll.cam.ac.uk/Publications/Published}.}

This radical revolution in parametric\is{parameters} theory, with well-established data-driven parameter\is{parameters} hierarchies, gives the \isi{Parametric Comparison Method} (PCM\is{Parametric Comparison Method}) a very solid ground for the reconstruction of language history. The PCM\is{Parametric Comparison Method} employs a parameter\is{parameters} grid with binary values (+/−) for each discrete parameter\is{parameters} in every language under investigation. This is based on a growing list of implicationally organized parameter\is{parameters} values consisting (in its most recent iteration) of 94 nominal \isi{parameters} tested on 69 languages (\citealt{mm:longobardi_principles_2017}; \citealt{mm:ceolin_formal_2020}), as well as at least 87 verbal \isi{parameters} for 39 languages (\citealt{mm:robertsbakermeelenetal2022}; \citealt{mm:bakerroberts2024}). These hierarchically organized implicational \isi{parameters} differ significantly from earlier attempts to use a list of binary features to encode “language structure” such as \citet{mm:nichols_linguistic_1992, mm:ringe_indo-european_2002, mm:dunn2005structural}. The latter only look at straightforward surface patterns without taking into account what kind of grammar could generate these patterns. Similarities in surface patterns are interesting but do not necessarily indicate that languages are historically closely related, even if the pattern in question is typologically unusual. \citet{mm:jongeling_comparing_2000} draws our attention to this listing numerous surface patterns in Welsh\il{Welsh (Modern)} that look remarkably similar to \ili{Biblical Hebrew} and finds an even closer parallel in \ili{Phoenician}, for example, with the unusual Determiner-Noun-Demonstrative order shown in (\ref{ex:NDEM1}):

\ea 
\ea
\settowidth\jamwidth{(Jongeling 2000:81)}
\gll y llances hon \\
the girl this\\ \jambox{[Welsh]}
\glt 'this girl'
\ex
\gll ha- nna\textsuperscript{c}ər\=a ha- z\=ot \\
the girl the this \\ \jambox{[Biblical Hebrew]}
\glt `this girl' 
\ex
\gll h gbr z'\textsuperscript{ɔ}  \\
the man this\\ \jambox{[Phoenician]}
\glt `this man' \jambox{(\cite[81]{mm:jongeling_comparing_2000})}
\z
\label{ex:NDEM1}
\z

\noindent Similarly, \citet{mm:roberts_review_1998}  notes that in \ili{Greek}, for example, the same linear and concord pattern (Noun-Genitive-Adjective) also exists in \ili{Hebrew}, but \citet{mm:longobardi_toward_2013} demonstrate that these similar surface patterns in fact reflect opposite values in two distinct parameters. In essence, the PCM\is{Parametric Comparison Method} here therefore requires the same amount of in-depth grammatical analysis as that shown by \citeauthor{mm:willis_reconstructing_2011}'s (\citeyear{mm:willis_reconstructing_2011}) reconstruction of the various stages of free relatives in British\il{Brittonic}: the question of how the surface patterns can be generated remains of crucial importance. 

A simple example of an implicational relation in the \isi{parameters} is shown by \citet{mm:longobardi_evidence_2009} as Nominal Parameter\is{parameters} 23 (Pn23), which distinguishes languages that exhibit concord (at least) in Number with a subject nominal in predicative constructions. This parameter\is{parameters} is contingent on the positive setting of two other \isi{parameters}, namely whether there is a Number feature on N (i.e., Parameter\is{parameters} 5 should be positive), which in turn can only be positive if there is a Number feature in the language in general (i.e., Parameter 2 should be positive). \citet{mm:longobardi_evidence_2009} note in their extensive Appendix listing all \isi{parameters} in detail that this is the parameter\is{parameters} that separates languages like English\il{English (Modern)}, \ili{German}, Irish\il{Irish (Modern)}, Welsh\il{Welsh (Modern)} and \ili{Farsi} with uninflected predicative adjectives from the rest of \ili{Indo-European}, as shown by examples of English\il{English (Modern)} and Irish\il{Irish (Modern)}, on the one hand (shown in \ref{ex:P23-}), and Standard \ili{Italian} and \ili{Triestino}\footnote{Many thanks to Christian Faggionato for these examples.} (Northern Italian variety spoken in Trieste), on the other (shown in \ref{ex:P23+} and \ref{ex:P23+bisiaco}):

\ea
\ea
\settowidth\jamwidth{[masc. predicative]}
\gll Tá an fear beag.\\
be.\textsc{3s} the man small\\\jambox{[masc. predicative]}
\glt `The man is small.'
\ex 
\gll Tá an bhean beag.\\
be.\textsc{3s} the man small\\\jambox{[fem. predicative]}
\glt `The woman is small.'
\ex
\gll Tá na daoine beag.\\
be.\textsc{3s} the people small\\ \jambox{[pl. predicative]}
\glt `The people are small.'
\ex
\gll Na daoine beag\textbf{a}.\\
the people small.\textsc{pl}\\  \jambox{[pl. attributive]}
\glt `the small people'
\z
\label{ex:P23-}
\z

\ea
\ea 
\settowidth\jamwidth{[masc. predicative]}
\gll L'uomo è piccol\textbf{o}.\\
the.man be.\textsc{3s} small.\textsc{sm}\\\jambox{[masc. predicative]}
\glt `The man is small.'
\ex 
\gll La donna è piccol\textbf{a}.\\
the woman be.\textsc{3s} small.\textsc{sf}\\\jambox{[fem. predicative]}
\glt `The woman is small.'
\ex 
\gll Le persone sone piccol\textbf{e}.\\
the people are small.\textsc{pl}\\\jambox{[pl. predicative]}
\glt `The people are small.'
\ex 
\gll le persone piccol\textbf{e}. \\
the people small.\textsc{pl}\\\jambox{[pl. attributive]}
\glt `the small people'
\z
\label{ex:P23+}
\z

\ea
\ea 
\settowidth\jamwidth{[masc. predicative]}
\gll L'omo xe pici\textbf{o}. \\   
the.man is small.\textsc{sm}\\ \jambox{[masc. predicative]}
\glt `The man is small.'
\ex 
\gll La dona xe pici\textbf{a}. \\
the woman is small.\textsc{sf}\\ \jambox{[fem. predicative]}
\glt `The woman is small.'
\ex 
\gll Le persone xe pici\textbf{e}. \\
the people are small.\textsc{pl}\\ \jambox{[pl. predicative]}
\glt `The people are small.'
\ex 
\gll le persone pici\textbf{e} \\
the people small.\textsc{pl}\\ \jambox{[pl. attributive]}
\glt `the small people'
\z
\label{ex:P23+bisiaco}
\z

\noindent The PCM\is{Parametric Comparison Method} can be used to compare a small number of individual \isi{parameters} (or even a single parameter\is{parameters}) related to one particular grammatical feature, as is the case of Number in the DP with parameters 2, 5 and 23 above. These can also be compared in different diachronic stages of a single language to trace syntactic changes over time (see \sectref{sec:PCMforhistcelt} below). 

However, another potential strength of the PCM\is{Parametric Comparison Method} lies in its ability to compare a large number of \isi{parameters} in many different languages to create a phylogeny that can tell us how languages are historically related to each other. One advantage is, for example, that this method can be applied to any set of languages, regardless of how similar or different they are, because the \isi{parameters} are drawn from a finite, universal list. Their discrete, binary nature lends itself to precise calculations of differences between languages. There are two main methods for these types of phylogenetic computational comparisons: distance-based methods (e.g., Neighbor joining, FITCH/KITCH, UPGMA) or character-based methods (e.g., Maximum Parsimony, Bayesian analysis). The latter type of methods are based on optimizing the probability of a proposed tree given the observed data, which mainly consists of databases of lexical cognates such as \citet{mm:dyen_indoeuropean_1992}. These methods have gained much popularity in computational historical linguistics in the last decades and are mainly used to show taxonomies as well as depth of splits of family branches (\citealt{mm:holden_bantu_2002}; \citealt{mm:atkinson_qd_curious_2005}; \citealt{mm:bouckaert2012mapping}, etc.). Character-based methods crucially rely on the assumption that characters (in this case lexical cognates) are independent. The arbitrary form-meaning pairing of cognates makes them plausible candidates for this type of probabilistic modeling. However, as \citet{mm:jager_phylogenetic_2013}, among others, has observed, this level of independence cannot be maintained when it comes to the regular sound changes: ``It rests on the simplifying assumptions that mutations at different positions are stochastically independent and that mutation probabilities are constant across lineages'' (\cite[286]{mm:jager_phylogenetic_2013}.) More recent character-based studies aim to address earlier issues with this method in various ways, e.g., through finding methods to create better and more reliable input data (especially for lesser-studied language families where extensive cognate databases are still a desideratum, see \citealt{mm:list_concepticon_2016}; \citeyear{mm:list_potential_2017}) or by adding specific priors (e.g., \citealt{mm:rama2018three}), weights or other additional information (e.g., \citealt{mm:neureiter2022detecting} attempting to include the possibility of horizontal transfer, i.e., language contact). Although these methods have achieved interesting results, with a precise representation of the history of a language family including potential dates of splits and information about how reliable each part of a given taxonomy is, \citet{mm:hammarstrom_computational_2019} and others have pointed out that some issues remain with forcing cognate classes to be binary characters (as is necessary for these models). Lexical cognates are not always perfect pairs of 1 form + 1 meaning, and often meanings have changed to the extent it is hard to establish cognacy. Finally, certain meanings (or lexical items) can be lost from the language entirely (\citealt{mm:meelenetal2022cognates}).

Parameters\is{parameters}, on the other hand, are binary by definition, but since they interact and are part of implicational hierarchies, they cannot form the basis of off-the-shelf methods as they would violate the independence of character assumption. The effect of these interdependent relations between \isi{parameters} is not negligible: compared to total independence, the number of possible languages is reduced by seven orders of magnitude (with a set of 75 \isi{parameters}) showing that implicational relationships between \isi{parameters} play a key role (see \citealt{mm:bortolussi_how_2011}  and \citealt{mm:longobardi_principles_2017} for tests of statistical significance of this and other parts of the PCM\is{Parametric Comparison Method}). Similarly, \citet{mm:ceolin_at_2021}  calculate that nearly 40\% of the taxonomic input is neutralized in a dataset of 94 nominal \isi{parameters} set for 58 languages.  For this and other reasons (see \citealt{mm:mcmahon_language_2005} and others), the PCM\is{Parametric Comparison Method} uses distance-based methods to create phylogenies instead. As opposed to character\hyp based algorithms, distance-based ones use pairwise distances between languages based on the identities and differences in the comparanda, i.e., each of the parameters set positively “+” or negatively “−”. In addition to these binary “\pm” values, parameter\is{parameters} settings could be redundant or irrelevant indicated by `0' as they are implied by a particular setting of other \isi{parameters} as we have seen in the interdependent relation of the \isi{}parameters referring to the Number feature above. Finally, the positive setting of a parameter\is{parameters} requires some specified empirical evidence (\citealt{mm:crisma_syntactic_2020}  and the discussion on the realization of features above), whereas negative settings represent default values. It follows then that neither pairs that contain `0', nor pairs that are both default (Language A “−” \textit{and} Language B “−”), should be part of the distance calculation, which is exactly what the Jaccard distance, shown in (\ref{ex:jaccard}), corresponds to (with $N$ as the number of pairs of parameter\is{parameters} settings in two languages A and B):\footnote{Note that in earlier presentations of the PCM\is{Parametric Comparison Method}, e.g., \citet{mm:longobardi_evidence_2009}, Normalized Hamming distances were used leaving out the potential significance of negative settings as the default. As work on the PCM\is{Parametric Comparison Method} develops and more research goes into the exact \isi{parameters} and their hierarchical relations, it is clear that Jaccard distances yield a better representation of the historical linguistic situation. See supplementary materials in \citet{mm:ceolin_at_2021} for further discussion.}
\begin{equation}
\Delta \text{Jaccard}(\text{Lang}_A,\text{Lang}_B) = \frac{[N_{-+} + N_{+-}]}{[N_{-+} + N_{+-} + N_{++}]}
\label{ex:jaccard}
\end{equation}

\noindent \citeauthor{mm:ceolin_formal_2020} (\citeyear{mm:ceolin_formal_2020}, \citeyear{mm:ceolin_at_2021}), \citet{mm:longobardi2023grammatical} and others demonstrate further applications of the PCM\is{Parametric Comparison Method} ranging from the possibilities for the reconstruction of ``deep language history'' to neural implementations, i.e., how can binary \isi{parameters} be conceivably implemented in the cortical circuitry in the human brain? However interesting these opportunities, I leave those aside in this chapter and instead, before moving on to more Celtic data, highlight one important observation that has direct implications for the syntactic history of a set of languages, namely, the “Anti-Babelic principle”:

\eanoraggedright
Anti-Babelic principle (\citealt{mm:guardiano_parametric_2005})\smallskip\\
Similarities among languages can be due either to historical causes (common origin or, at least, secondary convergence) or to chance; differences can only be due to chance (no one ever made language diverge on purpose)
\z

\noindent Although the strict version of the Anti-Babelic principle cannot hold (see, among others, examples in \cite{mm:thomason_language_2007}), deliberate language change is certainly not the norm and there is ample evidence that syntactic features in particular, unlike phonetic features, are less likely to be consciously adjusted (see, for example, \citealt{mm:smith2000synchrony}). The importance of this principle for the PCM\is{Parametric Comparison Method} is that it makes clear predictions about historically significant distributions of languages. If languages are known to be closely related, we expect a distance $\delta$ between 0 and 0.5, whereas the distances of other languages should tend towards 0.5 and there should be very few (if any) language pairs with a distance between 0.5 and 1. \citet{mm:bakerroberts2024} decide, based on this and calculations by \citet{mm:bortolussi_how_2011}, that any value of $\delta<0.20$ indicates probable genetic relatedness, whereas any value of $\delta>0.40$ is likely to be random. They illustrate this with results from 87 verbal parameters in known language families where Celtic, \ili{Slavic language family} and Romance\il{Romance language family} languages have an average $\delta=0.15$ and \ili{Sinitic} and \ili{Semitic} have an average $\delta=0.18$ and $\delta=0.19$, respectively. Larger families such as Germanic\il{Germanic language family} or all of the \ili{Indo-European} languages in their sample ($n=39$) have $\delta=0.22$ and $\delta=0.27$ respectively, whereas a comparison of all languages in their database yields $\delta=0.37$. By comparison, if we focus on just 56 parameters of the nominal domain, Celtic languages are even more closely related with $\delta=0.02$ (according to \cite[257]{mm:longobardi_principles_2017}). 

In the next few sections, I first show why Celtic languages are so important for the development of these types of Minimalist syntactic reconstruction methods.  Since there already are good examples in Celtic of using cognate functional items such as \citet{mm:willis_reconstructing_2011}'s reconstruction of British\il{Brittonic} free relatives summarized, I focus here on parametric\is{parameters} Minimalist methods only. For both the verbal and the nominal parametric\is{parameters} distance results discussed above, only data from Modern Irish\il{Irish (Modern)} and Modern Welsh\il{Welsh (Modern)} were taken into account, but here I extend this with data from older stages of these languages (Middle Welsh\il{Middle Welsh} and \il{Old Irish}) as well as occasional examples from other Celtic languages like Cornish\il{Cornish (Modern)} and Breton\il{Breton (Modern)}. I then show how the PCM\is{Parametric Comparison Method} in particular can help advance research on Celtic languages. Finally, I discuss some strengths and limitations of both types of Minimalist approach, ultimately arguing that we need both for a more comprehensive syntactic reconstruction.

\section{How Celtic can help Minimalist syntactic reconstruction}

The Celtic branch of \ili{Indo-European} forms an ideal testing ground for the PCM\is{Parametric Comparison Method}. First of all, it contains a number of languages, both with attested older stages and with languages that are still spoken in the present day. The latter gives us the opportunity to get native-speaker judgments on every possible parameter providing the kind of detailed syntactic information that is necessary for in-depth analyses. As for the older stages, Celtic has both inscriptions and manuscripts from a wide range of times and places. These show that while there is no doubt that languages like \ili{Lepontic}, \ili{Gaulish}, Irish\il{Irish (Modern)} and Welsh\il{Welsh (Modern)} are related, they exhibit numerous differences~-- especially in the syntax of their oldest stages. Whereas some branches of \ili{Indo-European} are relatively uniform in terms of, for example, “basic word order” (e.g., verb-second in Germanic), Continental Celtic inscriptions show important variation from their Insular Celtic counterparts. These differences within a set of languages that have been grouped together through the traditional comparative method of phonological reconstruction provide us with the opportunity to enhance and refine the set of parametric questions we ask. Finally, from a purely geographical point of view the Celtic languages are of crucial importance in the reconstruction of \ili{Indo-European} as they provide an excellent test case for what can happen at the forefront of \ili{Indo-European} expansion (\citealt{mm:ramprasadmeelen2022}). In what follows, I show how certain aspects of Celtic syntax can help to refine the set of parameters used in the PCM\is{Parametric Comparison Method}, and how evidence from some Celtic languages like Middle Cornish \il{Middle Cornish} can provide more insight into issues of syntactic transfer in contact situations.

\subsection{Celtic languages refining the PCM}
\label{sec:celticrefiningPCM}

From a syntactic point of view, the Insular Celtic languages in particular are interesting since their verb-initial nature sets them apart from other \ili{Indo-European} languages. Other syntactic features that are found in Celtic, but not in other Western European languages include inflected prepositions, shown in (\ref{ex:infprep}) with examples from Cornish\il{Cornish (Modern)} and Breton\il{Breton (Modern)}, and the lack of a `have'-like predicate to express possession in Welsh\il{Welsh (Modern)} and Irish\il{Irish (Modern)} in particular, shown in \REF{ex:have}.) 

\ea
warnav/-nas/-nodho/-nydhi/-nan/-nowgh/-nedhe \hspace{0.7cm} [Middle Cornish] \\
warnon/-nout/-nañ/-ni/-nomp/-noc'h/-nezho \hspace{1.5cm}[Breton] \\
\glt `on me/you/him/her/us/you(\textsc{pl})/them' (\citealt{mm:toorians_towards_2014} \& \cite[30]{mm:wmffre_central_1998})
\label{ex:infprep}
\z

\ea
\ea
\settowidth\jamwidth{[Welsh]}
\gll Mae gen i feic/annwyd\\
is with.\textsc{1s} me bike/cold\\ \jambox{[Welsh]}
\glt `I have a bike/cold.'
\ex
\gll Tá teach ag an mbean.\\
is house with the woman\\ \jambox{[Irish]}
\glt `The woman has a house.'
\z
\label{ex:have}
\z

\noindent Modern Breton\il{Breton (Modern)}, however, does have a lexical verb `have' namely \textit{kaout/endevout} (shown in \ref{ex:brethave1}), but this verb is historically derived from the verb \textit{bezañ} `be' + a prefixed personal pronoun. Breton\il{Breton (Modern)} can also indicate possession using a preposition \textit{gant} `with' (shown in \ref{ex:bretwith}), which is cognate with Welsh\il{Welsh (Modern)} \textit{gan} `with' above.

\ea 
\ea 
\settowidth\jamwidth{[Breton]}
\gll Bremañ Azenor ha Iona o deus un ti\\
now Azenor and Iona 3\textsc{p} have.\textsc{pres} a house\\ \jambox{[Breton]}
\glt `Azenor and Iona have a house now.' \jambox{(\cite[373]{mm:jouitteau_syntaxe_2005})}
\label{ex:brethave1}
\ex
\gll Ganti e oa teir yar.\\
with.\textsc{3sf} \textsc{prt} be.\textsc{impf}.\textsc{3s} three hen\\  \jambox{[Breton]}
\glt `She had three hens.' \jambox{(\cite[140]{mm:press_grammar_1986})}
\label{ex:bretwith}
\z
\z

\noindent However, these examples of syntactic variation within and beyond the Celtic language family are not covered by the currently-available sets of \isi{parameters} in the nominal and the verbal domains (see \citealt{mm:crisma_syntactic_2020} for 94 nominal \isi{parameters} and \citealt{mm:bakerroberts2024} for 87 verbal \isi{parameters}).

Furthermore, although insular Celtic languages show remarkable similarities in terms of syntax, especially in the nominal domain, there are also cases of (micro)variation. Another example of such variation that is currently not captured by the PCM\is{Parametric Comparison Method} framework concerns resumptive pronouns. Resumptive pronouns are generally not available in subject positions cross-linguistically, even in languages like Irish\il{Irish (Modern)} which usually allow resumptive pronouns in other positions, e.g., direct objects, shown in (\ref{ex:res}):

\ea
\ea
\settowidth\jamwidth{[Irish subject relative]}
\gll an fear a raibh (*sé) breoite\\
the man rel was he ill\\ \jambox{[Irish subject relative]}
\glt Intended: `the man who (he) was ill' \jambox{\citep[210]{mm:mccloskey_resumptive_1990}}
\ex 
\gll an fear ar bhuail tú é\\
the man rel struck you him\\ \jambox{[Irish object relative]}
\glt `the man that you struck (him)' \jambox{\citep[206]{mm:mccloskey_resumptive_1990}}
\z
\label{ex:res}
\z

\noindent In Modern Irish\il{Irish (Modern)}, resumptives are banned in subject position only in the main clause of a \textit{wh}-construction. This so-called ``Highest Subject Restriction'' is also found in \ili{Hebrew} and \ili{Palestinian Arabic} (\citealt{mm:shlonsky_resumptive_1992} and \citealt{mm:alexopoulou_resumption_2006}). In Modern Welsh\il{Welsh (Modern)}, on the other hand, resumptive pronouns are not allowed in subject position in main clauses and in those embedded clauses where a gap is possible (\cite[108]{mm:borsleyetal2007syntax}). 

\ea 
\ea 
\settowidth\jamwidth{[Welsh subject relative]}
\gll y dyn gafodd (*e) 'r wobr\\
the man get.\textsc{past}.\textsc{3s} (he) the prize\\ \jambox{[Welsh subject relative]}
\glt `the man who got the prize'
\ex 
\gll y ffrwydrad glywais i (*e) wedyn\\
the explosion hear.\textsc{past}.\textsc{1s} I (it) then\\ \jambox{[Welsh object relative]}
\glt `the explosion I heard then' \jambox{(\cite[119]{mm:borsleyetal2007syntax})}
\z
\label{ex:noresWelsh}
\z

\noindent Welsh\il{Welsh (Modern)} does allow resumptives in relatives based on possessives and objects of prepositions, as shown in (\ref{ex:resWelsh}):

\ea 
\ea 
\settowidth\jamwidth{[Welsh]}
\gll y myfyrwyr werthodd Ieuan y ceffyl iddyn \textbf{nhw}\\
the students sell.\textsc{past}.\textsc{3s} Ieuan the horse to.3P them\\ \jambox{[Welsh]}
\glt `the students who Ieuan sold the horse to'
\ex 
\gll y dyn welais i \textbf{ei} chwaer\\
the man see.\textsc{past}.\textsc{1s} I \textsc{3sm} sister\\ \jambox{[Welsh]}
\glt `the man whose sister I saw' \jambox{\citep[121]{mm:borsleyetal2007syntax}}
\z
\label{ex:resWelsh}
\z

\noindent This Welsh\il{Welsh (Modern)} pattern is also found in other languages, e.g., \ili{Czech} (\citealt{mm:toman_discussion_1998}) and \ili{Berber} and \ili{Turkish} (\citealt{mm:ouhalla_subject-extraction_1993}). \citet{mm:mccloskey_resumptive_1990} and \citet{mm:willis2000distribution}, among others, have analyzed this as a constraint on binding with respect to pronouns in certain domains (the A'-Disjointness Requirement). More research on this microvariation in the distribution of resumptive pronouns in languages like Irish\il{Irish (Modern)} and Welsh\il{Welsh (Modern)} can therefore clearly provide more comparanda for the PCM\is{Parametric Comparison Method}, which in turn will help it make better representations of differences and similarities between subbranches in particular, but also across language families.

When it comes to the nominal domain, the first set of large-scale parametric comparisons to create phylogenetic trees (\citealt{mm:guardiano_parametric_2005}) was based only on a set of 54 distinct \isi{parameters}. As research on the PCM\is{Parametric Comparison Method} evolved, these \isi{parameters} became more and more refined and further parametric\is{parameters} distinctions were added. This was partly due to new discoveries in the field of nominal syntax, but also because more languages, especially non-Indo-European ones, were added to the PCM\is{Parametric Comparison Method} databases. Even within the most recent publications using a total of 94 nominal parameters and a total number of 69 languages from a wide range of language families, Celtic is only represented by Modern Irish\il{Irish (Modern)} and Welsh\il{Welsh (Modern)}, however. It is therefore clear that more research on the syntax of Celtic languages, especially older stages as well as those not yet queried by the PCM\is{Parametric Comparison Method}, such as Scottish Gaelic\il{Scottish Gaelic (Modern)} and Modern Breton\il{Breton (Modern)}, can help improve Minimalist syntactic reconstruction.

\subsection{Celtic languages in contact}
\label{sec:celticcontact}

\noindent It is easier to detect similarities due to language contact with Neogrammarian lexical cognates. There is no clear equivalent to the regularity of sound laws in syntactic change, but parameter\is{parameters} hierarchies and the implicational relations found in theoretical syntax do provide some opportunities to distinguish inherited structure from transfer due to language contact. \citet{mm:ledgewayforth}, for example, show that syntactic changes due to contact do not follow the step-by-step parameter\is{parameters} hierarchy from macro- to meso-, micro- and nano- or vice versa. Recall from \sectref{sec:minapproaches} that nanoparameters involve features on individual lexical items, whereas microparameters\is{parameters} refer to a lexically definable subclass of functional heads, e.g., auxiliary verbs. Mesoparameters\is{parameters} deal with core functional categories or larger natural classes of functional heads and their effects. They differ from macroparameters\is{parameters} as they involve formal features of some but not \textit{all} functional heads acting together, such as head-movement in Germanic\il{Germanic language family} and Romance\il{Romance language family}. When it comes to endogenous changes, we expect \isi{parameters} to change step-by-step from one level to the next, as shown by Romance\il{Romance language family} active participle agreement. \citet{mm:ledgewayforth} observe that parametric comparison of Romance\il{Romance language family} and \ili{Greek} varieties spoken in the south of Italy instead show abrupt and saltatory movements through parametric\is{parameters} space indicating exogenous change, i.e., changes due to contact. The Romance\il{Romance language family} language \ili{Aspromonte Calabrese}, for example, which has been in contact with \ili{Greko} (a variety of Greek spoken in Italy), has lost auxiliary selection completely, ``jumping'' from a nanoparametric\is{parameters} setting with auxiliary selection sensitive to argument structure to a macroparametric\is{parameters} setting found in Greko that has no auxiliary selection at all. Similarly, in \ili{Old Calabrese} syncretic marking of internal arguments was restricted in a nanoparametric fashion to 1/2 person only. Endogenous changes in this and non-contact varieties in Romance\il{Romance language family} led to the loss of this 1/2 person restriction, moving one step up the hierarchy. However, contact with \ili{Greko} then led to a subsequent jump up two levels much closer (but not identical) to the level of \ili{Greko}, showing that exogenous changes can lead to hybridism which is more like approximation than convergence, but still predicted through the strict distinctions made by \isi{parameters}. Although the biggest leap or ``catapult effect'' from nano- to macroparametric\is{parameters} setting is actually predicted for some forms on endogenous change as well (e.g., passives), \citet{mm:ledgewayforth}  show that parameter\is{parameters} hierarchies can still be used as a good heuristic for detecting contact-induced change.

We can explore the issue of transfer further through investigating a group of closely-related languages, where we know some have undergone more contact-induced change than others. The British Celtic languages form an excellent case study, since both Breton\il{Breton (Modern)} and Cornish\il{Cornish (Modern)} are clearly heavily influenced by French\il{French (Modern)} and English\il{English (Modern)} respectively, with which they were/are in close contact. For Cornish\il{Cornish (Modern)}, \citet{mm:padel_english_2021} notes that ``Cornish shows extensive influence from English'' (256), even in the earliest texts going back to the 15th century, because Cornwall by that time had been an administrative county of England for five centuries already. Further evidence comes from the widespread use of Anglo-Saxon forenames in the area as well as the numerous loanwords that are fully assimilated into Cornish\il{Cornish (Modern)} found in the \textit{Old Cornish Vocabulary}, dating from the later twelfth century (\citealt{mm:graves_old_1962}). \citet{mm:humphreys_breton_1992}  finds similar patterns for Breton\il{Breton (Modern)}, which is heavily influenced by French\il{French (Modern)} already in the medieval period as it ceased to be the language of the ruling class in the 12th century.

When it comes to syntax, there are clear differences between Breton\il{Breton (Modern)} and Cornish\il{Cornish (Modern)}, on the one hand~-- sharing certain features with French\il{French (Modern)} and English\il{English (Modern)}, respectively~-- and Welsh\il{Welsh (Modern)}, on the other. This poses the question of whether these differences are due to language contact. Where Irish\il{Irish (Modern)} and Welsh\il{Welsh (Modern)} lack a lexical verb `have', as shown in \sectref{sec:celticrefiningPCM} above, both Breton\il{Breton (Modern)} and Cornish\il{Cornish (Modern)} have innovated these through the grammaticalization of an infixed object pronoun and a form of the verb `be', as shown in (\ref{ex:brethave}a) for Middle\il{Middle Breton} and (\ref{ex:brethave}b) for Modern Breton\il{Breton (Modern)}:

\ea
\ea
\settowidth\jamwidth{(B 176.766)}
\gll mil	uileny	diouz		tut			an	bro		ouz		bezo				huy\\
	1000	insult		from		people	the	land	\textsc{2pl}		be.\textsc{fut}.\textsc{3s}	you\\
	\glt `you'll have 1000 insults from the people of the land' \jambox{(B 176.766)}
\ex
 \gll Ur c'hi {ac'h eus}? - N {'em eus} ki ebet.\\
 Q dog have.\textsc{2s} - \textsc{neg} have.\textsc{1s} dog none\\
 \glt `Do you have a dog? - I haven't got a dog.' \jambox{\citep[57]{mm:press_colloquial_2004}}
\z
\label{ex:brethave}
\z

\noindent Although this grammaticalization could potentially be reconstructed for \ili{Proto-British} if occasional instances in early poetry show this innovation was incipient in Welsh\il{Welsh (Modern)} as well (\cite[50--52]{mm:lewis_llawlyfr_1966}), the increased level of contact with languages that did have lexical `have' like English\il{English (Modern)} and French could have accelerated this grammaticalization in Breton\il{Breton (Modern)} and Cornish\il{Cornish (Modern)}. The subsequent developments of using the verb ‘have’ to form a perfect is directly modeled on French\il{French (Modern)} (for Breton\il{Breton (Modern)}, cf. \citealt{mm:WillisForth}). For Cornish\il{Cornish (Modern)}, \citet[328]{mm:Williams2011} cites one example shown in (\ref{ex:Cornishperf}), which could be influenced by Middle English\il{Middle English}. However, \citet[258]{mm:padel_english_2021} argues that Cornish\il{Cornish (Modern)} never developed an analytic perfect with the verb ‘to have’, and (\ref{ex:Cornishperf}) shows a stative verb, so this subsequent change may only have occurred in Breton\il{Breton (Modern)}.

\ea 
\settowidth\jamwidth{(Middle Cornish)}
\gll flehys a m bes denethys \\
children \textsc{prt} \textsc{1s} have.\textsc{3s} beget.\textsc{ppp}\\ \jambox{(Middle Cornish)}
\glt `I have begotten children' \jambox{(CW 1979)}
\label{ex:Cornishperf}
\z 

\noindent Another example of grammaticalization is the development of the Cornish\il{Cornish (Modern)} volitional verb \textit{mynnes} `to wish' to a future marker. This is crosslinguistically common (e.g., \ili{Danish} \textit{ville}, \ili{Greek} \textit{tha} < \textit{thelo ina} `I wish that', \ili{Bulgarian} \textit{šte} < `I want that'), but not found in Welsh\il{Welsh (Modern)} or Breton\il{Breton (Modern)}. Old English\il{Old English} \textit{wile} `want, wish' (> PDE future \textit{will}) unlike other PDE auxiliaries that only grammaticalized later, could already have future meaning before the Middle English\il{Middle English} period (\citealt{mm:warner_english_1993}). This early grammaticalization in English\il{English (Modern)} makes it a good candidate for contact-induced, or at least contact-accelerated grammaticalization in Cornish\il{Cornish (Modern)}.

Similarly, if we compare the word order in declarative matrix clauses across the medieval British\il{Brittonic} languages, we can see a clear V2 pattern, which can be reconstructed to \ili{Proto-British} (alongside V1) (\citealt{mm:meelen_why_2016}, \citealt{mm:WillisForth}). In Middle Cornish\il{Middle Cornish}, however, there are many more exceptions to strict V2 with subject pronouns following initial constituents, shown by the following examples from \citet[151, 155]{mm:eska_remarks_2021}:

\ea
\ea
\settowidth\jamwidth{(PC 312.1150–1151)}
\gll an	dragan	me	a 			ra			 					guan \\
	the	dragon	I		\textsc{prt}		make.\textsc{prs}.\textsc{3s} 	weak\\ \jambox{[Middle Cornish]}
\glt `I shall make the dragon weak.' \jambox{(BM 232.4004)}
\ex 
\gll the	`th		scoforn		wharre		yehes		sur			me	a			re\\
	to	your	ear				soon			health	surely	I		\textsc{prt}		give.\textsc{prs}.\textsc{3s}\\
 \glt I shall give, surely, healing to your ear soon.'\jambox{(PC 312.1150–1151)}
\z
\z

\noindent These could be due to poetic meter (\citealt{mm:eska_remarks_2021}). However, the examples with unstressed subject pronouns are also found in neighboring Old English\il{Old English} and southern Middle English\il{Middle English} (\citealt{mm:krochetal2000middle}), language contact is very likely to have influenced this Cornish\il{Cornish (Modern)} innovation as well  (\cite{mm:WillisForth}).

An example of an innovation in Middle Breton\il{Middle Breton} and Cornish\il{Middle Cornish} that is very clearly modeled on French\il{French (Modern)} and English\il{English (Modern)}, respectively, is the passive. Similar to Irish\il{Irish (Modern)}, voice alternations in \ili{Brittonic} can be expressed through impersonal verbal morphology, which is present across all three medieval languages, e.g., Middle Cornish\il{Middle Cornish} \textit{gylwyr} `is called', \textit{gyllyr} `one can', Middle Breton\il{Middle Breton} \textit{mireur} `is watched', \textit{guilir} `one sees' and Middle Welsh\il{Middle Welsh} \textit{byddir} `one may be', \textit{lladawr} `will be killed'. When it comes to specific passive features, such as the use of `by'-phrases, there is variation with Middle Breton\il{Middle Breton} and Welsh\il{Middle Welsh} lacking by-phrases with impersonal morphology, where this may be found in Middle Cornish\il{Middle Cornish}, as shown in (\ref{ex:cornby}).  The use of the by-phrase in later Welsh\il{Welsh (Modern)} and Breton\il{Breton (Modern)} appears to be a parallel innovation due to contact with English\il{English (Modern)} and French\il{French (Modern)}, respectively (\citealt{mm:WillisForth}). 

\ea
\settowidth\jamwidth{[Middle Cornish]}
\gll maythogh keris gans lues\\
that love.\textsc{imp} by many\\\jambox{[Middle Cornish]}
\glt `that you are loved by many' \jambox{(BM 272)}
\label{ex:cornby}
\z 

\noindent Examples like the above in Cornish\il{Cornish (Modern)} with \textit{-is/-ys} can be ambiguous, however, between imperfect impersonal morphology and past participles with left-out auxiliary `be'. Periphrastic passives with `be' + past participle are also found in Breton\il{Breton (Modern)}, which just like in Cornish\il{Cornish (Modern)} is due to language contact (\cite[49--50]{mm:lewis_llawlyfr_1966}, \cite[125--126]{mm:leroux1957}).

\ea 
\ea
\settowidth\jamwidth{[Middle Cornish]}
\gll prak gans pup na vethaf lethys \\
why by all \textsc{neg} be.\textsc{1s} slay.\textsc{ppp}\\ \jambox{[Middle Cornish]}
\glt `why by all I may not be slain' \jambox{(O.595-96)}
\ex
\gll ha pa ez vezo laqueat mat euez \\
and when \textsc{prt} be.\textsc{fut}.\textsc{3s} put.\textsc{ppp} good attention\\ \jambox{[Middle Breton]}
\glt `and when good attention has been (will be) paid...'  \jambox{(Cathell 5)}
\z
\z

\noindent Middle Welsh\il{Middle Welsh}, on the other hand, innovates a \textit{cael-}passive, which is typologically similar to the English\il{English (Modern)} \textit{get-}passive, although this was probably an internal innovation (\citealt{mm:WillisForth}). \citet{mm:bakerroberts2024} call the verbal parameter that captures the distinction between periphrastic and non-periphrastic passives Pv38 (VSV) `Voice spread to V', which essentially asks whether voice is realized both on V and separately. The subtle distinctions between different types of periphrastic constructions exhibited by British Celtic languages are not captured by any specific parameter yet, however. Note that opportunities for further extension of the parameters in the passive domain could be based on the suggestion that the \textit{by}-phrase doubles the passive argument \textit{-en} (\cite[223]{mm:baker_passive_1989}), which is compatible with the observation that there is a distinction between transitive and unaccusative readings of, for example, `cost' in Welsh\il{Welsh (Modern)} when they are used in the impersonal form, as shown in the Modern Welsh\il{Welsh (Modern)} examples in (\ref{ex:cost}) (see \cite[133--136]{mm:arman_welsh_2015}  and \cite[425fn5]{mm:roberts_parameter_2019} for discussion):

\ea \label{ex:cost}
\ea 
\gll costiwyd y CD yn ddeg-punt gan y cwmni.\\
cost.\textsc{past}.\textsc{imp} the CD \textsc{pred} ten-pounds by the company\\ 
\glt `The CD was costed at 10 GBP by the company.'
\ex
\gll *Costiwyd (yn) deg-punt (gan y CD).\\
cost.\textsc{past}.\textsc{imp} \textsc{pred} ten-pounds by the CD\\
\z
\z

\noindent All of the examples in this section so far do show that when we know enough about the history of languages and their historical contact situations, this can inform our syntactic reconstructions. British Celtic languages form an excellent case study for this since we have medieval data for all three and despite the fact that they are closely related, we do see clear syntactic differences as well. All of these differences can be due to transfer from French\il{French (Modern)} (in Breton\il{Breton (Modern)}) and English\il{English (Modern)} (in Cornish\il{Cornish (Modern)}), because we know about the socio-historical situation in which these languages came into contact. Furthermore, we know enough about the historical stages of both French\il{French (Modern)} and English\il{English (Modern)} to be able to answer questions regarding syntactic parameters like Pv38 on periphrastic passives, which is set positively in both Middle English\il{Middle English} and Middle French\il{Middle French}.

The PCM\is{Parametric Comparison Method} offers another possible way to explore cases of (potential) syntactic transfer. Since loanwords are essentially cast aside before applying the comparative method to reconstruct language families, the resulting tree topologies are supposed to show vertical inherited lineages only, neglecting data that could indicate language contact. This means that even in large-scale Bayesian applications of phylogenetic reconstruction based on large cognate databases, historical cases of language contact (or, `horizontal transfer') can only be detected using additional measures (as done by, for example, \citealt{mm:neureiter2022detecting}). Since there is no such filtering out of contact features in the first stage of syntactic reconstruction using the PCM\is{Parametric Comparison Method}, this means the resulting topologies of the reconstructed trees would include similarities due to transfer in their vertical branches if there were any. Therefore, if we compare the traditional phonological and syntactic tree topologies, we would then expect them to differ exactly in those cases where language contact played a role. Since syntactic transfer is often argued to be more difficult than lexical borrowing, we may not see as many differences, but still, family trees and in particular subfamilies would be predicted to look slightly different. Although other factors could play a role here too, this prediction is certainly borne out in both the nominal and the verbal sets of parameters, as shown by \citet{mm:bakerroberts2024} in the case of Germanic\il{Germanic language family} in particular. Such instances of divergence between phonological and syntactic reconstructions force us to zoom into the exact cases of identity and difference in syntactic features, thus providing us with a detailed research program. In the next section I discuss more examples of how Minimalist syntactic reconstruction, and the PCM\is{Parametric Comparison Method} through the use of parameter hierarchies in particular, can help us gain more insight into (Celtic) language history.

\section{How Minimalist syntactic reconstruction can help Celtic}
\label{sec:MinhelpCeltic}

\noindent Although we do have older attestations of some modern Celtic languages like Irish\il{Irish (Modern)}, Breton\il{Breton (Modern)} and Welsh\il{Welsh (Modern)}, evidence is still scarce compared to other branches of the \ili{Indo-European} language family. The first Old Irish\il{Old Irish} and Old Welsh\il{Old Welsh} texts are of a much later date than the first attestations we have in \ili{Greek}, \ili{Latin} or \ili{Indo-Iranian}, which is one of the reasons why Celtic has always lagged behind in providing evidence for \ili{Proto-Indo-European} reconstruction. In addition, various (stages of) Celtic languages are not even attested at all, such as the language spoken by the first Celtic speakers moving to the British Isles or any of the modern languages that could have developed from later stages of \ili{Gaulish}, \ili{Celtiberian} or \ili{Lepontic}. Finally, extant evidence for any of the Continental Celtic languages is scarce, and mainly consists of short inscriptions, which severely restricts the possibilities of getting a comprehensive picture of the overall grammatical system of any of these languages. Issues like these are further complicated by ambiguous and sometimes contradictory evidence from historical phonology. Some scholars (e.g. \citealt{mm:kortlandt_italo-celtic_2007}; \citealt{mm:schrijver_studies_1997} and more recently \citealt{mm:weiss2022}) have postulated an \ili{Italo-Celtic} branch of \ili{Indo-European}, based on certain similarities in phonology and morphology such as sigmatic future with reduplication, substitution of \textit{*-bhos} for PIE\il{Proto-Indo-European} dat.\textsc{pl} \textit{*-mus} and the lack of innovation of an elaborate middle voice. More recent Bayesian approaches based on lexical cognates often place Celtic and Romance\il{Romance language family} languages close together as well, e.g., \citet[215]{mm:atkinson_qd_curious_2005} and \citet[7]{mm:neureiter2022detecting}. However, even \citet{mm:kortlandt_italo-celtic_2007} who presents a concrete reconstruction of \ili{Proto-Italo-Celtic}, acknowledges that ``specific \ili{Italo-Celtic} innovations are few'' (\cite[157]{mm:kortlandt_italo-celtic_2007}), which has led others (e.g., \citealt{mm:watkins_origin_1966}) to reject this \ili{Italo-Celtic} hypothesis at all originally (see \citealt{mm:weissforth}, however, for shared morphosyntactic features of preverbs and adpositions). Similarly, intermediate subgroups within Celtic itself are not always easy to establish. Insular Celtic languages like Welsh\il{Welsh (Modern)} and Irish\il{Irish (Modern)} exhibit a number of similarities \citep[463--465]{mm:schrijver_studies_1995}, but in some respects the British Celtic languages appear much closer to \ili{Gaulish} (e.g., as a P-Celtic language) leading scholars like \citet{mm:sims-williams_celtic_2003} to hypothesize that “British” could be inherited from both Gallo\hyp Brittonic as well as Insular Celtic.

In both of these cases comprehensive methods for syntactic reconstruction like the PCM\is{Parametric Comparison Method} offer opportunities to fill the gaps in our linguistic history. This is not just apparent for phylogenetic topology on a larger scale, but also for specific gaps in our knowledge of syntactic features of Celtic languages on a smaller scale. In this final section, I first focus on syntactic changes in the history of some insular Celtic languages specifically, to create a better picture of the grammars of intermediate stages like \ili{Proto-British}, but also to identify gaps in our knowledge and opportunities for future research. Finally, I  discuss examples of syntactic features shared by Celtic and Romance\il{Romance language family}, as well as where they differ based on a comparison of modern languages like present-day Irish\il{Irish (Modern)} and Welsh\il{Welsh (Modern)}, on the one hand, and French\il{French (Modern)}, \ili{Spanish} and both Northern and Southern Italian varieties, on the other, which leads to a final discussion of Celtic as a subfamily of \ili{Proto-Indo-European} in general.

\subsection{Reconstructing intermediate stages}
\label{sec:PCMforhistcelt}

\noindent In \sectref{sec:celticcontact}, we saw examples of syntactic innovations in the Voice system of British Celtic languages: both Middle Cornish\il{Middle Cornish} and Breton\il{Middle Breton} already show examples of the \textit{be}-passive, whereas Late Middle Welsh\il{Welsh (Middle Welsh)} innovated the \textit{cael-}passive, which behaves more like the English\il{English (Modern)} \textit{get}-passive. Old\il{Old Irish} and Middle Irish\il{Middle Irish}, on the other hand, did not have periphrastic passives, which would leave the verbal parameter Pv38 (VSV) in the negative setting as there is no evidence for a Voice feature being realized on both the verb and separately with an auxiliary. Present-day Irish\il{Irish (Modern)} may have innovated a periphrastic passive too, but this is so far limited to the perfective passive only (\cite[254--255]{mm:mccloskey_scope_1996}). Since the periphrastic passives are a later innovation in Cornish\il{Cornish (Modern)}, Breton\il{Breton (Modern)} and especially Welsh\il{Welsh (Modern)} (all of which still have morphological passives and impersonals), we should reconstruct the negative setting of this parameter for \ili{Proto-British} as well as \ili{Proto-Insular-Celtic} (if this is an intermediate stage we want to reconstruct at all). The potential positive setting for present-day Irish\il{Irish (Modern)} depends on the exact interpretation of the verb `be' as a possible auxiliary, which is debatable and it may mean this parameter is still set to negative even today. Note that this discussion reveals two important points about the PCM\is{Parametric Comparison Method}. First, the restriction to perfective passives in present-day Irish\il{Irish (Modern)} could lead us to refine the passive parameter hierarchy to allow for this type of microvariation. Second, it was the parametric\is{parameters} questions and the comparison of the cross-linguistic settings, which highlighted this gap in our knowledge about Irish\il{Irish (Modern)} to begin with.\footnote{Many thanks to Jim McCloskey and Ian Roberts for sharing their discussion notes, which brought this to my attention.}

A further example of systematic comparison driven by the PCM\is{Parametric Comparison Method} can be found in the domain of negation. \citet{mm:bakerroberts2024} posit Pv83 (NMU) as the parameter\is{parameters} for “Multiple Negation”, which essentially asks whether a language permits negative concord. Modern Irish\il{Irish (Modern)} and Welsh\il{Welsh (Modern)} differ in this respect, as shown in (\ref{ex:NMU-irish}) and (\ref{ex:NMU-welsh}), with only the latter overtly realizing this feature thus rendering a positive setting.

\ea
\settowidth\jamwidth{[Modern Welsh]}
\gll Níor chreideas é \\
\textsc{neg} believe.\textsc{past}.\textsc{1s} \textsc{3sm}.acc\\ \jambox{[Modern Irish]}
\glt `I did not believe him.'\jambox{(\citealt{mm:nolan2018encoding})}
\label{ex:NMU-irish}
\z

\ea 
\ea
\settowidth\jamwidth{[Modern Welsh]}
\gll D-oes neb yn ennill. \\
\textsc{neg}-be.\textsc{3s} no.one prog win.\textsc{vn}\\ \jambox{[Modern Welsh]}
\ex 
\gll *Mae neb yn ennill.\\
be.\textsc{3s} no.one prog win.\textsc{vn}\\
\glt `No one is winning.' \jambox{(\citealt{mm:willis_history_2013})}
\z
\label{ex:NMU-welsh}
\z

\noindent Middle Welsh\il{Middle Welsh} did not have negative concord, on the other hand, predominantly showing Stage I of Jespersen's Cycle with preverbal negation only, shown in (\ref{ex:negcon}). Middle Breton\il{Middle Breton} already appears to be in Stage II showing both preverbal negation as well as postverbal \textit{ket}, shown in (\ref{ex:negbret}), which has been spreading within the language as a Breton-specific innovation probably due to French\il{French (Modern)} (\cite[281--286]{mm:hemon_historical_1975}; \cite[103--105]{mm:poppe_negation_1995}; \citealt{mm:willis_negation_2006}, forthcoming).

\ea 
\ea
\settowidth\jamwidth{(Middle Breton)}
\gll \textbf{ny} wnn i pwy wyt-ti \\
\textsc{neg} know.pres.\textsc{1s} I who be.pres.\textsc{2s}-you\\\jambox{[Middle Welsh]}
\glt `I don't know who you are.'\jambox{(PKM 2.22-3)}
\label{ex:negcon}
\ex 
\gll Ne chome \textbf{ket} pell e wenneien en e c'hodell \\
\textsc{neg} stay.\textsc{3s} \textbf{ket} long \textsc{3sm} pennies in \textsc{3sm} pocket\\ \jambox{[Middle Breton]}
\glt `His pennies did not stay long in his pockets.' \jambox{(MAV p.29)}
\label{ex:negbret}
\z
\z

Cornish\il{Cornish (Modern)}, just like Irish\il{Irish (Modern)} in this case, only allows a preverbal marker \textit{ny} or \textit{neng/nyngs} (before forms of \textit{bos} `be' and \textit{mones} `go'), with or without preverbal subject, as shown in (\ref{ex:corneg}).

\ea
\ea 
\settowidth\jamwidth{(PA 121a / OM 185)}
\gll \textbf{Ny} lowenhaf in ow dythyow neffra lam\\
\textsc{neg} rejoice.\textsc{1s} in \textsc{1s} days never now\\\jambox{[Cornish]}
\glt `I shall not rejoice ever now in my days.'\jambox{(BK 2929–2930)}
\ex 
\gll \textbf{Nyng} ew ow thowl\\
\textsc{neg} be.\textsc{3s} \textsc{1s} intention\\
\glt `It is not my intention.'\jambox{(BK 293)}
\ex
\gll My \textbf{ny} won. / My a wor\\
I \textsc{neg} know.\textsc{1s} / I \textsc{ptc} know.\textsc{3s}\\
\glt `I don't know.'\jambox{(PA 121a / OM 185)}
\z
\label{ex:corneg}
\z

\noindent These examples reveal another curious fact about Cornish\il{Cornish (Modern)}, namely that verbs can exhibit agreement with pronominal subjects that precede the negation (\cite[48--49(§47)]{mm:lewis1946llawlyfr}), which is impossible in their positive counterparts. Pronouns cannot be overtly expressed in Cornish\il{Cornish (Modern)}, or most other Celtic languages, if there is subject-verb agreement (i.e., the so-called “complementarity principle”, \cite[292]{mm:stump_agreement_1984},  \citealt{mm:borsley_agreement_1989}; \citealt{mm:doron1988complementarity}). However, this principle clearly does not hold for subject\hyp initial negative clauses in Cornish\il{Cornish (Modern)}. It also does not hold for Welsh\il{Welsh (Modern)}, although in post-verbal position, only the weaker forms of the pronouns (echo pronouns) can be used, as shown in (\ref{ex:compprinc}). In Middle Welsh\il{Middle Welsh}, strong pronouns can be found with agreement, but crucially only when they occur in preverbal position, as shown in (\ref{ex:commid}). When it comes to nominal subject DPs, they can never agree with verbs in number (shown in \ref{ex:num}), which \citet[388--389]{mm:roberts_parameter_2019}  relates to the 3rd-person forms specifically.

\ea
\ea
\settowidth\jamwidth{[Modern Welsh]}
\gll dywedais i/*fi\\
say.\textsc{past}.\textsc{1s} I.\textsc{weak/strong}\\ \jambox{[Modern Welsh]}
\glt `I said'
\label{ex:compprinc}
\ex 
\gll mi a wnn pwy wyt ti ac ny chyuarchaf i\\
I.\textsc{strong} \textsc{ptc} know.\textsc{1s} who are.\textsc{2s} you and \textsc{neg} greet.\textsc{1s} I.\textsc{weak}\\ \jambox{[Middle Welsh]}
\glt `I know who you are, but I will not greet you.' (Pwyll p1rc2l18)
\label{ex:commid}
\ex
\gll *Canon / Canodd y plant.\\
sing.\textsc{past}.\textsc{3pl} / sing.\textsc{past}.\textsc{3s} the children.\textsc{pl}\\ \jambox{[Modern Welsh]}
\glt `The children sang.'
\label{ex:num}
\z
\z

\largerpage
\noindent These subtle distinctions among the Celtic languages in terms of agreement and the link to negation in Cornish\il{Cornish (Modern)} in particular reveal gaps in the grammars of the intermediate proto-languages we would like to reconstruct, showing us exactly where further research is necessary. When it comes to the use of multiple negation, however, it is clear that negative concord was an innovation in Welsh\il{Welsh (Modern)} and Breton\il{Breton (Modern)} only. This cannot be reconstructed to \ili{Proto-British} as they each exhibit different postverbal negative markers (Welsh\il{Welsh (Modern)} \textit{ddim} vs.\ Breton\il{Breton (Modern)} \textit{ket}), which are not lexical cognates. They do share the same syntactic features, however, but \citet{mm:willis_negation_2006} has shown that the changes in these features can be traced exactly through the history of Welsh\il{Welsh (Modern)}. For \ili{Proto-British}, as well as \ili{Proto-Insular-Celtic}, we therefore reconstruct the negative value for the Multiple Negation parameter, which is already found in Cornish\il{Cornish (Modern)}, Irish\il{Irish (Modern)}, as well as Old\il{Old Welsh} and Middle Welsh\il{Middle Welsh}. Note that this is also in line with the negative setting of this parameter in other \ili{Indo-European} languages like \ili{Latin}, \ili{Vedic} and \ili{Hittite}, which brings us to comparing Celtic parameters beyond the British Isles. Since evidence for Continental Celtic languages is scarce, the task of reconstructing its syntax seems daunting, especially from the perspective of reconstruction methods that rely on overt (lexical) cognacy of functional items as there are very few (if any) such items that can be found in both Continental and Insular Celtic languages. The PCM\is{Parametric Comparison Method} offers an opportunity here too since it not only provides a systematic approach seeking concrete answers to a fixed set of parametric questions in the nominal and clausal domains, it also allows for reconstruction based on more covert realizations, such as movement or Internal Merge due to requirements of specific features on functional heads. In addition, the strong hierarchical structure with built-in implicational parameters means that a number of parameter values can be inferred from the positive or negative setting of others. Combining these strengths of the PCM\is{Parametric Comparison Method} will provide experts on Continental Celtic with the opportunity to reconstruct a grammar of the proto-language that will cover more aspects of syntax than was ever possible before.

\subsection{Syntactic similarities in Italo-Celtic}
\label{sec:ItaloCeltic}
\largerpage
\noindent In this final section, I show examples of syntactic features that can be reconstructed for Celtic, Romance\il{Romance language family} and/or both to shed light on the “\ili{Italo-Celtic}” hypothesis from a new angle of Minimalist syntax. Obvious similarities are found for fundamental (macro)parameters, such as using bound morphemes (Pn1 FMG “grammaticalized morphology”), agreement (Pn2 FGA ``grammaticalized agreement'') and Case (Pn3 FGK ``grammaticalized Case''), which tend to distinguish \ili{Indo-European} languages from languages like \ili{Mandarin}, \ili{Japanese} and \ili{Wolof}. Much more interesting for our purposes are the similarities and differences found within the \ili{Indo-European} family and languages these came into contact with in the course of history, such as \ili{Basque}, \ili{Hungarian} and a range of \ili{Semitic} languages.

When focusing on Celtic, Romance\il{Romance language family} and Germanic\il{Germanic language family} in particular, there are various parameters with positive settings shared with some (or all) languages in those three subfamilies, such as grammaticalized person, number and gender in both the nominal and verbal domains. In addition, all three subfamilies share features to do with the Categorization Requirement (CR, cf. \citealt{mm:panagiotidis_categorial_2015}), which states that every c-selecting and c-selected functional head H must be uniquely identified in relation to an EP (Extended Projection). It is built on the assumption that non-categorizing heads do not have intrinsic V- or N-features, but instead bear FF [Cat:\_\_]. The higher heads in the EP then are categorially identified as belonging to that EP. \citet[164--166]{mm:roberts_parameter_2019} argues that UG makes available three mechanisms to satisfy the Categorization Requirement, yielding three types with Type I satisfying CR through roll-up, labeling all the functional categories as V or N giving rise to harmonic head-finality. Type II, on the other hand, satisfies CR through head-movement yielding head-initial languages, and, Type III satisfies CR by lexicalizing heads in the EP. The “purest” versions of these three types are rare and there is a possibility to combine types for different parts of the syntax. Celtic, Romance\il{Romance language family} as well as English\il{English (Modern)} combine Types II and III, but other West Germanic languages can also feature Type I, e.g., \ili{German} featuring all three types: roll up of VP around v, head movement as well as lexicalizing heads. \citet[163]{mm:roberts_parameter_2019} argues that both Welsh\il{Welsh (Modern)} and English\il{English (Modern)} allow intrinsic labeling followed by head-movement, as shown, for example, by the inflected auxiliary \textit{oedd} in example (\ref{ex:118c}) raising to a high functional position (e.g., some C head) when there are free lower aspectual heads, which could be considered some sort of “\textit{be}-insertion” (cf.\ English\il{English (Modern)} dummy “\textit{do}-support”):

\ea
\settowidth\jamwidth{[Welsh]}
\gll Oedd y bachgen wedi bod yn ymlad \\
be.\textsc{3s}\textsc{.past} the boy \textsc{perf} be.\textsc{inf} \textsc{prog} fight.\textsc{inf}\\ \jambox{[Welsh]}
\glt `The boy had been fighting.'
\label{ex:118c}
\z

Syntactic features that are even more interesting for investigating Celtic as a subfamily are those that distinguish the order of adjectives and nouns, for example, NM1 ``N under M1 Adjectives''. Given the cross-linguistically canonical sequence of structured adjectives, \citet[41]{mm:ceolin_at_2021} say that ``Manner1 adjectives'' (i.e., those describing quality or size as opposed to shape or color, which are `Manner2') can surface to the left of the noun in languages like \ili{Italian}, French\il{French (Modern)}, \ili{Spanish}, Germanic\il{Germanic language family}, \ili{Slavic language family} and Standard \ili{Greek}. In Welsh\il{Welsh (Modern)}, Irish\il{Irish (Modern)}, \ili{Farsi} or some Romance\il{Romance language family} dialects of Italy, however, they cannot, as shown in (\ref{ex:NM1}):

\ea
\ea
\settowidth\jamwidth{[Modern Welsh]}
\gll un bel nuovo vestito blu \\
a nice new dress blue\\ \jambox{[Italian]}
\ex
\gll ffrog las newydd neis \\
dress blue new nice\\\jambox{[Modern Welsh]}
\glt `a nice new blue dress'
\z
\label{ex:NM1}
\z

\largerpage
\noindent Although these word order facts are relatively clear, agreement with adjectives in Celtic is not as straightforward as it seems, which becomes apparent when asking these detailed parametric questions, especially for older stages of the Celtic languages. \citet{mm:borsleyetal2007syntax} claim that ``[I]n predicative position, adjectives never agree in gender or number with their noun'' (\citeyear[179]{mm:borsleyetal2007syntax}), citing example (\ref{ex:las}):

\ea 
\settowidth\jamwidth{[Modern Welsh]}
\gll Mae ei lygaid yn las / *leision. \\
be.\textsc{3s} \textsc{3sm} eyes \textsc{pred} blue.\textsc{general} / blue.\textsc{pl}\\ \jambox{[Modern Welsh]}
\glt `His eyes are blue.' 
\label{ex:las}
\z 

\noindent In spoken Welsh\il{Welsh (Modern)} corpora, however, it is possible to find examples, as shown in (\ref{ex:small}):

\ea 
\settowidth\jamwidth{[Modern Welsh]}
\gll roeddynt yn fychain \\
were.\textsc{3pl} \textsc{pred} small.\textsc{pl}\\ \jambox{[Modern Welsh]}
\glt `they were small' \jambox{(CEG corpus; \citealt{mm:meelen_adjectival_2020})}
\label{ex:small}
\z

\noindent \citet{mm:meelen_adjectival_2020} show that agreement in earlier stages of Welsh\il{Welsh (Modern)} was different, as shown in (\ref{ex:19th}), which is backed up by a corpus study of Middle\il{Middle Welsh} and Early Modern Welsh\il{Welsh (Modern)} with 17 out of 50 predicative adjectives (34\%) showing agreement. Since 13 out of the 17 examples stem from the Bible translation, some further research is necessary here before we can definitively show a parametric change, however.

\ea
\gll Mae y cyflenwadau o wenith Lloegr a Thramor yn fychain\\
be.\textsc{3s} the supplies of wheat English and overseas \textsc{pred} small.\textsc{pl}\\
\glt `the supplies of English and overseas wheat are small' (Papurau Newydd: Yr Amserau 12 Nov 1851)
\label{ex:19th}
\z

\noindent Other examples of parameters that put Welsh\il{Welsh (Modern)} and Irish\il{Irish (Modern)} close to the majority of Romance\il{Romance language family}, but also close to Germanic\il{Germanic language family} languages are Grammaticalized Aspect (GRA) and Mood (GRM) and Mood- and Voice-checking (MCV and VCV) in the verbal domain as well as Grammaticalized Specified Quantity (DGR) in the nominal domain (encoding definiteness).

There are, however, also various parameters that show differences between Celtic and Romance\il{Romance language family} languages, especially when we zoom in to the micro- and nano-levels. \citet{mm:roberts_parameter_2019} argues that verb-initial languages like Modern Welsh\il{Welsh (Modern)} and Irish\il{Irish (Modern)} copy $\varphi$-features (i.e., Person/Gender/Number features) of the verbal suffix in one of the other agreeing heads (enclisis), just like Northern \ili{Italian} Dialects (NIDs), where this is shown by subject (pro)clitics. Unlike NIDs, the Welsh weak/clitic pronouns actually combine both Person and Number features, creating a microparametric distinction. Other Romance varieties differ even more from Celtic in this respect, since they do not exhibit agreeing subject clitics (either enclitic or proclitic) altogether. Similarly, Welsh\il{Welsh (Modern)} and Irish\il{Irish (Modern)} differ from languages like French\il{French (Modern)} and \ili{Brazilian Portuguese} by optionally allowing the suppression of $\varphi$-features on T, making them into Partial Null-Subject languages (\cite[608]{mm:roberts_parameter_2019}). Other Romance\il{Romance language family} languages go one step further, not having obligatory EPP features on T at all, yielding them as Canonical Null-Subject languages with fully specified Person-features found on both T and D heads.

In the nominal domain, Celtic languages pattern with \ili{Semitic} and \ili{Icelandic} freely admitting bare singular count indefinite arguments (nominal Pn19 ``Weak Specified Quantity'', CGR), e.g., Hebrew \textit{kelev}, Welsh\il{Welsh (Modern)} \textit{ci} `\textbf{a} dog', where Romance\il{Romance language family} and many Germanic\il{Germanic language family} languages require an indefinite article, e.g.  \ili{Dutch} \textit{een hond}, French\il{French (Modern)} \textit{un chien} and \ili{Italian} \textit{un cane} `\textbf{a} dog'. Similarly parameter Pn47 (GFL) groups Celtic languages together with \ili{Greek}, most Slavic\il{Slavic language family} languages as well as \ili{German} and \ili{Icelandic} in which a non-adpositional, non-iterable Genitive Case appears to the right of canonically ordered adjectives, as shown in (\ref{ex:gfl}) from  \citet[31]{mm:ceolin_at_2021}:

\ea
\settowidth\jamwidth{[Modern Welsh]}
\gll portread hardd y plentyn\\
portrait beautiful the child\\ \jambox{[Modern Welsh]}
\glt `the child's beautiful portrait' 
\label{ex:gfl}
\z 

\noindent Finally, there are also parameters that set Welsh\il{Welsh (Modern)} and Irish\il{Irish (Modern)} apart from not just Romance\il{Romance language family}, but most of \ili{Indo-European}, such as Pn55 ``plural spread from cardinal quantifiers'' (PSC). In Welsh\il{Welsh (Modern)}, for example, singular nouns are used after cardinal quantifiers, shown in (\ref{ex:ffordd}), rendering a negative setting for this parameter, similar to \ili{Farsi}, Uralic\il{Uralic language family} and \ili{Turkic} languages (but unlike English\il{English (Modern)} or the rest of \ili{Indo-European}).

\ea
\settowidth\jamwidth{[Modern Welsh]}
\gll un ffordd / tri ffordd\\
one way / three way \\\jambox{[Modern Welsh]}
\glt `one way / three ways' 
\label{ex:ffordd}
\z 

\noindent If we combine the results from 87 verbal (\citealt{mm:bakerroberts2024}) and 94 nominal \isi{parameters} (\citealt{mm:ceolin_at_2021}), the first observation is that languages that are usually grouped together into subfamilies using the comparative method of phonological reconstruction also form subfamilies based on syntactic features. Within \ili{Indo-European}, we have a Slavic\il{Slavic language family} branch with \ili{Russian}, \ili{Polish}, \ili{Serbo-Croatian} and \ili{Slovenian}; a Germanic\il{Germanic language family} branch with \ili{Norwegian}, \ili{Danish}, \ili{German}, \ili{Dutch}, English\il{English (Modern)}, \ili{Afrikaans} and \ili{Icelandic}; a Romance\il{Romance language family} branch with \ili{Spanish}, \ili{Portuguese}, \ili{Italian} and French\il{French (Modern)}; and Modern Irish\il{Irish (Modern)} and Modern Welsh\il{Welsh (Modern)} are grouped together representing Celtic. The Hamming distance between those two modern Celtic languages, which counts both positive and negative settings that are similar as “identities”, is 0.039; the Jaccard distance, which ignores identical settings if they are both negative, is 0.167.\footnote{Note that \citet{mm:bakerroberts2024} only provide Hamming distances, whereas \citet{mm:ceolin_at_2021} change their previous Hamming metric to Jaccard to be able to distinguish negative similarities from positive ones. \citet{mm:ceolin_at_2021} report no difference at all, i.e., $\delta=0.0$ between Irish\il{Irish (Modern)} and Welsh\il{Welsh (Modern)}. Although the Insular Celtic DP indeed shows striking similarities, it is worth noting that 100\% similarity in 94 distinct parameters is very unusual even for languages that are very closely related. In fact it is the only language pair in their dataset that shows 0 differences for relevant comparanda.} In general, the phylogenetic tree topologies based on Hamming and Jaccard distances of all parameters combined are quite similar, although the Jaccard distances are about 0.13--0.2 larger. Note that with either metric, Irish\il{Irish (Modern)} and Welsh\il{Welsh (Modern)} clearly are below the $\delta<0.20$ threshold \citet{mm:bortolussi_how_2011} suggested for probable relatedness.

If we compare both Irish\il{Irish (Modern)} and Welsh\il{Welsh (Modern)} combined to other subfamilies, we see the average distance distributions in Table \ref{tab:nums}. Celtic, as represented by Irish\il{Irish (Modern)} and Welsh\il{Welsh (Modern)}, on average shares more syntactic features with the Slavic\il{Slavic language family} languages than with Romance\il{Romance language family}: both average Hamming and Jaccard distances are lower for Slavic\il{Slavic language family} than for Romance\il{Romance language family}. If the \isi{parameters} would solely be used for the purposes of detecting historical relatedness, this is a somewhat surprising result since family trees based on the traditional phonological comparative method would often posit a combined \ili{Italo-Celtic} subgroup (\citealt{mm:weiss2022}). \citet{mm:weissforth}, for example,  shows that shared innovations and retentions in the system of preverbs and adpositions provide additional evidence for this closer relation between the Celtic and Romance\il{Romance language family} languages (see also \sectref{sec:ItaloCeltic}). 

\begin{table}
\begin{tabular}{lcc}
\lsptoprule
 & Hamming & Jaccard \\ \midrule
Welsh vs Romance         & 0.204                           & 0.364                 \\
Welsh vs Germanic        & 0.218                  & 0.395                  \\\midrule
Irish vs Romance         & 0.196                  & 0.358                  \\
Irish vs Germanic        & 0.227                           & 0.429                           \\\midrule
Celtic vs Romance        & 0.200                           & 0.361                           \\
Celtic vs Germanic       & 0.227                           & 0.412                           \\
Celtic vs Slavic         & 0.185                           & 0.323                           \\
Celtic vs Uralic         & 0.262                           & 0.468                           \\
Celtic vs Semitic        & 0.233                           & 0.400                           \\
 \lspbottomrule
\end{tabular}
\caption{Hamming and Jaccard distances for averaged subfamilies}
\label{tab:nums}
\end{table}

This highlights an important point of interpretation and application of using the PCM\is{Parametric Comparison Method} for a variety of purposes. First, the results of detecting phylogenetic signals and establishing historical relations between subgroups should not necessarily be expected to be the same not only because lexical and phonological features on the one hand and grammatical features on the other are representing different parts of the language, but also because potential evidence for horizontal transfer (i.e., transfer due to contact, see \sectref{sec:celticcontact}) is systematically excluded for lexical/phonological reconstruction, but not for the grammatical features in the PCM\is{Parametric Comparison Method}. Second, the number and range of languages and features representing a subgroup is important in any statistical reconstruction method (whether Bayesian or not). The Celtic subgroup is only represented by two languages in the PCM\is{Parametric Comparison Method} and as they are less-studied some features in the nominal and verbal domain still have some question marks. The Romance\il{Romance language family} languages, on the other hand, are all studied in a high amount of detail and are represented even in its smallest iteration by at least five languages (French\il{French (Modern)}, \ili{Italian}, \ili{Spanish}, \ili{Portuguese} and \ili{Romanian}). Slavic\il{Slavic language family} languages currently fall somewhere in between, with four languages represented (\ili{Slovenian}, \ili{Serbo-Croatian}, \ili{Russian} and \ili{Polish}), some of which are well-studied from a syntactic point of view. Finally, it is worth bearing in mind that only the nominal and verbal parameters are part of the dataset yielding the above results. When clausal parameters that cover syntactic features of the C-domain are included, the overall results may look very different. Since \citet{mm:longobardi_evidence_2009} note that the nominal parameters are more likely to be robust over time, it is worthwhile zooming in on potential changes within the Celtic subfamily in the verbal domain. Modern Irish\il{Irish (Modern)} and Welsh\il{Welsh (Modern)} show a Jaccard $\delta=0.263$, which is much higher than the average of both nominal and verbal \isi{parameters} since \citet{mm:ceolin_at_2021} report 0 difference in the nominal domain. If we compare Middle Welsh\il{Middle Welsh} to Old Irish\il{Old Irish}, on the other hand, there are fewer syntactic distinctions with Jaccard $\delta=0.171$. Between Middle\il{Middle Welsh} and Modern Welsh\il{Welsh (Modern)} verbal parameters Jaccard $\delta=0.268$, whereas Irish\il{Irish (Modern)} appears to have changed fewer verbal parameters showing  Jaccard $\delta=0.229$.

Overall, these numbers are preliminary. As the number of \isi{parameters} increases, we can expect to see a refinement of the analysis. Despite this, it is clear that  subfamilies can be easily identified already and we can identify gaps in our knowledge of the old and modern Celtic languages. Even if the PCM\is{Parametric Comparison Method} eventually were to yield different results related to phylogenetic signal and subgrouping, we can see that it is a fruitful exercise for reconstructing the grammatical system of proto\hyp languages.

When comparing parameter\is{parameters} settings in both the modern and the medieval Celtic languages, we can get a much better picture of the syntactic developments going back to \ili{Proto-British} and \ili{Proto-Celtic}. For every reconstructed parameter whose values are different from those in the modern daughter languages, we should carefully address some of the remaining issues of syntactic reconstruction discussed in \sectref{sec:methods}, e.g., the radical-reanalysis and transfer problems. The latter was already discussed in \sectref{sec:celticcontact} and requires knowledge about the historical contact situation, as well as the application of methods to detect syntactic transfer in particular, such as those described by \citet{mm:bowern_syntactic_2008} and \citet{mm:daniels_method_2017}. The radical-reanalysis problem can only be addressed applying the careful step-by-step methodology presented by \citet{mm:willis_negation_2006}, which involves a critical analysis of how every stage of reanalyses and extensions could have led to the syntactic patterns we find in attested stages of all the daughter languages. This is more difficult when we are forced to reconstruct negative values of parameters, since they do not have overt realizations, forcing us to rely more on implicational generalizations known from (Minimalist) theoretical syntax such as those worked out in detailed parameter hierarchies. A large number of \isi{parameters}, however, (at least around a third for the verbal domain\footnote{Jim Baker, p.c.}) actually ask about direct PF realizations such as inflectional morphology, which can potentially be reconstructed using the traditional comparative method if they are true cognates, as done by \citeauthor{mm:walkden_abduction_2011} (\citeyear{mm:walkden_abduction_2011}, \citeyear{mm:Walkden2014}) and \citet{mm:willis_reconstructing_2011}. 

\section{Conclusion}
\largerpage[2]
%:DONE TO HERE
%All Celtic languages
%Proto-Celtic,
%Proto-British, 
%Italo-Celtic [AC: Use Proto-Italo-Celtic]
%DCxG (Diachronic construction Grammar)
%BCC, (Borer-Chomsky conjecture)
%PCM, (Parametric Comparison Method)
%parameters
%comparative method,
%reanalysis

In this chapter, I have shown that recent advances in Minimalist syntactic theory enable us to tackle the long-standing challenges of syntactic reconstruction. While various methods of syntactic reconstruction have been put forward in the last decades, Minimalist approaches that rely on the \isi{Borer-Chomsky Conjecture} (BCC) have been particularly successful in addressing problems of directionality, transfer and radical reanalysis. The \isi{Parametric Comparison Method} (PCM) has some additional benefits of providing a comprehensive overview of the syntax of a proto-language because of the discrete nature of parameters, at the same time avoiding the need to find potentially improbably phenomena (e.g., sound correspondences in restricted contexts) and reliance on vague cognacy judgments (e.g., due to semantic changes). The PCM\is{Parametric Comparison Method} is most powerful when it is combined with the careful step-by-step evaluation of the reanalyses and extensions in all reconstructed stages shown by \citet{mm:willis_reconstructing_2011}  and methods for detecting language contact in the domain of syntax (\citealt{mm:bowern_syntactic_2008, mm:daniels_method_2017}).

As I showed in \sectref{sec:celticrefiningPCM}, there are various ways in which the parametric\is{parameters} questions can be improved and extended, but the language family trees based on nominal or verbal domains (or both) show remarkable resemblances to the tree topologies based on lexical cognates using the traditional comparative method. Celtic languages can play an important role in the further development of the PCM\is{Parametric Comparison Method} due to their unique properties in syntactic features, even though only Modern Welsh\il{Welsh (Modern)} and Irish\il{Irish (Modern)} have been taken into account so far. This knowledge can help us refine and extend the PCM\is{Parametric Comparison Method}, on the one hand, but on the other, the PCM\is{Parametric Comparison Method} can help our understanding of historical syntax by providing a systematic set of questions that can shape a clearly-defined research program for the Celtic languages.

\printbibliography[heading=subbibliography,notkeyword=this]

\is{syntactic reconstruction|)}

\end{document}
