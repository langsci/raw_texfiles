\documentclass[output=paper,colorlinks,citecolor=brown,newtxmath]{langsci/langscibook}
\ChapterDOI{10.5281/zenodo.3764853}
%\bibliography{localbibliography}
%\usepackage{langsci-optional}
\usepackage{langsci-gb4e}
\usepackage{langsci-lgr}

\usepackage{listings}
\lstset{basicstyle=\ttfamily,tabsize=2,breaklines=true}

%added by author
% \usepackage{tipa}
\usepackage{multirow}
\graphicspath{{figures/}}
\usepackage{langsci-branding}

%
\newcommand{\sent}{\enumsentence}
\newcommand{\sents}{\eenumsentence}
\let\citeasnoun\citet

\renewcommand{\lsCoverTitleFont}[1]{\sffamily\addfontfeatures{Scale=MatchUppercase}\fontsize{44pt}{16mm}\selectfont #1}
  
\IfFileExists{../localcommands.tex}{
  \usepackage{langsci-optional}
\usepackage{langsci-gb4e}
\usepackage{langsci-lgr}

\usepackage{listings}
\lstset{basicstyle=\ttfamily,tabsize=2,breaklines=true}

%added by author
% \usepackage{tipa}
\usepackage{multirow}
\graphicspath{{figures/}}
\usepackage{langsci-branding}

  
\newcommand{\sent}{\enumsentence}
\newcommand{\sents}{\eenumsentence}
\let\citeasnoun\citet

\renewcommand{\lsCoverTitleFont}[1]{\sffamily\addfontfeatures{Scale=MatchUppercase}\fontsize{44pt}{16mm}\selectfont #1}
  
  %% hyphenation points for line breaks
%% Normally, automatic hyphenation in LaTeX is very good
%% If a word is mis-hyphenated, add it to this file
%%
%% add information to TeX file before \begin{document} with:
%% %% hyphenation points for line breaks
%% Normally, automatic hyphenation in LaTeX is very good
%% If a word is mis-hyphenated, add it to this file
%%
%% add information to TeX file before \begin{document} with:
%% %% hyphenation points for line breaks
%% Normally, automatic hyphenation in LaTeX is very good
%% If a word is mis-hyphenated, add it to this file
%%
%% add information to TeX file before \begin{document} with:
%% \include{localhyphenation}
\hyphenation{
affri-ca-te
affri-ca-tes
an-no-tated
com-ple-ments
com-po-si-tio-na-li-ty
non-com-po-si-tio-na-li-ty
Gon-zá-lez
out-side
Ri-chárd
se-man-tics
STREU-SLE
Tie-de-mann
}
\hyphenation{
affri-ca-te
affri-ca-tes
an-no-tated
com-ple-ments
com-po-si-tio-na-li-ty
non-com-po-si-tio-na-li-ty
Gon-zá-lez
out-side
Ri-chárd
se-man-tics
STREU-SLE
Tie-de-mann
}
\hyphenation{
affri-ca-te
affri-ca-tes
an-no-tated
com-ple-ments
com-po-si-tio-na-li-ty
non-com-po-si-tio-na-li-ty
Gon-zá-lez
out-side
Ri-chárd
se-man-tics
STREU-SLE
Tie-de-mann
}
  \togglepaper[6]%%chapternumber
}{}
%\togglepaper[6]

\title{Czech infinitival conditionals}
\author{Uwe Junghanns\affiliation{University of Göttingen}\lastand Hagen Pitsch\affiliation{University of Göttingen}}


\abstract{The paper deals with complex Czech sentences that consist of a finite matrix clause and an infinitival subordinate clause. The latter receives a conditional interpretation although there is no item to signal that function. Interestingly, an overt nominative subject can be realized within the infinitival clause. Conditional sentences of this type occur in Czech and Slovak only. The following issues are addressed in the paper: size of the infinitival clause, connection with the matrix, surface order, nominative case licensing, and interpretation. The proposal builds on the idea that finiteness is a conspiracy of tense and agreement marking \citep[see][]{Stowell1982,Stowell1995,Wurmbrand2001}. Infinitival structures come in two variants: tensed vs. untensed. Nominative subjects are realized in the former but not the latter. We assume an economical division of labour between null and overt subjects. The proposal invokes un/interpretable un/valued T-/\textPhi-features for an explanation \citep[see][]{PesetskyTorrego2001}. The issues of how the conditional interpretation comes about and what might possibly help to reach it are addressed as well.

\keywords{infinitives, conditionals, complex sentences, syntax, interpretation, nominative case licensing, non/finiteness, Czech}
}



\begin{document}
\maketitle

\il{Czech|(}
\section{Introduction}

Within the \ili{Slavic} branch, \ili{Czech} and \ili{Slovak} are the only languages to exhibit a peculiar non-finite, clause-like structure. It is used and interpreted as a conditional clause. Therefore, we will call it \textsc{infinitival conditional} (\isi{IC}); see the following example:

\ea\label{ex:start}
\gll [\hspace{-2pt} Já bý-t tebou], še-l by-ch k lékař-i.\\
     {} I.\textsc{nom} be-\textsc{inf} you.\textsc{ins.sg} go-\textsc{lpt.sg.m} \textsc{cond-1sg} to doctor-\textsc{dat.sg}\\
\glt `If I were you, I would go to the doctor.' \hfill (\ili{Czech}; \citealt[683]{Travnicek1951})
\z

\noindent Syntactically, ICs are peculiar in that they are headed by an \isi{infinitive}, but may nevertheless contain an overt subject in the \isi{nominative case} (\textsc{nom}). This pattern is suprising not only in \ili{Slavic} languages, but in \ili{Indo-European} in general.\footnote{Several \ili{Indo-European} languages feature non-finite -- but not infinitival -- structures that allow \textsc{nom}-subjects. Well-known examples are gerundial clauses in \ili{English} or \ili{Spanish}. \ili{Russian} exhibits non-canonical imperatives that combine with subjects of any person/number (see, a.\,o., \citealt{Xrakovskij2009}); it is, however, an open question whether the imperative (in this use) is a non-finite form. The same question pertains to the ``inflected \isi{infinitive}'' in European \ili{Portuguese} (see \citealt{Raposo1987}), as the relevant paradigm involves person/number markers.} Still, similar structures (though not conditional) exist, for instance in \ili{Tamil}; see \REF{ex:Tamil}:

\ea\label{ex:Tamil}
\gll Raman {[(}\hspace{-2pt} Vasu) {puuri-jæ} {porikk-æ}] maav\u{u} vaaŋg-in-aan.\\
	Raman.\textsc{nom} {} Vasu.\textsc{nom} puuri-\textsc{acc} fry-\textsc{inf} flour.{\textsc acc} buy-\textsc{pst-m.3sg}\\
\glt `Raman bought flour (for Vasu) to fry puuris.' \\ \hfill (\ili{Tamil}; \citealt[467]{McFaddenSundaresan2018})
\z

\noindent A number of authors have addressed the phenomenon of \ili{Czech} ICs descriptively, among them \citet{Svoboda1959,Svoboda1960a,Svoboda1960b,Svoboda1962,Poldauf1959,Porak1959,Kiparsky1960,Kiparsky1967,Dunn1982,Karlik2007,Meyer2010,Milotova2011,Milotova2012}. Apart from that, more or less detailed notes on ICs can be found in \citet{Krizkova1972,Machackova1980}, the \textit{Mluvnice spisovné češtiny} \citep{Travnicek1951}, \citet{BauerGrepl1972}, and \citet{GreplKarlik1998}. To the best of our knowledge, the present paper is the first attempt of a formal theoretical account of these structures.

The remainder of the paper is structured as follows: \sectref{sec:description} characterizes properties of the structure and provides relevant data. In \sectref{sec:syntax}, we discuss both the internal and external syntax of ICs, while \sectref{sec:nom} deals with \textsc{nom}-licensing. In \sectref{sec:interpretation}, we tackle the question of how the \isi{conditional interpretation} of ICs arises. \sectref{sec:summary} summarizes the paper.

\largerpage
\section{Description}\label{sec:description}

We provide only a brief characterization of \ili{Czech} \isi{IC} structures (see \citealt{JunghannsPitsch2019} for a more detailed description).

\begin{enumerate}

\item ICs minimally consist of an \isi{infinitive} (\textsc{inf}) which may be of all possible valency classes and both aspects.\footnote{\citet{Svoboda1959} mentions the example in \REF{ex:rain} with an \isi{IC} consisting of an \isi{infinitive} only:
\ea\label{ex:rain}
\gll [\hspace{-2pt} Prše-t], zůsta-l-i bych-om doma.\\
     {} rain-\textsc{inf} stay-\textsc{lpt-pl.m} \textsc{cond-1pl} {at.home}\\
\glt `If it rained, we would stay at home.' \hfill \citep[\ili{Czech};][167]{Svoboda1959}
\z}
Typically, but not obligatorily, the \textsc{inf} occupies the initial \isi{syntactic position} in the \isi{IC}; see \REF{ex:music}, \REF{ex:Germans}, and \REF{ex:money}:

\ea\label{ex:music}
\gll [\hspace{-2pt} Ne-mí-t sv-ou hudb-u], tak se tu z-blázní-m!\\
     {} \textsc{neg}-have-\textsc{inf} own-\textsc{acc.sg.f} music-\textsc{acc.sg.f} so \textsc{refl} here \textsc{pf}-{become.insane}-\textsc{1sg}\\
\glt `If I had not my music, I would become insane here.' \\ \hfill \citep[\ili{Czech};][4]{Milotova2012}
\z

\ea\label{ex:Germans}
\gll [\hspace{-2pt} Dozvědě-t se to Němc-i], tak nás {[\ldots]}\hspace{-2pt} po-stříle-l-i.\\
     {} learn;\textsc{pf}-\textsc{inf} \textsc{refl} this.\textsc{acc.sg.n} \ili{German}-\textsc{nom.pl} so we.\textsc{acc} {} \textsc{pf}-shoot-\textsc{lpt-pl.m}\\
\glt `If the Germans had learnt that, they would have shot us.' \\ \hfill \citep[\ili{Czech};][371]{Meyer2010}
\z

\ea\label{ex:money}
\gll [\hspace{-2pt} Já mí-t peníz-e], koupí-m to / koupi-l jsem to.\\
     {} I.\textsc{nom} have-\textsc{inf} money-\textsc{acc.pl} buy;\textsc{pf}-\textsc{1sg} this.\textsc{acc.sg} {} buy;\textsc{pf}-\textsc{lpt.sg.m} be.\textsc{1sg} this.\textsc{acc.sg}\\
\glt `If I had money, I would buy this.'/`If I had had money, I would have bought this.' \hfill \citep[\ili{Czech};][683]{Travnicek1951}
\z

In addition to the \textsc{inf}, ICs may contain \isi{sentential negation} (\textit{ne-}), objects, and adverbials. On the other hand, ICs \textit{never} contain a (conditional) subjunction or wh-pronouns.\footnote{The absence of a subjunction is a major difference in comparison with infinitival conditionals in \ili{Polish}, which obligatorily contain a conditional subjunction; see \REF{ex:Polish}:
\ea\label{ex:Polish}
\gll [\hspace{-2pt} Jeśli się przyjrze-ć bliżej], to widać na dole mał-e literk-i.\\
     {} if \textsc{refl} look.at-\textsc{inf} closer then visible at bottom little-\textsc{acc.pl} letter-\textsc{acc.pl}\\
\glt `If one takes a closer look, one can see little letters at the bottom.' \\ \hfill \citep[\ili{Polish};][455]{EngelKuc1999}
\z
Apart from this difference, \ili{Polish} infinitival conditional clauses exclude \textsc{nom}-subjects.}

\item ICs can have an overt \textsc{nom}-subject which may be a full nominal expression or a \isi{pronoun}; see \REF{ex:father} and \REF{ex:extraordinary}, respectively:

\ea\label{ex:father}
\gll [\hspace{-2pt} Bý-t otec doma], by-l-o by se to ne-sta-l-o.\\
     {} be-\textsc{inf} father.\textsc{nom} {at.home} be-\textsc{lpt-sg.n} \textsc{cond.3sg} \textsc{refl} this.\textsc{nom.sg.n} \textsc{neg}-happen-\textsc{lpt-sg.n}\\
\glt `If father had been at home, this would not have happened.' \\ \hfill \citep[\ili{Czech};][65]{Svoboda1960a}
\z

\ea\label{ex:extraordinary}
\gll [\hspace{-2pt} U-děla-t to někdo dnes], ne-by-l-o by na tom nic výjimečn-ého.\\
     {} \textsc{pf}-make-\textsc{inf} this.\textsc{acc} somebody.\textsc{nom} today \textsc{neg}-be-\textsc{lpt-sg.n} \textsc{cond.3sg} on this.\textsc{loc} nothing extraordinary-\textsc{gen.sg.n}\\
\glt `If somebody did this today, there would not be anything extraordinary about it.' \hfill \citep[\ili{Czech};][1]{Milotova2012}
\z

\item The subject of ICs can be coreferential or non-coreferential with the subject of the matrix clause; see, e.\,g., \REF{ex:Germans}, \REF{ex:money} and \REF{ex:father}, \REF{ex:extraordinary}, respectively.

\item ICs either precede or follow their matrix clause. Examples \REF{ex:start} and \REF{ex:music} to \REF{ex:extraordinary} illustrate the former case, while the latter one is exemplified by \REF{ex:war}:

\ea\label{ex:war}
\gll Ži-l-a by-s úplně jinak, [\hspace{-2pt} ne-bý-t válk-y]?\\
     live-\textsc{lpt-sg.f} \textsc{cond-2sg} completely differently {} \textsc{neg}-be-\textsc{inf} war-\textsc{gen.sg}\\
\glt `Would you have led a completely different life, if there had not been the war?' \hfill \citep[\ili{Czech};][7]{Milotova2012}
\z

\item ICs refer to specific situations in the extralinguistic world \citep[77]{Svoboda1960a}; see the finite conditional \textit{-li}-clause in \REF{ex:kettle-a} with a generic interpretation as opposed to the \isi{IC} in \REF{ex:kettle-b} which refers to a specific situation of coal burning:

\ea\label{ex:kettle}
\ea\label{ex:kettle-a}
\gll [\hspace{-2pt} Shoří-li 1 kg uhl-í na ohništ-i kotl-e parn-ího stroj-e], vydá 7000 velk-ých kalori-í.\\
     {} {burn.down};\textsc{pf}.\textsc{3sg}$=$if 1 kg.\textsc{nom} coal-\textsc{gen.sg} on grate-\textsc{loc.sg} boiler-\textsc{gen.sg} steam-\textsc{gen.sg} engine-\textsc{gen.sg} emit;\textsc{pf}.\textsc{3sg} 7000 big-\textsc{gen.pl} calory-\textsc{gen.pl}\\
\glt `If 1 kg of coal burns down on the grate of a boiler of a steam engine, it emits 7000 kilocalories.'
\ex\label{ex:kettle-b}
\gll [\hspace{-2pt} Shoře-t 1 kg uhl-í na ohništ-i kotl-e parn-ího stroj-e], vydá 7000 velk-ých kalori-í.\\
     {} {burn.down};\textsc{pf}-\textsc{inf} 1 kg.\textsc{nom} coal-\textsc{gen.sg} on grate-\textsc{loc.sg} boiler-\textsc{gen.sg} steam-\textsc{gen.sg} engine-\textsc{gen.sg} emit;\textsc{pf}.\textsc{3sg} 7000 big-\textsc{gen.pl} calory-\textsc{gen.pl}\\
\glt `If 1 kg of coal was burning down on the grate of a boiler of a steam engine, it would emit 7000 kilocalories.' \\ \hfill \citep[\ili{Czech};][77]{Svoboda1960a}
\z
\z

\item ICs express conditions and are therefore interpreted like finite conditional clauses marked by one of the subjunctions \textit{jestli(že)}, \textit{kdyby}, \textit{když}, \textit{-li}, \textit{pakliže}, or \textit{pokud} `if'. This is illustrated in \REF{ex:war2}, where the IC-example \REF{ex:war} from above is re-used and contrasted with a \textit{kdyby}-clause with the same meaning:

\ea\label{ex:war2}
\ea
\gll Ži-l-a by-s úplně jinak, [\hspace{-2pt} ne-bý-t válk-y]?\\
     live-\textsc{lpt-sg.f} \textsc{cond-2sg} completely differently {} \textsc{neg}-be-\textsc{inf} war-\textsc{gen.sg}\\
\glt `Would you have led a completely different life, if there had not been the war?'
\ex
\gll Ži-l-a by-s úplně jinak, [\hspace{-2pt} kdy-by ne-by-l-a válk-a]?\\
     live-\textsc{lpt-sg.f} \textsc{cond-2sg} completely differently  {} if=\textsc{cond.3sg} \textsc{neg}-be-\textsc{lpt-sg.f} war-\textsc{nom.sg.f}\\
\glt `Would you have led a completely different life, if there had not been the war?'
\z
\z

Finally, it should be noted that, unlike the above-mentioned types of conditional clauses with explicit conditional subjunctions, ICs never express real conditions, but are restricted to irreal (hypothetical and counterfactual) conditions.

\end{enumerate}

%----------------------------------------------------------------------------
%----------------------------------------------------------------------------

\section{Syntax of ICs}\label{sec:syntax}

In this section, we discuss and analyze the internal (\sectref{sec:syntax_internal}) and external (\sectref{sec:syntax_external}) syntax of \ili{Czech} ICs. Assumptions are kept as minimal as possible.

\subsection{Internal syntax}\label{sec:syntax_internal}

A number of facts provide evidence for the size of ICs.

A first thing to note is that given the \textsc{inf} is a \isi{transitive predicate}, ICs contain objects; see \REF{ex:music2}. Objects are base-generated in VP, so ICs are at least VPs.

\ea\label{ex:music2}
\gll [\hspace{-2pt} Ne-mí-t sv-ou hudb-u], tak se tu z-blázní-m!\\
     {} \textsc{neg}-have-\textsc{inf} own-\textsc{acc.sg.f} music-\textsc{acc.sg.f} so \textsc{refl} here \textsc{pf}-{become.insane}-\textsc{1sg}\\
\glt `If I had not my music, I would become insane here.' \\ \hfill \citep[\ili{Czech};][4]{Milotova2012}
\z

\noindent A second piece of evidence comes from the fact that ICs can be headed by infinitives of causative verbs. An example is \textit{udělat} `make' in \REF{ex:extraordinary} above. As causativity is commonly associated with the presence of the (non-deficient) head \textit{v} (see, e.\,g., \citealt{Marantz1999}), ICs must at least be \textit{v}Ps.

Furthermore, ICs can be negated. Standard assumptions link \isi{sentential negation} to the presence of a functional NegP \citep[see, e. g.,][432]{Blaszczak2009}. It follows that ICs are at least NegPs/PolPs.

Another crucial property of ICs is that they can contain overt \textsc{nom}-subjects, hence involve the option of \textsc{nom}-licensing. Examples above with \textsc{nom}-subjects are \REF{ex:start}, \REF{ex:Germans}, \REF{ex:money}, \REF{ex:father}, \REF{ex:extraordinary}, and \REF{ex:kettle-b}. Since it is the \isi{functional head} T which is often considered the locus of \textsc{nom}-licensing \citep[see, e. g.,][]{PesetskyTorrego2001}, this speaks in favor of analyzing ICs as TPs. The analysis of ICs as TPs is corroborated by the fact that these structures come with a propositional interpretation. By standard assumptions, TP is the syntactic equivalent of a proposition.

A final fact to observe is that ICs can be paraphrased by full-fledged (finite) conditional clauses. This might be regarded as evidence that, much like conditional clauses, ICs are full CPs. But the paraphrasability with conditional clauses is a weak \textit{syntactic} argument, as it merely restates the fact that ICs are \textit{interpreted} as conditions (see \sectref{sec:interpretation}). What is more, ICs never contain subjunctions or wh-pronouns. Due to these facts and the lack of substantial evidence for more syntactic structure, we analyze ICs as TPs.\footnote{We do not in principle exclude the possibility of a CP-layer in ICs. Under recent (phasal) assumptions, the C-head might be present but empty, submitting its features (and phase-hood) to T, thus effectively conflating with it. Under these assumptions, ICs would turn out as CPs and TPs at the same time.} \tabref{tab:1:format} summarizes the preceding considerations.

\begin{table}
\caption{Size of infinitival conditionals}
\label{tab:1:format}
 \begin{tabularx}{\textwidth}{lccccc}
  \lsptoprule
         	& VP & \textit{v}P & NegP & TP & CP\\
  \midrule
  availability of objects	& \ding{51} & & & &\\
  availability of causative verbs 	& & \ding{51} & & &\\
  availability of \isi{sentential negation} & & & \ding{51} & &\\
  \textsc{nom}-licensing and propositional interpretation & & & & \ding{51} &\\
  paraphrasability with conditional clauses & & & & & ?\\
  \lspbottomrule
 \end{tabularx}
\end{table}

%----------------------------------------------------------------------------

\subsection{External syntax}\label{sec:syntax_external}

There are two questions immediately arising with respect to the external syntax of ICs, namely (i) how and where ICs are connected with the matrix clause, and (ii) how the position of ICs relative to the matrix -- preposition vs. postposition -- can be explained.

\subsubsection{Connection between ICs and the matrix}\label{sec:syntax_external_connection}

The general question here is how and where ICs attach to the matrix structure. Our starting point to answer this question is a comparison between ICs and other clause types in \ili{Czech}.

A particular feature of \ili{Czech} syntax is that (at least a subset of) \isi{adverbial} clauses behave differently from \isi{relative} clauses and subject clauses. More precisely, where\-as \isi{relative} and subject clauses act as the syntactic host for second-position clitics in the matrix clause (see \citealt{Dokulil1956,Fried1994}), \isi{adverbial} clauses do not (see, a.\,o., \citealt[109]{Dokulil1956}; \citealt[150]{Lenertova2004}).

A subject clause example is cited in \REF{ex:Subjektsatz}. It shows that the \isi{second-position clitic} (in italics) is syntactically hosted by the subject clause (in square brackets):\footnote{Note that the direction of prosodic cliticization of \ili{Czech} second-position clitics may change; see \citet{Toman1996}.}

\ea\label{ex:Subjektsatz}
\gll [\hspace{-2pt} Že nikdo ne-protestova-l], \textit{ho} ne-překvapi-l-o.\\
	{} that nobody.\textsc{nom} \textsc{neg}-protest-\textsc{lpt-sg.m} he.\textsc{acc} \textsc{neg}-surprise-\textsc{lpt-sg.n}\\
\glt `The fact that nobody spoke up against it did not surprise him.' \\ \hfill \citep[\ili{Czech};][168]{Fried1994}
\z

\noindent An \isi{adverbial clause} example is given in \REF{ex:Adverbialsatz}. The \isi{second-position clitic} cannot be syntactically hosted by the \isi{adverbial clause}. Instead, it has to occur in a position following a stressed constituent of the matrix clause (here: \textit{odmlčel} `(he) fell silent'):

\ea\label{ex:Adverbialsatz}
\ea[*]{
\gll [\hspace{-2pt} Když domluvi-l], \textit{se} odmlče-l.\\
	{} when {finish-speaking}-\textsc{lpt-sg.m} \textsc{refl} {fall.silent}-\textsc{lpt-sg.m}\\
}
\ex[]{
\gll [\hspace{-2pt} Když domluvi-l], odmlče-l \textit{se}.\\
	{} when {finish-speaking}-\textsc{lpt-sg.m} {fall.silent}-\textsc{lpt-sg.m} \textsc{refl}\\
\glt `When he had finished speaking, he fell silent.' \\ \hfill \citep[\ili{Czech};][130--131]{Junghanns2002}
}
\z
\z

\noindent How do ICs behave with respect to second-position clitics in the matrix clause? The following data are based on example \REF{ex:father} above and reveal a parallel of ICs and \isi{adverbial} clauses:

\ea\label{ex:father2}
\ea[*]{
\gll [\hspace{-2pt} Bý-t otec doma], \textit{by} se to by-l-o ne-sta-l-o.\\
     {} be-\textsc{inf} father.\textsc{nom} {at.home} \textsc{cond.3sg} \textsc{refl} this.\textsc{nom.sg.n} be-\textsc{lpt-sg.n} \textsc{neg}-happen-\textsc{lpt-sg.n}\label{ex:father2-a}\\
}
\ex[]{
\gll [\hspace{-2pt} Bý-t otec doma], by-l-o \textit{by} se to ne-sta-l-o.\\
     {} be-\textsc{inf} father.\textsc{nom} {at.home} be-\textsc{lpt-sg.n} \textsc{cond.3sg} \textsc{refl} this.\textsc{nom.sg.n} \textsc{neg}-happen-\textsc{lpt-sg.n}\\
\glt `If father had been at home, this would not have happened.'\label{ex:father2-b}\\ \hfill \citep[\ili{Czech};][65]{Svoboda1960a}
}
\z
\z

\noindent The contrast in \REF{ex:father2} makes it clear that ICs are not like \isi{relative} and subject clauses, but rather like \isi{adverbial} clauses in that they do \textit{not} syntactically host second-position clitics. This leads us to the conclusion that ICs adjoin clause-externally to the matrix CP projection; see the structure in \REF{tree:CPadjunction}:\footnote{CP-adjunction is also proposed by \citet[138]{Reis1997} for all ``non-integrated'' \isi{adverbial} clauses in \ili{German}, while \citet[167\,ff]{ReisWoellstein2010} argue for a CP-adjunction analysis to capture \ili{German} V1-conditional clauses. More generally, \citet[71]{Haegeman2004} claims all non-integrated \isi{adverbial} clauses to be clause-external CP-adjuncts. On the other hand, integrated conditional clauses are usually analyzed as clause-internal TP- or \textit{v}P/VP-adjuncts; see, a.\,o., \citet[138]{Reis1997}, \citet[144--145,\,168]{ReisWoellstein2010}, and \citet[647]{BhattPancheva2006}.}

\ea\label{tree:CPadjunction}
{[\textsubscript{CP} \textbf{IC} CP ]}
\z

\noindent Given this CP-adjunction structure, the ungrammaticality of sentence \REF{ex:father2-a} with the \isi{second-position clitic} immediately following the \isi{IC} is due to the fact that the \isi{IC} is a clause-external adjunct. Assuming that the (split or non-split) CP is the domain of second-position cliticization in \ili{Czech} (see, a.\,o., \citealt{Junghanns2002,Lenertova2004}), the clause-external position of the \isi{IC} renders it ``invisible'' to the \isi{second-position clitic} in the matrix clause. The conclusion is that ICs cannot function as syntactic host for second-position clitics.\footnote{Further clause-external expressions that do not host second-position clitics in \ili{Czech} are gerundive expressions, vocatives, and external topics (see, a.\,o., \citealt{Dokulil1956}, \citealt{Travnicek1959}, \citealt[130--131]{Junghanns2002}).}
However, the structure in \REF{tree:CPadjunction} does not immediately capture data like \REF{ex:Pauline} where the \isi{IC} is apparently located \textit{within} the matrix CP.

\ea\label{ex:Pauline}
\gll Když jsme se tu by-l-i zjara projí-t s Pauline, prohodi-l-a, [\hspace{-2pt} že [\hspace{-2pt} ne-bý-t mě], vůbec by sem na ty smutn-é skál-y ne-doš-l-a], protože není romantik jako já.\\
	when be.\textsc{1pl} \textsc{refl} here be-\textsc{lpt-pl} {in-spring} walk-\textsc{inf} with Pauline remark-\textsc{lpt-sg.f} {} that {} \textsc{neg}-be-\textsc{inf} me.\textsc{gen} {at-all} \textsc{cond.3sg} here on these.\textsc{acc.pl} gloomy-\textsc{acc.pl} rock-\textsc{acc.pl} \textsc{neg}-go-\textsc{lpt-sg.f} because \textsc{neg}.be.\textsc{3sg} romantic.\textsc{nom.sg.m} like I.\textsc{nom}\\
\glt `When we went for a walk here in spring with Pauline, she remarked that, if it had not been for me, she would not at all have climbed these gloomy rocks, since she is not a romantic like me.' \\ \hfill (\ili{Czech}; Miloš Urban: \textit{Přišla z moře.} Praha: Argo 2014, 178--179.)
\z

\noindent Examples like \REF{ex:Pauline} are likely to involve CP-\isi{doubling} as depicted in \REF{tree:CPdoubling} (see \citealt{Kaspar2016} on the relevant syntatic structure).\footnote{Note that in \REF{ex:Pauline} neither the \isi{IC} nor the \isi{complementizer} can function as a syntactic host for the \isi{second-position clitic} \textit{by} (which is hosted by \textit{vůbec}). The CP-\isi{doubling} analysis in \REF{tree:CPdoubling} captures this because both the \isi{IC} and the \isi{complementizer} are clause-external relative to \textit{by}.\\ \citet{IatridouKroch1992} observe that CP-\isi{doubling} occurs under non-negative, non-irreal bridge verbs only. A reviewer remarks that the former condition does not seem to be correct; see \REF{ex:reviewer}:
\ea\label{ex:reviewer}
\gll Ne-vědě-l, že \minsp{[} ne-bý-t mě], vůbec by tu ne-by-l.\\
     \textsc{neg}-know-\textsc{lpt.sg.m} that {} \textsc{neg}-be-\textsc{inf} me.\textsc{gen} at-all \textsc{cond.3sg} here \textsc{neg}-be-\textsc{lpt.sg.m}\\
\glt `He didn't know that, if it was not for me, he would not be here at all.'
\z}

\ea\label{tree:CPdoubling}
{[\textsubscript{CP} \textit{že} [\textsubscript{CP} \textbf{IC} CP ]]}
\z

%----------------------------------------------------------------------------

\subsubsection{Position of ICs relative to the matrix clause}\label{sec:syntax_external_position}

If the CP-adjunction analysis in \REF{tree:CPadjunction} is on the right track, we are faced with two competing theoretical options concerning the position(ing) of the \isi{IC} -- pre- or postposition -- relative to the matrix clause:

\begin{itemize}

\item[(i)] There is only one base position for ICs: initial or final.\\ This might be characterized as the ``asymmetrical view'' (see, a.\,o., \citealt{kayne1994}) according to which all phrasal structures are uniformly linearized. It follows that deviating surface orders must be derived.

\item[(ii)] The initial and the final position of ICs are likewise base positions and thus syntactically on a par.\\ This might be characterized as the ``symmetrical view'' (see, a.\,o., \citealt{chomsky1986k,Chomsky2004}) according to which both left and right adjunction are equally permitted. It follows that statistically prevalent and/or marked orders must be explained by specific syntactic restrictions or so-called third factors.

\end{itemize}

\noindent As it does not require any derivations to take place, minimalist principles favor (ii). Assuming this, \REF{tree:left} and \REF{tree:right} are equally possible base positions of ICs:

\ea
\ea\label{tree:left}
{[\textsubscript{CP} \textbf{IC} CP ]}
\ex\label{tree:right}
{[\textsubscript{CP} CP \textbf{IC} ]}
\z
\z

\noindent Irrespective of which of these options is chosen, it is necessary to explain the empirical fact that ICs in initial position are more frequent than ICs in final position (see \citealt[74--75]{Svoboda1960a}, \citealt{Milotova2011,Milotova2012}). A first hint to a possible answer comes from \citet[75]{Svoboda1960a} who suggests that the position of ICs correlates with the theme/rheme (or information) structure of the relevant sentences. We build on his intuition, but will expand it: Indeed, the possible \isi{relative} positions of ICs have to do with information structure, but information structure alone is not sufficient to explain the facts.

Our proposal is that a ``third factor'' needs to be considered, namely the logical structure underlying the relevant utterances; see \REF{ex:logical}. To put it more precisely: It is the relation between the underlying \textit{logical structure} and the actual constituent order as following from \textit{information structure} which determines if an \isi{IC} is left- or right-adjoined to the matrix CP.

\ea\label{ex:logical}
condition $\rightarrow$ consequence \hfill (logical structure)
\z

\noindent According to Greenberg's universal 14 , the conditional clause precedes the consequence clause (\citealt[66]{Greenberg1963}). This is the normal order in conditional statements in all languages (see also \citealt[445--446]{Diessel2001}). Given this, the default or unmarked surface order of statements with ICs is one where the surface ordering of \isi{IC} and matrix clause iconically reflects the underlying logical structure; see \REF{tree:left}. The reverse ordering in \REF{tree:right} is (relatively) more marked.

In order to explain the relation between logical structure and information structure, it is necessary to introduce at least basic information-structural concepts: Our theoretical framework in this respect is the ``Leipzig model of information structure'' (see, a.\,o., \citealt{JunghannsZybatow2009}). According to this model, there are two coexisting, but not necessarily coextensive structures, namely the \textsc{topic/comment} and the \textsc{\isi{focus}/background} structure. Another crucial point to note in the current context is that (at least) two types of \isi{focus} have to be distinguished, namely (a) \textsc{natural focus} and (b) \textsc{contrastive focus}. While constituents associated with (a) appear in sentence-final position, constituents associated with (b) show a particular intonational contour and may occupy any \isi{syntactic position}.

With these few theoretical concepts, the following observations can be made with respect to conditional sentences containing ICs:

\begin{itemize}

\item Irrespective of its actual surface position in the sentence, the \isi{IC} is \textit{always} background material. This status is logically determined: Since conditions are logical premises, they correspond to background information in propositional utterances, viz. conditional clauses.

\item Depending on the context, various information-structural analyses are possible for sentences with the ``logically unmarked'' ordering in \REF{tree:left}, i.\,e. with the \isi{IC} in initial position.

\item By contrast, the matrix CP is always contrastively focused in sentences showing the ``marked'' order in \REF{tree:right}, i.\,e. with the \isi{IC} in final position; see \REF{ex:television-a} and its information-structural analysis in \REF{ex:television-b}:\footnote{It should be noted that contrastively focused constituents in sentence-initial position convey special emphasis as compared with contrastively focused constituents in other positions. This seems to be the reason why the initial positioning of contrastively focused constituents is the preferred strategy in spoken language.}

\end{itemize}

\ea\label{ex:television}
\ea\label{ex:television-a}
\gll Part-a by se rozpad-l-a, [\hspace{-2pt} ne-bý-t televiz-e].\\
     group-\textsc{nom.sg.f} \textsc{cond.3sg} \textsc{refl} {fall-apart}-\textsc{lpt-sg.f} {} \textsc{neg}-be-\textsc{inf} television-\textsc{gen.sg}\\
\glt `The group would (have) fall(en) apart, if there would not (have) be(en) television.' \hfill \citep[\ili{Czech};][4]{Milotova2012}
\ex\label{ex:television-b}
\gll {Parta by se R/OZpadla}, {[nebýt televize].}\\
	{\textsc{contrastive focus}}	{\textsc{background}}\\
\z
\z

\noindent Our conclusions concerning the relative position(ing) of ICs are as follows:

\begin{itemize}

\item ICs are right- or left-adjoined to the matrix CP projection, with neither of these possibilities being ``more basic'' than the other one (symmetrical view on adjunction).

\item Right-adjoined ICs -- see \REF{tree:right} -- are the (relatively more) marked option for two interrelated reasons:

	\begin{itemize}

	\item[(i)] Conditional sentences with ICs in final position do not match with the logical structure underlying conditional utterances (anti-iconicity).

	\item[(ii)] When the \isi{IC} appears in sentence-final position, the matrix CP is always contrastively focused (information structure).

	\end{itemize}

\end{itemize}

%----------------------------------------------------------------------------
%----------------------------------------------------------------------------

\section{Nominative case licensing}\label{sec:nom}

A syntactic peculiarity of ICs is the availability of overt \textsc{nom}-subjects. Not only for \ili{Slavic} languages is this fact surprising, given the common view that there is a 1-to-1 correspondence between \textsc{nom}-licensing and \isi{finiteness}, where the latter term usually refers to clausal structures that exhibit both verb-subject agreement and a \isi{tense} specification (see, e.\,g., \citealt[4]{Cowper2002}).

\ili{Czech} is a consistent \textit{pro}-drop language, so -- again by standard assumptions -- the following biconditionals (should) hold:

\ea\label{ex:finiteness}
\ea finite structure $\leftrightarrow$ \textsc{nom} \hfill $\rightarrow$ \textit{pro} or DP
\ex non-finite structure $\leftrightarrow$ \textsc{*nom} \hfill $\rightarrow$ \textsc{pro}
\z
\z

\noindent However, the picture drawn is too simple as it cannot explain the licensing of overt \textsc{nom}-subjects in \ili{Czech} ICs. To find a possible solution, we discuss three potential theoretical alternatives in \sectref{sec:nom_possibilities}. Building on one of these alternatives, we will present our proposal in \sectref{sec:nom_proposal}. Finally, we will discuss the conditions for \textsc{nom}-licensing in \ili{Czech} ICs in \sectref{sec:nom_conditions}.

%----------------------------------------------------------------------------

\subsection{Theoretical possibilities}\label{sec:nom_possibilities}

We can come up with three theoretical possibilities challenging the standard view on \textsc{nom}-licensing mentioned in \REF{ex:finiteness} so as to explain the availability of overt \textsc{nom}-subjects in ICs.

\subsubsection{Finiteness in C}\label{sec:nom_possibilities_1}

According to \citet{kayne1994}, \isi{finiteness} requires the incorporation of T[ense] into C[omp].\footnote{Already \citet{Raposo1987} argues for ``infl-to-comp'' movement.} In a similar vein, recent minimalist theory (a.\,o., \citealt{Chomsky2007} or \citealt{Richards2007}) assumes feature inheritance from C to T. Even more plainly, finite\-ness is analyzed as a part of the C-system in the cartographic framework of \citet{rizzi1997}.

The assumptions just mentioned have in common that \isi{finiteness} originates not in T, but in C. One theoretical advantage of this view is that the TP is furnished with \isi{finiteness} ``from outside'', so that there is no necessity to directly connect \isi{finiteness} or non-\isi{finiteness} to respective morphological markers appearing on the \isi{verb} in \textit{v}/V. This, in turn, gives rise to the theoretical possibility for clauses headed by a non-finite \isi{verb} to be finite.

Although the theory itself is quite appealing, it has one crucial disadvantage when applied to ICs: As discussed in \sectref{sec:syntax_internal}, there is no empirical evidence that ICs comprise a CP-layer. Assuming the category of \isi{finiteness} to originate in C would force one to stipulate the presence of a CP in ICs exclusively for theory-internal reasons. Since such an approach contradicts our general aim of keeping the body of assumptions as minimal as possible, other theoretical possibilities need to be taken into consideration.

\subsubsection{``UPro'' in non-finite structures}\label{sec:nom_possibilities_2}

\citet{Sundaresan2014} and \citet{McFaddenSundaresan2018} observe that the standard theory of \isi{pronominal} null subjects in non-finite structures faces severe empirical problems. They convincingly show that there are a number of unrelated languages that exhibit non-finite clauses with case-marked \textit{pro}-subjects (as well as there are languages with finite clauses that have \textsc{pro}-subjects). To account for these data, they propose that \textit{pro} and \textsc{pro} are based on one and the same lexical item -- an underspecified null \isi{pronoun} dubbed ``UPro'' --, the concrete realization of which is determined by its actual syntactic context.

Although the authors remain silent as regards case-licensing, their proposal allows for the possibility of \textsc{nom}-licensing in non-finite clauses. Moreover, their theory is sufficiently restrictive in that \textsc{nom}-licensing in non-finite environments can only take place under specific (syntactic) conditions, which effectively means that \textsc{pro} remains the default type of subject pronouns in non-finite clauses.

Clearly, the crucial advantage of this approach lies in its flexibility concerning the distribution of null subject pronouns in (not only) non-finite structures. With respect to ICs, however, there is also at least one severe disadvantage, namely the unclear connection between the realization of either \textsc{pro}, \textit{pro}, or a full DP and the syntactic mechanism of case-licensing (since the cited authors pursue a \isi{derivational} approach, they do not attribute the appearance of the elements in question to case-licensing at all). Moreover, the theory relates to null subject pronouns, leaving aside the possibility of overt DP subjects in non-finite structures, which is exactly what we find in \ili{Czech} ICs. This is why, despite its obvious theoretical advantages and rich empirical coverage, the ``UPro'' theory cannot explain ICs with \textsc{nom}-subjects, so a third theoretical option must be considered.

\subsubsection{Tensed non-finite structures}\label{sec:nom_possibilities_3}

The final theoretical possibility to be discussed goes back to \citet{Stowell1982,Stowell1995} and \citet{Wurmbrand2001}. According to it, there is no 1-to-1 correspondence between \textsc{nom}-licensing and \isi{finiteness} as suggested in \REF{ex:finiteness}. The crucial proposal is that non-finite clauses are not necessarily untensed, but may be tensed (which is taken to be the prerequisite for the \textsc{nom} to be licensed). In other words, being non-finite is not the same as being untensed, which gives rise to a much more flexible system as compared to standard assumptions.

To demonstrate this flexibility, we use the two features [\textsc{agr}] (`agreement') and [\textsc{tensed}] (`tensed') for a cross-classification; see \tabref{tab:2:crossclass}.\footnote{We prefer the feature [\textsc{agr}] to [\textsc{fin}] (`finite') because the grammatical concept of ``\isi{finiteness}'' seems to be a conspiracy of two distinct properties, namely \textsc{non-/agreement} (showing up on the \isi{verb}) and \textsc{un-/tensedness} (reflected in the non-/availability of a \textsc{nom}-subject), rather than a single grammatical property.}

\begin{table}
\caption{Classes of finiteness}
\label{tab:2:crossclass}
 \begin{tabular}{lcc}
  \lsptoprule
		& [\textsc{agr}] & [\textsc{tensed}]\\
  \midrule
  (i) & \ding{51} & \ding{51} \\
  (ii) & &  \\
  (iii) & \ding{51} & \\
  (iv) & & \ding{51} \\
  \lspbottomrule
 \end{tabular}
\end{table}

The prototypical constellations (at least in most \ili{Indo-European} languages) are the classes (i) and (ii). While class (i) represents prototypical finite structures, class (ii) represents prototypical non-finite structures. Classes (i) and (ii) correspond to \REF{ex:finiteness}. We should also like to point out that class (ii) is likely to capture \ili{Czech} ICs that do not contain an overt \textsc{nom}-subject: Assuming syntax to be a parsimonious device, one would clearly state that, if there is no necessity for an overt subject in an \isi{IC}, \textsc{nom}-licensing in terms of [\textsc{tensed}] should not take place either, leaving a \textsc{pro}-subject as the optimal choice.

However, two further options arise:

Class (iii) represents what might be called non-canonical (or apparent) finite structures. Since it does not immediately relate to our topic, we leave it to future empirical and theoretical research to find out whether or not structures of this kind exist.\footnote{Arguably, class (iii) captures \ili{Bulgarian} non-finite \textit{da}-clauses (see \citealt{KrapovaPetkov1999} for the proposal that \ili{Bulgarian} \textit{da}-clauses come in two varieties, [$+$T] or [$-$T]; see also \citealt{Pitsch2018}). Here, \textsc{nom}-licensing seems to be impossible despite the presence of a (``finite'') \isi{verb} agreeing in person and number with the (implicit) subject. It seems that this class is restricted to languages that lack an \isi{infinitive}, hence do not explicitly mark non-\isi{finiteness} (or, rather, non-agreement).}

Finally, class (iv) represents what might be called non-cano\-nical (or apparent) non-finite structure. This featural combination is of special importance for the present discussion, since it has the capacity to capture \ili{Czech} ICs with an overt \textsc{nom}-subject.

The overall advantage of this theoretical alternative is that, while the feature [\textsc{agr}] relates to the \isi{inflectional} morphology showing up on the \isi{verb} form in \textit{v}/V (which is always an \isi{infinitive} in case of ICs), the feature [\textsc{tensed}] relates to T. The two features are thus independent of one another (only for the two ``prototypical'' classes (i) and (ii) in \tabref{tab:2:crossclass} are they in ``harmony'').

We suspect that it is this splitting up of \isi{finiteness} into two distinct grammatical properties -- namely (i) being marked for agreement (\isi{verb} form in \textit{v}/V) and (ii) being tensed (\isi{functional head} T) -- which allows a consistent syntactic analysis of non-canonically behaving non-finite structures, among them \ili{Czech} ICs with overt \textsc{nom}-subjects. Essentially, we argue that ICs are underspecified as concerns the feature [\textsc{un/tensed}] and can, hence, belong to two different classes shown among those in \tabref{tab:2:crossclass}: class (ii) or class (iv).

%----------------------------------------------------------------------------
%----------------------------------------------------------------------------

\subsection{Proposal}\label{sec:nom_proposal}

Building on ideas by \citet{Stowell1982,Stowell1995} and \citet{Wurmbrand2001} (see \sectref{sec:nom_possibilities_3}), we argue that the concept of \isi{finiteness} is composed of two distinct grammatical properties, representable by the features [\textsc{agr}] and [\textsc{tensed}], respectively. Our specific claim with respect to \ili{Czech} ICs is that, due to their being headed by an \isi{infinitive}, these structures are always specified as lacking [\textsc{agr}], but are variable as concerns the specification of T with or without [\textsc{tensed}].

While this section concerns the technical details of our analysis, we will discuss the conditions under which T is (not) specified with [\textsc{tensed}] in \sectref{sec:nom_conditions}.

An adequate means to formalize our ideas is the twofold distinction between \textsc{interpretability} and \textsc{valuation} as proposed by \citet{PesetskyTorrego2001}. Accordingly, syntactic features can be interpretable or uninterpretable: Interpretable features are associated with semantic content and are thus relevant to interpretation, which is why they are transferred to the conceptual-intentional system. By contrast, uninterpretable features are purely formal and have to be deleted before spell-out. On the other hand, features of whatever grammatical category need a certain value. While some features come with a value, others need to be valued in the course of \isi{syntactic derivation}. Crucially, interpretable features can only be sent to the interface(s) if they have (received) a value.

Above, we used two simple features, [\textsc{agr}] and [\textsc{tensed}]. In the framework just described, these features translate as follows into more complex features with a position for a value (in square brackets):

\begin{itemize}

\item {[}\textsc{agr}] translates into \{i,u\}\textPhi[~~]

\item {[}\textsc{tensed}] translates into \{i,u\}T[~~]

\end{itemize}

\noindent For our analysis, the \isi{functional head} T, overt \textsc{nom}-subjects, \textsc{pro}, and the \isi{infinitive} in \textit{v}/V are the relevant syntactic objects.\footnote{Our analysis implies that \ili{Czech} ICs never contain \textit{pro}-subjects.} They have the following features:

\begin{enumerate}

\item Since it is the locus of semantic \isi{tense}, the \isi{functional head} T is equipped with an interpretable T-feature which can have one of two possible values, namely \textsc{tensed} or \textsc{untensed}.\footnote{In finite clauses, the T-value \textsc{tensed} T is further specified as present, past, or future. By contrast, tensed ICs lack further specification due to the absence of a \isi{tense} marker on the \isi{infinitive}. It follows that the \isi{relative} temporal orientation of ICs can only be inferred from the context.} On the other hand, the Φ-features of T are uninterpretable and unvalued.

\item Overt subjects have an uninterpretable T-feature, the morphological reflex of which is the \textsc{nom} (see \citealt{PesetskyTorrego2001}). Additionally, they enter syntax equipped with inherent interpretable and valued Φ-features.

\item The null \isi{pronoun} \textsc{pro} bears an uninterpretable T-feature with the value \textsc{untensed} (in other words, \textsc{pro} has no case). As to Φ-features, \textsc{pro} enters syntax without any value -- only in the course of \isi{syntactic derivation} is it supplied with values by its controller (usually a matrix argument).

\item The \isi{infinitive} in \textit{v}/V is equipped with an uninterpretable and unvalued T-feature. Its Φ-features are uninterpretable and unvalued, too.\footnote{T- and Φ-features of \isi{verb} forms in \textit{v}/V are generally uninterpretable, their interpretable counterparts being T and the subject, respectively. The basic idea behind this is that verbal \isi{inflectional} morphology is but a formal (semantically vacuous) reflex of the semantically relevant features of T and the subject.} This means that the \isi{infinitive} is completely void of grammatical information. Crucially, it does \textit{not} inherently bear the T-value \textsc{untensed}, but is unspecified in this respect (the value is set in the course of derivation against the respective value of T).\footnote{The unspecification of the \ili{Czech}/\ili{Slovak} \isi{infinitive} with respect to its T-value seems to be the crucial difference as compared to infinitives of other (non-)\ili{Slavic} languages that lack structures with infinitives and \textsc{nom}-subjects.}

\end{enumerate}

\tabref{tab:3:features} gives an overview of the features of the relevant syntactic objects.

\begin{table}
\caption{Syntactic features}
\label{tab:3:features}
 \begin{tabular}{lcc}
  \lsptoprule
		& T-feature & Φ-feature\\
  \midrule
  T-head & iT[\{\textsc{tensed,untensed}\}] & u\textPhi[~~~]\\
  \textsc{nom}-subjects & uT[\textsc{tensed}] & i\textPhi[Φ]\\
  \textsc{pro} & uT[\textsc{untensed}] & i\textPhi[~~~]\\
  \isi{infinitive} (\textit{v}/V) & uT[~~~] & u\textPhi[~~~]\\
  \lspbottomrule
 \end{tabular}
\end{table}

\largerpage[2]
From these assumptions it follows that \ili{Czech} infinitives in ICs may either receive the value \textsc{tensed} or \textsc{untensed}. This in turn means that ICs come in two variants.\footnote{A reviewer suggests that, unless there is good evidence for \textsc{pro}, it seems easier to postulate only one, \textit{tensed} structure for ICs (\textit{pro}). We agree that the status of \textsc{pro} is controversial, but the claim of \textit{pro} in ICs without overt subjects strikes us as theoretically disadvantageous, because if such ICs are tensed and contain \textit{pro}, this implies that potentially all non-finite structures are really tensed and contain \textit{pro} -- a rather bold claim. Moreover, untensed ICs with \textsc{pro} comply with the economy principle, as they can be derived and interpreted with minimal effort.} We give detailed analyses of both of them in the sections to follow.

\subsubsection{Non-finite ICs}\label{sec:nom_proposal_nonfinite}

The first possible variant of \ili{Czech} ICs is untensed, hence the T-head involved enters syntax with the T-feature iT[\textsc{untensed}]. The \isi{infinitive} in \textit{v}/V then receives the same value. From the lack of tensedness, it follows that the subject can only be \textsc{pro}. \textsc{pro} receives Φ-values (which are subsequently transmitted to T and \textit{v}/V) from its controller. In \REF{ex:nonfinite}, the line in (a) shows the situation as it is after base-merge, the line in (b) shows the relevant valuations, and the line in (c) shows the deletion of uninterpretable features before spell-out.

\ea\label{ex:nonfinite}
\ea {[}\textsubscript{TP} T\textsubscript{iT[\textsc{untensed}],u\textPhi[~~]} [\textsubscript{\textit{v}/VP} \textsc{pro}\textsubscript{uT[\textsc{untensed}],i\textPhi[~~~]} \textit{v}/V\textsubscript{uT[~~~~~~~~~~~~~~~~],u\textPhi[~~~]} ]]
\ex {[}\textsubscript{TP} T\textsubscript{iT[\textsc{untensed}],u\textPhi[Φ]} [\textsubscript{\textit{v}/VP} \textsc{pro}\textsubscript{uT[\textsc{untensed}],i\textPhi[Φ]} \textit{v}/V\textsubscript{uT[\textsc{untensed}],u\textPhi[Φ]} ]]
\ex {[}\textsubscript{TP} T\textsubscript{iT[\textsc{untensed}],\sout{u\textPhi[Φ]}} [\textsubscript{\textit{v}/VP} \textsc{pro}\textsubscript{\sout{uT[\textsc{untensed}]},i\textPhi[Φ]} \textit{v}/V\textsubscript{\sout{uT[\textsc{untensed}]},\sout{u\textPhi[Φ]}} ]]
\z
\z


\subsubsection{Finite ICs}\label{sec:nom_proposal_finite}

The second possible variant of ICs is tensed, hence the T-head involved enters syntax with the T-feature iT[\textsc{tensed}].\footnote{A reviewer asks what it means interpretationally for a T to be tensed, and suggests that tensed T's might be not (as) dependable in their interpretation on the matrix T. \citet{KrapovaPetkov1999} tackle this issue wrt. \ili{Bulgarian} \textit{da}-clauses which seem to come in two varieties, tensed and untensed, thus resembling ICs. However, \textit{da}-clauses are not restricted to conditionals, but occur in many contexts, which allows linguistic tests to reveal differences between their variants. The \isi{conditional interpretation} of ICs excludes such tests. As an example, it is impossible to use different temporal adverbials in the \isi{IC} and the main clause, since conditionals imply a temporal order that cannot be altered. Hence our belief that the un/tensedness of ICs only shows in the un/availability of an overt \textsc{nom}-subject.} The \isi{infinitive} in \textit{v}/V then receives the same value. The presence of a \isi{tense} specification licenses the \textsc{nom} on the subject (the subject's Φ-features subsequently value the Φ-features of T and \textit{v}/V). At the surface, the tensedness of this type of ICs is reflected by the morphological \textsc{nom}-marking of the subject, see \REF{ex:finite}:

\ea\label{ex:finite}
\ea {[}\textsubscript{TP} T\textsubscript{iT[\textsc{tensed}],u\textPhi[~~]} [\textsubscript{\textit{v}/VP} DP\textsubscript{uT[\textsc{tensed}],i\textPhi[Φ]} \textit{v}/V\textsubscript{uT[~~~~~~~~~~~~],u\textPhi[~~~]} ]]
\ex {[}\textsubscript{TP} T\textsubscript{iT[\textsc{tensed}],u\textPhi[Φ]} [\textsubscript{\textit{v}/VP} DP\textsubscript{uT[\textsc{tensed}],i\textPhi[Φ]} \textit{v}/V\textsubscript{uT[\textsc{tensed}],u\textPhi[Φ]} ]]
\ex {[}\textsubscript{TP} T\textsubscript{iT[\textsc{tensed}],\sout{u\textPhi[Φ]}} [\textsubscript{\textit{v}/VP} DP\textsubscript{\sout{uT[\textsc{tensed}]},i\textPhi[Φ]} \textit{v}/V\textsubscript{\sout{uT[\textsc{tensed}]},\sout{u\textPhi[Φ]}} ]]
\z
\z


%----------------------------------------------------------------------------

\subsection{Conditions for \textsc{nom}-licensing in ICs}\label{sec:nom_conditions}

This section concerns the conditions that require ICs to be tensed, hence under which \textsc{nom}-licensing in ICs takes place (see \sectref{sec:nom_proposal_finite}).

Two different scenarios -- (a) and (b) -- have to be distinguished. They can be characterized by two criteria:

\begin{itemize}
\item[(i)] Is the subject of the \isi{IC} overt or null?
\item[(ii)] Are the subjects in the compound conditional sentence in \isi{coreference} or in disjoint reference?
\end{itemize}

%----------------------------------------------------------------------------

\begin{itemize}
\item[\textbf{(a)}] \textbf{Null subject, coreference}

In this scenario, \isi{derivational} economy forces the insertion of \textsc{pro}, as it does not require \textsc{nom}-licensing and automatically yields \isi{coreference} with the matrix subject; see \REF{ex:music3}:

\ea\label{ex:music3}
\gll {[\textsc{pro}$_i$} Ne-mí-t sv-ou hudb-u], tak se tu {\textit{pro}$_i$} z-blázní-m!\\
     {} \textsc{neg}-have-\textsc{inf} own-\textsc{acc.sg.f} music-\textsc{acc.sg.f} so \textsc{refl} here {} \textsc{pf}-{become.insane}-\textsc{1sg}\\
\glt `If I had not my music, I would become insane here.'
\z

%----------------------------------------------------------------------------

\item[\textbf{(b)}] \textbf{Overt subject, \isi{coreference} or disjoint reference}

In this scenario, T has to bear the feature iT[\textsc{tensed}] to license the \textsc{nom} on the subject; see \REF{ex:father3}:

\ea\label{ex:father3}
\gll [\hspace{-2pt} Bý-t otec$_j$ doma], by-l-o by se to$_i$ ne-sta-l-o.\\
     {} be-\textsc{inf} father.\textsc{nom} {at.home} be-\textsc{lpt-sg.n} \textsc{cond.3sg} \textsc{refl} this.\textsc{nom.sg.n} \textsc{neg}-happen-\textsc{lpt-sg.n}\\
\glt `If father had been at home, this would not have happened.'
\z

\noindent Note that in \REF{ex:father3}, the subject is background material. However, subjects may also be contrastively focused. If so, they typically occur in clause-initial position as illustrated by \REF{ex:father4}:

\ea\label{ex:father4}
\gll [\hspace{-2pt} Otec$_j$ bý-t doma], by-l-o by se to$_i$ ne-sta-l-o.\\
     {} father.\textsc{nom} be-\textsc{inf} {at.home} be-\textsc{lpt-sg.n} \textsc{cond.3sg} \textsc{refl} this.\textsc{nom.sg.n} \textsc{neg}-happen-\textsc{lpt-sg.n}\\
\glt `If F/ATHER had been at home, this would not have happened.'
\z

\noindent A special subcase of this scenario involves overt subject pronouns, since these, in \ili{Czech}, are always contrastively focused (as opposed to non-em\-phat\-ic null pronouns); see \REF{ex:son-a} illustrating \isi{coreference} and \REF{ex:son-b} illustrating disjoint reference:

\ea\label{ex:son}
\ea\label{ex:son-a}
\gll [\hspace{-2pt} Já$_i$ bý-t váš syn], tak se \textit{pro}$_i$ ne-žení-m.\\
	{} I.\textsc{nom} be-\textsc{inf} your.\textsc{nom.sg.m} son.\textsc{nom.sg.m} so \textsc{refl} {} \textsc{neg}-marry-\textsc{1sg} \\
\glt `If /I were your son, I would not marry.'\\\xspace\hfill \citep[\ili{Czech};][12]{Milotova2012}
\ex\label{ex:son-b}
\gll [\hspace{-2pt} Já$_j$ bý-t doma], by-l-o by se to$_i$ ne-sta-l-o.\\
	{} I.\textsc{nom} be-\textsc{inf} {at.home} be-\textsc{lpt-sg.n} \textsc{cond.3sg} \textsc{refl} this.\textsc{nom.sg.n} \textsc{neg}-happen-\textsc{lpt-sg.n}\\
\glt `If /I had been at home, this would not have happened.'
\z
\z

\end{itemize}

\noindent The following generalizations can be made:

\begin{enumerate}

\item If the subject of the \isi{IC} is contrastively focused, it has to be overt, so the \textsc{nom} must be licensed and T has to be equipped with iT[\textsc{tensed}]; see \REF{ex:son-a}.\footnote{See \citet{Szabolcsi2009} for similar data from \ili{Hungarian}. The relevance of contrastive \isi{focus} is also mentioned by \citet[513]{McFaddenSundaresan2018}.}

\item If the subject of the \isi{IC} is not contrastively focused, \textsc{pro} is sufficient and the \textsc{nom} does not need to be licensed, hence T can enter syntax with iT[\textsc{untensed}]; see \REF{ex:music3}.

\item Irrespective of being contrastively focused or not, the subject of the \isi{IC} has to be overt in case its referent is not identifiable from the matrix clause or the preceding context. If so, the \isi{IC} must be tensed for the subject to be overtly realized; see \REF{ex:father3}.

\end{enumerate}

\noindent It should be noted that the \isi{coreference}/disjoint\,reference-criterion is merely an epiphenomenon. The primary criteria are (i) identifiability and/or (ii) contrastive focussing of the subject of the \isi{IC}.

%----------------------------------------------------------------------------
%----------------------------------------------------------------------------

\section{Interpretation}\label{sec:interpretation}

A crucial question with respect to ICs relates to interpretation: How does the \isi{conditional interpretation} of the compound sentence arise? There are two competing theoretical possibilities:

\begin{enumerate}

\item The compositional view:\footnote{See, e.g., the account of conditionals proposed by \citet{Kratzer1979,Kratzer1986,Kratzer1991}. Accordingly, even so-called bare conditionals (without explicit \isi{modal} operators) contain silent \isi{modal} (necessity) operators.}\\ Sentences with ICs contain a covert operator that induces the \isi{conditional interpretation}.

\item The pragmatic view:\footnote{See the analyses of \citet{Stalnaker1975,ByrneJohnsonLaird2010,ElderJaszczolt2016}. \citet[41]{ElderJaszczolt2016} give an overview of existing pragmatic analyses of conditionals.}\\ The \isi{conditional interpretation} arises due to specific lexical and grammatical triggers, the context, and/or world knowledge.

\end{enumerate}

\noindent The facts speak in favor of the second -- pragmatic -- view (see also \citealt{Milotova2011,Milotova2012}). First, ICs cannot be used in isolation to answer a question about the condition on the realization of a situation; see \REF{ex:frage-a} and \REF{ex:frage-b} for the question and answer, respectively:

\ea\label{ex:frage}
\ea[]{
\gll Pod jak-ou podmínk-ou by se to by-l-o ne-sta-l-o? \\
	under which-\textsc{ins.sg} condition-\textsc{ins.sg} \textsc{cond.3sg} \textsc{refl} this.\textsc{nom.sg.n} be-\textsc{lpt-sg.n} \textsc{neg}-happen-\textsc{lpt-sg.n} \\
\glt `Under what condition would this not have happened?'\label{ex:frage-a}
}
\ex[*]{
\gll Otec bý-t doma. \\
	father.\textsc{nom.sg} be-\textsc{inf} {at.home} \\
\glt \textit{intended:} `If father had been at home.'\label{ex:frage-b}
}
\z
\z

\noindent The reason for this exclusion is that ICs lack an ``unambiguous relation marking'' (``eindeutige Bezugskennzeichnung''; \citealt[135]{Reis1997}), i.\,e. a conditional subjunction.

Second, it is generally possible to identify specific signals in the matrix clause that trigger the irreal, viz. \isi{conditional interpretation}: (i) irrealis/conditional periphrasis; (ii) (realis/indicative) \isi{modal} verbs (e.\,g. \textit{moci} `can'); (iii) (inherently future-oriented) \isi{perfective} aspect; see \REF{ex:money}; (iv) lexical signals, e.g. \textit{tak} `so' in \REF{ex:son-a}, \textit{ted'} `now' (with past \isi{tense}) in \REF{ex:ruler}; (v) context; see \REF{ex:kontext}; (vi) world knowledge; see \REF{ex:weltwissen}.

\ea\label{ex:ruler}
\gll [\hspace{-2pt} Ne-bý-t	věc-í, s nimi-ž jsem zbytečně z-tráce-l čas], stá-l jsem ted' před vládc-em. \\
	{} \textsc{neg}-be-\textsc{inf} thing-\textsc{gen.pl} with which.\textsc{ins.pl}-\textsc{rel} be.\textsc{1sg} pointlessly \textsc{pf}-lose-\textsc{lpt-sg.m} time.\textsc{acc.sg} stand-\textsc{lpt-sg.m} be.\textsc{1sg} now {in.front.of} ruler-\textsc{ins.sg} \\
\glt `If there had not been things with which I pointlessly spent my time, I would now be standing in front of the ruler.’ \hfill \citep[\ili{Czech};][5]{Milotova2012}
\z

\ea\label{ex:kontext}
\gll {\ldots}\hspace{-2pt} kdy jsem tu přenocova-l, to však by-l-o opravdu jen ojediněl-é, protože to ne-stá-l-o za t-u rozmrzelost manželk-y. [\hspace{-2pt} Ne-bý-t jí], nocova-l jsem na chat-ě určitě častěji.\\
	{} when be.\textsc{1sg} here {stay.overnight}-\textsc{lpt-sg.m} this.\textsc{nom.sg.n} however be-\textsc{lpt-sg.n} really only sporadical-\textsc{nom.sg.n} because this.\textsc{nom.sg.n} \textsc{neg}-{be worth}-\textsc{lpt-sg.n} for this-\textsc{acc.sg.f} moroseness.\textsc{acc.sg.f} wife-\textsc{gen.sg} {} \textsc{neg}-be-\textsc{inf} she.\textsc{gen} {stay.the.night}-\textsc{lpt-sg.m} be.\textsc{1sg} at cottage-\textsc{loc.sg} surely more.often \\
\glt `\ldots when I stayed overnight here, this, however, happened really only sporadically, since it was not worth my wife's moroseness. If it had not been for her, I surely would have spent the night in the cottage more frequently.' \hfill \citep[\ili{Czech};][5]{Milotova2012}
\z

\ea\label{ex:weltwissen}
\gll [\hspace{-2pt} Ne-bý-t Clintonov-y administrativ-y], ne-jsme v NATO. \\
	{} \textsc{neg}-be-\textsc{inf} Clinton's-\textsc{gen.sg.f} administration-\textsc{gen.sg.f} \textsc{neg}-be-\textsc{1pl} in NATO \\
\glt `If it had not been for the Clinton administration, we would not be in the NATO.' \hfill \citep[\ili{Czech};][5]{Milotova2012}
\z

%----------------------------------------------------------------------------
%----------------------------------------------------------------------------

\section{Summary}\label{sec:summary}

The present investigation allows the following conclusions concerning \ili{Czech} ICs:\\

Semantically,
\begin{itemize}
	\item ICs express conditions without explicit marking as conditional\\ clauses;
	\item ICs yield their interpretation pragmatically through specific grammatical and lexical signals in the matrix clause (irrealis/conditional periphrasis, \isi{modal} verbs, temporal adverbs, etc.), the context, and/or world knowledge.
\end{itemize}

Regarding their use,
\begin{itemize}
	\item ICs are highly marked, expressive, and context-dependent expressions in \ili{Czech}, which is why they refer to specific situations only.
\end{itemize}

Syntactically,
\begin{itemize}
	\item ICs are (at least) TPs;
	\item ICs can be tensed (\textsc{nom}-subject) or untensed (\textsc{pro}), depending on whether or not their subject is identifiable from the context and/or contrastively focused;
	\item ICs are clause-external adjuncts to the matrix CP.
\end{itemize}

From a typological point of view,
\begin{itemize}
	\item ICs have cross-linguistic parallels, see \REF{ex:Tamil} from \ili{Tamil};
	\item \ili{Czech} and \ili{Slovak} are the only \ili{Slavic} languages that exhibit such structures; for \ili{Slovak} facts see \citet{Ruzicka1956} and \citet{Hirschova2005}.
\end{itemize}

\noindent A possible explanation for the latter fact is that the \ili{Czech} (and \ili{Slovak}) \isi{infinitive} is lexically unvalued with respect to its T- and Φ-features, so (i) it is not restricted to untensed clauses, but may also occur in tensed clauses, and (ii) it can ``invisibly agree'' with a \textsc{nom}-subject. The \isi{infinitive} in other \ili{Slavic} languages is perhaps more restricted.

Taken all together, these insights reveal that ICs are no ``construction'', but arise through syntactic structure building and are interpreted pragmatically.

\section*{Abbreviations}
\begin{tabularx}{.58\textwidth}{lX}
	1,2,3 & first, second, third person\\
 	\textsc{acc} & {accusative case}\\
	\textsc{aux} & auxiliary\\
    \textsc{cond} & conditional\\
	\textsc{dat} & {dative} case\\
	\textsc{f} & feminine \\
	\textsc{gen} & {genitive case}\\
	{IC} & infinitival conditional \\
    \textsc{inf} & {infinitive}\\
	\textsc{ins} & instrumental case\\
	\textsc{loc} & locative case\\
\end{tabularx}
\begin{tabularx}{.38\textwidth}{lX}
	\textsc{lpt} & \textit{l}-{participle} \\
	\textsc{m} & masculine\\
	\textsc{n} & neuter\\
 	\textsc{neg} & {negation}\\
	\textsc{nom} & {nominative case}\\
    \textsc{pf} & {perfective} aspect\\
	\textsc{pl} & plural\\
    \textsc{pst} & past {tense}\\
    \textsc{refl} & {reflexive}\\
    \textsc{rel} & {relative} \\
    \textsc{sg} & singular \\
\end{tabularx}



\section*{Acknowledgements}

We would like to thank the audience of FDSL 12.5 for discussion of issues concerning conditionals in Slavic. We are grateful to Mojmír Dočekal, Julie Goncharov, Wojciech Guz, Kateřina Milotová, Roumyana Pancheva, Yakov Testelec, and Jacek Witkoś for raising questions that helped to clear up the picture and get a better understanding of the peculiarities of Czech infinitival conditionals. We also would like to express our gratitude to two anonymous reviewers for helpful questions and comments.

{\sloppy
\printbibliography[heading=subbibliography,notkeyword=this]
}

\il{Czech|)}
\end{document}
