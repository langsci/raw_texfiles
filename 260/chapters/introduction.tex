\addchap{Introduction}

Advances in Formal Slavic Linguistics 2017 is a collection of fifteen articles on formal Slavic linguistics. The articles cover a wide array of topics, such as control verbs, instrumental arguments, and perduratives in Russian, comparatives, negation, n-words, negative polarity items, and complementizer ellipsis in Czech, impersonal se-constructions and complementizer doubling in Slovenian, prosody and the morphology of multi-purpose suffixes in Serbo-Croatian, and indefinite numerals and the binding properties of dative arguments in Polish. By exploring these phenomena in individual Slavic languages, the collection of articles in this volume makes a significant contribution to both Slavic linguistics and to linguistics in general.

The articles in the volume were prepared on the basis of talks given at the conference Formal Description of Slavic Languages 12.5, which was held on December 7-9, 2017, at the University of Nova Gorica. (FDSL was initially a biannual conference, hosted in turn by the University of Leipzig and University of Potsdam since 1995, with the University of Göttingen joining the main conference cycle in 2011 and Humboldt University Berlin in 2016. In 2006 FDSL was partly turned into an annual event when the University of Nova Gorica hosted FDSL 6.5. Eleven years later, after successful stops in Moscow in 2008 and Brno in 2010 and 2014, the 12.5 conference thus brought FDSL back to Nova Gorica, the place of its first halftime edition.) The 18 talks and 10 poster presentations given at FDSL 12.5 were selected out of 47 submitted abstracts. Following the conference, XX articles were submitted for inclusion in this volume, and XX successfully completed the review process in which every article was evaluated and commented on by two reviewers.

This volume would not have been possible without our extremely helpful reviewers: Nadira Aljović, Svitlana Antonyuk, Boban Arsenijević, Loren A. Billings, Petr Biskup, Joanna Błaszczak, Anna Bondaruk, Mojmír Dočekal, Jakub Dotlačil, Berit Gehrke, Guillaume Enguehard, Julie Goncharov, Hana Gruet-Skrabalova, Peter Jurgec, Dorota Klimek-Jankowska, Iliyana Krapova, Jonathan E. MacDonald, Christina Manouilidou, Tatjana Marvin, Natasa Miličević, Moreno Mitrović, Andrew Murphy, Zorica Puškar-Gallien, Jana Reifegerste, Branimir Stanković, Adrian Stegovec, Radek Šimík, Aida Talić, Neda Todorović, Barbara Tomaszewics, Ana Werkmann Horvat, Jacek Witkoś, Sławomir Zdziebko, Sašo Živanović. Olga
