\documentclass[output=paper,colorlinks,citecolor=brown,newtxmath]{langsci/langscibook}
\ChapterDOI{10.5281/zenodo.3764843}

%\usepackage{langsci-optional}
\usepackage{langsci-gb4e}
\usepackage{langsci-lgr}

\usepackage{listings}
\lstset{basicstyle=\ttfamily,tabsize=2,breaklines=true}

%added by author
% \usepackage{tipa}
\usepackage{multirow}
\graphicspath{{figures/}}
\usepackage{langsci-branding}

%
\newcommand{\sent}{\enumsentence}
\newcommand{\sents}{\eenumsentence}
\let\citeasnoun\citet

\renewcommand{\lsCoverTitleFont}[1]{\sffamily\addfontfeatures{Scale=MatchUppercase}\fontsize{44pt}{16mm}\selectfont #1}
  
\IfFileExists{../localcommands.tex}{
  \usepackage{langsci-optional}
\usepackage{langsci-gb4e}
\usepackage{langsci-lgr}

\usepackage{listings}
\lstset{basicstyle=\ttfamily,tabsize=2,breaklines=true}

%added by author
% \usepackage{tipa}
\usepackage{multirow}
\graphicspath{{figures/}}
\usepackage{langsci-branding}

  
\newcommand{\sent}{\enumsentence}
\newcommand{\sents}{\eenumsentence}
\let\citeasnoun\citet

\renewcommand{\lsCoverTitleFont}[1]{\sffamily\addfontfeatures{Scale=MatchUppercase}\fontsize{44pt}{16mm}\selectfont #1}
  
  %% hyphenation points for line breaks
%% Normally, automatic hyphenation in LaTeX is very good
%% If a word is mis-hyphenated, add it to this file
%%
%% add information to TeX file before \begin{document} with:
%% %% hyphenation points for line breaks
%% Normally, automatic hyphenation in LaTeX is very good
%% If a word is mis-hyphenated, add it to this file
%%
%% add information to TeX file before \begin{document} with:
%% %% hyphenation points for line breaks
%% Normally, automatic hyphenation in LaTeX is very good
%% If a word is mis-hyphenated, add it to this file
%%
%% add information to TeX file before \begin{document} with:
%% \include{localhyphenation}
\hyphenation{
affri-ca-te
affri-ca-tes
an-no-tated
com-ple-ments
com-po-si-tio-na-li-ty
non-com-po-si-tio-na-li-ty
Gon-zá-lez
out-side
Ri-chárd
se-man-tics
STREU-SLE
Tie-de-mann
}
\hyphenation{
affri-ca-te
affri-ca-tes
an-no-tated
com-ple-ments
com-po-si-tio-na-li-ty
non-com-po-si-tio-na-li-ty
Gon-zá-lez
out-side
Ri-chárd
se-man-tics
STREU-SLE
Tie-de-mann
}
\hyphenation{
affri-ca-te
affri-ca-tes
an-no-tated
com-ple-ments
com-po-si-tio-na-li-ty
non-com-po-si-tio-na-li-ty
Gon-zá-lez
out-side
Ri-chárd
se-man-tics
STREU-SLE
Tie-de-mann
}
  \bibliography{../localbibliography}
  \togglepaper[1]%%chapternumber
}{}
%\togglepaper[1]

\title{Object control: Hidden modals}

\author{Irina Burukina\affiliation{Eötvös Loránd University}}

\abstract{The paper proposes a novel analysis for object control verbs in Russian. First, I argue that object control verbs are not a homogeneous class, despite the common opinion advocated by \cite{FranksHornstein1992, Babby1998, Boeckx2010}, a.o. In Russian, desiderative object control verbs with a dative argument (\textit{velet’} `order’,  \textit{razrešit’} `allow’, etc.) differ significantly in their syntactic and semantic properties from implicative object control verbs with an accusative argument (\textit{zastavit’} `force’, \textit{ubedit’} `persuade’, etc.). However, this distinction does not match existing classifications. Second, I develop a structural description for dative control verbs and argue that (i) the dative argument and the embedded clause form a single constituent that excludes the matrix verb, and (ii) this constituent is headed by a silent modal element. This analysis accounts for many semantic and syntactic properties of dative object control verb including the unavailability of split control with dative control verbs and their distributional similarity with modal predicatives (\textit{možno} `allowed’, \textit{nado}  `necessary’).

\keywords{object control, non-finite complement, covert modality, dative, Russian}
}

 \IfFileExists{../localcommands.tex}{
  \usepackage{langsci-optional}
\usepackage{langsci-gb4e}
\usepackage{langsci-lgr}

\usepackage{listings}
\lstset{basicstyle=\ttfamily,tabsize=2,breaklines=true}

%added by author
% \usepackage{tipa}
\usepackage{multirow}
\graphicspath{{figures/}}
\usepackage{langsci-branding}

  
\newcommand{\sent}{\enumsentence}
\newcommand{\sents}{\eenumsentence}
\let\citeasnoun\citet

\renewcommand{\lsCoverTitleFont}[1]{\sffamily\addfontfeatures{Scale=MatchUppercase}\fontsize{44pt}{16mm}\selectfont #1}
  
  %% hyphenation points for line breaks
%% Normally, automatic hyphenation in LaTeX is very good
%% If a word is mis-hyphenated, add it to this file
%%
%% add information to TeX file before \begin{document} with:
%% %% hyphenation points for line breaks
%% Normally, automatic hyphenation in LaTeX is very good
%% If a word is mis-hyphenated, add it to this file
%%
%% add information to TeX file before \begin{document} with:
%% %% hyphenation points for line breaks
%% Normally, automatic hyphenation in LaTeX is very good
%% If a word is mis-hyphenated, add it to this file
%%
%% add information to TeX file before \begin{document} with:
%% \include{localhyphenation}
\hyphenation{
affri-ca-te
affri-ca-tes
an-no-tated
com-ple-ments
com-po-si-tio-na-li-ty
non-com-po-si-tio-na-li-ty
Gon-zá-lez
out-side
Ri-chárd
se-man-tics
STREU-SLE
Tie-de-mann
}
\hyphenation{
affri-ca-te
affri-ca-tes
an-no-tated
com-ple-ments
com-po-si-tio-na-li-ty
non-com-po-si-tio-na-li-ty
Gon-zá-lez
out-side
Ri-chárd
se-man-tics
STREU-SLE
Tie-de-mann
}
\hyphenation{
affri-ca-te
affri-ca-tes
an-no-tated
com-ple-ments
com-po-si-tio-na-li-ty
non-com-po-si-tio-na-li-ty
Gon-zá-lez
out-side
Ri-chárd
se-man-tics
STREU-SLE
Tie-de-mann
}
  \togglepaper[1]%%chapternumber
}{}

\begin{document}
\maketitle
\il{Russian|(}
\section{Introduction}\label{s1}

\sloppy The present paper investigates \ili{Russian} \textsc{\isi{object control} verbs} (OCVs) that require either a {dative} or an {accusative argument}: \textit{velet’} `order’, \textit{prikazat’} `order’, \textit{zastavit’} `make’ etc.  During the past decades several major theories of control have been developed \citep{W2001, Boeckx2010, Landau2015}, however, Slavic languages have not been sufficiently approached \citep{FranksHornstein1992, Babby1998}. Existing approaches usually draw a line between subject and \isi{object control} predicates, and the latter are treated as a homogeneous class. The most significant attempt for further sub-categorization has been made by \cite{Landau2015}, and has resulted in the development of the two-tiered theory of control. However, when the \ili{Russian} data is analyzed, the two-tiered theory of control makes a wrong prediction: the suggested attitude vs. non-attitude distinction does not correspond to the real availability of partial control. \par

The classification proposed in this paper captures the correlation between syntactic and semantic properties of \ili{Russian} OCVs, partially inheriting \cite{JC2003} idea to sub-categorize verbs of control according to their meaning. I distinguish between \isi{desiderative} \isi{dative predicate}s \REF{ex1a}, on the one hand, and \isi{implicative} \isi{accusative predicate}s \REF{ex1b}, on the other.\footnote{I use the terms ``\isi{desiderative}'' and ``\isi{implicative}'' to refer to these particular verbs following \cite{W2001} and \cite{Landau2013}.} As will be demonstrated, the two groups differ significantly in their behavior.

\begin{exe}
\ex\label{ex1} \begin{xlist}
\ex\label{ex1a}
\gll Petja razrešil Maše$_i$ PRO$_i$ vzjat’ kuklu.\\
    Petja.\textsc{nom} allowed Maša.\textsc{dat} {} take.\textsc{inf} doll \\
\glt `Petja allowed Maša to take the doll.'
\ex\label{ex1b}
\gll Petja zastavil Mašu$_i$ PRO$_i$ vzjat’ kuklu.\\
    Petja.\textsc{nom} forced Maša.\textsc{acc} {} take.\textsc{inf} doll \\
\glt `Petja forced Maša to take the doll.'
    \end{xlist}
\end{exe}

\noindent The paper continues with a novel two-part analysis for \ili{Russian} \isi{dative OCV}s: (i) the {dative argument} and the embedded clause form a single constituent that excludes the main predicate head, and (ii) this constituent is headed by a silent modal-like element that takes a \isi{non-finite clause} as its complement \REF{ex2}.\footnote{I leave the question about the size of the embedded \isi{non-finite clause} for further research and I mark it as a CP, adopting the traditional \cite{Lasnik1998} approach to infinitives in Slavic languages.} Therefore, the core claim is that the \isi{modal} item is not merely a part of semantic decomposition, but that it is present in the syntactic structure, separately from the main predicate. I further suggest that, in \ili{Russian}, this silent \isi{modal head} belongs to the existing class of \isi{modal predicatives} (\textit{možno} `allowed’, \textit{nado} `necessary’, etc.).

\begin{exe}
\ex\label{ex2}
\gll [\textsubscript{\textit{v}P} Petja [\textsubscript{VP} [\textsubscript{V} razrešil][\textsubscript{ModP} [\hspace{-2pt} Maše$_i$] [\textsubscript{{Mod}$'$} \textit{modal} [\textsubscript{CP} PRO$_i$ vzjat’ kuklu]]]]]\\
{} Petja {} {} allow {} Maša {} {} {} {} take doll\\
\glt `Petja allowed Maša to take the doll.'
\end{exe}

\noindent The rest of the paper is structured as follows: \sectref{s2} examines general properties of \ili{Russian} \isi{dative OCV}s in comparison with \isi{accusative OCV}s and addresses the constituency issue; and \sectref{s3} presents the analysis of sentences with a matrix {desiderative} control predicate. Finally, \sectref{s4} discusses the peculiar unavailability of split control in the presence of a {desiderative} OCVs.

\section{Desiderative control predicates and their properties}\label{s2}

Desiderative OCVs in \ili{Russian} include the verbs \textit{razrešit’} `allow', \textit{pozvolit’} `allow', \textit{zapretit’} `prohibit', \textit{prikazat’} `order', \textit{velet’} `order', \textit{predpisat’} `obligate', \textit{poručit’} `charge', \textit{skazat’} `tell'. They cannot assign accusative case and require a dative DP. OCVs that select an accusative argument are implicatives \textit{zastavit’} `force', \textit{vynudit’} `force', \textit{ugovorit’} `persuade', \textit{ubedit’} `persuade’ \REF{ex3}.

\begin{exe}
\ex\label{ex3} \begin{xlist}
\ex\label{ex3a}
\gll Vrač zapretil Ivanu$_i$ PRO$_i$ jest’ konfety.\\
    doctor.\textsc{nom} forbade Ivan.\textsc{dat} {} eat.\textsc{inf} candies \\
\glt `The doctor forbade Ivan to eat candies.'
\ex\label{ex3b}
\gll Vrač ubedil Ivana$_i$ PRO$_i$ ne jest’ konfety.\\
    doctor.\textsc{nom} persuaded Ivan.\textsc{acc} {} not eat.\textsc{inf} candies \\
\glt `The doctor persuaded Ivan not to eat candies.'
    \end{xlist}
\end{exe}

\noindent As demonstrated in \REF{ex3}, both types of control predicates under consideration can embed a \isi{non-finite clause}. Aside from this, \isi{desiderative OCV}s often embed a finite \isi{subjunctive} clause; importantly, a {dative} DP is still present and its referent can differ from that of the embedded subject \REF{ex4}.\footnote{
Accusative OCVs only marginally allow embedded finite clauses; in such cases the accusative DP and the embedded subject should be co-referential \REF{ex0}. It is beyond the scope of this paper to examine these constructions in details, and I will only briefly return to this problem in \sectref{s3}.

\begin{exe}
\ex\label{ex0} \begin{xlist}
\ex[??]{\label{ex0a}
\gll Vrač zastavil Ivana$_i$, čtoby on$_i$ ne jel konfety.\\
    doctor.\textsc{nom} forced Ivan.\textsc{acc} {so that} he.\textsc{nom} not eat.\textsc{sbjv} candies\\
\glt `The doctor forced Ivan not to eat candies.'}
\ex[*]{\label{ex0b}
\gll Vrač zastavil medsestru, čtoby Ivan ne jel konfety.\\
    doctor.\textsc{nom} forced nurse.\textsc{acc} {so that} Ivan.\textsc{nom} not eat.\textsc{sbjv} candies \\
\glt Intended: `The doctor told the nurse that Ivan must not eat candies.'}
    \end{xlist}
\end{exe}
}

\begin{exe}
\ex\label{ex4} \begin{xlist}
\ex\label{ex4a}
\gll Vrač zapretil Ivanu$_i$, čtoby on$_i$ jel konfety.\\
    doctor.\textsc{nom} forbade Ivan.\textsc{dat} {so that} he.\textsc{nom} eat.\textsc{sbjv} candies \\
\glt `The doctor forbade Ivan to eat candies.'
\ex\label{ex4b}
\gll Vrač zapretil medsestre, čtoby Ivan jel konfety.\\
    doctor.\textsc{nom} forbade nurse.\textsc{dat} {so that} Ivan.\textsc{nom} eat.\textsc{sbjv} candies \\
\glt `The doctor told the nurse that Ivan should not eat candies.'
    \end{xlist}
\end{exe}

\noindent Constructions with implicative and \isi{desiderative OCV}s in \ili{Russian} differ when it comes to structural relations established between a DP argument and an embedded clause. In sentences with an \isi{implicative OCV} an accusative DP and an embedded clause together do not pass constituency tests; they cannot be separated from the rest of a clause, excluding the main predicate. This is exemplified in \REF{ex5}, where attempted pseudo-cleft and short answer constructions are ungrammatical.

\begin{exe}
\ex\label{ex5} \begin{xlist}
\ex[*]{\label{ex5a}
\gll Čto ja zastavil, tak eto Petju pojti v kino.\\
    what I forced so that Petja.\textsc{acc} go.\textsc{dat} into cinema\\
\glt Intended `What I forced is that Petja would go to the cinema.'}
\ex[*]{\label{ex5b}
\gll Čto ty zastavil? Petju pomyt’ posudu.\\
    what you forced Petja.\textsc{acc} wash.\textsc{inf} dishes\\
\glt Intended: `What did you force him to do? I forced Petja to wash the dishes.'}
    \end{xlist}
\end{exe}

\noindent In contrast, a dative DP and an embedded clause apparently form a single constituent that excludes the main predicate in sentences with a \isi{desiderative OCV}; see the examples in \REF{ex6}, which are judged as acceptable by native speakers.

\begin{exe}
\ex\label{ex6} \begin{xlist}
\ex\label{ex6a}
\gll Čto ja razrešil, tak eto Pete pojti v kino.\\
    what I allowed so that Petja.\textsc{dat} go.\textsc{inf} into cinema\\
\glt `What I allowed is that Petja would go to the cinema.'
\ex\label{ex6b}
\gll Čto ty razrešil? Pete pojti v kino.\\
    what you allowed Petja.\textsc{dat} go.\textsc{inf} into cinema\\
\glt `What did you allow? I allowed Petja to go to the cinema.'
    \end{xlist}
\end{exe}

\noindent It is possible to suggest that the sentences in \REF{ex5} are ungrammatical because of the \isi{case assignment} problems: as a structural case, accusative is licensed by a \isi{functional head} that must be structurally present. Nevertheless, this does not directly affect the results of the constituency tests for sentences with \isi{desiderative} predicates, as we would not expected two unrelated constituents to be clefted or questioned. \par

Furthermore, the boundaries of the immediate constituent that includes the dative DP and the embedded clause and, apparently, does not contain the {matrix predicate}, become visible in multiple wh-questions and in case of \isi{quantifier stranding}. As a result of successive cyclic A-bar movement, an interrogative {pronoun} or a {quantifier} from a moved phrase can stay in an intermediate position; the position at the edge of an embedded clause is expected; however, there is another one, to the left of the {dative} DP. All possible positions where an interrogative {pronoun} or a {quantifier} can be realized are shown in \REF{ex7}.

\begin{exe}
\ex\label{ex7} \begin{xlist}
\ex\label{ex7a}
\gll Kto (\hspace{-2pt} kuda$_i$) razrešil 	[(\hspace{-2pt} kuda$_i$) emu [(\hspace{-2pt} kuda$_i$) pojti t$_i$]]?\\
    who {} where allowed {} where he.\textsc{dat} {} where go.\textsc{inf}\\
\glt `Who allowed him to go where?'
\ex\label{ex7b}
\gll (\hspace{-2pt} Vse) knigi$_i$ ja razrešil [(\hspace{-2pt} vse) Maše [(\hspace{-2pt} vse) pročitat’ t$_i$]].\\
    {} all books.\textsc{acc} I allowed {} all Maša.\textsc{dat} {} all read.\textsc{inf}\\
\glt `As for the books, I allowed Maša to read all of them.'
    \end{xlist}
\end{exe}

\noindent A possible way to account for the examples in \REF{ex7} is in terms of scrambling across the border of an embedded clause. However, as demonstrated by \cite{Bailyn2003}, a.o., the long-distance scrambling in \ili{Russian} is normally limited to a movement of a constituent from the embedded clause into the right \isi{focus} position of the matrix clause, and embedded constituents cannot move freely. Furthermore, the ungrammaticality of \REF{ex8a} suggests that it is also usually unacceptable to put a matrix constituent between the main predicate and the dative DP. Finally, the contrast in speakers’ judgments for sentences with a \isi{desiderative OCV} \REF{ex7} and an \isi{implicative OCV} \REF{ex9} provides additional support for the idea about the presence of a syntactic boundary.\footnote{24 out of 32 native speakers reported \REF{ex8a} to be ungrammatical; 6 native speakers said that it is `weird’.}

\ea\label{ex8}
\ea[*]{\label{ex8a}
\gll Kto kuda razrešil včera emu pojti?\\
    who where allowed yesterday he.\textsc{dat} go.\textsc{inf}\\
\glt Intended: `Who allowed him yesterday to go where?'}
\ex[]{\label{ex8b}
\gll Kto kuda včera razrešil emu pojti?\\
    who where yesterday allowed he.\textsc{dat} go.\textsc{inf}\\
\glt `Who allowed him yesterday to go where?'}
\z
\z

\ea\label{ex9}
\ea[???]{\label{ex9a}
\gll Kto zastavil kuda ego pojti?\\
    who forced where he.\textsc{acc} go.\textsc{inf}\\
\glt Intended: `Who forced him to go where?'}
\ex[*]{\label{ex9b}
\gll Knigi$_i$ ja zastavil [(\hspace{-2pt} vse) Mašu [\hspace{-2pt} pročitat’ t$_i$]].\\
    books.\textsc{acc} I forced {} all Maša.\textsc{acc} {} read.\textsc{inf}\\
\glt Intended: `As for the books, I forced Maša to read all of them.'}
\z
\z

\noindent In addition to the syntactic differences described above, desiderative and \isi{implicative OCV}s can be distinguished with regard to one additional property: availability of partial control for PRO. \isi{Desiderative OCV}s require exhaustive control, while \isi{implicative OCV}s allow partial \isi{coreference} between an embedded PRO and its matrix controller.\footnote{In case of partial control referent of the controller still must be included among referents of PRO.} Evidence for this comes from constructions with a semantically single dative or accusative controller and various embedded elements that require a semantically plural subject. For this purpose, first, collective predicates can be used; in \ili{Russian} many of those are derived using the pattern \textit{raz + sja}: \textit{razrugat’sja} `break up, quarrel’, \textit{razojtis’} `disperse’, etc. \REF{ex10}.\footnote{The same \textit{raz + sja} pattern is used to derive non-collective predicates with different meanings and (often) selection properties. For example, \textit{Petja razrugalsja} can be considered grammatical if the verb is interpreted as the homonymous one meaning `begin to swear at someone angrily’.}


\ea\label{ex10}
\ea[]{\label{ex10a}
\gll Mal’čiki razrugalis’.\\
    boys.\textsc{nom}.\textsc{pl} {broke up}\\
\glt `The boys broke up.'}
\ex[]{\label{ex10b}
\gll Komanda razrugalas’.\\
    team.\textsc{nom}.\textsc{sg} {broke up}\\
\glt `The team broke up.'}
\ex[*]{\label{ex10c}
\gll Petja razrugalsja.\\
    Petja.\textsc{nom} {broke up}\\
\glt Intended: `Petja broke up (with someone).'}
\z
\z

\noindent Placed in an embedded \isi{non-finite clause} with a single matrix controller, these verbs are allowed only if the matrix predicate is implicative \REF{ex11a}, and they are prohibited in sentences with a \isi{desiderative OCV} \REF{ex11b}.

\ea\label{ex11}
\ea[]{\label{ex11a}
\gll Ivan ubedil’ direktora razojtis’ v sem’.\\
    Ivan.\textsc{nom} persuaded director.\textsc{acc} disperse.\textsc{inf} at seven \\
\glt `Ivan persuaded the director that they should disperse at seven.'}
\ex[*]{\label{ex11b}
\gll Direktor razrešil Ivanu razojtis’ v sem’.\\
    director.\textsc{nom} allowed Ivan.\textsc{dat} disperse.\textsc{inf} at seven\\
\glt Intended: `The director allowed Ivan to disperse at seven.'}
\z
\z

\noindent Collective modifiers, for example, \textit{vmeste} `together’, behave the same way. When subject-oriented, \textit{vmeste} requires semantically plural PRO; in case of a semantically singular controller, \textit{vmeste} is permitted in constructions with a matrix \isi{implicative} verb \REF{ex12a}, but not a matrix \isi{desiderative} predicate \REF{ex12b}.

\ea\label{ex12}
\ea[]{\label{ex12a}
\gll Direktor zastavil Ivana poobedat’ vmeste.\\
    director.\textsc{nom} forced Ivan.\textsc{acc} {have lunch.\textsc{inf}} together\\
\glt `The director forced Ivan to have lunch together with him.'}
\ex[*]{\label{ex12b}
\gll Direktor velel Ivanu poobedat’ vmeste.\\
    director.\textsc{nom} ordered Ivan.\textsc{dat} {have lunch.\textsc{inf}} together\\
\glt Intended: `The director ordered Ivan to have lunch together with him.'}
\z
\z

\noindent Taking into account all the differences in the behavior of implicative and \isi{desiderative OCV}s in \ili{Russian}, I propose that the two groups require separate structural representations. In this paper I \isi{focus} on \isi{desiderative} dative OCVs and I proceed by suggesting a novel way to analyze these predicates. The core idea is that, aside from the main verb, an additional silent \isi{deontic} \isi{modal head} should be structurally introduced to connect a dative DP and an embedded clause.\footnote{At least two options might be suggested for a structural representation of \isi{implicative OCV}s: (i) an accusative DP and an embedded clause are both internal arguments of the main verb, located in SpecVP and CompVP, respectively (\citealt{Babby1998, bailyn2012} on \ili{Russian}), or (ii) an accusative DP and an embedded clause together form a small clause in the complement position of the matrix `causative-like’ predicate (\citealt{FranksHornstein1992, Landau2015}, a.o.). I am unable to provide a detailed comparison of these two approaches within the limits of this paper and I leave this problem for future investigations.}

\section{Proposed analysis}\label{s3}

The two prominent current frameworks that address the problem of detailed sub-categorization of control verbs, namely \citeposst{W2001} theory of restructuring configurations and \citeposst{Landau2015} two-tiered theory of control, cannot fully account for the \ili{Russian} data. According to Wurmbrand’s classification based on the structural properties of embedded clauses selected by various control predicates, \ili{Russian} \isi{implicative} and \isi{desiderative OCV}s fall into one category of restructuring irrealis predicates; the differences between the two types remain unexplained further. \par

\cite{Landau2015} distinguishes between attitude and non-attitude predicates, selecting attitude and non-attitude complements. The former refer to the world of the main actor’s beliefs; the later receives an interpretation with regard to the real world context. This semantic difference yields different syntactic structures, with additional functional projections above an embedded clause required by attitude predicates. Importantly, the theory predicts that attitude predicates (desideratives, propositionals) must support partial control, while non-attitude predicates (implicatives, modals) must require strict coreference between PRO and its controller. As was demonstrated in \sectref{s2}, as for the \ili{Russian} data, this prediction is not borne out: attitude \isi{desiderative} dative OCVs allow only exhaustive control, while non-attitude \isi{implicative} accusative OCVs do not prohibit partial coreference. \par

Thus, another way should be found to represent the structure of \isi{desiderative OCV}s. I propose that sentences with these predicates contain a hidden component that is responsible for their peculiar properties – a lexical \isi{deontic modal} head that, in \ili{Russian}, belongs to the class of the so-called \isi{modal} predicatives. \par

The results of the constituency tests provided in \sectref{s2} suggest that the dative controller and the embedded clause form a single constituent that excludes the matrix predicate. The question remains about the nature of this constituent; it might be suggested that the two form a small clause or there can be another lexical head that selects a dative DP and a clause as its arguments. I argue that the second option is more plausible and that this lexical head is a silent \isi{modal}. \par

In case of \isi{desiderative} predicates the embedded \isi{non-finite clause} is a fully saturated proposition; for example, it can be individually substituted by a proform \REF{ex13a} or an interrogative pronoun \REF{ex13b}.

\ea\label{ex13}
\ea\label{ex13a}
\gll Ja uže velel Pete eto.\\
    I already ordered Petja.\textsc{dat} this\\
\glt `I have already ordered Petja to do this.'
\ex\label{ex13b}
\gll Čto ty velel Pete?\\
    what you ordered Petja.\textsc{dat}\\
\glt `What did you order Petja to do?'
\z
\z

\noindent It is unlikely that a saturated embedded clause itself functions as a predicate of the \isi{dative argument}. This is further supported by the availability of an embedded finite clause; as was illustrated in \REF{ex4}, if a finite \isi{subjunctive} clause is selected, a dative DP argument is still available. Importantly, the latter does not have to be coreferent with the embedded subject \REF{ex4b}, which rules out possible copy-raising analyses. No semantic or syntactic difference can be found between a dative DP present together with an embedded infinitival construction and an argument selected simultaneously with a finite clause. Therefore, I assume that there is no reason to believe that the two are related to different predicates. \par

I propose that a dative DP and an embedded clause (either finite or non-finite) are selected together by a silent lexical \isi{modal head}; this \isi{modal} phrase is later merged as a complement of a \isi{desiderative OCV}. The structure is schematized in \REF{ex14}. In other words, I argue that \isi{deontic} modality, intuitively perceived in \isi{desiderative} predicates, is represented structurally. At least two properties of \isi{desiderative OCV} constructions support this idea.

\ea
\label{ex14}
{[\textsubscript{VP} [\textsubscript{V} desiderative][\textsubscript{ModP} [\hspace{-2pt} dative DP][\textsubscript{Mod$'$} modal [\textsubscript{CP} infinitival clause]]]]}
\z

\noindent Firstly, there is the possibility of ambiguous interpretation of examples with \isi{sentential negation}. Consider the example in \REF{ex15}, for which two readings \REF{ex15a} and \REF{ex15b} are available, while interpretation \REF{ex15c} is prohibited. However, simply assuming that \textit{razrešit’} `permit’ allows Neg-raising, we would expect \isi{negation} to scope either above the matrix verb (reading \REF{ex15a}) or above an embedded clause (reading \REF{ex15c}).

\ea\label{ex15}
\gll Petja	 ne	razrešal	Maše	ostat’sja.\\
    Petja.\textsc{nom} not allowed Maša.\textsc{dat} stay.\textsc{inf}\\
\glt `Petja didn't allow Maša to stay.'

\ea\label{ex15a}
`Petja said that for Maša it is not possible to stay.'
\ex\label{ex15b}
`Petja didn’t say that for Maša it is possible to stay.'
\ex\label{ex15c}
Not available: `Petja said that for Maša it is possible not to stay.'
\z
\z

\noindent Furthermore, according to \cite{vonFintelIatridou2007} and \cite{IatridouZeijlstra2013}, a.o., predicates denoting permission typically do not support neg-raising; see, for example, \ili{English} \isi{modal} verbs and \ili{Russian} \isi{modal} predicatives \REF{ex16}. Although this generalization is originally formulated for \isi{deontic modal}s, deontic modality is a part of constructions with \isi{desiderative OCV}s like order and permit, and an additional explanation is required for \REF{ex15} if we assume that this modality is encoded in \textit{razrešit’}  itself.

\ea\label{ex16}
\ea\label{ex16a}
Ivan cannot stay. \\
= Ivan must leave. ≠ Ivan may leave.
\ex\label{ex16b}
\gll Pete nel’zja ostavat’sja.\\
    Petja.\textsc{dat} {not allowed.\textsc{sg.n}} stay.\textsc{inf}\\
\glt `Petja is not allowed to stay here.' \\
	= Petja must leave. ≠ Petja may leave
\z
\z

\noindent Introducing a separate \isi{deontic} head, as shown in \REF{ex14}, splits the structure into two parts: the higher `communication’ component and the lower `permission’ constituent. In \REF{ex15} \isi{negation} can scope above either one of them yielding the interpretations \REF{ex15a} and \REF{ex15b}; however, the may-type \isi{modal} prohibit neg-raising and the interpretation \REF{ex15c} becomes impossible. \par

Second, almost all predicates of communication in \ili{Russian} have desiderative `counterparts’. As demonstrated in \REF{ex17a}, \ili{Russian} verbs of communication, similar to \ili{English} say, write, whisper, normally embed a finite indicative clause with the \textit{čto} complementizer. However, they can also co-occur with embedded non-finite or finite \isi{subjunctive} clauses, and such sentences receive a \isi{desiderative} (\isi{modal}) interpretation \REF{ex17b}--\REF{ex17c}. It is true that the contrast between \REF{ex17a} and \REF{ex17b}--\REF{ex17c} could, in principle, be accounted for by postulating two morphologically identical lexical entries for each of the verbs of information transfer. However, there is another possible explanation in line with the hidden \isi{modal} approach presented in this paper: verbs of communication always select a saturated proposition, that can be either a finite indicative CP or a saturated \isi{modal} phrase with two arguments \REF{ex18}. Although the behavior of verbs of communication alone does not prove that the proposed analysis is the correct one, taking into account the Neg-raising facts reported above, being able to capture both of these properties of sentences with a \isi{desiderative OCV} is an important advantage of the hidden \isi{modal} approach.


\ea\label{ex17}
\ea\label{ex17a}
\gll Petja \minsp{\{} skazal / napisal / šepnul\} Maše, čto Ivan pomyl posudu.\\
    Petja.\textsc{nom} {} said {} wrote {} whispered Maša.\textsc{dat} that Ivan.\textsc{nom} washed dishes\\
\glt `Petja \{said/wrote/whispered\} Maša that Ivan had washed the dishes.'
\ex\label{ex17b}
\gll Petja \minsp{\{} skazal / napisal / šepnul\} Maše pomyt’ posudu.\\
    Petja.\textsc{nom} {} said {} wrote {} whispered Maša.\textsc{dat} wash.\textsc{inf} dishes\\
\glt `Petja \{said/wrote/whispered\} that Maša should wash the dishes.'
\ex\label{ex17c}
\gll Petja \minsp{\{} skazal / napisal / šepnul\} Maše, čtoby ona pomyla posudu.\\
    Petja.\textsc{nom} {} said {} wrote {} whispered Maša.\textsc{dat} {so that} she.\textsc{nom} wash.\textsc{sbjv} dishes\\
\glt `Petja \{said/wrote/whispered\} that Maša should wash the dishes.'
\z
\z

\ea\label{ex18}
\gll Petja	\minsp{\{} skazal / napisal / šepnul\} [\hspace{-2pt} Maše Ø pomyt’ posudu].\\
    Petja.\textsc{nom} {} said {} wrote {} whispered {} Maša.\textsc{dat} necessary wash.\textsc{inf} dishes\\
\glt `Petja \{said/wrote/whispered\} that Maša should wash the dishes.' \\
= `Petja \{said/wrote/whispered\} that for Maša it is necessary to wash the dishes.'
\z

\noindent In \REF{ex19} the proposed structure is repeated; at this point I denote the complex \isi{modal} constituent as ModP and leave the exact size of it for future investigation.

\ea\label{ex19}
{[\textsubscript{VP} [\textsubscript{V} desiderative][\textsubscript{ModP} [\hspace{-2pt} dative DP][\textsubscript{Mod$'$} modal [\textsubscript{CP} infinitival clause]]]]}
\z

\noindent Furthermore, I argue that the embedded silent head belongs to the existing class of \isi{deontic modal}s. In \ili{Russian}, in addition to modal verbs, there is also a group of the so called \isi{modal} predicatives (\textit{nado} `necessary’, \textit{možno} `allowed’). Modal predicatives prohibit a nominative subject and require a dative DP argument \REF{ex20}; morphologically, they are invariant and usually end with a neutral singular ending -\textit{o}.

\ea\label{ex20}
\gll Ivanu možno ne rabotat’ segodnja.\\
    Ivan.\textsc{dat} allowed.\textsc{sg.n} not work.\textsc{inf} today\\
\glt `Ivan is allowed not to work today.'
\z

\noindent Similar to \isi{desiderative OCV}s, \isi{modal} predicatives embed a non-finite or a finite \isi{subjunctive} clause \REF{ex21}.

\ea\label{ex21}
\ea\label{ex21a}
\gll Petja velel Maše$_i$ \minsp{\{} ostat’sja / čtoby ona$_i$ ostalas’\}.\\
    Petja.\textsc{nom} ordered Maša.\textsc{dat} {} stay.\textsc{inf} {} {so that} she.\textsc{nom} stay.\textsc{sbjv}\\
\glt `Petja ordered Maša to stay.'
\ex\label{ex21b}
\gll Maše$_i$ nužno \minsp{\{} ostat’sja / čtoby ona$_i$ ostalas’\}.\\
    Maša.\textsc{dat} necessary.\textsc{sg.n} {} stay.\textsc{inf} {} {so that} she.\textsc{nom} stay.\textsc{sbjv}\\
\glt `Maša should stay.'
\z
\z

\noindent However, in contrast with sentences with \isi{desiderative OCV}s, in constructions with an overt \isi{modal} predicatives a dative DP and an embedded clause do not form a single constituent: the two cannot be separated together in clefts \REF{ex22} and short answers \REF{ex23}.

\ea[*]{ \label{ex22}
\gll Čto možno, tak eto Pete posmotret’ multiki.\\
    what allowed.\textsc{sg.n} so that Petja.\textsc{dat} watch.\textsc{inf} cartoons\\
\glt Intended: `What is allowed is for Petja to watch cartoons.'}
\z

\ea \label{ex23}
\gll Čto možno? *\hspace{-2pt} Pete posmotret’ multiki.\\
    what allowed.\textsc{sg.n} {} Petja.\textsc{dat} watch.\textsc{inf} cartoons\\
\glt Intended: `What is allowed? It is allowed for Petja to watch cartoons.'
\z

\noindent For \isi{deontic modal} predicatives, I propose the following structural representation \REF{ex24}. Importantly, as was already said about silent modals in \isi{desiderative} constructions, I argue that \isi{modal} predicatives are lexical heads, not functional elements. At first sight, this idea contradicts existing analyses of modals (\citealt{cinque1999,W1999}, a.o.); however the latter usually consider only \isi{modal} agreeing verbs, whereas the present paper discusses a different class of modal elements. Predicatives in \ili{Russian} select a wide variety of constructions as a complement; crucially, they select embedded finite \isi{subjunctive} clauses \REF{ex25}, which is a reflex of the lexical-semantic properties of the modal head.

\begin{exe}
\ex\label{ex24}
{[\textsubscript{ModP} DP [\textsubscript{Mod$’$} modal [\hspace{-2pt} embedded proposition]]]}
\end{exe}

\ea\label{ex25}
\gll Tebe neobxodimo, čtoby ty vypolnil eto zadanie.\\
    you.\textsc{dat} necessary.\textsc{sg.n} {so that} you.\textsc{nom} do.\textsc{sbjv} this task\\
\glt `It is necessary for you to do this task.'
\z

\noindent Going back to sentences with a \isi{desiderative OCV}, the final structural representation and an example are given in \REF{ex26}.

\ea\label{ex26}
\gll [\textsubscript{\textit{v}P} Petja [\textsubscript{VP} [\textsubscript{V} skazal][\textsubscript{ModP} [\hspace{-2pt} Maše$_i$ ][\textsubscript{Mod$'$} modal [\textsubscript{CP} PRO$_i$ vzjat’ kuklu ]]]]]\\
    {}  Petja.\textsc{nom}   {}  {}  said {} Mary.\textsc{dat}  {} {}  {}  {}  take.\textsc{inf}    doll.\textsc{acc} {}\\
\glt `Petja told Maša that for her it is necessary to take the doll.'
\z

\section{Split control}\label{s4}

The proposed analysis for \isi{desiderative OCV}s provides a straightforward explanation for the unavailability of split and partial coreference (\REF{ex27} reproduced from \REF{ex11} and \REF{ex12}).

\ea \label{ex27}
\ea[*]{\label{ex27a}
\gll Direktor razrešil Ivanu razojtis’ v sem’.\\
    director.\textsc{nom} allowed Ivan.\textsc{dat} disperse.\textsc{inf} at seven \\
\glt Intended: `The director allowed Ivan to disperse at seven.'}
\ex[*]{\label{ex27b}
\gll Direktor velel Ivanu poobedat’ vmeste.\\
    director.\textsc{nom} ordered Ivan.\textsc{dat} {have lunch.\textsc{inf}} together\\
\glt Intended: `The director ordered Ivan to have lunch together with him.'}
\z
\z

\noindent As illustrated in \REF{ex26}, the main predicate (interpreted as \isi{desiderative}) selects a propositional modal-headed constituent. Within this phrase the control relation is established strictly between the dative argument and the embedded PRO, adherent to the Minimal Distance Principle.

\section{Conclusion}\label{s5}

In this paper I have used the \ili{Russian} data to demonstrate that OCVs are not a homogeneous class, and that they can be sub-categorized based on their semantic and syntactic properties. Not rejecting \citeposst{Landau2015} attitude vs. non-attitude predicates dichotomy, I propose to distinguish between \isi{implicative} predicates, which require an accusative argument, and \isi{desiderative} predicates, which cannot assign accusative case and require a dative controller.

The developed structural representation for \isi{desiderative} dative OCVs are two-fold: (i) the dative argument and the embedded clause are united into a single constituent that excludes the matrix verb, and (ii) this constituent is headed by a silent \isi{deontic modal}. I suggest that the central idea of the proposed analysis – syntactic decomposition of \isi{desiderative} predicates into a verb of communication and a silent \isi{modal head} – can be further applied to other languages.

\section*{Abbreviations}
\begin{tabularx}{.45\textwidth}{lX}
\textsc{acc}&accusative\\
\textsc{dat}&dative\\
\textsc{inf}&infinitive\\
\textsc{n}&neuter\\
\textsc{nom}&nominative\\
\end{tabularx}
\begin{tabularx}{.45\textwidth}{lX}
OCV&object control verbs\\
\textsc{pl}&plural\\
\textsc{sbjv}&subjunctive\\
\textsc{sg}&singular\\
&\\
\end{tabularx}

\section*{Acknowledgements}

I  am  especially  grateful  to  Marcel  den  Dikken  for  discussion  and corrections. Many thanks to Maria Polinsky, Krisztina Szécsényi, and anonymous reviewers for commenting on earlier versions of the paper. I  also  want  to  thank  the  speakers  of  Russian  for  their  help  and judgments. All errors are my own.

\sloppy
\printbibliography[heading=subbibliography,notkeyword=this]

\il{Russian|)}
\end{document}
