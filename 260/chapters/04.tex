\documentclass[output=paper,colorlinks,citecolor=brown,newtxmath]{langsci/langscibook}
\ChapterDOI{10.5281/zenodo.3764849}
%\bibliography{localbibliography}
%\usepackage{langsci-optional}
\usepackage{langsci-gb4e}
\usepackage{langsci-lgr}

\usepackage{listings}
\lstset{basicstyle=\ttfamily,tabsize=2,breaklines=true}

%added by author
% \usepackage{tipa}
\usepackage{multirow}
\graphicspath{{figures/}}
\usepackage{langsci-branding}

%
\newcommand{\sent}{\enumsentence}
\newcommand{\sents}{\eenumsentence}
\let\citeasnoun\citet

\renewcommand{\lsCoverTitleFont}[1]{\sffamily\addfontfeatures{Scale=MatchUppercase}\fontsize{44pt}{16mm}\selectfont #1}
  



\title{Epistemic comparatives and other expressions of speaker's uncertainty}

\author{Julie Goncharov\affiliation{Hebrew University of Jerusalem / University of Göttingen}\orcid{}\lastand Monica Alexandrina Irimia\affiliation{University of Modena and Reggio Emilia}\orcid{0000-0003-3733-8163}}


\abstract{Our study endeavors to further our understanding of the ways the speaker’s perspective is expressed in natural language. We analyze a rarely discussed construction, namely epistemic comparatives and their interaction with inferential indirect evidentials and epistemic modals. We show that epistemic comparatives are incompatible with inferential indirect evidentials, but are well-formed with epistemic modals. We base our discussion on data from Bulgarian and we also show that similar facts hold in Romanian, thus strengthening the empirical coverage. On the theoretical side, we claim that inferential indirect evidentials are structurally distinguished from epistemic modals. This accounts for their different behavior with epistemic comparatives, thus providing further support to accounts which take indirect evidentials and epistemic modals to be separate categories.

\keywords{epistemic comparatives, indirect evidentials, epistemic modals, inferentials, Bulgarian, Romanian}
}

\IfFileExists{../localcommands.tex}{
  \usepackage{langsci-optional}
\usepackage{langsci-gb4e}
\usepackage{langsci-lgr}

\usepackage{listings}
\lstset{basicstyle=\ttfamily,tabsize=2,breaklines=true}

%added by author
% \usepackage{tipa}
\usepackage{multirow}
\graphicspath{{figures/}}
\usepackage{langsci-branding}

  
\newcommand{\sent}{\enumsentence}
\newcommand{\sents}{\eenumsentence}
\let\citeasnoun\citet

\renewcommand{\lsCoverTitleFont}[1]{\sffamily\addfontfeatures{Scale=MatchUppercase}\fontsize{44pt}{16mm}\selectfont #1}
  
  %% hyphenation points for line breaks
%% Normally, automatic hyphenation in LaTeX is very good
%% If a word is mis-hyphenated, add it to this file
%%
%% add information to TeX file before \begin{document} with:
%% %% hyphenation points for line breaks
%% Normally, automatic hyphenation in LaTeX is very good
%% If a word is mis-hyphenated, add it to this file
%%
%% add information to TeX file before \begin{document} with:
%% %% hyphenation points for line breaks
%% Normally, automatic hyphenation in LaTeX is very good
%% If a word is mis-hyphenated, add it to this file
%%
%% add information to TeX file before \begin{document} with:
%% \include{localhyphenation}
\hyphenation{
affri-ca-te
affri-ca-tes
an-no-tated
com-ple-ments
com-po-si-tio-na-li-ty
non-com-po-si-tio-na-li-ty
Gon-zá-lez
out-side
Ri-chárd
se-man-tics
STREU-SLE
Tie-de-mann
}
\hyphenation{
affri-ca-te
affri-ca-tes
an-no-tated
com-ple-ments
com-po-si-tio-na-li-ty
non-com-po-si-tio-na-li-ty
Gon-zá-lez
out-side
Ri-chárd
se-man-tics
STREU-SLE
Tie-de-mann
}
\hyphenation{
affri-ca-te
affri-ca-tes
an-no-tated
com-ple-ments
com-po-si-tio-na-li-ty
non-com-po-si-tio-na-li-ty
Gon-zá-lez
out-side
Ri-chárd
se-man-tics
STREU-SLE
Tie-de-mann
}
  \togglepaper[4]%%chapternumber
}{}
%\togglepaper[4]

\begin{document}
\maketitle%
\section{Introduction}\label{sect:intro}

\il{Bulgarian|(}
Recent research on \isi{epistemic modality} has brought to the fore a previously neglected construction, namely \textsc{\isi{epistemic} comparatives} (hereafter ECs; \citealt{finkra14,herrub14}). We illustrate this phenomenon with the example in \REF{BulgEC} from \ili{Bulgarian}. In this sentence, the speaker conveys that she believes the state of affairs where Ivan is in the office to be more plausible than the state of affairs where Ivan is at home.

	\ea \gll Ivan	po-skoro	e 	v	ofisa	otkolkoto vkušti.\\
	Ivan more-soon	is	in	office	than home \\
      \glt `According to the speaker, it is more plausible that Ivan is at work than at home.'	\hfill (\ili{Bulgarian})	  \label{BulgEC}
        \z

\noindent We contribute to the rising interest in ECs by examining some previously unreported properties of these constructions. Starting from \ili{Bulgarian}, we address an interesting puzzle raised by the interaction of ECs with other expressions of speaker's uncertainty. This puzzle  concerns the difference between \isi{epistemic} modals and grammaticalized evidentials. ECs are not possible with inferential grammaticalized evidentials, while they are well-formed with \isi{epistemic} modals. The sentence in \REF{BulgECEvid} contains a \isi{present perfect}, which in \ili{Bulgarian} obtains an \textsc{indirect evidential} (\isi{IE}) interpretation. The curious observation is that the \isi{EC} is not well-formed under the inferential reading of the indirect \isi{evidential}.


	\ea[??]{ \label{BulgECEvid}\gll Ivan bil po-skoro v offisa	otkolkoto	vkušti. \\
		Ivan be.\textsc{pst.ptcp} more-soon in office	than home \\
		\glt Intended: `Apparently, it is more plausible that Ivan is in the office than at home.' \hfill (\ili{Bulgarian})}
 	\z

\noindent However, ECs seem to be possible with \isi{epistemic} modals, as seen below:

	\ea \gll Ivan po-skoro moje da e v ofisa otkolkoto vkušti.\\
	Ivan	more-soon can \textsc{da} is in	office than home \\
        \glt `It is more plausible that Ivan might be at work than at home.' \hfill (\ili{Bulgarian})
        \z

\noindent These facts are not restricted to \ili{Bulgarian}. We show that similar patterns obtain in yet another language that has ECs and grammaticalized evidentiality, namely \ili{Romanian}. Parallel observations from outside the Slavic family make available a clearer perspective into the nature of ECs and evidentiality, highlighting the cross-linguistic uniformity of these phenomena. We provide a comprehensive analysis for both \ili{Bulgarian} and \ili{Romanian}.

More precisely, we propose to explain the puzzle by arguing that the ill-formed\-ness of ECs with grammaticalized evidentials boils down to competition for the same position. We build on a decompositional account of ECs \citep{herrub14} and an analysis of the \isi{Assert} operator as a degree relation \citep{grewol17}. We show that our proposal makes a number of correct predictions, including  the difference between inferential and \isi{hearsay} evidentials when combined with ECs.

The paper is organized as follows: \sectref{background} provides some background on ECs and evidentiality, \sectref{ECsIIEV} presents the core observation, \sectref{proposal} spells out our analysis of the facts described in  \sectref{ECsIIEV}, \sectref{ConcludRem} contains some concluding remarks and avenues for future research.


\section{Background on epistemic comparatives}\label{background}
ECs compare two propositions with respect to the probabilities assigned (usually) by the speaker.\footnote{Propositional comparatives can, in fact, be classified into three types: (i) metalinguistic comparatives, expressing appropriateness (\citealt{bre73,mcc88,emb07,mor11}; a.o.), (ii) preference comparatives, ranking choices \citep{giasta09,giayoo11}, and (iii) \isi{epistemic} comparatives, ordering speaker's probabilities \citep{finkra14,herrub14}. There is no consensus in the literature on whether these three types of comparatives should be treated uniformly or not (see \citealt{mor11} for discussion). We stay away from this debate here, as we \isi{focus} on ECs.

	\ea \ea She is more fit than slender. \hfill (metalinguistic)
	\ex I would rather die than marry him. \hfill (preference)
    \z \zlast
}\textsuperscript{,}\footnote{Relativization to the speaker is true only of unembedded ECs. When ECs are embedded under an attitude predicate, they are relativized to the attitude holder and when ECs occur in a question, they are relativized to the hearer, see \cite[562]{herrub14}.\label{fn:relativ}
}
As we already mentioned, in the \ili{Bulgarian} example in \REF{BulgEC}, the speaker communicates that she believes (or is committed to act as if she believes) the state of affairs where Ivan is in the office to be more plausible than the state of affairs where Ivan is at home.

One of the defining properties of ECs is that they employ a \isi{comparative} form of a temporal adverb. A fuller definition of ECs should be in terms of their meaning and distribution. For reasons of space, we only mention this salient characteristic here and refer the readers to the works on ECs mentioned in this paper. \REF{skoro} illustrates the \ili{Bulgarian} \textit{(po-)}\textit{skoro} with its usual temporal meaning.\footnote{How \isi{modal} uses of temporal adverbs like \textit{skoro} relate to their temporal uses is an important question. Although we do not address this issue here, we believe the diachronic account of the development of \isi{modal} uses of adverbs like \textit{skoro} (and \textit{rather}) from their temporal uses presented in \citet{ger16} is on the right track.
}

	\ea \gll Toj dojde	po-skoro	otkolkoto očakvah.\\
	he came	more-soon than expected.\textsc{1sg} \\
 	\glt `He came sooner than I expected.'\hfill (\ili{Bulgarian}) \label{skoro}
	\z

\noindent ECs have not been thoroughly investigated. There are four studies we are aware of: \citet{finkra14,herrub14,gon14,goniri17a}. \citet{finkra14} look at the properties  of ECs in \ili{German} without attempting a formal analysis. \citet{gon14} describes ECs in \ili{Russian}. \citet{herrub14} use  ECs to argue that possibility modals in \ili{German} are not gradable. We believe the analysis proposed by \citet{herrub14} is on the right track. We modify it slightly in  \sectref{proposal} to align it better with our assumptions. \citet{goniri17a} discuss the cross-linguistic variation of ECs, and provide an account for it.

In the next section we introduce the  puzzle that is the \isi{focus} of the present paper. The facts are based on novel  data related to ECs in \ili{Bulgarian}. We also show that similar facts obtain in \ili{Romanian}{.} Although not a Slavic language, \ili{Romanian} proves very useful for deepening our understanding of the \ili{Bulgarian} facts and for making cross-linguistic generalizations. \ili{Romanian} has ECs, as seen in \REF{RomEC}. Similarly to \ili{Bulgarian}, this sentence is constructed with an adverb (\textit{mai degrabă} `more soon'), which also has a temporal meaning, illustrated in \REF{RomTempAdv}.\footnote{ECs built from temporal adverbs can be  absent in certain languages, for example present day \ili{English}, as seen in \REF{rather}. ECs with temporal adverbs existed in Middle/Early Modern \ili{English}, as discussed in \citet{ger16}. ECs formed as `it is more likely to {\ldots} than' arguably have similar meaning, but for the purposes of this paper we take them to be a different constructions.

 	\ea[*]{John is \{sooner/more/rather\} at work than at home. \label{rather}}
 	\zlast
 	}

 	\ea
 	\ea\label{RomEC}\gll Ion este mai degrabă la birou decât acasă.\\
	Ion is more \textsc{adv}.early at office than home\\
	\glt `According to the speaker, it is more plausible that Ion is in the office than at home.'
	\ex\label{RomTempAdv}\gll Trebuie să vii mai degrabă.\\
	need \textsc{sbjv} come.\textsc{2sg} more  \textsc{adv}.early\\
	\glt `You need to come sooner.'\hfill (\ili{Romanian})
	\z\z

\noindent Importantly, \ili{Romanian} is part of the Balkan Sprachbund, just like \ili{Bulgarian}. Thus, it exhibits several features that are characteristic to this geographical area, among which: (i) presence of suffixed definiteness; (ii) lack of sequence-of-\isi{tense} phenomena; (iii) an analytic future constructed from the auxiliary \textit{want} (see especially \citealt{mistom06}). Another feature that \ili{Bulgarian} and \ili{Romanian} share, which is most important for us here, is the existence of grammaticalized evidentiality. Looking at two languages -- \ili{Bulgarian} (Slavic) and \ili{Romanian} (Romance) -- in a typological contact situation can give us relevant hints into the nature of the phenomena discussed in this paper. In the next section we show that ECs are not well-formed  with grammaticalized indirect evidentials, under inferential interpretations. We use data from both \ili{Bulgarian} and \ili{Romanian}.



\section{ECs and grammaticalized evidentials} \label{ECsIIEV}
Both \ili{Romanian} and \ili{Bulgarian} have verbal forms that express (indirect) evidentiality. We take the existence of these forms to support the claim that (indirect) evidentiality is grammaticalized in these languages.\footnote{In this paper we make a distinction between grammaticalized evidentials and what we will later call phrasal/lexical evidentials. See \sectref{ConcludRem} for some discussion.}

We provide two relevant examples in \REF{bulevid} and \REF{romevid}:

	\ea \ea \gll Ivan bil vkušti. \\
		Ivan be.\textsc{pst.ptcp=evid} home \\
		\glt `Apparently, Ivan is at home.'\hfill (\ili{Bulgarian}) \label{bulevid}
    \ex \gll Ion o fi acasă. \\
 		Ion \textsc{presmp} be at.home \\
		\glt `Apparently, Ion is at home.'\hfill (\ili{Romanian}) \label{romevid}
        \z \z

\noindent As mentioned in the introduction and illustrated in \REF{bulevid}, the \ili{Bulgarian} \isi{past participle} can carry \isi{IE} interpretations, see \citet{jakob156,comrie76,palmer86,izv97,alexetal03,mistom06b}; a.o. In \ili{Bulgarian}, the \isi{present perfect} is ambiguous. It can have a regular temporal interpretation or function as an \isi{evidential}. The \isi{present perfect} is formed from a present auxiliary and a \isi{past participle}. With the 3rd person the auxiliary can be omitted. In such cases, the \isi{present perfect} is unambiguously interpreted as \isi{evidential}, see \cite[fn. 7]{izv97}. In this paper, we use this strategy to rule out the temporal interpretation (\textit{bil} is \textsc{3sg.past.part}).\footnote{We are grateful to Roumyana Pancheva for clarifying this point.}

In the \ili{Romanian} example in \REF{romevid}, \isi{IE} is conveyed by the presumptive mood (\textsc{presmp}), as discussed especially by \citet{slavel56,cos76,reinhrip00,squ01,iri10,iri18}; a.o.\footnote{The \ili{Romanian} presumptive form \textit{o} `\textsc{\isi{presmp}.3sg}' is a \isi{modal} auxiliary which shows ambiguity between a \isi{modal} future reading and an \isi{evidential} reading. The presumptive mood in \ili{Romanian} can also be constructed from other \isi{modal} auxiliaries, e.g. conditional, \isi{subjunctive}.\label{fn:presmp}}

\largerpage[2]
The puzzle we discuss in this paper is that ECs are not well-formed in sentences with IEs, whereas they are grammatical with \isi{modal} auxiliaries.\footnote{We limit our investigation to indirect evidentiality because it has been convincingly demonstrated that ECs are unacceptable with direct evidence, just like regular \isi{epistemic} modals (see \citealt{herrub14}).}
Compare \REF{mobs1} with \REF{mobs2}:

	\ea \label{mobs1} \ea[??]{\gll Ivan bil po-skoro v offisa	otkolkoto	vkušti. \\
	 Ivan be.\textsc{pst.ptcp} more-soon	in	office	than home \\
	\glt Intended: `Apparently, it is more plausible that Ivan is in the office than at home.' \hfill(\ili{Bulgarian})} \label{IllformedECwithIIE}
	\newpage
	\ex[??]{\gll Ion o fi mai	degrabă la birou	decât 	acasă.  \\
	Ion \textsc{\isi{presmp}.fut}	be	more \textsc{adv}.early at	office	than	home \\
	\glt Intended: `Apparently, it is more plausible that Ion is in the office than at home.' \hfill (\ili{Romanian})}
 \label{IllformedECwithIIERom}
    \z \z

    \ea \label{mobs2} \ea \gll Ivan po-skoro moje da e 	v	ofisa	otkolkoto vkušti.   \\
    Ivan	more-soon
  can \textsc{da} is 	in	office	than home \\
      \glt  `It is more plausible that Ivan might be at work than at home.'\\\xspace\hfill (\ili{Bulgarian})
	\ex \gll Ion	poate 	fi		mai	degrabă	la 	birou decât acasă.\\
	Ion 	can.\textsc{3sg}	be.\textsc{inf}	more \textsc{adv}.early at office than home \\
	\glt `It is more plausible that Ion might be in the office than at home.'\\\xspace\hfill (\ili{Romanian})
	\z \z


\noindent One important observation is that the incongruence between  the indirect \isi{evidential} and the \isi{EC} only affects the \textsc{inferential indirect evidentials} (\isi{IIE}).\footnote{We are grateful to Roumyana Pancheva for the illuminating discussion on this point.}

It is well known that indirect evidentials come in two broad categories: inferential (the statement is based on the inference the speaker draws from available evidence) and \isi{hearsay} (the statement is based on somebody else's report). In \ili{Bulgarian} the present-perfect-as-an-\isi{evidential} in \REF{bulevid} is ambiguous. It can obtain both inferential and \isi{hearsay} readings (see especially \citealt{smir13,koev17}). Ill-formedness only arises under the inferential reading. The \isi{hearsay} interpretation (`based on what I heard, it is more plausible that Ivan is in the office than at home') is not deviant.\footnote{\cite[fn. 2]{koev17} notes that the inferential reading of the present-perfect-\isi{evidential} in \ili{Bulgarian} is more restricted than its \isi{hearsay} reading. There is also speaker variation in this respect. Therefore, speakers who only have the \isi{hearsay} reading will not perceive the contrast we are interested in.}
The \ili{Romanian} example in \REF{IllformedECwithIIERom} is not ambiguous. Hearsay readings of the \textsc{presmp} are normally constructed from conditional morphology on the auxiliary (e.g. \textit{ar}=\textsc{cond.3}), see \citet{iri18}.\footnote{One way to disambiguate the \isi{hearsay} readings from the conditional ones is by embedding them under overt \isi{hearsay} marking (`they say that'), as in the following example. Note that this sentence also illustrates the phenomenon of \isi{evidential} concord.

	\ea \gll \minsp{(} Se spune că) Ion ar fi fost la birou mai degrabă decât acasă.\\
	{} \textsc{se} says that Ion \textsc{cond=\isi{presmp}.3} be been at office more \textsc{adv}.early	than	home\\
	\glt `(They say that) based on \isi{hearsay}, it is more plausible that Ion was in the office rather than at home.'\z
}

 To summarize, the novel observation is that ECs are compatible with \isi{epistemic} modals, but are deviant with grammaticalized IIEs. The observation is based on \ili{Bulgarian} and \ili{Romanian}, two languages which grammaticalize evidentials using the \isi{present perfect} and the presumptive mood respectively.


\section{Proposal}\label{proposal}
In a nutshell, our proposal is that \textit{-er} in ECs and grammaticalized (inferential) evidentials are both degree modifiers of the (gradable) \textsc{speech act} (\isi{SA}) operator \isi{Assert}. Thus, they computationally compete for the same position. Therefore, the deviance of the examples in \REF{mobs1} is similar to \textit{*John is more that tall}, in which \textit{more} and  \textit{that} specify the degree of tallness. To flesh out our account we build on  the insights in \citet{grewol17}, who propose that  \isi{Assert} contains a \isi{degree argument} and thus has to compose with a \isi{degree modifier} (similarly to gradable adjectives). We combine this insight with the idea in \citet{davetal07} that evidentials reset the degree of credence of the speaker.

\subsection{Preliminary remarks} \label{PrelimRem}
We begin this subsection by discussing
\citet{grewol17} and then, spell-out some details of \citeposst{herrub14} analysis of ECs, on which we build our account.

\citet{grewol17} base their proposal on the idea that \isi{Assert} is modifiable. The evidence they use comes from the difference in distribution between \textsc{\isi{modal} adverbs} (MAdvs) and \textsc{\isi{modal} adjectives} (\isi{MAdjs}). It has been previously noticed in the literature that MAdvs and \isi{MAdjs} differ in (at least) the following respects.

 First, speaker orientation is stronger with MAdvs than \isi{MAdjs}, as seen in the contrast between \REF{madvena} and \REF{madvenb}, cited from \citet{grewol17}.

 	\ea \label{madven}
 	\ea\label{madvena}
 	\begin{xlist}
 	\exi{A:}[]{It is probable that they have run out of fuel.}
 	\exi{B:}[]{Whose opinion is this?}
    \end{xlist}
		\ex\label{madvenb}
	\begin{xlist}
	\exi{A:}[]{They have probably run out of fuel.}
	\exi{B:}[\#]{Whose opinion is this?}
			\end{xlist}
            \z\z

\noindent Second, only \isi{MAdjs} are possible in the antecedent of conditionals that do not have an assertive force by themselves (e.g. \citealt{kra91}), as shown in \REF{madvif}:

 	\ea \label{madvif} \ea[\#] {If John is possibly/probably/definitely in the office, I will call the office.}
		\ex[]{ If it is possible/probable/certain that John is in the office, I will call the office. \hfill \citep{grewol17}}
        \z\z

\noindent Finally, in confirmational tag-questions, \textit{yes} picks up the content of the proposition when \isi{MAdv} is used, see \REF{madvyesa}. On the other hand, MAdj confirms the degree of credence, see \REF{madvyesb}.

	\ea \ea\label{madvyesa}
	\begin{xlist}
	\exi{A:}[]{John is possibly/probably/certainly in the office, eh?/right?}
	\exi{B:}[]{Yes. (John is in the office.)}
		\end{xlist}
	\ex\label{madvyesb}
	\begin{xlist}
	\exi{A:}[]{It's possible/probable/certain that John is in the office, eh?/right?}
    \exi{B:}[]{Yes. (It's possible/probable/certain that John is in the office.)}
	\end{xlist}
        \z \z

\noindent \citet{grewol17} use these differences to suggest that MAdvs function as \isi{Assert} modifiers. To implement this idea compositionally they use the degree-semantics framework. More specifically, they propose the denotation of \isi{Assert} in \REF{assert}.\footnote{\citet{grewol17} also propose a more complex denotation of \isi{Assert} in terms of context update.
Although we believe that their implementation of the gradable \isi{Assert} is more adequate and can be easier incorporated into the existing \isi{SA} and \isi{evidential} systems, for the purpose of this paper we use a simplified denotation. This denotation is sufficient  to demonstrate the interaction between ECs and \isi{IIE}.}

In \REF{assert}, informally speaking, \cnst{Cr}$_{x}$ is a \isi{measure function} from propositions to degrees on the \isi{credence scale} (see \citealt{herrub14} and references cited there for some discussion). Additionally, we assume that `$x$'s credence in $p$' implies $x$'s commitment to behave as if ($x$ believes that) $p$. This is important for explaining the difference between \isi{IIE} and \isi{hearsay}.

	\ea \sib{\isi{Assert}$_{x}$}\textsuperscript{c} $= \lambda p\lambda d_d\,.\, \cnst{Cr}_{x}(p) \geq d $ \\
    where \cnst{Cr}$_{x}$ is a function that takes a proposition $p$ and returns a degree to which $x$ is committed to behave as if $x$ believes that $p$ \label{assert}
    \z

\noindent According to this system, simple sentences like \REF{simplea} have the LF in \REF{simpleb}, where the \isi{degree argument} is saturated by a contextually set \cnst{pos}(itive) operator, defined in \REF{sxpos}. This is similar to the standard treatment of gradable adjectives in simple sentences like \textit{John is tall} in degree-semantics \citep[e.g.][]{mcnken05}. \REF{simpled} computes the truth-conditions of \REF{simplea}. \REF{simplea} is true iff there is a contextually set degree $d$ such that the speaker has credence in the proposition that John is home at least to $d$.

    \ea \ea John is at home. \label{simplea}
    	 \ex LF: [ [ POS [ \isi{Assert} ] ] [John is at home] ] \label{simpleb}
         \ex \sib{POS}\textsuperscript{c} \(= \lambda G\lambda p\exists d[d \geq \cnst{standard}_{c} \wedge G(p)(d)]\) \label{sxpos}
         \ex $\big[$[\sib{POS}\textsuperscript{c} (\sib{\isi{Assert}$_{x}$}\textsuperscript{c})](\sib{J. is at home}\textsuperscript{c})$\big]$ = \\
      	$ = [[\lambda G\lambda p\exists d[d \geq \cnst{standard}_{c}  \wedge G(p)(d)]](\lambda p\lambda d\,.\, \cnst{Cr}_{x}(p) \geq d)]($\sib{J. is  at home}\textsuperscript{c}) = \\
         $= \exists d[d \geq \cnst{standard}_{c} \wedge \cnst{Cr}_{x}($\sib{J. is at home}\textsuperscript{c}) $\geq d ]$  \\
         (where $x$ = the speaker in $c$)	\label{simpled}
    \z \z

\noindent In more complex sentences with \isi{modal} adverbs as in \REF{clxa}, the \isi{degree argument} of \isi{Assert} is saturated by the \isi{MAdv}. Informally, we take MAdvs to set degrees on the \isi{credence scale}, e.g.\ \textit{possibly} $p$ holds iff \cnst{Cr}$_{x}(p) >$ 0, \textit{probably} $p$ holds iff \cnst{Cr}$_{x}(p) >$ 0.5, and \textit{certainly} $p$ holds iff \cnst{Cr}$_{x}(p) >$ 0.98, see also \citet{grewol17} for a more formal discussion. \REF{clxa} has the LF in \REF{clxb}. The denotation of \textit{probably} is given in \REF{clxc} and the truth-conditions for \REF{clxa} in \REF{clxd}. \REF{clxa} is true iff the degree of speaker's credence in the proposition that John is home is greater than 0.5 on the \isi{credence scale}.


 	\ea \ea John is probably at home. \label{clxa}
    	\ex LF: [[ probably [ \isi{Assert} ]] [John is at home] ] \label{clxb}
        \ex \sib{probably}\textsuperscript{c} $= \lambda G\lambda p\exists d[d > 0.5 \wedge G(p)(d)]$ \label{clxc}
        \ex   {[\sib{probably}\textsuperscript{c} (\sib{\isi{Assert}$_{x}$}\textsuperscript{c})]} (\sib{J. is at home}\textsuperscript{c}) = \\
		 $= [[\lambda G\lambda p\exists d[d > 0.5  \wedge G(p)(d)]](\lambda p\lambda d\,.\, \cnst{Cr}_{x}(p) \geq d)]$ (\sib{J. is at home}\textsuperscript{c}) \\
		 $= \exists d[d > 0.5 \wedge \cnst{Cr}_{x}$ (\sib{J. is at home}) $\geq d]$ \\
       (where $x$ = the speaker in $c$) \label{clxd}
 \z \z

\noindent A potential objection to the idea of gradable \isi{Assert} could be that \isi{SA}  operators are not normally part of the compositional derivation and cannot be  embedded. However, various contributions have shown that under certain conditions \isi{SA} operators can be embedded, see \citet{grewol17} for references.

Turning now to ECs, we follow the decompositional account of \citet{herrub14}, who analyze \ili{German} ECs of the type in \REF{ger1}. For \citet{herrub14} \textit{eher} is decomposed into a \isi{comparative} head \textit{-er} with the regular denotation in \REF{sxer} and an \isi{epistemic} component \textit{eh-}, which they take to be a believe-type predicate with a \isi{degree argument}, see \REF{sxeh}.


	\ea \gll Hans ist eher auf der Arbeit als zu Hause.\\
 	Hans is sooner at the work than at home\\
  	\glt `According to the speaker, it is more plausible that Hans is at work than at home.'  \hfill (\ili{German})  \label{ger1}
    \z

 	\ea  \label{hr} \ea \sib{-er} = $\lambda P\lambda Q\,.\,\cnst{max}(Q) > \cnst{max}(P)$ \label{sxer}
   	\ex \sib{eh-}\textsuperscript{z} = $\lambda p\lambda d\,.\,z$ is $d$-ready to believe $p$ \\
       (defined only if $z$ doesn't have direct evidence for $p$) \label{sxeh}
        \z \z

\noindent According to this system, the example in \REF{ger1} has the LF in \REF{ehera} and the truth-conditions in \REF{eherb}. \REF{hr} and \REF{eher} are from \cite[564--565]{herrub14}. With angle brackets ($\langle{\ldots}\rangle$), we signal the material that is not phonologically present.

	\ea \label{eher} \ea LF: [[-er [than $\langle$eh- Hans is$\rangle$ at home]] [eh- Hans is at work]] \label{ehera}
 	\ex \sib{-er}\textsuperscript{z}(\sib{than eh- Hans is at home}\textsuperscript{z})(\sib{eh- Hans is at work}\textsuperscript{z}) = \\
    $ \cnst{max}(\lambda d\,.\,z$ is $d$-ready to believe that Hans is at work) $ > \cnst{max}(\lambda d\,.\,z$ is $d$-ready to believe that Hans is at home) \\
    where $z$ is  the speaker \label{eherb}
	\z \z

\noindent Interesting support for the decompositional analysis of \textit{eher} comes from the fact that in \ili{Austrian} and \ili{Bavarian German} there is a discourse particle \textit{eh-} with a similar \isi{epistemic} interpretation, see \REF{ehAG} from \cite[ex.32]{herrub14}. See also \citet{zob17} for a detailed investigation of \textit{eher}.

 	\ea \gll Das ist auf regionaler Ebene eh m\"oglich. \\
    	that is on regional level eh possible \\
		\glt `That is anyways possible on a regional level.' \hfill (\ili{Austrian} \ili{German}) \label{ehAG}
	\z


\subsection{Analysis}
 We begin our analysis by discussing the interpretation of \textsc{inferential indirect evidentials} (IIEs). We propose that they function as degree modifiers of \isi{Assert} on a par with MAdvs (as discussed above). This claim is limited to IIEs; in this paper, we remain agnostic with respect to other types of evidentials, apart from \isi{hearsay} evidentials that, as we show below, are not modifiers of \isi{Assert}.  Furthermore, building on \citet{davetal07}, we assume that IIEs reset the threshold of the credence function from a contextually set value (set by POS) to the \isi{evidential} value, see \REF{sxiie}. This is illustrated for \ili{Romanian} in \REF{rom1}. \ili{Bulgarian} IIEs receive a similar account. For reasons of space, we provide only the LF and the truth-conditions.\footnote{We gloss over the mechanics of how the \isi{evidential} meaning comes about and how the source of evidence is encoded. These details are orthogonal to the point made in this paper, but see the discussion in \citet{koev17}.
 }

	\ea \sib{\isi{Evid}\textsubscript{IIE}}\textsuperscript{c} = $\lambda G\lambda p\exists d[ d = \mu^{c} (evid) \wedge G(p)(d)] $\\
    where $\mu$ maps the strength of evidence to a degree on the \isi{credence scale} in $c$ \label{sxiie}
    \z

 	\ea  \label{rom1} \ea \gll Ion o fi acasă. \\
 			Ion \textsc{presmp}=\textsc{IIE} be at.home \\
			\glt `Apparently, Ion is at home.' \hfill (\ili{Romanian})
 		\ex LF: [ [\isi{Evid}\textsubscript{IIE}[ \isi{Assert} ]] [Ion is at home]]
        \ex $\exists d[d = \mu_{c}(evid) \wedge  \cnst{Cr}_{x}$(\sib{Ion is at home}) $\geq d ]$ \\ (where $x$ = the speaker in $c$)
        \z \z


\noindent Our proposal for IIEs makes the immediate prediction that IIEs are incompatible with MAdvs, as they compete for the same position. This prediction is borne out for \ili{Romanian} \textit{posibil}, as shown in \REF{pofi}:

	\ea[??]{\gll Ion (posibil) o fi (posibil) acasă.\\
	 Ion possibly \textsc{presmp}=\textsc{iie} be (possibly)  at.home \\
	\glt Intended: `Possibly, Ion is apparently at home.'  \hfill (\ili{Romanian})} \label{pofi}
    \z

\noindent We now account for the core observation, namely that ECs are incompatible with IIEs. We propose that the  \isi{epistemic} component in ECs, expressed by \textit{eh-} in \ili{German} (see the observations above) can be assimilated to \citeposst{grewol17} \isi{Assert}. We generalize \citeposst{herrub14} analysis of \ili{German} to \ili{Bulgarian} and \ili{Romanian} and represent \textit{eh-} abstractly as
\isi{Epist} below. Both \isi{Assert} and \isi{Epist} are gradable and both manipulate (usually) the speaker's degree of credence in the proposition expressed by the prejacent. There is, however, an important difference between the two: \isi{Epist} is presuppositional, i.e.\ it is undefined if the speaker has direct evidence \citep[see the discussion in][]{herrub14}.

	 \ea \sib{\isi{Epist}$_{x}$}\textsuperscript{c} = $\lambda p\lambda d\,.\,\cnst{Cr}_{x}(p) \geq d$\\  (defined only if $x$ doesn't have direct evidence for $p$) \label{}
 	\z

\noindent To simplify the computation of comparatives and make it parallel to \isi{modal} adverbs, we slightly modify the structure advocated by \citet{herrub14} for eher-comparatives. We assume that \isi{EC} in \REF{coredataa} has the LF in \REF{coredatab}. We further assume that \textit{-er} has the denotation in \REF{coredatac}, where the \textit{than}--clause is a definite description of degrees (as assumed for gradable adjectives), see \REF{coredatad}. For reasons of space, we omit the details of how the meaning of the than-clause is obtained.
The truth-conditions for \REF{coredataa} (if defined) are given in \REF{coredatae} and paraphrased in \REF{coredataf}. We show this using \ili{Bulgarian}, but the same holds for \ili{Romanian}.

	\ea \ea \gll Ivan	po-skoro	e 	v	ofisa	otkolkoto vkušti.\\							Ivan	more-soon	is	in	office	than home \\
        \glt `According to the speaker, it is more possible that Ivan is at work than at home.' \hfill (\ili{Bulgarian}) \label{coredataa}
    	\ex {[[[}-er [than $\langle$\isi{Epist} Ivan is$\rangle$ at home]] \isi{Epist}] [Ivan is in the office]] \label{coredatab}
        \ex \sib{-er} $= \lambda d\lambda G\lambda p\exists d'[d' > d  \wedge G(p)(d')]$ \label{coredatac}
        \ex \sib{than \isi{Epist}$_{x}$ Ivan is at home}\textsuperscript{c} = $\cnst{max}(\{d: \cnst{Cr}_{x}$(\sib{Ivan is at home}) $\geq d \})$ \label{coredatad}
        \ex \sib{-er}(\sib{than \isi{Epist}$_{x}$ Ivan is at home}\textsuperscript{c}) = \\
        = $\lambda G\lambda p\exists d'[G(p)(d') \wedge\ d' > \cnst{max}(\{d: \cnst{Cr}_{x}$(\sib{Ivan is at home}) $\geq d \})]$ \\
        \sib{-er than \isi{Epist}$_{x}$ Ivan is at home}\textsuperscript{c}(\sib{\isi{Epist}$_{x}$}\textsuperscript{c}) = \\
        = $\lambda p\exists d'[\cnst{Cr}_{x}(p) \geq d' \wedge d' > \cnst{max}(\{d: \cnst{Cr}_{x}$(\sib{Ivan is at home}) $\geq d \})] $ \\
        \sib{-er than \isi{Epist}$_{x}$ Ivan is at home  \isi{Epist}$_{x}$}\textsuperscript{c}(\sib{Ivan is in the office}\textsuperscript{c}) =  \\
     $= \exists d'[\cnst{Cr}_{x}$(\sib{Ivan is in the office}) $\geq d' \wedge d' > \cnst{max}(\{d: \cnst{Cr}_{x}$(\sib{Ivan is at home}) $\geq d \})]$ \label{coredatae}

     \ex In prose: There is a degree to which $x$  believes Ivan is in the office is plausible and this degree is higher than the maximal degree to which $x$ believes that Ivan is at home is plausible (where $x$ is the speaker) \label{coredataf}
	\z \z

\noindent Given these assumptions, ECs are deviant with IIEs for the same reason \textit{posibil} is deviant with IIEs in \REF{pofi} above. That is to say, \isi{IIE} competes with \textit{-er} for the degree \isi{modifier position}. \REF{illfcdb} shows a simplified LF for the ill-formed \REF{illfcda} repeated from above (the underlined part shows the competition).



 	\ea \ea[*]{\gll Ivan	bil po-skoro	v	ofisa	otkolkoto vkušti.\\
		Ivan be.\textsc{pst.ptcp} more-soon in office than home \\
		\glt Intended: `Apparently, it is more plausible that Ivan is in the office than at home.'\hfill (\ili{Bulgarian})} \label{illfcda}
    	\ex[]{LF: *[[ \underline{ \{ [-er than $\langle$Ivan is$\rangle$ at home ] / \isi{Evid} \} } \isi{Epist} ] [Ivan is in the office]]  \label{illfcdb}}
        \z \z

\noindent Our account also explains why \isi{hearsay} evidentials are well-formed with ECs. Several researchers, among which \citet{fall02} and \citet{smir13} have pointed out that \isi{hearsay} evidentials do not require the speaker's commitment. In our system, this can be implemented by saying that \isi{hearsay} evidentials are not \isi{Epist}/As\-sert modifiers. Therefore, they do not compete with \textit{-er} in ECs for the degree \isi{modifier position}.

 As \isi{epistemic} modals are not degree modifiers of \isi{Epist}/\isi{Assert}, they are felicitous with ECs, see \REF{mightroma} repeated from above and its simplified LF in \REF{mightromb}.\footnote{Independent support for this comes from \citet{iri18} who has shown that there are important structural differences between the \isi{IIE} reading and the non-\isi{IIE} \isi{modal} reading of \ili{Romanian} \textsc{presmp}. Modal interpretations are obtained when the \isi{modal} features are merged in \isi{Mod}$^0$ and raised to T$^0$. \isi{IIE} interpretations are obtained by the merge of features related to the speaker's deictic location `now' in the  Sentience projection in the CP layer above the \isi{modal} in T$^0$. Note that according to this account \isi{IIE} features are interpreted higher than \isi{modal} features. One question would be why examples like \REF{IllformedECwithIIERom} are not well-formed under the future reading of the relevant morpheme. The situation with this auxiliary is more complex. First, not many speakers accept an interpretation of this morpheme which is purely future. For those speakers, though, for which the unmarked future reading is possible, no ill-formedness arises with \isi{EC}. For the majority of the other speakers, the question is what type of \isi{epistemic} future this auxiliary encodes that is distinct from both \isi{IIE}, as well as from a more well-behaved future, but at the same time is also ill-formed with ECs. We leave this issue for further research, as the exact readings need further attention (see also \citealt{mih13}).
}
The same holds for \ili{Bulgarian} in \REF{mightbul}.

 	\ea \ea \gll Ion poate fi mai degrabă la birou decât acasă.  \\
Ion 	can-\textsc{3sg} be-\textsc{inf} more \textsc{adv}.early at office than home \\
		\glt `It is more plausible that Ion might be in the office than at home.' \hfill (\ili{Romanian}) \label{mightroma}
   	 \ex LF: [[ { [-er than $\langle$Ion be$\rangle$ at home ] } \isi{Epist} ] [ might [Ion be in the office]]] \label{mightromb}
     \z \z

	\ea \gll Ivan po-skoro moje da e 	v	ofisa	otkolkoto vkušti.  \\
Ivan	more-soon	can \textsc{da} is 	in	office	than home \\
        \glt `It is more plausible that Ivan might be at work than at home.' \hfill (\ili{Bulgarian}) \label{mightbul}
        \z

\noindent Independent support for our proposal comes from the fact that ECs are also ill-formed with MAdvs. Recall that according to \citet{grewol17}, MAdvs are degree modifiers of the gradable \isi{Assert}. Thus, they are expected to compete with \textit{-er} in ECs, see \REF{rompblec}.

  \ea[*]{\gll Ion posibil este la birou mai degrabă decât acasă.\\
  	Ion possibly is at office more soon than home \\
 	\glt Intended: `According to the speaker, it is more plausible that Ion is possibly in the office rather than at home.'  \hfill (\ili{Romanian})} \label{rompblec}
    \z

\noindent To summarize, by assimilating the \isi{epistemic} component in ECs to gradable \isi{Assert} \citep{grewol17}, we derive the incompatibility of ECs and IIEs as a result of the competition for the degree \isi{modifier position}. This proposal assumes that (some) evidentials function as degree modifiers. This correctly predicts the difference between inferential and \isi{hearsay} indirect evidentials, assuming that the latter does not involve speaker's commitment. We, thus, identify three (overt) elements that can function as degree modifies for \isi{Epist}/\isi{Assert}: MAdvs, IIEs, and \textit{-er} in ECs.


\subsection{Predictions}\label{sec:predictions}
 Our account makes a number of correct predictions. The first prediction is that the \isi{IIE} is compatible with regular comparatives. In regular comparatives, \isi{IIE} scopes above \textit{-er} and the structure is grammatical, as  shown in \REF{lowera} for \ili{Romanian} and in \REF{lowerb} for \ili{Bulgarian}. In \REF{lowera} and \REF{lowerb}, \textit{-er} merges low as it compares degrees of tallness/happiness, rather than degrees of belief as in ECs. The simplified LF for  \REF{lowera} and \REF{lowerb} is illustrated in \REF{lowerc}.

	\ea \ea \gll Ion  o 		fi 		mai	\^inalt 	decât	Maria. \\
		Ion \textsc{\isi{presmp}=iie}	be more	tall than Mary \\
		\glt `Apparently, John is taller than Mary.' \hfill (\ili{Romanian}) \label{lowera}
    	\ex \gll Ivan bil po-stastliv ot Maria. \\
 		Ivan be-\textsc{pst.ptcp=iie} more-happy from Maria \\
		\glt `Apparently, Ivan is happier than Maria.' \hfill (\ili{Bulgarian}) \label{lowerb}
    	\ex LF: [[\isi{Evid}\textsubscript{IIE}(\isi{Assert})] [ [ -er [than Mary is $d$-tall/happy]] [John is $d'$-tall/happy]]] \label{lowerc}
        \z \z

\noindent The second prediction is that \isi{IIE} can co-occur with \isi{epistemic} attitude predicates like `believe'. This is illustrated in \REF{bela} for \ili{Romanian} and in \REF{belb} for \ili{Bulgarian}. We give the simplified LF for these examples in \REF{belc}.

    \ea \ea \gll Ion o fi crezând toate minciunile.\\
    Ion \textsc{\isi{presmp}=iie} be believe.\textsc{ger} all lie.the.\textsc{pl} \\
    \glt `Apparently, Ion believes all the lies.'  \hfill (\ili{Romanian}) \label{bela}
	\ex \gll Ivan bil vjarval na vsički l'ži.\\
   	 Ivan be.\textsc{pst.ptcp=iie} believe.\textsc{pfv} on all lies \\
    \glt `Apparently, Ivan believes all the lies.'   \hfill (\ili{Bulgarian}) \label{belb}
    \ex  {[[\isi{Evid}\textsubscript{IIE}(\isi{Assert})] [Ion believes all the lies]]} \label{belc}
    \z \z

\noindent These data support our account of the ill-formedness of ECs with IIEs. They also rule out alternative analyses according to which the deviance is due either to the incompatibility of evidentials and comparatives or to a potential conflict between evidentials and \isi{epistemic} attitudes.

\section{Concluding remarks and future research}\label{ConcludRem}

We have analyzed some previously unnoticed facts related to \isi{epistemic} modals and evidentials when they occur with \isi{epistemic} comparatives in \ili{Bulgarian} and \ili{Romanian}. We showed that ECs are incompatible with IIEs and explained this pattern by claiming that the two categories compete for the same position. Given that the ill-formedness does not arise with \isi{epistemic} modals, the  data examined here argue for accounts under which inferential evidentials are separated from \isi{epistemic} modals (\citealt{fall02}, \citealt{aikh14}, \citealt{murray10}; a.o.).

From a broader perspective, the observation presented in this paper and its account give rise to several  questions. In the remainder of the conclusion we briefly touch on three of them, leaving the detailed investigation for future research.

First, one expectation is that ECs should be ill-formed with indirect evidentials across-the-board. However, there appear to be cases in which ECs are well-formed with expressions that could be analyzed as having \isi{evidential} meaning.\footnote{We thank Sergei Tatevosov for this observation.}
We illustrate some examples below. In \REF{lexevidbul} and \REF{lexevidrom} we see that evidential-like adverbials like \textit{vidimo} and \textit{aparent} `apparently' are well-formed with the \isi{EC}.\footnote{As expected, adverbials with \isi{hearsay} semantics are well-formed, see below. Recall that \isi{hearsay} evidentials are not \isi{Assert} modifiers, thus do not compete with ECs.

	\ea \gll Kazvat Ivan po-skoro e v offisa otkolkoto  vkušti. \\
	they.say Ivan	more-soon is 	in	office	than home \\
	\glt `As they say, Ivan is at work rather than at home.' \hfill (\ili{Bulgarian})
	\z

    \ea \gll Cică Ion este la birou mai degrabă decât acasă. \\
	they.say John is at office more soon than home\\
	\glt `As they say, John is in the office rather than at home.' \hfill (\ili{Romanian})
	\z

	}

	\ea[?]{\gll Vidimo, Ivan po-skoro moje da e 	v	ofisa	otkolkoto vkušti. \\
	apparently, Ivan	more-soon can \textsc{da} is in	office than home \\
	\glt `It is more plausible that Ivan might be at work than at home.' \small(\ili{Bulgarian})} \label{lexevidbul}
    \z

	\ea \gll Aparent, Ion este la birou mai degrabă decât acasă. \\
	apparently Ion is at office more soon than home\\
	\glt `Apparently, Ion is in the office rather than at home.' \hfill (\ili{Romanian})\label{lexevidrom}
    \z

\noindent Examples of this type touch on an important issue, namely the difference between grammaticalized and phrasal evidentials. We take the former to be expressed by means of (\isi{inflectional}) morphology on the \isi{verb}. In the latter class we include \isi{adverbial} evidentials (like \textit{apparently}, etc.) and other phrasal units (like \textit{in my opinion}, etc.), which have \isi{evidential} semantics, see for example \citealt{aikh14}, a.o. We follow standard accounts for phrasal evidentials as having different syntax from grammaticalized evidentials (\citealt{fall02}, \citealt{aikh14}; a.o.). Thus, the well-formedness of \REF{lexevidbul} and \REF{lexevidrom} is not problematic for our account, as lexical evidentials do not compete with -\textit{er} for the \cnst{assert} \isi{modifier position}.

Second, we also observe that ECs can be embedded under expressions like \textit{I guess}, etc, that are sometimes claimed to have \isi{evidential} interpretations. Two examples from \ili{Romanian} are given in \REF{taglexevid}.


	\ea \label{taglexevid} \ea \gll Bănuiesc că Ion este la birou mai degrabă decât acasă.\\
		guess.\textsc{1.sg} that Ion be.\textsc{3sg} at office more soon than home\\
		\glt `I guess Ion is in the office rather than at home.'
		\ex \gll Cred că Ion este la birou mai degrabă decât acasă.\\
		believe.\textsc{1.sg} that Ion be.\textsc{3sg} at office more soon than home\\
		\glt `I believe Ion is in the office rather than at home.' \hfill (\ili{Romanian})
		\z \z

However, for cases like \REF{taglexevid}, there is independent evidence that they are biclausal (for example the presence of overt complementizers like \textit{că} `that'). Therefore, competition does not arise. It is also well known that ECs can be embedded under attitude predicates like \textit{believe, hope}, etc. (see \citealt{herrub14}, as well as the discussion in footnote \ref{fn:relativ}). We assume that the embedding under \textit{I guess} is amenable to a parallel analysis.\footnote{We thank an anonymous review for bringing to our attention \ili{Czech} data that support the same conclusion. We are also grateful to another anonymous reviewer who pointed out to us the connection between embedding under \textit{I guess} and attitude reports.
}

More surprisingly, embedding improves the ungrammaticality of grammaticalized evidentials with ECs. See the contrast in \REF{moremba} vs. \REF{morembb} and \REF{morembc} from \ili{Romanian}. This contrast deserves detailed attention and we leave it for further research.

	\ea \label{moremb} \ea[??]{\gll Ion o fi la birou mai degrabă decât acasă.\\
		Ion \textsc{presm.3.sg} be at office more soon than home\\
		\glt `I guess Ion is in the office rather than at home.'} \label{moremba}
		\ex[]{\gll Bănuiesc că Ion o fi la birou mai degrabă decât acasă.\\
		guess.\textsc{1.sg} that Ion \textsc{presm.3.sg} be at office more soon than home\\
		\glt `I guess Ion is in the office rather than at home.'} \label{morembb}
		\ex[]{\gll Cred că Ion o fi la birou mai degrabă decât acasă.\\
		believe.\textsc{1.sg} that Ion \textsc{presm.3.sg} be at office more soon than home\\
		\glt `I believe Ion is in the office rather than at home.' \hfill (\ili{Romanian}) \label{morembc}}
        \z \z

\noindent Finally, one of the anonymous reviewers makes the interesting observation that \ili{Polish} ECs are impossible in negated future contexts. The same point can be made using \ili{Romanian} data, as seen below:

\ea[??]{\gll Ion nu va fi la birou  mai degrabă decât acasă.\\
	Ion not \textsc{fut} be at office more 		soon than home\\
\glt Intended: `Ion will not be in the office rather than at home.' \hfill (\ili{Romanian})}
	\z

\noindent In \ili{Bulgarian} similar examples seem to be well-formed, see \REF{bulgfut}.

\ea \gll Ivan ne šte da byde na rabota, a po-skoro v kušti.\\
    Ivan not \textsc{fut} \textsc{da} be at work but more-soon at home\\
    \glt `Ivan will not be at work rather than at home.'  \hfill (\ili{Bulgarian}) \label{bulgfut}
    \z


\noindent However, the future marker \textit{šte} in \ili{Bulgarian} has been shown to be a versatile category with various types of interpretations \citep{rivsimeo14}. Thus, more refined diagnostics are needed to settle this problem. We leave a detailed account of this observation for further research.



\nopagebreak
\section*{Abbreviations}

\begin{tabularx}{.47\textwidth}{ll}
\textsc{adv}&adverb\\
\textsc{cond}&conditional\\
\textsc{da}&{modal} particle\\
{EC}&{epistemic} {comparative}\\
{Epist}&{epistemic}\\
{Evid}/\textit{evid}&{evidential}\\
\textsc{fut}&future\\
\textsc{ger}&gerund\\
{IE}&indirect {evidential}\\
{IIE}&inferential indirect\\
& {evidential}\\
\end{tabularx}
\begin{tabularx}{.47\textwidth}{ll}
\textsc{inf}&{infinitive}\\
{MAdv}&{modal} adverbs\\
{MAdjs}&{modal} adjectives\\
\textsc{pfv}&{perfective}\\
\textsc{pl}&plural\\
\textsc{presmp}&presumptive\\
{SA}&{speech act}\\
\textsc{pst}&past\\
\textsc{ptcp}&{participle}\\
\textsc{sbjv}&{subjunctive}\\
\textsc{sg}&singular\\
\end{tabularx}

\section*{Acknowledgements}
We would like to thank Roumyana Pancheva and Dimitar Kazakov for discussing the Bulgarian data with us. We are grateful to the audience of FDSL 12.5 and two anonymous reviewers for their helpful comments and questions. For Julie Goncharov, this research project was financially supported by the State of Lower-Saxony, Hannover, Germany (VWZN3181). For Monica Alexandrina Irimia, parts of this research have been funded by a grant from the University of Modena and Reggio Emilia. All errors are our own.

\sloppy
\printbibliography[heading=subbibliography,notkeyword=this]

\il{Bulgarian|)}
\end{document}
