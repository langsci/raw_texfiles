\documentclass[output=paper,colorlinks,citecolor=brown]{langscibook}
\ChapterDOI{10.5281/zenodo.13759978}
\title{Introduction} 
\author{Olga Kellert\orcid{}\affiliation{University of Göttingen} and Malte Rosemeyer\orcid{}\affiliation{Freie Universität Berlin}}

\abstract{In this introduction to the edited volume, we establish the theoretical framework for the synchronic and diachronic study of indefinites in Romance language varieties. Due to their flexibility in interpretation, the use of Romance indefinites is highly variable and subject to dynamic processes of language change. The present volume addresses fundamental linguistic questions about language variation and change in Romance indefinites. It focuses on quantificational expressions in language varieties that have not received much attention in the previous literature, such as Old Sardinian, Argentinian Spanish, Palenquero Creole and Cabindan Portuguese, Catalan, Romanian, and others. The studies united in this volume offer new data on these processes of variation and change.}

\IfFileExists{../localcommands.tex}{
   \addbibresource{../localbibliography.bib}
   % add all extra packages you need to load to this file

\usepackage{tabularx,multicol}
\usepackage{url}
\urlstyle{same}

\usepackage{listings}
\lstset{basicstyle=\ttfamily,tabsize=2,breaklines=true}

\usepackage{langsci-basic}
\usepackage{langsci-optional}
\usepackage{langsci-lgr}
\usepackage{langsci-osl}
% \usepackage{./langsci/styles/langsci-lgr}
% \usepackage{./langsci/styles/langsci-osl}
% \usepackage{langsci-gb4e}

\usepackage{tikz}
\usetikzlibrary{patterns,calc}
\pgfdeclarepatternformonly{south east lines}{\pgfqpoint{-0pt}{-0pt}}{\pgfqpoint{3pt}{3pt}}{\pgfqpoint{3pt}{3pt}}{
    \pgfsetlinewidth{0.6pt}
    \pgfpathmoveto{\pgfqpoint{0pt}{3pt}}
    \pgfpathlineto{\pgfqpoint{3pt}{0pt}}
    \pgfpathmoveto{\pgfqpoint{.2pt}{-.2pt}}
    \pgfpathlineto{\pgfqpoint{-.2pt}{.2pt}}
    \pgfpathmoveto{\pgfqpoint{3.2pt}{2.8pt}}
    \pgfpathlineto{\pgfqpoint{2.8pt}{3.2pt}}
    \pgfusepath{stroke}}
    
\usepackage{stmaryrd}
\usepackage{wasysym}
\usepackage{multirow}
\usepackage{caption}
\usepackage{subcaption}
\usepackage{mathrsfs}
\usepackage{qtree}

\usepackage{linguex}


   %pminos do not split footnotes
% \interfootnotelinepenalty=10000 %Footnote in Laporte chapters has to be split SN


%\DeclareIndexNameFormat{default}{%
%\nameparts{#1}%
%\usebibmacro{index:name}%
%{\index[names]}%
%{\namepartfamily}%
%{\namepartgiveni}%
% {}% L1
% {}% L2
%{\namepartprefix}% generates spurious space L3
%{\namepartsuffix}% generates spurious space L4
%}

%  {\DeclareIndexNameFormat{default}{%
%     \usebibmacro{index:name}{\index[names]}{#1}{#3}{#5}{#7}}}

%\DeclareIndexNameFormat{default}{%
%  \usebibmacro{index:name}{\sindex[nom]}{#1}{#3}{#5}{#7}}

%\DeclareIndexNameFormat{default}{%
%  \usebibmacro{index:name}{\sindex[person]}{#1}{#3}{#5}{#7}}
%\DeclareIndexNameFormat{default}{%
%\nameparts{#1} \usebibmacro{index:name}{\sindex[person]]}{\namepartfamily}{‌​\namepartgiven}{\nam‌​epartprefix}{\namepa‌​rtsuffix}}

%\newcommand{\smiley}{:)}

%\renewbibmacro*{index:name}[5]{%
%\usebibmacro{index:entry}{#1}%
%{\iffieldundef{usera}{}{\thefield{usera}\actualoperator}\mkbibindexname{#2}{#3}{#4}{#5}}}

% \newcommand{\noop}[1]{}

%remove for final
%\overfullrule=1mm

\newcommand{\tobi}[2]}}
\renewcommand{\S}[1]{\tobi{#1}{\textsc{*}}}

% this volume references
% puts: [this volume]
% already defined: \citetv
%\newcommand{\citepv}[1]{(\citeauthor{#1} \citeyear*{#1} [this volume])}
\newcommand{\citealtv}[1]{\citeauthor{#1} \citeyear*{#1} [this volume]}

%parentheses around example number
\newcommand{\pref}[1]{(\ref{#1})}

% in-text examples

\newcommand{\lnex}[1]{\textit{#1}} %target lang word
\newcommand{\lnlit}[1]{(lit.: `#1')} %literal reading
\newcommand{\lnlat}[1]{(#1)} % latinization
\newcommand{\lntrans}[1]{`#1'} %translation
\newcommand{\lnexl}[2]%
{\lnex{#1}{} \lnlat{#2}} % ex with latinization
\newcommand{\lnexlat}[3]{\lnex{#1}{} \lnlat{#2}{} \lntrans{#3}} % ex with latinization and tranl.

%ch01
\newcommand{\co}[1]{\mbox{\textbf{#1}}}

%ch09

\newcommand{\cyrbulg}[1]{\begin{otherlanguage*}{bulgarian}#1\end{otherlanguage*}}


%ch10
\newcommand{\nlp}{{\small NLP}}
\newcommand{\mwe}{{\small MWE}}
\newcommand{\rae}{{\small RAE}}
\newcommand{\lvc}{{\small LVC}}
\newcommand{\pos}{{\small P}o{\small S}}
%\newcommand{\todo}[1]{ \textcolor{red}{#1} }

%\renewcommand{\labelenumi}{\theenumi}
%\ainamefmt{{vv}{ll}{, ff}{, jj}} % fullname

\newcommand{\biberror}[1]{{\color{red}#1}}

\newcommand{\osenovaitem}{--~}
   %% hyphenation points for line breaks
%% Normally, automatic hyphenation in LaTeX is very good
%% If a word is mis-hyphenated, add it to this file
%%
%% add information to TeX file before \begin{document} with:
%% %% hyphenation points for line breaks
%% Normally, automatic hyphenation in LaTeX is very good
%% If a word is mis-hyphenated, add it to this file
%%
%% add information to TeX file before \begin{document} with:
%% %% hyphenation points for line breaks
%% Normally, automatic hyphenation in LaTeX is very good
%% If a word is mis-hyphenated, add it to this file
%%
%% add information to TeX file before \begin{document} with:
%% \include{localhyphenation}
\hyphenation{
    Beck-man
    Ngu-yen
    back-chan-nel
    back-chan-nels
    mo-not-o-nous
    ste-reo-typ-i-cal
}

\hyphenation{
    Beck-man
    Ngu-yen
    back-chan-nel
    back-chan-nels
    mo-not-o-nous
    ste-reo-typ-i-cal
}

\hyphenation{
    Beck-man
    Ngu-yen
    back-chan-nel
    back-chan-nels
    mo-not-o-nous
    ste-reo-typ-i-cal
}

   \boolfalse{bookcompile}
   \togglepaper[23]%%chapternumber
}{}

\begin{document}
\maketitle
\noindent \begin{sloppypar}
Indefinites are commonly described as linguistic elements that are used to present a referent in their scope as discourse-new \citep[1--15]{Lyons1999}. Consequently, indefinites typically occur in presentational (\ref{ex:kr1a}--c) and existential (\ref{ex:kr1d}) contexts. They can pertain to different syntactic classes, such as indefinite articles (\ref{ex:kr1a}), indefinite pronouns (\ref{ex:kr1b}), indefinite quantifiers (\ref{ex:kr1c}) and bare nouns (\ref{ex:kr1d}) (cf. also \citealt{Koch2012}).
\end{sloppypar}

\ea\label{ex:kr1}
    \ea\label{ex:kr1a} There is a book on the table.
    \ex\label{ex:kr1b} There is somebody outside of the house.
    \ex\label{ex:kr1c} There are some snakes in the building.
    \ex\label{ex:kr1d} There are snakes in Latin America.
    \z
\z

Indefinites have received much attention in the semantic literature, due to the fact that their interpretation may differ in terms of specificity. Consider, for instance, example (\ref{ex:kr2}), taken from \citet{FodorSag1982}; whereas in the context of (\ref{ex:kr2a}) the speaker appears to have had a specific referent in mind (`John') when uttering the first sentence, in (\ref{ex:kr2b}) she did not (cf. also \citealt{VonHeusinger2002}).

\ea\label{ex:kr2} A student in Syntax 1 cheated on the exam.
    \ea\label{ex:kr2a} His name is John.	[specific interpretation]
    \ex\label{ex:kr2b} We are all trying to figure out who it was. [non-specific interpretation]
    \z
\z

In his seminal work on the functions of indefinite pronouns, \citet[64]{Haspelmath1997} established an implicational hierarchy for English that has been taken up in many subsequent studies (see, for instance, \citealt{AloniPort2010} for an updated version). Among other things, the hierarchy predicts that uses of indefinite pronouns in which the referent is specific and known by the speaker (\ref{ex:kr3a}) are more basic than pronouns in which the referent is specific and unknown (\ref{ex:kr3b}), which in turn are more basic than non-specific irrealis uses (\ref{ex:kr3c}) and free-choice pronouns (\ref{ex:kr3d}) (examples from \citealt[3]{Haspelmath1997}).

\ea\label{ex:kr3}
    \ea\label{ex:kr3a} Somebody called while you were away: guess who!
    \ex\label{ex:kr3b} I heard something, but I couldn't tell what kind of sound it was.
    \ex\label{ex:kr3c} Please try somewhere else.
    \ex\label{ex:kr3d} Anybody can solve this simple problem.
    \z
\z

Haspelmath's implicational hierarchy combines semantic and syntactic parameters (e.g. appearance in the protasis of a conditional clause or comparatives). Recent studies have extended this description of indefinites by including further features such as plurality, scalarity and modality (see \citealt{Chierchia2006, Alonso-OvalleMenéndez-Benito2015, Falaus2015, Falaus2018, Kellert2021a}). Many of these studies use data from Romance languages. For instance, \citet{Chierchia2006} demonstrates that free choice items may display a quantificational force; in the Italian example (\ref{ex:kr4}), it is understood that the subject referent knocked all (not just some) doors with wooden shutters. Likewise, \citet{Alonso-OvalleMenéndez-Benito2015} maintain that indefinites such as Spanish \textit{cualquiera} carry a modal meaning component; a sentence such as (\ref{ex:kr5}) conveys that (a) María bought a gift and (b) she could have bought any gift (the modal meaning component). 

\ea\label{ex:kr4} 
\gll Mi sono me-sso a buss-are come un matto \textbf{a} \textbf{qualsiasi} \textbf{porta} con i battenti in legno.\\
     to.me	be.\textsc{prs.ind.1sg} start-\textsc{ptcp} to knock-\textsc{inf} like a madman to any door with the shutter in wood\\
\glt ‘I started knocking like a madman at any door with a wooden shutter.’
\z

\ea\label{ex:kr5} 
\gll María compr-ó \textbf{un} \textbf{regalo} \textbf{cualquiera}.\\
     Maria buy-\textsc{pst.pfv.3sg} a gift any\\
\glt ‘María bought a random gift.’
\z

Examples such as (\ref{ex:kr4}) and (\ref{ex:kr5}) demonstrate that that free/random choice elements such as Italian \textit{qualsiasi}, \textit{qualunque} and Spanish \textit{cualquiera} differ in their distribution from English indefinites such as \textit{any} or \textit{random}. A more obvious example of differences between Romance languages and English is the use of articles. For instance, partitives such as French \textit{des}, \textit{du} etc. can be translated into English using indefinites like \textit{some} (\ref{ex:kr6a}) or simply left out (\ref{ex:kr6b}). Here, too, we find variation between and within the various Romance languages (see \cite{GiustiCardinaletti2018}). As summarized in \citet{CarlierLamiroy2014}, the partitive is frequently optional in Italian, meaning that examples such as (\ref{ex:kr6a}) can be translated into Italian with or without using partitive articles (\ref{ex:kr7}). Given that “\textit{de} indicates that the referent of the NP is not wholly affected by the verbal action but only partially” \citep[55]{Carlier2013}, the difference between (\ref{ex:kr7a}) and (\ref{ex:kr7b}) might be that whereas in (\ref{ex:kr7a}), only a part of the available spinach was bought, (\ref{ex:kr7b}) is compatible with a situation in which all of the available spinach was bought.

\ea
    \ea\label{ex:kr6a} 
    \gll J'ai achet-é \textbf{des} \textbf{épinards}.\\
     I'have.\textsc{prs.1sg} buy-\textsc{ptcp} of.the spinach\\
    \glt ‘I bought some spinach.’\\
    \ex\label{ex:kr6b}
    \gll J'ai achet-é \textbf{du} \textbf{pain}.\\
     I'have.\textsc{prs.1sg} buy-\textsc{ptcp} of.the bread\\
\glt ‘I bought some bread.’ 
    \z
\ex\label{ex:kr7} 
    \ea\label{ex:kr7a} 
    \gll Ho compr-ato \textbf{dello} \textbf{spinacio}.\\
     have.\textsc{prs.1sg} buy-\textsc{ptcp} of.the spinach\\
    \glt ‘I bought some spinach.’\\
    \ex\label{ex:kr7b}
    \gll Ho compr-ato \textbf{spinacio}.\\
     have.\textsc{prs.1sg} buy-\textsc{ptcp} spinach\\
    \glt ‘I bought some spinach.’\\
    \z
\z

Likewise, substantial variation can be found regarding the interpretation of indefinite and definite articles. To give but one example, definite articles are generally less productive in Brazilian Portuguese than in other Romance languages and English, as bare nouns can have both generic (\ref{ex:kr8}) and specific-definite (\ref{ex:kr9}) interpretations. It stands to reason that the semantics of Brazilian Portuguese definite articles is more restricted to specific-definite interpretations than in other Romance languages. Another example is variation in the expression of the personal-impersonal distinction (see the papers united in \citealt{Posio2023}).

\ea\label{ex:kr8}
    \gll \textbf{Pedreiro} é preguiçoso.\\
     Bricklayer be.\textsc{prs.ind.3sg} lazy \\
    \glt ‘Bricklayers are lazy.’ (\citealt[108]{Dobrovie-SorinPires2008}) \\
\z

\ea\label{ex:kr9}
    \gll \textbf{Quintal} é cheio de terra.\\
     garden be.\textsc{prs.ind.3sg} full of soil \\
    \glt ‘The garden is full with soil.’ (\citealt[236]{Wall2013}) \\
\z

Finally, definiteness and indefiniteness also play a role in the verbal domain. For instance, some Romance languages allow the use of determiners before infinitives, as in (\ref{ex:kr10}) (both examples are taken from \cite{Rosemeyer2012}). The parameter of definiteness appears to be crucially related to whether or not the noun phrase is interpreted as an event (\ref{ex:kr10a}) or a fact (\ref{ex:kr10b}) (\citealt{Varela1979, DeMiguel1996, DemonteVarela1996, Vanderschueren2013, Schirakowski2021}), with indefinites constrained to the eventive interpretation.

\ea\label{ex:kr10} 
    \ea\label{ex:kr10a} 
\gll Lo devuelve a la realidad \textbf{un} \textbf{cruj-ir} \textbf{de} \textbf{hojas} \textbf{sueltas}\\
     him return.\textsc{prs.ind.3sg} to the reality a crackle-\textsc{inf} of leaves loose\\
\glt ‘A crackle of loose leaves brings him back to reality.’\\
    \ex\label{ex:kr10b} 
\gll ha sido un poco frustrante \textbf{el} \textbf{no} \textbf{haber} \textbf{podido} \textbf{dilucid-ar} \textbf{el} \textbf{mecanismo} \textbf{responsable} \textbf{de} \textbf{la} \textbf{superconductividad}\\
    have.\textsc{prs.ind.3sg} be.\textsc{ptcp} a bit frustrating the not have.\textsc{inf} can.\textsc{ptcp} explain-\textsc{inf} the mechanism responsible for the superconductivity \\
\glt ‘it has been a little frustrating that we have not been able to explain the mechanism responsible for superconductivity’
\z
\z

Likewise, the type of a noun can have a crucial impact on the meaning of its governing verb. For instance, the type of meaning expressed by the present perfect in Indo-European languages crucially hinges on the definiteness of the direct object. Thus, whereas (\ref{ex:kr11a}) is likely to be interpreted as a resultative perfect, an experiential reading seems more natural for (\ref{ex:kr11b}). This is due to the fact that the difference between resultative and experiential readings is governed by whether or not the resultant state still holds at event time \citep[155--156]{IatridouAnagnostopoulou2003}. Whereas the use of (\ref{ex:kr11a}) seems plausible in a situation in which the glasses are still lost, the pluractionality of (\ref{ex:kr11b}), understood as reference to multiple events \citep{VanGeenhoven2004}, appears to make such a reading implausible here. From a discourse perspective, one might argue that the resultant state of the events in (\ref{ex:kr11b}) is less likely to be relevant at speech time than the resultant state of the event in (\ref{ex:kr11a}).

\ea\label{ex:kr11} 
    \ea\label{ex:kr11a} 
\gll He perd-ido \textbf{mis} \textbf{gafas}.\\
     have lost-\textsc{ptcp} my glasses\\
\glt ‘I have lost my glasses.’\\
    \ex\label{ex:kr11b} 
\gll He perd-ido \textbf{muchas} \textbf{gafas}.\\
     have lost-\textsc{ptcp} many glasses\\
\glt ‘I have lost many glasses.’\\
\z
\z

Our brief exemplary survey has demonstrated that the semantics of Romance indefinites has been studied intensely both in the nominal and verbal domain. Far less attention has been given to the variation in terms of the differences in the use of indefinite elements between and within Romance languages. As of yet, there is no systematic comparative account of the use of indefinite elements in Romance.

To give but one example for such variation, in Brazilian Portuguese, indefinite articles are sometimes used in contexts in which no article would be used in English (see example \ref{ex:kr12} from Wall, p.c.). Notably, the indefinite determiner \textit{uma} in (\ref{ex:kr12}) would be translated into Spanish using the definite determiner \textit{la} (\ref{ex:kr13}). 

\ea\label{ex:kr12} 
\gll Ao fim da tarde eu gost-o de assist-ir \textbf{uma} \textbf{televisão}.\\
     at.the end of.the afternoon I like-\textsc{prs.ind.1sg} of watch-\textsc{inf} a television\\
\glt ‘In the evening I like to watch television.'\\
\z

\ea\label{ex:kr13} 
\gll En la tarde me gust-a mirar \textbf{la} \textbf{tele}\\
     in the afternoon me like-\textsc{prs.ind.3sg} watch-\textsc{inf} the television\\
\glt ‘In the evening I like to watch television.’\\
\z

We also document a distinct lack of studies on the historical development of Romance indefinites. Most analyses focus on processes by which indefinite articles in Romance were created or grammaticalized from transparent lexical words. For instance, \citet{BergarechePérez-Saldanya2011} demonstrate how the Old Catalan adjective \textit{diversos} (\ref{ex:kr14}) gradually experienced a category shift towards the use as an indefinite article (\ref{ex:kr15}) (examples from \citealt{BergarechePérez-Saldanya2011}). 

\ea\label{ex:kr14} 
\gll Los tartres són molt diversos de les altres gents, de manera i de custums\\
     the tartars be.\textsc{prs.ind.3pl} very different from the other people of manner and of habits\\
\glt ‘The Tartars are very different from other people in manners and habits’\\
\z

\ea\label{ex:kr15} 
\gll No és dupte que diversos altres sants no hagen fetes moltes altres semblants e mejors abstinències\\
     not be.\textsc{prs.ind.3sg} doubt that various other saints not have.\textsc{prs.sbj.3pl} do.\textsc{ptcp} many other similar and greater abstinences\\
\glt ‘There is no question that various other saints haven't performed many other similar and greater abstinences’\\
\z

A similar process has been posited for partitives, which cannot be described as determiners in Old French \citep{Carlier2013}, quantifiers such as Latin \textit{aliquis} ‘some’, \textit{nullus} ‘no’, and \textit{nemo} ‘no one’ \citep{Gianollo2018}, and free choice indefinites such as \textit{qualsiasi} \citep{DeganoAloni2021}. However, as mentioned by \citet[2]{DeganoAloni2021}, extremely little research has been done in this area. Most studies on the diachrony of quantifiers focus on the grammaticalization of quantifiers from transparent lexical elements into grammaticalized quantifiers (\citealt{Haspelmath1997, Verveckken2015, Gianollo2018, CompanyCompanyPozasLoyo2009}). Few studies concentrate on possible processes of degrammaticalization of quantifiers. 

The present volume fills this lacuna in the description of Romance languages, analyzing synchronic and diachronic data from French, Spanish, Italian, Portuguese, Catalan, Romanian, Sardinian, and Palenquero. The present studies describe variation in meaning and syntactic format of indefinites in the nominal and verbal domain and offer new data as to the historical development of these constructions. The contributions to this volume address the following questions:

\begin{itemize}
\item Which semantic and syntactic parameters are relevant to descriptions of the distribution of Romance indefinites?
\item Are their semantic properties lexically encoded or do they result from being used in a specific syntactic and/or pragmatic context?  
\item Which language-specific differences in the use of indefinite elements such as French \textit{quelques}\slash Italian \textit{alcuni}\slash Spanish \textit{algunos} `some' can be observed and how can these differences be explained?
\item Can we identify historical pathways of evolution of indefinites across Romance languages? Can we document word class changes (adjective > determiner, preposition > determiner, etc.)? 
\item Which Romance indefinites are the result of a (de)grammaticalization process? Is the process of grammaticalization the same in all Romance languages? 
\end{itemize}

The first three papers in the present volume establish comparative perspectives on the use of Romance indefinites. In their paper “Romanian \textit{niște} between non-specific and specific interpretations”, Jan Davatz and Elisabeth Stark analyze the meaning of the Romanian indefinite \textit{niște} ‘some’. Intriguingly, \textit{niște} is used in similar contexts as French and Italian partitive articles, which raises the question of the categoriality of this element. Davatz and Stark use a questionnaire study to provide a fine-grained description of the semantic and syntactic properties of \textit{niște}. Their results demonstrate that \textit{niște} does indeed share some properties with Italian partitive articles, such as its optionality, and lack of usage with preverbal subjects in generic contexts. However, \textit{niște} differs decisively from partitive articles in terms of its scopal properties and specificity. In particular, \textit{niște} can have scope over negation, and its use is frequent in anaphorical contexts, where the referent has already been introduced. The authors consequently propose to analyze \textit{niște} as a specificity marker, with properties unlike other Romance indefinite determiners. Their analysis also suggests a similarity and, possibly, competition between \textit{niște} and the Romanian differential object marker \textit{pe}.

The second paper “Argentinian Spanish \textit{cualunque} and Italian \textit{qualunque}”, by Marika Francia and Olga Kellert, addresses the question of language contact and language change. The authors discuss the difference between the Argentinian Spanish \textit{cualunque}, which has its origin in the Italian free choice indefinite \textit{qualunque}. They show that the Argentinian Spanish item has changed its meaning and syntactic category to an evaluative adjective with the meaning ‘ordinary’. This change from a functional category of an indefinite into a lexical category of an adjective is particularly interesting as it attests a case of degrammaticalization. Previous research has mainly focused on which categories and elements change from one quantificational category into another \citep{BergarechePérez-Saldanya2011, Verveckken2015} and on how quantifiers grammaticalize \citep{Haspelmath1997, CompanyCompanyPozasLoyo2009}. However, the question as to how quantifiers degrammaticalize has received little attention so far. The authors argue that this case of degrammaticalization is the result of a pragmatic implicature produced in specific contexts, which has caused the semantic meaning and syntactic category shift of \textit{cualunque}.

\begin{sloppypar}
The paper “Indefinite pronouns with \textsc{thing} and \textsc{person} in two Ibero-Romance\slash Kikongo varieties: Palenquero Creole and Cabindan Portuguese”, by Miguel Gutiérrez Maté, is also concerned with the question of language contact. Gutiérrez Maté studies the usage of the indefinites \textit{kusa} ‘thing’ and \textit{hende} ‘people’ in Palenquero, a Spanish-based creole spoken in San Basilio de Palenque (Colombia), on the basis of data from his own fieldwork. His analysis shows that while \textit{kusa} and \textit{hende} originally expressed quantificational readings, they are evolving into indefinite pronouns. Although this grammaticalization pathway is extremely common in creoles, Gutiérrez Maté argues that the distribution of \textit{kusa} and \textit{hende} cannot be explained in terms of language acquisition universals. In contrast, he proposes an explanation in terms of the original substrate languages of Palenquero, Kikongo (Bantu, Sub-Saharan), where the same processes are attested. Additional evidence for this assumption is given on the basis of a corpus-based analysis of Cabindan Portuguese (Angola). Gutiérrez Maté is unable to document the grammaticalization process attested for Palenquero indefinites in Cabindan Portuguese. This result is expected, given that no Portuguese-based creole has evolved in Angola, and lends further credibility to a substratist explanation.
\end{sloppypar}

Language contact is also an important issue for the three papers in this collective volume that analyze historical processes of change in the domain of Romance indefinites. In “On the diachrony of Catalan indefinite \textit{qualsevol}”, Olga Kellert and Andrés Enrique-Arias investigate the diachronic development of the Catalan indefinite \textit{qualsevol}. They refute the existent hypothesis according to which the Catalan indefinite \textit{qualsevol} is a loanword from a Latin indefinite and show instead that this indefinite has been grammaticalized into an indefinite from transparent lexical elements, similar to Spanish \textit{cualquiera}, as shown by \citet{CompanyCompanyPozasLoyo2009} and \citet{Kellert2021a, Kellert2021b}. The authors analyze the grammaticalization hypothesis on different levels: morphological agreement, syntactic and semantic, and show that there is strong evidence for the grammaticalization hypothesis on all three levels.     

The paper “Indefinites and quantifiers in Old Sardinian: A corpus-based study”, by Guido Mensching, establishes a systematic description of the inventory of Old Sardinian indefinites and quantifiers on the basis of corpus data, filling a lacuna in research on Sardinian and Romance. Mensching is particularly interested in the question of the influence of the superstratum languages Italian, Spanish and Catalan on Old Sardinian indefinites, as well as the interaction between syntactic and semantic parameters in their usage. Mensching demonstrates how a detailed analysis of the semantics and distribution of Old Sardinian indefinites can shed light on the possible origin of these forms; he argues, contra previous studies, that negative indefinites such as \textit{nullu} and \textit{perunu} ‘no (x)’ cannot be Italian loan words, whereas the quantifier \textit{cada} must be a loan from Spanish and Catalan, and \textit{omnia} ‘every/each’ is a Latinism. One crucial finding that supports his idea concerning negative indefinites is that while Old Sardinian was a strict negative concord language, borrowed negative indefinites frequently show a lack of negative concord in preverbal position. His analysis also uncovers historical processes of change in the system of Old Sardinian indefinites and quantifiers. In particular, the data suggest a gradual loss of agreement for the quantifier \textit{tot(t)u} ‘all/whole’. 

In the final paper of this volume, entitled “The Brazilian Portuguese present perfect: From nominal to verbal pluractionality”, Malte Rosemeyer and Martin Becker analyze the semantic change undergone by the present perfect in Brazilian Portuguese (BP), a compound tense, in a diachronic corpus of BP theater texts. On the basis of a previous analysis by \citet{AmaralHowe2012}, the authors hypothesize that the reanalysis of the perfect occurred in transitive contexts with a direct object that is inflected for masculine and singular, but can still be interpreted as expressing plural. Later, the interpretative property of pluractionality (originally derived from the nominal complement) came to be conventionally associated with the use of the perfect, leading to the readings that are typical for today’s use of the BP perfect. Their bottom-up approach towards calculating the likelihood for a given context of the present perfect to express pluractional readings confirms this hypothesis. In addition, their analysis reveals that register variation had an important influence on this change, leading them to hypothesize that the change towards pluractional readings in the BP present perfect was facilitated by intensive contact with European Portuguese during the second half of the 19\textsuperscript{th} century.

{\sloppy\printbibliography[heading=subbibliography,notkeyword=this]}
\end{document}
