\documentclass[output=paper,colorlinks,citecolor=brown]{langscibook}
\ChapterDOI{10.5281/zenodo.13759992}
\title[The Brazilian Portuguese present perfect]{The Brazilian Portuguese present perfect: From nominal to verbal pluractionality} 
\author{Malte Rosemeyer\orcid{}\affiliation{Freie Universität Berlin} and Martin Becker\orcid{}\affiliation{University of Cologne}}

%% Command for possessive-"'s" structures in bibliographical references 
\newcommand{\citeposs}[1]{\citeauthor{#1}'s (\citeyear{#1})}
\label{chapter/07}

\abstract{We analyze the semantic change undergone by the present perfect in Brazilian Portuguese (BP). While the present perfect in earlier stages of Portuguese is usually described as a resultative perfect, the construction has acquired iterative and durative readings in Modern Portuguese. We analyze the distribution of more than $n = 850$ occurrences of the present perfect in a diachronic corpus of BP theater texts, ranging from the 19\textsuperscript{th} until the 21\textsuperscript{st} century. We develop a bottom-up approach towards calculating the likelihood for a given context of the present perfect to express pluractional readings in the nominal and verbal domains. We measure the correlations between this parameter and time, as well as register. In doing so, we are able to establish a more precise model of the semantic change experienced by the present perfect in BP. The results from the analysis provide empirical evidence for \citeposs{AmaralHowe2012} claim that the reanalysis of the present perfect involved a transfer from nominal to verbal pluractionality. Additionally, our analysis reveals that register had an important influence on this change. This finding leads us to hypothesize that the change towards pluractional readings in the BP present perfect was facilitated by intensive contact with European Portuguese during the second half of the 19\textsuperscript{th} century, i.e. the so-called period of the Império do Brasil (‘Empire of Brazil’).}


\IfFileExists{../localcommands.tex}{
   \addbibresource{../localbibliography.bib}
   % add all extra packages you need to load to this file

\usepackage{tabularx,multicol}
\usepackage{url}
\urlstyle{same}

\usepackage{listings}
\lstset{basicstyle=\ttfamily,tabsize=2,breaklines=true}

\usepackage{langsci-basic}
\usepackage{langsci-optional}
\usepackage{langsci-lgr}
\usepackage{langsci-osl}
% \usepackage{./langsci/styles/langsci-lgr}
% \usepackage{./langsci/styles/langsci-osl}
% \usepackage{langsci-gb4e}

\usepackage{tikz}
\usetikzlibrary{patterns,calc}
\pgfdeclarepatternformonly{south east lines}{\pgfqpoint{-0pt}{-0pt}}{\pgfqpoint{3pt}{3pt}}{\pgfqpoint{3pt}{3pt}}{
    \pgfsetlinewidth{0.6pt}
    \pgfpathmoveto{\pgfqpoint{0pt}{3pt}}
    \pgfpathlineto{\pgfqpoint{3pt}{0pt}}
    \pgfpathmoveto{\pgfqpoint{.2pt}{-.2pt}}
    \pgfpathlineto{\pgfqpoint{-.2pt}{.2pt}}
    \pgfpathmoveto{\pgfqpoint{3.2pt}{2.8pt}}
    \pgfpathlineto{\pgfqpoint{2.8pt}{3.2pt}}
    \pgfusepath{stroke}}
    
\usepackage{stmaryrd}
\usepackage{wasysym}
\usepackage{multirow}
\usepackage{caption}
\usepackage{subcaption}
\usepackage{mathrsfs}
\usepackage{qtree}

\usepackage{linguex}


   %pminos do not split footnotes
% \interfootnotelinepenalty=10000 %Footnote in Laporte chapters has to be split SN


%\DeclareIndexNameFormat{default}{%
%\nameparts{#1}%
%\usebibmacro{index:name}%
%{\index[names]}%
%{\namepartfamily}%
%{\namepartgiveni}%
% {}% L1
% {}% L2
%{\namepartprefix}% generates spurious space L3
%{\namepartsuffix}% generates spurious space L4
%}

%  {\DeclareIndexNameFormat{default}{%
%     \usebibmacro{index:name}{\index[names]}{#1}{#3}{#5}{#7}}}

%\DeclareIndexNameFormat{default}{%
%  \usebibmacro{index:name}{\sindex[nom]}{#1}{#3}{#5}{#7}}

%\DeclareIndexNameFormat{default}{%
%  \usebibmacro{index:name}{\sindex[person]}{#1}{#3}{#5}{#7}}
%\DeclareIndexNameFormat{default}{%
%\nameparts{#1} \usebibmacro{index:name}{\sindex[person]]}{\namepartfamily}{‌​\namepartgiven}{\nam‌​epartprefix}{\namepa‌​rtsuffix}}

%\newcommand{\smiley}{:)}

%\renewbibmacro*{index:name}[5]{%
%\usebibmacro{index:entry}{#1}%
%{\iffieldundef{usera}{}{\thefield{usera}\actualoperator}\mkbibindexname{#2}{#3}{#4}{#5}}}

% \newcommand{\noop}[1]{}

%remove for final
%\overfullrule=1mm

\newcommand{\tobi}[2]}}
\renewcommand{\S}[1]{\tobi{#1}{\textsc{*}}}

% this volume references
% puts: [this volume]
% already defined: \citetv
%\newcommand{\citepv}[1]{(\citeauthor{#1} \citeyear*{#1} [this volume])}
\newcommand{\citealtv}[1]{\citeauthor{#1} \citeyear*{#1} [this volume]}

%parentheses around example number
\newcommand{\pref}[1]{(\ref{#1})}

% in-text examples

\newcommand{\lnex}[1]{\textit{#1}} %target lang word
\newcommand{\lnlit}[1]{(lit.: `#1')} %literal reading
\newcommand{\lnlat}[1]{(#1)} % latinization
\newcommand{\lntrans}[1]{`#1'} %translation
\newcommand{\lnexl}[2]%
{\lnex{#1}{} \lnlat{#2}} % ex with latinization
\newcommand{\lnexlat}[3]{\lnex{#1}{} \lnlat{#2}{} \lntrans{#3}} % ex with latinization and tranl.

%ch01
\newcommand{\co}[1]{\mbox{\textbf{#1}}}

%ch09

\newcommand{\cyrbulg}[1]{\begin{otherlanguage*}{bulgarian}#1\end{otherlanguage*}}


%ch10
\newcommand{\nlp}{{\small NLP}}
\newcommand{\mwe}{{\small MWE}}
\newcommand{\rae}{{\small RAE}}
\newcommand{\lvc}{{\small LVC}}
\newcommand{\pos}{{\small P}o{\small S}}
%\newcommand{\todo}[1]{ \textcolor{red}{#1} }

%\renewcommand{\labelenumi}{\theenumi}
%\ainamefmt{{vv}{ll}{, ff}{, jj}} % fullname

\newcommand{\biberror}[1]{{\color{red}#1}}

\newcommand{\osenovaitem}{--~}
   %% hyphenation points for line breaks
%% Normally, automatic hyphenation in LaTeX is very good
%% If a word is mis-hyphenated, add it to this file
%%
%% add information to TeX file before \begin{document} with:
%% %% hyphenation points for line breaks
%% Normally, automatic hyphenation in LaTeX is very good
%% If a word is mis-hyphenated, add it to this file
%%
%% add information to TeX file before \begin{document} with:
%% %% hyphenation points for line breaks
%% Normally, automatic hyphenation in LaTeX is very good
%% If a word is mis-hyphenated, add it to this file
%%
%% add information to TeX file before \begin{document} with:
%% \include{localhyphenation}
\hyphenation{
    Beck-man
    Ngu-yen
    back-chan-nel
    back-chan-nels
    mo-not-o-nous
    ste-reo-typ-i-cal
}

\hyphenation{
    Beck-man
    Ngu-yen
    back-chan-nel
    back-chan-nels
    mo-not-o-nous
    ste-reo-typ-i-cal
}

\hyphenation{
    Beck-man
    Ngu-yen
    back-chan-nel
    back-chan-nels
    mo-not-o-nous
    ste-reo-typ-i-cal
}

   \boolfalse{bookcompile}
   \togglepaper[23]%%chapternumber
}

\begin{document}
\maketitle

\section{Introduction} \label{sec:rb1}
In earlier stages of Portuguese, the present perfect (henceforth PPC, for the Portuguese term \textit{pretérito perfeito composto}), was used as a resultative perfect. For instance, in (\ref{ex:rb1}), the PPC expresses the present result of an event that has occurred exactly once in the past, and which was concluded a while before speech time.

\ea\label{ex:rb1}
Tycho Brahe Parsed Corpus of Historical Portuguese, 16\textsuperscript{th} c., \citet[28]{AmaralHowe2012}\\
\gll Eu ey por bem que Nicolaao Jusarte, fidalgo de minha casa, a que tenho ffeito merce da capitania de um dos navios que vão pera a India nesta armada d'outubro, vaa no navio do Porto \\
I have.\textsc{prs.ind.1sg} for good that Nicolaao Jusarte, nobleman of my.\textsc{f} house, to who have.\textsc{ind.prs.1sg} do.\textsc{ptcp} grace of.\textsc{det.def.f.sg} captaincy of \textsc{det.def.m.sg} of.\textsc{det.def.m.pl} ships that go.\textsc{ind.prs.3pl} to \textsc{det.def.f.sg} India in.\textsc{dem.f.sg} fleet of.october go.\textsc{ind.prs.3sg} in.\textsc{det.def.m.sg} ship of.\textsc{det.def.m.sg} Porto\\
\glt ‘I order that Nicolau Jusarte, nobleman of my house, whom I have awarded the honor of being the captain of one of the ships that will go to India in the October fleet, shall go on the ship from Porto’
\z
 
Examples such as (\ref{ex:rb1}) are ungrammatical in Modern Portuguese due to a historical change in the usage contexts of the PPC from resultative to durative or iterative readings. Consider, for instance, example (\ref{ex:rb2}), taken from the translation of an interview with designer Jony Ive, where the original English sentence is inflected for present perfect progressive (\textit{we’ve been doing}), whereas the Portuguese translation uses the PPC. The use of the present perfect progressive in the original text suggests that the speaker is talking about an event whose beginning lies in the past (Ive’s work with Apple) and has been continuing until speech time.

\ea\label{ex:rb2}
Interview with Jony Ive, \textit{Corpus do Português} (\citealt{Davies2016P}, original English text from \href{https://www.ft.com/content/0b20032e-98cf-11e9-8cfb-30c211dcd229}{https://www.ft.com/content/0b20032e-98cf-11e9- 8cfb-30c211dcd229}, accessed 20 September, 2020)\\
\gll Há algumas áreas que são paixões naturais para mim. O trabalho que tenho feito com a tecnologia de wearables – com a tecnologia a tornar-se mais pessoal, é uma inevitabilidade que se torne usada \\
have.\textsc{ind.prs.3sg} \textsc{det.indf.m.pl} areas that be.\textsc{ind.prs.3pl} passions natural for me. \textsc{det.def.m.sg} work that have.\textsc{ind.prs.1sg} do.\textsc{ptcp} with \textsc{det.def.f.sg} technology of wearables – with \textsc{det.def.f.sg} technology to become.\textsc{inf-refl.3} more personal be.\textsc{ind.prs.3sg} \textsc{det.indf.m.pl} inevitability that \textsc{refl.3} become.\textsc{sbj.prs.3sg} use-\textsc{ptcp.f}\\
\glt ‘There are some areas that are personal natural passions for me. The work that we’ve been doing with wearable technology – with technology becoming more personal, there is an inevitability that it becomes worn” [=original English text]’
\z

The change from resultative to durative or iterative readings has received great attention in the linguistic literature. In a recent proposal, \citet{AmaralHowe2012} analyze the development of the PPC in terms of the notion of “pluractionality”, understood as the potential for a sentence to express multiple events. They claim that the PPC acquired pluractional readings in contexts that are structurally ambiguous; in particular transitive contexts with a direct object that is inflected for masculine and singular, but can still be interpreted as expressing plural. For instance, in their example (\ref{ex:rb3}) below, the direct object \textit{hospital} is modified by the universal quantifier \textit{cada} ‘each’. As a result, the PPC \textit{tenho provido} in this example can be understood as either expressing a resultative reading (‘each hospital has a physician now’) or a pluractional reading (‘I have repeatedly granted the hospitals a physician’).

\ea\label{ex:rb3}
Tycho Brahe Parsed Corpus of Historical Portuguese, 16\textsuperscript{th} c. \citep[43]{AmaralHowe2012}\\
\gll \textbf{tenho} \textbf{provido} cada hospital de seu físico, que são os abades, retores, vigários e curas \\
have.\textsc{ind.prs.1sg} grant.\textsc{ptcp} each hospital of \textsc{poss.m.sg} physician, who be.\textsc{ind.prs.pl} \textsc{def.det.m.pl} abbots, rectors, vicars and priests\\
\glt ‘I have granted each hospital with a physician, who are the abbots, rectors, vicars, and priests’
\z

\citet{AmaralHowe2012} assume contexts such as (\ref{ex:rb3}) to be pivotal to the semantic change undergone by the PPC, in that hearers of such examples could have reanalyzed the PPC as expressing pluractionality in these contexts. This would involve a “transfer of semantic plurality in the nominal domain to the verbal one” \citep[51]{AmaralHowe2012}, in that an interpretive property formerly associated with an argument in the sentence would now come to be conventionally associated with the use of the PPC. 

Crucially, however, \citet{AmaralHowe2012} offer little quantitative evidence for the relevance of this mechanism and, indeed, do not map out the semantic change itself in diachrony. The present paper establishes a data-driven description of the historical change undergone by the PPC in Brazilian Portuguese. In doing so, it aims to test \citeauthor{AmaralHowe2012}’s (\citeyear{AmaralHowe2012}) hypothesis of a transfer from nominal to verbal pluractionality. In particular, this hypothesis makes the prediction that in earlier stages of the change, the PPC occurs in contexts that denote nominal pluractionality (i.e. transitives with direct objects that allow for a plural reading) with sufficient frequency as to enable a reanalysis that would then affect the usage of the PPC in all contexts. A second prediction would be that whereas the likelihood of the PPC to be used in nominal pluractionality contexts did not increase over time, its likelihood to be used in verbal pluractionality contexts (corresponding to the eventual conventional meaning of the PPC) did increase over time. 

We investigate these predictions in a large corpus of BP theater plays, which approximate the spoken language of the respective periods, between the 19\textsuperscript{th} and 21\textsuperscript{st} century. On the basis of a careful qualitative analysis of the use of the PPC in these plays, we establish a bottom-up quantitative operationalization of the likelihood for a given context to express nominal or verbal pluractionality. Our results confirm the predictions made by \citeauthor{AmaralHowe2012}’s (\citeyear{AmaralHowe2012}) hypothesis, according to which in earlier texts, the PPC is more likely to be used in contexts associated with nominal pluractionality than in later texts. At the same time, we find a general increase in the likelihood of the PPC to be used in contexts associated with verbal pluractionality.

Our corpus of BP theater plays also allows us to measure the sociolinguistic environment of the semantic change undergone by the PPC. This analysis allows for an explanation of a curious inconsistency in the development of the PPC towards usage in verbal pluractionality contexts, namely a notable decrease in speed of the change in texts from the first half of the 20\textsuperscript{th} century. In particular, we measure the degree to which each theater play represents conceptual orality, understood as the degree to which the scripted language in the plays is presented as highly planned and edited \parencite{KochOesterreicher1985}. A high degree of conceptual orality indicates a low adherence to the linguistic norm of written language at that time and, in contrast, a more authentic representation of actual spoken language. By integrating this measurement into ordinal logistic regression models, we are able to establish that the semantic change towards pluractionality proceeded in two qualitatively different stages: in a first stage (1850--1890), we find a strong increase of the use of the PPC in verbal pluractionality contexts only in texts with low conceptual orality. In contrast, in a second stage (1950--2016) of expansion towards verbal pluractionality contexts, this process is implemented to a stronger degree in texts with high conceptual orality. We interpret this finding as indicative of an influence of the stronger contact between Brazilian and European Portuguese during the 19\textsuperscript{th} century, i.e. the so-called period of the \textit{Império do Brasil} (‘Empire of Brazil’). Given that the PPC experienced the change towards expression of verbal pluractionality earlier in European than in Brazilian Portuguese, we hypothesize that the stronger contact with the prestigious European norm during the \textit{Império do Brasil} caused conservative authors to adopt usage patterns of the PPC associated with the European norm.

\section{Previous studies}\label{sec:rb2}
The Portuguese PPC is a well-described construction, which has been analyzed in a number of studies. Both grammars (\cite[180]{Hundertmark-SantosMartins1982}, \cite[23]{Gärtner1998}, \cite[249--250]{Perini2002}) and linguistic studies (\cite[127]{Boléo1936}, \cite[66]{Ilari2001a}, \cite{NovaesNespoli2014}, \cite{Santos2008}) establish that, in Modern Portuguese, the PPC can express either a durative (\ref{ex:rb4a}) or a non-durative reading, such as the iterative reading in (\ref{ex:rb4b}).\footnote{There are a number of non-durative readings that the PPC can obtain. The qualitative analysis in \sectref{sec:rb4} will describe these readings in more detail, using examples from our historical corpus.}

\ea
    \ea\label{ex:rb4a} Durative reading \parencite[180]{Hundertmark-SantosMartins1982}\\
     \gll   \textbf{Tem} \textbf{estado} muito calor.\\
            have.\textsc{ind.prs.3sg} be.\textsc{ptcp} much heat\\
     \glt ‘Recently, it has been very hot.’\\
    \ex\label{ex:rb4b} Iterative reading (\url{http://caras.sapo.pt/famosos/2011-07-04-claudia-vieira-tenho-vivido-bons-momentos-desde-que-a-maria-nasceu}, last access 4 November 2019)\\
     \gll   \textbf{Tenho} \textbf{vivido} bons momentos desde que a Maria nasceu.\\
            have.\textsc{ind.prs.1sg} live.\textsc{ptcp} good moments after that \textsc{det.def.f.3sg} Maria be.born.\textsc{ind-pst.pfv.3sg}\\
     \glt ‘I have experienced many good moments since Maria’s birth.’
    \z
\z

These prototypical readings are described as manifestations of an abstract semantics of the PPC that reunites several characteristics that have been shown to be relevant in the literature (\cite{Ilari1999,Ilari2001a,Ilari2001b}, \cite{Barbosa2008, Schmitt2010, Molsing2010, CabredoHofherrCarvalho2010, Barbosa2012, AmaralHowe2012, OliveiraLeal2012, Olbertz2018, Becker2020}):

\begin{itemize}
    \item The PPC expresses a plurality of events (or sub-events) of the type described by the predicate because it introduces an operator expressing indeterminate quantification of events (or sub-events) (see \cite[98]{Barbosa2008}, \cite[182]{Barbosa2012}, \cite{Becker2020}).
    \item The PPC introduces a temporal interval that extends from an initial point in the past (left boundary) to speech time (right boundary) and can, in the right context, even be interpreted as generating the implicature that the event extends until the future (\cite[32]{Becker2017}, \cite[489]{Olbertz2018}). This interval is asserted, equaling the so-called “perfect time span” \parencite[see, for instance,][]{Stechow1999}.
    \item The different readings depend on the aspectual properties of the verbal predicates, in that, for instance, without further modification stative predicates generate durative readings, as in (\ref{ex:rb4a}).
\end{itemize}

Recent analyses make use of the notion of pluractionality, which was introduced into semantics by \citet{Lasersohn1995}. Pluractionality can be described as an umbrella term for various kinds of readings that involve event plurality, such as iterative and habitual readings. There is no consensus regarding the exact definition of pluractionality. While \citet[83--84]{CabredoHofherrCarvalho2010} define pluractionality as an indeterminate plurality of events, thus excluding contexts with definite quantification such as cardinal quantification of the type ‘two times’, other authors such as \citet{BertinettoLenci2012} define the term more broadly. In their words, pluractionality indicates that “the same event repeats itself in a number of different situations”, not excluding definite quantification. In this paper, we will use this broader definition of pluractionality, and broaden it even further to contexts in which the pluralizing reading derives from a pragmatic inference. This inference has sometimes been described as a coercion mechanism (\cite{Pustejovsky1995, Michaelis2004}) in that pluractional readings are generated on the basis of the interaction between certain verb forms, predicate types and adverbial expressions \parencite{Laca2006}.
The semantic change of the PPC from resultative to pluractional readings, which stands in the center of interest of this paper, has likewise received some attention in the recent literature (\cite{Barbosa2014, AmaralHowe2012, Becker2017}, \cite[486--493]{Olbertz2018}). These studies differ as regards the variety of Portuguese that is being analyzed, the time frame, methodology (qualitative vs. quantitative), and research questions. 

Whereas \citet{AmaralHowe2012, Becker2017, Olbertz2018}, and \citet{Becker2020} are mostly interested in motivating the semantic change from resultative to pluractional readings, \citet{Barbosa2014} aims at establishing a complete description of the changes undergone by the PPC in Brazilian Portuguese between the 16\textsuperscript{th} and 20\textsuperscript{th} century.\footnote{\label{fn:rb2}As noted by \citet[85, fn 1]{Barbosa2014} herself, the label “Brazilian Portuguese” is somewhat misleading for texts written in the 16\textsuperscript{th} and 17\textsuperscript{th} century because at that time, a Brazilian Portuguese norm did not exist. Her “Brazilian Portuguese” texts from this period are texts written by European Portuguese authors living in Brazil.}  This is done by opposing the PPC and the simple past tense (the “pretérito perfeito simples”, or PPS), using variationist methodology. First, her results demonstrate that in relation to the PPS, the usage frequency of the PPC is higher in earlier stages of BP than after the 20\textsuperscript{th} century. Between the 18\textsuperscript{th} and 20\textsuperscript{th} century, the usage frequency of the PPC relative to the PPS drops from 14 percent to 5 percent. Perhaps due to the fact that texts from the 17\textsuperscript{th} and 18\textsuperscript{th} century do not really reflect a Brazilian Portuguese norm (see footnote~\ref{fn:rb2}), her results for these centuries display a somewhat mixed tendency. 

Second, \citet[93]{Barbosa2014} analyzes the change in the semantics of the PPC. Her results demonstrate the expected change from non-pluractional towards pluractional readings; whereas she analyzes 51 percent of the occurrences of the PPC in the 16\textsuperscript{th} century as expressing a (resultative) perfective reading, she documents a steady decrease of the usage frequency of the PPC in these contexts until the 20\textsuperscript{th} century, in which she does not find any tokens of the PPC expressing this reading. In contrast, she finds a mild increase in the relative frequency of iterative readings (which are found to be frequent already in early texts) and a strong increase in the relative frequency of durative readings (from 7 percent in the 16\textsuperscript{th} century to 40 percent in the 20\textsuperscript{th} century). 

While these numbers are very convincing, it has to be noted that they represent the author’s interpretation of the data and can consequently be considered as highly subjective. This is problematic because (a) sentences involving the PPC are frequently ambiguous between the different readings, a fact that is actually at the heart of the semantic reanalysis of the PPC (see the discussion of \cite{Becker2017} and \cite{AmaralHowe2012} below) and (b) there is no consensus in the literature with respect to the precise range of functions expressed by the PPC. For instance, \citet[487]{Olbertz2018} claims that “[i]t is only in the 20\textsuperscript{th}-century data that we have unambiguous cases of the iterative (or continuous) perfect, in which the event is implied to persist”. Third, \citet{Barbosa2014} also analyzes the influence of the degree of formality of the texts on the opposition between the PPC and the PPS, although her basis for the classification of the corpus into texts that are [+formal] or [$–$formal] is unclear. She claims that formality influenced the opposition only in the 16\textsuperscript{th} century and that, after the 17\textsuperscript{th} century, formality did not have an influence on the use of the PPC.

\citet{Becker2017} provides a fine-grained qualitative analysis of the semantics of the PPC, distinguishing several readings. He describes in greater detail uses of the PPC that he calls “summative”, in which the pluractional reading derives from an inference on the basis of the combination of adverbial expressions and sentence arguments. Consider \citeauthor{Becker2017}’s example (\ref{ex:rb5}), taken from the corpus do português \citep{Davies2016P}. Here, the author uses the PPC to describe various occurrences of the action ‘making a will’ that are situated in a time interval that ranges from some moment in the past until before the speech act (\textit{athé aqui} ‘up until here’). Note that the PPC in (\ref{ex:rb5}) indicates indefinite quantification, in that the exact number of testaments is unknown (and, indeed, irrelevant here). The pluractional reading mostly derives from the plurality of the direct object constituent, marked also by agreement on the participle \textit{feitos}. According to Becker, such uses of the PPC are extremely rare in the 16\textsuperscript{th} century and can be considered innovative.

\ea\label{ex:rb5}
\textit{Corpus do português}, 16\textsuperscript{th} c. \citep[29]{Becker2017}\\
\gll e ruogo todollos outros testamentos que athe aqui \textbf{hey} \textbf{feitos}\\
and ask.for.\textsc{ind.prs.1sg} all.\textsc{det.def.m.pl} other wills that until here have.\textsc{ind.prs.1sg} do.\textsc{ptcp.m.pl}\\
\glt ‘and I ask for all of the other wills that I have made until today’
\z

Crucially, \citet[33]{Becker2017} observes an increase in the usage frequency of such summative readings of the PPC in the 17\textsuperscript{th} century and claims that these readings can be considered the starting point of the semantic change of the PPC towards the expression of pluractionality. In particular, he observes that durative readings were developed only after the use of the PPC was firmly entrenched in summative contexts. 

While using different terminology, \citet{AmaralHowe2012} model the semantic reanalysis of the PPC in a very similar manner. In particular, they identify contexts such as example (\ref{ex:rb3}), repeated below as (\ref{ex:rb6}), as “onset” contexts for the change from resultative to pluractional readings. Example (\ref{ex:rb6}) seems to be a clear instance of the contexts called “summative” by \citet{Becker2017}; in particular, the pluractionality value of the PPC derives from a pragmatic inference on the basis of the implicit plurality of the direct object \textit{cada hospital}. Like (\ref{ex:rb5}), example (\ref{ex:rb6}) is ambiguous between this innovative pluractional reading and the original resultative reading, which is a hallmark of “bridging contexts”, i.e. contexts that are characterized as intermediate steps in processes of semantic change \parencite{Heine2002}.

\ea\label{ex:rb6}
TBCHP, 16\textsuperscript{th} c.  \citep[43]{AmaralHowe2012}\\
\gll \textbf{tenho} \textbf{provido} cada hospital de seu físico, que são os abades, retores, vigários e curas\\
have.\textsc{ind.prs.1sg} grant.\textsc{ptcp} each hospital of \textsc{poss.m.3sg} physician who be.\textsc{ind.prs.pl} the abbots rectors vicars and priests\\
\glt ‘I have granted each hospital with a physician, who are the abbots, rectors, vicars, and priests’
\z

\citet[40--48]{AmaralHowe2012} provide a testable hypothesis as to the nature of this semantic reanalysis. Crucially, the resultative and pluractional readings of example (\ref{ex:rb6}) presuppose the same kind of complex event structure, namely a process by which every hospital was assigned a physician. The difference between the two interpretations resides in the fact that the pluractional reading asserts that this assignment process was carried out multiple times with different referents (i.e., physician1 was assigned to hospital3, physician2 to hospital4, etc.). In contrast, the resultative reading merely asserts the result of these processes, i.e. that each hospital has been assigned a physician. Since in the resultative reading, the truth of the existence of these sub-events is not evaluated, it is perfectly compatible with both a single-event reading (all physicians were assigned in the same event) and a plural-event reading (each doctor was assigned separately). According to \citet[43--43]{AmaralHowe2012}, the change from a resultative to a pluractional reading can then easily take place in contexts in which the plural event interpretation is more likely for pragmatic reasons. For instance, in example (\ref{ex:rb6}), it seems unlikely that the subject referent was able to assign all physicians at the same time.

The idea that contexts of the type instantiated in (\ref{ex:rb6}) constitute bridging or, in \citegen{AmaralHowe2012} terms, “onset” contexts entails that the semantic change experienced by the PPC can also be described as change from this type of plur-actional reading to another type of pluractional reading, i.e. contexts that are no longer compatible with a resultative reading. As an example for such a “switch context” \citep[85]{Heine2002}, consider (\ref{ex:rb7}) taken from \citegen{Becker2017} study. Both instances of the PPC in (\ref{ex:rb7}) are incompatible with a resultative reading because the predicates \textit{pensar} and \textit{estar} do not entail resultant states. They also differ from example (\ref{ex:rb6}) in that the plurality of the event is not indicated by a plurality of one of the arguments of the verb.\pagebreak

\ea\label{ex:rb7}
\textit{Corpus do português}, 20\textsuperscript{th} c. \citep[37]{Becker2017}\\
\gll Esta é uma questão que \textbf{tenho} \textbf{pensado} muito ao longo de 7 anos que \textbf{tengo} \textbf{estado} na Microsoft\\
\textsc{dem.f.sg} be.\textsc{ind.prs.3sg} \textsc{det.indf.f.sg} question that have.\textsc{ind.prs.1sg} think.\textsc{ptcp} much at.\textsc{det.def.m.sg} long of seven years that have.\textsc{ind.prs.1sg} be.\textsc{ptcp} in.\textsc{det.def.f.sg} Microsoft\\
\glt ‘That’s a question that I have been thinking about a lot during the seven years that I have been at Microsoft’
\z

Consequently, the semantic change leading from examples such as (\ref{ex:rb6}) to examples such as (\ref{ex:rb7}) can be described as a conventionalization of the pluractional reading of the PPC. In other words, the pluractional reading of the PPC became independent from contextual cues such as plurality of the argument(s) in the sentence and has thus become part of the conventional meaning of the PPC. This change crucially involves a “transfer of semantic plurality in the nominal domain to the verbal one” \parencite[51]{AmaralHowe2012}, given that the trigger of the pluractional reading no longer resides in the nominal arguments in the sentence, but rather in the use of the PPC itself.

To summarize, studies on the development of the PPC assume that the semantic change undergone by the PPC can be interpreted in terms of a semantic reanalysis enabled by bridging contexts in which the pluractional reading derives via pragmatic inference from the (semantic) plurality of one or more of the arguments of the verb. The change from such “summative” to purely iterative and durative readings crucially involved a transfer from nominal to verbal plurality.

It is this hypothesis, namely the transfer from nominal to verbal plurality, that will be at the heart of the analysis presented in this paper. In particular, the assumption that contexts in which the pluractional reading derived via pragmatic inference from the presence of (semantically) plural nominal arguments served as bridging contexts for the conventionalization of the pluractional reading, makes the specific predictions that (a) the PPC is more likely to occur in contexts characterized by nominal pluractional readings in earlier texts than in later texts, and (b) likelihood for the PPC to be used in contexts associated with verbal pluractionality is expected to increase over time. Note that none of the papers discussed until now have given quantitative empirical evidence that might confirm these predictions.\largerpage

As is well known in historical linguistics, the sole mapping out of a reanalysis process does rarely do justice to the complexities of the description of the entire process of historical change, which is why recent approaches call for longitudinal descriptions that also analyze sociohistorical parameters. In particular, grammaticalization processes are non-teleological in the sense that they have more than one possible outcome \parencite{Collins2019}. As a result, it is extremely important to consider both the source constructions in such processes \parencite{Cristofaro2019} and map out the evolution of these processes over time, which can be affected by phenomena such as language contact (\citeauthor{Thomason2010} \citeyear{Thomason2010}, \citeauthor{PoplackDion2011} \citeyear{PoplackDion2011}, \citeauthor{AuweraGast2012} \citeyear{AuweraGast2012}) or socio-stylistic attitudes involved in processes such as latinization \parencite{Cornillie2019}. Consequently, we were also interested in the influence of one parameter, the degree of conceptual orality of the texts, on the development of the semantic change of the PPC, with the working hypothesis that the semantic change is expected to be implemented to a greater degree in texts that display a high degree of conceptual orality and consequently, a low degree of formality (see \citeauthor{Rosemeyer2019a} \citeyear{Rosemeyer2019a} for discussion).\footnote{Recall that the analysis regarding this parameter proposed in \citet{Barbosa2014} did not yield conclusive results.}  

\section{Data and periodization}\label{sec:rb3}

In order to provide quantitative evidence for the assumption that the semantic change of the PPC towards pluractionality crucially involved a transfer of pluractional readings in nominal contexts to verbal contexts, we extracted all tokens of the PPC from two diachronic corpora of Portuguese: the PorThea, a corpus of historical theater plays \parencite{Rosemeyer2019b}, and the Genre/Historical section of the Corpus do português, a multi-genre corpus with historical depth (\cite{Davies2016P}, henceforth CdP). Table \ref{tab:rb1} below gives an overview of the two corpora.\largerpage

\begin{table}[h]
\begin{tabularx}{\textwidth}{llQ}
\lsptoprule
 & PorThea & CdP \\
\midrule
Corpus size & \textasciitilde3.3 million words & \textasciitilde45 million words \\
\tablevspace
Varieties of Portuguese & EP, BP & EP, BP \\
\tablevspace
Genre & Theater plays & Different genres, including spoken language \\
\tablevspace
Time depth & 18\textsuperscript{th}--21\textsuperscript{st} c. & 13\textsuperscript{th}--20\textsuperscript{th} c. \\
\tablevspace
Automatic annotation & None & PoS tagging, lemmatization\\
\lspbottomrule
\end{tabularx}
\caption{Overview of the PorThea and CdP corpora}
\label{tab:rb1}
\end{table}

As evident in \tabref{tab:rb1}, the PorThea corpus is considerably smaller and less-well annotated than the CdP. Its main advantage, however, is its thematic consistency: given that theater plays approximate orality, a corpus of theater plays can be expected to give a better indication of patterns of change in the spoken language at that time than other written sources \parencite[432]{Kytö2011}. When using data from the CdP, it is possible that observed changes in usage frequency and distributional patterns are actually due to changes in the textual substrate, making such changes “apparent changes” (see \citeauthor{Szmrecsanyi2016} \citeyear{Szmrecsanyi2016}, \citeauthor{Rosemeyer2019a} \citeyear{Rosemeyer2019a}). For this reason, we used the data from the PorThea for the main analyses conducted in this paper and the CdP data merely for sanity checks.

Extraction of all tokens of the PPC from the Brazilian section PorThea led to a total result of $n = 857$ tokens using regular expressions. \figref{fig:rb1} visualizes the historical development of the usage frequency of the PPC in these data. 

\begin{figure}
\includegraphics[width=\textwidth]{figures/RosBeck1.png}
\caption{Historical development of the log-transformed usage frequency per 100,000 tokens of the PPC in the Brazilian section of the PorThea corpus. Points represent mean usage frequencies per year, whereas the line represents the result from a local polynomial regression analysis summarizing the trend.}
\label{fig:rb1}
\end{figure}

It is important to note that the PorThea does not contain texts for the period between 1750 and about 1830, which is why no data points are given in the plot for this period.\footnote{The corpus size in words by century is: \textit{n}\textsubscript{18\textsuperscript{th}} = 175,891, \textit{n}\textsubscript{19\textsuperscript{th}} = 787,015, \textit{n}\textsubscript{20\textsuperscript{th}} = 747,110, \textit{n}\textsubscript{21\textsuperscript{st}} = 948,485. Raw usage frequencies of the PPC per century are: \textit{n}\textsubscript{18\textsuperscript{th}} = 106, \textit{n}\textsubscript{19\textsuperscript{th}} = 570, \textit{n}\textsubscript{20\textsuperscript{th}} = 116, \textit{n}\textsubscript{21\textsuperscript{st}} = 65.} However, if we compare the mean usage frequencies of the PPC in plays dating from the first half of the 18\textsuperscript{th} century and the mid-19\textsuperscript{th} century, we find a weak increase. In contrast, the analysis suggests a strong and relatively linear decrease in the usage frequency of the PPC from the second half of the 19\textsuperscript{th} century to Present-Day BP. This finding, which reaches statistical significance,\footnote{Correlation testing was done using Kendall’s τ because both variables were not normally distributed \parencite[cf.][]{Gries2009}. There is a significant negative correlation between the date of the text and the log-transformed normalized frequency of the PPP per 100,000 words (Kendall’s $\tau = -0.371$, $z = -5.412$, $p\textsubscript{one-tailed} < 0.001$).} thus corroborates the previous findings by \textcite{Barbosa2014} as regards the decreasing usage frequency of the PPC in BP. 

Although usage frequency and grammatical productivity of a construction typically stand in a relationship to each other, a higher usage frequency does not always indicate higher grammatical productivity \parencite{Barðdal2008}. For instance, grammaticalization processes are usually assumed to involve a rise in usage frequency because the semantic change leads to an expansion of the use of the construction into new usage contexts (so-called “host class expansion”, \cite{Himmelmann2004}). In our case, such a “host-class expansion” could be understood as an expansion of PPC usage to more verb classes. However, it has also been shown that in incipient grammaticalization processes, prefabs, i.e. conventionalized form-function pairings which would usually be described as rather unproductive, lead the way \citep{BybeeCacoullos2009}. Likewise, lexicalization processes can involve an increase in usage frequency that does not reflect an increase in productivity. As a result, a complete description of any linguistic change needs to not only rely on usage frequencies, but also on productivity measures. 

One such productivity measure, which is common in corpus linguistics, is type-token ratio (TTR). The type-token ratio is calculated by dividing the total number of types (words, constructions etc.) by the total number of tokens \parencite[253]{McEneryHardie2012}, yielding a measure between 0 and 1. When analyzing verbal constructions, TTR can be a measure of productivity in terms of the degree to which that construction can be used with all verbs in a language. To achieve this, we divided the number of verb lemmas found in the PPC in each year of the corpus by the total number of the PPC found in that year. Whereas a higher TTR indicates a high productivity (wider range of verbs found in the PPC), a lower TTR indicates a lower productivity (lower range of verbs found in the PPC).

\figref{fig:rb2} visualizes the development of the TTR of the PPC by year in our corpus. It demonstrates an increase in TTR between 1850 and 1950, followed by a steep decrease after 1950.\footnote{Again, there is a significant correlation between TTR and year of publication of the play. In order to capture the non-linearity of the trend visualized in \figref{fig:rb2}, we calculated a linear regression model predicting TTR by year of publication. Given the distribution of TTR found in \figref{fig:rb2}, the variable \textsc{Year} was modeled as a third-degree polynomial. A significant effect was found ($F(3, 80) = 3.328,\allowbreak p<0.05$), with an R\textsuperscript{2} of 0.111.}

\begin{figure}
\includegraphics[width=\textwidth]{figures/RosBeck2.png}
\caption{Historical development of the type-token ratio (TTR) of the PPC in the Brazilian section of the PorThea corpus. Points represent TTR ratios per year, whereas the line represents the result from a local polynomial regression analysis summarizing the trend.}
\label{fig:rb2}
\end{figure}

If we compare the development of the TTR and the usage frequency of the PPC (see \figref{fig:rb1} above), we see that these two changes are not always correlated. Thus, between 1850--1950, we find that as the usage frequency of the PPC decreases, its TTR increases. In contrast, after 1950, the PPC decreases both in usage frequency and TTR. This finding suggests that the change by which the PPC decreased in usage frequency after 1850 might actually consist of two qualitatively different, subsequent, changes. 

As the visualizations in Figures \ref{fig:rb1} and \ref{fig:rb2} have shown, the Brazilian section of the PorThea corpus does not contain data for the time period between about 1750 and 1830, which is a problem for more fine-grained analyses of the distribution of the PPC. We consequently decided to eliminate all 18\textsuperscript{th} century tokens of the PPC from the subsequent quantitative analyses in \sectref{sec:rb5} and~\sectref{sec:rb6}, leading to a new total of $n = 751$ tokens.

The data from the CdP confirm the general trend towards a lower usage frequency of the PPC. Thus, the PPC reaches 530 words per million in the 18\textsuperscript{th} century CdP, 461 words per million in the 19\textsuperscript{th} century, and 356 words per million in the 20\textsuperscript{th} century. The historical data from the CdP do not distinguish between European and Brazilian Portuguese. However, it is possible to extract comparative frequency measures for the 20\textsuperscript{th} century; here the CdP data show the PPC to be much more frequent in EP than in BP (450 words per million in EP, 265 words per million in BP). This seems to suggest a greater productivity of the PPC in EP than in BP.

\section{Pluractional readings and (some of) its contexts}
\label{sec:rb4}
In this section, we try and establish the contextual parameters in the nominal and verbal domains that indicate a pluractional reading and identify some typical contexts and constellations of these pluractional readings arising. In many cases, several linguistic clues interact in producing a reading of plurality. All of the examples cited in this section are from our PorThea data.

In the nominal domain, a pluractional reading is inferred from the plurality of one or more of the arguments in the sentence (subject, direct object, indirect object). A very important role is played by the structure and semantics of the subject. A typical context combines a plural subject NP with a reflexive verb. In the example (\ref{ex:rb8}), the interaction of the plural subject with the reflexive verb produces a distributive reading. The subject NP \textit{nós todos} is interpreted as referring to every single individual (corresponding to \textit{cada um de nós}), which is predicated to be involved individually in a sub-event as part of an overarching collective commitment event. Consequently, the collective commitment event implies an indeterminate number or set of sub-events with its individual participants. This configuration corresponds to the structure in (\ref{ex:rb9}), which ascribes a commitment-sub-event to each member of the quantitatively undetermined we-group.

\ea\label{ex:rb8}
Martins Pena, \textit{O noviço}, 1845\\
\gll \textbf{Nós} \textbf{todos} nos \textbf{temos} \textbf{empenhado}\\
we all us.\textsc{refl.3pl} have.\textsc{ind.1pl} make.effort.\textsc{ptcp}\\
\glt ‘We have all made an effort’
\ex\label{ex:rb9}
\{(individual 1| commitment-subevent1), (individual 	2|commitment-subevent2), (..|..), (individual n| commitment-	subevent n)\}.
\z

The same -- though more subtle -- effect is produced by the interplay with collective subject-NPs, as in example (\ref{ex:rb10}). Here, the NP \textit{o povo} refers to a collective entity which is composed of individuals. At the same time, the temporal adverbial \textit{de algum tempo para cá} marks a typical perfect interval ranging from some moment in the past (left boundary) to the speech time (\textit{para cá}, i.e. the right boundary). It can be inferred from this constellation that the set of members referred to by the collective \textit{povo} changes over time as to its composition so that at each moment different members are involved in the macro-event of \textsc{showing\_democratic\_tendencies}. In other words, we can attribute different sets of individuals to each time point.

\ea\label{ex:rb10}
Artur Azevedo, \textit{A princesa dos cajueiros}, 1880\\
\gll \textbf{o} \textbf{povo} \textbf{tem} \textbf{mostrado} de algum tempo para cá certas tendências democráticas\\
\textsc{det.def.m.pl} people have.\textsc{ind.prs.3sg} show.\textsc{ptcp} from some time to here certain tendencies democratic\\
\glt ‘from time to time the people have shown certain democratic tendencies’
\z

A plural(ity) reading can also be triggered by a plural direct object NP. In example (\ref{ex:rb11}), the speaker not only refers to several acts of favors, but also enhances the quality of the acts (\textit{obséquios imensos}). This is a means of strengthening the inference that the speaker has been granted several favors, each of which required a certain amount of time.

\ea\label{ex:rb11}
José de Alencar, \textit{O crédito}, 1857\\
\gll não poderei pagar-lhe a amizade e os obséquios imensos que \textbf{nos} \textbf{têm} \textbf{sido} \textbf{feitos}\\
not can.\textsc{ind.fut.1sg} pay-you \textsc{det.def.f.sg} friendship and \textsc{det.def.m.pl} presents immense that to.us have.\textsc{ind.prs.3sg} be.\textsc{ptcp} make.\textsc{ptcp.m.pl} \\
\glt ‘I cannot make up to you the friendship and the limitless presents that we have been given’
\z

The direct object is not necessarily a plural noun but may also turn out to be a mass noun, as in example (\ref{ex:rb12}). What is important in these contexts, however, is the presence of a quantifying expression, i.e. of a quantifier phrase embedded in the NP. The quantifying expression implies that the macro-event in question falls into several stages in which a part or portion of the direct object is affected by the transitive process in question. In (\ref{ex:rb12}), the whole quantity of N (\textit{tôda a prata}) is affected by the \textsc{carry-in}-event (\textit{levar para dentro}). Our available world knowledge suggests that the whole event is structured by sub-events in which portions of the direct object referent partake. \citet{BertinettoCivardi2015} introduce a measure $\delta$, which indicates to what degree an incremental theme is affected by a telic event. In our example \textit{levar toda a prata}, the theme, as made explicit by the quantifier, is completely affected (so $\delta=1$). However, the completion process requires subsequent phases until its culmination point is reached (and the whole quantity of silver has been brought in).

\ea\label{ex:rb12}
Martins Pena, \textit{O usurário}, 1846\\
\gll Enquanto assim falam, os dois \textbf{têm} \textbf{levado} \textbf{tôda} \textbf{a} \textbf{prata} para dentro.\\
meanwhile so talk.\textsc{ind.prs.3pl} \textsc{det.def.m.pl} two have.\textsc{ind.prs.3pl} bring.\textsc{ptcp} all \textsc{det.def.f.sg} silver to inside\\
\glt ‘While chatting, the two of them have brought all of the silver inside.’
\z

It goes without saying that plurality effects can also be obtained by the indirect object. In example (\ref{ex:rb13}), the collective noun \textit{muita gente} refers to a set composed of individuals that are affected individually by the same kind of event.

\ea\label{ex:rb13}
Artur Azevedo, \textit{A capital federal}, 1897\\
\gll Volte, seu Figueiredo, volte, se não quer que lhe aconteça o mesmo que me sucedeu e \textbf{tem} \textbf{sucedido} \textbf{a} \textbf{muita} \textbf{gente}!\\
turn.back.\textsc{sbj.prs.3sg}, dear Figueiredo turn.back.\textsc{sbj.prs.3sg} if not want.\textsc{ind.prs.3sg} that to.you happen.\textsc{ind.prs.3sg} \textsc{det.def.m.sg} same that to.me happen.\textsc{ind.pst.pfv.3sg} and have.\textsc{ind.prs.3sg} happen.\textsc{ptcp} to many people\\
\glt ‘Turn back, dear Figueiredo, turn back, if you do not want that which has happened to me and to many people to also happen to you!’
\z

Another comparable case with an incremental theme is the one in (\ref{ex:rb14}). The quantifiers used in this example, namely \textit{um cento de} and \textit{outro tanto} (\textit{um cento de cartas}, \textit{outros tantos pedidos}), are very frequent in PPC contexts. Neither \textit{um cento de}, with the meaning ‘approximately one hundred’, nor \textit{(outro) tanto} (‘another x instances of N’) specify the precise number of instances of the incremental theme in question. This boils down to the fact that the number of instances of the writing-events (of letters and of requests) remains indeterminate. If the speaker had wanted to indicate a precise quantification of the instances (e.g. exactly 100 letters), he would have resorted to the simple preterit (\textit{Escrevi cem cartas e outro tanto pedidos} ‘I wrote a hundred letters and pleas’). 
  
\ea\label{ex:rb14}
Qorpo Santo, \textit{O marinheiro escritor}, 1866\\
\gll \textbf{Tenho} \textbf{escrito} um cento de cartas, \textbf{feito} outros tantos pedidos a pessoas que para lá vão; já mandei de propósito uma para tal fim, e nada tenho podido conseguir\\
have.\textsc{ind.prs.1sg} write.\textsc{ptcp} \textsc{det.indf.m.sg} hundred of letters, do.\textsc{ptcp} other so.many pleas to persons that to there go.\textsc{ind.prs.3pl} already send.\textsc{pst.pfv.1sg} of purpose \textsc{det.indf.f.sg} for such end and nothing have.\textsc{ind.prs.1sg} be.able.\textsc{ptcp} achieve.\textsc{inf}\\
\glt ‘I have written some hundred letters, made pleas to people who are going there; I already sent one there on purpose to achieve this aim, and I haven’t been able to accomplish anything’
\z

A particular pluractional setting may be produced in contexts with internal temporal modification, such as a negation operator (see \ref{ex:rb15}). Typically, the negation operator interacts with an explicit temporal expression. In this constellation, the negation operator is in the scope of an all-quantification inferable from the temporal expression. In (\ref{ex:rb15}), the speaker asserts that, for all time indexes i of an interval I, there is no event e of the given type, such that p holds, i.e. for all i $\in$ I: $\neg$ p(i). Therefore, this configuration yields the durative reading that the same state of affairs non-p holds during the whole relevant interval set by the expression \textit{até agora}. Once again, we are dealing with a perfect time span ranging from some point in the past to the utterance time of the speaker. 

\ea\label{ex:rb15}
Martins Pena, \textit{Os irmãos das almas}, 1844\\
\gll \textbf{Até} \textbf{agora} não \textbf{tenho} \textbf{sido} homem, mas era preciso sê-lo\\
until now not have.\textsc{ind.prs.1sg} be.\textsc{ptcp} human but be.\textsc{pst.ipfv.3sg} necessary be.\textsc{inf}-it\\
\glt ‘Until now I haven’t been human, but it was important to be it [=a dog]’
\z

A very similar effect can derive from the use of the direct object \textit{nada} ‘nothing’ (see \ref{ex:rb16}). As in example (\ref{ex:rb15}), the speaker asserts that a certain event did not happen within a certain time interval, a reading that seems to be highlighted by the resumptive preposing \parencite[908--911]{Leonetti2017} of \textit{nada}. 

 
\ea\label{ex:rb16}
Qorpo Santo, \textit{O marinheiro escritor}, 1866\\
\gll Tenho escrito um cento de cartas, feito outros tantos pedidos a pessoas que para lá vão; já mandei de propósito uma para tal fim, e \textbf{nada} \textbf{tenho} \textbf{podido} \textbf{conseguir}\\
have.\textsc{ind.prs.1sg} write.\textsc{ptcp} \textsc{det.indf.m.sg} hundred of letters, do.\textsc{ptcp} other so.many pleas to persons that to there go.\textsc{ind.prs.3pl} already send.\textsc{pst.pfv.1sg} of purpose \textsc{det.indf.f.sg} for such end and nothing have.\textsc{ind.prs.1sg} be.able.\textsc{ptcp} achieve.\textsc{inf}\\
\glt ‘I have written some hundred letters, made even more pleas to people who are going there; I already sent one there on purpose to achieve this aim, and I haven’t been able to accomplish anything’
\z

A very different situation can be observed in the verbal domain. Here, pluractionality is expressed by the use of adverbs or adverbials that express repetition or license a distributive interpretation, and the use of predicates that are compatible with the respective reading. 

The adverbial expression \textit{x vezes} (see example \ref{ex:rb17}) is one of the most frequent ones to indicate iterativity and plays an important role in the strengthening of the reanalysis of the PPC as a form imbued with pluractional semantics \parencite[169, 181]{Becker2020}. However, it is a noteworthy fact that in the 19\textsuperscript{th} century the PPC is still compatible with cardinal external quantification (see \ref{ex:rb18}). It is not until the 20\textsuperscript{th} century that the adverbial \textit{x vezes} is only compatible with indefinite quantification, therefore, reflecting the feature of indeterminacy inherent to the semantics of the PPC.

\ea\label{ex:rb17}
Joaquim Manoel de Macedo, \textit{O primo da Califórnia}, 1858\\
\gll também eles \textbf{têm-me} \textbf{recebido} \textbf{tantas} \textbf{vezes} em suas casas, que hoje por minha parte quero também recebê-los\\
also they have.\textsc{ind.prs.3pl}-me receive.\textsc{ptcp} so.many times in \textsc{poss.f.3pl} houses that today from \textsc{poss.f.1sg} part want.\textsc{ind.prs.1sg} also receive.\textsc{inf}-them\\
\glt ‘Morever, they have received me so many times in their houses that I now want to also receive them in mine’
\ex\label{ex:rb18}
Joaquim Manoel de Macedo, \textit{Luxo e Vaidade}, 1860\\
\gll já sei, \textbf{tens} \textbf{tirado} a sorte grande \textbf{cinco} \textbf{ou} \textbf{seis} \textbf{vezes}.\\
already know.\textsc{ind.prs.1sg} have.\textsc{ind.prs.2sg} draw.\textsc{ptcp} \textsc{det.def.f.sg} luck great five or six times\\
\glt ‘I (already) know, you have been lucky for five or six times.’
\z

A pluractional reading in the verbal domain can also arrive from modification with a locative adverbial with distributive meaning. Thus, the locative adverbial in (\ref{ex:rb19}) licenses the inference that the announcement event must have taken place several times.

\ea\label{ex:rb19}
Júnior França, \textit{As doutoras}, 1889\\
\gll Não é aqui que mora uma doutora que \textbf{tem} \textbf{anunciado} \textbf{nos} \textbf{jornais}?\\
not be.\textsc{ind.prs.3sg} here that live.\textsc{ind.prs.3sg} \textsc{det.indf.f.sg} doctor that have.\textsc{ind.prs.3sg} announce.\textsc{ptcp} in.\textsc{det.def.m.pl} newspapers\\
\glt ‘Doesn’t here live a doctor who has placed ads in the newspapers?’
\z


Finally, pluractionality in the verbal domain can arise on the basis of the type of predicate expressed by the verb. In particular, a pluractional interpretation is likely with durative predicates such as states (\ref{ex:rb20}) and atelic activities (\ref{ex:rb21}). In both cases, a durative reading arises. For instance, in (\ref{ex:rb21}) the question delimits a time interval that spans all of the life of the addressee, in which an event was repeated several times. Notably, such predicates co-occur more frequently with internal temporal modification (24 percent of all durative tokens) than other predicates (18 percent of all tokens from other predicate types). This difference reaches statistical significance ($\chi^2(1)=3.95$, $p<0.05$*), which implies that examples such as (\ref{ex:rb15}) above are rather typical.

\ea\label{ex:rb20}
José de Alencar, \textit{Mãe}, 1860\\
\gll \textbf{Tens} \textbf{tido} notícias dele?\\
have.\textsc{ind.prs.2sg} have.\textsc{ptcp} news of.him\\
\glt ‘Have you received any news about him?’
\ex\label{ex:rb21}
Júnior França, \textit{Caiu o ministério!}, 1883\\
\gll Que empregos \textbf{tem} \textbf{exercido}?\\
what work have.\textsc{ind.prs.2sghon} do.\textsc{ptcp}\\
\glt ‘What kind of work have you been doing?’
\z

\section{From nominal to verbal pluractionality}\label{sec:rb5}

Having established a typology of the pluractional readings of the PPC and the contextual parameters associated with these readings in the nominal and verbal domains, we are now in a position to analyze the historical distribution of the PPC in terms of these contextual parameters. As argued in \sectref{sec:rb2}, the hypothesis that the semantic reanalysis of the PPC involved a transfer of the pluractional semantics from the nominal to the verbal domain makes the prediction that the use of the PPC was more frequent in nominal pluractionality contexts in earlier texts, whereas over time, its use came to be preferred in verbal pluractionality contexts.

As was already suggested in \sectref{sec:rb2}, sentences involving the PPC can be ambiguous between a resultative and a pluractional reading, and it is exactly these bridging contexts that are assumed to be at the heart of the semantic reanalysis of the PPC as a marker of pluractionality. Indeed, such ambiguous cases are a necessary part of any semantic change. As a result, a seemingly direct approach towards modeling the semantic change experienced by the PPC, namely annotating by semantic function as was done in \citeauthor{Barbosa2014}’s (\citeyear{Barbosa2014}) study, is prone to subjectivity. In other words, it is possible that another researcher might come to different results regarding the annotation of the same data (cf. \citeauthor{Rosemeyer2016a} \citeyear{Rosemeyer2016a} for discussion).

Consequently, we adopted an indirect approach to the annotation of pluractionality in our data. Thus, we established aggregate variables for nominal and verbal pluractionality on the basis of a bottom-up categorization of each sentence in the data according to the contextual parameters identified as favoring these interpretations. \tabref{tab:rb2} and \ref{tab:rb3} summarize the annotation processes for these two variables, \textsc{NominalPluractionality} and \textsc{VerbalPluractionality}.

\begin{table}[p]
%\small
\begin{tabularx}{\textwidth}{Qp{.35\textwidth}}
\lsptoprule
{Condition} & {Transformation} {for} \newline\textsc{NominalPluractionality} \\\midrule
Subject inflected for plural (\textit{n} = 91) &  +1\\
\tablevspace
Theme refers to a distributive or plural referent (\textit{n} = 112) &  +1\\
\tablevspace
Indirect object refers to a distributive or plural referent (\textit{n} = 13) &  +1\\
\lspbottomrule
\end{tabularx}
\caption{Summary of the annotation process for \textsc{NominalPluractionality}}
\label{tab:rb2}
\end{table}

\begin{table}[p]
%\small
\begin{tabularx}{\textwidth}{Qp{.35\textwidth}}
\lsptoprule
 {Condition} & {Transformation} {for} \newline\textsc{VerbalPluractionality}  \\\midrule
Presence of an aspectual or temporal adverbial that expresses internal or external quantification (\textit{n} = 198)  &  +1\\
\tablevspace
Presence of a locative adverbial that implies quantification (\textit{n} = 13)  &  +1\\
\tablevspace
Durative predicate type (\textit{n} = 349)  & +1 \\
\lspbottomrule
\end{tabularx}
\caption{Summary of the annotation process for \textsc{VerbalPluractionality}}
\label{tab:rb3}
\end{table}

\begin{table}[p]
%\small
\begin{tabularx}{\textwidth}{lYYYY}
\lsptoprule
 & 0 & 1 & 2 & 3\\
\midrule
\textsc{NominalPluractionality} & 581 & 143 & 24 & 3 \\
\textsc{VerbalPluractionality} & 394 & 346 & 11 & 0\\
\lspbottomrule
\end{tabularx}
\caption{Distribution of the variables \textsc{NominalPluractionality} and \textsc{VerbalPluractionality}}
\label{tab:rb4}
\end{table}


\tabref{tab:rb4} describes the distribution of the resulting numerical variables \textsc{NominalPluractionality} and \textsc{VerbalPluractionality}. It demonstrates that nominal pluractionality contexts are less frequent than verbal pluractionality contexts, in that 77.4 percent of all tokens of the PPC received a score of 0 on the \textsc{NominalPluractionality} variable, whereas only 52.5 percent of all tokens of the PPC received a score of 0 on the \textsc{VerbalPluractionality}. 



Having established the operationalization of the variables \textsc{NominalPluractionality} and \textsc{VerbalPluractionality}, we proceeded to analyze the changes in the historical distribution of the PPC according to these two variables. \figref{fig:rb3} visualizes this distribution. High values of \textsc{NominalPluractionality} (upper plot) are frequent in 19\textsuperscript{th} century plays, where around 25 percent of all PPC tokens reach a score of at least one on this variable. In contrast, after the 20\textsuperscript{th} century, mean \textsc{NominalPluractionality} decreases notably. This linear trend reached statistical significance.\footnote{\label{fn:rb7}Statistical significance was tested using an ordinal logistic regression model (\citeauthor{JohnsonAlbert2004} \citeyear{JohnsonAlbert2004}, \citeauthor{Agresti2010} \citeyear{Agresti2010}) predicting \textsc{NominalPluractionality} from \textsc{Year}. The effect of \textsc{Year} reached statistical significance (log odds = $-$0.005, standard error = 0.002, $p<0.05$*). The modeling was realized in R \parencite{RDevelopmentCoreTeam2019}, using the ordinal package \parencite{Christensen2019}. We only tested for global significance of the trend, not significance of components.}

\begin{figure}
\includegraphics[width=\textwidth]{figures/RosBeck3.png}
\caption{Scores on the \textsc{NominalPluractionality} and \textsc{VerbalPluractionality} variables, by 20-year periods}
\label{fig:rb3}
\end{figure}

For \textsc{VerbalPluractionality}, the overall trend is somewhat more complex. In particular, we find an increase between 1830 and 1929, followed by a sharp decrease between 1930 and 1949, which is in turn again followed by an increase. Again, this distribution reaches statistical significance.\footnote{Statistical significance was tested using an ordinal logistic regression model predicting \textsc{VerbalPluractionality} from \textsc{Year}. \textsc{Year} was modeled as a third-degree polynomial in order to capture the nonlinearity of the trend visualized in \figref{fig:rb3}. Two of the three components of the third-degree polynomial reached statistical significance (log odds\textsubscript{poly1} = 4.733, standard error \textsubscript{poly1} = 2.031, $p \textsubscript{poly1}<0.05$*; log odds\textsubscript{poly2} = 1.141, standard error \textsubscript{poly2} = 1.026, $p \textsubscript{poly2}>0.05$; log odds\textsubscript{poly3} = 6.981, standard error \textsubscript{poly3} = 3.432, $p \textsubscript{poly3}<0.001$***). Cf. also \tabref{tab:rb5} in \sectref{sec:rb6}.}

\begin{sloppypar}
The historical distribution of the PPC in terms of the variables \textsc{NominalPluractionality} and \textsc{VerbalPluractionality} seems to confirm  \citegen{AmaralHowe2012} hypothesis that the semantic change towards pluractionality involved a transfer from nominal to verbal pluractionality. In particular, we find that nominal pluractional contexts are especially frequent in earlier plays. \figref{fig:rb3} suggests that as the frequency of verbal pluractional contexts increases, the frequency of nominal pluractional contexts decreases. However, there is one section of the data that does not seem to conform to this interpretation, namely the time period between 1930 and 1990. In this period, verbal pluractionality is less frequent than at the beginning of the 20\textsuperscript{th} century. The nonlinearity of the semantic change of the PPC strongly suggests that, apart from purely semantic factors, sociolinguistic factors may have been at work. As we shall see in \sectref{sec:rb6}, one candidate parameter that might explain this distribution is register.
\end{sloppypar}

\section{The role of register}\label{sec:rb6}

Recent approaches to modeling historical change using diachronic corpus data have demonstrated that apparent gaps and inconsistencies in historical trends are frequently the result of changes in the textual corpus from which the analyzed data is taken. For instance, in his analysis of the historical development of the Brazilian Portuguese system of \textit{wh}-interrogatives in data taken from the PorThea corpus, \citet{Rosemeyer2019a} documents an increase in the usage frequency of bare \textit{wh}-interrogatives such as \textit{O que?} ‘What?’. However, he argues that the use of such bare interrogatives is unlikely to undergo any major semantic or syntactic changes, as bare interrogatives are always strongly dependent on the previous context (e.g., the question \textit{Onde?} ‘where’ needs an antecedent in order for the hearer to be able to reconstruct a complete proposition such as \textit{Onde a gente viu a Maria}? ‘Where did we see Maria?’). Bare interrogatives are shown to be more typical for spoken language and texts that can be characterized as conceptually oral, i.e. approximating spoken language to a greater degree. Consequently, Rosemeyer tests the hypothesis that the increase in the usage frequency of bare interrogatives is due to a genre change in the corpus of theater plays; over time, theater plays have come to approximate orality to a greater degree, which is why bare interrogatives are used more frequently. Such a genre change can also be described as a change from more formal to less formal register. 

\citet[175]{Rosemeyer2019a} establishes a bottom-up measurement of the degree to which the theater plays approximate orality in the PorThea corpus. On the basis of the operationalization of the “involvement” dimension established in \citet{BiberFinegan2004}, the orality variable was defined as the joint log-transformed normalized usage frequency of a number of linguistic variables, namely (a) “private” verbs in present tense singular (e.g. \textit{achar} ‘to mean’ and \textit{pensar} ‘to think’ etc.), (b)~present progressives, (c) demonstrative neuter pronouns (\textit{isso} and \textit{isto} ‘this’), (d)~time and place adverbs (\textit{aqui} ‘here’ and \textit{agora} ‘now’) and (e) discourse markers such as \textit{bom} ‘well’ or \textit{pois} ‘so’). We assume with \citeauthor{BiberFinegan2004} that the use of these linguistic variables is typical for conceptually oral texts. The higher the score of a text for the resulting variable \textsc{logOrality}, the more a text is expected to approximate orality.

By controlling for \textsc{logOrality} while describing the changes in the usage frequency of bare interrogatives in the PorThea corpus, \citet{Rosemeyer2019a} is able to demonstrate that the frequency increase disappears when register is taken into account. In other words, the increase in the usage frequency of bare interrogatives is an “apparent change” that depends entirely on the composition of the textual corpus.

In order to investigate the possibility that the semantic change of the PPC towards verbal pluractional readings was an apparent change, we applied \citeauthor{Rosemeyer2019a}’s (\citeyear{Rosemeyer2019a}) variable \textsc{logOrality} to our data. Consequently, we were able to investigate whether the semantic change of the PPC towards verbal pluractional readings represents an actual change (in the sense that it is not due to changes in register over time) or apparent change (in the sense that the change is explained by the historical development of the genre of theater texts).

In addition, applying the variable \textsc{logOrality} to our data allowed us to gauge to which degree the semantic change towards verbal pluractionality readings is moderated by register. This is important because it gives an understanding of whether the change originated in texts with a high degree of conceptual orality or in texts with a low degree of conceptual orality. In the first case, the change can be classified as a “change from below”, i.e. a change that seems to have originated in spoken language first and then spread to more formal registers. In the second case, the change would represent a “change from above”, i.e. a change that originated in more formal registers and then spread to lower registers.

Let us first turn to the question of whether the semantic change towards plur-actionality constitutes actual or apparent change. As was mentioned in the description of the case study of bare interrogatives in \citet{Rosemeyer2019a}, in order for a change to be “explained away” by register differences, there needs to be a correlation between the variable whose distribution is being analyzed and the degree of conceptual orality. Only a weak marginally significant correlation was found for the variables \textsc{VerbalPluractionality} and \textsc{logOrality}.\footnote{Correlation testing was done using Kendall’s τ because both variables were not normally distributed \parencite[213]{Gries2009}. No significant correlation between \textsc{logOrality} and \textsc{VerbalPlur- actionality} was found (Kendall’s $\tau = -0.039$, $z = 1.317$, $p\textsubscript{one-tailed} < 0.1$).} As a result, it seems unlikely that the change towards the expression of verbal pluractionality found in the data is an artifact of the composition of the corpus.

In contrast, our analysis did find evidence for the assumption that the semantic change towards pluractionality was moderated by the degree to which the theater plays approximate spoken language. We expanded the ordinal logistic regression predicting \textsc{VerbalPluractionality} from \textsc{year}, whose results were presented in \sectref{sec:rb5} (see footnote~\ref{fn:rb7}), by including an interaction effect between \textsc{year} and \textsc{OralityHigh}. The variable \textsc{OralityHigh} assumed a value of “False” for low orality texts (where the value of \textsc{logOrality} was below the mean value of \textsc{log- Orality}) and “True” for high orality texts (where the value of \textsc{logOrality} was above the mean value of \textsc{logOrality}). \tabref{tab:rb5} summarizes the results from this model, including the original baseline model for comparison.

\begin{table}
\small
\begin{tabular}{l *3{S[table-format=1.1]} S[table-format=<1.3] S[table-format=-1.1] S[table-format=1.1] S[table-format=-1.1] S[table-format=<1.2]}
\lsptoprule
{Variable} & \multicolumn{4}{c}{{Baseline model}} & \multicolumn{4}{c}{{Extended model}} \\\cmidrule(lr){2-5}\cmidrule(lr){6-9}
 & {LO} & {SE} & {$z$} & {$p$} &  {LO} & {SE} & {$z$} & {$p$}\\ \midrule
 Main effects             &  &  &  &  &  &  &  &\\
 poly(\textsc{Year}, 3)1 & 4.7 & 2.0 & 2.3 & <.054 & 0.1  & 3.4 & 0.0  & >.05\\
 poly(\textsc{Year}, 3)2 & 1.1 & 2.0 & 0.6 & <.05  & -2.7 & 2.9 & -1.0 & >.05\\
 poly(\textsc{Year}, 3)3 & 7.0 & 2.0 & 3.4 & <.001 & 10.4 & 3.2 & 3.2  & <.01\\
 \textsc{OralityHigh}    &     &     &      &      & -0.3 & 0.2 & -1.7 & <.1\\
 \midrule
 Interaction effects      &  &  &  &  &  &  &  &\\
 poly(\textsc{Year}, 3)1  &  &  &  &  & 6.9 & 4.4 & 1.6 & >.05\\
  \textsc{OralityHigh}    &  &  &  &  & & & & \\
 poly(\textsc{Year}, 3)2\ &  &  &  &  & 9.7 & 4.4 & 2.2 & <.05\\
  \textsc{OralityHigh}    &  &  &  &  & & & & \\
 poly(\textsc{Year}, 3)3  &  &  &  &  & -7.6 & 4.3 & -1.8 & <.1\\
 \textsc{OralityHigh}     &  &  &  &  & & & & \\
\lspbottomrule
\end{tabular}
\caption{Summary of results from the two ordinal logistic regression models measuring the correlation between \textsc{VerbalPluractionality}, \textsc{Year}, and \textsc{OralityHigh}. Abbreviations: LO = Log odds, SE = standard error. \textsc{Year} was modeled as a third-degree polynomial using the function poly() in order to account for the fact that the increase in verbal pluractionality is a non-linear trend.}
\label{tab:rb5}
\end{table}

As evident in \tabref{tab:rb5}, the interaction between \textsc{Year} and \textsc{OralityHigh} reached statistical significance (cf. the last three lines in the table), which means that the degree of conceptual orality of the theater plays indeed significantly moderated the semantic change towards verbal pluractionality.

A pair-wise model comparison using ANOVA \parencite[see][285--293 for application]{Gries2013} found the extended model to explain significantly more variation than the baseline model, which also justifies inclusion of \textsc{OralityHigh} as a main and interaction effect. 

 \figref{fig:rb4} visualizes the historical distribution of verbal pluractionality when distinguishing between low and high orality texts, resulting from the interaction between \textsc{Year} and \textsc{OralityHigh} in the extended ordinal logistic regression model described in  \tabref{tab:rb5}.\footnote{The effect plots were produced using the effects package \parencite{FoxHong2009}.} It demonstrates significant differences in the semantic change of the PPC towards verbal pluractionality in texts scoring low or high on the dimension of conceptual orality. In particular, the curious “two-wave” distribution of the development towards verbal pluractionality documented in \sectref{sec:rb5} (see  \figref{fig:rb3}) can be explained in terms of register differences. Thus, the first increase in verbal pluractionality, between about 1840 and 1880, is restricted to low-orality texts; in high-orality texts we do not find an increase in verbal plur- actionality contexts until 1950. However, between 1880 and 1975 the trend towards verbal pluractionality in low-orality texts is actually inverted, such that the PPC is used less in verbal pluractionality contexts, evening out the differences between low- and high-orality texts. After 1950 (1975 for low-orality texts), we find a relatively uniform increase in verbal pluractionality in both low- and high-orality texts. In present-day theater plays, the degree of conceptual orality does not seem to affect the frequency of verbal pluractionality contexts.

\begin{figure}
\includegraphics[width=\textwidth]{figures/RosBeck4.png}
\caption{Effect plot for the interaction between \textsc{Year} (as 10-year periods) and \textsc{HighOrality} in the ordinal logistic regression model predicting \textsc{VerbalPluractionality}}
\label{fig:rb4}
\end{figure}


\section{Discussion and conclusions}\largerpage

In this paper, we have modeled the semantic change of the Brazilian Portuguese PPC from resultative to pluractional readings. We tested the hypothesis that nominal pluractionality contexts served as bridging contexts for this change, leading to a transfer of the pluractionality reading from nominal to verbal properties. Using quantitative data from a corpus of BP theater plays, we have been able to verify this hypothesis. Our data clearly shows a preference for the PPC to appear in contexts associated with nominal pluractionality in earlier stages of the change. Over time, its use became more likely in contexts associated with verbal pluractionality. The results from this paper thus confirm assumptions from previous studies based on qualitative analyses.

Closer inspection of the trend in terms of usage frequencies, type-token ratios and the degree of conceptual orality of a text has revealed that it proceeded in two qualitatively different phases. Thus, the PPC came to be associated more strongly with verbal pluractionality contexts in a first phase between 1840 and 1880. However, this change was restricted to low-orality texts, which correspond to a more formal register. Note also that the semantic change was correlated to a decreasing usage frequency of the PPC. This is surprising given that grammaticalization processes are usually expected to involve an increase in usage frequency (grammaticalization involves an extension of the use of the construction to new usage contexts, which leads to an increase in overall usage frequency). Indeed, inspection of the development of the type-token ratios of the PPC per year demonstrated that the mid-19\textsuperscript{th} century marks the beginning of the extension of the use of the PPC to new verb types and consequently, an increase in productivity.

How can we explain the fact that the PPC decreased in frequency at the same time that its productivity increased due to the semantic change towards pluractionality? One possible explanation, which was already alluded to at the end of \sectref{sec:rb2}, is contact. Consider the periodization of Brazilian Portuguese established in \citet{Galves2007}. Galves provides a summary of historical studies on several aspects of the grammar of European and Brazilian Portuguese. On the basis of this summary, she claims that BP grammar changed significantly with respect to EP in the first two centuries of colonization, which is why some studies find evidence for the emergence of a new grammar already in 18\textsuperscript{th} century texts.\largerpage

However, there is also evidence for a revival of EP influence on BP grammar in the 19\textsuperscript{th} century. Galves quotes the studies by \citet{Carneiro2005} and \citet{Pagotto1992}, which analyze the historical distribution of pronoun position (enclitic vs. proclitic). While BP had changed from preferred enclitic to proclitic position during the 19\textsuperscript{th} century, these authors document a return to preferred enclitic position. The three authors agree in attributing this change to the strong cultural influence of Portugal at the time of the so-called \textit{Império do Brasil} (1822--1889) and the first decades of the \textit{República Velha} (from 1889 onwards). Official documents (such as the Constitution of 1891 and other administrative documents, but also literary prose, aspired to imitate and even to exceed the \textit{norma culta} of the European Portuguese \parencite[see][51--53]{Pagotto1998}. The huge impact of the European Portuguese norm is evident especially in the text of the First Republican Constitution of 1891, which \citeauthor{Pagotto1992} (\citeyear{Pagotto1992}, \citeyear{Pagotto1998}) compared with the Constitution of the Empire of 1824 (\textit{Constitução do Império}). He concludes that the 1824 Constitution, in contrast to the 1891 Constitution, favors proclitic pronoun position and is, therefore, still closer to the classical Portuguese norm. In contrast, the 1891 Constitution clearly prefers enclitic pronoun position and thus consistently follows the European norm of the time, which tends towards a generalization of the enclitic pronominal position even in contexts where the proclisis was still common in classical language \parencite[51--53]{Pagotto1998}. 

Our data documenting the semantic evolution of the PPC seems to evince a similar tendency of increasing EP influence. As we have seen, several findings from this paper support the hypothesis that the semantic change of the PPC towards expression of pluractionality was fostered by the influence of EP grammar during the 19\textsuperscript{th} century. First, this hypothesis requires as a premise that the PPC displays a stronger tendency to express pluractionality in EP than in BP. The periodization analysis (see \sectref{sec:rb3}) may be interpreted as evidence for this premise. In particular, the usage frequency of the PPC is significantly higher in 20\textsuperscript{th} century EP texts than in BP texts, which might suggest that the semantic change has been implemented to a greater degree in EP than in BP. Second, such language contact is unlikely to have affected low register and informal language. Rather, we would expect this change to affect more formal language, which is what we find in this study. In particular, we only document a semantic change of the PPC towards pluractionality in more formal theater plays, i.e. plays that do approximate the language spoken in Brazil at that time. Note also that after the end of the \textit{Império do Brasil}, our data actually suggests a “de-pluractionalization” of the PPC in formal theater plays; in the first half of the 20\textsuperscript{th} century, authors seem to have reverted to the Brazilian Portuguese norm of using the PPC. Third, the assumption of a contact-induced change towards pluractionality is compatible with the finding of an overall decrease in the usage frequency of the PPC.  \figref{fig:rb5} uses the orality measure established in  \sectref{sec:rb6} to model more finely the development of usage frequencies in our 19\textsuperscript{th} century data. It demonstrates that the decrease in the usage frequency of the PPC in our 19\textsuperscript{th} century data is actually restricted to high-orality texts; in low-orality texts, the usage frequency of the PPC remains roughly similar until about 1880. It is only after 1880 that the use frequency of the PPC starts to decrease also in low-orality texts. Note that the year 1880 was also identified as the turning point with respect to the semantic change towards pluractionality in low-orality texts (see the discussion of  \figref{fig:rb4} in  \sectref{sec:rb5}), in that after the end of the 19\textsuperscript{th} century, verbal pluractionality readings actually became less frequent in low-orality texts. These findings seem to fit an explanation in terms of contact between EP and BP quite well.

\begin{figure}
\includegraphics[width=\textwidth]{figures/RosBeck5.png}
\caption{Historical development of the log-transformed usage frequency per 100,000 tokens of the PPC in the Brazilian section of the PorThea corpus, by orality. Points represent mean usage frequencies per year, whereas the line represents results of local polynomial regression analyses summarizing the trend.}
\label{fig:rb5}
\end{figure}

Let us now turn to the second phase of the semantic change, i.e. the increase in verbal pluractionality readings after 1950. This increase in verbal pluractionality was not found to be moderated by the degree of conceptual orality of the texts; in other words, the likelihood for the PPC to be used in contexts associated with verbal pluractionality readings increases both in low- and high-orality texts. At the same time, however, we find a decrease in the usage frequency of the PPC, as well as its type-token ratio, which strongly suggests a general decrease in productivity of the construction. Consequently, it appears that after about 1950, the PPC has experienced a specialization process by which its use has gradually been restricted to contexts that are strongly associated to verbal pluractional readings. The restriction of constructions to such functional niches is a hallmark of conservation processes, for instance in situations of language change in which one construction is being replaced by another, competing construction \parencite[see][]{Rosemeyer2016b}. In this case, it stands to reason that the gradual specialization of the PPC is due to the competition with the simple past (henceforth PPS). According to the variationist analysis by \citet{Barbosa2014}, already cited in  \sectref{sec:rb2}, the PPS is gradually ousting the PPC in BP. Thus, \citeauthor{Barbosa2014}’s (\citeyear{Barbosa2014}) data suggest an increase of the frequency of the PPS relative to the PPC of 85.6 percent in the 18\textsuperscript{th} century to 89.2 percent in the 19\textsuperscript{th} century and an almost-categorical 94.8 percent in the 20\textsuperscript{th} century. Another hallmark of historical replacement processes is that the usage of the competing construction can even end up expanding to those functional niches in which the replaced construction seems to still thrive. There is some evidence for the assumption that in BP the PPS is starting to be used in verbal pluractionality contexts, a change that might lead to the complete elimination of the PPC from BP grammar. The examples in (\ref{ex:rb22}--\ref{ex:rb23}) are taken from BP texts from the News on the Web (=NOW) section of the CdP, which includes 1.1 billion words from internet news texts dated between 2012 and 2019. In these examples, the PPS is used in contexts that clearly indicate verbal pluractionality and where, according to BP grammars, use of the PPC would be expected.

\ea\label{ex:rb22}
    BP examples of the syntagm \textit{fiz até agora} from the CdP, section NOW\\
 \ea\label{ex:rb22a}
 \gll Eu avalio com muita felicidade e gratidão tudo o que \textbf{fiz} \textbf{até} \textbf{agora}\\
 I assess.\textsc{ind.prs.1sg} with much happiness and gratitude everything \textsc{det.def.m.sg} that do.\textsc{ind.pst.pfv.1sg} until now\\
 \glt ‘I see everything I have achieved until now with much happiness and gratitude’\\
  \ex\label{ex:rb22b}
  \gll Venho apresentar minha defesa e dizer a verdade, como sempre \textbf{fiz} \textbf{até} \textbf{agora}\\
  come.\textsc{ind.prs.1sg} present my defense and say.\textsc{inf} \textsc{det.def.f.sg} truth, like always do.\textsc{ind.pst.pfv.1sg} until now\\
  \glt ‘I will now present my defense and tell the truth, as I have always done until now’

   \ex\label{ex:rb22c}
   \gll o terceiro álbum está a caminho e foi o melhor que já \textbf{fiz} \textbf{até} \textbf{agora}\\
   \textsc{det.def.m.sg} third album be.\textsc{prs.3sg} to way and be.\textsc{pst.pfv.3sg} \textsc{det.def.m.sg} best that already do.\textsc{ind.pst.pfv.1sg} until now\\
   \glt ‘The third album is on its way and it was [sic] the best that I have done until now’
 \z
\z

\ea\label{ex:rb23}
BP examples of the syntagm \textit{vivi até agora} from the CdP, section NOW\\
 \ea\label{ex:rb23a}
  \gll Deram- me um ano de vida, exatamente o que \textbf{vivi} \textbf{até} \textbf{agora}\\
  give.\textsc{ind.pst.pfv.3pl} me one year of life exactly \textsc{det.def.m.sg} that live.\textsc{pst.pfv.1sg} until now\\
  \glt ‘They gave me one [remaining] year to live, which is exactly how long I have lived until now’
  \pagebreak
  \ex\label{ex:rb23b}
  \gll “Talvez eu tenha mais vida para viver do que eu já \textbf{vivi} \textbf{até} \textbf{agora}”, brinca\\
  maybe I have.\textsc{sbj.prs.1sg} more life to live.\textsc{inf} of.\textsc{det.def.m.sg} that I already live.\textsc{pst.pfv.1sg} until now joke.ind.prs.3sg\\
  \glt ‘“Maybe I will have more life to live than I have lived until now", s/he jokes’
\z
\z

\tabref{tab:rb6} summarizes the distribution of the distribution of the PPS and the PPC in the contexts of the adverbial \textit{até agora} (‘until now’) in the NOW section of the CdP. 

\begin{table}
\begin{tabular}{l rr rr r}
\lsptoprule
{Query/Type}                 & \multicolumn{2}{c}{BP} & \multicolumn{2}{c}{EP} & {Total}\\\cmidrule(lr){2-3}\cmidrule(lr){4-5}
                             & \multicolumn{1}{c}{$n$} & \multicolumn{1}{c}{\%} & \multicolumn{1}{c}{$n$} & \multicolumn{1}{c}{\%} & \\\midrule
\_vis\% até agora (= PPS)    & 2974 & 90.0\%  & 2271 & 76.9\%  & 5245\\
TER \_vps* até agora (= PPC) & 297  & 9.1\%   & 684 & 23.1\%   & 981\\\addlinespace
Total                        & 3271 &         & 2955 &          & \\\midrule
\multicolumn{6}{c}{$\chi^2(1)=230.39, p<0.001$***}\\
\lspbottomrule
\end{tabular}
\caption{Summary of the distribution of the PPS and the PPC in the contexts of the adverbial \textit{até agora} (‘until now’) in the NOW section of the CdP. Percentages refer to the relative frequencies of the PPS and PPC within each dialect.}
\label{tab:rb6}
\end{table}
 
For the BP section of the data, \tabref{tab:rb6} demonstrates that in these contexts, which strongly suggest pluractional readings, the PPS is used in more than 90 percent of the cases, with the PPC relegated to a clear minority variant. While the overall pattern of the distribution is similar in EP, the asymmetry between the PPS and the PPC is less marked given that the PPC is used in about 23 percent of the cases. This finding, which reaches significance according to a $\chi$\textsuperscript{2} test, suggests that the PPC is more strongly established in verbal pluractionality contexts in EP than in BP and seems to resist replacement with the PPS to a greater degree than in EP. 

\pagebreak
{\sloppy\printbibliography[heading=subbibliography,notkeyword=this]}
\end{document}
