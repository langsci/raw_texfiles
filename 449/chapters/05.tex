\documentclass[output=paper,colorlinks,citecolor=brown]{langscibook}
\ChapterDOI{10.5281/zenodo.13759988}
\author{Olga Kellert\orcid{}\affiliation{University of Göttingen} and Andrés Enrique-Arias\orcid{}\affiliation{University of the Balearic Islands}}

\title{On the diachrony of Catalan indefinite \textit{qualsevol}}

\abstract{In this investigation, we aim to characterize the morphosyntactic and semantic properties of Catalan \textit{qualsevol} in diachrony. On the basis of almost 8000 examples extracted from texts encompassing a period from the 13\textsuperscript{th} to the 20\textsuperscript{th} century, we look at variables such as agreement properties, position with respect to the noun, grammatical function, and co-occurrence with strong determiners and quantifiers. Moreover, we analyze the historical evolution of the semantic interpretation of \textit{qualsevol}. This way we are able to trace a grammaticalization path for Catalan \textit{qualsevol} that is similar to the one proposed in the literature for Spanish \textit{cualquiera} \citep[see][]{CompanyCompanyPozasLoyo2009}. More specifically, our analysis reveals that in both languages \textit{qualsevol} and \textit{cualquiera} gradually changed their grammatical functions and semantic interpretations as a result of their origin as a relative clause. Due to the modificational function of the relative clause, it was reanalyzed as a nominal modifier similar to an adjective. When occurring in a prenominal position, the modifier was further reanalyzed as a determiner, whereas the postnominal one preserved its function as a modifier. While there are many works on similar items in Romance, English and other languages, this is the first systematic analysis of the diachrony of Catalan \textit{qualsevol}.}


\IfFileExists{../localcommands.tex}{
   \addbibresource{../localbibliography.bib}
   % add all extra packages you need to load to this file

\usepackage{tabularx,multicol}
\usepackage{url}
\urlstyle{same}

\usepackage{listings}
\lstset{basicstyle=\ttfamily,tabsize=2,breaklines=true}

\usepackage{langsci-basic}
\usepackage{langsci-optional}
\usepackage{langsci-lgr}
\usepackage{langsci-osl}
% \usepackage{./langsci/styles/langsci-lgr}
% \usepackage{./langsci/styles/langsci-osl}
% \usepackage{langsci-gb4e}

\usepackage{tikz}
\usetikzlibrary{patterns,calc}
\pgfdeclarepatternformonly{south east lines}{\pgfqpoint{-0pt}{-0pt}}{\pgfqpoint{3pt}{3pt}}{\pgfqpoint{3pt}{3pt}}{
    \pgfsetlinewidth{0.6pt}
    \pgfpathmoveto{\pgfqpoint{0pt}{3pt}}
    \pgfpathlineto{\pgfqpoint{3pt}{0pt}}
    \pgfpathmoveto{\pgfqpoint{.2pt}{-.2pt}}
    \pgfpathlineto{\pgfqpoint{-.2pt}{.2pt}}
    \pgfpathmoveto{\pgfqpoint{3.2pt}{2.8pt}}
    \pgfpathlineto{\pgfqpoint{2.8pt}{3.2pt}}
    \pgfusepath{stroke}}
    
\usepackage{stmaryrd}
\usepackage{wasysym}
\usepackage{multirow}
\usepackage{caption}
\usepackage{subcaption}
\usepackage{mathrsfs}
\usepackage{qtree}

\usepackage{linguex}


   %pminos do not split footnotes
% \interfootnotelinepenalty=10000 %Footnote in Laporte chapters has to be split SN


%\DeclareIndexNameFormat{default}{%
%\nameparts{#1}%
%\usebibmacro{index:name}%
%{\index[names]}%
%{\namepartfamily}%
%{\namepartgiveni}%
% {}% L1
% {}% L2
%{\namepartprefix}% generates spurious space L3
%{\namepartsuffix}% generates spurious space L4
%}

%  {\DeclareIndexNameFormat{default}{%
%     \usebibmacro{index:name}{\index[names]}{#1}{#3}{#5}{#7}}}

%\DeclareIndexNameFormat{default}{%
%  \usebibmacro{index:name}{\sindex[nom]}{#1}{#3}{#5}{#7}}

%\DeclareIndexNameFormat{default}{%
%  \usebibmacro{index:name}{\sindex[person]}{#1}{#3}{#5}{#7}}
%\DeclareIndexNameFormat{default}{%
%\nameparts{#1} \usebibmacro{index:name}{\sindex[person]]}{\namepartfamily}{‌​\namepartgiven}{\nam‌​epartprefix}{\namepa‌​rtsuffix}}

%\newcommand{\smiley}{:)}

%\renewbibmacro*{index:name}[5]{%
%\usebibmacro{index:entry}{#1}%
%{\iffieldundef{usera}{}{\thefield{usera}\actualoperator}\mkbibindexname{#2}{#3}{#4}{#5}}}

% \newcommand{\noop}[1]{}

%remove for final
%\overfullrule=1mm

\newcommand{\tobi}[2]}}
\renewcommand{\S}[1]{\tobi{#1}{\textsc{*}}}

% this volume references
% puts: [this volume]
% already defined: \citetv
%\newcommand{\citepv}[1]{(\citeauthor{#1} \citeyear*{#1} [this volume])}
\newcommand{\citealtv}[1]{\citeauthor{#1} \citeyear*{#1} [this volume]}

%parentheses around example number
\newcommand{\pref}[1]{(\ref{#1})}

% in-text examples

\newcommand{\lnex}[1]{\textit{#1}} %target lang word
\newcommand{\lnlit}[1]{(lit.: `#1')} %literal reading
\newcommand{\lnlat}[1]{(#1)} % latinization
\newcommand{\lntrans}[1]{`#1'} %translation
\newcommand{\lnexl}[2]%
{\lnex{#1}{} \lnlat{#2}} % ex with latinization
\newcommand{\lnexlat}[3]{\lnex{#1}{} \lnlat{#2}{} \lntrans{#3}} % ex with latinization and tranl.

%ch01
\newcommand{\co}[1]{\mbox{\textbf{#1}}}

%ch09

\newcommand{\cyrbulg}[1]{\begin{otherlanguage*}{bulgarian}#1\end{otherlanguage*}}


%ch10
\newcommand{\nlp}{{\small NLP}}
\newcommand{\mwe}{{\small MWE}}
\newcommand{\rae}{{\small RAE}}
\newcommand{\lvc}{{\small LVC}}
\newcommand{\pos}{{\small P}o{\small S}}
%\newcommand{\todo}[1]{ \textcolor{red}{#1} }

%\renewcommand{\labelenumi}{\theenumi}
%\ainamefmt{{vv}{ll}{, ff}{, jj}} % fullname

\newcommand{\biberror}[1]{{\color{red}#1}}

\newcommand{\osenovaitem}{--~}
   %% hyphenation points for line breaks
%% Normally, automatic hyphenation in LaTeX is very good
%% If a word is mis-hyphenated, add it to this file
%%
%% add information to TeX file before \begin{document} with:
%% %% hyphenation points for line breaks
%% Normally, automatic hyphenation in LaTeX is very good
%% If a word is mis-hyphenated, add it to this file
%%
%% add information to TeX file before \begin{document} with:
%% %% hyphenation points for line breaks
%% Normally, automatic hyphenation in LaTeX is very good
%% If a word is mis-hyphenated, add it to this file
%%
%% add information to TeX file before \begin{document} with:
%% \include{localhyphenation}
\hyphenation{
    Beck-man
    Ngu-yen
    back-chan-nel
    back-chan-nels
    mo-not-o-nous
    ste-reo-typ-i-cal
}

\hyphenation{
    Beck-man
    Ngu-yen
    back-chan-nel
    back-chan-nels
    mo-not-o-nous
    ste-reo-typ-i-cal
}

\hyphenation{
    Beck-man
    Ngu-yen
    back-chan-nel
    back-chan-nels
    mo-not-o-nous
    ste-reo-typ-i-cal
}

   \boolfalse{bookcompile}
   \togglepaper[23]%%chapternumber
}

\begin{document}
\tikzstyle{arrow} = [thick,->,>=stealth]
\maketitle

\section{Introduction} 

The Catalan indefinite \textit{qualsevol} is a Free Choice Item (FCI) similar to Spanish \textit{cualquier(a)}, Italian \textit{qualunque}, French \textit{quelconque} or Romanian \textit{oarecare} (see \cite{Alonietal2010}, and references therein). Free Choice indefinites are anti-specific and express referential vagueness. Specificity is given when “a speaker uses an indefinite noun phrase and intends to refer to a particular referent” (\citealt[][10]{Heusinger2011}). Consider the following example from Spanish with a simple indefinite that refers to a particular referent, namely Dr. Smith:

\ea \label{ex:ka1}
\gll Maria se cas-ó con un médico. En concreto con el Dr. Smith.\\
Maria \textsc{ref} marry-3\textsc{sg.past} with a doctor in concrete with the Dr. Smith\\
\glt ‘Maria married some doctor. Concretely, Dr. Smith.’\\
\z

Free Choice indefinites such as \textit{cualquier médico} ‘any doctor’ cannot refer to specific individuals (see \citealt{Alonso-OvalleMenéndez-Benito2010}), i.e. \textit{cualquier médico} does not refer to a certain doctor like Dr. Smith. Instead, they convey the meaning that all alternatives count as possible. Thus, the Spanish sentence in (\ref{ex:ka2}) conveys the meaning that for every possible book of your consideration you can choose that book.
\ea \label{ex:ka2}
\gll Pued-es eleg-ir cualquier libro.\\
    can-\textsc{prs.2sg} choose-\textsc{inf} any book\\
\glt ‘You can choose any book.’\\
\z

Like in Spanish, Catalan \textit{qualsevol} in prenominal position has the indefinite modal meaning exemplified in (\ref{ex:ka3a}), which has an FCI interpretation (see \cites[29]{Alonso-OvalleMenéndez-Benito2010}[]{Rivero2011}). At the same time, this prenominal indefinite is lexically identical to the postnominal one in example \REF{ex:ka3b}. However, the latter does not have the FCI interpretation but rather an evaluative one (EVAL): in example (\ref{ex:ka3b}) the speaker qualifies the \textit{home} ‘man’ as ‘unremarkable’ (see \cites[29]{Alonso-OvalleMenéndez-Benito2010}[]{Rivero2011} for this reading in Spanish \textit{cualquiera}):\largerpage

\ea \label{ex:ka3}
\ea\label{ex:ka3a} 
    \gll Pots port-ar=me qualsevol llibre.\\
    can.\textsc{prs.2sg} bring\textsc{-inf}=me any book\\
    \glt ‘You can bring me any book.’ = Every book is a possible option (FCI)\\
\ex\label{ex:ka3b} \gll És un home qualsevol\\
    be.\textsc{prs.3sg} a man any\\
    \glt ‘He is an unremarkable man’ = The man is unremarkable (EVAL)\\
\z
\z

Like Spanish \textit{cualquiera}, Catalan \textit{qualsevol} can be used as a pronoun and as a noun (see (\ref{ex:ka4}) and (\ref{ex:ka5}) below, respectively). Note that in the noun use, \textit{cualquiera} and \textit{qualsevol} have an EVAL interpretation, just like the postnominal \textit{cualquiera} and \textit{qualsevol} in (\ref{ex:ka3b}). We will show that this difference in interpretation correlates with the +/$-$ lexical category of \textit{qualsevol} as opposed to its grammatical function. Noun and postnominal \textit{qualsevol} is a lexical category, whereas the pronoun and the determiner \textit{qualsevol} are functional\slash grammatical categories (see \sectref{sec:kea4} and \sectref{sec:kea5}).

\ea \label{ex:ka4}
\ea Cualquiera puede hac-er=lo\\
\ex\gll Qualsevol pot f-er=ho\\
    any can.\textsc{prs.3sg} do-\textsc{inf}=it\\
    \glt ‘Anybody can do it’\\
\z

\ex \label{ex:ka5}
\ea Juan es un cualquiera\\
\ex \gll Joan és un qualsevol\\
    John be.\textsc{prs.3sg} an any\\
    \glt ‘John is a unremarkable man’\\
\z
\z

While similar items in Romance, English, and other languages have received considerable attention (see \cite{CompanyCompanyPozasLoyo2009, Rivero1988} on Spanish, \cite{Becker2014} on Italian and French, \cite{Stark2006} on Italian, \cite{Gianollo2018} on Romance, among others) there is not, as of yet, a systematic description of Catalan \textit{qualsevol}. This chapter aims at filling this gap by providing an empirically supported account of the morphosyntax and semantics of the Catalan Free Choice indefinite \textit{qualsevol} in diachrony. On the basis of some 8000 examples extracted from texts encompassing the 13\textsuperscript{th} to the 20\textsuperscript{th} century, we look at variables such as agreement properties, position with respect to the noun and the verb, grammatical function, and co-occurrence with quantifiers in order to identify the syntactic and semantic status of Catalan \textit{qualsevol}, especially with respect to its stage of grammaticalization, the changes in grammatical function, and the consequences of all of the above for its semantic interpretation.

The chapter is structured as follows: first we begin by providing a short introduction into the historical uses of \textit{qualsevol} and its morphological variants as described in grammars and dictionaries with a brief note on its etymology, including the different hypotheses on its origin (see \sectref{sec:kea2}). In \sectref{sec:kea3}, we present the corpus and methodology employed in the study. We analyze the corpus data qualitatively in \sectref{sec:kea4}. Then we move on to describe the diachronic evolution of \textit{qualsevol} with special attention to its grammaticalization path from a relative clause to nominal modifiers, pronouns and nouns (see \sectref{sec:kea5} and \sectref{sec:kea6}). A summary and outlook are presented in \sectref{sec:kea7}.

\section{Catalan \textit{qualsevol} and its morphological variants}\label{sec:kea2}

As noted in reference grammars and dictionaries (cf. \cite{Alcovermoll1962}, \cite[][VI: 889]{Coromines1985}; \cite[][184]{Moll2006}), \textit{qualsevol} derives from the combination *\textsc{quale se volet}, that is, it is a compound of relative \textit{qual} ‘which’, the impersonal pronoun \textit{se} ‘one’ and third person singular present indicative of \textit{voler} ‘want’. The variant \textit{qualsevulla}, which derives from a present subjunctive of \textit{voler} (*\textsc{quale se voleat}), appears also since medieval times and is still used nowadays, but only in formal written registers.

The traditional plural forms for \textit{qualsevol} and \textit{qualsevulla} are \textit{qualssevol} and \textit{qualssevulla}, respectively, in which the relative \textit{qual} is inflected for number. However, as final -\textit{s} becomes silent when combined with the initial \textit{s}- of \textit{se vol}, plural and singular forms are indistinguishable in actual speech and therefore number distinction is just a matter of a spelling convention. In old texts the plural of \textit{qualsevol} may appear written in three separate words: \textit{quals se vol} as in (\ref{ex:ka6a}) below. Likewise it is quite common to find examples with no double \textit{ss} (i.e. \textit{qualsevol} rather than \textit{qualssevol}) accompanying a plural noun phrase; this means that from early on plural inflection for \textit{qualsevol} was only sporadic (\ref{ex:ka6b}). Also, by the late 1300s there appears a new analogical plural with -\textit{s} at the end of the whole sequence: \textit{qualsevols}, as in example (\ref{ex:ka6c}) featured in \citet{Alcovermoll1962}. This new form, which is not allowed in standard normative Catalan, is evidence that by the 14\textsuperscript{th} century speakers had reanalyzed the sequence \textit{qualsevol} as one word. The verb following \textit{vol} is an indicator that at this time it was still compositional.

\ea \label{ex:ka6}
 \ea\label{ex:ka6a}(\textit{Spill} 431, ca. 1460)\\
     \gll Quantes s-ón vive-s | qual-s se vol s-ien\\
        how.many be-\textsc{prs.3pl} alive-\textsc{pl} | which-\textsc{pl} \textsc{ref} want be.\textsc{sbjv.3pl}\\
        \glt ‘How many are alive, whichever they may be’

\ex\label{ex:ka6b}(Pere IV, Cròn. 67 (ca. 1383))\\
\gll No contrastant-s qualsevol-{\emptyset} privilegi-s\\
        not withstanding-\textsc{pl} any-{\emptyset} privilege-\textsc{pl}\\
        \glt ‘Not withstanding any privileges’
\ex\label{ex:ka6c}(Hist. Sóller, II, 23 (1370))\\
        \gll Per qualsevol-s crim-s e exceso-s\\
        for any-\textsc{pl} crime-\textsc{pl} and excess-\textsc{pl}\\
        \glt ‘For any crimes and excesses’
\z
\z

As for gender, in principle the standard feminine form for \textit{qualsevol} is identical to the masculine form, as the relative \textit{qual} has no distinct feminine form. In dialectal non-standard speech, however, there exists the feminine form \textit{qualsevola} (cf. \cite{Alcovermoll1962}), which again evidences that the erstwhile compound is treated by speakers as one word. In the next section, we discuss different hypotheses about the origin of \textit{qualsevol}.


\section{Origin of \textit{qualsevol}}\label{sec:kea3}

There is some disagreement in the literature regarding the origin of Romance indefinites such as \textit{qualsevol} (see \cite[][1068--1088]{CompanyCompanyPozasLoyo2009} and references therein for a summary and discussion centered on Spanish indefinites). According to some linguists, Romance FCIs represent the direct continuation or the adaptation via calque of Latin FCI compounds \textit{quivis, quilibet, qualislibet}, and the like, a scenario that we will call the \textit{continuation hypothesis}. For instance, \citet{MenéndezPidal1928} assumes that \textit{qual quier, qui quier, qual-se-quiera,} and so forth represent the Old Spanish equivalents of the Latin \textit{quilibet, qualis-libet,} etc., that is, indefinite relatives compounded of a pronoun and an impersonal verb. According to \citet[][57]{MeyerLübke1899}, Italian \textit{qualunque} is derived from Latin \textit{qualiscumque}, composed from \textit{qualis} ‘which’ + \textit{cumque} ‘ever’ (\textit{qualiscumque > qual[is]-unqua[m]} > Old and Modern Italian \textit{qualunque}). As for French, etymological dictionaries assume that the indefinite pronoun \textit{quiconque} has its source in \textit{qui que + onques}, which was influenced by Latin \textit{quicumque} (see \cite{Becker2014} citing \cite[][vol. 6, 511]{Godefroy2006}; \cite[525]{BlochWartburg1975}; \cite[][737]{Gamillscheg1969}; \cite[][489]{Greimas1998}; \cite[][vol. 8, 91]{ToblerLommatzsch1925}).

According to an alternative hypothesis, (i.e. the \textit{grammaticalization hypothesis}), FCIs such as Spanish \textit{cualquiera} are a new Romance structure that emerged as the result of the evolution of relative clauses containing a verb of volition. The authors supporting this hypothesis argue for a grammaticalization process in which relative clauses were reanalyzed as indefinite noun phrases (\cite{Palomo1934, Rivero1988, Haspelmath1997, GirónAlconchel2012}, \cite[][§7.5.7]{Brucart1999}, \cite{CompanyCompanyPozasLoyo2009}, among others). However, there is no consensus about how exactly the grammaticalization of \textit{cualquiera} has taken place, the reason being that the indefinite \textit{cualquiera} was already documented in the earliest Old Spanish documents. The hypothesis of the grammaticalization of \textit{cualquiera} is thus (just) a hypothesis for which one can find good arguments (at best), but not proofs. According to one suggestion of the spell out of the grammaticalization path (cf.  \figref{fig:ka1}, based on \cite[][1086]{CompanyCompanyPozasLoyo2009}, Spanish \textit{cualquiera} starts out as a free relative clause introduced by the wh-element \textit{cual} ‘which’ and an NP + the volitional verb \textit{quiera} ‘want.3\textsc{sg.sbjv}’\footnote{The mood of the basic form for the derivation of \textit{cualquier} (whether it was subjunctive \textit{quiera} or indicative \textit{quiere}) is also controversial (see \cite{Pato2012}).} (see step 1). The second step is an adjacency structure between the wh-element and the verb \textit{quiera} (see step 2). The authors assume that the adjacency structure is [\textit{\textbf{qual} quier \textbf{NP} que…}] in which \textit{qual} is separated from the noun phrase NP. The adjacency and frequency of \textit{qual quier} has the effect that this sequence becomes reanalyzed and lexicalized as one word, which is no longer perceived as clausal. The indefinite acts as an argument of the main verb \textit{haga} (step 3). The biclausal structure (i.e. the main sentence and the free relative clause) at step 1 is reanalyzed as monoclausal at step 3.\largerpage

\begin{figure}
\includegraphics[width=\textwidth]{figures/KelEnri1.png}
\caption{Grammaticalization path of \textit{cualquier} in Spanish \citep[][1086]{CompanyCompanyPozasLoyo2009}}
\label{fig:ka1}
\end{figure}

The question now is whether the evolution of \textit{qualsevol} shows signs of grammaticalization. As with \textit{cualquiera}, our ability to answer this question is limited by the lack of direct documentation of spoken Latin and proto-Romance. By the time Catalan starts to be written consistently in the 13\textsuperscript{th} century, the grammaticalization of \textit{qualsevol} as a one word compound is fairly advanced. It is not easy then to ascertain whether the Free Choice meaning of Latin indefinites has continued in Catalan with the mere substitution of the verbal component (thus Latin \textit{–vis} or \textit{–libet} would have been replaced by Catalan \textit{se vol} or \textit{se vulla}), or conversely these indefinites derive from a wider sentence structure, namely a relative clause of the type \textit{en qual lloc se vulla} ‘in any place he wants’, which evolved to \textit{en qualsevulla lloc}. However, based on the data available to us in descriptive studies of medieval Catalan and etymological dictionaries (cf. \cite{Alcovermoll1962}; \cite[][552]{BatlleOcerinjauregui2016}; \cite[][184]{Moll2006}; \cite[][VI: 889]{Coromines1985}) we are more inclined to accept the Romance innovation scenario. Indeed \textit{qual se vol}-type compounds in early texts exhibit at least three features that are congruent with those of relative clauses:

First, there are numerous examples in which there is no adjacency between the relative and the verbal component of the compound. In these cases, there is virtually no way to tell the difference between a compound and a relative clause, as in example (\ref{ex:ka7a}) below (\cite[][552]{BatlleOcerinjauregui2016}; \cite[][226]{BadiaiMargarit2004}). Second, in the medieval data the verb \textit{voler} inflects according to the tense and modality of the clause (cf. \cite[][VI:889]{Coromines1985}); therefore we may find imperfect \textit{volia} as in (\ref{ex:ka7b}) or even future \textit{volrà} as in (\ref{ex:ka7c}). And third, the compounds have different forms depending on features of the antecedent. While the \textit{qual-} compounds (\textit{qualsevol, qualsevulla, qualsequer, qualsequira}) can be used with any antecedent, the forms with \textit{que-} (\textit{quesvol, quesvulla}) are restricted to inanimate antecedents and the \textit{qui-} form (\textit{quisvulla}) is used with human antecedents (cf. example (\ref{ex:ka7d}) from \citealt{Alcovermoll1962}). Only the universal compounds \textit{qualsevol} and \textit{qualsevulla} have had continuity into modern Catalan.

\ea \label{ex:ka7}
\ea\label{ex:ka7a}  
        \gll pot f-er e elég-er qual demanda=s vol\\
        can.\textsc{prs.3sg} make-\textsc{inf} and choose-\textsc{inf} which request=\textsc{ref} want.\textsc{prs.3sg}\\
        \glt ‘he can make and choose any request he wants’
\ex\label{ex:ka7b}  
        \gll arremir-en junte-s dos cavaller-s sarraïn-s a dos altre-s nostre-s, qual-s se vol-ia de la ost\\
        challenge-\textsc{pst.3pl} together-\textsc{pl} two knight-\textsc{pl} Moorish-\textsc{pl} to two other-\textsc{pl} our-\textsc{pl}, whichs\textsc{pl} \textsc{ref} want-\textsc{pst.3sg} of the army\\
		\glt ‘two Moorish knights challenged two other people among us, whoever 		they were, of our army’
\ex\label{ex:ka7c}  
        \gll altre qual-se-vol-rà que faç-a les- citacion-s\\
        another, which-\textsc{ref}-want-\textsc{fut.3sg} that do-\textsc{sbjv.3sg} the-\textsc{pl} summon-\textsc{pl}\\
        \glt‘another one, whoever it will be who will do the summons’ 
\ex\label{ex:ka7d}
        \gll Senyor caualler, qui-s-vull-a siau, Tirant\\
        Sir knight, who-\textsc{ref}-want-\textsc{sbjv.3sg} be.\textsc{sbjv.2sg}, Tirant\\
        \glt ‘Sir knight, whoever you may be, Tirant’\\
        Tirant lo Blanc, c. 60 (1490)
\z
\z

The preceding examples are evidence that the \textit{qualsevol}-type compounds derive from a sentence structure rather than a phrasal compound. If Catalan \textit{qualsevol} had emerged as a mere calque or replacement of \textit{quivis} and \textit{quilibet}, it would have been a cohesive compound from the beginning, which typically would involve certain restrictions, such as strict adjacency of its elements and lack of verb inflection motivated by elements in the sentence (see \cite[][1113]{CompanyCompanyPozasLoyo2009} for a similar argument concerning Spanish indefinites).

While, as we have said, the available data does not allow to trace the remote structural origins of these elements, we consider that the evidence in favor of a sentence level origin for \textit{qualsevol} type compounds, rather than a calque or adaptation of Latin compounds, is compelling. It is also true, however, that, from the earliest texts, the vast majority of occurrences of \textit{qualsevol} and \textit{qualsevulla} already exhibit the properties of a cohesive phrasal compound, such as a preference to be written together or the frequent loss of inflection for number.

In the pages to follow, we aim to provide an empirically supported account for the development of these phrasal compounds in diachrony. We look at the historical evolution of different properties, such as number agreement, syntactic function and position of \textit{qualsevol}, in order to trace the grammaticalization path of these compounds from relative clauses to indefinites wih an FCI meaning. Moreover, we will look at the different interpretations of indefinites such as the evaluative meaning, in order to see how change in meaning is correlated to the grammaticalization of indefinites such as \textit{qualsevol}.

We will restrict our investigation primarily to \textit{qualsevol}-type forms, as this is the compound that has a continuation into modern Catalan and constitutes the overwhelming majority of FCIs in the history of the language. Therefore, we will not trace the evolution of less frequent compounds such as \textit{quesvol, quesvulla, quisvulla} and the like.


\section{Corpus and Methodology}\label{sec:kea4}\largerpage

As we write this chapter there is only one publicly available historical corpus of Catalan, the Corpus Informatitzat del Català Antic (CICA) online at \url{http://cica.cat/index.php}. This corpus contains texts from ca. 1200--1599 for a total of 6.8 million words, with just a few texts from the 1600s. For later periods in the history of the language we have used two more corpora through personal communication. The first one, the Corpus Informatitzat de la Gramàtica del Català Modern (CIGCMod; \cite{AntolíMartinez2018}) encompasses texts produced between 1600--1832 for a total of 5.5 million words. The second one is the Corpus Textual Informatitzat de la Llengua catalana (CTILC) which covers the latest period, from 1833--2003, and with 82 million words is significantly larger than the two other corpora. As there is some overlap between CICA and CIGCM, we have checked carefully to make sure we eliminated all repeated examples. By combining these three corpora we were able to get data for the whole written history of the language.

We searched the corpora for all occurrences of the relevant forms of the free-choice indefinites: singular forms \textit{qualsevol, qualsevulla,} as well as \textit{qualsevolgués} and \textit{qualsequer}, formed with the preterite of \textit{voler} and the verb \textit{querir}, respectively;  and plural forms \textit{qualssevol, qualssevulla} and \textit{qualsevols}. As we are not concerned with spelling or phonetic variation we subsumed all graphic and dialectal variants (\textit{quansevol, colsevol, cualsebol, cualsevol,} and many others) under their corresponding normalized form. Likewise, in our intention of tracing the evolution of grammaticalized indefinites, we have limited our searches to one word compounds, and as such we have not conducted searches for the separate elements that may occur in the compound. Likewise, we left aside the handful of examples of non-universal indefinites, which only allow for either human antecedents, such as \textit{quisvulla}, or inanimate ones, such as \textit{quesvol}.\largerpage[2]

As expected in any large scale investigation that is primarily based on historical corpora we had to deal with some common methodological issues (cf. \cite{EnriqueArias2012, EnriqueArias2018}). For starters, we face the problem that not all historical periods of the Catalan language are represented equally. For instance, in the CICA corpus there are less than one million words for the 13\textsuperscript{th} century, as opposed to more than two million for the 14\textsuperscript{th} century and over three million for the 15\textsuperscript{th}, the reason being that the widespread use of written Catalan starts late in the 13\textsuperscript{th} century. As a result, in our combined corpus there were only 25 examples of FCIs for the 13\textsuperscript{th} century compared to several hundreds for the subsequent medieval centuries. At the same time, there were almost two thousand examples for the 19\textsuperscript{th} century and more than 17,000 for the 20\textsuperscript{th} century. This disparity in numbers is also related to the diverse size and scope of the different corpora and, more specifically, the disproportionate number of data for the 20\textsuperscript{th} century in the CTILC compared to the other centuries. To make sure that we do not overlook any important data, especially in the earlier centuries where numbers are relatively lower, we have decided to analyze all occurrences in the corpus except for the 20\textsuperscript{th} century where we have limited our analysis to a randomized sample of 2000 occurrences of \textit{qualsevol} plus 109 examples of \textit{qualsevulla}, which is the corresponding proportional share of this form relative to \textit{qualsevol} for this century. \tabref{tab:ka2} features the total number of tokens per century in our database.

\begin{table}[H]
%\small
    \begin{tabularx}{\textwidth}{Xrrrrrrrrr} 
    \lsptoprule
    Century & 13 & 14 & 15 & 16 & 17 & 18 & 19 & 20 & Total \\
    Tokens & 25 & 367 & 696 & 595 & 1011 & 1182 & 1842 & 2109 & 7827\\
    \lspbottomrule
    \end{tabularx}
    \caption{Number of tokens in the database sorted by century}
    \label{tab:ka2}
\end{table}

Another problem in corpus based linguistic research is that of the distribution of registers, genres and dialects. During the period known as \textit{The Decadence}, Catalan became less used in cultural contexts: as a result, the portion of the corpus for the 16\textsuperscript{th} to 18\textsuperscript{th} centuries has very few literary works and is primarily made up of notary documents and personal diaries. We are aware that this could have an impact on the results, as the creation and evolution of indefinite compounds is greatly determined by textual genre (cf. \cite[][1107]{CompanyCompanyPozasLoyo2009}). The same could be said of dialectal variation, as certain forms, such as \textit{qualsevulla}, may be associated with specific dialects. In future research it will be interesting to control for these variables by incorporating a more fine grained analysis of the data and with better control for genre and dialectal variation.

A final issue is that the editors of the CICA have normalized the medieval spellings and in doing so they have unified word separation. For instance, \textit{quals se vol sien} ‘whichever they may be’ in Jaume Roig’s \textit{Espill} (cf. \citealt[][s.v. qualsevol]{Alcovermoll1962}) is rendered \textit{qualssevol sien} on the CICA corpus. This type of normalization eliminates potentially interesting information concerning the grammaticalization of the compound, as the orthographic conventions used by scribes, that is, whether different forms are written separated or bound, and whether or not any constituents can intervene between them, are common criteria to determine the degree of fusion of the elements that take part in a compound. At any rate, as we will primarily be concerned with the evolution of \textit{qualsevol} once the compound is already set, this problem affects just a handful of examples.

Despite these problems we are confident that we have obtained the best historical corpus data available for Catalan. In total we have taken into account 7829 tokens of \textit{qualsevol}-type compounds, which, we consider, is a rather robust data base to extract some generalizations on the evolution of this structure.

Once we had extracted the examples, we coded for those factors that indicate relevant functional and semantic changes in the historical evolution of FCI and that allow us to trace the grammaticalization path of the compound. As we have already pointed out, there is no comprehensive account of the evolution of Catalan \textit{qualsevol};  therefore, in selecting the factors to be analyzed, we need to rely on previous studies for Spanish and other languages, as well as generalizations stemming from grammaticalization theory.

The first factor that we consider is allomorphic variation. As already explained, because \textit{qualsevol}-type structures originate in a relative clause, the medieval Catalan compounds may exhibit allomorphy  according to features of the clause, like tense (past \textit{qualsevolgués}, future \textit{qualsevolrá}) or modality (subjunctive \textit{qualsevulla}), or features of the antecedent, such as animacity (\textit{quisvol} with human antecedents, \textit{quesvol} with inanimate ones, and \textit{qualsevol} with either one). The reduction in the number of allomorphs is an indicator of the transition from a relative clause structure to a cohesive compound.

Second, we analyze number agreement considering two parameters: whether plural contexts trigger the presence of an overt plural marker and, if this is the case, what the morphological exponent of plural is. As we have explained before, it is quite common to find examples with no double \textit{ss} (i.e. \textit{qualsevol} rather than \textit{qualssevol} accompanying a plural noun phrase); this means that from early on, plural inflection for \textit{qualssevol} was only sporadic. In plural contexts with no inflection for plural (see example \ref{ex:ka6b} above) and when there is a plural marker we have several possibilities: on the one hand, there are the traditional plural forms \textit{qualssevol} and \textit{qualssevulla}, in which the relative \textit{qual} is inflected for number, but there is also the innovative plural form \textit{qualsevols} that features the plural \textit{–s} suffix at the end of the whole compound. The continuation of plural \textit{qualssevol} indicates that in some way speakers still analyze the compound as a combination of several distinct elements: \textit{quals se vol}. On the other hand the emergence of a new plural form \textit{qualsevols} is an indicator that speakers understand the compound as one word. There are at least two other features that are related to the degree of integration of the compound: the orthographic convention used by the scribe (whether the elements in the compound are written separated or bound) and interposition (whether or not any constituents can intervene between the relative wh-element and the verb). However, as the configuration of the texts in the corpus and the search engine do not allow this kind of investigation we will not consider these variables.

Next, we consider the position with respect to the noun, that is, wether \textit{qualsevol} precedes or follows the noun it modifies, as in examples (\ref{ex:ka8a}) and (\ref{ex:ka8b}), respectively, which we labeled as PRE (prenominal) or POST (postnominal). The reason why we looked at the position with respect to the noun is because we wanted to see the frequency distribution of \textit{qualsevol} as a modifier and whether the frequency is the same across periods in the prenominal and postnominal case. Our working hypothesis is that prenominal modifiers are different from postnominal modifiers in syntactic category. The former ones are of the determiner type, the latter ones are of the adjective type (see \sectref{sec:kea5}):


\ea \label{ex:ka8}
\ea\label{ex:ka8a} \gll renunci-ava a qualsevol pene-s per ell acusade-s a la dita ciutat\\
        renounce-\textsc{pst.3sg} to any penalty-\textsc{pl} by him demanded-\textsc{pl} to the said city\\
    \glt ‘he renounced to any penalties demanded by him to the said city’\\
        Manual de consells (1378-1379)
\ex\label{ex:ka8b} \gll que s-íe castig-ad cóm â criminal qualsevol\\
        that be-\textsc{sbjv.3sg} punish-\textsc{ptcp} like to criminal any\\
    \glt‘that he gets punished like any criminal’\\
        Febrer i Cardona, Antoni: Daniél ô el vertader cúlto de Dèu restablért en l'Oriént (1836) \\
\z
\z

We also consider the syntactic function of the \textit{qualsevol}-type element, that is, whether it works like a modifier, as in the examples (\ref{ex:ka8}a--b) above, or as a pronoun, as in (\ref{ex:ka9}), or even a noun as in (\ref{ex:ka10}).

\ea \label{ex:ka9}
    \gll Es clar, qualsevol hauria fet igual. (pronoun)\\
    be.\textsc{prs.3sg} clear, any have.\textsc{cond.3sg} do-\textsc{ptcp} same\\
    \glt‘It’s clear, anybody would have done the same’\\
    Ruyra Parada 27. (1919)
    
\ex \label{ex:ka10}
\gll Miris, no=s pens-i que jo sig-a un qualsevol (noun)\\
    look.\textsc{imo}, not-\textsc{ref} think-\textsc{imp} that I be-\textsc{prs.sbjv.1sg} a any\\
    \glt‘look, don’t think I’m a nobody’\\
    Baró, Teodor: No es or tot lo que llú (1872)\\
\z

Finally we looked at co-occurrences with other quantifiers and determiners in the same noun phrase, which we labeled as (+ DET), e.g. [determiner \textit{qualsevol} noun] or [determiner noun \textit{qualsevol}]. Moreover, we distinguished between +/$-$ strong quantifiers/determiners (see \cite{Zamparelli2000} on strong vs. weak determiners). The reason behind this parameter is because we wanted to see whether \textit{qualsevol} itself can be analyzed as a determiner or quantifier like ‘some/every’. If it is a determiner, then we expect it not to occur with other strong determiners or quantifiers like ‘every’, ‘none’, ‘some’, as a noun cannot be quantified or determined twice \textit{*every none girl} (see \cite{EtxeberriaGiannakidou2014}, among others, on double quantifiers). Thus, in a construction like [strong determiner \textit{qualsevol} noun] or [strong determiner noun \textit{qualsevol}], as in (\ref{ex:ka11}) below, \textit{qualsevol} cannot be a determiner itself (see \ref{sec:kea5} on the analysis):

\ea \label{ex:ka11}
\gll que mingun ortolà ni ninguna altra \textit{qualsevol} persona per si o per altri\\
    that no farmer nor no other any person by himself or by another\\
    \glt ‘that no farmer nor any other person by himself or through somebody else’\\
    (CA-MOD 5. Llibre del Mostassaf d'Elx) (1610)
\z


\section{Quantitative analyses}\label{sec:kea5}
\subsection{Morphological variants}\label{sec:kea5.1}

From the onset, \textit{qualsevol} is the most frequent form in the data. In the 13\textsuperscript{th} century CICA registers 60\% of \textit{qualsevol} (15/25) next to 36\% \textit{qualsequer} (9/25) and one example (4\%) of \textit{qualsevulla}. In the ensuing centuries \textit{qualsevol} will increase its frequency even more to become the only allomorph used in regular speech in modern Catalan. \tabref{tab:ka3} summarizes the distribution of the different allomorphs in the corpus:

\begin{table}
%\small
    \begin{tabular}{rrrrrr} 
    \lsptoprule
    Century & \textit{qualsequer} & \textit{qualsevolgués} & \textit{qualsevulla/s} & \textit{qualsevol/s} & Total\\
    \midrule
     13	& 9 & 0 & 1 (4\%) & 15 (60.0\%) & 25\\
     14	& 0 & 0 & 19 (5.1\%) & 348 (94.8\%) & 367\\
     15 & 0	& 4	& 126 (18.1\%) & 566 (81.3\%) & 696\\
     16 & 0	& 0	& 65 (10.9\%) & 530 (89.1\%) & 595\\
     17 & 0	& 0	& 22 (2.1\%) & 989 (97.8\%) & 1011 \\
     18 & 0 & 0	& 9 (0.08\%) & 1173 (99.2\%) & 1182 \\
     19 & 0	& 0	& 119 (6.5\%) & 1723 (93.5\%) & 1842 \\
     20 & 0 & 0 & 110 (5.3\%) & 1999 (94.7\%) & 2109 \\
     \midrule
     Total & 9 & 4 & 471 (6.2\%) & 7343 (93.8\%) & 7827\\
    \lspbottomrule
    \end{tabular}
    \caption{Distribution of the allomorphs of qualsevol registered in the database}
    \label{tab:ka3}
\end{table}    


In the earlier stages of the corpus there are a few cases of \textit{qualsequer} (9 examples or 36\%), but we find no examples beyond the 1200s. This could be related to the fact that the verb \textit{querir} ‘look for, want’ became very infrequent in Medieval Catalan and all but disappeared by the 15\textsuperscript{th} century \citep[][VI: 940]{Coromines1985}. \textit{Qualsevolgués} is also rather infrequent with only four examples in one text, \textit{Tirant lo Blanch}, from 1490. The lack of examples for other forms such as \textit{qualsequira, qualsevull} and \textit{qualsevulga} in our corpus confirms that these forms were also rather infrequent.

As for \textit{qualsevulla}, this is the only competitor of \textit{qualsevol} that has a continuous presence in the history of the language. This form, which \citet[][1551]{BrucartRigau2002} ascribes to the Valencian dialect of Catalan, experiences an increase in the Middle Ages, from just one example in the 13\textsuperscript{th} century to 5.1\% and 18.1\% for the 14\textsuperscript{th} and 15\textsuperscript{th} centuries, respectively. After the Middle Ages, \textit{qualsevulla} decreases until it almost disappears in the 18\textsuperscript{th} century. In recent times, \textit{qualsevulla} has experienced a modest increase, but the form is relegated to formal written registers \citep[][566]{ColominaiCastanyer2002}. As for its meaning, it is to all effects an equivalent of \textit{qualsevol} (\citealt{Alcovermoll1962}; \citealt[][1551]{BrucartRigau2002}).

An issue that has been the source of disagreement in the study of indefinites in Spanish is that of the mood of the verbal base, that is, whether \textit{qualquier} derives from subjunctive or indicative. \citet[][120--121]{CompanyCompanyPozasLoyo2009} consider that subjunctive was the verbal base because a non-factual meaning is better suited to convey the meaning of indifference and generalization in the indefinite. In the case of Catalan it is clear that both indicative (\textit{qualsevol, qualsequer}) and subjunctive (\textit{qualsevulla, qualsevulga, qualsequira}), contributed to the creation of the compound, but the form with indicative \textit{qualsevol} was always predominant and, in the end, the only one that continued in spontaneous speech. This outcome is somewhat expected, as third person singular of the present indicative is the most frequent, less marked and more basic form and thus it is the most likely candidate to become fixated once the compound loses autonomy and becomes one word.

In sum, the great variety of allomorphs of FCI formed with combinations of \textit{qual-\slash qui-\slash que-} + \textsc{verb of volition} in the earlier part of the data constitutes strong evidence of the origins of \textit{qualsevol}-type compounds in a sentence structure in which the verbal element of the compound was inflected in accordance with other components of the sentence. But as grammaticalization set in, the different elements in the compound lost autonomy and the inventory of allomorphs was reduced to the universal indefinite \textit{qualsevol} which is the less marked one: it allows any antecedent, whether animate or inanimate, and the verbal base uses unmarked third person indicative \textit{vol}. As we are about to see, \textit{qualsevol} also lost number agreement which, again, is an indicator of further grammaticalization.


\subsection{Plural agreement}\label{sec:kea5.2}

With respect to plural agreement we are looking at two parametes: first, whether plural contexts trigger overt number agreement morphology in \textit{qualsevol}, and second, when this is the case, whether we find traditional plural \textit{qualssevol}, in which the wh-element of the compound is inflected, or rather we find the new form \textit{qualsevols} which treats the compound as a single word. Regarding the first scenario, it seems that, from early on, there is a good number of plural contexts in which \textit{qualsevol} does not inflect for number, as in examples (\ref{ex:ka12}a--b) below where the indefinite exhibits no overt agreement:

\ea \label{ex:ka12}
\ea \gll A tot-s altre-s qualsevol-{\emptyset} contract-es\\
        to all-\textsc{pl} other-\textsc{pl} any-{\emptyset} contract-\textsc{pl}\\
     \glt ‘to any other contracts’\\
        Reintegració de la Corona de Mallorca a la Corona d'Aragó, Carta 264 (ca. 1300--1349)e
\ex \gll Per qualsevol-{\emptyset} person-es estrany-es\\
        For any-{\emptyset} people-\textsc{pl} strange-\textsc{pl}\\
     \glt   ‘For any strangers’\\
     Manual de Consells de la ciutat de València 1 (ca. 1300--1349)
    \z
\z

The data contains numerous examples attesting that, from the earliest periods recorded in the corpus, plural agreement for \textit{qualsevol} is rather unsteady (see \tabref{tab:ka4}). In the 13\textsuperscript{th} and 14\textsuperscript{th} century data less than half (47.5\%) of indefinites occurring in plural contexts are inflected for plural.\footnote{As there was only one plural example for the 13\textsuperscript{th} century -which exhibited plural agreement- we have collapsed the data from the 13\textsuperscript{th} and 14\textsuperscript{th} centuries.} This percentage gets even lower in the next two centuries (8.8\% in the 15\textsuperscript{th} century and 10.2\% in the 16\textsuperscript{th} century); this downward tendency, however, is reversed in the following centuries, which exhibit a steady increase in the percentage of forms inflected for number agreement: 19.2\%, 60.7\%, 85\% and 71\% for the 17\textsuperscript{th}, 18\textsuperscript{th}, 19\textsuperscript{th} and 20\textsuperscript{th} centuries, respectively (cf. \tabref{tab:ka4}).

\begin{table}
%\small
    \begin{tabularx}{\textwidth}{Qrrrrrrr} 
    \lsptoprule
    Century & 13-14 & 15 & 16 & 17 & 18 & 19 & 20 \\
     \midrule
     {No agreement} & 75 & 185 & 132 & 79 & 11 & 16	& 12 \\
     & 53.2\% & 91.2\%	& 89.8\% & 73.8\% & 39.3\% & 15.0\% & 28.6\%  \\
     \tablevspace
    \textit{qualsevols} & 0 & 3 & 7 & 11 & 17 & 91 & 9  \\
     & 0.0\% & 1.5\% & 4.8\% & 10.3\% & 60.7\% & 85.0\% & 21.4\% \\
      \tablevspace
     \textit{qualssevol}/ & 67 & 15 & 8 & 17 & 0 & 0 & 21 \\
     \textit{-ssevulla} & 47.5\% & 7.3\% & 5.4\% & 15.9\% & 0.0\% & 0.0\% & 50.0\%\\
     \tablevspace
     {Total agreement} & 67 & 18 & 15 & 28 & 17 & 91 & 30 \\ 
     & 47.5\% & 8.8\% & 10.2\% & 26.2\% & 60.7\% & 85.0\% & 71.4\% \\
     \midrule
     Total & 141 & 203 & 147 & 107 & 28	& 107 & 42\\
    \lspbottomrule
    \end{tabularx}
    \caption{Percentage of plural entities with a plural marker}
    \label{tab:ka4}
\end{table}    


If we leave aside the 20\textsuperscript{th} century which, as we discuss below, has its own peculiar evolution, the distribution of number agreement exhibits a V-shape progression: a steady decline during the Middle Ages and a rebound and increase in the Modern Era. These developments are concomitant with, and directly related to, the decline of traditional plural forms \textit{qualssevol} and \textit{qualssevulla} and the emergence and encroachment of the new form \textit{qualsevols}. The older form exhibits a steady decrease to the point of disappearing altogether in the 18\textsuperscript{th} and 19\textsuperscript{th} centuries. At the same time, the new form \textit{qualsevols} appears in the 15\textsuperscript{th} century data (although reference grammars mention examples already in the 1300s) and becomes the only plural form by the 18\textsuperscript{th} century. The loss of \textit{qualssevol} and its replacement with the new plural form evinces that the relative clause origin of \textit{qualsevol}-type forms is not apparent to speakers, who treat the erstwhile compound as a single word. The shift from \textit{qualssevol} to \textit{qualsevols} thus represents a further step in the grammaticalization path of \textit{qualsevol}. \figref{fig:ka2} summarizes the changes in the distribution of plural agreement forms over time.

\begin{figure}
% % % \includegraphics[width=\textwidth]{figures/KelEnri2.jpg}
\pgfplotsset{
    /pgfplots/bar cycle list/.style={/pgfplots/cycle list={
            {lsYellow,fill=lsYellow!50!white,mark=none},
            {lsSoftGreen,fill=lsSoftGreen!50!white,mark=none},
            {lsLightBlue,fill=lsLightBlue!90!white,mark=none}
        }
	}
}
\pgfplotstableread{data/ch5-fig2.csv}\FigureTwoData
    \begin{tikzpicture}
	\small
	\begin{axis}
		[
            axis lines*=left,
			bar width=6ex,
			font=\small,
			height=5cm,
			legend style={at={(0.5,-0.15)}, anchor=north},
			legend columns=-1,
			legend cell align=left,
			width=\textwidth,
			xtick=data,
			xticklabels from table={\FigureTwoData}{Data},
			x tick label style={font=\small},
			y tick label style={font=\small},
            ybar stacked, %=3pt,
			ylabel=\%,
			ylabel near ticks,
			ymin=0,
			ymax=100,
			ymajorgrids=true
		]
		\foreach \i in {{No agreement},{qualsevols},{qualssevol/-ssevulla}}
		  {
		  	\addplot table [x expr=\coordindex, x=Data, y=\i] {\FigureTwoData};
		    \edef\temp{\noexpand\addlegendentry{\i}}
		    \temp
 		  }
     \end{axis}
     \end{tikzpicture}
\caption{Evolution in the distribution of plural agreement forms for \textit{qualssevol} and \textit{qualsevols}}
\label{fig:ka2}
\end{figure}

The developments observed in the 20\textsuperscript{th} century, however, present a stark contrast with the evolution registered in the previous centuries. There is a tremendous decline of the plural form \textit{qualsevols} (as opposed to the upward oriented tendency until the end of the 19\textsuperscript{th} century) and now \textit{qualssevol} and \textit{qualssevulla} reemerge from zero to 50\% of the occurrences. This rather unnatural resurgence in the 20\textsuperscript{th} century of forms that had already disappeared in the previous centuries has to do with the written nature of the texts in the corpus, which are obviously affected by the changes in the written conventions for the language. The late 1800s ushered a renewed appreciation of Catalan as a language of culture; following the Decadence period of the 17\textsuperscript{th}, 18\textsuperscript{th} and early 19\textsuperscript{th} centuries, Catalan first appeared in newspapers and began a gradual entry into universities and scientific academies. At the same time, the Catalan language underwent an unprecedented process of normativization. In particular, Pompeu Fabra’s normative works established \textit{qualssevol} as the standard plural form for the literary language, while \textit{qualsevols} was condemned (cf. \cite[][VI: 889]{Coromines1985}). This circumstance explains why in the CTILC corpus non-standard \textit{qualsevols} diminished in the 20\textsuperscript{th} century, while standard \textit{qualssevol} has revived. We must keep in mind that we are dealing with a corpus of written works (literature, newspapers, magazines, essays and scientific and technical materials) that, starting in the early 1900s, are highly influenced by the new normative guidelines. But this recent increase has no repercussion in actual speech, as singular \textit{qualsevol} and plural \textit{qualssevol} are pronounced the same and only differ in the way they are spelled.

\subsection{Position and function}\label{sec:kea5.3}
The changes explained so far represent a typical grammaticalization path from a Relative Clause to a Nominal Modifier (see \cite{CompanyCompany2009} for a similar process in Spanish). The new structure, however, has undergone further changes which will be discussed in subsequent sections in detail.

From the earliest periods, \textit{qualsevol} has two basic syntactic functions, that of a pronoun (in the older texts always with a partitive Prepositional Phrase, such as \textit{dels regidors} ‘of the councilors’ as in (\ref{ex:ka13})), or that of a noun modifier, as in 
(\ref{ex:ka14}) and (\ref{ex:ka15}) (see \sectref{sec:kea5.4} below for a more detailed account of the syntactic status of prenominal \textit{qualsevol}).

\ea \label{ex:ka13}
\gll qualsevol dels regidor-s o principal-s de la ciutat\\
    any of.the councilor-\textsc{pl} or principal-\textsc{pl} of the city\\
\glt ‘Any of-the councilors or principals of the city’\\
    Corbatxo - page 67, line: 14 (1397)
    
\ex \label{ex:ka14}
\gll Ving-a qualsevol temptació,\\
    come-\textsc{sbjv.3sg} any temptation\\
\glt ‘Any temptation may come’\\
    Llull, Blanquerna 6, 7 (ca. 1283)
    
\ex \label{ex:ka15}
\gll Per qualsevol debilitació del cors,\\
    by any weakening of.the body\\
\glt ‘By any weakening of the body’\\
    Metge Somni I.  (1399)
\z

The distribution of the two functions remains fairly stable throughout most of the historical periods in the corpus. Between the 13\textsuperscript{th} and the 18\textsuperscript{th} centuries, the pronoun constitutes approximately 12\% to 15\% of the total, while the modifier hovers around 85\%--82\%. The percentage of pronouns goes up, however, to 29.7\% in the 19\textsuperscript{th} century, to then lower to 17.6\% in the 20\textsuperscript{th} century.

As summarized in \tabref{tab:ka5}, the modifier function is thus numerically dominant since the beginning and throughout all the periods in the history of the language up to contemporary times.

\begin{table}
%\small
    \begin{tabularx}{\textwidth}{Xrrrrrrrr} 
    \lsptoprule
     Century & 13-14 & 15 & 16 & 17 & 18 & 19 & 20 & Total\\
     \midrule
     Modifier & 85.3\% & 84.8\%	& 87.2\% & 85.2\% & 88.2\% & 70.3\%	& 82.4\% & 81.5\% \\
     Pronoun & 14.7\% & 15.2\% & 12.8\%	& 14.8\% & 11.8\% & 29.7\% & 17.6\% & 18.5\% \\
    \lspbottomrule
    \end{tabularx}
    \caption{Percentage of the distribution of modifier and pronoun function in the corpus}
    \label{tab:ka5}
\end{table}  


When \textit{qualsevol} is used as a pronoun with no nominal antecedent and without a partitive PP as in (\ref{ex:ka9}) above, which we reproduce here as (\ref{ex:ka16}) for convenience, it has the meaning of ‘any person’, ‘anybody’:

\ea \label{ex:ka16}
\gll Es clar, qualsevol hau-ria fet igual,\\
    be.\textsc{prs.3sg} clear any have-\textsc{cond.3sg} do-\textsc{ptcp} same\\
\glt ‘It’s clear, anybody would have done the same’\\
    Ruyra Parada 27. (1919)\\
\z

As illustrated in \tabref{tab:ka6} this pronominal use as in \REF{ex:ka16} already exists in the early texts in the corpus, but only with a few isolated examples; starting in the 16\textsuperscript{th} century, there is a slow but steady increase in the number of cases of the pronoun \textit{qualsevol} with no antecedent and with no partitive PP. The pronominal function grows considerably in the 19\textsuperscript{th} century:

\begin{table}
%\small
    \begin{tabular}{lrrrrrrr} 
    \lsptoprule
    Century & 13-14	& 15 & 16 & 17 & 18 & 19 & 20 \\
    \midrule
    N & 1 & 1 & 5 & 10 & 26 & 159 & 65 \\
    \% & 0.2\% & 0.1\% & 0.8\% & 1.0\%	& 2.2\%	& 8.6\%	& 3.1\% \\
    \lspbottomrule
    \end{tabular}
    \caption{Pronoun \textit{qualsevol} with no antecedent and no partitive PP}
    \label{tab:ka6}
\end{table}  


As for the position of \textit{qualsevol} as a noun modifier, from the beginning, it tended to appear in a predominantly prenominal (PRE) position where it had a Free Choice meaning, as in examples (\ref{ex:ka14}--\ref{ex:ka15}). During the early period that encompasses the 14\textsuperscript{th}--16\textsuperscript{th} centuries, however, it is possible to find a few postnominal occurrences of \textit{qualsevol} after bare nouns (POST) which are conjoined with other nouns as in \REF{ex:ka17}. In this early period, postnominal \textit{qualsevol} has only a Free Choice Interpretation as the prenominal \textit{qualsevol}:

\ea \label{ex:ka17}
\gll les ciutat-s, castell-s, terre-s e loch-s, baron-s, vasall-s e súbdit-s qualsevol\\
    the city-\textsc{pl}, castle-\textsc{pl}, land-\textsc{pl} and place-\textsc{pl}, baron-\textsc{pl}, vasal-\textsc{pl} and subject-\textsc{pl} any\\
\glt ‘any cities, castles, lands, places, barons, vassals, and subjects’\\
    Documents de la Cancelleria d'Alfons el Magnànim (15\textsuperscript{th} century)\\
\z

The first example of POST \textit{qualsevol} preceded by the indefinite determiner \textit{un} is found in the first decades of the 17\textsuperscript{th} century, as shown in (\ref{ex:ka18}):

\ea \label{ex:ka18}
\gll com si f-os un mort qualsevol de cascuna església\\
    as if be-\textsc{pst.sbjv.3sg} a dead any of each church\\
\glt ‘As if it were any dead person from each church’\\
    CA-MOD 120. Dietari de Pere Joan Porcar-I (ca. 1600-1622)\\
\z

As stated before, postponed \textit{qualsevol} is very rare in the early stages of the data. As illustrated in \tabref{tab:ka7}, this situation changed noticeably, starting in the 19\textsuperscript{th} century and continuing into the 20\textsuperscript{th} century data. In the last two centuries, postponed modifiers went from being sporadic to suddenly becoming a sizable proportion of near 10\% of the total occurrences of \textit{qualsevol} in its noun modifier function.

\begin{table}
%\small
    \begin{tabular}{lrrrrrrr} 
    \lsptoprule
    Century & 13-14	& 15 & 16 & 17 & 18 & 19 & 20 \\
     \midrule
    N & 4 & 5 & 2 & 3 & 2 & 114	& 201 \\
    \% &  1.2\%	& 0.8\%	& 0.4\%	& 0.3\%	& 0.2\%	& 8.8\%	& 11.6\% \\
    \lspbottomrule
    \end{tabular}
    \caption{Frequency of postponed modifier \textit{qualsevol} as opposed to preposed}
    \label{tab:ka7}
\end{table}  

Another development that happens in the 19\textsuperscript{th} century is the emergence of \textit{qualsevol} as a noun, preceded by an indefinite article. In this new function, \textit{un\slash una qualsevol} refers to a person of low moral or social status, as illustrated in (\ref{ex:ka19}a--b). The data exhibits no examples of this use prior to the mid 19\textsuperscript{th} century, for which we find 13 examples, followed by 16 examples in the 20\textsuperscript{th} century:

\ea \label{ex:ka19}
\ea\label{ex:ka19a} \gll Prefer-ia que ella pass-és per una qualsevol, \\
        prefer-\textsc{pst.3sg} that she pass-\textsc{pst.sbjv.3sg} for an any\\
    \glt ‘he’d rather make her look like a low-class woman’ \\
        Oller Febre, I, 154 (1890)
        
\ex\label{ex:ka19b} \gll el Clavell és un nuvi de-pega, un titella, un qualsevol \\
        the Clavell be.\textsc{prs.3sg} a boyfriend fake a puppet a any\\
    \glt ‘Clavell is a fake boyfriend, a puppet, a worthless man’\\
        Xavier Benguerel: El casament de la Xela (1937)
    \z
\z

This new function of \textit{qualsevol} as a noun with the evaluative meaning of ‘low class’ co-occurs with two different linguistic properties (see \citetv{chapters/03}). First, it correlates with the verbal mood and aspect, i.e. the verb needs to be a predicative verb like ‘look like’ or ‘to be’ and it needs to be in indicative present or past tense (see the verbs \textit{passar per} ‘pass for’ in (\ref{ex:ka19a}) and \textit{és} ‘to be’ in (\ref{ex:ka19b})). Second, \textit{un/a qualsevol} needs to refer to a person. If these two linguistic properties are not present, \textit{un/a qualsevol} is not interpreted as a noun with an evaluative function, but as an elliptical construction [\textit{un/a} N \textit{qualsevol}]. In this case, the noun rather refers to an entity, not necessarily a person, that was mentioned previously in the discourse. In the following example, \textit{una (altre) qualsevol} refers anaphorically to \textit{una creu} ‘a cross’:

\ea \label{ex:ka20}
\gll no mou á la ánima la contemplació d'una creu gòtica que la de una altre qualsevol de les que ara s=estil-en!\\
    not move.\textsc{prs.3sg} to the soul the contemplation of.a cross Gothic than the of an other any of the that now \textsc{ref}=be.in.style-\textsc{prs.3pl}\\
\glt ‘The contemplation of a Gothic cross doesn’t move the soul like any other of the ones that are now in style!’\\
    Norbert Font i Sagué, \textit{Datos pera la historia de les creus de pedra de Catalunya} (1894)
\z

In the next subsection, we will look into the grammatical status of postnominal \textit{qualsevol} in more detail, especially with respect to its co-occurrence with other determiners and quantifiers. Recall that the reason behind looking at other determiners is to see whether \textit{qualsevol} itself can be analyzed as a determiner or quantifier.


\subsection{\textit{Qualsevol} in co-occurrence with determiners and quantifiers}\label{sec:kea5.4}

\textit{Qualsevol} can co-occur with almost every possible determiner/quantifier\footnote{Usually, quantifiers and determiners are analyzed as two distinct categories. Whereas a determiner is a syntactic category represented as the head of the noun phrase (i.e. DP), a quantifier is primarily a semantic category that can be represented syntactically as a (strong) determiner (see \cite{Zamparelli2000}, and references therein). However, there are syntactic analyses that assume a syntactic position within the DP for quantifiers, so-called QPs.}: universal (\textit{tot}), existential (\textit{algú, un}), negational (\textit{ni}), bare noun, \textit{altre} ‘other’ + bare noun (see \ref{ex:ka21}--\ref{ex:ka28}).

We find data with \textit{qualsevol} N and universal quantifier \textit{tot} with and without coordination: \textit{Tot (i) qualsevol} N: ‘every N, whatever property/identity/kind N might have’. In example (\ref{ex:ka21}), \textit{qualsevol} has a different status than in (\ref{ex:ka22}), as it is not coordinated with the universal quantifier \textit{tot}:
	
\ea \label{ex:ka21}
    \gll al for de València y a tot altre qualsevol dret que ting-a introdu-hït en son favor.\\
    to.the law of Valencia and to all other any right that have-\textsc{sbjvg.3sg} introduce-\textsc{ptcp} in his favor\\
    \glt ‘To the law of Valencia and to all of any other rights that he may have been introduced in his favor’\\
    (CA-MOD 120. Dietari de Pere Joan Porcar-I) (1650--1666)    
    
    \ex \label{ex:ka22} 
    \gll y man-á á totes y qualsevol-s persone-s tant laique-s com eclesiástique-s, secular-s y regular-s\\
    and order-\textsc{pst.3sg} to all and any-\textsc{pl} person-\textsc{pl} so lay-\textsc{pl} like clergy-\textsc{pl}, secular-\textsc{pl} and regular-\textsc{pl}\\
    \glt ‘and he ordered all and any people whether lay, clergy, secular, or regular’\\
    (CA-MOD 120. Dietari de Pere Joan Porcar-I) (1894)\\
\z

In the following examples in (\ref{ex:ka23}--\ref{ex:ka24}), \textit{qualsevol} is used with negative determiners and quantifiers as in \textit{Ni/Ningun N qualsevol}: ‘no/nor N, whatever property/kind/identity N might have’:

\ea \label{ex:ka23}
    \gll que mingun ortolà ni ninguna altra qualsevol persona per si o per altri\\
    that no farmer nor no other any person by himself or by other\\
    \glt ‘that no farmer nor any other person by himself or through somebody else’\\
    (CA-MOD 5. Llibre del Mostassaf d'Elx) (1610)
    
    \ex \label{ex:ka24} 
    \gll ne per apellació ne per altra qualsevol raó, […]\\
    neither by appeal nor for other any reason\\
    \glt ‘neither by appeal nor for any other reason’\\
    Dietari o Llibre de Jornades, (ca. 1450--1499)
\z

In early periods, postnominal \textit{qualsevol} was used very often with bare nouns ‘other N, whatever property/kind/identity N might have’:\footnote{The semantic interpretation of bare nouns is a controversial topic in the literature. It is standardly assumed that bare nouns are interpreted generically (see \cite{Zamparelli2000} and references therein). We leave the study of bare nouns in Old Catalan for future research.}

\ea \label{ex:ka25}
    \gll tirar les dite-s pedre-s, axí corde-s com fusta e altre qualsevol cosa\\
    throw the said-\textsc{pl} stone-\textsc{pl}, like rope-\textsc{pl} as wood and other any thing\\
    \glt ‘throwing said stones, or strings or wood or any other thing\\
    Libre del Mostassaf de Mallorca, (ca. 1400--1449)
\z

In later periods, postnominal \textit{qualsevol} was very rarely used with bare nouns as in (\ref{ex:ka26}):

\ea \label{ex:ka26}
    \gll Y crid-e en vá, com dona qualsevulla o un aprenent de cuyna?\\
    and cry.\textsc{pst.1sg} in vain like woman any or a apprentice of kitchen?\\
    \glt ‘and I yelled in vain, like any woman or a kitchen apprentice?’\\
     Artur Masriera i Colomer, Hamlet príncep de Dinamarca, (1898)
\z

Instead, postnominal \textit{qualsevol} was often used with indefinite determiners such as \textit{Un N qualsevol} ‘some N, whatever property/kind/identity N might have’:

\ea \label{ex:ka27}
    \gll un xeval qualsevol, sig-a qui sig-a, lo que primé=t vingu-i á ma\\
    a lad any be-3\textsc{sg.sbjv} who be-3\textsc{sg.sbjv} the what first={dat.2sg} come-\textsc{sbjv.3sg} to hand\\
    \glt ‘any lad, whoever it may be, the first thing that comes to hand’\\
    Rossend Arús i Arderiu, Cartas á la dona (1877)
    
    \ex \label{ex:ka28}
    \gll ab l=excusa de f-er una pregunta qualsevol á la Sra. Pepa\\
    with the=excuse of make-\textsc{inf} a question any to the Ms. Pepa\\
    \glt ‘with the excuse of asking Ms Pepa any question’\\
    Narcís Oller, La papallona, (1882)
\z

\tabref{tab:ka8} shows the frequency of \textit{qualsevol} (Qlsv) with different types of determiners or quantifiers (represented by the upper number) and the calculated percentages of these frequencies (represented as decimal numbers). This table shows that \textit{qualsevol} co-occurs more often with bare nouns, less often with indefinite nouns, and even less often with universal, existential, and negative quantifiers (> represents the fall in frequency). The hierarchy is schematized in (\ref{ex:ka29}).

\ea \label{ex:ka29}
    bare > indefinite (\textit{un}) > universal quantifiers (\textit{tot}) > negative quantifiers (\textit{ni}) > existential (\textit{algun})\\
\z

\begin{sidewaystable}
%    \small 
    \begin{tabular}{llrrrrrrrrrr} 
    \lsptoprule 
    \multicolumn{2}{l}{Century} & 13 & 14 & 15 & 16 & 17 & 18 & 19 & 20 &  & Total\\
    \midrule
    {Bare N} & Qlsv & 5 & 40 & 54 & 58 & 80 & 102 & 247 & 204 & 790 & 1150\\
    & Altr & 71.43 & 31.25 & 34.62 & 35.58 & 32.79 & 25.00 & 40.49 & 40.56 & 35.62 &  69.81 \\
    \tablevspace
    & Altr & 0 & 64 & 48 & 51 & 82 & 202 & 38 & 6 & 491 & \multirow{4}{*}{}\\
    & Qlsv &  & 50.00 & 30.77 & 31.29 & 33.61 & 49.51 & 6.24 & 1.19 & 22.13 &\\
    \tablevspace
    & Bare & 1 & 0 & 1 & 5 & 10 & 26 & 159 & 65 & 265 &\\
    & Qlsv & 14.29 & & & & & & & & & \\
    \tablevspace
    \multicolumn{2}{l}{Un N Qlsv} & 0 & 0 & 0 & 0 & 1 & 0 & 120 & 220 & & 341\\
    &  &  &  &  &  & 0.41 &  & 19.70 & 43.74 &  & 15.37 \\
    \tablevspace
    \multicolumn{2}{l}{Tot Qlsv} & 0 & 14 & 30 & 25 & 22 & 51 & 32 & 2 & & 176 \\
    &  & 10.85 & 19.23 & 15.34 & 9.02 & 12.50 & 5.25 & 0.40 & & 7.94\\
    \tablevspace
    \multicolumn{2}{l}{Ni Qlsv} & 0 & 8 & 23 & 25 & 49 & 27 & 14 & 4 & & 151 \\
    & & 6.25 & 14.74 & 15.34 & 20.08 & 6.62 & 2.30 & 0.80 & & 6.81 \\
    \tablevspace
    \multicolumn{2}{l}{Algún Qlsv/ Qlsv Algún} & 0 & 2 & 0 & 0 & 0 & 0 & 0 & 2 & & 4\\
    & & 1.56 & & & & & & 0.40 & & 0.18 \\
    \tablevspace
    \multicolumn{2}{l}{Total} & 7 & 128 & 156 & 164 & 244 & 408 & 610 & 503 & & 2220 \\
    & 100.00 & 100.00 & 100.00 & 100.00 & 100.00 & 100.00 & 100.00 & 100.00 & & 100.00 \\
    \lspbottomrule
    \end{tabular}
    \caption{Distribution of quantifiers, bare nouns, and indefinite nouns with \textit{qualsevol}.}
    \label{tab:ka8}
\end{sidewaystable}  




Postnominal modifier \textit{cualquiera} in structures like [\textit{un/a} Noun \textit{cualquiera}] is often analyzed as an indefinite or quantificational determiner akin to ‘some’ (see \cite{ChoiMaribel2008}) or like ‘all/every’ \citep{Alonietal2010} while the status of the indefinite \textit{un/a} in [\textit{un/a} Noun \textit{cualquiera}] is simply ignored.

\ea \label{ex:ka30}
    [? \textit{un} Noun \textit{hombre} Determiner \textit{cualquiera}]\\
\z

These analyses are problematic for the \textit{qualsevol} data, given the co-occurence of postnominal \textit{qualsevol} with indefinite determiners and other quantifiers (see \tabref{tab:ka8}). As we have already shown, [\textit{Un} N \textit{qualsevol}] and [\textit{Un qualsevol}] rise in frequency from the 19\textsuperscript{th} century, which is also when the evaluative interpretation of [\textit{Un qualsevol}] as ‘unremarkable/low value’ appears. The determiner analysis of \textit{qualsevol} cannot explain the appearance of \textit{qualsevol} as a noun with an evaluative meaning.

Given the problematic analysis of \textit{qualsevol} as a determiner in [\textit{Un} N \textit{qualsevol}] or in [\textit{Un qualsevol}], we would like to suggest a different analysis of \textit{qualsevol} in these configurations. Based on diachronic data (i.e. co-occurrence of \textit{qualsevol} with other quantifiers, as shown in \tabref{tab:ka8}), we argue in the following \sectref{sec:kea6} that the postnominal as well as the nominal \textit{qualsevol} have the status of a \textit{predicate} with the Free Choice Interpretation in (\ref{ex:ka31}) or with the evaluative interpretation in (\ref{ex:ka32}) (see also \textcitetv{chapters/03}):

\ea \label{ex:ka31}
    \textit{algún/tot/ningún/un/}bare N \textit{qualsevol}\\
    ‘some/every/none/a/bare N’, ‘whatever identity/property/kind N one wants (literal) or N might have’ FCI
    
    \ex \label{ex:ka32}
    \textit{un} N \textit{qualsevol}\\
    ‘some ordinary/low value N’
\z

In the next section, we spell out the diachronic path followed in the evolution of \textit{qualsevol} and answer the question as to how the determiner \textit{qualsevol}, the nominal modifier and the noun \textit{qualsevol} emerged on this path.


\section{Diachronic analysis of \textit{qualsevol}}\label{sec:kea6}
In this section, we will analyze the grammaticalization of \textit{qualsevol}. We will first give a summary of diachronic evidence for the grammaticalization path of \textit{qualsevol} in \sectref{sec:kea6.1} and then provide an account for the reanalysis of \textit{qualsevol} into different grammatical functions in \sectref{sec:kea6.2}.

\subsection{Summary of diachronic evidence for the grammaticalization of \textit{qualsevol}}\label{sec:kea6.1}

Grammaticalization is commonly understood as the process by which a lexical form becomes a grammatical marker, or a grammatical form or construction assumes an even more grammatical function \citep[cf.][]{Kurylowicz1965, Lehmann1982, HopperTraugott2003}. There are a number of historical evolutions that have been identified as typical effects of grammaticalization processes, such as the loss of syntactic autonomy, the rigidification of positional patterns, the weakening of referential meaning, phonetic erosion, the reduction of contextual syntactic distribution, and often the change of grammatical status, including the tendency for the grammaticalized form to be integrated into new paradigms \citep{Lehmann1985, CompanyCompany2009}.

The historical developments that we have identified so far in the analysis of the evolution of \textit{qualsevol} correspond neatly with the processes that are commonly associated with grammaticalization-type changes. As we have seen, the nominal and verbal components of the indefinite compounds (i.e. the relatives \textit{qual-, que-, qui-,} and the verb \textit{voler-se}) lost autonomy, since both components stopped being free words to become morphemes of a compound that became a simple word. As such, the relative stopped being inflected for number and thus the plural form \textit{qualssevol} was replaced by a new form \textit{qualsevols}, in which the plural inflection \textit{-s} was affixed to the end of the verbal component, effectively treating the erstwhile compound as a single word.  Semantically, the relative clause \textit{qual se vol} lost its compositional meaning of an open proposition with a variable x, as being represented by the wh-pronoun \textit{qual} and the volitional verbal phrase \textit{se vol} [Rel. Cl. \textit{qual se vol}]= ‘one wants x’ (see \cite{Caponigro2004} and \cite{Kellert2015} on semantic interpretation of free relative clauses in synchrony). The new construct (i.e. the one-word-compound \textit{qualsevol}) acquired a new meaning, namely a set of alternatives that is interpreted with respect to a \textit{different} modal verb than the one provided diachronically earlier by the volitional verb. The alternatives of the new construct are interpreted with respect to the modal verb of the \textit{matrix clause} (e.g. \textit{pots portar qualsevol llibre} ‘you can bring any book’) and not with respect to the volitional modal verb of the relative clause, as it was the case at a prior stage, when the volitional verb was still part of the meaning (as in the relative clause structure \textit{qual se vol}). Clearly, the semantic change whereby indefinites such as \textit{qualsevol} change their interpretation with respect to the new modal verb provided by the matrix verb needs to be worked out in detail in the future (see \cite{Kellert2021c}).

To summarize, one important trigger for the grammaticalization of \textit{qualsevol} is the fact that the verbal component lost autonomy, since it changed from being inflected for mood and tense in old Catalan (\textit{qualsevolrà, qualsevolgués}), to become fixed in the invariable form \textit{qualsevol}. Likewise, there was a loss of autonomy in the pronominal component of \textit{qualsevol} since out of the various existing medieval forms \textit{que-, qui-, qual-,} only the latter survived. Moreover, the two original forms of the construction changed their categorical status, since both formatives were reinterpreted or reanalyzed as a simple indefinite pronoun: the components stopped being a relative pronoun and a verb, respectively, to become a new form of the Catalan pronominal system, namely the indefinite compound. Finally, there was also a process of paradigmatization as the new form was integrated into the Catalan paradigm for indefinites along with forms such as \textit{algú} ‘somebody’, \textit{cadascú} ‘each one’, \textit{tothom} ‘everyone’, and so on.

In sum, the evolution of \textit{qualsevol} represents a full grammaticalization path, which does not reflect the mere translation of Latin indefinite compounds as proposed in the Continuation Hypothesis. If that had been the case, we would have found a cohesive compound from the beginning. 


\subsection{Changes of \textit{qualsevol} into different grammatical functions}\label{sec:kea6.2}\largerpage
We assume that at the very first stage of \textit{qualsevol}, it was analyzed as a Relative Clause (RC) (see Grammaticalization Hypothesis in \sectref{sec:kea3}) represented as a complementiser clause (CP) (see \cite{Rivero1988}, among others). At this stage, \textit{qual} is a \textit{wh}-element with a \textit{wh}-feature [+wh], which simply marks an element as wh-relative or wh-interrogative pronoun (see \cite{Kellert2015} on \textit{wh}-features, \cite{Kellert2021c}.). This pronoun refers to the object argument of the finite verb \textit{vol} inside the finite verbal phrase (represented as TP for Temporal and Finite Phrase). We represent this reference to the object argument by an index j:

\ea{\label{ex:ka33}
    [\textsc{cp} \textit{qual}\textsubscript{j} [+wh] [\textsc{tp} \textit{se vol} j]]}\jambox*{First stage=transparent RC}
\glt    ‘what(ever) one wants.’
\z

The Relative Clause analysis explains the existence of examples with interposition such as \textit{qual N se vol} in the earlier documents, as in example (\ref{ex:ka7a}) repeated here as (\ref{ex:ka34}):

\ea\label{ex:ka34}
    \gll pot fer e eléger qual demanda’s vol\\
    can.\textsc{prs.3sg} make-\textsc{inf} and choose-\textsc{inf} which request=\textsc{ref} want-\textsc{prs.3sg}\\
    \glt ‘he can make and choose any request he wants’
\z

We analyze [\textit{qual} NP] as a specifier of a free Relative clause (RC) \citep{Kellert2021c}:

\ea{\label{ex:ka35}
    [\textsc{cp} [\textit{qual} NP]\textsubscript{j} [\textsc{+wh}] [\textsc{tp} \textit{se vol}\textsubscript{j}]]} \jambox*{First stage= RC}
\glt     ‘whatever NP one wants’ 
\z

At this stage, \textit{qual se vol} can also modify an overt NP \textit{outside} the Relative Clause or CP, as shown in (\ref{ex:ka36}). As \textit{qual} shows plural agreement in the structure in (\ref{ex:ka36}), we must assume that, at this point, the relative clause is still transparent, even though the grammaticalization process has already started, as shown by the orthographic representation of \textit{qualsevol} as a single word:

\ea \label{ex:ka36}
    [Det 	[NP [ModifP [\textsc{cp} quals\textsubscript{j} [\textsc{+wh}] [\textsc{tp} \textit{se vol}\textsubscript{j} ]]]]]\\
    e.g. tots deutes qualssevol\\
    ‘all debts whatever kind one wants’
\z
    
We assume that the relative clause in (\ref{ex:ka36}) denotes a \textit{property}, which describes individuals denoted by the noun phrase \textit{deutes} ‘debts’, as in (\ref{ex:ka37}):
 
\ea \label{ex:ka37}
    \textit{tots deutes qualssevol}: all x [debts’ (x) \& qualssevol’ (x)]
\z

In this analysis, \textit{qualssevol} has a similar syntactic and semantic status as an adjective with the meaning ‘common/ordinary’ (see \textcitetv{chapters/03}, \cite{Kellert2021c}):

\ea \label{ex:ka38}
    \textit{tots deutes comuns}: all x [debts’ (x) \& common’ (x)]\\
\z

The crucial point of our analysis of \textit{qualsevol} as a property in (\ref{ex:ka38}) is that this property is assigned to \textit{qualsevol} only at the level when it was reanalyzed as one syntactic category (i.e. a modifier), and not when it still was a relative clause. In other words, the meaning ‘common, unremarkable’ is part of the diachronic change that arises after the relative clause is no longer perceived as clausal (see \figref{fig:ka3}).

\begin{figure}
\caption{Analysis of \textit{Tots deutes qualssevol}.}
\label{fig:ka3}
\begin{forest}
[CP
    [{Det \\ tots},align=center]
    [NP {〈e,t〉}
        [{N \\ deutes \\  {〈e,t〉}},align=center]
        [{ModifP  {〈e,t〉},  {〈e,t〉}} [\textit{qualsevol}, roof]]
    ]
]
\end{forest}
\end{figure}

The next step is the loss of RC transparency and the lexicalization of the relative clause into a single compound word. At this stage, \textit{qual} is no longer transparent for plural agreement. The plural agreement is realized instead on the ending of \textit{qualsevol} as in \textit{qualsevols}. The indefinite is directly interpreted as a nominal modifier without the RC basis:

\ea \label{ex:ka39}
    Modifier \textit{qualsevols} \hfill{Second stage=loss of RC transparency}\\
    \ea \gll cònsols de qualsevol-s viles\\
    consuls of any.\textsc{pl} towns \\\jambox*{(16\textsuperscript{th} century)}
    \ex \gll y universal-s qualsevol-s\\
    and universal.\textsc{pl} any.\textsc{pl}\\\jambox*{(17\textsuperscript{th} century)}
    \z
\z

We assume that bare nouns in Catalan have been replaced by indefinite nouns as represented in (\ref{ex:ka40}) (see \citealt{Lapesa1975} for this assumption in Old Spanish), i.e. the indefinite determiner \textit{un} in (\ref{ex:ka40b}) replaced the (covert/empty) determiner of bare nouns in (\ref{ex:ka40a}):

\ea \label{ex:ka40}
   \ea\label{ex:ka40a} {[}\textsc{det} ∅ {[}\textsc{np} \textsc{n} {[}\textsc{modifp} \textit{qualsevol}{]}{]}{]}\\
    e.g. (\ref{ex:ka42}) [...] e súbdits qualsevol\\
    ‘and subjects whoever they are
    \ex\label{ex:ka40b} {[}\textsc{det} un {[}\textsc{np} \textsc{n} {[}\textsc{modifp} \textit{qualsevol}{]}{]}{]}\\
    e.g. (\ref{ex:ka18}) [...] un mort qualsevol\\
    ‘any dead person’
    \z
\z

\begin{figure}
\caption{Analysis of \textit{Com si fos un mort qualsevol.}}
\label{fig:ka4}

\begin{forest}
[CP
    [com si]
    [TP
        [fos]
        [DP
            [un]
            [NP
                [mort]
                [ModifP [\textit{qualsevol}, roof]]
            ]
        ]
    ]
]
\end{forest}
\end{figure}

In the mid 1800s, \textit{qualsevol} started to appear as a noun as in \textit{una/un qualsevol} ‘a female or male person with low status’. We leave it open as to whether nominalized elements can be interpreted as modifications of covert generic nouns with a gender specification like ‘male person’ and ‘female person’ (see (\ref{ex:ka41a})) or as real nominalizations where \textit{qualsevol} is interpreted as a noun (see (\ref{ex:ka41b})):

\ea \label{ex:ka41}
\ea\label{ex:ka41a}\relax [\textsc{det} un [N ‘person’ [+ male]] [\textsc{modifp} \textit{qualsevol} ]]]
\ex\label{ex:ka41b}\relax [\textsc{det} un [N \textit{qualsevol}]] \\
‘a male person with low status’
\z
\z

One way to test the two analyses in (\ref{ex:ka41}) is using coordination. Under the analysis in (\ref{ex:ka41b}), but not in (\ref{ex:ka41a}), one should be able to coordinate \textit{qualsevol} with nouns. This will be tested in future research.

Before turning to prenominal \textit{qualsevol}, we will show which data the nominal modifier analysis of \textit{qualsevol} covers so far, and why it is better than previous analyses in the literature (see our review of the literature regarding the analysis of structure  (\ref{ex:ka30})). As the postnominal \textit{qualsevol} is not a determiner, it can co-occur with other determiners. It also explains the postnominal position of \textit{qualsevol}, because \textit{qualsevol} has its origin in a relative clause, and relative clauses normally follow nouns. It also explains the adjective-like use of postnominal \textit{qualsevol}. It is a common assumption in the literature that postnominal adjectives in Italian or Romance in general have the syntactic structure of a relative clause (see \citealt{Cinque2010}). We have shown that postnominal \textit{qualsevol} originates as a relative clause and evolves into a nominal modifier. In that sense, there is a strong parallel between postnominal \textit{qualsevol} and postnominal adjectives. The occurrence of an evaluative meaning is easier to explain under the assumption that \textit{qualsevol} is a nominal modifier rather than a determiner due to its adjective-like and thus lexical status rather than its grammatical status (see \textcitetv{chapters/03}). A detailed analysis of the different readings of the modifier \textit{qualsevol} and how these readings evolved awaits future research.

The question now is whether the same modifier analysis as suggested in (\ref{ex:ka40}) can be applied to the prenominal \textit{qualsevol}. We suggest that the prenominal \textit{qualsevol} should be analyzed as a nominal modifier in sentences like in (\ref{ex:ka42}), where \textit{qualsevol} is predeced by a determiner and followed by noun:

\ea\label{ex:ka42}
    \gll al for de València y a tot altre qualsevol dret que ting-a introdu-hït en son favor.\\
    to.the law of Valencia and to all other any right that have-\textsc{sbjv.3sg} introduce-\textsc{ptcp} in his favor\\
    \glt ‘To the law of Valencia and to all of any other rights that he may have been introduced in his favor’\\
    (CA-MOD 120. Dietari de Pere Joan Porcar-I) (1650--1666)
\z

The DP \textit{tot altre qualsevol dret} in (\ref{ex:ka42}) is analyzed in (\ref{ex:ka42b}):

\ea \label{ex:ka42b}
    [\textsc{det} \textit{tot} [ \textit{altre} [\textsc{modifp} \textit{qualsevol} [\textsc{np} \textit{dret} ]]]]
\z

\begin{sloppypar}
However, in examples without any overt determiner like \textit{tot} in (\ref{ex:ka42}), the prenominal \textit{qualsevol} can be analyzed as a determiner-like attributive element, as demonstrated in (\ref{ex:ka43}).
\end{sloppypar}

\ea \label{ex:ka43}
    [\textsc{det} \textit{qualsevol} [\textsc{np} penes]]\\
    {[}...{]} a qualsevol penes {[}...{]} (cf.  \REF{ex:ka8a}).\\
    ‘to any penalties’
\z

We suggest a similar analysis of the determiner \textit{qualsevol} as shown in (\ref{ex:ka9}), which we reproduce here as (\ref{ex:ka44}), for pronoun uses of \textit{qualsevol}:

\ea %\label{ex:ka40}-->
\label{ex:ka44}
    \gll Es clar, qualsevol hau-ria f-et igual.\\
    be.\textsc{prs.3sg} clear any have-\textsc{cond.3sg} do-{ptcp} same\\\jambox*{(pronoun)}
    \glt ‘It’s clear, anybody would have done the same’\\
    Ruyra Parada 27. (1919)
\z

As for the pronoun use, the noun is analyzed as a generic noun with animate feature with the semantic interpretation of a ‘person’: 

\ea \label{ex:ka45}
    [\textsc{det} \textit{qualsevol} [\textsc{np} ‘person’]] hauria fet igual.\\
    ‘anybody would have done the same.’
\z

In \figref{fig:ka46} we summarize what we have shown in this section. The element \textit{qualsevol} originated as a relative clause, then it lexicalized into one word; then, depending on the prenominal or postnominal position, this new one word category was either reanalyzed as a lexical category (i.e. as a postnominal modifier or as a noun), or as a grammatical category (i.e. as a determiner or pronoun). Only the latter development can be defined as a process of grammaticalization in the sense of \citet{Lehmann1985}.

\begin{figure}
\caption{Evolution of \textit{qualsevol}}
\label{fig:ka46}
\begin{tikzpicture}[node distance=3cm]
\node (1) {Relative clause};
\node (2) [right of=1, align=center] {One word\\compound};
\node (3) [right of=2, xshift=5em] {};
\node (4) [above of=3, text width=10em, yshift=-2cm]{Postnominal modifier\slash noun (lexical meaning)};
\node (5) [below of=3, text width=10em, yshift=2cm]{Determiner\slash pronoun (grammatical function)};
\draw [arrow] (1) -- (2);
\draw [arrow] (2.east) to[in=180,out=0] (4.west);
\draw [arrow] (2.east) to[in=180,out=0] (5.west);
\end{tikzpicture}
\end{figure}


\section{Summary and outlook}\label{sec:kea7}

In this chapter, we have examined the morphosyntactic and semantic properties of Catalan \textit{qualsevol} in diachrony and we have proposed a grammaticalization path for this structure. We assumed that it started out as a relative clause and that due to its modificational function, it was reanalyzed as a nominal modifier similar to an adjective. The prenominal modifier was further reanalyzed as a determiner, whereas the postnominal one preserved its function as a modifier.

This chapter does not provide any detailed semantic analysis of Free Choice and the evaluative ‘unremarkable’ interpretation and how these two readings are interrelated (see \cite{Kellert2021c}). In future research, the syntactic functions of \textit{qualsevol} should be examined using contemporary oral data in order to see whether it has grammaticalized any further in Modern Catalan. Finally, it will be important to check in future investigations whether the changes in the grammaticalization path demonstrated for Catalan \textit{qualsevol} coincide with the development of other FCIs in Romance, such as Spanish \textit{cualquiera}, Italian \textit{qualunque}\slash\textit{qualsiasi}\slash\textit{qualsivoglia}, and French \textit{quelconque}.


\section*{Acknowledgements}\label{sec:kea8}
\begin{sloppypar}
Olga Kellert is thankful for the funding support from Deutsche Forschungs Gemeinschaft (DFG 256240798) for the project on quantification in Old Italian and other Romance languages and from the Habilitation grant from the Philosophical Faculty of the Georg-August-University of Göttingen. Andrés Enrique-Arias thankfully acknowledges funding from MICIU\slash AEI (\href{https://data.crossref.org/fundingdata/funder/10.13039/501100011033}{10.13039/501100011033}) FEDER, UE for the grants PID2020\hyp 116863GB\hyp I00 and PID2023-150917NB-I00. Thanks are also due to Manuel Pérez Saldanya and Jordi Antolí Martínez for their assistance in obtaining data from the Corpus Informatitzat de la Gramàtica del Català Modern (CIGCMod) and the Corpus Textual Informatitzat de la Llengua catalana (CTILC), respectively. Likewise, we are very grateful to the audience of the workshop “Indefinites in Romance. The limits of an unstable category” at the 36th Romanistentag in Kassel for their helpful feedback. We thank the editors and anonymous reviewers for their thoughtful comments.
\end{sloppypar}
{\sloppy\printbibliography[heading=subbibliography,notkeyword=this]}
\end{document}
