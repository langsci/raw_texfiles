\documentclass[output=paper,colorlinks,citecolor=brown]{langscibook}
\ChapterDOI{10.5281/zenodo.13759984}
\author{Marika Francia\orcid{}\affiliation{Georg-August-Universität Göttingen} and Olga Kellert\orcid{}\affiliation{Georg-August-Universität Göttingen}}

\title{Argentinian Spanish \textit{cualunque} and Italian \textit{qualunque}}

\abstract{\begin{sloppypar}In this article, we describe the syntactic and semantic properties of Argentinian Spanish \textit{cualunque} ‘common, unremarkable’, which originated from Italian  \textit{qualunque} ‘any’. As we will show, on the basis of different corpus data (mainly social media data), \textit{cualunque} has a syntactic distribution different from that of Italian \textit{qualunque}.\end{sloppypar}

Our main hypothesis is that Arg.Sp. \textit{cualunque} denotes a property of type 〈e,t〉, such as \textit{es cualunque}, and has the meaning of ‘unremarkable’. It describes a property of atomic individuals (e.g., people or objects) or their kinds \citep[see][]{Carlson1977} that do not have any particular or specific features that would distinguish them from other atomic individuals or their kinds. Thus, \textit{cualunque} entails a contrast to specific or particular (kinds of) individuals with special and distinguished properties. The crucial point of the analysis suggested in this paper is that the atomic individuals or their kinds described by \textit{cualunque} are evaluated on different scales, such as a frequency scale or some qualitative scale (e.g., scale of goodness). The neutral interpretation of \textit{cualunque} as ‘common’, ‘widespread’, or ‘normal’ is yielded when the individuals described by \textit{cualunque} are evaluated in the middle of these scales. The pejorative version of the meaning ‘common’ is analyzed as a predication over individuals or their kinds that are evaluated on the extreme end of a qualitative scale, such as the scale of goodness. On this interpretation, \textit{cualunque} means ‘not outstanding’ or ‘bad’. It is the evaluation on scales that has triggered the reanalysis of \textit{cualunque} as a gradable adjective or as a noun with the evaluative interpretation of ‘ordinary or bad person’ in Argentinian Spanish. Italian \textit{qualunque} also expresses the evaluation on scales, but only as a pragmatic implicature, which is not (yet) lexicalized.}


\begin{document}
\maketitle

\section{Introduction}\label{sec:fk1}
Italian had an invasive influence on the variety of Spanish spoken in the Río de la Plata area. Its effects are visible in all aspects of the language, especially in the lexicon, which has been enriched by many Italianisms (see \citealt{MeoZilio2001, DiTullio2003, EngelsKailuweit2011}, among others). Argentinian Spanish (Arg.Sp.) \textit{cualunque} is one of them.\largerpage[-1]

The influence of the Italian language and its dialects on Argentinian Spanish was the result of prolonged and intense linguistic contact between the local population and the Italian community that had emigrated to Argentina in the 19\textsuperscript{th} through mid-20\textsuperscript{th} century (see \citealt{MeoZilio1959, DiTullio2003}, \citealt[pt. 2.1]{Conde2011}).\footnote{Considering only the years of mass migration (1881--1914), 2,000,000 Italians arrived in Argentina. In 1914 Italians constitute ca. 12\% of the entire Argentinian population (see \cite[][32ff.]{Devoto2002}).} Most of these immigrants were illiterate, spoke only their local dialect, and had only a passive knowledge of Standard Italian (see \citealt{DeMauro2017} among others). In Argentina, they settled in urban environments, such as the city of Buenos Aires (see \cite[][pt. 2]{Baily1999}, \cite[][41]{Devoto2002}). Figure \ref{fig:fk1} offers an overview of this historical context. It shows two maps that represent the Italian regions most affected by the migratory flow and the areas of major settlement by Italian immigrants in Argentina. Among these, there is the city of Buenos Aires, where the argot known as Lunfardo\footnote{Lunfardo is an argot, about 40\% of whose vocabulary is derived from Italian. Its origin goes back to Buenos Aires in the second half of 19\textsuperscript{th} century. Its use later spread first throughout the entire region and subsequently across the whole country, penetrating the spoken language of the Spanish-speaking population (see \cite{Teruggi1974}, \cite{Conde2011}).} was born, and a region adjacent to that city, where its use spread in the first half of 20\textsuperscript{th} century (see \cite[32]{Teruggi1974}, \cite[pt. 1.2]{Conde2011}, among others).\footnote{The map on the right in \figref{fig:fk1} shows the Italian regions with the heaviest emigration from 1876 to 1915 as reported by the Italian Commissariat of Emigration. The map on the left in \figref{fig:fk1} shows the regions of settlement in Argentina according to the Third Census of the Argentinian Republic in 1914.}

\begin{figure}
% \includegraphics[width=\textwidth]{figures/FranKel1.jpg}
\includegraphics[width=\textwidth]{figures/ItalyArgentina.pdf}
\caption{Regions of emigration from Italy and regions of Italian settlement in Argentina. Left: Adaptation of Italians in Argentina (1914), from \url{https://commons.wikimedia.org/wiki/File:Italians_in_Argentina_(1914).png}. Right: Adaptation of Map 1 from \citet[][615]{Baily1999}.}
\label{fig:fk1}
\end{figure}

\begin{sloppypar}
Arg.Sp. \textit{cualunque} is one of the many Italianisms that enriched Rioplatanese Spanish. Its origin can be traced back to Italian (It.) \textit{qualunque} ‘any’ (see \cite{RealAcademiaEspañola2010}, \cite[][223]{HaenschWerner2000}) or to a similar-sounding dialectal form, such as Piedmontese \textit{qualonque} or Sicilian \textit{qualunchi} (\cite[][213]{Brero2001}, \cite[][1041]{PiccittoTropea1990}). The exact historical development of \textit{q\slash cualunque} is difficult to reconstruct due to a lack of sources of spoken Italian and Spanish from that time. The first available diachronic source of \textit{q/cualunque} starts with Cocoliche\footnote{Cocoliche refers to the variety of Spanish that was spoken by Italian immigrants of first generation in Argentina in their everyday communication in the late 19\textsuperscript{th} and early 20\textsuperscript{th} centuries. It was the result of language contact with colloquial Spanish spoken in the Río de la Plata area and Italian dialects at that time. It disappeared with the second generation. Cocoliche is characterized in the literature as an individual, spontaneous, and unconscious phenomenon with high instability and irregularity, as it could differ between speakers according to many factors, e.g., language of origin, level of education, permanence of residence in the country, and willingness to learn (see \cite[][73f.]{Cancellier2001}, \cite[][51f.]{DiTullio2003}, \cite[][44f.]{Engels2010}, \cite[][54]{Kailuweit2004}, \cite {MeoZilio1964}).},\largerpage[1]\ a \textit{learner variety} of Spanish spoken by Italian immigrants and its imitation in literary contexts, so-called literary Cocoliche.{\interfootnotelinepenalty=10000\footnote{In accordance with \citet{Kailuweit2004}, we refer to the imitation of Cocoliche in literary contexts, such as plays, novels, and short stories, as literary Cocoliche. \citet[][59]{Kailuweit2004} and \citet[][129f.]{Ennis2015} define this literary variety as a form of conceptual orality in the sense of \citet{KochOesterreicher2011}. It is based on the perception that each author has with respect to the Spanish spoken by Italian immigrants. As a result, it can vary from one author to another with respect to linguistic elements they consider characteristic. At the same time, these linguistic elements form a literary tradition from which subsequent authors can benefit \citep[66]{Kailuweit2004}.}}  The word was later attested in Lunfardo \citep[see][]{Conde2011} and then in an informal style of speech in Argentinian Spanish, in the sense of \citet{KochOesterreicher2011}.\footnote{We have found the first diachronic occurrences in texts of \textit{cocolichesco} characters in plays and novels dating back to the beginning of the 20\textsuperscript{th} century (see  Section \ref{sec:fk7}). Following the ethnolectal chain described by \citet[238]{EngelsKailuweit2011}, we assume that the term was first used by Italians speaking Cocoliche in their everyday attempts to communicate with the hispanophone population (\textit{first variety}). At the same time, it could be heard and read in popular media, such as in \textit{sainetes}, tangos, and popular magazines (\textit{secondary ethnolect}). The popularity of \textit{cocolichesco} characters led to its ironic imitation by native Spanish-speakers outside the context of the media (\textit{tertiary ethnolect}). In this way, \textit{cualunque} entered as an Italianism into the Lunfardo lexicon and later into the Argentinian Spanish informal style of speech.}
\end{sloppypar}


Arg. Sp. \textit{cualunque} has changed phonologically, semantically, and syntactically from the original source word It. \textit{qualunque}.

In Italian, \textit{qualunque} is an indefinite, which can be used as a prenominal determiner with a Free Choice (FC) interpretation in modal contexts. Thus, when embedded under an overt modal as in (\ref{ex:fk1}), it is interpreted as ‘every possibility is an option’ (see \cite{Chierchia2006, AloniPort2013, Alonso-OvalleMenéndez-Benito2017, Stark2006, Kellert2021a}, among others).

\ea\label{ex:fk1} Italian \citep[][2]{Kellert2021a}\\
    \gll    \textbf{Puoi} scegliere \textbf{qualunque} \textbf{libro}\\
            can.\textsc{prs.2sg} choose \textsc{qualunque} book\footnotemark\\
    \glt    ‘You can choose any book’\\
    	    Conventional meaning: ‘You can choose a book’ and
            FC meaning: ‘each book is a possible option’.
    \footnotetext{Since It. \textit{qualunque} and Arg.Sp. \textit{cualunque} have a number of different interpretations, we will not translate them in the gloss.}
\z

Italian \textit{qualunque} is also used as a postnominal indefinite with the meaning ‘ordinary/unremarkable’, often with copular verbs in the indicative present or past tense as in (\ref{ex:fk2}) and with indefinite nouns that assert the existence of some entity (see \cite{Kellert2021d} for Italian and \cite{Alonso-OvalleMenéndez-Benito2017} for Spanish \textit{cualquiera}). When postnominal \textit{qualunque} is used in a predicate position as in (\ref{ex:fk2}), it adds an evaluative interpretation, namely that ‘the book is unremarkable’.\largerpage

\ea\label{ex:fk2} Italian \citep[2]{Kellert2021a}\\
    \gll    The Stars è un libro qualunque\\
            The Stars be.\textsc{prs.3sg} a book \textsc{qualunque}\\
    \glt    The Stars is a book (existential inference)\\
            The Stars is unremarkable (evaluative meaning)
\z



Even though postnominal \textit{qualunque} is possible in non-modal contexts such as with indicative present or past tense copular verbs as in (\ref{ex:fk2})~-- something unpredicted by standard analyses of Free Choice Indefinites such as \citet{Chierchia2006}~-- \textit{qualunque} seems to still have the status of an indefinite rather than that of an adjective, as most standard syntactic tests targeting the category of adjectives do not apply to \textit{qualunque} in Standard Italian (see \cite{Kellert2021a}). For instance, It. \textit{qualunque} cannot be used as a predicate under copular verbs as in (\ref{ex:fk3}a), in comparative structures (\ref{ex:fk4}a,b), with degree modification (\ref{ex:fk5}), in coordination with other adjectives (\ref{ex:fk6}), and in uses with definite nouns (\ref{ex:fk7}) or quantifiers (\ref{ex:fk8}).\footnote{A linguistic experiment conducted in 2020 on 160 native Italian speakers from different regions of Italy suggests that It. \textit{qualunque} may be on its way to being reanalyzed as an adjective, as some informants accept uses of the word with coordination, degree modification, and comparatives (see \cite{Kellert2021b} for an analysis of an online survey conducted by \cite{Francia2020}). Further research needs to be conducted on It. \textit{qualunque}.}

\ea\label{ex:fk3} Predicative \textit{qualunque} (Italian)\\
    \ea[*]{
     \gll Gianni è qualunque\\
        Gianni be.\textsc{prs.3sg} \textsc{qualunque}\\ 
    \glt ‘Gianni is qualunque’
    }
    \ex[]{\gll Gianni è un uomo qualunque\\
        Gianni be.\textsc{prs.3sg} a man \textsc{qualunque}\\
    \glt ‘Gianni is an ordinary man’}
    \z
\ex \label{ex:fk4} Comparatives (Italian)\\
    \ea[*]{
     \gll Gianni è più/meno/tanto qualunque di/quanto te\\
        Gianni be.\textsc{prs.3sg} more/less/as \textsc{qualunque} than/as you\\
        \glt ‘Gianni is more/less/as qualunque than/as you’\\
    }
    \ex[*]{ 
    \gll Gianni è tanto qualunque quanto te\\
        Gianni be.\textsc{prs.3sg} as \textsc{qualunque} as you\\
    \glt ‘Gianni is as qualunque as you’
    }
    \z
\ex\label{ex:fk5} Degree modification (Italian)\\
    \gll Gianni è un uomo (*molto/*abbastanza/*poco) qualunque\\
    Gianni be.\textsc{prs.3sg} a man very/pretty/little \textsc{qualunque}\\
    \glt ‘Gianni is a (*very/*pretty/*not) ordinary man’
\ex \label{ex:fk6} Coordination (Italian)\\
    \gll Gianni è un uomo (*molto/*abbastanza/*poco) qualunque\\
    Gianni be.\textsc{prs.3sg} a man very/pretty/little \textsc{qualunque}\\
    \glt ‘Gianni is a (*very/*pretty/*not) ordinary man’
\ex\label{ex:fk7} Definites/Demonstratives (Italian)\\
    \gll *questa/*la ragazza qualunque\\
    this/the girl \textsc{qualunque}\\
    \glt ‘this/the girl qualunque’
\ex\label{ex:fk8} Quantifiers (Italian)
    \ea \gll {*tutte le}/*tante/*due ragazze qualunque\\
        all/many/two girls \textsc{qualunque}\\
       \glt ‘all/many/two girls qualunque’
     \ex \gll *nessuna/*qualche/*ogni ragazza qualunque\\
         no/some/each girl \textsc{qualunque}\\
        \glt ‘no/some/each girl qualunque’
    \z
\z


Unlike It. \textit{qualunque}, Argentinian Spanish \textit{cualunque} is used in all aforementioned syntactic contexts in (\ref{ex:fk3}--\ref{ex:fk8}). This is shown by degree modification in (\ref{ex:fk9}), a comparative clause in (\ref{ex:fk10}), coordination (\ref{ex:fk11}), morphological agreement with the noun \textit{cualunque} modified in (\ref{ex:fk12}), a definite determiner in (\ref{ex:fk13}), and a quantifier in (\ref{ex:fk14}). Our hypothesis is that \textit{cualunque} can be used as an adjective with the meaning ‘ordinary/common/unremarkable/low-class/bad’ (\cite{RealAcademiaEspañola2010}, \cite[223]{HaenschWerner2000}). It can also be used as a noun with a similar adjectival meaning of ‘an ordinary (male/female) person’ in (\ref{ex:fk13}). 

\ea\label{ex:fk9} Degree modification (Argentinian Spanish)\\
   \gll es un nombre re cualunque\\
    be.{\PRS.3\SG} a name very \textsc{cualunque}\\
   \glt ‘It’s a very ordinary name’\\
    (Twitter, 2019, \url{https://twitter.com/lauritula/status/1103348577324154880})
\ex \label{ex:fk10}Comparatives (Argentinian Spanish)\\
    \gll Más cualunque que gato atigrado\\
    more \textsc{cualunque} than cat tabby\\
    \glt ‘It’s more common than a tabby cat’\\
    (Twitter, 2019, \url{https://twitter.com/Felicitas73_/status/1126985847314767872})
\ex \label{ex:fk11}Coordination (Argentinian Spanish)\\
    \gll qué cartel [...] ordinario y cualunque\\
    what banner [...] ordinary and \textsc{cualunque}\\
    \glt ‘what an ordinary and unremarkable banner’\\
    (Twitter, 2020, \url{https://twitter.com/sebyo53/status/660538375246626816})
\ex \label{ex:fk12}Agreement (Argentinian Spanish)\\
    \gll A todos los periodistas [...] cualunques\\
    to {all.\PL} {the.\PL} {journalist.\PL} [...] {\textsc{cualunque}.\PL}\\
    \glt ‘to all ordinary journalists [...]’\\
    (Twitter, 2016, \url{https://twitter.com/anaopera/status/704785632200138752})
\ex\label{ex:fk13} Definites (Argentinian Spanish)\\
    \gll el cualunque es el que dice que la culpa la tienen otros.\\
    the.{\textsc{m}.\SG} {\textsc{cualunque}.\SG} {be.{\PRS.3\SG}} the that {say.{\PRS.3\SG}} that the fault it {have.{\PRS.3\SG}} others\\
    \glt ‘the ordinary man is the one who says that others are at fault.’\\
    (\textit{Tiempo Argentino}, 2018, \url{https://www.tiempoar.com.ar/politica/damian-selci-el-militante-es-el-producto-mas-civilizado-que-puede-tener-una-sociedad/})
\ex \label{ex:fk14} Quantifiers (Argentinian Spanish)\\
    \gll Todo virus cualunje\\
    all virus \textsc{cualunque}\\
    \glt ‘all common viruses’\\
    (Twitter, 2016, \url{https://twitter.com/InfoGeneral4p/status/772112368461545473})
\z

Arg.Sp. \textit{cualunque} can also be used as an indefinite with a Free Choice interpretation in modal contexts, such as imperatives (\ref{ex:fk15}), which we interpret as a modalized construction: ‘You can put any shirt’. However, \textit{cualunque} is less frequently used in modal contexts, as we will show using the frequency distribution in  \sectref{sec:fk4}. 

\ea\label{ex:fk15} 
    \gll Ponete una remerita cualunque y una calza: ¡listo!\\
    put.{\IMP.2\SG} a shirt \textsc{cualunque} and a leggings ready\\
    \glt ‘Put on any shirt and a pair of leggings: ready!’\\
    (Twitter, 2019, \url{https://twitter.com/LaSuvuelabajo/status/1179938401056952320})
\z

To sum up, Argentinian Spanish \textit{cualunque} can be used as a gradable adjective, a noun with a meaning similar to that of a gradable adjective, and as a Free Choice indefinite (noun modifier).\largerpage

It is important to note that Argentinian Spanish has another indefinite element, namely \textit{cualquiera} ‘any’. This word is normally used as a Free Choice pronoun in Arg.Sp. and in other Spanish varieties (e.g., \textit{Cualquiera puede hacerlo}. ‘Anybody can do this.’). However, in Argentinian Spanish, in contrast to European Spanish, it can also be used as an adjective, similar to \textit{cualunque} (see \cite{RizzoSalierno2013}, \cite{Kellert2021c}):

\ea\label{ex:fk16}
    \gll es re/tan/muy cualquiera\\
    be.{\PRS.3\SG} very \textsc{cualquiera}\\
    \glt ‘it’s very ordinary’
\ex \label{ex:fk17}
    \gll  Un día re/tan/muy cualquiera\\
    a day very \textsc{cualquiera}\\
  \glt  ‘a very ordinary day’
\z



\citet{RizzoSalierno2013} assumes that \textit{cualquiera} has changed into an adjective as a result of analogy to \textit{cualunque}. However, \textit{cualquiera} and \textit{cualunque} do not have similar uses in any respect. \textit{Cualquiera} in contrast to \textit{cualunque} can also be used as an indefinite under transitive verbs such as \textit{hacer} with the meaning ‘something bad’ and under \textit{mandar} with the meaning ‘incorrect/false’ (\cite{DiTullio2015}, \cite{Kellert2021c}).

\ea\label{ex:fk18}\citep{Kellert2021c}\\
    \gll Hizo cualquiera\\
    do.{\PST.3\SG} \textsc{cualquiera}\\
    \glt ‘She/He did something bad.’
\ex\label{ex:fk19}   \citep{Kellert2021c}\\
    \gll Es así o mande cualquiera?\\
    be.{\PRS.3\SG} like.this or say \textsc{cualquiera}\\
    \glt ‘Is it true or is it false?’
\z\largerpage[2]

Moreover, \textit{cualunque} and \textit{cualquiera} are diatopically different as the geolocation of the tweets with these lexical items have shown (see \sectref{sec:fk2}). The latter is used everywhere in Argentina, whereas \textit{cualunque} is used in a more restricted geographical area (in big cities such as Buenos Aires and Cordoba, where Italian immigrants have settled). Moreover, \textit{cualunque} is used by a particular social group that has some Italian background or identifies with this background as the user profile analysis as well as the content analysis of the tweets show (see \cite{Kellert2021b} for details).\footnote{The relation between cultural background and use of \textit{cualunque} needs to be tested in the future on a large number of quantitative results. All examples studied so far seem to show some relation to Italy or Italian products:
\ea nada que envidiar al Barilla, que son fideos cualunques en Italia.\\
‘measuring up to Barilla, which are ordinary noodles in Italy’\\
(Twitter, 2015, \url{https://twitter.com/ElTanqueMartin/status/641040975561904129})\z} For this reason, due to their syntactic, semantic, and sociolinguistic differences, we will refrain from pursuing a deeper analysis of \textit{cualquiera} and \textit{cualunque} in this article.

The main aim of this article is to understand the differences between modern Arg.Sp. \textit{cualunque} and modern It. \textit{qualunque} and to explain the conditions for variation and change with these lexical words.

This article is structured as follows:  \sectref{sec:fk2} provides a description of the methodology and corpora used for our investigation, as well as an overview of the graphical forms of Arg.Sp. \textit{cualunque}.  \sectref{sec:fk3} describes diaphasic and diastratic factors of linguistic variation and correlates them to different orthographic forms found in the corpora.  \sectref{sec:fk4} describes the frequencies of syntactic properties of Arg.Sp. \textit{cualunque}.  \sectref{sec:fk5} offers a semantic analysis of Arg.Sp. \textit{cualunque}.  \sectref{sec:fk6} suggests hypotheses of the semantic and syntactic change in Argentinian Spanish.  \sectref{sec:fk7} gives a summary and an outlook for future research.


\section{Data and methodology}\label{sec:fk2}\largerpage
In order to investigate the syntactic and semantic distribution of Arg.Sp. \textit{cualunque}, we mainly based our research on corpus data from different sources. The data were collected from the \citetitle{Davies2016E} (CDE) by Mark Davies,\footnote{\textit{Corpus del Español} by Mark Davies consists of five fully annotated and PoS tagged corpora: CDE (Genre/Historical); CDE Web/Dialect; CDE NOW (News On the Web); WordAndPhrase; Google Books $n$-grams (BYU). They are all accessible online via \url{https://www.corpusdelespanol.org}.}  in particular the CDE Web/Dialect and CDE News On the Web (CDE NOW). Both corpora are fully annotated and tagged for Part of Speech (PoS) and allow the user to limit the query to Argentinian Spanish. We also investigated data from a selected set of Google Books $n$-grams (BYU)\footnote{Google Books $n$-grams (BYU) is based on Google Books $n$-grams data. It consists of 45 billion words from 10 million books in Spanish published since the 19\textsuperscript{th} century. It allows searching by part of speech, lemma, word, wildcards, synonym, syntactic construction, and collocations. However, it does not allow one to select a geographical region. We therefore selected a sample of books published in Argentina. The corpus is accessible online at \url{http://www.english-corpora.org/googlebooks/}.} and from the online database \textit{The Internet Archive},\footnote{The digital library \textit{The Internet Archive} provides approximately 250,000 books and texts in Spanish. The online database allows a search query by single terms and phrases, including Boolean operators. The data can be filtered by publication date and language as well. It is available at \url{https://archive.org}.}  as well as data from Twitter localized in Argentina.\footnote{The search on Twitter (\url{https://twitter.com}) was limited to tweets in Spanish published between 2006 and 2019. We considered 238 examples that were geotagged from Argentina.}  We informally asked five native speakers of Argentinian Spanish about the use of \textit{cualunque}. As to graphical variants, in order to cover a large range of forms, we ran a search query for the sequence *alun* on CDE corpora, as well as for \textit{cualunque, cualunques, cualunche, cualunches, qualunque, qualunques, qualunche, qualunches, qualunje, qualunjes, cualungue, cualunques, gualunque, gualunques, gualungue, gualungues, gualunche, gualunches} on \textit{The Internet Archive} and on Twitter.\largerpage

As a result, we obtained a corpus of 360 occurrences of \textit{cualunque} in Argentinian Spanish, which includes the graphical forms listed in their order of frequency in  \tabref{tab:fk1}. 

\begin{table}
\begin{tabular}{lrr}
\lsptoprule
Graphical variants  & \multicolumn{1}{c}{$n$} & \multicolumn{1}{c}{\%} \\
    \midrule
    \textit{cualunque} & 239 & 66.4\% \\
    \textit{cualunche}   & 31 & 8.6\% \\
    \textit{cualunques}  & 30 & 8.3\% \\
    \textit{cualungue}   & 23 & 6.4\% \\
    \textit{qualunque}   & 17 & 4.7\% \\
    \textit{cualunches}  & 11 & 3.1\% \\
    \textit{cualunje}     & 5 & 1.4\% \\
    \textit{cualungues}   & 1 & 0.3\% \\
    \textit{gualunque}    & 1 & 0.3\% \\
    \textit{gualungue}    & 1 & 0.3\% \\
    \textit{qualunques}   & 1 & 0.3\% \\
    \midrule
    Total  & 360 & 100\% \\
\lspbottomrule
\end{tabular}
\caption{Graphical variants of Arg.Sp. \textit{cualunque}}
\label{tab:fk1}
\end{table}

\noindent The graphical variant \textit{cualunque} is the most widespread form. The form \textit{qualunque}, which is orthographically identical with It. \textit{qualunque}, is less frequent than \textit{cualunque}. The plural forms \textit{cualunques} and \textit{cualunches} are especially interesting, as Italian \textit{qualunque} is invariant. The form \textit{cualunche} with \textit{-che(s)} is a mixed representation of Sp. \textit{cual-} and It. \textit{-che} (lit. ‘that’), which might be explained in the future by the influence of Italian dialects spoken by Italian immigrants in Argentina or by an incorrect parsing of \textit{-que} as the Italian complementizer \textit{che} ‘that’.{\interfootnotelinepenalty=10000\footnote{The Sicilian dialect has the form \textit{qualunchi}  \citep[see][1041]{PiccittoTropea1990}. In order to explain \textit{cualunche} with the ending \textit{-che}, we might assume that Sicilian orthography plays a role in the Arg.Sp. form \textit{cualunche}. However, this assumption is problematic, given that it does not explain the phonological change of \textit{-i} to \textit{-e}. Another hypothesis is that speakers who use \textit{cualunche} have parsed \textit{-que} as the Italian complementizer \textit{che} ‘that’, not knowing that \textit{qualunque} is derived from Latin \textit{unquam} and thus cannot be decomposed into \textit{un-que}.}}  We will study these and other hypotheses that explain the variety of graphical forms in detail in the future.

In the following section, we looked at whether different graphical forms correlated with specific register variations.


\section{Correlation of form and register}\label{sec:fk3}
The use of the most widespread form \textit{cualunque} appears in contexts typically associated with an informal register, in sense of \citet{KochOesterreicher2011}. This informal register is spread across different text types or genres, including expository texts (e.g., essays (\ref{ex:fk20}) and newspapers (\ref{ex:fk21})); fiction (\ref{ex:fk22}); blogs about different topics (personal topics, politics, art reviews, sports, etc.), as in (\ref{ex:fk23}), from an educational blog, and in (\ref{ex:fk24}), from social media. 

\ea\label{ex:fk20}\citep{Silberstein1967}\\
\gll Esta fuerza de trabajo es una mercadería cualunque que tiene un valor que está lado por el tiempo de trabajo necesario\\
    this force of labor be.{\PRS.3\SG} a {commodity.\SG} \textsc{cualunque.sg} that have.{\PRS.3\SG} a value that be.{\PRS.3\SG} aside for the time of labor necessary\\
\glt‘This labor force is a commodity just like many other commodities that have a value that is set aside for the necessary labor time’
\ex\label{ex:fk21}
\gll Un militar de nombre cualunque, desconocido para el gran público. Murió el general Carlos A. Martínez.\\
    an officer of {name.\SG} \textsc{cualunque.sg} unknown to the big public die.{\PST.3\SG} the general Carlos A. Martínez\\
\glt    ‘A military officer of common name, unknown to the general public. General Carlos A. Martinez has died.’\\
(Télam S.E. Agencia Nacional de Noticias, 2013, \url{https://memoria.telam.com.ar/lesa-humanidad/202004/muri--el-general-carlos-a--mart-nez--ex-jefe-de-la-side-acusado-por-1200-cr-menes-de-lesa-humanidad_n2478})
\ex\label{ex:fk22}\citep[193]{Romano1999}\\
\gll Primero apareció una ramita cualunque entre dos ladrillos pero ahora hay toda una planta.\\
    first appear.{\PST.3\SG} a twig.\textsc{dim.sg} \textsc{cualunque.sg} between two bricks but now have.{\PRS.3\SG} all a plant\\
\glt  ‘First a normal little twig appeared between two bricks, but now there is a whole plant.’
\pagebreak
\ex\label{ex:fk23}
\gll ¿qué tipo de movimiento es? Un MRUV sencillo y cualunque como cualquier otro.\\
    what type of movement be.{\PRS.3\SG} a MRUV simple.\textsc{m.sg} and \textsc{cualunque.sg} as any other\\
\glt ‘First a normal little twig appeared between two bricks, but now there is a whole plant.’\\
(Blog post, \url{http://ricuti.com.ar/no_me_salen/cinematica/c3fis_22.html})
\ex\label{ex:fk24}
\gll A quién le puede interesar conocer los detalles de mi vida cualunque y cotidiana? La comparto como advertencia.\\
    to whom to.him canbe.{\PRS.3\SG} interest to.know the details of my life.\textsc{f.sg} \textsc{cualunque.sg} and {daily.\SG} it be.{\PRS.1\SG} as warning\\
\glt    ‘Who might be interested in knowing the details of my unremarkable and daily life? I share it as a warning.’\\
    (Twitter, 2015, \url{https://twitter.com/MxAsterion/status/644188778869993474})
\z

The Italian form \textit{qualunque} occurs in a more formal or written speech type in the sense of \citet{KochOesterreicher2011}. Uses of \textit{qualunque} in informal contexts are often related to the speaker’s intention to use a higher register, as in \REF{ex:fk26}. In other cases, they show the speaker’s confidence with the political connotation related to Italian \textit{uomo qualunque} ‘ordinary man’,\footnote{The Italian expression \textit{uomo qualunque} refers to an ‘ordinary or average person’, the ‘man in the street’. The term became popular in Italy in the mid-1940s, when it started being used in satirical and political magazines, in childrens’ literature, and in radio programs broadcast nationwide. However, it was first with Guglielmo Giannini’s satirical magazine \textit{L’Uomo qualunque} and later with his political party \textit{Fronte dell’Uomo Qualunque} that the expression rose in popularity \citep[see][1010]{CortellazzoZolli1996} and took on a political connotation referring to ‘a man that is indifferent with respect to political ideologies, especially in politics’ \citep[see][396]{Cocco2014}. Note that in our corpus the first case of \textit{uomo qualunque} is dated 1953, a few years after Giannini’s party was founded.}  as in (\ref{ex:fk25}).

\ea\label{ex:fk25}
\gll El fascismo explícito y la reivindicación de 
el uomo qualunque,\\
the fascism explicit and the vindication of the.M \textit{uomo.\SG} \textsc{qualunque.\SG}\\
\glt ‘Explicit fascism and the vindication of the “ordinary man”’\\
(\textit{Pagina/12} magazine, 2018, \url{https://www.pagina12.com.ar/147316-un-giro-mas-a-la-derecha-que-afecta-a-sudamerica})
\pagebreak
\ex\label{ex:fk26}
\gll Segundo, aclararle que el libro que le recomendé del Cardenal Ratzinger fue escrito cuando éste era Prefecto de la Congregación para la Doctrina de la Fé (no era un teólogo qualunque por ahí), y se lo cité porque Ud. afirmó y pidió alguien.\\
second clarify.you that the book that to.you {recommend.\PFV.3\SG}  of.the Cardinal Ratzinger was {write.\textsc{ptcp}} when this {be.\IPFV.3\SG} Prefect of the Congregation for the Doctrine of the Faith not {be.\IPFV.3\SG} a theologian.\textsc{sg} \textsc{qualunque.\SG} for there and refl. him {cite.\PFV.3\SG} because you {affirm.\PFV.3\SG} and {ask.\PFV.3\SG}someone\\
\glt ‘Second, to clarify that the book that I recommended to you from Cardinal Ratzinger was written when he was Prefect of the Congregation for the Doctrine of the Faith (he was not just any theologian out there), and I quoted it to you because you affirmed and asked for someone.’\\
(Blog post comment, 2011, \url{https://padrefabian.com.ar/la-guitarra-no-se-puede-usar-en-la-misa/#comment-3128})
\ex\label{ex:fk27}
\gll Nada bueno hay para la \#Argentina en la conjunciòn de un \#Qualunque y un \#Maoista\\
nothing good there.is for the Argentina in the conjuncture of un \textsc{qualunque.\SG} and a Maoist\\
\glt‘There is nothing good for \#Argentina in the conjuncture of a \#Qualunque and a \#Maoist’ (Twitter, 2015, \url{https://twitter.com/Rampa1970/status/663505695938125824})
\z


Plural variants have only been found in informal text forms, illustrated by examples (\ref{ex:fk28}--\ref{ex:fk30}) from Twitter.\largerpage[2]


\ea\label{ex:fk28}
\gll Son re cualungues los auriculares estos\\
be.{\PRS.3\PL} very {\textsc{cualunque}.\PL} {the.\PL} {headphone.\PL} {this.\PL}\\
\glt ‘These headphones are very ordinary’ (Twitter, 2013, \url{https://twitter.com/juanialvar3z/status/352174155883495426})
\ex\label{ex:fk29}
\gll Cuantas cualunches mal vestidas que hay en {shopping [sic]} por dios.\\
{how.many.\PL} {\textsc{cualunque}.\PL} bad {dressed.\PL} that have.{\PRS.3\SG} in shopping for god\\
\glt ‘How many poorly dressed ordinary people there are in the shopping center for God’s sake.’\\
(Twitter, 2013, \url{https://twitter.com/floraguerov/status/348231214219526145})
\ex\label{ex:fk30}
\gll asi es tuiter, {anonimo [sic]} para los cualunques como yo\\
so be.{\PRS.3\SG} Twitter anonymous for the.\textsc{m.pl} {\textsc{cualunque}.\PL} like me \\
\glt ‘so it’s Twitter, anonymous for common people like me’\\
(Twitter, 2019, \url{https://twitter.com/ratakmonodosico/status/1184171952736526336})
\z

In texts with a greater degree of informality and colloquiality (e.g. in comment areas and social media), \textit{cualunque} shows a greater degree of polysemy and syntactic functions, as will be described in the following sections. In these texts, it is mainly used as a Lunfardism, such as to convey irony or anger, or to build a sense of group belonging and familiarity,\footnote{\citet[334f.]{Teruggi1974} lists several reasons that can lead a person to choose a Lunfardism instead of the current Spanish word. Among these are the intention to establish contact and easy communication, as well as to express a social code. A Lunfardism can also be chosen for humorous effect or to cause irritation, to express a social criticism, to demonstrate tenderness, or even to denote scorn or contempt. For this reason, it is not uncommon to find expressive uses of Arg.Sp. \textit{cualunque}, especially in social media.}  as in the following example from a local magazine blog, where the author addresses the Argentinians, including himself.

\ea\label{ex:fk31}
Los argentinos no estamos acostumbrados a los sabores combinados, de ahí que las especias más usadas sean la pimienta y el pimentón, a lo sumo ají molido y orégano en algunos chimichurris. Pero hay vida más allá del salero. Muchas más especias, algunas realmente exóticas que pueden hacer que un arroz cualunque sea un plato gourmet.
\glt ‘We Argentinians are not used to combined flavors, so the spices used most are pepper and paprika, at most ground chili and oregano in some chimichurris. But there is life beyond the saltshaker. Many more spices, some really exotic, that can turn ordinary rice into a gourmet dish.’\\
(\textit{Perspectiva Sur}, 2016, \url{https://www.perspectivasur.com/3/59064-especias-esenciales-para-darle-onda-a-tus-platos})
\z

To summarize this section, the Arg.Sp. form \textit{cualunque} is usually used in informal speech and not in discussions of topics of a legal or scientific nature. 

It is important to note that Italian \textit{qualunque} shows a similar register or genre restriction when it appears in postnominal position, such as \textit{un uomo qualunque} ‘an ordinary man’, but not when it is in prenominal position, such as \textit{qualunque uomo} ‘any man’ \citep[see][]{Kellert2021a}. \citet{Kellert2021d} argues that the correlation between register variation, position, and meaning is not surprising, given that only postnominal \textit{qualunque} can have the evaluative meaning of ‘unremarkable’ and that expressions with this meaning can only be used in a certain register (informal) and genre type (non-scientific, non-legal genre). It seems that the most widespread form Arg.Sp. \textit{cualunque} mimics Italian postnominal \textit{qualunque} in this respect. This might be interpreted as a hint for the source of linguistic change (see  \sectref{sec:fk6}).

In the following section, we look at the frequencies of the syntactic functions of \textit{cualunque} that can give us a hint concerning its syntactic and semantic development. For this reason, we classified the collected data according to their morphosyntactic and semantic properties. As for morphosyntax, we analyzed the distribution within the nominal phrase (NP) and its co-occurrence with: a) nouns; b) determiners, such as definite and indefinite articles, possessives, and demonstratives; c) all types of quantifiers; d) prepositional phrases; e) modifiers; f) coordination with adjectives; and g) plural agreement. We also classified Arg.Sp. \textit{cualunque} with relation to its syntactic role and its co-occurrence with: a) different types of verbs, such as transitive, intransitive, copula, and modal verbs; b) different verb moods; and c) different clause types. As for the semantic properties, we analyzed a) \textit{cualunque}’s function and interpretation; b) the semantic features of the nouns, such as [+human] or [$-$human]; and c) the lexical semantics of the verbs. We then looked into correlations between different meanings and different syntactic structures. We will report the results in the following sections.

\section{Frequencies of syntactic properties of Arg.Sp. \textit{cualunque}}\label{sec:fk4}
In this section, we will mainly deal with the frequency and the extent to which they can tell us something about the variation and change of \textit{cualunque}. We decided to show just a few examples of each variable in the table (for more examples see Appendix~\ref{sec:fk9.1}).

\tabref{tab:fk2} shows the frequency distribution of syntactic functions of Arg.Sp. \textit{cualunque} in our corpus. The adjective (\ref{ex:fk32}) is the most frequent use, compared to the noun (\ref{ex:fk33}) and the indefinite use as in (\ref{ex:fk15}), repeated below as (\ref{ex:fk34}). Note that the indefinite use of It. \textit{qualunque} (see, e.g., (\ref{ex:fk1}),  \sectref{sec:fk1}) is the most frequent function in Italian \citep{Kellert2021a}.\pagebreak

\begin{table}
%\small
    \begin{tabular}{lrr}
    \lsptoprule
         & n & \% \\
        \midrule
        Adjective & 193 & 81\%\\
        Noun & 64 & 18\%\\
        Indefinite & 3 & 1\%\\
        \midrule
        Total & 360 & 100\% \\
        \lspbottomrule
    \end{tabular}
    \caption{Frequencies of syntactic functions of Arg.Sp. \textit{cualunque}}
    \label{tab:fk2}
\end{table}

\ea\label{ex:fk32}
    \gll en realidad era un lazo cualunque\\
    in reality {be.\IPFV.3\SG} a bond \textsc{cualunque.\SG}\\
    \glt ‘it was actually a common bond’\\
    (Blog comment, 2010, \url{http://todosgronchos.blogspot.com/2010/11/todos-somos-vanesa.html})
\ex \label{ex:fk33} 
    \gll Ése es el Cualunque que designaron para dirigir la Selección\\
    that {be.\PRS.3\SG} the \textsc{cualunque} that {appoint.\PFV.3\PL} for lead the National.Team\\
    \glt ‘That is the ordinary person that was appointed to lead the National Team’\\
    (Twitter, 2017, \url{https://twitter.com/domingo_melia/status/945444579267969025})
\ex \label{ex:fk34} 
    \gll Ponete una remerita cualunque y una calza: ¡listo!\\
    {put.\IMP.2\SG} a shirt \textsc{cualunque.\SG} and a leggings ready\\
    \glt ‘Put on any shirt and a pair of leggings: ready!’\\
    (Twitter, 2019, \url{https://twitter.com/LaSuvuelabajo/status/1179938401056952320})
\z

We suggest the following syntactic analysis of \textit{cualunque} in Arg.Sp. for (\ref{ex:fk32}) and (\ref{ex:fk33}).\footnote{As the indefinite use of \textit{cualunque} represents only 1\%, we do not analyze it in detail in this article.}

\ea\label{ex:fk35} [Determiner \textit{un/el} N Adj \textit{cualunque}] \z
\ea\label{ex:fk36} [Determiner \textit{un/el} N \textit{cualunque}] \z

\tabref{tab:fk3} shows the frequency distribution of different positions in which the adjectival \textit{cualunque} appears. The most frequent use is the adnominal position as in (\ref{ex:fk32}), followed by predicative position (\ref{ex:fk37}) and co-occurrence with a degree adverb (\ref{ex:fk38}).

\begin{table}
    \begin{tabular}{lrr}
    \lsptoprule
        Positions of Adj \textit{cualunque} & n & \% \\
        \midrule
        Adnominal (\textit{cualunque} N, N \textit{cualunque}) & 261 & 89\%\\
        Predicative (N \textit{es cualunque}) & 20 & 7\%\\
        With degree adverbs (\textit{re cualunque}) & 12 & 4\%\\
        \midrule
        Total & 293 & 100\% \\
        \lspbottomrule
    \end{tabular}
    \caption{Positions of Adj \textit{cualunque}}
    \label{tab:fk3}
\end{table}


\ea\label{ex:fk37}
 \gll el ataque de Macri a Lifschitz era “cualunque”\\
    the attack of Macri on Lifschitz {be.\IPFV.3\SG} \textsc{cualunque.\SG}\\
    \glt ‘Macri's attack on Lifschitz was “ordinary”’\\
    (\textit{Página\slash12} magazine, 2016, \url{https://www.pagina12.com.ar/diario/suplementos/rosario/10-55616-2016-07-22.html})
    \z
    
\ea\label{ex:fk38} \gll  Muy cualunque!\\
    very \textsc{cualunque.\SG}\\
   \glt  ‘very ordinary!’\\
    (Twitter, 2017, \url{https://twitter.com/leo_argentino/status/826620435257294850})
\z

\tabref{ex:fk4} shows the distribution of noun, adjective, and indefinite \textit{cualunque} with different types of determiners ($+/-$ definite as in, e.g., \textit{el/un} (N) \textit{cualunque}, quantifiers as in, e.g., \textit{todo/algún/dos} (N) \textit{cualunque(s)}).

\begin{table}
%\small
    \begin{tabularx}{\textwidth}{lYYYYYYYY}
    \lsptoprule
     & \multicolumn{2}{c}{Adj.} & \multicolumn{2}{c}{Noun} & \multicolumn{2}{c}{Indef.} & & \\
     & \multicolumn{2}{c}{\textit{cualunque}} & \multicolumn{2}{c}{\textit{cualunque}} & \multicolumn{2}{c}{\textit{cualunque}} & \multicolumn{2}{c}{Total} \\
     \cmidrule(lr){2-3}\cmidrule(lr){4-5}\cmidrule(lr){6-7}\cmidrule(lr){8-9}
        & n & {\%} & n & {\%} & n & {\%} & n & {\%} \\
        \midrule
      Det [$-$def] & 161 & {55\%} & 20 & {31\%} & 3 & {100\%} & 183 & {51\%} \\
      Det [$+$def] & 56 & {19\%} & 17 & {27\%} & 0 & {0\%} & 73 & {20\%} \\
      Det [Q] & 9 & {3\%} & 3 & {5\%} & 0 & {0\%} & 12 & {3\%} \\
      {[$-$Det]}  & 67 & {23\%} & 24 & {38\%} & 0 & {0\%} & 92 & {26\%} \\
        \midrule
        Total & 293 & {100\%} & 64 & {100\%} & 3 & {100\%} & 360 & {100\%} \\
        \lspbottomrule
    \end{tabularx}
    \caption{Distribution of Arg.Sp. \textit{cualunque} in co-occurrence with determiner types}
    \label{tab:fk4}
\end{table}



The co-occurrence of Arg.Sp. \textit{cualunque} with quantifiers and definite nouns is especially interesting, as this distribution is impossible in Italian \citep[see][]{Kellert2021a}. Unlike its Italian counterpart, Arg.Sp. \textit{cualunque} can occur with definite determiners, as in (\ref{ex:fk33}) and (\ref{ex:fk39}), cardinals (\ref{ex:fk40}), negative quantifiers (\ref{ex:fk41}), existential quantifiers, such as \textit{cualquier} ‘any’ (\ref{ex:fk42}) and \textit{algún} ‘some’ (\ref{ex:fk43}), universal quantifiers (\ref{ex:fk44}), and distributive quantifiers, as in (\ref{ex:fk45}).\largerpage[-1]\pagebreak

\ea\label{ex:fk39} demonstrative\\
    \gll este tirano cualunque\\
    this tyrant.\SG{} \textsc{cualunque}.\SG{}\\
    \glt ‘this ordinary tyrant’\\
    (\textit{La Nacion}, 2018, \url{https://www.lanacion.com.ar/espectaculos/teatro/la-farsa-dictaduras-nid2167685})
\ex\label{ex:fk40} cardinal number\\
    \gll los dos apellidos más cualunques de la Argentina\\
    the.\textsc{art.m.pl} two surname.\textsc{m.pl} more \textsc{cualunque}.\PL{} of the Argentina\\
    \glt ‘the two most ordinary surnames in Argentina.’\\
    (Twitter, 2019, \url{https://twitter.com/blackflag1974/status/1191379957919944710})
\ex \label{ex:fk41} negative quantifier\\
    \gll no tolera el pelo pecho en ningún hombre cualunque\\
    not tolerate.\PRS.3.\SG{} the hair.\textsc{m.sg} chest.\textsc{m.sg} in any man.\SG{} \textsc{cualunque}\\
    \glt ‘(she) does not tolerate chest hair on any ordinary man’\\
    (Twitter, 2012, \url{https://twitter.com/lasptefes/status/190825845210877953})

\ex\label{ex:fk42} existential quantifier \textit{cualquier} ‘any’\\\largerpage[1.5]
    \gll termino poniendo Favorito cualquier tweet cualungue.\\
    finish.\PRS.1\SG{} put.\textsc{ger} favorite any tweet \textsc{cualunque}\\
    \glt ‘I end up bookmarking any unimportant tweet’\\
    (Twitter, 2012, \url{https://twitter.com/Palladino_shu/status/233059298383626240})

\ex\label{ex:fk43} existential quantifier \textit{algún,a} ‘some’\\
    \gll Si querés saber qué anda diciendo la prensa, escuchá 3 minutos a algún cualunque del montón\\
    if want.\PRS.2\SG{} know.\INF{} what go.\PRS.3\SG{} say.\textsc{ger} the press listen.\IMP.2\SG{} three minute.\PL{} to some \textsc{cualunque} of.the pile\\
    \glt ‘If you want to know what the press is saying, listen for three minutes to some common person of the pile’\\
    (Twitter, 2015, \url{https://twitter.com/Bastion2008/status/655911353639636992})

\ex\label{ex:fk44} universal quantifier \textit{todo,a}\\
    \gll A todos los periodistas y/o humanos cualunques\\
    to all.\PL{} the.\PL{} journalist.\PL{} and/or human.being.\PL{} \textsc{cualunque}.\PL{}\\
    \glt ‘To all journalists and/or common human beings’\\
    (Twitter, 2016, \url{https://twitter.com/anaopera/status/704785632200138752})

\ex\label{ex:fk45}
distributive quantifier \textit{cada} ‘each’\\
    \gll Va cada qualunque, mira si no va a poder ir ella?\\
    go.\PRS.3\SG{} each \textsc{cualunque} look.\IMP.2\SG{} if not go.\PRS.3\SG{} to be.able go she\\
    \glt ‘Every unimportant person goes, let’s see if she is not going to be able to go?’\\
    (Twitter, 2019, \url{https://twitter.com/MarceMolt/status/1126438286564757504})
\z

The distribution of different quantifiers and determiners supports our syntactic analysis in (\ref{ex:fk35}); that is, the element \textit{cualunque} is an adjective and not a quantifier/determiner as suggested in the literature for It. \textit{qualunque} \citep[see][]{AloniPort2013}:

\ea\label{ex:fk46} [Quantifier/Determiner \textit{el/todo/algún} N Adj \textit{cualunque}]
\z

The following table summarizes the differences in the distribution of Arg.Sp. \textit{cualunque} and It. \textit{qualunque}. Arg.Sp. \textit{cualunque} can be used as a (gradable) adjective. It can also be used with definite articles. These two uses are impossible in Italian (see, e.g., (\ref{ex:fk5}) and (\ref{ex:fk7}) in  \sectref{sec:fk1}).

\begin{table}[h]
%\small
    \begin{tabularx}{\textwidth}{lCCC}
    \lsptoprule
         \multirow{2}{*}{Language/Use} & Determiner & Determiner & (Degree) \\
         & {[}$+$def{]} & {[}$-$def{]} & Adjective\\
    \midrule
        Arg.Sp  \textit{cualunque} & $+$ & $+$ & $+$\\
        It.  \textit{qualunque} & $-$ & $+$ & $-$\\
    \lspbottomrule
    \end{tabularx}
    \caption{Uses of Arg.Sp. \textit{cualunque} and It. \textit{qualunque}}
    \label{tab:fk5}
\end{table}


Let us now look at the distribution of Arg.Sp. \textit{cualunque} in relation to verbal mood. As \tabref{tab:fk6} shows, the indicative verbal mood, as in (\ref{ex:fk32}) or (\ref{ex:fk45}), is the most frequent one (see Appendix~\ref{sec:fk9.2} for further examples).\footnote{Sentences with verbal ellipsis have not been counted. The number of cases with verbal ellipsis amounts to 95, constituting 26\% of the total number of occurrences.}  Uses under the subjunctive, as in (\ref{ex:fk47}), the imperative, as in (\ref{ex:fk34}), or modal verbs, as in (\ref{ex:fk48}),\footnote{Cases with structures of the type [Vmod Vinf] and [Vfin Vinf] have also been counted under [infinitive]. They constitute 23\% and 15\% of [infinitive] cases, respectively, with 9 [Vmod Vinf] and 6 [Vfin Vinf] occurrences out of 40.}  are much less frequent.

\ea\label{ex:fk47}
    \citep[][287]{Medina1989}\\
    \gll Si vos fueras un cualunque \\
    if you.\textsc{sg} be.\textsc{sbjv.2sg} a \textsc{cualunque}\\
    \glt ‘If you were a common person’
\ex\label{ex:fk48}
    \gll no se puede ver ni un noticiero cualunque\\
    not \textsc{refl} can.\textsc{prs.3sg} see even a newscast \textsc{cualunque}\\
    \glt ‘you can’t even watch a regular newscast’\\
   (Twitter, 2018, \url{https://twitter.com/Buda48/status/1055953281929920522})
\z

\begin{table}
%\small
    \begin{tabular}{lrr}
    \lsptoprule
       V mood      & n   & \% \\
    \midrule
       Indicative  & 196 & 74\% \\
       Infinitive  & 40  & 15\% \\
       Subjunctive & 13  & 5\% \\
       Gerund      & 9   & 3\% \\
       Imperative  & 4   & 2\% \\ 
       Conditional & 3   & 1\% \\
    \midrule
       Total       & 265 & 100\% \\
    \lspbottomrule
    \end{tabular}
    \caption{Arg.Sp. \textit{cualunque} and verbal mood}
    \label{tab:fk6}
\end{table}


\tabref{tab:fk7} shows the distribution of Arg.Sp. \textit{cualunque} with respect to verbal mood indicative and tense\slash aspect (see Appendix~\ref{sec:fk9.2} for further examples). The most frequent use of \textit{cualunque} is with the indicative present tense, as in (\ref{ex:fk49}), and the past tense with perfect and imperfect aspect, as in (\ref{ex:fk50}) and (\ref{ex:fk51}), respectively.\largerpage[2]

\ea\label{ex:fk49}
    \gll No tenés una parrilla cualunque?\\
    no have.\textsc{prs.2sg} a grill \textsc{cualunque}\\
    \glt ‘Don’t you have a common grill?’\\
    (Twitter, 2018, \url{https://twitter.com/blackflag1974/status/1022158271384379392})
\ex\label{ex:fk50}\citep{Marchetti2014}\\
    \gll Me compré uno de esos jugos cualunques\\
    \textsc{refl} buy.\textsc{pfv.1sg} one of those juice.\textsc{pl} \textsc{cualunque}.\textsc{pl}\\
    \glt ‘I bought one of those bad juices’
\ex\label{ex:fk51}\citep[][45]{LozzaPugliese1985}\\
    \gll Iba con un traje cualunque.\\
    go.\textsc{ipfv.3sg} with a suit \textsc{cualunque}\\
    \glt ‘He was wearing a cheap suit.’
\z

Note that the distribution of verbal mood, tense, and aspect that co-occurs with \textit{cualunque} stands in contrast to the distribution of verbal mood, tense, and aspect of Free Choice elements such as the prenominal It. \textit{qualunque} or Engl. \textit{any}, which are only possible in modal contexts and are ungrammatical in non-modal contexts such as with verbs with perfective aspect, indicative mood present, and past tense \citep[see][]{Quer2000, AloniPort2013, GiannakidouQuer2013, Kellert2021c}. We thus suggest that Arg.Sp. \textit{cualunque} has undergone a process of word class change that is visible not only with respect to the type of Determiner Phrase (DP), but also with respect to Tense and Aspect. This change needs to be explained (see  \sectref{sec:fk5} and \sectref{sec:fk6}).

\begin{table}
%\small
    \begin{tabular}{lrr}
    \lsptoprule
       Indicative V tense and aspect      & n   & \% \\
    \midrule
       Present  & 142 & 72\% \\
       Perfect  & 31  & 16\% \\
       Imperfect & 20  & 10\% \\
       Future      & 3   & 2\% \\
    \midrule
       Total       & 196 & 100\% \\
    \lspbottomrule
    \end{tabular}
    \caption{Distribution of Arg.Sp. \textit{cualunque} with relation to indicative V tense}
    \label{tab:fk7}
\end{table}

To sum up, we have shown major differences in the use of Arg.Sp. \textit{cualunque} and It. \textit{qualunque} which are: a) Arg.Sp. co-occurs with definite determiners, and quantifiers, b) it can be used as a gradable adjective, and c) it is not restricted to modal contexts, but often appears with indicative mood in the present or past tenses (with perfective aspect).\largerpage[2]

\section{Semantic analysis of \textit{cualunque}}\label{sec:fk5}
In this section, we suggest a detailed semantic analysis of Arg.Sp. \textit{cualunque}. We first start with the notion of the anti-specificity of indefinites and quantifiers in order to explain how this notion is related to It. \textit{qualunque} or similar items in Italian dialects that were used by Italians who emigrated to Argentina. We will then postulate that a sort of pragmatic strengthening happened in later periods that has induced a semantic and syntactic change of \textit{cualunque} in Arg.Sp., which has been described in the present section.

\subsection{Anti-specificity}\label{sec:fk5.1}
When governed by a modal sentence, Free Choice indefinites such as It. \textit{qualunque} are interpreted as triggering a set of alternatives (see  \sectref{sec:fk1}). That is, under modals, \textit{qualunque libro} or \textit{un libro qualunque} denotes a set of possible books, such as this book or that book (see  \sectref{sec:fk1}). The phrase \textit{qualunque libro} or  \textit{un libro qualunque} ‘any book’ does not refer to a specific book, in contrast to \textit{questo libro} ‘this book’ or \textit{il libro che mi hai portato ieri} ‘the book you brought to me yesterday’. This contrast of +/$–$ specific nouns has been already observed in the literature (\cite[see][]{Heusinger2011, EtxeberriaGiannakidou2014, GiannakidouQuer2013} and references therein). Specificity is indicated when “a speaker uses an indefinite noun phrase and intends to refer to a particular referent” \citep[][10]{Heusinger2011}. Conversely, referential vagueness can be interpreted as an absence of specificity \citep{Heusinger2011, GiannakidouQuer2013}. Free Choice indefinites, such as \textit{qualunque} and \textit{cualquiera}, and epistemic indefinites, such as \textit{algún} in (\ref{ex:fk53}), are anti-specific and express referential vagueness. They thus signal a lack of referential intent \citep{Heusinger2011}. Referential vagueness can be described as a variation in possible values of the expression and in uncertainty about which one is the actual value. The uncertainty about the actual value follows from the Free Choice interpretation, which states that any value is possible (\cite[see][]{AloniPort2013}, among many others).

One way to test referential vagueness and anti-specificity is to use the specificity test, in which the referent of the indefinite noun is specified post-hoc, as in the \textit{it’s}-clause in (\ref{ex:fk52}):

\ea \label{ex:fk52}
    There is someone at the door. It’s Mr. Smith. 
\z

Such continuations are incompatible with \textit{algún}-phrases, as illustrated in (\ref{ex:fk53}) \citep[see][]{Menéndez-Benito2010, Alonso-OvalleMenéndez-Benito2010}:

\ea \label{ex:fk53}\citep[36]{Alonso-OvalleMenéndez-Benito2013}\\
    María se casó con algún médico. \# En concreto con el Dr. Smith.\\
    ‘María married some doctor. Concretely, Dr. Smith.’
\z

Free Choice indefinite phrases are also incompatible with continuations, such as (\ref{ex:fk53}) \citep[see][]{Chierchia2006, JayezTovena2007, Falaus2014}.

Anti-specific indefinites never modify singleton nouns, that is, nouns that denote a single atomic individuum such as a single person or thing (the so-called “anti-singleton condition”, \cite[cf.][]{Menéndez-Benito2010}). They must occur with nouns that denote a set with at least two members. This prediction follows from the Free Choice interpretation of \textit{qualunque}, which entails a consideration of at least two alternatives. This prediction is borne out empirically in our data, where N \textit{qualunque} does not occur with singleton nouns, as shown in (\ref{ex:fk54}). Here, N \textit{qualunque} entails at least two different individuals with the name Angela Merkel and at least two second world wars. However, this use is incompatible with our world, in which Angela Merkel refers to a single person, namely the German chancellor and in which the Second World War refers to a single war that occurred from 1940 to 1945:

\ea \label{ex:fk54}
    \# \textit{un,a} singleton N \textit{qualunque} ‘any singleton N’ (\# means ‘awkward’)\\
        a. \# \textit{una Angela Merkel qualunque} ‘any Angela Merkel’\\
        b. \# \textit{una Seconda Guerra Mondiale qualunque} ‘any second world war’\\
\z

Let us now see whether Arg.Sp. \textit{cualunque} can be described with the notion of anti-specificity. Recall that in modal contexts such as imperatives, \textit{cualunque} can have the Free Choice interpretation of ‘any’; that is, every alternative is possible, and the speaker does not distinguish between some alternative or another. \textit{Cualunque} as Free Choice is anti-specific:

\ea \label{ex:fk55}
    \gll Ponete una remerita cualunque y una calza: ¡listo!\\
    put.\textsc{imp.2sg} a shirt \textsc{cualunque} and a leggings ready\\
    \glt ‘Put on some shirt and a pair of leggings: ready!’\\
    (Twitter, 2019, \url{https://twitter.com/LaSuvuelabajo/status/1179938401056952320})\\
    
    \ex \label{ex:fk56}
    \gll Almorzamos en una pizzería cualunque y comemos postre en la casa de los azulejos, porque somos peronistas.\\
    eat.\textsc{prs.1pl} in a pizzeria \textsc{cualunque} and eat.\textsc{prs.1pl} dessert in the house of the azulejos because we.are peronistas.\\
    \glt ‘Let’s eat in a pizzeria no matter which one and eat dessert in the house of azulejos, because we’re Peronistas (= member of a particular political party).’\\
    (Twitter, 2016, \url{https://twitter.com/di__tir/status/813118305548570624})\\
\z

In (\ref{ex:fk21}), repeated below as (\ref{ex:fk57}), \textit{cualunque} is interpreted as ‘epistemically unknown’ (see \cite{Chierchia2013} on a similar kind of interpretation with It. \textit{qualunque}):

\ea \label{ex:fk57}
     Un militar de nombre cualunque, desconocido para el gran público. Murió el general Carlos A. Martínez.\\
    \ea\label{ex:fk57a} ‘A military officer of some name, unknown to the general public. The general Carlos A. Martínez has died.’\\
    \ex\label{ex:fk57b} ‘There is a military officer with a name that is unknown to the general public. This name is Carlos A. Martínez. He has died.’
    \z
    (Télam S.E. Agencia Nacional de Noticias, 2013, \url{https://memoria.telam.com.ar/lesa-humanidad/202004/muri--el-general-carlos-a--mart-nez--ex-jefe-de-la-side-acusado-por-1200-cr-menes-de-lesa-humanidad_n2478})
\z

The sentence in (\ref{ex:fk57b}) states that according to the general public (i.e. the agent of the epistemic modal base),\footnote{The difference between this interpretation and the “epistemic unknown” interpretation of It. \textit{qualunque} suggested in \citet{Chierchia2013} is that in Chierchia’s analysis, the epistemic unknown refers to the speaker and not to the agent as is the case with Arg.Sp. \textit{cualunque}.}  the officer’s name is unknown. From this it follows that the name can be every possible name. Thus, all names are the same, according to the general public’s knowledge. The Free Choice interpretation of \textit{cualunque} is compatible with the ignorance interpretation indicated by \textit{desconocido} ‘unknown’ given in the context of this example.\largerpage[2]

To sum up, Free Choice indefinites and epistemic indefinites are anti-specific, referentially vague, and never occur with singleton nouns.

As we will see in  \sectref{sec:fk5.2}, Free Choice Indefinites can lead to a \textit{pragmatic strengthening} (i.e. strengthening the unspecific meaning of \textit{cualunque}), which can eventually lexicalize and lead to semantic change. In  \sectref{sec:fk6}, we will claim that this is exactly what happened with Arg.Sp. \textit{cualunque}.


\subsection{Pragmatic strengthening}\label{sec:fk5.2}


We saw in  \sectref{sec:fk4} that \textit{cualunque} is used as an adjective and can have the neutral interpretation of ‘ordinary’, ‘common’, or ‘normal’, as in (\ref{ex:fk58}) or (\ref{ex:fk59}):

\ea \label{ex:fk58}
    \gll si sos cliente Black te atienden inmediatamente, si sos 
cualunque, esperà [sic] dos horas.\\
    if be.\textsc{prs.2sg} client Black you attend.\textsc{prs.3sg} immediately, if  be.\textsc{prs.2sg} \textsc{cualunque}, wait.\textsc{prs.3sg} two hours.\\
\glt ‘if you are the client Black they attend to you immediately, if you’re an ordinary person, you have to wait for two hours.’\\
(Twitter, 2013, \url{https://twitter.com/SuHerBre/status/292399334497058816})\\

\ex \label{ex:fk59}
     un dibujo clásico, tradicional, qualunque.\\
    ‘a classic, traditional, ordinary drawing’\\
    (Blog comment, 2013, \url{http://www.comiqueando.com.ar/secciones/el-podcast-de-comiqueando/programa-52/})
\z

The example of \textit{cualunque} in (\ref{ex:fk58}) is used as a predicate describing clients that do not have any particular property and thus represent a kind of person that stands in opposition to a certain and particular type of client, namely \textit{cliente Black}. The example \textit{un dibujo cualunque} in (\ref{ex:fk59}) refers to a kind of drawing described as classic or traditional that does not have any particular properties distinguishing it from very unusual or modern drawings (e.g., drawings by Picasso). To summarize what has been said so far. N \textit{cualunque} refers to a kind of people or object that does not have any specific or particular properties that would distinguish it from others of the same kind. This kind stands in a contrastive relation with another kind that describes people or objects that have particular or distinguishing properties such as \textit{cliente Black} or drawings by Picasso.

The neutral meaning of N \textit{cualunque} is used in a situation where \textit{cualunque} refers to a set of individuals that do not have any distinguished properties. These individuals with no distinguished properties are evaluated somewhere in the middle of a scale that ranks individuals according to their distinguishing properties. For instance, \textit{cualunque} in (\ref{ex:fk59}) refers to classic drawings that are opposed to distinguished drawings, which can be either especially good or especially bad (see also \citet{Alonso-OvalleRoyer2021} for an implicit ranking scale of \textit{komon} on the ‘unremarkable’ reading in the Mayan language Chuj).

The neutral meaning of N \textit{cualunque} often occurs when \textit{cualunque} is evaluated on a frequency scale. The \textit{cualunque}’s meaning as ‘common’ in \textit{un día cualunque} refers to some property of the day, which is very typical for many days and is thus very frequent, such as those filled with some usual activities such as waking up in the morning, going to work, and going to bed in the evening. This kind of a day is contrasted to some specific or particular day, such as a birthday, which may include unusual activities such as celebrating and staying home from work. Particular or specific days are rare because they have certain features that distinguish them from common days.
Actually, all examples in the corpus with N \textit{cualunque} that contain a noun denoting some expression of time, such as \textit{domingo} or \textit{día} as in (\ref{ex:fk60}), can be interpreted as situated in the middle of a frequency scale.

\ea \label{ex:fk60}
    \gll Noooo Bipolardo es lo mejor de este domingo cualunque. Tiene la pluma de oro el que redacta esos\\
    Noooo Bipolardo be.\textsc{prs.3sg} the best of this sunday \textsc{cualunque}. hold.\textsc{prs.3sg} the pen of gold the that edits these\\
   \glt ‘No Bipolardo is the best of this ordinary Sunday. The one who edits those tweets has the Golden Pen.’\\
    (Twitter, 2018, \url{https://twitter.com/caradecumbiaok/status/1046464719530467328})\\
\z

Another example with the neutral meaning of \textit{cualunque} as ‘common’ is given in (\ref{ex:fk61}), where the relative clause \textit{camina por la calle} restricts the set of people to those who walk on the street and who can be observed very frequently. This relative clause makes thus reference to normal or common people:

\ea \label{ex:fk61}
    \gll Al porteño cualunque que camina por la calle no le interesa lo de la base militar china.\\
    the {citizen of Buenos Aires} \textsc{cualunque}  who walk.\textsc{prs.3sg} on the street not him interest.{prs.3sg} the of the base military Chinese \\
   \glt ‘The common citizen of Buenos Aires who walks on the street is not interested in the Chinese military base.’\\
    (Twitter, 2015, \url{https://twitter.com/ton011972/status/562337257328508929})\\
\z

The common interpretation with reference to people is often given when the speaker refers to many people, including himself as \textit{como nosotros} ‘like us’ in (\ref{ex:fk62}) or as \textit{como yo} ‘like me’ in (\ref{ex:fk63}). Note that in the latter example, \textit{cualunque} is used in contrast to a distinguished type of individuals: \textit{gente de clase alta}.

\ea \label{ex:fk62}
    \gll [...], bardear a personas cualunques como nosotros en una red social por opinar distinto es muy pedorro.\\
    {} to.fence to people \textsc{cualunque} like us in a network social for to.think different be.\textsc{prs.3sg} very annoying\\
    \glt ‘[...], to fence normal people like us in a social network for having a different opinion is very annoying.’\\
    (Twitter, 2019, \url{https://twitter.com/Felicitas73_/status/1174698124197212161})\\
    
    \ex \label{ex:fk63}
    \gll Y no hablo de políticos o gente de clase alta, hablo de cualunques como yo\\
    I not talk.\textsc{prs.1sg} about politicians or people of class high, talk.\textsc{prs.1sg} about \textsc{cualunque} like me\\
    \glt ‘I’m not talking about politicians or upper-class people, I’m talking about common people like me’\\
    (Twitter, 2018, \url{https://twitter.com/KaroSci/status/1011244306898866176})\\
\z

The same analysis can be applied to \textit{un dibujo cualunque} in (\ref{ex:fk59}). Classic or traditional drawings are more common in this speech context than drawings with certain properties that are distinguished from classic or traditional drawings (e.g., drawings by Picasso). Clients with undistinguished properties referred to by \textit{cualunque} in (\ref{ex:fk58}) are more common than those with distinguished properties such as \textit{cliente Black}.

To sum up, the ‘common’ meaning of \textit{cualunque} is the result of considering individuals denoted by N \textit{cualunque} as being in the middle of some scale, such as a scale of frequency or some other scale and the result of contrasting these individuals to particular and rarely observable individuals.

The derogatory meaning is the result of evaluating the qualities of the kind of individuals denoted by \textit{cualunque} in contrast to specific individuals with positive qualities. Thus, the evaluation of \textit{cualunque} shifts from the middle of a scale to an extreme end as the result of contrasting \textit{cualunque} to a set of especially good individuals. The derogatory meaning of \textit{cualunque} in (\ref{ex:fk64}) and (\ref{ex:fk65}) introduces a contrast between exceptional people such as \textit{el actor original} or \textit{periodista operadora} and unexceptional people with no distinguished qualities or a simple activist:\largerpage[-1]\pagebreak

\ea \label{ex:fk64}
    \gll Pero que vuelva el actor original, no este cualunque sacado de Antares.\\
    but that return.\textsc{sbjv.3sg} the actor original, not this \textsc{cualunque} take.\textsc{pst.ptcp} from Antares\\
    \glt ‘But let the original actor return, not this poor one taken from Antares.’\\
    (Twitter, 2019, \url{https://twitter.com/AlanG996/status/1115053249197953024})\\
    
    \ex \label{ex:fk65}
    \gll Manguel cayó de periodista operadora K a simple militante cualunque. La degradaron.\\
    Manguel fall.\textsc{pst.3sg} from journalist operator K to simple militant \textsc{cualunque}. Her degrade.\textsc{pst.3pl}\\
    \glt ‘Manguel fell from journalist operator K to simple ordinary activist. She was demoted.’\\
    (Twitter, 2019, \url{https://twitter.com/La_gringai/status/1181608328549584896})\\
\z

The same shift towards a negative meaning of ‘common/normal/usual’ can be observed in diachrony. We find lexicalized pejorizations of the meaning ‘common’ or ‘average’ in other languages such as Fr. \textit{vulgaire}, Engl. \textit{vulgar} or Fr. \textit{médiocre} or Sp. \textit{mediocre} \citep[see][43]{KleparskiBorkowska2007}.

We leave it open for future research why \textit{cualunque} is never contrasted to specific individuals or types that are exceptionally bad on the non-neutral interpretation. One could possibly derive this fact from the Gricean maxim of \textit{informativity} (see \cite{Geurts2010}, Rosemeyer, p.c.). Thus, \textit{cualunque} is not contrasted with bad because such a contrast would not be informative enough according to Gricean maxim of informativity. This is why we do not hear sentences like “he is not a bad writer, but an ordinary writer,” because being an ordinary writer does not contrast sufficiently enough being a bad writer. However, the contrast between a good writer and an ordinary writer is informative enough, because being ordinary can mean being bad in certain contexts. This is why we do hear sentences like “he is not a good writer; he’s an ordinary writer.”\footnote{We thank Malte Rosemeyer for discussing this point with us.}

However, it seems to be a very general pattern that properties describing singleton nouns such as \textit{unique, particular, outstanding} and \textit{special} have a positive meaning more often than a negative one; for instance, \textit{a grade with distinction} is an especially good grade. \textit{John is a remarkable man} means that John has some positive properties that distinguish him from other men. \textit{An extraordinary day} is generally an extremely good day and not an extremely bad day unless uttered with a special intonation. This might explain why \textit{cualunque} is usually contrasted with exceptionally good individuals on the non-neutral meaning.

To sum up so far, Arg.Sp. \textit{cualunque} and It. \textit{qualunque} can have the neutral meaning of ‘common’ as ‘widespread’ or some pejorative version of ‘common’ with the meaning ‘common’ as ‘worse than extraordinary’. We analyzed these meanings as a property over (kinds of) individuals, which imply different scales: a frequency scale and a scale of goodness.

\section{Hypothesis on meaning change}\label{sec:fk6}

We assume that \textit{cualunque} underwent a change into a degree predicate \textit{cualunque} in (\ref{ex:fk66}) and a noun in (\ref{ex:fk67}) from postnominal indefinite \textit{cualunque}:

\ea \label{ex:fk66}
    It. postnominal indefinite \textit{qualunque} (original construction) $\gg$ Arg. Sp. postnominal \textit{cualunque} (by lexical borrowing from Italian) $\gg$ degree adjective \textit{cualunque} (by pragmatic strengthening and syntactic recategorization)\\
    \textit{e.g., cualunque} ‘any’ $\gg$ \textit{(re/tan) cualunque} ‘very ordinary/bad’
    
    \ex \label{ex:fk67}
    \textit{un,a qualunque $\gg$ un,a cualunque} (lexical borrowing from Italian) $\gg$ \textit{el/la cualunque} (semantic shift/recategorization)\\
    \ea \gll las leyes son para los cualunque, o sea nosotros.\\
    the laws be.\textsc{prs.3sg} for the \textsc{cualunque}, or be.\textsc{sbjv.3sg} us\\
       \glt  ‘the laws are for normal people, that is, for us.’\\
        (Twitter, 2019, \url{https://twitter.com/maggiepalacios4/status/1180456730028851201})\\
 \ex \gll Este cualunque que se cree presidente hasta cuando hay que soportar sus mentiras es un payaso como Maduro de Venezuela\\
    this \textsc{cualunque} who \textsc{refl} believe.\textsc{prs.3sg} president until when have.\textsc{prs.3sg} that support his lies be.\textsc{prs.3sg} a clown like Maduro of Venezuela\\
       \glt ‘This ordinary person that believes himself to be president until you have to believe his lies is a clown like Maduro of Venezuela’\\
        (Twitter, 2019, \url{https://twitter.com/Alfredo00649870/status/1085118326324514816})\\
    \z
\z

In order to explain the shift of \textit{cualunque} into a degree adjective as represented in (\ref{ex:fk66}) or as an evaluative noun in (\ref{ex:fk67}), we assume that the scale of goodness, which implies a degree scale (i.e., good is gradable), has been lexicalized in Argentinian Spanish. The lexicalization of the degree scale has driven the reanalysis of \textit{cualunque} as a degree adjective or as a nominal element with an evaluative interpretation:

\ea \label{ex:fk68}
    [DegP \textit{re/muy} ‘very’ [Adj \textit{cualunque}]] ‘very ordinary/bad’\\
    \ex \label{ex:fk69}
    [DP [Noun [+eval] \textit{cualunque}]] ‘someone ordinary/bad’\\
\z

As already mentioned in  \sectref{sec:fk1}, diachronic sources are very scarce. The first diachronic occurrences of \textit{cualunque} in texts written in literary Cocoliche (see \ref{ex:fk70}--\ref{ex:fk72}) have a Free Choice interpretation:

\ea \label{ex:fk70}
    Cocoliche\\
    \gll {Che cosa} volette? {-- [...]. --} Cualunque cosa,-\\
    what want.\textsc{prs.2pl} {} \textsc{cualunque} thing\\
    \glt ‘What do you want? -- [...] -- Any thing’\\
    (\textit{La Mujer} magazine, 1900, \url{https://archive.org/details/lamujer2140unse/page/n191?q=cualunque})
    
    \ex \label{ex:fk71}
    Cocoliche\\
    \gll Lu arquila per dos peso in cualunque montepío.\\
    it.\textsc{acc} pawn.\textsc{prs.3sg} for two pesos in \textsc{cualunque} pawnshop\\
    \glt ‘He pawns it for two pesos in any pawnshop.’\\
    (\textit{P.B.T.} magazine, 1906, \url{https://digital.iai.spk-berlin.de/viewer/fullscreen/861383842/105/})
    
    \ex \label{ex:fk72}
    Cocoliche\\
    \gll Il cochiyo, la fareñera, la finyinga, [...] cualunque de cueli [...] te vale mase que il cuore\\
    the knife the dagger the stab {} \textsc{cualunque} of these {} you be.worth.\textsc{3sg.pres} more than the heart\\
    \glt ‘The knife, the dagger, the stab, any of these are worth more than the heart’\\
    (\textit{Atlántida} magazine, 1930, \url{https://books.google.com.ar/books?id=cB8QAAAAIAAJ})
\z

Finally, the degree predicate \textit{cualunque} and the string [definite/demonstrative N \textit{cualunque}] represent a recent development in Argentinian Spanish. Younger people are more likely to use \textit{cualunque} as a degree adjective than older people, as our investigation of tweets and a linguistic poll on the Facebook group members of Lingüística Argentina has shown \citep{Kellert2021c}. This sociolinguistic contrast can be seen as one indicator of gradual language change \citep[see][]{Stein1990, SeilerEnkeMühlenbernd2018}.

We assume that the syntactic change schematized in (\ref{ex:fk67}) is a consequence of the lexicalization of the pragmatic scale on which speakers and/or hearers evaluate individuals denoted by N \textit{cualunque} in contrast to individuals with specific or special properties. This kind of process of lexicalization of pragmatic inferences is known in the literature on semantic change as \textit{pragmatic strengthening}, described as the process by which meanings tend to be enriched in pragmatic contexts and the resulted implicatures can be eventually conventionalized. This conventionalization is what triggers semantic change (see \citealt[][35]{Traugott1989}, among others). Moreover, Traugott observes that “meanings tend to become increasingly based in the speaker’s subjective belief state/attitude toward the proposition” (\cite[][35]{Traugott1989}). This process is known in the literature as \textit{subjectification} or \textit{pragmaticalization} (see \citep{Diewald2011}). We believe that \textit{cualunque} is another example of pragmatic strengthening, subjectification, and pragmaticalization, because its new meanings are the results of contextual interpretation by the speaker and hearer and the evaluation of individuals on different scales.\footnote{The exact syntactic analysis of the postnominal indefinite \textit{cualunque} in \REF{ex:fk66} (i.e. the original construction) needs to be studied in the future \citep[see][]{Kellert2021c}. One possibility is to assume a two-determiner-analysis of UN N \textit{cualunque}, as has been suggested for UN N \textit{qualunque} (see \cite{Zamparelli2000}, among others). On these accounts, \textit{qualunque} is considered to be a strong quantificational determiner on the same lines as ‘every’, ‘some’, or ‘none’. As a consequence, the indefinite determiner \textit{un} in \textit{un N qualunque} is analyzed as an empty or weak element with no semantic value. One weak point of the determiner analysis of \textit{qualunque} is that the change of Free Choice indefinites into degree predicates or evaluative nouns would be a case of degrammaticalization \citep[see][]{Kellert2021c}, because determiners are functional/grammatical categories that usually do not change into lexical categories. Usually, it is the opposite that occurs; that is, lexical categories change into functional/grammatical categories (see \cite{RobertsRoussou2003}). Another weak point of the determiner analysis is that determiners in Romance languages are usually used in prenominal positions, rarely in postnominal positions (see \cite{Stark2006} for some exceptions).}

Our analysis of \textit{cualunque} denoting a property with different readings such as ‘common’ and the derogatory version of it as ‘not outstanding’ can account for the adjectival behavior of \textit{cualunque} observed in  \sectref{sec:fk4}.

\section{Summary and outlook}\label{sec:fk7}

We have analyzed the synchronic variation of \textit{cualunque} in Arg.Sp. and \textit{qualunque} in Italian and have identified different syntactic categories of \textit{cualunque} with different interpretations:

\begin{description}
\item \textit{Cualunque} as a Free Choice indefinite ‘any’ or as an epistemic indefinite ‘some’ with the meaning ‘epistemically unknown’.
\item \textit{Cualunque} as a (gradable) adjective with the neutral meaning of ‘common’ as ‘widespread’ or ‘frequent’ or ‘normal’ or with the derogatory meaning of ‘common’ as ‘worse than outstanding’ or even ‘bad’. This syntactic use is not possible with Italian \textit{qualunque}, although both interpretations are possible in Italian in the predicative context of \textit{UN N qualunque}.
\item \textit{Cualunque} as a noun with either a neutral or depreciative meaning. The nominal use is only possible with indefinite determiners in Italian, not with definite determiners or quantifiers as in Argentinian Spanish.
\end{description}

The analysis suggested in this article can be applied to similar cases in other Romance languages that allow similar indefinites in non-modal contexts such as predicative position. This is shown for French \textit{n’importe quoi, quelconque,} and \textit{cualquiera} in Argentinian Spanish or the nominalized \textit{cualquiera} in European Spanish (see \cite{Kellert2021c}, see also \citetv{chapters/05} for Catalan \textit{qualsevol}):

\ea \label{ex:fk73}
    French\\
    \gll C’est du n’importe quoi.\\
    it.be.\textsc{prs.3sg} of.the \textsc{n'importe} \textsc{quoi}\\ 
    \glt ‘It’s total nonsense.’\\
    
    \ex \label{ex:fk74}
    French\\
    \gll C’est très quelconque.\\
    {it.be.\textsc{prs.3sg}} very \textsc{quelconque}\\
    \glt ‘It’s very ordinary.’\\
    
    \ex \label{ex:fk75} 
    Argentinian Spanish\\
    \gll Es (re) cualquiera.\\
    be.\textsc{prs.3sg} \textsc{re} \textsc{cualquiera}\\
    \glt ‘It’s really worthless/nothing special’\\
    
     \ex \label{ex:fk76}
     European Spanish\\
    \gll Juan es un cualquiera.\\
    Juan be.\textsc{prs.3sg} a \textsc{cualquiera}\\
    \glt ‘Juan is a nobody/low-class person.’\\
\z
In the future, we will study the question what determines the pejorative or the neutral interpretation of \textit{cualunque}. So far, it seems that one important feature that biases one interpretation or the other is the inclusion or exclusion of the speaker in the set of ordinary people denoted by the meaning of \textit{N cualunque} (see  \sectref{sec:fk6}). If the speaker includes herself in the set of ordinary people, \textit{cualunque} has a neutral meaning, whereas if the speaker does not, then \textit{cualunque} has only the pejorative interpretation:

\ea \label{ex:fk77}
 (exclusion of the speaker) pejorative meaning\\
    \gll sos una cualunque.\\
     be.\textsc{prs.2sg} a \textsc{cualunque}\\
    \glt ‘You are an unimportant/low-class woman’\\
    (Twitter, 2013, \url{https://twitter.com/BrendaCapello/status/369843684021252096})\\
    
\ex \label{ex:fk78}
(inclusion of the speaker) neutral meaning\\
\gll las leyes son para los cualunque, o sea nosotros\\
the laws be.\textsc{prs.2pl} for the.\textsc{2pl} \textsc{cualunque}, or be.\textsc{sbjv.3sg} us\\
\glt ‘the laws are for normal people, that is, for us’\\
(Twitter, 2019, \url{https://twitter.com/maggiepalacios4/status/1180456730028851201})\\
\z

Another important factor that biases the pejorative use is the lexical semantics of the noun modified by \textit{cualunque}. If the noun is a depreciative word, like \textit{negro} in (\ref{ex:fk79}) \citep[see][]{Kellert2021b}, and the speaker excludes herself from the set, \textit{cualunque} has a pejorative interpretation:

\ea \label{ex:fk79}
    \gll Sos un negro cualunque.\\
    be.\textsc{prs.2sg} a \textsc{negro} \textsc{cualunque}\\
    \glt ‘You’re a (just) a simple person/You’re nobody important’ (pejorative use)\\
    (Twitter, 2020, \url{https://twitter.com/LeluuArtero/status/233997516054601728})\\
\z

If the same lexical noun co-occurs with positive expressions such as \textit{te quiero} ‘I love you’, this creates an ironic meaning due to the use of depreciative nouns like \textit{negro cualunque} in a positive context:

\ea \label{ex:fk80}
    \gll luchiditatta si8siis te quiero negrito cualungue\\
    luchiditatta si8siis you love.\textsc{prs.1sg} \textsc{negrito} \textsc{cualunque}\\
    \glt ‘I love you little unimportant person’\\
    (Twitter, 2014, \url{https://twitter.com/facuundit/status/438433004042846208})\\
    
    \ex \label{ex:fk81} 
    \gll Nacho: Te amo negra cualungue♥\\
    Nacho: you love.\textsc{prs.1sg} \textsc{negra} \textsc{cualunque}\\
    \glt ‘Nacho: I love you unimportant female person’\\
    (Twitter, 40, \url{https://twitter.com/MelaCerioli/status/431786831042994176})\\
\z

\begin{sloppypar}
Another observation is that scalar focus particles can bias the derogatory meaning of \textit{cualunque} as well \citep[see][]{Kellert2021c}. The focus particle \textit{solo} ‘only/mere’ has the function of excluding high-value alternatives. In the following example, the focus particle \textit{solo} associates with a focus alternative \textit{una banda cualunque}, which leads to the exclusion of all other alternatives (‘particular bands’). This exclusion leads to the interpretation of \textit{banda cualunque} as a band being low on the scale of high-quality bands:
\end{sloppypar}

\ea \label{ex:fk82}
    \gll Pense que el sabado ibamos a ir a una re fiesta, y al final solo toca una banda cualunque.\\
    think.\textsc{pst.1sg} that the saturday go.\textsc{pfv.1pl} to go to a very party, and in end only play.\textsc{prs.3sg} a band \textsc{cualunque}\\
   \glt ‘I thought that on Saturday we were going to a good festival, but in the end only an ordinary band was playing.’\\
    (Twitter, 2016, \url{https://twitter.com/solariascelli/status/722543527159324672})
\z

An interesting shift in semantic interpretation can be observed with pets (see \ref{ex:fk83}). There, the meaning of \textit{cualunque} is ‘mixed-breed’, e.g., \textit{gato cualunque} (\ref{ex:fk84}) ‘mixed-breed cat’ or \textit{perro cualunque} (\ref{ex:fk85}) ‘mongrel’. We assume that this semantic shift is probably due to the meaning of \textit{cualunque} as ‘not special/not outstanding’. In the domain of pets, the meaning ‘not-outstanding’ means not belonging to a ‘pure breed’.

\ea \label{ex:fk83}
    \gll Mis bebas son cualungues/mestizas osea no son de raza solamente mi ahijado […] que se llama León y es un chihuahua.\\
    my babies be.\textsc{prs.3sg} \textsc{cualunque}/half-breed/mixed, i.e., not be.\textsc{prs.3pl} of pedigreed, only my godson [...] that \textsc{refl} be.called.\textsc{prs.3sg} Leon and be.\textsc{prs.3sg} a chihuahua.\\
    \glt ‘My babies are half-breed/mixed, i.e., they’re not pedigreed, only my godson [...] whose name is Leon and who’s a chihuahua.’\\
    (Twitter, 2018, \url{https://twitter.com/maruuchis85/status/968997312113586176})\\
    
     \ex \label{ex:fk84}\citep[][132]{Medina1989}\\
     \gll luego volvió a te con un gato cualunque\\
     later return.\textsc{pst.3sg} to you with a cat \textsc{cualunque}\\ 
    \glt ‘(he) later came back to you with a mixed-breed cat’
    
     \ex \label{ex:fk85}
     \gll A mi me gustan los perros cualunches.\\
     to me me like.\textsc{prs.3pl} the dogs \textsc{cualunque.pl}\\
    \glt ‘I like mongrels.’\\
    (Twitter, 2017, \url{https://twitter.com/QmaxiQ/status/892576809400848387})\\
\z

This hypothesis needs to be checked in future research. In the future, we will provide a detailed semantic analysis of the evaluative meaning of \textit{cualunque} (see \cite{Gutzmann2013}'s analysis of expressive elements).

\section*{Acknowledgements}\label{sec:fk8}
We thankfully acknowledge the funding provided by the Deutsche Forschungsgemeinschaft (DFG) for the project “Quantification in Old Italian”. A warm thanks goes to Andrés Saab and all our informants for their valuable comments on Argentinian Spanish. 

\appendixsection{Data reflecting \tabref{tab:fk3} in  \sectref{sec:fk4}}\label{sec:fk9.1}

\ea \label{ex:fk86}
    [cualunque N] \citep{Boot2012}\\
    \gll un cualunque maestro mayor de obra.\\
    a \textsc{cualunque} master senior of work\\
    \glt ‘a general contractor like any other’
    
    \ex \label{ex:fk87}
    [N cualunque]\\
    \gll un programa cualunque de televisión\\
    a program \textsc{cualunque} of television\\
    \glt ‘an ordinary TV program’\\
    (\textit{Rock and Ball} magazine, 2013, \url{https://rockandball.com.ar/punto-de-vista/futbol-para-todos-menos-para-los-que-van-a-la-cancha-76096/})
    
    \ex \label{ex:fk88}
    [N parece cualunque]\\
    \gll me pareció re cualunque la interpretación\\
    to.me appear.\textsc{pfv.3sg} very \textsc{cualunque} the interpretation\\
    \glt ‘I found the interpretation really bad’\\
    (Twitter, 2015, \url{https://twitter.com/LukeAKD/status/605901400157978624})
    
    \ex \label{ex:fk89}
    [N es cualunque]\\
    \gll tus transmisiones de fútbol son ordinarias y cualunques\\
    your broadcast.\textsc{pl} of soccer be.\textsc{prs.3pl} ordinary.\textsc{pl} and \textsc{cualunque}.\textsc{pl}\\
    \glt ‘your soccer broadcasts are ordinary and of low quality’\\
    (Twitter, 2016, \url{https://twitter.com/CCesaroni/status/696465325131460608})
    
    \ex \label{ex:fk90}
    [DegP muy ‘very’ [Adj cualunque]]\\
    \gll Día muy cualunque en la oficina.\\
    day very \textsc{cualunque} in the office\\
    \glt ‘a very ordinary day at the office’\\
    (Twitter, 2012, \url{https://twitter.com/IvanDawidowski/status/217632663304011776})
    
    \ex \label{ex:fk91}
    [DegP re ‘very’ [Adj cualunque]]\\
    \gll un tag re cualunque \\
    a tag very \textsc{cualunque}\\
    \glt ‘a very ordinary tag’\\
        (Twitter, 2016, \url{https://twitter.com/S4NFR4NC15C0/status/748491445858934785})
    \largerpage[2]
    \ex \label{ex:fk92}
    [DegP tan ‘so’ [Adj cualunque]]\\
    \gll El día que nací fue tan cualunque que [...]\\
    the day that be.born.\textsc{pfv.1sg} be.\textsc{pfv.3sg} so \textsc{cualunque} that\\
    \glt ‘the day I was born was so ordinary that [...]’\\
    (Twitter, 2013, \url{https://twitter.com/casicasiperono/status/371041190012919808})
    
    \ex \label{ex:fk93}
    [DegP más o menos ‘more or less’ [Adj cualunque]]\\
    \gll en un restaurante más o menos cualunque\\
    in a restaurant more or less \textsc{cualunque}\\
    \glt ‘in a more or less ordinary restaurant’\\
    (Twitter, 2018, \url{https://twitter.com/tamtenenbaum/status/1048597979416944640})

    \ex \label{ex:fk94} 
    [DegP medio ‘half’ [Adj cualunque]]\\
    \gll pero ese me parece medio cualunque… o no?\\
    but that to.me seem.\textsc{prs.3sg} half \textsc{cualunque} or not\\
    \glt ‘but that one seems kind of ordinary... or does it?’\\
    (Twitter, 2014, \url{https://twitter.com/GuilleSandrini/status/442802662879137792})
\z

\appendixsection{Data reflecting  \tabref{tab:fk6} and  \tabref{tab:fk7} in  \sectref{sec:fk4}}\label{sec:fk9.2}

\begin{exe}[(102)]
\ex\label{ex:fk95}
    indicative present\\
    \gll Pero, yo no soy un esquiador cualunque.\\
    but I not be.\textsc{prs.1sg} a skier \textsc{cualunque}\\
    \glt ‘but I am no ordinary skier’\\
    (\textit{La púrpura de tiro}, 2019, \url{https://www.lapurpuradetiro.com.ar/index.php/numeros-anteriores/item/1111-lo-siento-senor-griggs})
    
    \ex \label{ex:fk96}
    indicative past perfective\\
    \gll la crisis […] se trasladó al pueblo, a los cualunques\\
    the crisis {} \textsc{refl} pass.\textsc{pfv.3sg} to.the people to the \textsc{cualunque}.\textsc{pl}\\
    \glt‘the crisis […] passed to the people, to the lower-class people’\\
    (\textit{Tiempo Argentino}, 2018, \url{https://www.tiempoar.com.ar/politica/damian-selci-el-militante-es-el-producto-mas-civilizado-que-puede-tener-una-sociedad/})

    \ex \label{ex:fk97}
    indicative past imperfective\\
    \gll mirá como estaban ahí los escribas con una llamita cualunque\\
    look.\textsc{imp} how be.\textsc{ipfv.3pl} there the.\textsc{pl} scribe.\textsc{pl} with a flame.\textsc{dim.f.sg.} \textsc{cualunque}\\
    \glt ‘look how the scribes were there with a common little flame’\\
    (Twitter, 2016, \url{https://twitter.com/vanesagiselle_/status/786411452735352832})
    
    \ex \label{ex:fk98}
    indicative future \citep[][82]{Asis2000}\\
    \gll el resto de su vida será un vendedor cualunque.\\
    the rest of his life be.\textsc{fut.3sg} a salesman \textsc{cualunque}\\
    \glt ‘the rest of his life he will be an ordinary salesman.’
    
    \ex \label{ex:fk99}
    modal verb \textit{deber} ‘must’\\
    \gll ¿Se debe comparar a un equipo cualunque con una maquinaria ideal y perfecta como es el Barcelona?\\
    \textsc{refl} should.\textsc{prs.3sg} compare to a team \textsc{cualunque} with a machine ideal and perfect as be.\textsc{prs.3sg} the Barcelona\\
    \glt ‘Should we compare an ordinary team with an ideal and perfect machine like Barcelona?’\\
    (\textit{Clarín} magazine, 2016, \url{https://www.clarin.com/opinion/mania-discutirlo_0_EkX9qo-XW.html})
    \ex \label{ex:fk100}  
    infinitive\\
    \gll Me enferma {oir [sic]} a cualunques tratar de parecer finas hablando con arrastre.\\
    me make.sick.\textsc{prs.3sg} hear.\textsc{inf} to \textsc{cualunque}.\textsc{pl} try.\textsc{inf} of seem elegant.PL speaking with affectation\\
    \glt ‘It makes me sick to hear low-class people trying to look elegant speaking in an affected way.’\\
    (Blog post, 2018, \url{http://nanopoder.blogspot.com/2008/04/cosas-odiosas.html})
    
    \ex \label{ex:fk101}
    gerund\\
    \gll Te {estas [sic]} comiendo {una [sic]} postre cualunque de chocolate simil serenito. No te hagas....\\
    you be.\textsc{prs.2sg} eat.\textsc{ger} a dessert \textsc{cualunque} of chocolate similar cool not you do.\textsc{sbjv.2sg}\\
    \glt ‘You’re eating a shitty chocolate dessert like nothing. Don’t do it…’\\
    (Twitter, 2018, \url{https://twitter.com/prestoyvoila/status/960917953674956800})
    
    \ex \label{ex:fk102}
    conditional\\
    \gll con Suarseneguer u otro sería una peli cualunque\\
    with Schwarzenegger or other be.\textsc{cond.3sg} a film \textsc{cualunque}\\
    \glt ‘with Schwarzenegger or someone else would be an ordinary movie’\\
    (Twitter, 2019, \url{https://twitter.com/DiegolBarraza/status/1177618733788995584})
\end{exe}

{\sloppy\printbibliography[heading=subbibliography,notkeyword=this]}
\end{document}
