\documentclass[output=paper,colorlinks,citecolor=brown]{langscibook}
\ChapterDOI{10.5281/zenodo.13759986}
\author{Miguel Gutiérrez Maté\orcid{}\affiliation{University of Augsburg}}

\title[Indefinite pronouns with \textsc{thing} and \textsc{person}]{Indefinite pronouns with \textsc{thing} and \textsc{person} in two Ibero\hyp Romance\slash Kikongo varieties: Palenquero Creole and Cabindan Portuguese}

\abstract{This chapter deals with the origins of generic-noun-based indefinites (according to the terminology of the WALS) in two Ibero\hyp Romance varieties, the Spanish-lexified Palenquero Creole and the heterogeneous group of postcolonial varieties of Portuguese that are spoken in the exclave of Cabinda (Angola). Both varieties have in common the fact that they have been influenced by the same “substrate”, the Bantu H language Kikongo. Both substratal influence and language universals during first- and/or second-language acquisition seem to interact in the making of indefinite expressions, as they always interact in restructuring phenomena found in language contact ecologies. However, as regards indefinite expressions, there are substantial differences between Palenquero and Cabindan Portuguese, due to the fact that the latter often retains the special indefinites from the superstrate, whereas most Spanish special indefinites have been lost in the former (with some exceptions, most especially \textit{ná} < \textit{nada}). The two varieties studied here result from different language contact ecologies, which account for quantitative and qualitative differences between the two varieties.}


\IfFileExists{../localcommands.tex}{
   \addbibresource{../localbibliography.bib}
   % add all extra packages you need to load to this file

\usepackage{tabularx,multicol}
\usepackage{url}
\urlstyle{same}

\usepackage{listings}
\lstset{basicstyle=\ttfamily,tabsize=2,breaklines=true}

\usepackage{langsci-basic}
\usepackage{langsci-optional}
\usepackage{langsci-lgr}
\usepackage{langsci-osl}
% \usepackage{./langsci/styles/langsci-lgr}
% \usepackage{./langsci/styles/langsci-osl}
% \usepackage{langsci-gb4e}

\usepackage{tikz}
\usetikzlibrary{patterns,calc}
\pgfdeclarepatternformonly{south east lines}{\pgfqpoint{-0pt}{-0pt}}{\pgfqpoint{3pt}{3pt}}{\pgfqpoint{3pt}{3pt}}{
    \pgfsetlinewidth{0.6pt}
    \pgfpathmoveto{\pgfqpoint{0pt}{3pt}}
    \pgfpathlineto{\pgfqpoint{3pt}{0pt}}
    \pgfpathmoveto{\pgfqpoint{.2pt}{-.2pt}}
    \pgfpathlineto{\pgfqpoint{-.2pt}{.2pt}}
    \pgfpathmoveto{\pgfqpoint{3.2pt}{2.8pt}}
    \pgfpathlineto{\pgfqpoint{2.8pt}{3.2pt}}
    \pgfusepath{stroke}}
    
\usepackage{stmaryrd}
\usepackage{wasysym}
\usepackage{multirow}
\usepackage{caption}
\usepackage{subcaption}
\usepackage{mathrsfs}
\usepackage{qtree}

\usepackage{linguex}


   %pminos do not split footnotes
% \interfootnotelinepenalty=10000 %Footnote in Laporte chapters has to be split SN


%\DeclareIndexNameFormat{default}{%
%\nameparts{#1}%
%\usebibmacro{index:name}%
%{\index[names]}%
%{\namepartfamily}%
%{\namepartgiveni}%
% {}% L1
% {}% L2
%{\namepartprefix}% generates spurious space L3
%{\namepartsuffix}% generates spurious space L4
%}

%  {\DeclareIndexNameFormat{default}{%
%     \usebibmacro{index:name}{\index[names]}{#1}{#3}{#5}{#7}}}

%\DeclareIndexNameFormat{default}{%
%  \usebibmacro{index:name}{\sindex[nom]}{#1}{#3}{#5}{#7}}

%\DeclareIndexNameFormat{default}{%
%  \usebibmacro{index:name}{\sindex[person]}{#1}{#3}{#5}{#7}}
%\DeclareIndexNameFormat{default}{%
%\nameparts{#1} \usebibmacro{index:name}{\sindex[person]]}{\namepartfamily}{‌​\namepartgiven}{\nam‌​epartprefix}{\namepa‌​rtsuffix}}

%\newcommand{\smiley}{:)}

%\renewbibmacro*{index:name}[5]{%
%\usebibmacro{index:entry}{#1}%
%{\iffieldundef{usera}{}{\thefield{usera}\actualoperator}\mkbibindexname{#2}{#3}{#4}{#5}}}

% \newcommand{\noop}[1]{}

%remove for final
%\overfullrule=1mm

\newcommand{\tobi}[2]}}
\renewcommand{\S}[1]{\tobi{#1}{\textsc{*}}}

% this volume references
% puts: [this volume]
% already defined: \citetv
%\newcommand{\citepv}[1]{(\citeauthor{#1} \citeyear*{#1} [this volume])}
\newcommand{\citealtv}[1]{\citeauthor{#1} \citeyear*{#1} [this volume]}

%parentheses around example number
\newcommand{\pref}[1]{(\ref{#1})}

% in-text examples

\newcommand{\lnex}[1]{\textit{#1}} %target lang word
\newcommand{\lnlit}[1]{(lit.: `#1')} %literal reading
\newcommand{\lnlat}[1]{(#1)} % latinization
\newcommand{\lntrans}[1]{`#1'} %translation
\newcommand{\lnexl}[2]%
{\lnex{#1}{} \lnlat{#2}} % ex with latinization
\newcommand{\lnexlat}[3]{\lnex{#1}{} \lnlat{#2}{} \lntrans{#3}} % ex with latinization and tranl.

%ch01
\newcommand{\co}[1]{\mbox{\textbf{#1}}}

%ch09

\newcommand{\cyrbulg}[1]{\begin{otherlanguage*}{bulgarian}#1\end{otherlanguage*}}


%ch10
\newcommand{\nlp}{{\small NLP}}
\newcommand{\mwe}{{\small MWE}}
\newcommand{\rae}{{\small RAE}}
\newcommand{\lvc}{{\small LVC}}
\newcommand{\pos}{{\small P}o{\small S}}
%\newcommand{\todo}[1]{ \textcolor{red}{#1} }

%\renewcommand{\labelenumi}{\theenumi}
%\ainamefmt{{vv}{ll}{, ff}{, jj}} % fullname

\newcommand{\biberror}[1]{{\color{red}#1}}

\newcommand{\osenovaitem}{--~}
   %% hyphenation points for line breaks
%% Normally, automatic hyphenation in LaTeX is very good
%% If a word is mis-hyphenated, add it to this file
%%
%% add information to TeX file before \begin{document} with:
%% %% hyphenation points for line breaks
%% Normally, automatic hyphenation in LaTeX is very good
%% If a word is mis-hyphenated, add it to this file
%%
%% add information to TeX file before \begin{document} with:
%% %% hyphenation points for line breaks
%% Normally, automatic hyphenation in LaTeX is very good
%% If a word is mis-hyphenated, add it to this file
%%
%% add information to TeX file before \begin{document} with:
%% \include{localhyphenation}
\hyphenation{
    Beck-man
    Ngu-yen
    back-chan-nel
    back-chan-nels
    mo-not-o-nous
    ste-reo-typ-i-cal
}

\hyphenation{
    Beck-man
    Ngu-yen
    back-chan-nel
    back-chan-nels
    mo-not-o-nous
    ste-reo-typ-i-cal
}

\hyphenation{
    Beck-man
    Ngu-yen
    back-chan-nel
    back-chan-nels
    mo-not-o-nous
    ste-reo-typ-i-cal
}

   \boolfalse{bookcompile}
   \togglepaper[23]%%chapternumber
}

\begin{document}
\maketitle

\section{Introduction}
During my fieldwork in the village of San Basilio de Palenque (Bolívar, Colombia) in the summer of 2017, a traditional speaker of Palenquero Creole (RC, 84 years old, female) without being asked started to teach me some good old-fashioned Palenquero\footnote{In today’s San Basilio de Palenque, a village where the visits of scholars and students (or just tourists) have become quite regular in the last years, it would not be impossible that some informants, when asked for translation of a given Spanish sentence into PAL or for the correction of PAL sentences, would even make up their Palenquero in order to make it lexically and/or structurally more distant from Spanish, i.e. more “exotic”. However, this was clearly not the case here: first, the fragment above is taken from the second interview I made with this speaker, after our having developed some mutual trust the day before, which turned out to be relatively easy since I was accompanied by a youngster from the village whom she knew well; second, the conversation was not dealing with linguistic issues (it was just about her life and the history of the part of the village she was living in) and she was not even asked to teach us PAL: she rather invited us to bring the conversation to a new level, after we spoke in Spanish and she spoke in PAL for a while; third, and perhaps most importantly, she was not expecting any money for the interview and she never actually asked for it (in fact she seemed to enjoy the company quite a lot).} (or just \textit{Traditional Palenquero}, as opposed to the Palenquero varieties spoken by adult heritage speakers and the learner varieties that children learn at the local school; see \cite{Lipski2020} for an in-depth exploration of the cohabitation of all these varieties in the village alongside local Spanish). She corrected my Palenquero (henceforth PAL) on some occasions, as in the following example:

\begin{exe}
\exi{RC:}
    \gll    utere tá kombesá?\\
            you(\textsc{pl}) \textsc{prog} talk\\
    \glt    ‘are you talking?’
\exi{MG:}
    \gll    bueno... un poko\\
            well... a little\\
    \glt    ‘well... a little’
\exi{RC:}
    \gll    un poko kusa!\\
            a little thing\\
    \glt    ‘a little’
\exi{MG:}
    \gll    un poko kusa\\
            a little thing\\
    \glt    ‘a little’
\end{exe}

My mistake consisted in the use of \textit{un} \textit{poko} in the exact same way as its Spanish source \textit{un} \textit{poco}, i.e. as a well-formed indefinite expression -- which, by the way, is used adverbially in this context. When transferred into PAL, however, the structure is ill-formed: Sp. \textit{conversar} \textit{un} \textit{poco} → Pal. *\textit{kombesá} \textit{un} \textit{poko}. As the correction introduced by this particular informant clearly pointed out, indefinites require an explicit nominal head in PAL: Thus, when the indefinite is to be understood as a pronominal, not modifying any noun, the head has to be occupied by the generic nouns \textit{kusa} ‘thing’ (< Sp. \textit{cosa}), for inanimate expressions (\textit{aggú} \textit{kusa} ‘something’ [lit. ‘some-thing’]), or \textit{hende} ‘people’ (< Sp. \textit{gente}), for personal expressions (\textit{aggú} \textit{hende} ‘someone\slash somebody’ [lit. ‘some people’]). In fact, an anecdote of this kind makes me think that the degree of certainty with which the feature 21 of the \textit{Atlas of Pidgin and Creole Language Structures} (APICS) (“indefinite pronouns”) is codified as generic-noun-based in PAL should be changed from “certain” (as prudently indicated by \cite{Schwegler2013}) to “very certain”. It is true, however, that, due to the long-term bilingualism PAL/Spanish in the village, we can expect special indefinites like \textit{aggo} ‘something’ (< Sp. \textit{algo}) or \textit{agguno} ‘someone’ (< Sp. \textit{alguno}) to be incorporated in some varieties of PAL, especially in those by speakers whose bilingualism tends, for some reason or another, towards Spanish.

The above anecdote is meant to give a first idea of what the following pages are about. My aim in this chapter is to examine the distribution and origins of such generic-noun-based indefinites in PAL and, in doing so, reflect on the genesis of these kinds of indefinites in Creole languages, i.e. on its role during the process of creolization. To this end, the comparison of PAL indefinite expressions with those that are found in restructured Cabindan Portuguese (henceforth, CP) -- a complex of postcolonial varieties of Portuguese spoken alongside Kikongo in the exclave and province of Cabinda (Angola) -- will reveal itself as extraordinarily helpful as a means of understanding the limits between different types of language contact varieties (which includes setting quantitative and/or qualitative limits between creolization and second language acquisition processes). This is so due to the fact that both PAL and CP came into being when speakers of Kikongo varieties\footnote{Bantu H10-16 according to Guthrie’s most famous classification of Bantu languages \citep{Guthrie1967}, \textit{Kikongo Language Cluster} according to \citet{Bostoen2012} and \citet{BostoenDeSchryver2015}.}  somewhat “approximated” an Ibero\hyp Romance language (Spanish in the case of PAL, Portuguese in the case of CP), which led to more or less intense restructuring of Ibero\hyp Romance: see \citet{Holm2004} about \textit{partial restructuring} and \citet{HolmLorenzinoDeMello1999} about different degrees of restructuring, where Creoles are considered to be one end of the “restructuring continuum”. 

The term \textit{language approximations} \citep{Chaudenson2003} is often preferred to \textit{learner varieties}, since it would not be accurate to state that Creoles (like PAL) resulted from canonical language learning: First generations of Creole speakers did not really try to \textit{learn} another language in the modern sense of the term, but they just wanted to be able to communicate with other speakers with whom they did not share, at first, a common language. In doing so, it became clear, however, that speakers had – partially, at least – to give up their own first language in their effort to adapt to the primitive Creole societies (plantations with slaves from different African regions, maroon communities formed out of the reunion of bozals -- native speakers of African languages -- and black Creoles -- native speakers of regional varieties of European languages, etc.); it is only in the latter sense that we can relate creolization to second-language (L2) acquisition. 

If we extend the model of \citet{Michaelis2017} -- originally designed for the classification of Creole languages according to their lexifiers and substrates -- to any other vernacular resulting from a language contact situation, we should classify both PAL and CP as Ibero\hyp Romance\slash Bantu or, more precisely, Ibero\hyp Romance\slash Kikongo. Moreover, if we wanted to highlight the role of a particular Kikongo dialect in the formation of PAL we could possibly speak of a Spanish\slash Kiyombe Creole, since it is surely Kiyombe -- the autoctonous variety of the Mayombe forest, from which many Black slaves were taken and brought to the New World~-- that constitutes PAL’s main substrate (s. \cite{Moñino2017}, \cite{Schwegler2016a}, \citeyear{Schwegler2017}, \cite{GutiérrezMaté2020} and references therein).\footnote{It has to be noted that \citet[24--25]{Moñino2017} states that Vili (spoken along the Loango coast) was the language of the regional slave trade during the 17\textsuperscript{th} century and could therefore have become “the primary base of the Congo substrate of Palenquero”. Speakers of Kiyombe and other varieties could have easily learned Vili as an L2 or just have adapted to it (i.e. Vili might have served as the basis for a Kikongo koine). The hypothesis is suggestive, but -- as Moñino himself acknowledges -- we only have evidence of the genetic match between the inhabitants of Palenque and those from the Mayombe forest (no match between Palenqueros and the people from the Loango coast has been shown to date). Consequently, I assume that the Yombe people were the most important group in the primitive Palenquero society and that (L1-)Kiyombe had as good “adaptive” chances as (L2-)Vili in the New World setting. In reality, however, we can trace the origin of some PAL features back to Kiyombe and the origin of some other features back to several westernmost varieties of Kikongo (including Vili) \citep{GutMat}. When it comes to CP, the utilmate reason for focusing on Kiyombe-Portuguese bilinguals in my study is the fact that there are not so many Vili-Portuguese bilinguals in Cabinda, since Vili is only spoken in a small region bordering the Republic of the Congo.} To ensure comparability, unless indicated otherwise, the Cabindan examples selected for this chapter have been produced by Portuguese/Kiyombe bilinguals, who I interviewed in the Cabindan Mayombe (municípios of Buco Zau and Belize). However, I do not expect there to be substantial differences between the Kikongo varieties (certainly not between the Cabindan varieties of Kikongo) as regards the particular phenomenon analyzed throughout this chapter.

The orientation of my work is mainly typological, with focus on the emergence of new languages/varieties out of the contact between languages belonging to different types. My understanding of “generic-noun-based” indefinites follows the tradition of the \textit{World Atlas of Language Structures Online} (WALS, \citealt{Haspelmath2013}), which, for its part, follows the work of \citet{Haspelmath1997}. According to this author, indefinite pronouns are often derived from “generic ontological\hyp category nouns” such as \textsc{thing}, \textsc{person}, \textsc{place}, \textsc{manner}, etc.  \citep[26, 52]{Haspelmath1997}. In many cases, the pronominal status of such expressions is controversial, and a certain degree of idealism is needed:

\begin{quote}
for most languages with generic-noun-based indefinites, there is no good evidence available that these expressions are different from ordinary indefinite noun phrases. In fact, some descriptions explicitly deny that they are indefinite pronouns. […] In this chapter, such cases where we lack evidence for pronominal status have been lumped together with languages like Italian and English because it would have been very difficult to draw a line between the two types. The evidence for pronominal status can be rather subtle. For instance, French \textit{quelque chose}  ‘something’ at first glance looks like a regular indefinite noun phrase consisting of the indefinite determiner \textit{quelque} and the noun \textit{chose} ‘thing’. However, the notion ‘something good’ is expressed with a construction that is reserved for pronouns: \textit{quelque chose de bon} (cf. \textit{quoi de bon}  ‘what good’), whereas an ordinary noun phrase would be \textit{quelque bonne chose}  (‘some good thing’). \citep[§2.2]{Haspelmath2013}
\end{quote}

\begin{sloppypar}
At the very least, we could state that a given language exhibits “generic\hyp noun\hyp based indefinites” when it uses NPs with ontological\hyp category nouns in the phrasal head to convey the meaning of “what other languages express by means of indefinite pronouns” \citep[28]{Haspelmath1997}. We could assume that, prototypically, languages classified as having generic-noun-based indefinites should only exhibit generic-noun-based pronominals; however, it may be convenient to include other languages in the same group even if they have generic-noun-based indefinite expressions that 1) are not clearly grammaticalized (nor lexicalized) as such pronouns and/or 2) have not completely replaced the paradigm of special indefinites. 
\end{sloppypar}

Let us consider, for example, the case of European Portuguese: in this language, the most common indefinite expression meaning ‘something’ is \textit{alguma coisa} (‘some thing’), which is obviously generic-noun-based; however, unlike English \textit{something}, it does not actually seem to have undergone pronominalization, nor has it completely replaced the special indefinite \textit{algo}. Despite all this, the fact that an NP projected by a general ontological-category noun has become the indefinite expression \textit{par excellence} for conveying the meaning of ‘something’ should be factored into our typological characterization of Portuguese. This is also the reason why the WALS eventually classifies Portuguese as a language having “mixed indefinites” \citep[§2.4]{Haspelmath2013}; the latter means that, although Portuguese makes use of a special indefinite for the ‘someone’ word (\textit{alguém}), it does frequently exhibit a generic-noun-based expression (\textit{alguma coisa}) for the meaning of ‘something’. 

In other words, it is not only the categorical use of a given linguistic strategy, but also its productivity, that matters for linguistic typology. In fact, throughout my work, most especially as regards CP, I will be dealing with generic-noun-based indefinite expressions of which all I can say with certainty is that they are highly productive (cases like the above, in which a native speaker explicitly rejects an alternative strategy, are generally not found). 

The structure of this chapter is as follows: In \sectref{sec:gut2}, the PAL noun-based indefinites exemplified in this introduction will be presented in more detail, including a brief description of other uses of generic nouns, in order to have a general idea of their semantic heterogeneity and the ease with which they can undergo semantic bleaching and pronominalization. \sectref{sec:gut3} is about investigating the role of three aspects – which are all essential parts of any Creole – in the formation of indefinites in PAL: (a) linguistic universals during first and/or second language acquisition, (b) further development of tendencies that already existed in the lexifier (or superstrate) and (c) substratal influence. To further illustrate the latter (which I consider to be conditio sine qua non in the process of change analyzed here), indefinite pronominal expressions in CP will be examined in \sectref{sec:gut4}. The chapter ends with a summary of the main results and their contextualization within the framework of a wider research program (\sectref{sec:gut5}). 

The data of CP and Kikongo were collected \textit{in situ} as a result of my fieldwork in Cabinda in March/April 2019 and February/March 2020, whereas the data from PAL result from combining my own data (from fieldwork made in 2017) with those from the corpus published by \citet{FriedemannRosselli1983} and \citet{MagliaMoñino2015} and those from the first interviews made by A. Schwegler in San Basilio de Palenque (1985--1988).\footnote{A corpus of recordings I have been transcribing, digitizing, and analyzing since 2014 (cf. \cite{GutiérrezMaté2017}, \citeyear{GutiérrezMaté2020}).} The corresponding source is indicated after each PAL example.

\section{The Palenquero data}\label{sec:gut2}
\subsection{\textit{Kusa} (‘thing’) and \textit{hende} (‘people’) in indefinite expressions}\label{sec:gut2.1}

As stated above, indefinites in PAL are mainly built by using generic nouns: this is not only the case for indefinites meaning ‘something’ and ‘someone’, which are those that both the WALS and the APICS take into account for their typological classifications, but also for indefinites meaning ‘everything’, ‘a lot’, ‘a few’, etc. (all of them having in common the fact that they are quantificational expressions, be they “universal” or not, “evaluative” or not, etc.: \cite[501--502]{BosqueGutiérrez-Rexach2009}). When the indefinite is positive and non-personal, a determiner is always expressed and placed prenominally. A few examples of different types of generic-noun-based indefinites will suffice to demonstrate this trend in PAL. Particularly, as for the expression of ‘something’, PAL exhibits two forms, since the selection of the determiner seems to almost freely alternate between \textit{un} and \textit{aggú} (we cannot disregard the possibility that each form has different semantic nuances or is even conditioned by different syntactic constraints, but the available data do not allow any conclusions to be drawn yet):

\ea\textit{aggú/un kusa} ‘something’ (= Sp. \textit{algo}):
\ea\label{ex:gut1} \citep[48--33]{Schwegler2013}\\
\gll Bo a komblá aggú kusa? \\
    \textsc{2p.sg} \textsc{cpl} buy some thing \\
\glt ‘Did you buy something?’  (Sp. \textit{¿Compraste algo?})
\ex \label{ex:gut2}\citep[261]{MagliaMoñino2015}\\
\gll <antonse vamo a hacé> un kusa pa nu José akkansá suto \\
    <thus we-are-going to do> a thing for \textsc{neg} José reach us \\
\glt ‘then let’s do something so that José cannot reach us’\\(Sp. \textit{Entonces vamos a hacer algo para que José no nos alcance})
\z
\ex \textit{to kusa} ‘everything’ (= Sp. \textit{todo}):
\ea\label{ex:gut3}(Recordings by Armin Schwegler 1985--1988)\\
\gll Aora jue la mora tá ke hende a ten ke asé to kusa pa moná chikito \\
     Today \textsc{fp} the way be that people \textsc{cpl} have to do all thing for child small \\
\glt ‘Nowadays, there's a trend that people have to do everything for children’ (Sp. \textit{Ahora está la moda de que la gente tiene que hacer todo para los niños})
\ex\label{ex:gut4}(Recordings by Armin Schwegler 1985--1988)\\
\gll y’ase-[b]a hundá-lo to kusa \\
    1\textsc{p.sg. hab-imp} gather-3\textsc{p.obj} all thing \\
\glt ‘I used to gather everything together’ (Sp. \textit{solía juntarlo todo})
\z
\ex \textit{mucho kusa} ‘much\slash a lot’ (= Sp. \textit{mucho}):
\ea\label{ex:gut5}(Recordings by A. Schwegler 1985--1988)\\
\gll bo <biene> má pokke bo etá yebá mucho kusa \\
2\textsc{p.sg} <come> more because 2\textsc{p.sg} \textsc{prog} bring much thing \\
\glt ‘you will come back more often, because you are gaining (/learning) a lot from here’ (Sp. \textit{Vendrás más porque te estás llevando}[/\textit{estás aprendiendo}] \textit{mucho} [\textit{de aquí}]) 
\ex\label{ex:gut6}\citep[215]{FriedemannRosselli1983}\\
\gll kumo suto ten kampo nu, akí ta pelé mucho kusa \\
due.to 1\textsc{p.pl} have field \textsc{neg}, here \textsc{prog} lose much thing \\
\glt ‘since we don’t have fields, a lot is being lost here’ (Sp. \textit{Como no tenemos campos, aquí se está perdiendo mucho})
\z
\z

A different case is represented by the use of \textit{kusa} in negative polarity contexts. In these contexts, \textit{kusa} is generally employed without any indefinite determiner: 

\ea\label{ex:gut7}\citep[214]{MagliaMoñino2015}\\
\gll aora ma hende ta ase-ndo ele {<en la noche>} [...] pa nu hende nu ndá kusa\footnotemark\\ 
now \textsc{pl} people \textsc{prog} do-\textsc{ger} 3\textsc{p.obj} {[at night]} {} for \textsc{neg} people \textsc{neg} give thing \\
\glt ‘now, people are doing it at night [...] so as not to give anything’ \\
(Sp. \textit{Ahora la gente está haciéndolo de noche [casarse] para no dar nada}) \\ 
\footnotetext{Final sentences are quite exceptional in PAL, since they use only one negator, which is placed before the subject and immediately after \textit{pa} (< Sp. \textit{para}), as in the example (\ref{ex:gut2}) above. Consequently, in (\ref{ex:gut7}), we would have expected \textit{pa nu hende ndá kusa}. All other sentence types, including matrix sentences, have three possibilities to convey negation: with a preverbal negator, which is always placed \textit{after} the subject (\textsc{neg\oldstylenums{1}}), with both a preverbal and a sentence-final negator (\textsc{neg\oldstylenums{2}}) and with a sentence-final negator (\textsc{neg\oldstylenums{3}}) \citep{Schwegler2016b}. The additional use of the negator \textit{nu} between the subject and the verb might therefore have been influenced by two of the three regular negation types (\textsc{neg\oldstylenums{1}} and \textsc{neg\oldstylenums{2}}). On the other hand, the speaker might just be producing a spontaneous mixture of a Palenquero final sentence and a canonical Spanish sentence (with \textsc{neg\oldstylenums{1}}). Since final sentences have mostly remained unexplored in the literature on PAL negation, it is hard to tell whether this particular type of “preverbal double negation” is more common or even regular in some idiolects. From a purely structural viewpoint, I guess that scholars working on the generativist framework would treat this example as one of those exceptional cases in which a moved element (here, the negator) receives a phonological representation in the different structural positions that it has during the derivation: see the analysis of sentences like \textit{Wen meinst du wen Peter gewählt hat?} in German dialects by \citet[114]{GabrielMüllerFischer2018}.}
\ex\label{ex:gut8} \citep[242]{MagliaMoñino2015}\\
\gll Suto polé-ba asé kusa malo ante nu \\
1\textsc{p.pl} can-\textsc{imp} do thing bad before \textsc{neg} \\
\glt ‘we could not do anything wrong before.’ (Sp. \textit{No podíamos hacer nada malo antes)}
\z

It has to be noted that PAL also makes use of the special negative indefinite \textit{naa} or \textit{ná} (< Sp. \textit{nada}). Unlike other Spanish-like special indefinites, \textit{ná} seems to have been fully incorporated in the Creole \citep[234]{Schwegler2016b} and traditional speakers use it regularly (in some contexts, alongside the generic\hyp noun\hyp based form). The factors accounting for the variable of negative non\hyp personal indefinites (“\textit{ná} vs. \textit{kusa}”) in negative polarity contexts are yet to be determined.  

As regards \textit{hende}, we find a similar distribution to the one observed in the case of \textit{kusa}. The determiner decides the type of quantifying reading (in this particular example, we have the universal one: \textit{to hende} = ‘everyone’):

\ea\label{ex:gut9}(Fieldwork M. Gutiérrez Maté 2017)\\
\gll ma hende <preguntando> pu[sic] to hende: Raú kiene jue? \\
\textsc{pl} people asking for all people: Raúl who is? \\
\glt ‘people are asking everyone: “Who is Raúl?”’ \\
(Sp. \textit{La gente }[\textit{anda}] \textit{preguntando a todos}/\textit{todo el mundo: “Raúl quién es?”})
\z

Special indefinites like \textit{nadie} \textasciitilde\ \textit{narie} (< Sp. \textit{nadie}) and \textit{aggie} (< Sp. \textit{alguien}) are nearly non-existent in the traditional PAL varieties analyzed here. However, we have to deal with the fact that special indefinites with a partitive reading seem to be fully integrated in the Creole: \textit{aggú(n)} (< Sp. \textit{algún}), \textit{agguno} (< Sp. \textit{alguno}) and \textit{ninguno} (< Sp. \textit{ninguno}). The latter can also be used as an inherently negative word with the meaning of ‘nobody’\slash ‘no one’.\footnote{Let us consider the following example: 

\ea (Recordings by A. Schwegler, 1985--1988)\\
\gll ninguno sa[b]é ké kusa é bitibite nu \\
 no.one know what thing \textsc{cop} bitebite \textsc{neg} \\
\glt ‘no one knows what bitebite [a traditional food] is’
\z}

In addition, we have to deal with the peculiarity that, in some cases, we cannot decide whether we have an indefinite pronoun or a generic/arbitrary expression (similar to the Spanish source \textit{gente}). This is also a hint about the blurred limits between indefinite and generic expressions in many contexts (cf. Sp. \textit{No ha venido gente}\slash \textit{No ha venido nadie}) (I will come back to these issues in \sectref{sec:gut2.2}). The examples illustrate the use of \textit{hende} as a negative indefinite (without determiner): 

\ea\label{ex:gut10}\citep[287]{MagliaMoñino2015}\\
\gll Uto pueblo bo miná hende asina nu \\ 
other village 2\textsc{p.sg} look someone/people like.this \textsc{neg} \\
\glt ‘In other villages you don’t see anyone/people like that’ \\
(Sp. \textit{En otros pueblos no ves nadie así/gente así})
\ex\label{ex:gut11}\citep[288]{MagliaMoñino2015}\\
\gll Aki Palenge a ten kumina po lendro monte [...] ke ma hende kelé-lo nu \\
here Palenque \textsc{cpl} \textsc{exist} food over inside mountain {} that \textsc{pl} people want-it \textsc{neg} \\ 
\glt ‘Here in Palenque there is food in the forest that [...] nobody wants to [go and get]’
(Sp. \textit{Aquí en Palenque hay comida en el monte que nadie quiere} [\textit{ir a buscar}]) 
\z

As (\ref{ex:gut11}) shows, the pluralizer \textit{ma} often precedes \textit{hende}: actually, \textit{hende} and \textit{ma hende} alternate quite freely, a fact that might be related to the history and meaning of \textit{ma}. The source for this PAL item is the noun class 6 prefix in Kikongo (\cite{Schwegler2007}, \cite{Moñino2013}); in this language (as well as in Proto-Bantu), \textit{ma-} is a productive plural prefix\footnote{Not only is \textit{ma}- the plural of class 5 (\textit{ditoko}/\textit{matoko} ‘boy/s’) but also the plural of classes 14 (\textit{bwala}/\textit{maala} ‘village/s’) and 15 (\textit{kulu}/\textit{malu} ‘leg/s’).} and also the prefix used for liquids, masses and collectives (see \citealt[108--109]{Chicuna2018} for Kiyombe and \cite[51]{Maho1999} for Proto-Bantu). Since PAL does not distinguish noun classes morphologically, \textit{ma} has become the only pluralizer in the Creole, or, better said, an optional pluralizer, due to the fact that plural can also be interpreted contextually, with no need of marking it morphologically  \citep[42--43]{Moñino2013}. Interestingly, in PAL \textit{ma-} has been lost in mass nouns (\textit{agua} ‘water’, \textit{asuka} ‘sugar’, etc.) \citep[56--57]{Moñino2013}, but it can still be used for collectives: for instance, \textit{ma ngombe} (where \textit{ngombe} ‘cow’ is another Bantuism in PAL) can sometimes be better translated as ‘cattle’ than ‘cows’. Consequently, the alternation “\textit{hende} vs. \textit{ma hende}” can be seen as the result of two different issues that imply opposing tendencies in PAL: on the one hand, \textit{ma-} is mostly specialized as a plural marker but it can also retain other etymological meanings (including the collective); on the other hand, speakers tend to use \textit{ma} for conveying the aforementioned meanings but, if these can also be understood contextually, there is no need to use \textit{ma} at all.

\tabref{tab:gutmat1} sums up the indefinite expressions that are found in PAL.

\begin{table}
\small
\begin{tabularx}{\textwidth}{llXX}
\lsptoprule
{[$-$personal]}& {[affirmative]} & \textit{un\~{}aggú kusa} & ‘something’\\
 &  & \textit{to kusa} & ‘everything’\\
 &  & \textit{mucho kusa} & ‘a lot’\\
 &  & \textit{un poko kusa} & ‘a few, a little’\\
  \tablevspace
 & {[negative]} & \textit{ná} & ‘nothing’\\
 &  & (\textsc{neg}+) \textit{kusa} \newline \~{}(\textsc{neg}+) \textit{ná} (less frequent) & ‘(not...) anything’\\
 \tablevspace
 {[+personal]}& {[affirmative]} & \textit{un\~{}aggú hende} & ‘somebody\slash someone’\\
 &  & \textit{to hende} & ‘everyone’\\
 &  & \textit{mucho hende} & ‘a lot of people’\\
 &  & \textit{un poko hende} & ‘a few people’\\
  \tablevspace
 & {[negative]} & \textit{ninguno} & ‘nobody\slash no one’\\
 &  & (\textsc{neg}+) \textit{hende}\~{}\textit{ma hende} & ‘no one\slash (not...) anyone’\\
\lspbottomrule
\end{tabularx}
\caption{Generic-noun-based indefinites in PAL}
\label{tab:gutmat1}
\end{table}

\subsection{Other uses of \textit{kusa} and \textit{hende}}\label{sec:gut2.2}
Even though the nouns \textit{kusa} ‘thing’ and \textit{hende} ‘person/people’ are frequently used as indefinite pronominals in PAL, we have to acknowledge the fact that they have a wide spectrum of uses, some of which are more lexical and more prototypically nominal than others. In my view, one can easily admit that changes of the type ‘a/some thing’ > ‘something’ involve grammaticalization (as perhaps in the case of \textit{un}/\textit{aggú kusa} and similar indefinite expressions); as for those cases in which the generic noun has no determiner in negative contexts (as when \textit{kusa} means ‘anything’), we could analyze them as a particular case of semantic bleaching, i.e. in a similar fashion as the previous literature has analyzed the change Lat. REM > Old French \textit{rien(s)} > French \textit{rien} \citep[s.][364]{Roberts2012}. As in any other linguistic change, the innovative meaning did not suddenly replace (and may have never completely replaced) the traditional one \citep{Hopper1991}. In this section, I give a brief account of the uses in which \textit{kusa} and \textit{hende} are not to be understood as indefinite pronominals (which is not to say that all these uses are necessarily conservative).  
To begin with, \textit{kusa} is prototypically nominal when it refers to (specified or unspecified) material things (as in (\ref{ex:gut12}), but not in (\ref{ex:gut13})): 

\ea\label{ex:gut12}\citep[247]{MagliaMoñino2015}\\
\gll ma loke nu ten moná ju’ i ta jutá kusa akí \\
\textsc{art} \textsc{rel} \textsc{neg} have child \textsc{cop} \textsc{rel} \textsc{prog} steal thing here \\
\glt ‘those who have no children are those who are stealing things here’ \\
(Sp. \textit{Los que no tienen hijos son los que están robando cosas aquí}) 
\ex\label{ex:gut13}(Recordings by A. Schwegler 1985--1988)\\
\gll entonse kusa a ñamá mí <la atensión> pokke y’ a mina kúmo to ma pueblo <tenía[n]> karretera ané \\
thus thing \textsc{cpl} call 1\textsc{p.sg} the attention because 1\textsc{p.sg} \textsc{cpl} see how all \textsc{pl} village had road 3\textsc{pl.poss} \\
\glt ‘So one thing caught my attention because I saw that all the villages [around here] [already] had their roads’ (about cars getting stuck at the entrance of the unpaved road leading to San Basilio de Palenque) \\
(Sp. \textit{Entonces eso }[\textit{este hecho}] \textit{me llamó la atención porque yo vi cómo todos los pueblos tenían ya sus carreteras}) 
\z

In a similar vein, \textit{hende} is a canonical noun when it refers to some (specified or unspecified) person. As a matter of fact, the Spanish noun \textit{persona} did not really get into PAL, so \textit{hende} (< \textit{gente}) unites both the meaning of the Spanish generic noun \textit{gente} ‘people’ and the more concrete \textit{persona} ‘person’:\footnote{As one of the reviewers pointed out, Spanish speakers often use sentences in which the distinction generic/concrete is blurred: for example, \textit{Juan es muy buena gente} (with the literal meaning ‘Juan is very good people’ and the actual meaning ‘Juan is such a nice guy’). }

\ea\label{ex:gut14}(Fieldwork M. Gutiérrez Maté 2017)\\
\gll to e[se] ma hende i a konosé-lo nu \\
all that \textsc{pl} people 1\textsc{p.sg} \textsc{cpl} know-3\textsc{p.obj} \textsc{neg} \\
\glt ‘all those people [you have just mentioned] I do not know’ \\
(Sp. \textit{A toda esa gente }[/\textit{a todas esas personas}] \textit{no las conozco})
\ex\label{ex:gut15}(Recordings by A. Schwegler 1985--1988)\\
\gll entonse a tene-ba ndo hende nu-má’kí \\
then \textsc{cpl} \textsc{exist}.have-\textsc{imp} two people no-more’here \\
\glt ‘back then there were just two people here’ \\
(Sp. \textit{Entonces solo había dos personas aquí})
\z

Many uses of both \textit{kusa} and \textit{hende} (like those preceded by the indefinite determiner) seem to already find themselves on the limit between nominal and pronominal expressions: ‘a (given) thing\slash something’, ‘a (given) person\slash someone’ (the same is actually true for noun phrases with ‘thing’ and ‘person’ in Ibero\hyp Romance -- and many other languages -- when they have a non-specific reading, i.e. when the potential referents are interchangeable):

\ea\label{ex:gut16}(Recordings by A. Schwegler 1985--1988)\\
\gll pero aora bo temé di betí un kusa aí kueppo sí pokke bo polé biti-lo nu \\
but now 2\textsc{p.sg} be.afraid of wear a thing there body 2\textsc{p.sg.poss} because 2\textsc{p.sg} can wear-3\textsc{p.obj} \textsc{neg} \\
\glt ‘but now you are afraid to put something on your body, because you cannot wear it’ \\
(Sp. \textit{Pero ahora tienes miedo de ponerte una}/\textit{cualquier cosa}[/\textit{algo}] \textit{en tu cuerpo, porque no puedes vestirlo}) 
\ex \label{ex:gut17}\citep[209]{MagliaMoñino2015}\\
\gll si un hende andi mitá kaya asé-ba hablá suto-ba [...] \\
\textsc{comp}(\textsc{cond}) a person where middle street \textsc{hab-imp} tell 1\textsc{p.pl-imp} \\
\glt ‘if a person(/someone) in the middle of the street spoke to us, […]’ \\
(Sp. \textit{Si alguien en medio de la calle nos hablaba, }[...]) 
\z

As regards \textit{kusa}, we can easily observe other secondary uses, which can all be considered to be derived from the non-material meaning of the generic noun (see \ref{ex:gut13}). These other uses of \textsc{thing} can be very diverse cross-linguistically and adopt different discursive and informational values (I am thinking, for example, of uses like Eng. \textit{Thing is...}, German \textit{Hauptsache...}, etc., which highlight the utterance they are introducing and somehow contrast it with what has been previously said). It is not strange that such uses acquire connective properties which can be equally diverse. For instance, in (\ref{ex:gut18}) it can function as an inter-sentential connector (comparative or consecutive: ‘so that, in such a way’): 

\ea\label{ex:gut18}\citep[203]{FriedemannRosselli1983}\\
\gll kuando bo kabá ese punchera, bo a rregresá, gobbí yená punchera má pa gobbí salí, kusa kuando Tito ke paresé ri á Katajena, suto a tá lito \\
when 2\textsc{p.sg} finish that basin 2\textsc{p.sg} \textsc{cpl} come.back come fill basin more for again leave so.that when Tito \textsc{virt} appear from there Cartagena 1\textsc{p.pl} \textsc{comp} be ready \\
\glt ‘when you are finished with this bowl, you return and fill it again in order for you to (be able to) leave, so that when Tito comes back from there, from Cartagena, we are ready (to leave)’ \\
(Sp. \textit{Cuando uno ha acabado esa “ponchera”, uno regresa y vuelve a llenar la “ponchera” otra vez. Para volver a salir, de manera que cuando Tito aparezca de allá de Cartagena, nosotras estaremos listas})
\z

Besides the change of \textit{kusa} from a generic noun to an indefinite pronoun we also find the change to an interrogative pronoun (introduced by \textit{ké} < Sp. \textit{qué}): in other words, \textit{ké kusa} (lit. ‘what thing’) is a common variant of \textit{ké} (‘what’) (actually, what we usually find in these cases is the interrogative pronoun followed by the focus particle \textit{jue} [hwe]: \textit{ké (kusa) jue bo tá asé?} ‘what are you doing?’; cf. \cite{GutiérrezMaté2017}). However, generic-noun-based interrogative pronouns with the meaning of ‘who?’ do not seem to be allowed in PAL: in other words, the variant \textit{ké hende}? (‘what people’) has never been found, neither in my corpus nor in the other PAL corpora, and it is only the special interrogative \textit{kiene} (< Sp. \textit{quién}) that can be used for this meaning.

As for other uses of \textit{hende}, it can also expand semantically and adopt other meanings, including that of a \textit{generic} or \textit{arbitrary} subject pronoun (the latter being different from the former insofar as its reference explicitly excludes the speaker: cf. \cite[63--64]{Holmberg2009}): going one step further, \textit{hende} can even function as a sort of 1\textsc{p.pl} pronoun (cf. \cite{Schwegler1993}, \citeyear{Schwegler2002}). In (\ref{ex:gut19}) \textit{hende} is used as something between a generic/arbitrary noun and a 1\textsc{p.pl} pronoun, whereas in (\ref{ex:gut20}) it is clearly used as a 1\textsc{p.pl} possessive (possessives in PAL consist of independent personal pronouns placed in a post-nominal position). Consequently, the type of linguistic change we are dealing with parallels the one that has been taking place in Brazilian Portuguese from the 19\textsuperscript{th} century onwards \citep{Lopes2003}~-- the grammaticalization degree of \textit{gente} in Brazilian Portuguese\footnote{Even in some central and southern varieties of European Portuguese, as one of the reviewers pointed out.} being much more advanced than that of \textit{hende} in PAL:\largerpage[-1]

\ea\label{ex:gut19}(Recordings by A. Schwegler 1985--1988)\\
\gll asina jue hende asé abla-lo-ba akí… asina jue-ba \\
so \textsc{cop/fp} people \textsc{hab} say-3\textsc{p.acc-imp} here so \textsc{cop/fp-imp} \\
\glt ‘that's how people used to call it (/how it was called) here… so it was’ \\
(Sp. \textit{Así es como la gente solía} [/\textit{nosotros solíamos}] \textit{llamarlo aquí... así era})
\pagebreak
\ex\label{ex:gut20}(Recordings by A. Schwegler 1985--1988)\\
\gll bo etá bibí <con> un mujé asina kumo koló hende? \\
2\textsc{p.sg} \textsc{prog} live <with> \textsc{ind.art} woman like.this as colour 1\textsc{p.pl.poss} \\
\glt ‘are you living with a woman like that, who has the same color as us [=with a black woman]?’ \\
(Sp. \textit{¿Estás viviendo con una mujer así, que tiene el mismo color que nosotros?})
\z

All the above data is interesting because it shows that there are different processes of grammaticalization with regard to generic nouns taking place at the same time; in addition, this line of reasoning leaves the door open for the possibility that the change “generic noun > indefinite expression” (as any other linguistic change involving generic nouns) has occurred within the diachrony of PAL itself. According to such explanation, we would say, in traditional terms, that generic-noun-based indefinites result from an “internal” linguistic change. Be that as it may, there are other uses of \textit{kusa} and \textit{hende} in PAL that seem to be formally and/or semantically related to their use in indefinite expressions, even though we cannot know whether such a relation resulted from a chain of internal changes in PAL internal linguistic history or took place more or less at the beginning (during creolization). I particularly wish to highlight the structural resemblance of the generic-based-noun indefinites to those generic nouns that appear in phrases in which, in the lexifier language, there would be no overt generic noun but only an adjective, a free (or \textit{headless}) relative clause, etc. In other words, we expect many more nominal heads (preceded by determiner or not) to be filled up with a generic noun in PAL than in Spanish (this kind of construction is possible in PAL, but the use of \textit{kusa} and \textit{hende} in this context seems to be far more frequent than the use of, respectively, \textit{cosa} and \textit{gente} in Spanish in the same structural contexts):\largerpage[-1]

\ea\label{ex:gut21}\citep[247]{MagliaMoñino2015}\\
\gll Ndá ri kuenda un kusa lok’i tan ablá bo aora \\
Give of notice one thing \textsc{rel}-1\textsc{p.sg(cl)} \textsc{fut} tell 2\textsc{p.sg} now \\
\newline ?Ndá ri kuenda ∅ lok’i tan... \\
\glt‘Realize one thing that I am going to tell you now’\slash ‘Realize what I am going to tell you’ \\
(Sp. \textit{Date cuenta de lo que te voy a decir ahora})
\pagebreak
\ex\label{ex:gut22}(Fieldwork M. Gutiérrez Maté 2017)\\
\gll Ele é prieto... [ele é] hende... hende kumo suto \\
3\textsc{p.sg} \textsc{cop} black [3\textsc{p.sg} \textsc{cop}] people people like 1\textsc{p.pl} \\
\newline ?Ele é ∅ kumo suto \\
\glt ‘He is black… he is like us’ (an old woman speaking about Armin Schwegler, who is well known in the village) \\
(Sp. \textit{Él es negro} [\textit{en realidad}]\textit{, es como nosotros})
\z

Finally, some combinations of generic noun and adjectives seem to even have lexicalized, as when \textit{kusa} is modified by \textit{di belá belá} (lit. ‘of true true’) (\textit{kusa di belá belá} = ‘the actual/real truth) or when \textit{hende} is modified by \textit{ngande} ‘big’ (\textit{hende ngande} ‘adult(s)’):\footnote{\textit{Di belá belá} can be combined with other nouns (f.i., \textit{amigo di belá belá} ‘a true friend’: cf. \cite[][xvii]{Schwegler1996}), but \textit{kusa di belá belá} seems to be the lexical expression of ‘(real) truth’ (more than a simple truth, since in the latter case \textit{belá} would surely have been enough).}\largerpage[2]

\ea\label{ex:gut23}(Recordings by A. Schwegler 1985--1988)\\
\gll yo ju’i {te\footnotemark} ablá té belá, yo ju’i te ablá té kusa di belá belá \\
1\textsc{p.sg} \textsc{fp}-1\textsc{p.sg(cl)} \textsc{prog} tell 2\textsc{p.sg} truth, 1\textsc{p.sg} \textsc{fp}-1\textsc{p.sg(cl)} \textsc{prog} tell 2\textsc{p.sg} thing of truth truth \\
\glt ‘I am the one who is telling you the truth, I am the [only] one who is telling you the actual truth’ \\
(Sp. \textit{Yo soy el que te dice la verdad, yo soy el que realmente te está diciendo la verdad})
\z
\footnotetext{According to the context, the best way of interpreting this sentence is with a progressive verb tense. Such interpretation makes us think that preverbal \textit{te} is a variant of the TMA marker \textit{ta} (progressive), even though I am not sure whether I have even found such a variant before (neither in Schwegler’s oldest recordings nor in any other corpus). The form might result from (regressive) assimilation to the front vowel of postverbal 2\textsc{p} object \textit{té} (< Sp. \textit{usted}). Another possible interpretation would consist in thinking of \textit{te} as the result of joining the focus particle \textit{é} to the TMA \textit{tá} (tá + é = té), but, in PAL, the focus particle cannot be placed between the TMA and the verb (in Colombian Spanish, however, the sequence “auxiliary verb + focus particle(/“focalizing ser”) + main verb” is very common: \textit{estoy es diciéndote la verdad}) \citep[18]{GutiérrezMaté2017}. Finally we could consider \textit{te} to be the Spanish 2\textsc{p} clitic pronoun, inserted in a sentence that, for the most part, is constructed in PAL. According to this reading, the analysis would be:
\ea
\gll yo ju’i <te> ablá té belá, yo ju’i <te> ablá té kusa di belá belá \\  1\textsc{p.sg} \textsc{fp-1p.sg(cl)} \textsc{<2p.cl>}-\textsc{prog} tell \textsc{2p.sg} truth, 1\textsc{p.sg} \textsc{fp-1p.sg(cl)} \textsc{<2p.cl>} tell \textsc{2p.sg} thing of truth truth \\
\glt ‘I am the one who tells you the truth, I am the one who tells you the actual truth’
\z

However, there are three facts that make the latter interpretation rather unlikely. Firstly, there is no actual reason why the 2\textsc{p} object should be doubled (being expressed once in Spanish and once in PAL), especially in a (cleft) sentence that already has a focalized element (the subject \textit{yo}) and even seems to introduce a secondary focus at the end (precisely, by means of \textit{di belá belá}); in other words, there is no room for another sentence focus -- the doubling of the 2\textsc{p} pronoun being the formal consequence of such additional focalization -- i.e. a reading such as ‘I am the one who is telling you -- and only you -– the actual truth’ might just be too much. Secondly, despite the fact that preverbal object clitics from Spanish are often intertwined in PAL sentences (as much in the old recordings by Schwegler as in Lipski’s newer recordings: \cite[86--87, 115--116]{Lipski2020}), in such cases we would expect a Spanish conjugated verb (and not a clearly Creole verb, as in the example above). Thirdly, when the Spanish object clitic \textit{te} is inserted in a PAL sentence, it usually corresponds to PAL 2\textsc{p.sg} pronoun \textit{bo}, not to the pronoun \textit{té} (< Sp. \textit{usted}) (in the older recordings by Schwegler I find examples like \textit{bo <te va[s]> agüé?} ‘are you leaving today?’, but not examples like \textit{uté <te va[s]> agüé?}; the same is true for the examples presented by \cite[86]{Lipski2020}).}
\pagebreak
\ea\label{ex:gut24}(Recordings by A. Schwegler 1985--1988)\\
\gll to majanasito chikito a tá kamino ané i hende ngande tambié \\
all child small \textsc{cpl} be road 3\textsc{p.pl.poss} and people big also \\
\glt ‘all children are on their [own] journey, as are the adults’
\z



\section{The genesis of generic-noun-based indefinites throughout Creole languages}\label{sec:gut3}

\subsection{Universalist explanations}\label{sec:gut3.1}

The role of linguistic universals is traditionally acknowledged, by different approaches, as an essential part in the process of creolization. It is unclear, however, how exactly the role of universals in creolization is intertwined with that of superstratal and substratal influence and whether we are dealing with universals of first-language (L1) and/or second-language (L2) acquisition. The most famous approach that relates creolization to L1 acquisition is made by \citeauthor{Bickerton2016}: for this author, prototypical Creoles -- those being in the center of the \textit{Continuum of Creoles}\footnote{Obvioulsy, this concept is different from that of (post-)Creole continuum \citep{DeCamp1971}.}  \citep[][vii--viii]{Bickerton2016} -- have formed as a result of the nativization of pidgins. The idea that pidgins somehow carry the germ of Creoles is an old one,\footnote{In fact, \citet[215]{Schuchardt1888} already defined \textit{Jargon} -- close to what we call today a pidgin -- as “das Kreolische im Keim”, i.e. ‘a germinal Creole’.} yet Bickerton’s new approach consisted in making the process of creolization depend on the existence of a \textit{bioprogram} -- with which human beings are genetically provided -- whereby children who receive a pidgin as an input (L1) expand it naturally to a Creole. Bickerton developed the hypothesis of the bioprogram in his book \textit{Roots of Language} (\citeyear{Bickerton2016}) and contextualized it later in a wider context in \textit{Language and Species} (\citeyear{Bickerton1990}), in which he formulates a general theory of the human language. This theory is based on the distinction between \textit{protolanguages} -- a category in which he includes, alongside pidgins, the sign language of chimpanzees, the “talk” of human babies under two years of age, the talk of “Tarzans”, who did not receive any language input up to their adulthood, etc.~-- and \textit{languages}, namely the natural, completely developed languages, including Creoles. Both books together analyze the syntactic properties that distinguish languages from protolanguages, properties such as recursivity, interpretation of null categories according to syntactic rules, etc.

For other authors, like \citet{McWhorter2011}, the language contact between adults results in a structural simplification, which brings about new restructured varieties that can develop into Creoles. Since they are younger languages, Creoles are structurally simpler than non-Creoles, in which complexity has been developing throughout history -- often in connection with multi-secular written language cultures. It cannot be denied (no paradigm does that) that Creole languages are native languages (L1), but a great deal of the prototypically creole features are already present in the L2 or \textit{learner varieties} of the adults, which are later passed on to subsequent generations. By stating, for instance, that “situations involving second language acquisition include classroom learning, language shift leading to the formation of “indigenized” varieties of [...] languages, and creole formation” \citep[][127]{Winford2008}, it is not implied that creolization \textit{is} L2 acquisition, but rather that the latter process makes part of the former and, therefore, it makes sense to compare Creoles with other outcomes of L2 acquisition. In this regard, the discussion usually revolves around the role that children and their allegedly “creative” varieties play in the emerging creolophone communities -- which is critical for Bickerton, but not for many other authors. This is the key to understanding how much (or how little) of the interlanguage features can be transmitted to the following generation of Creole speakers.

\begin{sloppypar}
As for the most common developments of interlanguages, special attention has been paid to the overgeneralization of variants (generally, the analytic, more transparent ones) and the regularization of morphological irregularities \citep{McWhorter2011, Selinker1972}. As a matter of fact, such developments can account -- partially, at least -- for the phenomenon studied here: generic nouns are marginal but possible variants for conveying indefinite expressions in Ibero\hyp Romance (see \sectref{sec:gut3.2}) and might therefore become general in the interlanguages, in which they prevailed over special indefinites. This change would give rise to a bigger regularity of the system: “irregular” indefinite pronouns would be avoided in favor of the more regular indefinite NPs, in which common indefinite determiners are used. In addition, irregular morphology (such as the endings -\textit{ie} and -\textit{ien} of the Spanish indefinite pronouns \textit{alguien} and \textit{nadie~nadien}\footnote{The variant \textit{nadien} is documented in Caribbean texts written by \textit{semi-illiterate} Hispanic Creoles (white-descendents) during the colonial period (\citealt[][546, 548, 589]{GutiérrezMaté2018}) (remember that Colonial Caribbean Spanish is the actual lexifier of PAL).}) is avoided in interlanguages too.
\end{sloppypar}

In reality, the formation of generic-noun-based indefinites can be explained, to a great extent, both by principles of L1 and L2 acquisition. Certainly, both processes were present in the primitive creolophone communities and, consequently, both played a role in shaping Creoles’ grammar. In fact, even from a Bickertonian perspective, some structures of the interlanguages can be inherited during the nativization and remain unchanged afterwards. I will briefly elaborate on this last issue.

As Bickerton himself acknowledges in \textit{Language and Species} (\citeyear[][164--196]{Bickerton1990}), some elements of \textit{protolanguages} (like pidgins) can survive in \textit{languages} (like Creoles), despite the fact that the change from one to the other is supposed to be mainly an abrupt one. For example, the author considers that the interrogatives formed on the basis of generic nouns (‘what-man’ for ‘who’, ‘what-thing’ for ‘what’, ‘what-place’ for ‘where’, etc.) in many Creoles are relics of their “pidgin phase” (\citeyear[][183]{Bickerton1990}). In a protolanguage, it would have been practical to use one single non-referential element together with different (generic) nouns to form interrogative expressions.\footnote{Yet Bickerton is also aware that, on the other hand, generic nouns could have been too abstract for protolanguages (\citeyear[][182]{Bickerton1990}).} Furthermore, in the case of indefinites, we could formulate an analogous reasoning to justify the appearance of the analytic forms with \textsc{thing} and \textsc{person}. In addition, such linguistic change would be taking place in a general context, in which indefinites are supposed to represent absolutely necessary information in protolinguistic communities (\citeyear[][184--185]{Bickerton1990}) (protolanguages cannot simply leave the semantic category of indefiniteness open to contextual interpretation).

Moreover, the interpretation of null elements in protolanguages is not systematic (they rather require “guesswork identification” for their correct interpretation; \cite[][169]{Bickerton1990}), which could have naturally triggered the overtness of the nominal head, initially as a form of avoiding ambiguities with regard to the reference of the zero element: things/people that are mentioned in the previous discourse or can be identified situationally vs. things\slash people in general. Actually, if we assume the derivational link between pronouns and determiners (a framework that has a long tradition and has been applied even to account for the relationship between articles and personal pronouns: see \cite[][48--51, 179--191]{Bosque1989} and references therein), we can easily understand that this link extends to the case of indefinite expressions. As for the overtness of the phrase head, we have to acknowledge the fact that the syntactic configuration of the NP/DetP ([\textsc{dp} [Det [\textsc{np} N (Adj)]]) -- as well as the configuration of syntactic phrases in general~-- is considered to be an elementary aspect of the transition from protolanguages to languages \citep[][191]{Bickerton1990}; thus, the explicit marking of phrasal structure (avoiding null heads) could also be seen as a natural reflexion of the bioprogram, especially when the overt morphological features of the determiner had already been lost during the pidgin phase (in PAL there is \textit{un, aggú(n), mucho,} etc. but not \textit{*una/unos/unas, *alguna/algunos/algunas, *mucha/muchos/muchas,} etc.) (possibly, the same principle accounts for the frequent use of \textsc{thing} and \textsc{people} with non-pronominal NPs; see my comment on examples (\ref{ex:gut21}--\ref{ex:gut22})). To illustrate this, we can notice the differences -- and, most importantly, the structural correspondences -- between the following indefinite expressions in Spanish and PAL:


\begin{table}
\small
\begin{tabularx}{\textwidth}{QQQ}
\lsptoprule
           & nominal indefinite expression & pronominal indefinite expression\\
\midrule
Example & Sp: \textit{He bebido mucha leche} & Sp: \textit{He bebido mucho}\\
(constructed by linguist) & PAL: \textit{I a bebé mucho leche} & PAL: \textit{I a bebé mucho kusa}\\
& Eng: ‘I drank lot of milk’ & Eng: ‘I drank a lot’\\
\tablevspace
DetP structure in & \textit{mucha leche:} & \textit{mucho:}\\
Spanish & [DetP mucha [NP leche]] & [DetP mucho [NP ∅]\\
\tablevspace
DetP structure in & \textit{mucho leche:} & \textit{mucho kusa}\\
PAL & [DetP mucho [NP leche]] & [DetP mucho [NP kusa]\\
\lspbottomrule
\end{tabularx}\\
    \caption{Phrasal structure of (pro)nominal indefinite expressions in Spanish and PAL}
    \label{tab:gutmat2}
\end{table}


Of course, the same structural analysis can be extrapolated to other indefinite expressions analyzed throughout this chapter (like \textit{to kusa} ‘everything’). The same DetP structure is, therefore, valid to account for both indefinite NPs and pronouns in both the Creole and its lexifier. Specifically, assuming that the generic noun in PAL occupies the same syntactic position as the null element (∅) in Spanish is key to understand the hypothesis. 

In fact, we can even find languages/varieties emerged in (current or past) language contact settings that have developed the tendency towards the overt use of nominal heads in pronominal expressions beyond indefinites (for example, in demonstratives). This is the case, for example, of the \textit{quilombo} community of Jurussaca (Pará, Brazil), in which the proform \textit{um/uma} is regularly used in demonstrative pronominals: \textit{esse um/essa uma, aquele um/aquela uma} ‘that one’ (\textit{esse/essa, aquele/aquela} in Standard Portuguese) \citep{CamposVale2018} (this use resembles that of one in English and similar pro-forms in other languages: see also \cite[][29, 183--184]{Haspelmath1997} about ‘one’-based indefinite expressions).

In PAL, however, overt nominal heads (generic nouns) are mostly restricted to indefinite and, to a lesser extent, interrogative pronouns, i.e. we still regularly find ∅ in demonstrative, possessive and relative pronouns,\footnote{See the following examples: 
\ea (Fieldwork M. Gutiérrez Maté 2017)\\
    \gll ese ∅ nu má jue-ba\\
    that ∅ \textsc{neg} more be-\textsc{imp}\\
    \glt ‘it was only that’\\
    
    \ex (Recordings by A. Schwegler 1985--1988)\\
     \gll yo sí ten maílo nu pokke ∅ ri mí a morí\\
    \textsc{1p.sg} \textsc{aff} have husband \textsc{neg} because ∅ of me \textsc{prf} die\\
    \glt ‘I really have no husband, because mine has died’\\
    
    \ex (Fieldwork M. Gutiérrez Maté 2017) \\
    \gll kabeo liso, <dise> ma ∅ loke konosé-lo\\
    hair straight <say> \textsc{pl} ∅ \textsc{rel} know-\textsc{3p}\\
    \glt ‘[she had] straight hair, say those who knew her’
\z} which is possibly the main reason why language acquisition universals alone cannot account for the origin of PAL generic-noun-based indefinites. Most certainly, the type of changes studied in this chapter would not have taken place if the corresponding Spanish generic nouns had not occasionally been used with an unspecific (quasi-pronominal) reading, nor -- most importantly -- if Spanish had not come into contact with a language (Kikongo) that regularly employs generic-noun-based indefinites. The following two sections are devoted to these issues.

\subsection{Superstratist explanations}\label{sec:gut3.2}

The continuity between Creoles and their lexifiers has been highlighted and explained on a theoretical level predominantly -- although not exclusively -- by authors often referred to as “anti-exceptionalists” \citep[cf.][]{Mufwene2001}. In the particular case of Gallo-Romance linguistics, Chaudenson (whose impact on Mufwene is evident) was the author who, together with his disciples, worked the most along this line of investigation \citep[cf.][]{Chaudenson1992, Chaudenson2003}. From this perspective, Creoles derive from their lexifier languages, just as Romance languages derive from Latin. Specifically, Creoles are considered to have resulted from \textit{approximations des approximations}. These took place, for instance, when the African slaves from the French plantations in the Caribbean did not learn French from the white colonizers (the plantation owners) but from the foremen, who spoke L2 French themselves.

According to the above, it is necessary to wonder about indefinites in PAL’s superstrate: this leads us to investigate the use of the nouns \textit{cosa} ‘thing’ and \textit{gente} ‘people’ in 17\textsuperscript{th} century Northern Colombian Spanish, which we can consider as the authentic lexifier of Palenquero. To this effect, I rely on the corpus of colonial Caribbean documents that I gathered and transcribed for my PhD thesis and with which I have worked since then.\footnote{See \citet[][431--442]{GutiérrezMaté2013} for a description of the documents and their archivist references.} From this corpus of documents, I have selected those written in the governorate of Cartagena de Indias.

As regards the use of \textit{cosa}, it is common to find it together with an indefinite (mostly, postnominal) determiner in cases in which the entire nominal phrase can function as an indefinite pronoun (a reading that is especially clear when the reference of \textit{cosa} cannot be interpreted as a material thing). When the sentence polarity is positive, the determiner is \textit{algún/a} (\ref{ex:gut25}), which is also possible when the sentence polarity is negative and, therefore, the interpretation of the indefinite expression is also negative (\ref{ex:gut26}); nonetheless, the negative determiner \textit{ningún/a} can also be used with other overt markers of negative sentence polarity like the negation adverb (double negation) (\ref{ex:gut27}):

\ea \label{ex:gut25}
    (Cartagena de Indias 1694, p. 45r, ls. 6--8)\\
    \textit{se determino [...] que si se le aberiguara cosa alguna en este particular le diera la puniçion deuida a la naturaleça del delito}\\
    ‘it was decided that, if something was found out in this regard, it would be punished according to the nature of the crime’
    
     \ex \label{ex:gut26}     
     (Cartagena de Indias 1695, p. 238v, l. 22)\\
     \textit{No los he molestado en cosa alguna}\\
    ‘I have not bothered them at all’ (/...in any way)\\
    
     \ex \label{ex:gut27}(Cartagena de Indias 1672; p. 82r, 19--20)\\
     \textit{Embarcaron con orden de que no dejase sacar cosa ninguna del navío}\\
    ‘They embarked with an order not to let anything be taken out of the ship’
\z

In addition, \textit{cosa} can be used with no determiner (\ref{ex:gut28}), which does not preclude the possibility that it is modified by adjectives (\ref{ex:gut29}):

\ea \label{ex:gut28}
    (Cartagena de Indias 1694, p. 33r, ls. 9--10)\\
     \textit{No hay cosa que más se pueda temblar que unas hiervas}\\
     ‘There is nothing to fear more than some herbs’
    
     \ex \label{ex:gut29}
     (Cartagena de Indias 1693d, p. 5v, ls. 5--6)\\
     \textit{Se han puesto de mi parte sin otra causa de que ven mi limpieza y que no hago cosa injusta}\\
    ‘They have taken my side for no other reason than seeing my cleanliness and that I don't do anything unfair’
\z

Even in the absence of a quantitative study, it is clear that all these uses were also possible in other Spanish varieties at that time and even today (if anything happened to stand out in these examples, it would be the placement of the determiner \textit{algún/ningún} in the unusual postnominal position, but this particular word order seems to play a role for the pronoun-like use of generic nouns in Spanish: most especially, a prenominal determiner \textit{alguna} would not be compatible with the negative sentence polarity in (\ref{ex:gut26})). As for the use of \textit{gente}, there does not seem to be substantial differences with other Spanish varieties either. Especially as we do not register its use meaning ‘someone’ (the special indefinites \textit{alguien} or \textit{alguno} are used instead). However, a few interesting phenomena regarding the use of \textit{gente} have to be noted: firstly, some uses in negative contexts are certainly close to the meaning of ‘no one’ (i.e. \textit{gente} can be used instead of \textit{nadie/ninguno}, which were, in any case, the predominant forms in Colonial Caribbean Spanish):

\ea \label{ex:gut30}
    (Cartagena de Indias 1772, p. 513v, ls. 3--5)\\
    \textit{volvio a zalir a dicha Cassa la que encontró en silencio por haver reconozido no haver gente dentro de ella}\\
    ‘He returned to enter the aforementioned house, which he found in silence as he realized there was no one inside.’
\z

Secondly, when \textit{gente} was used with a non-arbitrary meaning, its reference adapted to the cultural and sociological idiosyncrasies of the colonial Caribbean: for example, \textit{gente} was commonly used for ‘militias’ against the enemies of the city and, in a more general fashion, to refer to certain social and/or ethnic groups, in which the speaker could include him/herself or not. Thirdly, in relation to what has just been said, \textit{gente} could adopt the meaning of a generic subject and even a generic subject with inclusion of the speaker \citep[][80--82]{GutiérrezMaté2013}, which resembles the use of a first person plural pronoun. Nevertheless, despite the fact that acknowledging that speakers used \textit{gente} to include themselves in the predication may well be revealing as regards the origin of some current uses of PAL \textit{hende} (see examples \ref{ex:gut19}--\ref{ex:gut20}), it does not say anything about its use as an indefinite pronoun. In other words, the linguistic change “generic noun > indefinite pronoun” is different from the change “generic noun > generic pronoun > 1\textsc{p.pl} pronoun”, even though one process does not prevent the other from taking place, as has actually happened in the case of PAL (not so in the case of Caribbean Spanish, since the marginal use of \textit{gente} as (something like) a 1\textsc{p.pl} pronoun does not seem to have really taken root, nor did it develop further after the colonial era).

Finally, the noun \textit{persona} deserves a separate comment, since it exhibits -- more frequently than in the case of \textit{gente} -- semantic readings close to that of an indefinite pronoun, especially in negative contexts (the phenomenon, evidently, is well known in other world languages, including French):

\ea \label{ex:gut31}
    (Cartagena de Indias 1694a, p. 3r, ls. 1--4)\\
    \textit{Domingo Criollo, con notiçia que havía tenido de dichos bandos, havía publicado uno en su palenque San Miguel, para que no saliesse negro d[e] él, ni tubiesse comunicaçión con persona}\\
    ‘Domingo Criollo, after having known about the aforementioned town proclamations, had announced one in his \textit{palenque} called San Miguel, so that no black person would leave it or have any communication with anyone [outside]’
\z

However, as we have seen, \textit{persona} does not seem to have entered (Traditional) Palenquero, neither in its canonical use as a noun nor in its potential use as an indefinite pronominal.

As it turns out, the superstratist explanation alone cannot account for the emergence of PAL generic-noun-based indefinites either. Moreover, if the drift towards generic-noun-based indefinites were somewhat inherent in the Spanish language, we could not explain why these have not become the main strategy for building indefinite expressions in other Hispanic varieties, not even in today’s Northern Colombian Spanish.

\subsection{Substratist explanations}\label{sec:gut3.3}\largerpage

\subsubsection{The data from the APICS}\label{sec:gut3.3.1}
\begin{sloppypar}
Today it is easy to prove that generic-based-noun indefinites are predominant in Creole languages; they clearly outnumber interrogative-based indefinites, which constitute the dominant group throughout the world languages. \tabref{tab:gutmat3} compares the data of the WALS, which contains a sample of 326 languages, with the data of the APICS, which codifies 74 languages (most of them being Creoles, together with a few other varieties resulting from language contact: mixed languages and partially restructured varieties). The APICS data are, in fact, taken from the WALS-like APICS (\url{https://apics-online.info/wals/21#2/30.4/9.8}), in which some values have been adapted in order to make them compatible with those that had previously served to codify the interlinguistic variation in the WALS: on the one hand, a “mixed” value is introduced in those cases in which the APICS specified the actual percentages of such “mixtures” (linguistic variables); on the other hand, the values “generic-noun-based indefinites” and “old generic-noun-based indefinites continuing somebody/something”,\footnote{This was the value for those cases in which generic-based-noun indefinites were already found in the lexifier, as in the case of English \textit{something}, French \textit{quelque chose,} etc.} which were distinguished in the APICS, are unified in the WALS-like APICS. The resulting data speak for themselves: whereas in non-Creole languages the proportion of generic-based-noun indefinites is approximately ¼, the proportion rises to about ¾ in Creoles. As is well known, the four basic values of the WALS (interrogative-based, generic-noun-based, special and existential constructions) rely on the distinction first introduced by \citet{Haspelmath1997}.
\end{sloppypar}

\begin{table}
    \small 
    \begin{tabularx}{\textwidth}{>{\raggedright\arraybackslash}p{.3\textwidth}YYYY} 
    \lsptoprule
    & \multicolumn{2}{l}{WALS (world languages)} & \multicolumn{2}{>{\raggedright\arraybackslash}p{.3\textwidth}}{APICS (world Creole languages and a few other language contact varieties)}\\
    \cmidrule(lr){2-3}\cmidrule(lr){4-5}
    & N & \% & N & \% \\
    \midrule 
    Interrogative-based (\textit{wh}- indefinite) & 194 & 59.5\% & 4 & 5.4\% \\
    \tablevspace
    Generic-noun-based (\textsc{thing}, \textsc{person}) & 85 & \cellcolor{lsLightGray} 26\% & 54 & \cellcolor{lsLightGray} 72.9\% \\
    \tablevspace
    Special (different lexemes or different morphology) & 22 & 6.7\% & 6 & 8.1\%  \\
    \tablevspace
    Existential constructions (‘there is one who…’ meaning ‘someone…’) & 2 & 0.6\% & 1 & 1.3\% \\
    \tablevspace
    Mixed (combination of two or more of the former) & 23 & 7\% & 9 & 12.1\% \\
    \midrule
    (language sample) & 326 & & 74 & \\
    \lspbottomrule
    \end{tabularx}
    \caption{Indefinite pronouns in the world languages and in Creoles}
    \label{tab:gutmat3}
\end{table}  

Initially, it may seem that Creoles naturally tend to form generic-noun-based indefinites, so that we could think of these as universals of creolization (in line with the arguments presented under \sectref{sec:gut3.1}). However, a closer look at the maps of the WALS and the WALS-like APICS shows that this is not the case. We can, for instance, observe three linguistic areas in which, according to the WALS, we find different strategies for the building of indefinite pronouns:\largerpage

\begin{enumerate}
    \item[1)] Sub-Saharan languages (mostly from the Niger-Congo family), which belong to the areas III and IV according to the classification of African languages by \citet{Güldemann2010}. In these languages (marked in blue in  \figref{fig:gut1}), the use of generic nouns prevails.
    \item[2)] Southern Indian languages (from the Dravidian family). In these languages (marked in red in  \figref{fig:gut1}), the use of interrogative-based indefinites prevails.
    \item[3)] Philippine languages (as a relatively homogeneous subgroup within the Austronesian languages). In these languages (marked in white in  \figref{fig:gut1}), the use of (pseudo)existential constructions prevails (when this strategy is combined with other types of indefinites, the languages are marked in grey in  \figref{fig:gut1}).
\end{enumerate}

\begin{figure}
\includegraphics[width=\textwidth]{figures/Figure1_MGM.png}
\caption{WALS map, feature 46A (“Indefinite pronouns”). Base map © OpenStreetMap contributors.}
\label{fig:gut1}
\end{figure}

For our purposes, the most interesting observation is that these three groups of languages also constituted the substrates of different groups of Creoles, which~-- with the exception of Group 1a (s.  \figref{fig:gut2}) -- formed \textit{grosso modo} in the same areas. If we compare the world languages from  \figref{fig:gut1} to the Creoles from  \figref{fig:gut2}, there seems to be no doubt about the substratal influence in the making of indefinite pronouns in Creoles. As for the particular case of the group 1a (Caribbean Creoles), it is well known that they came into being with participation of the same substrate languages that influenced group 1b: Caribbean Creoles emerged as a consequence of forced migration (slave trade) of speakers of Niger-Congo languages, who made contact with other exogenous languages (those spoken by the European colonists) in the New World. In addition, the few yellow spots in the area 1a (representing Creoles with special indefinites) should be taken carefully or even partially recoded; for instance, indefinites in Papiamentu (in yellow in Figure \ref{fig:gut2}) should be considered “mixed” rather than “special” for two reasons: (1) the forms \textit{un hende} \citep[][ex. 47--42]{Kouwenberg2013} and \textit{algun hende} \citep[][60]{Maurer1998} are consistently used as personal indefinite pronouns, and (2) the form \textit{un kos} \citep[][60]{Maurer1998} can also be used for the meaning of ‘something’ alongside the special indefinite \textit{algo} \citep[][ex. 47--52]{Kouwenberg2013}.\footnote{The origin of \textit{algo} in Papiamentu could be attributed to 17\textsuperscript{th} century Ibero\hyp Romance or to relexification by means of the canonical (Caribbean) Spanish form \textit{algo} at a later evolutionary stage of the Creole.}

\begin{figure}
\includegraphics[width=\textwidth]{figures/Figure2_MGM.png}
\caption{WALS-like APICS map, feature 21 (“Indefinite pronouns”). Base map © OpenStreetMap contributors.}
\label{fig:gut2}
\end{figure}

The\largerpage{} above does not mean that the formation of indefinite expressions in \textit{all} Creoles is directly inherited from their respective substrates, since the emergence of each Creole is idiosyncratic according to various factors: number of the substrate languages, homogeneity/heterogeneity of the typological characteristics of the substrates, degree of exposure to the lexifier language, and many other aspects concerning each particular language ecology. It is, however, evident that the preferred structural type of indefinite pronouns in many Creoles coincides with that of their substrates, which obliges us to take the hypothesis of substratal influence into account. Having made these clarifications, the relationship between Creoles and their substrates as regards indefinite pronouns is depicted in  \tabref{tab:gut4}.

\begin{table}
    \small \begin{tabularx}{\textwidth}{QcQ} 
    \lsptoprule
    Substrate languages (linguistic areas) & & Creoles \\
    \midrule
    Group 1. Areas III and IV of African languages according to \citet{Güldemann2010} & \Rightarrow & {Group 1a. Caribbean Creoles\newline Group 1b. African Creoles (both continental and insular)} \\
    \tablevspace
    Group 2. Southern Indian (Dravidian) languages & \Rightarrow & Group 2. Sri Lankan Creoles \newline (Sri Lankan Portuguese, Sri Lankan Malay) \\
    \tablevspace
    Group 3. Philippine languages & \Rightarrow & Group 3. Chabacano varieties (most \newline especially, the variety of Zamboanga) \\
    \lspbottomrule
    \end{tabularx}
    \caption{Indefinite pronouns in the world languages and in Creoles}
    \label{tab:gut4}
\end{table}  


The Creoles of the groups 2 and 3 have preferred the indefinite pronouns type of their substrates over the type displayed by their lexifiers. For example, the lexifiers of both Sri-Lankan Creoles do not use interrogative-based-indefinites: Malay makes use of generic nouns, whereas Portuguese combines special indefinites (\textit{alguém}) with generic-noun-based (\textit{alguma coisa}). As regards Chabacano, it becomes evident that the use of existential sentences with the meaning of indefinite expressions is not inherited from the lexifier (Spanish); in addition, this feature is, from a typological perspective, extremely unusual, so the likelihood that Chabacano (a Philippine/Spanish Creole) had developed this very feature “on its own” -- relying on universal principles -- are also extremely low.

Many Creoles spoken in the area 1b do not allow for the impact of the substrate languages to be proven, since both substrates and superstrates coincide in using generic-noun-based indefinites (for instance, Lingala has Bobangi, a Bantu C language, as its lexifier and other Central and West African Languages as its substrates -- cf. \cite{Meeuwis2013}, where both Bantu languages and Niger-Congo languages spoken in West Africa generally employ generic-noun-based-indefinites). As regards those Creoles spoken in the area 1a, we find a somewhat more relevant impact of the superstrates (remember the case of Papiamentu \textit{algo}), but substratal influence should be considered to be the most relevant factor in the formation of indefinite expressions throughout Caribbean Creoles. As depicted in \tabref{tab:gutmat3}, generic-noun-based indefinites are not strange in universal terms (¼ of the languages of the world employ it), but the extremely high percentage of its appearance in Creoles (up to ¾ of these languages) does not only seem to be the effect of universal dynamics, yet either the substrate or the superstrate must have played the decisive role. In the case of PAL, where the superstrate makes a rather marginal use of generic nouns as indefinites, substratal influence should be considered the ultimate trigger of the linguistic change analyzed here.

\subsubsection{The Kikongo data}\label{sec:gut3.3.2}

Unfortunately, Kikongo is not represented in  \figref{fig:gut1}. However, this language follows the general tendency found throughout Sub-Saharan languages of using generic nouns to form indefinite expressions. In the words of \citet[][118]{Kyala2013}: “Na ausência de palavras precisas para exprimir a noção exacta de pronome indefinido, o kikongo faz recurso às locuções pronominais indefinidas” (‘Due to the lack of precise words to express the exact notion of indefinite pronoun [=~special indefinites], Kikongo makes use of indefinite pronominal expressions [=~generic\hyp noun\hyp based indefinites]’). In Kikongo, there are in fact various words with the meaning of \textsc{thing} that can adopt an indefinite interpretation: The most common are \textit{kyuma} \textasciitilde\ \textit{kima}, \textit{diambu}, and \textit{kimvela}, although there are others as well \citep[cf.][150--151]{Laman1912}. \textit{Kyuma} seems to most frequently have the concrete meaning of ‘object’, which means that, when it is used in indefinite expressions, its potential referents are also material:

\ea \label{ex:gut32}
    (Fieldwork M. Gutiérrez Maté 2020)\\
    \textit{mwisi kyuma ko}\\
    \gll mu-isi ki-uma ko\\
    \textsc{nc}3-inhabitant.from \textsc{nc}7-thing \textsc{neg}\\
    \glt ‘there is nothing inside’ (lit. ‘nothing is from here’) (answering the question: \textit{what do you have in the box?})
\z

\textit{Diambu}, when used as generic noun, can have a more abstract value than \textit{kyuma} and be often translated with ‘problem’. Because of this, the potential referents of \textit{diambu} can also be more abstract (example (\ref{ex:gut33}) is the answer of an informant of Kiyombe when I asked him to translate \textit{eu vou te dizer algo} ‘I’m going to tell you something’ from Portuguese to Kiyombe):

\ea \label{ex:gut33}
    (Fieldwork M. Gutiérrez Maté 2020)\\
    \textit{minu diambu iakukamba}\\
    \gll minu di-ambu i-a-ku-kamb-a\footnotemark\\
    \textsc{1p.sg} \textsc{nc5}-thing/problem \textsc{1p.sg-pres/fut-2p.obj}-tell-\textsc{fv}\\
    \glt‘I am going to tell you something’
\z
\footnotetext{Future tense is not a common meaning of the circumfix \textit{a-...-a} throughout Kikongo dialects, but it can be found in some varieties and has even been described previously by missionary grammars \citep[cf.][170]{DomBostoen2015} and references therein.}

Lastly, \textit{kimvela} is the word that has the most ambiguous status of the three, perhaps being mostly restricted to its use as a pronoun (with nominal morphology, where \textit{ki-} is a prefix of class 7, the same as we find in \textit{kyuma}). It is defined by \citet[][s.v.]{Laman1936} as “pas un seul; pas un brin; rien; néant”, i.e. exclusively as a negative indefinite (‘nothing’). However, at least Kiyombe speakers (keep in mind that Laman describes a different dialect, which belongs to the Central Kikongo sub-clade: \cite[][147]{BostoenDeSchryver2015}) also use it as a positive indefinite (‘something’). As a matter of fact, when my informants were asked to translate this word into Portuguese, they answered primarily with the positive indefinite (\textit{alguma coisa} or \textit{algo}):

\ea \label{ex:gut34}
    (Fieldwork M. Gutiérrez Maté 2020)\\
    \textit{twala kimvela}\\
    \gll twala ki-mvela\\
    bring(\textsc{imp}) \textsc{nc7}-thing\\
    \glt ‘bring something!’
\z

As for personal indefinite pronouns, Kikongo uses the generic noun \textit{mu(n)tu} ‘person’ (\textsc{pl} \textit{ba(n)tu}): 

\ea \label{ex:gut35}(\cite[][117]{Kyala2013})\\
    \textit{Etata, muntu mosi ka wiza kunzo}\\
    \gll e-tata mu-ntu mosi k-a-(k)wiza ku-nzo\\
    \textsc{aum}-father \textsc{nc1}-person one \textsc{3p.sg-past}-come \textsc{nc17}(\textsc{loc})-house/home\\
    \glt ‘Father, someone had come home’\\
    (Port. translation by the author: \textit{Pai, tinha vindo alguém (alguma pessoa) em casa}))
    
    \ex \label{ex:gut36}(\citealt[][40]{Carter1999})\\
    \textit{kavàkala muntu ko}\\
    \gll ka-va-a-kala mu-ntu ko\\
    \textsc{neg-loc(nc16)}-\textsc{past-cop} \textsc{nc1}-person \textsc{neg}\\
    \glt ‘there was no one’
\z

When used as a positive indefinite (‘someone’), \textit{muntu} can be modified by the determiner \textit{mosi} (‘one, same’): \textit{muntu mosi} has even been described as the canonical form for ‘someone’ in some grammars \citep[][52--54]{Kyala2013}, but \textit{muntu} can also appear without further modification (according to some examples by \cite[][s.v. \textit{muntu}]{Laman1936}). In the French-Kikongo dictionary by \citet{BiyokoMabua2017} \textit{mutu mosi} is consistently translated as \textit{une personne} (‘one/some person’), whereas \textit{mutu} (without determiner) is translated as the actual indefinite pronoun (\textit{quelqu’un}). When \textit{muntu} is used as a negative indefinite (‘no one’) the generic noun seems, at first, to generally lack any determiner (this was so, at least, amongst my informants), but there is at least one source that indicates the two variants: \textit{ka muntu ko} and \textit{ka muntu mosi ko} (see \cite[][s.v. \textit{personne}]{Dereau1957}).  
In a similar way to PAL (s. examples \ref{ex:gut23}--\ref{ex:gut24}), attributive adjectives in copular sentences and similar constructions seem to often be accompanied by an explicit nominal head (a generic noun): 

\ea\label{ex:gut37}\citep[][150]{BiyokoMabua2017}\\
Bibila kima kimboti\\
\gll Bibila ki-ma ki-mboti\\
Bible \textsc{nc7}-thing \textsc{nc7}-good\\
\glt ‘The Bible is good’ (/‘The Bible is a good thing’)
\z

It is important to note that, just as today’s Kikongo, 17\textsuperscript{th} century Kikongo (the actual substrate of PAL) seems to only make use of generic-noun-based indefinites. In this paper, I cannot elaborate on these problems, which would demand a detailed analysis of the different Kikongo \textit{doculects} written in the 17\textsuperscript{th} and 18\textsuperscript{th} centuries \citep{BostoenDeSchryver2015, BostoenDeSchryver2018}, but a look at the very first source written in Kikongo (if we leave aside the few quoted words or sentences in this language that can be found in official documents written in Portuguese during the 16\textsuperscript{th} century) can quickly confirm the structural tendency outlined above. I am referring to the \textit{Doutrina Cristaã}, a bilingual catechism Portuguese → Kikongo, published in 1624: even though there are no cases of \textit{alguém, algo} or \textit{nada} in this text, we do find several instances of \textit{ninguém}, which are regularly translated as \textit{muntu} ‘person’ (in negative sentence contexts) (s. the edition by \citealt[][119, 145, 195]{Bontinck1978}).

Finally, it has to be noted that the grammaticalization of \textit{muntu} -- or its cognates in other closely related Bantu languages -- as a sort of focus particle\footnote{The change included several evolutionary stages of type “John is the \textsc{person} who did it” > “John \textsc{person} did it” > “The cat \textsc{person} did it”, etc.} does not seem to affect the vernacular varieties of Kikongo that I am dealing with (if anything, the change would find itself at its very early stages in Kiyombe). Such development has taken place in some languages of the Bantu B, C and H areas \citep{VanderWalManiacky2015}, but is especially characteristic of two vehiculars, Kituba and Lingala, which have often been classified as Creoles. Since these languages probably came into being as late as the second half of the 19\textsuperscript{th} century or at the turn of the 20\textsuperscript{th} century, we can easily understand that the linguistic change “generic noun \textsc{person} > focus particle” has not affected PAL in any way.

\section{A first account of the use of indefinite pronouns in Cabindan Portuguese}\label{sec:gut4}

Angolan Portuguese consists of a set of very heterogeneous varieties. Consequently, its analysis can be approached from many points of view (including the question of pluricentrism, i.e., by wondering about the possible endocentricity of Angolan Portuguese within the Lusophone World). In my case, however, it is only the “fossilized”\footnote{See \citet[][82--86]{Roche2013, Selinker1972}.} L2 varieties of Portuguese spoken by bilingual speakers with Kikongo (especially, Kiyombe) as a predominant L1 that are taken into account. All the examples I will present in this section were uttered by bilingual speakers.\footnote{Many examples come from old informants from the Cabinda province who learned Portuguese during the colonial period in a non-monitored way (although there was certainly contact with the European standard variety in their short time at school); thus, to give an example, one of the communities where I have done extensive fieldwork (Lites, municipality of Buco Zau, Cabinda) consisted of an old colonial \textit{fazenda} where the employees had learned Portuguese due to their contact with the white Portuguese settlers who owned the farms. Under these circumstances, Portuguese-based interlanguages, characterized by morphological simplification, the overgeneralization of analytic variants and various transfers from Kikongo, fossilized. It is in this sense that we can speak of the \textit{partially restructured} varieties of Portuguese that are spoken in Cabinda or, in a more general fashion, in Angola (cf. also \citealt{Inverno2009}, who also uses this concept with reference to the Portuguese varieties from Dundo, in the province of Lunda Norte).}

Especially after the colonial period, Portuguese became the authentic lingua franca in the country for several reasons.\footnote{One of the factors that had great impact on the generalization of Portuguese was the Angolan civil war (1975--2002). During this conflict, thousands of people had to migrate from one province to another (where a different indigenous language was spoken), which made the use of Portuguese more convenient.} It started being regularly used by almost the entire population in the cities and by a constantly increasing number or people in the countryside. Such \textit{vehicularization} of Portuguese did not necessarily bring the generalization of a given variety, nor the formation of a new one through koineization (at least not yet), even though there certainly are some features that have been widespread all over the country and are therefore characteristics of what we can today call “Angolan Portuguese”. Amongst these features, just to mention one that falls within the scope of indefinite expressions, we find the use of \textit{bué}, an indefinite determiner (and elative adverb) of Bantu origin,\footnote{The Umbundu etymology has been proposed on several occasions \citep[cf.][]{SchmidtRadefeldt}, but other source languages are also possible in theory, including Kikongo. The problem cannot be easily solved, since the possible sources of \textit{bué} that we are dealing with are formally close cognates. For instance, in Kiyombe, the indefinite determiner meaning ‘many’ is -\textit{phwedi}, as in the example below. A linguistic change Kiy. /ˈphwedi/ > Port. /ˈbwe/ seems also entirely possible:

\ea
    \textit{tsinzau tsi\textit{phwedi} tsidi mu Afrika}\\
    \gll ziN-zau zi-phwedi zi-idi mu Afrika\\
    \textsc{nc}10-elephant \textsc{nc}10-many \textsc{nc}10-\textsc{cop} \textsc{loc(nc18)} Africa\\
    \glt ‘there are many elephants in Africa’\\
    (fieldwork MGM, exercises of translation Portuguese → Kikongo)
\z} which has even spread out to other lusophone areas, including Portugal, as a marker of “youngspeak” \citep{Almeida2008}:

\ea \label{ex:gut38}
    \textit{Eu gosto bué\slash Lá tem bué de candongueiros}\\
    ‘I like it very much’\slash ‘There are many candongueiros [`informal shared taxis'] there’
\z

For the most part, however, the perception of Angolan speakers clearly distinguishes between different \textit{sotaques} ‘accents’, which, in reality, goes way beyond prosodic regional differences. Speakers can easily identify the (Umbundu-influenced) \textit{sulano} variety, the (Kikongo-influenced) variety of the Bakongo people from Northern Angola, the specific \textit{calão} spoken in Luanda (a folk concept under which two different realities can be understood: the Kimbundu-influenced varieties spoken in the city \citep[cf.][]{Mingas2000} and the varieties developed in the \textit{bairros sem identidade}, i.e., suburbs with no clear ethnic background like Lixeira-Sambizanga and others), etc. Accordingly, the linguistic features resulting from partial restructuring can also be different from one region to another, even though some commonalities are expected, insofar as the typological characteristics of the different \textit{línguas nacionais} spoken in Angola (mostly belonging to the Bantu H, K and R groups) are, to a great extent, the same.\largerpage

Creole languages never seem to have emerged in Angola (see \cite[][112--117]{GutiérrezMaté2020} and references therein). Therefore, we expect differences between the two Ibero\hyp Romance\slash Bantu vernaculars contrasted here, i.e. qualitative and quantitative differences that serve to determine the “degree of restructuring”. On the one hand, some features that were transferred from Kikongo to PAL do not make it into CP; on the other hand, some features that are transferred to both PAL and CP became grammar rules in the former, whereas they are just tendencies of use in the latter. The use of generic-noun-based indefinites can illustrate the latter principle, since it is by no means as systematic in CP as they are in PAL. Lastly, in Angolan Portuguese (including CP), we can find other phenomena that are primarily related to borrowing: most especially, amongst those speakers who employ Portuguese more frequently, the Kikongo influence is not so much the result of transfer (from the L1 to the L2) as the result of borrowing (from the L2 to the L1) (see \cite{ThomasonKaufman1988} about both directions of change in language contact scenarios). According to this idea, we can find some lexical and grammatical loanwords from Bantu languages, including the above-mentioned use of \textit{bué}, in CP as well as in other varieties of Angolan Portuguese. This phenomenon has to be distinguished from the one I am analyzing in this work, even though there may be some overlapping areas between the two phenomena: for example, if generic-based nouns were proven to be used more frequently in Angolan Portuguese than in other varieties of Portuguese, we would still have to decide whether this use results from transfer or from (structural) borrowing. In most cases, it is only the contact ecology and the particular type of bilingualism that allow us to solve the problem: in the particular case of my informants, who are more used to speaking Kikongo at an in-group level, one would assume that a higher frequency of generic-noun-based indefinites in Portuguese would be a consequence of transfer from Kikongo.

The variation between \textit{coisa} (the generic-noun-based indefinite) and \textit{algo} (the special indefinite) can be observed in an example like the following, recorded in a “paragem sem nome” (‘nameless stopover’) a few kilometers from Lândana (Cabinda):

\ea \label{ex:gut39}
    \textit{A (to B): essa menina vai vir a que hora?}\\
    ‘At what time will this girl come?’\\
    \textit{B (to C): [...] lhe mandaste…?}\\
    ‘you sent her…?’\\
    \textit{C (to B): foi comprá aí um coisa}\\
    ‘she went there to buy something’\\
    \textit{B (to A): foi comprar algo aqui}\\
    ‘she went there to buy something’
\z

In this case, both C (speaking to B) and B (speaking to A) refer to the same fact: the girl that A is asking for went to a nearby shop to buy something, which, deliberately, is left unspecified (it can be inferred from the context that it is a purchase of groceries that she repeats with a certain regularity). Thus, there is the same referent, yet an informant uses \textit{um coisa} (in which case the use of the masculine indefinite article instead of the feminine one represents another typical restructuring phenomenon), whereas the other informant employs \textit{algo}. It is probably no coincidence that the use of \textit{um coisa} takes place in the (in-group) communication between C and B (husband and wife respectively, both middle-aged), whereas \textit{algo} appears in the (out-group) communication with A, the foreign interviewer.\largerpage

Furthermore, the example is interesting because it demonstrates that the use of generic \textit{coisa} can be introduced by the indefinite article (\textit{um[a]} ‘a’) and not only by the indefinite determiner (\textit{alguma} ‘some’), which is anyway possible in CP as an alternative to \textit{algo}. As a matter of fact, \textit{algo} “is considered archaic and pragmatically highly marked in (Modern) Standard Portuguese” \citep[][ft. 21]{Cardoso2013}, so it has mostly been substituted by the generic-noun-based expression \textit{alguma coisa} (and also \textit{qualquer coisa}) ‘something’. As we have seen, this is also the reason why, as for the preferred strategy for building indefinite pronouns, the Portuguese language is classified -- accurately so -- as “mixed” in the WALS (see \figref{fig:gut1}). In Cabinda, \textit{algo} seems to be perceived as a more correct and less vernacular option than \textit{(alg)uma coisa}.\footnote{Whether the use of \textit{algo} is the result of an “archaism” or an \textit{idiomatization} \citep[cf.][]{Koch1997} out of official discourses – or both – still remains to be clarified: as stated before, Angolan Portuguese is many things, and it would not be surprising if some uses that are restricted to a few discourse traditions in Portugal had become more accepted in Angola through administrative documents, political discourses and the media -- which are often oriented to \textit{written} European Portuguese.} Possibly, it is perceived as a more polite form too (the following example was produced by a youngster who was trying to apologize on behalf of his grandfather when the latter was declining to be interviewed in a rather impolite manner):

\ea \label{ex:gut40}
    \textit{desculpa o se[nh]o[r], queria so falar algo, é que [...]}\\
    ‘Excuse me, sir. I just wanted to say something. The thing is […]’
\z

Further research should be able to determine the semantic nuances of all the possible variants that seem to convey the meaning of ‘something’ in CP, as well as the structural, stylistic and sociolinguistic factors that account for the use of one or the other variant. Alongside those forms that have already been mentioned (\textit{um coisa, uma coisa, alguma coisa} -- possibly \textit{algum coisa} too -- and \textit{algo}), we also find the generic noun with no article/determiner (\textit{coisa}) and the hybrid form \textit{um algo}. For example, the following sentences (quite similar in content) were recorded in an interview with the same speaker (male, 38, Buco Zau, Cabinda): 

\ea \label{ex:gut41}
    \textit{você tem coisa para falar? }\\
    ‘do you have something to say?’\\
    
    \ex \label{ex:gut42}
    \textit{tem um algo a dizer}\\
    ‘there is something to say’\\
\z

However, in this chapter, we focus on the use of those variants that are structurally closer to the grammar of the substrate, like the first example of (\ref{ex:gut39}) and, most especially, (\ref{ex:gut41}), in which \textit{coisa} is not modified by any indefinite article\slash determiner.

Just as we saw in the case of PAL, the generic noun with the meaning \textsc{thing} can also be used to form other indefinite pronouns in CP. Even though I have never heard \textit{toda coisa} (instead of \textit{tudo}), the indefinites \textit{pouca coisa} (\ref{ex:gut43}) and, most especially, \textit{muita coisa} (\ref{ex:gut44}--\ref{ex:gut46}) are very frequent:

\ea \label{ex:gut43}
    \textit{já tem setenta e dois ano que vou fazer, não é pouca coisa}\\
    ‘I am already about to be 72 years old, that’s no small thing’
    
    \ex \label{ex:gut44}
    \textit{naquela altura [...] eu trabalhei muinta coisa, fiz isso, fiz aquilo…}\\
    ‘at that time I worked a lot, I did this, I did that…’
    
    \ex \label{ex:gut45}
    \textit{quem fez a quarta classe [...] quer dizer que entende muita coisa }\\
    ‘those who completed the fourth class [...] it means that they were able to understand a lot [of the Portuguese language]’
    
    \ex \label{ex:gut46}
    \textit{já é muita coisa!}\\
    ‘that’s too much!’
\z

Interestingly, we often find \textit{muita} instead of \textit{muito} \textasciitilde\ \textit{muita coisa}: 

\ea \label{ex:gut47}
    \textit{dialeto é muita, agora o português so um bocado}\\
    ‘the dialect is a lot [=Kiyombe I know a lot], but I only know a little Portuguese’
    
    \ex \label{ex:gut48}
    \textit{papá não sabe muita, não sabe não}\\
    ‘dad doesn’t remember much [about that time], he just doesn’t remember much’
\z

Although it cannot be ruled out that the form \textit{muita} results from the neutralization of gender morphology, this hypothesis is rather unlikely, since my data clearly suggest that the masculine form prevails over the feminine in those cases in which the marking of grammatical gender has been restructured (as in the case of \textit{um coisa}). An alternative explanation is that \textit{muita} results from the routinization (grammaticalization) of \textit{muita coisa} in the restructured varieties of those speakers whose bilingualism leans towards regional Portuguese. I get this idea, partly, from examples like (\ref{ex:gut48}), a real case of an interview with an old father (91 years) and his son (60 years) in the ethnic neighborhood of the Bassolongo in the city of Cabinda (the Bassolongo are the indigenous group from the province of Zaïre, in this case from the city of Soyo): Whereas the father, who learned Portuguese during the colonial period, alternated between generic-noun-based indefinites and special indefinites, the son also produced examples like (\ref{ex:gut48}). Although both father and son are bilingual, it can be assumed that the latter was always a much more active “user” of Portuguese, since he did not really get to use Kikongo in his everyday Cabindan life (his family dialect, Kissolongo, is different from the one that is predominant in the city of Cabinda, Iwoyo). This hypothesis~-- which I note in a provisional way here -- would consist in the following chain of changes: (1) \textit{muito} > (2) \textit{muita coisa} > (3) \textit{muita}, where (1) would be the Standard Portuguese form, as learned at school today and at colonial times, (2) would be the prototypical variant resulting from language restructuing under the influence of Kikongo and (3) the variant used in the actual nativized variety of CP, which has -- partially, at least -- come into being out of the Kikongo-influenced restructured varieties of Portuguese.

Unlike PAL \textit{kusa}, CP \textit{coisa} does not seem to be used with a negative meaning (which does not mean that this use will be registered one day). What can in fact be often found -- at least, among older informants -- is the use of the NP \textit{um(a) coisa} in emphatic negative contexts (related to counter-expectation):

\ea \label{ex:gut49}
    \textit{oh: não vou te dizer uma coisa}\\
    ‘well, I am not going to tell you anything’
    
    \ex \label{ex:gut50}
    \textit{não trouxe uma coisa para mim?}\\
    ‘did you not bring me anything?’
\z

In these examples, the speaker is not only negating the propositional content of the sentence but also an inference: by using \textit{uma coisa} in (\ref{ex:gut49}), the speaker is explicitly contradicting the assumption made by the interlocutor (myself) that he was willing to tell me some anecdote of his life. As for the speaker of (\ref{ex:gut50}), he is not just asking me whether I did bring something or not, but also emphasizing that, under the circumstances of the conversation (and being already halfway through the conversation), I should in fact give something (a gift, or money).

For the most part, however, the indefinite with the meaning of ‘nothing’ in CP is \textit{nada}, i.e. the special indefinite that constitutes the canonical negative indefinite expression throughout Portuguese varieties: 

\ea \label{ex:gut51}
    \textit{ta rir é por qué? você que não sabe nada }\\
    ‘why are you laughing? You are the one who knows nothing’
\z

It has to be noted, though, that \textit{nada} can also be used in ways that are far from Standard Portuguese: these include its use as an “anaphoric” extra-sentential negator (= ‘no’) (\cite[][51--52]{Bosque1989}, \cite{Zanuttini1990}), as in answers to either positive or negative questions (s. examples (\ref{ex:gut52}) and (\ref{ex:gut53}), respectively), and its use as a second sentential negator (= ‘not’), which is placed in the sentence-final position (examples \ref{ex:gut54}--\ref{ex:gut55}). The latter may well be derived from the former and builds a special type of negation pattern, which is not only an alternative to the canonical Ibero\hyp Romance negation pattern (preverbal negation), but also to the double negation of the type “\textit{não} + V (+ X) + \textit{não}” (this is also found in some Cabindan speakers and has been extensively described in other Ibero\hyp Romance varieties like, most notably, Brazilian Portuguese and Palenquero; \cite{Schwegler2016b}, \cite{Schwenter2016}). Utterances with \textit{nada} appear to be somewhat more emphatic than those with \textit{não}. However, a closer look at the functional limits between both negators, \textit{não} and \textit{nada}, as well as the study of the reasons that led to the emergence of this very dichotomy of uses would be far beyond the scope of this chapter.\footnote{Suffice it to say that I consider this linguistic change to be related to language contact (in our particular case, with Kikongo, which formally distinguishes between anaphoric ‘no’ and sentential ‘not’ and also has double sentential negation) as well as to natural outcomes of L1-acquisition (after all, in the first stages of L1-acquisition there is only extra-sentential negators, which may develop later into sentential negators; \cite{Cameron-FaulknerThiekston2007}).}

\ea \label{ex:gut52}
    \textit{A: uma antiga doença era o} beri-beri, \textit{ouviram alguma vez} beri-beri\textit{?; B: nada!}\\
    ‘A: \textit{beri-beri} was an ancient disease, have you ever heard \textit{beri-beri}?; B: no!’  
  
    \ex \label{ex:gut53}
    \textit{A: Aqui não havia portugueses? B: na:da! havia, mas quer dizer eles aqui não pagava[m] imposto }\\
    ‘A: Were there no Portuguese here? B: no! there were some, but I mean that they didn’t have to pay taxes here’
    
    \ex \label{ex:gut54}
    \textit{mas aquela pessoa lhe deram tiro, não é? não morreu nada! }\\
    ‘[in that telenovela] they shot that guy, right? but he did not die!’
    
    \ex \label{ex:gut55} 
    \textit{A: não seria ``kibanga" em dialeto?; B: esta palavra eu não ouvi nada }\\
    ‘A: would that not be called kibanga in your dialect [=Kisundi]?; B: this word I have not heard’
\z

I will not elaborate here on the multiple uses (some more grammaticalized than others) that \textit{coisa} can adopt outside the domain of indefinites. However, I would like to note that \textit{coiso}, which is well known in many Portuguese varieties for referring to objects or people that speakers cannot -- or do not want to -- name,\footnote{The derivation of \textit{coiso} and even \textit{coisar} as a verb (‘to make something’), sometimes with sexual connotations, gives an idea of the wide spectrum of uses of the generic noun \textit{coisa} in many Portuguese varieties. } is extremely common in Angola (including Cabinda), where it has also developed as a sort of hesitation marker:\footnote{Example (\ref{ex:gut56}) was registered in naturalistic speech. It is noteworthy, too, that the examples that my informants made up when asked about the use of \textit{coiso} consisted, first and foremost, of its use as a hesitation marker, like \textit{Eu estava falá com... coiso... coiso... o Miguel!} ‘I was speaking to...eh.. eh... Miguel!’}

\ea \label{ex:gut56}
    \textit{município de... município de... tangente, de... coiso de... como lhe chamam aí?.. de... não sei... de Cabinda... não é?... é me[s]mo de... de Cabinda}\\
    ‘municipality of... municipality of… bordering on... like... how is it called? not sure... Cabinda, right? it is just Cabinda’
\z

Whether this particular use (which is certainly known in other varieties of Portuguese and Spanish (\textit{coso}), but maybe not to the exact same extent as it is in Angola) can be somewhat related to Kikongo and/or other Bantu languages requires further research.

As regards personal indefinites, we again find variation between generic-noun-based and special indefinites (\textit{alguém}) in CP. The generic noun that adopts the indefinite reading is \textit{pessoa} ‘person’ and not \textit{gente} ‘people’ (as has been shown in the cases of PAL and Kikongo, the limits between the indefinite and the generic reading are sometimes blurred; see also (\ref{ex:gut59}) further below). Even though the examples (\ref{ex:gut57}) and (\ref{ex:gut58}) are relatively similar as regards their propositional content and fit well into an existential quantificational reading (in both cases, it is about finding someone who can or cannot speak a given language), we also notice that, in some contexts at least, \textit{pessoa} can alternate with \textit{um(a) pessoa} quite freely (the data collected so far cannot help us to distinguish the precise semantic nuances of each form):

\ea \label{ex:gut57}
    \textit{é difícil você encontrar um pessoa da República Democrática do Congo conversar em francês}\\
    ‘it is difficult that you find someone (/a person) from the Democratic Republic of the Congo talking in French’
    
    \ex \label{ex:gut58}
    \textit{de vez em quando vais encontrá pessoa que entenham [sic] português, mai... mai só falem fiôte}\\
    ‘sometimes you are going to find someone/people who understands Portuguese, but only speaks Fiote (=general designation for the Cabindan dialects of Kikongo)’
\z

\begin{sloppypar}
The preference of \textit{pessoa} over \textit{gente} for indefinite expressions is not surprising from a universalistic perspective, but it is in fact surprising inasmuch as \textit{gente} is the selected generic noun for the making of indefinite pronouns meaning ‘someone’ in nearly all Niger\hyp Congo\slash Portuguese Creoles (cf. Cape Verdean \textit{algun djenti} < Port. \textit{alguma gente}, Sãotomense \textit{ũa ngê} < Port. \textit{uma gente}, etc.), including Papiamentu (even though in this case the generic noun seems to have been relexified by its cognate \textit{gente /hente/} in (Caribbean) Spanish: Pap. \textit{un hende} ‘someone\slash somebody’\footnote{Throughout Portuguese-based Creoles, it seems to be only Batavia Creole that chose the noun \textit{pessoa} from the lexifier: \textit{alung pesua} ‘somebody’.})  \citep{HaspelmathAPiCS2013}. In CP, \textit{gente} keeps being used as a kind of arbitrary generic noun, as in Standard European Portuguese. It is also interesting to note that, when my informants were asked to translate \textit{ba(n)tu} (plural of \textit{mu(n)tu} ‘person’) from Kikongo to Portuguese, it was by far \textit{pessoas}, not \textit{gente}, the first word that they could think of.\footnote{I learned this the hard way, so to speak: during my first interviews in Cabinda, I included a list of KIK words that speakers had to translate to Portuguese, in order for me to be able to elicit specific phonetic features. \textit{Batu} was on the list; I thought it would be translated with \textit{gente}, which would allow me to register realization of of /t/ before palatal vowel in Angolan Portuguese. However, I generally got \textit{pessoa[s]} when I was looking for \textit{gente}.}
\end{sloppypar}

In a similar fashion to PAL \textit{hende} (< \textit{gente}) in its nominal use, the status of \textit{pessoa} in CP can also work as a generic subject:
  
\ea \label{ex:gut59}
    \textit{aqui não tem, comboio aqui só [pode] encontrar pessoa nas outra[s] províncias}\\
    ‘here [in the province of Cabinda] there isn’t any... Trains here [in Angola] you/one can only find in the other provinces (/...people can only find...)’
\z

To complete the parallelism with PAL, it has to be noted that \textit{pessoa} can sometimes adopt an inclusive reading and therefore function as a first person plural expression (\textit{pessoa entende} = `people [here] understand'\slash`we understand'). In (\ref{ex:gut60}) we also see the active use of a special indefinite (\textit{alguns}) alongside the generic\slash arbitrary noun \textit{gente} ‘people’ which, since it is modified by \textit{daqui} ‘from here’, includes the reference to the speaker too).

\ea \label{ex:gut60}
    \textit{A: esses congoleses que vêm para cá, eles vêm...eles não falam português, não é?}\\
    ‘A: those Congolese who come here, they come... they don’t speak Portuguese, do they?’\\
    \textit{B: alguns fala[m]}\\
    ‘B: some do’\\
    \textit{A: ah!}\\
    ‘A: ah!’ \\
    \textit{B: aprende[m] e sabe[m] falar...Alguns mesmos fala[m] língua deles; como é próximo daqui, pessoa também entende}\\
    ‘they learn the language and speak... Some speak their language [Lingala] and, since the Congo is close to us, people here can also understand it’\\
    \textit{A: aha...e francês também, não é?}\\
    ‘A: aha! and French they speak too, don’t they?’\\
    \textit{B: yeah, francês, francês, só que é complicado para a gente daqui}\\
    ‘B: yeah, French, French also, but French is difficult for the people here’
\z

Finally, \textit{pessoa} can be used as a negative expression in some contexts. In the following example, we observe the alternation between \textit{pessoa} and \textit{ninguém} (the canonical negative indefinite): 

\ea \label{ex:gut61}
    \gll não pode estar pessoa sem ninguém trabalhá \\
    \textsc{neg} can be anyome without no.one work\\
    \glt ‘no one can stay [here] without working [/if no one works]’
\z

The main findings of this section are summarized in \tabref{tab:gut5}.

\begin{table}
    \small
\begin{tabularx}{\textwidth}{llQQ} 
    \lsptoprule 
    {[$-$ personal]} & {[affirmative]} & \textit{uma/alguma/qualquer coisa} \newline \~{} \textit{algo} (out-group, polite) & ‘something’ \\
    \tablevspace
    &  & \textit{tudo} & ‘everything’\\
    \tablevspace
    &  & \textit{muita coisa \~{} muita} \textit{\~{} muito} & ‘a lot’\\
    \tablevspace
    &  & \textit{pouca coisa \~{} pouco} & ‘few, little’\\
    \tablevspace
    & {[negative]} & \textit{nada} & ‘nothing’\\
    \tablevspace
    &  & (\textsc{neg} +) \textit{nada} \newline \~{} (\textsc{neg} +) \textit{uma coisa} \newline (counter-expectation) & ‘(not...) anything’ \\
    \tablevspace
    {[+ personal]} & {[affirmative]} & \textit{pessoa \~{} uma pessoa} \newline \textit{\~{} alguém}& ‘someone’ \\
    \tablevspace
    &  & \textit{todo o mundo \~{} toda a gente} \textit{\~{} todos} & ‘everyone’\\ 
    \tablevspace
    &  & \textit{muita gente} & ‘a lot of people’\\
    &  & \textit{pouca gente} & ‘a few people’ \\
    \tablevspace
    & {[negative]} & \textit{ninguém} & ‘nobody’\\
    \tablevspace
    &  &  (\textsc{neg} +) \textit{pessoa} & \~{} (\textsc{neg} +) \textit{ninguém} ‘(not...) anyone’\\
    \lspbottomrule
    \end{tabularx}
    \caption{Indefinite expressions found to date in CP}
    \label{tab:gut5}
\end{table}

\section{Summary and conclusions}\label{sec:gut5}

The data presented in this chapter are interesting in several ways: on the one hand, I have briefly described three varieties that have been insufficiently studied: Palenquero Creole, the “fossilized” learner varieties (\textit{variedades não nativas}, \cite{Gonçalves2010}) of Portuguese as spoken in the Angolan province of Cabinda, and, to a lesser extent, the Kiyombe dialect, which has traditionally been less well described than, for example, Southern Kikongo. On the other hand, I have analyzed a specific grammatical phenomenon from several perspectives, including that of the realization vs. non-realization of the very phenomenon; In other words, I have identified a series of linguistic variables that should be further investigated by future research (for example, \textit{un kusa} vs. \textit{aggú(n) kusa} ‘something’ or \textit{kusa} vs. \textit{ná} ‘nothing’ in Palenquero, \textit{mu(n)tu mosi} vs. \textit{mu(n)tu} ‘someone’ in Kikongo, \textit{muito} vs. \textit{muita coisa} vs. \textit{muita} ‘a lot’ in Cabindan Portuguese, etc.).

Nevertheless, I have collected all this data to make a (modest) contribution to the field of contact linguistics. More specifically, the process of creolization has been considered in terms of its relationship to the broader process of second-language acquisition, and both processes have been characterized here regarding the extent to which they are determined by the influence of substrate languages (L1) \citep[cf.][]{Winford2008, Winford2012}.

My line of reasoning consists of several steps:\largerpage

\begin{enumerate}
    \item[(1)] The use of generic-noun-based indefinites reveals itself as a much more frequent and idiomatic strategy in PAL than in its lexifier (Spanish); moreover, some indefinites of this type would surely be impossible in Spanish (starting with the example of \textit{un poko kusa} in the introductory section).
    \item[(2)] If we ask ourselves how this structural difference between the Creole and its lexifier emerged, we must pay attention to the three components of creolization: the linguistic universals (during first- and second-language acquisition), the further development of structural tendencies already existing in the lexifier, and the substratal influence. In any case, it is assumed that all three components of creolization always interact to some extent: according to \citet[][1]{NeumannHolzschuhSchneider2000}, “substrates and superstrates appear to offer structural possibilities from which elements of emerging structures are selected on the basis of universal preferences, typological affiliation or formal similarities”. In theory, at least, there is still another possibility, according to which PAL did not develop generic-based-noun indefinites right from the beginning (during its early formative period) but only at a later stage of its history -- a linguistic history that has already lasted about four centuries; unfortunately, this line of research is not feasible, since we do not have any written manifestation of PAL until the second half of the 20\textsuperscript{th} century (there is no such thing as “Palenquero historical linguistics” -- at least not yet\footnote{See, however, the texts analyzed by \citet{GutiérrezMaté2012}.}). However, it should not go unnoticed that PAL exhibits some grammatical uses of generic nouns that seem to be pretty much idiosyncratic (unknown to both substrate and superstrate, as well as to the universal tendencies of creolization described in the specialized literature) (see examples like (\ref{ex:gut18})). A little speculation: if, perhaps, the Creole developed such grammatical uses of generic nouns on its own, why could it not also have developed some other uses like generic-noun-based indefinites?
    \item[(3)] Each of the three components of creolization could, to some extent, account for the use of generic-noun-based indefinites in PAL in Creole (we exclude here the fourth kind of explanation outlined above, since there is no way to check it empirically). The universal preference for analytical structures in interlanguages may have triggered the aversion to special indefinites in language contact varieties. The avoidance of irregular morphology (like that of personal indefinite pronominals with the endings -\textit{ie}\slash-\textit{ien}) can also be seen as the result of typical interlanguage developments. In addition, it could be assumed that the universals of L1 acquisition favor the generalized use of the canonical phrase structure [\textsc{detp} Det [\textsc{np} N (Adj)]], which, again, leads to the overt use of nominal heads (such as \textit{kusa} in \textit{mucho kusa}). As for superstratal influence, it is noteworthy that Colonial Caribbean Spanish (as well as, surely, other varieties of Spanish, which, however, did not play any role in the formation of PAL) occasionally exhibited generic nouns with a meaning close to that of an indefinite pronoun; furthermore, we know other Romance languages that have generalized this type of indefinites (cf. French \textit{quelque chose}, Port. \textit{alguma coisa}, etc.) without apparently being influenced by other languages. Finally, in many Creoles, and most certainly in the one studied here, the substrate may have conditioned the use of generic-noun-based indefinites in the Creole, insofar as this is also the canonical strategy for the formation of indefinite expressions in Kikongo.
    \item[(4)] At first sight, it might seem that the very nature of the grammatical phenomenon studied here prevents the isolation of the real effect of substratal influence from the other two components of creolization: after all, generic nouns can easily adopt other semantic values in discourse, including that of indefinite expressions, which already lays the foundation for the linguistic change “generic noun > indefinite pronoun” (with no need to think of language contact). Nevertheless, there are two facts that give rise to the suspicion of substratal influence being the ultimate trigger of the linguistic change analyzed here: first, according to the WALS, only ¼ of the world’s languages employ generic nouns as the main strategy for the formation of indefinite expressions; ¼ is certainly a not negligible figure, but if Creole language structures were the result of genuinely “creative” universal changes, not conditioned in any way by the contributing languages, it would have been more likely that PAL would have developed the most common type of indefinites in universal terms, i.e. interrogative-based-indefinites, which are present in 60\% of the world’s languages. Secondly, if we focus on the contributing languages, it has to be acknowledged that no variety of Spanish seem to make a predominant use of generic nouns (most certainly not to the point of having displaced some special indefinites, as  happens in the case of the PAL); that is, if the change “generic noun > indefinite pronoun” were already somehow anticipated in Spanish itself, we would expect that some other Hispanic vernaculars would have spontaneously (without the conditioning of other languages) reached the same result of PAL. However, as far as I know, there is no such Hispanic variety. Finally, it has to be noted that understanding the origins of PAL indefinite expressions as the result of Kikongo language transfer is consistent with the general tendency found throughout Creole languages; according to this, Creoles adopt, to a large extent, their preferred type of indefinite pronouns from substrate languages (so it seems, at least, when comparing the materials of the WALS with those of the APICS).
    \item[(5)] The previous point allows us to assume that Kikongo is primarily responsible for the formation of PAL generic\hyp noun\hyp based indefinites (out of Spanish lexical materials). Thus, if we accept this hypothesis as valid, we are in a position to study the “transferability” of Kikongo indefinites also as a measure to compare different types of varieties resulting from the Ibero\hyp Romance\slash Kikongo language contact with each other and, relying on such comparison, to even set some quantitative and qualitative limits between such varieties (partially restructured vs. Creoles). It is this objective that leads us to take CP into account: as a matter of fact, PAL and CP are one of the very few cases worldwide in which we can find a Creole and a restructured variety that have in common both their substrate and their superstrate (if we accepted that the corresponding dialect continua – respectively, Kikongo and Ibero\hyp Romance – are homogeneous enough as to consider them as typological unities).
    \item[(6)] Unlike Spanish, Portuguese has developed towards the “mixed” type of indefinites, so that some generic-based-noun indefinites (like \textit{alguma coisa}) have partially displaced some special indefinites (like \textit{algo}); consequently, the typological unity of the two Ibero\hyp Romance languages is not complete. However, Portuguese still retains some of its special indefinites (\textit{alguém, ninguém, tudo,} etc.), which is enough to analyze possible restructuring phenomena. As for the particular case of CP, this variety does not exhibit uses of generic-based-noun indefinites that are really unknown in other varieties: even those uses that seem to “diverge” the most from Standard Portuguese varieties, like the example (\ref{ex:gut41}) \textit{Tem coisa para falar?}, can be heard, for instance, in Brazil. At the current state of this investigation, it is unclear whether uses like (\ref{ex:gut41}), or the frequent use of \textit{muita coisa}, or the common use of \textit{(uma) pessoa} instead of \textit{alguém} are really more frequent in Cabinda than in other lusophone regions; even though the answer to this question may be positive (this is also my first intuition about it), further research is much needed. Be that as it may, there is no doubt that special indefinites (\textit{alguém, nada, tudo,} etc. and, in some contexts at least, \textit{algo}: cf. \ref{ex:gut39}--\ref{ex:gut40}) are also relatively common in CP, even among elderly people. This is a remarkable difference between CP and PAL, which is surely related to the fact that there was far more contact with the superstrate in late-colonial and post-colonial Cabinda than in San Basilio de Palenque at the time of Creole formation (and this difference is independent of the fact that Portuguese, unlike Spanish, already makes general use of the generic-noun-based indefinite meaning ‘something’). Since contact with the non-restructured version of the European language was, to some extent, available, structural simplification (imposed by L2 acquisition) and linguistic “creativity” (during L1 acquisition) played a far less significant role in CP than in PAL. In addition, the language ecology of Cabinda is different from that of SBP for another reason: interlanguages were never “good” from an adaptive and evolutionary perspective \citep{Mufwene2001} and never became part of a new identity, the kind of “Creole identity” -- neither European nor exactly African\footnote{Today, SBP is considered to be \textit{un chito ri Afrika andi Amerika} (`a small piece of Africa in America') by many \textit{Palenqueros} (and by all local tourist guides), but this perception results from a simplification (and a re-ideologization) of the traditional Palenquero identity. It is very doubtful that the founders of SBP, which did not all come directly from Africa, would have had such a perception. If it were so, we would not be able to explain why Kikongo is no longer spoken in the village. In this regard, it should be remembered that the village arose from a gradual coming together of multiple \textit{palenques} (‘maroon communities’), all located in the nearby Sierra de María and/or the neighboring region opposite to the shores of the Magdalena river \citep{Navarrete2008, GutiérrezMaté2016}. This gradual process surely extends from the end of the 16\textsuperscript{th} century to the end of the 17\textsuperscript{th} century, that is, from the foundation of the so-called “palenque del Limón” (ca. 1580--1634) to the foundation of the palenque de San Basilio (from the remains of the palenque “San Miguel Arcángel”) or even to the peace treaty between Palenqueros and Hispanic civil authorities (1713--1714). According to this treaty, which I have recently consulted in the Archivo General de Indias, approximately half of the founders of SBP were “negros criollos”, i.e. Black Creoles -- born in the mountains, or in the haciendas nearby Cartagena. The socio-identitarian processes that took place in colonial \textit{palenques} were actually very complex, and the formation of SBP was even more idiosyncratic, insofar as Palenque was the first maroon community to obtain its freedom from a colonial administration.} -- that we find in SBP.
\end{enumerate}

This study lays the foundation for future qualitative and quantitative research on the loss of special indefinites in restructured Ibero\hyp Romance varieties spoken in current or former multilingual scenarios. Future studies will also have to solve some of the structural and variational issues that this work has not discussed. At a structural level, this chapter has not analyzed the distribution of generic-noun-based indefinites with indefinite determiners (‘a’, ‘some’, ‘any’); however, it could be assumed that this type of indefinites (like PAL \textit{un kusa} ‘something’) are a compromise (or convergence) between the substrate’s preferred strategy for building indefinite pronouns (i.e. bare generic-noun-based indefinites) and a secondary strategy used in Ibero\hyp Romance for the building of indefinite expressions (i.e. generic nouns modified by indefinite determiners). At a variational level, there are still other problems that need to be addressed in the future: as regards PAL, it is unclear how we can distinguish code-switched elements (including special indefinites) from Spanish elements that have been incorporated in the Creole (perhaps even centuries ago); as regards CP, it will be necessary to account for the fact that different sociolinguistic variables (age, gender, level of literacy, time of exposure to Portuguese, etc.) and idiolectal preferences result in different linguistic data. In this regard, the generalizations made throughout this chapter have to be understood within the wider context of a research project that deals with typological change from the perspective of language contact \citep{GutMat}, especially in those cases in which different contact varieties have the same substrate and the same superstrate.


\section*{Acknowledgments}
I would like to express my gratitude to the people and institutions that helped me obtain the necessary data to show the conclusions of this study. On the one hand, the collection of the data from Cabinda was possible thanks to two fieldwork trips funded by the \textit{Förderprogramm für den wissenschaftlichen Nachwuchs} of the University of Augsburg. On the other hand, obtaining the Palenquero data was possible thanks to Armin Schwegler, who showed his generosity in two aspects: firstly, he granted me access to his initial recordings in Palenque, which are especially valuable today, since they depict traditional varieties of Palenquero (these recordings formed the corpus of my postdoctoral project at the University of California, Irvine, in 2014 and 2015, funded by the the P.R.I.M.E. program of the German Academic Exchange Service and the \textit{European Research Council}); secondly, he facilitated contact with the locals, especially with his friend Víctor Simarra, who was an excellent (and necessary) collaborator available at all times during the interviews I made in the village in the summer of 2017. Lastly, I would like to thank Abel Massiala, my favorite informant from the Mayombe region and a great collaborator during my research stays in Cabinda, and Maximilian Rieder, a Master’s student who is working as a student research assistant (Ger. HiWi) for my project on Angolan Portuguese and has been hired thanks to the Philological\hyp Historical Faculty of the University of Augsburg (through its \textit{Haushaltsmittel für Forschungsvorhaben nach Typ A}).

\largerpage[2]
{\sloppy\printbibliography[heading=subbibliography,notkeyword=this]}
\end{document}
