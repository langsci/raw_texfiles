\documentclass[output=paper,colorlinks,citecolor=brown]{langscibook}
\ChapterDOI{10.5281/zenodo.13759990}
\author{Guido Mensching\orcid{}\affiliation{Georg-August-Universität Göttingen}}

\title{Indefinites and quantifiers in Old Sardinian: A corpus-based study}

\abstract{The Sardinian systems of indefinites and quantifiers are interesting, among other reasons, because of the great proportion of loans from (Old) Italian, Catalan and Spanish -- besides the preservation of a small number of Latin “archaisms” -- as well as because of some interesting word order and agreement-related phenomena. As far as diachrony is concerned, a systematic analysis of Old Sardinian indefinites and quantifiers has never been undertaken. This chapter is based on a study performed by using the new corpus ATLiSOr, which has been available since 2017, and presents some first results on Old Sardinian negative indefinites and universal quantifiers. In particular, it turns out that the distribution and the frequency of some indefinites and quantifiers provide new insights into the issue of whether these elements are loans or whether they are inherited from Latin. The chapter also contains some first insights into the syntax of the items at issue and particularly examines the agreement behavior of \textit{tot(t)u} ‘all’, a quantifier that is mostly invariable in Modern Sardinian. The study shows that agreement of adnominal \textit{tot(t)u} was still optional in Old Sardinian and points out an interesting exception, namely that agreement was obligatory when \textit{tot(t)u} was followed by a numeral.}

\IfFileExists{../localcommands.tex}{
   \addbibresource{../localbibliography.bib}
   \usepackage{langsci-optional}
\usepackage{langsci-gb4e}
\usepackage{langsci-lgr}

\usepackage{listings}
\lstset{basicstyle=\ttfamily,tabsize=2,breaklines=true}

%added by author
% \usepackage{tipa}
\usepackage{multirow}
\graphicspath{{figures/}}
\usepackage{langsci-branding}

   
\newcommand{\sent}{\enumsentence}
\newcommand{\sents}{\eenumsentence}
\let\citeasnoun\citet

\renewcommand{\lsCoverTitleFont}[1]{\sffamily\addfontfeatures{Scale=MatchUppercase}\fontsize{44pt}{16mm}\selectfont #1}
  
   %% hyphenation points for line breaks
%% Normally, automatic hyphenation in LaTeX is very good
%% If a word is mis-hyphenated, add it to this file
%%
%% add information to TeX file before \begin{document} with:
%% %% hyphenation points for line breaks
%% Normally, automatic hyphenation in LaTeX is very good
%% If a word is mis-hyphenated, add it to this file
%%
%% add information to TeX file before \begin{document} with:
%% %% hyphenation points for line breaks
%% Normally, automatic hyphenation in LaTeX is very good
%% If a word is mis-hyphenated, add it to this file
%%
%% add information to TeX file before \begin{document} with:
%% \include{localhyphenation}
\hyphenation{
affri-ca-te
affri-ca-tes
an-no-tated
com-ple-ments
com-po-si-tio-na-li-ty
non-com-po-si-tio-na-li-ty
Gon-zá-lez
out-side
Ri-chárd
se-man-tics
STREU-SLE
Tie-de-mann
}
\hyphenation{
affri-ca-te
affri-ca-tes
an-no-tated
com-ple-ments
com-po-si-tio-na-li-ty
non-com-po-si-tio-na-li-ty
Gon-zá-lez
out-side
Ri-chárd
se-man-tics
STREU-SLE
Tie-de-mann
}
\hyphenation{
affri-ca-te
affri-ca-tes
an-no-tated
com-ple-ments
com-po-si-tio-na-li-ty
non-com-po-si-tio-na-li-ty
Gon-zá-lez
out-side
Ri-chárd
se-man-tics
STREU-SLE
Tie-de-mann
}
   \boolfalse{bookcompile}
   \togglepaper[23]%%chapternumber
}{}
\begin{document}
\maketitle

\section{Introduction}\label{sec:men1}
\begin{sloppypar}
The systems of Modern Sardinian indefinites and quantifiers\footnote{\label{fn:men1}Note that, in traditional grammatical descriptions, quantifiers are subsumed under indefinites, but indefinites and quantifiers (and especially universal quantifiers) are usually kept apart in most modern linguistic frameworks, although both groups may share some properties and are sometimes diachronically derived from each other; see \citet[11--13]{Haspelmath1997} for discussion. For generative frameworks, see \citet{Heim1982}, \citet{BeghelliStowell1997}, and \citet{Szabolcsi1997}. I would like to thank an anonymous reviewer for providing these references.} are quite well known, mostly thanks to \citeauthor{Jones1993}’s (\citeyear{Jones1993}) \textit{Sardinian Syntax}; for Modern Sardinian indefinites, also see \citet{Mensching2005}. In contrast, very little research has been done on Old Sardinian, apart from  \citet{MeyerLübke1902}, who dedicated one paragraph to indefinites (\citeyear[40--41]{MeyerLübke1902}), \citeauthor{Wagner1938}’s (\citeyear{Wagner1938}) \textit{Flessione nominale del sardo antico e moderno}, where indefinites are dealt with on only four pages (\citeyear[128--132, §§40--46]{Wagner1938}), and a section of \citeauthor{BlascoFerrer2003}’s (\citeyear{BlascoFerrer2003}) analysis of the texts included in his anthology of Old Sardinian documents (\citeyear{BlascoFerrer2003}: 207--208, §39: “Quantificatori”).\footnote{In these works, quantifiers are treated together with indefinites; see footnote~\ref{fn:men1}.} These contributions mostly bear on the inventory of forms and their origin, including the issue of borrowing, and do not say much on syntax. 
\end{sloppypar}

Within the study of Romance indefinites and quantifiers, Sardinian is particularly interesting for at least two reasons: firstly, this language presents a remarkable number of indefinites and quantifiers that are loanwords from superstratum languages (Italian, Spanish, and Catalan), secondly, because some quantifiers show striking positional and agreement properties. 

A thorough analysis of Old Sardinian has become possible only recently, after the online publication of \CiteShortHand{ATLiSOr2017} (\textit{Corpus ATLiSOr: Archivio Testuale della Lingua Sarda delle Origini}) by Giovanni Lupinu in 2017. The aim of this chapter is to provide some first, mostly descriptive, results of a corpus analysis of  Old Sardinian indefinites and quantifiers. More precisely, this study aims to (i.) assess the state of the art with respect to the origin of these items, and, in particular, the issue of whether they are loans or autochthonous elements, and (ii.) to thoroughly describe the syntax of some of these items for the first time.\largerpage

The chapter is organized as follows:  \sectref{sec:men2} contains some information on Sardinian (\sectref{sec:men2.1}) and some notes on the state of research on Sardinian indefinites and quantifiers (\sectref{sec:men2.2}). \sectref{sec:men3} is concerned with the negative indefinites \textit{nemo(s)} ‘nobody’ and \textit{perunu}, \textit{niunu}/\textit{neunu}, \textit{nixunu}/\textit{nexunu} ‘nobody/no (X)’. \sectref{sec:men4} focusses on the universal quantifiers \textit{cada} and \textit{omnia/omni/ogni} ‘every/each’ as well as \textit{omnes} and \textit{tot(t)u} ‘all’. The element \textit{tot(t)u}, which is used to express the meaning of ‘all’ in Modern Sardinian, is widely documented in the medieval Sardinian texts. Today, it presents the striking property of being mostly uninflected for gender and number, a trait that has never been the focus of any study of Old Sardinian.  \sectref{sec:men5} therefore contains a study of the agreement properties of Old Sardinian \textit{tot(t)u}. The results of this study will show that agreement of \textit{tot(t)u} was still optional in Old Sardinian, except when it combines with numerals. This property will be discussed at the end of \sectref{sec:men5} from a typological perspective. 


\section{Sardinian}\label{sec:men2}\largerpage[2]

\subsection{A brief history of Sardinian}\label{sec:men2.1}
Sardinian is the Romance language that developed after Sardinia came under Roman rule as a result of the First Punic War. It has survived until today in two main dialect groups (Campidanese in the south and Logudorese in the north, the latter including the linguistically conservative Central Sardinian or Nuorese). All varieties of Sardinian are in diglossia with the official language, Italian. The number of speakers is estimated to be approximately 1 million (cf. \cite{Moseley2007}).

Sardinia belonged to the Byzantine Empire from the seventh century, but was neglected by Byzantium due to  attacks by the Saracens, which led to the development of independent political structures, the so-called Judicates of Cagliari, Torres, Arborea, and Gallura. The Saracen raids were finally stopped with the aid of Pisa and Genoa, who extended their power on the island, so that the Judicates (except Arborea) lost their autonomy (cf. \cite[][270]{MenschingRemberger2016}). The Italian dialects of these two city-states (i.e. Pisan, a variety of Tuscan, and Ligurian) constitute a first important superstratum for Sardinian, the second being Catalan, when the island was passed to the Crown of Aragon at the beginning of the fourteenth century. Catalan, in turn, was followed by (Castilian) Spanish after the unification of Castile and Aragon in 1479. In 1718, Sardinia was given to the House of Savoy and thus to Piedmont and to united Italy in 1861, leading to today’s diglossic situation and a heavy linguistic influence through Italian (cf. \cite{Rindler-Schjerve1987}; for further references see \cite[][270--217]{MenschingRemberger2016}).

Old Sardinian is documented from around 1050 to around 1400. Strikingly, and in contrast to the medieval documentation of other Romance languages, the documentation is exclusively legal and administrative, i.e. there are no writings at all of other genres, such as literary texts \citep[][80--83]{Wagner1997}. According to \citet[][250--251]{BlascoFerrer1995}, these documents can be divided into three types: (i.) letters of the chancelleries of the four Judicates, mostly containing legacies and donations; (ii.) the \textit{condaghes}, which are proceedings of transactions concerning property assets and housing stock of monasteries; (iii.) codifications of laws and municipal ordinances. The \CiteShortHand{ATLiSOr2017} corpus covers the whole documentation of all three types of documents. 


\subsection{Old and Modern Sardinian indefinites and quantifiers}\label{sec:men2.2}
\tabref{tab:men1} shows some indefinites and quantifiers of Modern Sardinian together with their origin as assumed in Wagner’s \textit{Dizionario Etimologico Sardo} (\CiteShortHand{DES}) and additional literature.

\begin{table}
\small
\begin{tabularx}{\textwidth}{lQ}
\lsptoprule
    Inherited from Latin &  \\
    \midrule
    \textit{tottu} ‘all' & < Late Latin \textsc{t$\breve\textsc{o}$ttum} (classical \textsc{t$\bar\textsc{o}$ttum}, \CiteShortHand{DES} 2: 500--501)\\
    \textit{nudda} ‘nothing’ & < Lat. \textsc{nulla} (\CiteShortHand{DES} 2: 175)\\
    \textit{nemos} ‘nobody’ & < Lat. \textsc{nemo} (\CiteShortHand{DES} 2: 161) \\
    \textit{donza/dogna} ‘every' & < Lat. \textsc{et omnia} (\CiteShortHand{DES} 2: 188) \\
    \textit{meta/meda} ‘much/many' & < Lat. \textsc{meta} ‘heap’ (\CiteShortHand{DES} 2: 112) \\
    \midrule 
    Loans & \\
    \midrule
    \textit{calchi/carchi} ‘some’ & < Ital. \textit{qualche} (\CiteShortHand{DES} 1: 269) \\
    \textit{cali(n)cunu} ‘some(body)’ & < Ital. \textit{qualcheduno} (\CiteShortHand{DES} 1: 269) \\
    \textit{nessunu} ‘no’ & < Ital. \textit{nessunu} (\CiteShortHand{DES} 2: 168) \\
    \textit{donzi /dogni} ‘every/each’ & < Ital. \textit{ogni} (\CiteShortHand{DES} 2: 188) \\
    \textit{tzertu/certu} ‘a certain' & < Ital. \textit{certo} (\CiteShortHand{DES} 1: 447) \\
    \midrule 
    Uncertain & \\
    \midrule
     \textit{algunu/argunu} ‘some(one)’ & < Span. \textit{alguno} (\CiteShortHand{DES} 1: 70--71), Cat. \textit{algú/algún} or Lat. \textsc{alicunus}?\\
     \textit{cada}  ‘every/each’ & < Lat. \textsc{cata} < Greek κατά (\CiteShortHand{DES} 1: 256, \CiteShortHand{MeyerLübke1992} 1755) or via Span. \textit{cada}?\\
     \textit{perunu} ‘nobody, no X’  &  < Old Italian \textit{veruno} or Lat. \textsc{per $\bar\textsc{u}$num}? (\CiteShortHand{DES} 2: 251 vs. \cite[][207]{BlascoFerrer2003})\\
     \textit{neunu/niunu} & < Old Italian \textit{neuno} or Lat. \textsc{n$\breve\textsc{e}$(c) $\bar\textsc{u}$num}? (\CiteShortHand{DES} 2: 168--169 vs. \cite[][207]{BlascoFerrer2003})\\
\lspbottomrule
\end{tabularx} \caption{Some indefinites and quantifiers of Modern Sardinian} \label{tab:men1}
\end{table}


All these items already existed in Old Sardinian, except for \textit{calchi/carchi} ‘some’. Instead of \textit{algunu} ‘some(one)’, Old Sardinian had \textit{alicunu}, which Wagner considers as inherited from Lat. \textsc{alicunus}, whereas he says that the more frequent Old Sardinian \textit{alcunu} is probably already an Italianism (\CiteShortHand{DES} 1: 70--71). The preferred form today, \textit{algunu/argunu}, comes from Spanish according to Wagner, who already documented \textit{algunu} in the \textit{Carta de Logu} of the Judicate of Arborea, a legal code issued in 1392, a date which would, however, indicate a Catalan rather than a Spanish origin. Note, however, that it is not excluded to consider both \textit{alcunu} and \textit{algunu} as inherited from Latin with a syncope of \textit{i} and preceding intervocalic sonorization in the second form. In the case of \textit{cada}, Wagner (\CiteShortHand{DES} 1: 256) claims that intervocalic sonorization occurs in the Old Sardinian texts in which he found this form (CSMB and CV), so that he sees no reason to consider it a Hispanicism. In contrast, he supposes that \textit{perunu} ‘no (X)’ stems from Ital. \textit{verunu}, maybe with the influence of \textit{per} (\CiteShortHand{DES} 2: 251), whereas \citet[][207]{BlascoFerrer2003} wants to derive it from Lat. \textsc{per $\bar\textsc{u}$num}. Similarly, Blasco Ferrer suggests that \textit{neunu} derives directly from Latin \textsc{n$\breve\textsc{e}$(c) $\bar\textsc{u}$num}, whereas Wagner believes it to be a loan from Italian. Today, \textit{nessunu} or \textit{nisciunu} are more widespread, which are clearly of Italian origin (\CiteShortHand{DES} 2: 168; \citealt[][207]{BlascoFerrer2003}). In \sectref{sec:men3}  and \sectref{sec:men4}, I will assess most of these controversial cases on the basis of corpus data.

The quantifiers \textit{meta/meda} ‘much/many’ and \textit{tot(t)u} ‘all’ show an interesting syntactic behavior in Modern Sardinian. Whereas Sardinian indefinites usually occur prenominally, the gender-invariable \textit{meta/meda} can occur both pre- and postnominally. In prenominal position, it usually agrees in number by taking an \textit{-s} in the plural, whereas for many speakers number agreement is lacking in postnominal position \citep[][36]{Jones1993}.\footnote{Examples:
 \ea metas/medas libros \\
    \ex libros meta(s)/meda(s)\\
        ‘many books’
    \z}
    For Old Sardinian, the corpus \CiteShortHand{ATLiSOr2017} has only three occurrences of \textit{meta} (see (\ref{ex:men1}) below), which had already been discovered by Wagner (\CiteShortHand{DES} 2: 112). As all are in the singular, nothing can be said about the agreement facts, but the two word orders are already attested, as can be seen from (\ref{ex:men1a}) vs. (\ref{ex:men1b}):\footnote{In this and the other examples, italics and round brackets come from the editors of the texts from which the examples were taken. For instance, in the examples in (1), the italics and round brackets stand for resolved abbreviations in the medieval manuscripts. I copied these markings as is from the \CiteShortHand{ATLiSOr2017} corpus. Please note that the corpus is composed of different editions with different standards (essentially round brackets vs. italics). Bold and underline are mine. I usually use bold to highlight the quantifiers and indefinites at issue and underline to highlight other properties mentioned in the explaining text.}

\ea \label{ex:men1}
    \ea\label{ex:men1a}(\CiteShortHand{CondSPS} 289, p. 252.2)\\
         renovo custu co(n)dake […], ki fuit de \textbf{te(m)p(us) meta}\\
        ‘I renew this \textit{condaghe}, which has existed for much time’
    \ex\label{ex:men1b}(\CiteShortHand{CondSNT} 1, p. 125.2) \\
         p\textit{ro} \textbf{meta servizu} ki lis feki\\
        ‘for the great amount of service that I rendered to them’
    \ex\label{ex:men1c}(\CiteShortHand{StSS} L. I-LXII, p. 30.24)\\
         çascatunu q(ui) aet cherre petha, \textbf{et paca (et) meta}\\
        ‘everybody who will want meat, either little or much’
    \z
\z

Contrary to \textit{meta}, the universal quantifier \textit{tot(t)u} is widely documented in the medieval texts. In Modern Sardinian, adnominal \textit{tot(t)u} is prenominal and followed by the definite article. It does not agree with the noun (cf. \cite[][37]{Jones1993}), as the following examples show:

\ea \label{ex:men2}
Modern Sardinian (Logudorese)
    \ea\label{men2a}
        \gll totu	s’	abba\\
         all	the	water-\textsc{f.sg} \\
    \ex\label{men2b}
        \gll totu	s’	ozu\\
         all	the	oil-\textsc{m.sg} \\
    \ex\label{men2c}
        \gll totu	sos	òmines\\
        all	the-\textsc{m.pl}	man-\textsc{m.pl} \\
    \ex\label{men2d}    
    \gll totu	sas	fèminas\\
        all	the-\textsc{f.pl}	woman-\textsc{f.pl}\\
    \z
\z

However, as \citet[][38]{Jones1993} observes, in some dialects “a plural form \textit{tottus} (invariable for gender) is used in some cases, in particular when this item occurs in isolation (see \citealt[270]{Farina1973}),” see \REF{ex:men3}:\footnote{Instead, there is no agreement when \textit{tot(t)u} precedes the participle:
\ea Sun tottu tuccàos.\  \citep[][38]{Jones1993}\z}

\ea\label{ex:men3}
    Modern Sardinian (Nuorese, \citealt[38]{Jones1993})\\
    \gll Sun tuccàos tottus.\\
     are	arrived-\textsc{m.pl} all-\textsc{pl}\\
\z

Whereas in Campidanese, the ending \textit{-us} corresponds to the regular masculine plural ending of nouns and adjectives with the singular in \textit{-u}, this is not the case in Logudorese and Nuorese, where the paradigm is \textit{-u} (sg.)/\textit{-os} (pl.). This is particularly interesting against the background of the diachronic data, as regular masculine and feminine plural forms  (\textit{tot(t)os/tot(t)as}) are attested in Old Sardinian. I will return to this issue in \sectref{sec:men5}.


\section{Negative indefinites}\label{sec:men3}

\subsection{Distribution in the corpus and the issue of borrowing}\label{sec:men3.1}
This section focuses on the Old Sardinian negative indefinites \textit{nemo(s)} ‘nobody’ and \textit{neunu} and \textit{nex(i)unu} ‘nobody/no (X)’ as well as \textit{nullu} and \textit{perunu} ‘no (X)’. \tabref{tab:men2} shows their distribution in the corpus.\footnote{In the left column, the texts that can be dated more or less exactly are arranged in chronological order, followed by another block of texts that extend over more than one century.}

\begin{table}
\small
    \begin{tabularx}{\textwidth}{Qccccc}
\lsptoprule
     & \textit{nemo(s)} & \textit{nullu} & \textit{perunu} & \textit{neunu} & \textit{nex(i)unu}  \\
    \midrule
    \CiteShortHand{CVolgAAC1} (1066--1074) &  & 1 & 4 &   &  \\
   
    \CiteShortHand{PrivLog} (1080--1085) &  & 1 &  &  &  \\

    \CiteShortHand{CartaarbGen2} (1112--1120) &  & 1 &  &  &  \\
   
    \CiteShortHand{Montecass22} (1136) &  & 1 &  &  &  \\

    \CiteShortHand{Montecass32} (1153) &   & 2 &  &  &  \\

    \CiteShortHand{Montecass35} (1170) &   & 2 &  &  &  \\

    \CiteShortHand{Montecass39} (1182–1183 ca.?) & 1 & 3 &  &  &  \\

    \CiteShortHand{Cartaarb} (1184) &  &  & 1  &  &  \\

    \CiteShortHand{CondBar} (1190) &  &  &   & 2 &  \\

    \CiteShortHand{Cartadon} (1211) &  &  & 5  &  &  \\

    \CiteShortHand{CVolgAAC11–21} 11--18 (1215--1217) &  &  & 13 & 2 &  \\
  
    \CiteShortHand{CartaBen} (1225) &  &  & 4  &  &  \\

    \CiteShortHand{CVolgAAC11–21} 19--21 (1225--1226) &  &  & 14  &  &  \\
 
    \CiteShortHand{StSS} (1316) &  &  &  & 145 & 11 \\

    \CiteShortHand{StCastel} (1334--1336?) &  &  &  & 3 & 44 \\ 

    \CiteShortHand{StCastel2} (1334--1336?) &  &  &  &  & 1 \\

    \CiteShortHand{CdLA} (end of 14\textsuperscript{th} c.) &  & 1\footnotemark & 11 & 2 & 24 \\

    \midrule
    \CiteShortHand{CondSPS} (end of 11\textsuperscript{th}--13\textsuperscript{th} c.)  & 1  & 1 &  & 8 &  \\

    \CiteShortHand{CondSNT} (12\textsuperscript{th}--13\textsuperscript{th} c.) &   &  & 1 &  &\\ 

    \CiteShortHand{CondSMB} (12\textsuperscript{th}--13\textsuperscript{th} c.) &   & 6 & 10 & 1 & 1 \\

    \midrule
    Total & 2 & 19 & 63 & 163 & 81 \\
    \lspbottomrule
    \end{tabularx}
    \caption{Negative indefinites in Old Sardinian (absolute occurrences)}\label{tab:men2}
\end{table}

\footnotetext{The Italianism \textit{nullo} ‘of no value’: “siat nullo et de neguna efficacia e valo(re)” (‘it shall be of no value and of no use’, \CiteShortHand{CdLA} LXXII 118, 9).}

\subsubsection{\textit{Nemo(s)}, \textit{nex(i)unu}, and \textit{neunu}}\label{sec:men3.1.1}
Sardinian belongs to the few Romance languages that still preserve Lat. \textsc{n$\bar\textsc{e}$mo} (besides Romanian, Corsican, and some Tuscan dialects; see \CiteShortHand{MeyerLübke1992} 5886 and \citealt[][81]{BertocchiMaraldiOrlandini2010}). Both the modern and the ancient form, according to the literature, is \textit{nemos} (Mod. Camp.: \textit{nemus}), where the \textit{-s} is considered as analogical to other indefinites (\textit{alikis, uniskis,} cf. \CiteShortHand{DES} 2: 161, \citealt[][131]{Wagner1938}). Wagner only cites one example, corresponding to \REF{ex:men4a} below, in which the item already appears with the analogical \textit{-s}. The corpus \CiteShortHand{ATLiSOr2017} now shows one additional example without the \textit{-s}, see \REF{ex:men4b}:

\ea\label{ex:men4}
    \ea\label{ex:men4a}(\CiteShortHand{CondSPS} 68, p. 130.5)\\
         ki non bi aet bias \textbf{nemos}\\
        ‘that nobody is entitled [to possess him]’
    \ex\label{ex:men4b}(\CiteShortHand{Montecass39})\\
        \textbf{nemo} no(n) 'de-llis levet, ni(n) ambilla ni(n) pischi\\
        ‘nobody may remove [from the river] neither eels nor fish’        
    \z
\z

Strangely enough, these are the only examples in the whole corpus. The reason for this might be that ‘nobody’ is too general for juridical texts, which tend to be precise, using indications such as ‘no man’, ‘no woman’, etc. This is actually borne out in the texts, where such expressions are found in the majority of negative references to indefinite persons. Some examples are given in \REF{ex:men5}.\largerpage[2]

\ea\label{ex:men5}
        \ea\label{ex:men5a}(\CiteShortHand{CdLA} XCVII, p. 136.2)\\
         <I>t<e>m hordinam(us) q(ui) \textbf{nexuna p(er)soni} de su regnu n(ost)ru d'Arborê no(n) usit nen deppiat deseredari sos figios […]\\
        ‘Likewise, we order that no person of our Kingdom of Arborea must disinherit the sons […]’
        \ex\label{ex:men5b}(\CiteShortHand{StSS}  L. I-XLIX, p. 25.35)\\
         Vivende su maritu, \textbf{neuna muçere} sensa paraula dessu maritu suo pothat nen deppiat facher alcunu c(on)tractu\\
        ‘When the husband is alive, no woman can or may make any contract’
        \ex\label{ex:men5c}(\CiteShortHand{StSS} L. I-LXXII, p. 33.8)\\
         \textbf{Neunu barberi} radat sas d(omi)nicas nen i(n) festas solle(m)pnes\\
        ‘No barber may shave on Sundays nor on solemn holidays’
        \ex\label{ex:men5d}(\CiteShortHand{Cartadon} 85, col. 1.29) \\
         Et icustu beni […], non apat balia nin po[te]stadi \textbf{p(er)unu Iuigi (et) nin p(er)una p(er)soni}, ki ad be(n)ni pust mei, a isfairi-llu\\
        ‘And (with respect to) this donation, no judge or no other person that will come after me can revoke it’
    \z
\z

For the elements \textit{nullu, neunu, nex(i)unu}\footnote{Other variants are \textit{nessiunu} and \textit{nensiunu}.}  and \textit{perunu}, 164 occurrences in the corpus show the constellation with a noun marked [+human]. In other words, although \textit{nemos} has survived until today, it was probably avoided in legal and administrative texts.\footnote{The same holds for ‘nothing’, which has no expression in Old Sardinian texts.}  However, this cannot be the only reason for the scarcity of \textit{nemo(s)}, as pronominal \textit{neunu} and \textit{nexiunu}, both synonyms of \textit{nemo(s)}, occur more frequently. Some examples are given in (\ref{ex:men6}):

\ea \label{ex:men6}
        \ea\label{ex:men6a}(\CiteShortHand{StSS} L. I-LXXVI, p. 34.43)\\
         Et i(n) una hora \textbf{neunu} pothat occhier plus de unu a(n)i(m)ale grussu\\
        ‘And in one hour, no one may kill more than one big animal’
        \ex\label{ex:men6b}(\CiteShortHand{StSS} L. I-CXXVI, p. 51.28)\\
         Qui \textbf{neunu} c(om)poret casu ov(er) lana si no(n) i(n) sa platha.\\
        ‘That no one should buy cheese or wool outside the market place.’
        \ex\label{ex:men6c}(\CiteShortHand{StCastel} CCXXI, p. 49.34)\\
         Et qui \textbf{nexiunu} non poçat vendere assos predittos venditores\\
        ‘And that no one can sell to the above-mentioned sellers’
    \z
\z

Nevertheless, the pronominal use of \textit{neunu} as in (\ref{ex:men6}) (33 occurrences) is exclusively documented in the \textit{Statuti Sassaresi} of 1316, a text that is known for its Italianizing tendencies. Pronomial \textit{neunu} can therefore definitely be classified as an Italianism. Similarly, \textit{nexiunu}, which has been clearly identified as an Italianism in the literature\footnote{As for \textit{nex(i)unu, nixunu, nisciunu} (where \textit{x(i)} and \textit{sci} represent [ʃ]), Wagner (\CiteShortHand{DES} 2: 168)  argues that they derive from the Old Italian form \textit{nexun(o)}, quoting \citet[][]{Monaci1955}. However, the texts of this chrestomathy in which this form (as well as \textit{nixun(o)}) appears all correspond to Lombard and Venetian dialects. Instead, \citet[][207]{BlascoFerrer2003} claims that \textit{nexiunu/nisciunu} are loans from the Pisan dialect. But note that Italian forms such as \textit{nesciuno/nisciuno} do not seem to be Tuscan, but are rather found in Southern Italy (\cite[cf.][215]{Rohlfs1969}).  What seems most plausible to me is that these Sardinian forms stem from Ligurian \textit{nesciun/nisciun}. Recall from \sectref{sec:men2.1} the influence of Genoa and the Ligurian dialect on Sardinian. In contrast, the variants \textit{nessiunu} and \textit{nensiunu} can stem from Tuscan.} (cf. \sectref{sec:men2.2}), in pronominal use, is found almost exclusively in the \textit{Statuti di Castelsardo} (1334--1336?) (27 occurrences), with one additional occurrence of the variant \textit{nixunu} in the \textit{Carta de Logu} of Arborea (end of the 14\textsuperscript{th} century). These two texts are also known for showing Italianisms. The items \textit{perunu} and \textit{nullu} do not occur as pronouns.

\textit{Neunu} (and its variant \textit{niunu}) and \textit{nex(i)unu/nix(i)unu} were used also as noun-modifying (adnominal) negative indefinites. Most occurrences of adnominal \textit{neunu} (119 out of 130) are found in the \textit{Statuti Sassaresi} (1316), but unlike pronominal \textit{neunu}, it is also sometimes found in some other texts, with the earliest two occurrences found in the \textit{Condaghe di Barisone II} (1190). The Italian influence on Sardinian started to become particularly palpable starting from the 13\textsuperscript{th} century (\cite[cf.][234--235]{Wagner1997}), but earlier influences cannot be excluded, so it seems likely that both pronominal and adnominal \textit{neunu} are to be considered as Italianisms. Even clearer is the case with \textit{nex(i)unu/nix(i)unu}, which is almost exclusively found in texts from the 13\textsuperscript{th} to the 14\textsuperscript{th} century,\footnote{\CiteShortHand{StSS} (1316), \CiteShortHand{StCastel} (1334--1336?), \CiteShortHand{CdLA} (end of 14\textsuperscript{th} c.).} with only one example that is possibly dated earlier (\textit{Condaghe di Santa Maria di Bonarcado}, 12\textsuperscript{th}--13\textsuperscript{th} c.). This occurrence (\textit{no li tolliant donamentu nixun fatu insoru} ‘they shall not take away any donation made to them’; \CiteShortHand{CondSMB}  33, p. 41.15) is a clear Italianism, as can be seen from the apocopated form \textit{nixun}, which does not conform to Old Sardinian grammar. In the late \textit{Carta de Logu} of Arborea, the masculine form even regularly shows the Italian ending \textit{-o} instead of \textit{-u}.

\subsubsection{\textit{Perunu} and \textit{nullu}}\label{sec:men3.1.2}
In contrast to \textit{neunu} and \textit{nex(i)unu}, the item \textit{perunu} already appears in the 11\textsuperscript{th} century (\textit{Carta volgare dell'Archivio Arcivescovile di Cagliari} n. 1, 1066--1074). Although the sea republics of Pisa and Genoa, after their victory over the Saracens in 1016, started to acquire privileges on the island over the course of the 11\textsuperscript{th} century (see, e.g. the document known as \textit{Privilegio logudorese} from the 1080s, \cite[cf.][233--234]{Wagner1997}), it is extremely improbable that an Italianism (in this case \textit{veruno}) appears integrated into Sardinian at such an early date in a form that shows irregular sound shift (\textit{peruno}, see \sectref{sec:men2.2}). I therefore tend to agree with \citet[][207]{BlascoFerrer2003} that \textit{perunu} should rather be considered as inherited from Latin. Blasco Ferrer does not account for the alleged etymon \textsc{per $\bar\textsc{u}$num}. I suggest that the origin of the Sardinian indefinite is rather \textsc{*per$\bar\textsc{u}$num}, with \textsc{per-} being the well-known Latin intensifying prefix used with adjectives and verbs (cf., among others, \CiteShortHand{FEW} 8, 213--214). Such a formation would thus be semantically equivalent to \textsc{v$\bar\textsc{e}$r$\bar\textsc{e}$ $\bar\textsc{u}$num} (> It. \textit{veruno}), “an emphatically reinforced form of the so-called pronominal adjective \textit{ūnus}, which often has an indefinite function” (\cite[][2]{Ramat1997}). Another detail that speaks against an interpretation of \textit{perunu} as an Italianism is the fact mentioned above that \textit{perunu} is not used as a pronoun, unlike Italian \textit{veruno} (\cite[cf.][9]{Ramat1997}). Actually, as  \tabref{tab:men2} shows, \textit{perunu} is documented rather constantly until the 13\textsuperscript{th} century, with the Italianism \textit{neunu} only sporadically popping up, starting from 1190 onwards. From the 13\textsuperscript{th} century onwards, \textit{neunu} competes with a second Italianism, \textit{nexunu}, which prevails in some texts.

As for \textit{nullu}, with only 19 occurrences, it appears to have been scarcely used. \citet[][132]{Wagner1938} considers it as an Italianism, but strikingly it mostly occurs in the earliest texts,\footnote{With one exception in \CiteShortHand{CdLA} LXXII, 118.9, in which, however, the ending \textit{-o} identifies the element as an independent Italianism. In addition, here, the whole construction \textit{X siat nullo} ‘X be of no avail’, in which \textit{nullo} is not adnominal (contrarily to the Old Sardinian use), actually calques an Italian model.} starting from the 11\textsuperscript{th} century, so it is either an autochthonous element or a Latinism.

\subsection{Syntax}\label{sec:men3.2}
The examples in \REF{ex:men4} in \sectref{sec:men3.1.1}, repeated here as (\ref{ex:men7}a,b), show that \textit{nemo(s)} undergoes negative concord\footnote{The term “negative concord” refers to cases in which there is “a single interpretation of negation in the face of multiple apparent \textit{negative} exponents" (\cite[][458]{Giannakidou2020}).} both in postverbal and in preverbal position:

\ea\label{ex:men7}
    \ea (\CiteShortHand{CondSPS} 68, p. 130.5)\\
    ki non bi aet bias \textbf{nemos}\\
    ‘that nobody is entitled [to possess him]’
    \ex (\CiteShortHand{Montecass39})\\
    \textbf{nemo} no(n) 'de-llis levet, ni(n) ambilla ni(n) pischi\\
    ‘nobody may remove [from the river] neither eels nor fish’
\z
\z

This suggests that Old Sardinian was a strict negative-concord language,\footnote{In “strict negative-concord languages”, a negative marker is obligatory with negative indefinites, independently of their position. Cf. \citet{Giannakidou2000, Giannakidou2006}, among many others. Note that Latin did not have negative concord, but was rather a “double negation” language, in which two negative items yielded a positive reading, cf. \citet{Gianollo2016}.} unlike modern Sardinian, which shows the negation only when the negative indefinite is postverbal (cf. \cite[][23]{Jones1993}). This can be confirmed with the items \textit{nullu} and \textit{perunu}, which are always accompanied by \textit{non} ‘not’ or \textit{nen/nin} ‘neither/nor’, even in the rare cases in which they occur preverbally (10.5\%; \textit{nullu}: 2 out of 19 cases; \textit{perunu} 4 out of 38 cases\footnote{I counted only the occurrences of \textit{perunu} with a negative meaning. For positive \textit{perunu} see below.}). The fact that these items occur mostly in post-verbal position is not surprising, given \citet{Wolfe2015} finding that Old Sardinian was fundamentally a V1-language (an insight to which I will return). Here are some examples for postverbal and preverbal \textit{nullu} and \textit{perunu} in (\ref{ex:men8}a,b) and (\ref{ex:men8}b,c) respectively (\textsc{neg} and V underlined):\largerpage[-1]\pagebreak

\ea \label{ex:men8}
    \ea (\CiteShortHand{CondSMB} 17, p. 20.17)\\
    Et \ul{\textit{non} appat} ausu \textbf{nullu\textit{m} homine\textit{m}} a ttollere\textit{n}dellos aliga\textit{n}do de servitiu de s\textit{an}c\textit{t}a Maria.\\
    ‘And no man dare (lit. not-have.3\textsc{sg-subjv} daring no man) to take them off the service of Saint Mary.’
    \ex (\CiteShortHand{CartaBen} 93, 1.17--18)\\
    {[Et]} \ul{no}·ndi \ul{levit} \textbf{pegus perunu} pro terra maina (et) \ul{ni} \textbf{atera causa p(er)una}\\
    ‘And he may not take away any cattle (lit. \textsc{neg}=from.it take.away.3\textsc{sg-subjv} cattle any) for clayey soil nor for any other thing’\footnote{The sense of this sentence is not clear. For \textit{terra maina} ‘black/clayey soil’, see \citet[][96, 237]{BlascoFerrer2003}.} 
    \ex (\CiteShortHand{Montecass39}) \\
    (et) pischi \textbf{nullu ho(m)i(n)e} mortale \ul{no(n)} 'de-llis \ul{levet}\\
    ‘and no mortal man may remove fish from them’
    \ex (\CiteShortHand{CdLA} V, p. 60.6)\\
    siat i(n)furchadu qui '(n)di mo(r)giat e \textbf{p(ro) dinari p(er)u(n)u} \ul{no(n) canpit}\\
    ‘(he) shall be hanged so that he dies, and he cannot live for any money’
\z
\z

Unlike \textit{nemo(s)} and \textit{nullu}, \textit{perunu} has a positive meaning (‘any’) in irrealis contexts in 9 cases out of 47 (19.1\%), both in preverbal (5 occ.) and in postverbal (4 occ.) position, so \textit{perunu} should rather be classified as a negative polarity item (NPI). Here are two examples:

\ea\label{ex:men9}
    \ea (\CiteShortHand{CVolgAAC11–21} 19, p. 313.18)\\
    Et si \textbf{p(er)unu} \textbf{tempus} illoi bolint torrari hominis ad istari in cussa billa, […]\\
    ‘And if (at) any time persons want to return here to live in this village, […]’

    \ex (\CiteShortHand{CVolgAAC11–21} 20, p. 315.14)\\
    et issu et totus sus piscobus […] bollant pasquiri cu(m) peguliu issoru, bollant arari, […] , ho piscari, ho fayri \textbf{peruna atera causa} \\
    ‘and he and all the bishops […] may pasture their cattle, may plow, or fish, or do any other thing’
\z
\z

This is a clear parallel to Italian, where \textit{veruno} developed from an NPI to a negative indefinite. Nevertheless, this fact need not be attributed to Italian influence, given the similar origin of \textit{veruno} and \textit{perunu} (see \sectref{sec:men3.1.2}), with neither of the two items containing a negative element. As \citet[][4]{Ramat1997} points out (citing \cite[][222]{Haspelmath1997}), in several languages, indefinites made up of an emphasizing or focalizing element and an item meaning ‘one’ have developed an exclusively negative meaning over time. This actually happened to both Sard. \textit{perunu} and It. \textit{veruno} in the modern stages of Sardinian and Italian, respectively.

Let us now look at \textit{neunu} and \textit{nex(i)unu}. Like Sardinian \textit{perunu} and the Old Italian items \textit{neuno} and \textit{nesciuno} (see \citealt{FrancoPoletto2016}, \cite{FrancoKellertMenschingPoletto2016}), Sardinian \textit{neunu} and \textit{nex(i)unu} sometimes show an NPI-like behavior. This occurs when they are used in postverbal position without negative concord,\footnote{In Old Italian, such occurrences of negative indefinites with an NPI-reading occur in a broader set of contexts (in addition to conditions, see also hypothetical free relative clauses and questions).} in which case they do not have a negative meaning, thus corresponding to English ‘any(body)’. However, this is found only eight times with \textit{neunu}+N, almost always in sentences that express a condition and that contain the verb \textit{kertare} ‘to bring a lawsuit’ as in (\ref{ex:men10a}). These eight occurrences are found in three texts, namely the \textit{Condaghe di Barisone} (1190), the \textit{Condaghe di San Pietro di Silki} (end of 11\textsuperscript{th}--13\textsuperscript{th} c.) and the \textit{Condaghe di Santa Maria di Bonarcado} (12\textsuperscript{th}--13\textsuperscript{th} c.). NPI-like \textit{nexunu} appears only in one conditional clause but with a different verb, see (\ref{ex:men10b}), from the \textit{Statuti Sassaresi}.

\ea\label{ex:men10}
    \ea\label{ex:men10a}(\CiteShortHand{StSS} L. I-CXVI, p. 48.34)\\
     ca\textit{m}biando cun bolu\textit{n}tate de pare e bocando·\textit{n}de de si 'nd\textit{e} kertavat \textbf{neunu ho\textit{min}e mo\textit{r}tale} o isse, d'ispiiaremila a ssanta Maria\\
    ‘interchanging [these properties] in joined commitment and with the condition that, should any mortal or he (himself) bring a lawsuit concerning this matter, it will be exempt from claims in favour of St. Mary’

    \ex\label{ex:men10b}(\CiteShortHand{StSS} L. II-59, p. 84.44)\\
     E si li ma(n)chat \textbf{bestia nexuna}, […], si paguet dae sos benes de su dictu comunargiu minore\\
    ‘And if any animal is missing, […], it shall be paid from the belongings of the minor herdsman at issue’
\z
\z

Now let us turn to negation of \textit{neunu} and \textit{nexuno}. Examples like (\ref{ex:men5a}) and (\ref{ex:men6c}), in which \textit{nexunu} occurs in a preverbal position, seem to confirm the status of Old Sardinian as a strict negative concord language, as does (\ref{ex:men11}) for \textit{neunu}: 

\ea\label{ex:men11}
(\CiteShortHand{StSS} L. II, p. 63.6)\\
Qui \textbf{neunu} \textbf{corssu} \ul{no(n)} \ul{pothat} aver officiu i(n) sa citadi de Sass(ar)i.\\
‘That no Corsican may hold office in the city of Sassari.’
\z

However, in strong contrast to \textit{perunu} and \textit{nullu}, these items often lack the negation in preverbal position, even when they are clearly negative, like in (\ref{ex:men6b}) (repeated as \ref{ex:men12a}) and (\ref{ex:men12b}):

\ea\label{ex:men12}
    \ea\label{ex:men12a}(\CiteShortHand{StSS} L. I-CXXVI, p. 51.28, repeated from (\ref{ex:men4b}))\\
    Qui \textbf{neunu} \ul{c(om)poret} casu ov(er) lana si no(n) i(n) sa platha.\\
    ‘That no one should buy cheese or wool outside the market place.’
    \ex\label{ex:men12b}(\CiteShortHand{StCastel} CCXXVI Rubr., p. 51.1)\\
     Qui \textbf{nexiunu} \ul{vendat} vinu a barile.\\
    ‘That nobody may sell wine by barrel.’
\z
\z

Actually, the presence of the negating element as in (\ref{ex:men11}) is quite rare with \textit{neunu}, in contrast to adnominal \textit{nexunu}, where 20 out of 36 relevant cases show the negation. In any case, the optionality of negation with preverbal n-words\footnote{I follow the terminology of \citet[1]{FrancoPoletto2016}, according to which n-words are “words morphologically starting with the negative morpheme \textit{n-}.”} is a property that is typically found in Old Italian (see \citealt{FrancoPoletto2016}; \cite{FrancoKellertMenschingPoletto2016}). It therefore seems that these items are not only loans from Italian, as the quantitative data presented in \sectref{sec:men3.1.1} suggest, but their syntax, too, is a calque from the corresponding Old Italian structures.

Strikingly, pronominal \textit{neunu} and \textit{nexunu} do not occur at all in the postverbal position. When these negative elements are used as modifiers, they are attested postverbally, but with an extremely low frequency. For \textit{neunu}, only the following four (out of 163 occ.) could be identified: 

\ea\label{ex:men13}
    \ea\label{ex:men13a}(\CiteShortHand{CondSPS} 410, p. 330.19)\\
     (et) simile \ul{no(n)} \ul{possa(n)t} laorare miglaresos, butones \ul{né} \textbf{neunu at(er)u lauru} qui siat minus dessa tocha […]\\
    ‘and similarly they cannot produce (either) \textit{miglaresos},\footnote{A type of silver work. Cf. \citet[][128]{Tola1850}.} (or) buttons or (lit.: nor) other products that are inferior to the alloy […]’
    \ex\label{ex:men13b}(\CiteShortHand{CondSPS} 410, p. 330.19)\\
     plachit a donnu Saltaro de Cherchi \ul{accordare}sende cun sos donnos \ul{kene} \textbf{kertu neunu}\\
    ‘it pleased Donnu Salataro de Cherchi to come to an agreement with the gentlemen without any legal dispute’
    \ex\label{ex:men13c}(\CiteShortHand{StSS} L. I-CXXV, p. 51.26)\\
     ma cussu sacrame(n)tu \ul{siat tentu} de facher, \ul{sensa} \textbf{neuna adpellatio(n)e}.\\
    ‘but he shall be obliged to do this sacrament without any appeal.’
    \ex\label{ex:men13d}(\CiteShortHand{StSS} L. II-VI, p. 66.29)\\
     et i(n) cussa q(ue)stio(n)e se \ul{p(ro)cedat}, \ul{sensa} \textbf{neunu atteru term(en)}\\
    ‘and one shall proceed in his issue without any other delay’
\z
\z

In each of these cases, the postverbal negative indefinite is licensed via negative concord, as is expected both in Old Italian and in Old Sardinian, either by \textit{non/ne} or by \textit{kene/sensa} ‘without’. For \textit{nexunu}, out of a total of 68 occurrences, there are only 11 occurrences in which this item appears postverbally: one with a postverbal direct object similar to (\ref{ex:men13a}) (“no li tollia\textit{n}t donam\textit{en}tu nixun fatu insoru” ‘they shall not take away any donation made to them’, \CiteShortHand{CondSMB} 33, p. 41.15, already cited in \sectref{sec:men3.1.1}), and two cases of prepositional phrases with an adverbial function (\textit{i[n] nessiunu modu/per modu nixunu} ‘in no way’, \CiteShortHand{StSS} L. I-CLII, p. 59.38 and \CiteShortHand{CdLA}  CXLIX, p. 184.4). These three occurrences are licensed by \textit{non}. In addition, there are three occurrences with \textit{sensa} ‘without’ similar to (\ref{ex:men13}c--d). Finally, there are eight cases of the following type, always with an expression meaning ‘of no value’, which show the lack of negative concord:

\ea\label{ex:men14}(\CiteShortHand{StSS} L. I-CXVI, p. 48.34)\\
Et si c(on)tra aet ess(er) factu, cussa accusa \ul{siat} \textbf{de nessiunu valore}\\
‘And should this be disregarded, the respective accusation will be of no value’
\z

This structure mirrors the behavior of Old Italian n-words in contexts with the meaning ‘no/little value’ (cf. \citealt{FrancoKellertMenschingPoletto2016}) and can also be considered as an influence of Italian.

The extremely low frequency of postverbal occurrences of the elements at issue diverges from Old Italian, where postverbal negative indefinites are frequently found. This is even more puzzling against the background of the behavior of \textit{nullu} and \textit{perunu}, and, generally, of \citeauthor{Wolfe2015}'s (\citeyear[][20--21]{Wolfe2015}) findings on Old Sardinian syntax:\footnote{In this study, Wolfe analyzes extracts from the \textit{Condaghe di San Nicola di Trullas} and the \textit{Condaghe di Santa Maria di Bonarcado}.} According to his study, Old Sardinian shows V1 word-order in around 73 percent of matrix clauses and all of the embedded clauses. Matrix clauses (but not embedded clauses) also appear with V2 (25\%) and -- very marginally -- with V3 (1.7\%) and V4 (0.5\%) order. Now, interestingly, in the sentences or clauses that contain the two indefinites at issue, V1 order only appears in the rare NPI cases mentioned above and in some of the cases with postverbal (negative) \textit{neunu} and \textit{nixuno}. 

\begin{table}
%\small
    \begin{tabular}{l *2{r@{~}r}} 
    \lsptoprule
         & \multicolumn{2}{c}{\textit{neunu}}  & \multicolumn{2}{c}{\textit{nexunu}} \\\midrule
     V1  & 10  & (5.9\%)  & 10 & (13.5\%) \\
     V2  & 114 & (67.5\%) & 57 & (77.0\%) \\
     V3  & 32  & (18.9\%) & 4  & (5.4\%) \\
     V4  & 13  & (7.7\%)  & 1  & (1.4\%) \\
     V5  & 0   & (0\%)    & 1  & (1.4\%) \\
     V6  & 0   & (0\%)    & 1  & (1.4\%) \\\midrule
     Total & 169 & & 74 & \\
     \lspbottomrule
    \end{tabular}
    \caption{Word order in sentences containing \textit{neunu} and \textit{nexunu}}
    \label{tab:men3}
\end{table}

Examples for V2 order can be seen  in (\ref{ex:men5}a,c), (\ref{ex:men6}b,c), (\ref{ex:men11}), (\ref{ex:men12}a,b); for V3 and V4 order, see (\ref{ex:men6a}) and (\ref{ex:men5b}), respectively. (\ref{ex:men15}) is an example with embedded V6 order. This is the only example with such order, which can be considered as ‘extreme’ in that sense:

\ea\label{ex:men15}(\CiteShortHand{StSS} L. II-XXXXI, 80.36)\\
ordinait qui [\textsubscript{1} dae como inantis] [\textsubscript{2} su pot(estade) q(ui) e(st) e pro temp(us) at ess(er)], [\textsubscript{3} a req(ue)sta \textbf{de nexiunu creditor(e)}] [\textsubscript{4} \textbf{nexiunu corp(us) mortu}], [\textsubscript{5} \textbf{p(er) nexiunu deppidu} de qualu(n)cha qua(n)titade siat obligadu], \ul{no(n) si poça(n)t ne(n) deppia(n)t}, i(n) sa dicta citade […] staxiri ne(n) inpedire
\glt ‘he ordered that, [\textsubscript{1} from now on], [\textsubscript{2} the potestate who is or will be in charge] [\textsubscript{3} on request of any creditor] cannot and must not either confiscate or block [\textsubscript{4} any dead body] [\textsubscript{5} for any debt of whatever quantity might be owed].
\z

In the data examined here, there is thus no matrix/subordinate clauses asymmetry with respect to V>1 word-order, unlike what Wolfe found in his corpus. These variations from V1-syntax are almost exclusively restricted to the \textit{Statuti Sassaresi}, the \textit{Statuti di Castelsardo} and the \textit{Carta de Logu d’Arborea}, three texts where Italianizing tendencies are expected, which, in this case, affect a core syntactic parameter. How about the puzzling fact that the relevant items almost never occur postverbally when they have a negative meaning? A future study might investigate whether this is due to the ambiguous status of these items in Old Italian (i.e. they could either undergo negative concord or function as NPIs). More particularly, it might be that in the postverbal position, these items were borrowed almost exclusively in their positive meaning. 

\section{Universal quantifiers}\label{sec:men4}
\subsection{\textit{Cada}}\label{sec:men4.1}
The item \textit{cada}, which is widespread in Modern Sardinian in the meaning ‘every\slash each’, ultimately derives from Greek κατά in its distributive meaning. The \CiteShortHand{FEW} (2:482) particularly mentions its use with time indications and numbers: κατὰ μῆνα ‘every\slash each month’, κατ\textit{$\prime$} ἐνιαυτóν ‘every year'/‘yearly’, καθ\textit{$\prime$} ἕν ‘one by one’, κατὰ τρεῖϛ ‘three each’ (also cf. \cite[][220]{Rohlfs1969}, \cite[][254--255]{Hofmann1972}). The item existed as a loan in Latin, starting from the 3rd century AD, with the same distributive meaning: \textit{cata mane mane} ‘morning by morning’, \textit{plica unum cata unum petalum} ‘fold the gold threads one by/after one’ (\CiteShortHand{FEW}, loc. cit.). This distributive use can still be found in some Romance varieties (e.g. in Romanian, cf. \CiteShortHand{MeyerLübke1992} 1755). In contrast, in Ibero-Romance and in Sardinian this item developed the meaning ‘every/each’. In other varieties, this meaning is only found in combinations with the word for ‘one’, like Old Northern Italian \textit{cad(a)uno}, Old Tuscan \textit{catuno}, with variants such the Old Pisan \textit{cateunu} (‘each one/everyone’, cf. \cite[][220--221]{Rohlfs1969}). 

Wagner (\CiteShortHand{DES} 1: 256) claims that both Modern and Old Sardinian \textit{cada} is an inherited form from Latin. But note that it is expected that the item at issue be pronounced */ˈkata/ in most of the modern central (Nuorese) dialects (which do not show sonorization of intervocalic Latin voiceless plosives).\footnote{The change from -\textit{t}- to -\textit{d}- in the Old Sardinian texts in which \textit{cada} occurs is not a problem, as intervocalic sonorization is a regular phenomenon in these texts (cf. \CiteShortHand{DES} 1: 256).} This is not borne out, yet: the sonorized form /ˈkada/ is found everywhere. Therefore, Wagner suggests that the pronunciation in the modern central varieties may have been influenced by Spanish. Our discussion below suggests another picture, according to which Old Sardinian \textit{cada} may directly stem from Latin, whereas its Modern Sardinian equivalent is most probably a loan from Catalan or Spanish.

As for the medieval documentation, Wagner (\CiteShortHand{DES} 1: 256) says that \textit{cada} figures various times in the \textit{Carte volgari dell'Archivio Arcivescovile di Cagliari} and the \textit{Condaghe di Santa Maria di Bonarcado}, quoting the examples \textit{cada VIII sollos}, lit. ‘every/each 8 silver coins’ (but see below), \textit{ankilla de cadadie} ‘maidservant in full time/full possession’, and \textit{serbus de cadadie} ‘serfs in full time/full possession’.

Looking at the corpus, we see that \textit{cada} as an independent word only occurs once:

\ea\label{ex:men16}(\CiteShortHand{CondSMB}, 67, p. 55.3)\\
Ego Cipari de Lac\textit{on} avia saltu cu\textit{n} s\textit{an}c\textit{t}u Augustinu et cu\textit{n} donnigella Maria, cia mea, \textbf{cada} .VII.\footnote{The modern edition used in \CiteShortHand{ATLiSOr2017} does say “VII”, whereas Wagner (see above) writes “VIII”, probably following an older edition.} sollos.\\
‘I, Cipari de Lacon, possessed some (wood)land together with (the monastery of) Sanctu Augustinu and the princess Maria, my aunt, (worth) 7 soldos \textbf{each}.’
\z

This is a clear distributive use, in which \textit{cada} does not modify a noun but is rather used adverbially in the sense of ‘in each case’, ‘for each’, ‘respectively’, and resembles rather the Greek usage as in κατὰ τρεῖϛ mentioned above. As \citet[][37]{Grandgent1907} assumes, “\textit{cata} was probably introduced, along the Mediterranean, by Greek merchants, in such [Latin] phrases as \textit{cata unum} = καθ' ἕνα, \textit{cata tres} = κατὰ τρεῖϛ.” In any case, the isolated occurrence of \textit{cada} in (\ref{ex:men16}) does, by all probability, not attest to the existence of a universal quantifier \textit{cada} in Old Sardinian, but is rather a reflex of Latin or even Byzantine Greek formulaic bookkeeping language. As for Greek, recall that Sardinia originally belonged to the Byzantine Empire (cf. \sectref{sec:men2.1}). The Judicates took up Byzantine administrative structures, and, as \citet[][165--174]{Wagner1997} demonstrates, the Old Sardinian chancellery language has multiple influences of Byzantine Greek.\footnote{Also note that another element, \textit{cana}, is slightly more frequent (4 occ.) and is used in exactly the same way as \textit{cada} in (\ref{ex:men16}):
\ea  (\CiteShortHand{CondSMB}, 100, p. 71.25)\\
     Partirus fiios d\textit{e} Justa de Scala et de Eizu de Esule: et levarus fiios de cussos \textbf{cana} .II.\\
    ‘We divided the sons of Justa de Scala and of Eizu de Esule: and we took from them two each.’
\z
\citet[][70]{MeyerLübke1902} proposed that this item is a blend of Lat. \textit{cata} (see above) and \textit{ana}, a late Latin loan from Greek ἀνά with the same distributive function as κατά; for Latin, see \citet[][254]{Hofmann1972} and, particularly, the example \textit{ana duas tunicas} ‘two tunics each/apiece’. Given the slightly better documentation of \textit{cana}, it is therefore not excluded that \textit{cada} in (\ref{ex:men16}) is a scribal error for \textit{cana}.} Thus, this isolated occurrence of \textit{cada} does not correspond to the modern adnominal use (\textit{cada} X ‘every/each X’).\largerpage[-1]

All other occurrences of \textit{cada} occur inside of what seems to be a compound word, \textit{cadadie/cadadia}, and are found exclusively in some documents of the \textit{Carte volgari dell’Archivio Arcivescovile di Cagliari} (dated 1215 and 1217), and always in the fixed expressions mentioned by Wagner: \textit{ankilla de cadadie} (3 occ.), \textit{serbu(s) de cadadie} (5 occ.) and \textit{serbus de cadadia} (1 occ.).\footnote{\label{fn:men23}The form of the item in the latter expression (\textit{cadadia}) is strange, as the word for day is \textit{die} in the whole documentation of Old Sardinian. It looks like a Hispanicism, which would be unexpected for this text, as the Catalan and Spanish influence did not take place before the 14\textsuperscript{th} century (Sardinia belonged to Aragon from 1326 onwards). But note that this form seems to stem from a 15\textsuperscript{th} c. copy. \CiteShortHand{CVolgAAC11–21} 11--21, whose edition was used for the corpus \CiteShortHand{ATLiSOr2017}, adds a footnote explaining that the parchment itself writes \textit{cadaia} (\CiteShortHand[306]{CVolgAAC11–21} 11--21). This form is probably corrupt and hence not conclusive.} The use of \textit{cada} in a compound does not prove the existence of this item as a quantifier in the Old Sardinian texts.\footnote{The compound itself is still enigmatic, an issue that cannot be resolved here. The development of -\textit{t}- > -\textit{d}- would indicate that the compound is an older lexicalization of a Latin *\textit{cata diem} that underwent this sound change. Alternatively, we could assume that the -\textit{d}- in \textit{cada} is due to a long-distance assimilation to the -\textit{d}- in \textit{die}. Finally, I would not exclude that \textit{cadadie} is a loan-blend of Greek καθʹ ἡημέραν/καθημέραν or καθεμερινóς ‘daily’ (\cite[][612]{Sophocles1900}), in which the -\textit{d}-  could stand for the interdental fricative -θ-. Note that \citet[][319]{Solmi1905}, in his lexical notes, says that \CiteShortHand{CVolgAAC11–21} 16, 307.5 has \textit{de catadie}, but the text says \textit{de cadadie}, and there is no note specifying a variant.}

It thus seems that a universal quantifier \textit{cada} ‘every/each’ is not attested in Old Sardinian. Instead, several derivations of Lat. \textsc{omnis} were used, which will be the subject of the next subsection. The modern Sardinian quantifier \textit{cada} is thus quite clearly a later loan from Catalan or Spanish.\footnote{A whole series of loans stemming from or related to Lat. \textit{cata} appears in the 14\textsuperscript{th} century (\textit{Statuti Sassaresi, Statuti di Castelsardo} and \textit{Carta de Logu d’Arborea}). The corpus shows \textit{casc(h)unu/-a, ciascunu,-a} and \textit{casc(h)adunu/-a, ciasc(h)adunu, ciascatunu, -a, çascatunu, -a, ciascu<d>unu/çascadunu, -a} ‘every/each’, more rarely ‘everybody’, all from O. Italian (see \cite[][220--221]{Rohlfs1969} for these and/or similar forms). The Sardinian loans sometimes appear with the original It. ending -\textit{o} instead of -\textit{u}. In contrast, \textit{cadiscuno, -a} is restricted to the \textit{Carta de Logu d’Arborea} (end of the 14\textsuperscript{th} century), where it occurs three times. It appears to be a loan from Cat. (\textit{cadascú/cadescú}) or O. Sp. \textit{cadascuno} (cf., among others, \cite[][396]{Malkiel1948}) or \textit{cadescuno}, which is rather O. Arag. (documented in a text of 1385--1396 edited by \cite{Cacho2003}).}

\subsection{Forms related to Lat. \textsc{omnis}}\label{sec:men4.2}\largerpage[-1]
In the medieval Sardinian texts, several forms related to Latin \textsc{omnis} can be found, which are shown in \tabref{tab:men4}.

\begin{table}
\small
    \begin{tabularx}{\textwidth}{l@{\hspace{2mm}}c@{\hspace{3mm}}c@{\hspace{3mm}}c@{\hspace{3mm}}c@{\hspace{3mm}}c@{\hspace{3mm}}c}
    \lsptoprule
         &  \textit{omnia}/ & \textit{do(n)nja}/ & \textit{(d)ogna} & \textit{omni}/ & \textit{(d)ogni}	& \textit{omnes}/ \\
         &  \textit{onnia} & \textit{donnia} &   & \textit{donnj} &  & \textit{onnes}\\
\midrule
\CiteShortHand{CVolgAAC1} (1066--1074) & 2 &  &  &  &  & 3\\
\CiteShortHand{PrivLog} (1080--1085) &  &  &  &  &  & 3 \\
\CiteShortHand{Montecass9} (1082--1112) &  &  &  &  &  & 3 \\
\CiteShortHand{CartaarbGen2} (1112--20) & 2 &  &  & 1 &  & \\
\CiteShortHand{Montecass10} (1113) & 1 &  &  &  &  & 3 \\
\CiteShortHand{Montecass5} (1120) &  &  &  &  &  & 4 \\
\CiteShortHand{Montecass12} (1120?) &  &  &  &  &  & 3 \\
\CiteShortHand{Montecass16} (ca. 1120) & 1 &  &  &  &  & 3 \\
\CiteShortHand{Montecass20} (1134?) &  &  &  &  &  & 2 \\
\CiteShortHand{Montecass22} (1136) &  &  &  & 1 &  & 3 \\
\CiteShortHand{Montecass32} (1153) &  &  &  &  &  & 3 \\
\CiteShortHand{Cartagall} (1173) & 3 &  &  &  &  &  \\
\CiteShortHand{Montecass39} (1182--83 ca.?) & 3 &  &  &  &  & 1 \\
\CiteShortHand{Cartaarb} (1184) &  &  &  &  &  & 1 \\
\CiteShortHand{CondBar} (1190) & 5 & 1 &  &  &  &  \\
\CiteShortHand{Cartadon} (1211) &  &  &  &  &  &  \\
\CiteShortHand{TrattPace} (1206) &  &  &  &  &  & 1 \\
\CiteShortHand{CVolgAAC11–21} 11--18 (1215--17) & 36 &  &  &  &  &  \\
\CiteShortHand{CartaBen} (1225) &  &  &  &  &  &  \\
\CiteShortHand{CVolgAAC11–21} 19--21 (1225--26) &  &  &  &  &  &  \\
\CiteShortHand{StSS} (1316) & 26 &  & 10 & 3 & 15 &  \\
\CiteShortHand{StCastel} (1334--1336?) &  &  & 2 & 1 & 1 &  \\
\CiteShortHand{CdLA} (end of 14\textsuperscript{th} c.) & 1 & 20 & 3 & 1 &  &  \\
\midrule
\CiteShortHand{CondSPS} (end of 11\textsuperscript{th}--13\textsuperscript{th} c.) & 31 & 3 &  & 1 &  &  \\
\CiteShortHand{CondSNT} (12\textsuperscript{th}--13\textsuperscript{th} c.) & 15 &  &  &  &  &  \\
\CiteShortHand{CondSMB} (12\textsuperscript{th}--13\textsuperscript{th} c.) & 100 & 6 &  & 3 &  & 9 \\
\midrule 
Total & 226 & 30 & 15 & 11 & 16 & 44 \\
\lspbottomrule
    \end{tabularx}
    \caption{Occurrences of forms related to Latin \textsc{omnis}}
    \label{tab:men4}
\end{table}

The most frequent form is \textit{omnia} and its variant \textit{onnia} ‘all/every/each’ (< Lat. \textsc{omnia}), which \citet[][129--130]{Wagner1938} considers an Italianism, given that it preserves the Latin -\textit{i}- (\cite[cf.][130]{Wagner1938}).\footnote{I think that this view is not conclusive. As \textsc{omnia} must have yielded a Vulgar Latin *[ˈɔn:ja], we have to look at other Vulgar Latin words with [-nj-]. An example is V. Lat. *[ˈβinjas] (Lat. \textsc{vineas}), which actually shows the -\textit{i}- (probably representing [j]) in O. Sard. \textit{vinias} (besides Latinizing \textit{vineas}) and the Italianizing spelling \textit{vi(n)gnas}, which is particularly frequent in the \textit{Statuti Sassaresi} and the \textit{Statuti di Castelsardo}. The former shows some isolated cases of \textit{vingias}, where -\textit{gi}- represents a palatal affricate, a later Sardinian development. It is probable that -\textit{j}- in \textit{vinjas} in the \textit{Carta de Logu d’Arborea} stands for the same sound. The fact that this development is usually not reflected in the results of \textsc{omnia} may be accidental or result from an impact of Latin spelling (but see \textit{donja} in the \textit{Carta de Logu d’Arborea}).} By this he seems to mean that \textit{omnia/onnia} are Latinizing spelling variants of O. It. \textit{ogna},\footnote{According to \citet[][219]{Rohlfs1969}, \textit{ogna} is mainly a Lombard, Venetian and Northern Tuscan form, thus being extremely rare in Old Florentine (only 8 occ. vs. 1042 occ. of \textit{ogne} and 11065 occ. of \textit{ogni}), whereas it is more frequent in Old Pisan (\textit{ogna}: 165, \textit{ogne}: 184, \textit{ogni}: 4051), according to the OVI corpus.} and he observes that the Italian spelling \textit{ogna} is often found in Old Sardinian texts. As \tabref{tab:men4} shows, this is, by far, not the case, given the only 15 occurrences of \textit{ogna} documented (vs. 226 occurrences of \textit{omnia/onnia}). In addition, while \textit{ogna} is frequently found in Old Pisan texts (165 occ. in the OVI corpus), there is no trace of the spellings \textit{omnia}\footnote{All 37 occurrences found in Old Pisan texts in OVI appear exclusively in Latin quotations. As a Romance element, \textit{omnia} is extremely rare in all medieval Italian dialects, although some isolated cases can be found in Old Lombard, Old Venetian, and Old Umbrian, all dialects that did not have any impact on Sardinian.} and \textit{onnia}. Therefore, either \textit{omnia} is a Latinism with a popular adaptation \textit{onnia}, or \textit{onnia} represents an inherited form, of which \textit{omnia} was a Latinizing spelling. The latter seems more probable, as (like O. It. \textit{ogna}) the originally neuter plural item is used here as a singular quantifier with the meaning ‘every/each’ preceding both masculine and feminine forms, as is shown in (\ref{ex:men17}):\footnote{\label{fn:men29}It occurs very rarely with a plural noun: \textit{de om(n)ia maiorales suos de locu} (‘of all members of the leading families of the place’, Carta Arb. Gen. 2 [1112--20], 104, 2.17), \textit{et cun omnia libertatos suos} (‘and with all its freed serfs’, \CiteShortHand{CondSMB} [12\textsuperscript{th}--13\textsuperscript{th} c.] 1, 9.11 and similarly in \CiteShortHand{CondSMB} 207, 131.22). In the Old Pisan texts, \textit{ogna} also rarely occurs with a plural noun, but only for indicating time intervals (of the type \textit{ogna sei mesi}). It is therefore probable that the expressions \textit{om(n)ia sex meses} (‘every six months’, \CiteShortHand{StSS} L. I-XLVII, 25.21) and \textit{om(n)ia duos me[s]es} (‘every two months’, \CiteShortHand{StSS} L. I-XXVIII, 13.33; similarly in L. I-XCIX, 40.39) are calques from O. It.}\largerpage[-1]

\settowidth\jamwidth{(m.)}
\ea\label{ex:men17}
\ea (\CiteShortHand{CVolgAAC1},  p. 43.2.2)\\
    (et) fazzant \textbf{o(mn)ia} serbici\textbf{u} \jambox{(m.)}
    ‘and they shall provide every service-\textsc{m.sg}'
\ex (\CiteShortHand{Cartagall}, p. 177, col. 1.28)\\
    cu(n) \textbf{onnia} p(er)tine(n)thi\textbf{a} issoro \jambox{(f.)}
    `with all their possession-\textsc{f.sg}'
\z
\z

Interpreting \textit{omnia} as a singular, is, in fact, a Vulgar Latin innovation that can be traced back to at least the 2nd c. AD, as the following example from \citet{Norberg1944} shows:\footnote{The remarks by \citet[][219]{Rohlfs1969} suggest that O. It. \textit{ogna} is due to the fact that the -\textit{a} in \textit{omnia} was interpreted as a feminine form, and hence \textit{ogna} is used with feminine nouns only. This is not true for the Old Pisan texts, which show both genders according to the OVI database. However, Rohlfs may be right for other Italo-Romance varieties, e.g. Old Venetian, for which the OVI corpus shows the feminine, with very scarce exceptions.}

\ea\label{ex:men18}
Late Latin (CIL 1, 583, 73, cf. \cite[][55]{Norberg1944}, quoted in \cite[][219]{Rohlfs1969}, fn. 2)\\
\textbf{omnia quod} ex hace lege factum non erit\\
‘all that will not have been done following from this law’
\z

Notably, a pronominal use in the sense of ‘all, everything’ is not found with O. Pisan \textit{ogna}, in contrast to Old Sardinian, in which \textit{omnia/onna} can be used as a pronoun, but only when it is restricted by a relative clause, i.e. exactly in syntactic contacts like (\ref{ex:men18}). However, as a relative pronoun, this language used \textit{cantu} (< Lat. \textsc{quantum}, \CiteShortHand{DES} 1: 289), thus diverging from the Latin construction.\footnote{The (written) Latin construction would be \textit{omnia qua}, or, in any case, \textit{omnia quanta}. Wagner (\CiteShortHand{DES} 2: 188) says that this structure might be a Latinism, but he is referring to \textit{omnia} itself, which has preserved the Latin meaning of ‘all’ in the sense of ‘everything’.} This structure makes up around 55 percent of the occurrences in the corpus (124 of 226 occurrences). It is generally found in the shape [\textit{omnia/onni cantu} … V], as in (\ref{ex:men19}a,b), with only two exceptions: In (\ref{ex:men19c}), the relative pronoun is \textit{ca} instead of \textit{cantu}, and in (\ref{ex:men19d}) the conjunction \textit{et} ‘and’ is located between the quantifier and the relative pronoun.\largerpage[2]

\ea\label{ex:men19}
    \ea\label{ex:men19a}(\CiteShortHand{CondSMB}  36, p. 47.10)\\
     Et confirmolli sa domo de sancta Barbara de Turre cum \textbf{omnia cantu} aet cun terras cum binias cum servos et ancillas\\
    ‘And I confirm to him the [possession of the] house of Saint Barbara de Turre with all that it has, [i.e.] with vineyards, with serfs and maids’
    \ex\label{ex:men19b}(\CiteShortHand{Cartagall}, pag. 177, col. 1.28)\\
     Co(m)porai-li a Gavini de Vare, su p(re)viteru de Bosove, \textbf{o(nn)ia ca(n)tu} vi avet i(n) balle de Bosove dave su molinu de Castra i(n) iosso\\
    ‘I bought from Gavini de Vare, the priest of Bosove, all that there is in the valley of Bosove from the mill of Castra downwards’
    \ex\label{ex:men19c}(\CiteShortHand{CondSPS} 139, 170.2)\\
     MAXIMILLA ABBATISSA dessu monasteriu de S(an)c(t)u Pet(ru) de Silki, ki ponio in ecustu condake pro \textbf{o(mn)ia \ul{ca}} 'nke parai in sa domo\\
    ‘MAXIMILLA, ABBESS of the monastery of Saint Peter of Silki, whom I mention in this \textit{condaghe} for all that I acquired in the house’
    \ex\label{ex:men19d}(\CiteShortHand{CVolgAAC11–21} 20, 316.4)\\
     Et daulloy assu do(n)nu miu s(an)c(t)u Antiogu d'iscla de Sulchis \textbf{o(m)nia \ul{et} cantu} apu dessu saltu\\
    ‘And I give to my lord, Saint Antiogu of the island of Sulchis, all that I have of the (wood)land’
\z
\z

The fact that this construction is totally absent from Old Italian can be taken as proof that \textit{omnia/onnia} is not an Italianism. This argument is further corroborated by the data of the clear Italianism \textit{(d)ogna}, which appears late (14\textsuperscript{th} c.), and exclusively in the three texts that we have already identified as being prone to Italianisms, the \textit{Statuti Sassaresi}, the \textit{Statuti di Castelsardo}, and the \textit{Carta de Logu d’Arborea}. Notably, *\textit{(d)ogna cantu} is not attested, which falls in place due to the absence of similar structures in Old Italian. The initial \textit{d}- in \textit{dogna}\footnote{Wagner (\CiteShortHand{DES} 2: 188) accounts for this \textit{d}- as the result of a wrong segmentation of \textit{(d)ed omnia}, which, according to him, is frequently found in the Old Sardinian texts. However, the corpus shows no occurrences of this string. But a similar hypothesis is possible for \textit{et omnia/onnia}, considering that the final -\textit{t} of \textit{et} could be sonorized before vowels (see the alternative spelling \textit{ed} in this phonological contexts in the corpus).} (4 cases in the \textit{Statuti Sassaresi}) is not restricted to the Italianism, but occurs earlier with the autochthonous element \textit{omnia/onnia}, staring from the end of the 12\textsuperscript{th} century both as a pronoun followed by \textit{cantu} as in (\ref{ex:men20a}) and as a modifier, see (\ref{ex:men20}b,~c). In the late \textit{Cartu de Logu} it is almost generalized and appears in the spellings \textit{donnja and donja}, like in (\ref{ex:men20c}), where the -\textit{j}- may represent a palatal affricate already indicating the modern development of Vulgar Latin /nj/ to /ndʒ/ reported in \citet[58]{Wagner1907}.

\ea\label{ex:men20}
    \ea\label{ex:men20a}(\CiteShortHand{CondBar} II, p. 63.22)\\
     Ego, iudike Barusone, conp(or)ai-li a Mariane de Varru su de Usone \textbf{do(n)nia ca(n)tu} bi aviat in I(n)nobiu de vineas, et terras, et saltos, (et) corte, et ho(m)i(n)es.\\
    ‘I, the judge Barusone, bought from Mariane de Varru, the one from Usone, all that there was in Innobiu of vineyards, (wood)land, courtyards and people.’
    \newpage
    \ex\label{ex:men20b}(\CiteShortHand{CondSMB}, 30, 38.2)\\
     cu\textit{n} lassando pa\textit{r}te \textbf{a ffiios e a do\textit{n}nia frate suo}\\
    ‘by leaving a part to the sons and each brother of theirs’
    \ex\label{ex:men20c}(\CiteShortHand{CdLA} CV, p. 144.15)\\
     q(ui) ad bendere cu(n) att(e)ra mesura si no de cusas qui naradas su(n)t paguit p(er) \textbf{donja volta} (sollos) VI\\
    ‘who sells using a measure other than those which are listed shall pay for each time six soldos’
\z
\z 

The form \textit{omni} is rarely found, and partially appears in Latin or Latinizing formulae, such as in \textit{om(n)i opera bona} (Carta Arb. Gen. 2, p. 104, 2.25), \textit{de o(mn)i op(er)a} (\CiteShortHand{Montecass22}, p. 170, 1.10), \textit{cessante omni iustu impedimentu} (‘every legal obstacle having ceased to exist’, \CiteShortHand{StCastel} CLXV, p. 37.24).\footnote{But see \CiteShortHand{StSS} 80,40 with the Italianism \textit{ogni} and another syntax: \textit{ogni inpedimentu cessante} ‘every obstacle having ceased to exist’.} Here, \textit{omni} seems to be a fossilized Latin ablative. The same could be said for the pronominal use in \textit{cum omni cantu at} (‘with all that he has’, \CiteShortHand{CondSMB}  36, p. 46.22 and sim. in \CiteShortHand{CondSMB} 36, p. 46.21).\footnote{The sense of \CiteShortHand{CondSMB} 36, 46.18 is not totally clear to me:
\ea
    \gll E de omni apat fine a su fine in seculum.\\
    and	of	all	have.\textsc{subjv-3sg} end	to	the	end	in saeculum\\\z
\citet[][113]{Virdis2003} translates “E tutto ciò in perpetuo.” (‘And all this forever.’)} These occurrences must be distinguished from the clearly Romance expressions like \textit{om(n)i annu} ‘every year’ (\CiteShortHand{StSS} L. I-XIX, p. 10.47 and \CiteShortHand{StSS}  L. I-CVIII Rubr., p. 46.14, \CiteShortHand{StSS} L. I-CXLlX, p. 59.12; sim. in \CiteShortHand{CondSPS} 426, p. 342.20), which might be Italianisms (from it. \textit{ogni anno}, see below), as was suggested by Wagner (\CiteShortHand{DES} 2: 188), with a Latinizing spelling. Finally, there is one isolated case with the parasitic \textit{d}- that we have seen above: \textit{p(er) don(n)j bolta} (\CiteShortHand{CdLA} CI, p. 140.23).

\textit{Ogni} is a clear Italianism, which exclusively appears in the \textit{Statuti Sassaresi} and the \textit{Statuti di Castelsardo}. Some examples are \textit{per ogni cavallu} (‘for each horse’, \CiteShortHand{StSS} L. II-XLVIII, p. 82.38), \textit{de ogni atheru po(r)chu} (‘of every other pig’, \CiteShortHand{StSS} L. II-L, p. 83.4),  and the time expressions\footnote{See the similar use of \textit{ogna} in footnote~\ref{fn:men29}.} \textit{ogni annu} (‘every year’, several times, e.g. in \CiteShortHand{StSS} L. II-XLVI, p. 82.15), \textit{ogni die} (‘every day’, \CiteShortHand{StCastel} LVIII, p. 31.23). Like in Italian, this item is only used with singular nouns and does not exist as a pronoun. Similarly to \textit{ogna}, the \textit{Statuti Sassaresi} show three occurrences with initial \textit{d}- (\textit{dogni}), a blend of It. \textit{ogni} and the probably autochthonous \textit{donnia} (see above).

Finally, the corpus shows a total of 45 occurrences of \textit{omnes} and its variant \textit{onnes} (< Lat. \textsc{omnes}). As far as we can judge from the texts that can be dated more or less precisely, this item ceased to be used at the beginning of the 13\textsuperscript{th} century.  However, a productive use of \textit{omnes} is only found three times in the \textit{Privilegio logudorese} (1080--1085):

\ea\label{ex:men21}
    \ea\label{ex:men21a}(\CiteShortHand{PrivLog}, p. 253.2)\\
     Ego iudice Mariano de Lacon, fazo ista(m) carta ad onore de \textbf{om(ne)s homines} de Pisas\\
    ‘I, the judge Mariano de Lacon, make this document for the honor of all men of Pisa’
    \ex\label{ex:men21b}(\CiteShortHand{PrivLog}, p. 254.9–10)\\
     ego feci-nde-lis carta pro honore de xu piscopu(m) Gelardu e de Ocu Biscomte e de \textbf{om(ne)s consolos} de Pisas e ffeci-la pro honore de \textbf{om(ne)s ammicos meos} de Pisas: […]\\
    ‘I make this document for the honor of the bishop Gelardu, and Ocu Biscomte, and all consules of Pisa, and I make it for the honor of all my friends from Pisa: […]’
\z
\z

We cannot tell whether this is evidence for an earlier use of \textit{omnes} in Sardinian that had become almost obsolete at the beginning of the written documentation or whether the writer(s) of the \textit{Privilegio logudorese} used \textit{omnes} as a Latinism. In any case, in other 11\textsuperscript{th} century texts, even of a slightly earlier date, \textit{omnes} only appears followed by the adjective or noun \textit{sanctu} (Sard.)/\textit{sanctus} (Lat.) ‘holy, Saint’ and preceded by \textit{i(n) grat(tia) de} ‘in thanks to’, as in \textit{i(n) grat(tia) de […] o(mn)es s(anc)tos P(ro)ph(et)as} (‘in thanks to […] all holy prophets’, \CiteShortHand{CVolgAAC1}, p. 43,1.16), \textit{i(n) grat(tia) de […] o(mne)s s(anc)ti Martires} (‘in thanks to […] all holy martyrs’, with the Latin nominative plural \textit{sancti}, \CiteShortHand{CVolgAAC1}, p. 43, 1.21), and \textit{i(n) grat(tia) de […] o(mne)s s(anc)tos et s(anc)tas Dei} (‘in thanks to […] all saints of God’, \CiteShortHand{CVolgAAC1}, p. 43, 1.22). In all the later documentation, too, \textit{omnes} (and 4 occurrences of \textit{onnes}, all stemming from the Sardinian documents of the monastery of Montecassino)\footnote{For the tight relationship of the monastery (located on the Italian mainland) to Sardinia and its activities on the island as well as the documents at issue (mostly donation letters, see \cite{Saba1927}).} is only found in such fixed formulaic expressions, mostly in strongly Latinizing (parts of) texts. The relevant expressions are almost exclusively two formulae. The first is the formula that we have already seen, \textit{omnes/onnes sanctos et sanctas Dei}, the second \textit{omnes/onnes frates meos e fideles meos testes} ‘with all my brothers and my stalwarts as witnesses’.

To summarize, we can say that Latin \textsc{omnis} yielded the universal quantifier \textit{(d)omnia} (from the Latin neuter plural \textsc{omnia}) as an inherited word, which is found in two functions: first, as a pronoun meaning ‘all, everything’, which must, however, be restricted by a relative clause; second, like some Italianisms also derived from \textsc{omnis}, for quantifying over an individual expressed by a singular NP (‘every/each X)’. For the plural, we find a small number of early occurrences of \textit{omnes}. However, ‘all.\textsc{pl}’ was mostly expressed by \textit{tot(t)u}, as we shall see in the following subsection.

\subsection{\textit{Tot(t)u} ‘all’}\label{sec:men4.3}
The item \textit{tot(t)u} stems from late Latin \textsc{t$\bar\textsc{o}$ttus}, a variant of \textsc{t$\bar\textsc{o}$tus} (\CiteShortHand{MeyerLübke1992} 8815, \CiteShortHand{DES} 2: 500). Its main functions are universal plural quantification like in (\ref{ex:men22}) and universal quantification of singular mass, collective and abstract nouns as shown in (\ref{ex:men23}).

\ea\label{ex:men22}
\ea\label{ex:men22a}(\CiteShortHand{CondSMB} 209, p. 134.21)\\
     denanti dess'altari suo, ue era\textit{n}t \textbf{totu sos monagos}\\
    ‘in front of his altar, where all the monks were’
\ex\label{ex:men22b}(\CiteShortHand{CVolgAAC1}, p. 43, 2.44)\\
     a ponner curadores et maiores suos i(n) \textbf{totas billas} dessus paniliu<s>.\\
    ‘to put officials and principals of his in all villages of semifree serfs.’
\z
\ex \label{ex:men23}
\ea\label{ex:men23a}(\CiteShortHand{CVolgAAC1}, p. 43, 2.10)\\
     Et \textbf{totu custu serbiciu} fage(n)ta fina ad icomo ad su Re(n)nu.\\
    ‘And they have been doing all this service to the Kingdom until now.’
\ex\label{ex:men23b}(Carta Mars. 2, p. 72, 2.12)\\
 (et) de \textbf{totu bi[l]la} de Maara\\
    ‘and of all the village of Maara’\\
\ex\label{ex:men23c}(\CiteShortHand{StSS} L. I-XLIII, p. 23.42)\\
     q(ui) \textbf{totta s'abba} de cussas co(n)ças se vochet foras dessa terra de Sass(ar)i\\
    ‘that all the (waste)water of these tanneries should be poured away outside the territory of Sassari’
\z
\z

Let us consider some aspects of the syntax of Old Sardinian \textit{tot(t)u} (deferring the lack of agreement in examples such as (\ref{ex:men22a}) and (\ref{ex:men23b}) until \sectref{sec:men5}). Unlike modern Sardinian, the determiner following \textit{tot(t)u} (either a definite article as in  (\ref{ex:men22a}) and (\ref{ex:men23c}) or a demonstrative as in (\ref{ex:men23a})) was not obligatory, see  (\ref{ex:men22b}) and (\ref{ex:men23b}). Occasionally, \textit{tot(t)u} occurs to the right of the NP or DP, e.g. \textit{fiios suos tottu} ‘all her children’ (\CiteShortHand{CondSPS} 205, p. 210.17) besides \textit{cu(n) tottu fiios suos} ‘with all her children’ (ibidem, p. 210.33). As both examples are found in the same context (a long list of names of freed serfs), there does not seem to be any semantic or pragmatic difference between the two. Strikingly, postposed \textit{tot(t)u} mostly appears when a relative clause or some restricting phrase follows:

\ea\label{ex:men24}
\ea (\CiteShortHand{CVolgAAC11–21} 20, p. 315.16)\\
Et dau illoy \textbf{su saltu miu de genna de Codrigla totu} \ul{in qua si segat}.\\
‘And I give him my all my (woodland) pasture of Genna de Codrigla up to where it is delimited.’
\ex (\CiteShortHand{CondSMB} 32, p. 40.16)\\
et \textbf{fundamentu suu totu} \ul{c'aviat in Calcaria d\textit{e} Com\textit{ita} d\textit{e} Muru}\\
‘and all his land that he had in Calcaria de Comita de Muru’
\ex (\CiteShortHand{CondSMB} 202, p. 127.13)\\
et \textbf{ipsa binia sua tota} \ul{de Tommanu}\\
‘and all his vineyard of Tommanu’
\ex (\CiteShortHand{StSS} L. I-X, 7.38)\\
\textbf{Sos bandos tottu} \ul{in custu breve c(on)tentos, missos (et) ma(n)datos p(er) issu}\\
‘All bans contained, issued and authorized by him’
\z
\z

I provisionally interpret these examples as structures in which the property that determines the set expressed by ‘all’ is spelled out right adjacent to the quantifier. Since this is a phrasal constituent or, in generative terms, a maximal projection, it cannot be inserted in the standard head position (Q°) of a quantifier phrase and it must therefore be generated in a right peripheral position.\footnote{It is striking that there is a possessive adjective in (\ref{ex:men24}a--c). However, the existence of the possessive did not obligatorily trigger postnominal \textit{tot(t)u}, as the following example shows: \textit{\textbf{totu} sa t(er)ra n(ost)ra de Caralis} ‘all our land of Cagliari’ (\CiteShortHand{CVolgAAC1}, p. 44, 1.30). It seems that the structure seen in (\ref{ex:men24}a--d) has a kind of partitive meaning, e.g. for (\ref{ex:men24}b): ‘the complete part of his land that he had in Calcaria de Comita de Muru’. Thanks to Olga Kellert for pointing this out to me.}

Instead of being part of a NP or DP, \textit{tot(t)u} could be used predicatively, with the meaning ‘entirely’. As (\ref{ex:men25b}) shows, this sense could additionally be made explicit by the item \textit{intre(g)u} ‘entire’:

\ea\label{ex:men25}
\ea\label{ex:men25a}(CVolg. AAC 9, p. 63, 2.11)\\
 una domu \textbf{totu} fabrigada (et) cob(er)ta\\
‘a house entirely built and covered'
\ex\label{ex:men25b}(\CiteShortHand{CondSMB} 28, p. 35.25)\\
 et Ioh\textit{ann}e d\textit{e} Urri ramasit a s\textit{an}c\textit{t}u Georgii \textbf{totu} \ul{intreu}\\
‘and Iohanne of Urri remained entirely in the possession of St. George’
\z
\z

Examples like those in (\ref{ex:men26}) are similar, but unlike those in (\ref{ex:men25}), they can be seen as cases of quantifier floating.

\ea\label{ex:men26}
\ea (\CiteShortHand{CondSMB} 26, p. 34.13)\\
Et sa pa\textit{r}te de sa mugiere, si obierit sine filiis, remaneat \textbf{tota} assa domo d\textit{e} s\textit{an}c\textit{t}a Maria p\textit{ro} s'a\textit{n}i\textit{m}a sua.\\
‘And the wife’s part, if she dies without children, shall all be left to Saint Mary for (the well-being of) her soul.’
\ex (\CiteShortHand{CdLA} VI, p. 60.25)\\
et issos b(e)nis suos \textbf{tottu} siant (con)flischados assa co(r)ti n(ost)ra\\
‘and their possessions shall all be confiscated by our court’
\ex (Carta Mars. 2, p. 72, 1.20)\\
ca fuit \textbf{totu} de S(an)c[tu Satur]ru su saltu\\
‘because the (woodland) pasture belonged all to Saint Saturru’\\
\z
\z

Of the 545 occurrences of \textit{tot(t)u} in the corpus,\footnote{Among these, I counted one occurrence of \textit{tutu} and one of \textit{tuta}, where the -\textit{u}- is probably an Italian influence. I did not count four occurrences of the plural form \textit{tuti}, which is clearly Italian.} only around 20 cases are of the types in (\ref{ex:men25}) and (\ref{ex:men26}).\footnote{Also including a small number of cases in which \textit{tot(t)u} is used predicatively with an empty subject (pro or PRO) as an antecedent.} In another 35 cases, \textit{tot(t)u} is clearly used as a pronoun. In contrast to adnominal \textit{tot(t)u}, whose documentation begins in the 11\textsuperscript{th} century, the pronominal use does not seem to be attested earlier than the beginning of the 13\textsuperscript{th} century.\footnote{With the caveat that there are three occurrences in the \textit{Condaghe di San Pietro di Silki} (end of 11\textsuperscript{th}--13\textsuperscript{th} c.), which cannot be dated exactly.} Usually, pronominal \textit{tot(t)u} is uninflected and is used either as a singular (‘all, everything’ as in (\ref{ex:men27}a--c) (24 cases) or as a plural (‘all [of them]’) as in (\ref{ex:men27}d,e).\footnote{As for the plural reading ‘all (of them)’, the inflected form \textit{totos} is only documented twice as a pronoun in the whole corpus (\CiteShortHand{CondSMB} 131, 86.6; \CiteShortHand{CondSMB} 1, 8.10).} The singular \textit{tot(t)u} clearly competes with \textit{omnia} and \textit{donia} (see \sectref{sec:men4.2}), but unlike the latter normally occurs without a restricting relative clause (except for some rare cases such as (\ref{ex:men27c}).\footnote{In addition, we find \textit{tottu (is)su chi} … ‘all this that …’, but here \textit{tot(t)u} is adnominal and not a pronoun.}

\ea\label{ex:men27}
\ea\label{ex:men27a}(\CiteShortHand{CondSMB} 23, p. 30.12)\\
 \textbf{Totu} lu dam\textit{us} a s\textit{an}cta Maria d\textit{e} Bona\textit{r}cadu p\textit{ro}ssas a\textit{n}i\textit{m}as n\textit{ost}ras.\\
‘We give all to Saint Mary of Bonarcado for (the well-being of) our souls.’
\ex\label{ex:men27b}(\CiteShortHand{CondSMB} 66, p. 54.13)\\
 et e\textit{st} \textbf{totu} puspare .XXX. sollos\\
‘and it is all together (worth)  thirty soldos’
\ex\label{ex:men27c}(\CiteShortHand{CVolgAAC11–21} 11, p. 294.28)\\
 Istimo(n)ius […] de \textbf{totu} \ul{ca(n)tu} narat ista carta, do(n)nu Riccu su archipiscobu miu de Pluminus, et […]\\
‘Witnesses […] of all that is said in this document [are] Donnu Riccu, my Archbishop of Pluminus, and […]’
\ex\label{ex:men27d}(\CiteShortHand{StSS}  L. I-I, p.  4.21)\\
 Iustithia açes facher ad \textbf{tottu}, man(n)os et piçinnos\\
‘You have to do justice to all (of them), adults and children’
\ex\label{ex:men27e}(\CiteShortHand{CdLA} XVI, p. 72.11)\\
 si \textbf{totu} o sa maiore parte non esserent in concordia no siant credidos\\
‘if all (of them) or the majority do not accord, they shall not be believed’
\z
\z

In the \textit{Statuti Sassaresi}, the singular reading also appears in formulaic expressions containing \textit{in tottu}, such as \textit{in tottu et per tottu} ‘in all and for all’ (i.e. entirely, in all respects) and in \textit{tottu over in parte} ‘totally or partially’, which are calques from the equivalent Italian expressions \textit{in tutto e per tutto} and \textit{in tutto ovver’ in parte}. 


\section{Agreement patterns of Old Sardinian \textit{tottu}+DP/NP}\label{sec:men5}

As mentioned in \sectref{sec:men4.3}, adnominal \textit{tottu} often appears as invariable in Old Sardinian, i.e. without agreement with the DP or NP. Interestingly, structures with and without agreement of \textit{tot(t)u} can be found in the corpus (contra \citeauthor{BlascoFerrer1984}'s \citeyear[][93]{BlascoFerrer1984} observations).\footnote{“[Il lat. volg. \textsc{tottus}] [...] si è cristallizato sin dalle prime documentazioni nella forma invariabile /tóttu/".} In fact, both structures, with and without agreement of \textit{tot(t)u}, are documented in the texts: \newpage

\ea\label{ex:men28}
\ea fem. sg. [+agreement] (\CiteShortHand{CondSPS} 44, p. 118.7)\\
e llevarun \textbf{tott\ul{a}} s\textbf{a} cas\textbf{a} issoro\\
‘and they took away all their possessions’
\ex fem. sg. [-agreement] (\CiteShortHand{CondSMB} 104, p. 74.19)\\
Parsit iustitia a \textbf{tot\ul{u}} coron\textbf{a} de logu\\
‘It seemed just to the whole court’
\z
\ex\label{ex:men29} 
\ea fem. pl. [+agreement] (\CiteShortHand{CdLA} CXXV, p. 166.3)\\
sas dominiguas de totu s'an(n)o et \textbf{tot\ul{as}} s\textbf{as} fest\textbf{as} de sant\textit{a} Mari\textit{a}\\
‘the Sundays of the whole year and all the feasts of Saint Mary’
\ex fem. pl. [-agreement] (\CiteShortHand{StSS} L. I-XXXVII, p. 21.21)\\
deppiat satisfacher sa mesitate d(e) \textbf{tott\ul{u}} s\textbf{as} ispes\textbf{as}\\
‘[he] had to cover half of all the expenses'
\z
\ex\label{ex:men30} 
\ea masc. pl. [+agreement] (\CiteShortHand{CondSMB} 33, p. 41.2)\\
ad honore de Deus et de sancta Maria et de \textbf{tot\ul{os}} s\textbf{os} sant\textbf{os}\\
‘to the honor of God and of Saint Mary and of all the saints’ 
\ex masc. pl. [-agreement] (\CiteShortHand{CondSNT} 1, p. 64.14)\\
T\textit{estes}: Simio d'Elices e \textbf{tot\ul{u}} bicin\textbf{os} su\textbf{os}.\\
‘Witnesses: Simio d’Elices and all his neighbours’
\z
\z

Let us look at the distribution of agreeing and non-agreeing adnominal \textit{tot(t)u}, shown in \tabref{tab:men5}.

\begin{table}
\footnotesize
\begin{tabularx}{\textwidth}{Qcccccccc}
\lsptoprule         
&  \multicolumn{3}{c}{{Singular}}  &  \multicolumn{4}{c}{{Plural}}  &  \\
\cmidrule(lr){2-4} \cmidrule(lr){5-8}
 &  {m.}  &  \multicolumn{2}{c}{{f.}}  &  \multicolumn{2}{c}{{m.}}  &  \multicolumn{2}{c}{{f.}}  & \\
\cmidrule(lr){3-4} \cmidrule(lr){5-6} \cmidrule(lr){7-8} 
{Texts/}  &    &  {+agr}  & {-agr}   &  {+agr}  &  {-agr}  & {+agr}  & {-agr}  & \\

 {Period}  &  \textit{tottu}  &  \textit{totta}  &  \textit{tottu}  &  \textit{tottos}  & \textit{tottu}   &  \textit{tottas} &  \textit{tottu}  & {Total}\\

\midrule

div. texts 	&	4    &  8   &  1  &  1   & 4   & 1  & 0   & 19 \\
	(1050–1150)&	    &  (88.9\%)   &  (11.1\%)   &  (20\%)   & (80\%)   & (100\%)   & (0\%)   & \\
   \tablevspace
div. texs  	&	16   &  1  &  1  & 1   &  10  & 0   &  7  &  36  \\
(1150–1225)& & (50\%)  & (50\%)   & (9.1\%)   & (90.9\%)  & (0\%)   & (100\%)   & \\
  \tablevspace
\CiteShortHand{CondSPS} &	35   &  34  &  0  &  5  & 10   & 0   & 2   & 86 \\
(end 11\textsuperscript{th}–mid 13\textsuperscript{th} c.)	&    & (100\%)   & (0\%)   & (50\%)   & (50\%)   &  (0\%)  & (100\%)   & \\
  \tablevspace
\CiteShortHand{CondSMB} 	&	26   &  39  &  7  & 8   &  12  &  0  & 0   & 92 \\
(12\textsuperscript{th}– 13\textsuperscript{th} c.) &    &  (84.8\%)  &  (15.2\%)  &  (40\%)  &  (60\%)  &  (0\%)  &  (0\%)  & \\
  \tablevspace
\CiteShortHand{CondSNT}	&	9   &  22  &  0  &  13  &  9  &  2  & 0   & 55 \\
(1\textsuperscript{st} quarter 12\textsuperscript{th}–2nd half 13\textsuperscript{th} c.)&    &  (100\%)  &  (0\%)  & (62\%)   &  (38\%)  &(100\%)   &  (0\%)  & \\
  \tablevspace
\CiteShortHand{StSS}	&	24   & 10   & 1   &  0  &  40  & 0   & 33   & 108 \\
(1316)  &    & (90.9\%)   &  (9.1\%)  &  (0\%)  & (40\%)   & (0\%)   & (100\%)   & \\
  \tablevspace
\CiteShortHand{StCastel}	&	2   &  2  & 0   &  0  & 16  & 0   &  14  & 34 \\
(1334–1336?) &    &  (100\%)  &  (0\%)  & (100\%)   &  (100\%)  &  (0\%)  &  (100\%)  & \\
  \tablevspace
\CiteShortHand{CdLA}  & 7   &   5 &  5  &   1 &   24 &  5  &  2  & 49  \\
(end 14\textsuperscript{th} c.) &    &  (50\%)  & (50\%)   & (4\%)   & (96\%)   & (71.4\%)   & (28.6\%)   & \\

\midrule

{Total}  & {123}   &  {121}  &  {15}  &  {29}  & {125}   &  {8}  & {58}   &  {479}\\
  &  & (89\%) & (11\%) & (18.8\%) & (81.2\%) & (12.1\%) & (87.9\%) & \\
\lspbottomrule
    \end{tabularx}
    \caption{Agreement of adnominal  \textit{tot(t)u}}
    \label{tab:men5}
\end{table}



If we first look at the last line, we see that, very strikingly, \textit{tot(t)u} agreed in the overwhelming majority of cases (89\%) in the feminine singular, whereas it rather rarely agreed in the feminine plural (ca. 12\%) and not very frequently either in the masculine plural (ca. 19\%). Due to the fact that the \textit{condaghes} (\CiteShortHand{CondSPS}, \CiteShortHand{CondSMB}, \CiteShortHand{CondSNT}) contain texts that extend over great time-spans, it is rather difficult to make any statement concerning the diachronic development. We can however say that, at least by tendency, agreement in the feminine singular seems to have been constantly predominant (ca. 90\%--100\%) until the end of the 14\textsuperscript{th} century, when it suddenly drops to 50\% in the latest text. The masculine plural form seems to have had at least some significant vitality between the end of the 11\textsuperscript{th} and the second half of the 13\textsuperscript{th} c. and was practically inexistent in the 14\textsuperscript{th} c. Together with the drop of the frequency of the feminine singular form, we could interpret this as the beginning of a tendency that would ultimately lead to the modern situation without agreement. There are, however, two issues that have to be addressed concerning the plural forms.

First, as we have seen, the feminine plural form is scarcely found throughout most of the documentation, but strikingly, the latest text of the end of the 14\textsuperscript{th} century (the \textit{Carta de Logu}) shows 5 occurrences of this form (\textit{totas}) vs. 2 of the non-agreeing form (\textit{tot[t]u}) in feminine plural contexts.\footnote{This unexpected rise is only partially explained because three of the five occurrences of the agreeing forms stem from the same passage containing three times the same pattern:
\ea (\CiteShortHand{CdLA} CXXV, p. 166.3--4)\\
In p(ri)mis sas dominiguas de totu s'an(n)o et \textbf{totas sas festas} de santa Maria; item \textbf{totas} <\textbf{sas}> \textbf{festas} de sos apostollos e \textbf{tot(a)s sas festas} de sos evangellistes; […]\\
    ‘First, the Sundays of the whole year and all the feasts of Saint Mary, then all the feasts of the Apostles, and all the feasts of the Evangelists; […]’
\z} We have already seen in previous sections that the \textit{Carta de Logu} is among those texts that are particularly prone to Italianisms. However, it would not be plausible to consider the form \textit{totas} or at least the tendency to have agreement in the feminine plural as an Italianism, as the other strongly Italianizing texts (\textit{Statuti Sassaresi, Statuti di Castelsardo}) do not show this phenomenon, i.e., in these texts, \textit{tot(t)u} never agrees in the plural. The only reasonable conclusion seems to me to consider this as a Catalanism, also taking into account that the corresponding Old Catalan form was actually \textit{totas}.\footnote{The O. Cat. paradigm of this quantifier was \textit{tot} (m. sg.), \textit{tota} (f. sg.), \textit{totz} (m. pl.), \textit{totas} (f. pl.). For Catalan influences in the \textit{Carta de Logu}, see \citet[][180-181]{LoiCorvetto1992}, where the author discusses some ideas by \citet[][136]{Sanna1975}. Even if this author thinks that some rather clear Catalanisms such as \textit{desviadu, mescladura, biage} (var. \textit{biagio, biatgio}) and the spellings \textit{que-, gue-} for It. \textit{che-, ghe-} might be explained otherwise (e.g. as influences of Genovese), he admits the possibility of an Aragonese scribe having been involved in the writing of the manuscript.}\largerpage[2]

\begin{sloppypar}
Second, when we look at the occurrences of the masculine plural form (\textit{tot([t]os}), we observe that, interestingly, 22 of the 29 occurrences are all followed by a numeral. More particularly, most of these occurrences show a structure of the type ‘\textit{all} (\textit{the}) numeral N’, as shown in (\ref{ex:men31}):
\end{sloppypar}

\ea\label{ex:men31}
\ea\label{ex:men31a}(\CiteShortHand{CondSPS} 33, p. 108.36)\\
 torraitimilos iudike \textbf{tottos .VI. sos fiios} de Barbara Rasa\\
‘the judge gave back to me all six sons of Barbara Rasa’
\ex\label{ex:men31b}(\CiteShortHand{CondSNT} 1, p. 107.2)\\
 et a Petru de Nurki et a \textbf{totos .III. sos connatos} co\textit{m}porailis su pede de Iorgi de Contra\\
‘and from Petru de Nurki and all his three brothers-in-law, I bought a quarter of Iorgi de Contra’
\ex\label{ex:men31c}(\CiteShortHand{CondSMB} 133, p. 89.9)\\
 Mandei pro·llos et benneruntimi \textbf{totos tres frates fiios} de Gostantine Stapu: Orçoco et Comida et Iohanne.\\
‘I summoned them, and there came all three brothers, sons of Gostantine Stapu: Orçoco, Comida, and Iohanne.’
\z
\z

Conversely, lack of agreement of \textit{tot(t)u} is not found at all whenever a numeral follows. This even turns out to be true for the feminine, where only one occurrence with a numeral is found in the corpus, showing agreement:

\ea\label{ex:men32}(\CiteShortHand{CondSNT} 1, 94.5)\\
Conporailis ad Ytçoccor Mavronti et assos fr\textit{ate}s, die de Pale Pirinione, et die in Istefane Pira, et .iii. dies in \textbf{totas .iii. sas filias}: […]\\
‘I bought from Ytçoccor Mavronti and from his brothers one day of Pale Pirinione, and one day of Istefane Pira, and three days of all three daughters: […]’ 
\z

We can thus summarize the results as follows: agreement of \textit{tot(t)u} was optional in Old Sardinian, with a strong preference towards agreement in the singular. In the plural, agreement is only marginally attested but was obligatory when \textit{tottu} was followed by a numeral.

Some more comments can be made with respect to the structures in (\ref{ex:men31}) and (\ref{ex:men32}), which are interesting from a typological perspective. Within the modern Romance languages, there are basically two patterns, which are shown for Italian in (\ref{ex:men34}): the structure illustrated in (\ref{ex:men34a}), with the conjunction \textit{e} ‘and’ before the article (situated between the numeral and the noun), and the option in (\ref{ex:men34b}), without the conjunction and the article preceding the sequence numeral + noun. Whereas French does not allow at all the combination of \textit{tous} ‘all’ with a numeral (cf. \cite[][210]{Doetjes1997}, see ex. (\ref{ex:men33})), Spanish only allows the option corresponding to (\ref{ex:men35b}), whereas (\ref{ex:men35a}) is ungrammatical:


\ea\label{ex:men33} 
French
\ea\label{ex:men33a}  *Tous (et) trois les étudiants ont  lu le livre.
\ex\label{ex:men33b}  *Tous les trois étudiants ont lu le livre.\\
‘All three students read the book.’
\z 
\ex\label{ex:men34}
Italian (cf. \cite[][7]{Balsadella2017})
\ea\label{ex:men34a} Tutti e tre gli studenti hanno letto il libro.  
\ex\label{ex:men34b} Tutti i tre studenti hanno letto il libro.\\
‘All three students read the book.’
\z
\ex\label{ex:men35}
Spanish (cf. \cite[][173]{Cirillo2009})
\ea\label{ex:men35a} *Todos y tres (los) estudiantes leyeron el libro.
\ex\label{ex:men35b} Todos los tres estudiantes leyeron el libro.\\
‘All three students read the book.’
\z
\z

In contrast, Old Sardinian had another structure, as witnessed in (\ref{ex:men31}a,b) to (\ref{ex:men32}), which was similar to (\ref{ex:men34a}) but lacking the coordinating conjunction and with the definite article being optional (see \ref{ex:men31c}). The same word order can be shown to have existed in other medieval Romance languages:

\ea\label{ex:men36}
Middle Italian (\emph{\citetitle{OVI2021}}, \textit{Bibbia volgare} Ez 41, p. g575)\\
e due porte erano da \textbf{tutti due li lati} degli usci\\
‘and two doors were at both two sides of the exits’

\ex\label{ex:men37}
Old Spanish (\emph{\citetitle{CORDE}}, Anónimo, c1414)\\
\textbf{todos	quatro	los	caualleros}	mobieron\\
 ‘all three knights moved on’\\

\ex\label{ex:men38}
Old French (\textit{Guillaume d'Orange}, v. 792, ed. \cite[134]{Jonckbloet1854})\\
Quant	il 	connut 	\textbf{toz	trois	les	compaignons}\\
‘When he recognized all three companions’
\z

What distinguishes Old Sardinian from these languages is that agreement on the universal quantifier was optional and becomes obligatory exactly in this structure involving a numeral. On a more typological level, outside the Romance languages, the Old Sardinian structure is identical to the option (\ref{ex:men39b}) of Modern Dutch. Interestingly, we find a similar agreement pattern (with agreeing \textit{all}, in contrast to \ref{ex:men39a}):


\ea\label{ex:men39}
Modern Dutch (\cite[][160]{Cirillo2009})
\ea\label{ex:men39a} \gll \textbf{Al} \ul{de} \ul{drie} studenten	hebben	het	boek	gelezen.\\
all	the	three	student-\textsc{pl}	have the book read\\
\ex\label{ex:men39b} \gll \textbf{Alle} \ul{drie}	\ul{de}	studenten	hebben	het	boek gelezen.\\
all-\textsc{pl}	three 	the	student-\textsc{pl}	have the book read\\
\z
\z

\citet[][160]{Cirillo2009} analyzed these structures as follows:

\ea\label{ex:men40}
\ea\label{ex:men40a} {[}\textsubscript{QP} all {[}\textsubscript{DP} the  {[}\textsubscript{CardP} three {[}\textsubscript{NP} students{]}{]}{]}{]}
\ex\label{ex:men40b} {[}\textsubscript{QP} all three {[}\textsubscript{DP} the {[}\textsubscript{CardP} ∅ {[}\textsubscript{NP} students{]}{]}{]}{]}
\z
\z\largerpage

In (\ref{ex:men40a}), which corresponds to the word order in (\ref{ex:men39a}) as well as to that of the Italian structure in (\ref{ex:men34b}) and  the Spanish structure in (\ref{ex:men35b}), the quantifier is generated in its regular position within a quantifier phrase (QP) preceding the DP, and the cardinal number is generated beneath the determiner in a cardinal phrase (CardP). (\ref{ex:men40b}) represents the word order of the Dutch example (\ref{ex:men39b}), which is the same as in the Old Sardinian examples in (\ref{ex:men31}) and (\ref{ex:men32}) and of the other Old Romance languages shown in (\ref{ex:men36}) to (\ref{ex:men38}). Here, in \citeauthor{Cirillo2009}'s  analysis, the numeral is generated together with the quantifier (creating a complex quantifier head, the “universal numeric quantifier (∀NumQ)”,\footnote{Note that he does not derive (\ref{ex:men40b}) from (\ref{ex:men40a}) because head-movement from the lower Card-Position would have to cross D° and thus violate the HMC. The structure in (\ref{ex:men40b}) is, however, somewhat awkward because of the empty Card head.} \citeauthor{Cirillo2009} does not account for the agreement behavior, yet. An explanation for the presence of agreement in (\ref{ex:men39b}) is found in \citet{Corver2010}: According to him, the ∀NumQ expression \textit{alle drie} ‘all three’ is also generated as a complex head (Num°).\footnote{The NumP has the same position of \citeauthor{Cirillo2009}’s CardP: [\textsubscript{NumP} [\textsubscript{Num°} all five] [\textsubscript{NP} women]].} The derivation is performed in two steps: first, the NP is raised to the specifier of NumP, and this is where agreement is realized, via specifier head agreement. Then, the lower part of the NumP moves to the specifier of DP.\footnote{This step is problematic because it involves movement of an X'-constituent, which should not be allowed in modern generative frameworks. In \citet{Mensching2023} I discuss \citegen{Corver2010} assumptions in more detail and suggest a similar analysis (applied to the Old Sardinian structures) that does not have this problem. I propose that the NP moves out of the CardP or NumP to a functional category, and the remaining part of NumP/CardP (containing the ∀NumQ item) undergoes remnant movement to QP. Obligatory agreement is explained withing the minimalist framework following \citet{Chomsky2000}: The Q head has an unvalued phi-probe that probes the NumP remnant, and thus needs valued phi features on ∀NumQ.}

I have included this brief discussion because it shows that the Old Sardinian data can contribute to some interesting issues of a general linguistic interest, but I will not go any further into the formal analysis of this construction. Let us, instead, have a brief look at Modern Sardinian. Recall from \sectref{sec:men2.2} that despite the general tendency of Modern Sardinian \textit{tot(t)u} to show lack of agreement, a modern plural form \textit{tot(t)us} can be observed in some syntactic contexts. First of all, as I already said in \sectref{sec:men2.2}, this form is striking, because in the modern Logudorese and Nuorese varieties, this form with the ending \textit{-us} cannot be derived from the old plural form in \textit{-os}, as these varieties preserve \textit{-o} in word-final syllables. In addition, as we have seen, the old plural forms were almost not used at the end of the medieval period and, in any case, when they occur, there was a clear distinction between a masculine form ending in \textit{-os} and a feminine form ending in \textit{-as}, whereas the modern plural form in \textit{-us} is invariable for gender. It is therefore very probable that the modern form \textit{tot(t)us} is an innovation. Why this innovation arose in contexts such as that in example (\ref{ex:men3}) of \sectref{sec:men2.2} must be left for future research. However, \citet[][38]{Jones1993} mentions another syntactic context in which some speakers use the plural form, namely in connection with numerals:\largerpage

\ea\label{ex:men41}
\gll tottus tres ómines\\
all-\textsc{pl} three men\\
\glt ‘all three men’
\z

Note that, if this is a ∀NumQ construction, agreement is expected (once the language has a plural form), if an analysis along the lines of \citet{Corver2010} is on the right track. Unfortunately, since Jones’s examples lack the article, we cannot exactly determine its structure.\footnote{The Modern Sardinian structures need more research. An informal inquiry that I made with three speakers of Logudorese and Nuorese varieties suggests that, when the article is present, the following options are possible:

\ea\label{ex:menfootnote50}
\ea \gll totu(s)	e	tres 	sos	òmines.\\
all(-\textsc{pl})	and	three	the	men \\
\ex \gll totu	sos	tres òmines.\\
 all	the	three men \\
\z
\z

These options correspond to the Italian patterns in (\ref{ex:men34}a,b) and are therefore certainly to be considered as Italianisms, particularly because they are not found in the medieval documentation. Interestingly (and coherent with what we have said about $\forall$NumQ-constructions), only (i.a) optionally appears to admit agreement for some speakers.}

\section{Summary and outlook}\label{sec:men6}

In the present chapter, I have presented some preliminary results of a corpus analysis on indefinites in Old Sardinian, a language that had been understudied in this regard. My interest in such a study was mostly motivated by (i.) the fact that a part of the inventory of indefinites of Modern Sardinian is known to contain indefinites that are loanwords mostly taken from older stages of Italian, Catalan, and Spanish, and (ii.), that some Modern Sardinian quantifiers show lack of agreement. In both cases, for Old Sardinian, i.e. the language documented in medieval legal and administrative documents, these issues had been considered before in the literature, but only on a superficial level, which had led to hypotheses that had never been matched against quantitative data. With respect to (i.), the quantitative methods applied here (on the basis of the corpus \CiteShortHand{ATLiSOr2017}) led to results that can be summarized as follows:

\begin{enumerate}
    \item Wagner (1938--1939) and in his \CiteShortHand{DES} tended to consider the negative indefinites \textit{nullu}, \textit{perunu} ‘no (X)’ \textit{neunu}, and \textit{nexunu} ‘nobody/no (X)’ (and their variants) as Italianisms. I have shown that the distribution of these items over time strongly suggests that only the latter two are loans from Old Italian. Whereas \textit{nullu} should be considered either a Latinism or an item inherited from Latin (maybe rather the former given its scarce documentation), \textit{perunu} must definitely be interpreted as an autochthonous element derived from Latin.
    \item As for \textit{cada} (‘every/each’), Wagner (\CiteShortHand{DES} 1: 256) used the Old Sardinian documentation for arguing that this item is probably not a Hispanicism. However, when looking at the corpus, it becomes evident that \textit{cada} does not appear at all in the corpus in its modern sense. Apart from one occurrence with a distributive sense of ‘X N each’ (with X a numeral), this item is only found in the lexicalized compound \textit{cadadie} ‘daily’ in expressions meaning ‘full-time serf/maid’. I therefore concluded that \textit{cada} in the sense of ‘every/each’ must be a later loan from Catalan and Spanish.
    \item A small number of early occurrences of \textit{omnes/onnes} ‘all.\textsc{pl}’ might indicate that this item was inherited from Latin but was almost obsolete at the beginning of the Old Sardinian documentation. In any case, later occurrences of these items are clear Latinisms that are only found in Latinizing texts and formulae.
    \item My analysis strongly suggests that the state of the art concerning \textit{omnia} and \textit{onnia} ‘every/each’ must be revised as follows: they are neither Latinisms nor Italianisms. I rather consider \textit{onnia} as a form inherited from \textsc{omnia} in its Vulgar Latin singular use and \textit{omnia} as a Latinizing spelling variant. Wagner’s idea that \textit{omnia} and \textit{onnia} are spelling variants of the Italianism \textit{ogna} is contradicted by the fact that \textit{ogna} appears rather late in some Old Sardinian texts that are known for their Italianizing tendencies and that neither \textit{omnia} nor \textit{onnia} were usual in Old Italian.
\end{enumerate}

Apart from these findings, the corpus study also provided an occasion to look at some aspects of the syntax of these elements. In this respect, an innovative finding is that Old Sardinian seems to have been a strict negative concord language, unlike Modern Sardinian, where preverbal negative indefinites lack negative concord. Contrarily to the items \textit{nullu} and \textit{perunu}, which I have argued to be autochthonous elements, the borrowed negative indefinites often show the lack of negative concord when they appear preverbally, which seems to indicate that this property has been adopted from Old Italian together with the items itself. More generally, the occurrence of these items in preverbal positions also indicates foreign influence, given that Old Sardinian was mostly a V1 language, a fact reflected quite well in the syntax of \textit{perunu} and \textit{nullu}.\largerpage

Finally, I have been looking at the agreement behavior of \textit{tot(t)u} ‘all, whole’, inherited from late Latin \textsc{t$\bar\textsc{o}$ttus}. In modern Sardinian, this item standardly lacks agreement. For Old Sardinian, an in-depth study of this item was missing until now and has been provided in this article for the first time. The results show that agreement of \textit{tot(t)u} was optional in Old Sardinian, with a strong preference for agreement in gender in the singular (\textit{tot(t)u} vs. \textit{tot(t)a}), whereas, in the plural ((\textit{tot(t)os/tot(t)as}) agreement in both genders was strongly dispreferred (in favor of the default form \textit{tot(t)u}). The agreement property of this item must have vanished altogether after the Middle Ages, and the gender-neutral plural form \textit{tot(t)us}, which is occasionally observed in some modern varieties, must be considered as an innovation. I have also been able to detect an exception to the optional agreement of Old Sardinian \textit{tot(t)u}, namely a structure involving numerals of the type ‘\textit{all} Numeral (Det) N’ (in this word order), where I have identified the sequence ‘\textit{all} Numeral’ as an instance of universal numeric quantifiers according to \citet{Cirillo2009}. Here, number agreement in the plural was obligatory, in conformity with observations that have been made for other languages which show this kind of structure.

All in all, I hope to have demonstrated with this study that a thorough corpus-based analysis on Old Sardinian can bring forth important insights in the field of indefinites, both for Romance and for general linguistics.\largerpage

\section*{Acknowledgments}
This chapter was written within the framework of a joint project with Cecilia Poletto (“Quantification in Old Italian”) financed by the Deutsche Forschungsgemeinschaft (DFG).

{\sloppy
\printbiblist[heading=subbibliography,keyword={MenschingPrimary},title={Sources and dictionaries}]{shorthand}}

\printbibliography[heading=subbibliography,notkeyword=MenschingPrimary]
\end{document}
