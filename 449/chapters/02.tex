\documentclass[output=paper,colorlinks,citecolor=brown]{langscibook}
\ChapterDOI{10.5281/zenodo.13759982}

\author{Jan Davatz\affiliation{University of Zurich} and Elisabeth Stark\affiliation{University of Zurich}}
\title{Romanian \textit{ni{ș}te} between non-specific and specific interpretations}

\abstract{\begin{sloppypar}Our contribution focuses on the semantics of the intriguing and highly under\-studied indefinite element \textit{niște} (approximately English ‘some’) in Romanian. Given the apparent distributional similarities between \textit{niște} and the so-called \textit{partitive articles} (PAs) of French and Italian (both preceding either an indefinite singular mass or an indefinite plural count noun, like in French \textit{du vin}, ‘wine’), we discuss whether and to what extent \textit{niște} and PAs are semantically comparable. The results of a pilot study conducted in 2018 with 33 Romanian native speakers show, first, that similarly to the Italian and unlike the French PAs, the use of \textit{niște} is in no context obligatory, but rather strongly connected to individual preferences. Second, \textit{niște} is never used with preverbal subjects in generic contexts, behaving thus similarly to the Italian PA (facts for the French PA are unclear). Third, and most intriguingly, next to narrow scope readings of \textit{niște} with respect to negation, wide scope is not excluded, neither with plural count nor with mass nouns. This last behavior is unattested for both the Italian and the French singular PA (and related Gallo-Romance languages such as Franco-Proven\c{c}al). Fourth, two specificity-related properties of \textit{niște} seem to depend on the task: While the speakers clearly prefer a noun determined by \textit{niște} over a bare noun in the case of epistemic specificity, they do not produce it actively in the translations. A similar observation can be made for the combination with the (specificity marking) DOM-marker \textit{pe}, which is considered grammatical by the majority, but apparently not preferred in active production. In sum, its scope properties and its apparently strong connection to epistemic specificity make \textit{niște} an element of its own, not comparable with any other Romance indefinite determiner.\end{sloppypar}}

\IfFileExists{../localcommands.tex}{
   \addbibresource{../localbibliography.bib}
   % add all extra packages you need to load to this file

\usepackage{tabularx,multicol}
\usepackage{url}
\urlstyle{same}

\usepackage{listings}
\lstset{basicstyle=\ttfamily,tabsize=2,breaklines=true}

\usepackage{langsci-basic}
\usepackage{langsci-optional}
\usepackage{langsci-lgr}
\usepackage{langsci-osl}
% \usepackage{./langsci/styles/langsci-lgr}
% \usepackage{./langsci/styles/langsci-osl}
% \usepackage{langsci-gb4e}

\usepackage{tikz}
\usetikzlibrary{patterns,calc}
\pgfdeclarepatternformonly{south east lines}{\pgfqpoint{-0pt}{-0pt}}{\pgfqpoint{3pt}{3pt}}{\pgfqpoint{3pt}{3pt}}{
    \pgfsetlinewidth{0.6pt}
    \pgfpathmoveto{\pgfqpoint{0pt}{3pt}}
    \pgfpathlineto{\pgfqpoint{3pt}{0pt}}
    \pgfpathmoveto{\pgfqpoint{.2pt}{-.2pt}}
    \pgfpathlineto{\pgfqpoint{-.2pt}{.2pt}}
    \pgfpathmoveto{\pgfqpoint{3.2pt}{2.8pt}}
    \pgfpathlineto{\pgfqpoint{2.8pt}{3.2pt}}
    \pgfusepath{stroke}}
    
\usepackage{stmaryrd}
\usepackage{wasysym}
\usepackage{multirow}
\usepackage{caption}
\usepackage{subcaption}
\usepackage{mathrsfs}
\usepackage{qtree}

\usepackage{linguex}


   %pminos do not split footnotes
% \interfootnotelinepenalty=10000 %Footnote in Laporte chapters has to be split SN


%\DeclareIndexNameFormat{default}{%
%\nameparts{#1}%
%\usebibmacro{index:name}%
%{\index[names]}%
%{\namepartfamily}%
%{\namepartgiveni}%
% {}% L1
% {}% L2
%{\namepartprefix}% generates spurious space L3
%{\namepartsuffix}% generates spurious space L4
%}

%  {\DeclareIndexNameFormat{default}{%
%     \usebibmacro{index:name}{\index[names]}{#1}{#3}{#5}{#7}}}

%\DeclareIndexNameFormat{default}{%
%  \usebibmacro{index:name}{\sindex[nom]}{#1}{#3}{#5}{#7}}

%\DeclareIndexNameFormat{default}{%
%  \usebibmacro{index:name}{\sindex[person]}{#1}{#3}{#5}{#7}}
%\DeclareIndexNameFormat{default}{%
%\nameparts{#1} \usebibmacro{index:name}{\sindex[person]]}{\namepartfamily}{‌​\namepartgiven}{\nam‌​epartprefix}{\namepa‌​rtsuffix}}

%\newcommand{\smiley}{:)}

%\renewbibmacro*{index:name}[5]{%
%\usebibmacro{index:entry}{#1}%
%{\iffieldundef{usera}{}{\thefield{usera}\actualoperator}\mkbibindexname{#2}{#3}{#4}{#5}}}

% \newcommand{\noop}[1]{}

%remove for final
%\overfullrule=1mm

\newcommand{\tobi}[2]}}
\renewcommand{\S}[1]{\tobi{#1}{\textsc{*}}}

% this volume references
% puts: [this volume]
% already defined: \citetv
%\newcommand{\citepv}[1]{(\citeauthor{#1} \citeyear*{#1} [this volume])}
\newcommand{\citealtv}[1]{\citeauthor{#1} \citeyear*{#1} [this volume]}

%parentheses around example number
\newcommand{\pref}[1]{(\ref{#1})}

% in-text examples

\newcommand{\lnex}[1]{\textit{#1}} %target lang word
\newcommand{\lnlit}[1]{(lit.: `#1')} %literal reading
\newcommand{\lnlat}[1]{(#1)} % latinization
\newcommand{\lntrans}[1]{`#1'} %translation
\newcommand{\lnexl}[2]%
{\lnex{#1}{} \lnlat{#2}} % ex with latinization
\newcommand{\lnexlat}[3]{\lnex{#1}{} \lnlat{#2}{} \lntrans{#3}} % ex with latinization and tranl.

%ch01
\newcommand{\co}[1]{\mbox{\textbf{#1}}}

%ch09

\newcommand{\cyrbulg}[1]{\begin{otherlanguage*}{bulgarian}#1\end{otherlanguage*}}


%ch10
\newcommand{\nlp}{{\small NLP}}
\newcommand{\mwe}{{\small MWE}}
\newcommand{\rae}{{\small RAE}}
\newcommand{\lvc}{{\small LVC}}
\newcommand{\pos}{{\small P}o{\small S}}
%\newcommand{\todo}[1]{ \textcolor{red}{#1} }

%\renewcommand{\labelenumi}{\theenumi}
%\ainamefmt{{vv}{ll}{, ff}{, jj}} % fullname

\newcommand{\biberror}[1]{{\color{red}#1}}

\newcommand{\osenovaitem}{--~}
   %% hyphenation points for line breaks
%% Normally, automatic hyphenation in LaTeX is very good
%% If a word is mis-hyphenated, add it to this file
%%
%% add information to TeX file before \begin{document} with:
%% %% hyphenation points for line breaks
%% Normally, automatic hyphenation in LaTeX is very good
%% If a word is mis-hyphenated, add it to this file
%%
%% add information to TeX file before \begin{document} with:
%% %% hyphenation points for line breaks
%% Normally, automatic hyphenation in LaTeX is very good
%% If a word is mis-hyphenated, add it to this file
%%
%% add information to TeX file before \begin{document} with:
%% \include{localhyphenation}
\hyphenation{
    Beck-man
    Ngu-yen
    back-chan-nel
    back-chan-nels
    mo-not-o-nous
    ste-reo-typ-i-cal
}

\hyphenation{
    Beck-man
    Ngu-yen
    back-chan-nel
    back-chan-nels
    mo-not-o-nous
    ste-reo-typ-i-cal
}

\hyphenation{
    Beck-man
    Ngu-yen
    back-chan-nel
    back-chan-nels
    mo-not-o-nous
    ste-reo-typ-i-cal
}

   \boolfalse{bookcompile}
   \togglepaper[23]%%chapternumber
}{}

\begin{document}
\maketitle

\section{Introduction}\label{sec:ds1}

The Romanian indefinite element \textit{niște} is often associated with the so-called \textit{partitive articles} (\textit{PA} in the following) of Romance languages such as French or Italian, as it seems to have a similar distribution:

\ea\label{ex:ds1}
    \ea\label{ex:ds1a} French\\
    Je bois *(du) vin.\\
    \ex\label{ex:ds1b} Italian\\
    Bevo (del) vino.\\
    \ex\label{ex:ds1c} Romanian \\
    Beau (niște) vin.\\
     \glt ‘I drink (some) wine.’\\
    \z
\z

\ea\label{ex:ds2}
    \ea\label{ex:ds2a} French\\
    Je vois *(des) enfants.
    \ex\label{ex:ds2b} Italian\\
    Vedo (dei) bambini.\\
    \ex\label{ex:ds2c} Romanian\\
    V\u{a}d (pe niște) tineri.\\
     \glt ‘I see (some) children.’
    \z
\z

As we can see from the examples in (\ref{ex:ds1}) and (\ref{ex:ds2}), only French does not allow bare arguments, not even for singular mass nouns (cf. \ref{ex:ds1a}). Semantically, PAs in French and Italian are described as having a narrow scope/non-specific reading in the singular, but wide scope and specific readings available for the plural (cf. \citealt{Ihsane2008}; \cite{CardinalettiGiusti2016}).

In this contribution, which is based on a collection of experimental data (fieldwork, translation, interpretation, grammaticality judgments, cf. \cite{CornipsPoletto2005}), we aim at a detailed semantic description of \textit{niște}, also in a comparative (Romance) perspective.

The paper is structured as follows: After a short summary of semantic descriptions of \textit{niște} available in the literature (\sectref{sec:ds2}), including an introduction to several different notions of specificity and our working definition and a glance at the Romanian DOM-marker \textit{pe}, we present our methodology and data in \sectref{sec:ds3}. \sectref{sec:ds4} summarizes the main results of our fieldwork study, with a focus on the behavior of \textit{niște} with respect to specificity. These results are discussed in \sectref{sec:ds5}, especially in \sectref{sec:ds5.2} and \sectref{sec:ds5.3}, before a short conclusion in \sectref{sec:ds6}, emphasizing the idiosyncratic character of \textit{niște} in a pan-Romance perspective.

\section{State of the art}\label{sec:ds2}
This section is subdivided in five subsections, building the basis for the remainder of the article. After a short presentation of the etymology of \textit{niște} and its first uses in Old Romanian documents in \sectref{sec:ds2.1}, main insights from the spare existing contributions concentrating on its semantics are summed up in \sectref{sec:ds2.2}. \sectref{sec:ds2.3} and \sectref{sec:ds2.4} introduce the notion of \textit{specificity} and the DOM-marker \textit{pe} with its specificity-related properties, both crucial to the understanding of the semantics of \textit{niște}. \sectref{sec:ds2.5} discusses the research questions underlying this article.

\subsection{The origin of \textit{niște}}\label{sec:ds2.1}
The etymology of \textit{niște} is most likely to be found in the Latin expression \textsc{nescio} \textsc{quid} `I don’t know what’.\footnote{An alternative etymology which has been put forward is the evolution from abbreviated Romanian \textit{nu știu ce} ‘I don’t know what’, i.e. \textit{nuș’ ce}, to \textit{niște} \citep[200]{Stan2006}.} The use of \textit{niște} is attested, often also under the form \textit{nește}, already in the first Old Romanian documents from the 16/17\textsuperscript{th} century (cf. \citealt{Stan2006}). From the first attestations onwards, \textit{niște} is used both with plural count (\ref{ex:ds3}) and mass nouns (\ref{ex:ds4}), the second use being, however, less frequent \citep[299, 354]{PanaDindelegan2016}

\ea\label{ex:ds3} Old Romanian \citep[299]{PanaDindelegan2016}\\
\gll Au    venit           nește boiari.\\
     have.3\textsc{pl} come.\textsc{ptcp} \textsc{niște} boyar.\textsc{pl}\\
\glt ‘Some boyars came.’ 
\z

\ea\label{ex:ds4} Old Romanian \citep[354]{PanaDindelegan2016}\\
\gll nește oloi\\
     \textsc{niște}  oil\\
\glt ‘some oil’ 
\z

In addition to these two uses, which still exist in present-day Romanian, another use of \textit{niște} is attested. According to \citet[354]{PanaDindelegan2016}, this use is excluded in Modern Romanian. Consider example (\ref{ex:ds5}), where \textit{niște} precedes a numeral with a plural count noun:

\ea\label{ex:ds5} Old Romanian \citep[354]{PanaDindelegan2016}\\
\gll Niște trei   voinici \\
     \textsc{niște}  three heroe.\textsc{pl}\\
\glt ‘some three heroes’ 
\z


\subsection{Semantic descriptions of \textit{niște}}\label{sec:ds2.2}

First of all, following the existing literature, we have to distinguish between the use of \textit{niște} in combination with mass nouns (conjugated as singular) from the cases where \textit{niște} precedes a plural count noun. For reasons of convenience, we are going to refer to the first one as \textit{niște}\textsc{\textsubscript{sg}} and to the second one as \textit{niște}\textsc{\textsubscript{pl}}. As for the semantics of \textit{niște}\textsc{\textsubscript{sg}}, we find a first approximation of its semantics in  \citet{Gutu2005}:

\begin{quote}
    Pentru a exprima aproximarea nonspecific\u{a}, se folosește în contextul substantivelor masive [...] \textit{niște}, intrând în opoziție semantic\u{a} cu \textit{mult}, \textit{puțin}, care exprim\u{a} aproximarea specific\u{a}.\hfill\hbox{\citep[261]{Gutu2005}}
    
    [In order to express non-specific approximation, in the context of mass nouns \textit{niște} is used, which enters a semantic opposition with \textit{mult}, \textit{puțin}, which express  specific approximation.]
\end{quote}

According to the author, \textit{niște}\textsc{\textsubscript{sg}} thus expresses “non-specific approximation”, which is semantically opposed to other quantifiers expressing “specific approximation”, like \textit{mult} ‘much’ or \textit{puțin} ‘little’. We interpret this in terms of the difference between specific and non-specific quantification; contrary to \textit{mult} (‘much’ = big amount) vs. \textit{puțin} (‘little’ = small amount), \textit{niște} does not inform about the size of the amount of the substance at issue.\footnote{See, however, \citet{DavatzStark2019}, where we show that the speakers seem to have a clear idea of the amount denoted by \textit{niște}. According to our findings, \textit{niște} denotes a small quantity and can be situated on the continuum between \textit{mult} (‘much’) and \textit{puțin} (‘little’).} Following this description, the semantic contribution of \textit{niște}\textsc{\textsubscript{sg}} seems to be restricted to the mere assertion of a certain amount which, however, is left unspecified. This seems to be fine with concrete mass nouns, but more difficult to conceive of with abstract nouns like \textit{talent} ‘talent’ (cf. \citealt[2]{Nedelcu2003}).

Concerning \textit{niște}\textsc{\textsubscript{\textsc{pl}}}, \citet[65]{Dobrovie-SorinGiurgea2013} states the following in their reference grammar:

\begin{quote}
[…] with plural count nouns, \textit{niște} ‘some’ may introduce an individualized plural entity, distinguishable from other plural entities of the same kind.
\end{quote}

The same authors provide two examples to illustrate the difference with respect to \textit{niște}\textsc{\textsubscript{sg}}:\largerpage[-1]\pagebreak

\ea\label{ex:ds6} Modern Romanian\\
\gll Maria a v\u{a}zut niște filme, iar Ion a v\u{a}zut altele.\\
     Mary have.3\textsc{sg} see.\textsc{ptcp} \textsc{niște} movie.\textsc{pl} but John have.\textsc{prs.3sg} see.\textsc{ptcp}  other.\textsc{pl}\\
\glt ‘Mary has seen some movies, but John has seen others.’
\ex\label{ex:ds7} Modern Romanian\\
\gll \#Maria a b\u{a}ut niște vin, iar Ion a b\u{a}ut altul.\\
     Maria have.3\textsc{sg} drink.\textsc{ptcp} \textsc{niște} wine but John have.\textsc{prs.3sg} drink.\textsc{ptcp} other\\
\glt ‘Mary has drunk some wine, but John has drunk another.’\footnote{The sentence is grammatically fine, but the contrast exemplified in (\ref{ex:ds6}) is much harder to obtain with mass nouns.}
\z

In \citegen{Heusinger2002} terms, this seems to be indicative of \textit{referential} or \textit{epistemic specificity} for \textit{niște}\textsc{\textsubscript{\textsc{pl}}}, i.e. the referents of the respective DP have already been introduced in the discourse universe and/or are known by the speaker.

\begin{sloppypar}
Next to its property of introducing an individualized plural entity, \textit{niște}\textsubscript{\textsc{pl}} also seems to be -- at least to a certain degree -- compatible with generic readings, even in preverbal position. \citet[207]{Nedelcu2009} gives the following example:
\end{sloppypar}

\ea\label{ex:ds8} {Modern Romanian}\\
\gll Nu pot s\u{a} cred c\u{a} acei doi sunt milionari. Niște milionari nu c\u{a}l\u{a}toresc la clasa a doua.\\
\textsc{neg} can.\textsc{prs.1sg} \textsc{comp} believe.\textsc{prs.1sg} \textsc{comp} \textsc{dem.mpl} two be.\textsc{prs.3pl} milionnaire.\textsc{pl} \textsc{niște} millionaire.\textsc{pl} \textsc{neg} travel.\textsc{prs.3pl} at class.\textsc{art} second\\
\glt ‘I can’t believe those two are millionaires. Millionaires don’t travel economy class.’
\z

This use is, however, not uncontroversial in the literature. According to \citet[82]{Avram1986}, in the singular both the definite and the indefinite article can be used with a generic reading, like in many Indo-European languages, whereas in the plural, only the definite article can be used with generic interpretation, \textit{niște}\textsc{\textsubscript{\textsc{pl}}} being excluded from the subject position in generic contexts. Its use in (\ref{ex:ds8}) might be explained by the fact that the respective DP (\textit{niște milionari}) does not refer to millionaires in general, but rather to a certain group (of millionaires) (cf. \citealt[207]{Nedelcu2009}). 

Furthermore, unlike bare nouns, i.e. nouns without any determiner, nouns preceded by \textit{niște}\textsc{\textsubscript{\textsc{pl}}} can also have wide scope with respect to intensional predicates. Consider the two readings (a) and (b) of (\ref{ex:ds9}) illustrating narrow scope and wide scope of the DP preceded by \textit{niște}\textsc{\textsubscript{\textsc{pl}}}, respectively \citep[example taken from][63]{Dobrovie-SorinGiurgea2013}:

\ea\label{ex:ds9} Modern Romanian\\
\gll Maria crede     c\u{a}  Petre a     furat niște c\u{a}rți.\\
     Mary believe.\textsc{prs.3sg} \textsc{comp} Peter have.\textsc{prs.3sg} steal.\textsc{ptcp} \textsc{niște} book.\textsc{pl}\\
\glt a. ‘Mary believes that Peter stole books (no matter    which ones).’
\glt b. ‘There are books of which Mary believes that  Peter stole them.’
\z

\subsection{The notion of specificity}\label{sec:ds2.3}

As could easily be seen in example (\ref{ex:ds9}), DPs introduced by \textit{niște} seem to be able to show \textit{scopal specificity}, i.e. specificity that is induced by the interaction of an indefinite with other operators in the sentence, e.g. the predicate ‘believe’ in the case of (\ref{ex:ds9}), universal quantifiers, negation etc. This \textit{scopal specificity} is, however, only one out of four different concepts of \textit{specificity} described in the semantic literature.

Next to \textit{scopal specificity}, which is conditioned by the presence of a variable-binding operator, there is the so-called \textit{epistemic specificity}, which can be best described as a specificity dependent on the speaker’s knowledge. By \textit{epistemically specific} DPs we understand (indefinite) DPs which are “inherently” referential because their referents are known by the speaker at the time of uttering the sentence. Consider (\ref{ex:ds10}) for an English example, given by \citet[260]{Heusinger2002}, where the student referred to as “a student” is known to the speaker:

\ea\label{ex:ds10} English\\
A student in Syntax 1 cheated on the exam. His name is John.\\
\z

In addition to \textit{scopal} and \textit{epistemic specificity}, the literature further lists \textit{partitive specificity} and \textit{relative specificity} as types of specificity (cf. \cite{Farkas1995} and \cite{Heusinger2002}). As for \textit{partitive specificity}, it induces a strong (presuppositional) interpretation of an indefinite DP (both the existence of a superset to which the partitive specific DPs belong, and the existence of their referents, is presupposed). \textit{Partitively specific} indefinite DPs always have wide scope with respect to other operators and can be considered the equivalent of what \citet{Milsark1974} called \textit{strong indefinites}. The sentence in (\ref{ex:ds11}) shows one such example, again taken from \citet[260]{Heusinger2002}:

\ea\label{ex:ds11} English\\
Some ghosts live in the pantry, others live in the kitchen.\\
\z

As clearly illustrated by the sentence in (\ref{ex:ds11}), this partitive interpretation is most easily induced by using a “complementary” pronoun such as ‘others’.\footnote{Note that \textit{partitive specificity} has to be distinguished from cases like Example (\ref{ex:ds6}), where the two DPs do not belong to a common discourse-given set.} In the absence of such a pronoun, the most natural interpretation would be a weak (existential) indefinite, the strong one being limited to cases of non-canonical intonation (\textsc{some} \textit{ghosts live in the pantry}). Importantly, the referent of the indefinite DP is presuppositional, i.e. its existence cannot be negated. However, it does not necessarily have to be known to the speaker.

The last type of specificity discussed in the literature is the so-called \textit{relative specificity}. Relatively specific indefinite DPs are neither wide scope nor referential, but still “specific” as they refer independently from the matrix predicate \citep[cf.][262]{Heusinger2002}:

\ea\label{ex:ds12} English\\
James said that George met a certain student of his. \\
\z

Note that in (\ref{ex:ds12}) there is “referential co-variation” of the variable introduced by the specific indefinite DP \textit{a certain student} and the proper name \textit{George}: The value for \textit{a certain student} is dependent on George, for whom the referent is necessarily specific. On the contrary, James does not have to be able to identify the student.

Against the wide discussion of such examples and consensus in the literature about the existence of these four different types of specificity, we would like to follow the unifying approach of \citet{Heusinger2002}, according to whom \textit{specificity} is best understood as \textit{referential anchoring}:

\begin{quote}
Specificity indicates that an expression is referentially anchored to another object in the discourse. “Referentially anchored” means that the referent of the specific DP is functionally dependent on the referent of another expression.\hfill\hbox{\citep[268]{Heusinger2002}}
\end{quote}

This conception enables us to reconcile three of the four types of specificity, as \citeauthor{Heusinger2002} clearly illustrates with the following example (\citeyear[269]{Heusinger2002}):

\ea\label{ex:ds13} English
    \ea\label{ex:ds13a} Bill gave each student a (certain) task\textsubscript{speaker} to work on.\\
    \ex\label{ex:ds13b} Bill gave each student a (certain) task\textsubscript{Bill} to work on.\\
    \ex\label{ex:ds13c} Bill gave each student (x) a (certain) task\textsubscript{x} to work on.\\
\z
\z

As these different interpretations show, an indefinite DP, or more precisely its index, can be linked to different established indices resulting in different types of specificity. The first interpretation in (\ref{ex:ds13a}) corresponds to the epistemic specificity, where the indefinite DP \textit{a (certain) task} is anchored to the speaker index and thus completely independent of other elements in the sentence. In (\ref{ex:ds13b}), it is anchored to the subject index, i.e. in this case there is a certain task that Bill gave to each student but which may be obscure to the speaker. Its interpretation corresponds thus to a relatively specific DP. In the third reading, the specific DP is anchored to the quantified DP \textit{each student} and the task varies thus from student to student. Accordingly, we are dealing with \textit{scopal specificity} in this case.

\subsection{The DOM-marker \textit{pe} as a specificity-marking element}\label{sec:ds2.4}

An element strongly connected to the notion of specificity is the Romanian DOM-marker \textit{pe}, which is presented briefly in this section. Knowing its properties helps to understand its interaction with \textit{niște}\textsubscript{\textsc{pl}}, which seems, as we have seen, to be linked to (epistemic) specificity as well (cf. \sectref{sec:ds2.2}). As the interplay (and grammaticality) of \textit{pe} and \textit{niște}\textsubscript{\textsc{pl}} is undescribed in the literature, our pilot study involves several examples of a combination of these two elements, allowing us to draw a clearer picture of the semantics of \textit{niște}.

Specificity is a grammatically relevant feature in Romanian, where the DOM-marker \textit{pe} is highly sensitive to the specificity of the referent. Put differently, non-specificity blocks the appearance of \textit{pe} (\cite[303]{ChiriacescuHeusinger2010}; \cite[42]{Stark2011}). Consider (\ref{ex:ds14a}) with a specific DP marked by \textit{pe} vs. (\ref{ex:ds14b}) with a non-specific DP, which has to be unmarked \citep[examples taken from][303--304]{ChiriacescuHeusinger2009}.\footnote{The referent of the indefinite DP in (\ref{ex:ds14b}) could also be interpreted specifically, but a non-specific interpretation of (\ref{ex:ds14a}) is excluded.} Note that the marker \textit{pe} is “tightly related to Clitic Doubling” \citep[393]{HillMardale2017}, cases of \textit{pe} without a co-occurring clitic being considered “marginal” \citep[7]{ChiriacescuHeusinger2009}. In (\ref{ex:ds14a}), the feminine singular clitic \textit{o} doubles the DP \textit{o secretar\u{a}} ‘a secretary’, whereas clitic doubling is excluded in the absence of \textit{pe}, as (\ref{ex:ds14b}) shows: \newpage

\ea\label{ex:ds14} Modern Romanian
    \ea\label{ex:ds14a}
    \gll Ion   o   caut\u{a}          pe o secretar\u{a}.\\
       John \textsc{cl} search.\textsc{prs.3sg} \textit{pe} a  secretary\\
    \glt ‘John is looking for a secretary (whom he knows).’
    \ex\label{ex:ds14b}
    \gll Ion   caut\u{a}             o secretar\u{a}.\\
       John search.\textsc{prs.3sg} a  secretary\\
    \glt ‘John is looking for a secretary.’
    \z
\z

The contrast exemplified in (\ref{ex:ds14}) thus concerns scopal specificity, i.e. the fact that the indefinite DP can scope over the predicate \textit{search}. In fact, it is not possible for a \textit{pe}-marked direct object to scope under extensional/intensional operators. Note that in von Heusinger’s framework, the index of the secretary in (\ref{ex:ds14a}) is referentially anchored to the index of the subject, John.

As for transparent contexts without any operators, epistemic specificity may occasionally trigger \textit{pe}-marking, too. Consider (\ref{ex:ds15}), where according to \citet[443]{HeusingerChiriacescu2013} both the version with and without the marker \textit{pe} could receive the continuation ‘I do not know the friend’ or ‘I do know the friend’:

\ea\label{ex:ds15} Modern Romanian\\
\gll Petru (l-) a               vizitat      (pe) un prieten.\\
     Peter  \textsc{cl} have.\textsc{prs.3sg} visit.\textsc{ptcp} \textit{pe}   a   friend\\
\glt ‘Peter visited a friend.’
\z


To put it in von Heusinger’s framework, the \textit{pe}-marked indefinite direct object can be anchored to the speaker of the utterance, but does not have to be. The same holds for the unmarked indefinite DP.\footnote{If we want to follow \citet[443]{HeusingerChiriacescu2013}, the (subtle) difference between the two forms can be explained by introducing a discourse-based parameter. According to them, “\textit{pe}-marking signals a higher referential persistence”. However, similarly to the observations concerning the specificity effects of \textit{pe}, “the lack of \textit{pe}-marking does not necessarily signal a lower level of referential persistence” \citep[315]{ChiriacescuHeusinger2010}.}

We can thus conclude that “if an indefinite noun phrase is \textit{pe}-marked, it must be scopally [in combination with extensional operators] or referentially [in combination with intensional operators] specific” \citep[305]{ChiriacescuHeusinger2010}. It does not have to be necessarily epistemically specific.

\subsection{Research questions}\label{sec:ds2.5}

This article seeks to contribute to the description of the semantics of \textit{niște}\textsubscript{\textsc{pl}} by closing several gaps identified in the literature and asking new questions. First and foremost, we want to further explore the difference between a bare plural count noun and a DP introduced by \textit{niște}\textsubscript{\textsc{pl}} (in analogy to the discussion in Ita- lian, where DPs introduced by a “partitive article” are semantically opposed to bare nouns). Based on the observation that \textit{niște}\textsubscript{\textsc{pl}} “may introduce an individua- lized plural entity” (cf. \sectref{sec:ds2.2}), we hypothesize that (epistemic) specificity might be a crucial factor when it comes to its use. In other words: (epistemically) specific DPs might tend to be marked by \textit{niște}\textsubscript{\textsc{pl}}. As the same holds true for DPs marked by the DOM-marker \textit{pe} (cf. \sectref{sec:ds2.4}), the question has to be asked whether a combination of the two elements is a priori possible and, if yes, whether this has any effects on the interpretation of the DP.

Second, we want to address the controversial question of \textit{niște}\textsubscript{\textsc{pl}} in generic contexts, and, third, complete the descriptions regarding the scope properties of \textit{niște} (cf. \sectref{sec:ds2.2}), which lack two fundamental aspects: (i) How does \textit{niște}\textsubscript{\textsc{pl}} behave with respect to other quantifiers, and (ii) how does \textit{niște}\textsubscript{} behave with respect to negation? Finding answers to these three questions will allow us to compare \textit{niște}\textsubscript{} with the French and Italian “partitive article” from a semantic point of view.

\section{Methodology and database}\label{sec:ds3}
The following section will present the methodology we used to collect our data (\sectref{sec:ds3.1}) and the data on which our findings are based (\sectref{sec:ds3.2}).

\subsection{Methodology}\label{sec:ds3.1}
In order to gather data that could be used for the description of the semantic pro- perties of \textit{niște}, we designed a questionnaire consisting of four different tasks: (i) translation, (ii) interpretation, (iii) preference and (iv) grammaticality judgments. The tasks had to be done by the speakers in the order just mentioned, avoiding thus a bias in the translation task. (i) was composed of 31 German sentences which had to be translated into Romanian. The 31 sentences contained, all in all, 9 mass nouns and 17 count nouns without any kind of determiner (some sentences containing both types of nominals) which in principle could be translated either by a bare noun or a DP introduced by \textit{niște}. Two mass nouns (\textit{etwas Kürbis} ‘some squash’, \textit{ein wenig Wein} ‘some wine’) and 3 count nouns (\textit{ein paar wenige Fehler} ‘some few mistakes’, \textit{einige Leute} ‘some people’, \textit{ein paar wenige Krümel} ‘some few breadcrumbs’) were introduced by one or more quantifiers, which could be translated by \textit{niște} or other quantifiers. Thirty-one nominals could thus in principle be translated by a DP introduced by \textit{niște}. Additionally, 2 mass nouns which were part of a partitive construction (\textit{von diesem Kuchen} ‘of this cake’, \textit{von seinem Bier} ‘of his beer’) and one count noun introduced by a (colloquially modified) numeral (\textit{so} \textit{drei Idioten} ‘(some) three idiots’) were added as fillers. 

Both mass nouns and count nouns were tested in direct object position, as prepositional and presentational complements, with stage-level predicates (expressing transitory properties, cf. \cite{Carlson1977}) and in generic and negative contexts. The goal of the translation task was thus to identify possible syntactic contexts where \textit{niște} is obligatory.

The interpretation task (ii) consisted of 6 different Romanian sentences whose respective interpretations had to be indicated by the participants.\footnote{Four out of these 6 sentences, all of them with a binary choice for the participants regarding their interpretation, served to test the scope properties of \textit{niște} (cf. \sectref{sec:ds4.4}). Additionally, one sentence focused on the collective vs. distributive interpretation of \textit{niște}\textsc{\textsubscript{\textsc{pl}}} (cf. footnote~\ref{fn17}) and one on the quantitative interpretation of \textit{niște}\textsc{\textsubscript{sg} }(cf. \citealt{DavatzStark2019}).}

The preference task (iii) was composed of 9 different sentences: 6 sentences contrasted the use of a bare noun, i.e. a noun without any kind of determiner, and the use of a DP introduced by \textit{niște}. Half of these sentences contained a mass noun in different syntactic contexts (preverbal subject of a generic sentence, presentational complement and direct object), half of them contained a count noun (two times in direct object position, once as a presentational complement). The spea- kers had to indicate whether they prefer the version with \textit{niște} or the one without. One sentence focused on the presence vs. absence of the DOM-marker \textit{pe} (and clitic doubling) in combination with \textit{niște}, i.e., \textit{niște} was present in all three versions of the sentence. In addition to these two types, there were two sentences testing word order properties, which are not discussed in the remainder of this article. The preference task was meant to complement the translation task and check whether the participants behave according to their active productions.

Test set IV consisted of 40 sentences containing an occurrence of \textit{niște}, whose grammaticality had to be judged by the speakers on a reduced Likert scale from 0 (= “I don’t understand the sentence”) to 3 (= “the sentence is well formed”). Reducing the scale to 4 values should prevent the speakers from spending too much time on thinking about slight and – for our purposes – irrelevant differences regarding the “usualness” of a sentence and allow them to focus on the difference between grammatical and ungrammatical. However, it seemed important to us to give them the possibility to indicate in case they had not understood the sentence (which could point to its ungrammaticality) or if a sentence is grammatical, but (very) unusual. One of our main concerns here was the possibility of a combination of \textit{niște} with the DOM-marker \textit{pe} (cf. \sectref{sec:ds2.4}). The participants could take the time they needed to answer the questionnaire in written form.

\subsection{Data}\label{sec:ds3.2}
The questionnaire presented above was used in fieldwork in March 2018 at the Babeș-Bolyai University in Cluj (Romania). All in all, we have data from 61 students of German philology, 32 of them Romanian native speakers and 29 Hungarian native speakers. In the remainder of this article, only the former will be considered.\footnote{For a discussion of the results of the Hungarian participants see \citet{Davatz2018}.} 

The 32 questionnaires from the Romanian native speakers contain altogether 908 valid translations of the 29 relevant nominals in the translation task, of which 44 are introduced by \textit{niște}. For the task regarding the preference of the presence\slash absence of \textit{niște}, which in principle should have generated a total of 192 responses ($32 \times 6$), we count 199 responses, 126 of which contain \textit{niște}. This is due to the fact that various speakers left out some examples, while others accepted both versions, especially with the ones with presentational constructions: \textit{Nou\u{a} ne place când este (niște) z\u{a}pad\u{a}} ‘We like it when there is (some) snow’ and \textit{Sunt (niște) oameni pe lumea asta care nu te-ar ajuta niciodat\u{a}} ‘There are (some) people in this world who would never help you’. Counting only the examples where one single version is indicated as correct, there are 110 occurrences of \textit{niște} and 57 occurrences of bare nouns.

As far as the interpretation task and the grammaticality judgment are concerned, we have valid and unambiguous data from all speakers.


\section{Results}\label{sec:ds4}
This section presents the findings concerning the semantic properties of \textit{niște} that result from our study. In \sectref{sec:ds4.1} we first show some general insights from the translation task regarding the use of \textit{niște} in active production. \sectref{sec:ds4.2} is concerned with the results regarding the use of \textit{niște} with generic nominals in the translation task. The following \sectref{sec:ds4.3}. treats the use of \textit{niște} with specific nominals and is divided into \sectref{sec:ds4.3.1}, focusing on the results of the preference task, and \sectref{sec:ds4.3.2}, showing the findings from the translation task. \sectref{sec:ds4.4} presents the scope properties of \textit{niște} resulting from the interpretation task. Finally, \sectref{sec:ds4.5} concerning the combination of \textit{niște} with the DOM-marker \textit{pe} is again divided in two subsubsections \sectref{sec:ds4.5.1} and \sectref{sec:ds4.5.2}, which show the results of the preference task and the grammaticality judgment task, respectively.

\subsection{Generalities}\label{sec:ds4.1}
The first, very general, but nonetheless important and new finding resulting from the translation task is that there does not seem to be any syntactic context in which the use of \textit{niște} is obligatory. \textit{Niște} is used only in 44 out of the relevant 908 translations in the respective task, which amounts to 5\%. Interestingly, there is no difference regarding the frequency of its use between (singular) mass nouns and (plural) count nouns: With count nouns, \textit{niște} is used in 31 out of 620 translations (= 5\%), with mass nouns in 13 out of 288 possible cases, which amounts to exactly the same ratio. The results show thus that in the vast majority of the cases the participants (i) prefer a bare noun to the use of \textit{niște} and (ii) would rather use a quantifier different from \textit{niște} in active production. As far as its apparent optionality is concerned, it resembles \textit{prima facie} the PA of Standard Italian, which use is traditionally said to be non-obligatory as well.

These preliminary findings are, however, not confined to the simple observation that \textit{niște} is never obligatory, but show furthermore that its use seems to be strongly connected to individual preferences. There are two crucial numbers supporting this observation: (i) Only 19 out of 32 speakers used \textit{niște} at least once in their translations, which means that more than one third of the speakers did not make use of it at all, and (ii) only 5 out of the 32 speakers (= 16\%) are responsible for 27 of the 44 occurrences (= 61\%). Since the only controlled sociolinguistic variable was the education of the speakers, other extralinguistic variables such as diastratic or diatopic factors might play a role in the use of \textit{niște}.\footnote{With plural abstract nouns, the use of \textit{niște} is generally not recommended \citep[79]{Avram1986} and to be understood as a stylistic means to express irony or, in some cases, admiration \citep[4--6]{Nedelcu2003}. However, according to \citet[5]{Nedelcu2003}, there is a tendency in colloquial registers, and even in the media, that \textit{niște} is used also with plural abstract nouns in stylistically unmarked contexts, being reduced to its function of a mere indefinite determiner. It is thus by no means excluded that the use of \textit{niște} could be influenced either by diastratic or stylistic factors also with plural concrete nouns.} See \tabref{tab:ds1}, which shows the distribution of \textit{niște} over the different sentences among the 9 speakers using it more than just once.


\begin{table}
\begin{tabular}{cccccccccccccc}
\lsptoprule
 & \multicolumn{12}{c}{Number of the sentence in the questionnaire} & \\\cmidrule(lr){2-13}
Speaker & 3 & 5 & 6 & 12 & 13 & 17 & 18 & 25 & 27 & 28 & 29 & 30 & Total\\\midrule
 1 & \ding{52} & \ding{52} &  & \ding{52} &  & \ding{52} & \ding{52} &  &  & \ding{52} & \ding{52} & \ding{52} & 8\\
 2 & \ding{52} & \ding{52} &  & \ding{52} &  &  &  &  & \ding{52} &  &  & \ding{52} & 5\\
 3 & \ding{52} &  &  &  &  &  & \ding{52} &  & \ding{52} &  & \ding{52} & \ding{52} & 5\\
 4 &  & \ding{52} &  &  & \ding{52} & \ding{52} &  &  &  & \ding{52} &  & \ding{52} & 5\\
 5 & \ding{52} &  &  &  &  &  &  & \ding{52} & \ding{52} &  &  & \ding{52} & 4\\
 6 &  & \ding{52} &  &  &  &  &  &  &  & \ding{52} &  &  & 2\\
 7 & \ding{52} &  &  &  & \ding{52} &  &  &  &  &  &  &  & 2\\
 8 & \ding{52} &  &  &  & \ding{52} &  &  &  &  &  &  &  & 2\\
 9 & \ding{52} &  & \ding{52} &  &  &  &  &  &  &  &  &  & 2\\
\midrule
 & 7 & 4 & 1 & 2 & 3 & 2 & 2 & 1 & 3 & 3 & 2 & 5 & 35\\
\lspbottomrule
\end{tabular}
\caption{Usages of \textit{niște} over the different sentences among the speakers using it more than once (translation task) \citep[cf.][39]{Davatz2018}}
\label{tab:ds1}
\end{table}

\subsection{Use of \textit{niște} with generic nominals}\label{sec:ds4.2}

The results concerning the relevant input sentence in the translation task show clearly that the use of \textit{niște} does not seem to be compatible with a generically interpreted nominal. In 27 out of the 28 valid translations we find the definite article, and there is no single translation making use of \textit{niște}. Consider the German input sentence in (\ref{ex:ds16}) and in (\ref{ex:ds17}) an example of a typical translation by the informants:\footnote{To avoid overloading the questionnaire of this pilot study, we did not test every context using all the different tasks. The results of the translation task are thus the only results we have concerning the possibility to use \textit{niște} in combination with generic plural nominals. The preference task includes, however, a generically interpreted singular nominal in a sentence translated as “Rice is more nourishing than polenta”. The results show that \textit{niște} seems to be possible in such cases but that the use of the definite article is strongly preferred. Furthermore, \textit{niște} seems to indicate rather a (small) quantity than pure indefiniteness (cf. \cite{Davatz2018}, \cite{DavatzStark2019}).}

\ea\label{ex:ds16} German\\
Ich kann nicht glauben, dass diese zwei Millionäre sind. Millionäre reisen nicht in der zweiten Klasse.\\
\glt ‘I can’t believe those two are millionaires. Millionaires don’t travel economy class.’
\ex \label{ex:ds17} Modern Romanian\\
    \gll Nu pot s\u{a} cred c\u{a} \u{a}știa doi sunt milionari. Milionarii nu c\u{a}l\u{a}toresc       cu clasa a doua.\\
     \textsc{neg} can.\textsc{prs}.1\textsc{sg} \textsc{comp} believe.\textsc{prs}.1\textsc{sg} \textsc{comp} \textsc{dem.mpl}  two be.\textsc{prs.}3\textsc{pl} millionaire.\textsc{pl} millionaire.\textsc{pl.art} \textsc{neg} travel.\textsc{prs}.3\textsc{pl} with class.\textsc{art} second\\
\z

\subsection{Use of \textit{niște} with specific nominals}\label{sec:ds4.3}

In order to allow for a solid comparison between the results of the different tasks, the results concerning the use of \textit{niște} with specific nominals are subdivided in a subsection presenting the results of the preference task and one dealing with the findings from the translation task. This also allows us to highlight the importance of the type of task the informants are given.

\subsubsection{Preference task}\label{sec:ds4.3.1}
There are two different preference tasks in the questionnaire concerning the use of \textit{niște} with specific nominals. The first one contrasts a bare noun and a noun phrase preceded by \textit{niște} in the context of an epistemically specific object. (\ref{ex:ds18}) illustrates the two options between which the speakers had to decide:

\ea\label{ex:ds18} Modern Romanian\\
\gll Mama a \^{\i}nt\^{a}lnit (niște) membri ai parlamentului: Șerban și Ioan.\\
     mother.\textsc{art} have.\textsc{prs.}3\textsc{sg} meet.\textsc{ptcp} \textsc{niște}    member.\textsc{pl} \textsc{agr} parliament.\textsc{art.gen}  Șerban and Ioan\\
\glt ‘Mum has met some members of the parliament: șerban and Ioan.’
\z

The numbers show a predominant preference for \textit{niște} with such specific object nominals, as \tabref{tab:ds2} clearly illustrates.


\begin{table}
%\small
\begin{tabular}{ll}
\lsptoprule
Bare noun & \textit{niște}\\
\midrule
4 (12\%) & 29 (88\%)\\
\lspbottomrule
\end{tabular}
\caption{Preferences with respect to presence/absence of \textit{niște} with a specific object nominal}
\label{tab:ds2}
\end{table}

The second preference input item contrasts again a bare noun with a noun preceded by \textit{niște}, but in this case, the context is slightly different. The object noun phrase does not necessarily denote a specific referent, but it is resumed by an anaphoric pronoun. Example (\ref{ex:ds19}) shows the two options the informants were given, one with \textit{niște} and one without:

\ea\label{ex:ds19} Modern Romanian\\
\gll Am v\u{a}zut (niște) tineri \^{i}n fața cl\u{a}dirii. Numai doi  dintre ei m-au       salutat.\\
     have.\textsc{prs}.1\textsc{sg} see.\textsc{ptcp} \textsc{niște} teenager\textsc{.pl} in face.\textsc{art} building.\textsc{gen} only     two of.between they me.\textsc{acc}{}-have.\textsc{prs}.3\textsc{pl} greet.\textsc{ptcp}\\
\glt ‘I saw some teenagers in front of the building. Only two of them greeted me.’
\z

The numbers are again very clear. In the overwhelming majority of cases, the speakers opted for the version containing \textit{niște}, as \tabref{tab:ds3} shows.

\begin{table}
\begin{tabular}{cc}
\lsptoprule
Bare noun & \textit{niște}\\\midrule
2 (6\%) & 30 (94\%)\\
\lspbottomrule
\end{tabular}
\caption{Preferences with respect to presence/absence of \textit{niște} with ca- taphorical DPs}
\label{tab:ds3}
\end{table}

\subsubsection{Translation task}\label{sec:ds4.3.2}
The results of the translation task concerning the use of \textit{niște} with specific nominals stem from two different sentences in the questionnaire. In one sentence, the respective DP is in direct object position (\ref{ex:ds20}), in the other the DP is the agent adjunct of the passivized verb \textit{einladen} ‘to invite’ (\ref{ex:ds21}). Note that the DP is in both cases epistemically specific, in that the person uttering the respective sentence knows the persons the respective nominals denote:

\ea\label{ex:ds20} German\\
\gll Im Restaurant habe ich Nachbarn getroffen, die du auch kennst: Paul und Erich.\\
     in.\textsc{art} restaurant   have.\textsc{prs}.1\textsc{sg} I    neighbor.\textsc{pl} meet.\textsc{ptcp} \textsc{rel} you also   know.\textsc{prs}.2\textsc{sg} Paul  and Eric\\
\glt ‘In the restaurant I met neighbors you know, too: Paul and Eric.’
\ex\label{ex:ds21} German\\
\gll Gestern wurde ich von Freunden eingeladen, die ich letztes Jahr kennengelernt habe, Lena und Marc.\\
     yesterday become.\textsc{pst.1sg} I by friend.\textsc{pl} invite.\textsc{ptcp} \textsc{rel} I    last   year get.to.know.\textsc{ptcp} have.\textsc{prs.}1\textsc{sg} Lena and Marc\\
\glt ‘Yesterday I was invited by friends whom I got to know last year, Lena and Marc.’
\z

The numbers in \tabref{tab:ds4} show not only that about half of the informants opt for a definite article in the Romanian translation, but also that the use of a bare noun is still more frequent than the use of \textit{niște}.

\begin{table}
\small
\begin{tabularx}{\textwidth}{lXXXXX}
\lsptoprule
Syntactic function & Definite article & Bare noun & \textit{niște} & Numeral & Demon-strative\\
\midrule
Direct object    & 15 (48\%) & 9 (29\%) & 4 (13\%) & 2 (7\%) & 1 (3\%)\\
Agent complement & 15 (50\%) & 8 (27\%) & 6 (20\%) & 1 (3\%) &  --\\
\lspbottomrule
\end{tabularx}
\caption{Translations of epistemically specific indefinite German DPs}
\label{tab:ds4}
\end{table}

\subsection{Scope properties}\label{sec:ds4.4}
The results concerning the scope properties of \textit{niște} are subdivided in one subsection dealing with the scope properties of \textit{niște} with respect to negation (\sectref{sec:ds4.4.1}) and one focusing on contexts where \textit{niște} interacts with quantifiers (\sectref{sec:ds4.4.2}). The results of this whole section only stem from interpretation tasks.

\subsubsection{Scope properties with respect to negation}\label{sec:ds4.4.1}

As for the scopal behavior of \textit{niște} in the context of negation, we have to distinguish between the results concerning \textit{niște} in combination with a mass noun (\ref{ex:ds22}) and the ones dealing with \textit{niște} preceding a plural count noun (\ref{ex:ds23}):

\ea\label{ex:ds22} Modern Romanian\\
\gll N-am b\u{a}ut niște vin.\\
     \textsc{neg}{}-have\textsc{.prs.}1\textsc{sg} drink.\textsc{ptcp} \textsc{niște} wine\\
\glt ‘I didn’t drink (any) wine.’
\pagebreak
\ex \label{ex:ds23} Modern Romanian\\
\gll   N-am v\u{a}zut niște tineri.\\
     \textsc{neg{}-}have.\textsc{prs}.1\textsc{sg} see.\textsc{ptcp} \textsc{niște} teenager.\textsc{pl}\\
\glt ‘I didn’t see (any) teenagers.’
\z

In both cases the speakers were given two possible continuations of the respective sentence, one corresponding to a narrow-scope interpretation of \textit{niște} (and thus putting the DP on a par with the corresponding bare noun), the other one to a reading where \textit{niște} takes wide scope over the negation. In other words, in the latter interpretation, there was some (kind of) wine that was not drunk and some teenagers who were not seen, respectively.

The figures in \tabref{tab:ds5} show rather clearly that the former reading is preferred both with mass and plural count nouns, but that -- at least for some speakers -- \textit{niște} can also take wide scope with respect to negation.

\begin{table}
%\small
\begin{tabular}{lll}
\lsptoprule
Type of noun & Narrow scope & Wide scope\\
\midrule
Mass noun & 27 (84\%) & 5 (16\%)\\
(Plural) count noun & 25 (78\%) & 7 (22\%)\\
\lspbottomrule
\end{tabular}
\caption{Scope properties of \textit{niște} with respect to negation (mass vs. count nouns)}
\label{tab:ds5}
\end{table}

\subsubsection{Scope properties with respect to quantifiers}\label{sec:ds4.4.2}
The results concerning the scopal behavior of \textit{niște} in the context of a quantifier stem from an interpretation task focusing exclusively on plural count nouns. As in the task described in the previous subsection, the speakers were given two sentences and two different readings from which they had to choose the one they preferred. As the position of the subject plays a crucial role in information structure in Romanian \citep[cf.][902]{Leonetti2017} and might thus distort the findings, both postverbal (\ref{ex:ds24}) and preverbal (\ref{ex:ds25}) subjects have been included in the test:

\ea\label{ex:ds24} Modern Romanian\\
\gll   \^{I}n fiecare duminic\u{a} vin niște prieteni s\u{a} ne viziteze.\\
     in every Sunday come.\textsc{prs.}3\textsc{pl} \textsc{niște} friend.\textsc{pl} \textsc{comp} us.\textsc{acc} visit.\textsc{prs}.\textsc{sbjv}.3\textsc{pl}\\
\glt ‘Every Sunday some friends come to visit us.’
\z

\ea\label{ex:ds25} Modern Romanian\\
\gll Niște copii vin s\u{a} se joace aici în fiecare zi.\\
     \textsc{niște} kid.\textsc{pl} come.\textsc{prs}.3\textsc{pl} \textsc{comp} \textsc{refl} play.\textsc{prs.sbjv}.3\textsc{pl} here in every day\\
\glt ‘Some kids come to play here every day.’
\z

Unlike in the task focusing on the scopal behavior with respect to negation, the two options contained the two different possible interpretations the sentence can have, rather than possible continuations. One interpretation corresponded to a reading where \textit{niște} takes narrow scope over the quantifier and the other one to a wide-scope behavior of \textit{niște}. The results differ clearly from the results concerning the scope properties of \textit{niște} in the context of negation, in that \textit{niște} does not seem to show any scope preferences at all -- neither with a preverbal nor with a postverbal subject.\footnote{As a reviewer pointed out, it might also be possible that the participants just weren’t sensitive to these distinctions.} This is illustrated in \tabref{tab:ds6}, which contains the result for the reactions to (\ref{ex:ds24}) and (\ref{ex:ds25}).

\begin{table}
%\small
\begin{tabular}{lll}
\lsptoprule
Subject position & Narrow scope & Wide scope\\
\midrule
preverbal & 16 (50\%) & 16 (50\%)\\
postverbal & 17 (53\%) & 15 (47\%)\\
\lspbottomrule
\end{tabular}
\caption{Scope properties of \textit{niște} with respect to quantifiers}
\label{tab:ds6}
\end{table}

\subsection{Combination with the DOM-marker \textit{pe}}\label{sec:ds4.5}
As far as the possible combination of \textit{niște} with the DOM-marker \textit{pe} is concerned, we have results from two different tasks: \sectref{sec:ds4.5.1} presents the one from the preference task, \sectref{sec:ds4.5.2} the one from the grammaticality judgment task. Again, the findings resulting from the different tasks differ remarkably from one another.

\subsubsection{Preference task}\label{sec:ds4.5.1}

There are, at least in principle, three conceivable possibilities when it comes to direct objects referring to a human being, depending on its degree of specificity: (i) One version with the DOM marker \textit{pe} but no additional clitic (which is, according to the literature, only marginally accepted, cf. \sectref{sec:ds2.4}); (ii) one with both clitic doubling and \textit{pe}; and (iii) one with neither of them. The results of the preference task stem from one sentence in the questionnaire presenting these different versions, of which the informants had to indicate the one they preferred. Example (\ref{ex:ds26}) subsumes the versions (i) and (ii), (\ref{ex:ds27}) shows version (iii):

\ea\label{ex:ds26} Modern Romanian\\
\gll   Ieri     (i{}-)a             v\u{a}zut     pe niște studenți    în  bibliotec\u{a}.\\
     yesterday \textsc{cl}{}-have.\textsc{prs}.3\textsc{sg} see.\textsc{ptcp} \textit{pe} \textsc{niște} student.\textsc{pl} in library\\
\glt ‘Yesterday he saw some students in the library.’
\ex\label{ex:ds27} Modern Romanian\\
\gll Ieri     a        v\u{a}zut       niște studenți   în bibliotec\u{a}.\\
     yesterday have.\textsc{prs}.3\textsc{sg} see.\textsc{ptcp} \textsc{niște} student.\textsc{pl} in library\\
\glt ‘Yesterday he saw some students in the library.’
\z

The results in \tabref{tab:ds7} show a clear preference for the absence of \textit{pe} (iii) in combination with \textit{niște}. However, version (ii) with \textit{pe} and clitic doubling seems acceptable as well. Interestingly, two speakers even prefer version (i).

\begin{table}
%\small
\begin{tabular}{lll}
\lsptoprule
(i) [+\textit{pe}] & (ii) [+\textsc{cl}, +\textit{pe}] & (iii) [-\textit{pe}]\\
\midrule
2 (6\%) & 7 (20\%) & 25 (74\%)\\
\lspbottomrule
\end{tabular}
\caption{Preferences with respect to the presence/absence of \textit{pe} and \textsc{CL} in combination with \textit{niște}}
\label{tab:ds7}
\end{table}

\subsubsection{Grammaticality judgment task}\label{sec:ds4.5.2}
The results stemming from the grammaticality judgment task show a somewhat different picture insofar as the combination of \textit{niște} and the DOM-marker \textit{pe} and an additional clitic is not only regarded as “grammatical, but uncommon”, but even as “unproblematic” by the majority of the speakers. Consider the sentence (\ref{ex:ds28}) and its judgments (\tabref{tab:ds8}):\largerpage

\ea\label{ex:ds28} Modern Romanian\\
\gll Ieri      i-a       v\u{a}zut       pe niște nepoți           de-ai   s\u{a}i        în bibliotec\u{a}.\\
     Yesterday \textsc{cl}{}-have.\textsc{prs.3sg}   see.\textsc{ptcp} \textit{pe} \textsc{niște} grandchild.\textsc{pl} of-\textsc{agr}   \textsc{poss}.\textsc{mpl} in library\\
\glt ‘Yesterday he saw some of his grandchildren in the library.’
\z

\begin{table}
%\small
\begin{tabularx}{\textwidth}{XXXX}
\lsptoprule
Incomprehensible & Impossible & Possible, \linebreak but unusual & Fully \linebreak \mbox{unproblematic}\\
\midrule
1 (3\%) & 1 (3\%) & 4 (13\%) & 26 (81\%)\\
\lspbottomrule
\end{tabularx}
\caption{Grammaticality judgment concerning the combination of \textit{niște} with \textit{pe} and a clitic (direct object \textit{in situ})}
\label{tab:ds8}
\end{table}

The usualness of the combination seems, however, to depend at least partially also on the syntactic function. If the noun phrase is in object predicative complement position (of verbs such as ‘to consider as’), the combination is considered equally grammatical, but much more unusual (\tabref{tab:ds9}).

\begin{table}
%\small
\begin{tabularx}{\textwidth}{XXXX}
\lsptoprule
Incomprehensible & Impossible & Possible, \linebreak but unusual & Fully \linebreak \mbox{unproblematic}\\
\midrule
-- & 1 (3\%) & 13 (41\%) & 18 (56\%)\\
\lspbottomrule
\end{tabularx}
\caption{Grammaticality judgment concerning the combination of \textit{niște} with \textit{pe} and a clitic (predicative complement \textit{in situ})}
\label{tab:ds9}
\end{table}

A similar effect can be observed in the case of clitic left-dislocation, as in sentence (\ref{ex:ds29}).

\ea\label{ex:ds29} Modern Romanian\\
\gll Mi{}-a       zis       c\u{a}       pe niște copii   îi cunoaște   de foarte mult   timp.\\
     me.\textsc{dat-}have.\textsc{prs.3sg} tell.\textsc{ptcp} \textsc{comp} \textit{pe} \textsc{niște} child.\textsc{pl}   \textsc{cl} know.\textsc{prs.3sg}   of very    much time\\
\glt ‘He told me that he’d known some kids for a very long time.’
\z

Consider \tabref{tab:ds10}, containing the results of the judgments for (\ref{ex:ds29}) and two other sentences with a clitic left-dislocated noun phrase preceded by \textit{pe} and \textit{niște}.\largerpage

\begin{table}[H]
%\small
\begin{tabularx}{\textwidth}{XXXX}
\lsptoprule
Incomprehensible & Impossible & Possible, \linebreak but unusual & Fully \linebreak \mbox{unproblematic}\\
\midrule
1 (1\%) & 14 (15\%) & 23 (24\%) & 57 (60\%)\\
\lspbottomrule
\end{tabularx}
\caption{Grammaticality judgment concerning the combination of \textit{niște} with \textit{pe} (clitic left-dislocated direct object)}
\label{tab:ds10}
\end{table}

Finally, when the noun phrase is clitic left-dislocated and its referent expli- citly contrasted with another referent, average judgments of grammaticality sink significantly. Consider sentence (\ref{ex:ds30}) and \tabref{tab:ds11}, showing the results of the respective judgments given by the speakers:

\ea\label{ex:ds30} Modern Romanian\\
\gll Pe niște copii  i-am          v\u{a}zut,      restul clasei     era           deja   plecat\u{a}.\\
     \textit{pe} \textsc{niște} kid.\textsc{pl} \textsc{cl}{}-have.\textsc{prs.1sg} see.\textsc{ptcp} rest.\textsc{art} class.\textsc{art.gen}   be.\textsc{pst}.3\textsc{sg} already   leave.\textsc{ptcp.fsg}\\
\glt ‘I saw some kids; the rest of the class had already left.’
\z

\begin{table}
%\small
\begin{tabularx}{\textwidth}{XXXX}
\lsptoprule
Incomprehensible & Impossible & Possible, \linebreak but unusual & Fully \linebreak \mbox{unproblematic}\\
\midrule
-- & 9 (30\%) & 7 (23\%) & 14 (47\%)\\
\lspbottomrule
\end{tabularx}
\caption{Grammaticality judgment concerning the combination of \textit{niște} with \textit{pe} (clitic left dislocated contrasted direct object)}
\label{tab:ds11}
\end{table}

\section{Discussion}\label{sec:ds5}

The aim of this section is to discuss the results presented in the previous chapter in some further detail. \sectref{sec:ds5.1} is primarily concerned with the discussion of general findings, such as the non-obligatoriness – or, put differently, the frequent preference of a bare noun over a noun preceded by \textit{niște} – and the apparent impossibility of using \textit{niște} with generic nominals. However, it also tries to shed light on the general semantics of \textit{niște} by discussing data stemming from introspection, i.e. the comments which the speakers were asked to make in the questionnaire. In \sectref{sec:ds5.2} we turn our attention to the actual core topic of this chapter: the specificity-related properties of \textit{niște}. We discuss the points supporting an analysis of \textit{niște} as a specificity marker and the counterarguments some of our results represent. \sectref{sec:ds5.3} discusses the scope properties of \textit{niște} and compares them with the scope properties reported for the French and Italian PA. 


\subsection{Generalities}\label{sec:ds5.1}
The first general and important observation emerging from the results of the translation task is the fact that \textit{niște} is always optional, regardless of the syntactic function of the DP of which it is a part.\footnote{For the sake of completeness, it has to be mentioned that there is, in fact, one context where the use of \textit{niște} seems obligatory, namely in combination with the comparative adverbial \textit{ca} ‘like’. As a consequence, \textit{ca} \textit{niște st\u{a}pâni} ‘like (some) rulers’ is different from \textit{ca} \textit{st\u{a}pâni} ‘as rulers’ \citep[cf.][82]{Avram1986}. This context is, however, not part of our study.} It is only rarely used in active production and seems to be subject to individual preferences. In this respect, \textit{niște} resembles the PA of Standard Italian, the use of which is said to be optional as well, at least from a purely syntactic viewpoint.

Let us now have a look at the sentences containing a plural count noun, which were translated at least three times by making use of \textit{niște}.\footnote{The other two sentences which generated three or more translations showing \textit{niște} contain a quantifier preceding a mass noun. In fact, the DP \textit{etwas Kürbis} ‘some squash’ produced the highest number of occurrences of \textit{niște} (12 occurrences). As this paper is more concerned with specificity-related properties of \textit{niște}, we will not discuss this data any further here.} 

The six input sentences in (\ref{ex:ds31}--\ref{ex:ds36}) induced 23 occurrences of \textit{niște}, which is more than half of all the occurrences found in the translation task. For reasons of convenience, the two sentences already given in (\ref{ex:ds20}) and (\ref{ex:ds21}) are repeated here as (\ref{ex:ds31}) and (\ref{ex:ds32}).\largerpage

\ea\label{ex:ds31} German\\
\gll Im Restaurant habe ich Nachbarn getroffen, die du auch kennst: Paul und Erich.\\
     in.\textsc{art} restaurant   have.\textsc{prs}.1\textsc{sg} I neighbor.\textsc{pl} meet.\textsc{ptcp} \textsc{rel} you also   know.\textsc{prs.}2\textsc{sg}  Paul  and Eric\\
\glt ‘In the restaurant I met neighbors you know too: Paul and Eric.’ 
\ex\label{ex:ds32} German\\
\gll Gestern wurde ich von Freunden eingeladen, die ich letztes Jahr kennengelernt habe, Lena und Marc.\\
     yesterday become.\textsc{pst.1sg} I      by   friend.\textsc{pl}  invite.\textsc{ptcp} \textsc{rel} I    last   year get.to.know.\textsc{ptcp} have.\textsc{prs}.1\textsc{sg} Lena and Marc\\
\glt ‘Yesterday I was invited by friends whom I got to know last year, Lena and Marc.’
\ex \label{ex:ds33} German\\
\gll Da waren Kinder im Laden, die ihre Mutter suchten.\\
     there be.\textsc{pst.3pl} child.\textsc{pl}   in.\textsc{art} store    \textsc{rel} their mother search.\textsc{pst.3pl}\\
\glt ‘There were children in the store who were looking for their mother.’
\ex\label{ex:ds34} German\\
\gll Äpfel hätte ich auch noch gerne.\\
     apple.\textsc{pl} have.\textsc{cond.1sg} I   also   still  please \\
\glt ‘I’d also like to have some apples.’
\ex\label{ex:ds35} German\\
\gll Auf diesem Teller gibt es Eier.\\
     on  \textsc{dem.masc.sg} plate  give.\textsc{prs.3sg} it  egg.\textsc{pl}\\
\glt ‘There are eggs on this plate.’
\ex\label{ex:ds36} German\\
\gll Es sind nur ein paar wenige Krümel übriggeblieben im Teller.\\
     it be.\textsc{prs.3pl} only a    few   little.\textsc{pl} crumb.\textsc{pl} leave.over.\textsc{ptcp} in.\textsc{art} plate\\
\glt ‘There are only some few crumbs left in the plate.’
\z

As has already been mentioned in \sectref{sec:ds4.3.2}, in the translations of (\ref{ex:ds31}) and (\ref{ex:ds32}), \textit{niște} competes not only with a bare noun, but to an even bigger extent with the definite article. The sentences (\ref{ex:ds31}) and (\ref{ex:ds32}) produce a total of 10 occurrences of \textit{niște} (in 63 valid translations, which equals 16\%). In the sentences (\ref{ex:ds33}) to (\ref{ex:ds35}), \textit{niște} is predominantly in competition with a bare noun: There are 10 cases of \textit{niște} and 62 occurrences of bare nouns.\footnote{Additionally, there is one use of \textit{unii} ‘certain’ and \textit{câțiva} ‘some’, respectively, for (\ref{ex:ds33}), and one use of \textit{câteva} ‘some’ for (\ref{ex:ds34}). The rate of occurrence of \textit{niște} in the translations of these three sentences is thus 13\% (10 out of 75).} As for sentence (\ref{ex:ds36}), there is more variation: \textit{niște} (3 occurrences) is considered an alternative to the quantifier \textit{câteva} ‘some’, which clearly dominates in the translations (22 occurrences).\footnote{In addition to \textit{niște}, there are also three occurrences of \textit{puține} ‘few’ as well as one use of a bare noun and \textit{ceva} ‘some’ respectively. This amounts to a frequency of 10\% with which \textit{niște} is used in the translations (3 out of 30).} 

We already mentioned that five persons used \textit{niște} particularly often in their translations. By zooming in on these five speakers (abbreviated by “Sp.”), we can easily illustrate that the use of \textit{niște} is strongly connected to individual preferences (\tabref{tab:ds12}).

\begin{table}
\begin{tabular}{lcccccc}
\lsptoprule
 & \REF{ex:ds31} & \REF{ex:ds32} & \REF{ex:ds33} & \REF{ex:ds34} & \REF{ex:ds35} & \REF{ex:ds36} \\
\midrule
Sp. 1 & \ding{52} & \ding{52} & \ding{52} & \ding{52} & \ding{52} & \\
Sp. 2 &           & \ding{52} & \ding{52} & \ding{52} &  & \ding{52}\\
Sp. 3 &           & \ding{52} &  &  &  & \ding{52}\\
Sp. 4 & \ding{52} & \ding{52} & \ding{52} &  & \ding{52} & \\
Sp. 5 &           & \ding{52} &  &  &  & \ding{52}\\
\cmidrule(lr){2-3}\cmidrule(lr){4-6}\cmidrule(lr){7-7}
 & \multicolumn{2}{c}{7/10=70\%} & \multicolumn{3}{c}{7/15=47\%} & 3/5=60\% \\
\lspbottomrule
\end{tabular}
\caption{Distribution of the use of \textit{niște} over the sentences (\ref{ex:ds31}) to (\ref{ex:ds36}) among the five speakers using it most frequently}
\label{tab:ds12}
\end{table}

The figures show that (i) 17 of the 23 occurrences generated by these six sentences stem from these five speakers and that (ii) the ratio of the use of \textit{niște} in (\ref{ex:ds31}) to (\ref{ex:ds36}) is considerably higher among these speakers (47--70\% vs. 10--16\%). In these contexts, the use of \textit{niște} is apparently a valid or even the preferred option for these five speakers.\footnote{One might object that the presence/absence of \textit{niște} in (\ref{ex:ds33}) could be caused by a difference between a distributive/collective reading and has nothing to do with individual preferences. Indeed, \textit{niște} seems to strongly favor a collective reading: a distributive interpretation is, however, not excluded (contra \cite[208]{Nedelcu2009}; cf. \cite{Davatz2018} for further details).}\largerpage[1.5]

The obvious question which now arises regards the nature of the semantic difference between a bare noun and a noun preceded by \textit{niște}, i.e. the question what \textit{niște} contributes semantically to the meaning of the respective DP. In order to answer this question and analyze the semantics of \textit{niște}\textsubscript{\textsc{pl}}, we shall look now at the comments made by the speakers in the preference task. Consider again the sentence given in (\ref{ex:ds18}), repeated below as (\ref{ex:ds37}):

\ea\label{ex:ds37} Modern Romanian\\
\gll Mama     a        întâlnit   (niște) membri      ai parlamentului:   Șerban și    Ioan.\\
     mum.\textsc{art} have.prs.3\textsc{sg} meet.\textsc{ptcp}   \textsc{niște}   member.\textsc{pl} \textsc{gen} parliament.\textsc{art.gen}   Șerban and Ioan\\
\glt ‘Mum has met (some) members of the parliament: șerban and Ioan.’
\z

As was illustrated in \sectref{sec:ds4.3.1}, the speakers showed a clear preference for the version where \textit{niște} precedes the specific direct object. However, the reasons for why they choose one or the other version varied considerably between the informants: (i) Three people explained their preference for \textit{niște} by the (implicit) marking of a (low) quantity in the example, (ii) three other people explained it by the referential specificity of the direct object and (iii) one person motivated her preference for the non-use of \textit{niște} by the fact that the referents of the noun \textit{membri} ‘members’ are known, i.e. that they are epistemically specific. Consider the respective statements in \ref{it1}--\ref{it3}:\largerpage

\begin{enumerate}
    \item   Numind 2 oameni (Șerban și Ioan), avem nevoie de o marc\u{a} a cantit\u{a}ții.\\
            ‘Naming two people (S. and I.), we need a marker of the quantity.’\label{it1}
    \item   Membrii sunt specificați, deci \textit{niște} se potrivește.\\
            ‘The members are specified, so \textsc{niște} fits.’\label{it2}
    \item   Wir wissen schon, welche.\\
            ‘We already know who [it is about].’\label{it3}
\end{enumerate}

Another general observation which can be made is that \textit{niște}\textsubscript{\textsc{pl}} is not actively used with generic nominals (cf. \sectref{sec:ds4.2}). This finding, resulting from the translation task, supports the claims made by \citet[82]{Avram1986}, stating that \textit{niște} cannot have a “generic value”, which contradicts \citet[207]{Nedelcu2009}.

\textit{Niște}\textsubscript{\textsc{pl}} is thus distinct from the French plural PA, the use of which is possible with contrastive generic preverbal subjects (cf. \citealt[69]{VogeleerTasmowski2005} and \cite[165]{Wilmet2003}):

\ea\label{ex:ds38} French\\
\gll Des moutons n’ont          \textup{jamais} cinq pattes!\\
     \textsc{pa}   sheep.\textsc{pl} \textsc{neg}{}-have.\textsc{prs.3pl} never  five  paw.\textsc{pl}\\
\glt ‘Sheep \textit{never} have five legs!’
\z

However, there are in fact other Romance varieties showing PAs which behave similarly to \textit{niște}\textsubscript{\textsc{pl}}. Recent fieldwork in the Aosta Valley reveals that PAs with preverbal generic nominals are systematically translated by a definite article and never produced actively in the local Franco-Provençal varieties (see \citealt{StarkGerards2020}, \citealt{Ihsane2018}).\footnote{As for the use of Italian partitive articles in these contexts, see \citet[77]{CardinalettiGiusti2016}.\label{fn17}} 

The complete absence of \textit{niște}\textsubscript{\textsc{pl}} with generic nominals in active production does not, however, necessarily imply an actual ungrammaticality of this use. It remains to be tested whether and to what extent it is considered grammatical.\footnote{Recent fieldwork conducted by the authors reveals that, though categorically avoided in active production, generic PA-subjects are not considered ungrammatical by all the speakers in grammaticality judgments (see \cite{DavatzIhsaneStark2023}).}

\subsection{\textit{Niște}\textsubscript{\textsc{pl}} as a specificity marker?}\label{sec:ds5.2}\largerpage
The question that has to be asked now is: Could the above-discussed absence (or even impossibility) of \textit{niște} preceding a generic nominal in subject position be due to properties of specificity which are inherent to it? Providing an answer to this question using the collected data is, however, far from easy, as the results concerning the extent to which \textit{niște}\textsubscript{\textsc{pl}} can (or has to) be considered a marker of specificity differ remarkably depending on the task. When the speakers have the choice between a bare noun and a DP introduced by \textit{niște}, they clearly opt for the latter, whereas when they are given a German bare noun, they tend to prefer a translation with a bare noun. The fact that German indefinite plural count nouns are undetermined might have an important influence on the translation; a comparison with translations of French sentences with indefinite plural count nouns determined by the so-called “partitive article” would allow us to further explore this possible factor. Another conceivable explanation for the difference between the results of the two tasks is the educational background of the participants: Even though asked to translate the sentences as naturally as possible into Romanian, many of them might have aimed at a stylistically high rather than a “spontaneous” and “natural” translation.

The preference task shows that the speakers clearly prefer (88\%) the use of \textit{niște}\textsc{\textsubscript{\textsc{pl}}} over a bare noun if the respective noun phrase in direct object position is epistemically specific (cf. \sectref{sec:ds4.3.1}). As shown in the previous section, the reasons why they do so seem to differ, but one reason mentioned by various participants was the “specificity” of the noun phrase. Other speakers assign their choice for \textit{niște}\textsubscript{\textsc{pl}} to the given (and highly restricted) number of members of the parliament, which are denoted by the noun phrase in the respective example (cf. \ref{ex:ds37}). This implies that \textit{niște}\textsubscript{\textsc{pl}} is preferentially used to denote smaller quantities, whereas a bare noun is not specified at all for the quantity of referents it denotes. Three further comments made by the speakers seem to support this observation, the first one in \ref{it4} related to (\ref{ex:ds37}), \ref{it5} and \ref{it6} to (\ref{ex:ds19}):

\begin{enumerate}\setcounter{enumi}{3}
    \item   Mama a întâlnit câțiva membri, puțini.\\
            ‘The mother has met \textit{some} members, \textit{few}.’\label{it4}
    \item   \textit{Niște} are rolul de a indica atât un num\u{a}r redus de indivizi, cât și de a oferi specificitate complementului direct.\\
            ‘\textit{Niște} has both the role of \textit{indicating a reduced number of individuals} and of \textit{conferring specificity} on the direct complement.’\label{it5}
    \item   Ohne die Angabe \textit{niște} kann die Rede von 2 oder 20 Jugendlichen sein.\\
            ‘Without the indication \textit{niște} it can be about 2 or 20 teenagers.’\label{it6}
\end{enumerate}

Considering the numbers (\tabref{tab:ds3}) and comments for the sentence in (\ref{ex:ds19}), repeated under (\ref{ex:ds39}), which shows a very high rate of uses of \textit{niște}\textsubscript{\textsc{pl}} despite the non-epistemic specificity of the direct object, it seems that a small quantity might indeed be the more important factor than epistemic specificity when it comes to the use of \textit{niște}\textsubscript{\textsc{pl}}:

\ea\label{ex:ds39} Modern Romanian\\
\gll Am v\u{a}zut (niște) tineri în fața cl\u{a}dirii. Numai doi dintre ei m-au salutat.\\
have.\textsc{prs}.1\textsc{sg} see.\textsc{ptcp} \textsc{niște} teenager\textsc{.pl} in face.\textsc{art} building.\textsc{gen} only      two of.between they me.\textsc{acc}{}-have.\textsc{prs}.3\textsc{pl} greet.\textsc{ptcp}\\
\glt ‘I saw some teenagers in front of the building. Only two of them greeted me.’
\z

Judging from the comment in \ref{it5}, the two notions of “small quantity” and “specificity” seem strongly intertwined for the speakers. It is, in our opinion, not excluded that the “specificity” effect is an implicature of \textit{niște} referring to a small quantity: What is reduced in number implicates a higher degree of specificity, a higher probability of “referential anchoring” (see \sectref{sec:ds2.3}). Another conceivable factor for the preferred use of \textit{niște}\textsubscript{\textsc{pl}} in (\ref{ex:ds39}) with the anaphoric pronoun \textit{ei} is one that has been brought up for the DOM-marker \textit{pe} (cf. \cite{ChiriacescuHeusinger2009}, \citeyear{ChiriacescuHeusinger2010}; \cite{HeusingerChiriacescu2013}), namely discourse prominence. It might be that marking a DP by \textit{niște}\textsubscript{\textsc{pl}} increases “the potential to generate further co-referential expressions”, as \citet[13]{ChiriacescuHeusinger2009} state it for \textit{pe}. 

The results of the translation task (cf. \sectref{sec:ds4.3.2}) show that \textit{niște}\textsubscript{\textsc{pl}} is often omitted in similar contexts and used less frequently than a bare noun (\tabref{tab:ds4}). This suggests that epistemic specificity might be more a result of the presence of \textit{niște}\textsubscript{\textsc{pl}} than a (strong) trigger for its use, similarly to what \citet{KleinSwart2011} stated for DOM-markers (and confirming the findings of \cite{Dobrovie-SorinGiurgea2013}).

Interestingly, however, the combination with the DOM-marker \textit{pe} marking scopal and epistemic specificity is clearly dispreferred in the preference task (cf. \sectref{sec:ds4.5.1}.). Yet this combination is not considered ungrammatical, but seems to be restricted to very specific cases where \textit{niște}\textsubscript{\textsc{pl}} is interpreted partitively.\footnote{Regarding the acceptability of the sentence \textit{Îi consider\u{a} inteligenți pe niște copii} ‘He considered some children intelligent’, one speaker commented that it would be acceptable \textit{doar dac\u{a} DOAR pe unii dintre ei} ‘only if \textsc{only} some of them’.}

\subsection{\textit{Niște}\textsubscript{\textsc{sg/pl}} with surprising scope properties}\label{sec:ds5.3}
Beyond the insights presented in the two previous subsections and the apparent difficulties in pinning down the exact semantics of \textit{niște}, our pilot study additionally shows that \textit{niște}\textsubscript{} differs from other indefinite determiners like the PA in French and Italian with respect to (some of) its scope properties.

The results of the interpretation task, designed specifically to reveal the scopal behavior of \textit{niște}\textsubscript{} show (i) an apparent scopal indifference with respect to other quantifiers for \textit{niște}\textsc{\textsubscript{\textsc{pl}}}, and (ii) a clear domination of narrow scope of \textit{niște}\textsc{\textsubscript{sg/pl}} with respect to negation, with, however, \textit{no systematic exclusion of wide scope} (cf. \sectref{sec:ds4.4}). 

As far as (i) is concerned, the literature shows that the same holds true for the French PA  \citep[cf.][139]{Ihsane2008}. The second observation is, however, much more surprising, in that \textit{niște} seems, at least for some speakers, to be able to scope over the negating element both with plural count nouns \textit{and mass nouns}. While the former is also true for the Italian PA \citep[cf.][60]{CardinalettiGiusti2016}, the latter is attested neither for the French nor the Italian singular PA (cf. \cite[139f.]{Ihsane2008} for French; \cite[60]{CardinalettiGiusti2016} for Italian). And \textit{niște}\textsc{\textsubscript{\textsc{pl}}} is not systematically associated with wide scope, as one might expect from its apparent preference for specific DPs: It \textit{can} be interpreted having wide scope, but does not have to be.

\section{Conclusion}\label{sec:ds6}
Coming back to the general research question building the background of this article, i.e. the question whether \textit{niște} is semantically comparable to the so-called “partitive articles” of French and Italian, we can state similarities and differences. \sectref{sec:ds4} and \ref{sec:ds5} have helped answer our three detailed research questions set up in \sectref{sec:ds2.5}. First, compared to bare plural count nouns, \textit{niște}\textsubscript{\textsc{pl}} seems to favour a specific interpretation, but is maybe not always compatible with the DOM-marker \textit{pe} (plus clitic-doubling). Second, \textit{niște}\textsubscript{\textsc{pl}} is incompatible with generic readings/contexts, and third, the scope properties of \textit{niște} are quite idiosyncratic. 

Even if there are thus some characteristics which \textit{niște} seems to share with the so-called “partitive articles” of French and Italian -- like the impossibility of use with generic subjects or possible wide scope with respect to quantifiers in the plural -- there are two crucial properties which clearly distinguish it from them. First and foremost, \textit{niște}\textsubscript{\textsc{sg}} is apparently able to scope over negation, whereas wide scope with respect to negation is unattested both for the French and Italian singular PA. The second property distinguishing \textit{niște}\textsubscript{\textsc{pl}} from the two other plural PAs is the fact that it seems to be used preferentially in the context of epistemic specificity or subsequent anaphoric pronouns. However, the compatibility with the direct object marker \textit{pe}, reported to be a marker of specificity, is relatively low. The comments made by the speakers insinuate that the meaning of \textit{niște}\textsubscript{\textsc{pl}} is slightly different in this case: \textit{Niște}\textsubscript{\textsc{pl}} seems to denote a part of a whole. It seems conceivable that \textit{niște}\textsubscript{\textsc{pl}}, similarly to the DOM-marker \textit{pe}, is a marker of specificity or, maybe even more to the point, of discourse prominence (cf. \citealt{ChiriacescuHeusinger2009, ChiriacescuHeusinger2010, HeusingerChiriacescu2013}), and that their co-occurrence thus leads to a clash due to redundancy. The mechanics causing the different interpretation of \textit{niște} in this context are, however, yet to be understood, and an analysis of their precise interaction is called for in future research. What is clear already at this stage is that \textit{niște}\textsubscript{\textsc{sg/pl}} has to be considered as an element of its own and is only partially comparable to other Romance indefinite determiners.


%%\section*{Abbreviations}
%%\begin{tabularx}{.45\textwidth}{lQ}
%%... & \\
%%... & \\
%%\end{tabularx}
%%\begin{tabularx}{.45\textwidth}{lQ}
%%... & \\
%%... & \\
%%\end{tabularx}


{\sloppy\printbibliography[heading=subbibliography,notkeyword=this]}
\end{document}
