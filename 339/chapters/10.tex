\part{Clitic climbing}
\label{part3}
\chapter{Approaches to clitic climbing}
\label{Approaches to clitic climbing}
\section{Introduction}
\label{Introduction:11}

Most works on CLs in Bosnian, Croatian, and Serbian address the nature of 2P effect mainly within formal theoretical frameworks (primacy of syntactic vs prosodic processes; for an overview see Chapter \ref{Theoretical Aproaches to the Study of Clitics in BCS}). One of the controversial issues in the literature concerns CC. An example of CC out of an infinitive complement is given in (\ref{(10.1)}).

\begin{exe}\ex\label{(10.1)}
\gll I mi  \textbf{se}\textsubscript{2} planiramo\textsubscript{1} baciti\textsubscript{2} na posao.\\
and we \textsc{refl} plan.1\textsc{prs}	throw.\textsc{inf} at work \\
\glt ‘We also plan to throw ourselves into work.’ 
\hfill [hrWaC v2.2]

\ex\label{(10.2)}
\gll Bojim\textsubscript{1} \textbf{se}\textsubscript{1} testirati\textsubscript{2} \textbf{ih}\textsubscript{2}.\\
be.afraid.1\textsc{prs} \textsc{refl} test.\textsc{inf} them.\textsc{acc}\\
\glt ‘I am afraid to test them.’
\hfill  [hrWaC v2.2]
\end{exe}

\noindent CC occurs in constructions containing two or more verbal elements. In example (\ref{(10.1)}), the reflexive CL \textit{se} which belongs to the infinitive verb form \textit{baciti} ‘throw’ is realised in the second position of the matrix clause. This is quite puzzling because the CL seems to have “climbed” from the infinitive complement into the matrix clause. Example (\ref{(10.2)}), where the pronominal CL \textit{ih} ‘them’ stays in the infinitive complement, shows that CC does not always occur. As we discuss in more detail below, CC is indeed a major source of variation in the usage of pronominal and reflexive CLs in BCS. 

Part \ref{part3} of the book is dedicated to CC and its constraints in BCS, a hitherto underresearched topic. To our knowledge there are only four studies dealing specifically with CC in BCS \citep{Caink04, Stjepanovic04,  Aljovic04, Aljovic05}, and only the latter three address the question of constraints on CC. Besides these studies, some scattered information can be found in various works \citep[e.g.][]{CamdzicHudson2002, Todorovic12}, as we show later. In comparison, for Czech the syntactic conditions for CC are much better described by: \citet{Rezac99, Rezac05, Junghanns02, Dotlacil04, Rosen01, Rosen14, Hana07, Lenertova04}, who discuss a whole series of constraints on CC in this West Slavonic language.\footnote{Cf. also \citet[247]{FranksKing00} on CC in Slovene. We will not take into account works on CC in Romance languages.} Of them, \citet{Junghanns02} undoubtedly offers the most comprehensive account. 
\textcolor{black}{In Section \ref{Approaches to 2P effects: syntax, phonology and information structure} we show that, unlike in Czech, next to strictly syntactic approaches \citep[e.g.][]{Progovac96, Franks97}, also phonological \citep[e.g.][]{RadanovicKocic88, RadanovicKocic96} and mixed approaches \citep[e.g.][]{Schutze94, Boskovic00, Boskovic01} to CL placement exist. However, as we discuss in Section \ref{The limits of phrase splitting in BCS standard varieties}, even the strict phonological approaches need to use syntax to explain the variation in CL placement. Moreover, some scholars argue against a purely phonological approach to the 2P phenomenon. \citet[441]{CavarWilder94}, for instance, question the assumption of phonological rules which have the power to move material around in phonological representations in order to capture marginal cases like phrase splitting. In a similar vein, we thus conclude that syntactic constraints on CL placement are relevant in BCS too.} As we assume the CL systems of both languages to show many common features, we use the Czech constraints as a test ground for BCS (see Chapter \ref{Constraints on clitic climbing in Czech compared to Bosnian, Croatian and Serbian (theory and observations)}).

We preliminarily define constraints on CC as structural features or combinations of features blocking the realisation in the matrix of a CL belonging to the embedding. Constraints sensu stricto can only be detected by testing minimal pairs, providing negative evidence where one sentence is evaluated as acceptable and the other as unacceptable. Nearly all the scholars who have worked on constraints on CC in BCS discussed below rely exclusively on their own linguistic intuition as native speakers, which may entail certain problems (see a more detailed discussion of those problems in Sections \ref{Introduction:4} and \ref{Pros and cons of judgment data}). This does not hold for the work for Czech of \citet{Junghanns02}, which is based on examples found in corpora.  

In part \ref{part3} of the book we first give a brief presentation of the main theoretical accounts of CC. Afterwards, we zoom in on the linguistic data. At first glance, the distribution of CC shows a confusingly high degree of variability. We follow the research scheme presented in Section \ref{Chosen strategy}: intuition/theory – observation – experiment. Part \ref{part3} of the book, dedicated to CC, has the following structure: On the basis of the existing research literature we present constraints in Czech and compare these data to BCS (intuition/theory) – see Chapter \ref{Constraints on clitic climbing in Czech compared to Bosnian, Croatian and Serbian (theory and observations)}. We then present two empirical corpus studies on CC (observation). The studies are based on a common methodology, which we explain in Chapter \ref{Introductory remarks to corpus studies on CC}.

First we present a corpus-based study of CC out of \textit{da}\textsubscript{2}-complements which are characterised by the presence of an element sometimes interpreted as a complementiser and of an inflected verb (Chapter \ref{A corpus-based study on CC in da constructions and the raising-control distinction (Serbian)}).\footnote{For more information on \textit{da}\textsubscript{2}-complements see Section \ref{Types of complements}.} This is an interesting topic because CC out of other complements with inflected verbs is a rare phenomenon. Here, we focus exclusively on Serbian because \textit{da}\textsubscript{2}-complements are much more frequently used in Serbian than in Croatian, especially in the context of raising and subject control verbs.\footnote{More information on the raising--control dichotomy can be found in Section \ref{The control vs raising distinction}.}

Next we present an empirical in-depth study on diaphasic variation with respect to the raising--control dichotomy and its impact on CC out of infinitive complements (Chapter \ref{A corpus-based study on clitic climbing in infinitive complements in relation to the raising-control dichotomy and diaphasic variation (Croatian)}).\footnote{For more information on diaphasic variation, see Section \ref{Systemic vs functional microvariation}.} This study focuses on corpora which contain texts with standard Croatian on the one hand and colloquial Croatian language features on the other. Two reasons motivated us to choose Croatian as our target language. First, in Croatian infinitive complements are used not only with raising, but also with subject and object control CTPs (which is not the case in Serbian and Bosnian). Second, only for Croatian are there electronically stored and publicly available big corpora compiled not only for colloquial, but also for standard language.\footnote{For an overview of corpora available for BCS see Chapter \ref{Corpora for Bosnian, Croatian and Serbian}.}

On the basis of the observation chapters (with corpus studies) we proceed to Chapter \ref{Experimental study on constraints on clitic climbing out of infinitive complements}. In that chapter we conduct a full-fledged psycholinguistic experiment consisting of acceptability judgment tasks. With this study we want to contribute new, experimentally collected data to CC research. Namely, our aim is to broaden the set of structures considered in accounts of CC in BCS. The general question which lies behind the psycholinguistic study is whether on the one hand any particular contexts can be recognised as triggers for obligatory CC, and on the other hand whether there are any features which can be detected as constraints on CC. Just like in the corpus study presented in Chapter \ref{A corpus-based study on clitic climbing in infinitive complements in relation to the raising-control dichotomy and diaphasic variation (Croatian)}, sentences in this test contain CTPs with infinitive complements only. As already mentioned, in contrast to Serbian and Bosnian where many CTPs favour \textit{da}\textsubscript{2}-complements, in Croatian infinitive complements are used not only with raising, but also with control CTPs. Therefore, to match our corpus-linguistic data on CC out of infinitive complements, we conducted the psycholinguistic study exclusively for Croatian. 

Next we discuss our data on haplology, clusters, and pseudodiaclisis. We bring together the findings from corpus studies and the experiment. We discuss what determines CC in BCS in terms of complexity in Chapter \ref{On the heterogeneous nature of constraints on clitic climbing: complexity effects}. \textcolor{black}{The idea of system complexity offers a unified explanation for different types of constraints on CC in BCS without the necessity of assuming particular syntactic mechanisms like clause union or restructuring (see below), which by some authors is considered a sufficient, and by others only a necessary condition for CC. Although we do not negate the relevance of restructuring for CC, theory of complexity is in our view more adequate for the empirical data. It allows for explaining the considerable amount of variation we identify also in restructuring environments. This is achieved by incorporating non-systemic factors (diaphasic variation) into the model.}

\section{Theoretical approaches to clitic climbing}
\subsection{Definitions of clitic climbing}

CC has been analysed only within formal frameworks. Surprisingly, there are no functional or cognitive accounts. In the theory-neutral survey of CL systems in different languages by \citet[162]{SpencerLuis12}, CC is linked to “constructions in which the clitic is associated with a verb complex in a subordinate clause but is actually pronounced in constructions with a higher predicate (for instance, the matrix verb which selects that subordinate clause), even though it may have no obvious semantic or syntactic connection to that verb”. We refrain from giving an overview of the research literature, confining ourselves to a small selection of definitions of CC by various authors:\footnote{For a \textcolor{black}{literature} overview, we refer readers to the abovementioned textbook \citet{SpencerLuis12}, and to the concise article \citet{Dotlacil16}, which sums up the findings of works related to generative grammar and minimalism.}

\largerpage[-1]
\begin{itemize}
\item \citet[122]{Hana07} on Czech: “In a clause, clitics governed by the highest non-clitic governor (usually a non-auxiliary finite verb, [\dots]) obligatorily occur in Wackernagel position – in the main clitic cluster. However, there can be other clitic clusters in the domain of more embedded phrases. Clitics governed by those words can, or even tend, under certain circumstances to occur in the clitic clusters of less embedded governors, possibly in the main one. Within a finite clause, clitics governed by infinitives [\dots], adjectives [\dots], adverbs, and numerals [\dots] can climb up into a higher clitic cluster.”
\item \citet[58]{Junghanns02} on Czech: “In komplexen syntaktischen Ausdrücken bewegt sich ein klitisches Pronomen aus der Einbettung in die Matrix. Eindeutiges Indiz für die Bewegung ist die Tatsache, dass das der Einbettung entstammende Pronomen in der Satzoberfläche links vom Matrixverb erscheint. Diese Bewegung von Klitika ist “Clitic Climbing” (CC) genannt worden [\dots].”\footnote{In complex syntactic expressions, a clitic pronoun moves from the embedding into the matrix. A clear indication of the movement is that the pronoun originating in the embedding appears in the sentence surface left of the matrix verb. This movement of clitics is called “Clitic Climbing” (CC).}
\item \citet[7]{Rezac05} on Czech: “Clitic climbing refers to a phenomenon whereby the clitic argument of an infinitive shows up within the clause of a c-commanding verb.”
\item \citet[70]{Dotlacil04} on Czech: “Clitic climbing (realization of a clitic in a clause higher than the one in which the clitic originates) [ \dots].” 
\item \citet[169]{Aljovic04} on BCS: “Clitic climbing is the phenomenon whereby clitics climb out of the clause containing the verb they are arguments of, and attach to a higher predicate.” 
\item \citet[152]{Slodowicz08} on Polish: “a process found in sentences in which the object of the complement predicate is realized by means of a pronominal clitic appearing in the matrix clause.”
\end{itemize}

If we abstract from the theoretical embedding of individual works, this small selection of definitions shows that most authors agree that CC is associated with matrix complement structures and with the positioning of a CL in the matrix clause and not in the complement in which it originates. If we discuss these definitions, we find the following pitfalls or even shortcomings: 

\begin{itemize}
\item \citet{Hana07} and \citet{Rezac05} focus on infinitive complements because in Czech CC is attested exclusively out of them. Serbian, however, is different because it allows CC out of \textit{da}\textsubscript{2}-complements containing a verbal form inflected for person and number. 
\item \citet{SpencerLuis12} and  \citet{Aljovic04} assume that the complement forms a clause on its own. As the following section on restructuring shows, this is a contentious issue. 
\item Some authors assume a specific operation responsible for moving the element from one position to another \citep{Hana07, Aljovic04, Junghanns02} whereas others do not \citep{Rezac05, Dotlacil04, Slodowicz08}.
\item There is no consensus as to the relationship between  the CL and the verb of the complement: government \citep{Hana07}, CL as argument \citep{Rezac05} or object \citep{Slodowicz08}. Other authors remain agnostic on this point. This is also our position because not only argumental pronominal CLs undergo CC; so do lexical reflexives like \textit{se} in \textit{smijati se} ‘laugh’, which can hardly be interpreted as objects. 
\end{itemize}

Summing up, we propose the following definition: \textsc{clitic climbing} (CC) refers to a phenomenon whereby a clitic is not realised in a position contiguous to elements of the embedding to which it belongs, but in a position contiguous to elements of the matrix. 

\subsection{Clitic climbing and optionality}
\label{Clitic climbing and optionality}

It is well-known that formal syntactic models rely on the axiom of parsimony. Therefore, it comes as no surprise that many authors working in a formal framework try to treat CC as an ordinary case of CL placement. Namely, they attempt to explain away the peculiarities of CC in order to formulate a unified theory of cliticisation; e.g. \citet[350]{CamdzicHudson2002} see CC and cliticisation in general “as a very simple and natural extension of ordinary grammar.” In some approaches CC is associated with the syntactic process of “restructuring”, in others with “clause union”.\footnote{The term “restructuring” is used in the analysis of certain infinitive complements which lack clausal properties when they appear as complements of restructuring verbs \citep[cf.][2]{Aljovic05}. For a more detailed discussion of restructuring and clause union see next section.}  

Scholars differ when it comes to the relation between restructuring and CC. Some claim that restructuring is a necessary but insufficient condition for CC, while others are convinced that CC is contingent upon restructuring. Thus we can clearly see that there are two major streams in research on CC. 

On the one hand, there are authors who claim that CC is always optional, which means that if the conditions for restructuring are fulfilled, CC can, but does not have to occur (for BCS see \citealt{Progovac93}, \citealt{Progovac96}, \citealt{CavarWilder94}, \citealt{Stjepanovic04}, for Czech see \citealt{Rezac05}).\footnote{These scholars are convinced that the existence of cases where other processes such as licensing of negative polarity items, object preposing and wh-movement out of complement signal restructuring are present while  at the same time CLs do not climb demonstrates that CC is optional. For more on restructuring tests see \citet[50--53]{Progovac94} and \citet[178--179]{Stjepanovic04}.} In probably the best-known paper dealing specifically with CC in BCS, \citeauthor{Stjepanovic04} starts out from the hypothesis that CC is not obligatory either out of infinitive or out of \textit{da}\textsubscript{2}-complements \citep[cf.][181, 186, 205]{Stjepanovic04}. For more on technical details of this account see section below. 

\citet{Aljovic05} gives a comprehensive account of the discussion on the connection between CC and restructuring and on the most controversial question of whether CC is optional or obligatory. \citet{Aljovic04, Aljovic05} convincingly points out the weak spots of an approach that considers CC to be optional within the restructuring context. If the lack of restructuring is not the reason for the lack of CC, some special mechanisms of CL placement would have to exist, and various ad-hoc explanations would be called for \citep[3]{Aljovic05}. A further consequence of this approach is the lack of a (unified) theory of cliticisation. Moreover, it fails to predict, or needs special solutions to explain cases of obligatory CC which is observed in some languages \citep[cf.][3]{Aljovic05}. 

\begin{sloppypar}
\citet[6]{Aljovic05} addresses three questions: “Why is clitic climbing sometimes unavailable (blocking effects)? Why is sometimes clitic climbing obligatory? Why does clitic climbing sometimes appear to be optional?” She identifies “the size of the clausal complement” as the deciding factor in CC. “Clitics climb from domains that are functionally poor”, i.e., do not contain elements such as sentence negation and interrogatives (for more details see below; \citealt[cf.][]{Aljovic05}).\footnote{In this respect, she does not differ for instance from \citet{Rezac05} who claims that in Czech, CC is possible only from VP (verbal phrase) complements, while CPs (complementiser phrases) and TPs (tense phrases) do not allow climbing.} She suggests that there is no optionality for CC. CLs climb obligatorily in contexts of restructuring infinitives because these restructuring complements lack the functional structure necessary to keep CLs in their original phrasal domain. 
\end{sloppypar}

\subsection{Clitic climbing, restructuring (or clause union) and movement}

In the following, we will give some technical theory-internal details showing how the authors implement the conceptual issues delineated in Section \ref{Clitic climbing and optionality}. in their models of grammar. Due to lack of space we cannot provide a full presentation of individual theories. This chapter is addressed to readers with some basic knowledge of minimalism or related frameworks. 

There are many accounts of restructuring. The main idea is that there exist types of predicates which differ as to the complement they select. Restructuring predicates are found among modal verbs, motion verbs, aspectual verbs, causative verbs and some propositional attitude verbs. They vary not only cross-linguistically but also among speakers of one language \citep[cf.][2]{Aljovic05}. 

\citet{Progovac93, Progovac96}, \citet{Stjepanovic04}, \citet{Aljovic04, Aljovic05}, and \citet{Todorovic11, Todorovic12, Todorovic15} are, as far as we can see, the first to extend the notion of restructuring from infinitives to complements introduced by the element \textit{da} containing a verbal form inflected for number and person, but not for tense: so-called \textit{da}\textsubscript{2}-complements.\footnote{Some of them use other terminology like S- and I-verbs or subjunctive and indicative complements, but the idea behind the different terms is the same.}  

\citet[175--179]{Stjepanovic04} provides further data for the distinction of sub\-junctive-se\-lec\-ting (S-) and indicative-selecting (I-) verbs.\footnote{In our terminology S-verbs are equivalent to verbs with \textit{da}\textsubscript{2}-complements and I-verbs, to verbs with \textit{da}\textsubscript{1}-complements. For more information on those differences see Section \ref{Types of complements}.} Long object preposing from passivised embeddings is possible with S-verbs, but not with I-verbs, as with the latter the passive reading is lost. With S-verbs, multiple wh-fronting with one wh-phrase originating in the matrix clause and another being licenced in the embedding is possible with any order of the wh-phrases, provided that the embedded subject is not overtly realised. With I-verbs, the structurally lower wh-phrase has to follow the matrix one. \citet[179]{Stjepanovic04} concludes that S-verbs are restructuring verbs and that domain extension is restructuring. 

As \citet[186--198]{Stjepanovic04} notes, there are two major lines of thinking about restructuring: The first considers restructuring to be a transformational process, whereby two clauses are rearranged into one. According to the second, restructuring constructions are generated as a single clause from the beginning. Stjepanović argues for the latter based on different referential features of a singular subject and an embedded collective verb, compare (\ref{(10.3a)}) and (\ref{(10.3b)}).

\begin{exe}\ex\begin{xlist}
\ex[]{\label{(10.3a)}
\gll Petar \textbf{je} odlučio da \textbf{se} okupe u parku\\
Petar be.3\textsc{sg} decide.\textsc{ptcp.sg.m} that \textsc{refl} gather.\textsc{3prs} in park\\ 
\glt ‘Petar decided to gather in the park.’}
\ex[*]{\label{(10.3b)}
\gll Petar \textbf{je} pokušao da \textbf{se} okupe u parku\\
Petar be.\textsc{3sg} try.\textsc{ptcp.sg.m} that \textsc{refl} gather.\textsc{3prs} in park\\
\glt Intended: ‘Petar tried to gather in the park.’ \\ }
\hfill (BCS; \citealt[193]{Stjepanovic04})
\end{xlist}
\end{exe}

\noindent In (\ref{(10.3a)}) the singular subject \textit{Petar} is not strictly referential with the embedded subject of \textit{se okupe} ‘(they) gather’, which has a collective meaning and can therefore be used with the verbal plural form \textit{se okupe}. However, (\ref{(10.3a)}) is grammatical in contrast to (\ref{(10.3b)}), the latter showing a mismatch between the singularity of \textit{Petar} and the collective semantics of \textit{se okupe}. Referring to \citet{Wurmbrand99}, Stjepanović takes such cases for determining what restructuring is. The reasoning behind this may be summed up as “one clause per subject argument”: The \textit{da}-complement in (\ref{(10.3a)}) has a phonetically empty subject, big PRO, which agrees with \textit{okupe}. As subjects are licenced by clause, the \textit{da}-complement builds a CP on its own. Hence, non-restructuring instances like (\ref{(10.3a)}) are bi-clausal. Conversely, restructuring constructions like (\ref{(10.3b)}) have only one subject, as indicated by their ungrammaticality, i.e. there is no PRO, as they are mono-clausal. From this it follows that CC is just an instance of regular CL movement. As CC is optional in restructuring contexts with both infinitive and \textit{da}\textsubscript{2}-complements, \citet[206]{Stjepanovic04} concludes that restructuring is not the driving force for CC. Furthermore, \citet[198--204]{Stjepanovic04} observes that restructuring verbs behave like raising verbs. She also argues that under S-verbs \textit{da}  is not a real complementiser, but that it belongs to the verbal domain, similarly to the English infinitive marker \textit{to} and German \textit{zu} \citep[205f]{Stjepanovic04}.

In her PhD thesis \citet{Todorovic12} further elaborates on indicative and subjunctive complements and claims that CC out of \textit{da}-complements is restricted to pronominal CLs which are hosted by \textit{da} [−veridical] and is impossible in the case of \textit{da} [+veridical]. Following \citet{Progovac05}, \citet{Todorovic12} eliminates CPs from the clausal structure of Serbian, thus giving rise to a mono-clausal analysis of \textit{da}-complementation. Therefore, \textit{da} is not viewed as a complementiser at all. Todorović assumes two different structural positions for the indicative and subjunctive \textit{da} in the syntactic tree. In indicative \textit{da}-complements, CLs successively move with the verb through all functional heads on its way from VP to TSP. Thus, auxiliary and pronominal CLs cluster together, while the lower copy of the verb is pronounced. In subjunctive complements, CLs climb on their own, since [−veridical] verbs lack tense and hence do not move to TSP for checking purposes. At this point, CLs attach to the matrix verb. According to \citet{Todorovic12}, there is no CC at all, because the notion of climbing is based on the assumption of CLs moving from an embedded to a matrix clause. However, the elimination of clausal boundaries between matrix verb and non-veridical \textit{da}-complement allows CC to be interpreted in the more general terms of CL positioning within the clause. 

\citet{Progovac93, Progovac96} proposes that BCS CLs are right-adjoined to the head of the CP. Thus, 2P CLs always appear in the second position when being hosted by material that appears either in the specifier position of a CP or in the C-head. CC is considered to be the effect of domain extension by S-verbs, which is technically the deletion of the embedded clause’s CP or inflectional phrase (IP). As a result, CLs right-adjoin to the head of the matrix CP. By contrast, I-verbs do not extend their domain and the embedded CP is preserved, so CLs cannot climb and thus remain in situ. 

\citet[245]{FranksKing00} consider CC to be associated with restructuring, in that the matrix and the embedded verbs’ domains are combined into one. CLs are then positioned with respect to this single domain. Restructuring is considered to be a lexical matter in principle and may be optional, obligatory, or impossible, with respective implications for CC. 

\citet{Rezac05} discusses the syntactic structure which is necessary for CC in Czech, demonstrating that CC is possible only when the constituent out of which a CL climbs is a bare VP complement and showing that CPs, TPs and “small vPs” (verb shells) do not allow CC. He argues that CC depends on restructuring contexts of raising and control verbs (both subject and object), with CC taking place only with restructuring infinitives but not with non-restructuring infinitives, where CLs remain in situ. With restructuring infinitives CLs climb as arguments in order to check case and φ-features such as person or number. Furthermore, he argues that these conditions affect both CLs and full NPs equally \citep[cf.][]{Rezac05}.

\begin{sloppypar}
\citet{Rosen14} who analyses CC of reflexives and their haplology within HPSG, uses clause union as an independently motivated mechanism for explaining mainly word-order phenomena. According to this solution, CLs may climb due to optional raising of arguments. Argument raising is a lexically specified option. 
\end{sloppypar}

A different approach to CC is offered by \citet{Junghanns02}, who is the only author who systematically studies the environments enabling or blocking CC. His approach does not rely on restructuring or any other highly abstract notion, but on a mechanism of CL movement which is susceptible to information structure. According to him, CC is possible in Czech in raising, subject control matrix clauses, and in ECM environments, where the climbing CL is generated in an infinitive embedding. \citet[66]{Junghanns02} notes that CLs may climb from complements as well as from adjuncts and argues that there is therefore no clause union in instances of CC. As heads, CLs may only climb if there is a free verbal head above the embedded infinitive that they can use as a landing site \citep[cf.][85f]{Junghanns02}. The lack of a free verbal head blocks CC, so CL movement from subordinations under NPs, APs, and PPs is blocked. \citet{Junghanns02} treats instances of CC out of infinitive complements under noun/determiner and predicative adjective phrases as special cases of incorporation into the verbal head. However, \citet[82f]{Junghanns02} admits that syntax proper does not explain the seeming optionality of CC in Czech. He proposes using information structure in order to explain CC and in situ realisation of CLs. Namely, he suggests that CC takes place if the CL belongs to the background of the sentence. In contrast, CLs remain in the infinitive phrase (in situ) if they are part of the topic or focus.

\subsection{Outlook}

Summing up, the presented works on CC revolve around the (non-)obligatoriness of CC and the attempts to reconcile CC with a unified account of CL positioning. CC involves complex structures containing two verbal elements. One popular idea is to assume a process which unites two clauses into one. If there is no boundary between the matrix and the embedded structure, CLs do not climb, but appear in their usual, second position. Many authors, even those who do not refer to clause union or restructuring in their argumentation, assume that the embedding has a functionally poor structure. Moreover, these authors point out that CC is possible only with specific types of matrix predicates like restructuring verbs or S-verbs. The extension of this class of predicates, however, remains unclear. The distinction between raising and control seems to be relevant. Further, there is no consensus as to the structure of the embedding. An empirically more adequate approach explains CC in relation to the syntactic environments blocking landing sites for the movement of CLs. 

Generally speaking, no theory of CC as such is presented. As CLs show up in positions where according to the formal models they should not, the authors discuss how this aberrant behaviour can be reconciled with the axioms of their respective model. Factors beyond sentence structure in formal models are not taken into account.

It is not an exaggeration to say that all theoretical accounts of CC in BCS are based on heavy data reduction in the sense that they are based on a small selection of syntactic environments where CC occurs. We see that most authors focus mainly on instances of climbing by single CLs and do not address the potential interaction between different types of CLs. Neither do they ask whether there are differences between, for example, pronominal and reflexive CLs. A further open question concerns different types of matrix predicates. It is \citet{Junghanns02} who strives for a complete account of environments enabling or blocking CC in Czech. As his approach refrains from assuming a single highly abstract mechanism, it covers a much wider range of data than the other models. Therefore, it will serve as the basis for the following chapter on constraints on CC in Czech and BCS. 
