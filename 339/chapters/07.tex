\chapter[Clitics in dialects]{Clitics in dialects (Bosnian, Croatian, Serbian)}
\label{Clitics in dialects}
\section{Introduction}

Research on BCS clitics in the theoretical literature is mainly based on standard varieties. To the best of our knowledge there are no earlier studies devoted to CLs in BCS dialects. Data from dialects are included quite rarely, mainly in descriptions of phenomena which do not occur in standard languages like CL doubling (see Section \ref{Clitic doubling}). Moreover, in dialectological studies CLs are only sporadically mentioned in the sections dedicated to morphology and syntax. Thus, this study is most probably the first attempt to give an overview of the CL system in BCS dialects. 

In Chapters \ref{Clitics and variation in grammaticography and related work}, \ref{Clitics in a corpus of a spoken variety}, and \ref{A corpus-based study on clitic climbing in infinitive complements in relation to the raising-control dichotomy and diaphasic variation (Croatian)} we show that even the data from standard varieties display a certain degree of variation with respect to CLs. We strongly believe that this variation has its source in dialects. Namely, standard varieties emerge via a complex process of selection and normativisation of specific dialect(s) which are chosen as a basis for the standard. Standard varieties are therefore inseparable from local idioms. Additionally, standard varieties are learned at school. Moreover, as we already demonstrated in Chapter \ref{Clitics and variation in grammaticography and related work}, some native speakers do not completely acquire certain rules in respect of CLs in the standard variety. Looking into the data from dialects can give us a clearer picture of the detected variation and can help understand why native speakers make what normativists consider “mistakes” when they use CLs in standard varieties. 

Due to the mentioned lack of previous studies focusing on CLs in BCS dialects, we start with the first step of the research strategy presented in Section \ref{Chosen strategy}. Chosen strategy: intuition/theory. We summarise and critically synthesise explicit findings from dialectological literature. Since CLs behave completely differently in Kajkavian and Čakavian, and human and other resources are limited, we concentrate only on the data from Štokavian dialects. This dialect is our focus because it is more widespread than Čakavian and Kajkavian. Moreover, some Štokavian dialects serve as the base for the three standard varieties of BCS. However, since in Croatia Kajkavian and Čakavian dialects are in contact with Štokavian, sometimes we could not neglect data from Kajkavian and Čakavian. Furthermore, data from Kajkavian and Čakavian are sometimes used to show parallels or divergences in the CL systems. 

The second step of our empirical approach, i.e. observation, was applied only partially. First, as explained in Section \ref{Available data}, only some dialectological works contain transcripts and only some transcripts are valuable sources of data. Second, the lack of transcripts in digital form slowed us down. Since no quantitative analysis was possible, the transcripts of Štokavian dialects were analysed only qualitatively and the exact data on the distribution of certain structures are lacking. In our qualitative analysis we focused on interesting examples with phenomena detected as parameters of variation in Section \ref{Parameters of microvariation}.    

Since we assume that not all readers are familiar with basic dialectological concepts, Section \ref{An overview of BCS Štokavian dialects} presents a compact introduction to BCS Štokavian dialects which is followed by a short overview of available data in Section \ref{Available data}. The following sections bring a comprehensive account of parameters of CL variation. Section \ref{Inventory:8} gives exhaustive data on variation in the CL inventory, while Section \ref{Internal organisation of the clitic cluster:8} introduces interesting findings concerning the internal organisation of the CL cluster, which includes divergent patterns in CL cluster formations and morphonological processes within the cluster. The position of the CL or CL cluster (1P, 2P, DP, phrase splitting, endoclitics) is discussed in Section \ref{Position of the clitic or clitic cluster:8}. Sections \ref{Clitic climbing:8} and \ref{Diaclisis:8} present data on CC and diaclisis in dialects. We attempt to describe clitic doubling in Štokavian dialects in Section \ref{Clitic doubling}. For the sake of comparison, the status of each parameter in standard varieties is thoroughly described in Chapter \ref{Clitics and variation in grammaticography and related work}. An empirical study of parameters of microvariation carried out on material from spoken Bosnian is presented in the next chapter.  

\section{An overview of BCS Štokavian dialects}
\label{An overview of BCS Štokavian dialects}
Three main groups of Slavonic dialects are spoken on the territory of Bosnia and Herzegovina, Croatia, Serbia, Kosovo and Montenegro: Štokavian, Čakavian and Kajkavian. The latter two are used only in Croatia, while Štokavian is used in all the abovementioned countries.   

An overview of Štokavian dialects based on three classifications is presented in Table \ref{T7.1}. The reader has to bear several things in mind. First of all classifications differ as to the number of Štokavian dialects. For instance \citet{IPR01} and \citet[160f]{Lisac03}, who use modified versions of \citet{Ivic88} map, list the \textit{Smederevsko-vršački} dialect among Štokavian dialects, unlike \citet[318f]{Okuka08}. Moreover, \citet[160f]{Lisac03} and \citet[318f]{Okuka08} include \textit{Istočnobosanski}, i.e. \textit{Srednjobosanski}, whereas according to the map by \citet{IPR01} such a dialect does not exist. 

Besides the differences in the number of Štokavian dialects, there are certain differences in terminology. Very often one and the same dialect is differently labelled by different authors: compare for instance the names for the \textit{Zapadni} dialect in Table \ref{T7.1} (page \pageref{T7.1}). Throughout this chapter we will use the terms which are in small caps format in the  table. 

\begin{sidewaystable}
\caption{An overview of Štokavian dialects\label{T7.1}}
\fittable{\begin{tabular}{llll}
\lsptoprule
Countries &  Ivić (2001) & Lisac (2003) & Okuka (2008) \\
\midrule
&\multicolumn{3}{c}{Neo-Štokavian dialects} \\
\cmidrule(rl){2-4}
Ba, Hr, Me, Rs & \textsc{Istočnoher\-cegovački}  & \textit{Istočnohercegovačko-krajiški} & \textit{Istočno\-hercegovačko-krajiški} \\
Ba, Hr, Me, Rs  & \textsc{Šumadijsko-vojvođanski} & \textsc{Šumadijsko-vojvođanski} & \textsc{Šumadijsko-vojvođanski} \\
Ba, Hr, Rs  & \textit{Mlađi ikavski}   & \textsc{Zapadni}    & \textit{Zapadnohercegovačko-primorski} \\
\cmidrule(rl){2-4}
&\multicolumn{3}{c}{Old Štokavian dialects} \\
\cmidrule(rl){2-4}
Me, Rs, Kos   & \textit{Zetsko-sjenički}   & \textsc{Zetsko-južno\-sandžački} & \textit{Zetsko-raški} \\
Kos, Rs  & \textsc{Kosovsko-resavski}  & \textsc{Kosovsko-resavski}  & \textsc{Kosovsko-resavski} \\
Rs   & \textsc{Smederevsko-vr\-šač\-ki}  & \textsc{Smederevsko-vršački}  & ---  \\
Ba, Hr   & ---     & \textit{Istočno\-bosanski}   & \textsc{Srednjo\-bosanski} \\
Ba, Hr, Rs  & \textit{Ekavski slavonski,}  \newline \textit{Posavski ikavski} & \textsc{Slavonski} & \textsc{Slavonski} \\
\cmidrule(rl){2-4}
&\multicolumn{3}{c}{Middle Štokavian dialects/Torlac dialects} \\
\cmidrule(rl){2-4}
Rs, Kos  & \textsc{Prizrensko-južnomoravski} &  & \textsc{Prizrensko-južnomoravski} \\
Rs   & \textsc{Svrljiško-zaplanjski}  &   \textit{Torlački}    & \textsc{Svrljiško-zaplanjski} \\
Rs   & \textsc{Timočko-lužnički}  &       & \textsc{Timočko-lužnički} \\
\lspbottomrule
\end{tabular}}
\end{sidewaystable}

In this book we mainly use the terminology proposed by \citet{Okuka08}, with three exceptions: for the sake of brevity, instead of \textit{Istočnohercegovačko-krajiški} we use \textit{Istočnohercegovački}. The term \textit{Zapadnohercegovačko-primorski} is also replaced by the shorter term \textit{Zapadni} dialect. And instead of using \textit{Zetsko-raški}, whose second element refers to the medieval name of the region, we will \textit{Zetsko-južnosandžački} because we believe that the latter term is more transparent for those who are not that familiar with the medieval history of the region now called Sandžak. While many dialects have names corresponding to the regions in which they are used, the reader has to bear in mind that the names of dialects do not always completely overlap with the names of the regions. For instance, \textit{Istočnohercegovački} is not used only in Eastern Herzegovina but also in Western Bosnia, North-Eastern Montenegro, Western Serbia and Eastern Croatia. Moreover, in Eastern Herzegovina not only is \textit{Istočnohercegovački} spoken, but also \textit{Srednjobosanski}. For the sake of clarity we would like to state that we use region names only to refer to regions, and dialect names only to refer to dialects. 

The spatial distribution of Štokavian dialects from Table \ref{T7.1} is shown in \figref{M7.1}.

\begin{figure}[ht]
\caption{Štokavian dialects. Author: Dr. sc. Branimir Brgles}
\label{M7.1}
\includegraphics[width=0.3\textwidth]{LEGENDA.pdf}
\includegraphics[width=0.65\textwidth]{FinalV1no_signs.pdf}
\end{figure}

\newpage
As may be seen in \figref{M7.1}, Štokavian is spoken in nearly half of Croatia, and in all of Bosnia and Herzegovina, Montenegro, and Serbia \citep[15]{Lisac03}. Additionally, Štokavian is spoken in Italy (Molise), Austria (Vlahija in Burgenland), Hungary (various settlements), Romania (Rekaš), Slovenia (Bojanci and Marindol), Kosovo and Macedonia. 

The internal classification of Štokavian dialects is usually done according to the following criteria: accents, reflex of the vowel [ě] (jat), and šćakavism or štakavism in some words.\footnote{Neo-Štokavian idioms have the same accent system as the standard varieties, with four kinds of accent based on the combination of two features: pitch accent (rising or falling) and length (short or long). The combination of these two features results in four accents (long rising\ \  ́, long falling\ \  ̑, short rising\ \  ̀, short falling\ \  ̏). In Neo-Štokavian dialects non-accented long syllables (marked with ˉ) appear only after \textcolor{black}{accented syllables}. In contrast, in Old Štokavian dialects non-accented long syllables can appear also before \textcolor{black}{accented syllables} \citep[cf.][23]{Lisac03}.}\textsuperscript{,}\footnote{The Proto-Slavonic vowel [*ě] was probably a long open front vowel. It is possible that even in this early period of the Slavonic languages’ development the pronunciation of this vowel varied greatly between dialects. In further development the vowel underwent changes and in the BCS area it was replaced by vowels [e], [i] or [ie] (the last is a diphthong). Thus the Proto-Slavonic *\textit{dětь} ‘child’ became \textit{dete} in ekavian, \textit{dite} in ikavian and \textit{dijete} in ijekavian.}\textsuperscript{,}\footnote{Šćakavian dialects use \textit{šć} (\textit{ognjišće}) and \textit{žđ} (\textit{zvižđi}) while in štakavian dialects the same words have \textit{št} (\textit{ognjište} ‘fireplace’) and \textit{žd} (\textit{zviždi} ‘s/he whistles’).} Apart from features which are important as factors which help us distinguish different Štokavian dialects, there are common Štokavian features which clearly differentiate them from Čakavian and Kajkavian dialects. \citet[17f]{Lisac03} lists the following as the main features of Štokavian:

\begin{enumerate}
\item \textit{što}/\textit{šta} ‘what’ as an interrogative pronoun;
\item differentiation between two short and two or three long (with rising and falling intonation) types of accent;
\item non-accented long syllables are preserved (but not equally in all idioms);
\item morphonological alternations: \textit{jt} → \textit{ć} (e.g. \textit{pojti} → \textit{poći} ‘go’), \textit{jd} → \textit{đ} (\textit{pojdem} → \textit{pođem} ‘I go’);
\item new iotation of dental and labial consonants with a lot of exceptions especially in Bosnia and Slavonia (\textit{djevojka} → \textit{đevojka} ‘girl’);
\item the loss of the phoneme [h] (with some exceptions);
\item the ending \textit{-a} in the genitive plural for feminine and masculine nouns, with a lot of exceptions ;
\item the ending \textit{-u} for masculine and neuter nouns in the locative singular;
\item the infix \textit{-ov-}\slash\textit{-ev-} for masculine nouns in the plural, with a lot of exceptions, especially between the Neretva and Dubrovnik;
\item syncretism of the dative, locative and instrumental plural for nouns, with many exceptions;
\item preservation of the aorist, with some exceptions around Dubrovnik;
\item numerous borrowings from Turkish.
\end{enumerate}

For more information on the distinctive phonological, morphological and syntactic features of Štokavian dialects see \citet[19--26]{Lisac03}. 

\section{Available data}
\label{Available data}
\subsection{Types of available data}

Dialectology is an important research field at universities and research institutions in Bosnia, Croatia and Serbia: for instance, it is an obligatory part of the study programme for students who decide to study Croatian language and literature in Zagreb, Osijek, Rijeka and Split. The leading Institut za hrvatski jezik i jezikoslovlje (Institute of Croatian language and linguistics) has a separate department of dialectology. Similarly the Institut za srpski jezik Srpske akademije nauka i umetnosti (Institute for the Serbian language of SASA) conducts dialectological projects. This tradition goes back to pre-Yugoslavian times and for instance the anthology series \textit{Srpski dijalektološki zbornik} was first published in 1905 and \textit{Hrvatski dijalektološki zbornik}, in 1956. Besides these anthologies there are special dialectological journals such as \textit{Čakavska rič, Kaj} etc. Moreover, many PhD dissertations are dedicated to idioms of individual villages or regions. 

Published studies are usually based on data collected during fieldwork carried out by the researcher. In most cases these are interviews with NORM (non-mobile old rural male) speakers and the focus is on phonetics and phonology. Morphology and syntax are not studied thoroughly and the parts which are dedicated to these subdisciplines mainly consist of incompletely structured observations on archaic or innovative forms for cases, tenses, word order and sometimes agreement. Lexis is also usually poorly studied in general studies of some idioms. Scholars normally limit themselves to observations on archaic vocabulary, and German, Hungarian, Italian, Romanian or Turkish lexical borrowings. Nonetheless, some specialised studies such as \citet{Plotnikova97}, \citet{BozovicBascarevic05}, \citet{MLK18}, \citet{VuleticSkracic18}, \citet{Horvat18}, and \citet{Filipan13} focus exclusively on lexis. 

Studies published in journals are rarely accompanied by transcripts of interviews in the appendix, while some studies published in \textit{Srpski dijalektološki zbor\-nik} and \textit{Hrvatski dijalektološki zbornik} provide only excerpts from interviews. PhD dissertations usually do not even provide excerpts and since in most cases researchers have to finance their fieldwork themselves, they tend to be unwilling to share their materials. Some excerpts can be found in \citet{Lisac03, Lisac09} and \citet{Okuka08} dialectological handbooks and in \citet{MenacCelinic12} dialectological reader. Importantly, as far as we know no dialectological transcripts have been digitalised. \citet{MenacCelinic12} are the only who  attach an audio CD.

While investigating the literature, we focused first of all on three dialectology handbooks of Štokavian: \citet{IPR01}, \citet{Lisac03} and \citet{Okuka08}. Next, we concentrated on the most extensive and popular sources of dialectological studies, \textit{Hrvatski dijalektološki zbornik} and \textit{Srpski dijalektološki zbornik}. In this case we decided to take into account only the anthologies published after 1950. The reason is very simple: if the study was published before 1950, then the fieldwork was conducted even earlier and the informants were most probably born in the 19\textsuperscript{th} century, and thus the language they spoke may have been very different from the language of the speakers who live in the same area now. Additionally we decided to include all open-access papers which included data on CLs in dialects and were available through the journal portal \textcolor{black}{\textit{Hrčak}}.\footnote{\textcolor{black}{For more information visit} \url{https://hrcak.srce.hr/}.} Furthermore, the data were supplemented with the dialectological literature (mainly journals and PhD dissertations) available at the library of Institute of Croatian language and linguistics in Zagreb, which is mostly inaccessible outside Croatia.  

\subsection{Data quality}

In this section we would like to comment on the quality of publicly available printed excerpts of dialectological texts based on examples (\ref{(7.1)}--\ref{(7.4)}), which are taken from three different handbooks published in this century.
 
\begin{exe}
\ex\label{(7.1)}
 /0.00/ – Kȁ smo c̐ìnili maturḁ̄lnu vèc̐eru, i tȏ smo ȉšli u Dùbrovac̐ku Rȋ̯éku. Tada bila tete Jele. Ti nec̐eš to znat ʒ̐e je to. /0.08/ I tamo, sve to skupa, bili, bälali. Vidimo mi: profesur ti s ńǫ bälḁ stḁlno. /0.14/ A onda, ajde pjevaj – Kḁte pjevḁ ko staglin, razumi̯eš, izjutra rano, sve do zore smo ostali. /0.20/ I posļe smo se uputili, a oni dvoje skupa. Nama se odmḁ štaklo. A l̦eti se vjenc̐ali, razumi̯eš.
\begin{flushright}Marija Matana, Dubrovnik. Recorded by Martina Lobaš, 1989 \\ \citep[70]{MenacCelinic12}\end{flushright}
\ex
\label{(7.2)} Mlàdić bi se vjérō, pa bi hódō nȅkol̕iko vrȅmena. Pȍslije bi dòšō u ròditējā, da se pȋtā, da se prȍsī. Dohòdijo mu je òtac, ili brȁt, òni dvȃ bi dòšli u pròšńu. Mlàdīć i nèkā ńègova svȏjta – ako nè bi ȉmo òca, ȉmō bi dȗnda. Ȍnda bi se ugovóril̕o nakon kȍlikō će se vjènča. I ȍnda bi bȋla proglášēńa, trȋ púta i bȋla bi svȁdba – kad bi se odlúčilo. Tȗ bi se pjȅval̕o, pȋl̕o. Pjȅvali su, názdravl̦ali: Lȉjepō ime Áne, Bog jē žívijo, mnȍgo ljȇtā srȅtna bíla, mnȍgo ljȇta žívjel̕a…
\begin{flushright}BHDZ, VII, 1996, 238, 241, 243  \\ \citep[113f]{Lisac03}\end{flushright}
\ex
\label{(7.3)} Žívi̭la sam ù slami ȍsam gȍdīna. A pȕno đècē. Dȅsetoro đecē sam ròdila. I skȕpla ònu đȅcu, sȁd nȅko i̭e ì i̭umrlo, odránla sam i sȅdmero. I u tôj slȁmi sam i odránla. Bȉi̭edno, bȉi̭edno odránla. I mȁlo pòmalo i đèca, kȁko kòi̭e mȁlo jȁče, ono pòmāže, pòmogne mi. I tàkō ȉ i̭onda jâ prêđem óvdekā. Tȁm sam bíla kò šume. Pȍšl̕e, vála bȍgu, đèca dòbra, pȁmetna mi đèca, dòbro mi.\\
\begin{flushright} Đúja Todórović. 82. yr., Trnjaci [Semberija], Subotić 1973, 125  \\ \citep[109]{Okuka08}\end{flushright}
\ex\label{(7.4)} 
	Jednostavno ne samo što gubimo materijalna dobra, mi, mi imam... mi gubimo žrtv... ne, narod uopšte, ne, ne samo svoje bližnje, nego uopšte narod…
\begin{flushright}  The Tübingen corpus of spoken Bosnian language, Transcript BH\end{flushright}
\end{exe}

\noindent The first three examples represent \textit{Istočnohercegovački} dialect. In (\ref{(7.1)}) the stress is marked only above the words in the first sentence, while in other excerpts it is marked consistently everywhere. The name of the speaker is given in examples (\ref{(7.1)}) and (\ref{(7.3)}), while age is given only in the latter. Both examples provide the year in which the recording was taken. In (\ref{(7.2)}) only information on the source of the excerpt is given, while finding all other data requires access to the original. Information on the year and the place of the fieldwork, like the sex, age and education of informants, is crucial and it should always be provided by the dialectologist. 

In some cases informants told folk stories, which were recorded and transcribed as dialectological material. Although we believe that this kind of material is valuable for many reasons, much speaks against including it in the analysis of a given dialect. Above all, this kind of material is often learned by heart from senior members of the community, and contains structures which are not actively in use. The interpretation of structures found in this kind of material as a distinctive features of the dialect/idiom of interest may give a distorted overall picture of that dialect/idiom. 

We would like to address one more data quality problem. If we compare the examples presented in (\ref{(7.1)}--\ref{(7.3)}) with the one in (\ref{(7.4)}), we see that the excerpt presented in (\ref{(7.4)}) provides a more natural depiction of the speech flow. (\ref{(7.4)}) is an example of the speech of a mobile non-rural middle aged woman from Bosnia (born in Serbia). It is hard to believe that the speech flow of rural informants can be as fluent as is presented in examples (\ref{(7.1)}--\ref{(7.3)}). Therefore, we assume that those who transcribed the interviews undertook certain interventions and made the texts look more like written texts and less like free, unprepared spoken language.   

\subsection{Examples in this chapter}

Since as we demonstrated, dialectologists do not present their data uniformly, we decided to present the examples provided in this chapter in their original form, i.e. according to the transcription which was used by the author we quote. In the glossed examples we also provide the dialect or local idiom name. The names of the dialects are used according to the terminology proposed in the previous Section \ref{An overview of BCS Štokavian dialects}. Some authors write about the idioms of certain villages and they do not specify to which dialect these belong. In such cases, we estimated the dialect according to the borders of dialects in dialectological maps in \citet{IPR01}, \citet{Lisac03} and \citet{Okuka08}. Sometimes in the reviewed literature, the dialectological material was described not only with respect to the dialect, but also the subdialect or even the local idiom. Whenever these data are available, we provide them in the running text for the sake of future research. As we show below, the CL system may vary even between local idioms which belong to the same subdialect or even dialect. 

Some examples presented here come from dialectological handbooks or papers and they are cited from running texts and not from transcripts. As it turns out, in many cases the quoted authors do not use full sentences, i.e. they start their examples without a capital letter and finish them without punctuation marks. We quote such examples exactly as they appear in the original running text. However, when we quote examples from transcripts and if we do not need the whole sentence to illustrate our argumentation, we use the symbol [\dots] to indicate that the beginning and/or the end of the sentence is missing.

\section{Inventory}\label{Inventory:8}
\subsection{Pronominal clitics}

In the following subsections we concentrate on CL forms which diverge from those used in BCS standard varieties. Furthermore, we discuss CL forms which caused many disputes among scholars in ex-Yugoslavia, for instance the third person feminine accusative CL \textit{ju} and reflexive CL \textit{si}. Note that we did not find data for pronominal CLs of the first and second person singular. Dialectologists usually find only archaic or innovative features or features which are divergent in some way to be worth noting and commenting. We assume that in such cases the forms very probably correspond to those in the respective standard variety. This would indicate that pronominal CLs of the first and second person singular do not diverge from forms in BCS standard varieties.

Every subsequent empirical study of CLs in BCS dialects has to begin with the established inventory of the units which will be examined. In that respect, the following subsections on the inventory of CLs may provide valuable information. However, readers who are not interested in this comprehensive account of the inventory can skip this part.

\subsubsection{Feminine pronominal clitics}
\label{Feminine pronominal clitics}



\subsubsubsection{Feminine pronominal clitics in the accusative}

In Section \ref{Inventory of pronominal clitics in BCS standard varieties} we discussed certain differences in the third person singular feminine accusative CL in standard varieties. While standard Bosnian and Serbian use \textit{je} as the default, an increase in the use of \textit{ju} has been observed in standard Croatian since the end of the 20\textsuperscript{th} century. If we look at Table \ref{T7.2}, we see that both \textit{ju} and \textit{je} forms are attested in Old and Neo-Štokavian dialects.\footnote{For some CLs we did not find any information in the reviewed dialectological literature. We indicate these cases with “data NA” in the relevant table fields.} 

\begin{table}[t]
\caption{CL forms of the third person singular feminine pronoun\label{T7.2}}
\fittable{\begin{tabular}{lll}
\lsptoprule
Dialect (subdialect or idiom) & her.\textsc{acc} & her.\textsc{dat} \\\midrule
\textit{Šumadijsko-vojvođanski} (Ilok) &\textit{ju}	& data NA  \\
\textit{Istočnohercegovački} &\textit{je}	&\textit{jon} \\
\textit{Zapadni} (Sinj, Bitelić) &\textit{je} &\textit{jon}	 \\
\textit{Slavonski} 	&\textit{ju} & data NA	 \\
\textit{Srednjobosanski} (Fojnica) (Sarajevo)	&\textit{je\slash ju je} &\textit{joj} \\
\textit{Zetsko-južnosandžački} (Istočnocrnogorski) 	&\textit{ju} 	&data NA \\
\textit{Kosovsko-resavski} 	&\textit{ju} &\textit{ ju} \\
\textit{Timočko-lužnički} (Vlasinski) & \textit{gu} beside \textit{ju} & \textit{đu} beside \textit{voj} \\
\textit{Timočko-lužnički }(Lužnički) &\textit{ju} &\textit{voj}	 \\
\textit{Svrljiško-zaplanjski} &\textit{ju} &\textit{voj}, \textit{vu} \\
\textit{Prizrensko-južnomoravski} &\textit{gu}, \textit{ga}, \textit{ja}, \textit{je}, \textit{ju}, \textit{u}  	&\textit{gu}, \textit{gi}, \textit{i}, \textit{je} \\
\lspbottomrule
\end{tabular}}
\end{table}

While in the far East of Croatia in the local idiom of Ilok (Neo-Štokavian \textit{Šumadijsko-vojvođanski} dialect), in the idioms of Baranja (Old Štokavian \textit{Slavonski} dialect) and in Zagreb (\textit{Turopoljski} Kajkavian dialect) the form \textit{ju} is in use, Croats in Southern Croatian areas such as Sinj, Bitelić and Imotski (Neo-Što\-ka\-vi\-an \textit{Zapadni} dialect) prefer \textit{je} (cf. \citealt[130]{Lisac03}, \citealt[331]{Sekeres77}, \citealt[64]{Hoyt12}, \citealt[185]{Curkovic14}, \citealt[120]{Simundic71}).\footnote{This does not mean that all idioms of \textit{Šumadijsko-vojvođanski} employ \textit{ju} as a CL; we are aware of certain differences. For instance, \citet[259]{Radovanovic06} claims that the form \textit{ju} is not attested in the language of her informants from Kolubara.} In the neighbouring Neo-Štokavian \textit{Istočnohercegovački} dialect the CL \textit{je} dominates as well (see \ref{(7.5)}), although there are some local idioms such as Grude where only \textit{ju} is attested (cf. \citealt[174]{Halilovic96}, \citealt[200]{Peco07a}, \citealt[311]{Peco07b}).\footnote{According to Peco (cf. \citealt[200]{Peco07a}, \citealt[311]{Peco07b}) the Grude idiom is part of the \textit{Istočnohercegovački} dialect, whereas \textcolor{black}{dialectological map in} \citet[162]{Lisac03} has Grude as a part of the \textit{Zapadni} dialect. In both cases it is Neo-Štokavian.} In some local idioms of \textit{Istočnohercegovački} dialect, for instance in Banja Vrućica, and in the local idiom spoken in the Neretva river valley, the CL \textit{ju} does not exist at all (cf. \citealt[371]{Dragicevic07}, \citealt[142]{VuksaNahod14}). 

\begin{exe}\ex\label{(7.5)}
\gll Bog  \textbf{jē}  žívijo \\
God her.\textsc{acc}  live.\textsc{ptcp.sg.m} \\
\glt ‘May God give her a long life’                                    
\hfill  (Istočnohercegovački; \citealt[113]{Lisac03})
\end{exe}

\noindent In contrast to the mixed \textit{je}/\textit{ju} distribution in Neo-Štokavian dialects, according to the dialectological data it seems that in Old and Middle Štokavian dialects the form \textit{ju} is dominant. It is the only variant found in the \textit{Istočnocrnogorski} idiom (\textit{Zetsko-južnosandžački} dialect) and in the idioms of North Metohija (\textit{Kosovsko-resavski} dialect) (cf. \citealt[200]{Peco07a}, \citealt[223]{Bukumiric03}).\footnote{\citet[200]{Peco07a} uses the term \textit{Zetsko-gornjopolimski} dialect instead of \textit{Zetsko-južnosandžački} dialect.} \citet[174]{Halilovic96} even assumes that this form originates from idioms spoken in Montenegro.

\begin{exe}\ex\label{(7.6)}
\gll [\dots] da \textbf{ju} izvūčȅ iz jȁmē. \\
{} that  her.\textsc{acc} pull.out.3\textsc{prs}  from pit \\
\glt ‘[\dots] to pull her out of the pit.’
\hfill  (Zetsko-južnosandžački; \citealt[189]{Okuka08})
\end{exe}

\noindent However, it seems that the form \textit{je} is prevalent in the Old Štokavian \textit{Srednjobosanski} dialect (cf. \citealt[126]{Brozovic07}, \citealt[58]{HTS09}). In two Middle Štokavian idioms of the \textit{Timočko-lužnički} dialect, \textit{Vlasinski} and \textit{Lužnički}, like in the \textit{Svrljiško-zaplanjski} dialect, the third person singular feminine accusative CL is \textit{ju} \citep[cf.][272f, 254]{Okuka08}. In contrast to the latter two Torlac dialects, the \textit{Prizrensko-južnomoravski} dialect additionally uses various other forms such as \textit{gu}, \textit{ga}, \textit{ja}, \textit{je} and \textit{u} besides \textit{ju} (cf. \citealt[111]{Stevanovic50}, \citealt[237]{Okuka08}, \citealt[51]{Mladenovic10}). There are, however, certain differences among local idioms of that dialect. \citet[111]{Stevanovic50} claimed that in the local idiom of Đakovica the pronominal CL \textit{je} is never used and data from the beginning of the 21\textsuperscript{st} century do not indicate any changes \citep[cf.][51]{Mladenovic10}. However, some other idioms of \textit{Prizrensko-južnomoravski} do have the CL \textit{je} \citep[cf.][51]{Mladenovic10}.


\subsubsubsection{Feminine pronominal clitic in the dative}

As may be seen in Table \ref{T7.2}, the dative forms of the feminine pronominal CL vary between dialects even more than the accusative ones. The speakers of the local idiom of the Neretva valley use the CL \textit{jon} exclusively \citep[cf.][143]{VuksaNahod14}, whereas among speakers of the local idiom of Bitelić only the older generation uses this variant while the language of younger speakers is slowly changing in the direction of the standard variety in this regard  \citep[cf.][186]{Curkovic14}.\footnote{\citet[120]{Simundic71} claims that \textit{jon} is the younger form.} The dative CL \textit{jon} is attested in the local idiom of Dubrovnik as well: see example (\ref{(7.7)}).

\protectedex{\begin{exe}\ex\label{(7.7)}
\gll nema  \textbf{jon}  spasa \\
 \textsc{neg}.have.3\textsc{prs}  her.\textsc{dat}  salvation \\
\glt ‘she cannot be saved’
\hfill  (Istočnohercegovački; \citealt[301]{Okuka08})
\end{exe}
}

\noindent All the local idioms mentioned are Neo-Štokavian. While local idioms spoken in the Neretva valley and in Dubrovnik belong to \textit{Istočnohercegovački}, the local idiom of Bitelić belongs to the \textit{Zapadni} dialect. 

In Middle Štokavian Torlac dialects such as \textit{Timočko-lužnički} and \textit{Svrljiško-zaplanjski} the third person singular feminine dative CL is \textit{voj} \citep[cf.][272, 254]{Okuka08}.\footnote{However, it is not the only possible form for the third person feminine dative CL since in the \textit{Vlasinski} idiom speakers also use \textit{đu} besides \textit{voj} \citep[cf.][273]{Okuka08}.} Additionally, the form \textit{vu} is also attested in the latter \citep[cf.][254]{Okuka08}. As in the case of the third person singular feminine accusative, various forms of the dative feminine pronoun are attested in the \textit{Prizrensko-južnomoravski} dialect: \textit{gu}, \textit{gi}, \textit{i} and \textit{je} (cf. \citealt[237]{Okuka08}, \citealt[51]{Mladenovic10}). 

In contrast, in some Old Štokavian dialects, such as \textit{Srednjobosanski}, the CL \textit{joj}, the standard form in all varieties of BCS, is used \citep[cf.][58]{HTS09}.

\subsubsection{Masculine pronominal clitics}

In Table \ref{T7.3} we give a short overview of different CL forms for the third person singular masculine pronoun. Since there are no data on the dative CL form, we assume that it does not differ from the form used in the standard variety. The accusative forms mentioned in the dialectological literature will be discussed in detail below. 

\begin{table}
\caption{CL forms of the third person singular masculine pronoun\label{T7.3}}
\begin{tabular}{lll}
\lsptoprule
Dialect (subdialect or idiom)& him.\textsc{acc}& him.\textsc{dat} \\\midrule
\textit{Zapadni} (Sinj, Bitelić) & \textit{nj}& data NA \\
\textit{Istočnohercegovački} (Grude, Neretva) & \textit{nj}, \textit{jn} & data NA \\
                                              & \textit{nje}& \\
\lspbottomrule
\end{tabular}
\end{table}

As mentioned in Section \ref{Inventory of pronominal clitics in BCS standard varieties} the CL \textit{nj}, which is used exclusively with prepositions, is considered archaic in contemporary standard Serbian, but seems to be frequent in some Neo-Štokavian idioms. 

Speakers of \textit{Istočnohercegovački} dialect use the third person singular masculine accusative CL \textit{nj} for both animate and inanimate referents (cf. \citealt[200]{Peco07a}, \citealt[311]{Peco07b}). In the local idiom of Bitelić the CL \textit{nj} is used only after prepositions with an additional vowel [a], whereas in the local idiom spoken in the Neretva valley it is used after all prepositions (cf. \citealt[185]{Curkovic14}, \citealt[143]{VuksaNahod14}). Besides the CL form \textit{nj}, in the local idiom spoken in the Neretva valley the equivalent form \textit{jn} exists as well \citep[cf.][142f]{VuksaNahod14}.\footnote{\citet{VuksaNahod14} does not specify in what context speakers of the local idiom of the Neretva valley use the CL \textit{jn}. Moreover she does not state whether there are other differences between \textit{nj} and \textit{jn} besides formal ones.} In addition to the two mentioned accusative forms, in the local idiom of the Neretva valley the variant \textit{nje} is found \citep[cf.][143]{VuksaNahod14}. This form is employed only after prepositions, as in example (\ref{(7.8)}).\footnote{\citet{VuksaNahod14} does not specify whether the CL \textit{nje} follows only a certain type of preposition, or any preposition used with the accusative case.}

\begin{exe}
\ex\label{(7.8)}
\gll ná  \textbf{nje}  \textbf{se}  mȅtnē  \\
on him.\textsc{acc} \textsc{refl}  put.3\textsc{prs}  \\
\glt ‘one puts it on him/it’
\hfill  (Istočnohercegovački; \citealt[143]{VuksaNahod14})
\end{exe}

\noindent At the end of this subsection we would like to point out that in the dialectological literature reviewed we did not find any information on CL forms used without prepositions for the third person masculine pronoun in the accusative. Therefore, we assume that it does not differ from the form used in BCS standard varieties, where only non-clitic form may be used in this context.


\subsubsection{Plural pronominal clitics}
\label{Plural pronominal clitics in the accusative}
\subsubsubsection{Plural pronominal clitics in the accusative}

In some Štokavian dialects accusative pronominal plural forms differ strikingly from the forms used in the contemporary standard varieties of BCS. A short overview of these forms is presented in Table \ref{T7.4}.

\begin{table}
\caption{CL forms of the accusative plural pronouns\label{T7.4}}
\fittable{\begin{tabular}{llll}
\lsptoprule
Dialect (subdialect or idiom) & we.\textsc{acc}	& you.\textsc{acc}	& they.\textsc{acc} \\\midrule
\textit{Šumadijsko-vojvođanski} (Kolubarski)						& \textit{ne}		& \textit{ve}		& data NA \\
\tablevspace
\textit{Istočnohercegovački} &\textit{ne} 	& \textit{ve}		& \textit{hi, hig, hin, ih,} \\
(Zapadnocrnogorski) (Kula, Cuce) &&& \textit{i, ig, ik, ji, jig, jih } \\
\tablevspace
\textit{Kosovsko-resavski}								& \textit{ne}		& \textit{ve}		& \textit{i, ju} \\
\tablevspace
\textit{Slavonski}										& data NA		& data NA		& \textit{je, i} \\
\tablevspace
\textit{Zetsko-južnosandžački}								& \textit{ne}		& \textit{ve}		& data NA \\
\tablevspace
\textit{Svrljiško-zaplanjski}								& \textit{ni}		& \textit{vi}		& \textit{i} \\
\tablevspace
\textit{Prizrensko-južnomoravski}							& \textit{ne}		& \textit{ve}		& \textit{gi, ge, giv, i, i(h), ji} \\
\lspbottomrule
\end{tabular}}
\end{table}

\largerpage[-2]
\citet[74]{Okuka08} and \citet[152]{Pesikan65} claim that speakers of the \textit{Zapadnocrnogorski} subdialect use the CL form \textit{ne} instead of the first person plural accusative CL \textit{nas}. However, following \citet{Peco07a} we must emphasize that \textit{ne} is not typical of the entire territory of the \textit{Istočnohercegovački} dialect. Specifically, this form is not present in local idioms of Eastern Herzegovina \citep[cf.][297]{Peco07a}. Additionally, \citet[197]{Peco07a} claims that he found a single example of the archaic second person plural accusative CL \textit{ve} on the territory of Eastern Herzegovina, in the local idiom of Kula, as in example (\ref{(7.9)}).

\begin{exe}\ex\label{(7.9)}
\gll kȁmo  ve  jȍš?\\
where you.\textsc{acc} still\\
\glt ‘where are you going?
\hfill  (Istočnohercegovački; \citealt[197]{Peco07a})
\end{exe}


\largerpage[-2]
\noindent \citet[141]{Okuka08} lists \textit{ne} and \textit{ve} as the plural CL accusative forms used in the Neo-Štokavian \textit{Kolubarski} subdialect (\textit{Šumadijsko-vojvođanski} dialect). The mentioned forms are preserved in the Middle Štokavian \textit{Prizrensko-južnomoravski} dialect as well (cf. \citealt[110]{Stevanovic50}, \citealt[46]{Mladenovic10}). The CL forms \textit{ne} and \textit{ve} are also characteristic of the Old Štokavian \textit{Zetsko-južnosandžački} dialect \citep[cf.][88]{Barjaktarevic66}. Moreover, they are a trait connecting the idioms of the \textit{Zetsko-južnosandžački} dialect to the Old Štokavian idioms of the \textit{Kosovsko-resavski} dialect and to idioms in the southeastern part of the Neo-Štokavian \textit{Istočnohercegovački} dialect area \citep[cf.][121]{Lisac03}. Additionally, \citet[221]{Bukumiric03} claims that the old CL forms \textit{ne} and \textit{ve} are well preserved in idioms of North Metohija (\textit{Kosovsko-resavski} dialect).

According to \citet[255]{Okuka08} the first and second person plural accusative CLs \textit{ni} and \textit{vi} are preserved in the Middle Štokavian \textit{Svrljiško-zaplanjski} dialect. These forms are used in the \textit{Timočko-lužnički} dialect as well. Moreover, \citet[201]{Ivic57} claims that as accusative CLs, \textit{ni} and \textit{vi} are older than \textit{ne} and \textit{ve}. The usage of the CL \textit{i} (and not \textit{gi}) for the third person plural accusative differentiates the \textit{Svrljiško-zaplanjski} dialect from the neighbouring Eastern and Southeastern Serbian idioms \citep[cf.][255]{Okuka08}.\footnote{The existence of the third person plural accusative CL \textit{gi} is a trait which connects Eastern and Southeastern Serbian idioms with Northeastern Macedonian and Western Bulgarian idioms \citep[cf.][20]{Okuka08}.} In contrast to the Torlac dialects just mentioned, which do not show such a great degree of variation with respect to the third person plural accusative CL, in the \textit{Prizrensko-južnomoravski} dialect the following forms are attested: \textit{gi}, \textit{ge}, \textit{giv}, \textit{i}, \textit{i(h)} and \textit{ji} (cf. \citealt[237]{Okuka08}, \citealt[52]{Mladenovic10}).

Variation affects the third person plural accusative CL in the \textit{Istočnohercegovački} dialect as well, where scholars attest dozens of such forms. Peco (cf. \citealt[202]{Peco07a}, \citealt[311]{Peco07b}) lists the following CL forms of the third person plural in the accusative and genitive: \textit{hi}, \textit{hig}, \textit{hin}, \textit{i}, \textit{ig}, \textit{ih}, \textit{ik}, \textit{ji}, \textit{jig}, and \textit{jih}.\footnote{It seems that \textit{hi} is a very old CL form. \citet[14]{HTS09} state that the CL \textit{hi} was used in the local idiom of Sarajevo in the 18\textsuperscript{th} century. They class this CL as a general Bosnian phenomenon, especially in more archaic idioms \citep[cf.][14]{HTS09}. When speaking about the local idiom of Sarajevo in the 19\textsuperscript{th} century, \citet[21]{HTS09} claim that there were differences between Muslim, Orthodox and Catholic speakers with respect to the CL form they tended to use. So, for instance Muslim speakers purportedly tended to use the enclitic \textit{hi} the most often, \textit{hin} more rarely, and \textit{ih} the least, while Orthodox and Catholic speakers tended to use \textit{i(h)} \citep[21]{HTS09}.}\textsuperscript{,}\footnote{Some of those forms, such as \textit{hi}, \textit{hin}, and \textit{him}, are used in the \textit{Srednjobosanski} dialect as well \citep[41]{Halilovic05}.} As claimed by \citet[202f]{Peco07a}, not all of these forms are equally widespread, and sometimes the same informant can switch from one form to another during one session. The sentences below exemplify the CL forms in the local idioms of Kula (\ref{(7.10)}), Dabar (\ref{(7.11)}) and Divin (\ref{(7.12)}). 

\begin{exe}\ex\label{(7.10)}
\gll [\dots] a  đèca  sve  vȋčū: ȅto  \textbf{hi}. \\
{} and children all yell.3\textsc{prs}  there them.\textsc{gen} \\
\glt ‘[\dots] and the children are all yelling: there they are.’ \\
\hfill  (Istočnohercegovački; \citealt[281]{Peco07a} )
\ex\label{(7.11)}
\gll Bílo  \textbf{ig}  \textbf{je}  prȉje   vȉšē.  \\
be.\textsc{ptcp.sg.n} them.\textsc{gen} be.\textsc{3sg}  before more \\
\glt ‘There were more of them before.’
\hfill  (Istočnohercegovački; \citealt[285]{Peco07a})
\ex\label{(7.12)}
\gll Ovī  \textbf{ji}  fìno  {dòčekajū [\dots]} \\
these them.\textsc{acc} nicely welcome.3\textsc{prs} \\
\glt ‘They welcome them nicely.’
\hfill  (Istočnohercegovački; \citealt[287]{Peco07a})
\end{exe}

\noindent Besides the third person plural accusative form \textit{i}, which appears in several Štokavian dialects, the \textit{Slavonski} dialect preserves the old accusative plural CL form \textit{je} \citep[cf.][48]{FarkasBabic11}. 

\subsubsubsection{Plural pronominal clitics in the dative}

In some dialects and their local idioms, dative plural CLs are the same as those which are part of standard varieties, as for example in the local idiom of Sarajevo (\textit{Srednjobosanski} dialect) \citep[cf.][58]{HTS09}. However, Table \ref{T7.5} reveals a great deal of variation in respect of dative plural CL forms. The data is reconstructed from the scattered information we found in dialectological literature. 

\begin{table}
\caption{CL forms of the plural pronouns in the dative\label{T7.5}}
\fittable{
\begin{tabular}{llll}
\lsptoprule
Dialect   & we.\textsc{dat} & you.\textsc{dat} & they.\textsc{dat}\\
(subdialect or idiom) & \\\midrule
\textit{Srednjobosanski} & \textit{nam} & \textit{vam} & data NA\\
(Sarajevo) & & & \\\tablevspace
\textit{Istočnohercegovački} & \textit{nam, ni} (rarely) & \textit{vi, vam} & \textit{him, im, jim, jin}\\
(Nikšićka Župa, Cuce) & & & \\\tablevspace
\textit{Šumadijsko-vojvođanski} & \textit{ni} & \textit{vi} & data NA\\
(Kolubarski, Šumadija) & & & \\\tablevspace
\textit{Zetsko-južnosandžački} & \textit{ni} & \textit{vi} & data NA\\
\textit{Kosovsko-resavski} & \textit{ni} & \textit{vi}  & \textit{ju}\\
\textit{Svrljiško-zaplanjski} & data NA & data NA & \textit{im}\\
\textit{Prizrensko-južnomoravski} & \textit{ni}  & \textit{vi} & \textit{gi, gim, giv, i, im, ji, mgi}\\
\lspbottomrule
\end{tabular}
}
\end{table}


In the \textit{Istočnohercegovački} dialect the archaic CL form \textit{ni} for the first person plural dative is rather rare; for instance, it can be found in the Montenegro village Nikšićka Župa and in the Cuce tribe. In contrast, the archaic CL form \textit{vi} for the second person plural dative is quite common and appears in everyday language, as does the form \textit{vam}, which is in use in standard varieties (cf. \citealt[152]{Pesikan65}, \citealt[196f]{Peco07a}). The example presented in (\ref{(7.13)}) is from the local idiom of Divin.

\protectedex{\begin{exe}\ex\label{(7.13)}
\gll [\dots] ali \textbf{vi}  ne  mògu  dat  {ȍdgovōr [\dots].} \\
{} but you.\textsc{dat} \textsc{neg} can.1\textsc{prs}  give.\textsc{inf}  answer \\
\glt ‘[\dots] but I cannot give you the answer [\dots].’ \\
\hfill  (Istočnohercegovački; \citealt[287]{Peco07a})
\end{exe}
}

\noindent \citet[63, 72]{Okuka08} partially agrees with \citet[196f]{Peco07a} and underlines that the main trait of idioms in Eastern Herzegovina is the usage of the CL \textit{vi} instead of \textit{vam}. However, he admits that not all idioms of the \textit{Istočnohercegovački} dialect use this feature. For instance, in the \textit{Jugozapadnosrbijanski} subdialect the CL \textit{vi} appears only optionally, while in the \textit{Sjevernozapadnosrbijanski} subdialect only \textit{vam} is used as a CL \citep[cf.][78]{Okuka08}. Lisac's (\citeyear[103]{Lisac03}) opinion differs slightly from Okuka's (\citeyear{Okuka08}) and Peco's (\citeyear{Peco07a}); according to him, generally speaking in the \textit{Istočnohercegovački} dialect as a whole the second person plural dative CL is \textit{vam}, while the CL \textit{vi} appears only in idioms spoken in Eastern Herzegovina. 

However, authors (e.g. \citealt[88]{Barjaktarevic66}, \citealt[121]{Lisac03}, \citealt[177]{Okuka08}) do agree that the CLs \textit{ni} and \textit{vi} are characteristic of the Old Štokavian \textit{Zetsko-južnosandžački} dialect.\footnote{In some subdialects of the \textit{Zetsko-južnosandžački} dialect, such as in \textit{Sjeničko-novopazarski} subdialect, dative CLs for the 1\textsuperscript{st} and 2\textsuperscript{nd} person are replaced by accusative ones \citep[cf.][186]{Okuka08}.} The mentioned trait connects idioms of the \textit{Zet\-sko-juž\-no\-san\-džački} dialect to the southeastern idioms of \textit{Istočnohercegovački} and to idioms of \textit{Kosovsko-resavski} (cf. \citealt[121]{Lisac03}, \citealt[205]{Okuka08}). \citet[221]{ Bukumiric03} claims that the CL forms \textit{ni} and \textit{vi} are well preserved in idioms of North Metohija (\textit{Kosovsko-resavski} dialect) and that the forms \textit{nam} and \textit{vam} are quite rare. These CLs are also present in the \textit{Prizrensko-južnomoravski} Torlac dialect and in the Neo-Štokavian \textit{Šumadijsko-vojvođanski} dialect (in idioms of central Šumadija and in the Kolubarski subdialect) (cf. \citealt[110]{Stevanovic50}, \citealt[46]{Mladenovic10}, \citealt[291]{Remetic85}, \citealt[141, 237]{Okuka08}).

Besides the third person plural dative CL \textit{im}, Peco (\citealt[203]{Peco07a}, \citealt[311]{Peco07b}) mentions \textit{jim} and \textit{jin} as forms present in the Neo-Štokavian \textit{Istočnohercegovački} dialect. Below are examples from the local idioms of Nevesinje (\ref{(7.14)}) and Borač (\ref{(7.15)}). 

\begin{exe}\ex\label{(7.14)}
\gll Mȅne  šćȅri  zòvū  da  \textbf{jim}  ȉdēm. \\
me.\textsc{acc} daughters call.3\textsc{prs}  that  them.\textsc{dat} go.1\textsc{prs}  \\
\glt ‘My daughters are calling me to go to them.’ \\
\hfill  (Istočnohercegovački; \citealt[282]{Peco07a})
\ex\label{(7.15)}
\gll Švábo  \textbf{him}  né  šće   nȉšta.  \\
German them.\textsc{dat} \textsc{neg} fut.3\textsc{sg} nothing \\
\glt ‘The German will not do anything to them.’ \\
\hfill (Istočnohercegovački; \citealt[283]{Peco07a})
\end{exe}

\noindent Whereas in the Middle Štokavian \textit{Svrljiško-zaplanjski} Torlac dialect the third person plural dative CL is \textit{im} \citep[cf.][255, 237]{Okuka08}, speakers of the \textit{Prizrensko-južnomoravski} Torlac dialect use several variants such as: \textit{gi}, \textit{gim}, \textit{giv}, \textit{i}, \textit{im}, \textit{ji}, \textit{mgi} (cf. \citealt[237]{Okuka08}, \citealt[34]{Mladenovic10}). One example with the CL \textit{gi} from the local idiom of Prizren is presented in (\ref{(7.16)}).

\begin{exe}\ex\label{(7.16)}
\gll [\dots] otíša da  \textbf{gi}  č̕estíta. \\
 {} leave.\textsc{ptcp.sg.m} that  them.\textsc{dat} congratulate.3\textsc{prs}  \\
\glt ‘[\dots] he left to congratulate them.’ \\
\hfill  (Prizrensko-južnomoravski; \citealt[247]{Okuka08})
\end{exe}

\noindent \citet[202]{Ivic57} claims that the CLs \textit{ju} and \textit{gi} as a third person plural dative form are in use only in those idioms which use \textit{ni}, \textit{vi} and/or \textit{ne}, \textit{ve} as first and second person plural dative and accusative CLs. 

%\subsection{Verbal clitics}

\subsection{Verbal clitics: Aoristal/conditional clitics of the verb \textit{biti} ‘be’}
\label{Verbal clitics:8}
In contrast to pronominal CLs it seems that verbal CLs do not vary much. In the \textit{Istočnohercegovački}, \textit{Zapadni}, \textit{Šumadijsko-vojvođanski}, \textit{Slavonski} and \textit{Kosovsko-resavski} dialects the CL \textit{bi} ‘would’ of the conditional auxiliary is used for all persons (cf. \citealt[331]{Peco07b}, \citealt[96]{KurtovicBudja09}, \citealt[302]{Radovanovic06}, \citealt[327]{Remetic85}, \citealt[377]{Dragicevic07}, \citealt[106]{Golic93}, \citealt[267]{Bukumiric03}). \citet[42]{Lisac12} admits that in the majority of Croatian idioms the usage of the CL form \textit{bi} for all persons prevails, but he emphasises that in the local idiom of Dubrovnik this form and forms appearing in the standard are used equally often.\footnote{\citet[42]{Lisac12} adds that the speakers of Čakavian also use, among others, \textit{bin}, \textit{biš}, \textit{bi}, \textit{bimo}, \textit{bite}, and \textit{bi}, while in Kajkavian one form, \textit{bi}, is usually used for all persons (for Čakavian see \citealt{Menac89}).} \citet[331]{Peco07b} believes that the CL form \textit{bi} used for all persons is spreading as a trait from dialects into standard language; if not in its written, then certainly in its spoken registers.\footnote{For the results of our analysis of spoken Bosnian with respect to this matter see Section \ref{Verbal clitics:9}.} Accordingly, \citet[165]{Aladrovic11} reports this feature in the written language of elementary school students from Požega.\footnote{According to the dialectological map, Požega belongs to the Old Štokavian \textit{Slavonski} dialect, but younger generations probably speak the \textit{Istočnohercegovački} dialect.}

Some variation with respect to conditional auxiliary CLs was detected in the local idiom spoken in the valley of the river Fojnica (\textit{Srednjobosanski} dialect) and in the Cuce tribe (\textit{Istočnohercegovački} dialect). Namely, the CL \textit{bišĕ} ‘they would’ was attested there (cf. \citealt[137]{Brozovic07}, \citealt[171]{Pesikan65}). Furthermore, \citet[171]{Pesikan65} attested \textit{bihu} as a third person plural auxiliary in the Bjelice and Zagarač Montenegrin tribes. 

%\subsubsection{Present/future tense clitics of the verb \textit{ht(j)eti} ‘will’}
%
%In the \textit{Vlasinski} subdialect of the Middle Štokavian Torlac \textit{Timočko-lužnički} dialect, the first person singular future CL of \textit{ht(j)eti} is \textit{če}. However, speakers of the \textit{Lužnički} subdialect, which is a part of the same dialect, use the CL \textit{ču} for the first person singular \citep[cf.][272f]{Okuka08}.
%
%The well-known difference between standard Bosnian and Serbian on the one hand, and Croatian on the other, is the way in which the future auxiliary CL is written when it follows \textit{-ti} infinitives (for details, see Section \ref{Inventory of verbal clitics in BCS standard varieties}). In contrast to the rules of standard orthography, according to which the future auxiliary CL and \textit{-ći} infinitives are always written separately, in some Štokavian dialects (\textit{Istočnohercegovački}, \textit{Zapadni}, \textit{Srednjobosanski} and \textit{Kosovsko-resavski}) the future auxiliary CL merges with \textit{-ći} infinitives (cf. \citealt[376]{Dragicevic07}, \citealt[96]{KurtovicBudja09}, \citealt[14]{HTS09}, \citealt[264]{Bukumiric03}).\footnote{\citet[164f]{Aladrovic11} records this dialectal feature in the spoken and written language of elementary school students in Požega.} The example presented in (\ref{(7.45)}) is from the local idiom of Sarajevo. 
%
%\protectedex{\begin{exe}\ex\label{(7.45)}
%\gll do\textbf{ću}  \\
%come.\textsc{inf}+\textsc{fut}.1\textsc{sg} \\
%\glt ‘I will come’
%\end{exe}\vspace{\vSpaceForLanguageExamples}\hfill  Srednjobosanski, (\citealt[14]{HTS09})
%}
%
%However, as example (\ref{(7.46)}) from the local idiom of Ložišće on the island Brač below shows, this kind of merging is attested not only in Štokavian, but also in Čakavian dialects. 
%
%\protectedex{\begin{exe}\ex\label{(7.46)}
%\gll lȅ\textbf{ću}  \\
%lie.\textsc{inf}+\textsc{fut}.1\textsc{sg} \\
%\glt ‘I will lie’
%\end{exe}\vspace{\vSpaceForLanguageExamples}\hfill  Južnočakavski, (\citealt[194]{Galovic13})
%}
%
%According to \citet[194]{Galovic13}, in the Čakavian idiom of Ložišće on Brač the CL forms of the auxiliary verb \textit{tȉt} ‘will’ are added to the infinitive base when they appear after the verb.

\subsection{Reflexive clitic \textit{si}}
\label{Reflexive clitic si}

The \textsc{refl\textsubscript{2nd}} CL \textit{si} is only part of standard Croatian, while Bosnian and Serbian normativists do not include it in the inventory of standard Bosnian and standard Serbian (for more information see Section \ref{Reflexive markers se and si in BCS standard varieties}).\footnote{Our typology of reflexives is presented in Section \ref{Conclusion: how many types of se do we need to distinguish?}.} However, as we will show in this section, this form is present in Štokavian dialects spoken on Bosnian and Serbian territory.\footnote{Moreover, the mentioned form is also present in the spoken variety of Bosnian: for more details see Section \ref{Reflexive clitics:9}.}

The \textsc{refl\textsubscript{2nd}} CL \textit{si} is quite common in Kajkavian dialects: see the example from the \textit{Gornjolonjski} Kajkavian dialect presented in (\ref{(7.17)}) \citep[cf.][242]{BrlobasLoncaric12}.

\protectedex{\begin{exe}\ex\label{(7.17)}
\gll mọ̃ram  \textbf{si}  ma̍le  počinọ̍ti \\
must.1\textsc{prs}  \textsc{refl}  little rest.\textsc{inf}  \\
\glt ‘I have to rest a little’
\hfill  (Gornjolonjski; \citealt[243]{BrlobasLoncaric12})
\end{exe}
}

\noindent This form is also widely used in the local idiom of Zagreb \citep[65]{Hoyt12}, and in the local idiom of Žumberak \citep[cf.][25]{Tezak85} – see example (\ref{(7.18)}) below.\footnote{In the idiom of Žumberak the features of all three dialects, Štokavian, Kajkavian and Čakavian, are present.} 


\protectedex{\begin{exe}\ex\label{(7.18)}
\gll Kúpijo  \textbf{sam}  \textbf{si}  knjȉgu. \\
buy.\textsc{ptcp.sg.m} be.1\textsc{sg} \textsc{refl} book \\
\glt ‘I bought myself a book.’
\hfill (Žumberak idiom; \citealt[255]{Tezak85})
\end{exe}
}

\noindent According to the dialectological data, the CL \textit{si} is not very typical of Štokavian and Čakavian dialects, although it is used occasionally.\footnote{\citet[158]{Vranic03} claims that in the Čakavian idioms of Pag island the long reflexive form is used more often than the CL one. Moreover, she provides Čakavian examples in which the reflexive in the dative is replaced with the construction: ‘preposition $+$ reflexive in accusative’ or ‘preposition $+$ personal pronoun in accusative’ \citep[cf.][158]{Vranic03}. \citet[158]{Vranic03} claims that such substitutions are quite frequent.} The reflexive CL \textit{si} does not exist in the following Neo-Štokavian idioms: in the local idioms of Bitelić and Imotski (\textit{Zapadni} dialect), in the local idioms of the Neretva valley and in the local idiom of Banja Vrućica (\textit{Istočnohercegovački} dialect), in idioms of central Bosnia (\textit{Zapadni} dialect) and in idioms of Kolubara (\textit{Šumadijsko-vojvođanski} dialect) (cf. \citealt[192]{Curkovic14}, \citealt[120]{Simundic71}, \citealt[142]{VuksaNahod14}, \citealt[371]{Dragicevic07}, \citealt[207]{Peco90}, \citealt[255]{Radovanovic06}).\footnote{Some central Bosnian idioms belong to the Old Štokavian \textit{Srednjobosanski} dialect.} Those idioms only have the full reflexive form \textit{sebi} in the dative. However, the dative \textsc{refl\textsubscript{2nd}} CL \textit{si} can be found in some Neo-Štokavian idioms, e.g. in Western Herzegovina (\textit{Zapadni} and \textit{Istočnohercegovački} dialect), although it is claimed to be rare \citep[cf.][311]{Peco07b}. This form is also found in some Old Štokavian idioms of Northeastern Bosnia (\textit{Slavonski} dialect) \citep[cf.][269]{Peco85}. In contrast, in the Middle Štokavian \textit{Svrljiško-zaplanjski} Torlac dialect the usage of the dative CL \textit{si} is frequent \citep[cf.][255]{Okuka08}. \citet[205]{Ivic57} claims that the mentioned CL can be found in the Middle Štokavian \textit{Prizrensko-južnomoravski} dialect as well. \citet[45]{Mladenovic10} later corroborated this claim and found this CL in six out of nine investigated idioms of the \textit{Prizrensko-južnomoravski} dialect.\footnote{\textcolor{black}{For a recent corpus linguistic study on the reflexive CL \textit{si} in Torlak dialect, see \citet[]{Cirkovic21}}.} Furthermore, it seems that Old Štokavian idioms are closer to the Middle Štokavian idioms with respect to the reflexive CL \textit{si}. Specifically, the form in question is also found in some Old Štokavian idioms of Northeast Bosnia (\textit{Slavonski} dialect) and in Novopazarsko-sjenički idioms (\textit{Zetsko-južnosandžački} dialect) (cf. \citealt[269]{Peco85}, \citealt[90]{Barjaktarevic66}). 

\subsection{Stress on clitics in BCS dialects}

It is a well-known fact that CLs behave differently in Kajkavian and Čakavian dialects. Therefore, it should not come as a surprise that both pronominal and verbal CLs can be stressed there, which is a consequence of the general rule of moving stress to the penultimate syllable of the stress unit. An example of this feature from the Kajkavian local idiom of Virje is presented in (\ref{(7.20)}).

\protectedex{\begin{exe}\ex\label{(7.20)}
\gll samo  \textbf{sȅm}  \textbf{se}  obudȋla \\
just be.1\textsc{sg}  \textsc{refl}  wake.up.\textsc{ptcp.sg.f} \\
\glt ‘I just woke up’
\hfill  (Podravski; \citealt[463]{Maresic11})
\end{exe}
}

\noindent However, something similar is present in Štokavian dialects as well. For instance, in the Neo-Štokavian local idiom of Bitelić (\textit{Zapadni} dialect), the CL for the third person plural accusative can be a long syllable \citep[cf.][186]{Curkovic14}, as may be seen in example (\ref{(7.21)}). 

\protectedex{\begin{exe}\ex\label{(7.21)}
\gll ôn  \textbf{ī}  \textbf{je}  pítā  \\
he them.\textsc{acc}  be.3\textsc{sg}  ask.\textsc{ptcp.sg.m} \\
\glt ‘he asked them’
\hfill  (Zapadni; \citealt[186]{Curkovic14})
\end{exe}
}

\noindent Moreover, in the Middle Štokavian \textit{Svrljiško-zaplanjski} Torlac dialect the stress can be placed on any syllable in a word, i.e. it can also be placed on the CL \citep[254]{Okuka08}. This trait differentiates the \textit{Svrljiško-zaplanjski} dialect from its neighbouring \textit{Timočko-lužnički} Torlac dialect \citep[cf.][257]{Okuka08}.

\section{Internal organisation of the clitic cluster}
\label{Internal organisation of the clitic cluster:8}
\subsection{Clitic ordering within the cluster}
\label{Clitic ordering within the cluster:8}
In many dialects the order of CLs in the cluster can differ from the order in BCS standard varieties.\footnote{For CL order in the cluster in BCS standard varieties see Section \ref{Clitic ordering within the cluster}.} The most common difference concerns the order of the reflexive CL \textit{se} and verbal CL \textit{je}, and is attested in both Old and Neo-Štokavian dialects.\footnote{As we already pointed out in Section \ref{Haplology of unlikes} although the CL sequence \textit{se je} is (hypothetically) possible in BCS standard varieties, in contrast to the CL sequence \textit{je se} which is not possible there, it is discussed rather controversially by normativists. Specifically, some grammarians recommend deletion of the verbal CL \textit{je}, i.e. haplology of unlikes. However, as we show in the next section, haplology is not restricted only to standard varieties. It also occurs in varieties which are not under the direct influence of language norms, i.e. in dialects. When it occurs, the process also solves the problem of reversed order.} With respect to the former, \citet[150]{Brozovic07} reports the reversed \textit{je} \textit{se} order for the local idiom spoken in the Fojnica valley (\ref{(7.22)}) and \citet[46]{Kolenic99}, for the local idiom of Ilača (\ref{(7.23)}).\footnote{This is in accordance with Baotić's (\citeyear[371]{Baotic85}) observations on the construction in question in Northern Bosnia. It is attested by him in local idioms of the \textit{Slavonski} and \textit{Srednjobosanski} dialects.}

\begin{exe}
\ex\label{(7.22)}
\gll òna  \textbf{je}  \textbf{se}  obúkla  \\
she be.\textsc{3sg}  \textsc{refl}  dress.\textsc{ptcp.sg.f} \\
\glt ‘she dressed herself’
\hfill  (Srednjobosanski; \citealt[150]{Brozovic07})
\ex\label{(7.23)}
\gll ônda  \textbf{je}  \textbf{se}  glȅdalo  \\
then be.\textsc{3sg}  \textsc{refl}  watch.\textsc{ptcp.sg.n} \\
\glt ‘then it was watched’
\hfill  (Slavonski; \citealt[46]{Kolenic99})
\end{exe}

\noindent Examples of \textit{je} \textit{se} CL order are found in the Neo-Štokavian \textit{Šumadijsko-voj\-vo\-đan\-ski} dialect, which is a neighbouring dialect of the Old Štokavian \textit{Slavonski} dialect (cf. \citealt[279]{Nikolic66}, \citealt[136]{Okuka08}). The example in (\ref{(7.113)}) is from the local idiom of Petnica. 

\begin{exe}\ex\label{(7.113)}
\gll Majka  \textbf{mi}  \textbf{je}  \textbf{se}  zvala  \\
mother me.\textsc{dat}  be.3\textsc{sg}  \textsc{refl}  call.\textsc{ptcp.sg.f} \\
\glt ‘My mother was called’
\hfill  (Šumadijsko-vojvođanski; \citealt[136]{Okuka08})
\end{exe}

\noindent \citet[58]{Lisac03}, \citet[33]{Halilovic05}, and \citet[309]{Curkovic14} provide examples of the reversed \textit{je} \textit{se} order in the \textit{Zapadni} dialect. \citet[58]{Lisac03} even claims that the verbal CL \textit{je} consistently appears before the reflexive CL \textit{se} in the \textit{Zapadni} dialect – see example from the local idiom of Derventa in (\ref{(7.24)}).

\protectedex{\begin{exe}\ex\label{(7.24)}
\gll Àntūn  \textbf{je}  \textbf{se}  čúvō  ùvīk. \\
Antun be.3\textsc{sg}  \textsc{refl}  guard.\textsc{ptcp.sg.m} always \\
\glt ‘Antun always guarded himself.’
\hfill  (Zapadni; \citealt[58]{Lisac03})
\end{exe}
}

\noindent This kind of reversed CL order can also be found in the speech of younger generations. \citet[165]{Aladrovic11} mentions it as a dialectal feature in the language of elementary school students from Požega. Furthermore, the reversed \textit{je se} CL cluster can be found in Štokavian idioms which are spoken in the territory of other dominant languages. Ivić’s example (\ref{(7.25)}) comes from the idiom of Galipolje Serbs.

\protectedex{\begin{exe}\ex\label{(7.25)}
\gll Ako  \textbf{je}  \textbf{se}  {uženȉla [\dots]} \\
if be.3\textsc{sg}  \textsc{refl}  marry.\textsc{ptcp.sg.f} \\
\glt ‘If she got married [\dots]’
\hfill  (Galipolje idiom; \citealt[395]{Ivic57} )
\end{exe}
}

\noindent \citet[395]{Ivic57} claims that this kind of reversed order is far more common in BCS dialects than the one presented below in examples (\ref{(7.33)})--(\ref{(7.36)}), where the verbal CL \textit{je} precedes pronominal CLs. In Ivić’s opinion the \textit{je se} order developed as a consequence of haplology of unlikes, which first resulted in \textit{se je} > \textit{sē} \citep[395]{Ivic57}. Afterwards, the verbal CL \textit{je} was restored in front of the reflexive CL \textit{se} by analogy with other verbal CLs \citep[cf.][395]{Ivic57}. 

However, the divergent position of the CL \textit{se} in the CL cluster is not always connected to its position relative to the verbal CL \textit{je}. \citet[91]{Okuka08} reports cases of \textit{se li} (\ref{(7.26)}) and \textit{se ga} (\ref{(7.27)}) cluster strings in the Neo-Štokavian \textit{Lički} subdialect.

\begin{exe}\ex\label{(7.26)}
\gll oće  \textbf{se}  \textbf{li}  nȅđe  mùći  potòpiti  \\
\textsc{fut}.\textsc{3sg}  \textsc{refl}  \textsc{q}  somewhere can.\textsc{inf}  flood.\textsc{inf}  \\
\glt ‘will flooding be possible somewhere’ \\
\hfill  (Istočnohercegovački; \citealt[91]{Okuka08} )
\ex\label{(7.27)}
\gll jȃ  \textbf{se}  \textbf{ga}  sjȅćām  \\
I \textsc{refl}  him.\textsc{gen}  remember.1\textsc{prs}  \\
\glt ‘I remember him’
\hfill  (Istočnohercegovački; \citealt[91]{Okuka08} )
\end{exe}

\noindent The \textsc{refl\textsubscript{lex}} \textit{se} can precede a genitive pronominal CL not only in the \textit{Istočnohercegovački}, but also in the \textit{Šumadijsko-vojvođanski} dialect \citep[cf.][280]{Nikolic66}. The order presented in (\ref{(7.27)}) can be found not only in Štokavian, but also in Čakavian dialects. In \citet[41]{Lisac09} we found a similar example from the \textit{Buzetski} Čakavian dialect (\ref{(7.28)}) with an accusative CL after \textsc{refl\textsubscript{lex}} \textit{se}.

\protectedex{\begin{exe}\ex\label{(7.28)}
\gll dok  \textbf{sɛ}  \textbf{γa}  nɛ  pupijɛ  \\
until \textsc{refl}  him.\textsc{gen}  \textsc{neg} drink.up.2\textsc{prs}  \\
\glt ‘until it is drunk up’
\hfill  (Buzetski; \citealt[41]{Lisac09})
\end{exe}
}

\noindent Reversed order of the verbal CL \textit{bi} and a pronominal CL in the cluster is attested in the Old Štokavian local idiom of Crmnica (\ref{(7.29)}) and in the Neo-Štokavian local idiom of Bitelić (\ref{(7.30)}) (cf. \citealt[180]{Okuka08}, \citealt[309]{Curkovic14}).\footnote{The original version in \citet{Okuka08} is: “\textit{Ne-bik-ti ja to učinijo, pa-da-mi-bi-da iljadu dinara}”.}


\begin{exe}\ex\label{(7.29)}
\gll Ne bik \textbf{ti} ja  to učinijo, pa da \textbf{mi} \textbf{bi} da  iljadu  dinara. \\
\textsc{neg} \textsc{cond}.1\textsc{sg} you I that do.\textsc{ptcp.sg.m} well that me.\textsc{dat}  \textsc{cond} give.\textsc{ptcp.sg.m} thousand dinars \\
\glt ‘I wouldn’t do that, even if you gave me a thousand dinars.’ \\
\hfill  (Zetsko-južnosandžački; \citealt[180]{Okuka08})

\ex\label{(7.30)}
\gll Ȍnda  \textbf{ti}  \textbf{bi}  dȍli  {dòšlo [\dots].} \\
then you.\textsc{dat} \textsc{cond} down come.\textsc{ptcp.sg.n} \\
\glt ‘Then (they) would come down [\dots].’
\hfill  (Zapadni; \citealt[309]{Curkovic14})
\end{exe}

\noindent The example from the local idiom of Bitelić in (\ref{(7.31)}) indicates that in the \textit{Zapadni} dialect CL order in this kind of cluster does not always differ from the order in standard BCS varieties. Therefore, we conclude that the reversed order is just a possibility and not a rule in these dialects.

\protectedex{\begin{exe}\ex\label{(7.31)}
\gll Stȃrī  \textbf{bi}  \textbf{ti}  ljûdi  {ȕvečē [\dots].} \\
old \textsc{cond}  you.\textsc{dat} people in.the.evening \\
\glt ‘Old people would in the evening [\dots].’
\hfill  (Zapadni; \citealt[309]{Curkovic14})
\end{exe}
}

\noindent \citet[209]{Pesikan65} also reports reversed order of \textsc{refl\textsubscript{lex}} and conditional auxiliary CLs in the local idiom of Rijeka (\ref{(7.32)}).\footnote{He suggests that from the diachronic perspective cases similar to (\ref{(7.29)}) and (\ref{(7.32)}) signal that the conditional auxiliary CL is younger than the pronominal and reflexive CLs. The assumption that the order of CLs in the cluster was influenced by their relative age can be found in diachronic literature (e.g. \citealt[95]{Grickat72}, \citealt[189]{ZimmerlingKosta13}) and seems plausible. The relative order of the reflexive CL \textit{se} and the conditional auxiliary CLs attested in (\ref{(7.32)}) was already present in OCS. However, in texts written in the Croatian redaction of Church Slavonic, i.e. the younger variety, the reflexive CL \textit{se} follows the conditional auxiliary CLs \citep[cf.][318]{GKMPRSV14}.}

\protectedex{\begin{exe}\ex\label{(7.32)}
\gll ȍna  \textbf{se}  \textbf{bi}  prepȁla  \\
she \textsc{refl}  \textsc{cond}  get.scared.\textsc{ptcp.sg.f} \\
\glt ‘she would get scared’
\hfill  (Zetsko-južnosandžački; \citealt[209]{Pesikan65})
\end{exe}
}

\noindent The verbal CL \textit{je} can appear not only before reflexive CLs (see examples (\ref{(7.23)})--(\ref{(7.25)}) presented above), but also in front of pronominal CLs (cf. \citealt[394]{Ivic57}, \citealt[210]{Pesikan65}, \citealt[279]{Nikolic66}, \citealt[256]{Okuka08}).\footnote{This kind of reversed CL order in which pronominal CLs are preceded by the verbal CL \textit{je} is also attested in colloquial Serbian and in \{bs, hr, sr\}WaC corpora. For more information see Section \ref{Clitic ordering within the cluster in BCS standard varieties}.} Examples (\ref{(7.33)}) and (\ref{(7.34)}) below are from the local idioms of Pričinović and Zagrač, while (\ref{(7.35)}) is from the \textit{Svrljiško-zaplanjski} dialect. 

\begin{exe}\ex\label{(7.33)}
\gll komèsija  \textbf{je}  \textbf{me}  pítala  \\
commission be.\textsc{sg}  me.\textsc{acc}  ask.\textsc{ptcp.sg.f} \\
\glt ‘the commission asked me’
\hfill  (Šumadijsko-vojvođanski; \citealt[279]{Nikolic66})

\ex\label{(7.34)}
\gll ȍn  \textbf{je}  \textbf{mu}  rȅkā  \\
he be.3\textsc{sg}  him.\textsc{dat}  tell.\textsc{ptcp.sg.m} \\
\glt ‘he told him’
\hfill  (Zetsko-južnosandžački; \citealt[210]{Pesikan65})

\ex\label{(7.35)}
\gll kuj  \textbf{e}  \textbf{ga}  volel  \\
who be.3\textsc{sg}  him.\textsc{acc}  love.\textsc{ptcp.sg.m}\\
\glt ‘who loved him’
\hfill  (Svrljiško-zaplanjski dialect; \citealt[256]{Okuka08})

\ex\label{(7.36)}
\gll Kȍ  \textbf{je}  \textbf{mu}  \textbf{se}  poklonȉo. \\
who be.3\textsc{sg}  him.\textsc{dat} \textsc{refl}  bow.\textsc{ptcp.sg.m} \\
\glt ‘Who bowed to him.’
\hfill  (Galipolje idiom; \citealt[394]{Ivic57})
\end{exe}

\noindent \citet[152]{Stevanovic50} claims that pronominal CLs can stand in front of verbal CLs (the present tense of \textit{biti} ‘be’) in the local idiom of Đakovica:

\protectedex{\begin{exe}\ex\label{(7.37)}
\gll Dvá  \textbf{mu}  \textbf{su}  sína  žȅneta. \\
two him.\textsc{dat} be.3\textsc{pl}  sons married \\
\glt  ‘Two of his sons are married.’ \\
\hfill  (Prizrensko-južnomoravski; \citealt[152]{Stevanovic50})
\end{exe}
}

\noindent The CLs in example (\ref{(7.38)}) from the local idiom of Vitina found in \citet[345]{Peco07b} are ordered according to cluster ordering rules of the standard varieties, with one exception – there are two dative CLs in one cluster. 

\protectedex{\begin{exe}\ex\label{(7.38)}
\gll Vèlik  \textbf{ti}  \textbf{mu}  \textbf{je}  národ  {dohòdijo, [\dots].}  \\
big you.\textsc{dat}  him.\textsc{dat}  be.3\textsc{sg}  people come.\textsc{ptcp.sg.m} \\
\glt ‘A great mass of people came to him, you know [\dots].’ \\
\strut\hfill  (Istočnohercegovački; \citealt[345]{Peco07b})
\end{exe}
}

\noindent According to the cluster ordering rules in standard BCS varieties there is only one slot for dative CLs. In his theoretical work on CLs \citet[62]{Boskovic01} mentions the possibility of two dative CLs in one cluster and claims that when both the ethical and argumental dative are present in a sentence, the former must precede the latter. Peco’s example from the Neo-Štokavian \textit{Istočnohercegovački} dialect seems to nicely support Bošković’s theory.

\subsection{Morphonological processes within the cluster}
\label{Morphological processes within the cluster}
\subsubsection{Suppletion}

We discuss suppletion in standard BCS varieties in Section \ref{Morphonological processes within the cluster} and in Section \ref{Suppletion}. This phenomenon seems to be restricted only to standard BCS varieties. Specifically, in the dialectological literature and \textcolor{black}{revised} transcripts the CL cluster \textit{ju je} is attested only in dialects in which \textit{ju} is the default and only accusative form for the third person singular feminine – see example (\ref{(7.39)}).\footnote{Moreover, our personal communication with dialectologists resulted in the same conclusion. A search of their transcripts for examples of suppletion had no positive results. \textcolor{black}{However, more robust data on this matter is needed here.}}

\protectedex{\begin{exe}\ex\label{(7.39)}
\gll [\dots] kə\textsuperscript{a}(d) \textbf{ju}  \textbf{e}  vȉdio  da  idê  k  {ńȅmu [\dots].} \\
{} when her.\textsc{acc}  be.3\textsc{sg} see.\textsc{ptcp.sg.m} that  go.3\textsc{prs} to him.\textsc{dat}  \\
\glt ‘[\dots] when he saw her coming towards him [\dots].’ \\
\strut\hfill  (Zetsko-južnosandžački; \citealt[188]{Okuka08})
\end{exe}
}

\noindent Since speakers of \textit{Zetsko-južnosandžački} employ the CL \textit{ju} as the default form, even in sentences without the verbal CL \textit{je}, (\ref{(7.39)}) cannot be considered an example of suppletion. 

\subsubsection{Haplology}
\label{Haplology:8}
Unlike standard BCS varieties, which in this particular case resolve the problem of repeated morphs with suppletion, some dialects use haplology (see Section \ref{Morphonological processes within the cluster}). In the example from the local idiom of Bitelić (\textit{Zapadni} dialect) presented in (\ref{(7.40)}), instead of two \textit{je} CLs, the verbal ‘is’ and pronominal ‘her’, there is only one. 

\protectedex{\begin{exe}\ex\label{(7.40)}
\gll ôn  \textbf{je}  pítā  \\
he her.\textsc{acc}/be.3\textsc{sg}  ask.\textsc{ptcp.sg.m} \\
\glt ‘he asked her’
\hfill  (Zapadni; \citealt[185]{Curkovic14})
\end{exe}
}

\noindent Furthermore, some dialects allow repetition of \textit{je} CLs. This is found in the local idioms of Imotski (\textit{Zapadni} dialect), Tuholj (Srednjobosanski dialect) and Pag (\textit{Srednjočakavski} Čakavian dialect) (cf. \citealt[120]{Simundic71}, \citealt[322]{Halilovic90}, \citealt[165]{Vranic03}):\footnote{\citet[109]{Golic93} also provides an example of a \textit{je} \textit{je} sequence from the local idiom of Donji Miholjac (\textit{Slavonski} dialect), but in her example the pronominal CL \textit{je} is the old form for the third person plural genitive/accusative. She states that this form is only preserved in the language of the older generation.}  

\begin{exe}\ex\label{(7.41)}
\gll dà  \textbf{jē}  \textbf{je}  vȉdio, zóvnijo  \textbf{bi}  \textbf{nās}  \\
that  her.\textsc{acc}  be.3\textsc{sg}  see.\textsc{ptcp.sg.m} call.\textsc{ptcp.sg.m} \textsc{cond}  us.\textsc{acc} \\
\glt ‘if he had seen her, he would have called us’ \\
\hfill  (Zapadni dialect; \citealt[120]{Simundic71})

\ex\label{(7.42)}
\gll Ȏn  \textbf{in}  \textbf{je}  \textbf{je}  ukrâl  \\
he them.\textsc{dat}  her.\textsc{acc} be.3\textsc{sg}  steal.\textsc{ptcp.sg.m} \\
\glt ‘He stole her from them’
\hfill  (Srednjočakavski; \citealt[165]{Vranic03})
\end{exe}

\subsubsection{Haplology of unlikes}

As described in Section \ref{Clitic ordering within the cluster:8}, in many idioms the reflexive CL \textit{se} and the verbal CL \textit{je} ‘is’ appear in an order which diverges from the one found in standard BCS varieties. In contrast, some idioms such as the local idiom of Nevesinje adopt haplology of unlikes, i.e. the solution used in BCS standard varieties – see example in (\ref{(7.43)}).\footnote{For an explanation of the haplology of unlikes, see Section \ref{Morphonological processes within the cluster}.}

\protectedex{\begin{exe}\ex\label{(7.43)}
\gll Kad  \textbf{se}  oslobòdila  {kmetàrija [\dots].} \\
when \textsc{refl}  free.\textsc{ptcp.sg.f} serfs \\
\glt ‘When the serfs freed themselves [\dots].’ \\
\hfill  (Istočnohercegovački; \citealt[281]{Peco07a})
\end{exe}
}

\noindent However, as examples (\ref{(7.22)})--(\ref{(7.25)}) reveal, haplology of unlikes does not always occur. Moreover, even idioms which belong to the same dialect sometimes differ with respect to this phenomenon: compare examples (\ref{(7.43)}) and (\ref{(7.44)}). In contrast to examples (\ref{(7.22)})--(\ref{(7.25)}), CLs in example (\ref{(7.44)}) are ordered just like they would be in BCS standard varieties. This is attested in the \textit{Istočnobosanski} subdialect.

\protectedex{\begin{exe}\ex\label{(7.44)}
\gll kad  \textbf{se}  \textbf{je}  zàratilo \\
when \textsc{refl}  be.3\textsc{sg} start.war.\textsc{ptcp.sg.n} \\
\glt ‘when the war started’
\hfill  (Istočnohercegovački; \citealt[77]{Okuka08})
\end{exe}
}

\section{Position of the clitic or the clitic cluster}
\label{Position of the clitic or clitic cluster:8}
\subsection{Second position}
\label{Second position:8}
CLs can definitely follow phrases with two content words in Neo-Štokavian dialects. As we pointed out in Section \ref{Second position, second word and delayed placement}. Second position, second word and DP, this is also the case in standard Bosnian and Serbian, while the Croatian norm recommends phrase splitting or DP. While browsing dialectological literature we find examples such as those from the local idioms of Studenci (\ref{(7.47)}) and Zvirići (\ref{(7.48)}). However, bear in mind that 2P is not the only option in the \textit{Istočnohercegovački} dialect: phrase splitting is also possible (see examples (\ref{(7.72)}), (\ref{(7.73)}) and (\ref{(7.74)}) below). 

\begin{exe}\ex\label{(7.47)}
\gll Čȉtavo  pȍlje  \textbf{se}  mȍre  tòpit  \\
entire field \textsc{refl}  can.3\textsc{prs}  flood.\textsc{inf}  \\
\glt ‘The entire field can be flooded’
\hfill  (Istočnohercegovački; \citealt[341]{Peco07b})

\ex\label{(7.48)}
\gll Mȏj  dȇdo  \textbf{bi}  {rȅkā: [\dots].} \\
my grandfather \textsc{cond} say.\textsc{ptcp.sg.m} \\
\glt ‘My grandfather would say: [\dots].’ 
\hfill  (Istočnohercegovački; \citealt[342]{Peco07b})
\end{exe}

\noindent The typical position of CLs in the local idiom of Sarajevo does not involve phrase splitting, as shown in example (\ref{(7.49)}) \citep[62]{HTS09}.

\protectedex{\begin{exe}\ex\label{(7.49)}
\gll ĺȇvā  cìpela  \textbf{mi}  \textbf{je}  malèhna  \\
left shoe me.\textsc{dat}  be.3\textsc{sg}  small \\
\glt ‘my left shoe is small’
\hfill  (Srednjobosanski; \citealt[62]{HTS09})
\end{exe}
}

\subsection{Delayed placement of clitics}
\label{Delayed placement of clitics}
According to \citet[273]{Raguz16} the Old Štokavian local idiom of the village Bogdanovci does not display the tendency common in standard Croatian to place the CL after the first stressed word. In (\ref{(7.50)}) and (\ref{(7.51)}) there are no barriers which would prevent the reflexive CL \textit{se} from taking 2P. Similar examples are also found in the local idiom of Bizovac \citep[cf.][144]{Klaic07}. Both of these idioms belong to the \textit{Slavonski} dialect. 

\begin{exe}\ex\label{(7.50)}
\gll Ti nȅ  mōraš \textbf{se}  ùdat.  \\
you \textsc{neg} have.to.2\textsc{prs}  \textsc{refl}  marry.\textsc{inf}  \\
\glt ‘You do not have to get married.’
\hfill  (Slavonski; \citealt[273]{Raguz16})

\ex\label{(7.51)}
\gll A za svȁtove nísu  pògače  \textbf{se}  pèkle. \\
and for wedding \textsc{neg}.be.\textsc{3pl}  breads \textsc{refl}  bake.\textsc{ptcp.pl.f} \\
\glt ‘And for the wedding, breads were not baked.’ \\
\hfill (Slavonski; \citealt[273]{Raguz16})
\end{exe}

\noindent Examples with DP can be found in other idioms of the \textit{Slavonski} dialect, such as those spoken in South Baranja. However, if initial constituents are heavy, DP is not the only possible position for CLs in this dialect. Namely, in idioms of South Baranja and in the local idiom of Našice CLs can also follow heavy constituents (cf. \citealt[412f]{Sekeres77}, \citealt[264]{Sekeres66}).

\citet[165]{Aladrovic11} reports DP as a dialectal trait in the written language of elementary school students from Požega (\ref{(7.52)}). We can thus conclude that DP as a dialectal feature is not present only in the language of the older population. 

\protectedex{\begin{exe}\ex\label{(7.52)}
\gll ja  želim  \textbf{se}  vratiti  \\
I want.\textsc{1prs}  \textsc{refl}  return.\textsc{inf}  \\
\glt ‘I want to return’
\hfill  (Istočnohercegovački; \citealt[165]{Aladrovic11})
\end{exe}
}

\noindent \citet[279]{Nikolic66} and \citet[136]{Okuka08} also report divergence in CL positioning in some idioms of the Neo-Štokavian \textit{Šumadijsko-vojvođanski} dialect. Instead of moving towards the 2P, CLs move towards the end of the sentence, as in the examples from the local idioms of Kolubara (\ref{(7.53)}) and Banat (\ref{(7.54)}). CLs in the Banat area might be influenced by the neighbouring Romanian language. Nevertheless, the example from the Kolubara area shows that peculiarities of CL positioning cannot be ascribed exclusively to the influence of other languages, as the area is placed in the middle of the \textit{Šumadijsko-vojvođanski} dialect area (in central Serbia).

\begin{exe}\ex\label{(7.53)}
\gll od  one  kože prave opanke \textbf{nam}  \\
from that leather make.\textsc{3prs}  shoes us.\textsc{dat}  \\
\glt  ‘from that leather, they make shoes for us’ \\
\hfill  (Šumadijsko-vojvođanski; \citealt[136]{Okuka08})

\ex\label{(7.54)}
\gll Šta vi sad \textbf{se} oblačite? \\
what you  now \textsc{refl}  get.dressed.2\textsc{prs}  \\
\glt ‘Why are you getting dressed now?’ \\
\hfill (Šumadijsko-vojvođanski; \citealt[136]{Okuka08})
\end{exe}

\noindent Although \citet[279]{Nikolic66} claims that CLs in the \textit{Šumadijsko-vojvođanski} dialect often split semantically tightly bound phrases (see examples (\ref{(7.77)}) and (\ref{(7.80)}) in Section \ref{Phrase splitting}), examples with DP are easily found in his material, such as (\ref{(7.59)}) from the local idiom of Pričinović.

\protectedex{\begin{exe}\ex\label{(7.59)}
\gll a  tȃj  domàćin  narédio  \textbf{im}  \\
and that host order.\textsc{ptcp.sg.m} them.\textsc{dat}  \\
\glt ‘and that host ordered them’
\hfill  (Šumadijsko-vojvođanski; \citealt[280]{Nikolic66})
\end{exe}
}

\noindent As we can see, the phenomenon of DP occurs in various dialects. In Okuka's (\citeyear[77]{Okuka08}) words: in the \textit{Istočnobosanski} subdialect CLs shift backwards. The examples of DP presented below are from the \textit{Istočnobosanski} subdialect (\ref{(7.55)}) and from the local idiom of Borač (\ref{(7.56)}). 

\begin{exe}\ex\label{(7.55)}
\gll sva ogubala \textbf{se} \\
entire become.leprous.\textsc{ptcp.sg.f} \textsc{refl}  \\
\glt ‘she became completely covered in warts ’ \\
\hfill  (Istočnohercegovački; \citealt[77]{Okuka08})

\ex\label{(7.56)}
\gll Prȉje trídes  gȍdīnā Àhmet Cȋk \textbf{je}  kȍsio  nà  tāj  dan.\\
before thirty years Ahmet Cik be.3\textsc{sg}  scythe.\textsc{ptcp.sg.f} on that day\\
\glt ‘Thirty years ago on that day Ahmet Cik was mowing grass.’ \\
\hfill  (Istočnohercegovački; \citealt[285]{Peco07a})
\end{exe}

\noindent In the idiom of Galipolje Serbs spoken in the Macedonian city Pehčevo, CLs have to follow the negative present tense form of \textit{biti}, and therefore examples of DP such as the one in (\ref{(7.60)}) below can occur.

\protectedex{\begin{exe}\ex\label{(7.60)}
\gll Takȍ  ȏn  nījȅ  \textbf{je}  poslȕšavo. \\
so he \textsc{neg}.be.3\textsc{sg}  her.\textsc{acc}  listen.\textsc{ptcp.sg.m} \\
\glt ‘So he did not listen to her.’
\hfill  (Galipolje idiom; \citealt[395]{Ivic57})
\end{exe}
}

\noindent Many examples with DP concur with the use of \textit{da} particle, such as those from the local idiom of Pričinović presented in (\ref{(7.57)}) and (\ref{(7.58)}).\footnote{In this respect dialects definitely diverge from standard Bosnian and Serbian, in which CLs must follow any complementisers. Moreover, those examples speak strongly against claims that the only possible and correct position of CLs is directly after the \textit{da} particle and are in accordance with the results of our study on CC out of \textit{da}\textsubscript{2}-complements. For more information see Sections \ref{Placement with regard to different types of hosts in BCS standard varieties} and \ref{Results:da}.} 

\begin{exe}\ex\label{(7.57)}
\gll ȍće  d’ ȕbije  \textbf{me}  \\
want.3\textsc{prs}  that  kill.3\textsc{prs} me.\textsc{acc} \\
\glt ‘he wants to kill me’
\hfill  (Šumadijsko-vojvođanski; \citealt[280]{Nikolic66})

\ex\label{(7.58)}
\gll nȇće  da  donèsē  \textbf{ga}  \\
\textsc{neg}.want.3\textsc{prs} that  bring.3\textsc{prs} him.\textsc{acc} \\
\glt ‘he does not want to bring him’ \\
\hfill  (Šumadijsko-vojvođanski; \citealt[280]{Nikolic66})
\end{exe}

\subsection{Phrase splitting}
\label{Phrase splitting}
Phrase splitting is attested in both Old and Neo-Štokavian dialects, and even in some Kajkavian dialects. In most cases, verbal CLs split attributes from their head nouns, but there are some examples in which dative pronominal CLs cause phrase splitting.\footnote{In this respect there are not many differences between BCS standard varieties and dialects. For more information, see Section \ref{The limits of phrase splitting in BCS standard varieties}.}

Examples in which the attribute and its noun are split can be found in the Old Štokavian \textit{Slavonski} dialect (cf. \citealt[412f]{Sekeres77}, \citealt[103]{Golic93}) and in the Neo-Štokavian \textit{Istočnohercegovački} dialect \citep[cf.][283]{Peco07a}. For the latter, see the example from the local idiom of Borač presented in (\ref{(7.72)}).\footnote{This kind of split phrase is found to be acceptable in both standard Croatian and standard Serbian. For more information, see Section \ref{The limits of phrase splitting in BCS standard varieties}.} In this idiom we also find examples in which adverbs are split from a noun in the genitive (\ref{(7.73)}).\footnote{Examples in which a noun in the genitive case is split from its head are controversially discussed in the theoretical syntactic literature, see Section \ref{The limits of phrase splitting in BCS standard varieties}.} Note that we find similar examples (\ref{(7.37)}) with a split quantified phrase (\textit{dva sina} ‘two sons’) in the Middle Štokavian \textit{Prizrensko-južnomoravski} dialect.

\begin{exe}\ex\label{(7.72)}
\gll Ȍvā  \textbf{je}  država  plamírala  da  prȁvī  škȏrlu.   \\
this be.3\textsc{sg} country plan.\textsc{ptcp.sg.f} that  make.3\textsc{prs}  school.\textsc{acc}  \\
\glt ‘This country planned to build a school.’ \\
\hfill  (Istočnohercegovački; \citealt[283]{Peco07a})

\ex\label{(7.73)}
\gll Vȉše  \textbf{smo}  hȁjra  {vȉđeli [\dots].} \\
more be.1\textsc{pl}  benefit see.\textsc{ptcp.pl.m} \\
\glt ‘We saw more benefit [\dots].’
\hfill  (Istočnohercegovački; \citealt[283]{Peco07a})
\end{exe}

\noindent Furthermore, in the \textit{Istočnohercegovački} dialect CLs can be inserted between parts of compound pronouns like \textit{tko god} (\ref{(7.74)}).\footnote{Although this kind of split phrase is found to be acceptable in both standard Croatian and standard Serbian, it is considered to be quite uncommon in both – see Section \ref{The limits of phrase splitting in BCS standard varieties}. Moreover, it is attested in the corpus Bosnian Interviews.}

\protectedex{\begin{exe}\ex\label{(7.74)}
\gll kȍ  \textbf{se}  god  bòjī, slȁbo  će  próći  \\
who \textsc{refl}  ever fear.3\textsc{prs}  poorly \textsc{fut}.3\textsc{sg} pass.\textsc{inf}  \\
\glt ‘whoever is afraid will fare poorly’ \\
\hfill  (Istočnohercegovački; \citealt[338]{Sekeres77})
\end{exe}
}

\noindent In contrast, in idioms of Baranja (\textit{Slavonski} dialect) and of Bačka (\textit{Zapadni} dialect) CLs normally follow compound pronouns \citep[cf.][340, 413]{Sekeres77}.

In the local idiom spoken in the Neretva valley CLs can split prepositional phrases, like in example (\ref{(7.75)}).\footnote{This phenomenon is rather controversial in the theoretical literature on CL placement, see Section \ref{The limits of phrase splitting in BCS standard varieties}.}  

\protectedex{\begin{exe}\ex\label{(7.75)}
\gll [\dots] slȁđā \textbf{je}  od  šèćera  bíla  \\
{} sweeter be.3\textsc{sg}  than sugar be.\textsc{ptcp.sg.f} \\
\glt ‘[\dots] she was sweeter than sugar’ \\
\hfill  (Istočnohercegovački; \citealt[188]{VuksaNahod14})
\end{exe}
}

\noindent As we pointed out in Section \ref{The limits of phrase splitting in BCS standard varieties} splitting a forename from a last name is not recommended in standard Serbian. However, in dialects CLs can split the tightly bound forename from the surname: see example (\ref{(7.76)}) from the local idiom of Kolašin and (\ref{(7.77)}) from the local idiom of Banovo Polje (cf. \citealt[67]{Okuka08}, \citealt[279]{Nikolic66}). Moreover, we would like to emphasise that the latter example is attested in a dialect spoken on Serbian territory. In the previous century \citet[209]{Pesikan65} reported the same for \textit{Starocrnogorski} idioms.\footnote{According to dialectological maps some \textit{Starocrnogorski} idioms are part of Neo-Štokavian \textit{Istočnohercegovačk}i, and others of the Old Štokavian \textit{Zetsko-južnosandžački} dialect.}

\begin{exe}\ex\label{(7.76)}
\gll Vuk  \textbf{mi}  \textbf{je}  Šćepanović  reko. \\
Vuk me.\textsc{dat}  be.3\textsc{sg}  Šćepanović  say.\textsc{ptcp.sg.m} \\
\glt ‘Vuk Šćepanović told me.’
\hfill  (Istočnohercegovački; \citealt[67]{Okuka08})

\ex\label{(7.77)}
\gll Mìlan  \textbf{mi}  \textbf{je}  Pȅcić  bȉo  kao  ùpravnik  bólnice. \\
Milan me.\textsc{dat} be.3\textsc{sg} Pecić be.\textsc{ptcp.sg.m} like director hospital \\
\glt ‘Milan Pecić was like a hospital director to me.’ \\
\hfill  (Šumadijsko-vojvođanski; \citealt[279]{Nikolic66})
\end{exe}

\noindent Here we would like to point out that in both examples provided above, it is a CL cluster that splits the phrase, which theoretical syntacticians \citet{Progovac96} and \citet{RadanovicKocic96} strongly dislike (for more information, see Section \ref{The limits of phrase splitting in BCS standard varieties}).

In the \textit{Istočnohercegovački} and \textit{Šumadijsko-vojvođanski} dialects one more rather controversial structure is attested. Namely, \citet[67, 74]{Okuka08} and \citet[209]{Pesikan65} report cases in which verbal CLs split conjoined phrases: see examples (\ref{(7.78)})--(\ref{(7.80)}) presented below.\footnote{For controversial discussion on this structure from the theoretical point of view, see Section \ref{The limits of phrase splitting in BCS standard varieties}.}

\begin{exe}\ex\label{(7.78)}
\gll Ja  \textbf{smo}  i  on  zajedno  utekli  \\
I be.1\textsc{pl}  and  he together escape.\textsc{ptcp.pl.m} \\
\glt ‘He and I escaped together’
\hfill  (Istočnohercegovački; \citealt[67]{Okuka08})

\ex\label{(7.79)}
\gll Mi  \textbf{ćemo}  i  oni  to  zajedno  \\
we \textsc{fut}.\textsc{1sg}  and they that together \\
\glt ‘We and they are going (to do) that together’ \\
\hfill  (Istočnohercegovački; \citealt[67]{Okuka08})
\end{exe}

\noindent According to \citet[67]{Okuka08}, this construction is widespread in the \textit{Istočnohercegovački} dialect. This, however, is not the only Neo-Štokavian dialect in which the construction in question is attested. Example (\ref{(7.80)}) below is from the local idiom of Pričinović, i.e. the \textit{Šumadijsko-vojvođanski} dialect. 

\protectedex{\begin{exe}\ex\label{(7.80)}
\gll jȃ  \textbf{smo}  i  žèna  sámi  \\
I be.1\textsc{pl}  and woman alone \\
\glt ‘my wife and I are alone’
\hfill  (Šumadijsko-vojvođanski; \citealt[279]{Nikolic66})
\end{exe}
}

\noindent In the Neo-Štokavian \textit{Banatsko-pomoriški} subdialect, possessive CLs lean on attributes and split phrases (\ref{(7.81)}) \citep[cf.][148]{Okuka08}. 

\protectedex{\begin{exe}\ex\label{(7.81)}
\gll pòkōjni  \textbf{mi}  mûž  \\
deceased me.\textsc{dat}  husband \\
\glt ‘my deceased husband’
\hfill (Šumadijsko-vojvođanski; \citealt[148]{Okuka08})
\end{exe}
}

\noindent It seems that language contact does not negatively influence phrase splitting. In the idiom of Galipolje Serbs CL clusters can be inserted between the attribute and the noun (\ref{(7.110)}).\footnote{A few pages later, \citet[397]{Ivic57} even claims that in the Galipoljski idiom CLs usually follow the first member of the noun phrase.} As we can see from these two examples, phrase splitting is attested even in idioms which are in direct contact with other languages which have CLs: the \textit{Banatsko-pomoriški} subdialect is spoken on the Romanian border and the idiom of Galipolje Serbs is in direct contact with Macedonian. In addition, phrase splitting is attested even in the Old Štokavian local idiom of Vršenda, which is in direct language contact with Hungarian \citep[cf.][111f]{Gorjanac11}: see examples (\ref{(7.82)}) and (\ref{(7.83)}) below.

\begin{exe}\ex\label{(7.82)}
\gll Jedna  \textbf{je}  krava  bila  za  čitavu  družinu.  \\
one be.3\textsc{sg}  cow be.\textsc{ptcp.sg.f} for entire group \\
\glt ‘There was one cow for the entire group.’
\hfill  (Slavonski; \citealt[111]{Gorjanac11})

\ex\label{(7.83)}
\gll Naša  \textbf{je}  stara  imala  petero  {dice {[\dots].}} \\
our be.3\textsc{sg} old  have.\textsc{ptcp.sg.f} five children \\
\glt ‘Our mother had five children [\dots].’
\hfill  (Slavonski; \citealt[112]{Gorjanac11})
\end{exe}

\noindent As we already mentioned, phrase splitting is attested even in Kajkavian dialects, in which both auxiliary CLs (\ref{(7.84)}) and pronominal CLs (\ref{(7.85)}) can be placed between an attribute and its noun. The former example is from the local idiom of Turopolje, while the latter is from the \textit{Gornjolonjski} dialect.

\begin{exe}\ex\label{(7.84)}
\gll Mȍja  \textbf{je}  mȁma  {govorȉla [\dots].} \\
my be.3\textsc{sg} mother speak.\textsc{ptcp.sg.f} \\
\glt ‘My mother was saying [\dots].’
\hfill  (Turopoljski; \citealt[139]{Loncaric94})

\ex\label{(7.85)}
\gll ova  \textbf{mi}  peć  nigdar  neće  dobre  goreti.  \\
this me.\textsc{dat}  stove never \textsc{neg.fut.3sg} good burn.\textsc{inf}  \\
\glt ‘This stove of mine will never burn well.’ \\
\strut\hfill  (Gornjolonjski; \citealt[242]{BrlobasLoncaric12})
\end{exe}


\subsection{Clitic first (1P)}
\label{Clitic first}
As described in Section \ref{Position of the clitic or clitic cluster} CLs cannot have sentence-initial position in any of the BCS standard varieties, which are all based on the Neo-Štokavian \textit{Istočnohercegovački} dialect. Furthermore, in BCS standard varieties CLs cannot follow the conjunctions \textit{i} and \textit{a} ‘and’. However, in the dialectological literature we find examples from Neo-Štokavian dialects showing different behaviour:

\protectedex{\begin{exe}\ex\label{(7.61)}
\gll jâ  i  \textbf{smo}  tî  ȍsam  dánā  da  krȕva  nè  vidi  \\
I and be.1\textsc{pl}  you eight days that  bread \textsc{neg} see.3\textsc{prs}  \\
\glt ‘it happened that you and I did not see bread for eight days’ \\
\strut\hfill  (Istočnohercegovački; \citealt[91]{Okuka08} )
\end{exe}
}

\noindent In example (\ref{(7.61)}) recorded in the \textit{Lički} subdialect of the Neo-Štokavian \textit{Istočnohercegovački} dialect, the verbal CL \textit{smo} ‘are’ follows the conjunction \textit{i}.\footnote{Readers have to bear in mind that this is not an example of conjoined phrase splitting. A CL which splits a conjoined phrase follows the first element of that phrase. In this example the CL follows the unaccented coordinative conjunction \textit{i}.} We may speculate that this peculiar CL order could have been triggered by the neighbouring \textit{Srednjočakavski} dialect.\footnote{According to \citet[113]{Lisac09}, in the \textit{Srednjočakavski} subdialect CLs can take 1P in a sentence, as is usual for Čakavian dialects. Furthermore, CLs can even bear stress in this dialect \citep[cf.][]{Lisac09}.} Similar examples are found in the Neo-Štokavian \textit{Šumadijsko-vojvođanski} dialect: see (\ref{(7.62)}). According to \citet[136]{Okuka08} in this dialect enclitics can turn into proclitics, and often follow the conjunction \textit{i}.

\protectedex{\begin{exe}\ex\label{(7.62)}
\gll Došo  i  mi  pusti  \textbf{ga},  i  \textbf{su}  \textbf{ga}  strél̕ali.\\
come.\textsc{ptcp.sg.m} and we let.\textsc{inf}  him.\textsc{acc}  and be.3\textsc{pl}  him.\textsc{acc}  shoot.\textsc{ptcp.pl.m}\\
\glt ‘He came and we let him go, and they shot him.’ \\
\strut\hfill  (Šumadijsko-vojvođanski; \citealt[136]{Okuka08})
\end{exe}
}

\noindent We can assume that the trait in question is spreading from the eastern edge of \textit{Šumadijsko-vojvođanski} to other parts. The reason for such an assumption is Okuka's (\citeyear[148]{Okuka08}) report that the \textit{Banatsko-pomoriški} subdialect shows Romanian influences, one of which is proclitisation of CLs. He provides the following example (\ref{(7.63)}) with an auxiliary CL which takes absolute initial position in the sentence.

\protectedex{\begin{exe}\ex\label{(7.63)}
\gll \textbf{Su}  bíli  u  célo   sèlo. \\
be.3\textsc{pl} be.\textsc{ptcp.pl.m} in entire village \\
\glt ‘They were in the entire village.’ \\
\hfill  (Šumadijsko-vojvođanski; \citealt[148]{Okuka08})
\end{exe}
}

\noindent Further in the east, in the local idiom of Rekaš (\textit{Kosovsko-resavski} dialect) in Romania, CLs can take 1P: see example (\ref{(7.64)}). However, such behaviour is not the rule, since they can appear in 2P as well \citep[cf.][171]{Vulic09}.

\protectedex{\begin{exe}\ex\label{(7.64)}
\gll \textbf{se}  moli  Boga \\
\textsc{refl}  pray.3\textsc{prs}  God \\
\glt ‘he is praying to God’
\hfill  (Kosovsko-resavski; \citealt[171]{Vulic09})
\end{exe}
}

\noindent 1P is reported for verbal (\ref{(7.65)}), pronominal (\ref{(7.66)}) and reflexive CLs (\ref{(7.67)}) in the local idiom of Đakovica (\textit{Prizrensko-južnomoravski} dialect). \citet[152]{Stevanovic50} claims that dative pronominal CLs are more common in 1P than other pronominal CLs. 

\begin{exe}\ex\label{(7.65)}
\gll \textbf{Su}  bíle pét  brȁća. \\
be.3\textsc{pl} be.\textsc{ptcp.pl.f} five brothers \\
\glt ‘They were five brothers.’ \\
\hfill  (Prizrensko-južnomoravski; \citealt[152]{Stevanovic50})

\ex\label{(7.66)}
\gll \textbf{Mu}  víkam  mojému  strícu. \\
him.\textsc{dat}  yell.1\textsc{prs}  mine uncle \\
\glt ‘I am telling my uncle.’
\hfill  (Prizrensko-južnomoravski; \citealt[152]{Stevanovic50})

\ex\label{(7.67)}
\gll \textbf{Se}  zvȁo   táko.  \\
\textsc{refl}  call.\textsc{ptcp.sg.m} like.that \\
\glt ‘He was called that.’
\hfill  (Prizrensko-južnomoravski; \citealt[152]{Stevanovic50})
\end{exe}

\noindent Karaševo-Croats, who live in seven villages in the Romanian part of Banat and speak the \textit{Timočko-lužnički} Torlac dialect, place CLs in the proclitic position under the influence of Romanian \citep[cf.][147]{Lisac03} – see the example presented in (\ref{(7.68)}).

\protectedex{\begin{exe}\ex\label{(7.68)}
\gll \textbf{smo}  \textbf{sɛ}  nazdravili  \\
be.1\textsc{pl}  \textsc{refl}  toast.\textsc{ptcp.pl.m}\\
\glt ‘we made a toast’
\hfill  (Timočko-lužnički dialect; \citealt[147]{Lisac03})
\end{exe}
}

\noindent Language contact with Macedonian and Bulgarian is probably also the reason why speakers of the \textit{Prizrensko-južnomoravski} and \textit{Timočko-lužnički} dialects can place CLs in the sentence-initial position \citep[cf.][239--267]{Okuka08}. The example in (\ref{(7.69)}) below is from \textit{Timočko-lužnički}.

\protectedex{\begin{exe}\ex\label{(7.69)}
\gll \textbf{Če}  dojde  \textit{li}? \\
\textsc{fut}.3\textsc{sg} come.3\textsc{prs}  \textsc{q}  \\
\glt ‘Will he come?’
\hfill  (Timočko-lužnički; \citealt[267]{Okuka08})
\end{exe}
}

\noindent Although \citet[27]{Lisac03} claims that Štokavian’s interesting feature, turning enclitics into proclitics, is due to language contact, it seems that at least some of the examples with CLs in the initial position or after conjunctions such as \textit{i} and \textit{a} cannot be explained through the influence of other languages. \citet[150]{Brozovic07} provides examples from the local idiom spoken in the Fojnica valley (\textit{Srednjobosanski} dialect) in which CLs do not follow the first stressed word, such as those quoted in (\ref{(7.70)}) and (\ref{(7.71)}).

\begin{exe}\ex\label{(7.70)}
\gll \textbf{se}  mȍžĕš  priládit  \\
\textsc{refl}  can.2\textsc{prs}  catch cold \\
\glt ‘you can catch a cold’
\hfill  (Srednjobosanski; \citealt[150]{Brozovic07})

\ex\label{(7.71)}
\gll i  \textbf{je}  rȅkō  \\
and  be.3\textsc{sg}  say.\textsc{ptcp.sg.m}\\
\glt ‘and he told them’
\hfill  (Srednjobosanski; \citealt[150]{Brozovic07})
\end{exe}

\noindent The \textit{Srednjobosanski} dialect borders neither non-Štokavian dialects nor any other languages. Therefore, we assume that its atypical positioning is language-in\-ter\-nal\-ly caused and may have something to do with the fact that it is an Old Štokavian dialect.

\subsection{Endoclitics}

\textsc{endoclitics} (a term proposed by \citealt{RadanovicKocic88}) are a phenomenon very similar to phrase splitting, involving the insertion of a CL in a morphological word, i.e. between affix and stem, like in examples (\ref{(7.86)})--(\ref{(7.88)}) from the local idioms of Derventa, Neretva valley, and Zmijanje (cf. \citealt[58]{Lisac03}, \citealt[67]{Okuka08}, \citealt[195]{VuksaNahod14}). In addition, the occurrence of endoclitics is reported by \citet[279]{Nikolic66} and \citet[23]{Halilovic05} for the \textit{Šumadijsko-vojvođanski} and \textit{Srednjobosanski} dialects. 

\begin{exe}\ex\label{(7.86)}
\gll nȃj  \textbf{mi}  \textbf{je}  drȁžī  \\
most me.\textsc{dat}  be.3\textsc{sg}  dearer \\
\glt ‘he is my dearest’
\hfill  (Zapadni; \citealt[58]{Lisac03})

\ex\label{(7.87)}
\gll [\dots] a rȉbu  nâj  \textbf{san}  vȉšē  ùvatijā  jèdnu  {nôċ [\dots].} \\
{} and fish most be.1\textsc{sg} more catch.\textsc{ptcp.sg.m} one night \\
\glt ‘[\dots] and I caught the most fish one night [\dots].’ \\
\hfill  (Istočnohercegovački; \citealt[195]{VuksaNahod14})

\ex\label{(7.88)}
\gll naj  \textbf{bi}  bolje  uspjevo  krompijer  \\
most \textsc{cond} better succeed.\textsc{ptcp.sg.m} potato \\
\glt ‘the potato would succeed the best’ \\
\hfill  (Istočnohercegovački; \citealt[67]{Okuka08})
\end{exe}

\noindent As we can see, in Neo-Štokavian dialects superlative forms can be split not only by a verbal CL as in (\ref{(7.87)}) and (\ref{(7.88)}), but also by a CL cluster  as in example (\ref{(7.86)}). Furthermore, in the local idiom of Retkovci a pronominal CL can be inserted into negated forms in the present tense of the verb \textit{biti} (\ref{(7.89)}). 

\protectedex{\begin{exe}\ex\label{(7.89)}
\gll Nȉ  \textbf{mi}  \textbf{je}  znô  kãst. \\
\textsc{neg} me.\textsc{dat}  be.3\textsc{sg} know.\textsc{ptcp.sg.m} tell.\textsc{inf} \\
\glt ‘He was not able to tell me.’
\hfill (Slavonski; \citealt[18]{KolenicBilic04})
\end{exe}
}

\noindent Moreover, it seems that endoclitics exist in Istrian Čakavian dialects as well.\footnote{Since Kalsbeek does not provide additional information about his data, we could not determine exactly from which Čakavian dialect those examples originated. } \citet[107]{Kalsbeek03} documented cases of a CL inserted between parts of a negative imperative, like in example (\ref{(7.90)}). 

\protectedex{\begin{exe}\ex\label{(7.90)}
\gll Ne  \textbf{ga}  muõj  zvāljȁt! \\
\textsc{neg} him.\textsc{acc}  \textsc{imp}.2\textsc{sg}  dirty.\textsc{inf}  \\
\glt ‘Don’t dirty it!’
\hfill  (Čakavian; \citealt[107]{Kalsbeek03})
\end{exe}
}

\noindent Finally, in Istrian Čakavian the CL \textit{li} can be placed between the stem and the ending of the future auxiliary (cf. Kalsbeek 2003: 107), like in the example presented in (\ref{(7.91)}).

\protectedex{\begin{exe}\ex\label{(7.91)}
\gll Ćȅ\textbf{li}š  jȕtre  rivȁt  tȍ  storȉt? \\
\textsc{fut.q.2sg}  tomorrow manage.\textsc{inf}  that get.done.\textsc{inf} \\
\glt ‘Will you be able to get that done tomorrow?’ \\
\hfill  (Čakavian; \citealt[107]{Kalsbeek03})
\end{exe}
}

\noindent Similar examples with the interrogative CL \textit{li} inserted between parts of the verb \textit{htjeti} are found in the Neo-Štokavian local idiom of Imotski – see example (\ref{(7.92)}) below.

\protectedex{\begin{exe}\ex\label{(7.92)}
\gll [\dots] òće \textbf{li}  mo  lȅtriku. \\
~ want \textsc{q}  1\textsc{sg}  electricity \\
\glt ‘[\dots] do we want electricity.’
\hfill  (Zapadni; \citealt[212]{Simundic71})
\end{exe}
}

\section{Clitic climbing}
\label{Clitic climbing:8}
CC is not discussed in the dialectological literature, but as the central part (Part \ref{part3}) of this monograph is dedicated to this topic, we tried to find some examples of CC in the transcripts from dialectological literature.\footnote{See Section \ref{Clitic climbing} and Chapter \ref{Approaches to clitic climbing} for a basic explanation of the phenomenon of Clitic Climbing.} Although most of the transcripts include mainly simple structures, we found some examples of CTPs and their infinitive complements.\footnote{See Section \ref{Complement-taking predicates} for basic information on CTP types.} In example (\ref{(7.93)}) from the local idiom of Divin, the archaic pronominal dative CL \textit{vi} climbs over the raising matrix CTP \textit{moći}. Furthermore, we found CC of the pronominal CL \textit{me} (\ref{(7.94)}) in \textit{Sremski} idioms and CC of the \textsc{refl\textsubscript{lex}} \textit{se} in the \textit{Lički} subdialect (\ref{(7.95)}). 

\begin{exe}\ex\label{(7.93)}
\gll [\dots] ali \textbf{vi}\textsubscript{2}  ne  mògu\textsubscript{1}  dat\textsubscript{2}  ȍdgovōr  [\dots]. \\
{} but you.\textsc{dat} \textsc{neg} can.1\textsc{prs}  give.\textsc{inf}  answer \\
\glt ‘[\dots] but I cannot give you the answer [\dots].’ \\
\hfill  (Istočnohercegovački; \citealt[287]{Peco07a})

\ex\label{(7.94)}
\gll kad  \textbf{me}\textsubscript{2}  stȁnū\textsubscript{1}  grčevi  vȁtati\textsubscript{2}  \\
when me\textsc{.acc} start.\textsc{3prs}  cramps catch.\textsc{inf}  \\
\glt ‘when I start getting cramps’
\hfill  (Šumadijsko-vojvođanski; \citealt[368]{Nikolic64})
\ex\label{(7.95)}
\gll ako   \textbf{se}\textsubscript{2} mȉslī\textsubscript{1}  žènit\textsubscript{2} \\
if \textsc{refl}  think.\textsc{3prs}  marry.\textsc{inf}  \\
\glt ‘if he plans to get married’
\hfill  (Istočnohercegovački; \citealt[90]{Okuka08})
\end{exe}

\noindent While the CTPs in examples (\ref{(7.93)}) and (\ref{(7.94)}) are raising predicates (modal and phasal), the CTP in example (\ref{(7.95)}) is a subject control predicate. 

CC is found in Old Štokavian \textit{Slavonski} dialect as well (see (\ref{(7.89)}) above). In this example, the pronominal CL \textit{mi} climbs out of a subject-controlled infinitive and splits the negative present tense form of \textit{biti} ‘be’. Besides examples with CC, we also find examples without CC. These come from the local idiom spoken in the Neretva valley. In addition, we find examples such as (\ref{(7.100)}) regarding which we cannot say for sure whether CC occurred. In the latter example the pronominal CL \textit{ga} is placed directly in front of the infinitive, i.e. it does not climb over the subject matrix predicate \textit{ići}.\footnote{\citet[67]{Junghanns02} says one cannot be sure whether the CL has really climbed if it is placed directly in front of the infinitive, i.e. it can still be embedded. For more information see Section \ref{Clitic climbing}.}

\protectedex{\begin{exe}\ex\label{(7.100)}
\gll [\dots] ìšā\textsubscript{1}	\textbf{ga}\textsubscript{2} 	ùbit\textsubscript{2} 	{i 	[\dots].}  \\
{} go.\textsc{ptcp.sg.m}	him.\textsc{acc} 	kill.\textsc{inf} 	and	more \\
\glt ‘[\dots] he went to kill him [\dots].’ \\
\hfill  (Istočnohercegovački; \citealt[195]{VuksaNahod14})
\end{exe}
}

\noindent Furthermore, we find examples such as (\ref{(7.101)}) from the local idiom of Pričinović, where something quite the reverse of CC happens. In this example, the pronominal CL \textit{ti} generated by the matrix verb \textit{valja} appears after the embedded infinitive complement \textit{ići}. This may be an instance of an retrospective (afterthought) frequently found in spoken language.\footnote{For more information and examples, see Sections \ref{Principles of analysis of spoken language} and \ref{Impact}.}

\protectedex{\begin{exe}\ex\label{(7.101)}
\gll štȁ 	\textbf{š} 	\textbf{se} 	rasprémati 	kad vàljā\textsubscript{1} 	òpet 	ìći\textsubscript{2} 	\textbf{ti}\textsubscript{1}  \\
what	\textsc{fut.2sg} 	\textsc{refl} 	get.undressed.\textsc{inf} 	when  ought.3\textsc{prs} 	again	go.\textsc{inf} 	you.\textsc{dat} \\
\glt 	‘why would you get undressed when you ought to go again’ \\
\strut\hfill  (Šumadijsko-vojvođanski; \citealt[280]{Nikolic66})
\end{exe}
}

\noindent In Chapter \ref{A corpus-based study on CC in da constructions and the raising-control distinction (Serbian)} we present our study of CC out of \textit{da}\textsubscript{2}-complements in srWaC. This very rare phenomenon appears in \textit{Sremski} idioms. The examples of CC out of \textit{da}\textsubscript{2}-complements presented below contain both raising (\ref{(7.96)})--(\ref{(7.97)}) and subject control (\ref{(7.98)})--(\ref{(7.99)}) CTPs.

\begin{exe}\ex\label{(7.96)}
		\gll al  \textbf{su}  \textbf{me}\textsubscript{2}  mórali\textsubscript{1}  da  pȕstē\textsubscript{2}  kȕći  \\
		but be.3\textsc{pl}  me.\textsc{acc}  must.\textsc{ptcp.pl.m} that  let.3\textsc{prs}  home \\
		\glt ‘but they had to let me go home’ \\
	\hfill  (Šumadijsko-vojvođanski; \citealt[368]{Nikolic64})

	\ex\label{(7.97)}
		\gll e  sȁd  jȃ  \textbf{vam}\textsubscript{2}  nè  mogu\textsubscript{1}  tȁčno  da  kȃžēm\textsubscript{2}  \\
		well now I you.\textsc{dat}  \textsc{neg} can.1\textsc{prs} exactly that  say.1\textsc{prs}\\
		\glt ‘well now, I cannot tell you exactly’ \\
	\hfill  (Šumadijsko-vojvođanski; \citealt[368]{Nikolic64})

	\ex\label{(7.98)}
		\gll nȉje  \textbf{ga}\textsubscript{2}  ni  stȉgō\textsubscript{1}  da  vȉdī\textsubscript{2}  \\
		\textsc{neg}.be.3\textsc{sg}  him.\textsc{acc}  \textsc{neg} get\textsc{.ptcp.sg.m} that  see\textsc{.3prs} \\
		\glt ‘he did not even get to see him’ \\
	\hfill  (Šumadijsko-vojvođanski; \citealt[368]{Nikolic64})

	\ex\label{(7.99)}
		\gll kako  \textbf{vas}\textsubscript{2}  \textbf{je}  jèdān  tȅo\textsubscript{1}  da  túčē\textsubscript{2}  \\
		how you.\textsc{acc} be.3\textsc{sg}  one want.\textsc{ptcp.sg.m} that  beat.3\textsc{sg} \\
		\glt ‘how one of them wanted to beat you’ \\
	\hfill  (Šumadijsko-vojvođanski; \citealt[368]{Nikolic64})
\end{exe}

\section{Diaclisis}
\label{Diaclisis:8}
We find examples of diaclisis in several dialects.\footnote{For a definition of diaclisis, see Section \ref{Diaclisis and pseudodiaclisis}.} Examples (\ref{(7.102)}) and (\ref{(7.103)}) are from the \textit{Zapadni} dialect.

\begin{exe}\ex\label{(7.102)}
\gll [\dots] i	ȍnda 	ka 	\textbf{se} 	dòbro j\textsuperscript{e} 	{svârī [\dots].}  \\
{} and	then	when	\textsc{refl} 	good	her.\textsc{acc} 	cook.3\textsc{prs} \\
\glt ‘[\dots] and then when it is cooked well [\dots].’
\hfill  (Zapadni; \citealt[285]{Curkovic14})

\ex\label{(7.103)}
\gll [\dots] kad 	\textbf{b\textsuperscript{i}} 	ȍnī 	stârci 	\textbf{se} 	skȕpili 	{ȕv\textsuperscript{e}če [\dots].} \\
{} when	\textsc{cond.3pl} 	those	old.men	\textsc{refl} 	gather.\textsc{ptcp.pl.m} in.the.evening \\
\glt ‘[\dots] when those old men would gather in the evening [\dots].’ \\
\hfill  (Zapadni; \citealt[320]{Curkovic14})
\end{exe}

\noindent Nevertheless, we must point out that diaclisis is not the rule in the \textit{Zapadni} dialect, since examples with CL clusters like the one presented in (\ref{(7.104)}) are also attested.

\protectedex{\begin{exe}\ex\label{(7.104)}
\gll [\dots] Òbūkl\textsuperscript{i}	\textbf{b} 	\textbf{se} 	ù 	onū 	svȁsku 	{rȍbu [\dots].} \\
{} dress.\textsc{ptcp.pl.m}	\textsc{cond.3pl} 	REFL 	in	those	wedding	clothes \\
\glt ‘[\dots] They would dress in those wedding clothes [\dots].’ \\
\hfill  (Zapadni; \citealt[284]{Curkovic14})
\end{exe}
}

\noindent Diaclisis is also attested in the local idiom of Pričinović and in the local idiom of Babina Greda: 

\begin{exe}\ex\label{(7.105)}
\gll Pa 	\textbf{je} 	\textbf{l} 	múzē 	\textit{se}? \\
well	be.3\textsc{sg} 	\textsc{q} 	milk.3\textsc{prs} 	\textsc{refl}  \\
\glt ‘Well, is it being milked?’
\hfill  (Šumadijsko-vojvođanski; \citealt[279]{Nikolic66})

\ex\label{(7.106)}
\gll U 	dva 	sata 	\textbf{smo} 	noću 	\textbf{se} 	dizali. \\
at	two	o’clock	be.1\textsc{pl} 	at.night	\textsc{refl} 	get.up.\textsc{ptcp.pl.m} \\
\glt ‘At two o’clock at night we got up.’
\hfill  (Slavonski; \citealt[149]{FarkasBabic11})
\end{exe}

\section{Clitic doubling}
\label{Clitic doubling}
CL doubling is considered to be a feature specific to the Balkan languages; it involves structures where pronominal CLs double overtly expressed direct or indirect objects.\footnote{We did not include CL doubling among other parameters in Chapter \ref{Our terms and concepts} because it is not attested in the standard varieties.} It is common to all Slavonic and non-Slavonic South-Eastern Balkan languages \citep[114]{Stevanovic50}. \citeauthor{MiseskaTomic06} (\citeyear[239]{MiseskaTomic06}, \citeyear[426]{MiseskaTomic08}) claims that pronominal CLs do not double direct or indirect objects in standard varieties of BCS. The same observation applies to the Northern Serbian dialects. Conversely, pronouns are CL-doubled in all the South-eastern Serbian dialects \citep[463]{MiseskaTomic08}. Whereas all indirect objects are regularly CL-doubled in the western periphery of the south-eastern Serbian dialect area, in the southeasternmost parts of the area both direct and indirect lexical objects are only optionally CL-doubled \citep[463]{MiseskaTomic08}. 

\citet[113f]{Stevanovic50} provides examples of CL doubling in the \textit{Prizrensko-južnomoravski} dialect and claims that such examples appear in the \textit{Kosovsko-resavski} dialect too. One of his examples from the local idiom of Đakovica is presented in (\ref{(7.108)}). \citet[112]{Barjaktarevic66} finds CL doubling in the \textit{Zetsko-južno\-sandžač\-ki} dialect: example (\ref{(7.109)}) is from the local idiom of Trnava. Furthermore, \citet[356]{Ivic57} claims that CL doubling can be found in the idiom of Galipolje Serbs (\ref{(7.110)}). 

\begin{exe}\ex\label{(7.108)}
\gll Dáj 	\textbf{mu} 	\minsp{[} njȅmu]. \\
 give.\textsc{imp} 	him.\textsc{dat} {}	him.\textsc{dat}  \\
\glt ‘Give him.’
\hfill  (Prizrensko-južnomoravski; \citealt[113]{Stevanovic50})

\ex\label{(7.109)}
\gll \minsp{[} Mène] 	\textbf{me} 	 zovȇ. \\
 {} me.\textsc{acc} 	me.\textsc{acc} 	call.\textsc{prs}  \\
\glt ‘He calls me.’
\hfill  (Zetsko-južnosandžački dialect; \citealt[112]{Barjaktarevic66})

\ex\label{(7.110)}
\gll Uvȃ 	\textbf{mi} 	\textbf{je} 	divȏjka 	poznáta 	\minsp{[} minȅ]. \\
 this	me.\textsc{dat} 	be.3\textsc{sg} 	girl	familiar {} me.\textsc{dat}  \\
\glt 	‘This girl is familiar to me.’
\hfill  (Galipolje idiom; \citealt[356]{Ivic57})
\end{exe}

\noindent Nonetheless, CL doubling is not obligatory in the latter idiom, i.e. constructions without CL doubling are used as well \citep[357]{Ivic57}. In addition, \citet[357]{Ivic57} argues that CL doubling is far more common in the neighbouring \textit{Prizrensko-južnomoravski} dialect than in the idiom of Galipolje Serbs. He assumes that CL doubling in the idiom of Galipolje Serbs is not a result of direct Macedonian influence. He finds arguments for this assumption in differing word order \citep[cf.][357]{Ivic57}. \citet[357]{Ivic57} believes that CL doubling was already present in this idiom when Galipolje Serbs lived in Barjamič and that it was the result of Greek influence.   
CL doubling is also attested among the Croatian population in Janjevo and Letnica (\ref{(7.111)}), who speak the \textit{Prizrensko-južnomoravski} Torlac dialect, and in the \textit{Moliški} dialect (\ref{(7.112)}):\footnote{There are several theories about the origins of Croatian speakers from Molise. However, all of the theories agree that they most probably came to Molise from Štokavian territory.}

\begin{exe}\ex\label{(7.111)}
\gll on 	\textbf{me} 	\minsp{[} mene] 	videja  \\
he	me.\textsc{acc} {}	me.\textsc{acc} 	see.\textsc{ptcp.sg.m} \\
\glt ‘he saw me’
\hfill  (Prizrensko-južnomoravski; \citealt[149]{Lisac03})

\ex\label{(7.112)}
\gll \minsp{[} njega]	su	\textbf{ga}	ubili \\
 {} him.\textsc{acc} 	be.3\textsc{pl} 	him.\textsc{acc}	kill.\textsc{ptcp.pl.m} \\
\glt ‘they killed him’
\hfill  (Moliški; \citealt[59]{Lisac03})
\end{exe}

\section{Summary}
\subsection{Inventory}


We notice a considerable number of forms for pronominal CLs. First, we see that in many idioms of both Old and Neo-Štokavian dialects both forms \textit{ju} and \textit{je} are attested for the feminine singular accusative. We come across dialects which use \textit{je} exclusively, \textit{ju} exclusively, and those which use both. Second, somewhat unexpectedly we find a large number of varying forms of other pronominal CLs which have not found their way into any of the three standard vairieties. In this respect the \textit{Prizrensko-južnomoravski} dialect spoken in Southern Serbia and Kosovo turns out to be the most varied as it shows the greatest number of forms (i.e. four forms for her.\textsc{dat}, six forms for her.\textsc{acc}, six forms for they.\textsc{acc} and seven forms for they.\textsc{dat}). Without going into much detail, we observe an uneven distribution of variation according to the person and number categories. Namely, dialects tend to show more variants for pronouns of the third person plural than of first and second person plural. In contrast, it seems that pronominal forms for the first and second person singular in dialects do not differ much from forms attested in standard BCS varieties.

As mentioned in Section \ref{Reflexive markers se and si in BCS standard varieties} the standard varieties differ with respect to the reflexive CL \textit{si}. However, our dialectological overview shows that this form is found not only on Croatian, but also on Bosnian and Serbian language territory: it is attested in a scattered area comprising some idioms of Western Herzegovina, Northern Bosnia, South Eastern Serbia and Montenegro (\textit{Zapadni}, \textit{Istočnohercegovački}, \textit{Svrljiško-zaplanjski}, \textit{Prizrensko-južnomoravski}, \textit{Slavonski} and \textit{Zetsko-južnosandžački} dialects). Furthermore, this form is also present in spoken Bosnian – for more details, see Section \ref{Reflexive clitics:9}.

We do not find a great deal of variation as to the inventory of verbal CLs. As in the spoken varieties, in many dialects (e.g. \textit{Istočnohercegovački}, \textit{Zapadni}, \textit{Šumadijsko-vojvođanski}, \textit{Slavonski} and \textit{Kosovsko-resavski}) the conditional auxiliary form \textit{bi} is used for all persons.

\subsection{Internal organisation of the clitic cluster}

In many dialects we find non-standard order in the CL cluster. The most common divergent pattern concerns the order of the reflexive \textit{se} and its position relative to the verbal CL \textit{je}, conditional auxiliary CL \textit{bi}, the polar question marker CL \textit{li} and pronominal CLs. Furthermore, in some Serbian dialects (e.g. \textit{Šu\-ma\-dij\-sko-voj\-vo\-đan\-ski}, \textit{Zetsko-južnosandžački}, \textit{Svrljiško-zaplanjski} and the idiom of Ga\-li\-po\-lje Serbs) the verbal CL \textit{je} appears in front of pronominal CLs just like all other verbal CLs. Here the CL cluster contains a single slot for all verbal CLs and is thus simpler than in the standard languages. In one idiom belonging to the \textit{Prizrensko-južnomoravski} dialect pronominal CLs can stand in front of verbal CLs (the present tense of \textit{biti}) and in the \textit{Zapadni} and \textit{Zetsko-južnosandžački} dialects pronominal CLs can stand before conditional auxiliary CLs. In addition, we would like to point out that although in many dialects some kind of non-standard CL order in the CL cluster is attested, sometimes a CL order which does not diverge from the standard is attested besides that “reversed” order. 

Dialects present a varied picture of the use of the pronominal CL \textit{je} (third person singular feminine accusative CL) and the homophone verbal CL \textit{je} (present tense third person singular of \textit{biti} ‘be’). Some local idioms (e.g. of Imotski, Tuholj and Pag) do not exhibit repeated morph constraint, i.e. allow the repetition of \textit{je}, while others (e.g. of Bitelić) use haplology. The same variability is found in the case of the co-occurrence of \textit{se je} versus haplology of unlikes. 

\subsection{Position of the clitic or the clitic cluster}

Unlike in standard varieties of BCS, in dialects CLs can take the sentence-initial position, and they can follow the conjunctions \textit{i} and \textit{a} ‘and’. Such occurrences are mainly attested in dialects neighbouring with varieties which do allow 1P like Čakavian or the Romanian language, but as examples from the \textit{Srednjobosanski} dialect indicate, not all 1P occurrences can be ascribed to language contact. Similarly, DP is a relatively widespread feature found not only in contact varieties. In this respect dialects definitely differ from the standard Bosnian and Serbian varieties since in the former, CLs do not always follow the \textit{da}-complementiser. As those examples show, placement of CLs directly after the \textit{da}-complementiser is not the only correct possibility, which is in accordance with the results of our study on CC out of \textit{da}\textsubscript{2}-complements (for more information see Sections \ref{Placement with regard to different types of hosts in BCS standard varieties} and \ref{Results:da}).

A further finding concerns phrase splitting, which is attested in both Old and Neo-Štokavian dialects. In most cases attributes are split from their nouns by a verbal CL, but there are some examples with pronominal CLs in the dative. Many types of phrase splitting attested in dialects (e.g. splitting of a prepositional phrase, conjoined phrase, quantificational phrase, forename and surname) are controversially discussed in the theoretical syntactic literature. In addition, in dialects we found examples of CL clusters splitting phrases, which theoretical syntacticians \citet{Progovac96} and \citet{RadanovicKocic96} judge to be unacceptable (for more information see Section \ref{The limits of phrase splitting in BCS standard varieties}).\footnote{\textcolor{black}{This feature is detectable not only from dialectological data but also from the corpus of Spoken Bosnian \textit{Bosnian Interviews}, see Section \ref{Inventory of clitics participating in phrase splitting}}}\textsuperscript{,}\footnote{\textcolor{black}{We do not claim that dialectological data is superior to the data of \citet{Progovac96} and \citet{RadanovicKocic96} but rather that it is different. Namely, we point out the differences between the varieties described in works of a theoretical character and works on dialects. These differences indicate that the theoretical models of BCS CLs most likely do not hold for dialectological data. Nevertheless, the fact that formal theories cannot account for dialectological data does not make such data inferior to or less valid than data provided by formal linguists. Dialectological data is valid for itself, that is, for varieties classified as dialects. Moreover, the fact that dialectological data is usually collected to describe phonetic and phonological properties, with only a superficial interest in morphology and practically none in syntax, speaks for itself. In addition, the fieldwork is usually conducted by descriptive linguists. We believe that in a way, the latter two facts ensure that there is no theoretical agenda that could have distorted the dialectological data in any way.}} Moreover, we even came across one type of split not attested in the standard languages: endoclitics, i.e. CLs that split one morphological word form.

It is interesting to note that we do find single examples of diaclisis. Due to the small number of instances we cannot draw any further conclusions. Although we have only several examples of CC, we can claim that it is attested not only from infinitive complements, but in the Serbian \textit{Šumadijsko-vojvođanski} dialect also from \textit{da}\textsubscript{2}-complements. These findings are in accordance with the results of our corpus study presented in Chapter \ref{A corpus-based study on CC in da constructions and the raising-control distinction (Serbian)}.

Finally, we can speculate that the \textit{Prizrensko-južnomoravski} dialect has a different CL system than the majority varieties of BCS because it has not only differing ordering patterns but also the possibility of regular 1P and uses CL doubling.
