\chapter{Introductory remarks to corpus studies on clitic climbing}
\label{Introductory remarks to corpus studies on CC}
\section{Corpus-driven studies on clitic climbing}

In the next two chapters we present two corpus studies designed to investigate some tentative constraints on CC in BCS, formulated in the previous chapter where we compared CC in BCS and in Czech. The two studies are methodologically similar. As explained in Section \ref{Empirical approach in the current study}, our approach to CC is empirical and inductive. Instead of assuming any universal rules, we are interested in statistical tendencies and regularities. We do not reject any constructions before examining real instances of them in corpora, and potentially testing them further on informants. We suspect that structures have often been rejected due to their infrequency. Hence, we first try to retrieve all permuted structures from large corpora, also the supposedly incorrect ones.\footnote{\textcolor{black}{For more information on large corpora} see Section \ref{WAC}.}

In order to do that, in Section \ref{Operationalising the constructions in question} we define the construction which we investigated in terms of CC and show how we formulated the appropriate queries. In Section \ref{Choice of corpora}, we argue which sources are the most suitable for retrieving the studied constructions. Section \ref{Choice of matrix verbs} explains how complement-taking predicates (CTPs) were sampled for the study. Data analysis is presented in the next chapters; in this chapter we explain only the data collection process.

\section{Operationalising the constructions in question}
\label{Operationalising the constructions in question}

As explained in Chapter \ref{Approaches to clitic climbing}, we study constructions containing two verbal elements. These are \textit{da}\textsubscript{2}-constructions in Serbian (Chapter \ref{A corpus-based study on CC in da constructions and the raising-control distinction (Serbian)}) and infinitive complement constructions in Croatian (Chapter \ref{A corpus-based study on clitic climbing in infinitive complements in relation to the raising-control dichotomy and diaphasic variation (Croatian)}).\footnote{See Section \ref{Types of complements} for more information on \textit{da}\textsubscript{2}-constructions.} In both constructions, several positions of the CL complement in relation to the complement-taking predicate and verbal complement are potentially possible, as shown in Table \ref{T12.1}.\footnote{For more information on complement-taking predicates see Section \ref{Complement-taking predicates}.}

\begin{table}
\caption{Permuted constructions with verbal complements.\label{T12.1}}
\fittable{\begin{tabular}{lllll}
\lsptoprule
& \multicolumn{2}{c}{noCC} & \multicolumn{2}{c}{CC}\\
\cmidrule(lr){2-3}\cmidrule(lr){4-5} 
Complement type & Variant 1 & Variant 2 & Variant 3 & Variant 4\\\midrule
\textit{da}\textsubscript{2} & \textsc{ctp da comp cl}& \textsc{ctp da cl comp} & \textsc{ctp cl da comp} & \textsc{cl ctp da comp} \\
infinitive & \textsc{ctp inf cl}& NA & \textsc{ctp cl inf} & \textsc{cl ctp inf} \\
\lspbottomrule
\end{tabular}}
\end{table}

Occurrences of each variant of the constructions in question can be retrieved from the corpora by means of CQL queries in which we can combine morphosyntactic tags with word form and lemma-based attribute search.\footnote{For a basic description of CQL see \url{https://www.sketchengine.eu/documentation/cql-basics/}.} To explain the logic behind the CQL queries in our studies, we provide an example of variant 4 with an infinitive complement from Chapter \ref{A corpus-based study on clitic climbing in infinitive complements in relation to the raising-control dichotomy and diaphasic variation (Croatian)}, where only the third person pronominal and reflexive CLs were retrieved, while CTPs were restricted to the present tense form. Following the template from Table \ref{T12.1}, the basic CQL query, according to the current MSD index used for hrWaC, should be as follows:\footnote{\textcolor{black}{For the current MSD index used for hrWaC visit} \url{http://nl.ijs.si/ME/V6/msd/html/msd-hbs.html\#sd.msds-hbs}.}


\noindent\begin{verbatim*}
[(word="([mj]u)|(joj)|(i[hm])|(ga)|(se)|(je)|(si)"][tag="Vmr.*"]
[tag="V.n"]
\end{verbatim*}

%Old:
%[(word=$"$([mj]u)|(joj)|(i[hm])|(ga)|(se)|(je)|(si)$"$][tag=$"$Vmr.*$"$][tag=$"$V.n$"$]

\noindent The first segment in the query encodes the expression for third person pronominal and reflexive CLs via their word forms, while the query for a present tense indicative form of a CTP (second segment) and an infinitive (third segment) is performed via their morphosyntactic tags.\footnote{Each segment is specified in square brackets.}

This, however, is not enough, as for example the forms of two CLs, the pronominal CL \textit{je} and the reflexive CL \textit{si}, are ambiguous since they are homographic with verbal CLs.\footnote{A list of all CL forms used in BCS standard varieties can be found in Section \ref{Inventory:7}.} The correction shown below decreases the likelihood that non-target CLs, i.e. verbal instead of pronominal or reflexive, will appear in the result of a query:


\noindent\begin{verbatim*}
[(word="([mj]u)|(joj)|(i[hm])|(ga)|(se)")|(word="(je)|(si)"&
tag!="V.*")]
\end{verbatim*}

%Old:
%[(word=$"$([mj]u)|(joj)|(i[hm])|(ga)|(se)$"$)|(word=$"$(je)|(si)$"$ \& tag!=$"$V.*$"$)]

\noindent This is, naturally, under the assumption that the tagger has high accuracy. Prior to using tag attributes we examined frequency lists to ensure that a given tag is not too prone to error. Still, errors cannot be eliminated completely. For example, the tagger sometimes interprets the verbal CL \textit{je} ‘is’ as the accusative form of the third person feminine pronominal, as in (\ref{(12.1)}). A similar error occurs for the verbal CL \textit{si} ‘are’, which is interpreted in (\ref{(12.2)}) as the reflexive CL in the dative. In order to better explain the problem, we introduce an additional line of glosses containing morphosyntactic tags from hrWaC.

\begin{exe}\ex\label{(12.1)}
\glll Prva stvar koju trebate zapamtiti \textbf{je} ne paničariti.\\
 first thing which have.to.\textsc{2prs} remember.\textsc{inf} be.\textsc{3sg} \textsc{neg} panic.\textsc{inf} \\
 Mlofsn Ncfsn Pi-fsa* Vmr2p Vmn Pp3fsa* QZ Vmn \\
\glt ‘The first thing you have to remember is not to panic.’
\hfill [hrWaC v2.2]

\ex\label{(12.2)}
\glll Onoga, koga treba banirati \textbf{si} upravo {ti [\dots].} \\
 that which have.to.\textsc{3prs} ban.\textsc{inf} be.\textsc{3sg} exactly you\\
 Pd-nsg* Pi3m-a* Vmr3s Vmn Px--sd* Rgp Pp2-sn \\
\glt ‘You are the one who should be banned [\dots].’
\hfill  [hrWaC v2.2]
\end{exe}

\noindent The first two words in example (\ref{(12.1)}) are tagged correctly, whereas the relative pronoun \textit{koju} ‘which’ is misclassified as an indefinite pronoun. It is worth pointing out that although the relative pronoun class does exist in the tagset description, the query \texttt{[tag=$"$Pr.*$"$]} returns an empty result: hence, the tagger does not recognise this class. The misclassified relative pronoun is followed by two correctly tagged words. However, the verbal CL \textit{je} ‘is,’ which has the function of copula, is due to its homographic form and its position misclassified as a pronominal CL. This is because the automatic tool does not perform a syntactic analysis and the infinitive \textit{zapamtiti} ‘remember’ is more likely to be followed by an accusative phrase (a direct object) than by a copula. Thus, it is wrongly tagged as the third person singular feminine personal pronoun in the accusative. The last two words in example (\ref{(12.1)}) are tagged correctly.  

In example (\ref{(12.2)}), three out of seven words are mistagged. The tagger assigned the wrong case and gender to the demonstrative pronoun: \textit{onoga} ‘that’ is masculine and not neuter. It is in the accusative and not in the genitive case. Similarly to \textit{koju} in example (\ref{(12.1)}), the relative pronoun \textit{koga} ‘which’ is misclassified as indefinite. Whereas the infinitive, adverb and personal pronoun are tagged correctly, the verbal CL \textit{si} ‘are’ is misclassified due to its homographic form and its position, since the infinitive \textit{banirati} ‘ban’ is more likely to be followed by a dative phrase (the indirect object). The verbal copula is falsely interpreted as the reflexive CL \textit{si} in the dative. 

The basic query shown above, consisting only of \textsc{core} \textsc{elements}, that is, the elements belonging to the target construction, gives low recall, particularly in the case of rarely occurring variants of target constructions. Therefore, in our queries we introduced \textsc{free} \textsc{elements} (defined below as \texttt{[]\{0,4\}}) appearing between the core elements of the query:\footnote{The number of free elements allowed between core elements differs between the studies. For example, in the case of the infinitive complement construction labelled as variant 3 in Table \ref{T12.1}, free elements between the infinitive and the CL are obligatory in order to ensure that the variant of the construction is an instance of CC. Note that, as already mentioned in Section \ref{Clitic climbing}, \textcite[67]{Junghanns02} warns that if the CL is placed directly in front of an infinitive, we cannot be sure whether CC really occurred or whether the CL is still in the complement. Obligatory free elements separating a CL from an infinitive should guarantee that CC really occurred. In contrast, free elements are not needed in the case of the \textit{da}\textsubscript{2}-complement for the construction labelled as variant 3 in Table \ref{T12.1}, since \textit{da} itself stands between the CL and the semifinite verbal part of the complement.}


\noindent\begin{verbatim*}
[(word="([mj]u)|(joj)|(i[hm])|(ga)|(se)")|(word="(je)|(si)"&
tag!="V.*")][]{0,4}[tag="Vm.*"][]{0,4}[tag="V.n"]
\end{verbatim*}

%Old:
%$[(word="([mj]u)|(joj)|(i[hm])|(ga)|(se)")|(word="(je)|(si)" \& tag!="V.*")][]\{0,4\}[tag="Vm.*"] []\{0,4\} [tag="V.n"]$

\noindent Next, by analysing the tag-based frequency lists we determined the expressions which should be excluded from free elements in order to gradually increase the complexity of queries. Our aim was to eliminate as much noisy data as possible, but at the same time to keep possibly many instances of the constructions in question. Therefore, free elements could contain neither additional core elements nor expressions that would most probably mark sentence or clause crossing, such as conjunctions (queried as \texttt{tag=$"$C.*$"$}), punctuation (queried as \texttt{tag=$"$\textbackslash{}Z"}), accidental omission of a space after a full stop (queried as \texttt{word=$"$.*\textbackslash{}..*$"$}), indefinite and interrogative pronouns (queried as \texttt{tag=$"$P[iq].*$"$}), participles (queried as \texttt{tag=$"$Rr"}), negative verbal forms (queried as \texttt{tag=$"$V.*y"}), and most auxiliary and copula forms. Example (\ref{(12.3)}) is an illustration of a false positive due to an accidental omission of a space after a full stop. The infinitive \textit{predočiti} ‘envisage’ is written together with the personal pronoun \textit{mi} ‘we’ which belongs to the next sentence in the text. The verb is correctly tagged as an infinitive. However, it is lemmatised as a non-existent infinitive form *\textit{predočiti.mi}.

\protectedex{\begin{exe}\ex\label{(12.3)}
\gll [\dots] ne možemo \minsp{*} predočiti.Mi \textbf{mu}  logički moramo  {pripisati [\dots].}\\
{} \textsc{neg} can.\textsc{2prs} {} envisage.\textsc{inf}.we him.\textsc{dat} logically must.\textsc{2prs} attribute.\textsc{inf} \\
\glt ‘[\dots] we cannot envisage. Logically, we have to attribute to him [\dots].’ \\
\strut\hfill [hrWaC v2.2]
\end{exe}
}

\noindent Finally, we introduced obligatory free elements at the beginning and at the end of the query. This allowed us to eliminate unwanted results where the CL belongs either to the preceding, or to the following predicate. An example of the former is (\ref{(12.4)}), in which the CL \textit{mu} ‘him’ is part of the structure \textit{su} \textit{mu} \textit{u pripremi} and is not goverened by the infinitive \textit{pročitati} ‘read’. In (\ref{(12.5)}) the CL \textit{ih} ‘them’ is a complement of \textit{pronaći} ‘find’ and not of \textit{potruditi se} ‘try’; that is, it is governed by the following predicate, and not by the target predicate.\footnote{Both infinitives in that example are written without the final vowel \textit{-i}, which is a feature of colloquial BCS. Furthermore, the infinitive \textit{pronaći} ‘find’ is also written without a diacritic, which is a feature of the language used in user generated content.}

\begin{exe}
\ex\label{(12.4)}
\gll Pitanja koja \textbf{su}\textsubscript{1}  \textbf{mu}\textsubscript{1}  već u pripremi\textsubscript{1}  možete\textsubscript{2} pročitati\textsubscript{3} pod {komentarima [\dots].} \\
questions which be.\textsc{3pl} him.\textsc{dat} already in preparation can.\textsc{2prs} read.\textsc{inf} under comments \\
\glt ‘You can read the questions for him which are already in preparation in the comments [\dots].’
\hfill [hrWaC v2.2]

\ex\label{(12.5)}
\gll Mislim da \textbf{se}\textsubscript{2} za tratorie i  restorane treba\textsubscript{1}  malo potrudit\textsubscript{2} \textbf{ih}\textsubscript{3} {pronac\textsubscript{3} [\dots].}\\
think.\textsc{1prs} that \textsc{refl} for trattorias and restaurants have.to.\textsc{2prs} little try.\textsc{inf} them.\textsc{acc} find.\textsc{inf} \\
\glt ‘I think that for trattorias and restaurants one has to try a little bit to find them [\dots].’
\hfill [hrWaC v2.2]
\end{exe}


\noindent As a result, we obtained more complex but better performing queries (in this particular case, for variant 4 from Table \ref{T12.1}). This allowed us to extract more and better data for our studies. Regardless of the improvements, one should bear in mind that some level of error is unavoidable. The full query is shown below:


\noindent\begin{verbatim*}
[!(word="(me)|([mj]u)|(joj)|(i[hm])|(ga)|([nv]a[sm])|(se)"|
(word="(je)|(si)"&tag!="V.*")|(word="[mt]i"&tag!="(Pp[12]-[sp]n)|
(Pd-mpn)")|(word="te"&tag!="(Pd-[fm][sp][nga])|(Cc)"))]{1,2}
\end{verbatim*}



\noindent\begin{verbatim*}
[(word="([mj]u)|(joj)|(i[hm])|(ga)|(se)")|(word="(je)|(si)"&
tag!="V.*")]
\end{verbatim*}



\noindent\begin{verbatim*}
[!(tag="C.*"|lemma="\Z"|tag="P[iq].*"|tag="V.*"|tag="Rr"|word=
".*\..*"|lemma="što"|word="(me)|([mj]u)|(joj)|(i[hm])|(ga)|
([nv]a[sm])|(se)"|(word="(je)|(si)"&tag!="V.*")|(word="[mt]i"&
tag!="(Pp[12]-[sp]n)|(Pd-mpn)")|(word="te"&tag!="(Pd-[fm][sp]
[nga])|(Cc)"))]{0,4}
\end{verbatim*}



\noindent\begin{verbatim*}
[lemma="sramiti" & tag="V.r.*"]
\end{verbatim*}



\noindent\begin{verbatim*}
[!(tag="C.*"|lemma="\Z"|tag="P[iq].*"|tag="V.*"|tag="Rr"|word=
".*\..*"|lemma="što"|word="(me)|([mj]u)|(joj)|(i[hm])|(ga)|
([nv]a[sm])|(se)"|(word="(je)|(si)"&tag!="V.*")|(word="[mt]i"&
tag!="(Pp[12]-[sp]n)|(Pd-mpn)")|(word="te"&tag!="(Pd-[fm][sp]
[nga])|(Cc)"))]{0,4}
\end{verbatim*}



\noindent\begin{verbatim*}
[tag="V.n" & lemma!="biti"]
\end{verbatim*}



\noindent\begin{verbatim*}
[!(tag="C.*"|lemma="\Z"|tag="P[iq].*"|tag="V.*"|tag="Rr"|word=
".*\..*"|lemma="što"|word="(me)|([mj]u)|(joj)|(i[hm])|(ga)|
([nv]a[sm])|(se)"|(word="(je)|(si)"&tag!="V.*")|(word="[mt]i"&
tag!="(Pp[12]-[sp]n)|(Pd-mpn)")|(word="te"&tag!="(Pd-[fm][sp]
[nga])|(Cc)"))]{1,2}within<s/>
\end{verbatim*}


%old, lines should be re-done
%[!(word=$"$(me)|([mj]u)|(joj)|(i[hm])|(ga)|([nv]a[sm])|(se)$"$ | (word=$"$(je)|(si)$"$}
%\noindent\texttt{\small\&tag!=$"$V.*$"$) | (word=$"$[mt]i"\&tag!=$"$(Pp[12]-[sp]n)|(Pd-mpn)$"$) | (word=$"$te"}
%\noindent\texttt{\small(word=$"$te" \&tag!=$"$(Pd-[fm][sp][nga])|(Cc)$"$))]\{1,2\}}
%\noindent\texttt{\small[(word=$"$([mj]u)|(joj)|(i[hm])|(ga)|(se)$"$) | (word=$"$(je)|(si)$"$ \& tag!=$"$V.*$"$)]}
%\noindent\texttt{\small[!(tag=$"$C.*$"$ | lemma=$"$\textbackslash{}Z"| tag=$"$P[iq].*$"$ | tag=$"$V.*$"$ | tag=$"$Rr$"$ | word=$"$.*\textbackslash{}..*$"$|}
%\noindent\texttt{\small lemma=$"$što$"$ | word=$"$(me)|([mj]u)|(joj)|(i[hm])|(ga)|([nv]a[sm])|(se)$"$|}
%\noindent\texttt{\small (word=$"$(je)|(si)$"$ \& tag!=$"$V.*$"$) | (word=$"$[mt]i$"$ \& tag!=$"$(Pp[12]-[sp]n)|(Pd-mpn)$"$) |}
%\noindent\texttt{\small (word=$"$te$"$ \& tag!=$"$(Pd-[fm][sp][nga])|(Cc)$"$))]\{0,4 \}[lemma=$"$sramiti$"$ \& tag=$"$V.r.*$"$]}
%\noindent\texttt{\small[!(tag=$"$C.*$"$ | lemma=$"$\textbackslash{}Z$"$ | tag=$"$P[iq].*$"$ | tag=$"$V.*$"$ | tag=$"$Rr$"$ | word=$"$.*\textbackslash{}..*$"$|}
%\noindent\texttt{\small lemma=$"$što$"$ | word=$"$(me)|([mj]u)|(joj)|(i[hm])|(ga)|([nv]a[sm])|(se)$"$|}
%\noindent\texttt{\small (word=$"$(je)|(si)$"$ \& tag!=$"$V.*$"$)| (word=$"$[mt]i$"$ \& tag!=$"$(Pp[12]-[sp]n)|(Pd-mpn)$"$)|}
%\noindent\texttt{\small (word=$"$te$"$ \& tag!=$"$(Pd-[fm][sp][nga])|(Cc)$"$))]\{0,4\} [tag=$"$V.n$"$ \& lemma!=$"$biti"]}\\
%\noindent\texttt{\small[!(tag=$"$C.*$"$ | lemma=$"$\textbackslash{}Z$"$ | tag=$"$P[iq].*$"$ | tag=$"$V.*$"$ | tag=$"$Rr$"$| word=$"$.*\textbackslash{}..*$"$|}
%\noindent\texttt{\small lemma=$"$što$"$ | word=$"$(me)|([mj]u)|(joj)|(i[hm])|(ga)|([nv]a[sm])|(se)$"$|}
%\noindent\texttt{\small(word=$"$(je)|(si)$"$ \&tag !=$"$V.*$"$) | (word=$"$[mt]i"\&tag!=$"$(Pp[12]-[sp]n) | (Pd-mpn)$"$)}
%\noindent\texttt{\small(|(word=$"$te$"$ \& tag!=$"$(Pd-[fm][sp][nga])|(Cc)$"$))]\{1,2\} within<s/>}

\noindent In our last study on infinitive complements in Croatian, we compared data from the Forum subcorpus of hrWaC with data from corpora of standard Croatian: Riznica and CNC. The latter corpus uses an older tag set, and for some reason does not allow comparably complex queries.\footnote{\textcolor{black}{For the tag set used in CNC visit} \url{http://nl.ijs.si/ME/V4/msd/html/msd-hr.html}.} Therefore, the results from CNC were obtained via a simplified procedure which involved multiple filtering. We first retrieved all instances of a given CTP in the present tense form with the query \texttt{[lemma=$"$CTP$"$ \& msd=$"$Vmip.*$"$]}. Within the results, we filtered out all instances containing a CL within seven words of the CTP. After that we identified instances of embedded complements which were no further than ten tokens after the CTP, and then excluded the occurrences of \textit{da} up to 10 tokens after the CTP to avoid the \textit{da}-complements. This simplified procedure of data collection with multiple filtering was also the reason why CNC was used only as a complementary source of standard Croatian.

\section{Choice of corpora}
\label{Choice of corpora}
A consequence of using big data is the necessity of relying on search engine efficiency and precision of queries. As explained in Chapter \ref{Corpora for Bosnian, Croatian and Serbian}, the next prerequisite of a good corpus after size is availability of a search mechanism. This boils down to a search engine, filtering options and extensive, precise morphosyntactic annotation of structures in the database.

Finally, as the focus of the study is not on one relatively homogeneous language, but on three closely related languages, the examined material should be comparable in at least some aspects, such as age of texts, size of data, and possibly text type. Bearing these factors in mind, in Section \ref{Clitic climbing in BCS} we conclude that the most suitable source of data for studying CC in BCS is, in our view, available web corpora. Therefore, in both studies we used the WaC family.

In Serbian, the \textit{da}\textsubscript{2}-complement is a construction that competes against infinitive complements. This is why in the case of \textit{da}\textsubscript{2}-complements of CTPs, CC and noCC structure variants  were retrieved from srWaC. Additionally, we used the Serbian version \citep{AdamovicovaVavrin20} of InterCorp \citep{CermakRosen12} (which has an identical tag set) to establish the set of matrix verbs with which this construction appears the most often.\footnote{A detailed discussion of what motivated this choice can be found in \citet*{JHK17a}.}

Our decision concerning the data source for CC out of infinitive complements was based on the relative frequency of infinitive complements in comparison to \textit{da}\textsubscript{2}-complements. Unlike in Serbian, in Croatian infinitive complements dominate with raising and subject control CTPs, and are possible to a certain extent even in the case of object control matrix predicates.\footnote{For more information on the difference between raising, subject and object control matrix predicates see Section \ref{The control vs raising distinction}.}  Therefore, the data were taken from hrWaC. Additionally, in the second corpus study we collected data from Riznica and CNC so that we could test whether diaphasic variation as factor has an impact on the frequency of CC. As explained in Section \ref{Clitic climbing in BCS}, in 2018 the accessibility of Riznica improved, which allowed us to use this corpus. However, as described in the previous section, CNC was treated only as a complementary source of standard Croatian, since the simplified queries with multiple filtering do not allow for full equivalence of queries. 

\section{Choice of matrix verbs}\label{Choice of matrix verbs}
\begin{sloppypar}
When studying \textit{da}\textsubscript{2}-constructions and infinitive complements, we first constructed CTP frequency lists from which we chose CTPs. In the chronologically first study, we distinguished only three types of CTPs, basing on the raising--control distinction existing in Czech for constraints on CC.\footnote{For more information on this topic see Section \ref{Constraints related to the raising-control distinction}.}
\end{sloppypar}

Each set of queries was performed separately for each matrix verb. One position in the query thus became more specific, which had a positive impact on the precision of queries.
 
In the case of \textit{da}\textsubscript{2}-constructions, the list of CTPs was based on the core version of the Serbian InterCorp \citep{CermakRosen12}, which contains only original Serbian literary works and is manually aligned. In order to retrieve the list of CTPs, we constructed four CQL queries based on the four possible variants given in Table \ref{T12.1}. We obtained a frequency list with the 42 matrix verbs from InterCorp which have \textit{da}-complements. From this list, we first removed matrix predicates which have \textit{da}\textsubscript{1} complements.\footnote{For more information on the difference between \textit{da}\textsubscript{1}- and \textit{da}\textsubscript{2}-complements see Section \ref{Types of complements}.} Next, we excluded two unwanted types of \textit{da}\textsubscript{2}-predicates: reflexive and polyfunctional. This allowed us to avoid impersonal and passive constructions in the case of reflexive verbs. We also wanted to avoid sentences with pseudo-twins, which usually lead to pseudodiaclisis or haplology/haplology of unlikes.\footnote{For more information on haplology and pseudodiaclisis see Sections \ref{Morphonological processes within the cluster} and \ref{Diaclisis and pseudodiaclisis}.} Polyfunctionality influences the type of complement on one hand and the syntactic type of the matrix on the other. Since CTPs such as \textit{znati} ‘know’, \textit{ht(j)eti} ‘want/will’ and \textit{morati} ‘must’ may take both \textit{da}\textsubscript{2}- and \textit{da}\textsubscript{1}-complements, they were not excluded in the first step with other CTPs which take \textit{da}\textsubscript{1}-complements. However, although they can take \textit{da}\textsubscript{2}-complements, due to their polyfunctionality we decided to exclude them in order to avoid excessive manual filtering of unwanted utterances with \textit{da}\textsubscript{1}-complements. Further, to eliminate polyfunctionality with respect to syntactic type, we avoided CTPs such as \textit{učiti}, which can mean both ‘learn’ and ‘teach’. Whereas in the former meaning it is a simple subject control predicate, in the latter it is an object control predicate.

In general, raising verbs are much more frequent than object control verbs. Therefore, we avoided the most frequent raising verbs and included two of their subtypes: modal and phasal. This resulted in 15 lemmata, 5 per each syntactic type. 

\largerpage
However, the number of object control predicates that met our query conditions in InterCorp was small and in comparison to raising and subject control verbs their frequencies were lower. The latter might be due to the fact that the population of object control lemmata is bigger than the one of raising lemmata, so each token appears with lower frequency. To increase the recall for object control verbs from srWaC we added two object control verbs which are not present in InterCorp, but are quite frequent in srWaC. The list of CTPs which emerged as a result of this procedure and which was used to collect data on CC out of \textit{da}\textsubscript{2}-complements for Chapter \ref{A corpus-based study on CC in da constructions and the raising-control distinction (Serbian)} is given in Table \ref{T12.2}.\footnote{Since Serbian orthography allows for both the ekavian (\textit{smeti}) and the ijekavian (\textit{smjeti} ‘be allowed’) pronunciation, we took this into account when querying srWaC.} Basing on the results of the four CQL queries for the variants from Table \ref{T12.1} in srWaC, we calculated the estimated frequencies of CTPs in srWaC.\footnote{For more information on calculating estimated frequencies in that study see Section \ref{Results:da}.}

\clearpage
\begin{table}
\caption[CTPs selected for study of CC out of \textit{da}\textsubscript{2}-complements]{CTPs selected for study of CC out of \textit{da}\textsubscript{2}-complements\label{T12.2}}

\begin{tabularx}{\textwidth}{Xllrrl}
\lsptoprule
No. & Verb & Translation & \multicolumn{2}{c}{Frequency of \textit{da}\textsubscript{2}}  & Syntactic type \\\cmidrule(lr){4-5}
    &      &             & InterCorp & srWaC\footnote{(estimated)}\\\midrule
1.&\textit{moći} &‘can’&69&37526&\\
2.&\textit{nastaviti}&‘continue’&2&1028&\\
3. &\textit{početi} &‘start’ &26 & 6546& raising\\
4. &\textit{prestati} &‘stop’ &5 &1203&\\
5. &\textit{sm(j)eti}  &‘be allowed’ &7 &2027&\\
\tablevspace
6.&\textit{nam(j)eravati}  &‘intend’ &3 &465&\\
7. &\textit{nastojati}  &‘strive’ &7 &721&simple\\
8. &\textit{pokušati}  &‘try’ &7 &4794&subject\\
9. &\textit{um(j)eti}  &‘be able to’ &8 &1209&control\\
10. &\textit{usp(j)eti}  &‘succeed’ &9 &4331&\\
\tablevspace
11. &\textit{dozvoliti}  &‘allow’ &7 &2528&\\
12. &\textit{narediti}  &‘order’ &5 &1174&object\\
13. &\textit{nat(j)erati}  &‘force’ &2 &502&control\\
14. &\textit{zamoliti}  &‘ask’ &3 &1584& with dative\\
15. &\textit{pustiti}  &‘let’ &3 &534&and accusative\\
16. &\textit{primorati}  &‘compel’ &0 &248&controllers\\
17. &\textit{pomoći}  &‘help’ &0 &331&\\
\lspbottomrule
\end{tabularx}
\end{table}


To prepare the study of infinitive complements presented in Chapter \ref{A corpus-based study on clitic climbing in infinitive complements in relation to the raising-control dichotomy and diaphasic variation (Croatian)} we used frequency lists from hrWaC.  Since Hansen, Kolaković \& Jurkiewicz-Rohr\-ba\-cher (\citeyear{HKJ18}: 266) suggested that the reflexivity of the CTP might be a constraint on CC out of stacked infinitives, we decided to include a group of reflexive subject control CTPs in that study. The same CTPs, presented in Table \ref{T12.3}, were also used  in our psycholinguistic experiment (see Chapter \ref{Experimental study on constraints on clitic climbing out of infinitive complements}) for easier comparison of the results of the two studies.

\begin{table}[t]
\caption{CTPs selected for study of CC out of infinitive complements. Lemma frequency is taken from hrWaC v2.2 and expressed per million.\label{T12.3}}
\begin{tabular}{lllrl}
\lsptoprule
No. & Verb & Translation & Frequency/mil. words & Syntactic type\\\midrule
1.&\textit{moći} &‘can’&4056.5&\\
2.&\textit{trebati}&‘have to’&1761.4&\\
3. &\textit{morati}&‘must’ &1224.8&\\
4. &\textit{smjeti}&‘be allowed’ &208.0&\\
5. &\textit{počinjati}&‘start’ &125.0 &raising\\
6.&\textit{kretati}&‘go/start’ &101.2 &\\
7. &\textit{nastavljati}&‘continue’ &77.7 &\\
8. &\textit{prestajati}&‘stop’ &24.9  &\\
\tablevspace
9. &\textit{znati}&‘know/can’ &1584.6&\\
10. &\textit{željeti}&‘want/will’&859.1&\\
11. &\textit{pokušavati}&‘try’ &139.6&simple\\
12. &\textit{planirati}&‘plan’ &105.7 &subject\\
13. &\textit{nastojati}&‘strive’ &76.2 &control\\
14. &\textit{odlučivati}&‘decide’ &54.9  &\\
15. &\textit{odbijati}&‘refuse’ &32.6  &\\
16. &\textit{uspijevati}&‘succeed’ &31.9  &\\
\tablevspace
17. &\textit{bojati se}&‘be afraid’ &106.0 &\\
18. &\textit{sjetiti se}&‘remember’ &79.1  &\\
19. &\textit{truditi se}&‘try’ &53.3  &reflexive\\
20. &\textit{sramiti se}&‘be ashamed’ &12.4  &subject\\
21. &\textit{usuđivati se}&‘dare’ &5.6 &control\\
22. &\textit{stidjeti se}&‘be ashamed’ &4.2  &\\
23. &\textit{libiti se}&‘hesitate’ &3.0  &\\
24. &\textit{ustručavati se}&‘hesitate’ &2.5 &\\
\lspbottomrule
\end{tabular}
\end{table}


The list of all verbs which have an infinitive as a complement was extracted from hrWaC v2.2 in three steps. First, we applied the query \texttt{[tag=$"$Vm.*$"$]}, which let us find all examples with a main verb in all their possible forms. After that, the Filter function was applied. We used the query \texttt{[tag=$"$V.n"]} as a positive filter, which allowed us to extract only those examples with the main verbs which have an infinitive to their right (within 1 to 5 positions). In the next step we applied the Frequency function in NoSketchEngine and sorted the verbs according to their lemma form. The list obtained was downloaded as a .txt file, and opened for further editing in Excel. Next to each lemma we noted the predicate type, with three major groups: raising, subject control and object control. In the next step we classified subject control predicates into two further groups: simple subject control predicates such as \textit{planirati} ‘plan’ and reflexive subject control predicates such as \textit{bojati se} ‘be afraid’, and formed separate lists of predicates according to this classification.\footnote{The same procedure applies also to object control predicates which were later subclassified to object control predicates into four groups according to their controllers: object control predicates with pronominal controllers in dative and accusative and object control predicates with \textsc{refl}\textsubscript{2nd} controllers \textit{se} and \textit{si}. The predicates from the four object control groups are not listed in this chapter because they were not used in the study presented in Chapter \ref{A corpus-based study on clitic climbing in infinitive complements in relation to the raising-control dichotomy and diaphasic variation (Croatian)}. They can be found in Section \ref{Selection of matrix verbs}.}

\section{Data collection}
\label{Data collection Ch 12}

Having designed the CTP list and the queries we proceeded to data collection. Due to processing problems arising from recall and precision, and since manual revision of all retrieved examples would have exceed our human capacities, we decided to work with random samples of maximally 100 examples per structure variant per CTP.\footnote{In the case of less frequent simple and reflexive subject control predicates when entered query resulted in less than 100 examples all retrieved examples were checked manually.} Maximally one hit from one web page or text was taken for the sample. To achieve this, we first applied the 1\textsuperscript{st} hit in doc function of NoSketchEngine  and then its Sample function. The samples were downloaded as .txt files and revised manually in Excel. The clean and manually revised data were then used in the analyses described in the next two chapters.
