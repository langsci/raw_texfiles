
\chapter[Experimental study on clitic climbing out of infinitive complements]
        {Experimental study on constraints on clitic climbing out of infinitive complements (Croatian)}
\label{Experimental study on constraints on clitic climbing out of infinitive complements}

\largerpage
\section{Introduction}

As we have already pointed out, some of the data on CLs based on linguists’ informal judgments have turned out to be flawed.\footnote{For more information see Chapters \ref{Empirical approach to clitics in BCS}, \ref{Parameters of variation: conclusions}, \ref{Constraints on clitic climbing in Czech compared to Bosnian, Croatian and Serbian (theory and observations)}, and \ref{A corpus-based study on CC in da constructions and the raising-control distinction (Serbian)}.} Therefore our goal is to provide data which do not suffer from bias, unreliability, and narrowness by testing a part of the constraints on CC previously discussed in the syntactic literature.\footnote{We warn our readers that the reception of this chapter requires an acquaintance with various phenomena which are closely related to CC. Therefore, we advise our readers to at least become closely acquainted with the contents of Chapters \ref{Our terms and concepts}, \ref{Empirical approach to clitics in BCS}, \ref{Approaches to clitic climbing}, and \ref{Constraints on clitic climbing in Czech compared to Bosnian, Croatian and Serbian (theory and observations)} before reading this chapter. Ideally, one should have first read all the chapters dedicated to CC in this part \ref{part3} of the book before reading this chapter.} To supplement the results of our corpus linguistic studies on CC, we decided to broaden the available information on CC out of infinitive complements to include empirical data collected through acceptability judgment tasks.\footnote{For more information on the corpora and queries used in \textcolor{black}{our corpus linguistic studies on CC} see \textcolor{black}{Sections \ref{Clitic climbing in BCS} and \ref{Operationalising the constructions in question}. The results are presented in \citet*{HKJ18}, Chapters \ref{A corpus-based study on CC in da constructions and the raising-control distinction (Serbian)}, and \ref{A corpus-based study on clitic climbing in infinitive complements in relation to the raising-control dichotomy and diaphasic variation (Croatian)}.} }

This study complements our investigations described in the previous chapters, in accordance with the principle of triangulation of methods described in Section \ref{Triangulation of methods}. As pointed out, we follow the scheme: intuition/theory – observation – experiment. Chapter \ref{Constraints on clitic climbing in Czech compared to Bosnian, Croatian and Serbian (theory and observations)} presents the first step in this procedure, in which we give an exhaustive account of the constraints scattered across the literature, and pretest them for BCS by retrieving naturally occurring constructions and performing informal acceptability judgment tasks on them. We further tested some of the constraints on CC mentioned in that chapter in more exhaustive corpus linguistic studies presented in \citet*{HKJ18}, Chapters \ref{A corpus-based study on CC in da constructions and the raising-control distinction (Serbian)}, and \ref{A corpus-based study on clitic climbing in infinitive complements in relation to the raising-control dichotomy and diaphasic variation (Croatian)}. These studies belong to the second step. We are aware that corpus studies cannot provide negative evidence and that control over influencing factors in corpus studies is limited.\footnote{For more information on the drawbacks of corpus studies see \ref{Limitations of corpus linguistics}.} To some extent, we can overcome these problems with experimental manipulation and presentation of stimuli. This is the third step in our triangulation of methods. The acceptability judgment experiment allow us to test the hypotheses formulated during the intuition/theory and observation (corpus) steps with a high level of control over individual factors.

As mentioned in Section \ref{Pros and cons of judgment data}, there are two main advantages of judgment experiments. First, they can provide negative data and data which cannot be collected otherwise. In other words, introspection experiments such as acceptability judgments make possible the investigation of rare phenomena that fail to appear even in a very large corpus (such as web corpus). Low acceptability of a structure is considered negative evidence, i.e., it indirectly indicates that such a structure is very probably not used by native speakers. Second, if the test is designed properly, judgment data have internal validity, i.e. these kinds of studies allow unambiguous causal inferences. 

In what follows we describe the process of systematic collection of data which fulfil all the requirements of inferential statistical methods. This allows more robust generalisations on some of the constraints on CC. In Section \ref{RQ:16} we generate research questions concerning CC basing on constraints previously put forward in this book. Section \ref{The test set-up} brings exhaustive information on the test set-up: selection of matrix verbs (CTPs), test design, production of stimuli, and participants. The experimental procedure together with data preprocessing is explained in Section \ref{Data analysis}. Our results with respect to each research question are thoroughly discussed in Sections \ref{Results:16} and \ref{Discussion:16}. In Section \ref{Reaction time analysis} we analyse the reaction time, our control measure, for accepted sentences. Section \ref{Conclusions:16} puts forward an overview of the results summed up in general conclusions about CC in BCS.

\largerpage
\section{Research questions}
\label{RQ:16}

Building on the previous research on CC in Czech and BCS summarised in Chapter \ref{Constraints on clitic climbing in Czech compared to Bosnian, Croatian and Serbian (theory and observations)}, we turn to the present study, in which we further explore the impact of the raising--control distinction and selected mechanisms mentioned in Sections \ref{Constraints related to the raising-control distinction} and \ref{Constraints related to mixed clitic clusters}.\footnote{We exclude the constraint tightly connected to object control and animacy of the CL referent described in Section \ref{Object control and animacy of the referent of the clitic}, which lies beyond the scope of this study. However, bearing in mind that this factor may be important, we kept it constant through all experimental situations.}\textsuperscript{,}\footnote{For basic information on different predicate types with respect to the raising--control dichotomy see Section \ref{The control vs raising distinction}.}

Our first research question addresses the raising--control distinction, which has been reported as crucial for CC in Czech. Our next six research questions concern fine differences between CTP (sub)types. The last two research questions address the type and case of the infinitive CL. Some of those differences have only been discussed in the literature on CC in Czech, while others have been partially addressed in some of our papers or in other chapters of this book. 

In Chapter \ref{A corpus-based study on CC in da constructions and the raising-control distinction (Serbian)} we empirically show that the raising--control dichotomy plays an important role in CC out of \textit{da}\textsubscript{2}-complements in Serbian.\footnote{For more information on \textit{da}-complements see Section \ref{Types of complements}.} We formulate our first research question with the aim of experimentally investigating whether matrix predicate types are relevant also with respect to CC out of infinitive complements.

To the best of our knowledge, we are the first to compare differences in CC rates of simple subject control predicates on the one hand and reflexive subject control predicates on the other in corpora of standard and colloquial Croatian (\citealt[cf.][]{KJH19}, see Chapter \ref{A corpus-based study on clitic climbing in infinitive complements in relation to the raising-control dichotomy and diaphasic variation (Croatian)}). Moreover, the study by \citet*[][]{HKJ18} on CC out of stacked infinitives also showed that reflexivity of the matrix predicate influences CC. Our second research question is intended to test differences in CC rates of simple subject and reflexive subject control predicates via acceptability judgment tasks. 

Authors working on CC in Czech mention the object control case constraint. Namely, object control predicates with a dative controller block only the climbing of dative pronominal CLs, while object control predicates with an accusative controller build even stronger barriers, and block not only the climbing of dative but also of accusative pronominal CLs. Since obtaining evidence from corpora is rather difficult and would require excessive manual filtering and checking, an acceptability judgment task seems to be the most suitable approach for examining this topic empirically. Our third research question thus addresses this constraint in the context of CC out of infinitive complements in Croatian.

Scholars who work on the object control case constraint on CC in Czech base their discussion only on object control predicates with pronominal or NP controllers. Since reflexivity is recognised as an important factor in CC and since object control predicates with a reflexive controller have not previously been included in the discussion of CC, we formulate this as our fourth research question. Moreover, we are the first to directly compare the behaviour of the two \textsc{refl\textsubscript{2nd}} CLs, \textit{se} and \textit{si}. 

The literature review shows that object control predicates with pronominal or NP controllers in the dative trigger restrictions only on the climbing of dative CLs. Furthermore, in this case reflexivity might be an important factor in limiting the range of CC. Our fifth research question involves a comparison of object control predicates with pronominal CL controllers in the dative and object control predicates with the reflexive controller \textit{si} in the dative. Analogously, the sixth research question is dedicated to a comparison of pronominal CL controllers in the accusative with the reflexive controller \textit{se}.

Next, reflexive subject control predicates and object control predicates with the \textsc{refl\textsubscript{2nd}} controller \textit{se} have not been compared with respect to CC in the previous works. Moreover, the mentioned reflexives are not only of different types (\textsc{refl\textsubscript{\textsc{lex}}} vs \textsc{refl\textsubscript{2nd}}), but they also appear with different matrix predicates (subject vs object control). Our seventh research question addresses CC in the context of these differences.\footnote{For more information on types of reflexives see Section \ref{Different types of reflexives}.}

The literature on CC in Czech indicates that the type of infinitive CL (pronominal vs reflexive) plays an important role in CC. It has been claimed that unlike pronominal CLs, reflexives cannot climb out of object-controlled infinitives.\footnote{For Czech examples see Section \ref{Object control reflexive constraint}.} These claims motivated us to formulate our eighth research question.

Scholars working on CC in Czech indicate that pronominal infinitive CL complements in the accusative are less restricted in the climbing than pronominal infinitive CL complements in the dative, provided that the matrix predicate is of the object control type.\footnote{For Czech examples see Section \ref{Object control constraint related to case}.} We address this in our ninth research question.

\begin{sloppypar}
\citet[][]{Junghanns02} and \citet[][]{Rosen14} investigate the problem of phonologically identical/different and morphologically identical/different CLs with different governors in respect of CC.\footnote{For more information and Czech examples see Section \ref{Pseudo-twins}.} \citet[][]{Rosen14} offers haplology as a solution to CC in such contexts.\footnote{For more information on haplology see Section \ref{Morphonological processes within the cluster}.} However, due to the design of our experiment we cannot test such sentences. Addressing this phenomenon properly would require systematic investigation of the factors that influence haplology, i.e. manipulating the position of CLs, understanding which of two CLs is being eliminated etc. As a consequence, the number of sentences on the list would increase beyond a size which is reasonable for participants. Therefore, we made an informed decision to leave haplology for separate, future research. 
\end{sloppypar}

Our set of research questions thus targets the following variables:

\begin{itemize}
\item type of matrix verb (including the reflexivity of the predicate),
\item number of CLs in a sentence,
\item type of infinitive CL,
\item case of infinitive CL,
\item position of the infinitive CL (CC vs no CC).
\end{itemize}

An exhaustive description of dependent and independent variables and their levels can be found in Sections \ref{Selection of matrix verbs} and \ref{Design}.

In this chapter we address the following nine research questions:

\begin{enumerate}[label=RQ\arabic*:,ref=RQ\arabic*]
\item\label{RQ1} Do raising, subject and object control matrix predicates behave the same with respect to CC out of their infinitive complements?
\item\label{RQ2} Do simple subject control predicates (such as \textit{planirati} ‘plan’) and reflexive subject control predicates (such as \textit{bojati se} ‘be afraid’) behave the same with respect to CC out of their infinitive complements?
\item\label{RQ3} Do object control predicates with a pronominal CL controller in the dative and those with a pronominal CL controller in the accusative behave the same with respect to CC out of their infinitive complements?
\item\label{RQ4} Do object control predicates with a \textsc{refl\textsubscript{2nd}} CL \textit{si} controller and those with a \textsc{refl\textsubscript{2nd}} CL \textit{se} controller behave the same with respect to CC out of their infinitive complements?
\item\label{RQ5} Do object control predicates with a pronominal CL controller in the dative and those with a \textsc{refl\textsubscript{2nd}} CL \textit{si} controller behave the same with respect to CC out of their infinitive complements? 
\item\label{RQ6} Do object control predicates with a pronominal CL controller in the accusative and those with a \textsc{refl\textsubscript{2nd}} CL \textit{se} controller behave the same with respect to CC out of their infinitive complements? 
\item\label{RQ7} Do reflexive subject control predicates with the reflexive CL \textit{se} and object control predicates with a \textsc{refl\textsubscript{2nd}} CL \textit{se} controller behave the same with respect to CC out of their infinitive complements? 
\item\label{RQ8} Do pronominal and reflexive (\textsc{refl\textsubscript{\textsc{lex}}} CL \textit{se} and \textsc{refl\textsubscript{2nd}} CL \textit{se} and \textit{si}) infinitive CLs behave the same with respect to CC if the type of matrix predicate is constant? 
\item\label{RQ9} Do dative and accusative infinitive CLs behave the same with respect to CC provided that the type of matrix predicate is constant? 
\end{enumerate}

\noindent These ten research questions are operationalised in the form of null hypotheses as follows:
\begin{enumerate}[label=H\textsubscript{0.\arabic*}:,ref=H\textsubscript{0.\arabic*}]\sloppy
\item\label{H1} Raising, subject and object control predicates behave the same with respect to CC out of their infinitive complements.
\item\label{H2} Simple subject control predicates (such as \textit{planirati} ‘plan’) and reflexive subject control predicates (such as \textit{bojati se} ‘be afraid’) behave the same with respect to CC out of their infinitive complements.
\item\label{H3} Object control predicates with a pronominal CL controller in the dative and those with a pronominal CL controller in the accusative behave the same with respect to CC out of their infinitive complements.
\item\label{H4} Object control predicates with a \textsc{refl\textsubscript{2nd}} CL \textit{si} controller and those with a \textsc{refl\textsubscript{2nd}} CL \textit{se} controller behave the same with respect to CC out of their infinitive complements.
\item\label{H5} Object control predicates with a pronominal CL controller in the dative and those with a \textsc{refl\textsubscript{2nd}} CL \textit{si} controller behave the same with respect to CC out of their infinitive complements.
\item\label{H6} Object control predicates with a pronominal CL controller in the accusative and those with a \textsc{refl\textsubscript{2nd}} CL \textit{se} controller behave the same with respect to CC out of their infinitive complements.
\item\label{H7} Reflexive subject control predicates with the reflexive CL \textit{se} and object control predicates with a \textsc{refl\textsubscript{2nd}} CL \textit{se} controller behave the same with respect to CC out of their infinitive complements.
\item\label{H8} Pronominal and reflexive (\textsc{refl\textsubscript{\textsc{lex}}} CL \textit{se} and \textsc{refl\textsubscript{2nd}} CLs \textit{se} and \textit{si}) infinitive CLs behave the same with respect to CC if the type of matrix predicate is constant.
\item\label{H9} Dative and accusative infinitive CLs behave the same with respect to CC provided the type of matrix predicate is constant.
\end{enumerate}

\section{The test set-up}
\label{The test set-up}

Since the value of acceptability judgment data depends on the validity of the experimental procedures \citep[cf.][]{Myers17}, we decided to take all necessary steps in following all recommendations possible with respect to the test set-up, test design, stimulus production, selection of participants, and procedure. These aspects will be discussed in the following sections.

\subsection{Selection of matrix verbs}
\label{Selection of matrix verbs}

The research questions presented in the previous section form the main guidelines for designing the experiment and later for analysis of the data. The question of potential constraints is explored through sentence processing, i.e. an acceptability judgment task.\footnote{An explanation of why this and not some other psycholinguistic test was chosen can be found in Section \ref{Experiments chosen for our study}.}

Each sentence is a carefully developed stimulus in the experiment. Basing on the research questions from Section \ref{RQ:16} we will now discuss the elements which the stimuli must contain. For the raising--control constraint the following three major predicate types are used in the stimuli: raising (e.g. \textit{moći} ‘can’), subject (e.g. \textit{pokušavati} ‘try’) and object control (e.g. \textit{pomagati} ‘help’). Furthermore, since we investigate the role reflexivity plays in CC, the latter two groups must be further divided.\footnote{For more information on the role of reflexivity see Sections \ref{Infinitives as complements of agreeing predicative adjectives}, \ref{Infinitives as complements of non-agreeing predicatives}, \ref{Object control reflexive constraint}, \ref{Phonologically identical pronominal and reflexive clitics with different governors}, \ref{Morphologically different clitics with similar syntactic function and different governors}, the results of our studies presented in \citet*{HKJ18}, and in Chapter \ref{A corpus-based study on clitic climbing in infinitive complements in relation to the raising-control dichotomy and diaphasic variation (Croatian)}.} Thus, in the subject control group we have:

\begin{itemize}
\item simple subject control verbs (e.g. \textit{planirati} ‘plan’),
\item reflexive subject control verbs, i.e. verbs with the \textsc{refl\textsubscript{\textsc{lex}}} CL \textit{se} (e.g. \textit{bojati se} ‘be afraid’).
\end{itemize}

The object control group includes predicates which have a pronominal CL as a controller and those which have a reflexive CL as a controller. Since reflexivity and case of the matrix complement (i.e. controller) are addressed, the group of object control predicates is divided into four subgroups: 

\begin{itemize}
\item object control matrix predicates with a pronominal CL controller in the dative (e.g. \textit{pomagati} ‘help’),
\item object control matrix predicates with a pronominal CL controller in the accusative (e.g. \textit{poticati} ‘encourage’),
\item object control matrix predicates with the \textsc{refl\textsubscript{2nd}} CL \textit{si} controller (e.g. \textit{dozvoljavati si} ‘allow oneself’),
\item object control matrix predicates with the \textsc{refl\textsubscript{2nd}} CL \textit{se} controller (e.g. \textit{prisiljavati se} ‘force oneself’).
\end{itemize}

Summing up, the following seven types of matrix verbs are distinguished for the purpose of stimulus preparation: 

\begin{itemize}
\item raising,
\item simple subject control verbs,
\item reflexive subject control verbs, i.e. verbs with the \textsc{refl\textsubscript{\textsc{lex}}} CL \textit{se},
\item object control verbs with a pronominal CL controller in the dative,
\item object control verbs with a pronominal CL controller in the accusative,
\item object control verbs with the \textsc{refl\textsubscript{2nd}} CL \textit{si} controller,
\item object control verbs with the \textsc{refl\textsubscript{2nd}} CL \textit{se} controller. 
\end{itemize}

We chose the verbs according to the procedure described in Section \ref{Choice of matrix verbs}.
Since the use of tenses other than the present tense implies the use of auxiliary CLs in the matrix clause, we constructed the stimuli using matrix predicates in the present tense only.\footnote{In BCS there are other simple tenses besides the present tense, such as the aorist and imperfect. However, their usage in everyday language is stylistically restricted and it would not make much sense to construct stimuli for acceptability judgment tasks with them. In fact, it might even be dangerous, since those tenses could influence the evaluation of the stimuli.} We avoided stimuli with auxiliary CLs since we are not sure of their impact on CC. Further we narrowed our list down to imperfective verbs only.\footnote{The use of present tense as actual present in simple and main clauses requires imperfective aspect. In a few cases we used perfective verbs, but we then used temporal adverbials implying habituality (see \cite{Dickey00} on the aspect of BCS habituals).} In the case of verbs with the same stem but different prefixes such as \textit{po}-, \textit{za}-, \textit{od}- in \textit{počinjati}, \textit{započinjati}, \textit{otpočinjati} ‘begin’, we take the one with the least complex lexical meaning. In the case of \textit{započinjati} and \textit{otpočinjati} prefixes put additional emphasis on the beginning, that is, on the first phase of the situation expressed by those verbs. Therefore, we decided on the most neutral variant \textit{počinjati}. That was usually also the most frequent variant. The group of raising CTPs is very small and contains only imperfective predicates with obligatory raising of the subject (i.e. modal and phasal verbs). Its eight members are listed in Table \ref{T16.1}.


\begin{table}
\caption{Raising predicates selected for the acceptability judgment experiment. The frequencies of the lemmas are taken from hrWaC v2.2 and expressed per million words.\label{T16.1}}
\begin{tabular}{lllr}
\lsptoprule
Nr.&Verb&Translation&Frequency\\\midrule
1.& \textit{moći} & ‘can’ &4056.5 \\
2.& \textit{trebati} & ‘have to’ & 1761.4 \\
3.& \textit{morati} & ‘must’ & 1224.8 \\
4.& \textit{smjeti} & \mbox{‘be allowed’} & 208.0 \\
5.& \textit{počinjati} & ‘start’ & 125.0 \\
6.& \textit{kretati} & \mbox{‘go/start’} & 101.2 \\
7.& \textit{nastavljati} & \mbox{‘continue’} & 77.7 \\
8.& \textit{prestajati} & ‘stop’ & 24.9 \\
\lspbottomrule
\end{tabular}
\end{table}


During the preparatory phase various decisions had to be taken. First, in all seven groups of predicates the same number of matrix verbs had to be selected. Since the number of raising predicates extracted to design stimuli on the first experimental list was eight, it defined the maximal number of predicates on each of the other six experimental lists.

It is commonly known from psycholinguistic studies that the frequency of a word has a wide impact (\citealt[][]{BMR16}, \citealt{BMK18}). Therefore, in an ideally designed experiment the frequencies of selected verbs should match across different predicate type groups. Unfortunately in our case this was not possible, since subject control predicates and object control predicates which can take infinitive complements are much less frequent than raising predicates.\footnote{For more information on complement types in BCS, see Section \ref{Types of complements}. Most object control predicates in Croatian actually take \textit{da}\textsubscript{2}-complements.} Therefore, the most frequent verbs are usually taken. We justify our decision below. 

Table \ref{T16.2} shows the subject control predicates selected for the experiment; non-reflexive on the left side of the table and reflexive on the right. 

\begin{table}
\caption{Subject control predicates selected for the acceptability judgment experiment\label{T16.2}}
\fittable{\begin{tabular}{l@{~}llrllr}
\lsptoprule
Nr.&Verb&Translation&Freq.&Verb&Translation&Freq.\\\midrule
1.& \textit{znati}& ‘know/can’ &1584.6 & \textit{bojati se}& ‘be afraid’ &106.0 \\
2. &\textit{željeti}& ‘want/will’ &859.1 & \textit{sjetiti se}& ‘remember’& 79.1 \\
3.& \textit{pokušavati} &‘try’ &139.6 & \textit{truditi se}& ‘try’ &53.3 \\
4.& \textit{planirati}& ‘plan’ &105.7 & \textit{sramiti se}& ‘be ashamed’ &12.4 \\
5.& \textit{nastojati}& ‘strive’ &76.2 & \textit{usuđivati se}& ‘dare’& 5.6 \\
6.& \textit{odlučivati}& ‘decide’ &54.9 & \textit{stidjeti se}& ‘be ashamed’& 4.2 \\
7.& \textit{odbijati}& ‘refuse’ &32.6 & \textit{libiti se}& ‘hesitate’ &3.0 \\
8.& \textit{uspijevati}& ‘succeed’ &31.9 & \textit{ustručavati se}& ‘hesitate’ &2.5 \\
\lspbottomrule
\end{tabular}}
\end{table}

\begin{table}[b]
\caption{Object control predicates selected for the acceptability judgment experiment\label{T16.3}}
\begin{tabular}{l@{~}llrllr}
\lsptoprule
Nr.&Verb (\textsc{dat})&Translation&Freq.&Verb (\textsc{acc})&Translation&Freq.\\\midrule
1. &\textit{pomagati}& ‘help’ &112.5& \textit{učiti} &‘teach’ & 104.4 \\
2. &\textit{omogućivati}& ‘enable’& 64.7& \textit{poticati}& ‘encourage’ &59.2 \\
3. &\textit{savjetovati}& ‘advise’& 38.8& \textit{tjerati} &‘force’& 29.9 \\
4. &\textit{preporučivati}& ‘recommend’& 34.5& \textit{puštati}& ‘let/allow’ &26.4 \\
5. &\textit{dopuštati}& ‘allow’ &33.2& \textit{obvezivati}& ‘oblige’& 9.9 \\
6. &\textit{dozvoljavati} &‘allow’& 19.8& \textit{prisiljavati} &‘force’ &6.2 \\
7. &\textit{zabranjivati}& ‘forbid’ &12.1 &\textit{požurivati}& ‘hurry up’ &1.2\\
8. &\textit{naređivati}& ‘order’& 3.5 &\textit{primoravati} &‘compel’ &0.6\\
\lspbottomrule
\end{tabular}
\end{table}

Some frequent subject control verbs do not appear on the list. Although \textit{htjeti} ‘will/ want’ is the most frequent subject control verb, we exclude it from the study since it is predominantly used as a future tense auxiliary. Instead we take \textit{željeti} ‘want/wish’. Since the list of potential CTP candidates is long, we avoid partial synonyms. For example, the choice of \textit{planirati} rules out \textit{namjeravati} ‘intend/ plan’ as these verbs have very similar meanings. Furthermore, we exclude the quite frequent verb \textit{misliti} ‘intend’, since it is more common with \textit{da}\textsubscript{1}-complements in its other meaning ‘think’. We also exclude all verbs of motion such as \textit{ići} ‘go’, \textit{dolaziti} ‘come’ and \textit{ostajati} ‘stay’ as they are often used with final subordinate clauses and with a \textit{da} complementiser. 

Table \ref{T16.3} shows the object control predicates selected for the experiment.\footnote{In certain contexts, the verb \textit{učiti} in table \ref{T16.3} can mean ‘learn’. The problem of polysemy was solved through context, i.e. from a given sentence it was clear that the meaning ‘teach’ was employed.} The list includes both object control predicates with a dative controller (left side of the table) and object control predicates with an accusative controller (right side of the table). 



The object control predicates with a \textsc{refl\textsubscript{2nd}} controller selected for the experiment are presented in Tables \ref{T16.4} and \ref{T16.5}.\footnote{In Tables \ref{T16.2}--\ref{T16.5}, frequency refers to the frequency of lemmas without reflexive markers since in hrWaC v2.2 all reflexives are annotated separately and it is not possible to extract the exact data on the frequency of the lemma with a reflexive. The high frequency of \textit{braniti} is a result of homonymy. There are actually two lemmas: \textit{braniti} ‘defend’ and \textit{braniti} ‘forbid.’} The former contains object control predicates with the \textsc{refl\textsubscript{2nd}} controller \textit{si}, while the latter is object control predicates with the \textsc{refl\textsubscript{2nd}} controller \textit{se}. 

\begin{table}
\caption{Object control predicates with the \textsc{refl\textsubscript{2nd}} CL \textit{si} controller selected for the acceptability judgment experiment\label{T16.4}}
\begin{tabular}{l@{~}llr}
\lsptoprule
Nr.& Verb (\textsc{dat}) &Translation& Frequency\\
\midrule
1. &\textit{braniti si}& ‘forbid oneself’& 81.6 \\
2. &\textit{omogućivati si}&‘enable oneself’& 64.7\\
3. &\textit{dopuštati si}& ‘allow oneself’& 33.2\\
4. &\textit{priuštiti si}& ‘allow oneself’& 21.5\\
5. &\textit{dozvoljavati si}& ‘allow oneself’ &19.8\\
6. &\textit{olakšavati si }&‘make something easier for oneself’& 11.7\\
7. &\textit{naređivati si }&‘assign oneself’ &3.5\\
8. &\textit{uskraćivati si}& ‘deprive oneself’ &3.3\\
\lspbottomrule
\end{tabular}
\end{table}


Object control predicates with the \textsc{refl\textsubscript{2nd}} CL \textit{si} controller occur quite rarely. Although some of them might sound slightly odd, like \textit{naređivati si} ‘assign oneself’ and \textit{braniti si} ‘forbid oneself’, all are attested in corpora. A similar problem is encountered on the list of object control predicates with a \textsc{refl\textsubscript{2nd}} CL \textit{se} controller, although most of the verbs in Table \ref{T16.5} are used in everyday language. Only \textit{ovlašćivati se} ‘authorise’ is an exception: it is typical of the administrative register. Nevertheless, also the verbs in this group are all attested in corpora.

\begin{table}
\caption{Object control predicates with a \textsc{refl\textsubscript{2nd}} CL \textit{se} controller selected for the acceptability judgment experiment\label{T16.5}}
\begin{tabular}{l@{~}llr}
\lsptoprule
Nr.& Verb (\textsc{acc}) &Translation& Frequency\\\midrule
1. &\textit{učiti se}& ‘teach oneself’& 104.4\\
2. &\textit{poticati se}& ‘encourage oneself’& 59.2\\
3. &\textit{spremati se}& ‘prepare oneself’ &39.6\\
4. &\textit{prisiljavati se}& ‘force oneself’ &6.2\\
5. &\textit{ohrabrivati se}& ‘encourage oneself’& 5.1\\
6. &\textit{navikavati se}& ‘accustom oneself’& 1.7\\
7. &\textit{ovlašćivati se}& ‘authorise oneself’& 0.7\\
8. &\textit{primoravati se}& ‘compel oneself’ &0.6\\
\lspbottomrule
\end{tabular}
\end{table}

\subsection{Experiment design}\label{Design}

In stimulus design, a fully crossed factorial design is usually aimed for. This implies that each level of an independent variable is crossed with each level of other independent variables. Such a design provides the highest level of methodological rigour. However, since our stimuli are extracted from natural language, applying a fully crossed design was not possible. In other words, certain combinations of factor levels do not exist in language, or are too rare for enough examples to be found and build a fully fledged list of stimuli. Although it is sometimes possible to artificially construct critical examples, this is not the rule but rather an exception. 

For example, the variables of highest interest to us are type of matrix verb (raising, subject control, object control), number of CLs (one, two), type of matrix and infinitive CL (personal pronoun, \textsc{refl\textsubscript{2nd}}, \textsc{refl\textsubscript{\textsc{lex}}}), and case of matrix and infinitive CL (dative, accusative). However, achieving a fully crossed factorial design with all these variables is not possible, as for example due to their argument structure matrix verbs of the raising type do not have CL complements. Therefore, they do not appear in constructions with two CLs. In contrast, object control matrix verbs always form such constructions as they have their own CL complements (i.e. controllers) and their infinitive complements also have CL complements.\footnote{We are aware that object control predicates can have a NP instead of CL complement/controller. But the fact that they have one more complement than raising predicates still remains. } Similarly, some subject control matrix predicates have the \textsc{refl\textsubscript{\textsc{lex}}} CL \textit{se}, whereas object control matrix verbs have either pronominal or \textsc{refl\textsubscript{2nd}} complements (i.e. controllers).\footnote{We are aware that some subject control predicates like \textit{obećati} ‘promise’ are polyvalent and that they can have a NP or CL complement in the dative, for more information see Section \ref{The control vs raising distinction}. But the fact that subject control predicates differ from the raising and object control predicates with respect to their complements still remains.} Furthermore, even if all the combinations were present in language, given the number of our variables of interest, permuting them would give us a very high number of stimuli. Such a large number of sentences would be too demanding for experiment participants. This would not only decrease the reliability of the observed data (given the fatigue level), but also present ethical issues. Therefore, our design was a compromise between methodological rigour, availability of language material, and operational capabilities of participants. At the same time, we could also call it an optimal solution for tackling the research questions given the language structure and operational capabilities of participants.

With all this in mind, we developed a design which enables us to examine the relationship between the dependent variable (sentence acceptability) and the four independent variables mentioned at the end of Section \ref{RQ:16}. These are summarised in Table \ref{T16.X1}.

\begin{sidewaystable}
\caption{List of variables\label{T16.X1}}
\begin{tabular}{llll}
\lsptoprule
Variable& Type& Levels& Class\\\midrule
sentence acceptability& binary& 0 (unacceptable) & dependent \\
 &&1 (acceptable) & \\
\tablevspace
 && raising & independent  \\
 && simple subject control &   \\
 && reflexive subject control & \\
 && object control with pronominal CL &   \\
\tablevspace
predicate type& categorical& controller in dative & independent \\
 && object control with pronominal CL &  \\
 && controller in accusative &  \\
 && object control with \textsc{refl\textsubscript{2nd}} CL \textit{si} controller &  \\
 && object control with \textsc{refl\textsubscript{2nd}} CL \textit{se} controller &  \\
\tablevspace
type of infinitive CL& categorical & pronominal & independent \\
 && \textsc{refl\textsubscript{2nd}} &  \\
 && \textsc{refl\textsubscript{\textsc{lex}}} &  \\
\tablevspace
case of infinitive CL& categorical & dative & independent \\
 && accusative &  \\
\tablevspace
CC& binary& 0 (CC absent) & independent \\
 && 1 (CC present) &  \\
\lspbottomrule
\end{tabular}
\end{sidewaystable}

In accordance with the explanations from Section \ref{Selection of matrix verbs}, the first independent variable is predicate type with seven levels (raising verbs, simple subject control verbs, reflexive subject control verbs with the \textsc{refl\textsubscript{\textsc{lex}}} CL \textit{se}, object control verbs with a pronominal CL controller in the dative, object control verbs with a pronominal controller in the accusative, object control verbs with the \textsc{refl\textsubscript{2nd}} CL \textit{se} controller, object control verbs with the \textsc{refl\textsubscript{2nd}} CL \textit{si} controller). For the reasons discussed above, this independent variable was introduced as a between-participants (different participants were presented with different predicate types) and between-items factor (as one verb cannot belong to multiple predicate types). 

The second independent variable was type of the infinitive CL, which had three levels (pronominal, \textsc{refl\textsubscript{2nd}}, \textsc{refl\textsubscript{\textsc{lex}}}). The third independent variable was case of the infinitive CL, with two levels (dative, accusative). This factor was nested in two levels of the second independent variable (pronominal and \textsc{refl\textsubscript{2nd}}), as \textsc{refl\textsubscript{\textsc{lex}}} does not have grammatical case. Within a given matrix verb type, both the second and the third factor were introduced as within-participant and between-item factors. 

Finally, we manipulated the position of the infinitive CL, i.e., we introduced the fourth independent variable, CC, which had two levels (CC present, CC absent). This variable was introduced as both a within-participant and a within-item factor. Given that the phenomenon of CC is central to our study, we found it crucial to allow for the comparison of acceptability scores of the same sentence in two conditions: with and without CC. We counterbalanced the position of the critical CL by applying the Latin square design, as we will describe in more detail in the next section.\footnote{The critical CL is the CL of interest, i.e. the infinitive CL complement whose climbing is being tested.}

In order to control for the effects of the additional variables that are not subject to manipulation in this research, we introduced some additional restrictions. First, we controlled for the animacy of the CL referents and constructed sentences from which it is clear that the CL referents are animate.\footnote{For more information on the importance of this factor see Section \ref{Object control and animacy of the referent of the clitic}.} Next, we controlled for the person of the critical pronominal CL and constructed only sentences with third person pronominal CLs.\footnote{For more information on the importance of this factor see Section \ref{Object control person-case constraint}.}\textsuperscript{,}\footnote{As the third person singular feminine accusative pronominal CL, we had both \textit{ju} and \textit{je} forms in our stimuli, because some speakers prefer \textit{ju} while others prefer \textit{je}. In this way we tried to avoid their personal preferences based on their dialects or idiolects affecting the rating of our stimuli. For the status of the third person singular accusative feminine CL \textit{ju} and \textit{je} in standard Croatian see Section \ref{Inventory of pronominal clitics in BCS standard varieties} and for its status in Štokavian dialects see Section \ref{Feminine pronominal clitics}.} Additionally, we controlled for the length of the sentences across conditions, grammatical number and gender of the CL where applicable, as we will describe in the next section.\footnote{This type of design on the one hand allows the researcher to test the effect of each independent variable separately and on the other it allows the researcher to look at possible interactions between the independent variables. For these reasons it is more cost-effective than conducting various separate experiments on each independent variable. In addition, using this type of design also allows the researcher to determine if the effect of one independent variable depends on the value of another independent variable \citep[cf.][121]{AGM13}.}\textsuperscript{,}\footnote{A design like ours with two or more independent variables (factors) is called a factorial design. One of the main advantages of such designs is that they help control for unintended differences between the conditions \citep[][14]{StoweKaan06}.} 

\subsection{Stimuli}
\label{Stimuli}
\subsubsection{Stimulus design}

The stimuli, that is, sentences evaluated in the experiment, were designed with the matrix verbs listed in Section \ref{Selection of matrix verbs} in the present tense and supplemented with infinitives which had pronominal and reflexive CLs as complements.\footnote{We prepared core elements of target sentences as \citet[][50]{Cowart97} recommends. } In contrast to matrix verbs, which were the independent variable of the greatest interest to us, infinitives were not treated as variables. Both the matrix predicate verbs and the infinitives were extracted from hrWaC v2.2 using CQL queries and the Frequency function. Whenever we were unable to find infinitives with a given pronominal or reflexive CL complement in a given case in hrWaC v2.2, we turned to the Institute of Croatian Language and Linguistics where an e-dictionary of verb valencies is being developed \citep[cf.][]{BBR17}. We paid a lot of attention to the creation of stimulus sentences, as it is well known that

\begin{quotation}
[\dots] an informant’s response to an individual sentence may be affected by many different lexical, syntactic, semantic and pragmatic factors, together with an assortment of extralinguistic influences that become haphazardly associated with linguistic materials and structures. \hfill \citep[][46]{Cowart97}
\end{quotation}


To deal with all confounding factors, scholars recommend paradigm-like token sets as a safe strategy. This ensures that all the abovementioned unwanted and hazardous factors are uniformly spread across all tested sentences. This in turn guarantees that the differences in ratings can be attributed exclusively to the phenomenon under investigation \citep[][13, 47, 52]{Cowart97}. Furthermore, we created multiple lexical encodings of each condition to minimise the effects of particular lexical items on the results, as recommended in the methodological literature \citep[cf.][39]{SchutzeSprouse13}.\footnote{We avoid the term “lexicalization” used by \citet[][39]{SchutzeSprouse13} due to its ambiguity.}

We created seven experimental lists, each containing only one matrix predicate subtype (raising, simple subject control, reflexive subject control, etc.). The structure of stimuli without CC (henceforth noCC stimuli) is presented in Table \ref{T16.X2}.\footnote{More examples of noCC stimulus sentences for each matrix predicate type can be found in the Appendix \ref{Appendix}.}\textsuperscript{,}\footnote{Here are the item translations from Table \ref{T16.X2}:
\begin{enumerate}[label=(I\thechapter.\arabic*)]
    \item[\refstepcounter{enumi}\REF{I16.1}]  ‘We are entirely stopping complaining about the bad company he keeps.’
	\item[\refstepcounter{enumi}\REF{I16.2}] ‘I am even trying to invite him to the monthly meetings.’
	\item[\refstepcounter{enumi}\REF{I16.3}] ‘I truly hesitate to please myself in every way.’
	\item[\refstepcounter{enumi}\REF{I16.4}] ‘You have been allowing him to hide from curious glances since always.’
	\item[\refstepcounter{enumi}\REF{I16.5}] ‘I visibly hurry her to voice her opinion on the presented suggestions.’
\end{enumerate}
}

\begin{table}
\caption[Structure of noCC stimuli]{Structure of noCC stimuli\label{T16.X2}}
\fittable{\begin{tabular}{N@{~}llllll} %column N is defined in localcommands to contain nothing but a counter that we can label. This requires some \multicolumn{1} tricks, but in the end it is faster and looks cleaner this way.
\lsptoprule
\multicolumn{1}{l}{}&\multicolumn{6}{c}{Position}\\
\cmidrule(lr){2-7}
\multicolumn{1}{l}{Item} & Adverb & CL\textsubscript{1} & Matrix.\textsc{prs}\textsubscript{1} & Infinitive\textsubscript{2} & CL\textsubscript{2} & PP Complement/Adjunct \\
\midrule
\label{I16.1} & Posve & NA & prestajemo\textsubscript{1} & prigovarati\textsubscript{2} & \textit{mu}\textsubscript{2} & zbog lošeg društva. \\
\label{I16.2} & Čak & NA & nastojim\textsubscript{1} & pozivati\textsubscript{2} & \textit{ga}\textsubscript{2} & na mjesečne sastanke. \\
\label{I16.3} & Zaista & se\textsubscript{1} & ustručavam\textsubscript{1} & ugađati\textsubscript{2} & \textit{si}\textsubscript{2} & u svakom pogledu. \\
\label{I16.4} & Oduvijek & \textit{mu}\textsubscript{1} & dozvoljavaš\textsubscript{1} & skrivati\textsubscript{2} & \textit{se}\textsubscript{2} & od znatiželjnih pogleda. \\
\label{I16.5} & Vidno & \textit{je}\textsubscript{1} & požurujem\textsubscript{1} & očitovati\textsubscript{2} & \textit{se}\textsubscript{2} & o iznesenim prijedlozima. \\
\lspbottomrule
\end{tabular}}
\end{table}

\begin{sloppypar}For each experimental list, eight different matrix verbs (see position Matrix.\textsc{prs}\textsubscript{1}) were used multiple times.\footnote{Lists with the eight verbs for each of the seven experimental lists can be found in Tables \ref{T16.1}--\ref{T16.5}.} Since two of our independent variables are critical (infinitive) CL type (pronominal vs reflexive) and case (dative vs accusative), in each of the seven experimental lists we had:\end{sloppypar}

\begin{itemize}
\item eight sentences with dative pronominal CLs, 
\item eight sentences with accusative pronominal CLs, 
\item eight sentences with the \textsc{refl\textsubscript{2nd}} CL \textit{si}, 
\item eight sentences the \textsc{refl\textsubscript{2nd}} CL \textit{se} and 
\item sixteen sentences with the \textsc{refl\textsubscript{\textsc{lex}}} CL \textit{se} (see position CL\textsubscript{2}). 
\end{itemize}

As governors, we used eight different infinitives (see Position Infinitive\textsubscript{2}) per critical CL subtype.\footnote{In other words, due to the recommendation to use different lexical materialisations as mentioned above, each of the eight accusative pronominal CLs was governed by a different infinitive. The same applies to dative pronominal CLs and to \textsc{refl\textsubscript{2nd}} and \textsc{refl\textsubscript{\textsc{lex}}} CLs, with the exception that the latter CL depended on 16 different infinitives.} Each sentence on an experimental list had a unique adverb at the beginning and a unique (prepositional) complement/adjunct at the end (see Positions Adverb and PP Complement/Adjunct). The first CL, which is generated by the matrix predicate, was not present on the first two experimental lists, on which we presented the raising and simple subject control predicates (see position CL\textsubscript{1}).\footnote{As we already pointed out in Section \ref{Design}, unlike reflexive subject control and object control predicates, raising and simple subject control predicates do not have their own CLs, as will become obvious from examples (\ref{(16.11a)}) and (\ref{(16.13a)}).}

As we explain in the section below, each sentence was rated in its CC and its noCC version. In the CC version, the critical CL (CL\textsubscript{2}) climbs and takes 2P directly following the adverb. If the matrix predicate has its own CL, the matrix CL (CL\textsubscript{1}) and critical CL (CL\textsubscript{2}) clusterise. CL\textsubscript{1} appears in the cluster first, and is followed by CL\textsubscript{2}.\footnote{The exceptions to that CL cluster sequence are examples in which CL\textsubscript{1} is reflexive (for instance in the case of reflexive subject control predicates or object control predicates with a \textsc{refl\textsubscript{2nd}} CL controller). In that case the pronominal CL\textsubscript{2} appears in the cluster first, and is then followed by the reflexive CL\textsubscript{1}. This was done in order to follow the patterns of CL ordering in a cluster – for the relative order of CLs in the CL cluster in standard BCS varieties see Section \ref{Clitic ordering within the cluster}. In sentences with two reflexive CLs, the order in the cluster was as usual: CL\textsubscript{1} followed by CL\textsubscript{2}.}

We now briefly present the stimuli. The comparison of stimuli is based on the different CL\textsubscript{2} subtypes. In the first two items, I16.1 and I16.2 presented in Table \ref{T16.X2}, the infinitive governs the third person pronominal CL in the dative and accusative, while in the second two items, I16.3 and I16.4, the infinitive governs the \textsc{refl\textsubscript{2nd}} CLs \textit{si} and \textit{se}. In the last item, I16.5, the \textsc{refl\textsubscript{\textsc{lex}}} CL \textit{se} is the critical CL. For presentation purposes we deliberately chose stimuli with matrix predicates which belong to different types to show how some of them have their own CL\textsubscript{1} (see items \ref{I16.1} and \ref{I16.2}), while others do not (see items \ref{I16.3}--\ref{I16.5}). 

\begin{table}[ht]
\caption[Comparison of noCC stimuli across seven experimental lists]{Comparison of noCC stimuli across seven experimental lists\label{T16.X3}}
\fittable{\begin{tabular}{N@{~}llllllll}
\lsptoprule
\multicolumn{1}{l}{}&\multicolumn{8}{c}{Position}\\
\cmidrule(lr){2-9}
\multicolumn{1}{l}{Item} & Adverb & CL\textsubscript{1} & Matrix.\textsc{prs}\textsubscript{1} & Infinitive\textsubscript{2} & CL\textsubscript{2} & \multicolumn{3}{l}{PP Complement/Adjunct} \\
\midrule
\label{I16.6} & Stoga & NA & krećem & pozivati & ga & na & mjesečne & sastanke.\\
\label{I16.7}& Čak & NA& nastojim & pozivati& ga & na & mjesečne & sastanke.\\
\label{I16.8}& Nekako & se & ustručavamo & pozivati & ga & na & mjesečne & sastanke.\\
\label{I16.9} & Uporno& mi & naređuju & pozivati & ga & na & mjesečne & sastanke.\\
\label{I16.10} & Javno& ih & obvezujem & pozivati & ga & na & mjesečne & sastanke.\\
\label{I16.11} & Ujedno & si & dopuštam & pozivati & ga & na & mjesečne & sastanke.\\
\label{I16.12} & Nevoljko & se & prisiljavamo & pozivati & ga & na & mjesečne & sastanke.\\
\lspbottomrule
\end{tabular}}
\end{table}

As can be seen in Table \ref{T16.X3} (compare items \ref{I16.6}--\ref{I16.12}) and Appendix \ref{Appendix}, sentences are designed in such a way that they differ only as to adverbs and matrix predicates (and consequently also as to matrix CLs if available), and at the same time they contain the same infinitives, critical CLs and PP complements/adjuncts.\footnote{\textcolor{black}{Here are the item translations from Table \ref{T16.X3}:}
\begin{enumerate}[label=(I\thechapter.\arabic*)] \setcounter{enumi}{5}
    \item[\refstepcounter{enumi}\REF{I16.6}] ‘Therefore, I am starting to invite him to the monthly meetings.’
	\item[\refstepcounter{enumi}\REF{I16.7}] ‘I am even trying to invite him to the monthly meetings.’
	\item[\refstepcounter{enumi}\REF{I16.8}] ‘We kind of hesitate to invite him to the monthly meetings.’
	\item[\refstepcounter{enumi}\REF{I16.9}] ‘They are persistently ordering me to invite him to the monthly meetings.’
	\item[\refstepcounter{enumi}\REF{I16.10}] ‘I publicly oblige them to invite him to the monthly meetings.’
	\item[\refstepcounter{enumi}\REF{I16.11}] ‘At the same time I am allowing myself to invite him to the monthly meetings.’
	\item[\refstepcounter{enumi}\REF{I16.12}] ‘We begrudgingly force ourselves to invite him to the monthly meetings.’
\end{enumerate}
} 

As we already pointed out in Section \ref{Different types of judgment tasks} it is important for participants to be exposed to polarised sentences, otherwise they will start to evaluate acceptable sentences as unacceptable. Therefore, in each experiment, besides the 48 target sentences, participants had to evaluate 48 target-like syntactically and morphologically ill-formed sentences. Those sentences were deliberately constructed via disruption of obvious grammatical rules unrelated to our study. Furthermore, the stimuli must be counterbalanced. 

\begin{quotation}
Counterbalancing aims to distribute both the idiosyncratic and the systematic structural effects that arise in a single sentence across the whole experiment in such a way that the systematic effects can be reliably discriminated from the background blur of idiosyncratic effects.\hfill \citep[93]{Cowart97}
\end{quotation}

The first rule of counterbalancing is that a participant is never to see a sentence twice, i.e. s/he is never exposed to more than one member of a token set \citep[][50f, 93]{Cowart97}.\footnote{This recommendation should be followed because the second (or any further) encounter with the same sentence will be influenced by the first one; this danger exists even in the case of similar sentences \citep[cf.][50]{Cowart97}.}\textsuperscript{,}\footnote{In our case a token set is one of the 48 target sentences structured as in Tables \ref{T16.X2} and \ref{T16.X3}. A member of a token set is the CC or the noCC version of a particular sentence. } Latin square design helped us fulfil this requirement, i.e., it enabled us to distribute items across participants’ lists properly (cf. \citealt[][49]{StoweKaan06}, \citealt[][121]{AGM13}). In our experiment, each list contains one sentence with each of the conditions, and no list contains more than one version of each sentence. Moreover, the application of Latin square design means that for each sentence, half of the participants saw a noCC version (like the one presented in (\ref{(16.8a)})), while the other half saw a CC version (like the one presented in (\ref{(16.8b)})), and that each participant saw both CC and noCC sentences.
%\footnote{Henceforth in this chapter the Latin letter \textit{a} is assigned to an example stands for the noCC version, while the letter \textit{b} indicates the CC version.}

\begin{exe}\ex
\begin{xlist}
\ex\label{(16.8a)}
\gll Potpuno \textbf{ih}\textsubscript{1} primoravam\textsubscript{1} angažirati\textsubscript{2} \textbf{je}\textsubscript{2} u političkoj kampanji.\\
 categorically them.\textsc{acc} compel.1\textsc{prs} hire.\textsc{inf} her.\textsc{acc} in political campaign \\
\ex\label{(16.8b)}
\gll Potpuno \textbf{ih}\textsubscript{1} \textbf{je}\textsubscript{2} primoravam\textsubscript{1} angažirati\textsubscript{2} u političkoj kampanji. \\
 categorically them.\textsc{acc} her.\textsc{acc} compel.1\textsc{prs} hire.\textsc{inf} in political campaign \\
\end{xlist}
\glt ‘I am categorically compelling them to hire her in the political campaign.’
\end{exe}

\noindent The second rule of counterbalancing is that one should obtain a subject’s judgments on all the relevant factor combinations \citep[cf.][50, 93]{Cowart97}. The third rule of counterbalancing is that every sentence in every token set should be judged by a participant \citep[][93]{Cowart97}. 

Since participants can form implicit hypotheses on the aim of the experiment, which could potentially distort or affect their judgments \citep[cf.][51f, 93f]{Cowart97}, syntactically and morphologically well- and ill-formed fillers were included in the experiment. \citet[][39]{SchutzeSprouse13} name two more roles of fillers in addition to this important one. First, they can help us to ensure that all the possible responses are used about equally often. Second, they can be used to collect data for other research questions. For the latter reason, we used target sentences from the research of Dóra Vuk on agreement (80 grammatically well-former and 65 syntactically and morphologically ill-formed sentences) as fillers.\footnote{Dóra Vuk’s PhD thesis \textit{Kongruenz in der kroatischen Herkunftssprache in Ungarn und Österreich} ‘Agreement in Croatian heritage language in Hungary and Austria’ was financially supported by the Graduate School for East and Southeast European Studies. In her research she concentrated on gender agreement in conjoined phrases and its realisation in adjectives in nominal predicate and in past participle. Her sentences were also constructed for use in an acceptability judgment task.} Since Vuk could not provide us with enough filler sentences, they were supplemented by an additional 20 syntactically and morphologically well-formed sentences from hrWaC and an additional 35 syntactically and morphologically ill-formed sentences, which had the structure of her target sentences (see Cowart's \citeyear[][52]{Cowart97} recommendations). The latter were obtained via permutation of sentences attested in the aforementioned corpus. The filler-stimulus ratio was 2:1. Below are examples of syntactically and morphologically well-formed (\ref{(16.9)}) and ill-formed (\ref{(16.10)}) filler sentences.

\begin{exe}\ex\label{(16.9)}
\gll Sestra i mama \textbf{su} slične.\\
 sister and mother be.3\textsc{pl} similar\\
\glt ‘Sister and mother are similar.’

\ex[*]{\label{(16.10)}
\gll Ruka i noga \textbf{su} \textbf{mu} podignuta.\\
 arm and leg be.3\textsc{pl} him.\textsc{dat} elevated\\}
\glt Intended: ‘His arm and leg were elevated.’
\end{exe}

\noindent All target sentences in the experiment have similar length in order to control for this extraneous variable, so that differences in judgments can be attributed solely to differences in structure \citep[cf.][45]{Cowart97}.

In order to avoid effects of fatigue, boredom and response strategies which participants develop during the experiment, the order of sentences presented to participants was randomised (cf. \citealt[][51, 94]{Cowart97}, \citealt[][82]{KrugSell13}). Furthermore, randomisation is important because the preceding sentence can influence the judgment of the following sentence \citep[cf.][51f]{Cowart97}. It was carried out with the algorithm of the software we used – OpenSesame version 3.1.9 \textit{Jazzy James} \citep[][]{MST12}.\footnote{For more information on OpenSesame visit \url{http://osdoc.cogsci.nl/}.} The order of stimulus presentation was shuffled in the experimental part and in the practise session (see Cowart's (\citeyear[][96]{Cowart97}) recommendations for randomisation). 

The greatest advantages of computer-based acceptability judgment tasks are that two measures can be taken at the same time (reaction time and acceptability rating) and that participants cannot go back and change previous answers. However, as in all experiments, there is the problem that only a few members of a speech community are willing to participate in such studies since it means that they have to come to a certain place at a certain time.\footnote{At the time when we were collecting these data, online solutions were still being developed, and they were neither as wide-spread nor as well-tested as they are today: the pandemics gave these solutions an additional boost. Currently, reaction time can quite reliably be collected online, and there is even a way to present participants with a reward; for more information, see \citet{filipovic21}. However, one advantage of in-person testing is control over testing conditions. For us, in-person testing was also crucial because we could check whether the participants were really speakers of the Neo-Štokavian dialect.} In other words, it is harder to bring the participants to the lab than to give them a paper questionnaire that they can fill in on the spot. 

\subsubsection{Ecological validity of stimuli in our study}
\label{Ecological validity of stimuli in our study}

We tried to improve the ecological validity of our stimuli as far as possible.\footnote{Ecological validity is a problem of experimental data; for more information see Sections \ref{Triangulation of methods} and \ref{Research validity}.} When constructing the stimuli, we used corpora to make them sound more natural. For instance, we always searched them for adverbials or (prepositional) complements/adjuncts often appearing to the right of the infinitive used in our stimuli: see position PP Complement/Adjunct in Table \ref{T16.X3}, examples \REF{I16.6}--\REF{I16.12}. Further, we looked for the most frequent adverbs to appear left of matrix verbs – see position Adverb in example \REF{I16.6}--\REF{I16.12}. Additionally, in order to be sure that the adverbs at the beginning of the sentence can serve as hosts for CLs, we checked how well the chosen adverbs were attested with pronominal CLs such as \textit{ga}. All adverbs which had less than 100 hits with pronominal CLs in the whole hrWaC were replaced with adverbs more likely to appear as hosts.

Some may object that the object control matrix verbs (see Tables \ref{T16.3}--\ref{T16.5}) chosen for the study do not sound natural with the infinitive and they might prefer the \textit{da}\textsubscript{2}-construction instead. However, we emphasise that the stimuli were constructed exclusively with object control verbs which were attested with infinitive complements in hrWaC.\footnote{We compared our 16 object control verbs with the verbs listed in \citet[][]{GnjatovicMatasovic13}, a study on verbs with obligatory control in Croatian. Only three of them (\textit{savjetovati} ‘advise’, \textit{preporučivati} ‘recommend’, \textit{požurivati} ‘hurry’) were not mentioned in this article.}

Moreover, \textit{naređivati si} ‘order yourself/give yourself a command’ may be considered objectionable and odd-sounding by some. Indeed, not all of the eight object control matrix predicates with the \textsc{refl\textsubscript{2nd}} \textit{si} (see Table \ref{T16.4}) selected for one of the seven experimental lists are completely satisfactory. However, it was not possible to find eight more appropriate representative verbs with the \textsc{refl\textsubscript{2nd}} \textit{si} controller that were attested with infinitive complements. The verbs chosen were therefore a compromise that enabled us to fully cover the experimental design. 

As a last step, we conducted a pilot study where we asked native speakers to evaluate our target sentences. The results of their feedback were used to improve stimuli to sound as natural as possible.

We are aware that the constructed stimuli can never achieve the ecological validity of data produced spontaneously. However, the abovementioned steps, i.e. the double check in the corpus, which provided us with model sentences for the acceptability experiment and the pilot study, allow us to reject the claim that the constructed examples are entirely artificial. In other words, they are likely to be similar to sentences which appear in real-life situations.

\subsection{Participants}\label{Participants}

Methodological literature recommends avoiding linguists as potential participants for several reasons.\footnote{It must be said that contrary to the abovementioned reasons against using linguists as participants, some argue that professional linguists’ expert knowledge may increase their reliability and perhaps also their sensitivity, since they are able to detect fine-grained distinctions which inexperienced participants simply cannot perceive (see \citealt[][61, 66]{Newmeyer83}, \citealt[][397]{Newmeyer07}, \citealt[][354]{Fanselow07}, \citealt[][497--500]{Devitt06}, \citealt[][860f]{Devitt10}). There are several contradictory studies regarding this issue. On the one hand \citet[][]{Spencer73}, \citet[][]{GordonHendrick97}, and \citet[][]{Dabrowska10} point out that there are differences in ratings between linguist and non-linguist populations, while on the other hand \citet[][]{SprouseAlmeida12} and \citet[][]{SSA13} found strong agreement in ratings by linguists and non-linguists. For more information on this problem see Section \ref{Introduction:4}.} First of all, they have been exposed to a great deal of language contact and therefore may have different intuitions than non-linguists. On the one hand, their linguistic knowledge may lead them to under- or over-report on marginal structures or features \citep[cf.][78]{KrugSell13}. On the other hand, since they are probably aware of the theoretical impact of their judgments, they may be consciously or subconsciously biased to judge in accordance with their theoretical viewpoints (cf. \citealt[][372]{Ferreira05}, \citealt[][1483]{WasowArnold05}, \citealt[][233]{GibsonFedorenko10}, \citealt[][88f, 98f]{GibsonFedorenko13}). Apart from linguists, we decided to exclude language teachers and all students of languages since they can demonstrate rather prescriptive attitudes and may rely heavily on the notion of a narrowly defined standard language usage \citep[cf.][78]{KrugSell13}. Moreover, we wanted to control for dialect, since CLs behave differently in the Čakavian and Kajkavian dialects. Therefore we chose only speakers of Neo-Štokavian dialects \citep[cf.][45]{Cowart97}.\footnote{For more information on CLs in Štokavian dialects see Chapter \ref{Clitics in dialects}.}


\begin{table}[t]
\small
\caption{Higher education institutions attended by our participants\label{T16.6}}
\begin{tabularx}{\textwidth}{XrXr} %using Y instead of r made it worse.
\lsptoprule
Higher education institutions& $N$ & Higher education institutions & $N$\\
\midrule
Faculty of Agriculture \newline Faculty of Forestry & 55 & Academy of Music & 24\\
\tablevspace
Faculty of Chemical Engineering \newline and Technology \newline Faculty of Food Technology \newline and Biotechnology \newline Faculty of Science \newline Department of Biology \newline Department of Physics \newline Department of Ecology, Agriculture \newline and Aquaculture \newline School of Medicine & 53 & Faculty of Kinesiology& 23 \\
\tablevspace
Faculty of Electrical Engineering \newline Faculty of Electrical Engineering \newline and Computing \newline Faculty of Mechanical Engineering \newline and Naval Architecture& 50& Faculty of Economics and Business \newline Faculty of Economics \newline Zagreb School of Economics \newline and Management & 21 \\
\tablevspace
Faculty of Transport and Traffic Sciences& 33 & Faculty of Humanities and Social Sciences & 18 \\
\tablevspace
Polytechnic Lavoslav Ružička Vukovar \newline Polytechnic of Požega \newline Polytechnic of Šibenik \newline Polytechnic of Zagreb & 6 & Faculty of Law \newline Faculty of Political Science & 2 \\
\tablevspace
Catholic Faculty of Theology & 50 & Faculty of Tourism and \newline Hospitality Management & 1 \\
\lspbottomrule
\end{tabularx}
\end{table}


The experiment was conducted in three university cities: Zagreb, Split, and Osijek. Although the dialect of native speakers of Croatian from Zagreb is not Neo-Štokavian, the research was conducted in Zagreb because it is the city with the biggest and oldest university in Croatia. As such it is attractive to students from other cities and regions. Therefore, many Neo-Štokavian speakers can be found in Zagreb. Unlike Zagreb, Split and Osijek are cities where Neo-Štokavian is spoken and they were chosen precisely for that reason.

The average age of our speakers is 21.5 years. We recruited participants from 30 different higher education institutions located in seven different cities in Croatia.\footnote{Although the experiment was conducted in three university cities, the participants attended higher education institutions located in seven different cities in Croatia. Some participants were recruited while visiting their friends in one of the university cities. Others were recruited during the Christmas holidays in their home villages Rokovci and Andrijaševci where one of the authors also stayed in winter 2017.} For details see Table \ref{T16.6} below.

\subsection{Recruiting participants}
\label{Recruiting participants}

Since relying on volunteers only turned out to be inefficient (in particular in terms of time), participants were rewarded with cinema coupons for participating in the experiment. Information about the experiment (time, place, mode of procedure) was distributed among students by university teachers, on Facebook study groups, online learning platforms, official faculty web sites, flyers and official faculty email addresses.\footnote{This would not be possible without the help of many enthusiastic university teachers and administrators, who are all listed in the Acknowledgments.} Participants were able to schedule their appointment and to book their place via Google spreadsheets in which they signed up for the experiment with a pseudonym. 

The yes/no task requires forty participants to reach 80\% coverage (statistical power) \citep[][40]{SchutzeSprouse13}. This was found to be the minimum number of participants needed to achieve the given statistical power, assuming that each participant provides only one response per condition. In our case, participants provided us with multiple responses per condition. However, in order to stay on the safe side, we kept the minimum recommended sample size. 

\subsection{Procedure}
\label{Procedure}

As recommended, we followed basic ethical guidelines. We provided participants with general information on the purpose of the study.\footnote{This does not mean that we told participants that we are investigating CLs and word order. We simply informed them that we wanted to investigate certain structures in Croatian and that their native speaker intuition was an invaluable tool for us when distinguishing between acceptable and unacceptable structures.} We also obtained written consent to using their data, i.e. we were assured that their participation was voluntary. We guaranteed our participants complete anonymity and gave them our contact details. This is so that participants can have access to the research findings \citep[cf.][71]{KrugSell13}.

Prior to conducting the experiment we explicitly informed the participants that the data would be used for scientific purposes only. We emphasized that it was important for the data to be reliable and that the experiment required a great deal of concentration. 

Since most of the participants had never taken part in acceptability judgment tasks, they had to be familiarised with the method. Therefore, before collecting real data, the participants had to complete a training session in which they rated 24 sentences in order to prepare them for the task in the experiment.\footnote{In the training session participants rated six target sentences, six syntactically and morphologically ill-formed target-like sentences, six filler sentences, and six syntactically and morphologically ill-formed filler-like sentences. As indicated in Section \ref{Data preprocessing}, all sentences from the practise session were removed from the analysis.} The instruction as to how to complete the task were provided in writing and orally, and repeated twice: before showing of the training set and of the experimental set. Below are the written instructions in Croatian.\footnote{A string of words in the form of a sentence will appear on the screen. Your task is to determine whether the given string of words, i.e. the given sentence, is acceptable as a Croatian-language sentence. 

It is important for you to know that this is not a formal test of Croatian language knowledge. It is exceptionally important to us that you answer in accordance with your personal sense of language.

We would ask you to read each sentence carefully, but not to spend too much time thinking about it, instead you should answer by following your first impulse.

If you consider the sentence to be acceptable, press the left mouse button with your pointer finger. If you consider the sentence to be unacceptable in your language, press the right mouse button with your middle finger. Therefore, hold the mouse as you usually do when working on a computer.

Try to answer as quickly and as accurately as possible.

At the beginning you will take part in a short exercise. If you have any questions you may ask them after the exercise, i.e. before the experiment.

Press any key to start the exercise.

\center{*******}

This was the exercise. Press any key to start the experiment.
}

\begin{quotation}
Na zaslonu će se pojaviti niz riječi u obliku rečenice. Vaš je zadatak da procijenite je li dani niz riječi, tj. dana rečenica prihvatljiva kao rečenica hrvatskoga jezika. 

Važno je da znate da ovo nije formalni test znanja hrvatskoga jezika. Nama je iznimno važno da odgovorite u skladu s vlastitim jezičnim osjećajem. 

Molimo Vas da svaku rečenicu pažljivo pročitate, ali da ne provedete previše vremena razmišljajući o njoj, nego da odgovorite prateći svoj prvi impuls. 

Smatrate li da je rečenica prihvatljiva, kažiprstom pritisnite lijevu tipku miša. Smatrate li da dana rečenica nije prihvatljiva u Vašemu jeziku, srednjim prstom pritisnite desnu tipku miša. Miš, dakle, držite onako kako ga inače držite kad radite na računalu. 

Pokušajte odgovarati što brže i što točnije. 

Na početku ćete imati kratku vježbu. Budete li imali pitanja, možete ih postaviti poslije vježbe, odnosno prije eksperimenta.

Za početak vježbe pritisnite bilo koju tipku.

\center{*******}

Ovo je bila vježba. Za početak eksperimenta pritisnite bilo koju tipku.
\end{quotation}

Since there was a danger that the participants’ answers would be a compromise between actual language usage and the socially desired answers, we decided to make it explicit in the instruction that this was not a formal test of Croatian language knowledge.\footnote{\textcolor{black}{The danger that participants’ answers would be a compromise between actual language usage and the socially desired answers} is often the case when vernacular, non-standard forms and usages are stigmatised as a hallmark of uneducated people.} That means that there were no desirable answers and no wrong answers per se (cf. \citealt[][75]{KrugSell13}, \citealt[][103]{Hoffmann13}). Additionally, since at this point we were not interested in possible diaphasic variation, we instructed orally participants to rate whether each sentence presented could be said or written by a native Croatian speaker, i.e., considering the Croatian language as a whole (not comparing the sentences in the experiment with a specific Croatian variety).\footnote{Participants were told this orally (in addition to reading it in the instructions themselves) to help them familiarize themselves with the task and to make sure that they were fully aware what the task entailed. The basic idea was that the participants should understand that they would be making judgments based on their own experience of the language (which arises from their contact, oral and written, with other native speakers), not the formal knowledge of grammar taught in schools. It was crucial to emphasise this part since sometimes people use constructions not approved in the norm. We wanted them to know that we would not stigmatise any language use.} 

Each trial began with the presentation of the fixation cross in the centre of the screen for 2000 ms in order to draw the participant’s eyes to neutral position. Next, a sentence was presented in the centre of the screen. The presentation time for each sentence was until the participant’s response or time-out, which was set to 8000 ms. The participants had to give their answer by pressing the left mouse key with their index finger to judge a sentence as acceptable or the right mouse key with their middle finger to judge a sentence as unacceptable.\footnote{This was done to ensure that the dominant motor action is mapped to the yes response, to reduce the variation in response time that can be attributed to the execution of the motoric component of the response action.} The response and response time were recorded automatically. If a participant did not make a response within 8000 ms, the trial was aborted. As already stated, each participant received 24 practice items before the experimental session started. In the experimental session each participant rated 296 sentences.

The participants rated the sentences in a quiet room (classroom or office) at respective faculties on four laptops provided by a member of the study team, under the supervision of that member. Some faculties provided us with additional laptops so that we could conduct the experiment faster (test more than 4 participants at the same time) and some even allowed us to install OpenSesame and experimental files on their computers and to use their computer labs. Each acceptability judgment experiment session took about 30 minutes per participant.\footnote{Although it seems rather long, \citet[][283]{Rosenbach13} claims that 30--45 minutes is the ideal time span for holding a subject’s full concentration without tiring them out.}

\section{Data analysis}
\label{Data analysis}
\subsection{Data preprocessing}
\label{Data preprocessing}
This stage involved two steps: identification of participants who did not perform the task correctly and identification of extreme outliers in the responses. %1sentence paragraph

In the first step, we removed all the data collected during the practice sessions and all the filler sentences from our dataset. Next, we focused on obviously syntactically and morphologically ill-formed sentences (unrelated to our study design) and we analysed by-participant accuracy for these sentences.\footnote{We were not able to rely on total accuracy, as the acceptability of our target sentences was itself the subject of our inquiry. Therefore, it was impossible to use their acceptability as the criterion for engagement in the task.} Participants who accepted obviously grammatically ill-formed stimuli, and did so multiple times, were most likely not paying attention to the task. Therefore, participants who accepted more than 25\% of the sentences from the subset of clearly unacceptable sentences were excluded from further analyses. This resulted in the exclusion of 18 participants. After this, the subset of clearly unacceptable sentences was removed from the set, and all the analyses were conducted on the main set of target sentences described in Section \ref{Stimuli}.

We also inspected the distribution of timeout data, i.e., situations in which participants failed to respond by either “yes” or “no” to the sentence in question. We did so to make sure that the sentences which were not considered acceptable were not predominantly those that went unrated, i.e. those that were timed out instead of receiving a “no” response. The data revealed that this was not the case – only around 1\% of responses were timeouts and they were evenly distributed across conditions. We also conducted parallel statistical analyses with and without timeout data and found no substantial differences between them. Therefore, we decided to keep the timeout data and treat them as lack of acceptance.

In the next step, we inspected the reaction time (RT) distribution. Before a\-na\-ly\-sing RT, we removed five data points that were below 700ms and clearly outliers. Finally, the RT data were log-transformed to approach normality (as suggested by \citealt[][]{BaayenMilin10}).

\subsection{Statistical analysis}
\label{{Statistical analysis}}
The data were analysed in the R statistical environment (R Core Team 2017) using the \textsc{lmer}4 \citep[][]{BMBW15}, \textsc{lmertest} \citep[][]{KBC17} and \textsc{lsmeans} \citep[][]{Lenth16} packages. We applied mixed-effects regression with participants and sentence endings as random variables.\footnote{These included the matrix verb followed by the rest of the sentence (see positions Matrix.\textsc{prs}\textsubscript{1}, Infinitive\textsubscript{2}, CL\textsubscript{2}, PP Complement/Adjunct in Table \ref{T16.X3} examples (\ref{I16.6})--(\ref{I16.12}), which were kept constant while the CL type and CL position were manipulated.} This statistical method has become the golden standard in psycholinguistic research (\citealt[][]{Baayen12}, \citealt[][]{BDB08}, \citealt[][]{Jaeger08}). 

In this analysis the random effects of participants and items are taken into account simultaneously. Unlike the fixed effects which are tested deliberately by the researcher, participants and items are considered to be the source of random effects. These are the differences that originate in factors that are beyond control in the current experiment, i.e. overall processing speed among participants, differences among language items that are as yet unknown and so on. 

By applying mixed effects analysis, the fixed effects of the variables which were manipulated (or introduced in some other way) by the researcher can be generalised beyond the set of stimuli presented in the experiment and beyond the speakers who participated. In other words, these effects can be generalised to other language stimuli of the same kind and to other speakers who belong to the same population, i.e. healthy speakers of the given language. 

\begin{sloppypar}
Wherever possible, we included random slopes of the variables that were tested as fixed effects, as suggested by \citet[][]{BLST13}. However, when convergence of the model was not achievable, we included random slopes one by one as suggested by \citet[][]{Barr13} and tested whether their inclusion in the model was justified by the data (cf. \citealt*{BKVB15}, \citealt[][]{MKVBB17}).\end{sloppypar}

In each analysis we fitted two parallel models: one for acceptance and one for reaction time. For the models that were fitted to the binomial variable of acceptance ($1=\text{accept}$; $0=\text{reject}$), we used binomial distribution as the underlying functional form and we fitted the model by using generalised linear mixed effects regression (\textsc{glmer}). For the models that were fitted to reaction time, we used Gaussian distribution as the appropriate underlying functional form and we fitted the model using the \textsc{lmer} function. 

\section{Results: Regression model for acceptance rates}
\label{Results:16}
\label{Regression model for acceptance rates}

In Table \ref{T16.7}--\ref{T16.7d} and Figure \ref{Fig.15.1A}--\ref{F16.1} we report the results of the acceptance rate analysis. The pattern of results reveals a three-way interaction of: 

\begin{itemize}
\item CC (CC, noCC), 
\item matrix verb type (raising, simple subject control, reflexive subject control with the \textsc{refl\textsubscript{\textsc{lex}}} CL \textit{se}, object control with the pronominal CL controller in the dative, object control with the pronominal CL controller in the accusative, object control with the \textsc{refl\textsubscript{2nd}} CL \textit{si} controller, and object control with the \textsc{refl\textsubscript{2nd}} CL \textit{se} controller), 
\item and infinitive CL type (pronominal, \textsc{refl\textsubscript{2nd}}, \textsc{refl\textsubscript{\textsc{lex}}}). 
\end{itemize}

Detailed data from the generalised mixed effects regression model are presented in Tables \ref{T16.7}--\ref{T16.7d}.\footnote{The model formula is: Acceptance $\sim$ CC * Matrix verb * Infinitive CL type + (1|Participant) + (1|Item); for explanation of statistical measures in Tables \ref{T16.7}--\ref{T16.7d} see Appendix \ref{s}.} Those interested in the tables who cannot easily follow their content can use the explanations in this section as a guide. Those who are not interested in the details of the generalised mixed effects regression model can skip the table and go directly to following subsections, in which we present the results according to the order of the research questions presented in Section \ref{RQ:16}.

\begin{table}
\caption{Random effects from generalised mixed effects regression model fitted to acceptance data ($1=\text{acceptable}$; $0=\text{unacceptable}$).\label{T16.7}}
\begin{tabular}{lrr}
\lsptoprule
                                     & \multicolumn{1}{c}{Var.} & \multicolumn{1}{c}{SD} \\\midrule
participants (intercept adjustments) & 1.410 & 1.187\\
items (intercept adjustments) & 0.631 & 0.794\\
\lspbottomrule
\end{tabular}
\end{table}

We describe the observed results in more detail by referring to the fixed effects in Tables \ref{T16.7a}--\ref{T16.7d}. The first row presents the intercept of the model, and in this case it corresponds to noCC sentences with raising predicates whose infinitive complement governs pronominal CLs (leftmost bar in Figure \ref{Fig.15.1A}). In comparison, there is a statistically significant rise in acceptance of the same sentences with CC (the estimated coefficient is above zero and statistically significant; see row 2 in Table \ref{T16.7a}). In the next six rows (row 3–row 8) the noCC sentences with raising predicates whose infinitive complement governs a pronominal CL (intercept) are compared to the same sentences (pronominal CL, noCC) with the six remaining matrix verb types. Based on these rows it can be inferred which of the matrix verb types make the noCC sentences whose infinitive complement governs a pronominal CL more/less acceptable compared to the same sentences with raising predicates (whose infinitive complement governs a pronominal CL, noCC). Compared to the intercept, i.e. raising predicates, the acceptance rate is higher for the same sentence variants in which the matrix verb is a simple subject control verb, reflexive subject control verb with the \textsc{refl\textsubscript{\textsc{lex}}} CL \textit{se}, object control verb with a \textsc{refl\textsubscript{2nd}} CL \textit{si} governor, and object control verb with a \textsc{refl\textsubscript{2nd}} CL \textit{se} governor. In contrast, the acceptance rate for the sentence variants with an object control matrix verb whose governor is a pronominal CL in the dative or with an object control matrix verb whose governor is a pronominal CL in the accusative is the same as for the sentences with raising-type matrix verbs. 

In rows 9 and 10 noCC sentences with raising matrix verbs whose infinitive complement governs a pronominal CL are compared to noCC sentences with the same matrix verb type, but a different critical CL type: the \textsc{refl\textsubscript{\textsc{lex}}} CL \textit{se} (row 9) and the \textsc{refl\textsubscript{2nd}} CL \textit{si/se} (row 10). None of the two differences is significant (see also the three black bars in Figure \ref{F16.1}A). 

\begin{table}
\caption{Rows 1--10 of fixed effects from generalised mixed effects regression model fitted to acceptance data ($1=\text{acceptable}$; $0=\text{unacceptable}$).\label{T16.7a}}
\begin{tabular}{l@{~}>{\raggedright}p{0.3\linewidth} S[table-format=-1.3]S[table-format=1.3]S[table-format=-1.3]S[table-format=<1.4] @{\,}l}
\lsptoprule
&                                                                & {Est.} & {SE} & {$z$} & {$\text{Pr}(>|z|)$} & \\\midrule
1 & intercept (CC – no; matrix verb – raising; infinitive CL type – pronominal CL) &0.063 &0.294 &0.214 &0.830 & \\
2 & CC – yes & 2.593 &0.36 & 7.194 & <0.0001 & ***\\
3 & matrix verb – simple subject control & 1.096 &0.421 & 2.605 &0.009 & **\\
4 & matrix verb – reflexive subject control & 2.454 &0.438 & 5.608 & <0.0001 & ***\\
5 & matrix verb – object control with pronominal CL controller in dative &0.450 &0.415 & 1.083 &0.279 & \\ 
6 & matrix verb – object control with pronominal CL controller in accusative &0.443 &0.414 & 1.069 &0.285 & \\ 
7 &  matrix verb – object control with \textsc{refl\textsubscript{2nd} CL \textit{si} controller} & 0.985 & 0.417 &  2.363 & 0.018 &  * \\
8 &  matrix verb – object control with \textsc{refl\textsubscript{2nd}} CL \textit{se} controller & 1.738 &0.423 & 4.111 &  <0.0001 &  ***\\
9 & infinitive CL type – \textsc{refl\textsubscript{\textsc{lex}}} CL \textit{se} & -0.141 &0.330 & -0.427 & 0.669 & \\
10& infinitive CL type – \textsc{refl\textsubscript{2nd}} CL \textit{si/se} & -0.210 &0.331 & -0.636 & 0.525 &\\
\lspbottomrule
\end{tabular}
\end{table}

Rows 11--16 (Table \ref{T16.7b}) inform us whether the relationship between acceptance rates of noCC and CC sentence variants which was observed for sentences with raising-type predicates whose infinitive complement governs a pronominal CL is the same for the remaining six types of matrix verbs whose infinitive complement governs a pronominal CL. Statistical significance associated with a coefficient presented in one of these rows indicates that this relationship differs. Based on the results observed, we can infer that the described ratio is the same for simple subject verb predicates, whereas for all of the remaining matrix verb types, the relationship is different – sentences with CC are not favoured over the corresponding noCC variants (compare leftmost pairs of bars across the panels in Figures \ref{Fig.15.1A}--\ref{F16.1}). 

The numbers in rows 17--18 (Table \ref{T16.7b}) allow similar comparisons: here, the relationship between CC and noCC variants of the sentences with raising predicates whose infinitive complements govern a pronominal CL is compared with the same relationship for sentences with raising matrix verbs whose infinitive complements govern \textsc{refl\textsubscript{2nd}} CLs \textit{se} and \textit{si} and the \textsc{refl\textsubscript{\textsc{lex}}} CL \textit{se}. The fact that none of the two coefficients are significant tells us that the type of the complementing CL does not affect the relationship between acceptance of CC and noCC variants in the case of sentences with raising predicates. 

\begin{table}
\caption{Rows 11--18 of fixed effects from generalised mixed effects regression model fitted to acceptance data ($1=\text{acceptable}$; $0=\text{unacceptable}$).\label{T16.7b}}
\begin{tabularx}{\textwidth}{l@{~}Q S[table-format=-1.3]S[table-format=1.3]S[table-format=-2.3]S[table-format=<1.4] @{\,}l}
\lsptoprule
&& {Est.} & {SE} & {$z$} & {$\text{Pr}(>|z|)$} & \\
\midrule
11 & CC – yes : matrix verb – simple subject control & -0.549 & 0.520 & -1.055 & 0.291 & \\
12 & CC  –  yes  :  matrix  verb  –  reflexive  subject  control & -3.066 &  0.518 &  -5.924 & <0.0001 & *** \\
13 & CC –  yes  :  matrix  verb – object control with pronominal CL controller in dative & -4.984 &  0.500 &  -9.964 & <0.0001 & *** \\
14 & CC – yes : matrix verb – object control with pronominal CL controller in accusative & -6.365 & 0.532 & -11.962 & <0.0001 & *** \\
15 & CC – yes : matrix verb – object control with \textsc{refl\textsubscript{2nd} CL \textit{si} controller} & -5.684 & 0.502 & -11.333 & <0.0001 & *** \\
16 & CC – yes : matrix verb – object control with \textsc{refl\textsubscript{2nd}} CL \textit{se} controller & -4.160 & 0.496 & -8.385 & <0.0001 & *** \\
17 & CC – yes : infinitive CL type – \textsc{refl\textsubscript{\textsc{lex}}} CL \textit{se} & 0.699 & 0.521 & 1.342 & 0.180 & \\
18 & CC – yes : infinitive CL type – \textsc{refl\textsubscript{2nd}} CL \textit{si/se} & 0.417 & 0.511 & 0.816 & 0.414 & \\
\lspbottomrule
\end{tabularx}
\end{table}

Rows 19--24 (Table \ref{T16.7c}) inform us whether the finding which relates noCC sentences with raising predicates whose infinitive governs a pronominal CL and the same sentence type whose infinitive governs the \textsc{refl\textsubscript{\textsc{lex}}} CL \textit{se} is affected by the change in predicate type. The results indicate that the absence of difference between noCC raising sentences whose infinitives govern a pronominal CL and those whose infinitives govern the \textsc{refl\textsubscript{\textsc{lex}}} CL \textit{se} which we found (i.e. the acceptance rates for the two sentence groups are identical, as indicated in row 9), can be generalised to sentences in which the \textsc{refl\textsubscript{\textsc{lex}}} CL \textit{se} is governed by infinitive complements of simple subject control predicates, sentences with object control predicates with a \textsc{refl\textsubscript{2nd}} CL \textit{si} controller, and object control predicates whose infinitives govern the \textsc{refl\textsubscript{2nd}} CL \textit{se}. However, this does not hold for the remaining predicate types. In the case of noCC sentences with reflexive subject control matrix predicates, the acceptance rates are lower for sentences with the \textsc{refl\textsubscript{\textsc{lex}}} CL \textit{se} as the infinitive complement than for those containing a pronominal CL in the same position. In contrast, in the case of noCC sentences with object control matrix predicates and whose controller is a pronominal CL in the dative or the accusative, the acceptance rates are higher for sentences whose infinitive complement contains the \textsc{refl\textsubscript{\textsc{lex}}} CL \textit{se} compared to those that contain a pronominal CL as an infinitive complement. 

Similarly, rows 25--30 (Table \ref{T16.7c}) inform us whether the finding which relates noCC sentences with a pronominal CL as the infinitive complement with those whose infinitive governs the \textsc{refl\textsubscript{2nd}} CL is affected by the change in predicate type. The results show that the acceptance rates for the noCC sentence variants whose infinitives govern a pronominal CL on the one hand and whose infinitives govern the \textsc{refl\textsubscript{2nd}} CL on the other hand differ only within sentences with reflexive subject control matrix predicates. Namely, sentences whose infinitives govern the \textsc{refl\textsubscript{2nd}} CL \textit{si/se} are less acceptable than those with a pronominal CL in the same position. 

\begin{sidewaystable}\small
\caption{Rows 19--30 of fixed effects from generalised mixed effects regression model fitted to acceptance data ($1=\text{acceptable}$; $0=\text{unacceptable}$).\label{T16.7c}}
\begin{tabularx}{\textwidth}{l@{~}Q S[table-format=-1.3]S[table-format=1.3]S[table-format=-1.3]S[table-format=1.3]  @{\,}l}
\lsptoprule
   & & {Est.} & {SE} & {$z$} & {$\text{Pr}(>|z|)$} & \\\midrule
19 & matrix verb – simple subject control : infinitive CL type – \textsc{refl\textsubscript{\textsc{lex}}} CL \textit{se} & 0.203 & 0.475 & 0.427 & 0.670 & \\
20 & matrix verb – reflexive subject control : infinitive CL type – \textsc{refl\textsubscript{\textsc{lex}}} CL \textit{se} & -1.323 & 0.487 & -2.716 & 0.007 & ** \\
21 & matrix verb – object control with pronominal CL controller in dative : infinitive CL type – \textsc{refl\textsubscript{\textsc{lex}}} CL \textit{se} & 0.849 & 0.470 & 1.805 & 0.071 & . \\
22 & matrix verb – object control with pronominal CL controller in accusative : infinitive CL type – \textsc{refl\textsubscript{\textsc{lex}}} CL \textit{se} & 0.922 & 0.469 & 1.966 & 0.049 & * \\
23 & matrix verb – object control with \textsc{refl\textsubscript{2nd}} CL \textit{si} controller : infinitive CL type – \textsc{refl\textsubscript{\textsc{lex}}} CL \textit{se} & 0.593 & 0.475& 1.248 & 0.212 & \\
24 & matrix verb – object control with \textsc{refl\textsubscript{2nd}} CL \textit{se} controller : infinitive CL type – \textsc{refl\textsubscript{\textsc{lex}}} CL \textit{se} & -0.673 & 0.475 & -1.418 & 0.156 &\\
25 & matrix verb – simple subject control : infinitive CL type – \textsc{refl\textsubscript{2nd}} CL \textit{si/se} & 0.257 & 0.475 & 0.541 & 0.588 & \\
26 & matrix verb – reflexive subject control : infinitive CL type – \textsc{refl\textsubscript{2nd}} CL \textit{si/se} & -1.079 & 0.488 & -2.209 &0.027 & * \\
27 & matrix verb – Object control with pronominal CL controller in dative : Infinitive CL type – \textsc{refl\textsubscript{2nd}} CL \textit{si/se} & 0.090 & 0.468 & 0.193 & 0.847 & \\
28 & matrix verb – object control with pronominal CL controller in accusative : infinitive CL type – \textsc{refl\textsubscript{2nd}} CL \textit{si/se} & 0.079 & 0.465 & 0.171 & 0.864 & \\
29 & matrix verb – object control with \textsc{refl\textsubscript{2nd}} CL \textit{si} controller : infinitive CL type – \textsc{refl\textsubscript{2nd}} CL \textit{si/se} & -0.161 & 0.470 & -0.342 & 0.732 &\\
30 & matrix verb – object control with \textsc{refl\textsubscript{2nd}} CL \textit{se} controller : infinitive CL type – \textsc{refl\textsubscript{2nd}} CL \textit{si/se} & -0.701 & 0.475 & -1.476 & 0.140 &\\
\lspbottomrule
\end{tabularx}
\end{sidewaystable}

\begin{sidewaystable}\small
\caption{Rows 31--42 of fixed effects from generalised mixed effects regression model fitted to acceptance data ($1=\text{acceptable}$; $0=\text{unacceptable}$).\label{T16.7d}}
\begin{tabularx}{\textwidth}{l@{~}Q S[table-format=-1.3]S[table-format=1.3]S[table-format=-1.3]S[table-format=<1.4]  @{\,}l}
\lsptoprule
&& {Est.} & {SE} & {$z$} & {$\text{Pr}(>|z|)$} & \\\midrule
31 &CC – yes : matrix verb – simple subject control : infinitive CL type – \textsc{refl\textsubscript{\textsc{lex}}} CL \textit{se} & -0.445 &0.747 & -0.595 &0.552 &\\
32 & CC – yes : matrix verb – reflexive subject control : Infinitive CL type – \textsc{refl\textsubscript{\textsc{lex}}} CL \textit{se} & -3.271 &0.728 & -4.492 & <0.0001 &  *** \\
33 & CC – yes : matrix verb – object control with pronominal CL controller in dative : infinitive CL type – \textsc{refl\textsubscript{\textsc{lex}}} CL \textit{se} & -1.188 & 0.712 & -1.668 &0.095 &  . \\
34 & CC – yes : matrix verb – object control with pronominal CL controller in accusative : infinitive CL type – \textsc{refl\textsubscript{\textsc{lex}}} CL \textit{se} & -0.205 &0.738 & -0.279 &0.781 & \\
35 & CC – yes : matrix verb – object control with \textsc{refl\textsubscript{2nd}} CL \textit{si} controller : Infinitive CL type–- \textsc{refl\textsubscript{\textsc{lex}}} CL \textit{se} & -2.541 &0.742 & -3.423 & 0.001 & ** \\
36 & CC – yes : matrix verb – object control with \textsc{refl\textsubscript{2nd}} CL \textit{se} controller : Infinitive CL type–- \textsc{refl\textsubscript{\textsc{lex}}} CL \textit{se} & -2.334 &0.713 & -3.271 &0.001 & ** \\
37 & CC – yes : matrix verb – simple subject control : infinitive CL type – \textsc{refl\textsubscript{2nd}} CL \textit{si/se} & -0.166 &0.739 & -0.224 &0.823 & \\
38 & CC – yes : matrix verb – reflexive subject control : infinitive CL type – \textsc{refl\textsubscript{2nd}} CL \textit{si/se} &  -3.566 &0.727 & -4.905 & <0.0001 & *** \\
39 & CC – yes : matrix verb – object control with pronominal CL controller in dative : infinitive CL type – \textsc{refl\textsubscript{2nd}} CL \textit{si/se} & -0.778 &0.710& -1.096 &0.273 & \\
40 & CC – yes : matrix verb – object control with pronominal CL controller in accusative : infinitive CL type – \textsc{refl\textsubscript{2nd}} CL \textit{si/se} &0.961 &0.730 & 1.317 &0.188 & \\
41 & CC – yes : matrix verb – object control with \textsc{refl\textsubscript{2nd}} CL \textit{si} controller : iInfinitive CL type–- \textsc{refl\textsubscript{2nd}} CL \textit{si/se} & -0.536 &0.711 & -0.754 &0.451 & \\ 
42 & CC – yes : matrix verb – object control with \textsc{refl\textsubscript{2nd}} CL \textit{se} controller : infinitive CL type–- \textsc{refl\textsubscript{2nd}} CL \textit{si/se} & -2.273 &0.710 & -3.200 &0.001 & ** \\
\lspbottomrule
\end{tabularx}
\end{sidewaystable}


\hspace*{-1mm}The remaining rows make analogous comparisons for sentences with CC. Rows 31--36 (Table \ref{T16.7d}) show whether matrix predicate type modifies the relationship between CC sentences whose infinitives govern a pronominal CL on the one hand and those whose infinitives contain the \textsc{refl\textsubscript{\textsc{lex}}} CL \textit{se} on the other hand. As observed for sentences with raising matrix predicates (row 17), the two sentence types are equally acceptable in the case of CC sentences with simple subject control matrix predicates and object control predicates whose controller is a pronominal CL in the accusative. For all the remaining predicate types, CC sentences whose infinitives contain the \textsc{refl\textsubscript{\textsc{lex}}} CL \textit{se} are less acceptable than the same sentence type whose infinitives govern a pronominal CL in the same position.

Finally, rows 37--42 (Table \ref{T16.7d}) present similar comparisons between CC sentences whose infinitives govern a pronominal CL on the one hand and the \textsc{refl\textsubscript{2nd}} CL \textit{si/se} on the other. The results show that the two are different with respect to acceptability only in the case of CC sentences with reflexive subject control matrix predicates and CC sentences with object control matrix predicate whose controller is the \textsc{refl\textsubscript{2nd}} CL \textit{se}. Within each of the two CC sentence types, those whose infinitives govern the \textsc{refl\textsubscript{2nd}} CL \textit{si/se} are less acceptable than those whose infinitive governs a pronominal CL in the same position.


Given the complex structure of the results, we organise the presentation according to the research questions presented in Section \ref{RQ:16}.

%\begin{figure}
%    \centering
%    \includegraphics[width=.97\textwidth]{NEWF151ABC}
%    \caption{A) Sentences with raising-type predicates \\ B) Sentences with subject control predicates: B1 (left) – simple, B2 (right) – reflexive \\ C) Sentences with object control predicates: C1 and C2 (upper row) – pronominal CL controller, C3 and C4 (bottom row) – \textsc{refl\textsubscript{2nd}} CL controller; C1 and C3 (left) – dative, C2 and C4 (right) – accusative \\ Acceptance rates (proportion of “yes” responses) of corresponding CC and noCC stimuli for different matrix predicates and different types of critical CLs: Pers. pr. – pronominal CLs; Refl. 2ND. – \textsc{refl\textsubscript{2nd}} CLs \textit{se} and \textit{si}; Refl. LEX. – \textsc{refl\textsubscript{\textsc{lex}}} CL \textit{se}. The horizontal black line marks a 50\% acceptance rate.}\label{F16.1}}

%    \label{fig:my_label}
%\end{figure}



\section{Discussion}
\label{Discussion:16}
\subsection{Raising, subject and object control predicates}
\label{Raising, subject and object control predicates}
The first research question was formulated at the highest level of generality with the aim of comparing CC and noCC stimuli which contain one of three types of predicates: raising-type predicates, subject control predicates and object control predicates. As can be seen in Figures \ref{Fig.15.1A}--\ref{F16.1}, there are striking differences in the profiles of acceptance for CC and noCC variants for the three predicate types. These differences are particularly pronounced for CC sentences. Sentences with raising matrix predicates show a convincing preference for the acceptance of the CC variants (Figure \ref{Fig.15.1A}: compare examples (\ref{(16.11a)}) and (\ref{(16.11b)})). Object control predicate sentences, meanwhile, show the opposite pattern – there is a clear preference for acceptance of the noCC sentence variants (Figure \ref{F16.1}: compare examples (\ref{(16.12a)}) and (\ref{(16.12b)})).\footnote{\textcolor{black}{The} type of sentence \textcolor{black}{presented in (\ref{(16.12b)})} was accepted in 20\% of cases: see first grey column in Panel C1 of Figure \ref{F16.1}. Therefore we do not mark the example with *, which is usually used to indicate that a sentence is unacceptable and/or grammatically incorrect. Moreover, in this chapter we avoid using * for our stimuli since all sentence types were acceptable at least to some degree. Instead, we speak of graded acceptability. } 

\begin{figure}
\caption{Sentences with raising-type predicates}
\label{Fig.15.1A}
\includegraphics[height=.35\textheight]{figures/Fig.15.1A.pdf}
\end{figure}

\begin{figure}
\caption{Sentences with subject control predicates: \\ Panel B1 (left) – simple, Panel B2 (right) – reflexive}
\label{Fig.15.2B}
\includegraphics[height=.35\textheight]{figures/Fig.15.2B.pdf}
\end{figure}

\begin{figure}
\caption{Sentences with object control predicates: Panels C1 and C2 (upper row) – pronominal CL controller, Panels C3 and C4 (bottom row) – \textsc{refl\textsubscript{2nd}} CL controller; Panels C1 and C3 (left) – dative, Panels C2 and C4 (right) – accusative \\ Acceptance rates (proportion of “yes” responses) of corresponding CC and noCC stimuli for different matrix predicates and different types of critical CLs: Pers. pr. – pronominal CLs; Refl. 2ND. – \textsc{refl\textsubscript{2nd}} CLs \textit{se} and \textit{si}; Refl. LEX. – \textsc{refl\textsubscript{\textsc{lex}}} CL \textit{se}. The horizontal black line marks a 50\% acceptance rate.}\label{F16.1}

\includegraphics[height=.35\textheight]{figures/Fig.15.3C.pdf} \\\medskip
\includegraphics[height=.35\textheight]{figures/Fig.15.4C.pdf}
\end{figure}

\newpage

\begin{exe}\ex\begin{xlist}
\ex\label{(16.11a)}
\gll Polako počinju\textsubscript{1} konkurirati\textsubscript{2} \textbf{joj}\textsubscript{2} na drugim područjima. \\
 slowly start.3\textsc{prs} compete.\textsc{inf} her.\textsc{dat} on other areas \\
\ex\label{(16.11b)}
\gll Polako \textbf{joj}\textsubscript{2} počinju\textsubscript{1} konkurirati\textsubscript{2} na drugim područjima. \\
slowly her.\textsc{dat} start.3\textsc{prs} compete.\textsc{inf} on other areas \\
\end{xlist}
\glt ‘They are slowly starting to compete with her in other areas.’ 

\ex
\begin{xlist}
\ex[]{\label{(16.12a)}
\gll Vjerojatno \textbf{im}\textsubscript{1} omogućuju\textsubscript{1} konkurirati\textsubscript{2} \textbf{joj}\textsubscript{2} na drugim područjima. \\
 probably them.\textsc{dat} enable.3\textsc{prs} compete.\textsc{inf} her.\textsc{dat} on other areas \\}
\ex[?]{\label{(16.12b)}
\gll Vjerojatno \textbf{im}\textsubscript{1} \textbf{joj}\textsubscript{2} omogućuju\textsubscript{1} konkurirati\textsubscript{2} na drugim područjima. \\
probably them.\textsc{dat} her.\textsc{dat} enable.3\textsc{prs} compete.\textsc{inf} on other areas \\}
\end{xlist}
\glt‘They are probably making it possible for those ones to compete with her in other areas.’
\end{exe}

\noindent In the case of raising predicates, the preference of CC stimuli over the noCC variants was expected and is in line with the previous theoretical studies on CC (cf. \citealt[][]{Aljovic05}, \citealt[][]{Rezac99, Rezac05}, \citealt[][]{Dotlacil04}, \citealt[][]{Hana07}) and with our corpus studies on CC out of \textit{da}\textsubscript{2}-complements, infinitives and stacked infinitives presented in Chapters \ref{A corpus-based study on CC in da constructions and the raising-control distinction (Serbian)}, \ref{A corpus-based study on clitic climbing in infinitive complements in relation to the raising-control dichotomy and diaphasic variation (Croatian)}, and in \citet*{HKJ18}. 

In contrast to sentences with raising and object control matrix predicates, sentences with subject control predicates could be seen as the middle ground between the two extremes. It must not be forgotten that these sentences can appear in two conditions: as simple subject control predicates (of the \textit{planirati} ‘plan’ type) and as reflexive subject control predicates (of the \textit{bojati se} ‘be afraid’ type). Once we account for this difference, as illustrated in Figure \ref{Fig.15.2B}, it becomes obvious that the two differ drastically, thus revealing a much more complex pattern of results. This issue, which as far as we know has not been tackled before, is the subject of RQ 2.

Finally, including the type of both matrix predicate and infinitive CL in the analysis reveals an additional complexity in the results, as demonstrated by the three-way interaction in the statistical model presented in Tables \ref{T16.7}--\ref{T16.7d}. This complexity is addressed separately and explored through the relevant research questions that we presented in Section \ref{RQ:16}.

\subsection[Simple subject control predicates and subject control predicates with \textsc{refl\textsubscript{\textsc{lex}}} CL se]{Simple subject control predicates and subject control predicates with \textsc{refl\textsubscript{\textsc{lex}}} CL \textit{se}}
\label{Simple subject control predicates and subject control predicates with refllex CL se}

The second research question addresses the relationship between CC and noCC variants for sentences containing simple subject control predicates and reflexive subject control predicates. When compared, the two variants of subject control predicates reveal different patterns of CC: compare Panels B1 and B2 from Figure \ref{Fig.15.2B}.

In the case of simple subject control predicates (of the \textit{planirati} ‘plan’ type), speakers show a clear preference for the CC stimuli, as compared to the corresponding noCC stimuli (compare examples (\ref{(16.13a)}) and (\ref{(16.13b)})). This preference is independent of the infinitive CL type, i.e. the observed advantage is strikingly similar for pronominal CLs (69\% noCC vs 89\% CC), \textsc{refl\textsubscript{2nd}} CLs \textit{se} and \textit{si} (68\% noCC vs 91\% CC), and the \textsc{refl\textsubscript{\textsc{lex}}} CL \textit{se} (69\% noCC vs 91\% CC). 

\begin{exe}\ex\begin{xlist}
\ex\label{(16.13a)}
\gll Polako nastojite\textsubscript{1} konkurirati\textsubscript{2} \textbf{joj}\textsubscript{2} na drugim područjima.\\
 slowly try.2\textsc{prs} compete.\textsc{inf} her.\textsc{dat} on other areas \\
\ex\label{(16.13b)}
\gll Polako \textbf{joj}\textsubscript{2} nastojite\textsubscript{1} konkurirati\textsubscript{2} na drugim područjima.\\
 slowly her.\textsc{dat} try.2\textsc{prs} compete.\textsc{inf} on other areas \\
\end{xlist}\glt ‘You are slowly trying to compete with her in other areas.’ 
\end{exe}

\noindent However, the pattern of results changes drastically for sentences with reflexive subject control predicates (of the \textit{bojati se} ‘be afraid’ type). Here, the acceptability of the CC stimuli highly depends on the critical CL type. If an infinitive governs a pronominal CL, the CC stimulus is acceptable (82\%), but no longer favoured over its noCC variant (88\%). In fact, we can say that both versions (CC and noCC) of sentences with reflexive subject control predicates whose infinitives govern a pronominal CL are, statistically speaking, equally acceptable (compare examples (\ref{(16.14a)}) and (\ref{(16.14b)})). 

\begin{exe}\ex\begin{xlist}
\ex\label{(16.14a)}
\gll Ipak \textbf{se}\textsubscript{1} trudim\textsubscript{1} konkurirati\textsubscript{2} \textbf{joj}\textsubscript{2} na drugim područjima.\\
 still \textsc{refl} try.\textsc{1prs} compete.\textsc{inf} her.\textsc{dat} on other areas \\ 
\ex\label{(16.14b)}
\gll Ipak \textbf{joj}\textsubscript{2} \textbf{se}\textsubscript{1} trudim\textsubscript{1} konkurirati\textsubscript{2} na drugim područjima.\\
 still her.\textsc{dat} \textsc{refl} try.\textsc{1prs} compete.\textsc{inf}  on other areas \\ 
\end{xlist}\glt ‘Still, I am trying to compete with her in other areas.’ 
\end{exe}

\noindent However, if the critical CL is \textsc{refl\textsubscript{2nd}} or \textsc{refl\textsubscript{\textsc{lex}}}, then the CC stimuli are evidently unacceptable to the respondents (13\% and 17\% compared to 72\% and 69\% for their noCC versions). Compare examples (\ref{(16.15a)}) and (\ref{(16.15b)}) with the critical CL \textsc{refl\textsubscript{\textsc{lex}}} \textit{se}.

\begin{exe}\ex
\begin{xlist}
\ex[]{\label{(16.15a)}
\gll Silno \textbf{se}\textsubscript{1} boje\textsubscript{1} očitovati\textsubscript{2} \textbf{se}\textsubscript{2} o iznesenim prijedlozima.\\
 immensely \textsc{refl} be.afraid.3\textsc{prs} declare.\textsc{inf} \textsc{refl} about presented suggestions \\ }
\ex[?]{\label{(16.15b)}
\gll Silno \textbf{se}\textsubscript{1} \textbf{se}\textsubscript{2} boje\textsubscript{1} očitovati\textsubscript{2} o iznesenim prijedlozima.\\
 immensely \textsc{refl} \textsc{refl} be.afraid.3\textsc{prs} declare.\textsc{inf} about presented suggestions \\}
\end{xlist}
\glt‘They are immensely afraid to voice their opinion on the presented suggestions.’
\end{exe}

\noindent As we already reported in Section \ref{Constraints related to the raising-control distinction}, both \citet[][]{Rezac05} and \citet[][]{Hana07} clearly state that in Czech there are no restrictions on CC out of infinitive complements not only of raising but also of subject control predicates. Therefore it should not come as a surprise that our experiment participants favoured CC versions of sentences with simple subject control predicates. Moreover, the results of the acceptability judgment experiment do not diverge from the corpus linguistic data on CC out of \textit{da}\textsubscript{2}-complements, and infinitives presented in Chapters \ref{A corpus-based study on CC in da constructions and the raising-control distinction (Serbian)} and \ref{A corpus-based study on clitic climbing in infinitive complements in relation to the raising-control dichotomy and diaphasic variation (Croatian)} – in the case of simple subject control predicates, CC sentences are more frequent than noCC sentences. 

However, as our results show, for CC it does not only matter whether CTPs are of the raising or subject control type, it also matters whether the latter are simple or reflexive. Namely, as Figure \ref{F16.1} suggests, in the case of simple subject control CTPs there are no constraints on CC. However, if CTPs are of the reflexive subject control type, only pronominal CLs can climb out of the infinitive complement, whereas the \textsc{refl\textsubscript{2nd}} CL \textit{se}, \textsc{refl\textsubscript{2nd}} CL \textit{si} and \textsc{refl\textsubscript{\textsc{lex}}} CL \textit{se} do not climb. These results confirm \citeauthor{Junghanns02}' (\citeyear[][79]{Junghanns02}) pseudo-twins constraint.\footnote{For more information on the pseudo-twins constraint see Section \ref{Pseudo-twins}.} However, the mentioned phenomenon has not hitherto been linked to the control phenomenon. 

Further, \citet[][]{Hana07} recognises only the object control reflexive constraint, while in his opinion subject control CTPs do not prevent (reflexive) CLs from climbing. However, according to our findings, the list of constraints on CC should clearly be updated, at least with respect to BCS. There is one more type of control constraint: the subject control reflexive constraint. Namely, if the matrix predicate is of the reflexive subject control type, reflexive infinitive CLs cannot climb.

%We have to add an additional comment in respect of sentences in which two reflexive CLs appear. Not only are CC sentences which lead to a \textit{se} \textit{se} string unacceptable: such noCC sentences also score only around 70\% – see the second and third column pair in Figure \ref{F16.2} panel B2. Moreover, as our corpus linguistic data gathered for the study presented in Chapter \ref{A corpus-based study on clitic climbing in infinitive complements in relation to the raising-control dichotomy and diaphasic variation (Croatian)} reveal, structures similar to (\ref{(16.15a)}) do not often appear in corpora. Therefore we can speculate that there is another, more acceptable structure. This could be either haplology (deletion of one \textit{se} CL) or the \textit{da}\textsubscript{2}-construction. However, without an additional acceptability judgment test we cannot verify which of these three structures is the most acceptable variant.

\subsection{Object control predicates with a pronominal CL controller in the dative and with a pronominal CL controller in the accusative}
\label{Object control predicates with a pronominal CL controller in the dative and with a pronominal CL controller in the accusative}
The third research question aims to compare CC and noCC stimuli which contain object control matrix predicates whose controllers are a pronominal CL in the dative on the one hand and a pronominal CL in the accusative on the other. Sentences with object control predicates which have a pronominal CL controller in the dative (of the \textit{dopuštati} ‘allow’ type) show an identical pattern of results regarding the acceptability rate of CC as those with a pronominal CL controller in the accusative (of the \textit{prisiljavati} ‘force’ type) – compare Panels C1 and C2 from Figure \ref{F16.1}.



The CC sentences are consistently rated as unacceptable, regardless of the controller (dative, accusative) and type of climbing CL (pronominal CL, \textsc{refl\textsubscript{2nd}}, \textsc{refl\textsubscript{\textsc{lex}}}). NoCC sentences ((\ref{(16.12a)}) and (\ref{(16.16a)})) were clearly favoured over their permuted CC variants. 

\begin{exe}\exr{(16.12a)}
\gll Vjerojatno \textbf{im}\textsubscript{1} omogućuju\textsubscript{1} konkurirati\textsubscript{2} \textbf{joj}\textsubscript{2} na drugim područjima.\\
 probably them.\textsc{dat} enable.3\textsc{prs} compete.\textsc{inf} her.\textsc{dat} on other areas\\
\glt ‘They are probably making it possible for those ones to compete with her in other areas.’

\ex\label{(16.16a)}
\gll Zakonski \textbf{te}\textsubscript{1} primoravamo\textsubscript{1} konkurirati\textsubscript{2} \textbf{joj}\textsubscript{2} na drugim područjima.\\
 legally you.\textsc{acc} compel.1\textsc{prs} compete.\textsc{inf} her.\textsc{dat} on other areas\\
\glt ‘We are legally compelling you to compete with her in other areas.’ 
\end{exe}
 
\noindent Following the discussion from Section \ref{Constraints related to the raising-control distinction} of strong restrictions on CC out of object-controlled infinitive complements, the experimental data indicate that we can generalise this rule to CC in Croatian too. Moreover, the case of the pronominal CL which is the controller has no impact on acceptability. These results are in line with our corpus linguistic study on CC out of \textit{da}\textsubscript{2}-complements presented in Chapter \ref{A corpus-based study on CC in da constructions and the raising-control distinction (Serbian)}. Both production data on CC out of \textit{da}\textsubscript{2}-complements from srWaC and acceptability judgments, i.e., comprehension data on CC out of infinitive complements made by native speakers of Croatian, reveal the same pattern. Namely, CC out of object-controlled infinitive complements with either a dative or an accusative pronominal CL controller is highly restricted.

To sum up: we cannot say that one of the abovementioned object control predicate types is more likely to trigger CC. However, the theoretical discussion of CC in Czech indicates that the constraints on CC in the case of object control do not depend only on object control or on the case of the controller itself. Constraints most probably involve a combination of several features. Therefore, the results for Croatian presented in this section are supplemented in Section \ref{Dative and accusative infinitive CL complements} with more detailed analyses, where not only the case of the matrix CL, but also the case of the critical CL are considered. This allows us to test the special object control person-case constraint described in Section \ref{Object control person-case constraint}. Finally, when the critical CL is of the \textsc{refl\textsubscript{\textsc{lex}}} type (see (\ref{(16.17a)}) and (\ref{(16.18a)}) below), the contrast between acceptance rates of the noCC and CC stimuli seems to be even bigger than for pronominal or \textsc{refl\textsubscript{2nd}} critical CLs: compare the third bar pair with the first and second in Panels C1 and C2 from Figure \ref{F16.1}.

\begin{exe}\ex\label{(16.17a)}
\gll Čak \textbf{nam}\textsubscript{1} dopuštaš\textsubscript{1} očitovati\textsubscript{2} \textbf{se}\textsubscript{2} o iznesenim prijedlozima.\\
 even us.\textsc{dat} allow.2\textsc{prs} declare.\textsc{inf} \textsc{refl} about presented suggestions\\
\glt ‘You even allow us to voice our opinion on the presented suggestions.’

\ex\label{(16.18a)}
\gll Vidno \textbf{je}\textsubscript{1} požurujem\textsubscript{1} očitovati\textsubscript{2} \textbf{se}\textsubscript{2} o iznesenim prijedlozima.\\
 visibly her.\textsc{acc} hurry.1\textsc{prs} declare.\textsc{inf} \textsc{refl} about presented suggestions \\
\glt ‘I visibly hurry her to voice her opinion on the presented suggestions.’ 
\end{exe}
 
\noindent In other words, whereas we observe no effect of the matrix CL type (i.e. controller), we do observe an effect of the infinitive CL type. Although this result is only marginally significant, it nicely fits \citet[][129f]{Hana07} object control reflexive constraint. Even though \citet[][]{Hana07} does not distinguish between different kinds of reflexives in Czech, our data show that the constraint seems to be slightly more salient in the case of \textsc{refl\textsubscript{\textsc{lex}}} than in the case of \textsc{refl\textsubscript{2nd}}. 

\subsection{Object control predicates with a \textsc{refl\textsubscript{2nd}} CL \textit{si} controller and with a \textsc{refl\textsubscript{2nd}} CL \textit{se} controller}
\label{Object control predicates with a refl2nd CL si controller and with a refl2nd CL se controller}

The fourth research question addresses the relationship between CC and noCC stimuli for sentences with matrix object control predicates which have the \textsc{refl\textsubscript{2nd}} CL controller \textit{si} (of the \textit{dozvoljavati si} ‘allow yourself’ type) and the \textsc{refl\textsubscript{2nd}} CL controller \textit{se} (of the \textit{prisiljavati se} ‘force yourself’ type). Contrasting these sentences reveals an overall similarity with respect to CC acceptance with one exception: object control sentences with a \textsc{refl\textsubscript{2nd}} CL \textit{se} controller whose infinitives govern a pronominal CL: compare Panels C3 and C4 from Figure \ref{F16.1}.

Regardless of the case of the reflexive controller, noCC sentences are clearly favoured over the same CC sentences. See examples of noCC (\ref{(16.19a)}) and CC (\ref{(16.19b)}) sentences with the reflexive dative controller \textit{si} presented below. 

\begin{exe}\ex\begin{xlist}
\ex[]{\label{(16.19a)}
\gll Vjerojatno \textbf{si}\textsubscript{1} dozvoljavate\textsubscript{1} konkurirati\textsubscript{2} \textbf{joj}\textsubscript{2} na drugim područjima.\\
 probably \textsc{refl} allow.2\textsc{prs} compete.\textsc{inf} her.\textsc{dat} on other areas\\}
\ex[?]{\label{(16.19b)}
\gll Vjerojatno \textbf{joj}\textsubscript{2} \textbf{si}\textsubscript{1} dozvoljavate\textsubscript{1} konkurirati\textsubscript{2} na drugim područjima.\\
 probably her.\textsc{dat} \textsc{refl} allow.2\textsc{prs} compete.\textsc{inf} on other areas\\}
\end{xlist}
\glt ‘You are probably allowing yourselves to compete with her in other areas.’ 
\end{exe}

\noindent In their CC version, sentences with an object control matrix predicate whose controller is a \textsc{refl\textsubscript{2nd}} CL are highly unacceptable in all conditions but one: when the \textsc{refl\textsubscript{2nd}} CL \textit{se} is the controller and the infinitive governs a pronominal CL: see example (\ref{(16.20b)}) presented below. This is clearly visible in Panel C4 in Figure \ref{F16.1}: compare the first bar pair with the second and the third.

\begin{exe}\ex\begin{xlist}
\ex\label{(16.20a)}
\gll Sada \textbf{se}\textsubscript{1} prisiljavate\textsubscript{1} zahvaliti\textsubscript{2} \textbf{im}\textsubscript{2} na nesebičnoj pomoći.\\
 now \textsc{refl} force.2\textsc{prs} thank.\textsc{inf} them.\textsc{dat} on unselfish help \\
\ex\label{(16.20b)}
\gll Sada \textbf{im}\textsubscript{2} \textbf{se}\textsubscript{1} prisiljavate\textsubscript{1} zahvaliti\textsubscript{2} na nesebičnoj pomoći. \\
 now them.\textsc{dat} \textsc{refl} force.2\textsc{prs} thank.\textsc{inf} on unselfish help \\
\end{xlist}
\glt ‘Now you are forcing yourselves to thank them for their unselfish help.’
\end{exe}

\largerpage[2]%longdistance
\noindent CC sentences like (\ref{(16.20b)}) are more acceptable than any other CC sentences containing an object control matrix predicate with either a \textsc{refl\textsubscript{2nd}} CL \textit{si} or a \textsc{refl\textsubscript{2nd}} CL \textit{se} controller. However, we must point out that even in this condition, CC sentences similar to (\ref{(16.20b)}) are less favoured than their noCC variants (similar to (\ref{(16.20a)})), and are acceptable to only slightly over 50\% of the participants. 

Summing up, with our data we extend the existing discussion on the object control constraint to matrix predicates with \textsc{refl\textsubscript{2nd}} CL \textit{si} and \textsc{refl\textsubscript{2nd}} CL \textit{se} controllers. The empirical analysis shows that in the case of object control matrix predicates, the type of controller (pronominal CL or \textsc{refl\textsubscript{2nd}}) apparently does not play an important role in CC. However, as already stated above, there is one exception. Namely, the climbing of pronominal CLs in the context of a \textsc{refl\textsubscript{2nd}}CL \textit{se} controller (sentences like (\ref{(16.20b)})) is somewhat acceptable to our participants. Nevertheless, even in this case the acceptance rate barely reached the threshold of 50\%.\footnote{For more information on acceptability thresholds see Section \ref{Acceptability thresholds for different types of judgment tasks}.} This finding is interesting for several reasons. First, we must emphasise that this setup is the only case in which the climbing of a pronominal CL out of an object-controlled infinitive seems to be possible. Second, in respect of CC, the \textsc{refl\textsubscript{2nd}} CL \textit{se} as controller behaves more like its phonetically closer \textsc{refl\textsubscript{\textsc{lex}}} CL \textit{se} than its morphologically and syntactically closer \textsc{refl\textsubscript{2nd}} CL \textit{si}: compare Panel B2 from Figure \ref{Fig.15.2B} and Panels C4 with C3 from Figure \ref{F16.1}.

One more finding merits attention in this section. The very low acceptance rates of CC sentences with the strings \textit{se se} and \textit{si se} (see second and third CCyes bars in both Panels C1 and C3 from Figure \ref{F16.1}) confirm \citet[][79]{Junghanns02} pseudo-twins constraint (see Section \ref{Pseudo-twins}). Since in corpora we do not find evidence for such CC structures (mixed clusters with pseudo-twins) and the acceptability of noCC stimuli (pseudodiaclisis of pseudo-twins) lies well below 80\%, we assume there should exist another, more acceptable construction.\footnote{For basic information on mixed clusters see Section \ref{Clitic ordering within the cluster}.}\textsuperscript{,}\footnote{For more information on pseudodiaclisis see Section \ref{Diaclisis and pseudodiaclisis}.} This could be either haplology/haplology of unlikes or the \textit{da}\textsubscript{2}-construction.\footnote{For more information on haplology of unlikes see Section \ref{Morphonological processes within the cluster}.} However, an additional acceptability judgment test would be needed to establish the most acceptable variant. 

\subsection{Object control predicates with a pronominal CL controller in the dative and with a \textsc{refl\textsubscript{2nd}} CL \textit{si} controller}
\label{Object control predicates with a pronominal CL controller in the dative and with a refl2nd CL si controller}
The fifth research question was formulated to compare CC and noCC sentences containing object control matrix predicates with a pronominal CL controller in the dative (\ref{(16.12a)}) and analogous sentences containing object control matrix predicates with a \textsc{refl\textsubscript{2nd}} CL \textit{si} controller (\ref{(16.19a)}). Our results revealed a similar pattern of acceptance with respect to CC: compare Panels C1 and C3 from Figure \ref{F16.1}. 

\begin{exe}
\exewidth{(12a)}
\exr{(16.12a)}
\gll Vjerojatno \textbf{im}\textsubscript{1} omogućuju\textsubscript{1} konkurirati\textsubscript{2} \textbf{joj}\textsubscript{2} na drugim područjima. \\
 probably them.\textsc{dat} enable.3\textsc{prs} compete.\textsc{inf} her.\textsc{dat} on other areas\\
\glt ‘They are probably making it possible for those ones to compete with her in other areas.’

\exr{(16.19a)}
\gll Vjerojatno \textbf{si}\textsubscript{1} dozvoljavate\textsubscript{1} konkurirati\textsubscript{2} \textbf{joj}\textsubscript{2} na drugim područjima.\\
 probably \textsc{refl} allow.2\textsc{prs} compete.\textsc{inf} her.\textsc{dat} on other areas\\
\glt ‘You are probably allowing yourselves to compete with her in other areas.’ 
\end{exe}


\noindent Participants clearly favoured the noCC sentence variants over the CC variants, and perceived the CC stimuli as highly unacceptable. Therefore, we can conclude on empirical grounds that CLs cannot climb into an object-control matrix clause whose controller is in the dative case, regardless of the controller type (pronominal vs \textsc{refl\textsubscript{2nd}} CL). 

\subsection{Object control predicates with a pronominal CL controller in the accusative and with a \textsc{refl\textsubscript{2nd}} CL \textit{se} controller}
\label{Object control predicates with a pronominal CL controller in the accusative and with a refl2nd CL se controller}

The sixth research question was intended to capture the difference between noCC and CC sentences with object control predicates whose controller is a pronominal CL in the accusative (\ref{(16.21)}) and object control sentences whose controller is the \textsc{refl\textsubscript{2nd}} CL \textit{se} (\ref{(16.22)}). The latter type is discussed above (Section \ref{Object control predicates with a refl2nd CL si controller and with a refl2nd CL se controller}, Panels C1 and C3 in Figure \ref{F16.1}.). Our comparison revealed comparable preference of noCC sentences (similar to (\ref{(16.21a)}) and (\ref{(16.22a)})) and very low acceptability of CC sentences (similar to (\ref{(16.21b)}) and (\ref{(16.22b)})) in all conditions but one: compare Panels C2 and C4 from Figure \ref{F16.1}. 

\begin{exe}\ex\label{(16.21)}
\begin{xlist}
\ex[]{\label{(16.21a)}
\gll Javno \textbf{ih}\textsubscript{1} obvezujem\textsubscript{1} pozivati\textsubscript{2} \textbf{ga}\textsubscript{2} na mjesečne sastanke.\\
 publicly them.\textsc{acc} oblige.1\textsc{prs} invite.\textsc{inf} him.\textsc{acc} on monthly meetings\\}
\ex[?]{\label{(16.21b)}
\gll Javno \textbf{ih}\textsubscript{1} \textbf{ga}\textsubscript{2} obvezujem\textsubscript{1} pozivati\textsubscript{2} na mjesečne sastanke.\\
 publicly them.\textsc{acc} him.\textsc{acc} oblige.1\textsc{prs} invite.\textsc{inf} on monthly meetings\\}
\end{xlist}
\glt ‘I publicly oblige them to invite him to the monthly meetings.’

\ex\label{(16.22)}\begin{xlist}
\ex\label{(16.22a)}
\gll Nevoljko \textbf{se}\textsubscript{1} prisiljavamo\textsubscript{1} pozivati\textsubscript{2} \textbf{ga}\textsubscript{2} na mjesečne sastanke.\\
 unwillingly \textsc{refl} force.1\textsc{prs} invite.\textsc{inf} him.\textsc{acc} on monthly meetings\\
\ex\label{(16.22b)}
\gll Nevoljko \textbf{ga}\textsubscript{2} \textbf{se}\textsubscript{1} prisiljavamo\textsubscript{1} pozivati\textsubscript{2} na mjesečne sastanke.\\
 unwillingly him.\textsc{acc} \textsc{refl} force.1\textsc{prs} invite.\textsc{inf} on monthly meetings\\
\end{xlist}\glt ‘We begrudgingly force ourselves to invite him to the monthly meetings.’
\end{exe}


\largerpage[2]%longdistance
%noindent needed since LaTeX moved the figure.
\noindent As already shown in Section \ref{Object control predicates with a refl2nd CL si controller and with a refl2nd CL se controller}, object control sentences with a \textsc{refl\textsubscript{2nd}} CL \textit{se} controller and a pronominal infinitive CL are the only category of object control sentences which are somewhat acceptable in their CC version. The acceptance rate for such sentences (see example presented in (\ref{(16.22b)}) above) is above 50\%. However, as already pointed out in Section \ref{Object control predicates with a refl2nd CL si controller and with a refl2nd CL se controller}, even for this sentence type the noCC variant is still preferred over the CC variant.

Our study is the first attempt to include the difference in controller type (pro\-no\-mi\-nal vs reflexive) in the discussion of constraints on CC in the context of object control matrix predicates. As we showed in the previous section, CC is blocked no matter if the controller is a dative pronominal CL or the \textsc{refl\textsubscript{2nd}} CL \textit{si} in the dative. In contrast, when the controller is in the accusative, CC is blocked only if the controller is a pronominal CL, whereas in the case of the \textsc{refl\textsubscript{2nd}} CL \textit{se} controller, only the climbing of \textsc{refl\textsubscript{\textsc{lex}}} and \textsc{refl\textsubscript{2nd}} CLs out of the infinitive is truly blocked. Pronominal CLs can climb out of an infinitive which is controlled by the \textsc{refl\textsubscript{2nd}} CL \textit{se}. Nevertheless, it must be pointed out that the CC stimuli in the mentioned setup barely reached the threshold of acceptability. 

\subsection{Reflexive subject control predicates and object control predicates with a \textsc{refl\textsubscript{2nd}} CL \textit{se} controller}
\label{Reflexive subject control predicates and object control predicates with a refl2md CL se controller}

\largerpage[2]

The seventh research question was formulated to establish the relationship between CC and noCC variants for sentences containing reflexive subject control matrix predicates and object control matrix predicates with the \textsc{refl\textsubscript{2nd}} CL \textit{se} controller, which have already been discussed in Sections \ref{Simple subject control predicates and subject control predicates with refllex CL se}, \ref{Object control predicates with a refl2nd CL si controller and with a refl2nd CL se controller}, and \ref{Object control predicates with a pronominal CL controller in the accusative and with a refl2nd CL se controller} As we can see from Panel B2 in Figure \ref{Fig.15.2B} and C4 in Figure \ref{F16.1}, the ratings of these sentences give strikingly similar patterns. 


If an infinitive governs the \textsc{refl\textsubscript{2nd}} CLs \textit{se} and \textit{si} or the \textsc{refl\textsubscript{\textsc{lex}}} CL \textit{se}, the two contrasted sentence categories showed identical patterns of acceptance: CC sentence variants were accepted very rarely (13\%, 17\%, 12\%, 15\%) and were outranked by noCC sentence variants (72\%, 69\%, 65\%, 67\% respectively). For examples of the latter see the sentences in (\ref{(16.15a)}) and (\ref{(16.23a)}) presented below. 

\begin{exe}\exr{(16.15a)}
\gll Silno \textbf{se}\textsubscript{1} boje\textsubscript{1} očitovati\textsubscript{2} \textbf{se}\textsubscript{2} o iznesenim prijedlozima.\\
 immensely \textsc{refl} be.afraid.3\textsc{prs} declare.\textsc{inf} \textsc{refl} about presented suggestions\\
\glt ‘They are immensely afraid to voice their opinion on the presented suggestions.’

\ex\label{(16.23a)}
\gll Redovito \textbf{se}\textsubscript{1} ovlašćujem\textsubscript{1} očitovati\textsubscript{2} \textbf{se}\textsubscript{2} o iznesenim prijedlozima.\\
 regularly \textsc{refl} authorise.1\textsc{prs} declare.\textsc{inf} \textsc{refl} about presented suggestions\\
\glt ‘I regularly authorise myself to voice my opinion on the presented suggestions.’
\end{exe}

\noindent As argued in Sections \ref{Simple subject control predicates and subject control predicates with refllex CL se} and \ref{Object control predicates with a refl2nd CL si controller and with a refl2nd CL se controller}, the acceptability of noCC stimuli (pseudodiaclisis), which is well below 80\%, indicates that either haplology (deletion of one \textit{se} CL) or the \textit{da}\textsubscript{2}-construction might be more acceptable. Nevertheless, without an additional acceptability judgment test we can only speculate which construction is the most acceptable variant. 

Furthermore, in Section \ref{Object control predicates with a refl2nd CL si controller and with a refl2nd CL se controller} and this section we deliver empirical proof supporting Rosen’s claim that the pseudo-twins constraint is “blind” to different types of reflexives \citep[cf.][104]{Rosen14}. As may be seen from Panel B2 of Figure \ref{Fig.15.2B} and Panels C3 and C4 of \ref{F16.1}, neither the \textsc{refl\textsubscript{2nd}} CLs \textit{se} and \textit{si} nor the \textsc{refl\textsubscript{\textsc{lex}}} CL \textit{se} can climb out of the infinitive if there is already a reflexive in the matrix clause.\footnote{The reader should bear in mind that we did not test sentences with haplology. This was not possible due to our experiment design. In order to test if CC is possible in such structures separate experiments should be designed. For more information on CC in the context of reflexives and haplology see Section \ref{Pseudo-twins}.} If they are part of the matrix clause, both tested types of reflexives, \textsc{refl\textsubscript{\textsc{lex}}} CL \textit{se} and \textsc{refl\textsubscript{2nd}} CL \textit{si} and \textit{se}, block the climbing of other infinitive reflexive CLs. In other words, sentences with mixed reflexive clusters have very low acceptance rates. Since this constraint appears with both subject and object control matrix predicates, we can conclude that it does not depend on the difference between the mentioned predicates.

In contrast, in the case of reflexive subject control sentences with the \textsc{refl\textsubscript{\textsc{lex}}} CL \textit{se} and object control sentences with the \textsc{refl\textsubscript{2nd}} CL \textit{se} controller, CC sentence variants in which infinitives govern a pronominal CL tend to be acceptable: see examples of such sentences in (\ref{(16.24b)}) and (\ref{(16.25b)}). 

\begin{exe}
\ex\label{(16.24b)}
\gll Barem \textbf{ih}\textsubscript{2} \textbf{se}\textsubscript{1} usuđujem\textsubscript{1} zaposliti\textsubscript{2} na neodređeno.\\
 at.least them.\textsc{acc} \textsc{refl} dare.1\textsc{prs} hire.\textsc{inf} on indefinitely\\
\glt ‘At least I dare to employ them indefinitely.’

\ex\label{(16.25b)}
\gll Doslovno \textbf{ih}\textsubscript{2} \textbf{se}\textsubscript{1} primoravam\textsubscript{1} zaposliti\textsubscript{2} na neodređeno.\\
 literally them.\textsc{acc} \textsc{refl} compel.1\textsc{prs} hire.\textsc{inf} on indefinitely\\
\glt ‘I literally compel myself to employ them on a permanent basis.’ 
\end{exe}
 
\noindent The only difference between the two is that acceptance of CC sentences with a reflexive subject control matrix predicate and an infinitive which governs a pronominal CL is significantly higher (82\%; see example (\ref{(16.24b)}) provided above) than acceptance of CC sentences with an object control matrix predicate with a \textsc{refl\textsubscript{2nd}} CL \textit{se} controller and an infinitive which governs a pronominal CL (54\%; see example (\ref{(16.25b)}) provided above). Although Panel B2 in Figure \ref{Fig.15.2B} and Panel C4 in Figure \ref{F16.1} suggest that there are some differences among the acceptance rates for noCC sentences, these differences were not statistically significant (88\% and 79\%). 

Thus, as we already discussed in the previous section, in the case of object control predicates the \textsc{refl\textsubscript{2nd}} CL \textit{se} controller appears to be more similar to the \textsc{refl\textsubscript{\textsc{lex}}} CL \textit{se} than to the \textsc{refl\textsubscript{2nd}} CL \textit{si} controller when it comes to CC. 

\subsection{Pronominal and reflexive (\textsc{refl\textsubscript{\textsc{lex}}} CL \textit{se} and \textsc{refl\textsubscript{2nd}} CLs \textit{se} and \textit{si}) infinitive CLs}
\label{Pronominal and reflexive infinitive CLs}
The eighth research question addressed the importance of infinitive CL type for CC. Although (due to the observed three-way interaction) we have already discussed the role of infinitive CL type in the previous sections, we will briefly summarise the findings here. We can distinguish three effect types: sentences containing only one CL, reflexive subject control sentences and object control sentences with a \textsc{refl\textsubscript{2nd}} CL \textit{se} controller, and lastly object control sentences with a \textsc{refl\textsubscript{2nd}} CL \textit{si} or a pronominal CL controller. 

First, for sentences containing only one CL, i.e. sentences with raising (examples \ref{(16.11b)}, \ref{(16.26b)}, \ref{(16.27b)}) and simple subject control (examples \ref{(16.13b)}, \ref{(16.28b)}, \ref{(16.29b)}) matrix predicates, the type of infinitive CL plays no role. In these two sentence categories, the effect of CC is universal, i.e. consistent across all types of infinitive CLs (pronominal CLs, \textsc{refl\textsubscript{2nd}} CLs \textit{se} and \textit{si}, \textsc{refl\textsubscript{\textsc{lex}}} CL \textit{se}). 

\begin{exe}\ex\label{(16.26b)}
\gll Zato \textbf{se}\textsubscript{2} moram\textsubscript{1} braniti\textsubscript{2} od apsurdnih optužbi. \\
 for.that.reason \textsc{refl} must.1\textsc{prs} defend.\textsc{inf} from absurd accusations \\
\glt ‘For that reason, I have to defend myself from absurd accusations.’

\ex\label{(16.27b)}
\gll Aktivno \textbf{se}\textsubscript{2} nastavljaju\textsubscript{1} baviti\textsubscript{2} tim pitanjem. \\
 actively \textsc{refl} continue.3\textsc{prs} deal.\textsc{inf} that question \\
\glt ‘They are actively continuing to deal with that question.’

\ex\label{(16.28b)}
\gll Zato \textbf{se}\textsubscript{2} odlučuje\textsubscript{1} braniti\textsubscript{2} od apsurdnih optužbi. \\
 for.that.reason \textsc{refl} decide.3\textsc{prs} defend.\textsc{inf} from absurd accusations \\
\glt ‘For that reason, s/he decides to defend herself/himself from absurd accusations.’

\ex\label{(16.29b)}
\gll Aktivno \textbf{se}\textsubscript{2} pokušavaju\textsubscript{1} baviti\textsubscript{2} tim pitanjem. \\
 actively \textsc{refl} try.3\textsc{prs} deal.\textsc{inf} that question \\
\glt ‘They are actively trying to deal with that question.’
\hfill 
\end{exe}

\noindent Second, as already discussed in the previous section, for reflexive subject control sentences (\ref{(16.15b)}, \ref{(16.37b)}, \ref{(16.38b)}) and object control sentences with a \textsc{refl\textsubscript{2nd}} CL \textit{se} controller (\ref{(16.39b)}, \ref{(16.40b)}, \ref{(16.41b)}), the effect of CC is identical for the infinitive CLs: \textsc{refl\textsubscript{\textsc{lex}}} CL \textit{se} (\ref{(16.15b)}, \ref{(16.39b)}) and \textsc{refl\textsubscript{2nd}} CL \textit{se} (\ref{(16.37b)}, \ref{(16.40b)}) and \textit{si} (\ref{(16.38b)}, \ref{(16.41b)}). No reflexive-type CLs can climb out of the infinitive into the matrix clause. The strings \textit{se se} and \textit{si se} are ruled out. 

\begin{exe}
\ex[?]{\label{(16.37b)}
\gll Pogotovno \textbf{se}\textsubscript{2} \textbf{se}\textsubscript{2} boji\textsubscript{1} zaštititi\textsubscript{2} od smrtonosnih bolesti.\\
 especially \textsc{refl} \textsc{refl} be.afraid.3\textsc{prs} protect.\textsc{inf} from lethal diseases\\
\glt ‘He/she is especially afraid to protect himself/herself from lethal diseases.’}

\ex[?]{\label{(16.38b)}
\gll Pogotovno \textbf{si}\textsubscript{2} \textbf{se}\textsubscript{2} boji\textsubscript{1} nauditi\textsubscript{2} takvom odlukom.\\ 
 especially \textsc{refl} \textsc{refl} be.afraid.3\textsc{prs} protect.\textsc{inf} that.kind decision\\
\glt ‘He/she is especially afraid of harming himself/herself with such decision.’}

\ex[?]{\label{(16.39b)}
\gll Potpuno \textbf{se}\textsubscript{1} \textbf{se}\textsubscript{2} prisiljavaš\textsubscript{1} oslanjati\textsubscript{2} na njegovu riječ. \\
 even \textsc{refl} \textsc{refl} force.2\textsc{prs} rely.\textsc{inf} on his word\\
\glt ‘You completely force yourself to rely on his word.’}

\ex[?]{\label{(16.40b)}
\gll Čak \textbf{se}\textsubscript{1} \textbf{se}\textsubscript{2} prisiljava\textsubscript{1} zaštititi\textsubscript{2} od smrtonosnih bolesti. \\
 even \textsc{refl} \textsc{refl} force.3\textsc{prs} protect.\textsc{inf} {} lethal diseases\\
\glt ‘He/she even forces himself/herself to protect himself/herself from lethal diseases.’}

\ex[?]{\label{(16.41b)}
\gll Nesumnjio \textbf{si}\textsubscript{2} \textbf{se}\textsubscript{2} prisiljavamo\textsubscript{1} štedjeti\textsubscript{2} za sigurnu budućnost. \\
 Undoubtedly \textsc{refl} \textsc{refl} force.1\textsc{prs} save.\textsc{inf} for secure future\\
\glt ‘Undoubtedly we force ourselves to save for a secure future.’}
\end{exe}

\noindent However, the contrast between CC and noCC stimuli is smaller for infinitive pronominal CLs. In other words, pronominal CLs can climb out of the infinitive in the case of reflexive subject control matrix predicates (\ref{(16.24b)}) and object control matrix predicates with the \textsc{refl\textsubscript{2nd}} controller \textit{se} (\ref{(16.25b)}).

\begin{exe}
\exr{(16.24b)}
\gll Barem \textbf{ih}\textsubscript{2} \textbf{se}\textsubscript{1} usuđujem\textsubscript{1} zaposliti\textsubscript{2} na neodređeno. \\
 at.least them.\textsc{acc} \textsc{refl} dare.1\textsc{prs} hire.\textsc{inf} on indefinitely \\
\glt ‘At least I dare to employ them indefinitely.’

\exr{(16.25b)}
\gll Doslovno \textbf{ih}\textsubscript{2} \textbf{se}\textsubscript{1} primoravam\textsubscript{1} zaposliti\textsubscript{2} na neodređeno. \\
 literally them.\textsc{acc} \textsc{refl} compel.1\textsc{prs} hire.\textsc{inf} on indefinitely \\
\glt ‘I literally compel myself to employ them indefinitely.’
\end{exe}
 
\noindent  Third, for object control sentences with a pronominal CL controller in the dative (examples \ref{(16.30b)} and \ref{(16.31b)}), the accusative (examples \ref{(16.32b)} and \ref{(16.33b)}), or a \textsc{refl\textsubscript{2nd}} CL \textit{si} controller (examples \ref{(16.34b)} and \ref{(16.35b)}), the effect of CC is identical for pronominal and \textsc{refl\textsubscript{2nd}} \textit{si} and \textit{se} infinitive CLs: they cannot climb out of the infinitive into the matrix clause.

\begin{exe}\ex[?]{\label{(16.30b)}
\gll Opet \textbf{im}\textsubscript{1} \textbf{mu}\textsubscript{2} dopuštate\textsubscript{1} prigovarati\textsubscript{2} zbog lošeg društva.\\
 again them.\textsc{dat} him.\textsc{dat} allow.2\textsc{prs} complain.\textsc{inf} because bad company\\
\glt ‘You are allowing them to complain about the bad company he keeps again.’} 

\ex[?]{\label{(16.31b)}
\gll Strogo \textbf{mi}\textsubscript{1} \textbf{se}\textsubscript{2} preporučuješ\textsubscript{1} odijevati\textsubscript{2} po najnovijoj modi.\\
 strictly me.\textsc{dat} \textsc{refl} suggest.2\textsc{prs} dress.\textsc{inf} on latest fashion\\
\glt ‘You are strictly recommending me to dress according to the latest fashion.’}

\ex[?]{\label{(16.32b)}
\gll Opet \textbf{ga}\textsubscript{1} \textbf{mu}\textsubscript{2} prisiljavaš\textsubscript{1} prigovarati\textsubscript{2} zbog lošeg društva.\\
 again him.\textsc{acc} him.\textsc{dat} force.2\textsc{prs} complain.\textsc{inf} because bad company\\
\glt ‘You are forcing him to complain about the bad company that one keeps again.’}

\ex[?]{\label{(16.33b)}
\gll Zbilja \textbf{ga}\textsubscript{1} \textbf{se}\textsubscript{2} tjeraš\textsubscript{1} odijevati\textsubscript{2} po najnovijoj modi.\\
 truly him.\textsc{acc} \textsc{refl} force.2\textsc{prs} dress.\textsc{inf} on latest fashion\\
\glt ‘You are truly forcing him to dress according to the latest fashion.’}

\ex[?]{\label{(16.34b)}
\gll Opet \textbf{mu}\textsubscript{2} \textbf{si}\textsubscript{1} dopuštaš\textsubscript{1} prigovarati\textsubscript{2} zbog lošeg društva.\\
 again him.\textsc{dat} \textsc{refl} allow.2\textsc{prs} complain.\textsc{inf} because bad company\\
\glt ‘You are allowing yourself to complain about the bad company he keeps again.’}

\ex[?]{\label{(16.35b)}
\gll Ponekad \textbf{si}\textsubscript{1} \textbf{se}\textsubscript{2} priuštim\textsubscript{1} odjenuti\textsubscript{2} po najnovijoj modi.\\
 sometimes \textsc{refl} \textsc{refl} afford.1\textsc{prs} dress.\textsc{inf} on latest fashion\\
\glt ‘Sometimes I allow myself to dress according to the latest fashion.’}
\end{exe}

\noindent The above empirical findings can be summarised as follows. In the case of raising and simple subject control predicates, the type of the infinitive CL does not influence CC at all. In the case of reflexive subject control matrix predicates and object control matrix predicates with the \textsc{refl\textsubscript{2nd}} controller \textit{se}, we observe an effect of CL type on CC: pronominal infinitive complements can climb whereas the remaining two reflexive CL types cannot.

\subsection{Dative and accusative infinitive CL complements}
\label{Dative and accusative infinitive CL complements}

The ninth research question tackles the relationship between the case of the infinitive CL and CC. To answer it, we performed additional regression analyses with acceptance as dependent variable and CC (CC, noCC), infinitive CL type (pronominal CL, \textsc{refl\textsubscript{2nd}} CL), and infinitive CL case (dative, accusative) as independent variables. Sentences containing the \textsc{refl\textsubscript{\textsc{lex}}} CL \textit{se} as critical CL were excluded from the analyses. To avoid a possible four-way interaction which we would not be able to interpret, we conducted the analyses separately for each predicate type.

\begin{table}
\caption[Generalised mixed effects regression model fitted to acceptance data ($1=\text{acceptable}$; $0=\text{unacceptable}$) for sentences containing an object control matrix predicate whose controller is a pronominal CL in the dative.]{Generalised mixed effects regression model fitted to acceptance data ($1=\text{acceptable}$; $0=\text{unacceptable}$) for sentences containing an object control matrix predicate whose controller is a pronominal CL in the dative.\label{T16.8}}
\begin{tabularx}{\textwidth}{Q *{4}{r}@{\,}l}
\lsptoprule
Random Effects & & &  Var & SD & \\\midrule
\multicolumn{3}{Q}{participants (intercept adjustments)} & 0.770 & 0.877 & \\
\multicolumn{3}{Q}{items (intercept adjustments)} & 0.463 & 0.680 & \\\midrule
Fixed effects & {Est.} & \multicolumn{1}{c}{SE} & \multicolumn{1}{c}{$z$} & {$\text{Pr}(>|z|)$} &\\\midrule
intercept (CC - no; infinitive complement case - accusative) & 1.461 & 0.237 & 6.167 & $<0.0001$ & *** \\
CC - yes & $-2.715$ & 0.245 & $-11.098$ & $<0.0001$ & ***\\
infinitive complement case - dative & $-1.964$ & 0.238 & $-8.244$ & $<0.0001$ & *** \\
\lspbottomrule
\end{tabularx}
\end{table}

\begin{figure}
\caption[Effect of infinitive CL case on CC acceptance rate: Panel D1 (left) – sentences with object control matrix verb and pronominal dative CL controller, Panel D2 (right) – sentences with object control matrix verb and pronominal accusative CL controller.]{Effect of infinitive CL case on CC acceptance rate: Panel D1 (left) – sentences with object control matrix verb and pronominal dative CL controller, Panel D2 (right) – sentences with object control matrix verb and pronominal accusative CL controller.}
\label{F16.9}
\includegraphics[width=0.85\textwidth]{figures/Fig.15.8.pdf}
\end{figure}

Regression models reveal that the effect of the case of the critical CL is significant only for object control matrix predicates with pronominal CL controllers.\footnote{Figure \ref{F16.9} presents data from models presented in Table \ref{T16.8} and \ref{T16.9}. As we already emphasised at the beginning of this section, we conducted analyses separately for each predicate type to avoid a possible four-way interaction.} Therefore, we now report only these results. They are shown in Figure \ref{F16.9} and Tables \ref{T16.8} (dative controller) and \ref{T16.9} (accusative controller).

As presented in Table \ref{T16.8} and Panel D1 from Figure \ref{F16.9}, if an object control matrix predicate with a pronominal CL controller in the dative is followed by an infinitive which governs either a pronominal CL in the accusative or the \textsc{refl\textsubscript{2nd}} CL \textit{se}, the CC sentence (such as those in (\ref{(16.42b)}) and (\ref{(16.43b)})) is more acceptable than the CC sentence in which the infinitive governs a pronominal CL in the dative or the \textsc{refl\textsubscript{2nd}} CL \textit{si} (such as those in (\ref{(16.30b)}) and (\ref{(16.44b)})).\footnote{The formula of the reported model is: Acceptance $\sim$ CC + Infinitive complement case + (1|Participant) + (1|Item).} 

\begin{exe}\ex[?]{\label{(16.42b)}
\gll Uporno \textbf{mi}\textsubscript{1} \textbf{ga}\textsubscript{2} naređuju\textsubscript{1} pozivati\textsubscript{2} na mjesečne sastanke.\\
 persistently me.\textsc{dat} him.\textsc{acc} order.3\textsc{prs} invite.\textsc{inf} on monthly meetings\\
\glt ‘They are persistently ordering me to invite him to the monthly meetings.’}

\ex[?]{\label{(16.43b)}
\gll Rado \textbf{nam}\textsubscript{1} \textbf{se}\textsubscript{2} pomažete\textsubscript{1} postaviti\textsubscript{2} prema takvom ponašanju. \\
 gladly us.\textsc{dat} \textsc{refl} help.2\textsc{prs} set.up.\textsc{inf} towards such behaviour \\ 
\glt ‘You gladly help us stand up to such behaviour.’}

\exr{(16.30b)}[?]{
\gll Opet \textbf{im}\textsubscript{1} \textbf{mu}\textsubscript{2} dopuštate\textsubscript{1} prigovarati\textsubscript{2} zbog lošeg društva. \\
 again them.\textsc{dat} him.\textsc{dat} allow.2\textsc{prs} complain.\textsc{inf} because bad company \\ 
\glt ‘You are allowing them to complain about the bad company he keeps again.’}

\ex[?]{\label{(16.44b)}
\gll Odnedavno \textbf{mu}\textsubscript{1} \textbf{si}\textsubscript{2} omogućuju\textsubscript{1} posvetiti\textsubscript{2} dovoljno vremena.\\
 recently him.\textsc{dat} \textsc{refl} enable.3\textsc{prs} devote.\textsc{inf} enough time\\ 
\glt ‘Recently, they have been making it possible for him to devote enough time to himself.’}
\end{exe}

\noindent A slightly different pattern of results is observed for sentences containing an object control predicate with a pronominal CL controller in the accusative followed by an infinitive whose complement is a pronominal CL in the accusative or the \textsc{refl\textsubscript{2nd}} CL \textit{se}. As presented in Table \ref{T16.9} and Panel D2 from Figure \ref{F16.9}, in this case we observe an interaction of CC and the case of the infinitive CL. Again, CC variants are less acceptable than the same sentences without CC. Sentences with infinitive CLs in the accusative are more acceptable than sentences with infinitive CLs in the dative, but only in CC sentence variants (in noCC sentence variants infinitive CL case has no effect). 

\begin{exe}\exr{(16.33b)}[?]{
\gll Zbilja \textbf{ga}\textsubscript{1} \textbf{se}\textsubscript{2} tjeraš\textsubscript{1} odijevati\textsubscript{2} po najnovijoj modi.\\
 truly him.\textsc{acc} \textsc{refl} force.2\textsc{prs} dress.\textsc{inf} on latest fashion\\ 
\glt ‘You are truly forcing him to dress according to the latest fashion.’}
\end{exe}
 
\noindent In other words, stimuli like the one presented in (\ref{(16.33b)}) above are more acceptable than sentences like those presented below. It means that sentences with an object control matrix predicate which have a dative (\ref{(16.12b)}) or an accusative (\ref{(16.16b)}) pronominal CL controller are, just like those with the \textsc{refl\textsubscript{2nd}} CL \textit{si} controller (\ref{(16.19b)}), less acceptable in their CC version if the infinitive CL is in the dative. 
 
\begin{exe}\exr{(16.12b)}[?]{
\gll Vjerojatno \textbf{im}\textsubscript{1} \textbf{joj}\textsubscript{2} omogućuju\textsubscript{1} konkurirati\textsubscript{2} na drugim područjima.\\
 probably them.\textsc{dat} her.\textsc{dat} enable.3\textsc{prs} compete.\textsc{inf} on other areas\\ 
\glt ‘They are probably making it possible for those ones to compete with her in other areas.’}

\ex[?]{\label{(16.16b)}
\gll Zakonski \textbf{joj}\textsubscript{2} \textbf{te}\textsubscript{1} primoravamo\textsubscript{1} konkurirati\textsubscript{2} na drugim područjima.\\
 legally her.\textsc{dat} you.\textsc{acc} compel.1\textsc{prs} compete.\textsc{inf} on other areas\\ 
\glt ‘We are legally compelling you to compete with her in other areas.’}

\exr{(16.19b)}[?]{
\gll Vjerojatno \textbf{joj}\textsubscript{2} \textbf{si}\textsubscript{1} dozvoljavate\textsubscript{1} konkurirati\textsubscript{2} na drugim područjima.\\
 probably her.\textsc{dat} \textsc{refl} allow.2\textsc{prs} compete.\textsc{inf} on other areas\\ 
\glt ‘You are probably allowing yourselves to compete with her in other areas.’}
\hfill 
\end{exe}

\noindent Furthermore, object control sentences with an accusative pronominal CL controller are less acceptable in their CC version if the infinitive CL is a pronoun in the accusative, like in the example (\ref{(16.45b)}) presented below.

\begin{exe}\ex[?]{\label{(16.45b)}
\gll Uistinu \textbf{me}\textsubscript{1} \textbf{ih}\textsubscript{2} potičeš\textsubscript{1} tužiti\textsubscript{2} za medijsku klevetu.\\
 truly me.\textsc{acc} them.\textsc{acc} encourage.2\textsc{prs} sue.\textsc{inf} for media slander\\ 
\glt ‘You are truly encouraging me to sue them for media slander.’}
\end{exe}

\begin{table}
\caption{Generalised mixed effects regression model fitted to acceptance data ($1=\text{acceptable}$; $0=\text{unacceptable}$) for sentences containing an object control matrix predicate whose governor is a pronominal CL in the accusative.\label{T16.9}}
\begin{tabularx}{\textwidth}{Q S[table-format=-1.4] *{2}{r} S[table-format=<1.4]@{\,}l}
\lsptoprule
Random Effects & & & Var. & {SD} &\\\midrule
\multicolumn{3}{l}{participants (intercept adjustments)} & 0.508 & 0.713 & \\
\multicolumn{3}{l}{items (intercept adjustments)} & 0.669 & 0.818 & \\\midrule
Fixed Effects & {Est.} & \multicolumn{1}{c}{SE} & \multicolumn{1}{c}{$z$} & {$\text{Pr}(>|z|)$} & \\\midrule
intercept (CC - no; infinitive complement case - accusative) & 0.718 & 0.261 & 2.747 & 0.006 & **\\
CC - yes & -2.486 & 0.359 & $-6.930$ & <0.0001 & ***\\
infinitive CL case - dative& -0.5623 & 0.335& $-1.678$ & 0.093 & \\
CC - yes: infinitive CL case - dative & -1.309 & 0.550 & $-2.379$ & 0.017 & *\\
\lspbottomrule
\end{tabularx}
\end{table}

%noindent, because LaTeX moved the table above
\noindent However, although the accusative case of the infinitive CL is a statistically relevant factor for CC with respect to object control matrix predicates with a pronominal CL controller, we still cannot claim that there is no constraint on the climbing of pronominal accusative CLs in such a context (pace \citealt[][]{Dotlacil04}, \citealt[][]{Rezac05}). Namely, as can be seen in Figure \ref{F16.9}, even if the climbing CL is in the accusative, object control sentences with dative and accusative pronominal CL controllers still do not reach the acceptability threshold of 50\% in their CC versions. 

\section{Reaction time analysis}
\label{Reaction time analysis}

One reason for choosing the speeded yes-no acceptability judgment task was the possibility of obtaining a second, control measure – reaction time (see Section \ref{Types of psycholinguistic tasks}). In the present section we analyse the reaction time for accepted sentences.

The acceptance rates for almost all two-CL sentences are very low. In other words, the number of observations available for this part of the analysis is very small and does not allow the impact of all previously discussed predictors on reaction time to be checked. 

Therefore, we pooled the data for all stimuli with two CLs, i.e. all sentences containing a reflexive subject control matrix predicate and all sentences with object control matrix predicates. Although the acceptance rates for stimuli with one CL were high, in order to make the conditions comparable, we also pooled the data for all stimuli with one CL, i.e. sentences with raising and simple subject control matrix predicates. We thus compared processing latencies for one-CL and two-CL CC and noCC sentences. 

\begin{figure}
\caption{The effect of CC and number of CLs on processing latencies in accepted sentences. Sentences with one CL -- bars on the left; sentences with two CLs -- bars on the right. NoCC sentences in black; CC sentences in grey.}
\label{F16.11}
\includegraphics[width=0.7\textwidth]{figures/Fig.15.9.pdf}
\end{figure}

As presented in Figure \ref{F16.11}, stimuli with only one CL took less time to be accepted than sentences with two CLs. Additionally, within the one-CL sentences, stimuli with CC were accepted faster than the corresponding noCC stimuli. In other words, CC sentences with raising and simple subject control matrix predicates were accepted faster than the noCC version of these sentences with the same matrix predicates. Two-CL sentences (i.e. sentences with reflexive subject control and object control matrix predicates) show no such advantage, as both variants are equally demanding in terms of processing.

The results of regression analysis summarised in Table \ref{T16.11} confirm these results.\footnote{The formula of the model is: {RT $\sim$ Number of CLs * CC + Infinitive CL type + (1|Participant) + (1|Item)}} Number of CLs and CC for stimuli with one CL are significant factors, and so is their interaction. There seems to be a numerical trend suggesting that two-CL CC sentences are processed even longer than their noCC counterparts, but this difference is not statistically significant. 

\begin{table}
\caption[Mixed-effect regression model fitting CC and number of CLs to processing latencies for accepted sentences]{Mixed-effect regression model fitting CC and number of CLs to processing latencies for accepted sentences}
\label{T16.11}
\begin{tabularx}{\textwidth}{Q S[table-format=-1.3] r S[table-format=3.3] S[table-format=<1.4]@{\,}l}
\lsptoprule
Random Effects &&& {Var.} & {SD} &\\\midrule
\multicolumn{3}{l}{participants (intercept adjustments)} & 0.050& 0.224&\\
\multicolumn{3}{l}{items (intercept adjustments)} & 0.009& 0.094&\\\midrule
Fixed Effects& {Est.} & {SE} & {$z$} & {$\text{Pr}(>|z|)$}&\\\midrule
intercept (number of CLs – two; CC – no;  infinitive CL type – pronominal CL)& 8.296 & 0.018 & 455.547 & <0.0001 & *** \\
number of CLs – one& -0.126 & 0.032 & -3.968 & <0.0001 & *** \\
CC – yes & 0.019 & 0.014 & 1.382 & 0.167 & \\
infinitive CL type – \textsc{refl\textsubscript{\textsc{lex}}} CL \textit{se} & -0.033 & 0.012 & -2.684 & 0.008 & ** \\
infinitive CL type – \textsc{refl\textsubscript{2nd}} & -0.016 & 0.012 & -1.286 & 0.199 & \\
number of CLs – one : CC – yes & -0.109 & 0.022 & -4.940 & <0.0001 & *** \\
\lspbottomrule
\end{tabularx}
\end{table}


Our findings clearly indicate a processing difference with respect to CC in two types of sentences. The reaction time analysis results for accepted sentences are in accordance with acceptance rate analysis results. One-CL sentences, i.e. sentences with raising and simple subject control matrix predicates, are more acceptable and processed faster in their CC version than in their noCC version. Similar patterns cannot be observed for sentences with two CLs.

\section{Conclusions}
\label{Conclusions:16}

Hitherto studies of CC in Slavonic languages have had a weak empirical basis. Our study is the first psycholinguistic study which delivers experimental evidence for this syntactic phenomenon. We believe that our results contribute to the discussion on obligatoriness of CC in BCS and the constraints on it. The summarised results of our experimental study on CC in Croatian, in which native speakers rated the acceptability of CC and noCC sentences with raising, subject control and object control matrix predicates are presented comprehensively in Table \ref{T16.12}. In Chapter \ref{On the heterogeneous nature of constraints on clitic climbing: complexity effects} we compare the results of this study with the results of our corpus studies presented in Chapters \ref{A corpus-based study on CC in da constructions and the raising-control distinction (Serbian)} and \ref{A corpus-based study on clitic climbing in infinitive complements in relation to the raising-control dichotomy and diaphasic variation (Croatian)}.

\begin{sidewaystable}
\caption{Summary of the observed effects: stars denote significant effect of the given factor.\label{T16.12}}
\begin{tabularx}{\textwidth}{lQcccQ}
\lsptoprule
Num. CLs      & Predicate type & CC & \multicolumn{2}{c}{infinitive CL}                & Description\\\cmidrule(lr){4-5}
              &                &    & Type                                      & Case & \\\midrule
one & raising & $*$ &&& CC preferred \\
 & simple subject control & $*$ &&& CC preferred \\
\tablevspace
two & reflexive subject control & $*$ & $*$ && noCC preferred in general; CC somewhat more preferred for sentences with pronominal infinitive CLs than for sentences with reflexive infinitive CLs\\
& object control with pronominal controller in the dative & $*$ & $*$ & $*$ & noCC preferred; contrast marginally higher if infinitive CL is \textsc{refl\textsubscript{\textsc{lex}}} CL \textit{se}; pronominal CL and \textsc{refl\textsubscript{2nd}} CL more likely to climb if in accusative \\
 & object control with pronominal controller in accusative & $*$ & $*$ & $*$ & noCC preferred; contrast marginally higher if infinitive CL is \textsc{refl\textsubscript{\textsc{lex}}} CL \textit{se}; \textsc{refl\textsubscript{2nd}} infinitive CL more likely to climb if in accusative \\
 & object control with \textsc{refl\textsubscript{2nd}} CL \textit{si} controller & $*$ & $*$ && noCC preferred; contrast marginally higher if infinitive CL is \textsc{refl\textsubscript{\textsc{lex}}} CL \textit{se} \\
 &object control predicate with \textsc{refl\textsubscript{2nd}} CL \textit{se} controller & $*$ & $*$ && noCC preferred; CC somewhat more preferred for sentences with pronominal infinitive CL than for sentences with reflexive infinitive CL\\
 \lspbottomrule
\end{tabularx}
\end{sidewaystable}

We have solid reasons to believe that our data are reliable. As we already pointed out in Section \ref{Stimuli}, besides 48 target sentences we had 48 target-like syntactically and morphologically ill-formed sentences, which served us as control items. Through them we could establish whether our participants’ responses had the necessary quality. The majority of our 336 participants rejected these straightforwardly ill-formed control items. As we already indicated in Section \ref{Data preprocessing}, participants who accepted these clearly ill-formed sentences (18 in total) were excluded from further analysis.

As can be seen in Figure \ref{F16.1}, results are internally consistent, i.e. parallel structures obtained similar scores in each mode. Moreover, the data collected in the second phase of the field research, which was conducted in March 2018 in Osijek, were not significantly different from the data collected in Zagreb and Split in December 2017 in the first phase of the field research. 

In the following we will briefly discuss how our results contribute to the discussion on obligatoriness of CC in BCS. It has become clear that CC is not a unified phenomenon. To start with, \citet[][]{Aljovic05} for instance claims that in BCS CC is obligatory with restructuring predicates. In contrast, we have shown that CC is not obligatory in the context of raising and subject control predicates since the acceptability rate for noCC versions of sentences with the mentioned matrix predicates was around 50\%.\footnote{Although we are aware that restructuring and raising predicates are not identical, we would like to point out that \citet[][198--204]{Stjepanovic04} observes that restructuring verbs behave like raising verbs.} However, we must point out that our participants clearly did favour CC over noCC versions of sentences with the mentioned matrix predicate types – see the summary of results in Table \ref{T16.12}. Moreover, CC versions of sentences with raising and subject control predicates were processed faster than their noCC counterparts. 

The results for object control matrix predicates are also partially in accordance with claims made for CC in Czech. As we already pointed out in Chapter \ref{Constraints on clitic climbing in Czech compared to Bosnian, Croatian and Serbian (theory and observations)}, the object control constraint has been controversially discussed in the literature. With some exceptions, which we will briefly comment on, our findings are generally in line with what \citet[][]{Thorpe91} and \citet[][]{Junghanns02} claimed: CLs cannot climb out of object-controlled infinitives. As our empirical study shows, this claim made for Czech also holds for Croatian. In contrast to \citet[][]{Thorpe91} and \citet[][]{Junghanns02}, \citet[][]{Dotlacil04} and \citet[][]{Rezac05} claim that in Czech CC out of object-controlled infinitives is possible in some special cases. For instance, in the theoretical syntactic literature it is claimed that in Czech, the climbing of accusative CLs is not blocked if the controller of the object control matrix predicate is in the dative \citep[cf.][]{Rezac05}. In our CC experiment the ninth research question addressed this issue in Croatian. Our results revealed that the effect of the case of the infinitive CL was indeed significant for CC out of object-controlled infinitives. However, as can be seen in Figure \ref{F16.9}, even if the controller is in the dative and the infinitive CL is in the accusative, CC is far from absolutely acceptable. Even in that configuration, the acceptability of CC sentences does not reach 50\%. Although we did not conduct a systematic corpus experiment which would cover CC in the context of the mentioned matrix predicate types, as we pointed out in Chapter \ref{Constraints on clitic climbing in Czech compared to Bosnian, Croatian and Serbian (theory and observations)}, some examples for the climbing of accusative CLs out of object-controlled infinitives with a dative controller can be found in hrWaC v2.2. Conversely, we did not come across examples of accusative CLs climbing out of object-controlled infinitives with an accusative controller. As can be seen in Figure \ref{F16.9}, acceptability of the latter CC sentences was even lower than acceptability of CC sentences in which the accusative CL climbed out of an object controlled infinitive with a dative controller. 

Furthermore, we can submit empirical evidence for an object control reflexive constraint in Croatian, which was reported for Czech by \citet[][]{Hana07}. As shown in Table \ref{T16.12} above, for object control matrix predicates the contrast between the acceptability of CC and noCC versions of sentences was higher if the infinitive CL was \textsc{refl\textsubscript{\textsc{lex}}}. 

The novelty of our study lies also in the range of matrix predicates included. Namely, as far as we know, the authors who studied CC in Czech included neither reflexive subject control matrix nor object control matrix predicates with the \textsc{refl\textsubscript{2nd}} \textit{se} and \textsc{refl\textsubscript{2nd}} \textit{si} controllers in their discussion of CC. Our study revealed interesting insights into how the mentioned three matrix predicate types influence CC. 

First, we can say that in respect of CC, object control matrix predicates with a \textsc{refl\textsubscript{\textsc{lex}}} CL \textit{si} controller do not behave differently than object control matrix predicates with dative and accusative pronominal CL controllers. In other words, all of them block CC out of the infinitive. Second, our empirical study showed that reflexive subject control matrix predicates behave the same in respect of CC as object control matrix predicates with the \textsc{refl\textsubscript{2nd}} CL \textit{se} controller. In the case of both matrix predicates, the climbing of reflexive CLs, both the \textsc{refl\textsubscript{2nd}} CLs \textit{se} and \textit{si} and the \textsc{refl\textsubscript{\textsc{lex}}} CL \textit{se}, is blocked. These results are in line with the observation of \citet[][]{Junghanns02} on the pseudo-twins constraint reported for Czech. While climbing of reflexive CLs is blocked, the mentioned matrix predicates with \textsc{refl\textsubscript{\textsc{lex}}} CL \textit{se} and \textsc{refl\textsubscript{2nd}} CL \textit{se} allow pronominal CLs to climb out of infinitives. However, we must point out that noCC sentences were preferred by our participants. At first glance, the Panel B2 in Figure \ref{Fig.15.2B} and C4 in Figure \ref{F16.1} might suggest that there are differences in the acceptance rates of CC sentences between those two predicate types since the acceptance rate for the climbing of pronominal CLs is somewhat higher in the case of reflexive subject control predicates than in the case of object control predicates with the \textsc{refl\textsubscript{2nd}} \textit{se} controller. However, those differences were not statistically significant. 
