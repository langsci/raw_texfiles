\part{Final remarks}
\label{part4}
\chapter{Overall summary}
\label{Summary and outlook}



\section{Scope of the book}

At the end of this data-oriented, empirical in-depth study of the BCS CLs we would like to repeat that CLs remain an extremely interesting subject for both data- and theory-driven syntactic research. This is because ``the study of clitics can shed light on the interfaces between syntactic, morphological, and phonological linguistic representations'' \citep[12]{FJL04}. Yet, as we have pointed out, nearly all theoretical models assume a more or less stable and homogeneous CL system in BCS. However, it is no exaggeration to say that in many works data quality leaves much to be desired, which leads to contradictory statements about the acceptability of certain structures. Through our focus on empirical study of microvariation within the CL system in BCS we show that CLs are also an ideal test case for variationist approaches and the theory of linguistic complexity. 



The book is divided into three main parts and a part with final remarks. Part \ref{part1} functions as an introduction containing definitions of our terms and concepts in the context of the briefly presented main assumptions of a selected number of theoretical approaches to CLs in BCS (Chapter \ref{Our terms and concepts}). However, we tried to avoid premature theoretical commitments and as far as possible we used definitions as mere descriptive labels. As explained in Chapter \ref{Our terms and concepts}, we identified three main parameters of variation:
\begin{enumerate}
	 \item inventory, 
	\item internal organisation of the CL cluster (CL ordering and morphonological processes within the cluster),
	\item position of the CL or CL cluster (i.e. second position, delayed placement and phrase splitting). 
\end{enumerate}

Moreover, Part \ref{part1} contains the main tenets of our methodology (Chapters~\ref{Empirical approach to clitics in BCS} and~\ref{Corpora for Bosnian, Croatian and Serbian}).


Part \ref{part2} targets systemic microvariation and selected cases of microvariation in the diatopic and the diaphasic dimensions. In it, we gave an empirical account of variation in all three standard varieties: Bosnian, Croatian, and Serbian (Chapter \ref{Clitics and variation in grammaticography and related work}), and in Štokavian dialects spoken on the territory of the former Yugoslavia and in some neighbouring countries (Chapter \ref{Clitics in dialects}). Furthermore, we investigated the CL system in a spoken variety (Chapter \ref{Clitics in a corpus of a spoken variety}). Due to the lack of comparable resources for all three varieties we restricted ourselves only to spoken Bosnian. We detected some global patterns of microvariation in the three above-mentioned parameters. 



Our in-depth analysis of the existing literature on Czech showed CC to be a major source of variation which causes considerable disagreement among scholars (Chapter \ref{Constraints on clitic climbing in Czech compared to Bosnian, Croatian and Serbian (theory and observations)}). As we are convinced that it is a highly important feature affected by systemic microvariation, we dedicated all of Part \ref{part3} to CC (Chapters \ref{Approaches to clitic climbing}--\ref{On the heterogeneous nature of constraints on clitic climbing: complexity effects}). 

In Chapter \ref{Introduction:11} we also offered our definition of CC as a phenomenon whereby a CL is not realised in a position contiguous to the elements of the embedding to which it belongs, but in a position contiguous to elements of the matrix. Finally, we tried to explain the constraints on CC in terms of complexity (Chapter \ref{On the heterogeneous nature of constraints on clitic climbing: complexity effects}).



\section{Empirical approach}



As to the empirical approach chosen, the current work is language-use oriented and is based on the triangulation of methods: intuition/theory – observation – experiment. The first step always involved a thorough analysis of the whole body of existing research literature, independently of the respective theoretical framework. We put theory-driven studies on an equal footing with normative and descriptive work. We thus strove to overcome the deplorable lack of exchange of thought between formal syntacticians and descriptive linguists, who tend to ignore each other. For the features showing some degree of microvariation we used empirical data collected in the years 2015--2018 from large web corpora \{bs,hr,sr\}WaC – our first source of observations. A selection of hypotheses concerning factors determining variation in the usage of CLs, formulated on the basis of corpus material, were further tested in acceptability judgment experiment where the level of control could be adjusted for individual factors. The whole book was actually inspired by \citet[60]{DFZ09} who emphasised the need for empirical data in research on CLs. In their own words, the ``[c]urrent research has [...] relied heavily on native speaker judgments that have been culled primarily from previously published work, or from interrogating native speaker linguists. While these are not uncommon methods in theoretical linguistics, it is well worth augmenting the database with other sources'' \citep[60]{DFZ09}.

Here we would like to underline that we do not entirely reject introspection as a linguistic method, but we argue in favour of a thorough documentation. The reader should bear in mind that the first step of our research (intuition/theory) would not be possible without an informal judgment taks. However, we do believe that this as the only method of obtaining data might not be robust enough to permit generalisation, especially in a study dealing with variation, since the judgments of one or a few people cannot truly account for it. Since we are aware of the risks related to data gathering via intuition, we chose to design our studies in a way that avoids the traps mentioned in the methodological literature.
We are well aware that ``[s]cientific research is collaborative and incremental in nature, with researchers building on and extending each other’s work'' \citep[133f]{Stefanowitsch20}. Therefore we have tried hard to be as transparent as possible concerning our data and methods. Namely, we have done our best to describe our research designs, research materials, and procedures associated with them explicitly and in sufficient detail. This should allow other researchers to retrace and check the correctness of each step of our analyses \citep[cf.][133f]{Stefanowitsch20}. Such a practice, with some exceptions like \citet{DFZ09}, \citet{ZFD17}, and \citet{DiesingZec17}, is rarely applied in the study of BCS CLs.

\section{The role of prescriptivism in the BCS clitic systems}
\label{The role of prescriptivism in the BCS CL systems}
Presumably like many scholars working on BCS CLs, we are also profoundly interested in the role played by prescriptivism in CL placement. The only way we were able to address this, at least indirectly, besides examining CL placement in BCS standard varieties (Chapter \ref{Clitics and variation in grammaticography and related work}), was to observe CL placement in non-standard varieties, i.e. dialects (Chapter \ref{Clitics in dialects}), spoken Bosnian (Chapter \ref{Clitics in a corpus of a spoken variety}), and colloquial Croatian (Chapter \ref{A corpus-based study on clitic climbing in infinitive complements in relation to the raising-control dichotomy and diaphasic variation (Croatian)}). However, we are aware that it is not the same to compare CL placement in standard varieties (Chapter \ref{Clitics and variation in grammaticography and related work}), whose descriptions are mainly based on written language, with CL placement in non-standard spoken varieties (Chapters \ref{Clitics in dialects} and \ref{Clitics in a corpus of a spoken variety}).

In our search for variation, we tried our best not to depart from the standard assumption that prescriptive linguistic norms are much more rigorous in Croatia than in Serbia. While often cited in scholarly literature, we believe that this assumption is something of an unsubstantiated misconception.\footnote{We would like to point out that no grammar book of Croatian language contains the word ``normative'' or ``standard'' in its title. On the other hand, a closer look at \textit{Normativna gramatika srpskog jezika} \citep{PiperKlajn14} will reveal to the careful reader that not only Croatian but also Serbian has strict normative rules} The fact that the rules of standard Serbian differ from the Croatian norm does not make them any less normative. In Chapter \ref{Clitics and variation in grammaticography and related work} we show how even in the period of the so-called Serbo-Croatian language Serbian and Croatian linguists recommended different CL placement. The Serbian linguist \citet[308]{Pesikan58} claimed that it is better to place CLs after a two-word phrase than to use DP of CLs. Moreover, the same linguist openly argued that the Croatian tendency to insert CLs after the first stressed word and to split phrases is, in his term, an ``exaggeration'' \citep[cf.][309]{Pesikan58}.

The fact that standard Serbian chooses a different variant from the possible variants than Croatian does not make it less normative. According to dialectological literature (Chapter \ref{Clitics in dialects}), Serbian speakers do split different kinds of phrases, some of which did not find their way into the Serbian norm, such as first name and surname, and coordinative phrases. Similarly, in the same chapter, we show that the reflexive \textit{si} and the pronominal CL \textit{ju} do occur in some Serbian dialects and that the latter is not always necessarily restricted to the context of the morphonological process of suppletion. These features actually vanished from standard Serbian due to prescriptivism. In contrast, in standard Croatian they are recognized as legitimate variants available in the language system.

We can, of course, speculate that depending on teachers and their beliefs and attitudes to the norm, speakers of Croatian, just like speakers of Serbian, might have been instructed in school to place CLs according to the rules described in their standard grammar books. Nevertheless, normative instructions are one thing, and the internalisation of such rules is another. While obeying the rules is more likely in written language, it is very hard to control for in spoken language: ``[l]evel of formality (style) may be easier to manipulate in performing for the linguist than pronunciation, which is easier to manipulate than morphological or syntactic behavior'' \citep[26]{Stefanowitsch20}. As our descriptive and empirical studies on standard and non-standard varieties in Chapters \ref{Clitics and variation in grammaticography and related work}, \ref{Clitics in dialects}, \ref{Clitics in a corpus of a spoken variety}, and \ref{A corpus-based study on clitic climbing in infinitive complements in relation to the raising-control dichotomy and diaphasic variation (Croatian)} indicate, normative instructions and their internalisation do not always coincide.

\begin{sloppypar}To control for and to avoid potential prescriptive attitudes towards CL placement as a confounding variable in our psycholinguistic experimental study (Chapter \ref{Experimental study on constraints on clitic climbing out of infinitive complements}), we deliberately excluded linguists and students of language studies as participants. It is well known that precisely these groups of participants can demonstrate rather prescriptive attitudes and may rely heavily on the notion of a narrowly defined standard language usage \citep[cf.][]{KrugSell13}. Indeed, if we had not made this decision, it would have been much simpler to find participants for our experiment. However, as in many other respects, we deliberately chose to avoid the easier way in favour of one which would guarantee us better data and consequently give us better insights into the CL system in BCS.\end{sloppypar}



\section{Results: Parameters of variation}

\subsection{Inventory}



We now briefly present our main results. First, it is interesting to note that all varieties of BCS seem to have a CL inventory which comprises four structural types: the polar question marker \textit{li}, pronominal, reflexive and verbal CLs. No other types of CLs have been found. We detected only minor diatopic variation, mainly involving the reflexive CL \textit{si} and the third person singular feminine accusative CL \textit{ju}. 

In the standard varieties only Croatian grammarians accept the reflexive CL \textit{si}. The analysis of the dialectological literature, however, clearly shows that this form is found not only on Croatian, but also on Bosnian and Serbian language territory. 

Further, Croatian and Serbian authors differ in their recommendations for the usage of the third person singular feminine accusative CL \textit{ju}. Some Croatian authors treat \textit{ju} as a separate unit of the inventory and not as a case of morphological suppletion (repeated morph constraint). Moreover, in the corpus of spoken Bosnian the CL \textit{ju} is not attested at all. However, dialects give a very varied picture in this respect. Many idioms belonging to Old and Neo-Štokavian dialects have both the forms \textit{ju} and \textit{je} for the feminine singular accusative. Nevertheless, we also encounter dialects which use \textit{je }exclusively or \textit{ju} exclusively. 

Finally, some dialects exhibit forms of pronominal CLs which have not found their way into any of the three standard norms: of them the \textit{Prizrensko-južno\-mo\-rav\-ski} dialect spoken in Southern Serbia and Kosovo shows the greatest number of forms. 

With respect to variation in the usage of verbal CLs we would like to emphasise that the conditional auxiliary form \textit{bi} used for all persons (without inflection) is a case of allomorphy and not a true difference in the inventory.  





\subsection{Internal organisation of the clitic cluster}



We find more variation in the parameter of internal organisation of the CL cluster. As to CL ordering in the cluster, the potential co-occurrence of the reflexive CL \textit{se} and the verbal CL \textit{je} is a clear case of microvariation. Whereas in standard Bosnian and Croatian both haplology and the cluster \textit{se je} are allowed, the Serbian normative grammar book \citet{PiperKlajn14} accepts only haplology. Further, we find ample evidence for the reversed CL order \textit{je se}, which is attested in the central BCS territory, e.g. in the \textit{Šumadijsko-vojvođanski}, \textit{Zapadni}, \textit{Slavonski}, \textit{Srednjobosanski} and \textit{Istočnohercogevački} dialects. In the data from the corpus of spoken Bosnian the \textit{je se} cluster sequence is four times more frequent than \textit{se je} prescribed in standard Bosnian and Croatian. 

The verbal CL \textit{je} is affected by similar type of variation: according to our data, in some Serbian dialects it can precede pronominal CLs. Hence, it looka as if in these idioms the CL cluster was simpler than in the standard languages as it contains a single slot for all verbal CLs. To confirm this observation a separate study, involving bigger and better quality data, would be necessary.

Furthermore, in a very small number of Štokavian dialects, contrary to the CL ordering in the standard BCS variaties, pronominal CLs can stand in front of verbal CLs (the present tense and conditional forms of \textit{biti}). 

\subsection{Morphonological processes within the cluster}

Another major source of variation is related to morphonological processes, once again involving the CLs \textit{se} and \textit{je}. As we show in Chapter \ref{Clitics and variation in grammaticography and related work}, haplology of unlikes is a normatively regulated feature, and even within standard varieties it is treated differently by different authors. The uneven distribution is clearly visible in our empirical data from the corpus of spoken Bosnian where haplology of unlikes is far more common than the co-occurrence of these two CLs (Chapter \ref{Clitics in a corpus of a spoken variety}). Namely, haplology (with only the CL \textit{se} occurring) is found in 68.8\% of cases; a CL cluster (the CLs co-occur; the sequence \textit{je se} is more frequent than \textit{se je}), in 25.4\% of cases; and in 5.6\% of the cases the reflexive CL \textit{se} and the verbal CL \textit{je} appear in (pseudo)diaclisis. Since non-omission is well attested in non-standard varieties (Chapters \ref{Clitics in dialects} and \ref{Clitics in a corpus of a spoken variety}, we come to the conclusion that the processes in place to avoid the repetition of morphemes are rather unstable. For BCS we can conclude that the repeated morph constraint is a question of preference. Finally, there are some clues that the constraint on the string \textit{se je} is not exclusively morphonological in nature. As we note in Chapter \ref{Clitics and variation in grammaticography and related work}, \citet[564]{Ridjanovic12} argues that the deletion does not affect the verbal CL \textit{je}, which is a copula, and the genitive CL \textit{je} as an argument in the string \textit{je se} is not affected either. Hence, in standard varieties only the auxiliary \textit{je}  seems susceptible to this process.



\subsection{Position of clitics and clitic clusters}

Summing up our discussion of the parameter position of clitics and clitic clusters, we present interesting findings on three levels. First, absolute 1P is ruled out in all of the three standard varieties. Accordingly, absolute 1P is not found in the corpus of spoken Bosnian either. We have only come across sentences in which CLs follow insertions, DSEs and retrospectives. However, the dialectal survey shows that absolute 1P is attested in idioms which have been in intensive language contact with Romanian (\textit{Šumadijsko-vojvođanski}) or with Macedonian (\textit{Kosovsko-resavski, Prizrensko-južnomoravski} and \textit{Timočko-lužnički}). These findings strongly suggest that absolute 1P is more probable in Štokavian contact varieties. 

One further observation concerns CLs after the conjunctions \textit{a} and \textit{i}. Unlike standard BCS varieties, some dialects spoken in Croatia and Serbia allow the positioning of CLs directly after these coordinative conjunctions. Second, in standard Croatian and standard Bosnian the second position rule is understood as 2W, whereas in the literature on standard Serbian it is emphasised that 2P is normally understood as the position posterior to the first phrase (which may or may not be compound). Croatian and Bosnian standards allow the insertion of CLs into far more types of phrases than the Serbian standard does. However, both dialectal data and the data from the corpus of spoken Bosnian show ample evidence for splitting of conjoined NPs and quantificational phrases, which is not only widespread in Bosnian and Croatian territory, but can also be found in Serbian language territory. Contrary to some claims from the theoretical literature, in the spoken Bosnian variety not only subject phrases but also prepositional phrases can be split. Furthermore, we show that in some dialects and in the spoken Bosnian variety more than one CL can be inserted into a phrase. These data contradict \citet{Progovac96} and \citet{RadanovicKocic88, RadanovicKocic96}, who claim that clusters are not used as splitting elements. 

Third, we are the first to empirically measure the heaviness of the initial constituent in spoken Bosnian. This feature is claimed to be a factor responsible for DP. Although many authors do mention heaviness, most of them do not provide any information on how to distinguish initial constituents which are heavy and cause DP from those which are not heavy and allow 2P. Inspired by the Czech linguists \citet{KCN18} we conducted an empirical study on spoken Bosnian. In the results we see a strong tendency towards 2W. The typical CL position in the clause is after the first word (77\% of all observations), which is most frequently two graphemes long. The most frequent initial constituent in DP is three graphemes long, but in general its length is not limited, while the most frequent host in DP is four graphemes long; thus, it is longer than the initial constituent. This seems to suggest that DP depends not only on the heaviness of the first constituent, but might also be related to its phonological properties. 


Although this was not in the focus of our study, in the spoken Bosnian data we find evidence that the polar question marker \textit{li }differs from all the other CL types as in 100\% of the cases it is placed in 2P after one short word (only 2 to 4 graphemes long). Its positioning is thus far more uniform than the positioning of verbal, pronominal and reflexive CLs. 



\subsection{Constraints on clitic climbing}

Part \ref{part3} is dedicated to CC and its constraints in BCS, a hitherto underresearched topic. CC deserves a separate part because the data from the grammar handbooks, dialectological sources and available corpora of spoken language are too scarce to allow for any sound conclusions. Since CC in Czech has received much more attention than in BCS, as a starting point we can draw on the body of research on CC in Czech. Because CC shows considerable similarity in both languages, we can start from Junghanns' (\citeyear{Junghanns02}) findings. Our use of findings on CC in Czech as a point of departure does not mean that we assume that the constraints on CC in Czech can be automatically postulated and carried over to BCS. Instead, we transparently work bottom-up: we generate all possible hypotheses to check whether something from Czech might hold up for BCS. In other words, observations made for Czech serve as a starting point for hypothesis generation and not as assumptions that the very same mechanisms are present in BCS. The amount of empirical data in this book effectively proves that we have not tried to carry assumptions over from one language to another.

For the study of constraints on CC we combine all three types of methods, starting with intuition/theory based on the existing literature (mainly on Czech), through observation based on empirical corpus studies, and ending with a large psycholinguistic experiment involving acceptability judgments. In the corpus study we look into the highly controversial question whether CLs can climb out of \textit{da}$_2$-complements. Whereas some authors treat such sentences as completely normal, others reject them outright. We show that CC is indeed marginally possible, but exclusively with raising and simple subject control predicates. This study and a further study conducted by \citet*{HKJ18} on stacked infinitives, i.e. matrix predicates with multiple embeddings, serve as a basis for the test set-up of the psycholinguistic experiment. The test, which comprised 296 sentences, was carried out on 336 participants from various university institutions in Croatia. In the experiment we were able to test the impact of the following factors:

\largerpage[-1]

\begin{enumerate}
	\item type of matrix verb (including the reflexivity of the predicate),
	\item number of CLs in a sentence (one, two),
	\item CL type, CL case,
	\item position of infinitive CL (CC vs noCC).
\end{enumerate}  

We analysed the data, applying mixed-effects regression with participants and sentence endings as random variables, a statistical method which has become the golden standard in psycholinguistic research.

In addition to the systemic and non-systemic diatopic factors triggering microvariation, at least for Croatian we detected a higher frequency of CC in the standard than in the informal variety presented in web fora. 

The findings presented in Chapter \ref{On the heterogeneous nature of constraints on clitic climbing: complexity effects} show that the constraints belong to various levels of syntax.  First, we fully agree with  \citet[205]{Rosen01} that ``several factors interact to determine their [CLs] position'' and ``only some generalizations concerning their ordering behaviour can be expressed by strict rules, while other properties have to be stated as mere preferences''. Most factors interact with some other factor. The factors CL type and CL case interact with the factor matrix predicate type and are not active in themselves. As to preferences, we saw certain patterns of CC which show graded acceptability. 

Therefore, in our view, a single syntactic mechanism like restructuring cannot account for all of constraints to CC. Instead, we offer a solution alternative to the already existing theoretical approaches on CC in BCS, which enables us to account the whole broad spectrum of variation in empirical data we identify. We argue that the heterogeneous nature of CC can be best accounted for by complexity effects, which in the domain of syntax belong primarily to the constitutional and organisational subtypes of ontological complexity. Building on the approach by \citet{Rescher98} who gives a consistent typology of modes of complexity, we construct a series of hierarchies for the probability of CC based on the interaction between three factors: island {\textgreater} CTP type {\textgreater} mixed cluster effect. We propose a division of syntactic islands, a sort of locality constraint, into two subtypes: true islands and tied islands. The latter marginally allow CC (\textit{da}$_2$-complements). 

Next to the structural constraints, we identify the diaphasic factor regulating CC in BCS. Diaphasic variation represents operational subtype of ontological complexity.


It goes without saying that we have not covered all the problems involving microvariation in the CL system of BCS. Especially in the vast field of CC, we leave to future research the similarity constraint and the haplology of pseudo-twins, for example. What we do plan to study in more detail, however, is haplology of unlikes and the constraints on CC related to the infinitive complement, in particular, stacked infinitives.  
