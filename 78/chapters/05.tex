\chapter[Word classes]{Word classes}\label{Para_5}
\section{Introduction}\label{Para_5.1}

This chapter discusses the Papuan Malay word classes, or parts of speech. Some of the word classes are examined in more detail in separate chapters.



The notion of “word class” is defined as a class of “words that share morphological or syntactic properties” {\citep[5188]{Asher.1994}}. In general, pertinent criteria for establishing class membership are a “word’s distribution, its range of syntactic functions, and the morphological or syntactic categories for which it is specifiable” \citep[1–2]{Schachter.2007}. In Papuan Malay, morphological criteria do not play a major role in distinguishing different word classes, given the lack of inflectional \isi{morphology} and the rather limited productivity of derivational patterns (see §\ref{Para_3.1}). Instead the main criteria for defining distinct word classes are their syntactic properties.



Based on their syntactic properties, three open and several closed lexical classes are distinguished. It is acknowledged, however, that Papuan Malay has membership overlap between a number of categories (see §\ref{Para_5.14}). Most of this \isi{variation} involves verbs, including overlap between verbs and nouns, which is typical of Malay varieties and other western \ili{Austronesian} languages. In discussing lexical and syntactic categories in western \ili{Austronesian} languages,\footnote{More specifically, \citet[112]{Himmelmann.2005} refers to western \ili{Austronesian} “symmetrical voice languages”, that is languages that have “at least two voice alternations marked on the \isi{verb}, neither of which is clearly the basic form”.}
 \citet[127]{Himmelmann.2005} points out, that “the syntactic distinction between nouns and verbs is often somewhat less clearly delineated in that word-forms which semantically appear to be verbs easily and without further morphological \isi{modification} occur in nominal functions and vice versa”. This applies especially to languages with “multifunctional lexical bases'', that is, “lexical bases which occur without further \isi{affixation} in a variety of syntactic functions” \citep[129]{Himmelmann.2005}.



As for the analytical consequences of such overlap, \citet[128]{Himmelmann.2005} notes that most authors “assume underlying syntactic differences based on the semantics of the forms”, analyzing such instances of \isi{variation} “as involving zero conversion”. As far as the description of \ili{regional Malay varieties} is concerned, this approach is accepted, for example, by \citet{vanMinde.1997} in his grammar of \ili{Ambon Malay}, \citet{Stoel.2005} in his description of \ili{Manado Malay}, and \citet[250]{Paauw.2009} in his discussion of \ili{regional Malay varieties} such as \ili{Banda Malay}, \ili{Kupang Malay}, or \ili{Larantuka Malay}. Some authors, however, “argue for a basic lack of a morphosyntactic \isi{noun}/\isi{verb} distinction”, as \citet[128]{Himmelmann.2005} points out. Examples for this alternative approach are Gil’s (\citeyear*{Gil.2013}) description of \ili{Riau Indonesian} (see also \citealt{Gil.1994}), Himmelmann’s (\citeyear*{Himmelmann.2008}) analysis of \ili{Tagalog} (see also \citealt{Himmelmann.1991}), and Litamahuputty’s (\citeyear*{Litamahuputty.2012}) grammar of \ili{Ternate Malay}.



In discussing Papuan Malay lexical and syntactic categories in this grammar, nouns and verbs are analyzed as belonging to distinct word classes, in spite of the attested \isi{variation} in membership, discussed in §\ref{Para_5.14}. This approach is chosen because of the distinct syntactic properties of the categories under discussion, as shown in more detail throughout this chapter. In cases of \isi{variation}, the category membership of a given lexeme can usually be deduced from the context in which an utterance occurs. Rather than proposing additional special word classes for lexical items with dual distribution, the lexemes in question are analyzed as having dual class membership and the \isi{variation} as involving zero conversion.



In the next two sections, the two major open lexical classes of nouns and verbs are discussed. The class of nouns, described in §\ref{Para_5.2}, includes common nouns, proper nouns, location, and direction nouns. Verbs, discussed in §\ref{Para_5.3}, are divided into \isi{trivalent}, \isi{bivalent}, and \isi{monovalent} verbs, with the class of \isi{monovalent} verbs including dynamic and stative verbs. Adverbs, discussed in §\ref{Para_5.4}, constitute the third open word class. The closed word classes are then described, that is, personal pronouns in §\ref{Para_5.5}, demonstratives in §\ref{Para_5.6}, locatives in §\ref{Para_5.7}, interrogatives in §\ref{Para_5.8}, numerals in §\ref{Para_5.9}, quantifiers in §\ref{Para_5.10}, prepositions in §\ref{Para_5.11}, and conjunctions in §\ref{Para_5.12}. Tags, \isi{placeholder} and hesitation makers, interjections, and ideophones are presented in §\ref{Para_5.13}. The categories with \isi{variation} in \isi{word class membership} are discussed (§\ref{Para_5.14}).The main points of this chapter are summarized in (§\ref{Para_5.15}).


\section{Nouns}\label{Para_5.2}
\largerpage
Papuan Malay has a large open class of nouns which refer to persons, things, and places, as well as abstract concepts and ideas. Typically, nouns have head-function in \isi{noun} phrases and argument function in verbal clauses.

\largerpage
Based on their syntactic properties, the nouns are divided into common nouns (§\ref{Para_5.2.1}), proper nouns (§\ref{Para_5.2.2}), location nouns (§\ref{Para_5.2.3}), and direction nouns (§\ref{Para_5.2.4}). Their defining syntactic and functional properties are discussed in more detail in the respective sections.



Morphological properties do not play a major role in defining nouns as a distinct word class. This is due to the lack of inflectional \isi{morphology} and the limited role of derivational processes. The latter include \isi{reduplication}, and, to a limited extent, \isi{affixation} with suffix -\textitbf{ang} or prefix \textscItal{pe(n)-} (for details see §\ref{Para_3.1.3} and §\ref{Para_3.1.4}, respectively).



Nouns are distinct from other word classes such as verbs (§\ref{Para_5.3}), adverbs (§\ref{Para_5.4}), personal pronouns (§\ref{Para_5.5}), and demonstratives (§\ref{Para_5.6}) in terms of the following distributional properties. Some of these properties, however, do not apply to all four \isi{noun} types. The exceptions are mentioned below and discussed in more detail in the respective sections on the different \isi{noun} types.


\begin{enumerate}
\item 
Nouns are distinct from verbs (a) in terms of their predominant functions as heads in \isi{noun} phrases and as arguments in verbal clauses, (b) in that they can be quantified with numerals and quantifiers (this only applies to common and proper nouns), and (c) in that they are only negated with \textitbf{bukang} ‘\textsc{neg}’.
\item 
Unlike adverbs, nouns (a) have predicative uses, and (b) can modify other nouns.
\item 
Nouns are distinct from personal pronouns, in that nouns (a) can be modified with personal pronouns, while personal pronouns are not modified with nouns, (b) can be modified with numerals/quantifiers in pre- or posthead position, while personal pronouns are only modified with numerals or quantifiers in posthead position, and (c) can express the possessum in adnominal possessive constructions, while personal pronouns do not take this slot. (Most of these properties only apply to common and (to a lesser extent to) proper nouns but not to location and direction nouns.)
\item 
Nouns can be modified with demonstratives, whereas demonstratives cannot be modified with nouns.

\end{enumerate}

The following sections describe the four \isi{noun} types in more detail: common nouns are discussed in §\ref{Para_5.2.1}, proper nouns in §\ref{Para_5.2.2}, location nouns in §\ref{Para_5.2.3}, and direction nouns in §\ref{Para_5.2.4}. Also included are brief descriptions of time-denoting nouns in §\ref{Para_5.2.5}, classifying nouns in §\ref{Para_5.2.6}, and kinship terms in §\ref{Para_5.2.7}.


\subsection{Common nouns}
\label{Para_5.2.1}
Common nouns have general reference, in that they “do not refer to individual entities (‘tokens’) but only connote classes (‘types') of entities” \citep[58]{Givon.2001}. They have the following defining syntactic and functional properties:



\begin{enumerate}
\item 
Head function in \isi{noun} phrases is predominant (\chapref{Para_8}); in addition, they also have \isi{predicative function} in nonverbal clauses (\chapref{Para_12}).

\item 
Argument function (subject or object) in verbal clauses is predominant (\chapref{Para_11}).\footnote{As \citet[59]{Givon.2001} points out, it is technically speaking “not the \isi{noun} but rather the \textit{\isi{noun} phrase }that assumes the various grammatical roles [{\ldots}] However, within the \isi{noun} phrase, a \isi{noun} is typically the syntactic and semantic \textit{head}, defining the type of entity involved. All other elements in the \isi{noun} phrase are \textit{modifiers }of that head \isi{noun}”.}

\item 
Quantification (with numerals and quantifiers) and \isi{modification} with adnominal constituents (including other nouns, verbs, personal pronouns, demonstratives, locatives, interrogatives, \isi{noun} phrases, prepositional phras\-es, and/or relative clauses) (\chapref{Para_8}).

\item 
Negation only with \textitbf{bukang} ‘\textsc{neg}’ (§\ref{Para_13.1.2}).

\item 
In adnominal possessive constructions, common nouns can express the possessor and/or the possessum (\chapref{Para_9}).

\end{enumerate}

Cross-linguistically, two types of common nouns can be distinguished, count nouns and mass nouns. While a count \isi{noun} designates “a separate, one of a number of such entities which can be counted”, a mass \isi{noun} “denotes a quantity or mass of unindividuated material” \citep[5108, 5144]{Asher.1994}. Examples of Papuan Malay count and mass nouns, both concrete and abstract, are presented in \tabref{Table_5.1}.


\begin{table}
\caption{Count and mass nouns}\label{Table_5.1}

\begin{tabular}{llll}
\lsptoprule

\multicolumn{2}{l}{ Concrete count nouns} & \multicolumn{2}{l}{ Abstract count nouns}\\
\midrule
\textitbf{ana} & ‘child’ & \textitbf{adat} & ‘tradition’\\
\textitbf{bawang} & ‘onion’ & \textitbf{berkat} & ‘blessing’\\
\textitbf{celana} & ‘trouser’ & \textitbf{dosa} & ‘sin’\\
\textitbf{daung} & ‘leaf’ & \textitbf{jatwal} & ‘schedule’\\
\textitbf{hutang} & ‘forest’ & \textitbf{kwasa} & ‘power’\\
\textitbf{jaring} & ‘net’ & \textitbf{pamali} & ‘taboo’\\
\textitbf{sumur} & ‘well’ & \textitbf{tanggal} & ‘date’\\
\textitbf{tikus} & ‘rat’ & \textitbf{tuju} & ‘goal’\\
\midrule
\multicolumn{2}{l}{ Concrete mass nouns} & \multicolumn{2}{l}{ Abstract mass nouns}\\
\midrule
\textitbf{ampas} & ‘waste’ & \textitbf{cinta} & ‘love’\\
\textitbf{busa} & ‘foam’ & \textitbf{baw} & ‘smell’\\
\textitbf{dara} & ‘blood’ & \textitbf{dana} & ‘funds’\\
\textitbf{garam} & ‘salt’ & \textitbf{duka} & ‘grief’\\
\textitbf{minyak} & ‘oil’ & \textitbf{hikmat} & ‘wisdom’\\
\textitbf{nasi} & ‘cooked rice’ & \textitbf{iman} & ‘faith’\\
\textitbf{susu} & ‘milk’ & \textitbf{ongkos} & ‘expenses’\\
\textitbf{te} & ‘tea’ & \textitbf{umur} & ‘age’\\
\lspbottomrule
\end{tabular}
\end{table}

Count nouns can be modified with numerals as in (\ref{Example_5.1}) and (\ref{Example_5.2}), or with quantifiers as in (\ref{Example_5.3}) to (\ref{Example_5.6}). The numerals and quantifiers can occur in prehead position, as in (\ref{Example_5.1}), (\ref{Example_5.3}), or (\ref{Example_5.5}), or in posthead position as in (\ref{Example_5.2}), (\ref{Example_5.4}), or (\ref{Example_5.6}). (Concerning the position of adnominal numerals vis-à-vis their head nominal and their semantics, see §\ref{Para_5.9} and §\ref{Para_8.3.1}.)


\begin{styleExampleTitle}
Count nouns\footnote{Documentation: \textitbf{dua} ‘two’ 080919-001-Cv.0022, BR111017-002.003, \textitbf{banyak} ‘many’ 081006-023-CvEx.0007, 081029-004-Cv.0021, \textitbf{sedikit} ‘few’ BR111021.014, BR111021.015.}
\end{styleExampleTitle}
 
\ea
\label{Example_5.1}
\gll \textitbf{dua} \textitbf{orang}\\ %
two person\\
\glt ‘two people’
\z

\ea
\label{Example_5.2}
\gll \textitbf{orang} \textitbf{dua}\\ %
person two  \\
\glt ‘both people’
\z 
\ea
\label{Example_5.3}
\gll \textitbf{banyak} \textitbf{orang}\\ %
many person  \\
\glt ‘many people’\\
\z 
\ea
\label{Example_5.4}
\gll \textitbf{orang} \textitbf{banyak}\\ %
person many  \\
\glt ‘many people’\\
\z 
\ea
\label{Example_5.5}
\gll \textitbf{sedikit} \textitbf{orang}\\ %
few person\\
\glt ‘few people’\\
\z 
\ea
\label{Example_5.6}
\gll \textitbf{orang} \textitbf{sedikit}\\ %
person few\\
\glt ‘few people’\\
\z 

Mass nouns can be modified with quantifiers, which always occur in posthead position, as in (\ref{Example_5.7}) and (\ref{Example_5.8}). That is, the quantifiers cannot occur in prehead position, as shown with the elicited ungrammatical constructions in (\ref{Example_5.9}) and (\ref{Example_5.10}). Also, mass nouns cannot co-occur with numerals, neither in pre- nor in posthead position, as shown with the elicited ungrammatical examples in (\ref{Example_5.11}) and (\ref{Example_5.12}). (As for the position of adnominal quantifiers vis-à-vis their head nominal and the semantics involved, see §\ref{Para_5.10} and §\ref{Para_8.3.2}.)


\begin{styleExampleTitle}
Mass nouns\footnote{Documentation: \textitbf{banyak} ‘many’ BR111021.015, BR111021.017, \textitbf{sedikit} ‘few’ BR111021.016, BR111021.018, \textitbf{dua} ‘two’ BR111021.019, BR111021.020.}
\end{styleExampleTitle}
 
\ea
\label{Example_5.7}
\gll \textitbf{sagu} \textitbf{banyak}\\
sago many\\
\glt ‘lots of sago’\\
\z 
\ea
\label{Example_5.8}
\gll \textitbf{sagu} \textitbf{sedikit}\\ %
sago few\\
\glt ‘little sago’\\
\z
 
\ea
\label{Example_5.9}
\gll \textitbf{*banyak} \textitbf{sagu}\\ %
many sago\\
\glt Intended reading: ‘lots of sago’\\
\z
 
\ea
\label{Example_5.10}
\gll \textitbf{*sedikit} \textitbf{sagu}\\ %
few sago\\
\glt Intended reading: ‘little sago’\\
\z
 
\ea
\label{Example_5.11}
\gll \textitbf{*dua} \textitbf{sagu}\\ %
two sago\\
\glt (‘two sago’)\\
\z
 
\ea
\label{Example_5.12}
\gll \textitbf{*sagu} \textitbf{dua}\\ %
sago two\\
\glt (‘two sago’)\\
\z
 

\subsection{Proper nouns}
\label{Para_5.2.2}
Proper nouns have specific reference in that they “refer to individual entities (or specific groups)” \citep[58]{Givon.2001}. Hence, proper nouns are distinct from common nouns, which have general reference. More specifically, proper nouns express the names of specific people and geographical places. In Papuan Malay proper nouns are distinct from common nouns in terms of the following properties:




\begin{enumerate}
\item 
Proper nouns can be modified with the following constituents: \isi{monovalent} stative verbs, personal pronouns, demonstratives, locatives, interrogatives, numerals, quantifiers, and/or relative clauses (\chapref{Para_8}). Unlike common nouns, they are not readily modified with other nouns, \isi{noun} phrases, or prepositional phrases.

\item 
Proper nouns always occur as bare nouns; reduplicated proper nouns are unattested (§\ref{Para_4.1.1.1}).

\item 
Proper nouns typically express the possessor but not the possessum in adnominal possessive constructions (\chapref{Para_9}).

\end{enumerate}

Some examples of person and place names attested in the corpus are presented in \tabref{Table_5.2}. Original Papuan Malay names, however, do not exist as such. The person names are very commonly taken from the Bible or originate from European languages. Family or clan names and place names originate from local languages, such as the Papuan language \ili{Isirawa} (see also §\ref{Para_1.4}). The examples in \tabref{Table_5.2} also illustrate that person names with more than two syllables are most commonly shortened to two-syllable names.


\begin{table} 
\caption{Proper nouns: Person and place names}\label{Table_5.2}
\begin{tabular}{llll}
\lsptoprule

\multicolumn{2}{c}{ Male person names} & \multicolumn{2}{c}{ Female person names}\\
 Long form & Short form & Long form &  Short form\\
\midrule
\textitbf{Abimelek} & \textitbf{Abi} & \textitbf{Antonia} & \textitbf{Anto}\\
\textitbf{Benyamin} & \textitbf{Beni} & \textitbf{Fransiska} & \textitbf{Siska}\\
\textitbf{Dominggus} & \textitbf{Domi} & \textitbf{Gerice} & \textitbf{Ice}\\
\textitbf{Edwart} & \textitbf{Edo} & \textitbf{Hendrika} & \textitbf{Ika}\\
\textitbf{Hermanus} & \textitbf{Herman} & \textitbf{Isabela} & \textitbf{Ise}\\
\textitbf{Kornelius} & \textitbf{Kori} & \textitbf{Magdalena} & \textitbf{Magda}\\
\textitbf{Lodowik} & \textitbf{Lodo} & \textitbf{Pawlina} & \textitbf{Pawla}\\
\textitbf{Martinus} & \textitbf{Tinus} & \textitbf{Samalina} & \textitbf{Lina}\\
\textitbf{Pontius} & \textitbf{Ponti} & \textitbf{Sarlota} & \textitbf{Ota}\\
\textitbf{Sokarates} & \textitbf{Ates} & \textitbf{Yohana} & \textitbf{Hana}\\
\midrule
\multicolumn{2}{c}{ Clan and family names} & \multicolumn{2}{c}{ Place names}\\
\midrule
\textitbf{Aweta} & \textitbf{Manierong} & \textitbf{Arbais} & \textitbf{Mararena}\\
\textitbf{Cawem} & \textitbf{Merne} & \textitbf{Betaf} & \textitbf{Rotea}\\
\textitbf{Catwe} & \textitbf{Sefanya} & \textitbf{Dabe} & \textitbf{Sarmi}\\
\textitbf{Domanser} & \textitbf{Sope} & \textitbf{Karfasia} & \textitbf{Takar}\\
\textitbf{Kaywor} & \textitbf{Yapo} & \textitbf{Liki} & \textitbf{Webro}\\
\lspbottomrule
\end{tabular}
\end{table}



Modification of proper nouns with \isi{monovalent} stative verbs, personal pronouns, de\-monstratives, locatives, interrogatives, numerals, quantifiers, and relative clauses is illustrated in (\ref{Example_5.13}) to (\ref{Example_5.20}), respectively.\footnote{Documentation: \isi{verb} 081011-024-Cv.0142, personal \isi{pronoun} 080916-001-CvNP.0003, \isi{demonstrative} 080917-008-NP.0043, \isi{locative} 080917-008-NP.0118, \isi{interrogative} 080922-001a-CvPh.1245, \isi{numeral} 080922-002-Cv.0052, \isi{quantifier} 080922-010a-NF.0269, relative clause 080919-006-CvNP.0017.}

 
\ea
\label{Example_5.13}
\gll \textitbf{Jayapura} \textitbf{besar} \textitbf{itu}\\ 
Jayapura be.big. \textsc{d.dist}.  \\
\glt ‘that big (city of) Jayapura’\\
\z 
\ea
\label{Example_5.14} 
\gll  \textitbf{Iskia} \textitbf{de}\\
Iskia \textsc{3sg}  \\
\glt ‘Iskia’ (Lit. ‘he Iskia’)\\
\z 
\ea
\label{Example_5.15}
\gll  \textitbf{Sarmi} \textitbf{itu}\\
\ili{Sarmi} \textsc{d.dist}.  \\
 \glt ‘that (city of) \ili{Sarmi}’\\
\z 
\ea
\label{Example_5.16}
\gll  \textitbf{Paynete} \textitbf{situ}\\
Paynete \textsc{l.med} \\
 \glt ‘Paynete there’\\
\z 
\ea
\label{Example_5.17}
\gll \textitbf{Muay} \textitbf{mana?}\\
Muay where? \\
 \glt ‘which Muay?’\\
\z 
\ea
\label{Example_5.18}
\gll \textitbf{Suebu} \textitbf{satu} \textitbf{ni}\\ 
Suebu one \textsc{d.prox}  \\
 \glt ‘this certain (member of the) Suebu (family)’\\
\z 
\ea
\label{Example_5.19}
\gll \textitbf{Sope} \textitbf{banyak}\\
Sope many  \\
\glt ‘many Sope (family members)’\\
\z 
\ea
\label{Example_5.20}
\gll \textitbf{Wili} \textitbf{yang} \textitbf{tinggal}\\
Wili \textsc{rel} stay\\
\glt ‘Wili who’ll stay’\\
\z 


When addressing interlocutors or talking about others, speakers very commonly introduce person names with common nouns that indicate kinship relations or are used as honorifics, as shown in \tabref{Table_5.3}. Likewise, place names are often preceded by common nouns denoting geographical entities.


\begin{table}
\caption[Introduced person and place names]{Introduced person and place names\footnote{Documentation of person names: 080922-001a-CvPh.1096, 081011-024-Cv.0123, 081014-005-Cv.0002, 081014-014-CvNP.0084. 
Documentation of place names: 080922-002-Cv.0049, 080917-008-NP.0018, 081025-008-Cv.0008, 080917-008-NP.0126.}\label{Table_5.3}}

\begin{tabular}{ll}
\lsptoprule

\multicolumn{2}{c}{ Introduced person names}\\
\midrule
\textitbf{ade Aris} & ‘younger sibling Aris’\\
\textitbf{mama Sance} & ‘mama Sance’\\
\textitbf{bapa-tua Fredi} & ‘uncle Fredi’\\
\textitbf{tete Daut} & ‘grandfather Daut’\\
\textitbf{mace Agustina} & ‘Ms. Agustina’\\
\textitbf{pace Alpeus} & ‘Mr. Alpeus’\\
\midrule
\multicolumn{2}{c}{ Introduced places names}\\
\midrule
\textitbf{kampung Harapang} & ‘Harapang village’\\
\textitbf{kota Sarmi} & ‘\ili{Sarmi}’ city\\
\textitbf{kali Biri} & ‘Biri’ river\\
\textitbf{pulow Sarmi} & ‘\ili{Sarmi} island’\\
\lspbottomrule
\end{tabular}
\end{table}

\subsection{Location nouns}
\label{Para_5.2.3}
Location nouns, or \isi{locative} nouns, designate locations rather than physical objects. The Papuan Malay location nouns are given in \tabref{Table_5.4}, together with their token frequencies in the corpus.


\begin{table}
\caption{Papuan Malay location nouns}\label{Table_5.4}

\begin{tabular}{llr}
\lsptoprule
 \multicolumn{1}{c}{Item} & \multicolumn{1}{c}{Gloss} &  \# tokens\\
\midrule
\textitbf{atas} & ‘top’ &  146\\
\textitbf{bawa} & ‘bottom’ &  116\\
\textitbf{blakang} & ‘backside’ &  92\\
\textitbf{dalam} & ‘inside’ &  230\\
\textitbf{depang} & ‘front’ &  102\\
\textitbf{luar} & ‘outside’ &  79\\
\textitbf{pinggir} & ‘border’ &  23\\
\textitbf{samping} & ‘side’ &  24\\
\textitbf{sebla} & ‘side’ &  110\\
\textitbf{sekitar} & ‘vicinity’ &  17\\
\textitbf{tenga} & ‘middle’ &  42\\
\lspbottomrule
\end{tabular}
\end{table}

Location nouns are distinct from common nouns (§\ref{Para_5.2.1}) in terms of the following properties:



\begin{enumerate}
\item 
In their nominal uses, location nouns (a) only occur in prepositional phrases, (b) can be modified with nouns, demonstratives, or locatives, but with no other constituents, and (c) do not take the possessor or possessum slots in adnominal possessive constructions.\footnote{The exception is \textitbf{blakang} ‘backside’. It also has the body part meaning ‘back’. As such it can denote the possessum in an \isi{adnominal possessive construction} such as \textitbf{sa pu blakang} ‘\textsc{1sg} \textsc{poss} backside’ ‘my back’ [081015-005-NP.0032].}

\item   In their adnominal uses, location nouns are juxtaposed to common nouns only; that is, unlike common nouns, they cannot be stacked.

\end{enumerate}

Location nouns are distinct from direction nouns (§\ref{Para_5.2.4}) in that they can be modified with juxtaposed adnominal nouns, while direction nouns cannot be modified in this way.



The nominal uses of the location nouns are discussed in §\ref{Para_5.2.3.1} and their adnominal uses in §\ref{Para_5.2.3.2}.


\subsubsection[Nominal uses]{Nominal uses}
\label{Para_5.2.3.1}
In their nominal uses, the nouns always occur inside prepositional phras\-es and are typically modified with a juxtaposed adnominal \isi{noun} such that ``\textsc{prep} \textsc{n.loc} \textsc{n}''. Semantically, \textsc{n.loc} \textsc{n} \isi{noun} phrases are characterized by the subordination of the adnominal \isi{noun} in \textsc{N2} position under the head nominal location \isi{noun} in N1 position (see also §\ref{Para_8.2.2}).



Generally speaking, the main function of location nouns is to specify the spatial relationship between a figure and the ground \citep[3]{Levinson.2006}, with the ground being encoded by the juxtaposed adnominal \isi{noun}. The same applies to the Papuan Malay location nouns, in that they more fully specify the spatial relationship between figure and ground than is achieved by a bare \isi{preposition} that introduces the ground. This is illustrated with the contrastive examples in (\ref{Example_5.21}) to (\ref{Example_5.23}) and in (\ref{Example_5.24}) and (\ref{Example_5.25}).


\begin{styleExampleTitle}
``\textsc{prep} \textsc{n.loc} \textsc{n}'' versus ``\textsc{prep} \textsc{n}'' prepositional phrases\footnote{Documentation: 081006-023-CvEx.0061, 081109-002-JR.0002, 081006-023-CvEx.0080, 081011-001-Cv.0167, 080919-004-NP.0030.}
\end{styleExampleTitle}


\ea
\label{Example_5.21}
\gll {\textitbf{di}} {\textitbf{atas}} {\textitbf{pohong}}\\ %
at  top  tree    \\
\glt

‘at the top of the tree’\\
\z


\ea
\label{Example_5.22}
\gll  \textitbf{di}  \textitbf{bawa}  \textitbf{pohong}\\
 at  bottom  tree    \\
\glt

‘under the tree’\\
\z


\ea
\label{Example_5.23}
\gll \textitbf{di} \textitbf{pohong}\\ %
at    tree\\
\glt

‘in the tree’\\
\z


\ea
\label{Example_5.24}
\gll {\textitbf{di}} {\textitbf{pinggir}} {\textitbf{kali}}\\ %
at  border  river    \\
\glt

‘alongside the river’\\
\z


\ea
\label{Example_5.25}
\gll {\textitbf{di}} {\textitbf{kali}}\\ %
 at    river    \\
\glt

‘in the river’\\
\z



More examples illustrating the nominal uses of locations nouns in prepositional phrases are given in (\ref{Example_5.26}) to (\ref{Example_5.36}).


\begin{styleExampleTitle}
Location nouns with nominal modifier\footnote{Documentation: \textitbf{atas} ‘top’ 081025-008-Cv.0162, \textitbf{bawa} ‘bottom’ 081025-009b-Cv.0018, \textitbf{blakang} ‘backside’ 081106-001-Ex.0002, \textitbf{dalam} ‘inside’ 081025-006-Cv.0039, \textitbf{depang} ‘front’ 081115-001a-Cv.0139, \textitbf{luar} ‘outside’ 081025-003-Cv.0159, \textitbf{pinggir} ‘border’ 080918-001-CvNP.0060, \textitbf{samping} ‘side’ 081014-014-CvNP.0046, \textitbf{sebla} ‘side’ 081109-001-Cv.0026, \textitbf{sekitar} ‘vicinity’ 081011-024-Cv.0140, \textitbf{tenga} ‘middle’ 080927-009-CvNP.0037.}
\end{styleExampleTitle}

\eabox[-1.9em]{\label{Example_5.26}
\aktab{atas}{`top'}{\textitbf{dari} \textitbf{atas} \textitbf{kursi}}{from top chair}{‘from the top of the chair’}}

\eabox[-1.9em]{\label{Example_5.27}
\aktab{bawa}{`bottom'}{\textitbf{di} \textitbf{bawa} \textitbf{meja}}{at bottom table}{‘below the table’}}

\eabox[-1.9em]{\label{Example_5.28}
\aktab{blakang}{`backside'}{\textitbf{dengang} \textitbf{blakang} \textitbf{kapak}}{with backside axe}{‘with the backside of the axe’}}

\eabox[-1.9em]{\label{Example_5.29}
\aktab{dalam}{`inside'}{\textitbf{di} \textitbf{dalam} \textitbf{kamar}}{at inside room}{‘inside the room’}}

\eabox[-1.9em]{\label{Example_5.30}
\aktab{depang}{`front'}{\textitbf{di} \textitbf{depang} \textitbf{greja} \textitbf{tu}}{at front church \textsc{d.dist}}{‘in front of that church’}}

\eabox[-1.9em]{\label{Example_5.31}
\aktab{luar}{`outside'}{\textitbf{ke} \textitbf{luar} \textitbf{negri}}{to outside country}{‘abroad’}}

\eabox[-1.9em]{\label{Example_5.32}
\aktab{pinggir}{`border'}{\textitbf{di} \textitbf{pinggir} \textitbf{jalang}}{at border walk}{‘alongside the road’}}

\eabox[-1.9em]{\label{Example_5.33}
\aktab{samping}{`side'}{\textitbf{di} \textitbf{samping} \textitbf{ruma}}{at side house}{‘beside the house’}}

\eabox[-1.9em]{\label{Example_5.34}
\aktab{sebla}{`side'}{\textitbf{ke} \textitbf{sebla} \textitbf{darat}}{to side land}{‘landwards’}}

\eabox[-1.9em]{\label{Example_5.35}
\aktab{sekitar}{`vicinity'}{\textitbf{di} \textitbf{sekitar} \textitbf{Pante-Barat}}{at vicinity Pante-Barat}{‘in the vicinity of Pante-Barat’}}

\eabox[-1.9em]{\label{Example_5.36}
\aktab{tenga}{`middle'}{\textitbf{di} \textitbf{tenga} \textitbf{hutang}}{at middle forest}{‘in the middle of the forest’}}


In the examples in (\ref{Example_5.26}) to (\ref{Example_5.36}), the ground, encoded by the adnominal \isi{noun} in \textsc{N2} position, is mentioned overtly. If the ground is understood from the context, though, the adnominal \isi{noun} denoting it can be omitted and the location \isi{noun} is used as an independent nominal as in (\ref{Example_5.37}) to (\ref{Example_5.40}). In (\ref{Example_5.37}) the ground is understood from the speech situation: it is the house where the speech acts occurs. In (\ref{Example_5.38}) to (\ref{Example_5.40}) the ground is understood from the discourse: it is \textitbf{kitorang tiga} ‘we three’ in (\ref{Example_5.38}), \textitbf{sumur} ‘well’ in (\ref{Example_5.39}), and \textitbf{bandara} ‘airport’ in (\ref{Example_5.40}).


\begin{styleExampleTitle}
Location nouns with omitted nominal modifier
\end{styleExampleTitle}
\ea
\label{Example_5.37}
\gll {tida} {{usa}} {{kamu}} {{duduk}} {\bluebold{di}} {{\bluebold{depang},}}\\ %
 \textsc{neg}  {need.to}  {\textsc{2pl}}  {sit}  at  {front}\\
\gll  {ana}  {prempuang}  {itu}  {duduk}  \bluebold{di}  \bluebold{blakang}\\
 {child}  {woman}  {\textsc{d.dist}}  {sit}  at  backside\\
\glt 
‘it’s not necessary that you sit \bluebold{in front (of the house)}, as for girls, (they) sit \bluebold{in the back (of the house)}’ \textstyleExampleSource{[081115-001a-Cv.0316]}
\z
\ea
\label{Example_5.38}
\gll {kitorang} {tiga} {\ldots} {naik} {di} {motor} {\ldots} {Martina} {\bluebold{di}} {\bluebold{tenga}}\\ %
 \textsc{1pl}  three  { }  ascend  at  motorbike  {}  Martina  at  middle\\
\glt 
‘we three {\ldots} got onto the motorbike {\ldots} Martina was \bluebold{in the middle}’ \textstyleExampleSource{[081015-005-NP.0020]}
\z
\ea
\label{Example_5.39}
\gll {sumur} {itu} {masi} {ada} {{\ldots}} {\bluebold{di}} {\bluebold{dalam}} {\bluebold{tu}} {ada} {senjata}\\ %
 well  \textsc{d.dist}  still  exist { }   at  inside  \textsc{d.dist}  exist  rifle\\
\glt 
‘that well still exists {\ldots} \bluebold{inside there} are rifles’ \textstyleExampleSource{[080922-010a-CvNF.0120-0121]}
\z
\ea
\label{Example_5.40}
\gll {{pas}} {{turung}} {{bandara}} {{Sentani}} {{pas}} {{de}} {{ketemu}} {dengang}\\ %
 {precisely}  {descend}  {airport}  {Sentani}  {precisely}  {\textsc{3sg}}  {meet}  with\\
\gll   {Wamena}  {dorang,}  {pas}  {Wamena}  dong  {\bluebold{di}}  {\bluebold{pinggir}}  {\bluebold{situ}}\\
 {Wamena}  {\textsc{3pl}}  {precisely}  {Wamena}  \textsc{3pl}  {at}  {border}  {\textsc{l.med}}\\
\glt 
‘the moment (he) landed (at) \ili{Sentani} airport, he met the Wamena people, right then the Wamena people were (sitting) \bluebold{alongside (the airstrip)} \bluebold{there}’ \textstyleExampleSource{[081109-009-JR.0003]}
\z


The examples in (\ref{Example_5.39}) and (\ref{Example_5.40}) also illustrate that an independently used location \isi{noun} can be modified with a \isi{demonstrative} or a \isi{locative}, respectively.



As shown so far, location nouns more fully specify the spatial relationship between a figure and the ground than is achieved by a bare \isi{preposition} that introduces the ground. If the specific spatial relationship can be deduced from the context, though, the location \isi{noun} can be omitted as illustrated with elided \textitbf{atas} ‘top’ in (\ref{Example_5.41}) and \textitbf{dalam} ‘inside’ in (\ref{Example_5.42}).


\begin{styleExampleTitle}
Omitted location nouns
\end{styleExampleTitle}

\ea
\label{Example_5.41}
\gll {de} {kas} {turung} {mama} {Petrus} \bluebold{dari} \bluebold{\st{atas}} \bluebold{kursi} {to?}\\ %
 \textsc{3sg}  give  descend  mother  Petrus  from  \st{top}  chair  right?\\
\glt 
‘he (the evil spirit) threw mother Petrus \bluebold{from (the top of her) chair}, right?’ \textstyleExampleSource{[081025-008-Cv.0158]}
\z
\ea
\label{Example_5.42}
\gll {dong} {mandi} {\bluebold{di}} \bluebold{\st{dalam}} {\bluebold{kamar}} {\bluebold{mandi}} {\bluebold{sana}}\\ %
 \textsc{3pl}  bathe  at  \st{inside}  room  bathe  \textsc{l.dist}\\
\glt
‘they were bathing \bluebold{in(side of) the bathroom over there}’ \textstyleExampleSource{[081109-001-Cv.0081]}

\z

\subsubsection[Adnominal uses]{Adnominal uses}
\label{Para_5.2.3.2}
In their adnominal uses, the location nouns are juxtaposed to common nouns or, although much less frequently, to common nouns with juxtaposed adnominal personal pronouns, such that ``\textsc{n} (\textsc{pro}) \textsc{n.loc}''. In their adnominal uses, they signal \isi{locational} relations. Overall, though, the adnominal uses of location nouns are marginal: of a total of 981 tokens, only 35 (4\%) have adnominal uses, whereas 946 have nominal uses (96\%).



In designating \isi{locational} relations, the location nouns have restrictive function. That is, they signal that the referent encoded by the head nominal is precisely the one situated in the spatial location designated by the location \isi{noun}. Thereby, the location \isi{noun} aids the hearer in the identification of the referent, as in \textitbf{jalang atas} ‘upper road’ in (\ref{Example_5.43}), \textitbf{rem blakang} ‘rear brakes’ in (\ref{Example_5.44}), or \textitbf{tetangga dong sebla} ‘the neighbors next door’ in (\ref{Example_5.49}). The \isi{locational} relation can also be figurative as in \textitbf{generasi bawa} ‘next generation’ in (\ref{Example_5.51}), or in \textitbf{dunia luar} ‘outside world’ in (\ref{Example_5.46}), or temporal as in \textitbf{bulang depang} ‘next month’ in (\ref{Example_5.52}). Adnominal uses for \textitbf{sekitar} ‘vicinity’ are unattested in the corpus.


\begin{styleExampleTitle}
Locational relations: Spatial and figurative\footnote{Documentation: \textitbf{atas} ‘top’ BR111031-001.005, \textitbf{blakang} ‘backside’ 081022-002-Cv.0013, \textitbf{dalam} ‘inside’ 081025-006-Cv.0023, \textitbf{luar} ‘outside’ 081029-002-Cv.0033, \textitbf{pinggir} ‘edge’ 080923-010-CvNP.0010, \textitbf{samping} ‘side’ BR111031-001.004, \textitbf{sebla} ‘side’ 081006-035-CvEx.0067, \textitbf{tenga} ‘middle’ 081014-006-Pr.0037, \textitbf{bawa} ‘bottom’ 081011-024-Cv.0148, \textitbf{depang} ‘front’ 080921-011-Cv.0012.}
\end{styleExampleTitle}



\eabox[-1.9em]{\label{Example_5.43}
\aktab{atas}{`top'}{\textitbf{jalang} \textitbf{atas}}{walk top}{‘the upper road’ (Lit. ‘walk on top xxx’)}}

\eabox[-1.9em]{\label{Example_5.44}
\aktab{\textitbf{blakang}}
{‘backside’}%
{\textitbf{rem} \textitbf{blakang}}%  
{brake backside}%
{‘rear brakes’}}

\eabox[-1.9em]{
\label{Example_5.45}
\aktab{\textitbf{dalam}\vphantom{j}}
{‘inside’}
{\textitbf{kolor} \textitbf{dalam}}
{shorts inside}
{‘undershorts’}}

\eabox[-1.9em]{
\label{Example_5.46}
\aktab{\textitbf{luar}}
{‘outside’}
{\textitbf{dunia} \textitbf{luar}\vphantom{j}} 
{world outside}
{‘outside world’}}

\eabox[-1.9em]{
\label{Example_5.47}
\aktab{\textitbf{pinggir}}
{‘border’}
{\textitbf{tana} \textitbf{pinggir}}
{ground border}
{‘the ground along the side’}}

\eabox[-1.9em]{
\label{Example_5.48}
\aktab{\textitbf{samping}}
{‘side’}
{\textitbf{sak} \textitbf{samping}}
{bag side}
{‘side pocket’}}

\eabox[-1.9em]{
\label{Example_5.49}
\aktab{\textitbf{sebla}}
{‘side’}
{\textitbf{tetangga} \textitbf{dong} \textitbf{sebla}} 
{neighbor \textsc{3pl} side}
{‘the neighbors next door’}}

\eabox[-1.9em]{
\label{Example_5.50}
\aktab{\textitbf{tenga}}
{‘middle’}
{\textitbf{kolam} \textitbf{tenga}}
{big.hole middle}
{‘the pond in the middle’}}

\eabox[-1.9em]{
\label{Example_5.51}
\aktab{\textitbf{bawa}\vphantom{j}} 
{‘bottom’}
{\textitbf{generasi} \textitbf{bawa}}
{generation bottom }
{‘next generation’ (Lit. ‘generation at bottom’)}}

\eabox[-1.9em]{
\label{Example_5.52}
\aktab{\textitbf{depang}}
{‘front’}
{\textitbf{bulang} \textitbf{depang}}
{month front}
{‘next month’ (Lit. ‘month in front’)}}


\subsection{Direction nouns}
\label{Para_5.2.4}
Direction nouns express cardinal directions and relative directions. The former designate the four principal compass points, while the latter express left-right orientation. The Papuan Malay direction nouns are presented in \tabref{Table_5.5}, together with their token frequencies in the corpus (given their low token frequencies, most examples in this section are elicited).


\begin{table}
\caption{Papuan Malay cardinal and relative directions}\label{Table_5.5}


\begin{tabular}{llr}
\lsptoprule
 \multicolumn{1}{c}{Item} & \multicolumn{1}{c}{Gloss} &  \# tokens\\
\midrule
\textitbf{utara} & ‘north’ &  {}-{}-{}-\\
\textitbf{slatang} & ‘south’ &  {}-{}-{}-\\
\textitbf{barat} & ‘west’ &  10\\
\textitbf{timur} & ‘east’ &  5\\
\textitbf{kiri} & ‘left’ &  1\\
\textitbf{kanang} & ‘right’ &  2\\
\lspbottomrule
\end{tabular}
\end{table}

\newpage 
Direction nouns have the following distributional properties:


\begin{enumerate}
\item 
Direction nouns occur in prepositional phrases as independent heads of the \isi{noun} phrase within the \isi{prepositional phrase}; they are unattested as head nominals in unembedded \isi{noun} phrases.

\item 
Direction nouns have adnominal uses; that is, they occur in \isi{noun} phrases with a preceding \isi{noun} as nominal head.
\item 
Direction nouns can be modified with adnominally used demonstratives or locatives.
\end{enumerate}

Direction nouns are distinct from common nouns (§\ref{Para_5.2.1}) and location nouns (§\ref{Para_5.2.3}) in terms of the following properties:


\begin{enumerate}
\item 
Contrasting with common nouns, direction nouns (a) are unattested as heads of unembedded \isi{noun} phrases, (b) are only modified with demonstratives and locatives, and (c) are unattested in adnominal possessive constructions, neither as the possessor nor as the possessum.

\item  
Contrasting with location nouns, direction nouns with juxtaposed adnominal nouns are unattested when employed as nominals in prepositional phrases.
 
\end{enumerate}

Direction nouns typically occur as complements in prepositional phrases, as shown with the four cardinal directions in (\ref{Example_5.53}) to (\ref{Example_5.56}) and the two relative directions in (\ref{Example_5.57}) and (\ref{Example_5.58}). Direction nouns can be modified with demonstratives as in \textitbf{utara ini} ‘this north’ in (\ref{Example_5.53}) or \textitbf{kiri ini} ‘this left’ in (\ref{Example_5.57}), or with locatives as in \textitbf{slatang sana} ‘south over there’ in (\ref{Example_5.54}) or \textitbf{kanang sana} ‘right over there’ in (\ref{Example_5.58}).


\begin{styleExampleTitle}
Direction nouns as complements in prepositional phrases
\end{styleExampleTitle}


\ea
\label{Example_5.53}
\gll {sa} {pu} {prahu} {hanyut} {sampe} {ke} {\bluebold{utara}} {\bluebold{ini}}\\ %
 \textsc{1sg}  \textsc{poss}  boat  drift  reach  to  north  \textsc{d.prox}\\
\glt 
‘my boat drifted up to the \bluebold{north here}’ \textstyleExampleSource{[Elicited BR130103.018]}
\z

\ea
\label{Example_5.54}
\gll {pohong} {gaharu} {tu} {paling} {banyak} {di} \bluebold{slatang} \bluebold{sana}\\ %
 tree  agarwood  \textsc{d.dist}  most  many  at  south  \textsc{l.dist}\\
\glt 
‘agarwood trees are most common in the \bluebold{south over there}’ \textstyleExampleSource{[Elicited BR130103.017]}
\z
\ea
\label{Example_5.55}
 {de}  {blang,}  {a},  sa  datang  dari  \bluebold{barat}\\
 \textsc{3sg}  {say}  {ah!}  \textsc{1sg}  come  from  west\\
\glt 
‘he said, ``ah, I come from the \bluebold{west}''' \textstyleExampleSource{[080922-010a-CvNF.0237]}
\z
\ea
\label{Example_5.56}
\gll {pesawat} {ini} {de} {terbang} {ke} {\bluebold{timur}} {dulu}\\ %
 airplane  \textsc{d.prox}  \textsc{3sg}  fly  to  east  first\\

\glt 
‘this plane it flies to the \bluebold{east} first’ \textstyleExampleSource{[Elicited BR130103.014]}
\z
\ea
\label{Example_5.57}
\gll {{pesawat}} {{de}} {{terbang}} {dari} {\bluebold{kiri}} {\bluebold{ini},} {baru} {lewat}\\ %
 {airplane}  {\textsc{3sg}}  {fly}  from  left  \textsc{d.prox}  and.then  pass.by\\
 \gll sana  {trus}  {ke}  {Wamena}\\
 \textsc{l.dist}  {next}  {to}  {Wamena}\\
\glt 
‘the plane flies from the \bluebold{left here} and passes by over there (and) then (it flies on) to Wamena’ \textstyleExampleSource{[Elicited BR130103.022]}
\z
\ea
\label{Example_5.58}
\gll {{ko}} {jalang} {{trus,}} {baru} {ko} {putar}\\ %
 {\textsc{2sg}}  walk  {be.continuous}  and.then  \textsc{2sg}  turn.around\\
\gll  ke  {\bluebold{kanang}}  {\bluebold{sana}}\\
 to  {right}  {\textsc{l.dist}}\\
\glt 
‘you walk on, only then you turn to the \bluebold{right over there}’ \textstyleExampleSource{[Elicited BR130103.005]}
\z


In (\ref{Example_5.57}) and (\ref{Example_5.58}) the \isi{preposition} is obligatory. With motion verbs that also express direction, however, the allative \isi{preposition} \textitbf{ke} ‘to’ may also be omitted. This is illustrated in (\ref{Example_5.59}) with the motion \isi{verb} \textitbf{belok} ‘turn’. (For details on the \isi{elision} of prepositions encoding location, see §\ref{Para_10.1.5}.)


\begin{styleExampleTitle}
Elision of the \isi{preposition}
\end{styleExampleTitle}

\ea
\label{Example_5.59}
\gll {di} {{jembatang}} {{depang}} {{ko}} {{belok}} {Ø} {\bluebold{kanang}} {trus} {di} {jembatang}\\ %
 at  {bridge}  {front}  {\textsc{2sg}}  {turn} {}   right  next  at  bridge\\
\gll  {depang}  lagi  ko  {belok}  {Ø}  {\bluebold{kiri}}\\
 {front}  again  \textsc{2sg}  {turn}  {}  {left}\\
\glt 
‘at the bridge ahead you turn \bluebold{right}, and then at the next bridge you turn \bluebold{left}’ \textstyleExampleSource{[Elicited BR130103.002]}
\z



In their adnominal uses, the direction nouns are juxtaposed to a head nominal. Semantically, these \isi{noun} phrases designate ``subtype-of'' relations as in \textitbf{bagiang barat} ‘western part’ and \textitbf{bagiang timur} ‘eastern part’ in (\ref{Example_5.60}), or they denote \isi{locational} relations as in \textitbf{sebla kiri} ‘left side’ in (\ref{Example_5.61}), or in \textitbf{tangang kanang} ‘right hand/arm’ in (\ref{Example_5.62}).


\begin{styleExampleTitle}
Adnominal uses of direction nouns
\end{styleExampleTitle}
\ea
\label{Example_5.60}
\gll {kalo} {{\bluebold{bagiang}}} {{\bluebold{barat}}} {{itu}} {{kasiang}} {prempuang} {tokok} {prempuang}\\ %
 if  {part}  {west}  {\textsc{d.dist}}  {pity}  woman  tap  woman\\
\gll  {ramas}  tapi  {kalo}  {\bluebold{bagiang}}  {\bluebold{timur}}  {tida}\\
 {press}  but  {if}  {part}  {east}  {\textsc{neg}}\\
\glt 
[About regional differences within the regency:] ‘as for the \bluebold{western part} there, (it’s a) pity, the women tap (and) the women press (the sagu) but as for the \bluebold{eastern part} (it’s) not (like that)’ \textstyleExampleSource{[081014-007-CvEx.0025-0026]}
\z
\ea
\label{Example_5.61}
\gll {lapangang} {bola} {kaki} {ada} {di} {\bluebold{sebla}} {\bluebold{kiri}}\\ %
 field  ball  foot  exist  at  side  left\\
\glt 
‘the football field is on the \bluebold{left side}’ \textstyleExampleSource{[Elicited BR130103.011]}
\z
\ea
\label{Example_5.62}
\gll {tulang} {yang} {\bluebold{tangang}} {\bluebold{kanang}} {ini} {su} {kluar} {ke} {samping}\\ %
 bone  \textsc{rel}  hand  right  \textsc{d.prox}  already  go.out  to  side\\
\glt
[About an accident:] ‘the bone of the \bluebold{right arm} here already stuck out sideways’ \textstyleExampleSource{[081108-003-JR.0006]}
\z


\subsection{Time-denoting nouns}
\label{Para_5.2.5}
The label ``time-denoting nouns'' refers to nouns which denote time units (§\ref{Para_5.2.5.1}), the periods of the day (§\ref{Para_5.2.5.2}), the days of the week and months of the year (§\ref{Para_5.2.5.3}), and \isi{relative time} (§\ref{Para_5.2.5.4}). Time-denoting nouns have the same syntactic properties as common nouns (for details see §\ref{Para_5.2.1}.)


\subsubsection[Time units]{Time units}
\label{Para_5.2.5.1}
 \tabref{Table_5.6} lists the different time-denoting nouns that divide a year into smaller units.


\begin{table}
\caption{Time units}\label{Table_5.6}


\begin{tabular}{ll@{\hspace*{8mm}}ll}
\lsptoprule
 \multicolumn{1}{c}{Item} & \multicolumn{1}{c}{Gloss} & \multicolumn{1}{c}{Item} &  \multicolumn{1}{c}{Gloss}\\
\midrule
\textitbf{titik} & ‘second’ & \textitbf{minggu} & ‘week’\\
\textitbf{minit} & ‘minute’ & \textitbf{bulang} & ‘month’\\
\textitbf{jam} & ‘hour’ & \textitbf{taung} & ‘year’\\
\textitbf{hari} & ‘day’ &  & \\
\lspbottomrule
\end{tabular}
\end{table}

The time units listed in \tabref{Table_5.6}  are count nouns that can be modified with numerals or quantifiers, as illustrated in (\ref{Example_5.63}) and (\ref{Example_5.64}). In addition to designating a \isi{time unit}, \textitbf{minggu} ‘week’ also denotes a day of the week, namely ‘Sunday’ (see \tabref{Table_5.8} ).


\ea
\label{Example_5.63}
\gll {bapa} {bilang} {begini,} {tunggu} {\bluebold{lima}} {\bluebold{blas}} {\bluebold{minit}} {to?}\\ %
 father  say  like.this  wait  five  teens  minute  right?\\
\glt 
‘father said like this, ``wait \bluebold{fifteen minutes}, right?!''' \textstyleExampleSource{[081025-006-Cv.0173]}
\z
\ea
\label{Example_5.64}
\gll  brapa   bulang  dorang skola        ka  mace,  \bluebold{lima}  \bluebold{bulang}  ka?\\
 several month \textsc{3.pl}  go.to.school or woman five month or\\
\glt 
‘for how many months have they been going to school, Madam, for \bluebold{five months}, right?’ \textstyleExampleSource{[081025-003-Cv.0207]}
\z


\subsubsection[Periods of the day]{Periods of the day}
\label{Para_5.2.5.2}
 \tabref{Table_5.7} presents the time-denoting nouns for the four periods of the day. More specifically, \textitbf{pagi} ‘morning’ designates the period from just after midnight until about eleven o’clock, while \textitbf{siang} ‘midday’ refers to the time from about eleven o’clock until about fourteen hours. The next period, \textitbf{sore} ‘afternoon’, lasts until about eighteen hours when darkness sets in, while \textitbf{malam} ‘night’ denotes nighttime.


\begin{table}
\caption{Periods of the day}\label{Table_5.7}
\begin{tabular}{ll@{\hspace*{8mm}}ll}
\lsptoprule
\multicolumn{1}{c}{Item} &\multicolumn{1}{c}{Gloss} &\multicolumn{1}{c}{Item} & \multicolumn{1}{c}{Gloss}\\
\midrule
\textitbf{pagi} & ‘morning’ & \textitbf{sore} & ‘afternoon’\\
\textitbf{siang} & ‘midday’ & \textitbf{malam} & ‘night’\\
\lspbottomrule
\end{tabular}
\end{table}

The four periods-of-the-day expressions are count nouns that can be modified with numerals or quantifiers, as shown in (\ref{Example_5.66}) and (\ref{Example_5.67}). In addition, these expressions are also used as modifiers within \isi{noun} phrases, as in (\ref{Example_5.68}) to (\ref{Example_5.70}).


\begin{styleExampleTitle}
Head and modifier functions
\end{styleExampleTitle}

\ea
\label{Example_5.66}
\gll {saya} {hanya} {bisa} {makang}, {kasi} {makang} {dorang} {\bluebold{satu}} {\bluebold{malam}} {saja}\\ %
 \textsc{1sg}  only  be.able  eat  give  eat  \textsc{3pl}  one  night  just\\
\glt 
‘I can only eat, feed them just \bluebold{one night}’ \textstyleExampleSource{[081011-020-Cv.0080]}
\z

\ea
\label{Example_5.67}
\gll {ko} {harus} {\bluebold{setiap}} {\bluebold{pagi}} {harus} {jalang} {trus}\\ %
 \textsc{2sg}  have.to  every  morning  have.to  walk  be.continuous\\
\glt 
[About attending school:] ‘you have to (go to school) \bluebold{every morning}, (you) have to go regularly’ \textstyleExampleSource{[080917-007-CvHt.0004]}
\z
\ea
\label{Example_5.68}
\gll {tra} {ada} {\bluebold{snek}} {\bluebold{pagi}}\\ %
 \textsc{neg}  exist  snack  morning\\
\glt 
‘there was no \bluebold{morning snack}’ \textstyleExampleSource{[081025-008-Cv.0079]}
\z
\ea
\label{Example_5.69}
\gll  \bluebold{hari}  \bluebold{sening}  \bluebold{sore}  \bluebold{itu}  smua  harus  hadir\\
 day  Monday  afternoon  \textsc{d.dist}  all  have.to  attend\\
\glt 
[About volleyball training:] ‘\bluebold{next Monday afternoon} everyone has to attend’ \textstyleExampleSource{[081109-001-Cv.0053]}
\z
\ea
\label{Example_5.70}
\gll {dari} {jam} {dua} {blas} {tong} {makang} {sampe} {\bluebold{jam}} {\bluebold{satu}} {\bluebold{siang}}\\ %
 from  hour  two  teens  \textsc{1pl}  eat  until  hour  one  midday\\
\glt 
‘we ate from twelve o’clock until \bluebold{one o’clock midday}’ \textstyleExampleSource{[081025-008-Cv.0085]}
\z



Within the clause, the four expressions typically occur at clause boundaries. Most often, they occur in clause-initial position where they set the temporal stage for the entire clause. Alternatively, although less often, the temporal expressions occur in clause-final position, where they are less prominent. This is illustrated in (\ref{Example_5.71}) to (\ref{Example_5.74}) with near contrastive examples. The time expression \textitbf{pagi} ‘morning’ occurs in clause-initial position in (\ref{Example_5.71}) and in clause-final position in (\ref{Example_5.72}). Likewise, \textitbf{malam} ‘night’ occurs in clause-initial position in (\ref{Example_5.73}) and in clause-final position in (\ref{Example_5.74}).


\begin{styleExampleTitle}
Positions within the clause
\end{styleExampleTitle}
\ea
\label{Example_5.71}
\gll {\bluebold{pagi}} {kitong} {datang} {lagi}, {dong} {kasi} {makang}\\ %
 morning  \textsc{1pl}  come  again  \textsc{3pl}  give  eat\\
\glt 
[About a youth retreat:] ‘\bluebold{in the morning}, we came again, they fed (us)’ \textstyleExampleSource{[081025-009a-Cv.0024]}
\z
\ea
\label{Example_5.72}
\gll {kemaring} {sa} {datang} {\bluebold{pagi}}\\ %
 yesterday  \textsc{1sg}  come  morning\\
\glt 
‘yesterday I came \bluebold{in the morning}’ \textstyleExampleSource{[080922-002-Cv.0021]}
\z
\ea
\label{Example_5.73}
\gll {{\ldots}} {\bluebold{malam}} {sa} {berdoa}\\ %
 { }   night  \textsc{1sg}  pray\\
\glt 
‘[when they said (that) he was very very sick,] \bluebold{in the evening} I prayed (for him)’ \textstyleExampleSource{[080923-015-CvEx.0010]}
\z
\ea
\label{Example_5.74}
\gll {pas} {bapa} {berdoa} {\bluebold{malam}} {\bluebold{itu},} {pagi} {de} {meninggal}\\ %
 precisely  father  pray  night  \textsc{d.dist}  morning  \textsc{3sg}  die\\
\glt 
‘(my) father prayed \bluebold{that evening}, and right away in the morning he (the boy) died’ \textstyleExampleSource{[081025-009b-Cv.0039]}
\z


The periods-of-the-day expressions are also used in greetings, as illustrated in (\ref{Example_5.75}) to (\ref{Example_5.78}).


\begin{styleExampleTitle}
Usage in greetings
\end{styleExampleTitle}

\ea
\label{Example_5.75}
\gll {slamat} {\bluebold{pagi}} {pak}\\ %
 be.safe  morning  father\\
\glt 
‘good \bluebold{morning} Sir’ \textstyleExampleSource{[080923-011-Cv.0002]}
\z
\ea
\label{Example_5.76}
\gll  slamat  \bluebold{siang}  ana\\
 be.safe  midday  child\\
\glt 
‘good \bluebold{midday} child’ \textstyleExampleSource{[080922-001a-CvPh.1260]}
\z
\ea
\label{Example_5.77}
\gll  slamat  \bluebold{sore}  smua\\
 be.safe  afternoon  all\\
\glt 
‘good \bluebold{afternoon} (you) all’ \textstyleExampleSource{[081110-002-Cv.0001]}
\z
\ea
\label{Example_5.78}
\gll {slamat} {\bluebold{malam}} {pak} {pendeta}\\ %
 be.safe  night  father  pastor\\
\glt
‘good \bluebold{evening} Mr. Pastor’ \textstyleExampleSource{[080925-003-Cv.0240]}
\z


\subsubsection[Days of the week and months of the year]{Days of the week and months of the year}
\label{Para_5.2.5.3}
The seven days of the week and the twelve months of the year are listed in \tabref{Table_5.8}.


\begin{table}
\caption{Days of the week and months of the year}\label{Table_5.8}

\begin{tabular}{llll}
\lsptoprule
\multicolumn{4}{c}{ Days of the week}\\
\midrule
 \multicolumn{1}{c}{Item} & \multicolumn{1}{c}{Gloss} & \multicolumn{1}{c}{Item} & \multicolumn{1}{c}{Gloss}\\
 \midrule
\textitbf{sening} & ‘Monday’ & \textitbf{jumat} & ‘Friday’\\
\textitbf{slasa} & ‘Tuesday’ & \textitbf{saptu} & ‘Saturday’\\
\textitbf{rabu} & ‘Wednesday’ & \textitbf{minggu} & ‘Sunday’\\
\textitbf{kamis} & ‘Thursday’ &  & \\
\midrule
\multicolumn{4}{c}{ Months of the year}\\
\midrule
 \multicolumn{1}{c}{Item} & \multicolumn{1}{c}{Gloss} & \multicolumn{1}{c}{Item} & \multicolumn{1}{c}{Gloss}\\
 \midrule
\textitbf{januari} & ‘January’ & \textitbf{juli} & ‘July’\\
\textitbf{februari} & ‘February’ & \textitbf{agustus} & ‘August’\\
\textitbf{maret} & ‘March’ & \textitbf{september} & ‘September’\\
\textitbf{april} & ‘April’ & \textitbf{oktober} & ‘October’\\
\textitbf{mey} & ‘May’ & \textitbf{nofember} & ‘November’\\
\textitbf{juni} & ‘Juni’ & \textitbf{desember} & ‘December’\\
\lspbottomrule
\end{tabular}
\end{table}

Typically, the days of the week and the months of the year occur in \textsc{n1n2} \isi{noun} phrases, headed by the common nouns \textitbf{hari} ‘day’ and \textitbf{bulang} ‘month’, respectively (see \tabref{Table_5.6}; see also §\ref{Para_8.2.2}). Examples for the days of the week are given in (\ref{Example_5.79}) and (\ref{Example_5.80}) and for the months of the year in (\ref{Example_5.81}). Occasionally, however, speakers omit \textitbf{hari} ‘day’ or \textitbf{bulang} ‘month’ as with \textitbf{rabu} ‘Wednesday’ in (\ref{Example_5.80}) and with \textitbf{oktober} ‘October’ and \textitbf{januari} ‘January’ in (\ref{Example_5.82}), respectively.


\ea
\label{Example_5.79}
\gll {yo} {bapa,} {\bluebold{hari}} {\bluebold{minggu}} {sa} {datang}\\ %
 yes  father  day  Sunday  \textsc{1sg}  come\\
\glt 
‘yes father, \bluebold{on Sunday} I’ll come’ \textstyleExampleSource{[080922-001a-CvPh.0344]}
\z
\ea
\label{Example_5.80}
\gll \bluebold{hari} \bluebold{slasa} {{itu}} {{\ldots}} {{de}} {pu} {ana} {prempuang} {meninggal}\\ %
 {day}  {Tuesday}  \textsc{d.dist} {}    \textsc{3sg} \textsc{poss}  {child}  woman  die\\
 \gll  jadi  {tong}  {tinggal}  di  {ruma}  {sampe}  \bluebold{rabu}\\
 so  {\textsc{1pl}}  {stay}  at  {house}  {until}  {Wednesday}\\
\glt 
‘that \bluebold{Monday} {\ldots} his daughter died, so we stayed at home until \bluebold{Wednesday}’ \textstyleExampleSource{[080925-003-Cv.0001]}
\z
\ea
\label{Example_5.81}
\gll {ko} {pu} {alpa} {banyak} {di} {\bluebold{bulang}} {\bluebold{oktober}}\\ %
 \textsc{2sg}  \textsc{poss}  be.absent  many  at  month  October\\
\glt 
‘you have lots of (unexcused) absences in \bluebold{October}’ \textstyleExampleSource{[081023-004-Cv.0015]}
\z
\ea
\label{Example_5.82}
\gll {o} {nanti} {\bluebold{oktober}} {e} {\bluebold{januari}} {baru} {kitong} {antar}\\ %
 oh!  very.soon  October  uh  January  and.then  \textsc{1pl}  bring\\
\glt
[About wedding customs:] ‘oh later in \bluebold{October} uh \bluebold{January}, and then we’ll bring (our daughter to your house)’ \textstyleExampleSource{[081110-005-CvPr.0049]}
\z


\subsubsection[Relative time]{Relative time}
\label{Para_5.2.5.4}
Relative time is expressed with the three time-denoting nouns and two phrasal expressions presented in \tabref{Table_5.9}.


\begin{table}
\caption{Relative time}\label{Table_5.9}
{\setlength\tabcolsep{1pt}
\begin{tabular}{llllll}
\lsptoprule
\multicolumn{5}{c}{Item} & \multicolumn{1}{c}{Gloss}\\
\midrule
\multicolumn{3}{l}{\textitbf{kemaring}} &  \multicolumn{2}{l}{\textitbf{dulu}} & ‘the day before yesterday’\\
\multicolumn{3}{l}{yesterday} &\multicolumn{2}{l}{ be.prior} \\
\\
\multicolumn{5}{l}{\textitbf{kemaring}}   & ‘yesterday, sometime ago’\\
\\
\textitbf{hari} & \textitbf{ini} & & & & ‘today’\\
day & \multicolumn{5}{l}{\textsc{d.prox}}\\
\\
\multicolumn{2}{l}{\textitbf{besok}} & & & &  ‘tomorrow, sometime in the future’\\
\multicolumn{2}{l}{\textitbf{lusa}} & & &  & ‘the day after tomorrow’\\
\lspbottomrule
\end{tabular}
}
\end{table}

Within the clause, the relative-time denoting expressions typically occur in clause-initial position. Here they set the temporal stage for the entire clause, similar to the nouns denoting periods of the day, discussed in §\ref{Para_5.2.5.2}. This is illustrated with the examples in (\ref{Example_5.83}) to (\ref{Example_5.85}). Alternatively, but less often, the relative-time expressions directly precede the predicate where they are less prominent, as shown in (\ref{Example_5.86}). The contrast in meaning conveyed by the different positions within the clause is illustrated with \textitbf{besok} ‘tomorrow’ in the near contrastive examples in (\ref{Example_5.85}) and (\ref{Example_5.86}). By fronting \textitbf{besok} ‘tomorrow’ in (\ref{Example_5.85}), the speaker accentuates the temporal setting of the entire clause. This is not the case in (\ref{Example_5.86}), where \textitbf{besok} ‘tomorrow’ directly precedes the predicate, where it is less salient.



The examples in (\ref{Example_5.83}) and (\ref{Example_5.85}) also illustrate that the temporal scope of \textitbf{kemaring} ‘yesterday’ and \textitbf{besok} ‘tomorrow’ is larger than the preceding or following 24-hour period, respectively. Generally speaking \textitbf{kemaring} ‘yesterday’ denotes a past point in time such as \textitbf{kemaring} ‘some time ago’ in (\ref{Example_5.83}). Along similar lines, \textitbf{besok} ‘tomorrow’ refers to a future point in time which in (\ref{Example_5.85}) is \textitbf{besok} ‘next year’.


\begin{styleExampleTitle}
Positions within the clause
\end{styleExampleTitle}

\ea
\label{Example_5.83}
\gll {{\bluebold{kemaring}}} {{\bluebold{dulu}}} {{sa}} {deng} {nene}, {nene} {jam} {dua}\\ %
 {yesterday}  {be.prior}  {\textsc{1sg}}  with  grandmother  grandmother  hour  two\\
 \gll  malam  {datang}  {deng}  {menangis}\\
 night  {come}  {with}  {cry}\\
\glt 
‘\bluebold{the day before yesterday} I and grandmother, at two in the morning grandmother came crying {\ldots}’ \textstyleExampleSource{[081014-008-CvNP.0001]}
\z
\ea
\label{Example_5.84}
\gll  yo,  \bluebold{hari}  \bluebold{ini}  suda  ko  su  skola\\
 yes  day  \textsc{d.prox}  already  \textsc{2sg}  already  go.to.school\\
\glt 
‘yes, \bluebold{today} you already went to school’ \textstyleExampleSource{[080917-003a-CvEx.0006]}
\z
\ea
\label{Example_5.85}
\gll {kalo} {{\bluebold{besok}}} {de} {{itu}} {hadir} {ke} {sana} {tu}\\ %
 if  {tomorrow}  \textsc{3sg}  {\textsc{d.dist}}  attend  to  \textsc{l.dist}  \textsc{d.dist}\\
 \gll  {biking}  de  {sperti}  {bos}\\
 {make}  \textsc{3sg}  {similar.to}  {boss}\\
\glt 
[About an event planned for the next year:] ‘if \bluebold{next year} he (the mayor), what’s-its-name, (comes and) attends (the retreat) over there, treat him like a boss’ \textstyleExampleSource{[081025-009a-Cv.0172]}
\z
\ea
\label{Example_5.86}
\gll {bapa} {nanti} {\bluebold{besok}} {hadir} {di} {ini} {retrit} {pemuda}\\ %
 father  very.soon  tomorrow  attend  at  \textsc{d.prox}  retreat  youth\\
\glt 
[About an event planned for the next year:] ‘you (‘father’) (have to) attend, what’s-its-name, the youth retreat \bluebold{next year}’ \textstyleExampleSource{[081025-009a-Cv.0175]}
\z


In addition, the corpus includes a small number of utterances in which the nouns designating relative-time occur as subjects in nonverbal clauses. This is illustrated with \textitbf{besok} ‘tomorrow’ and \textitbf{lusa} ‘the day after tomorrow’ in (\ref{Example_5.87}).


\begin{styleExampleTitle}
Subject-function in nonverbal clauses
\end{styleExampleTitle}

\ea
\label{Example_5.87}
\gll {\bluebold{besok}} {hari} {kamis} {\bluebold{lusa}} {hari} {jumat} {baru} {{\ldots}}\\ %
 tomorrow  day  Thursday  day.after.tomorrow  day  Friday  and.then  \\
\glt 
‘\bluebold{tomorrow} is Thursday, \bluebold{the day after tomorrow} is Friday and then {\ldots}’ \textstyleExampleSource{[080917-003a-CvEx.0006]}
\z


Like other nouns, relative-time denoting nouns also have adnominal uses as shown in (\ref{Example_5.88}) and (\ref{Example_5.89}). In their adnominal uses, they occur in posthead position and have restrictive function. That is, they specify whether the period or point in time encoded by the head nominal is situated in the future or in the past, as in \textitbf{hari minggu besok} ‘next Sunday’ in (\ref{Example_5.88}) or \textitbf{taung kemaring} ‘a few years back’ in (\ref{Example_5.89}).


\begin{styleExampleTitle}
Adnominal uses
\end{styleExampleTitle}

\ea
\label{Example_5.88}
\gll {yo} {memang} {\bluebold{hari}} {\bluebold{minggu}} {\bluebold{besok}} {sa} {datang}\\ %
 yes  indeed  day  Sunday  tomorrow  \textsc{1sg}  come\\
\glt 
‘yes, indeed, \bluebold{next Sunday} I’ll come’ \textstyleExampleSource{[080922-001a-CvPh.0346]}
\z
\ea
\label{Example_5.89}
\gll {{banyak}} {{mati}} {di} {{lautang}} {kas} {tenggelam} {sampe} {\bluebold{taung}} {\bluebold{kemaring}}\\ %
 {many}  {die}  at  {ocean}  give  sink  until  year  yesterday\\
 \gll  taung  {{\ldots}}  dua  {ribu}  {dua}\\
 year  {}  two  {thousand}  {two}\\
\glt 
[About people in a container who died in the ocean:] ‘many died in the (open) ocean, (the murderers) sank (the containers), (many died in the open ocean) until \bluebold{a few years back}, (until) the year 2002’ \textstyleExampleSource{[081029-002-Cv.0025]}
\z


Relative-time expressions also occur as complements in prepositional phrases as, for instance, in \textitbf{sampe besok} ‘until the next day’ (literally ‘until tomorrow’) in (\ref{Example_5.90}). This example also illustrates that \textitbf{besok} ‘tomorrow’ denotes \isi{relative time}. As the events described here happened in the past, \textitbf{besok} ‘tomorrow’ refers to a future point in time relative to the narrated events. Hence, \textitbf{besok} translates as ‘the next day’. (Prepositions encoding time are discussed in more detail in §\ref{Para_10.1}.)


\begin{styleExampleTitle}
Complements in prepositional phrases
\end{styleExampleTitle}
\ea
\label{Example_5.90}
\gll {sa} {minum} {lagi} {trus} {sa} {tinggal} {\bluebold{sampe}} {\bluebold{besok}}\\ %
 \textsc{1sg}  drink  again  next  \textsc{1sg}  stay  until  tomorrow\\
\glt
[About recovering from an accident:] ‘I took (medicine) again, then I stayed \bluebold{until the next day}’ (Lit. ‘\bluebold{until tomorrow}’) \textstyleExampleSource{[081015-005-NP.0042-0043]}
\z


\subsection{Classifying nouns}
\label{Para_5.2.6}
Papuan Malay has a very reduced inventory of classifying nouns. Attested is only one, namely the common \isi{noun} \textitbf{ekor} ‘tail’ which is used to count animals. In this function, it always follows a posthead \isi{numeral}, as shown in (\ref{Example_5.91}).


\begin{styleExampleTitle}
Enumeration of animals
\end{styleExampleTitle}

\ea
\label{Example_5.91}
\gll {dong} {{dua}} {{dapat}} {{ikang}} {{ini}} {{\bluebold{tiga}}} {\bluebold{ekor}}\\ %
 \textsc{3pl}  {two}  {get}  {fish}  {\textsc{d.prox}}  {three}  tail\\
 \gll  {dapat}  {\bluebold{ikang}}  {\bluebold{tiga}}  {\bluebold{ekor}}  dong  {dua}  {{\ldots}}\\
 {get}  {fish}  {three}  {tail}  \textsc{3pl}  {two}  {}\\
\glt 
‘the two of them get these fish, \bluebold{three (of them)}, having gotten \bluebold{three fish}, the two of them {\ldots}’ (Lit. ‘\bluebold{three tails}’) \textstyleExampleSource{[081109-011-JR.0003]}
\z


As a classifying \isi{noun}, \textitbf{ekor} ‘tail’ does not refer to the entities themselves being counted but rather to their form, as is rather common in Malay and other \ili{Austronesian} varieties; see for instance \ili{Ambon Malay} \citep[153]{vanMinde.1997}, \ili{Ternate Malay} \citep[62]{Litamahuputty.1994}, Tetun \citep{vanKlinken.1999}, or Standard Malay \citep[321–323]{Mintz.2002}.



Enumeration of people and objects, by contrast, is done without a classifier as illustrated in (\ref{Example_5.92}) and (\ref{Example_5.93}), respectively.


\begin{styleExampleTitle}
Enumeration of people and objects
\end{styleExampleTitle}

\ea
\label{Example_5.92}
\gll {jadi} {saya} {\bluebold{empat}} {\bluebold{ana}}\\ %
 so  \textsc{1sg}  four  child\\
\glt 
‘so I (have) \bluebold{four children}’ \textstyleExampleSource{[081006-024-CvEx.0002]}
\z
\ea
\label{Example_5.93}
\gll {{orang}} {{Sarmi}} {{harus}} {{siap}} {{untuk}} {orang} {Sorong}\\ %
 {person}  {Sarmi}  {have.to}  {provide}  {for}  person  Sorong\\
 \gll  \bluebold{spulu}  {\bluebold{kaing}}  {\bluebold{itu}}  {\bluebold{kaing}}  {\bluebold{adat}}  {\bluebold{itu}}\\
 ten  {cloth}  {\textsc{d.dist}}  {cloth}  {tradition}  {\textsc{d.dist}}\\
\glt
[About bride-prices:] ‘a \ili{Sarmi} person has to provide a Sorong person with \bluebold{those ten cloths}, \bluebold{those traditional cloths}’ \textstyleExampleSource{[081006-029-CvEx.0012]}
\z


\subsection{Kinship terms}
\label{Para_5.2.7}
This section presents the most common Papuan Malay terms for consanguineal and \isi{affinal kin}. An initial investigation of the kinship system indicates that Papuan Malay uses a combination of Iroquois and Hawaiian terminologies and makes a relative age discrimination.



Before presenting the Papuan Malay kinship terms, \tabref{Table_5.10} lists the standard symbols used to abbreviate basic terms.


\begin{table}
\caption{Symbols for kinship terms}\label{Table_5.10}


\begin{tabular}{llllll}
\lsptoprule
 Terms & Symbols & Terms & Symbols & Terms &  Symbols\\
\midrule
father & F & brother & B & husband & H\\
mother & M & sister & Z & wife & W\\
parent & P & sibling & Sb & spouse & Sp\\
son & S & older & o &  & \\
daughter & D & younger & y &  & \\
child & C &  &  &  & \\
\lspbottomrule
\end{tabular}
\end{table}

More complex kinship terms are expressed by chains of these abbreviations, such as FZ for ‘father’s sister’ or MF for ‘mother’s father’.


\subsubsection[Consanguineal kin]{Consanguineal kin}
\label{Para_5.2.7.1}
The kinship system is Iroquois, in that Papuan Malay makes a distinction in the first ascending generation between same-sex and cross-sex parents’ siblings in a bifurcate merging pattern, as demonstrated in \tabref{Table_5.11}. Contrasting with typical Iroquois systems, however, the cross-parallel distinction only applies to parents’ younger siblings. That is, only parents’ same-sexed younger siblings are considered as consanguines: \textitbf{bapa-ade} ‘uncle’ (literally ‘younger father’) and \textitbf{mama-ade} ‘aunt’ (literally ‘younger mother’). Parents’ opposite-sexed younger siblings are called \textitbf{om} ‘uncle’ and \textitbf{tanta} ‘aunt’; both terms are loanwords from \ili{Dutch}. By contrast, Papuan Malay does not distinguish between parents’ older siblings of opposite sex. That is, all parents’ older siblings are considered as consanguines regardless of their sex: \textitbf{bapa-tua} ‘uncle’ (literally ‘old father’) and \textitbf{mama-tua} ‘aunt’ (literally ‘old mother’). The six consanguineal terms also extend to \isi{affinal kin}, as discussed in §\ref{Para_5.2.7.2}.



With respect to other generations, the kinship system is Hawaiian, in that it extends bilaterally, without making distinctions between lineal and collateral consanguines, or between cross and parallel consanguines. Consequently, Papuan Malay does not distinguish between siblings and cousins, as shown in \tabref{Table_5.11}. That is, children of parents’ siblings are also classified as siblings. In addition, the system makes a relative age discrimination. Older siblings and children of parents’ older siblings are called \textitbf{kaka} ‘older sibling’ while younger siblings and children of parents’ younger siblings are called \textitbf{ade} ‘younger sibling’. The same relative age discrimination applies to cousins in the second degree of collaterality: their relative ages are determined by the ages of the linking grandparents. With the exception of the reference term \textitbf{orang-tua} ‘parent’, speakers use the consanguineal terms, listed in \tabref{Table_5.11}, both for reference and for address.


\begin{table}
\caption{Papuan Malay kinship terms: Consanguineal kin}\label{Table_5.11}

\begin{tabular}{llll}
\lsptoprule
  \multicolumn{1}{c}{Item} &  \multicolumn{1}{c}{Gloss} &  \multicolumn{1}{c}{Symbol} &   \multicolumn{1}{c}{Relation}\\
\midrule
\textitbf{bapa} & ‘father’ & F & father\\
\textitbf{mama} & ‘mother’ & M & mother\\
\textitbf{orang-tua} & ‘parent’ & P & parent\\
\textitbf{ana} & ‘child’ & C & child\\
\midrule
\textitbf{kaka} & ‘older sibling’ & oSb & older sibling\\
&  & PoSbC & parent’s older sibling’s child\\
\textitbf{ade} & ‘younger sibling’ & ySb & younger sibling\\
&  & PySbC & parent’s younger sibling’s child\\
\textitbf{bapa-tua} & ‘uncle’ & PoB & parent’s older brother\\
\textitbf{bapa-ade} & ‘uncle’ & FyB & father’s younger brother\\
\midrule
\textitbf{om} & ‘uncle’ & MyB & mother’s younger brother\\
\textitbf{mama-tua} & ‘aunt’ & PoZ & parent’s older sister\\
\textitbf{mama-ade} & ‘aunt’ & MyZ & mother’s younger sister\\
\textitbf{tanta} & ‘aunt’ & FyZ & father’s younger sister\\
\midrule
\textitbf{tete} & ‘grandfather’ & PF & parent’s father\\
&  & PPB & parent’s parent’s brother\\
\textitbf{nene} & ‘grandmother’ & PM & parent’s mother\\
&  & PPZ & parent’s parent’s sister\\
\textitbf{cucu} & ‘grandchild’ & CC & child’s child\\
\lspbottomrule
\end{tabular}
\end{table}

To signal the gender of a sibling or child, the kinship terms \textitbf{kaka} ‘older sibling’, \textitbf{ade} ‘younger sibling’, and \textitbf{ana} ‘child’ are modified with the common nouns \textitbf{laki{\Tilde}laki} ‘man’ or \textitbf{prempuang} ‘woman’, giving \textitbf{kaka laki{\Tilde}laki} ‘older brother’, \textitbf{ade prempuang} ‘younger sister’, or \textitbf{ana laki{\Tilde}laki} ‘son’.


\subsubsection[Affinal kin]{Affinal kin}
\largerpage
\label{Para_5.2.7.2}
The Papuan Malay affinal terms, listed in \tabref{Table_5.12}, include two terms for spouse, that is, \textitbf{paytua} ‘husband’ and \textitbf{maytua} ‘wife’, and two terms for in-laws, namely \textitbf{mantu} ‘(parent/child) in-law’ and \textitbf{ipar} ‘sibling in-law’. Speakers employ these terms for both reference and address.



Papuan Malay distinguishes between in-laws belonging to different generations and those belonging to the same generation, as illustrated in \tabref{Table_5.12}.



The expression for in-laws belonging to the first ascending or descending generation is the self-recipro\-cal term \textitbf{mantu} ‘(parent/child) in-law’. This term, however, is unattested on its own. It is always modified with the common nouns \textitbf{bapa} ‘father’, \textitbf{mama} ‘mother’, or \textitbf{ana} ‘child’ to specify the affinal relationship, giving \textitbf{bapa mantu} ‘father in-law’, \textitbf{mama mantu} ‘mother in-law’, or \textitbf{ana mantu} ‘child in-law’.




The term for same-generation in-laws is \textitbf{ipar} ‘sibling in-law’. This self-reciproc\-al term extends to spouses’ siblings and those siblings’ spouses, as well as to children’s spouses’ parents (co-parents-in-law). Again, a relative age discrimination is made similar to that for siblings: \textitbf{kaka ipar} ‘older sibling in-law’ and \textitbf{ade ipar} ‘younger sibling in-law’.


\begin{table}
\caption{Papuan Malay kinship terms: Affinal kin}\label{Table_5.12}

\begin{tabular}{llll}
\lsptoprule
 \multicolumn{1}{c}{Item} &  \multicolumn{1}{c}{Gloss} &  \multicolumn{1}{c}{Symbol} &   \multicolumn{1}{c}{Relation}\\
\midrule
\textitbf{paytua} & ‘husband’ & H & husband\\
\textitbf{maytua} & ‘wife’ & W & wife\\
\textitbf{mantu} & ‘(parent/child) in-law’ & SpP & spouse’s parents\\
&  & CSp & child’s spouse\\
\textitbf{ipar} & ‘sibling in-law’ & SbSp & sibling’s spouse\\
&  & SpSb & spouse’s sibling\\
&  & SpSbSp & spouse’s sibling’s spouse\\
&  & CSpP & child’s spouse’s parents\\
\lspbottomrule
\end{tabular}
\end{table}

The six consanguineal terms that distinguish between same-sex and cross-sex parents’ siblings in the first ascending generation, mentioned in §\ref{Para_5.2.7.1}, also extend to \isi{affinal kin}, as shown in \tabref{Table_5.13}.


\begin{table}
\caption{Papuan Malay consanguineal terms extending to affinal kin}\label{Table_5.13}

\begin{tabular}{llll}
\lsptoprule
\multicolumn{1}{c}{Item} &  \multicolumn{1}{c}{Gloss} &  \multicolumn{1}{c}{Symbol} &   \multicolumn{1}{c}{Relation}\\
\midrule
\textitbf{bapa-tua} & ‘uncle’ & PoZH & parent’s older sister’s husband\\
\textitbf{bapa-ade} & ‘uncle’ & MyZH & mother’s younger sister’s husband\\
\textitbf{om} & ‘uncle’ & FyZH & father’s younger sister’s husband\\
\textitbf{mama-tua} & ‘aunt’ & PoBW & parent’s older brother’s wife\\
\textitbf{mama-ade} & ‘aunt’ & FyBW & father’s younger brother’s wife\\
\textitbf{tanta} & ‘aunt’ & MyBW & mother’s younger brother’s wife\\
\lspbottomrule
\end{tabular}
\end{table}
\section{Verbs}
\label{Para_5.3}
Papuan Malay has a large open class of verbs which express actions, events, and processes, as well as states or more time-stable properties. They have the following defining syntactic and functional properties:


\begin{enumerate}
\item 
Valency: each \isi{verb} takes a specific number of arguments (§\ref{Para_5.3.1}).
\item 
Predicative function is predominant; besides, verbs also have attributive uses in \isi{noun} phrases (§\ref{Para_5.3.2}).
\item 
Modification with adverbs, including \isi{intensification} and \isi{grading} (§\ref{Para_5.3.4} and §\ref{Para_5.3.5}).
\item 
Negation only with \textitbf{tida} ‘\textsc{neg}’ or \textitbf{tra} ‘\textsc{neg}’ (§\ref{Para_5.3.6}).\footnote{As for the occurrence of \textitbf{bukang} ‘\textsc{neg}’ in verbal clauses, see Footnote \ref{Footnote_5.155} in §\ref{Para_5.3.6} (p. \pageref{Footnote_5.155}).}
\item 
Occurrence in \isi{causative} and in reciprocal constructions (§\ref{Para_5.3.7} and §\ref{Para_5.3.8}).
\end{enumerate}

Morphological properties play only a minor role in defining verbs as a distinct word class, due to the lack of inflectional \isi{morphology} and the limited role of derivational processes. The latter include \isi{reduplication} (for details see §\ref{Para_4.2.2}), and, to a limited extent, \isi{affixation} with prefix \textscItal{ter-} or suffix -\textitbf{ang} (§\ref{Para_5.3.9}; see also §\ref{Para_3.1}).



Verbs are divided into three classes on the basis of their \isi{valency} and their tendency to function predicatively, namely \isi{trivalent}, \isi{bivalent}, and \isi{monovalent} verbs. In turn, \isi{monovalent} verbs are further divided into dynamic and stative verbs. That is, Papuan Malay does not have a distinct class of adjectives. Instead, \isi{monovalent} stative verbs encode, what \citet[4]{Dixon.2004b} calls “the four core semantic types” of dimension, age, value, and color which are “typically associated with the word class adjective”. The two criteria of \isi{valency} and prevalent \isi{predicative function} also account for the other properties of verbs, listed above and discussed in more detail in the following sections.



Verbs are distinct from nouns (§\ref{Para_5.2}) and adverbs (§\ref{Para_5.4}) in terms of the following distributional properties:

\begin{enumerate}
\item 
Contrasting with nouns, verbs (a) have \isi{valency},\footnote{It is acknowledged that some authors maintain that nouns have \isi{valency}. \citet{vanValin.1997}, for instance, discuss the “layered structure of adpositional and \isi{noun} phrases” (\citeyear*[52–67]{vanValin.1997}) and the “semantic representation of nouns and \isi{noun} phrases” (\citeyear[184–195]{vanValin.1997}), and \citet[89–92]{vanValin.2001} examines “[t]ypes of dependencies”. See also Croft’s (\citeyear*[62–79]{Croft.1991}) discussion on “structural markedness and the semantic prototypes”, as well as \citet{Allerton.2006}, \citet{Sommerfeldt.1983}, and \cite{vanDurme.1997}.} and (b) are negated with \textitbf{tida}/\textitbf{tra} ‘\textsc{neg}’. In addition (c) verbs, except for \isi{monovalent} stative ones, occur as predicates in reciprocal constructions, and (d) \isi{monovalent} stative and \isi{bivalent} verbs occur as predicates in comparative constructions.
\item 
Unlike adverbs, verbs (a) are used predicatively, and (b) can modify nouns.

\end{enumerate}

The following sections explore the characteristics and properties of verbs in more detail. As for their syntactic properties the following topics are discussed: \isi{valency} in §\ref{Para_5.3.1}, predicative and attributive functions in §\ref{Para_5.3.2}, adverbial \isi{modification} in §\ref{Para_5.3.3}, \isi{intensification} in §\ref{Para_5.3.4}, \isi{grading} in §\ref{Para_5.3.5}, \isi{negation} in §\ref{Para_5.3.6}, occurrences in \isi{causative} constructions in §\ref{Para_5.3.7}, and uses in reciprocal constructions in §\ref{Para_5.3.8}. Finally, the morphological properties of verbs are briefly examined in §\ref{Para_5.3.9}. In each section, dynamic verbs are discussed first, and stative verbs second. Dynamic verbs, in turn, are described in order from those with three arguments to those with one argument. Each section also discusses the type and token frequencies in the corpus for the respective properties and summarizes these frequencies in a table. These tables form the basis for the summary in §\ref{Para_5.4.11}.


\subsection{Valency}
\label{Para_5.3.1}
Valency is defined as a “weighting or quantification of verbs in terms of the number of dependents (or arguments or valents) they take” \citep[5185]{Asher.1994}. Papuan Malay verbs are classified into three classes on the basis of their \isi{valency}, namely verbs with one, two, or three core arguments. Examples are given in \tabref{Table_5.14} and \tabref{Table_5.14a}: verbs that have two or three arguments are listed first, followed by verbs with one argument. Monovalent verbs are further distinguished according to their semantics into dynamic and stative verbs, and other properties, discussed in the following sections.


\begin{table}
\caption{Tri- and \isi{bivalent} verbs}\label{Table_5.14}

\begin{tabular}{llll}
\lsptoprule

\multicolumn{4}{c}{ Trivalent verbs}\\
\midrule
\textitbf{ambil} & ‘fetch’ & \textitbf{kasi} & ‘give’\\
\textitbf{bawa} & ‘bring’ & \textitbf{kirim} & ‘send’\\
\textitbf{bli} & ‘buy’ & \textitbf{minta} & ‘request’\\
\textitbf{ceritra} & ‘tell’ &  & \\
\midrule
\multicolumn{4}{c}{ Bivalent verbs}\\
\midrule
\textitbf{antar} & ‘bring’ & \textitbf{kubur} & ‘bury’\\
\textitbf{bunu} & ‘kill’ & \textitbf{lawang} & ‘oppose’\\
\textitbf{cabut} & ‘pull out’ & \textitbf{maki} & ‘abuse (verbally)’\\
\textitbf{dorong} & ‘push’ & \textitbf{mara} & ‘feel angry (about)’\\
\textitbf{ejek} & ‘mock’ & \textitbf{naik} & ‘ascend’\\
\textitbf{ganas} & ‘feel furious (about)’ & \textitbf{pake} & ‘use’\\
\textitbf{ganggu} & ‘disturb’ & \textitbf{rabik} & ‘tear’\\
\textitbf{hela} & ‘haul’ & \textitbf{simpang} & ‘store’\\
\textitbf{ikut} & ‘follow’ & \textitbf{tarik} & ‘pull’\\
\textitbf{jual} & ‘sell’ & \textitbf{usir} & ‘chase away’\\
\lspbottomrule
\end{tabular}
\end{table}
\begin{table}
\caption{Monovalent verbs}\label{Table_5.14a}

\begin{tabular}{llll}
\lsptoprule
\multicolumn{4}{c}{ Monovalent dynamic verbs}\\
\midrule
\textitbf{bernang} & ‘swim’ & \textitbf{kembali} & ‘return’\\
\textitbf{bocor} & ‘leak’ & \textitbf{lari} & ‘run’\\
\textitbf{datang} & ‘come’ & \textitbf{maju} & ‘advance’\\
\textitbf{duduk} & ‘sit’ & \textitbf{mandi} & ‘bathe’\\
\textitbf{gementar} & ‘tremble’ & \textitbf{oleng} & ‘shake’\\
\textitbf{guling} & ‘roll over’ & \textitbf{pergi} & ‘go’\\
\textitbf{hidup} & ‘live’ & \textitbf{sandar} & ‘lean’\\
\textitbf{hosa} & ‘pant’ & \textitbf{sante} & ‘relax’\\
\textitbf{jalang} & ‘walk’ & \textitbf{terbang} & ‘fly’\\
\textitbf{jatu} & ‘fall’ & \textitbf{tinggal} & ‘stay’\\
\midrule
\multicolumn{4}{c}{ Monovalent stative verbs}\\
\midrule
\textitbf{abu} & ‘be dusty’ & \textitbf{muda} & ‘be young’\\
\textitbf{bagus} & ‘be good’ & \textitbf{nyamang} & ‘be comfortable’\\
\textitbf{cantik} & ‘be beautiful’ & \textitbf{panas} & ‘be hot’\\
\textitbf{dinging} & ‘be cold’ & \textitbf{puti} & ‘be white’\\
\textitbf{enak} & ‘be pleasant’ & \textitbf{renda} & ‘be low’\\
\textitbf{gila} & ‘be crazy’ & \textitbf{sakit} & ‘be sick’\\
\textitbf{hijow} & ‘be green’ & \textitbf{swak} & ‘be exhausted’\\
\textitbf{jahat} & ‘be bad’ & \textitbf{tinggi} & ‘be tall’\\
\textitbf{kecil} & ‘be small’ & \textitbf{tua} & ‘be old’\\
\textitbf{lema} & ‘be weak’ & \textitbf{waras} & ‘be sane’\\
\lspbottomrule
\end{tabular}
\end{table}

Trivalent verbs have three core arguments, that is, a subject and two grammatical objects. This is illustrated with \textitbf{kasi} ‘give’ in (\ref{Example_5.94}). It is important to note, however, that the attested \isi{trivalent} verbs allow but do not require three syntactic arguments. (For details, see §\ref{Para_11.1.3}.)


\begin{styleExampleTitle}
Trivalent verbs with three core arguments
\end{styleExampleTitle}

\ea
\label{Example_5.94}
\gll {dia} {\bluebold{kasi}} {kitong} {daging}\\ %
 \textsc{3sg}  give  \textsc{1pl}  meat\\
\glt 
‘he \bluebold{gave} us (fish) meat’ \textstyleExampleSource{[080919-004-NP.0061]}
\z


Bivalent verbs have two core arguments, a subject and one grammatical object. This is shown with \textitbf{pukul} ‘hit’ in (\ref{Example_5.95}) and \textitbf{mara} ‘feel angry (about)’ in (\ref{Example_5.96}). Bivalent verbs also allow, but do not require two syntactic arguments. (For details see §\ref{Para_11.1.2}.)


\begin{styleExampleTitle}
Bivalent verbs with two core arguments
\end{styleExampleTitle}

\ea
\label{Example_5.95}
\gll {bapa} {de} {\bluebold{pukul}} {sa} {deng} {pisow}\\ %
 father  \textsc{3sg}  hit  \textsc{1sg}  with  knife\\
\glt 
‘(my) husband \bluebold{hit }me with a knife’ \textstyleExampleSource{[081011-023-Cv.0167]}
\z
\ea
\label{Example_5.96}
\gll {{\ldots}} {jadi} {sa} {\bluebold{mara}} {dia}\\ %
  { }  so  \textsc{1sg}  feel.angry(.about)  \textsc{3sg}\\
\glt 
‘[he doesn’t report to me in a good way,] so I \bluebold{feel angry about} him’ \textstyleExampleSource{[081011-020-Cv.0107]}
\z


Monovalent verbs have only one core argument. They are further divided into dynamic and stative verbs. Dynamic verbs such as \textitbf{lari} ‘run’ in (\ref{Example_5.97}) denote actions involving one participant, while stative verbs, such as \textitbf{besar} ‘be big’ or \textitbf{kecil} ‘be small’ in (\ref{Example_5.98}), express states or more time-stable properties.


\begin{styleExampleTitle}
Monovalent verbs with one core argument
\end{styleExampleTitle}
\ea
\label{Example_5.97}
\gll  Nofita  de  \bluebold{lari}  dari  saya\\
 Nofita  \textsc{3sg}  run  from  \textsc{1sg}\\
\glt 
‘Nofita \bluebold{ran (away)} from me’ \textstyleExampleSource{[081025-006-Cv.0322]}
\z
\ea
\label{Example_5.98}
\gll {kepala} {ni} {\bluebold{besar}} {baru} {badang} {ni} {\bluebold{kecil}}\\ %
 head  \textsc{d.prox}  be.big  and.then  body  \textsc{d.prox}  be.small\\
\glt 
‘(his) head here \bluebold{is big} but (his) body here \bluebold{is small}’ \textstyleExampleSource{[081025-006-Cv.0278]}
\z


In the corpus, the class of \isi{trivalent} verbs is the smallest one with seven, as shown in \tabref{Table_5.15}. A small majority of attested verbs are \isi{bivalent} with 535 entries (52\%), while 490 verbs are \isi{monovalent} (48\%). Most of the \isi{monovalent} verbs are stative (351/490 – 72\%), while 139 verbs are dynamic (28\%).


\begin{table}
\caption{Verb type frequencies}\label{Table_5.15}

\begin{tabular}{lrr}
\lsptoprule
 & \multicolumn{2}{c}{ Frequencies}\\
\midrule
Verb class &  \# &  \%\\
\midrule
\textsc{v.tri} &  7 &  0.7\%\\
\textsc{v.bi} &  535 &  51.8\%\\
\textsc{v.mo} &  490 &  47.5\%\\
\hspace{2mm}\textsc{v.mo}(\textsc{dy}) &  (139) &  (28.4\%)\\
\hspace{2mm}\textsc{v.mo}(\textsc{st}) &  (351) &  (71.6\%)\\
\midrule
Total &  1,032 &  100\%\\
\lspbottomrule
\end{tabular}
\end{table}

In addition, the corpus contains 43 derived \isi{monovalent} verbs prefixed with \textscItal{ter-} that denote accidental or unintentional actions or events (167 tokens). These lexemes are examined in detail in §\ref{Para_3.1.2}, and briefly reviewed in §\ref{Para_5.3.9}; therefore, they are not further discussed in this section.


\subsection{Predicative and attributive functions}
\label{Para_5.3.2}
Verbs can function predicatively as well as attributively. The identified \isi{verb} classes display clear distributional preferences, however. Dynamic verbs usually function predicatively, and less frequently attributively. Monovalent stative verbs, by contrast, typically occur as adnominal modifiers in \isi{noun} phrases, although they also have \isi{predicative function}.



In their predicative uses, verbs act “as ‘comment’ on a given \isi{noun} as ‘topic’”, using Dixon’s (\citeyear*[31]{Dixon.1994}) terminology. This typical function of dynamic verbs is demonstrated with \isi{bivalent} \textitbf{bunu} ‘kill’ in (\ref{Example_5.99}). The predicative use of \isi{monovalent} stative verbs is illustrated with \textitbf{tinggi} ‘be high’ in (\ref{Example_5.100}). In the corpus, all dynamic verbs have \isi{predicative function}, while only 40\% of the stative verbs (139/351) are used predicatively.


\begin{styleExampleTitle}
Predicative uses
\end{styleExampleTitle}

\ea
\label{Example_5.99}
\gll {bapa} {Iskia} {dong} {\bluebold{bunu}} {\bluebold{babi}}\\ %
 father  Iskia  \textsc{3pl}  kill  pig\\
\glt 
‘father Iskia and his companions \bluebold{killed a pig}’ \textstyleExampleSource{[080917-008-NP.0120]}
\z
\ea
\label{Example_5.100}
\gll {glombang} {itu} {\bluebold{tinggi}}\\ %
 wave  \textsc{d.dist}  be.high\\
\glt 
‘that wave \bluebold{was high}’ \textstyleExampleSource{[080923-015-CvEx.0016]}
\z


In their \isi{attributive function} within \isi{noun} phrases, the modifying verbs serve to specify or restrict “the reference of the \isi{noun}”, in Dixon’s (\citeyear*[31]{Dixon.1994}) terminology. Cross-linguistically, this is achieved in one of two ways, as \citet{Dixon.1994} points out. One option is verb-via-\isi{juxtaposition} \isi{modification}; that is, the modifying \isi{verb} is directly juxtaposed to a \isi{noun} in a \isi{noun} phrase. The second option is “verb-via-relative-clause \isi{modification}” {\citep[19]{Dixon.2004b}; that is, \isi{modification} is achieved by means of a relative clause. }Papuan Malay also makes use of these two options, as illustrated in (\ref{Example_5.101}) to (\ref{Example_5.105})



The first option of verb-via-\isi{juxtaposition} \isi{modification} is illustrated in (\ref{Example_5.101}). The examples show that all \isi{verb} types can occur in \isi{noun} phrases as adnominal modifiers in posthead position, both with agentive and non-agentive head nominals (the examples in (\ref{Example_5.101a}) and (\ref{Example_5.101b}) are elicited).


\begin{styleExampleTitle}
Attributive uses: Verb-via-\isi{juxtaposition} \isi{modification}
\end{styleExampleTitle}



\ea
\label{Example_5.101}
 \hspace{0.5 cm} Trivalent verbs\\
\ea
\label{Example_5.101a}

\gll {sifat} {kasi}\\
   spirit  give      \\
\glt    {‘disposition of giving’}\\
\vspace{5pt}
\ex
\label{Example_5.101b}
\gll {tukang} {bli}\\ %
craftsman  buy\\
\glt     {‘one who likes to buy’}\\

\vspace{10 pt}{Bivalent verbs}\vspace{0pt}\\
\ex
\label{Example_5.101c}

\gll   {ana}  {angkat}\\
   child  lift    \\
\glt  {‘adopted child’}   \\
\vspace{5pt}
\ex
\label{Example_5.101d}
\gll   {tukang}  {minum}\\
  craftsman  drink\\
\glt   {‘drunkard’}\\


\newpage 

{Monovalent dynamic verbs}\vspace{0 pt}\\
\ex
\label{Example_5.101e}

\gll   {sabung}  {mandi} \\
   soap  bathe      \\
\glt   {‘bathing soap’}   \\
\vspace{5pt}
\ex
\label{Example_5.101f}
\gll {tukang}  {jalang}\\
craftsman  walk\\
\glt    {‘one who likes to walk around’}\\

\vspace{10 pt} {Monovalent stative verbs} \vspace{0 pt}\\
\ex
\label{Example_5.101g}

\gll {bua}  {mera} \\   
   fruit  be.red     \\
\glt   {‘red fruit’}    \\
\vspace{5pt}
\ex
\label{Example_5.101h}
\gll   {orang}  {tua}\\
 person  be.old\\
 \glt   {‘old person’}\\
 \z
 \z


The second option of modifying nouns within a \isi{noun} phrase is by placing the \isi{verb} within a relative clause, as illustrated in (\ref{Example_5.102}) to (\ref{Example_5.105}). This verb-via-relative-clause \isi{modification} typically applies to dynamic verbs, such as (monotransitively used) \isi{trivalent} \textitbf{bawa} ‘bring’ in the elicited example in (\ref{Example_5.102}), \isi{bivalent} \textitbf{kawing} ‘marry unofficially’ in (\ref{Example_5.103}), or \isi{monovalent} dynamic \textitbf{tinggal} ‘stay’ in (\ref{Example_5.104}).


\begin{styleExampleTitle}
Attributive uses: Verb-via-relative-clause \isi{modification}
\end{styleExampleTitle}

\ea
\label{Example_5.102}
\gll {ojek} {\bluebold{yang}} {\bluebold{bawa}} {tete} {tu} {su} {pulang}\\ %
 motorbike.taxi  \textsc{rel}  bring  grandfather  \textsc{d.dist}  already  go.home\\
\glt 
‘that motorbike taxi driver \bluebold{who brought} grandfather has already returned home’ \textstyleExampleSource{[Elicited MY131119.001]}
\z

\ea
\label{Example_5.103}
\gll {orang} {Papua} {\bluebold{yang}} {\bluebold{kawing}} {orang} {pendatang} {de} {tinggal} {{\ldots}}\\ %
 person  Papua  \textsc{rel}  marry.inofficially  person  stranger  \textsc{3sg}  stay  \\
\glt 
‘a Papuan person \bluebold{who married} a stranger, he/she’ll stay (in Papua)’ \textstyleExampleSource{[081029-005-Cv.0046]}
\z

\ea
\label{Example_5.104}
\gll {{\ldots}} {buat} {sodara{\Tilde}sodara} {\bluebold{yang}} {\bluebold{tinggal}} {di} {kampung}\\ %
  { } for  \textsc{rdp}{\Tilde}sibling  \textsc{rel}  stay  at  village\\
\glt 
‘[we cut (the pig meat) up that day, we divided (it) for us who cut (it) up that day, (and) then] for the relatives and friends \bluebold{who live} in the village’ \textstyleExampleSource{[080919-003-NP.0014]}
\z

\ea
\label{Example_5.105}
\gll {de} {ada} {potong} {ikang} {\bluebold{yang}} {\bluebold{besar}} {di} {pante}\\ %
 \textsc{3sg}  exist  cut  fish  \textsc{rel}  be.big  at  coast\\
\glt 
‘at the beach he was cutting up a fish \bluebold{that was big}’ \textstyleExampleSource{[080919-004-NP.0061]}
\z


All \isi{verb} types can be used attributively. However, when comparing the attested attributively used \isi{verb} tokens across the two types of \isi{noun} phrase \isi{modification}, a clear pattern emerges, shown in \tabref{Table_5.16}. The vast majority of attributively used \isi{monovalent} stative verbs occur in \isi{noun} phrases involving verb-via-juxtapos\-ition \isi{modification}, although stative verbs also occur in verb-via-relative-clause \isi{modification}, such as \textitbf{besar} ‘be big’ in (\ref{Example_5.105}). The vast majority of attributively used dynamic verbs, by contrast, occur in \isi{noun} phrases involving verb-via-relative-clause \isi{modification}. Cross-linguistically these preferences are rather common \citep[31]{Dixon.1994}.





\begin{table}

\caption[Attributive uses of verbs within \isi{noun} phrases]{Attributive uses of verbs within \isi{noun} phrases\footnote{As percentages are rounded to one decimal place, they do not always add up to 100\%.}}\label{Table_5.16}
\centering
\begin{tabular}{l*{5}{r}}
\lsptoprule
 & \multicolumn{2}{c}{ Token frequencies} & & \multicolumn{2}{c}{Type frequencies}\\
\midrule
 & \multicolumn{2}{c}{ Via juxtaposition} &  & \multicolumn{2}{c}{Different verbs}\\
Verb class &  \# &  \%  & &  \# &  \%\\
\textsc{v.tri} &  1 &  0.2\%  & &  1 &  1.0\%\\
\textsc{v.bi} &  61 &  10.0\% &  &  27 &  26.0\%\\
\textsc{v.mo}(\textsc{dy}) &  30 &  4.9\%  & &  10 &  9.6\%\\
\textsc{v.mo}(\textsc{st}) &  520 &  85.0\% &  &  66 &  63.5\%\\
\midrule
Total &  612 &  100.0\%  & &  104 &  100.0\%\\
\midrule
& \multicolumn{2}{c}{ Via relative-clause}  & & \multicolumn{2}{c}{ Different verbs}\\
Verb class &  \# &  \% &  &  \# &  \%\\
\textsc{v.tri} &  35 &  4.2\% & &   5 &  2.2\%\\
\textsc{v.bi} &  371 &  44.5\% &  &  119 &  51.5\%\\
\textsc{v.mo}(\textsc{dy}) &  140 &  16.8\% &  &  27 &  11.7\%\\
\textsc{v.mo}(\textsc{st}) &  288 &  34.5\% &  &  80 &  34.6\%\\
\midrule
Total &  834 &  100.0\% & &   231 &  100.0\%\\
\midrule
& \multicolumn{2}{c}{ Overall totals} &  & \multicolumn{2}{c}{ Overall totals}\\
Verb class &  \# &  \% & &   \# &  \%\\
\textsc{v.tri} &  36 &  2.5\%  & &  6 &  1.8\%\\
\textsc{v.bi} &  432 &  29.9\%  & &  146 &  43.6\%\\
\textsc{v.mo}(\textsc{dy}) &  170 &  11.8\% & &   37 &  11.0\%\\
\textsc{v.mo}(\textsc{st}) &  808 &  55.9\% &   & 146 &  43.6\%\\
\midrule
Total &  1,446 &  100.0\% &  &  335 &  100.0\%\\
\lspbottomrule
\end{tabular}

\end{table}

So far 612 \isi{noun} phrases have been identified which involve verb-via-juxtapos\-ition \isi{modification}, and 834 \isi{noun} phrases with verb-via-relative-clause \isi{modification}. This total of 1,446 \isi{noun} phrases involves 36 \isi{noun} phrases (2.5\%) which are formed with seven distinct \isi{trivalent} verbs, 432 \isi{noun} phrases (29.9\%) formed with 146 distinct \isi{bivalent} verbs, 170 \isi{noun} phrases (11.8\%) formed with 37 distinct \isi{monovalent} dynamic verbs, and 808 \isi{noun} phrases (55.9\%) formed with 146 distinct \isi{monovalent} stative verbs. About two thirds of the attested 808 attributively used \isi{monovalent} stative \isi{verb} tokens occur in \isi{noun} phrases with verb-via-juxtapos\-ition \isi{modification} (520/808 – 64\%), while only about one third occurs in \isi{noun} phrases with verb-via-relative-clause \isi{modification} (288/808 – 36\%). The opposite holds for dynamic verbs. The vast majority of attributively used dynamic \isi{verb} tokens occur in \isi{noun} phrases with verb-via-relative-clause \isi{modification}: 35/36 \isi{trivalent} \isi{verb} tokens (97\%), 371/432 \isi{bivalent} \isi{verb} tokens (86\%), and 140/170 \isi{monovalent} dynamic \isi{verb} tokens (82\%). By contrast, only few dynamic verbs are used in \isi{noun} phrases with verb-via-\isi{juxtaposition} \isi{modification}: 1/36 \isi{trivalent} \isi{verb} tokens (3\%), 61/432 \isi{bivalent} \isi{verb} tokens (14\%), and 30/170 \isi{monovalent} dynamic \isi{verb} tokens (18\%).



\subsection{Adverbial modification}
\label{Para_5.3.3}
In their predicative uses, tri-, bi-, and \isi{monovalent} verbs can be modified with adverbs, as shown in (\ref{Example_5.106}) to (\ref{Example_5.113}). In (\ref{Example_5.106}) to (\ref{Example_5.109}), the temporal ad\isi{verb} \textitbf{langsung} ‘immediately’ modifies \isi{trivalent} \textitbf{kasi} ‘give’, \isi{bivalent} \textitbf{tanya} ‘ask’, \isi{monovalent} dynamic \textitbf{pulang} ‘go home’, and stative \textitbf{basa} ‘be wet’, respectively; the example in (\ref{Example_5.106}) is elicited.


\begin{styleExampleTitle}
Adverbial \isi{modification} with temporal ad\isi{verb} \textitbf{langsung} ‘immediately’
\end{styleExampleTitle}

\ea
\label{Example_5.106}
\gll {pace} {dong} {\bluebold{langsung}} {\bluebold{kasi}} {dia} {senter}\\ %
 man  \textsc{3pl}  immediately  give  \textsc{3sg}  flashlight\\
\glt 
‘the men \bluebold{immediately gave} him a flashlight’ \textstyleExampleSource{[Elicited BR130221.013]}
\z

\ea
\label{Example_5.107}
\gll  sa  \bluebold{langsung}  \bluebold{tanya}  dorang\\
 \textsc{1sg}  immediately  ask  \textsc{3pl}\\
\glt 
`I \bluebold{immediately asked} them’ \textstyleExampleSource{[080919-007-CvNP.0045]}
\z

\ea
\label{Example_5.108}\gll  sa  \bluebold{langsung}  \bluebold{pulang}\\
 \textsc{1sg}  immediately  go.home\\
\glt 
‘I \bluebold{went home immediately}’ \textstyleExampleSource{[081014-008-CvNP.0018]}
\z

\ea
\label{Example_5.109}
\gll {bapa} {\bluebold{langsung}} {\bluebold{diam}}\\ %
 father  immediately  be.quiet\\
\glt 
‘the gentleman \bluebold{was immediately quiet}’ \textstyleExampleSource{[080917-010-CvEx.0213]}
\z


Along similar lines, frequency ad\isi{verb} \textitbf{lagi} ‘again, also’ modifies the verbs in (\ref{Example_5.110}) to (\ref{Example_5.113}); the example in (\ref{Example_5.110}) is elicited. (For more details on adverbs see §\ref{Para_5.4}.)


\begin{styleExampleTitle}
Adverbial \isi{modification} with frequency ad\isi{verb} \textitbf{lagi} ‘again, also’
\end{styleExampleTitle}

\ea
\label{Example_5.110}
\gll {Dodo} {\bluebold{ambil}} {Agus} {air} {\bluebold{lagi}}\\ %
 Dodo  fetch  Agus  water  again\\
\glt 
‘Dodo \bluebold{fetched} water for Agus \bluebold{again}’ \textstyleExampleSource{[Elicited BR130409.001]}
\z

\ea
\label{Example_5.111}
\gll  sa  \bluebold{tampeleng}  dia  \bluebold{lagi}\\
 \textsc{1sg}  slap.on.face/ears  \textsc{3sg}  again\\
\glt 
‘I \bluebold{slapped} him \bluebold{across the face again}’ \textstyleExampleSource{[081013-002-Cv.0007]}
\z

\ea
\label{Example_5.112}
\gll  nanti  Lodia  dong  \bluebold{datang}  \bluebold{lagi}\\
 very.soon  Lodia  \textsc{3pl}  come  again\\
\glt 
‘very soon Lodia and her companions will \bluebold{also come}’ \textstyleExampleSource{[081006-016-Cv.0010]}
\z

\ea
\label{Example_5.113}
\gll {{\ldots}} {sampe} {mungking} {dua} {taung} {baru} {\bluebold{rame}} {\bluebold{lagi}}\\ %
 { }    until  maybe  two  year  and.then  be.crowded  again\\
\glt
‘[it goes on like that] for maybe two years before (the situation gets) \bluebold{lively again}’ \textstyleExampleSource{[081025-004-Cv.0102]}
\z


\subsection{Intensification}
\label{Para_5.3.4}
In their predicative uses, \isi{monovalent} stative and \isi{bivalent} verbs can be intensified with the degree adverbs \textitbf{skali} ‘very’ or \textitbf{terlalu} ‘too’, as shown in (\ref{Example_5.114}) to (\ref{Example_5.117}). While \textitbf{skali} ‘very’ follows the \isi{verb} as in (\ref{Example_5.114}) and (\ref{Example_5.115}), \textitbf{terlalu} ‘too’ precedes it as in (\ref{Example_5.116}) and (\ref{Example_5.117}). Intensification of predicatively used \isi{monovalent} dynamic and \isi{trivalent} verbs is unattested in the corpus. Furthermore, \isi{intensification} of attributively used verbs is unattested. (For details on degree adverbs see §\ref{Para_5.4.7}.)


\begin{styleExampleTitle}
Intensification
\end{styleExampleTitle}

\ea
\label{Example_5.114}
\gll {sa} {\bluebold{snang}} {\bluebold{skali}} {dong} {pu} {cara} {masak}\\ %
 \textsc{1sg}  feel.happy(.about)  very  \textsc{3pl}  \textsc{poss}  manner  cook\\
\glt 
‘I \bluebold{very (much) enjoy} their way of cooking’ \textstyleExampleSource{[081014-017-CvPr.0029]}
\z

\ea
\label{Example_5.115}
\gll {Aris} {\bluebold{tinggi}} {\bluebold{skali}}\\ %
 Aris  be.high  very\\
\glt 
‘Aris\bluebold{ is very tall}’ \textstyleExampleSource{[080922-001b-CvPh.0026]}
\z

\ea
\label{Example_5.116}
\gll {\ldots} {ade} {kecil} {\bluebold{terlalu}} {\bluebold{menangis}} {kitorang}\\ %
  { } ySb  be.small  too  cry  \textsc{1pl}\\
\glt 
‘[Hana’s husband didn’t come along,] the small younger sibling \bluebold{cried too much} (for) us’ \textstyleExampleSource{[080921-002-Cv.0008]}
\z

\ea
\label{Example_5.117}
\gll {sa} {liat} {mama} {\bluebold{terlalu}} {\bluebold{baik}}\\ %
 \textsc{1sg}  see  mother  exceedingly  be.good\\
\glt 
‘I see you (‘mother’) \bluebold{are too good}’ \textstyleExampleSource{[081115-001a-Cv.0324]}
\z


As mentioned, \isi{intensification} of \isi{monovalent} dynamic verbs is unattested in the corpus. According to one consultant, though, it is possible to intensify them with the expressions \textitbf{terlalu banyak} ‘too much’ or \textitbf{terlalu sedikit} ‘too little’, as in the elicited examples in (\ref{Example_5.118}) and (\ref{Example_5.119}).


\begin{styleExampleTitle}
Grading of \isi{monovalent} dynamic verbs with \textitbf{terlalu banyak}/\textitbf{sedikit} ‘too much\-/little’
\end{styleExampleTitle}

\ea
\label{Example_5.118}
\gll {Dodo} {de} {\bluebold{terlalu}} {\bluebold{banyak}} {\bluebold{tidor}}\\ %
 Dodo  \textsc{3sg}  too  many  sleep\\
\glt 
‘Dodo \bluebold{sleeps too much}’ \textstyleExampleSource{[Elicited BR130410.005]}
\z

\ea
\label{Example_5.119}
\gll {Dodo} {de} {\bluebold{terlalu}} {\bluebold{sedikit}} {\bluebold{lari}}\\ %
 Dodo  \textsc{3sg}  too  few  run\\
\glt 
‘Dodo \bluebold{runs too little}’ \textstyleExampleSource{[Elicited BR130410.008]}
\z


In addition, one of the consultants came up with the two examples in (\ref{Example_5.120}) and (\ref{Example_5.121}), respectively, in which dynamic \textitbf{lari} ‘run’ and \textitbf{tunduk} ‘bow’ are directly modified with \textitbf{terlalu} ‘too’. In (\ref{Example_5.120}), however, \textitbf{lari} means ‘deviate’ rather than ‘run’, and \textitbf{tunduk} ‘bow’ in (\ref{Example_5.121}) receives the stative reading ‘be obedient’.


\begin{styleExampleTitle}
Grading of \isi{monovalent} dynamic verbs with \textitbf{terlalu} ‘too’
\end{styleExampleTitle}

\ea
\label{Example_5.120}
\gll {{prahu}} {{ini}} {pu} {ukurang} {\bluebold{terlalu}} {\bluebold{lari}} {dari} {ukurang}\\ %
 {boat}  {\textsc{d.prox}}  \textsc{poss}  measurement  too  run  from  measurement\\
\gll yang  {ko}  {kasi}\\
 \textsc{rel}  {\textsc{2sg}}  {give}\\
\glt 
‘the size of this boat \bluebold{deviates too much} from the size that you gave’ \textstyleExampleSource{[Elicited BR130410.017]}
\z

\ea
\label{Example_5.121}
\gll {Agus} {de} {\bluebold{terlalu}} {\bluebold{tunduk}}\\ %
 Agus  \textsc{3sg}  too  bow\\
\glt 
‘Agus \bluebold{is too obedient}’ \textstyleExampleSource{[Elicited BR130410.004]}
\z


When examining the attested intensified \isi{monovalent} stative and \isi{bivalent} \isi{verb} tokens as to whether they are intensified with \textitbf{skali} ‘very’ or with \textitbf{terlalu} ‘too’, the data shows clear distributional preferences, presented in \tabref{Table_5.17}. The corpus contains 155 \isi{verb} phrases, made up of 80 different verbs, in which \textitbf{skali} ‘very’ intensifies a \isi{verb}. Most of these verbs are stative ones (81\%), accounting for 80\% of the \textitbf{skali}{}-\isi{intensification} tokens. The corpus also contains 33 \isi{verb} phrases, formed with 27 different verbs, in which \textitbf{terlalu} ‘too’ intensifies a \isi{verb}. Again, most of the intensified verbs are stative ones (74\%) accounting for 73\% of the \textitbf{terlalu}{}-\isi{intensification} tokens.


\begin{table}
\caption{Intensification of verbs}\label{Table_5.17}
\begin{tabular}{l*{5}{r}}
\lsptoprule
& \multicolumn{2}{c}{ Token frequencies} &  & \multicolumn{2}{c}{ Type frequencies}\\
\midrule
 & \multicolumn{2}{c}{ \textitbf{skali}{}-intensification} &  & \multicolumn{2}{c}{ Different verbs}\\
Verb class &  \# &  \% &  &   \# &  \%\\
\textsc{v.tri} &  0 &  {}-{}-{}- & &   0 &  {}-{}-{}-\\
\textsc{v.bi} &  31 &  20.0\% & &   15 &  18.7\%\\
\textsc{v.mo}(\textsc{dy}) &  0 &  {}-{}-{}- & &   0 &  {}-{}-{}-\\
\textsc{v.mo}(\textsc{st}) &  124 &  80.0\% & &   65 &  81.3\%\\
\midrule
Total &  155 &  100.0\% & &   80 &  100.0\%\\
\midrule
& \multicolumn{2}{c}{ \textitbf{terlalu}{}-intensification} &  & \multicolumn{2}{c}{ Different verbs}\\
Verb class &  \# &  \% &  &  \# &  \%\\
\textsc{v.tri} &  0 &  {}-{}-{}- & &   0 &  {}-{}-{}-\\
\textsc{v.bi} &  9 &  27.3\% &  &  7 &  25.9\%\\
\textsc{v.mo}(\textsc{dy}) &  0 &  {}-{}-{}- &   & 0 &  {}-{}-{}-\\
\textsc{v.mo}(\textsc{st}) &  24 &  72.7\% &  &  20 &  74.1\%\\
\midrule
Total &  33 &  100.0\% &  &  27 &  100.0\%\\
\lspbottomrule
\end{tabular}
\end{table}
\subsection{Grading}
\label{Para_5.3.5}

In their predicative uses, \isi{monovalent} stative and \isi{bivalent} verbs can occur with \isi{grading} adverbs, as shown in (\ref{Example_5.122}) to (\ref{Example_5.125}), whereas \isi{grading} of \isi{monovalent} dynamic and \isi{trivalent} verbs is unattested. The comparative degree is marked with the \isi{grading} ad\isi{verb} \textitbf{lebi} ‘more’ and the superlative degree with \textitbf{paling} ‘most’; both adverbs precede the \isi{verb}. (For details on degree adverbs see §\ref{Para_5.4.7}; for details on comparative clauses see §\ref{Para_11.5}.)


\begin{styleExampleTitle}
Grading of \isi{bivalent} verbs
\end{styleExampleTitle}

\ea
\label{Example_5.122}
\gll {a,} {dong} {mala} {\bluebold{lebi}} {\bluebold{sayang}} {saya}\\ %
 ah!  \textsc{3pl}  in.fact  more  love  \textsc{1sg}\\
\glt 
‘ah, they actually \bluebold{loved} me \bluebold{more}’ \textstyleExampleSource{[Elicited BR130221.034]}\footnote{The elicited example in (\ref{Example_5.122}) is the corrected version of the original recording \textitbf{dong mana lebi sayang saya} ‘they actually[\textsc{spm}] loved me more’ [081110-008-NPHt.0021]. That is, the speaker mispronounced \textitbf{mala} ‘in.fact’, realizing it as \textitbf{mana}.}
\z

\ea
\label{Example_5.123}
\gll  tempat  itu  sa  \bluebold{paling}  \bluebold{takut}\\
 place  \textsc{d.dist}  \textsc{1sg}  most  feel.afraid(.of)\\
\glt 
‘that place I \bluebold{feel most afraid of}’ \textstyleExampleSource{[081025-006-Cv.0285]}
\z


\begin{styleExampleTitle}
Grading of \isi{monovalent} stative verbs
\end{styleExampleTitle}

\ea
\label{Example_5.124}
\gll  yo  kaka,  itu  yang  \bluebold{lebi}  \bluebold{baik}  untuk  saya\\
 yes  oSb  \textsc{d.dist}  \textsc{rel}  more  be.good  for  \textsc{1sg}\\
\glt 
[Talking about her husband:] ‘yes older sibling, that (is the one) who is \bluebold{better} for me’ \textstyleExampleSource{[081110-008-CvNP.0178]}
\z

\ea
\label{Example_5.125}
\gll {puri} {tu} {\bluebold{paling}} {\bluebold{besar}}\\ %
 anchovy-like.fish  \textsc{d.dist}  most  be.big\\
\glt 
‘that anchovy-like fish is \bluebold{the biggest}’ \textstyleExampleSource{[080927-003-Cv.0002]}
\z


Again, \isi{monovalent} dynamic verbs differ from \isi{monovalent} stative and \isi{bivalent} verbs in that they are not directly modified with a \isi{grading} ad\isi{verb}. Instead they are modified with \textitbf{lebi banyak} ‘(do s.th.) more’ to indicate comparative degree, as in the elicited example in (\ref{Example_5.126}), or with \textitbf{paling banyak} ‘(do s.th.) most’ to indicate superlative degree, as in the elicited example in (\ref{Example_5.127}).

‘that anchovy-like fish is \bluebold{the biggest}’
\begin{styleExampleTitle}
Grading of \isi{monovalent} dynamic verbs
\end{styleExampleTitle}

\ea
\label{Example_5.126}
\gll {Dodo} {\bluebold{lebi}} {\bluebold{banyak}} {\bluebold{bertriak}} {dari} {Agus}\\ %
 Dodo  more  many  scream  with  Agus\\
\glt 
‘Dodo \bluebold{screams more} than Agus’ \textstyleExampleSource{[Elicited BR130221.025]}
\z

\ea
\label{Example_5.127}
\gll {Dodo} {\bluebold{paling}} {\bluebold{banyak}} {\bluebold{tertawa}}\\ %
 Dodo  most  many  scream\\
\glt 
‘Dodo \bluebold{laughs most}’ \textstyleExampleSource{[Elicited BR130221.030]}
\z



\begin{table}[b]
\caption{Grading of verbs}\label{Table_5.18}
\begin{tabular}{l*{5}{r}}
\lsptoprule
& \multicolumn{2}{c}{ Token frequencies} &  & \multicolumn{2}{c}{ Type frequencies}\\
\midrule
 & \multicolumn{2}{c}{ \textsc{cmpr}{}-constructions} &  & \multicolumn{2}{c}{ Different verbs}\\
Verb class &  \# &  \%  & &  \# &  \%\\
\textsc{v.tri} &  0 &  {}-{}-{}- & &   0 &  {}-{}-{}-\\
\textsc{v.bi} &  6 &  11.1\% &  &  5 &  22.7\%\\
\textsc{v.mo}(\textsc{dy}) &  0 &  {}-{}-{}- & &   0 &  {}-{}-{}-\\
\textsc{v.mo}(\textsc{st}) &  48 &  88.9\% &  &  17 &  77.3\%\\
\midrule
Total &  54 &  100\% &  &  22 &  100\%\\
\midrule
& \multicolumn{2}{c}{ \textsc{supl}{}-constructions} &  & \multicolumn{2}{c}{ Different verbs}\\
Verb class &  \# &  \% &  &  \# &  \%\\
\textsc{v.tri} &  0 &  {}-{}-{}- & &   0 &  {}-{}-{}-\\
\textsc{v.bi} &  8 &  17.4\% &  &  6 &  20.0\%\\
\textsc{v.mo}(\textsc{dy}) &  0 &  {}-{}-{}- &  &  0 &  {}-{}-{}-\\
\textsc{v.mo}(\textsc{st}) &  38 &  82.6\% & &   24 &  80.0\%\\
\midrule
Total &  46 &  100\% &  30 &  &  100\%\\
\lspbottomrule
\end{tabular}
\end{table}

With respect to the frequencies of the \isi{monovalent} stative and \isi{bivalent} verbs in comparative constructions, the data indicates a clear pattern, presented in \tabref{Table_5.18}. The vast majority of graded verbs are \isi{monovalent} stative ones. The corpus contains 54 \textitbf{lebi}{}-comparative constructions, formed with 22 different verbs. Of these, 77\% are \isi{monovalent} stative, accounting for 89\% of the attested comparative constructions. In addition, the corpus contains 46 \textitbf{paling}{}-superlative constructions, formed with 30 different verbs. Again, most of these verbs are \isi{monovalent} stative (80\%) which account for 83\% of the superlative constructions. Cross-linguistically, this distributional pattern corresponds to the “prototypical comparative scheme” in which the parameter of comparison “is typically expressed by an adjective, in a language with a large open class of adjectives; or else by a stative \isi{verb} (with an adjective-like meaning)” {\citep[787]{Dixon.2008}}.


\subsection{Negation}
\label{Para_5.3.6}
Verbs are negated with \textitbf{tida} ‘\textsc{neg}’ or \textitbf{tra} ‘\textsc{neg}’.\footnote{\label{Footnote_5.155}The negator \textitbf{bukang} ‘\textsc{neg}’ also occurs in verbal clauses. However, it does not negate the \isi{verb} as \textitbf{tida/tra} ‘\textsc{neg}’ does. Instead, \textitbf{bukang} ‘\textsc{neg}’ has scope over the entire proposition and expresses contrastive \isi{negation} of that proposition as a whole (for details see §\ref{Para_13.1.2}).} This is demonstrated with \isi{trivalent} \textitbf{kasi} ‘give’ in (\ref{Example_5.128}), \isi{bivalent} \textitbf{pake} ‘use’ in (\ref{Example_5.129}), \isi{monovalent} dynamic \textitbf{datang} ‘come’ in (\ref{Example_5.130}), and \isi{monovalent} stative \textitbf{baik} ‘be good’ in (\ref{Example_5.131}). These examples also illustrate that both negators are used interchangeably (for more details on \isi{negation} see §\ref{Para_13.1}).


\ea
\label{Example_5.128}
\gll {kaka} {su} {bilang} {de} {begitu,} {sa} {\bluebold{tra}} {\bluebold{kasi}} {ko} {jempol}\\ %
 oSb  already  say  \textsc{3sg}  like.that  \textsc{1sg}  \textsc{neg}  give  \textsc{2sg}  thumb\\
\glt 
‘I (‘older sibling’) already told him like that, ``I \bluebold{won’t give} you a thumbs up''' \textstyleExampleSource{[081115-001a-Cv.0042]}
\z

\ea
\label{Example_5.129}
\gll {kalo} {saya} {berburu} {\bluebold{tida}} {\bluebold{pake}} {anjing} {malam} {hari} {saya} {kluar}\\ %
 if  \textsc{1sg}  hunt  \textsc{neg}  use  dog  night  day  \textsc{1sg}  go.out\\
\glt 
‘if I hunt without \bluebold{taking} dogs, I leave at night’ \textstyleExampleSource{[080919-004-NP.0002]}
\z

\ea
\label{Example_5.130}
\gll {de} {\bluebold{tra}} {datang} {{\ldots}} {de} {\bluebold{tida}} {datang}\\ %
 \textsc{3sg}  \textsc{neg}  come   { }  \textsc{3sg}  \textsc{neg}  come\\
\glt 
‘she did \bluebold{not} come {\ldots} she did \bluebold{not} come’ \textstyleExampleSource{[081010-001-Cv.0204-0205]}
\z

\ea
\label{Example_5.131}
\gll {nanti} {dia} {pikir} {saya} {\bluebold{tida}} {\bluebold{baik}}\\ %
 very.soon  \textsc{3sg}  think  \textsc{1sg}  \textsc{neg}  be.good\\
\glt
‘later he’ll think (that) I’m \bluebold{not good}’ \textstyleExampleSource{[080919-004-NP.0052]}
\z


\subsection{Causative constructions}
\label{Para_5.3.7}
Papuan Malay syntactic causatives are monoclausal V\textsubscript{1}V\textsubscript{2} constructions. A caus\-ative \isi{verb} V\textsubscript{1}, encodes the notion of cause, while the V\textsubscript{2} denotes the notion of effect. Two full verbs both of which are still used synchronically function as \isi{causative} verbs, namely \isi{trivalent} \textitbf{kasi} ‘give’, with its short form \textitbf{kas}, and \isi{bivalent} \textitbf{biking} ‘make’. Syntactic causatives have \isi{monovalent} or \isi{bivalent} bases, while \isi{causative} constructions with \isi{trivalent} verbs are unattested.



In \textitbf{kasi}{}-causatives the V\textsubscript{2} can be \isi{bivalent} or \isi{monovalent}, while in \textitbf{biking}{}-causatives the V\textsubscript{2} is always \isi{monovalent}. (See §\ref{Para_11.2} for a detailed discussion of \isi{causative} constructions.)



Causative constructions with \textitbf{kasi} ‘give’ are presented in (\ref{Example_5.132}) to (\ref{Example_5.134}). The V\textsubscript{2} is \isi{bivalent} \textitbf{masuk} ‘enter’ in (\ref{Example_5.132}), \isi{monovalent} dynamic \textitbf{bangung} ‘wake up’ in (\ref{Example_5.133}), and stative \textitbf{sembu} ‘be healed’ in (\ref{Example_5.134}). (For more details on \textitbf{kasi}{}-causatives, see §\ref{Para_11.2.1.2}.)


\begin{styleExampleTitle}
Causative constructions with \textitbf{kasi} ‘give’
\end{styleExampleTitle}

\ea
\label{Example_5.132}
\gll {dong} {\bluebold{kas}} {\bluebold{masuk}} {korek} {di} {sini}\\ %
 \textsc{3pl}  give  enter  matches  at  \textsc{l.prox}\\
\glt 
‘they \bluebold{inserted} matches here’ (Lit. ‘\bluebold{give to enter}’) \textstyleExampleSource{[081025-006-Cv.0180]}
\z

\ea
\label{Example_5.133}
\gll {sa} {takut} {skali} {jadi} {sa} {\bluebold{kas}} {\bluebold{bangung}} {mama}\\ %
 \textsc{1sg}  feel.afraid(.of)  very  so  \textsc{1sg}  give  wake.up  mother\\
\glt 
‘I felt very afraid, so I \bluebold{woke up} you (‘mother’)’ (Lit. ‘\bluebold{give to wake up}’) \textstyleExampleSource{[080917-008-NP.0031]}
\z

\ea
\label{Example_5.134}
\gll {ko} {\bluebold{kasi}} {\bluebold{sembu}} {sa} {punya} {ana} {ini}\\ %
 \textsc{2sg}  give  be.healed  \textsc{1sg}  \textsc{poss}  child  \textsc{d.prox}\\
\glt 
‘[Addressing an evil spirit:] ‘you \bluebold{heal} this child of mine!’ (Lit. ‘\bluebold{give to be healed}’) \textstyleExampleSource{[081006-023-CvEx.0031]}
\z



In causatives with \textitbf{biking} ‘make’, the V\textsubscript{2} is always \isi{monovalent}. Most often, the \isi{monovalent} \isi{verb} is stative, such as \textitbf{pusing} ‘be dizzy, be confused’ in (\ref{Example_5.135}). However, \textitbf{biking}{}-causatives can also be formed with non-agentive dynamic bases, such as \textitbf{tenggelam} ‘sink’ in the elicited example in (\ref{Example_5.136}). If the causee is inanimate, or animate but helpless, the base can also be agentive dynamic, such as \textitbf{hidup} ‘live’ in the elicited example in (\ref{Example_5.137}). (For more details on \textitbf{biking}{}-causatives, see §\ref{Para_11.2.1.3}; see also examples (\ref{Example_11.55}) and (\ref{Example_11.57}) in §\ref{Para_11.2.1.2}, p. \pageref{Example_11.55}.)


\begin{styleExampleTitle}
Causative constructions with \textitbf{biking} ‘make’
\end{styleExampleTitle}

\ea
\label{Example_5.135}
\gll {yo,} {dong} {dua} {deng} {Wili} {tu} {\bluebold{biking}} {\bluebold{pusing}} {mama}\\ %
 yes  \textsc{3pl}  two  with  Wili  \textsc{d.dist}  make  be.dizzy  mother\\
\glt 
‘yes!, he and Wili there \bluebold{worried} (their) mother’ (Lit. ‘\bluebold{make to be dizzy/confused}’) \textstyleExampleSource{[081011-003-Cv.0002]}
\z

\ea
\label{Example_5.136}
\gll {banyak} {mati} {di} {lautang,} {\bluebold{biking}} {\bluebold{tenggelam}}\\ %
 many  die  at  ocean  make  sink\\
\glt 
[About people in a container who died in the ocean:] ‘many died in the (open) ocean, (the murderers) \bluebold{sank} (the containers)’ \textstyleExampleSource{[Elicited BR131103.003]}
\z

\ea
\label{Example_5.137}
\gll {{\ldots}} {tapi} {dong} {\bluebold{biking}} {bangkit} {dia} {lagi,} {\bluebold{biking}} {\bluebold{hidup}} {dia}\\ %
{}   but  \textsc{3pl}  make  be.resurrected  \textsc{3sg}  again  make  live  \textsc{3sg}\\
\glt 
[About sorcerers who can resurrect the dead:] ‘[he’s already (dead),] but they \bluebold{resurrect} him again, \bluebold{make} him \bluebold{live}’ \textstyleExampleSource{[Elicited BR131103.005]}
\z



\begin{table}
\caption{Causative constructions with \textitbf{kasi} ‘give’ and \textitbf{biking} ‘make’}\label{Table_5.19}
\begin{tabular}{l*{5}{r}}
\lsptoprule
& \multicolumn{2}{c}{ Token frequencies}  & & \multicolumn{2}{c}{ Type frequencies}\\
\midrule
 & \multicolumn{2}{c}{ \textitbf{kasi}{}-causatives} &  & \multicolumn{2}{c}{ Different verbs}\\
Verb class &  \# &  \% & &   \# &  \%\\
\textsc{v.tri} &  0 &  {}-{}-{}- &   & 0 &  {}-{}-{}-\\
\textsc{v.bi} &  327 &  68.4\% &   & 39 &  48.1\%\\
\textsc{v.mo}(\textsc{dy}) &  115 &  24.1\% & &   18 &  22.2\%\\
\textsc{v.mo}(\textsc{st}) &  36 &  7.5\% &  &  24 &  29.6\%\\
\midrule
Total &  478 &  100\% &  & 81 &  100\%\\
\midrule
& \multicolumn{2}{c}{ \textitbf{biking}{}-causatives}  & & \multicolumn{2}{c}{ Different verbs}\\
Verb class &  \# &  \% & &   \# &  \%\\
\textsc{v.tri} &  0 &  {}-{}-{}- & &   0 &  {}-{}-{}-\\
\textsc{v.bi} &  0 &  {}-{}-{}- & &   0 &  {}-{}-{}-\\
\textsc{v.mo}(\textsc{dy}) &  0 &  {}-{}-{}- &   & 0 &  {}-{}-{}-\\
\textsc{v.mo}(\textsc{st}) &  25 &  100\% & &   16 &  100\%\\
\midrule
Total &  25 &  100.0\% & &  16 &  100.0\%\\
\lspbottomrule
\end{tabular}
\end{table}

Concerning the frequencies of mono- and \isi{bivalent} verbs in \isi{causative} constructions, the following pattern emerges. Most of the attested \isi{verb} tokens in \textitbf{kasi}{}-\isi{causative} constructions are \isi{bivalent} or \isi{monovalent} dynamic ones, whereas the attested verbs in \textitbf{biking}{}-causatives are always \isi{monovalent} stative ones, as shown in \tabref{Table_5.19}. The corpus contains 478 \textitbf{kasi}{}-\isi{causative} constructions, formed with 81 different verbs. Most verbs in \textitbf{kasi}{}-constructions are dynamic one (78\%), including 48\% \isi{bivalent} and 22\% \isi{monovalent} dynamic verbs. Together the attested dynamic verbs account for 92\% of the \textitbf{kasi}{}-causatives. By contrast, \textitbf{biking}{}-causatives are always formed with \isi{monovalent} stative verbs. In all, the corpus contains 25 \textitbf{biking}{}-causatives, formed with 16 different verbs.

\subsection{Reciprocal constructions}
\label{Para_5.3.8}
Verbs can occur in reciprocal constructions in which the \isi{reciprocity marker} \textitbf{baku} ‘\textsc{recp}’ precedes the \isi{verb} (for more details on reciprocal constructions, see §\ref{Para_11.3}). This is illustrated with \isi{trivalent} \textitbf{ceritra} ‘tell’ in the elicited example in (\ref{Example_5.138}), \isi{bivalent} \textitbf{gendong} ‘hold’ in (\ref{Example_5.139}), and \isi{monovalent} dynamic \textitbf{saing} ‘compete’ in (\ref{Example_5.140}). Reciprocal constructions with \isi{monovalent} stative verbs are unattested.

\largerpage

\ea
\label{Example_5.138}
\gll {Markus} {deng} {Yan} {dong} {\bluebold{baku}} {\bluebold{ceritra}}\\ %
 Markus  with  Yan  \textsc{3pl}  \textsc{recp}  tell\\
\glt 
‘Markus and Yan were \bluebold{talking to each other}’ \textstyleExampleSource{[Elicited BR130601.001]}\footnote{The corpus contains one \isi{reciprocal construction} formed with \isi{trivalent} \textitbf{ceritra} ‘tell’, similar to the elicited example in (\ref{Example_5.138}). Most of the utterance is unclear, however, as the speaker mumbles.}
\z

\ea
\label{Example_5.139}
\gll {kitong} {\bluebold{baku}} {\bluebold{gendong}} {to?} \bluebold{baku} \bluebold{gendong}\\ %
 \textsc{1pl}  \textsc{recp}  hold  right?  \textsc{recp}  hold\\
\glt 
‘we’ll \bluebold{hold each other}, right?, (we’ll) \bluebold{hold each other}’ \textstyleExampleSource{[080922-001a-CvPh.0695]}
\z

\ea
\label{Example_5.140}
\gll {ade-kaka} {\bluebold{baku}} {\bluebold{saing}}\\ %
 ySb-oSb    \\
 \\
 siblings  \textsc{recp}  compete\\
\glt 
‘the siblings were \bluebold{competing with each other}’ \textstyleExampleSource{[080919-006-CvNP.0001]}
\z





\begin{table}
\caption{Reciprocal constructions}\label{Table_5.20}

\begin{tabular}{l*{5}{r}}
\lsptoprule


 & \multicolumn{2}{c}{ Token frequencies} & &  \multicolumn{2}{c}{ Type frequencies}\\
 \midrule
 & \multicolumn{2}{c}{ \textsc{recp}{}-constructions} &  & \multicolumn{2}{c}{ Different verbs}\\
Verb class &  \# &  \%  & &  \# &  \%\\
\textsc{v.tri} &  1 &  1.0\% & &   1 &  2.4\%\\
\textsc{v.bi} &  95 &  94.1\% & &   37 &  88.1\%\\
\textsc{v.mo}(\textsc{dy}) &  5 &  5.0\% &  &  4 &  9.5\%\\
\textsc{v.mo}(\textsc{st}) &  0 &  {}-{}-{}- & &   0 &  {}-{}-{}-\\
\midrule
Total &  101 &  100\% & &   42 &  100\%\\
\lspbottomrule
\end{tabular}
\end{table}


The data in the corpus indicates the following frequency patterns for reciprocal constructions, as shown in \tabref{Table_5.20}. The corpus contains 101 reciprocal constructions formed with 42 different verbs. Most of these verbs are \isi{bivalent} (88\%), accounting for 94\% of the reciprocal constructions.

\subsection{Morphological properties}\label{Para_5.3.9}

Papuan Malay has only two somewhat productive affixes, as discussed in \chapref{Para_3}, prefix \textscItal{ter-} ‘\textsc{acl}’ and suffix -\textitbf{ang} ‘\textsc{nmlz}’. Mono- and \isi{bivalent} verbs can be prefixed with \textscItal{ter-} ‘\textsc{acl}’ to derive verbs which denote accidental or unintentional actions or events. Examples are given in \tabref{Table_5.21}, such as \isi{bivalent} \textitbf{angkat} ‘lift’ and \textitbf{lempar} ‘throw’, \isi{monovalent} dynamic \textitbf{jatu} ‘fall’, and stative \textitbf{lambat} ‘be slow’ and \textitbf{sala} ‘be wrong’. Likewise, mono- and \isi{bivalent} verbs can be suffixed with-\textitbf{ang} ‘\textsc{nmlz}’ to derive nouns, such as \isi{bivalent} \textitbf{jual} ‘sell’ and \textitbf{pake} ‘use’, \isi{monovalent} dynamic \textitbf{jalang} ‘walk’ and \textitbf{libur} ‘take vacation’, and stative \textitbf{pica} ‘be broken’ and \textitbf{sial} ‘be unfortunate’. Some lexemes suffixed with-\textitbf{ang} ‘\textsc{nmlz}’ also function as verbs, such as \textitbf{jualang} ‘merchandise, to sell’. Affixation of \isi{trivalent} verbs is unattested. (For details on \isi{affixation} with \textscItal{ter-} ‘\textsc{acl}’ and-\textitbf{ang} ‘\textsc{nmlz}’ see §\ref{Para_3.1.2} and §\ref{Para_3.1.3}, respectively).


\begin{table}[p]
\caption{Affixation of verbs}\label{Table_5.21}

\begin{tabular}{llll}
\lsptoprule
 \multicolumn{1}{c}{BF} & \multicolumn{1}{c}{Gloss} & \multicolumn{1}{c}{Item} &  \multicolumn{1}{c}{Gloss}\\
 \midrule
\multicolumn{4}{l}{Prefix \textscItal{ter-}: Derived verbs denoting accidental actions}\\
\midrule
\textitbf{angkat} & ‘lift’ & \textitbf{trangkat} & ‘be lifted up’\\
\textitbf{lempar} & ‘throw’ & \textitbf{talempar} & ‘be thrown’\\
\textitbf{jatu} & ‘fall’ & \textitbf{terjatu} & ‘be dropped, fall’\\
\textitbf{lambat} & ‘be slow’ & \textitbf{terlambat} & ‘be late’\\
\textitbf{sala} & ‘be wrong’ & \textitbf{tasala} & ‘be mistaken’\\
\midrule
\multicolumn{4}{l}{Suffix-\textitbf{ang}: Derived nouns}\\
\midrule
\textitbf{jual} & ‘sell’ & \textitbf{jualang} & ‘merchandise, to sell’\\
\textitbf{pake} & ‘use’ & \textitbf{pakeang} & ‘clothes’\\
\textitbf{jalang} & ‘walk’ & \textitbf{jalangang} & ‘route’\\
\textitbf{libur} & ‘take vacation’ & \textitbf{liburang} & ‘vacation’\\
\textitbf{pica} & ‘be broken’ & \textitbf{picaang} & ‘splinter’\\
\textitbf{sial} & ‘be unfortunate’ & \textitbf{sialang} & ‘s.o. unfortunate/ill-fated’\\
\lspbottomrule
\end{tabular}
\end{table}


\begin{table}[p]
\caption{Affixation of verbs}\label{Table_5.22}

\begin{tabular}{l*{5}{r}}
\lsptoprule
& \multicolumn{2}{c}{ Token frequencies} & {} & \multicolumn{2}{c}{ Type frequencies}\\
\midrule
 & \multicolumn{2}{c}{ \textscItal{ter-}affixation} &  & \multicolumn{2}{c}{ Different verbs}\\
Verb class &  \# &  \% &  &  \# &  \%\\
\textsc{v.tri} &  0 &  {}-{}-{}- & &   0 &  {}-{}-{}-\\
\textsc{v.bi} &  153 &  91.6\% &  &  38 &  88.4\%\\
\textsc{v.mo}(\textsc{dy}) &  1 &  0.6\% & &   1 &  2.3\%\\
\textsc{v.mo}(\textsc{st}) &  13 &  7.8\% &  &  4 &  9.3\%\\
\midrule
Total &  167 &  100.0\% &   & 43 &  100.0\%\\
\midrule
& \multicolumn{2}{c}{ \textitbf{{}-ang} affixation}  & & \multicolumn{2}{c}{ Different verbs}\\
Verb class &  \# &  \% & &   \# &  \%\\
\textsc{v.tri} &  0 &  {}-{}-{}- &  &  0 &  {}-{}-{}-\\
\textsc{v.bi} &  357 &  88.6\% & &   62 &  89.9\%\\
\textsc{v.mo}(\textsc{st}) &  12 &  3.0\% &  &  3 &  4.3\%\\
\textsc{v.mo}(\textsc{dy}) &  34 &  8.4\% &  &  4 &  5.8\%\\
\midrule
Total &  403 &  100.0\% & &   69 &  100.0\%\\
\lspbottomrule
\end{tabular}
\end{table}


In the corpus, \isi{affixation} of \isi{bivalent} bases occurs much more often than that of \isi{monovalent} bases, as shown in \tabref{Table_5.22}. Regarding prefix \textscItal{ter-} ‘\textsc{acl}’, the corpus includes 43 lexemes derived from verbal bases with a total of 166 tokens. Most of them are \isi{bivalent} verbs (88\%), accounting for 92\% of all \textscItal{ter-}tokens. As for suffix-\textitbf{ang} ‘\textsc{nmlz}’, the corpus contains 69 lexemes with verbal bases, with a total of 403 tokens. Again, most of the verbal bases are \isi{bivalent} (90\%), accounting for 89\% of all-\textitbf{ang}{}-tokens.


\subsection{Summary}
\label{Para_5.4.11}
Tri-, bi-, and \isi{monovalent} verbs have partially distinct and partially overlapping properties, which are summarized in \tabref{Table_5.23}  (in this table bi- and \isi{trivalent} verbs are listed summarily in the column ``Valency of 2 or 3''). They are distinct from each other in terms of two main criteria, namely their \isi{valency} and their function, which is mainly predicative. Related to the criterion on \isi{valency} is the ability of verbs to occur in \isi{causative} and reciprocal expressions and to be affixed. Therefore, \tabref{Table_5.23} lists these characteristic under the label ``Valency''. The criterion of function has to do with the predicative (\textsc{pred}) and attributive (\textsc{attr}) uses of the verbs, their \isi{negation}, and adverbial \isi{modification}. Hence, \tabref{Table_5.23} lists these characteristics under the label ``Function''.

\begin{table}[b]

\setlength{\tabcolsep}{5pt}
\caption[Properties of tri-, bi-, and \isi{monovalent} verbs]{Properties of tri-, bi-, and \isi{monovalent} verbs\footnote{See \citet[51–53]{vanKlinken.1999} for a similar approach to distinguishing different \isi{verb} classes.}}\label{Table_5.23}
\centering
\begin{tabularx}{\textwidth}{clXXX}

\lsptoprule

 \multicolumn{1}{c}{Main}  & \multicolumn{1}{c}{Properties} & \multicolumn{1}{c}{Valency} & \multicolumn{2}{c}{ Valency of 1}\\
\multicolumn{1}{c}{criteria} &  & \multicolumn{1}{c}{ of 2 or 3} & \multicolumn{1}{c}{dynamic} &  \multicolumn{1}{c}{stative}\\
 \midrule
\multirow{4}{*}{\sidewaystext{Function}} & Adverbial \isi{modification} & Yes & Yes &  Yes\\
 & Negation (\textitbf{tida/tra}) & Yes & Yes &  Yes\\
& \textsc{pred} uses & Most often & Most often &  Less often\\
& \textsc{attr} uses (via relative clause) & Most often & Less often &  Less often\\
\midrule
\multirow{6}{*}{\sidewaystext{Valency}} & Base for \textscItal{ter-}\isi{affixation} & Most often & Less often &  Less often\\
 & Base for \textitbf{-ang}-\isi{affixation} & Most often & Less often &  Less often\\
& Causative (\textitbf{kasi}) & Most often & Less often &  Less often\\
& Reciprocal & Most often & Less often &  No\\
& Valency {\textgreater}1 & Yes & No &  No\\
& Causative (\textitbf{biking}) & Less often & No &  Most often\\\midrule
\multirow{5}{*}{\sidewaystext{Function}} & \textsc{attr} uses (via \isi{juxtaposition}) & Less often & No &  Most often\\
 & Intensification (\textitbf{skali}) & Less often & No &  Most often\\
& Intensification (\textitbf{terlalu}) & Less often & No &  Most often\\
& Grading (\textitbf{lebi}) & Less often & No &  Most often\\
& Grading (\textitbf{paling}) & Less often & No &  Most often\\
\lspbottomrule
\end{tabularx}

\end{table}

In terms of \isi{valency}, Papuan Malay has three \isi{verb} classes, mono-, bi- and \isi{trivalent} verbs. Related to the \isi{valency} criterion is the ability of verbs to be used in \isi{causative} constructions. All three \isi{verb} types occur in causatives formed with \textitbf{kasi} ‘give’. Most often, however, \textitbf{kasi}{}-causatives are formed with bi- or \isi{trivalent} verbs. By contrast, \isi{causative} constructions with \textitbf{biking} ‘make’ are typically formed with stative verbs; dynamic verbs are unattested in \textitbf{biking}{}-causatives. Also related to the \isi{valency} criterion is the ability of bi- and \isi{trivalent} verbs to occur in reciprocal expressions. Monovalent dynamic verbs, by contrast, occur only rarely in such expressions, while reciprocal constructions with stative verbs are unattested. Finally, with respect to \isi{affixation}, it is typically \isi{bivalent} verbs that form the bases for lexemes prefixed with \textscItal{ter-} or suffixed with \textitbf{-ang}.



With respect to their function, all verbs are used predicatively, dynamic verbs much more often though, than stative verbs. In their predicate uses, all three \isi{verb} types can be modified adverbially and all verbs are negated with \textitbf{tida}/\textitbf{tra} ‘\textsc{neg}’. Less often, verbs have \isi{attributive function} in \isi{noun} phrases. Verb-via-\isi{juxtaposition} \isi{modification} most commonly applies to stative verbs, while \isi{modification} with dynamic verbs typically involves verb-via-relative-clause \isi{modification}. Related to their attributive uses is the \isi{intensification} and \isi{grading} of verbs. Typically, this applies to \isi{monovalent} stative verbs, while \isi{intensification} and \isi{grading} of \isi{bivalent} verbs occurs much less often. Monovalent dynamic and \isi{trivalent} verbs are neither intensified nor graded.


\section{Adverbs}
\largerpage
\label{Para_5.4}
Papuan Malay has a large open class of adverbs, which modify constituents other than nouns. Their main function is to indicate aspect, frequency, \isi{affirmation} and \isi{negation}, modality, time, focus, and degree. Within the clause, the adverbs most commonly occur in prepredicate position. Unlike the other two open lexical classes of nouns and verbs, Papuan Malay adverbs are not used predicatively.



Cross-linguistically, \citet[66]{Haser.2006} note that in terms of their semantics and \isi{morphology}, “adverbs are most closely related to adjectives, from which they are often derived”. With the restriction that Papuan Malay has a class of \isi{monovalent} stative verbs instead of adjectives (see §\ref{Para_5.3.1}), this observation also seems to apply to the Papuan Malay adverbs. First, a number of adverbs are related to \isi{monovalent} stative verbs, such as the temporal ad\isi{verb} \textitbf{dulu} ‘first, in the past’ which is related to stative \textitbf{dulu} ‘be prior’ (see §\ref{Para_5.4.5}), or the focus ad\isi{verb} \textitbf{pas} ‘precisely’ which is related to stative \textitbf{pas} ‘be exact’ (see §\ref{Para_5.4.6}; see also §\ref{Para_5.14}). Second, manner is expressed through stative verbs (see §\ref{Para_5.4.8}). Third, reduplicated verbs can receive an adverbial reading due to an \isi{interpretational shift}. Examples are \textitbf{taw{\Tilde}taw} ‘just now’ with its base \textitbf{taw} ‘know’ (see §\ref{Para_5.4.5}; see also §\ref{Para_4.2.2.8}). In Papuan Malay this link with verbs extends to dynamic verbs, in that reduplicated dynamic verbs can also receive an adverbial reading. Examples are the modal adverbs \textitbf{kira{\Tilde}kira} ‘probably’ and \textitbf{taw{\Tilde}taw} ‘suddenly’ which are related to the respective dynamic verbs \textitbf{kira} ‘think’ and \textitbf{taw} ‘know’ (see §\ref{Para_5.4.4}; see also §\ref{Para_5.14}).



In addition to this prominent link with verbs, Papuan Malay adverbs are also related to nouns, although this link appears to be less prominent. First, a number of modal adverbs are \isi{historically derived} from nouns by unproductive \isi{affixation} with -\textitbf{nya} ‘\textsc{3possr}’. Examples are \textitbf{artinya} ‘that means’ (literally ‘the meaning of’), \textitbf{katanya} ‘it is being said’ (literally ‘the word of’), or \textitbf{maksutnya} ‘that is to say’ (literally ‘the purpose of’). Second, reduplicated nouns can receive an adverbial reading due to an \isi{interpretational shift} (see §\ref{Para_4.2.1.4}).


\newpage
The adverbs occur in different positions within the clause. They can take a prepredicate or postpredicate position, with the prepredicate position being the most common. There are also a fair number of adverbs which can occur in both positions. For the prepredicate adverbs two positions are possible, directly preceding the predicate and preceding the subject. Likewise, two positions are possible for the postpredicate adverbs, directly following the predicate and, in clauses with peripheral adjuncts, following the adjunct. Depending on their positions within the clause, the adverbs differ in terms of their semantic effect. Generally speaking, prepredicate adverbs which precede the subject have scope over the entire proposition. The semantic effect of prepredicate adverbs which directly precede the predicate and of postpredicate adverbs is more limited. On the whole, however, these distinctions are subtle, as shown with the temporal ad\isi{verb} \textitbf{langsung} ‘immediately’ in §\ref{Para_5.4.5}.



The following sections describe the adverbs in terms of their positions within the clause and their overall semantic functions. Aspect adverbs are discussed in §\ref{Para_5.4.1}, frequency adverbs in §\ref{Para_5.4.2}, \isi{affirmation} and \isi{negation} adverbs in §\ref{Para_5.4.3}, modal adverbs in §\ref{Para_5.4.4}, temporal adverbs in §\ref{Para_5.4.5}, focus adverbs in §\ref{Para_5.4.6}, and degree adverbs in §\ref{Para_5.4.7}. Papuan Malay does not have manner adverbs; instead, manner is expressed through stative verbs which always follow the main \isi{verb}, as briefly discussed in §\ref{Para_5.4.8}. Each of these sections includes a table which lists the different adverbs and indicates whether they take a prepredicate (\textsc{pre-pred}) and/or postpredicate (\textsc{post-pred}) position within the clause (empty cells signal unattested constituent combinations). The different positions are also illustrated with (near) contrastive examples. An investigation of the semantic effects encoded by these positions, however, is left for future research. Also left for future research is the question of which adverbs can co-occur and in which positions.



Following the description of the different types of adverbs, §\ref{Para_5.4.9} summarizes the main points of this section, especially with respect to the interplay between syntactic properties and functions of the adverbs.


\subsection{Aspectual adverbs}
\label{Para_5.4.1}
Aspectual adverbs provide temporal information about the event or state denoted by the \isi{verb} in terms of their “duration or completion” {\citep[5094]{Asher.1994}}. Thereby they differ from the temporal adverbs which designate temporal points \citep[91–92]{Givon.2001}. The Papuan Malay aspectual adverbs are presented in \tabref{Table_5.24}.



Aspectual \textitbf{blum} ‘not yet’ and \textitbf{masi} ‘still’ have prospective meanings; that is, they point “forward to possible transitions in the future”, using Smessaert and ter Meulen’s (\citeyear*{Smessaert.2004}) terminology. More specifically, \textitbf{blum} ‘not yet’ indicates that the event or state denoted by the \isi{verb} is not yet completed or has not yet occurred, while \textitbf{masi} ‘still’ signals that the event or state is still continuing. Aspectual \textitbf{suda} ‘already’, by contrast, has a retrospective meaning; that is, it marks “a realized transition in the past”, again employing Smessaert and ter Meulen’s (\citeyear*[221]{Smessaert.2004}) terminology (\textitbf{suda} ‘already’ is very often shortened to \textitbf{su}). Besides, \textitbf{suda} ‘already’ can signal \isi{imperative} mood, in which case it occurs in clause-final position, as discussed in §\ref{Para_13.3}. Progressive aspect is not encoded by an ad\isi{verb} but with the existential \isi{verb} \textitbf{ada} ‘exist’; for expository reasons, however, the progressive marking function of \textitbf{ada} ‘exist’ is discussed here (existential clauses are discussed in §\ref{Para_11.4}).


 
The three adverbs always occur in prepredicate position, as shown in \tabref{Table_5.24}. This applies to their uses in verbal clauses, as in (\ref{Example_5.141}) and (\ref{Example_5.142}), and in nonverbal clauses, as in (\ref{Example_5.145}) to (\ref{Example_5.147}). Likewise, adverbially used \textitbf{ada} ‘exist’ precedes the predicate, as shown in (\ref{Example_5.143}), (\ref{Example_5.144}), and (\ref{Example_5.148}).


\begin{table}
\caption{Aspectual adverbs and adverbially used \textitbf{ada} ‘exist’ and their positions within the clause}\label{Table_5.24}
\begin{tabular}{llcc}
\lsptoprule

\multicolumn{1}{c}{Item} & \multicolumn{1}{c}{Gloss} & \multicolumn{2}{c}{ Position}\\
&  & \textsc{pre-pred} &  \textsc{post-pred}\\
\midrule

\textitbf{blum} & ‘not yet’ & X & \\
\textitbf{masi} & ‘still’ & X & \\
\textitbf{suda} & ‘already’ & X & \\
\textitbf{ada} & ‘exist’ & X & \\
\lspbottomrule
\end{tabular}
\end{table}

In verbal predicate clauses, the aspectual adverbs and adverbially used \textitbf{ada} ‘exist’ modify dynamic verbs, as in (\ref{Example_5.141}) and (\ref{Example_5.143}), or stative verbs as in (\ref{Example_5.142}) and (\ref{Example_5.144}).


\begin{styleExampleTitle}
Aspectual adverbs and adverbially used \textitbf{ada} ‘exist’ modifying verbal predicates
\end{styleExampleTitle}

\ea
\label{Example_5.141}
\gll {a}, {mama} {\bluebold{blum}} {mandi,} {mama} {\bluebold{masi}} {bangung} {tidor}\\ %
 ah!  mother  not.yet  bathe  mother  still  wake.up  sleep\\
\glt 
‘ah, I (‘mother’) have \bluebold{not yet} bathed, I (‘mother’) am \bluebold{still} waking up’ \textstyleExampleSource{[080924-002-Pr.0007]}
\z

\ea
\label{Example_5.142}
\gll {ana} {itu} {de} {\bluebold{suda}} {besar} {betul,} {de} {\bluebold{suda}} {besar} {{\ldots}}\\ %
 child  \textsc{d.dist}  \textsc{3sg}  already  be.big  be.true  \textsc{3sg}  already  be.big  \\
\glt 
‘(when) that child is \bluebold{already} really grown-up, (when) he/she’s \bluebold{already} grown-up, {\ldots}’ \textstyleExampleSource{[081006-025-CvEx.0005]}
\z

\ea
\label{Example_5.143}
\gll {sa} {pu} {maytua} {\bluebold{ada}} {\bluebold{tidor}} {karna} {hari} {blum} {siang}\\ %
 \textsc{1sg}  \textsc{poss}  wife  exist  sleep  because  day  not.yet  day\\
\glt 
‘my wife \bluebold{was sleeping} because it wasn’t daylight yet’ \textstyleExampleSource{[080919-004-NP.0026]}
\z

\ea
\label{Example_5.144}
\gll {dong} {bilang,} {a} {de} {\bluebold{ada}} {\bluebold{sakit}}\\ %
 \textsc{3pl}  say  ah!  \textsc{3sg}  exist  be.sick\\
\glt 
‘they said, ``ah, he’s \bluebold{being sick}''' \textstyleExampleSource{[080919-007-CvNP.0025]}
\z


The examples in (\ref{Example_5.145}) to (\ref{Example_5.148}) demonstrate the uses of the aspectual adverbs and adverbially used \textitbf{ada} ‘exist’ in nonverbal predicate clauses. (An alternative analysis of clauses with \textitbf{ada} ‘exist’, such as the one in (\ref{Example_5.148}), is presented in §\ref{Para_11.4.1}.)


\begin{styleExampleTitle}
Aspectual adverbs modifying nonverbal predicates
\end{styleExampleTitle}

\ea
\label{Example_5.145}
\gll {itu} {kang} {\bluebold{blum}} {musim} {ombak}\\ %
 \textsc{d.dist}  you.know  not.yet  season  wave\\
\glt 
[About traveling by high or low tide:] ‘that is \bluebold{not yet} the wavy season, you know’ \textstyleExampleSource{[080927-003-Cv.0020]}
\z

\ea
\label{Example_5.146}
\gll  Roni  \bluebold{masi}  deng  de  pu  temang{\Tilde}temang\\
 Roni  still  with  \textsc{3sg}  \textsc{poss}  \textsc{rdp}{\Tilde}friend\\
\glt 
‘Roni is \bluebold{still} with his friends’ \textstyleExampleSource{[081006-031-Cv.0011]}
\z

\ea
\label{Example_5.147}
\gll {sa} {\bluebold{su}} {di} {Arare} {sama} {Pawla}\\ %
 \textsc{1sg}  already  at  Arare  to  Pawla\\
\glt 
‘I (would) \bluebold{already} be in Arare with Pawla’ \textstyleExampleSource{[081025-009a-Cv.0110]}
\z

\ea
\label{Example_5.148}
\gll {ana{\Tilde}ana} {prempuang} {dong} {\bluebold{ada}} {di} {depang}\\ %
 \textsc{rdp}{\Tilde}child  woman  3\textsc{pl}  exist  at  front\\
\glt
‘the girls \bluebold{are being} in front’ \textstyleExampleSource{[080921-004a-CvNP.0066]}
\z


\subsection{Frequency adverbs}
\label{Para_5.4.2}
Frequency adverbs “typically indicate the number of times something happened” during a given time interval \citep[688]{Doetjes.2007}. The Papuan Malay frequency adverbs are listed in \tabref{Table_5.25} ; they always occur in prepredicate position.\footnote{In the corpus only \textitbf{biasanya} ‘usually’ and \textitbf{perna} ‘ever’ are attested in the clause-initial position; for the remaining frequency adverbs, their uses in this position were established by means of elicitation.}


\begin{table}[b]

\caption{Frequency adverbs and their positions within the clause}\label{Table_5.25}
\centering
\begin{tabular}{llcc}
\lsptoprule
 \multicolumn{1}{c}{Item} & \multicolumn{1}{c}{Gloss} & \multicolumn{2}{c}{Position}\\
&  & \multicolumn{1}{c}{\textsc{pre-pred}} &  \multicolumn{1}{c}{\textsc{post-pred}}\\
\midrule

\textitbf{biasanya}\footnote{The ad\isi{verb} \textitbf{biasanya} ‘usually’ is \isi{historically derived}: \textitbf{biasa-nya} ‘be.usual-\textsc{3poss}’ (for details on suffixation with \textitbf{{}-nya} ‘\textsc{3poss}’, see §\ref{Para_3.1.6}).} & ‘usually’ & X & \\
\textitbf{perna} & ‘ever, once’ & X & \\
\textitbf{jarang} & `rarely’ & X & \\
\textitbf{kadang({\Tilde}kadang)} & ‘sometimes’ & X & \\
\textitbf{slalu} & ‘always’ & X & \\
\textitbf{sring} & ‘often’ & X & \\
\lspbottomrule
\end{tabular}

\end{table}


The prepredicate position of the frequency adverbs is illustrated in (\ref{Example_5.149}) to (\ref{Example_5.152}). The adverbs can directly precede the predicate, such as \textitbf{kadang{\Tilde}kadang} ‘sometimes’ in (\ref{Example_5.149}) or \textitbf{perna} ‘ever’ in (\ref{Example_5.151}), or they can precede the subject, such as \textitbf{kadang({\Tilde}kadang)} ‘sometimes’ in (\ref{Example_5.150}) or \textitbf{perna} ‘ever’ in (\ref{Example_5.152}). These examples also show that frequency adverbs not only modify verbal predicates as in (\ref{Example_5.149}) to (\ref{Example_5.151}), but also nonverbal predicates as in (\ref{Example_5.152}). The semantics conveyed by the different positions have to do with scope.


\begin{styleExampleTitle}
Frequency adverbs in clause-initial and prepredicate positions
\end{styleExampleTitle}

\ea
\label{Example_5.149}
\gll {yo,} {de} {\bluebold{kadang{\Tilde}kadang}} {terlalu,} {ini,} {egois}\\ %
 yes  \textsc{3sg}  \textsc{rdp}{\Tilde}sometimes  too  \textsc{d.prox}  be.egoistic\\
\glt 
‘yes, she’s \bluebold{sometimes} too, what’s-its-name, egoistic’ \textstyleExampleSource{[081115-001a-Cv.0218/0220]}
\z

\ea
\label{Example_5.150}
\gll {\bluebold{kadang}} {sa} {sa} {buang} {bola} {sama} {Wili} {deng} {Klara} {to?}\\ %
 sometimes  \textsc{1sg}  \textsc{1sg}  discard  ball  to  Wili  with  Klara  right?\\
\glt 
‘\bluebold{sometimes} I, I threw the ball to Wili and Klara, right?’ \textstyleExampleSource{[081006-014-Cv.0005]}
\z

\ea
\label{Example_5.151}
\gll {de} {\bluebold{perna}} {kasi} {makang} {sa} {pu} {ana}\\ %
 \textsc{3sg}  ever  give  eat  \textsc{1sg}  \textsc{poss}  child\\
\glt 
‘she \bluebold{once} fed my child’ \textstyleExampleSource{[081110-008-CvNP.0050]}
\z

\ea
\label{Example_5.152}
\gll {{\ldots}} {\bluebold{perna}} {kitong} {dua} {di} {apa} {kantor} {Golkar}\\ %
 { }  ever  \textsc{1pl}  two  at  what  office  Golkar\\
\glt
‘[so I and, what’s-his-name, Noferus here,] \bluebold{once} the two of us were at, what-is-it, the Golkar office’ \textstyleExampleSource{[080923-009-Cv.0050]}
\z


\subsection{Affirmation and {negation} adverbs}
\label{Para_5.4.3}
The \isi{affirmation} and \isi{negation} adverbs listed in \tabref{Table_5.26}  indicate general \isi{affirmation}, \isi{negation}, or prohibition, and provide responses to polar questions (see also \chapref{Para_13}).


\begin{table}
\caption{Papuan Malay \isi{affirmation} and \isi{negation} adverbs}\label{Table_5.26}

\begin{tabular}{llcc}
\lsptoprule
\multicolumn{1}{c}{Item} & \multicolumn{1}{c}{Gloss} & \multicolumn{2}{c}{Position}\\
&  & \textsc{pre-pred} &  \textsc{post-pred}\\
\midrule
\textitbf{yo} & ‘yes’ & X & \\
\textitbf{bukang} & ‘\textsc{neg}, no’ & X & \\
\textitbf{tida}/\textitbf{tra} & ‘\textsc{neg}, no’ & X & \\
\textitbf{jangang} & ‘\textsc{neg.imp}, don’t’ & X & \\
\lspbottomrule
\end{tabular}
\end{table}

The four adverbs always take a prepredicate position. Affirmative \textitbf{yo} ‘yes’ is always fronted, while the negative and \isi{prohibitive} adverbs directly precede the predicate. Affirmative \textitbf{yo} ‘yes’ is often realized as \textitbf{ya}, and negative \textitbf{jangang} ‘\textsc{neg.imp}’ is quite commonly shortened to \textitbf{jang}. Examples are provided in (\ref{Example_5.153}) to (\ref{Example_5.156}): \isi{affirmation} with \textitbf{yo} ‘yes’ in (\ref{Example_5.153}), \isi{negation} with interchangeably used \textitbf{tra} ‘\textsc{neg}’ and \textitbf{tida} ‘\textsc{neg}’ in (\ref{Example_5.154}), and with \textitbf{bukang} ‘\textsc{neg}’ (\ref{Example_5.155}), and prohibition with \textitbf{jangang} ‘\textsc{neg.imp}’ in (\ref{Example_5.156}). Negation and prohibition are discussed in more detail in §\ref{Para_13.1} and §\ref{Para_13.3.3}, respectively.


\begin{styleExampleTitle}
Affirmation and \isi{negation} adverbs: Examples
\end{styleExampleTitle}

\ea
\label{Example_5.153}
\gll {\bluebold{yo},} {tikus} {de} {loncat} {ke} {klapa} {lagi}\\ %
 yes  rat  \textsc{3sg}  jump  to  coconut  again\\
\glt 
‘\bluebold{yes}, the rat also jumped over to the coconut tree’ \textstyleExampleSource{[080917-003b-CvEx.0025]}
\z

\ea
\label{Example_5.154}
\gll {de} {\bluebold{tra}} {datang} {{\ldots}} {de} {\bluebold{tida}} {datang}\\ %
 \textsc{3sg}  \textsc{neg}  come {}   \textsc{3sg}  \textsc{neg}  come\\
\glt 
‘she did \bluebold{not} come {\ldots} she did \bluebold{not} come’ \textstyleExampleSource{[081010-001-Cv.0204-0205]}
\z

\ea
\label{Example_5.155}
\gll {saya} {\bluebold{bukang}} {anjing} {hitam}\\ %
 \textsc{1sg}  \textsc{neg}  dog  be.black\\
\glt 
‘(the situation is) \bluebold{not} (that) I am a black dog’ \textstyleExampleSource{[081115-001a-Cv.0266]}
\z

\ea
\label{Example_5.156}
\gll {Nofi} {\bluebold{jangang}} {ganggu} {kaka,} {ade} {tu,} {e?}\\ %
 Nofi  \textsc{neg.imp}  disturb  oSb  ySb  \textsc{d.dist}  e?\\
\glt
‘Nofi \bluebold{don’t} bother that older relative, younger relative, eh?’ \textstyleExampleSource{[081011-009-Cv.0013]}
\z


\subsection{Modal adverbs}
\label{Para_5.4.4}
Modal adverbs “express the subjective evaluation of the speaker toward a state of affairs” {\citep[751]{Bussmann.1996}}. This includes “epistemic” adverbs which “denote the speaker’s attitude toward the truth, certainty or probability of the state or event” and “evaluative” adverbs which express “the speaker’s \textstyleChItalic{evaluative} attitudes, i.e. judgments of \textstyleChItalic{preference} for or \textstyleChItalic{desirability} of a state or event” \citep[92–93]{Givon.2001}.



The Papuan Malay modal adverbs are presented in \tabref{Table_5.27}.
% \todo{This table looks strange to me. Last column empty, and uneven columns}
 Most of them are \isi{historically derived} by (unproductive) \isi{affixation} (for details on \isi{affixation} see §\ref{Para_3.1}). All Papuan Malay modal adverbs take a pre-pred\-icate position. Besides the adverbs listed in  \tabref{Table_5.27}, degree ad\isi{verb} \textitbf{paling} ‘most’ also has an epistemic function when it precedes the subject, as discussed in §\ref{Para_5.4.7}.


\begin{table}
\caption{Papuan Malay modal adverbs and their positions within the clause}\label{Table_5.27}

\begin{tabularx}{\textwidth}{llllllcl}
\lsptoprule
\multicolumn{2}{c}{ Item} & \multicolumn{2}{c}{ Literal} & \multicolumn{2}{c}{ Gloss} & \multicolumn{2}{c}{Position}\\
\multicolumn{2}{l}{} & \multicolumn{2}{l}{} & \multicolumn{2}{l}{} & \textsc{pre-pred} &  \textsc{post-pred}\\
\midrule
\multicolumn{8}{l}{Epistemic adverbs}\\
\midrule
& \textitbf{kata-nya} & \multicolumn{2}{l}{‘word-\textsc{3possr}’} & \multicolumn{2}{l}{‘it is being said’} & X & \\
& \textitbf{kira{\Tilde}kira} & \multicolumn{2}{l}{‘\textsc{rdp{\Tilde}}think’} & \multicolumn{2}{l}{‘probably’} & X & \\
& \textitbf{memang} & \multicolumn{2}{l}{} & \multicolumn{2}{l}{‘indeed’} & X & \\
& \textitbf{misal-nya} & \multicolumn{2}{l}{‘example-\textsc{3possr}’} & \multicolumn{2}{l}{‘for example’} & X & \\
& \textitbf{mungking} & \multicolumn{2}{l}{} & \multicolumn{2}{l}{‘maybe’} & X & \\
& \textitbf{pasti} & \multicolumn{2}{l}{} & \multicolumn{2}{l}{‘definitely’} & X & \\
& \textitbf{pokok-nya} & \multicolumn{2}{l}{‘main-\textsc{3possr}’} & \multicolumn{2}{l}{‘the main thing is’} & X & \\
& \textitbf{sebenar-nya} & \multicolumn{2}{l}{‘one:be.true-\textsc{3possr}’} & \multicolumn{2}{l}{‘actually’} & X & \\
& \textitbf{sperti-nya} & \multicolumn{2}{l}{‘similar.to-\textsc{3possr}’} & \multicolumn{2}{l}{‘it seems’} & X & \\
& \textitbf{arti-nya} & \multicolumn{2}{l}{‘meaning-\textsc{3possr}’} & \multicolumn{2}{l}{‘that means’} & X & \\
& \textitbf{maksut-nya} & \multicolumn{2}{l}{‘purpose-\textsc{3possr}’} & \multicolumn{2}{l}{‘that is to say’} & X & \\
\midrule
\multicolumn{8}{l}{Evaluative adverbs}\\
\midrule
& \multicolumn{2}{l}{\textitbf{akir-nya}} & \multicolumn{2}{l}{‘end-\textsc{3possr}’} & ‘finally’ & X & \\
& \multicolumn{2}{l}{\textitbf{coba}} & \multicolumn{2}{l}{‘try’} & ‘if only’ & X & \\
& \multicolumn{2}{l}{\textitbf{harus-nya}} & \multicolumn{2}{l}{‘have.to-\textsc{3possr}’} & ‘appropriately’ & X & \\
& \multicolumn{2}{l}{\textitbf{muda{\Tilde}muda-ang}} & \multicolumn{2}{l}{‘\textsc{rdp}{\Tilde}be.easy-\textsc{pat}’} & ‘hopefully’ & X & \\
& \multicolumn{2}{l}{\textitbf{taw{\Tilde}taw}} & \multicolumn{2}{l}{‘\textsc{rdp}{\Tilde}know’} & ‘suddenly’ & X & \\
\lspbottomrule
\end{tabularx}
\end{table}

The prepredicate position of the modal adverbs is demonstrated in (\ref{Example_5.157}) to (\ref{Example_5.160}). Typically, they precede the subject. This is illustrated with epistemic \textitbf{memang} ‘indeed’ in (\ref{Example_5.157}) and \textitbf{pasti} ‘definitely’ in (\ref{Example_5.158}), and with evaluative \textitbf{akirnya} ‘finally’ in (\ref{Example_5.159}) and \textitbf{taw{\Tilde}taw} ‘suddenly’ in (\ref{Example_5.160}). Functioning at clause level, the epistemic adverbs introduce propositions which offer explanations and clarifications for the events depicted in the preceding discourse, while the evaluative adverbs provide an evaluation of the events described in the preceding discourse.


\begin{styleExampleTitle}
Modal adverbs in prepredicate position preceding the subject
\end{styleExampleTitle}

\ea
\label{Example_5.157}
\gll {kas} {tinggal,} {\bluebold{memang}} {de} {nakal}\\ %
 give  stay  indeed  \textsc{3sg}  be.mischievous\\
\glt 
‘let it be, \bluebold{indeed}, he is mischievous’ \textstyleExampleSource{[081015-001-Cv.0027]}
\z

\ea
\label{Example_5.158}
\gll {\bluebold{pasti}} {de} {pulang}\\ %
 definitely  \textsc{3sg}  go.home\\
\glt 
‘\bluebold{certainly}, she’ll come home’ \textstyleExampleSource{[081006-019-Cv.0010]}
\z

\ea
\label{Example_5.159}
\gll {\bluebold{akirnya}} {asap{\Tilde}asap} {naik,} {langsung} {api} {menyala}\\ %
 finally  \textsc{rdp}{\Tilde}smoke  ascend  immediately  fire  flame\\
\glt 
‘\bluebold{finally} smoke ascended, immediately the fire flared up’ \textstyleExampleSource{[080922-010a-CvNF.0079]}
\z

\ea
\label{Example_5.160}
\gll {\bluebold{taw{\Tilde}taw}} {orang} {itu} {tida} {keliatang}\\ %
 \textsc{rdp}{\Tilde}know  person  \textsc{d.dist}  \textsc{neg}  be.visible\\
\glt 
‘\bluebold{suddenly} that person wasn’t visible (any longer)’ \textstyleExampleSource{[080922-002-Cv.0123]}
\z


While evaluative modal adverbs always precede the subject, most epistemic adverbs can take two prepredicate positions. Besides preceding the subject, as in (\ref{Example_5.157}) and (\ref{Example_5.158}), they can also directly precede the predicate. The exceptions are \textitbf{artinya} ‘that means’ and \textitbf{maksutnya} ‘that is to say’, both of which always precede the subject. This position directly preceding the predicate is illustrated with \textitbf{memang} ‘indeed’ in (\ref{Example_5.161}) and with \textitbf{pasti} ‘definitely’ in (\ref{Example_5.162}) (compare both examples with the examples in (\ref{Example_5.157}) and (\ref{Example_5.158}), respectively). Both examples also show that modal adverbs not only occur in verbal clauses as in (\ref{Example_5.162}), but also in nonverbal clauses, as in (\ref{Example_5.161}).


\begin{styleExampleTitle}
Modal adverbs in prepredicate position preceding the predicate
\end{styleExampleTitle}

\ea
\label{Example_5.161}
\gll {jangang} {ko} {singgung,} {tapi} {ini} {\bluebold{memang}} {bukti}\\ %
 \textsc{neg.imp}  \textsc{2sg}  offend  but  this  indeed  proof\\
\glt 
[About problems with the local elections:] ‘don’t feel offended but this is \bluebold{indeed} the proof’ \textstyleExampleSource{[081011-024-Cv.0150]}
\z

\ea
\label{Example_5.162}
\gll {{\ldots}} {tapi} {de} {\bluebold{pasti}} {kasi} {swara}\\ %
 { }  but  \textsc{3sg}  definitely  give  voice\\
\glt
[About meeting strangers in remote areas:] ‘[most likely, he/she won’t know your name yet,] but he/she’ll \bluebold{definitely} call (you)’ \textstyleExampleSource{[080919-004-NP.0078]}
\z


\subsection{Temporal adverbs}
\label{Para_5.4.5}
Temporal adverbs designate temporal points \citep[91–92]{Givon.2001}. Thereby they differ from aspectual adverbs which provide temporal information about the event or state denoted by the \isi{verb} in terms of their completion or duration \citep[5094]{Asher.1994}. The Papuan Malay temporal adverbs are listed in \tabref{Table_5.28}. Within the clause, almost all of them occur in prepredicate or in postpredicate position. The exceptions are \textitbf{baru} ‘recently’ and \textitbf{baru}{\Tilde}\textitbf{baru} ‘just now’ which only occur in prepredicate position.\footnote{Three of the adverbs listed in  \tabref{Table_5.28}  have dual \isi{word class membership} with \isi{monovalent} stative verbs: \textitbf{baru} ‘recently’, \textitbf{dulu} ‘be prior’, and \textitbf{skarang} ‘now’ (\isi{variation} in \isi{word class membership} is discussed in §\ref{Para_5.14}).}


\begin{table}
\caption{Temporal adverbs and their positions within the clause}\label{Table_5.28}

\begin{tabular}{llcc}
\lsptoprule
 \multicolumn{1}{c}{Item} & \multicolumn{1}{c}{Gloss} & \multicolumn{2}{c}{Position}\\
&  & \textsc{pre-pred} &  \textsc{post-pred}\\
\midrule

\textitbf{dulu} & ‘first, in the past’ & X &  X\\
\textitbf{lama{\Tilde}lama} & ‘gradually’ & X &  X\\
\textitbf{langsung} & ‘immediately’ & X &  X\\
\textitbf{nanti} & ‘very soon’ & X &  X\\
\textitbf{sebentar} & ‘in/for a moment’ & X &  X\\
\textitbf{skarang} & ‘now’ & X &  X\\
\textitbf{tadi} & ‘earlier’ & X &  X\\
\textitbf{baru} & ‘recently’ & X & \\
\textitbf{baru{\Tilde}baru} & ‘just now’ & X & \\
\lspbottomrule
\end{tabular}
\end{table}

Examples for the prepredicate position are given in (\ref{Example_5.163}) to (\ref{Example_5.166}), and for the postpredicate position in (\ref{Example_5.167}) to (\ref{Example_5.170}). The different meaning aspects conveyed by both positions are discussed in connection with the examples in (\ref{Example_5.172}) to (\ref{Example_5.174}).

In prepredicate position, the adverbs can directly precede the predicate, such as \textitbf{langsung} ‘immediately’ in (\ref{Example_5.163}) and \textitbf{nanti} ‘very soon’ in (\ref{Example_5.165}), or precede the subject, such as \textitbf{langsung} ‘immediately’ in (\ref{Example_5.164}) and \textitbf{nanti} ‘very soon’ in (\ref{Example_5.166}).


\begin{styleExampleTitle}
Temporal adverbs in prepredicate position
\end{styleExampleTitle}

\ea
\label{Example_5.163}
\gll {de} {\bluebold{langsung}} {ke} {asrama} {polisi} {cari} {bapa}\\ %
 \textsc{3sg}  immediately  to  dormitory  police  search  father\\
\glt 
‘he (went) \bluebold{immediately} to the police dormitory to look for father’ \textstyleExampleSource{[081011-022-Cv.0242]}
\z

\ea
\label{Example_5.164}
\gll {wa,} {ko} {datang,} {\bluebold{langsung}} {ko} {lapar?}\\ %
 wow!  \textsc{2sg}  come  immediately  \textsc{2sg}  be.hungry\\
\glt 
‘wow!, you come (here, and) \bluebold{immediately} you’re hungry?’ \textstyleExampleSource{[081110-002-Cv.0049]}
\z

\ea
\label{Example_5.165}
\gll {{\ldots}} {dang} {ko} {\bluebold{nanti}} {kena} {picaang}\\ %
   { } and  \textsc{2sg}  very.soon  hit  splinter\\
\glt 
‘[don’t (go down to the beach, (it’s) dirty,] and \bluebold{later} you’ll run into broken glass and cans’ \textstyleExampleSource{[080917-004-CvHt.0002]}
\z

\ea
\label{Example_5.166}
\gll {{\bluebold{nanti}}} {{bapa}} {mo} {brangkat,} {\bluebold{nanti}} {bapa} {kas} {taw}\\ %
 {very.soon}  {father}  want  leave  very.soon  father  give  know\\
 \gll  sama  {bapa-ade}  {pendeta}\\
 with  {uncle}  {pastor}\\
\glt 
‘\bluebold{very soon} I (‘father’) will leave (and) \bluebold{then} I (‘father’) will inform uncle pastor’ \textstyleExampleSource{[080922-001a-CvPh.0339]}
\z


The postpredicate position is illustrated in (\ref{Example_5.167}) to (\ref{Example_5.170}). In clauses with peripheral adjuncts, the ad\isi{verb} follows the predicate and precedes the adjunct, such as \textitbf{nanti} ‘very soon’ in the elicited example in (\ref{Example_5.167}) and \textitbf{langsung} ‘immediately’ in (\ref{Example_5.169}). Clauses, in which the temporal ad\isi{verb} follows the peripheral adjunct are either ungrammatical, such as \textitbf{nanti} ‘very soon’ in the elicited examples in (\ref{Example_5.168}), or only marginally grammatical such as \textitbf{langsung} ‘immediately’ in the elicited contrastive examples in (\ref{Example_5.170}).


\begin{styleExampleTitle}
Temporal adverbs in postpredicate position
\end{styleExampleTitle}

\ea
\label{Example_5.167}
\gll {tong} {pergi} {\bluebold{nanti}} {ke} {Sarmi}\\ %
 \textsc{1pl}  go  very.soon  to  \ili{Sarmi}\\
\glt 
‘we’ll go \bluebold{very soon}’ to \ili{Sarmi}’ \textstyleExampleSource{[Elicited MY131113.001]}
\z

\ea
\label{Example_5.168}
\gll {*} {tong} {pergi} {ke} {Sarmi} {\bluebold{nanti}}\\ %
  { }  \textsc{1pl}  come  to  \ili{Sarmi}  very.soon\\
\glt 
Intended reading: ‘we’ll go to \ili{Sarmi} \bluebold{very soon}’ \textstyleExampleSource{[Elicited MY131113.002]}
\z

\ea
\label{Example_5.169}
\gll {{\ldots}} {tak!,} {masuk} {\bluebold{langsung}} {di} {bawa} {meja} {sana}\\ %
  { }    bang!  enter  immediately  at  bottom  table  \textsc{l.dist}\\
\glt 
[About a small boy who had a collision with an evil spirit:] ‘whump!, \bluebold{immediately} (the kid) went under the table over there’ \textstyleExampleSource{[081025-009b-Cv.0029]}
\z

\ea
\label{Example_5.170}
\gll {??} {{\ldots}} {tak!,} {masuk} {di} {bawa} {meja} {sana} {\bluebold{langsung}}\\ %
  { } { }    bang!  enter  at  bottom  table  \textsc{l.dist}  immediately\\
\glt 
Intended reading: ‘whump!, (the kid) went under the table over there \bluebold{immediately}’ \textstyleExampleSource{[Elicited MY131113.003]}
\z


The meaning aspects conveyed by the different positions of the temporal adverbs have to do with scope. This is demonstrated with \textitbf{langsung} ‘immediately’ in three (near) contrastive examples: the prepredicate position following the subject is shown in (\ref{Example_5.171}), the prepredicate position preceding the subject in (\ref{Example_5.172}), and the postpredicate position in (\ref{Example_5.173}).


\begin{styleExampleTitle}
Positions and scope of temporal adverbs
\end{styleExampleTitle}

\ea
\label{Example_5.171}
\gll {bapa} {\bluebold{langsung}} {diam}\\ %
 father  immediately  be.quiet\\
\glt 
‘the gentleman was \bluebold{immediately} quiet’ \textstyleExampleSource{[080917-010-CvEx.0186]}
\z

\ea
\label{Example_5.172}
\gll {\bluebold{langsung}} {dong} {diam}\\ %
 immediately  \textsc{3pl}  be.quiet\\
\glt 
‘\bluebold{immediately} they were quiet’ \textstyleExampleSource{[080922-003-Cv.0085]}
\z

\ea
\label{Example_5.173}
\gll {bapa} {de} {diam} {\bluebold{langsung}}\\ %
 father  \textsc{3sg}  be.quiet  immediately\\
\glt 
‘the gentleman was quiet \bluebold{immediately}’ \textstyleExampleSource{[080917-010-CvEx.0191]}
\z


Only one temporal ad\isi{verb} has clear distinct meanings depending on its positions, namely \textitbf{dulu} ‘first, in the past’. Prepredicate \textitbf{dulu} translates with ‘in the past’, whereas postpredicate \textitbf{dulu} translates with ‘first’, as shown in (\ref{Example_5.174}).


\begin{styleExampleTitle}
Temporal \textitbf{dulu} ‘first, in the past’ in clause-initial and postpredicate positions
\end{styleExampleTitle}

\ea
\label{Example_5.174}
\gll {\bluebold{dulu}} {kitong} {pu} {{orang-tua}} {itu} {tida} {bisa}\\ %
 first  \textsc{1pl}  \textsc{poss}  {parent}  \textsc{d.dist}  \textsc{neg}  be.able\\
 \gll  {berhubungang}  {\bluebold{dulu}}\\
 {have.sexual.intercourse}  {first}\\
\glt 
‘\bluebold{in the past} our parents couldn’t have sex \bluebold{first} (before getting married)’ \textstyleExampleSource{[081110-006-CvEx.0012]}
\z


Temporal \textitbf{baru} ‘recently’ and \textitbf{baru{\Tilde}baru} ‘just now’ only occur in prepredic\-ate position, as in (\ref{Example_5.175}) and (\ref{Example_5.176}). While \textitbf{baru} ‘recently’ directly precedes the predicate, \textitbf{baru{\Tilde}baru} ‘just now’ precedes the subject.


\begin{styleExampleTitle}
Temporal \textitbf{baru} ‘recently’ and \textitbf{baru{\Tilde}baru} ‘just now’ in prepredicate position only
\end{styleExampleTitle}

\ea
\label{Example_5.175}
\gll {kariawang} {dong} {\bluebold{baru}} {lewat}\\ %
 employee  \textsc{3pl}  recently  pass.by\\
\glt 
‘the employees \bluebold{recently} walked by’ \textstyleExampleSource{[080922-001a-CvPh.0830]}
\z

\ea
\label{Example_5.176}
\gll {\bluebold{baru{\Tilde}baru}} {de} {masuk} {ruma-sakit}\\ %
 \textsc{rdp}{\Tilde}recently  \textsc{3sg}  enter  hospital\\
\glt
‘\bluebold{just now}, he got into hospital’ \textstyleExampleSource{[081115-001a-Cv.0070]}
\z


\subsection{Focus adverbs}
\label{Para_5.4.6}
Focus adverbs indicate “an accentual peak or stress which is used to contrast or compare [{\ldots} an] item either explicitly or implicitly with a set of alternatives” \citep[52]{Hoeksema.1991}. That is, focus adverbs highlight information and signal some kind of restriction, thereby adding emphasis to an utterance. Hence, they are also known as “emphatic” adverbs \citep[94]{Givon.2001}. In Papuan Malay, almost all focus adverbs take a prepredicate position, as shown in \tabref{Table_5.29}. The exceptions are \textitbf{juga} ‘also’, \textitbf{lagi} ‘again, also’, and \textitbf{saja} ‘just’ which take a postpredicate position. While the latter two only occur in postpredicate position, \textitbf{juga} ‘also’ also takes a prepredicate position.


\begin{table}[b]

\caption[Focus adverbs and their positions within the clause]{Focus adverbs and their positions within the clause}\label{Table_5.29}

\begin{tabular}{llcc}
\lsptoprule
 \multicolumn{1}{c}{Item} & \multicolumn{1}{c}{Gloss} & \multicolumn{2}{c}{Position}\\
&  & \textsc{pre-pred} &  \textsc{post-pred}\\
\midrule
\textitbf{apalagi} & ‘moreover’ & X & \\
\textitbf{kecuali} & ‘except’ & X & \\
\textitbf{kususnya}\footnote{The ad\isi{verb} \textitbf{kususnya} ‘especially’ is \isi{historically derived}: \textitbf{kusus-nya} ‘be.special-\textsc{3poss}’ (for details on suffixation with \textitbf{{}-nya} ‘\textsc{3poss}’, see §\ref{Para_3.1.6}).} & ‘especially’ & X & \\
\textitbf{cuma} & ‘just’ & X & \\
\textitbf{hanya} & ‘only’ & X & \\
\textitbf{justru} & ‘precisely’ & X & \\
\textitbf{mala} & ‘instead’ & X & \\
\textitbf{pas}\footnote{The focus ad\isi{verb} \textitbf{pas} ‘precisely’ has dual \isi{word class membership} with the \isi{monovalent} stative \isi{verb} \textitbf{pas} ‘be exact’ (\isi{variation} in \isi{word class membership} is discussed in §\ref{Para_5.14}).} & ‘precisely’ & X & \\
\textitbf{juga} & ‘also’ & X &  X\\
\textitbf{lagi} & ‘again, also’ &  &  X\\
\textitbf{saja} & ‘just’ &  &  X\\
\lspbottomrule
\end{tabular}


\end{table}

The prepredicate position of the focus adverbs is illustrated in (\ref{Example_5.177}) to (\ref{Example_5.181}). Focus adverbs typically precede the subject. This is shown with \textitbf{cuma} ‘just’ in (\ref{Example_5.177}) and \textitbf{hanya} ‘only’ in (\ref{Example_5.179}). Most of them can also take a prepredicate position directly preceding the predicate; the exceptions are \textitbf{apalagi} ‘moreover’, \textitbf{kecuali} ‘except’ and \textitbf{kususnya} ‘especially’ which always precede the subject. The position directly preceding the predicate is shown with \textitbf{cuma} ‘just’ in (\ref{Example_5.178}) and \textitbf{hanya} ‘only’ in (\ref{Example_5.180}), respectively. Another exception is prepredicate \textitbf{juga} ‘also’, which always directly precedes the predicate, as in (\ref{Example_5.181}); for its postpredi\-cate uses see (\ref{Example_5.182}). These examples also illustrate that focus adverbs not only modify verbal predicates, as in (\ref{Example_5.177}), and (\ref{Example_5.179}) to (\ref{Example_5.181}), but also nonverbal predicates, such as the \isi{numeral} predicate \textitbf{dua} ‘two’ in (\ref{Example_5.178}).


\begin{styleExampleTitle}
Focus adverbs in clause-initial and prepredicate positions
\end{styleExampleTitle}

\ea
\label{Example_5.177}
\gll {baru{\Tilde}baru} {de} {su} {turung,} {\bluebold{cuma}} {de} {su} {pulang}\\ %
 \textsc{rdp}{\Tilde}recently  \textsc{3sg}  already  descend  just  \textsc{3sg}  already  go.home\\
\glt 
[Reply to an interlocutor who is looking for someone:] ‘just now he already came by, (it’s) \bluebold{just} (that) he already went home’ \textstyleExampleSource{[080922-001a-CvPh.0554]}
\z

\ea
\label{Example_5.178}
\gll {{\ldots}} {tapi} {[sa} {pu} {alpa]} {\bluebold{cuma}} {dua}\\ %
 { }  but  \textsc{1sg}  \textsc{poss}  be.absent  just  two\\
\glt 
[About unexcused school absences:] ‘[I was absent many times,] but I had just two (official) absences’ (Lit. ‘my being absent was \bluebold{just} two’) \textstyleExampleSource{[081023-004-Cv.0014]}
\z

\ea
\label{Example_5.179}
\gll {jadi} {{kalo}} {nika} {di} {kantor} {itu} {begitu,} {\bluebold{hanya}} {dong} {bilang}\\ %
 so  {if}  marry  at  office  \textsc{d.dist}  like.that  only  \textsc{3pl}  say\\
 \gll  {nika}  {sipil}\\
 {marry}  {be.civil}\\
\glt 
[About marrying civically:] ‘so if (one) marries at the office it’s like that, \bluebold{only} (that) they call (it) ``marrying civically''' \textstyleExampleSource{[081110-007-CvPr.0030]}
\z

\ea
\label{Example_5.180}
\gll {prempuang} {\bluebold{hanya}} {duduk} {makang} {pinang} {saja}\\ %
 woman  only  sit  eat  betel.nut  just\\
\glt 
‘the girls \bluebold{just} sit (around) and eat betel nut’ \textstyleExampleSource{[081014-007-CvEx.0045]}
\z

\ea
\label{Example_5.181}
\gll {Ise} {dong} {\bluebold{juga}} {duduk} {di} {sana}\\ %
 Ise  \textsc{3pl}  also  sit  at  \textsc{l.dist}\\
\glt 
‘Ise and the others are \bluebold{also} sitting over there’ \textstyleExampleSource{[081025-009b-Cv.0075]}
\z


Three focus adverbs take a postpredicate position, namely \textitbf{juga} ‘also’, \textitbf{lagi} ‘again, also’, and \textitbf{saja} ‘just’. This is demonstrated with the examples in (\ref{Example_5.182}) to (\ref{Example_5.184}). (As shown in (\ref{Example_5.181}), \textitbf{juga} ‘also’ can also take a prepredicate position.) In clauses with peripheral adjuncts, the three adverbs can directly follow the predicate, such as the first \textitbf{juga} ‘also’ token in (\ref{Example_5.182}) and \textitbf{lagi} ‘again, also’ in (\ref{Example_5.183}). Alternatively, they can follow the adjunct, such as the second \textitbf{juga} ‘also’ token in (\ref{Example_5.182}) and \textitbf{lagi} ‘again, also’ in (\ref{Example_5.184}). Focus ad\isi{verb} \textitbf{saja} ‘just’ has the same distributional properties as \textitbf{lagi} ‘again, also’. The semantics expressed with the different positions again have to do with scope.


\begin{styleExampleTitle}
Focus adverbs in postpredicate position
\end{styleExampleTitle}

\ea
\label{Example_5.182}
\gll {{dari}} {{sini}} {{deng}} {{Papua-Lima,}} {{kembali}} {{\bluebold{juga}}} {deng} {{Papua-Lima}}\\ %
 {from}  {\textsc{l.prox}}  {with}  {Papua-Lima}  {return}  {also}  with  {Papua-Lima}\\
 \gll {\ldots}  ke  sana  {deng}  {Papua-Lima}  {kembali}  {deng}  {Papua-Lima}  \bluebold{juga}\\
 { }   to  \textsc{l.dist}  {with}  {Papua-Lima}  {return}  {with}  {Papua-Lima}  also\\
\glt 
‘(I’ll leave) from here with the Papua-Lima (ship) and return \bluebold{also} with the Papua-Lima (ship) {\ldots} (I’ll get) over there with the Papua-Lima (ship and) return with the Papua-Lima (ship) \bluebold{also}’ \textstyleExampleSource{[080922-001a-CvPh.0483/0493]}
\z

\ea
\label{Example_5.183}
\gll {de} {kembali} {\bluebold{lagi}} {ke} {Papua}\\ %
 \textsc{3sg}  return  again  to  Papua\\
\glt 
‘he came back \bluebold{again} to Papua’ \textstyleExampleSource{[081025-004-Cv.0008]}
\z

\ea
\label{Example_5.184}
\gll {sa} {pulang} {ke} {Waim} {\bluebold{lagi}}\\ %
 \textsc{1sg}  go.home  to  Waim  again\\
\glt
‘I went home to Waim \bluebold{again}’ \textstyleExampleSource{[081015-005-NP.0051]}
\z


\subsection{Degree adverbs}
\label{Para_5.4.7}
Degree adverbs “describe the extent of a characteristic”, that is, they “emphasize that a characteristic is either greater or less than some typical level” \citep[209]{Biber.2002}. Amplifiers or intensifiers “increase \isi{intensity}”, while diminishers or downtoners “decrease the effect of the modified item” (\citeyear*[209–210]{Biber.2002}).




The Papuan Malay degree adverbs are presented in \tabref{Table_5.30}. The table includes four amplifiers/intensifiers and two diminishers/downtoners. Most of the adverbs occur in prepredicate position. The exception is \textitbf{skali} ‘very’, which takes a postpredicate position. Two of the amplifiers modify gradable verbs, namely \textitbf{lebi} ‘more’ and \textitbf{paling} ‘most’. The former signals comparative degree while the latter marks superlative degree.


\begin{table}
\caption{Degree adverbs and their positions within the clause}\label{Table_5.30}

\begin{tabular}{llcc}
\lsptoprule
\multicolumn{1}{c}{Item} & \multicolumn{1}{c}{Gloss} & \multicolumn{2}{c}{Position}\\
\multicolumn{2}{l}{} & \textsc{pre-pred} &  \textsc{post-pred}\\
\midrule

\multicolumn{4}{c}{Amplifiers/intensifiers}\\
\midrule
\textitbf{lebi} & ‘more’ & X & \\
\textitbf{paling} & ‘most’ & X & \\
\textitbf{terlalu} & ‘too’ & X & \\
\textitbf{skali} & ‘very’ &  &  X\\
\midrule
\multicolumn{4}{c}{Diminishers/downtoners}\\
\midrule
\textitbf{agak} & ‘rather’ & X & \\
\textitbf{hampir} & ‘almost’ & X & \\
\lspbottomrule
\end{tabular}
\end{table}

The four amplifiers modify \isi{monovalent} stative and \isi{bivalent} verbs, as discussed in §\ref{Para_5.3.4} and §\ref{Para_5.3.5} (comparative constructions are discussed in §\ref{Para_11.5}). The amplifiers occur in prepredicate position, following the subject, such as \textitbf{paling} ‘most’ in (\ref{Example_5.185}). Furthermore, \textitbf{paling} ‘most’ can precede the subject, although not very often. In this clause-initial position it functions as an epistemic modal ad\isi{verb} which has scope over the entire proposition, as in (\ref{Example_5.186}) (modal adverbs are discussed in §\ref{Para_5.4.4}).


\begin{styleExampleTitle}
Amplifier degree adverbs
\end{styleExampleTitle}

\ea
\label{Example_5.185}
\gll {ana} {ini} {\bluebold{paling}} {bodo}\\ %
 child  \textsc{d.prox}  most  be.stupid\\
\glt 
‘this child is \bluebold{most} stupid’ \textstyleExampleSource{[081011-005-Cv.0035]}
\z

\ea
\label{Example_5.186}
\gll {{waktu}} {{saya}} {bilang} {{sa}} {{mo}} {biking} {acara,} {\bluebold{paling}} {sa} {tra}\\ %
 {time}  {\textsc{1sg}}  say  {\textsc{1sg}}  {want}  make  ceremony  most  \textsc{1sg}  \textsc{neg}\\
\gll kerja,  {sa}  {sebagey}  {kepala}  {acara}\\
 work  {\textsc{1sg}}  {as}  {head}  {ceremony}\\
\glt 
‘when I say, I want to hold a festivity, \bluebold{most likely} I won’t (have to) work, I’ll be the head of the festivity’ \textstyleExampleSource{[080919-004-NP.0068]}
\z


The intensifier \textitbf{terlalu} ‘too’ also occurs in prepredicate position, as in (\ref{Example_5.187}). By contrast, \textitbf{skali} ‘very’ takes a postpredicate position, as illustrated in (\ref{Example_5.188}) to (\ref{Example_5.190}). In clauses with peripheral adjuncts, as in (\ref{Example_5.189}), \textitbf{skali} ‘very’ follows the predicate, such as \textitbf{enak} ‘be pleasant’ in (\ref{Example_5.189}). Clauses in which \textitbf{skali} ‘very’ follows the peripheral adjunct, as in the elicited example in (\ref{Example_5.190}), are ungrammatical.


\begin{styleExampleTitle}
Intensifier degree adverbs
\end{styleExampleTitle}

\ea
\label{Example_5.187}
\gll {a,} {ko} {\bluebold{terlalu}} {bodo}\\ %
 ah!  \textsc{2sg}  too  be.stupid\\
\glt 
‘ah, you are \bluebold{too} stupid’ \textstyleExampleSource{[080917-003a-CvEx.0009]}
\z

\ea
\label{Example_5.188}
\gll {ade} {bongso} {jadi} {ko} {sayang} {dia} {\bluebold{skali}} {e?}\\ %
 ySb  youngest.offspring  so  \textsc{2sg}  love  \textsc{3sg}  very  eh\\
\glt 
‘(your) youngest sibling, so you love her \bluebold{very much}, eh?’ \textstyleExampleSource{[080922-001a-CvPh.0302]}
\z

\ea
\label{Example_5.189}
\gll {kamu} {orang-tua} {enak} {\bluebold{skali}} {di} {sana}\\ %
 \textsc{2pl}  parent  be.pleasant  very  at  \textsc{l.dist}\\
\glt 
‘you, the parents, (have) \bluebold{very} pleasant (lives) over there’ (Lit. ‘you {\ldots} are \bluebold{very} pleasant’) \textstyleExampleSource{[081115-001a-Cv.0106]}
\z

\ea
\label{Example_5.190}
\gll {*} {kamu} {orang-tua} {enak} {di} {sana} {\bluebold{skali}}\\ %
  { }  \textsc{2pl}  parent  pleasant  at  \textsc{l.dist}  very\\
\glt 
Intended reading: ‘you, the parents, (have) \bluebold{very} pleasant (lives) over there’ \textstyleExampleSource{[Elicited MY131113.004]}
\z


The diminishers \textitbf{agak} ‘rather’ and \textitbf{hampir} ‘almost’ also occur in prepredicate position, as illustrated in (\ref{Example_5.191}) to (\ref{Example_5.194}). Always directly preceding the \isi{verb}, \textitbf{agak} ‘rather’ modifies stative verbs, as in (\ref{Example_5.191}). Clauses in which \textitbf{agak} ‘rather’ precedes the subject, as in the elicited example in (\ref{Example_5.192}), are ungrammatical. Diminisher \textitbf{hampir} ‘almost’ typically modifies dynamic verbs, as in (\ref{Example_5.193}) and (\ref{Example_5.194}).\footnote{According to one consultant, some Papuan Malay speakers also use \textitbf{hampir} ‘almost’ to modify stative verbs. Much more often though they employ a construction with \textitbf{su mulay} ‘already start to’ as in (\ref{Footnote_Example_5.1}) below:
\vspace{-5pt}
\ea
\label{Footnote_Example_5.1}
\gll {baru} {kita} {pergi} {skola,} \bluebold{suda} \bluebold{mulay} {sembu}\\
 {and.then} \textsc{1pl} {go} {school} {already} {start} {be.healed} \\
\glt [After an accident:] ‘and then we went (back) to school, (our wounds) were  \bluebold{almost} healed’ (Lit. ‘\bluebold{already started to} be healed’) [081014-012-NP.0005]\\
\z
}

The ad\isi{verb} can directly precede the predicate, as in the elicited example in (\ref{Example_5.193}), or precede the subject, as in (\ref{Example_5.194}). In the corpus, \textitbf{hampir} ‘almost’ always occurs in the latter position, where the ad\isi{verb} has scope over the entire proposition.



\begin{styleExampleTitle}
Diminisher/downtoner degree adverbs
\end{styleExampleTitle}

\ea
\label{Example_5.191}
\gll {sa} {su} {\bluebold{agak}} {besar}\\ %
 \textsc{1sg}  already  rather  be.big\\
\glt 
[About the speaker’s childhood:] ‘I was already \bluebold{rather} big’ \textstyleExampleSource{[080922-008-CvNP.0025]}
\z

\ea
\label{Example_5.192}
\gll {*} {\bluebold{agak}} {sa} {su} {besar}\\ %
 { }   rather  \textsc{1sg}  already  be.big\\
\glt 
Intended reading: ‘I was already \bluebold{rather} big’ \textstyleExampleSource{[Elicited MY131113.006]}
\z

\ea
\label{Example_5.193}
\gll {dong} {\bluebold{hampir}} {bunu} {bapa}\\ %
 \textsc{3pl}  almost  kill  father\\
\glt 
‘they \bluebold{almost} killed (my) father’ \textstyleExampleSource{[Elicited MY131113.005]}
\z

\ea
\label{Example_5.194}
\gll {\bluebold{hampir}} {dong} {bunu} {bapa}\\ %
 almost  \textsc{3pl}  kill  father\\
\glt
‘(it) \bluebold{almost} (happened that) they killed (my) father’ \textstyleExampleSource{[081011-022-Cv.0210]}
\z


\subsection{Expressing manner}
\label{Para_5.4.8}
Papuan Malay does not have manner adverbs. Instead, manner is expressed through stative verbs, as shown in (\ref{Example_5.195}) to (\ref{Example_5.200}). The modifying stative verbs always take a postpredicate position. In (\ref{Example_5.195}), for instance, postpredicate stative \textitbf{kras} ‘be harsh’ modifies stative \textitbf{sakit} ‘be sick’, and in (\ref{Example_5.198}) \textitbf{trus} ‘be continuous’ modifies \textitbf{tatap dia} ‘observe him’. In verbal clauses with peripheral adjuncts, the modifying stative \isi{verb} can directly follow the predicate as in (\ref{Example_5.199}), or follow the adjunct, as in (\ref{Example_5.200}).


\ea
\label{Example_5.195}
\gll {baru} {satu} {kali} {sa} {sakit} {\bluebold{kras}}\\ %
 and.then  one  time  \textsc{1sg}  be.sick  be.harsh\\
\glt 
‘but then one time I was \bluebold{badly} sick’ \textstyleExampleSource{[080922-008-CvNP.0009]}
\z

\ea
\label{Example_5.196}
\gll  e,  kam  mandi  \bluebold{cepat}  suda!\\
 hey!  \textsc{2pl}  bathe  be.fast  already\\
\glt 
‘hey, you bathe \bluebold{quickly}!’ \textstyleExampleSource{[080917-008-NP.0128]}
\z

\ea
\label{Example_5.197}
\gll  dong  dua  lari  \bluebold{trus}\\
 \textsc{3pl}  two  run  be.continuous\\
\glt 
[About a motorbike trip:] ‘the two of them drove \bluebold{continuously}’ \textstyleExampleSource{[081015-005-NP.0011]}
\z

\ea
\label{Example_5.198}
\gll {langsung} {sa} {tatap} {dia} {\bluebold{trus}}\\ %
 immediately  \textsc{1sg}  gaze.at  \textsc{3sg}  be.continuous\\
\glt 
‘immediately I gazed at him \bluebold{continuously}’ \textstyleExampleSource{[081006-035-CvEx.0071]}
\z

\ea
\label{Example_5.199}
\gll {de} {buka} {\bluebold{trus}} {siang} {malam}\\ %
 \textsc{3sg}  open  be.continuous  day  night\\
\glt 
[About opening hours of an office] ‘it is open \bluebold{continuously} day and night’ \textstyleExampleSource{[081005-001-Cv.0003]}
\z

\ea
\label{Example_5.200}
\gll {{\ldots}} {terendam} {di} {air} {\bluebold{trus}}\\ %
 { }  be.soaked  at  water  be.continuous\\
\glt
[About a motorbike that got stuck in a river:] ‘[(the motorbike) is still there {\ldots},] (it) is immersed in water \bluebold{continuously}’ \textstyleExampleSource{[081008-003-Cv.0026]}
\z


\subsection{Summary}
\label{Para_5.4.9}
The Papuan Malay adverbs take different positions within the clause, that is, they can occur in prepredicate or in postpredicate position. The most common position, however, is the prepredicate one. There are also a fair number of adverbs which can occur in both positions.



For the prepredicate adverbs two positions are attested, one directly preceding the predicate and one preceding the subject. A fair number of prepredicate adverbs can occur in both positions. Likewise, for the postpredicate adverbs two positions are attested, one directly following the predicate and, in clauses with peripheral adjuncts, one following the adjunct. Most postpredicate adverbs can occur in both positions. In terms of their functions, the adverbs designate aspect, frequency, \isi{affirmation} and \isi{negation}, modality, time, focus, and degree; manner is expressed through stative verbs in postpredicate position.



Listed according to their semantic functions, the adverbs have the following distributional preferences.

\begin{enumerate}
\item 
Aspect adverbs

\begin{styleIiI}
They only occur in prepredicate position, directly preceding the predicate.
\end{styleIiI}


\item 
Frequency adverbs

\begin{styleIiI}
They only occur in prepredicate position where they directly precede the predicate or the subject.
\end{styleIiI}


\item 
Affirmation and \isi{negation} adverbs

\begin{styleIiI}
They always occur in a predicate position. The \isi{affirmation} ad\isi{verb} always precedes the subject, while the three \isi{negation} adverbs directly precede the predicate.
\end{styleIiI}


\item 
Modal adverbs

\begin{styleIiI}
All epistemic and evaluative adverbs take a prepredicate position, preceding the subject. Besides, most of the epistemic adverbs can also directly precede the predicate; the exceptions are \textitbf{artinya} ‘that means’ and \textitbf{maksutnya} ‘that is to say’ which always precede the subject.
\end{styleIiI}


\item 
Temporal adverbs

\begin{styleIiI}
All but two can occur in pre- or in postpredicate position. In prepredicate position, the adverbs can directly precede the predicate or the subject. In postpredicate position, they always follow the predicate and, in clauses with peripheral adjuncts, precede the adjunct. Two adverbs only occur in prepredicate position, namely \textitbf{baru} ‘recently’ and \textitbf{baru{\Tilde}baru} ‘just now’.
\end{styleIiI}


\item 
Focus adverbs

\begin{styleIiI}
All but three only occur in prepredicate position where they can directly precede the predicate or the subject. The exceptions are \textitbf{juga} ‘also’, \textitbf{lagi} ‘again, also’, and \textitbf{saja} ‘just’, which take a postpredicate position. While \textitbf{lagi} ‘again, also’, and \textitbf{saja} ‘just’ only occur in postpredicate position, \textitbf{juga} ‘also’ also takes a prepredicate position. In postpredicate position, the three adverbs can either directly follow the predicate or, in clauses with peripheral adjuncts, follow the adjunct.
\end{styleIiI}


\item 
Degree adverbs

\begin{styleIiI}
All but one only take a prepredicate position, where most of them directly precede the predicate. The exception is \textitbf{hampir} ‘almost’ which can also precede the subject. The one degree ad\isi{verb} which is unattested in prepredicate position is \textitbf{skali} ‘very’. It only occurs in postpredicate position, directly following the predicate.
\end{styleIiI}
\end{enumerate}

These distributional preferences are summarized in \tabref{Table_5.31}.


\begin{table}
\caption{Papuan Malay adverbs and their positions within the clause}\label{Table_5.31}

\begin{tabular}{lll}
\lsptoprule
 \multicolumn{1}{c}{Ad\isi{verb} type} & \multicolumn{2}{c}{ Positions within the clause}\\
  & \textsc{pre-pred} &  \textsc{post-pred}\\
\midrule
Aspect & all \textsc{adv} & none\\
Frequency & all \textsc{adv} & none\\
Affirmation/\isi{negation} & all \textsc{adv} & none\\
Modal & all \textsc{adv} & none\\
Temporal & all \textsc{adv} & most \textsc{adv}\\
Focus & most \textsc{adv} & three \textsc{adv}\\
Degree & most \textsc{adv} & one \textsc{adv}\\
\lspbottomrule
\end{tabular}
\end{table}

As for those adverbs which can take more than one position within the clause, the semantic distinctions conveyed by the different positions have to do with scope. Overall, however, these distinctions are subtle and require further investigation.



Papuan Malay does not have manner adverbs. Instead, manner is expressed with \isi{monovalent} stative verbs which always take a postpredicate position.




\section{Personal pronouns}
\label{Para_5.5}
The Papuan Malay personal \isi{pronoun} system distinguishes singular and plural numbers and three persons; the personal pronouns do not mark case, clusivity, gender, or politeness. Referring to animate and inanimate entities, they allow the unambiguous identification of their referents. They do so by signaling not only the person-number values of their referents, but also their definiteness.

The Papuan Malay personal pronouns are presented in \tabref{Table_5.32}.


\begin{table}
\caption{Personal \isi{pronoun} system with long and short forms}\label{Table_5.32}

\begin{tabular}{lll}
\lsptoprule
 & Long forms &  Short forms\\
 \midrule
\textsc{1sg} & \textitbf{saya} & \textitbf{sa}\\
\textsc{2sg} & \textitbf{ko} & \textitbf{{}-{}-{}-}\\
\textsc{3sg} & \textitbf{dia} & \textitbf{de}\\
\textsc{1pl} & \textitbf{kitong} & \textitbf{tong}\\
& \textitbf{kita} & \textitbf{ta}\\
& \textitbf{kitorang} & \textitbf{torang}\\
\textsc{2pl} & \textitbf{kamu} & \textitbf{kam}\\
\textsc{3pl} & \textitbf{dorang} & \textitbf{dong}\\
\lspbottomrule
\end{tabular}
\end{table}

\newpage 
Each personal \isi{pronoun}, except for \textsc{2sg}, has at least one long and one short form. The use of the long and short \isi{pronoun} forms does not mark grammatical distinctions but represents speaker preferences. These distributional preferences are discussed in detail in §\ref{Para_6.1.1}.



The Papuan Malay personal pronouns have the following distributional properties:



\begin{enumerate}
\item 
Substitution for \isi{noun} phrases (pronominal uses) (§\ref{Para_6.1}).
\item 
Modification with demonstratives, locatives, numerals, quantifiers, prepositional phrases, and/or relative clauses (pronominal uses) (§\ref{Para_6.1}).
\item 
Co-occurrence with \isi{noun} phrases (adnominal uses): \textsc{n/np} \textsc{pro} (§\ref{Para_6.2})\textsc{.}

\end{enumerate}

Personal pronouns are distinct from other word classes such as nouns (§\ref{Para_5.2}) and demonstratives (§\ref{Para_5.6}) in terms of the following distributional properties:



\begin{enumerate}
\item 
Personal pronouns are distinct from nouns in that personal pronouns (a) very commonly modify nouns, while nouns do not modify personal pronouns, (b) are modified with numerals or quantifiers in posthead position, while with nouns the modifying numerals/quantifiers can also occur in prehead position, and (c) only designate the possessor in adnominal possessive constructions, while nouns can also express the possessum.

\item 
Unlike demonstratives, personal pronouns (a) express person and number, (b) signal definiteness, while demonstratives indicate specificity,\footnote{\label{Footnote_5.17}According to \citet[148]{Andrews.2007}, definiteness indicates that “an \textsc{np} has [...] a referent uniquely identifiable to the hearer”. Hence, the hearer is expected to be in a position to identify the referent. Specificity, by contrast, signals that “the speaker is referring to a particular instance of an entity as opposed to any instance of it” \citep[148]{Andrews.2007}. That is, the identifiability of the referent is not presupposed. Instead, the speaker makes the entity under discussion identifiable to the hearer by pointing out “a particular instance of an entity” among other possible referents (\citeyear*[148]{Andrews.2007}). (See also \citealt{Abbot.2006}.)\label{Footnote_5.164}} and (c) cannot be stacked.

\end{enumerate}

The personal pronouns have pronominal and adnominal uses. This is illustrated with two examples. The utterance in (\ref{Example_5.201}) demonstrates the pronominal uses of short \textitbf{sa} ‘\textsc{1sg}’ and long \textitbf{dia} ‘\textsc{3sg}’, while the example in (\ref{Example_5.202}) shows the adnominal uses of short \textitbf{dong} ‘\textsc{3pl}’. The personal pronouns are discussed in detail in \chapref{Para_6}.


\ea
\label{Example_5.201}
\gll {ana} {itu} {\bluebold{sa}} {paling} {sayang} {\bluebold{dia}} {\bluebold{tu}} {ana} {itu}\\ %
 child  \textsc{d.dist}  \textsc{1sg}  most  love  \textsc{3sg}  \textsc{d.dist}  child  \textsc{d.dist}\\
\glt 
‘that child, \bluebold{I} love \bluebold{her (}\blueboldSmallCaps{emph}\bluebold{)} most, that child’ \textstyleExampleSource{[081011-023-Cv.0097]}
\z

\ea
\label{Example_5.202}
\gll {\bluebold{Natanael}} {\bluebold{dong}} {menang}\\ %
 Natanael  \textsc{3pl}  win\\
\glt
[About a volleyball game:] ‘\bluebold{Natanael and his friends} won’ \textstyleExampleSource{[081109-001-Cv.0002]}
\z


\section{Demonstratives}
\label{Para_5.6}
Papuan Malay has a two-term \isi{demonstrative} system: proximal \textitbf{ini} ‘\textsc{d.prox}’ and distal \textitbf{itu} ‘\textsc{d.dist}’, together with their reduced fast-speech forms \textitbf{ni} ‘\textsc{d.prox}’ and \textitbf{tu} ‘\textsc{d.dist}’. As deictic expressions they orient the hearers and signal specificity. That is, they draw the hearers’ attention to particular occurrences of an entity in the surrounding situation or in the discourse. While \textitbf{ini} ‘\textsc{d.prox}’ indicates proximity of this entity, \textitbf{itu} ‘\textsc{d.dist}’ signals its distance – in spatial and in nonspatial terms.



Papuan Malay demonstratives have the following distributional properties:


\begin{enumerate}
\item 
Co-occurrence with \isi{noun} phrases (adnominal uses): \textsc{n}/\textsc{np} \textsc{dem} (§\ref{Para_5.6.1}).

\item 
Substitution for \isi{noun} phrases (pronominal uses) (§\ref{Para_5.6.2}).

\item 
Modification with relative clauses (pronominal uses (§\ref{Para_5.6.2}).

\item 
Co-occurrence with verbs or adverbs (adverbial uses) (§\ref{Para_5.6.3}).

\item 
Stacking of demonstratives: \textsc{dem} \textsc{dem} and \textsc{n} \textsc{dem} \textsc{dem} (§\ref{Para_5.6.4}).

\end{enumerate}

Demonstratives are distinct from other word classes such as personal pronouns (§\ref{Para_5.5}) and locatives (§\ref{Para_5.7}) in terms of the following syntactic properties:


%\setcounter{itemize}{0}
\begin{enumerate}
\item 
Demonstratives are distinct from personal pronouns, in that demonstratives (a) have adverbial uses, (b) can be stacked, (c) can take the possessum slot in adnominal possessive constructions, and (d) signal specificity, while personal pronouns express definiteness.\footnote{Concerning the semantic distinctions between the notion of definiteness and the notion of specificity see Footnote \ref{Footnote_5.17} in §\ref{Para_5.5} (p. \pageref{Footnote_5.17}).}

\item 
Contrasting with locatives, demonstratives (a) are employed as independent nominals in unembedded \isi{noun} phrases, (b) occur in adnominal possessive constructions either as the possessor or the possessum, and (c) can be stacked.

\end{enumerate}

The adnominal uses of the demonstratives are discussed in §\ref{Para_5.6.1}, their pronominal uses in §\ref{Para_5.6.2}, their adverbial uses in §\ref{Para_5.6.3}, and stacking of demonstratives in §\ref{Para_5.6.4}. A full discussion of the Papuan Malay demonstratives is presented in §\ref{Para_7.1}.


\subsection{Adnominal uses}
\label{Para_5.6.1}
Adnominally used demonstratives occur in posthead position at the right periphery of the \isi{noun} phrase. That is, all \isi{noun} phrase constituents occur to the left of the \isi{demonstrative}, with the \isi{demonstrative} having scope over the entire \isi{noun} phrase as illustrated in (\ref{Example_5.203}) to (\ref{Example_5.205}). Constituents occurring to the right of the demonstratives such as \textitbf{liar} ‘be wild’ in (\ref{Example_5.206}) are not part of the \isi{noun} phrase: \textitbf{liar} ‘be wild’ is a clausal predicate. The examples in (\ref{Example_5.203}) and (\ref{Example_5.204}) show that the demonstratives signal specificity (and not definiteness). The \isi{noun} phrase \textitbf{tanta dia itu} ‘that aunt’ (literally ‘that she aunt’) designates a specific and \isi{definite} referent with distal \textitbf{itu} ‘\textsc{d.dist}’ indicating specificity while adnominally used \textitbf{dia} ‘\textsc{3sg}’ signals definiteness (§\ref{Para_5.5}). By contrast, the \isi{noun} phrase \textitbf{ana kecil satu ini} ‘this particular small child’ in (\ref{Example_5.204}) denotes a specific but \isi{indefinite} referent with proximal \textitbf{ini} ‘\textsc{d.prox}’ again indicating specificity while posthead \textitbf{satu} ‘one’ signals \isi{indefiniteness} (see also §\ref{Para_5.9.4}).


\begin{styleExampleTitle}
Posthead demonstratives: Scope
\end{styleExampleTitle}

\ea
\label{Example_5.203}
\gll {Wili} {ko} {jangang} {gara{\Tilde}gara} {\bluebold{tanta}} {\bluebold{dia}} {\bluebold{itu}}!\\ %
 Wili  \textsc{2sg}  \textsc{neg.imp}  \textsc{rdp}{\Tilde}irritate  aunt  \textsc{3sg}  \textsc{d.dist}\\
\glt 
‘you Wili don’t irritate \bluebold{that aunt}!’ \textstyleExampleSource{[081023-001-Cv.0038]}
\z

\ea
\label{Example_5.204}
\gll {{baru}} {{\bluebold{ana}}} {{\bluebold{kecil}}} {{\bluebold{satu}}} {{\bluebold{ini}}} {de} {tra} {gambar}\\ %
 {and.then}  {child}  {be.small}  {one}  {\textsc{d.prox}}  \textsc{3sg}  \textsc{neg}  draw\\
\gll  \bluebold{ana}  {\bluebold{murit}}  {\bluebold{satu}}  {\bluebold{ni}}  {de}  tra  {gambar}\\
 child  {pupil}  {one}  {\textsc{d.prox}}  {\textsc{3sg}}  \textsc{neg}  {draw}\\
\glt 
‘but then \bluebold{this particular small child}, he doesn’t draw, \bluebold{this particular school kid}, he doesn’t draw’ \textstyleExampleSource{[081109-002-JR.0002]}
\z

\ea
\label{Example_5.205}
\gll {Papua-Satu} {ada} {muncul} {dari} {\bluebold{laut}} {\bluebold{sana}} {\bluebold{itu}}\\ %
 Papua-Satu  exist  appear  from  sea  \textsc{l.dist}  \textsc{d.dist}\\
\glt 
‘(the ship) Papua-Satu is appearing from \bluebold{the sea over there (}\blueboldSmallCaps{emph}\bluebold{)}’ \textstyleExampleSource{[080917-008-NP.0129]}
\z

\ea
\label{Example_5.206}
\gll {{\ldots}} {karna} {\bluebold{babi}} {\bluebold{ini}} {liar}\\ %
 { }   because  pig  \textsc{d.prox}  be.wild\\
\glt 
‘{\ldots} because \bluebold{this pig is wild}’ \textstyleExampleSource{[080919-004-NP.0019]}
\z


Demonstratives can also modify constituents other than nouns, namely personal pronouns as in (\ref{Example_5.207}), interrogatives as in (\ref{Example_5.208}), or locatives as in (\ref{Example_5.209}).


\begin{styleExampleTitle}
Posthead demonstratives: Modifying personal pronouns, interrogatives, or locatives
\end{styleExampleTitle}

\ea
\label{Example_5.207}
\gll {\bluebold{ko}} {\bluebold{itu}} {manusia} {yang} {tra} {taw} {bicara} {temang}\\ %
 \textsc{2sg}  \textsc{d.dist}  human.being  \textsc{rel}  \textsc{neg}  know  speak  friend\\
\glt 
‘\bluebold{you (}\blueboldSmallCaps{emph}\bluebold{)} are a human being who doesn’t know how to talk (badly about) friends’ \textstyleExampleSource{[081115-001a-Cv.0245]}
\z

\ea
\label{Example_5.208}
\gll {ana} {laki{\Tilde}laki} {ini} {de} {mo} {ke} {\bluebold{mana}} {\bluebold{ni}}\\ %
 child  \textsc{rdp}{\Tilde}husband  \textsc{d.prox}  \textsc{3sg}  want  to  where  \textsc{d.prox}\\
\glt 
‘this boy, \bluebold{where (}\blueboldSmallCaps{emph}\bluebold{)} does he want to (go)?’ \textstyleExampleSource{[080922-004-Cv.0017]}
\z

\ea
\label{Example_5.209}
\gll {di} {\bluebold{sini}} {\bluebold{tu}} {ada} {orang} {swanggi} {satu}\\ %
 at  \textsc{l.prox}  \textsc{d.dist}  exist  person  nocturnal.evil.spirit  one\\
\glt
‘\bluebold{here (}\blueboldSmallCaps{emph}\bluebold{)} is a certain evil sorcerer’ \textstyleExampleSource{[081006-022-CvEx.0150]}
\z

\subsection{Pronominal uses}
\label{Para_5.6.2}
In their pronominal uses, the demonstratives stand for \isi{noun} phrases, as illustrated in (\ref{Example_5.210}) to (\ref{Example_5.215}). They occur in all syntactic positions within the clause. In (\ref{Example_5.210}), a \isi{demonstrative} takes the subject slot, in (\ref{Example_5.211}) the direct object slot, and in (\ref{Example_5.212}) the oblique object slot.


\begin{styleExampleTitle}
Pronominal uses in argument position
\end{styleExampleTitle}

\ea
\label{Example_5.210}
\gll {yo,} {\bluebold{itu}} {mo} {putus}\\ %
 yes  \textsc{d.dist}  want  break\\
\glt 
[About redirecting a river for a street building project:] ‘yes, \bluebold{it} (the river) is going to get dispersed’ \textstyleExampleSource{[081006-033-Cv.0064]}
\z

\ea
\label{Example_5.211}
\gll {ko} {suka} {makang} {\bluebold{ini}}?\\ %
 \textsc{2sg}  like  eat  \textsc{d.prox}\\
\glt 
[About fried bananas:] ‘do you like to eat \bluebold{these}?’ \textstyleExampleSource{[081006-023-CvEx.0071-0072]}
\z

\ea
\label{Example_5.212}
\gll {dong} {percaya} {\bluebold{sama}} {\bluebold{itu}}\\ %
 \textsc{3pl}  trust  to  \textsc{d.dist}\\
\glt 
[About believing in evil spirits:] ‘they believe \bluebold{in those}’ \textstyleExampleSource{[081006-023-CvEx.0001]}
\z


In their pronominal uses, the demonstratives can be modified with relative clauses, as in the elicited example in (\ref{Example_5.213}).


\begin{styleExampleTitle}
Modification of pronominally used demonstratives with relative clauses
\end{styleExampleTitle}

\ea
\label{Example_5.213}
\gll {sa} {{pili}} {\bluebold{ini}} {yang} {mera,} {ade} {pili} {\bluebold{itu}} {yang}\\ %
 \textsc{1sg}  {choose}  \textsc{d.prox}  \textsc{rel}  be.red  ySb  choose  \textsc{d.dist}  \textsc{rel}\\
 \gll {warna}  {puti}\\
 {color}  {be.white}\\
\glt 
[About buying new shirts:] ‘I chose \bluebold{this} (one) which is red, (my) younger sibling chose \bluebold{that} (one) which is (of) white color’ \textstyleExampleSource{[Elicited MY131119.004]}
\z


Pronominally used demonstratives also occur in adnominal possessive constructions (see \chapref{Para_9}). They can designate the possessor as in (\ref{Example_5.214}) or the possessum as in (\ref{Example_5.215}).


\begin{styleExampleTitle}
Pronominal uses in adnominal possessive constructions
\end{styleExampleTitle}

\ea
\label{Example_5.214}
\gll {bapa} {masi} {kenal} {{\ldots}} {\bluebold{ini}} {\bluebold{pu}} {\bluebold{muka}}?\\ %
 father  still  know  {}  \textsc{d.prox}  \textsc{poss}  face\\
\glt 
‘do you (‘father’) still know {\ldots} \bluebold{this (one)’s face}?’ \textstyleExampleSource{[080922-001a-CvPh.1123]}
\z

\ea
\label{Example_5.215}
\gll {ko} {ambil} {dulu} {\bluebold{ade}} {\bluebold{pu}} {\bluebold{itu}}\\ %
 \textsc{2sg}  fetch  first  ySb  \textsc{poss}  \textsc{d.dist}\\
\glt
‘you pick up (the fish) first, \bluebold{that (fish) of the younger sister}’ (Lit. ‘\bluebold{younger sibling’s that}’) \textstyleExampleSource{[081006-019-Cv.0002]}
\z


\subsection{Adverbial uses}
\label{Para_5.6.3}
In their adverbial uses, the demonstratives co-occur with verbs as in \textitbf{pikir ni} ‘think (\textsc{emph})’ in (\ref{Example_5.216}) or with adverbs as in \textitbf{pasti tu} ‘exactly (\textsc{emph})’ in (\ref{Example_5.217}).


\ea
\label{Example_5.216}
\gll {de} {\bluebold{pikir}} {\bluebold{ni},} {dong} {ribut} {apa} {ka}\\ %
 \textsc{3sg}  think  \textsc{d.prox}  \textsc{3pl}  trouble  what  or\\
\glt 
‘he \bluebold{thought} \bluebold{(}\blueboldSmallCaps{emph}\bluebold{)}, ``what are they troubled about?''' \textstyleExampleSource{[081014-005-Cv.0036]}
\z

\ea
\label{Example_5.217}
\gll {yo,} {brita} {\bluebold{pasti}} {\bluebold{tu}} {yang} {sa} {bilang}\\ %
 yes  new  definitely  \textsc{d.dist}  \textsc{rel}  \textsc{1sg}  say\\
\glt
‘yes, the news are \bluebold{exactly} \bluebold{(}\blueboldSmallCaps{emph}\bluebold{)} as I told (you)’ \textstyleExampleSource{[080922-001a-CvPh.0767]}
\z


\subsection{Stacking of demonstratives}
\label{Para_5.6.4}
Papuan Malay also allows the stacking of demonstratives. Typically, only identical demonstratives are stacked, as in (\ref{Example_5.218}) and (\ref{Example_5.219}). To some degree, however, non-identical demonstratives can also be stacked, as shown in (\ref{Example_5.220}) and (\ref{Example_5.221}).\footnote{Juxtaposed \textitbf{ini ni} ‘\textsc{d.prox d.prox}’ and \textitbf{itu tu} ‘\textsc{d.dist d.dist}’ are not taken as instances of partial \isi{reduplication}. As discussed in §\ref{Para_4.1.2}, partial \isi{reduplication} of the stems \textitbf{ini} ‘\textsc{d.prox}’ and \textitbf{itu} ‘\textsc{d.dist}’ should result in the reduplicated forms \textitbf{in{\Tilde}ini} ‘\textsc{d.prox{\Tilde}d.prox}’ and \textitbf{it{\Tilde}itu} ‘\textsc{d.dist}{\Tilde}\textsc{d.dist}’, respectively. Therefore, \textitbf{ini ni} ‘\textsc{d.prox d.prox}’ and \textitbf{itu tu} ‘\textsc{d.dist} \textsc{d.dist}’ are taken as instances of \isi{demonstrative} stacking. Reduplication of demonstratives does occur, however, as discussed in §\ref{Para_4.2.5.2}.}



In (\ref{Example_5.218}), short proximal \textitbf{ni} ‘\textsc{d.prox}’ modifies the pronominally used long proximal \isi{demonstrative}, such that ``\textsc{dem} \textsc{dem}''. In (\ref{Example_5.219}), short distal \textitbf{tu} ‘\textsc{d.dist}’ modifies a nested \isi{noun} phrase with the adnominally used long distal \isi{demonstrative}, such that ``[[\textsc{n} \textsc{dem}] \textsc{dem}]''.


\ea
\label{Example_5.218}
\gll {ada} {segala} {macang} {tulang} {dia} {buang} {[\bluebold{ini}} {\bluebold{ni}]}\\ %
 exist  all  variety  bone  \textsc{3sg}  throw(.away)  \textsc{d.prox}  \textsc{d.prox}\\
\glt 
‘there were all kinds of bones, he threw away \bluebold{these very (ones)}’ \textstyleExampleSource{[080922-010a-CvNF.0101]}
\z

\ea
\label{Example_5.219}
\gll {waktu} {kitorang} {masuk} {di} {[[\bluebold{ruma}} {\bluebold{itu}]} {\bluebold{tu}]} {{\ldots}}\\ %
 when  \textsc{1pl}  go.in  at  house  \textsc{d.dist}  \textsc{d.dist}  \\
\glt 
‘when we moved into \bluebold{that very house}, {\ldots}’ \textstyleExampleSource{[081006-022-CvEx.0167]}
\z



While unattested in the corpus, speakers do allow one combination of non-identical \isi{demonstrative} stacking in elicitation. Acceptable is the order of proximal \textitbf{ini} ‘\textsc{d.prox}’ followed by short distal \textitbf{itu} ‘\textsc{d.dist}’, as shown in the elicited example in (\ref{Example_5.220}). The reverse order is not permitted by speakers even in elicitation, as illustrated in (\ref{Example_5.221}). At this point in the research on Papuan Malay, however, the semantics of ``\textsc{n} \textitbf{ini tu}'' constructions as compared to ``\textsc{n} \textitbf{ini ni}''' and ``\textsc{n} \textitbf{itu tu}'' constructions remain uncertain.


\begin{styleExampleTitle}
Non-identical stacked demonstratives\footnote{The elicited examples are based on the example in (\ref{Example_7.23}) in §\ref{Para_7.1.2.3} (p. \pageref{Example_7.23}).}
\end{styleExampleTitle}

\ea
\label{Example_5.220}
\gll {\bluebold{orang}} {\bluebold{ini}} {\bluebold{tu}} {percaya} {sama} {Tuhang} {Yesus}\\ %
 person  \textsc{d.prox}  \textsc{d.dist}  trust  to  God  Jesus\\
\glt 
‘\bluebold{that person here} believes in God Jesus’ \textstyleExampleSource{[Elicited BR111017.009]}
\z

\ea
\label{Example_5.221}
\gll {*} {\bluebold{orang}} {\bluebold{itu}} {\bluebold{ni}} {percaya} {sama} {Tuhang} {Yesus}\\ %
 { }   person  \textsc{d.dist}  \textsc{d.prox}  trust  to  God  Jesus\\
\glt
Intended reading: ‘\bluebold{this person there} believes in God Jesus’ \textstyleExampleSource{[Elicited BR111017.010]}
\z


\section{Locatives}
\label{Para_5.7}
Papuan Malay has a distance oriented three-term \isi{locative} system: proximal \textitbf{sini} ‘\textsc{l.prox}’, medial \textitbf{situ} ‘\textsc{l.med}’, and distal \textitbf{sana} ‘\textsc{l.dist}’. The locatives provide orientation to the hearer in the outside world and in the speech situation by signaling distance, both spatial and nonspatial. Hence, they are similar to the demonstratives. The demonstratives, however, draw the hearer’s attention to specific entities in the discourse or surrounding situation. The locatives, by contrast, focus the hearer’s attention to the specific location of these entities and the relative distance of this location to the deictic center.



The distributional properties of the locatives are as follows:



\begin{enumerate}
\item 
Substitution for \isi{noun} phrases embedded in prepositional phrases (pronominal uses) (§\ref{Para_5.7.1}).
\item 
Modification with demonstratives or relative clauses (pronominal uses) (§\ref{Para_5.7.1}).
\item 
Co-occurrence with \isi{noun} phrases (adnominal uses): \textsc{n}/\textsc{np} \textsc{loc} (§\ref{Para_5.7.2}).

\end{enumerate}

Locatives are distinct from other word classes such as personal pronouns (§\ref{Para_5.5}) or demonstratives (§\ref{Para_5.6}) in terms of the following syntactic properties:


\begin{enumerate}
\item 
Locatives  only occur in prepositional phrases; that is, they are unattested as nominal heads in unembedded \isi{noun} phrases.
\item 
Locatives can be modified with adnominally used demonstratives and relatives clauses, but with no other adnominal modifier; hence, locatives cannot be stacked.
\item 
Locatives are unattested in adnominal possessive constructions as possessor or as possessum.

\end{enumerate}

The pronominal uses of the locatives are discussed in §\ref{Para_5.7.1} and their adnominal uses in §\ref{Para_5.7.2}. Generally speaking, the pronominally used locatives provide additional information about the location of an entity, information nonessential for its identification. Adnominally used locatives, by contrast, limit the referential scope of their head nominals and thereby assist in the identification of their referents. A full discussion of the Papuan Malay locatives is found in §\ref{Para_7.2}.


\subsection{Pronominal uses}
\label{Para_5.7.1}
In their pronominal uses the locatives substitute or stand for \isi{noun} phrases, as illustrated with the four elicited contrastive examples in (\ref{Example_5.222}). Distal \textitbf{sana} ‘\textsc{l.dist}’ in (\ref{Example_5.222b}) substitutes for the \isi{noun} phrase \textitbf{ruma yang paling di bawa itu} ‘that house that’s the furthest down’ in (\ref{Example_5.222a}). The ungrammatical construction in (\ref{Example_5.222c}) shows that the \isi{locative} replaces the entire \isi{noun} phrase and not only its nominal head \textitbf{ruma} ‘house’.


\begin{styleExampleTitle}
Pronominal uses: Substitution for \isi{noun} phrases
\end{styleExampleTitle}

\ea
\label{Example_5.222}
\ea
\label{Example_5.222a}
\gll  {sa} {tinggal} {di} {\bluebold{ruma}} {\bluebold{yang}} {\bluebold{paling}} {\bluebold{di}} {\bluebold{bawa}} {\bluebold{itu}}\\ %
   \textsc{1sg}  stay  at  house  \textsc{rel}  most  at  bottom  \textsc{d.dist}\\
\glt ‘I live in \bluebold{the house that’s the furthest down there}’ \textstyleExampleSource{[Elicited FS120314-001.007]}\\
\vspace{5pt}

\ex
\label{Example_5.222b}
 \gll  sa  tinggal  di  \bluebold{sana}\\
   \textsc{1sg}  stay  at  \textsc{l.dist}\\
\glt ‘I live \bluebold{over there}’ \textstyleExampleSource{[Elicited FS120314-001.008]}\\
\vspace{5pt}

\ex
\label{Example_5.222c}
 \gll {*}  {sa}  {tinggal}  {di}  \bluebold{sana}  \bluebold{yang}  \bluebold{paling}  \bluebold{di}  \bluebold{bawa}  \bluebold{itu}\\
  {  }   \textsc{1sg}  stay  at  \textsc{l.dist}  \textsc{rel}  most  at  bottom  \textsc{d.dist}\\
\glt Intended reading: ‘I live \bluebold{over there that’s the furthest down}’ \textstyleExampleSource{[Elicited FS120314-001.010]}\\
\z
\z



Locatives are always embedded in prepositional phrases. The \isi{prepositional phrase} can be a peripheral adjunct, as in the first clause in (\ref{Example_5.223}) or in (\ref{Example_5.225}), a prepositional predicate, as in the second clause in (\ref{Example_5.223}), or an adnominal \isi{prepositional phrase}, as in (\ref{Example_5.224}). Usually, the locatives are introduced with an overt \isi{preposition} as in (\ref{Example_5.223}) or (\ref{Example_5.224}). The \isi{preposition} may, however, also be elided as in (\ref{Example_5.225}): the omitted \isi{preposition} is allative \textitbf{ke} ‘to’ (the \isi{elision} of prepositions is discussed in §\ref{Para_10.1.5}).


\begin{styleExampleTitle}
Pronominal uses in prepositional phrases
\end{styleExampleTitle}

\ea
\label{Example_5.223}
\gll {ko} {datang} {\bluebold{ke}} {\bluebold{sini},} {nanti} {bapa} {\bluebold{ke}} {\bluebold{situ}} {{\ldots}}\\ %
 \textsc{2sg}  come  to  \textsc{l.prox}  very.soon  father  to  \textsc{l.med}  \\
\glt 
‘you come \bluebold{here}, later I (‘father’) (go) \bluebold{there} {\ldots}’ \textstyleExampleSource{[080922-001a-CvPh.0462]}
\z

\ea
\label{Example_5.224}
\gll {\bluebold{orang}} {\bluebold{dari}} {\bluebold{sana}} {\bluebold{itu}} {{\ldots}} {dorang} {itu} {kerja} {sendiri}\\ %
 person  from  \textsc{l.dist}  \textsc{d.dist}   {}  \textsc{3pl}  \textsc{d.dist}  work  be.alone\\
\glt  ‘\bluebold{those people from over there}, {\ldots}. they work by themselves’ \textstyleExampleSource{[081014-007-CvEx.0050]}
\z

\ea
\label{Example_5.225}
\gll {kam} {datang} {\bluebold{Ø}} {\bluebold{sini},} {kam} {biking} {kaco} {saja}\\ %
 \textsc{2pl}  come   {}  \textsc{l.prox}  \textsc{2pl}  make  be.confused  just\\
\glt ‘you come \bluebold{here}, you’re just stirring up trouble’ \textstyleExampleSource{[081025-007-Cv.0013]}
\z



The pronominally used locatives can be modified with demonstratives or relative clauses. Modification with the demonstratives typically involves short distal \textitbf{tu} ‘\textsc{d.dist}’ as in (\ref{Example_5.226}), while \isi{modification} with long distal \textitbf{itu} ‘\textsc{d.dist}’ is only attested for the non-proximal locatives, as in \textitbf{sana itu} ‘over there (\textsc{emph})’ in (\ref{Example_5.227}). These distributional patterns still require further investigation. Modification with proximal \textitbf{ini} ‘\textsc{d.prox}’ is unattested but possible, as shown in the elicited example in (\ref{Example_5.228}). Modification with relative clauses is also possible, as illustrated for proximal \textitbf{sini} ‘\textsc{l.prox}’ in (\ref{Example_5.229}) and medial \textitbf{situ} ‘\textsc{l.med}’ in (\ref{Example_5.230}). In the corpus, however, such \isi{modification} is rare and unattested for distal \textitbf{sana} ‘\textsc{l.dist}’.


\begin{styleExampleTitle}
Modification of pronominally used locatives
\end{styleExampleTitle}

\ea
\label{Example_5.226}
\gll {sampe} {di} {\bluebold{sini}} {\bluebold{tu}} {dia} {langsung} {sakit} {karna} {{\ldots}}\\ %
 reach  at  \textsc{l.prox}  \textsc{d.dist}  \textsc{3sg}  immediately  be.sick  because  \\
\glt 
‘having arrived \bluebold{here (}\blueboldSmallCaps{emph}\bluebold{)}, he was sick immediately because (he hadn’t eaten)’ \textstyleExampleSource{[081025-008-Cv.0050]}
\z

\ea
\label{Example_5.227}
\gll {dong} {lobe} {ke} {\bluebold{sana}} {\bluebold{itu}}\\ %
 \textsc{3pl}  walk.searchingly.with.lamp  to  \textsc{l.dist}  \textsc{d.dist}\\
\glt 
‘they walk searchingly with lights \bluebold{to over there (}\blueboldSmallCaps{emph}\bluebold{)}’ \textstyleExampleSource{[081108-001-JR.0002]}
\z

\ea
\label{Example_5.228}
\gll {di} {\bluebold{sini}} {\bluebold{ni}} {orang} {tida} {taw} {makang} {pinang}\\ %
 at  \textsc{l.prox}  \textsc{d.prox}  person  \textsc{neg}  know  eat  betel.nut\\
\glt 
‘\bluebold{here (}\blueboldSmallCaps{emph}\bluebold{)} people don’t habitually eat betel nuts’ \textstyleExampleSource{[Elicited BR111017.001]}
\z

\ea
\label{Example_5.229}
\gll {di} {\bluebold{sini}} {\bluebold{yang}} {tra} {banyak}\\ %
 at  \textsc{l.prox}  \textsc{rel}  \textsc{neg}  many\\
\glt 
[About logistic problems:] ‘(it’s) \bluebold{here where} there weren’t many (passengers)’ \textstyleExampleSource{[081025-008-Cv.0140]}
\z

\ea
\label{Example_5.230}
\gll {{\ldots}} {sa} {mandi} {di} {situ,} {di} {\bluebold{situ}} {\bluebold{yang}} {mungking} {nangka}\\ %
  { }  \textsc{1sg}  bathe  at  \textsc{l.med}  at  \textsc{l.med}  \textsc{rel}  maybe  jackfruit\\
\glt
‘[I saw (the poles),] I bathed there, \bluebold{there where} (there are) maybe jackfruits’ \textstyleExampleSource{[080922-010a-CvNF.0298]}
\z


\subsection{Adnominal uses}
\label{Para_5.7.2}
Adnominally used locatives always occur in posthead position. Most commonly, they occur in \isi{noun} phrases embedded in prepositional phrases, as illustrated in (\ref{Example_5.231}). In (\ref{Example_5.231a}), proximal \textitbf{sini} ‘\textsc{l.prox}’ modifies the \isi{locational} \isi{noun} \textitbf{sebla} ‘side’; the \isi{noun} phrase is introduced with allative \textitbf{ke} ‘to’. In (\ref{Example_5.231b}), distal \textitbf{sana} ‘\textsc{l.dist}’ modifies the \isi{noun} \textitbf{laut} ‘sea’; the \isi{preposition} is \isi{locative} \textitbf{di} ‘at, in’.


\begin{styleExampleTitle}
Adnominal uses in embedded \isi{noun} phrases
\end{styleExampleTitle}
\ea
\label{Example_5.231}
\ea
\label{Example_5.231a}
\gll {ke} {sebla} {\bluebold{sini}}\\
  to  side  \textsc{l.prox} \\
\glt ‘to the side {\bluebold{here}’ \textstyleExampleSource{[081011-001-Cv.0148]}}  \\
\vspace{5pt}

\ex
\label{Example_5.231b}
\gll   {di} {laut} {\bluebold{sana}}\\ %
  at  sea  \textsc{l.dist}\\
 \glt ‘in the sea {\bluebold{over there}’ \textstyleExampleSource{[080917-006-CvHt.0004]}}\\
\z
\z


Adnominally used locatives also occur in unembedded \isi{noun} phrases as in (\ref{Example_5.232}), although considerably less frequently. In (\ref{Example_5.232a}), proximal \textitbf{sini} ‘\textsc{l.prox}’ modifies the personal \isi{pronoun} \textitbf{dong} ‘\textsc{3pl}’, while in (\ref{Example_5.232a}) medial \textitbf{situ} ‘\textsc{l.med}’ modifies the \isi{noun} phrase \textitbf{orang kantor} ‘office employees’.


\begin{styleExampleTitle}
Adnominal uses in unembedded \isi{noun} phrases
\end{styleExampleTitle}

\ea
\label{Example_5.232}
\ea
\label{Example_5.232a}
\gll {dong} {\bluebold{sini}}\\
\textsc{3pl}  \textsc{l.prox}\\
\glt ‘they {\bluebold{here}’ \textstyleExampleSource{[080922-001a-CvPh.0556]}}\\
\vspace{5pt}

\ex
\label{Example_5.232b}
\gll {orang} {kantor} {\bluebold{situ}}\\
person  office  \textsc{l.med}\\
\glt ‘the office employees \bluebold{there}’ \textstyleExampleSource{[081005-001-Cv.0018]}
\z
\z


\section{Interrogatives}
\label{Para_5.8}
Papuan Malay has six interrogatives which serve to form content questions. That is, marking a clause as a question, they signal to the hearer which piece of information is being asked for.



The Papuan Malay interrogatives and their functions within the clause are presented in \tabref{Table_5.33} . All of them are used pronominally. Most of them also have predicative uses; the exception is \textitbf{kapang} ‘when’. Besides, the majority of interrogatives also have adnominal uses, except for \textitbf{bagemana} ‘how’, \textitbf{kapang} ‘when’, and \textitbf{knapa} ‘why’ which are unattested. Furthermore, three interrogatives are used as placeholders. In their pronominal and adnominal uses, the interrogatives typically remain in-situ, that is, in the position of the constituents they replace.



Besides, the mid-range \isi{quantifier} \textitbf{brapa} ‘several’ (§\ref{Para_5.10}) also functions as an \isi{interrogative}, which questions quantities in the sense of ‘how many’. For expository reasons, this \isi{interrogative} function of \isi{quantifier} \textitbf{brapa} ‘several, how many’ is discussed here.


\begin{table}
\caption{Papuan Malay interrogatives and their functions within the clause}\label{Table_5.33}
\begin{tabular}{llcccc}
\lsptoprule
 \multicolumn{1}{c}{Item} & \multicolumn{1}{c}{Gloss} & \multicolumn{4}{c}{ Functions within the clause}\\
&  & \textsc{pronom} & \textsc{adnom} & \textsc{pred} &  \textsc{pl-hold}\\
\midrule

\textitbf{siapa} & ‘who’ & X & X & X &  X\\
\textitbf{apa} & ‘what’ & X & X & X &  X\\
\textitbf{mana} & ‘where, which’ & X & X & X & \\
\textitbf{bagemana} & ‘how’ & X &  & X &  X\\
\textitbf{kapang} & ‘when’ & X &  &  & \\
\textitbf{knapa} & ‘why’ & X &  & X & \\
\textitbf{brapa} & ‘how many’ &  & X & X & \\
\lspbottomrule
\end{tabular}
\end{table}

In their predicative uses, most of the five interrogatives can take two positions, as shown in \tabref{Table_5.34} ; the same applies to \isi{quantifier} \textitbf{brapa} ‘several, how many’. That is, all of them can remain in-situ, in the unmarked clause-final position, following the clausal subject. Besides, most of them can also be fronted to the marked clause-initial position, preceding the subject; the exception is \textitbf{knapa} ‘why’. In its \isi{interrogative} uses, \isi{quantifier} \textitbf{brapa} ‘several, how many’ can also take two positions. That is, it can remain in-situ, in the clause-final position, or be fronted to the marked clause-initial position. These positions and the semantics they convey are discussed in detail in the following sections.


\begin{table}
\caption{Predicatively used interrogatives and their positions within the clause}\label{Table_5.34}
\begin{tabular}{llcc}
\lsptoprule
 \multicolumn{1}{c}{Item} & \multicolumn{1}{c}{Gloss} & \multicolumn{2}{c}{ Position within the clause}\\
&  & \textsc{cl-initial} &  \textsc{cl-final}\\
\midrule
\textitbf{siapa} & ‘who’ & X &  X\\
\textitbf{apa} & ‘what’ & X &  X\\
\textitbf{mana} & ‘where, which’ & X &  X\\
\textitbf{bagemana} & ‘how’ & X &  X\\
\textitbf{knapa} & ‘why’ &  &  X\\
\textitbf{brapa} & ‘how many’ & X &  X\\
\lspbottomrule
\end{tabular}
\end{table}

In the following, the interrogatives are described in turn, \textitbf{siapa} ‘who’ in §\ref{Para_5.8.1}, \textitbf{apa} ‘what’ in §\ref{Para_5.8.2}, \textitbf{mana} ‘where, which’ in §\ref{Para_5.8.3}, \textitbf{bagemana} ‘how’ in §\ref{Para_5.8.4}, \textitbf{kapang} ‘when’ in §\ref{Para_5.8.5}, and \textitbf{knapa} ‘why’ in §\ref{Para_5.8.6}, and \isi{quantifier} \textitbf{brapa} ‘several, how many’ in §\ref{Para_5.8.7}. Some of the interrogatives also express non-\isi{interrogative} \isi{indefinite} meanings; this function is summarily discussed in §\ref{Para_5.8.8}.


\subsection{\textitbf{siapa} ‘who’}
\label{Para_5.8.1}
The \isi{interrogative} \textitbf{siapa} ‘who’ questions the identity of human referents. Its pronominal uses are illustrated in (\ref{Example_5.233}) to (\ref{Example_5.240}), its adnominal uses in (\ref{Example_5.241}) and (\ref{Example_5.242}), and its predicative uses in (\ref{Example_5.243}) and (\ref{Example_5.244}). In addition, \textitbf{siapa} ‘who’ serves as a \isi{placeholder} as shown in (\ref{Example_5.245}). Furthermore, the \isi{interrogative} is also used in one-word utterances.

In its pronominal uses, \textitbf{siapa} ‘who’ occurs in all syntactic positions, as shown (\ref{Example_5.233}) to (\ref{Example_5.240}), typically remaining in-situ. In the \isi{verbal clause} in (\ref{Example_5.233}), \textitbf{siapa} ‘who’ takes the subject slot. In the corpus, however, verbal clauses with \textitbf{siapa} ‘who’ in the subject slot are rare. Typically, speakers use equative nominal clauses when they want to question the identity of the clausal subject. In such nonverbal clauses, \textitbf{siapa} ‘who’ takes the subject slot while a headless relative clause takes the predicate slot. This is shown with the elicited contrastive example in (\ref{Example_5.234}). In this equative clause, \textitbf{siapa} ‘who’ is the subject, while the headless relative clause \textitbf{yang suru {\ldots}} ‘(the one) who ordered {\ldots}’ is the predicate. Likewise in (\ref{Example_5.235}), the \isi{interrogative} takes the subject slot while the headless relative clause \textitbf{yang datang {\ldots}} ‘(the one) who came {\ldots}’ takes the predicate slot.\footnote{\label{Footnote_5.168} Alternatively, as one anonymous reviewer points out, one could argue that in (\ref{Example_5.234}) and (\ref{Example_5.235}) the typical subject-predicate word order is inverted and that \textitbf{siapa} ‘who’ does not take the subject but the predicate slot, whereas the relative clause introduced with \textitbf{yang} ‘\textsc{rel}’ takes the subject slot.
} (For details on relative clauses see §\ref{Para_14.3.2}.)


\begin{styleExampleTitle}
Pronominal uses of \textitbf{siapa} ‘who’: Subject slot
\end{styleExampleTitle}

\ea
\label{Example_5.233}
\gll {e,} {\bluebold{siapa}} {suru} {kam} {minum{\Tilde}minum} {di} {sini?}\\ %
 hey  who  order  \textsc{2pl}  \textsc{rdp}{\Tilde}drink  at  \textsc{l.prox}\\
\glt 
‘hey, \bluebold{who} told you to keep drinking here’ \textstyleExampleSource{[081014-005-Cv.0006]}
\z

\ea
\label{Example_5.234}
\gll {e,} {\bluebold{siapa}} {yang} {suru} {kam} {minum{\Tilde}minum} {di} {sini?}\\ %
 hey  who  \textsc{rel}  order  \textsc{2pl}  \textsc{rdp}{\Tilde}drink  at  \textsc{l.prox}\\
\glt 
‘hey, \bluebold{who} (is the one) who told you to keep drinking here’ \textstyleExampleSource{[Elicited MY131112.004]}
\z

\ea
\label{Example_5.235}
\gll {\bluebold{siapa}} {yang} {datang} {jemput} {saya?}\\ %
 who  \textsc{rel}  come  pick.up  \textsc{1sg}\\
\glt 
‘\bluebold{who} (is the one) who came (and) picked me up?’ \textstyleExampleSource{[080918-001-CvNP.0001]}
\z


Interrogative \textitbf{siapa} ‘who’ also takes non-subject slots. In (\ref{Example_5.236}), \textitbf{siapa} ‘who’ takes the direct object slot in a mono\isi{transitive clause}, in (\ref{Example_5.237}) a direct object slot in a \isi{double-object} construction, in (\ref{Example_5.238}) the oblique object slot, and in (\ref{Example_5.239}) the peripheral adjunct slot.\footnote{Alternatively, one could argue that in (\ref{Example_5.238}) \textitbf{siapa} ‘who’ does not take an oblique object but an adjunct slot.} In addition, \textitbf{siapa} ‘who’ questions the possessor’s identity in adnominal possessive constructions, as in (\ref{Example_5.240}).


\begin{styleExampleTitle}
Pronominal uses of \textitbf{siapa} ‘who’: Non-subject slots
\end{styleExampleTitle}

\ea
\label{Example_5.236}
\gll {dong} {cari} {\bluebold{siapa}?}\\ %
 \textsc{3pl}  search  who\\
\glt 
‘for \bluebold{whom} are they looking?’ \textstyleExampleSource{[080921-010-Cv.0010]}
\z

\ea
\label{Example_5.237}
\gll {kwe} {mo} {pi} {kasi} {\bluebold{siapa}} {di} {sana?}\\ %
 cake  want  go  give  who  at  \textsc{l.dist}\\
\glt 
‘as for the cake, \bluebold{to whom} do (you) want to go and give (it) over there?’ \textstyleExampleSource{[080922-001a-CvPh.0670]}
\z

\ea
\label{Example_5.238}
\gll {{\ldots}} {ke} {mana?,} {ke} {kampung?,} {deng} {\bluebold{siapa}?}\\ %
  { }  to  where  to  village  with  who\\
\glt 
[Talking to her young son:] ‘[do you want to leave today?,] where to?, to the village?, with \bluebold{whom}?’ \textstyleExampleSource{[080917-003a-CvEx.0048-0044]}
\z

\ea
\label{Example_5.239}
\gll {baru} {nanti} {minggu} {keduanya} {sembayang} {di} {\bluebold{siapa}?}\\ %
 and.then  very.soon  week  second:\textsc{3possr}  worship  at  who\\
\glt 
‘then later in the second week (of this month), (we’ll) worship at \bluebold{whose} (place)?’ (Lit. ‘at \bluebold{who}’) \textstyleExampleSource{[081011-005-Cv.0037]}
\z

\ea
\label{Example_5.240}
\gll {\bluebold{siapa}} {\bluebold{pu}} {mata} {yang} {buta?}\\ %
 who  \textsc{poss}  eye  \textsc{rel}  be.blind\\
\glt 
‘\bluebold{whose} eyes are blind?’ \textstyleExampleSource{[080922-001a-CvPh.0142]}
\z



As a nominal modifier, \textitbf{siapa} ‘who’ takes the position of a modifying \isi{noun} which it replaces. This illustrated with the \isi{interrogative} clauses (\ref{Example_5.241}) and (\ref{Example_5.242}.


\begin{styleExampleTitle}
Adnominal uses of \textitbf{siapa} ‘who’
\end{styleExampleTitle}

\ea
\label{Example_5.241}
\gll {[\bluebold{prempuang}} {\bluebold{siapa}]} {biking} {sa} {jadi} {bingung?}\\ %
 woman  who  make  \textsc{1sg}  become  be.confused\\
\glt 
‘\bluebold{which woman} made me become confused?’ \textstyleExampleSource{[080922-004-Cv.0028]}
\z

\ea
\label{Example_5.242}
\gll {skarang} {sa} {tanya,} {[\bluebold{orang}} {\bluebold{siapa}]} {yang} {benar?}\\ %
 now  \textsc{1sg}  ask  person  who  \textsc{rel}  be.true\\
\glt 
‘now I asked, ``\bluebold{which person} (is the one) who is right?''' \textstyleExampleSource{[080917-010-CvEx.0197]}
\z


In its predicative uses, \textitbf{siapa} ‘who’ occurs in equative nominal predicate clauses where it questions the identity of the clausal subject, as shown in (\ref{Example_5.243}) and (\ref{Example_5.244}) (for more details on nominal clauses see §\ref{Para_12.2}). The \isi{interrogative} can remain in-situ, in the clause-final position, following the subject as in (\ref{Example_5.243}), or it can be fronted to the marked clause-initial position, preceding the subject, as in (\ref{Example_5.244}). In the corpus, the token frequencies for both positions are about the same. When speakers want to accentuate the subject, such as \textitbf{ini} ‘\textsc{d.prox}’ in (\ref{Example_5.243}), the \isi{interrogative} remains in-situ, where it is less prominent. When, by contrast, speakers want to stress the questioning of the subject’s identity, they front \textitbf{siapa} ‘who’ to the clause-initial position, as in (\ref{Example_5.244}). Besides their different functions, the contrastive examples in (\ref{Example_5.243}) and (\ref{Example_5.244}) also have distinct intonation contours. When \textitbf{siapa} ‘who’ remains in-situ, as in (\ref{Example_5.243}), the clause has a rising intonation, typical of interrogatives. When it is fronted, the clause has a falling intonation, typical of declaratives. In both cases, \textitbf{siapa} ‘who’ is marked with a slight increase in pitch of its stressed penultimate syllable (“~\'{~}~”).


\begin{styleExampleTitle}
Predicative uses of \textitbf{siapa} ‘who’
\end{styleExampleTitle}

\ea
\label{Example_5.243}
\glll {\textstyleChBold{{} -- {} -- }}\hspace{7mm} {\textstyleChBold{\textsuperscript{{} — }}\textstyleChBold{\textsubscript{{} — }}}\hspace{7mm}  {\textstyleChBold{{} -- {} -- }}\hspace{7mm}  {\textstyleChBold{\textsuperscript{{} — }}\textstyleChBold{\textsubscript{{} — }}}\\ %
 {ini}  {\bluebold{siápa}?}  {ini}  {\bluebold{siápa}?}\\
 {\textsc{d.prox}}  {who}  {\textsc{d.prox}}  {who}\\
\glt 
‘\bluebold{who} is this? \bluebold{who} is this?’ \textstyleExampleSource{[080916-001-CvNP.0006]}
\z

\ea
\label{Example_5.244}
\glll {\textstyleChBold{{} -- {} -- }}\hspace{7mm} {\textstyleChBold{{} -- }\textbf{\textsubscript{{\textbackslash}}}}\\ %
 \bluebold{siápa}  ini?\\
 who  \textsc{d.prox}\\
\glt 
‘\bluebold{who} is this?’ \textstyleExampleSource{[081011-023-Cv.0104]}
\z


In addition, \textitbf{siapa} ‘who’ functions as a \isi{placeholder} when speakers do not recall a referent’s name, as in (\ref{Example_5.245}).


\begin{styleExampleTitle}
Placeholder uses of \textitbf{siapa} ‘who’
\end{styleExampleTitle}

\ea
\label{Example_5.245}
\gll {Sarles} {antar} {\bluebold{siapa},} {Bolikarfus}\\ %
 Sarles  bring  who  Bolikarfus\\
\glt 
‘Sarles gave a ride to, \bluebold{who-is-it}, Bolikarfus’ \textstyleExampleSource{[081002-001-CvNP.0031]}
\z


Rather commonly, speakers also employ the \isi{interrogative} in one-word utterances, when they question the identity of a referent, in the sense of ‘who (do you mean)?’


\subsection{\textitbf{apa} ‘what’}
\label{Para_5.8.2}
The \isi{interrogative} \textitbf{apa} ‘what’ questions the identity of nonhuman referents, name\-ly entities and events; in addition, it can question reason. The pronominal uses of \textitbf{apa} ‘what’ are illustrated in (\ref{Example_5.246}) to (\ref{Example_5.251}), its adnominal uses in (\ref{Example_5.252}) and (\ref{Example_5.253}), and its predicative uses in (\ref{Example_5.254}) and (\ref{Example_5.255}). The \isi{interrogative} is also used as a \isi{placeholder} as shown in (\ref{Example_5.256}). Besides, \textitbf{apa} ‘what’ is also used in one-word utterances.

In its pronominal uses, \textitbf{apa} ‘what’ occurs in all syntactic positions, as demonstrated in (\ref{Example_5.246}) to (\ref{Example_5.251}), always remaining in-situ. In the elicited \isi{verbal clause} in (\ref{Example_5.246}), \textitbf{apa} ‘what’ takes the subject slot. While this construction is grammatically correct and acceptable, verbal clauses with \textitbf{apa} ‘what’ in the subject slot are unattested in the corpus. Instead, speakers always use equative clauses when they want to question the identity of the clausal subject, similar to the questions formed with \textitbf{siapa} ‘who’ in (\ref{Example_5.234}) and (\ref{Example_5.235}) (§\ref{Para_5.8.1}). This is demonstrated with the contrastive equative clause in (\ref{Example_5.247}). In this example, \textitbf{apa} ‘what’ takes the subject slot, while the headless relative clause \textitbf{yang su gigit {\ldots}} ‘(the one) who has already bitten {\ldots}’ takes the predicate slot.\footnote{Similar to the comment in Footnote \ref{Footnote_5.168} in §\ref{Para_5.8.1} (p. \pageref{Footnote_5.168}), one could argue that in (\ref{Example_5.246}) the subject-predicate word order is inverted and that \textitbf{apa} ‘what’ fills the predicate rather than the subject slot, while the relative clause introduced with \textitbf{yang} ‘\textsc{rel}’ takes the subject slot.} (For details on relative clauses see §\ref{Para_14.3.2}.)


\begin{styleExampleTitle}
Pronominal uses of \textitbf{apa} ‘what’: Subject slot
\end{styleExampleTitle}

\ea
\label{Example_5.246}
\gll {\bluebold{apa}} {su} {gigit} {sa} {pu} {lutut}\\ %
 what  already  bite  \textsc{1sg}  \textsc{poss}  knee\\
\glt 
‘\bluebold{what} has bitten my knee?’ \textstyleExampleSource{[Elicited MY131112.005]}
\z

\ea
\label{Example_5.247}
\gll {\bluebold{apa}} {yang} {su} {gigit} {sa} {pu} {lutut?}\\ %
 what  \textsc{rel}  already  bite  \textsc{1sg}  \textsc{poss}  knee\\
\glt 
‘\bluebold{what} (is it) that has bitten my knee?’ \textstyleExampleSource{[080916-001-CvNP.0004]}
\z


Interrogative \textitbf{apa} ‘what’ also takes non-subject slots. In (\ref{Example_5.248}), \textitbf{apa} ‘what’ takes the direct object slot, in (\ref{Example_5.249}) the oblique object slot, and in (\ref{Example_5.250}) the peripheral adjunct slot.\footnote{Alternatively, one could argue that in (\ref{Example_5.249}) \textitbf{apa} ‘what’ does not take an oblique object but an adjunct slot.} Besides, speakers use \textitbf{apa} ‘what’ to question reasons or motives, as in (\ref{Example_5.251}).


\begin{styleExampleTitle}
Pronominal uses of \textitbf{apa} ‘what’: Non-subject slots
\end{styleExampleTitle}

\ea
\label{Example_5.248}
\gll {kam} {cari} {\bluebold{apa}?}\\ %
 \textsc{2pl}  search  what\\
\glt 
‘\bluebold{what} are you looking for?’ \textstyleExampleSource{[080917-006-CvHt.0001]}
\z

\ea
\label{Example_5.249}
\gll {tokok} {sagu} {tu} {deng} {\bluebold{apa}} {ini?}\\ %
 tap  sago  \textsc{d.dist}  with  what  \textsc{d.prox}\\
\glt 
‘\bluebold{what} are you pounding that sagu with?’ \textstyleExampleSource{[081014-006-CvPr.0014]}
\z

\ea
\label{Example_5.250}
\gll {sa} {tra} {taw} {tugu} {ini} {dari} {\bluebold{apa}?}\\ %
 \textsc{1sg}  \textsc{neg}  know  monument  \textsc{d.prox}  from  what\\
\glt 
‘I don’t’ know from \bluebold{where} this monument comes?’ (Lit. ‘from \bluebold{what} (place)’) \textstyleExampleSource{[080917-008-NP.0003]}
\z

\ea
\label{Example_5.251}
\gll {de} {bilang,} {ko} {tidor} {\bluebold{apa}?}\\ %
 \textsc{3sg}  say  \textsc{2sg}  sleep  what\\
\glt 
‘he said, ``\bluebold{why} are you sleeping?''' \textstyleExampleSource{[081006-034-CvEx.0022]}
\z


In its adnominal uses, \textitbf{apa} ‘what’ takes the position of a nominal modifier which it replaces, such as the name of a weekday in (\ref{Example_5.252}), or the name of a clan in (\ref{Example_5.253}).


\begin{styleExampleTitle}
Adnominal uses of \textitbf{apa} ‘what’
\end{styleExampleTitle}

\ea
\label{Example_5.252}
\gll {[\bluebold{hari}} {\bluebold{apa}]} {baru} {sa} {minta} {ijing?}\\ %
 day  what  and.then  \textsc{1sg}  request  permission\\
\glt 
[A school boy asking his mother:] ‘on \bluebold{which day} will I ask for permission (to be absent)?’ \textstyleExampleSource{[080917-003a-CvEx.0003]}
\z

\ea
\label{Example_5.253}
\gll {dia} {tanya} {saya,} {ko} {[\bluebold{marga}} {\bluebold{apa}]?}\\ %
 \textsc{3sg}  ask  \textsc{1sg}  \textsc{2sg}  clan  what\\
\glt 
‘he asked me, ``\bluebold{which clan} do you (belong to)?''' (Lit. ‘you are \bluebold{what clan}’) \textstyleExampleSource{[080922-010a-CvNF.0281]}
\z


In its predicative uses, \textitbf{apa} ‘what’ questions the identity or pertinent characteristics of the clausal subject (for more details on nominal predicate clauses see §\ref{Para_12.2}). Like \textitbf{siapa} ‘who’ (§\ref{Para_5.8.1}), \textitbf{apa} ‘what’ can remain in the unmarked clause-final position, as in (\ref{Example_5.254}). Alternatively, it can be fronted to the marked clause-initial position, as in (\ref{Example_5.255}), where it stresses the questioning of the subject’s identity or characteristics. The contrastive clauses in (\ref{Example_5.254}) and (\ref{Example_5.255}) have the same distinct intonation contours as the corresponding questions with \textitbf{siapa} ‘who’ in (\ref{Example_5.243}) and (\ref{Example_5.244}) (§\ref{Para_5.8.1}). Clauses with in-situ \textitbf{apa} ‘what’, as in (\ref{Example_5.254}), have the typical rising \isi{interrogative} intonation. Clauses with fronted \textitbf{apa} ‘what’ have the typical falling declarative intonation. Like \textitbf{siapa} ‘who’, \textitbf{apa} ‘what’ is marked with a slight increase in pitch of its stressed penultimate syllable (“~\'{~}~”).


\begin{styleExampleTitle}
Predicative uses of \textitbf{apa} ‘what’
\end{styleExampleTitle}

\ea
\label{Example_5.254}
\glll {\textstyleChBold{{} --{\hspace{3pt}}-- }}\hspace{5mm}  {\textstyleChBold{\textsuperscript{{} — }}\textstyleChBold{\textsubscript{{} — }}}\\ %
\textit{ini} \textitbf{ápa}?\\
 \textsc{d.prox}  what\\
\glt 
‘\bluebold{what} is this?’ \textstyleExampleSource{[081109-001-Cv.0012]}
\z
\ea
\label{Example_5.255}
\glll{\textstyleChBold{{} -- }\textbf{\textsubscript{{\textbackslash}}}}\hspace{5mm} {\textstyleChBold{{} --{\hspace{3pt}}-- }}\hspace{5mm} {\textstyleChBold{{}-- }\textbf{\textsubscript{{\textbackslash}}}}\\ %
\textit{adu,}   \textitbf{ápa}  ini?\\
 oh.no!  what  \textsc{d.prox}\\
\glt 
‘oh no!, \bluebold{what} is this?’ \textstyleExampleSource{[081109-001-Cv.0012]}
\z


In addition, \textitbf{apa} ‘what’ functions as a \isi{placeholder}, when speakers do not recall the name of a lexical item, as in (\ref{Example_5.256}).


\begin{styleExampleTitle}
Placeholder uses of \textitbf{apa} ‘what’
\end{styleExampleTitle}

\ea
\label{Example_5.256}
\gll {de} {bisa} {bantu} {deng} {\bluebold{apa},} {ijasa}\\ %
 \textsc{3sg}  be.able  help  with  what  diploma\\
\glt 
‘he can help (us) with, \bluebold{what-is-it}, the diploma’ \textstyleExampleSource{[081011-023-Cv.0107]}
\z


Speakers also employ \textitbf{apa} ‘what’ in one-word utterances to question the overall situation in the sense of ‘what (is wrong)?’, or to signal lack of understanding, in the sense of ‘what?’, for example during phone conversations with a bad connection.


\subsection{\textitbf{mana} ‘where, which’}
\label{Para_5.8.3}
The \isi{interrogative} \textitbf{mana} ‘where, which’ questions locations and single items. Its pronominal uses are illustrated in (\ref{Example_5.257}) and (\ref{Example_5.258}), its adnominal uses in (\ref{Example_5.259}) and (\ref{Example_5.260}), and its predicative uses in (\ref{Example_5.261}) to (\ref{Example_5.263}). The \isi{interrogative} is also used in one-word utterances, as shown in (\ref{Example_5.264}).



In its pronominal uses as the head of a \isi{noun} phrase, \textitbf{mana} ‘where, which’ questions locations, as in (\ref{Example_5.257}) and (\ref{Example_5.258}). More specifically, it functions as the complement in a \isi{prepositional phrase} which is headed by a \isi{preposition} encoding location (details on prepositional phrases are provided in \chapref{Para_10}).


\begin{styleExampleTitle}
Pronominal uses of \textitbf{mana} ‘where, which’
\end{styleExampleTitle}

\ea
\label{Example_5.257}
\gll {ko} {tinggal} {\bluebold{di}} {\bluebold{mana}?}\\ %
 \textsc{2sg}  stay  at  where\\
\glt 
‘\bluebold{where} do you live?’ \textstyleExampleSource{[080922-010a-CvNF.0237]}
\z

\ea
\label{Example_5.258}
\gll {ko} {datang} {\bluebold{dari}} {\bluebold{mana}?}\\ %
 \textsc{2sg}  come  from  where\\
\glt 
‘\bluebold{from where} do you come?’ \textstyleExampleSource{[080922-010a-CvNF.0236]}
\z


In its adnominal uses, \textitbf{mana} ‘where, which’ questions single entities among larger numbers of identical or similar entities expressed by its referents, as in (\ref{Example_5.259}) and (\ref{Example_5.260}).


\begin{styleExampleTitle}
Adnominal uses of \textitbf{mana} ‘where, which’
\end{styleExampleTitle}

\ea
\label{Example_5.259}
\gll {kalo} {{[\bluebold{ana}}} {{\bluebold{mana}]}} {{yang}} {{sa}} {{duduk}} {ceritra} {deng} {dia,}\\ %
 if  {child}  {where}  {\textsc{rel}}  {\textsc{1sg}}  {sit}  tell  with  \textsc{3sg}\\
\gll {itu}  {ana}  {itu,}  {de}  {hormat}  {torang}\\
 {\textsc{d.dist}}  {child}  {\textsc{d.dist}}  {\textsc{3sg}}  {respect}  {\textsc{1pl}}\\
\glt 
[Conversation about a certain teenager:] ‘as for \bluebold{which kid} with whom I sit and talk with, that is that kid, she respects us’ \textstyleExampleSource{[081115-001a-Cv.0282]}
\z

\ea
\label{Example_5.260}
\gll {dong} {bilang,} {[\bluebold{badang}} {\bluebold{mana}]} {yang} {sakit?}\\ %
 \textsc{3pl}  say  body  where  \textsc{rel}  be.sick\\
\glt 
‘they said, ``\bluebold{which (part of your) body} (is the one) that is hurting?''' \textstyleExampleSource{[081015-005-NP.0031]}
\z


In its predicative uses, \textitbf{mana} ‘where, which’ occurs in prepositional predicates which question the subject’s location (for details on prepositional predicates see §\ref{Para_12.4}). Like predicate clauses with \textitbf{siapa} ‘who’ (§\ref{Para_5.8.1}) and \textitbf{apa} ‘what’ (§\ref{Para_5.8.2}), prepositional predicates with \textitbf{mana} ‘where, which’ can take two positions. They can remain in-situ, following the clausal subject, as in (\ref{Example_5.261}). Alternatively, they can be fronted to the marked clause-initial position, where they stress the questioning of the subject’s location, as in the elicited example in (\ref{Example_5.262}). In the corpus, though, the \isi{preposition} is always omitted from fronted prepositional predicates, as in (\ref{Example_5.263}).


\begin{styleExampleTitle}
Predicative uses of \textitbf{mana} ‘where, which’
\end{styleExampleTitle}

\ea
\label{Example_5.261}
\gll {sabung} {mandi} {di} {\bluebold{mana}?}\\ %
 soap  bathe  at  where\\
\glt 
‘\bluebold{where} is (our) soap?’ \textstyleExampleSource{[081025-006-Cv.0026]}
\z

\ea
\label{Example_5.262}
\gll {di} {\bluebold{mana}} {sabung} {mandi?}\\ %
 at  where  soap  bathe\\
\glt 
‘\bluebold{where} is (our) soap?’ \textstyleExampleSource{[Elicited MY131112.006]}
\z

\ea
\label{Example_5.263}
\gll {Nofi,} {Ø} {\bluebold{mana}} {kitong} {pu} {ikang{\Tilde}ikang?}\\ %
 Nofi { }   where  \textsc{1pl}  \textsc{poss}  \textsc{rdp}{\Tilde}fish\\
\glt 
‘Nofi, \bluebold{where} are our fish?’ \textstyleExampleSource{[080917-006-CvHt.0002]}
\z


Quite commonly, the \isi{interrogative} is used to form one-word utterances in which case it questions an entire proposition, as in (\ref{Example_5.264}).


\begin{styleExampleTitle}
One-word utterances with \textitbf{mana} ‘where, which’
\end{styleExampleTitle}

\ea
\label{Example_5.264}
\gll {Speaker-2:} {di} {\bluebold{mana}?}\\ %
  { } at  where\\
\glt
[Speaker-1: ‘(I used to) stay with my aunt Marta’]\\
Speaker-2: ‘\bluebold{where}?’ \textstyleExampleSource{[080922-002-Cv.0029-0030]}
\z


\subsection{\textitbf{bagemana} ‘how’}
\label{Para_5.8.4}
The \isi{interrogative} \textitbf{bagemana} ‘how’ questions manner or circumstance in the sense of ‘how, what (is it) like’. The \isi{interrogative} has pronominal uses, as illustrated in (\ref{Example_5.265}) and (\ref{Example_5.267}), and predicative uses, as shown in (\ref{Example_5.268}) to (\ref{Example_5.271}). In addition, \textitbf{bagemana} ‘how’ has \isi{placeholder} uses as in (\ref{Example_5.272}). It also occurs in one-word utterances, as in (\ref{Example_5.273}) and (\ref{Example_5.274}).

In its pronominal uses, \textitbf{bagemana} ‘how’ can remain in-situ, in the unmarked clause-final position, or can occur in the marked clause-initial position. In the clause-final position, the \isi{interrogative} questions the specific manner of an event or activity such as the best way of transporting a pig in (\ref{Example_5.265}). In the clause-initial position, the scope of \textitbf{bagemana} ‘how’ is larger. Here it questions an entire proposition, as in (\ref{Example_5.266}) and (\ref{Example_5.267}), and not only a specific manner, as in (\ref{Example_5.265}). The example in (\ref{Example_5.267}) also shows that, depending on the context, fronted \textitbf{bagemana} ‘how’ also question reasons.


\begin{styleExampleTitle}
Pronominal uses of \textitbf{bagemana} ‘how’
\end{styleExampleTitle}

\ea
\label{Example_5.265}
\gll {{\ldots}} {adu,} {babi} {ni} {sa} {harus} {angkat} {\bluebold{bagemana}?}\\ %
 { }   oh.no!  pig  \textsc{d.prox}  \textsc{1sg}  have.to  lift  how\\
\glt 
‘[the pig was very big, I alone could not transport it, I thought,] ``oh no!, this pig, \bluebold{how} am I going to transport it?''' \textstyleExampleSource{[080919-003-NP.0008]}
\z

\ea
\label{Example_5.266}
\gll {\bluebold{bagemana}} {kitong} {mo} {dapat} {uang?}\\ %
 how  \textsc{1pl}  want  get  money\\
\glt 
‘\bluebold{how} are we going to get money?’ \textstyleExampleSource{[080927-006-CvNP.0041]}
\z

\ea
\label{Example_5.267}
\gll {de} {tanya} {juga,} {\bluebold{bagemana}} {ko} {bisa} {kasi} {ana} {ini?}\\ %
 \textsc{3sg}  ask  also  how  \textsc{2sg}  be.able  give  child  \textsc{d.prox}\\
\glt 
[About bride-price children:] ‘she also asked (me), ``\bluebold{how} can you give this child (of yours away)?''' \textstyleExampleSource{[081006-026-CvEx.0003]}
\z


When used predicatively, \textitbf{bagemana} ‘how’ can remain in-situ, as in (\ref{Example_5.268}) and (\ref{Example_5.270}), or can be fronted, as in (\ref{Example_5.269}) and (\ref{Example_5.271}). Similar to the predicative uses of the interrogatives discussed in the previous sections, the clause-final in-situ position is the unmarked one where the \isi{interrogative} is less prominent in comparison to the clause-initial subject, as shown in (\ref{Example_5.268}). When placed in the marked clause-initial position, by contrast, \textitbf{bagemana} ‘how’ accentuates the questioning of the subject’s circumstance, as in (\ref{Example_5.269}). In addition, predicatively used \textitbf{bagemana} ‘how’ inquires about the well-being of one’s interlocutor(s) as in (\ref{Example_5.270}) and (\ref{Example_5.271}).


\begin{styleExampleTitle}
Predicative uses of \textitbf{bagemana} ‘how’
\end{styleExampleTitle}

\ea
\label{Example_5.268}
\gll {dong} {tida} {taw} {itu,} {Yesus} {itu,} {injil} {itu} {\bluebold{bagemana}?}\\ %
 \textsc{3pl}  \textsc{neg}  know  \textsc{d.dist}  Jesus  \textsc{d.dist}  Gospel  \textsc{d.dist}  how\\
\glt ‘they don’t know, what’s-his-name, Jesus, (they don’t know) \bluebold{what} the Gospel (is like)’ (Lit. ‘the gospel is \bluebold{how}?’) \textstyleExampleSource{[081006-023-CvEx.0005]}
\z

\ea
\label{Example_5.269}
\gll {{\ldots}} {susa} {liat} {setang} {itu,} {\bluebold{bagemana}} {rupa} {setang}\\ %
 { }   be.difficult  see  evil.spirit  \textsc{d.dist}  how  form  evil.spirit\\
\glt [About evil spirits:] ‘[but for us who {\ldots} already believe in Jesus, we can’t,] (for us) it is difficult to see that evil spirit, \bluebold{what} the evil spirit’s face (is like)’ (Lit. ‘\bluebold{how} (is) the evil spirit’s form?’) \textstyleExampleSource{[081006-022-CvEx.0069]}\\
\z

\ea
\label{Example_5.270}
\gll {yo,} {ko} {Herman} {\bluebold{bagemana}?}\\ %
 yes  \textsc{2sg}  Herman  how\\
\glt [Greeting a visitor:] ‘yes, \bluebold{how} are you, Herman?’ \textstyleExampleSource{[081014-011-CvEx.0072]}\\
\z

\ea
\label{Example_5.271}
\gll {eh,} {\bluebold{bagemana}} {ipar?,} {sore,} {dari} {Jayapura?}\\ %
 hey!  how  sibling-in-law  afternoon  from  Jayapura\\
\glt [Greeting a visitor:] ‘hey, \bluebold{how} (is it going) brother-in-law?, (good) afternoon! (did you just get here) from Jayapura?’ \textstyleExampleSource{[081110-002-Cv.0003]}\\
\z


Another use of \textitbf{bagemana} ‘how’ is that of a \isi{placeholder}, as shown in (\ref{Example_5.272}).


\begin{styleExampleTitle}
Placeholder uses of \textitbf{bagemana} ‘how’
\end{styleExampleTitle}

\ea
\label{Example_5.272}
\gll {{\ldots}} {sa} {macang,} {sa} {macang} {\bluebold{bagemana},} {e,} {rasa} {sa} {{\ldots}}\\ %
 { }  \textsc{1sg}  variety  \textsc{1sg}  variety  how  uh  feel  \textsc{1sg}  \\
\glt 
‘[so when I (went) to \ili{Biak} there, I felt very strange] I kind of, I kind of, \bluebold{what-is-it}, uh, felt (that) I {\ldots}’ \textstyleExampleSource{[081011-013-Cv.0009]}
\z


In one-word utterances, \textitbf{bagemana} ‘how’ questions the circumstances of an event or state, as in (\ref{Example_5.273}), or signals lack of understanding as in (\ref{Example_5.274}).


\begin{styleExampleTitle}
One-word utterances with \textitbf{bagemana} ‘how’
\end{styleExampleTitle}

\ea
\label{Example_5.273}
\gll {saya} {tanya} {saya} {punya} {bapa,} {\bluebold{bagemana}?}\\ %
 \textsc{1sg}  ask  \textsc{1sg}  \textsc{poss}  father  how\\
\glt 
‘I asked my father, ``\bluebold{how} (did this happen)?''' \textstyleExampleSource{[080921-011-Cv.0012]}
\z

\ea
\label{Example_5.274}
\gll {\bluebold{bagemana}?} {\bluebold{bagemana}?}\\ %
 how  how\\
\glt
[During a phone conversation with a bad connection:] ‘\bluebold{what}?, \bluebold{what}?’ \textstyleExampleSource{[080922-001b-CvPh.0027]}
\z


\subsection{\textitbf{kapang} ‘when’}
\label{Para_5.8.5}
The \isi{interrogative} \textitbf{kapang} ‘when’ questions time. Always used pronominally, \textitbf{kapang} ‘when’ usually occurs in clause-initial position, as shown with its first and third occurrences in (\ref{Example_5.275}). Here, \textitbf{kapang} ‘when’ questions the temporal setting of the events or states expressed by the entire clause. When the temporal setting is less important, \textitbf{kapang} ‘when’ occurs in clause-final position, as shown with the second \textitbf{kapang} ‘when’ token in (\ref{Example_5.275}). Hence, the different positions of \textitbf{kapang} ‘when’ within the clause have functions which parallel those of the time-denoting nouns which the \isi{interrogative} replaces (see §\ref{Para_5.2.5}). Alternatively, but rarely, the \isi{interrogative} occurs between the subject and the predicate, as in (\ref{Example_5.276}). According to one consultant, this position of \textitbf{kapang} ‘when’ is acceptable, although the semantics conveyed by this position are still ill understood.


\begin{styleExampleTitle}
Pronominal uses of \textitbf{kapang} ‘when’
\end{styleExampleTitle}

\ea
\label{Example_5.275}
\gll {\bluebold{kapang}} {kita} {{mo}} {{antar?,}} {kitong} {antar} {\bluebold{kapang}?} {{\ldots}}\\ %
 when  \textsc{1pl}  {want}  {deliver}  \textsc{1pl}  deliver  when  \\
 \gll \bluebold{kapang}  {kitong}  {antar}  {dia?}\\
 when  {\textsc{1pl}}  {bring}  {\textsc{3sg}}\\
\glt 
[Discussing when the bride’s parents will bring their daughter to the groom’s parents:] ‘[they (the bride’s parents) start asking, ``{\ldots},] \bluebold{when} should we bring her? we bring her \bluebold{when}?, {\ldots} \bluebold{when} do we bring her?''' \textstyleExampleSource{[081110-005-CvPr.0043-0044]}
\z

\ea
\label{Example_5.276}
\gll {kasiang,} {sa} {\bluebold{kapang}} {mandi} {deng} {dorang} {lagi} {e?}\\ %
 pity  \textsc{1sg}  when  bathe  with  \textsc{3pl}  again  eh\\
\glt
[About a sick boy:] ‘what a pity, \bluebold{when} will I bathe with them (my friends) again, eh?’ \textstyleExampleSource{[081025-009b-Cv.0044]}
\z


\subsection{\textitbf{knapa} ‘why’}
\label{Para_5.8.6}
The \isi{interrogative} \textitbf{knapa} ‘why’ questions reasons and motives. Its pronominal uses are illustrated in (\ref{Example_5.277}) to (\ref{Example_5.279}), its predicative uses in (\ref{Example_5.280}), and its uses in one-word utterances in (\ref{Example_5.281}).



Typically, \textitbf{knapa} ‘why’ is used pronominally. Most often it occurs in clause-initial position, as in (\ref{Example_5.277}). In clauses marked with an initial \isi{conjunction}, \textitbf{knapa} ‘why’ follows the \isi{conjunction} as in (\ref{Example_5.278}). Alternatively, but rarely, \textitbf{knapa} ‘why’ occurs between the subject and the predicate, as in (\ref{Example_5.279}). According to one consultant, this position of \textitbf{knapa} ‘why’ is acceptable; the semantics of this position still need to be investigated, though.


\begin{styleExampleTitle}
Pronominal uses of \textitbf{knapa} ‘why’
\end{styleExampleTitle}

\ea
\label{Example_5.277}
\gll {e,} {\bluebold{knapa}} {kam} {kas{\Tilde}kas} {bangung} {dia?}\\ %
 hey!  why  \textsc{2pl}  \textsc{rdp}{\Tilde}give  wake.up  \textsc{3sg}\\
\glt 
‘hey, \bluebold{why} do you keep waking him up?’ \textstyleExampleSource{[080918-001-CvNP.0039]}
\z

\ea
\label{Example_5.278}
\gll {tapi} {\bluebold{knapa}} {ana} {ini} {sakit?}\\ %
 but  why  child  \textsc{d.prox}  be.sick\\
\glt 
‘but \bluebold{why} is this child sick’ \textstyleExampleSource{[080917-010-CvEx.0133]}
\z

\ea
\label{Example_5.279}
\gll {{\ldots}} {Matius} {itu} {dia} {\bluebold{knapa}} {maju} {begitu?}\\ %
 { }   Matius  \textsc{d.dist}  \textsc{3sg}  why  advance  like.that\\
\glt 
‘[as for Matius, I’m very surprised,] Matius there, \bluebold{how come} he could advance like that?’ \textstyleExampleSource{[081006-032-Cv.0025]}
\z


Interrogative \textitbf{knapa} ‘what’ can also be used predicatively. In this case, \textitbf{knapa} ‘what’ remains in-situ, in the clause-final position, following the subject, as in (\ref{Example_5.280}).


\begin{styleExampleTitle}
Predicative uses of \textitbf{knapa} ‘why’
\end{styleExampleTitle}

\ea
\label{Example_5.280}
\gll {bapa} {ko} {\bluebold{knapa}?}\\ %
 father  \textsc{2sg}  why\\
\glt 
[After an accident]: ‘Sir, \bluebold{what happened}?’ (Lit. ‘you father (are) \bluebold{why}?’) \textstyleExampleSource{[081108-001-JR.0005]}
\z


The \isi{interrogative} can also form one-word utterances in which case it questions an entire proposition, as in (\ref{Example_5.281}).


\begin{styleExampleTitle}
One-word utterances with \textitbf{knapa} ‘why’
\end{styleExampleTitle}

\ea
\label{Example_5.281}
\gll {Speaker-2:} {} {e,} {\bluebold{knapa}?}\\ %
  {} {}  hey!  why\\
\glt
[About the birth of twins] [Speaker-1: ‘{\ldots} as for the girl, they say it’s an evil spirit, so they kill (her)’]\\
Speaker-2: ‘hey, \bluebold{why}?’ \textstyleExampleSource{[081011-022-Cv.0147-0151]}
\z


\subsection{Interrogative uses of mid-range {quantifier} \textitbf{brapa} ‘several’}
\label{Para_5.8.7}
In its \isi{interrogative} uses, the mid-range \isi{quantifier} \textitbf{brapa} ‘several’ receives the reading ‘how many’. It questions quantities of countable entities and, in combination with the mid-range \isi{quantifier} \textitbf{banyak} ‘many’, of non-countable entities. Its adnominal uses are shown in (\ref{Example_5.282}) to (\ref{Example_5.284}), and its predicative uses in (\ref{Example_5.286}) and (\ref{Example_5.289}).

Most often, \textitbf{brapa} ‘several, how many’ functions as a nominal modifier which takes the position of the \isi{numeral} or \isi{quantifier} it replaces. Corresponding to the syntax of adnominally used numerals and quantifiers, it precedes or follows its head nominal, as in (\ref{Example_5.282}) to (\ref{Example_5.284}). In prehead position of countable referents, \textitbf{brapa} ‘several, how many’ questions the absolute numbers of items denoted by the head nominals, as in (\ref{Example_5.282}). In posthead position of countable referents, it questions unique positions within series, as in (\ref{Example_5.283}). When following mass nouns, the \isi{interrogative} questions the nonnumeric amounts of its referents, as in the elicited example in (\ref{Example_5.284}). Like other quantifiers, \textitbf{brapa} ‘several, how many’ is unattested in prehead position of mass nouns. If the referent’s identity is known from the context, the head nominal can be omitted, as with numerals and other quantifiers. This is illustrated in (\ref{Example_5.285}), where the omitted head is \textitbf{rupia} ‘rupiah’. (Details on numerals and quantifiers are given in §\ref{Para_5.9} and §\ref{Para_5.10}, respectively.)


\begin{styleExampleTitle}
Adnominal uses of \textitbf{brapa} ‘how many’
\end{styleExampleTitle}

\ea
\label{Example_5.282}
\gll {\bluebold{brapa}} {\bluebold{bulang}} {dorang} {skola} {ka?}\\ %
 several  month  \textsc{3pl}  go.to.school  or\\
\glt 
‘(for) \bluebold{how many months} will they go to school?’ \textstyleExampleSource{[081025-003-Cv.0207]}
\z

\ea
\label{Example_5.283}
\gll {jadi} {mama,} {mama} {pulang} {\bluebold{jam}} {\bluebold{brapa}?}\\ %
 so  mother  mother  go.home  hour  several\\
\glt 
‘so mama, \bluebold{what time} will you (‘mother’) come home?’ (Lit. ‘\bluebold{how manyeth hour}’) \textstyleExampleSource{[080924-002-Pr.0002]}
\z

\ea
\label{Example_5.284}
\gll {ko} {minta} {\bluebold{minyak}} {\bluebold{brapa}} {\bluebold{banyak}?}\\ %
 \textsc{2pl}  request  oil  several  many\\
\glt 
‘\bluebold{how much oil} do you request’ \textstyleExampleSource{[Elicited BR120520.001]}
\z

\ea
\label{Example_5.285}
\gll {kemaring} {dapat} {\bluebold{brapa}} {\bluebold{Ø}?}\\ %
 yesterday  get  several  \\
\glt 
[Collecting money for a project:] ‘\bluebold{how many (rupiah)} did (you) get yesterday?’ \textstyleExampleSource{[080925-003-Cv.0090]}
\z


The predicative uses of \textitbf{brapa} ‘several, how many’ are shown in (\ref{Example_5.286}) to (\ref{Example_5.289}). Like \textitbf{siapa} ‘who’ (§\ref{Para_5.8.1}), \textitbf{apa} ‘what’ (§\ref{Para_5.8.2}), and \textitbf{mana} ‘where, which’ (§\ref{Para_5.8.3}), \textitbf{brapa} ‘several, how many’ can remain in the unmarked clause-final position, as in (\ref{Example_5.286}) and \ref{Example_5.288}), or it can be fronted to the marked clause-initial position, as in the elicited example in (\ref{Example_5.287}) and \ref{Example_5.289}). Again, the fronting of the \isi{interrogative} serves to emphasize the questioning, namely of numeric quantities in (\ref{Example_5.287}), and of nonnumeric quantities in (\ref{Example_5.289}). These two examples are elicited, though, as interrogatives with fronted \textitbf{brapa} ‘several, how many’ are unattested in the corpus. (See also §\ref{Para_12.3} for details on \isi{numeral} and \isi{quantifier} predicate clauses.)


\begin{styleExampleTitle}
Predicative uses of \textitbf{brapa} ‘how many’
\end{styleExampleTitle}

\ea
\label{Example_5.286}
\gll {bapa} {pu} {ana{\Tilde}ana} {\bluebold{brapa}?}\\ %
 father  \textsc{poss}  \textsc{rdp}{\Tilde}child  several\\
\glt 
‘\bluebold{how many} children do you (‘father’) have?’ (Lit. ‘father’s children are \bluebold{how many}?’) \textstyleExampleSource{[080923-009-Cv.0010]}
\z

\ea
\label{Example_5.287}
\gll {\bluebold{brapa}} {bapa} {pu} {ana{\Tilde}ana?}\\ %
 several  father  \textsc{poss}  \textsc{rdp}{\Tilde}child\\
\glt 
‘\bluebold{how many} children do you (‘father’) have?’ \textstyleExampleSource{[Elicited MY131112.007]}
\z

\ea
\label{Example_5.288}
\gll {tong} {pu} {uang} {\bluebold{brapa}?}\\ %
 \textsc{1pl}  \textsc{poss}  money  several\\
\glt 
‘\bluebold{how much} money do we have?’ (Lit. ‘our money is \bluebold{how many}?’) \textstyleExampleSource{[081006-017-Cv.0015]}
\z

\ea
\label{Example_5.289}
\gll {\bluebold{brapa}} {tong} {pu} {uang?}\\ %
 several  \textsc{1pl}  \textsc{poss}  money\\
\glt
‘\bluebold{how much} money do we have?’ \textstyleExampleSource{[Elicited MY131112.008]}
\z


\subsection{Interrogatives denoting {indefinite} referents}
\label{Para_5.8.8}
Cross-linguistically, interrogatives may also function as “general indefinites” by referring “to a general population, of unknown size” {\citep[401]{Dixon.2010}}. In this case, the interrogatives translate with ‘whoever’, ‘whatever’, ‘wherever’, etc.

In Papuan Malay, the \isi{indefinite} reading is achieved by juxtaposing the focus ad\isi{verb} \textitbf{saja} ‘just’ to the \isi{interrogative}, as illustrated in (\ref{Example_5.290}) to (\ref{Example_5.295}). In the corpus, this function of the interrogatives is only attested for \textitbf{siapa} ‘who’, \textitbf{apa} ‘what’, and \textitbf{mana} ‘where, which’, as shown in (\ref{Example_5.290}) to (\ref{Example_5.293}). The elicited respective examples in (\ref{Example_5.293}) to (\ref{Example_5.295}) illustrate, however, that \textitbf{bagemana} ‘how’, \textitbf{kapang} ‘when’, and \textitbf{knapa} ‘why’ can also have this function.


\ea
\label{Example_5.290}
\gll {kalo} {ko} {liat} {\bluebold{ko}} {\bluebold{pu}} {\bluebold{sodara}} {\bluebold{siapa}} {\bluebold{saja},} {kalo} {dia} {{\ldots}}\\ %
 if  \textsc{2sg}  see  \textsc{2sg}  \textsc{poss}  sibling  who  just  if  \textsc{3sg}  \\
\glt 
‘when you see \bluebold{your relatives whoever} (they are), when he/she {\ldots}’ \textstyleExampleSource{[080919-004-NP.0078]}
\z

\ea
\label{Example_5.291}

\gll  bicara  \bluebold{apa}  \bluebold{saja},  bicara  saja\\
 speak  what  just  speak  just\\
\glt 
‘speak (to me about) \bluebold{whatever}, just speak (to me)’ \textstyleExampleSource{[080922-001a-CvPh.1174]}
\z

\ea
\label{Example_5.292}

\gll  di  \bluebold{mana}  \bluebold{saja}  bapa  bisa  tinggal,  di  \bluebold{tempat}  \bluebold{mana}  \bluebold{saja}\\
 at  where  just  father  be.able  stay  at  place  where  just\\
\glt 
‘I (‘father’) can live \bluebold{wherever}, (I can) live in \bluebold{whatever place}’ \textstyleExampleSource{[080922-001a-CvPh.1116]}
\z

\ea
\label{Example_5.293}
\gll {sa} {tra} {mo} {taw!,} {\bluebold{bagemana}} {\bluebold{saja}} {ko} {harus} {pigi} {skola!}\\ %
 \textsc{1sg}  \textsc{neg}  want  know  how  just  \textsc{2sg}  have.to  go  school\\
\glt 
[Addressing a child who does not want to go to school for various reasons:] ‘I don’t want to know!, you have to go to school, \bluebold{no matter what}!’ (Lit. ‘\bluebold{however}’) \textstyleExampleSource{[Elicited MY131112.001]}
\z

\ea
\label{Example_5.294}

\gll  \bluebold{kapang}  \bluebold{saja}  ko  bisa  datang\\
 when  just  \textsc{2sg}  be.able  come\\
\glt 
‘you can come \bluebold{whenever}’ \textstyleExampleSource{[Elicited MY131112.002]}
\z

\ea
\label{Example_5.295}
\gll {\bluebold{knapa}} {\bluebold{saja}} {sa} {pu} {kaka} {de} {mo} {pulang} {Jayapura}\\ %
 why  just  \textsc{1sg}  \textsc{poss}  oSb  \textsc{3sg}  want  go.home  Jayapura\\
\glt
‘my older sibling wants to return to Jayapura, \bluebold{for whatever reason}’ (Lit. ‘\bluebold{whyever}’) \textstyleExampleSource{[Elicited MY131112.002]}
\z


\subsection{Summary}\label{Para_5.8.9}

In requesting specific types of information, the Papuan Malay interrogatives have a variety of functions within the clause. All of them have pronominal uses. In addition, five them also have predicative uses; the exception is \textitbf{kapang} ‘when’. Furthermore, three of them also have adnominal and/or \isi{placeholder} uses.



In their pronominal and adnominal uses, the interrogatives typically remain in-situ. Besides, pronominally used \textitbf{kapang} ‘when’ and \textitbf{knapa} ‘why’ may also occur in a clause-internal position, between the subject and the predicate. This position, however, is rare and the semantics conveyed still need to be investigated. In their predicative uses, four of the interrogatives can take two positions. That is, they can remain in situ, that is, in the clause-final position, or they can be fronted to the clause-initial position. When speakers want to accentuate the subject, the \isi{interrogative} remains in-situ in the unmarked clause-final position. When, by contrast, speakers want to emphasize the fact that they are requesting specific types of information, such as the identity of the subject or its location, they front the \isi{interrogative} to the marked clause-initial position where it is more salient. The exception is \textitbf{knapa} ‘why’, which is unattested clause-initially.



The mid-range \isi{quantifier} \textitbf{brapa} ‘several’ also functions as an \isi{interrogative}, questioning quantities in the sense of ‘how many’. It has, however, only adnominal and predicative, and no pronominal uses. In its adnominal uses, it remains in-situ, while in its predicative uses it can take two positions: it can remain in-situ or be fronted to the clause-initial position.


\section{Numerals}
\label{Para_5.9}
As numeric expressions, the Papuan Malay numerals designate countable divisions of their referents. Cardinal numbers are presented in §\ref{Para_5.9.1}, ordinal numbers in §\ref{Para_5.9.2}, and distributive numbers in §\ref{Para_5.9.3}. In §\ref{Para_5.9.4} an additional non-enumerating function of the \isi{numeral} \textitbf{satu} ‘one’ is presented.


\subsection{Cardinal numerals}
\label{Para_5.9.1}
Papuan Malay has a decimal \isi{numeral} system. The basic cardinal numerals, along with some examples of how they are combined, are presented in \tabref{Table_5.35}.


\begin{table}

\caption[Basic Papuan Malay cardinal numerals]{Basic Papuan Malay cardinal numerals\footnote{The numerals \textitbf{sratus} ‘one hundred’ and \textitbf{sribu} ‘one thousand’ are \isi{historically derived} by unproductive \isi{affixation} with the prefix \textitbf{s(e)-}.}}\label{Table_5.35}

\begin{tabularx}{\textwidth}{rp{35mm}rX}
\lsptoprule
 \# &\multicolumn{1}{c}{Numbers} &  \multicolumn{1}{c}{\#} &  \multicolumn{1}{c}{Numbers}\\
\midrule

 1 & \textitbf{satu} &  100 & \textitbf{sratus}\\
&  &  & one:hundred\\\tablevspace
 2 & \textitbf{dua} &  102 & \textitbf{sratus dua}\\
&  &  & one:hundred two\\\tablevspace
 3 & \textitbf{tiga} &  200 & \textitbf{dua ratus}\\
&  &  & two hundred\\\tablevspace
 4 & \textitbf{empat} &  234 & \textitbf{dua ratus tiga pulu empat}\\
&  &  & two hundred three tens four\\\tablevspace
 5 & \textitbf{lima} &  1,000 & \textitbf{sribu}\\
&  &  & one:thousand\\\tablevspace
 6 & \textitbf{enam} &  1,004 & \textitbf{sribu empat}\\
&  &  & one:thousand four\\\tablevspace
 7 & \textitbf{tuju} &  2,000 & \textitbf{dua ribu}\\
&  &  & two thousand\\\tablevspace
 8 & \textitbf{dlapang} &  2,013 & \textitbf{dua ribu tiga blas}\\
&  &  & two thousand three teens\\\tablevspace
 9 & \textitbf{sembilang} &  10,000 & \textitbf{spulu ribu}\\
&  &  & one:tens thousand\\\tablevspace
 10 & \textitbf{spulu} &  32,000 & \textitbf{tiga pulu dua ribu}\\
& one:tens &  & three tens two thousand\\\tablevspace
 11 & \textitbf{seblas} &  980,000 & \textitbf{sembilang ratus dlapang pulu ribu}\\
& one:teens &  & nine hundreds eight tens thousand\\\tablevspace
 12 & \textitbf{dua blas} &  1,000,000 & \textitbf{satu juta}\\
& two teens &  & one million\\\tablevspace
 20 & \textitbf{dua pulu} &  1,000,000,000 & \textitbf{satu milyar}\\
& two tens &  & one billion\\\tablevspace
 21 & \textitbf{dua pulu satu} &  zero & \textitbf{kosong}\\
& two tens one &  & be empty\\\tablevspace
 30 & \textitbf{tiga pulu} &  & \\
& three tens &  & \\
\lspbottomrule
\end{tabularx}

\end{table}

As illustrated in \tabref{Table_5.35}, complex numerals are formed by indicating the number of units of the highest power of ten, followed by the number of units of the next lower power down to the simple units or digits of one to ten. The individual components of complex numbers are combined by \isi{juxtaposition}. The formulas for forming complex numerals are presented in (\ref{Example_5.296}) and (\ref{Example_5.297}):


\begin{styleExampleTitle}
Formulas for complex numerals
\end{styleExampleTitle}

\ea
\label{Example_5.296}
\textit{Complex numerals with tens (\textitbf{pulu})}\\ %
{(\textsc{digit} \textitbf{juta}) (\textsc{digit} \textitbf{ribu}) (\textsc{digit} \textitbf{ratus}) (\textsc{digit} \textitbf{pulu}) \textsc{digit}}\\
\z
\ea
\label{Example_5.297}
\textit{Complex numerals with teens (\textitbf{blas}).}\\ %
{(\textsc{digit} \textitbf{juta}) (\textsc{digit} \textitbf{ribu}) (\textsc{digit} \textitbf{ratus}) \textsc{digit} \textitbf{blas}}\\
\z


Most often, cardinal numerals are used adnominally to enumerate entities. In this function they may precede or follow their head nominal. With a preposed \isi{numeral}, the \isi{noun} phrase signals the absolute number of items denoted by the head nominal, as in \textitbf{lima orang} ‘lima people’ in (\ref{Example_5.298}). Thereby the composite nature of countable referents is underlined. Posthead numerals, by contrast, express exhaustivity of \isi{definite} referents such as \textitbf{pace dua ini} ‘both of these men’ in (\ref{Example_5.299}), or denote unique positions within a series. (For details on the adnominal uses of numerals see §\ref{Para_8.3.1}.)


\begin{styleExampleTitle}
Adnominally used numerals
\end{styleExampleTitle}

\ea
\label{Example_5.298}
\gll {mungking} {\bluebold{lima}} {\bluebold{orang}} {mati}\\ %
 maybe  five  person  die\\
\glt 
‘about \bluebold{five people} died’ \textstyleExampleSource{[081025-004-Cv.0033]}
\z

\ea
\label{Example_5.299}
\gll {\bluebold{pace}} {\bluebold{dua}} {\bluebold{ini}} {dong} {dua} {dari} {pedalamang}\\ %
 man  two  \textsc{d.dist}  \textsc{3pl}  two  from  interior\\
\glt 
‘\bluebold{both these men}, the two of them are from the interior’ \textstyleExampleSource{[081109-010-JR.0001]}
\z


When the identity of the referent was established earlier or can be deduced from the context, the head nominal can be omitted, as in (\ref{Example_5.300}).


\begin{styleExampleTitle}
Numerals with omitted head nominal
\end{styleExampleTitle}

\ea
\label{Example_5.300}
\gll {Ika} {biasa} {angkat} {itu} {\bluebold{dlapang}} {\bluebold{pulu}} {\bluebold{sembilang}} {Ø}\\ %
 Ika  be.usual  lift  \textsc{d.dist}  eight  ten  nine  \\
\glt 
‘Ika usually lifts, what’s-its-name, \bluebold{eighty-nine} (kilogram)’ \textstyleExampleSource{[081023-003-Cv.0004]}
\z


The examples in (\ref{Example_5.298}) and (\ref{Example_5.300}) also illustrate that numerals can be used with countable nouns that are animate or inanimate, respectively.



In addition to their adnominal uses, numerals are used predicatively. In (\ref{Example_5.301}), for example, the \isi{numeral} \textitbf{dua blas} ‘twelve’ functions as a predicate that provides information about the numeric quantity of its subject \textitbf{de} ‘\textsc{3sg}’ (‘the moon’). (For details on \isi{numeral} predicate clauses see §\ref{Para_12.3}).


\begin{styleExampleTitle}
Predicatively used numerals
\end{styleExampleTitle}

\ea
\label{Example_5.301}
\gll {di} {kalender} {de} {\bluebold{dua}} {\bluebold{blas}}\\ %
 at  calendar  \textsc{3sg}  two  teens\\
\glt 
‘in the calendar\bluebold{ }there are\bluebold{ twelve} (moons)’ (Lit. ‘it (the moon) \bluebold{is twelve}’) \textstyleExampleSource{[081109-007-JR.0002]}
\z


The basic mathematical functions of the cardinal numerals are presented in \tabref{Table_5.36} .


\begin{table}
\caption{Mathematical functions}\label{Table_5.36}
\begin{tabular}{lll}
\lsptoprule
 Item & Sign &  Gloss\\
\midrule

\textitbf{tamba} & + & ‘plus’\\
add &  & \\
\textitbf{kurang} & – & ‘minus’\\
lack &  & \\
\textitbf{kali} & x & ‘times’\\
time &  & \\
\textitbf{bagi} & / & ‘divide’\\
divide &  & \\
\lspbottomrule
\end{tabular}
\end{table}

In natural conversations, however, calculations occur only very rarely. Therefore, the following examples are elicited: the function of addition is presented in (\ref{Example_5.302}), subtraction in (\ref{Example_5.303}), multiplication in (\ref{Example_5.304}), and division in (\ref{Example_5.305}).


\begin{styleExampleTitle}
Addition
\end{styleExampleTitle}

\ea
\label{Example_5.302}
\gll {dua} {babi} {\bluebold{tamba}} {tiga} {babi} {sama} {dengang} {lima} {babi}\\ %
 two  pig  add  three  pig  be.same  with  five  pig\\
\glt 
‘two pigs \bluebold{plus} three pigs are five pigs’ \textstyleExampleSource{[Elicited BR120820.001]}
\z

\begin{styleExampleTitle}
Subtraction
\end{styleExampleTitle}
\ea
\label{Example_5.303}
\gll {lima} {babi} {\bluebold{kurang}} {tiga} {babi} {sama} {dengang} {dua} {babi}\\ %
 five  pig  lack  three  pig  be.same  with  two  pig\\
\glt 
‘five pigs \bluebold{minus} three pigs are two pigs’ \textstyleExampleSource{[Elicited BR120820.002]}
\z

\begin{styleExampleTitle}
Multiplication
\end{styleExampleTitle}

\ea
\label{Example_5.304}
\gll {dua} {babi} {\bluebold{kali}} {tiga} {babi} {sama} {dengang} {enam} {babi}\\ %
 two  pig  time  three  pig  be.same  with  six  pig\\
\glt 
‘two pigs \bluebold{times} three pigs are six pigs’ \textstyleExampleSource{[Elicited BR120820.003]}
\z

\begin{styleExampleTitle}
Division
\end{styleExampleTitle}

\ea
\label{Example_5.305}
\gll {enam} {babi} {\bluebold{bagi}} {tiga} {babi} {sama} {dengang} {dua} {babi}\\ %
 six  pig  divide  three  pig  be.same  with  two  pig\\
\glt
‘six pigs \bluebold{divided} (by) three pigs are two pigs’ \textstyleExampleSource{[Elicited BR120820.004]}
\z


\subsection{Ordinal numerals}
\label{Para_5.9.2}
Papuan Malay employs two strategies to express the notion of ordinal numerals. For kinship terms the concept of ordinal numerals is encoded by a ``NNum'' \isi{noun} phrase headed by the \isi{noun} \textitbf{nomor} ‘number’, as shown in (\ref{Example_5.306}) and (\ref{Example_5.307}). This \isi{noun} phrase ``\textitbf{nomor} Num'' gives the ordinal reading ``Num-th'' such as \textitbf{nomor tiga} ‘third’ in (\ref{Example_5.306}) or \textitbf{nomor empat} ‘fourth’ in the elicited example in (\ref{Example_5.307}).


\begin{styleExampleTitle}
Inherited strategy
\end{styleExampleTitle}

\ea
\label{Example_5.306}
\gll {saya} {{tida}} {bole} {{kasi}} {{sama}} {{bapa}} {{punya}} {{sodara}}\\ %
 \textsc{1sg}  {\textsc{neg}}  may  {give}  {to}  {father}  {\textsc{poss}}  {sibling}\\
\gll {ana}  {prempuang}  {yang}  {sa}  {bilang}  {\bluebold{nomor}}  \bluebold{tiga}\\
 {child}  {woman}  {\textsc{rel}}  {\textsc{1sg}}  {say}  {number}  three\\
\glt 
[About bride-price children:] ‘I shouldn’t have given to father’s sibling the daughter that, as I said, was (my) \bluebold{third (child)}’ (Lit. ‘\bluebold{number three}’) \textstyleExampleSource{[081006-024-CvEx.0088]}
\z

\ea
\label{Example_5.307}
\gll {Aleks} {ini} {sa} {pu} {tete} {pu} {ade} {\bluebold{nomor}} {\bluebold{empat}}\\ %
 Aleks  \textsc{d.prox}  \textsc{1sg}  \textsc{poss}  grandfather  \textsc{poss}  ySb  number  four\\
\glt 
‘Aleks here is my grandfather’s \bluebold{fourth} youngest sibling’ (Lit. ‘\bluebold{number four}’) \textstyleExampleSource{[Elicited BR120821.002]}
\z


According to one consultant, the strategy presented in (\ref{Example_5.306}) and (\ref{Example_5.307}) is the inherited Papuan Malay strategy to express the notion of ordinal numbers. This strategy used to be employed not only for kinship terms but for countable nouns in general. With the increasing influence of \ili{Standard Indonesian}, however, Papuan Malay speakers have started employing ordinal numbers of Indonesian origins more frequently. Hence, in the corpus the ordinal numbers for countable nouns other than kinship terms are of \ili{Standard Indonesian} origins, such as \textitbf{kedua} ‘second’ in (\ref{Example_5.308}) or \textitbf{ketiga} ‘third’ in the elicited example in (\ref{Example_5.309}).\footnote{In \ili{Standard Indonesian}, ordinal numerals, with the exception of \textitbf{pertama} ‘first’, are derived by prefixing \textitbf{ke-} to the cardinal number (for details see \citealt[293]{Mintz.1994}).}


\begin{styleExampleTitle}
Borrowed strategy
\end{styleExampleTitle}

\ea
\label{Example_5.308}
\gll {distrik} {\bluebold{kedua}} {di} {mana}\\ %
 district  second  at  where\\
\glt 
‘where is the \bluebold{second} district?’ \textstyleExampleSource{[081010-001-Cv.0071]}
\z

\ea
\label{Example_5.309}
\gll {ini} {bibit} {nangka} {yang} {\bluebold{ketiga}} {yang} {sa} {bli}\\ %
 \textsc{d.prox}  \textsc{1sg}  \textsc{poss}  \textsc{rel}  third  \textsc{rel}  \textsc{1sg}  buy\\
\glt
‘this is the \bluebold{third} jackfruit seedling that I bought’ \textstyleExampleSource{[Elicited BR120821.003]}
\z


\subsection{Distributive numerals}
\label{Para_5.9.3}
Distributive numerals express that “a property or action” applies “to the individual members of a group, as opposed to the group as a whole” {\citep[154]{Crystal.2008}}. In Papuan Malay, this notion of ``one by one'' or ``two by two'' is expressed through \isi{reduplication} of the \isi{numeral}. This is illustrated with \textitbf{satu{\Tilde}satu} ‘one by one’ or ‘in groups of one each’ in (\ref{Example_5.310}), and with \textitbf{dua{\Tilde}dua} ‘two by two’ or ‘in groups of two’ in (\ref{Example_5.311}). (See also §\ref{Para_4.2.4}.)


\ea
\label{Example_5.310}
\gll {tong} {tiga} {cari} {jalang} {\bluebold{satu{\Tilde}satu}}\\ %
 \textsc{1pl}  three  search  walk  \textsc{rdp}{\Tilde}one\\
\glt 
‘the three of us looked for a path (through the river) \bluebold{one-by-one}’ \textstyleExampleSource{[081013-003-Cv.0003]}
\z

\ea
\label{Example_5.311}
\gll {tong} {minum} {\bluebold{dua{\Tilde}dua}} {\bluebold{glas}} {ato} {\bluebold{tiga{\Tilde}tiga}} {\bluebold{glas}}\\ %
 \textsc{1pl}  drink  \textsc{rdp}{\Tilde}two  glass  or  \textsc{rdp}{\Tilde}three  glass\\
\glt
[About the lack of water during a retreat:] ‘we drank \bluebold{two glasses each} or \bluebold{three glasses each} (per day)’ (Lit. ‘\bluebold{two by two} or \bluebold{three by three}’) \textstyleExampleSource{[081025-009a-Cv.0069]}
\z


\subsection{Additional function of \textitbf{satu} ‘one’}
\label{Para_5.9.4}
In addition to its enumerating function in postposed position, adnominally used \textitbf{satu} ‘one’ is employed to encode “specific \isi{indefiniteness}”, using Crystal’s (\citeyear*[444]{Crystal.2008}) terminology. That is, in \textsc{nn}um-\textsc{np}s adnominal \textitbf{satu} ‘one’ denotes specific but nonidentifiable referents, giving the specific \isi{indefinite} reading \textscItal{n} \textitbf{satu} ‘a certain \textsc{n}’. The specific \isi{indefinite} referent may be animate human such as \textitbf{ade satu} ‘a certain younger sibling’ in (\ref{Example_5.312}) or inanimate such as \textitbf{kampung satu} ‘a certain village’ in (\ref{Example_5.313}). The referent of \textitbf{ojek satu} in (\ref{Example_5.313}) can be interpreted as the animate referent ‘motorbike taxi driver’, or as the inanimate referent ‘motorbike taxi’.


\ea
\label{Example_5.312}
\gll {ada} {\bluebold{ade}} {\bluebold{satu}} {di} {situ}\\ %
 exist  ySb  one  at  \textsc{l.med}\\
\glt 
‘(there) is \bluebold{a certain younger sibling} there’ \textstyleExampleSource{[080922-004-Cv.0018]}
\z

\ea
\label{Example_5.313}
\gll {sa} {{pas}} {{jalang}} {{kaki}} {sampe} {di} {\bluebold{kampung}} {\bluebold{satu}} {Wareng}\\ %
 \textsc{1sg}  {precisely}  {walk}  {foot}  reach  at  village  one  Wareng\\
 \gll {ada}  {\bluebold{ojek}}  {\bluebold{satu}}  {turung}\\
 {exist}  {motorbike.taxi}  {one}  {descend}\\
\glt
‘right at the moment when I was walking on foot as far as \bluebold{a certain village} (named) Wareng, there was \bluebold{a certain motorbike taxi (driver)} that(/who) came down (the road)’ \textstyleExampleSource{[080923-010-CvNP.0001]}
\z


\section{Quantifiers}
\label{Para_5.10}
As nonnumeric expressions, the Papuan Malay quantifiers denote \isi{definite} or \isi{indefinite} quantities of their referents. The universal and mid-range quantifiers are discussed in §\ref{Para_5.10.1}, and distributive quantifiers in §\ref{Para_5.10.2}.


\subsection{Universal and mid-range quantifiers}
\label{Para_5.10.1}
The Papuan Malay quantifiers are listed in \tabref{Table_5.37}, following Gil’s (\citeyear*{Gil.2013b})  distinction of universal and mid-range quantifiers.\footnote{Following {\citet[1]{Gil.2013b}}, the expression \textitbf{sembarang} ‘any’ is a “free-choice \isi{universal quantifier}”.}


\begin{table}
\caption{Papuan Malay quantifiers}\label{Table_5.37}

\begin{tabular}{ll}
\lsptoprule

\multicolumn{2}{c}{ Universal quantifiers}\\
\midrule
\textitbf{masing-masing} & ‘all’\\
\textitbf{segala} & ‘all’\\
\textitbf{sembarang} & ‘any (kind of)’\\
\textitbf{(se)tiap} & ‘every’\\
\textitbf{smua} & ‘all’\\
\midrule
\multicolumn{2}{c}{ Mid-range quantifiers}\\
\midrule
\textitbf{banyak} & ‘many’\\
\textitbf{brapa} & ‘several’\\
\textitbf{sedikit} & ‘few’\\
\textitbf{stenga} & ‘half’\\
\lspbottomrule
\end{tabular}
\end{table}

Noun phrases with adnominal quantifiers have syntactic properties similar to those with adnominal numerals, as illustrated in (\ref{Example_5.314}) to (\ref{Example_5.322}). Noun phrases with prehead quantifiers (``\textsc{q}t\textsc{n-np}'') express nonnumeric amounts or quantities of the items indicated by their head nominals. Thereby, the composite nature of countable referents is accentuated. Posthead quantifiers, by contrast, may denote exhaustivity of \isi{indefinite} referents or signal unknown positions within series or sequences; they modify countable as well as uncountable referents.



The data in (\ref{Example_5.314}) to (\ref{Example_5.322}) shows that not all quantifiers occur in all positions. Only five quantifiers occur in either pre- or posthead position, namely \textitbf{banyak} ‘many’, \textitbf{brapa} ‘several’, \textitbf{masing-masing} ‘each’, \textitbf{sedikit} ‘few’, and \textitbf{smua} ‘all’. While \textitbf{banyak} ‘many’, \textitbf{sedikit} ‘few’, and \textitbf{smua} ‘all’ can modify both count and mass nouns, \textitbf{brapa} ‘several’ and \textitbf{masing-masing} ‘each’ only modify count nouns. The remaining four quantifiers occur in prehead position only, namely \textitbf{segala} ‘all’, \textitbf{sembarang} ‘any (kind of)’, \textitbf{(se)tiap} ‘every’, and \textitbf{stenga} ‘half’. These quantifiers modify count nouns only.



Five of the quantifiers are used with either animate or inanimate referents, namely \textitbf{banyak} ‘many’, \textitbf{brapa} ‘several’, \textitbf{masing-masing} ‘each’, \textitbf{sedikit} ‘few’, and \textitbf{smua} ‘all’. By contrast, \textitbf{sembarang} ‘any (kind of)’ is only used with animate referents, and \textitbf{(se)tiap} ‘every’ and \textitbf{stenga} ‘half’ only with inanimate referents.\footnote{To express the notion of ‘every person’, speakers prefer quantification with \textitbf{masing-masing} ‘each’.} Universal \textitbf{segala} ‘all’ is only used in combination with the \isi{noun} \textitbf{macang} ‘variety’. (The mid-range \isi{quantifier} \textitbf{brapa} ‘several’ also functions as an \isi{interrogative}, which questions quantities in the sense of ``how many'', as discussed in §\ref{Para_5.8.7}). (For details on the adnominal uses of the quantifiers see §\ref{Para_8.3.2}.)
 
 


\begin{styleExampleTitle}
Adnominal quantifiers in preposed and postposed position\footnotetext{Documentation: \textitbf{banyak} ‘many’ 081006-023-CvEx.0007, 081029-004-Cv.0021, 081011-001-Cv.0240; \textitbf{brapa} ‘several’ 080919-001-Cv.0049, 080923-008-Cv.0012; \textitbf{masing-masing} ‘each’ BR111021.010, BR111021.009; \textitbf{sedikit} ‘few’ BR111021-001.004, BR111021-001.006, 081006-035-CvEx.0050; \textitbf{segala} ‘all’ 081006-032-Cv.0017; \textitbf{sembarang} ‘any’ 080927-006-CvNP.0035; \textitbf{setiap} ‘every’ 080923-016-CvNP.0002; \textitbf{smua} ‘all’ 081006-030-CvEx.0009, 080921-004b-Cv.0026, BR111021.012; \textitbf{stenga} 081115-001b-Cv.0056.}
\end{styleExampleTitle}

\newcommand{\akexbox}[2][-.5\baselineskip]{ 
    \parbox[t]{.4\textwidth}{
      \vspace{#1}
      #2
     }  
}


\ea
\label{Example_5.314}
\ea
\akexbox{
Count \textsc{n}: Prehead position\\
\label{Example_5.314a}
\gll  \bluebold{banyak} orang\\
     many person\\
\glt      ‘\bluebold{many} people’ \\
}
\akexbox{
Count \textsc{n}: Posthead position\\
\gll  orang \bluebold{banyak}\\ %
  person many\\
\glt ‘\bluebold{many} people’\\
}

\medskip

 
 \ex
\akexbox{~}
\akexbox{
Mass \textsc{n}: Posthead position\\
 \label{Example_5.314b}
 \gll te \bluebold{banyak}\\
       tea many\\
\glt        ‘\bluebold{lots} (of) tea’\\
}
\z
\z


\ea
\label{Example_5.315}
\ea
\akexbox{
Count \textsc{n}: Prehead position\\
\label{Example_5.315a}
\gll  \bluebold{sedikit} orang\\
few person \\
\glt  ‘\bluebold{few} people’ \\
}
\akexbox{
Count \textsc{n}: Posthead position\\
\gll  kladi \bluebold{sedikit}\\
 taro.root few\\
\glt  ‘\bluebold{few} taro roots’\\
}
\medskip
\ex
\akexbox{~}
\akexbox{
Mass \textsc{n}: Posthead position\\
\label{Example_5.315b}
\gll air \bluebold{sedikit}\\
water few\\
\glt ‘\bluebold{little} water’\\
}
\z
\z

\ea
\label{Example_5.316}
\ea
\akexbox{
\label{Example_5.316a}
Count \textsc{n}: Prehead position\\
\gll  \bluebold{smua} masala\\
all problem \\
\glt  ‘\bluebold{all} problems’ \\
}
\akexbox{
Count \textsc{n}: Posthead position\\
\gll pemuda \bluebold{smua}\\
 youth all\\
\glt ‘\bluebold{all} (of) the young people’\\
}
\medskip
\ex
\akexbox{~}  
\akexbox{
\label{Example_5.316b}
Mass \textsc{n}: Posthead position\\
\gll  gula \bluebold{smua}\\
 sugar all\\
\glt  ‘\bluebold{all} (of the) sugar’\\
}
\z

\z

\ea
\label{Example_5.317}
\akexbox{
Count \textsc{n}: Prehead position\\
\gll  \bluebold{brapa} orang \\
several person\\
\glt  ‘\bluebold{several} people’ \\
}
\hspace{.5cm}
\akexbox{
Count \textsc{n}: Posthead position\\
\gll dorang \bluebold{brapa}\\
 \textsc{3pl} several\\
\glt  ‘\bluebold{several} (of) them’\\
}
\z


\ea
\label{Example_5.318}
\akexbox{
Count \textsc{n}: Prehead position\\
\gll \bluebold{masing-masing} trek\\
each truck\\
\glt  ‘\bluebold{each} truck’ \\ 
}
\hspace{.5cm}
\akexbox{
Count \textsc{n}: Posthead position\\
\gll trek \bluebold{masing-masing}\\
 truck each\\
\glt  ‘\bluebold{each} truck’\\
}
\z


\ea
\label{Example_5.319}
Count \textsc{n}: Prehead position\\
\gll \bluebold{segala} macang \\
 all variety \\
\glt  ‘\bluebold{every}thing, what\bluebold{ever}’
\z



\ea
\label{Example_5.320}
Count \textsc{n}: Prehead position\\
\gll \bluebold{sembarang} orang\\
any person\\
\glt  ‘\bluebold{any} person, \bluebold{any}body’ \\
\z

\ea
\label{Example_5.321}
Count \textsc{n}: Prehead position\\
\gll \bluebold{setiap} lagu \\
every song\\
\glt  ‘\bluebold{every} song’ 
\z

\ea
\label{Example_5.322}
Count \textsc{n}: Prehead position\\
\gll  \bluebold{stenga} jam\\
 half hour\\
\glt  ‘\bluebold{half} an hour’
\z





When the identity of the referent was established earlier or can be deduced from the context, the head nominal can be omitted. Not all quantifiers, however, are used in \isi{noun} phrases with elided head nominal. Attested are only \textitbf{banyak} ‘many’ as in (\ref{Example_5.323}), \textitbf{brapa} ‘several’ as in (\ref{Example_5.324}), \textitbf{sedikit} ‘few’ as in (\ref{Example_5.325}), and \textitbf{smua} ‘all’ as in (\ref{Example_5.326}).


\begin{styleExampleTitle}
Quantifiers with omitted head nominal
\end{styleExampleTitle}

\ea
\label{Example_5.323}
\gll {\bluebold{banyak}} {Ø} {mati} {di,} {e,} {di} {di} {pulow{\Tilde}pulow,} {\bluebold{banyak}} {Ø} {mati} {di} {lautang}\\ %
 many {}   die  at  uh  at  at  \textsc{rdp}{\Tilde}island  many {}    die  at  ocean\\
\glt 
‘[there are many Papuans who died,] \bluebold{many} (Papuans) died on, uh, on on the islands, \bluebold{many} (Papuans) died on the ocean’ \textstyleExampleSource{[081029-002-Cv.0024-0025]}
\z

\ea
\label{Example_5.324}
\gll {kalo} {{suda}} {ambil} {satu} {Ø,} {kasiang,} {kitong} {hanya}\\ %
 if  {already}  take  one {}   pity  \textsc{1pl}  only\\
 \gll {\bluebold{brapa}}  Ø  {saja}\\
 {several} {}   {just}\\
\glt 
‘once (they) have taken one (of our children), what a pity, we (have) just a \bluebold{few} (children left)’ (Lit. ‘only \bluebold{several}’) \textstyleExampleSource{[081006-024-CvEx.0070]}
\z

\ea
\label{Example_5.325}
\gll {di} {sini} {yo} {fam} {Yapo} {ini} {ada} {\bluebold{sedikit}} {Ø}\\ %
 at  \textsc{l.prox}  yes  family.name  Yapo  \textsc{d.prox}  exist  few  \\
\glt 
‘here, yes, there are (only) \bluebold{few} Yapo family (members)’ (Lit. ‘this Yapo family is \bluebold{few} (people)’) \textstyleExampleSource{[080922-010a-CvNF.0274]}
\z

\ea
\label{Example_5.326}
\gll {{\ldots}} {mobil} {blakos,} {Ø} {\bluebold{smua}} {naik} {di} {blakang}\\ %
  { }  car  pickup.truck  {}  all  climb  at  backside\\
\glt
‘[we took] a pickup truck, \bluebold{all} (of the passengers) got onto its loading space’ \textstyleExampleSource{[081006-017-Cv.0001]}
\z


In addition to their adnominal uses, quantifies are also used predicatively. In (\ref{Example_5.327}), for instance, predicatively used \textitbf{banyak} ‘many’ conveys information about the nonnumeric quantity of its subject \textitbf{picaang} ‘splinter’. (For details on \isi{quantifier} predicate clauses see §\ref{Para_12.3}).


\begin{styleExampleTitle}
Predicatively used \isi{quantifier}
\end{styleExampleTitle}

\ea
\label{Example_5.327}
\gll {{\ldots}} {picaang} {juga} {\bluebold{banyak}}\\ %
  { }  splinter  also  many\\
\glt
‘[at the beach] there are also\bluebold{ lots} (of) splinters’ (Lit. ‘the splinters (are) also\bluebold{ many}’) \textstyleExampleSource{[080917-006-CvHt.0008]}
\z


\subsection{Distributive quantifiers}
\label{Para_5.10.2}
The notion of ``little by little'' or ``many by many'' is expressed through \isi{reduplication} of the \isi{quantifier}, similar to the formation of distributive numerals presented in (\ref{Example_5.310}) and (\ref{Example_5.311}) (§\ref{Para_5.9.3}; see also §\ref{Para_4.2.4}). Distributive quantifiers denote events that affect an \isi{indefinite} number of members of a group or set at different points in time. In (\ref{Example_5.328}), for example, \textitbf{uang banyak{\Tilde}banyak} denotes ‘sets of lots of money’, while in (\ref{Example_5.329}) \textitbf{sedikit{\Tilde}sedikit} designates ‘sets of little (food)’.


\ea
\label{Example_5.328}
\gll {bapa} {kirim} {\bluebold{uang}} {\bluebold{banyak{\Tilde}banyak}}\\ %
 father  send  money  \textsc{rdp}{\Tilde}many\\
\glt 
[Phone conversation:] ‘father send \bluebold{lots of money at regular intervals}’ (Lit. ‘\bluebold{lots by lots of money}’) \textstyleExampleSource{[080922-001a-CvPh.0440]}
\z

\ea
\label{Example_5.329}
\gll {dong} {blum} {isi} {selaing} {dong} {isi} {\bluebold{sedikit{\Tilde}sedikit}} {to?}\\ %
 \textsc{3pl}  not.yet  fill  besides  \textsc{3pl}  fill  \textsc{rdp}{\Tilde}few  right?\\
\glt
[About organizing the food distribution during a retreat:] ‘they haven’t yet filled (their plates), moreover they’ll fill (their plates only) with \bluebold{little} (food), right?’ (Lit. ‘\bluebold{little by little (food)}’) \textstyleExampleSource{[081025-009a-Cv.0081]}
\z


\section{Prepositions}
\label{Para_5.11}
Papuan Malay has eleven prepositions which serve to denote grammatical and semantic relations between their complements and the predicate. The prepositions have the following defining characteristics:

\begin{enumerate}
\item 
Prepositions introduce prepositional phrases with an overt \isi{noun} phrase complement which may neither be fronted nor omitted (see \chapref{Para_10}).

\item 
All prepositions introduce peripheral adjuncts within the clause (see \chapref{Para_10}).

\item 
Most of the prepositions also introduce oblique arguments and/or nonverbal predicates (see §\ref{Para_11.1.3.2} and §\ref{Para_12.4}, respectively).

\item 
Some of the prepositions also introduce prepositional phrases that function as modifiers within \isi{noun} phrases (see §\ref{Para_8.2.7}).

\end{enumerate}

The Papuan Malay prepositions are presented in \tabref{Table_5.38}. Three groups of prepositions are distinguished according to the semantic relations between their complements and the predicate: prepositions encoding (1) location in space and time, (2) \isi{accompaniment}/instruments, goals, and \isi{benefaction}, and (3) comparisons.


\begin{table}[b]

\caption[Papuan Malay prepositions according to the semantic relations...]{Papuan Malay prepositions according to the semantic relations between their complements and the predicate}\label{Table_5.38}
\centering
\begin{tabularx}{\textwidth}{lllllp{7cm}}
\lsptoprule

\multicolumn{6}{l}{\stepcounter{InTableCounter0} \arabic{InTableCounter0}. Location in space and time (§\ref{Para_10.1})}\\
\midrule
& \multicolumn{2}{c}{Preposition} & \multicolumn{1}{c}{Gloss} & \multicolumn{2}{c}{Semantic relations}\\
\multicolumn{2}{l}{} & \textitbf{di} & ‘at, in’ &  & static location\\
\multicolumn{2}{l}{} & \textitbf{ke} & ‘to’ &  & movement toward a referent\\
\multicolumn{2}{l}{} & \textitbf{dari} & ‘from’ &  & movement from a source location\\
\multicolumn{2}{l}{} & \textitbf{sampe} & ‘until’ &  & movement toward a nonspatial temporal endpoint\\
\midrule
\multicolumn{6}{p{12cm}}{\stepcounter{InTableCounter0} \arabic{InTableCounter0}. Accompaniment/instruments, goals, and \isi{benefaction} (§\ref{Para_10.2})\footnote{Both \textitbf{untuk} ‘for’ and \textitbf{buat} ‘for’ introduce beneficiaries and benefactive recipients. Benefactive \textitbf{untuk} ‘for’, however, has a wider distribution and more functions than \textitbf{buat} ‘for’ in that \textitbf{untuk} ‘for’ (1) combines with demonstratives, (2) introduces inanimate referents, and (3) introduces circumstance. For details see §\ref{Para_10.2.3} and §\ref{Para_10.2.4}.}}\\
\midrule
& \multicolumn{2}{c}{Preposition} & \multicolumn{1}{c}{Gloss} & \multicolumn{2}{c}{Semantic relations}\\
\multicolumn{2}{l}{} & \textitbf{dengang} & ‘with’ &  & \isi{accompaniment}\\
\multicolumn{2}{l}{} & \textitbf{sama} & ‘to’ &  & goal\\
\multicolumn{2}{l}{} & \textitbf{untuk} & ‘for’ &  & \isi{benefaction}\\
\multicolumn{2}{l}{} & \textitbf{buat} & ‘for’ &  & \isi{benefaction}\\
\midrule
\multicolumn{6}{l}{\stepcounter{InTableCounter0} \arabic{InTableCounter0}. Comparison (§\ref{Para_10.3})\footnote{Both \textitbf{kaya} ‘like’ and \textitbf{sperti} ‘similar to’ signal likeness in terms of appearance or behavior. They differ in scope, however. Similative \textitbf{kaya} ‘like’ signals overall resemblance between the two bases of comparison. The scope of \textitbf{sperti} ‘similar to’, by contrast, is more limited: it signals likeness or resemblance in some, typically implied, respect. For details see §\ref{Para_10.3.1} and §\ref{Para_10.3.2}.}}\\
\midrule
& \multicolumn{2}{c}{Preposition} & \multicolumn{1}{c}{Gloss} & \multicolumn{2}{c}{Semantic relations}\\
\multicolumn{2}{l}{} & \textitbf{sperti} & ‘similar to’ &  & \isi{similarity}\\
\multicolumn{2}{l}{} & \textitbf{kaya} & ‘like’ &  & \isi{similarity}\\
\multicolumn{2}{l}{} & \textitbf{sebagey} & ‘as’ &  & equatability\\
\lspbottomrule
\end{tabularx}

\end{table}

The complement in a \isi{prepositional phrase} is obligatory. If the semantic relationship between this complement and the predicate can be deduced from the context, two of the prepositions of location may be omitted, \isi{locative} \textitbf{di} ‘at, in’ and allative \textitbf{ke} ‘to’. A full discussion of the Papuan Malay prepositions and prepositional phrases is given in \chapref{Para_10}.

A considerable number of the prepositions have dual \isi{word class membership}, two have trial class membership. That is, three prepositions are also used as verbs: \textitbf{buat} ‘for’, \textitbf{sama} ‘to’, and \textitbf{sampe} ‘until’ (see §\ref{Para_5.3}). Six prepositions are also used as conjunctions: \textitbf{dengang} ‘with’, \textitbf{kaya} ‘like’, \textitbf{sama} ‘to’, \textitbf{sampe} ‘until’, \textitbf{sperti} ‘similar to’, and \textitbf{untuk} ‘for’ (see §\ref{Para_5.12} and \chapref{Para_14}). (Variation in \isi{word class membership} is discussed in §\ref{Para_5.14}.)

\clearpage 
\section{Conjunctions}
\label{Para_5.12}
Papuan Malay has 23 conjunctions which serve to connect words, phrases, or clauses. They have the following defining characteristics:



\begin{enumerate}
\item 
Conjunctions combine different constituents, namely clauses, \isi{noun} phrases, pre\-po\-sitional phrases, and verbs; they do not head phrases.

\item 
Conjunctions occur at the periphery of the constituents they mark.

\item 
Conjunctions form intonation units with the constituents they mark, although they do not belong to them semantically.

\end{enumerate}

The Papuan Malay conjunctions can be divided into two major groups, namely those combining same-type constituents, such as clauses with clauses, and those linking dif\-fer\-ent-type constituents, such as verbs with clauses.

Conjunctions combining clauses are traditionally divided into coordinating and subordinating ones \citep[45]{Schachter.2007}. With respect to clause linking in Papuan Malay, however, there is no formal marking of this distinction. That is, in terms of their morphosyntax and word order, clauses marked with a \isi{conjunction} are not distinct from those which do not have a \isi{conjunction}. (For details see \chapref{Para_14}.)

\tabref{Table_5.39} gives an overview of the Papuan Malay conjunctions attested in the corpus. They are grouped in terms of the types of constituents they combine and the semantic relations they signal. Two of the conjunctions are listed twice as they mark more than one type of semantic relation, namely \textitbf{baru} ‘and then, after all’, and \textitbf{sampe} ‘until’.

\begin{table}
\caption{Papuan Malay conjunctions}\label{Table_5.39}
\small
\begin{tabularx}{\textwidth}{llXXX}
\lsptoprule
\multicolumn{5}{l}{\stepcounter{InTableCounter0} \Roman{InTableCounter0}. Conjunctions combining same-type constituents (§\ref{Para_14.2})}\\
\midrule
 & \multicolumn{4}{l}{\stepcounter{InTableCounter1} \arabic{InTableCounter1}. Conjunctions marking addition (§\ref{Para_14.2.1})}\\
 \midrule
& & {Conjunction} & {Gloss} & {Semantic relations}\\
& & {\textitbf{dengang}} & {‘with’} & Addition\\
& & {\textitbf{dang}} & {‘and’} & Addition\\
& & {\textitbf{sama}} & {‘to’} & Addition\\
\midrule
& \multicolumn{4}{l}{\stepcounter{InTableCounter1} \arabic{InTableCounter1}. Conjunctions marking alternative (§\ref{Para_14.2.2})}\\
\midrule
& & {Conjunction} & {Gloss} & {Semantic relations}\\
& & {\textitbf{ato}} & {‘or’} & Alternative\\
& & {\textitbf{ka}} & {‘or’} & Alternative\\
 \midrule
& \multicolumn{4}{l}{\stepcounter{InTableCounter1} \arabic{InTableCounter1}. Conjunctions marking time and/or \isi{condition} (§\ref{Para_14.2.3})}\\
 \midrule
& & {Conjunction} & {Gloss} & {Semantic relations}\\
& & {\textitbf{trus}} & {‘next’} & Sequence (neutral)\\
& & {\textitbf{baru}} & {‘and then’} & Sequence (contrastive)\\
& & {\textitbf{sampe}} & {‘until’} & Anteriority\\
& & {\textitbf{seblum}} & {‘before’} & Anteriority\\
& & {\textitbf{kalo}} & {‘if, when’} & Posteriority / Condition\\
 \midrule
& \multicolumn{4}{l}{\stepcounter{InTableCounter1} \arabic{InTableCounter1}. Conjunctions marking consequence (§\ref{Para_14.2.4})}\\
 \midrule
& & {Conjunction} & {Gloss}  & {Semantic relations}\\
& & {\textitbf{jadi}} & {‘so, since’} & Result / Cause\\
& & {\textitbf{supaya}} & {‘so that’} & Purpose\\
& & {\textitbf{untuk}} & {‘for’} & Purpose\\
& & {\textitbf{sampe}} & {‘with the result that’} & Result\\
& & {\textitbf{karna}} & {‘because’} & Cause (neutral)\\
& & {\textitbf{gara-gara}} & {‘because’} & Cause (emotive)\\
 \midrule
& \multicolumn{4}{l}{\stepcounter{InTableCounter1} \arabic{InTableCounter1}. Conjunctions marking contrast (§\ref{Para_14.2.5})}\\
 \midrule
& & {Conjunction} & {Gloss}  & {Semantic relations}\\
& & {\textitbf{tapi}} & {‘but’} & Contrast\\
& & {\textitbf{habis}} & {‘after all’} & Contrast\\
& & {\textitbf{baru}} & {‘after all’} & Contrast\\
& & {\textitbf{padahal}} & {‘but actually’} & Contrast\\
& & {\textitbf{biar}} & {‘although’} & Concession\\
% \multicolumn{5}{l}{
% \stepcounter{InTableCounter0}\Roman{InTableCounter0}. 
% Conjunctions combining same-type constituents (§\ref{Para_14.2})}\\
\midrule
& \multicolumn{4}{l}{{
\stepcounter{InTableCounter1} \arabic{InTableCounter1}. 
Conjunctions marking \isi{similarity} (§\ref{Para_14.2.6})}}\\
 \midrule
& & {Conjunction} & {Gloss} & {Semantic relations}\\
& & {\textitbf{sperti}} & {‘similar to’} & Similarity (partial)\\
& & {\textitbf{kaya}} & {‘like’} & Similarity (overall)\\ 
% \lspbottomrule
% \end{tabularx}
% \end{table}
% 
% \begin{table}
% \caption{Papuan Malay conjunctions continued}\label{Table_5.39a}
% \small
% \begin{tabularx}{\textwidth}{llXXX}
% \lsptoprule
% \multicolumn{5}{l}{\stepcounter{InTableCounter0}\Roman{InTableCounter0}. Conjunctions combining same-type constituents (§\ref{Para_14.2})}\\
% \midrule
% & \multicolumn{4}{l}{{\stepcounter{InTableCounter1} \arabic{InTableCounter1}. Conjunctions marking \isi{similarity} (§\ref{Para_14.2.6})}}\\
%  \midrule
% & & {Conjunction} & {Gloss} & {Semantic relations}\\
% & & {\textitbf{sperti}} & {‘similar to’} & Similarity (partial)\\
% & & {\textitbf{kaya}} & {‘like’} & Similarity (overall)\\
% \midrule
\midrule
\multicolumn{5}{l} {II. Conjunctions combining dif\-fer\-ent-type constituents (§\ref{Para_14.3})}\\
\midrule
& & {Conjunction} & {Gloss} & {Syntactic function}\\
& & {\textitbf{bahwa}} & {‘that’} & Complementizer\\
& & {\textitbf{yang}} & {‘\textsc{rel}’} & Relativizer\\
\lspbottomrule
\end{tabularx}
\end{table}

A substantial number of the conjunctions have dual \isi{word class membership}, two have trial class membership. More specifically, seven of them are also used as verbs, namely \textitbf{biar} ‘although’, ‘\textitbf{buat} ‘for’, \textitbf{habis} ‘after all’, \textitbf{jadi} ‘so, since’, \textitbf{sama} ‘to’, \textitbf{sampe} ‘until’, and \textitbf{trus} ‘next’ (see §\ref{Para_5.3}). Six conjunctions are also used as prepositions, namely \textitbf{dengang} ‘with’, \textitbf{kaya} ‘like’, \textitbf{sama} ‘to’, \textitbf{sampe} ‘until’, \textitbf{sperti} ‘similar to’, and \textitbf{untuk} ‘for’ (see §\ref{Para_5.11} and \chapref{Para_10}). Besides, alternative-marking \textitbf{ka} ‘or’ is also used to mark \isi{interrogative} clauses (see §\ref{Para_13.2.3}). (For details on \isi{variation} in \isi{word class membership} see §\ref{Para_5.14}.)

\clearpage 
\section[Tags, placeholders etc.]{Tags, placeholders and hesitation markers, interjections, and onomatopoeia}
\label{Para_5.13}
\subsection{Tags}
\label{Para_5.13.1}
Papuan Malay has three tags, \textitbf{to} ‘right?’, \textitbf{e} ‘eh?’ and \textitbf{kang} ‘you know?’, as shown in (\ref{Example_5.330}) to (\ref{Example_5.336}). They are short questions that are “tagged” onto the end of an utterance and have a rising intonation. Their main function is to confirm what is being said.

With \textitbf{to} ‘right?’, speakers ask for agreement or disagreement, as in (\ref{Example_5.330}) and (\ref{Example_5.331}),\footnote{The \isi{tag} \textitbf{to} ‘right?’ is a loanword from \ili{Dutch}, which uses \textitbf{toch} ‘right?’ as a \isi{tag}.} while with \textitbf{kang} ‘you know?’ speakers assume their interlocutors to agree with their statements, as in (\ref{Example_5.331}) and (\ref{Example_5.332}). Speakers use \textitbf{to} ‘right?’ at the end of an utterance. When employing \textitbf{kang} ‘you know?’, by contrast, they usually continue their utterance and add further information related to the issue under discussion. In this context, \textitbf{kang} ‘you know?’ quite often co-occurs with \textitbf{to} ‘right?’, as in (\ref{Example_5.331}).


\begin{styleExampleTitle}
Tags: \textitbf{to} ‘right?’ and \textitbf{kang} ‘you know’
\end{styleExampleTitle}

\ea
\label{Example_5.330}
\gll {sebentar} {pasti} {hujang} {karna} {awang} {hitam} {\bluebold{to}?}\\ %
 in.a.moment  definitely  rain  because  cloud  be.black  right?\\
\glt 
‘in a bit it will certainly rain because of the black clouds, \bluebold{right?}’ \textstyleExampleSource{[080919-005-Cv.0016]}
\z

\ea
\label{Example_5.331}
\gll {de} {suda} {tidor,} {\bluebold{kang}?,} {dia} {hosa} {\bluebold{to}?}\\ %
 \textsc{3sg}  already  sleep  you.know  \textsc{3sg}  pant  right?\\
\glt 
‘she was already sleeping, \bluebold{you know?}, she has breathing difficulties, \bluebold{right?}’ \textstyleExampleSource{[080916-001-CvNP.0005]}
\z

\ea
\label{Example_5.332}
\gll {dong} {bilang} {soa-soa} {\bluebold{kang}?,} {kaya} {buaya} {begitu}\\ %
 \textsc{3pl}  say  monitor.lizard  you.know  like  crocodile  like.that\\
\glt 
‘they call (it) a monitor lizard, \bluebold{you know?}, (it’s) like a crocodile’ \textstyleExampleSource{[080922-009-CvNP.0053]}
\z


Like \textitbf{to} ‘right?’, \textitbf{e} ‘eh?’ occurs at the end of an utterance, and like \textitbf{kang} ‘you know’, it assumes agreement. Its uses seem to be more restricted, though, than those of the two other tags. Speakers tend to employ \textitbf{e} ‘eh?’ as a marker of assurance, that is, when they want to give assurance, as in (\ref{Example_5.333}), or ask for assurance as in (\ref{Example_5.334}) and (\ref{Example_5.335}). As an extension of this assurance-marking function, \textitbf{e} ‘eh?’ is also used to mark imperatives, as in (\ref{Example_5.336}) (see also §\ref{Para_13.3.1}).


\begin{styleExampleTitle}
Tags: \textitbf{e} ‘eh?’
\end{styleExampleTitle}

\ea
\label{Example_5.333}
\gll {saya} {cabut} {ko} {dari} {skola} {itu} {\bluebold{e}?}\\ %
 \textsc{1sg}  pull.out  \textsc{2sg}  from  school  \textsc{d.dist}  eh\\
\glt 
‘I’ll take you out of school there, \bluebold{eh?}’ \textstyleExampleSource{[080922-001a-CvPh.0199]}
\z

\ea
\label{Example_5.334}
\gll {bapa} {datang} {\bluebold{e}?} {bapa} {datang} {\bluebold{e}?}\\ %
 father  come  eh  father  come  eh\\
\glt 
‘you (‘father’) will come (here), \bluebold{eh?}, you (‘father’) will come (here), \bluebold{eh?}’ \textstyleExampleSource{[080922-001a-CvPh.1072]}
\z

\ea
\label{Example_5.335}
\gll {ade} {bongso} {jadi} {ko} {sayang} {dia} {skali} {\bluebold{e}?}\\ %
 ySb  youngest.offspring  so  \textsc{2sg}  love  \textsc{3sg}  very  eh\\
\glt 
‘(your) youngest sibling, so you love her very much, \bluebold{eh?}’ \textstyleExampleSource{[080922-001a-CvPh.0302]}
\z

\ea
\label{Example_5.336}
\gll {hari} {minggu} {ko} {ke} {ruma} {\bluebold{e}?} {ke} {Siduas} {punya} {ruma} {\bluebold{e}?}\\ %
 day  Sunday  \textsc{2sg}  to  house  eh  to  Siduas  \textsc{poss}  house  eh\\
\glt
‘on Sunday you go to the house, \bluebold{eh?!}, to Siduas’ house, \bluebold{eh?!}’ \textstyleExampleSource{[080922-001a-CvPh.0341]}
\z


\subsection{Placeholders and hesitation markers}
\label{Para_5.13.2}
Papuan Malay has five placeholders, namely the three interrogatives \textitbf{siapa} ‘who’, \textitbf{apa} ‘what’ and \textitbf{bagemana} ‘how’, and the two demonstratives \textitbf{ini} ‘\textsc{d.prox}’ and \textitbf{itu} ‘\textsc{d.prox}’. Their main function is to substitute for lexical items that the speaker has temporarily forgotten. The five placeholders are discussed in the respective sections on interrogatives (§\ref{Para_5.8}) and demonstratives (§\ref{Para_7.1.2.6}).

Hesitation markers, by contrast, have no lexical meaning. As vocal indicators they mainly serve to fill pauses. The main Papuan Malay \isi{hesitation marker} is \textitbf{e(m)} ‘uh’, as in (\ref{Example_5.337}); alternative realizations are \textitbf{u(m)} ‘uh’ as in (\ref{Example_5.338}), or \textitbf{a(m)}, \textitbf{mmm}, or \textitbf{nnn} ‘uh’.


\ea
\label{Example_5.337}
\gll {kalo} {sa} {su} {pake,} {\bluebold{em},} {kaca-mata} {tu} {mungking} {{\ldots}}\\ %
 if  \textsc{1sg}  already  use  uh  glasses  \textsc{d.dist}  maybe  \\
\glt 
‘if I’d been wearing, \bluebold{uh}, those (sun)glasses, maybe {\ldots}’ \textstyleExampleSource{[080919-005-Cv.0007]}
\z

\ea
\label{Example_5.338}
\gll {pace} {Oktofernus} {de,} {\bluebold{u},} {masi} {urus} {dorang} {sana}\\ %
 man  Oktofernus  \textsc{3sg}  uh  still  arrange  \textsc{3pl}  \textsc{l.dist}\\
\glt
‘Mr. Oktofernus, \bluebold{uh}, was still taking care of them over there’ \textstyleExampleSource{[081025-008-Cv.0121]}
\z


\subsection{Interjections}
\label{Para_5.13.3}
Interjections typically “constitute utterances by themselves and express a speaker’s current mental state or reaction toward an element in the linguistic or extralinguistic context” \citep[743]{Ameka.2006}. Hence, “interjections are context-bound linguistic signs” (\citeyear*[743]{Ameka.2006}). That is, their interpretation depends on the specific context in which they are uttered. This also applies to Papuan Malay, as illustrated with the \isi{interjection} \textitbf{adu} ‘oh no!, ouch!’. Depending on the context, the \isi{interjection} expresses disappointed surprise, ‘oh no!’, or pain, ‘ouch!’.

Cross-linguistically, two major types of interjections are distinguished, namely primary and secondary interjections \citep{Ameka.2006}. Primary interjections are defined as “little words or ‘non-words’, which [{\ldots}] do not normally enter into construction with other word classes” (\citeyear*[744]{Ameka.2006}). Secondary interjections, by contrast, are defined as “words that have an independent semantic value but which can be used conventionally as nonelliptical utterances by themselves to express a mental attitude or state” (\citeyear*[744]{Ameka.2006}).

Papuan Malay primary interjections are presented in \tabref{Table_5.40} and in the examples in (\ref{Example_5.339}) to (\ref{Example_5.341}), and secondary interjections in  \tabref{Table_5.41}  and in the examples in (\ref{Example_5.342}) to (\ref{Example_5.344}).

The primary interjections, listed in \tabref{Table_5.40} , include words used for expressing emotions such as \textitbf{ba} ‘humph!’, getting attention such as \textitbf{e} ‘hey’, or addressing animals, such as \textitbf{ceh} ‘shoo’.

\begin{table}
\caption{Papuan Malay primary interjections}\label{Table_5.40}

\begin{tabularx}{\textwidth}{llp{7cm}}
\lsptoprule
 \multicolumn{1}{c}{Item} & \multicolumn{1}{c}{Gloss} &  \multicolumn{1}{c}{Semantics: Interjection used {\ldots}}\\
\midrule
\textitbf{a} & ‘ah!, oh boy!, ugh!’ & to express emotions ranging from contentment to acute discomfort or annoyance\\
\textitbf{adu} & ‘oh no!, ouch!’ & to express disappointed surprise or pain\\
\textitbf{ale} & ‘wow!’ & to express surprise or to attract attention\\
\textitbf{ay} & ‘aah!, aw!’ & to express surprise or affection\\
\textitbf{ba} & ‘humph!’ & to express disgust or denigration\\
\textitbf{ceh} & ‘shoo!’ & to chase something away\\
\textitbf{e} & ‘hey!, ha!, eh?’ & to express emphasis or astonishment or to attract attention\\
\textitbf{ha} & ‘huh?’ & to express surprise, disbelief, or confusion\\
\textitbf{hm} & ‘pfft’ & to express sarcasm or disagreement\\
\textitbf{hura} & ‘hooray!’ & to express joy, approval, or encouragement\\
\textitbf{i} & ‘ugh!, oh no!, oh!’ & to express disgust, irritation or disappointed surprise\\
\textitbf{isss} & ‘stop!’ & to stop someone/-thing or to attract attention\\
\textitbf{mpfff} & ‘ugh!’ & to express displeasure, or incredulity\\
\textitbf{na} & ‘well’ & to introduce a comment or statement, or to resume a conversation\\
\textitbf{o} & ‘oh!’ & to express surprise\\
\textitbf{oke} & ‘OK’ & to express agreement\\
\textitbf{prrrt} & ‘pfft!’ & to express sarcasm or disagreement\\
\textitbf{sio} & ‘alas!’ & to express sorrow or pity\\
\textitbf{sss} & ‘pfft!’ & to express sarcasm or disagreement\\
\textitbf{sssyyyt} & ‘shhh!’ & to silence someone\\
\textitbf{tsk-tsk} & ‘tsk-tsk’ & to express disapproval\\
\textitbf{uy} & ‘o boy!’ & to express surprise or to attract attention\\
\textitbf{wa} & ‘wow!’ & to express surprise or exasperation\\
\lspbottomrule
\end{tabularx}
\end{table}

Examples of primary interjections in natural discourse are presented in (\ref{Example_5.339}) to (\ref{Example_5.341}).


\begin{styleExampleTitle}
Primary interjections
\end{styleExampleTitle}

\ea
\label{Example_5.339}
\gll {\bluebold{a},} {saya} {bisa} {pulang} {karna} {sa} {su} {dapat} {babi}\\ %
 ah!  \textsc{1sg}  be.able  go.home  because  \textsc{1sg}  already  get  pig\\
\glt 
‘\bluebold{ah!}, I can return home because I already got the pig’ \textstyleExampleSource{[080919-004-NP.0024]}
\z

\ea
\label{Example_5.340}
\gll  \bluebold{mpfff},  Yonece  de  liat\bluebold{{\Tilde}}liat  sa  smes  di  net  to?\\
 ugh!  Yonece  \textsc{3sg}  \textsc{rdp}{\Tilde}see  \textsc{1sg}  smash  at  (sport.)net  right?\\
\glt 
[About a volleyball game:] ‘\bluebold{ugh!}, Yonece saw (that) I was going to smash, right?’ \textstyleExampleSource{[081109-001-Cv.0160]}
\z

\ea
\label{Example_5.341}
\gll {\bluebold{o},} {dong} {mara} {e}?\\ %
 oh!  \textsc{3pl}  feel.angry(.about)  eh\\
\glt 
‘\bluebold{oh!}, they’ll be angry, eh?’ \textstyleExampleSource{[080917-008-NP.0054]}
\z
%\end{table}

Examples of the Papuan Malay secondary interjections, listed in \tabref{Table_5.41}, include words for expressing emotions such as \textitbf{sunggu} ‘good grief’, as well as routine expressions for thanking, greetings, or leave-taking, such as \textitbf{da} ‘goodbye’. Some of them also have independent uses in Papuan Malay, such as \textitbf{bahaya} ‘be dangerous’ (see the column ‘Basic meaning’ in \tabref{Table_5.41}). Others, by contrast, are only used as interjections, such as \textitbf{ayo} ‘come on!’. Remarkably, many secondary interjections are loanwords, such as \textitbf{bahaya} ‘great!, be dangerous’ (\ili{Sanskrit}), \textitbf{mama} ‘oh boy, mother’ (\ili{Dutch}), or \textitbf{sip} ‘that’s fine’ (English).


\begin{table}[t]
\caption{Papuan Malay secondary interjections}\label{Table_5.41}

\begin{tabular}{llll}
\lsptoprule
 \multicolumn{1}{c}{Item} & \multicolumn{1}{c}{Gloss} & \multicolumn{1}{c}{Basic meaning} &  \multicolumn{1}{c}{Source language}\\
\midrule
\textitbf{bahaya} & ‘great!’ & ‘be dangerous’ & \ili{Sanskrit}\\
\textitbf{damay} & ‘my goodness!’ & ‘peace’ & \\
\textitbf{mama} & ‘oh boy!’ & ‘mother’ & \ili{Dutch}\\
\textitbf{sialang} & ‘damn it!’ & ‘bad luck’ & \\
\textitbf{sunggu} & ‘good grief!’ & ‘be true’ & \\
\textitbf{tobat} & ‘go to hell!’ & ‘repent’ & \ili{Arabic}\\
\textitbf{tolong} & ‘please!’ & ‘help!’ & \\
\textitbf{amin} & ‘amen!’ &  & \ili{Arabic}\\
\textitbf{ayo} & ‘come on!’ &  & \\
\textitbf{da} & ‘goodbye!’ &  & \ili{Dutch}\\
\textitbf{enta} & ‘who knows!’ &  & \\
\textitbf{haleluya} & ‘hallelujah!’ &  & \ili{Hebrew} via \ili{Dutch}\\
\textitbf{halow} & ‘hello!’ &  & \ili{Dutch}\\
\textitbf{shalom} & ‘peace be with you!’ &  & \ili{Hebrew} via \ili{Dutch}\\
\textitbf{sori} & ‘excuse me!’ &  & English\\
\textitbf{sip} & ‘that’s fine!’ &  & English\\
\textitbf{trima-kasi} & ‘thank you!’ &  & \\
\lspbottomrule
\end{tabular}
\end{table}

Examples of secondary interjections are presented in (\ref{Example_5.342}) to (\ref{Example_5.344}).

%
\begin{styleExampleTitle}
Secondary interjections
\end{styleExampleTitle}

\ea
\label{Example_5.342}
\gll {\bluebold{damay},} {sa} {bulang} {oktober} {sa} {pu} {alpa} {cuma} {dua} {saja}\\ %
 peace  \textsc{1sg}  month  October  \textsc{1sg}  \textsc{poss}  be.absent  just  two  just\\
\glt 
‘\bluebold{my goodness!}, in October I, I had just only two absences’ \textstyleExampleSource{[081023-004-Cv.0014]}
\z

\ea
\label{Example_5.343}
\gll  sa  {bilang,}  {o}  {\bluebold{sunggu}},  ini  kalo  Hendro  ini  de\\
 \textsc{1sg}  {say}  {oh!}  {be.true}  \textsc{d.prox}  if  Hendro  \textsc{d.prox}  \textsc{3sg}\\
 \gll {su}  {angkat}  {deng}  {piring}\\
 {already}  {lift}  {with}  {plate}\\
\glt 
‘I said, ‘oh \bluebold{good grief!}, what’s-his-name, as for this Hendro, he would already have taken (all the cake) with the plate’ \textstyleExampleSource{[081011-005-Cv.0028]}
\z

\ea
\label{Example_5.344}
\gll {kasi} {nasi} {suda,} {\bluebold{ayo}}\\ %
 give  cooked.rice  already  come.on!\\
\glt
‘give me rice!, \bluebold{come on!}’ \textstyleExampleSource{[080922-001a-CvPh.1208]}
\z
%\end{table}

\subsection{Onomatopoeia}\label{Para_5.13.4}

Papuan Malay has a large set of onomatopoeic words which serve to imitate the natural sounds associated with their referents. Quite a few of the onomatopoeic words presented in \tabref{Table_5.42} emulate the sound of a sudden percussion, such as \textitbf{cekkk} ‘wham’. Other words are \textitbf{fuuu} ‘fooo’ which imitates the sound of blowing air, or \textitbf{piiip} ‘beep’ which emulates the blowing of a horn.


\begin{table}
\caption{Papuan Malay onomatopoeic words}\label{Table_5.42}

\begin{tabularx}{\textwidth}{lp{8cm}}
\lsptoprule
 \multicolumn{1}{c}{Item} &  \multicolumn{1}{c}{Semantics}\\
 \midrule
\textitbf{cekkk} & Sound of a heavy blow\\
\textitbf{dederet} & Sound of a drum\\
\textitbf{fuuu} & Sound of blowing air\\
\textitbf{kkkhkh} & Sound of an object falling or collapsing with a dull or heavy sound\\
\textitbf{mmmuat} & Sound of kissing\\
\textitbf{ngying{\Tilde}ngyaung} & Sound of a cockatoo calling\\
\textitbf{pak, tak, tang, wreeek} & Sound of banging, of a punch to the jaw, or of colliding bodies, slamming objects\\
\textitbf{piiip} & Sound of blowing a horn\\
\textitbf{syyyt} & Sound of an object moving through air or water\\
\textitbf{srrrt} & Sound of pulling, tearing or cutting\\
\textitbf{ssst} & Sound of vomiting\\
\textitbf{tak} & Sound of knocking\\
\textitbf{tpf} & Sound of spitting out a mouthful of liquid\\
\textitbf{trrrt} & Sound of running feet\\
\textitbf{wruaw} & Sound of heavy breathing or suffocation\\
\textitbf{wuuu} & Sound of shouting\\
\lspbottomrule
\end{tabularx}
\end{table}

Examples of onomatopoeic words in context are presented in (\ref{Example_5.345}) to (\ref{Example_5.347}).


\ea
\label{Example_5.345}
\gll {sa} {ayung} {dia} {tiga} {kali,} {\bluebold{pak}} {\bluebold{pak}} {\bluebold{pak}}\\ %
 \textsc{1sg}  hit  \textsc{3sg}  three  time  bang!  bang!  bang!\\
\glt 
‘I hit him three times, \bluebold{bang!}, \bluebold{bang!}, \bluebold{bang!}’ \textstyleExampleSource{[080923-010-CvNP.0018]}
\z

\ea
\label{Example_5.346}
\gll  {\ldots}  {kitong}  {liat,}  uy  cahaya  {\bluebold{syyyt}}  de  datang  sperti\\
 { }   {we}  {see}  boy!  glow  {swish}  \textsc{3sg}  come  similar.to\\
\gll {lampu}  {itu}  {petromaks}  {itu}\\
 {lamp}  {\textsc{d.dist}}  {kerosene.lantern}  {\textsc{d.dist}}\\
\glt 
‘[when the evil spirit comes from afar,] we see, oh boy!, a glow, \bluebold{swish!}, he/she comes (with a noise) like that, what’s-its-name, kerosene pressure lantern’ \textstyleExampleSource{[081006-022-CvEx.0153]}
\z

\ea
\label{Example_5.347}
\gll {de} {{pegang}} {di} {{batang}} {leher} {baru} {de} {ramas} {tete,}\\ %
 \textsc{3sg}  {hold}  at  {stick}  neck  and.then  \textsc{3sg}  press  grandfather\\
\gll {tete}  {\bluebold{wruaw}}  {\bluebold{wruaw}}\\
 {grandfather}  {wheeze!}  {wheeze!}\\
\glt 
‘he held (grandfather) by (his) throat, and then he pressed grandfather(’s throat and) grandfather (went) ``\bluebold{wheeze!}, \bluebold{wheeze!}''' \textstyleExampleSource{[081015-001-Cv.0012/0014]}
\z


Across languages, onomatopoeic words belong to the larger class of ideo\-pho\-nes, that “report an extralinguistic event like a sound, a smell, a taste, a visual impression, a movement, or a psychic emotion” \citep[510]{KilianHatz.2006}. As for the Papuan Malay corpus, however, extralinguistic events other than the onomatopoeic sound imitations presented in \tabref{Table_5.42}  have not been identified.


\section{Variation in word class membership}
\label{Para_5.14}
Papuan Malay has \isi{variation} in \isi{word class membership} between (1) verbs and nouns, (2) verbs and adverbs, (3) verbs and conjunctions, (4) verbs and prepositions, and (5) prepositions and conjunctions.


\newpage 
Cross-linguistically, the shift of word categories occurs quite commonly. Generally speaking, it “is a unidirectional process; that is, it leads from less grammatical to more grammatical forms and constructions” \citep[4]{Heine.2002}. Or in other words, “the shift from major categories to minor ones (N {\textgreater} Preposition/Conjunction, V {\textgreater} Auxiliary/Preposition) is much more frequent crosslinguistically than its opposite”, as {\citet[133]{Wischer.2006}} points out.



Therefore, in discussing \isi{variation} in Papuan Malay \isi{word class membership} between verbs and adverbs, verbs and prepositions, and verbs and conjunctions, the verbs are taken as the source forms from which the respective adverbs, prepositions, and conjunctions derived. As for \isi{variation} between prepositions and conjunctions, \citet[4]{Heine.2002} note that cross-linguistically “[p]repositions often develop into conjunctions”. Very likely, this observation also applies to the \isi{variation} between prepositions and conjunctions in Papuan Malay. The dual membership of lexemes as verbs and nouns, however, is less clear-cut, as discussed in Paragraph 1 below.


%\setcounter{itemize}{0}
\begin{enumerate}
\item 
Verbs and nouns (see §\ref{Para_5.2} and §\ref{Para_5.3})


A number of lexemes have dual membership as verbs and nouns. So far, 41 such lexemes have been identified, some of which are listed in \tabref{Table_5.43}, together with the token frequencies of their uses as verbs and nouns.



The identified lexemes fall into two groups. First, verbs and their associated instrument, result, patient, agent, or location nouns. The corpus contains 32 such verb-\isi{noun} pairs. In most cases, the \isi{verb} is \isi{bivalent} (29 verbs), while only few are \isi{monovalent} (3 verbs). \tabref{Table_5.43} presents eight of these verb-\isi{noun} pairs. The first four lexemes are most often used as verbs, that is, \textitbf{gambar} ‘draw’, \textitbf{jalang} ‘walk’, \textitbf{jubi} ‘bow shoot’, and \textitbf{skola} ‘go to school’. The remaining four lexemes are most often used as nouns, that is, \textitbf{dayung} ‘paddle’, \textitbf{musu} ‘enemy’, \textitbf{pana} ‘arrow’, and \textitbf{senter} ‘flashlight’.



The second group of lexemes with dual membership are affixed items: two items suffixed with-\textitbf{ang} and four prefixed with \textscItal{pe(n)-}. Structurally, the six lexemes are nouns. In their actual uses, however, four of them are (more) often used as verbs (for a detailed discussion on \isi{affixation} see §\ref{Para_3.1}).

\begin{table}[t]

\caption{Variation in \isi{word class membership} between nouns and verbs}\label{Table_5.43}
\setlength{\tabcolsep}{3.7pt}
%\begin{tabularx}{\textwidth}{lp{2.5cm}lrcp{2.5cm}lr}
\begin{tabularx}{\textwidth}{lllrcllr}
\lsptoprule
 & \multicolumn{3}{c}{\textsc{verb}} &  & \multicolumn{3}{c}{ \textsc{noun}}\\
 \multicolumn{1}{c}{Item} & \multicolumn{1}{c}{Gloss} &  & \# & {\textgreater}/{\textless} & \multicolumn{1}{c}{Gloss} &  &  \multicolumn{1}{r}{\#}\\
\midrule

\textitbf{gambar} & ‘draw’ & \textsc{v.bi} &  21 & {\textgreater} & ‘drawing’ & \textsc{res} &  2\\
\textitbf{jalang} & ‘walk’ & \textsc{v.mo} &  398 & {\textgreater} & ‘road’ & \textsc{loc} &  71\\
\textitbf{jubi} & ‘bow shoot’ & \textsc{v.bi} &  20 & {\textgreater} & ‘bow and arrow’ & \textsc{ins} &  14\\
\textitbf{skola} & ‘go to school’ & \textsc{v.mo} &  148 & {\textgreater} & ‘school’ & \textsc{loc} &  94\\
\textitbf{dayung} & ‘paddle’ & \textsc{v.bi} &  3 & {\textless} & ‘paddle’ & \textsc{ins} &  8\\
\textitbf{musu} & ‘hate’ & \textsc{v.bi} &  3 & {\textless} & ‘enemy’ & \textsc{pat} &  7\\
\textitbf{pana} & ‘bow shoot’ & \textsc{v.bi} &  13 & {\textless} & ‘arrow’ & \textsc{ins} &  39\\
\textitbf{senter} & ‘light with flashlight’ & \textsc{v.bi} &  5 & {\textless} & ‘flashlight’ & \textsc{ins} &  11\\
\textitbf{jualang} & ‘sell’ & \textsc{v.bi} &  7 & {\textgreater} & ‘merchandise’ & \textsc{pat} &  1\\
\textitbf{latiang} & ‘practice’ & \textsc{v.bi} &  12 & {\textgreater} & ‘practice’ & \textsc{pat} &  5\\
\textitbf{pamalas} & ‘be very list\-less’ & \textsc{v.mo} &  12 & {\textgreater} & ‘lazy person’ & \textsc{agt} &  2\\
\textitbf{panakut} & ‘be very fear\-ful (of)’ & \textsc{v.bi} &  2 & {\textgreater} & ‘coward’ & \textsc{agt} &  1\\
\textitbf{pandiam} & ‘be very quiet’\footnote{The corpus includes only one token of \textitbf{pandiam} ‘taciturn person / be very quiet’; its reading is ambiguous, that is, it can receive a verbal or a nominal reading (see example (\ref{Example_3.29}) in §\ref{Para_3.1.4.2}, p. \pageref{Example_3.29}).}
 & \textsc{v.mo} &  (1) & OR & ‘taciturn person’ & \textsc{agt} &  (1)\\
\textitbf{pencuri} & ‘steal (\textsc{emph})’ & \textsc{v.bi} &  5 & {\textless} & ‘thief’ & \textsc{agt} &  7\\
\lspbottomrule
\end{tabularx}

\end{table}

\item 
Verbs and adverbs (see §\ref{Para_5.3} and §\ref{Para_5.4})

Some verbs also have adverbial function. Five such lexemes have been identified so far, as listed in \tabref{Table_5.44}. All but one of them are more often used as adverbs than as verbs. The exception is \isi{bivalent} \textitbf{coba} ‘try’ which is more often used as a \isi{verb} and less often as an evaluative modal ad\isi{verb} (§\ref{Para_5.4.4}).

In addition, the corpus includes six adverbs which are reduplicated verbs: \textitbf{baru{\Tilde}ba\-ru} ‘just now’, \textitbf{kira{\Tilde}kira} ‘probably’, \textitbf{lama{\Tilde}lama} ‘gradually’, \textitbf{muda{\Tilde}mudaang} ‘hopefully’, and \textitbf{taw{\Tilde}taw} ‘suddenly’. Their respective base words are \textitbf{baru} ‘be new’, \textitbf{kira} ‘think’, \textitbf{lama} ‘be long (of duration), \textitbf{muda} ‘be easy’, and \textitbf{taw} ‘know’

\begin{table}
\caption{Variation in \isi{word class membership} between verbs and adverbs}\label{Table_5.44}

\begin{tabular}{lllllll}
\lsptoprule
 & \multicolumn{3}{l}{Source form: \textsc{verb}} & {\textgreater} & \multicolumn{2}{l}{ Derived form: \textsc{adv}}\\
 \multicolumn{1}{c}{Item} & \multicolumn{1}{c}{Gloss} &  & \# &  & \multicolumn{1}{c}{Gloss} &  \#\\
\midrule
\textitbf{baru} & ‘be new’ & \textsc{v.mo} &  24 & {\textless} & ‘recently’ &  66\\
\textitbf{dulu} & ‘be prior’ & \textsc{v.mo} &  63 & {\textless} & ‘first, in the past’ &  286\\
\textitbf{pas} & ‘be exact’ & \textsc{v.mo} &  26 & {\textless} & ‘precisely’ &  110\\
\textitbf{skarang} & ‘be current’ & \textsc{v.mo} &  21 & {\textless} & ‘now’ &  282\\
\textitbf{coba} & ‘try’ & \textsc{v.bi} &  36 & {\textgreater} & ‘if only’ &  14\\
\lspbottomrule
\end{tabular}
\end{table}



\newpage 
\item 
Verbs and conjunctions (see §\ref{Para_5.3} and §\ref{Para_5.12})


Some verbs are zero-derived into the \isi{conjunction} class, namely four \isi{monovalent} stative and three \isi{bivalent} verbs, as listed in \tabref{Table_5.45}. Again, the lexemes differ in terms of the relative token frequencies of the source forms and the derived conjunctional forms. For the first three items, the verbal source forms have higher token frequencies, whereas the last four lexemes are predominantly used as conjunctions.


\begin{table}
\caption{Variation in \isi{word class membership} between verbs and conjunctions}\label{Table_5.45}

\begin{tabular}{lllllll}
\lsptoprule
& \multicolumn{3}{l}{ Source form: \textsc{verb}} & {\textgreater} & \multicolumn{2}{l}{ Derived form: \textsc{cnj}}\\
 \multicolumn{1}{c}{Item} & \multicolumn{1}{c}{Gloss} &  & \# &  & \multicolumn{1}{c}{Gloss} &  \#\\
\midrule
\textitbf{biar} & ‘let’ & \textsc{v.bi} &  67 & {\textgreater} & ‘although’ &  39\\
\textitbf{habis} & ‘be used up’ & \textsc{v.mo} &  48 & {\textgreater} & ‘after all’ &  21\\
\textitbf{sama} & ‘be same’ & \textsc{v.mo} &  60 & {\textgreater} & ‘to’ &  8\\
\textitbf{baru} & ‘be new’ & \textsc{v.mo} &  24 & {\textless} & ‘and then, after all’ &  986\\
\textitbf{jadi} & ‘become’ & \textsc{v.bi} &  173 & {\textless} & ‘so, since’ &  1,213\\
\textitbf{sampe} & ‘reach’ & \textsc{v.bi} &  251 & {\textless} & ‘until’ &  257\\
\textitbf{trus} & ‘be continuous’ & \textsc{v.mo} &  70 & {\textless} & ‘next’ &  396\\
\lspbottomrule
\end{tabular}
\end{table}

\newpage 
\item 
Verbs and prepositions (see §\ref{Para_5.3} and §\ref{Para_5.11})


One \isi{preposition} is derived from a \isi{monovalent} \isi{verb} and two from \isi{bivalent} verbs:


\begin{itemize}
\item 
The goal \isi{preposition} \textitbf{sama} ‘to’ is derived from \isi{monovalent} \textitbf{sama} ‘be same’.

\item 
The benefactive \isi{preposition} \textitbf{buat} ‘for’ is derived from \isi{bivalent} \textitbf{buat} ‘make’.

\item 
The temporal \isi{preposition} \textitbf{sampe} ‘until’ is derived from \isi{bivalent} \textitbf{sampe} ‘reach’.

\end{itemize}

\item 
Prepositions and conjunctions (see §\ref{Para_5.11} and §\ref{Para_5.12})


Six Papuan Malay prepositions are also used as conjunctions:

\begin{itemize}
\item 
Temporal \textitbf{sampe} ‘until’ also functions as a \isi{conjunction} that introduces temporal or result clauses.
\item 
Comitative \textitbf{dengang} ‘with’ and goal \isi{preposition} \textitbf{sama} ‘to’ also function as conjunctions that combine \isi{noun} phrases; occasionally, \textitbf{dengang} ‘with’ also links \isi{verb} phrases.
\item 
Benefactive \textitbf{untuk} ‘for’ also functions as a \isi{conjunction} that introduces purpose clauses.
\end{itemize}
\begin{itemize}
\item 
Similative \textitbf{sperti} ‘similar to’ and \textitbf{kaya} ‘like’ also function as conjunctions that introduce simulative clauses.

\end{itemize}
\end{enumerate}
Papuan Malay displays \isi{variation} in \isi{word class membership}, most of which involves verbs. Overall, the observed \isi{variation} corresponds to similar processes observed cross-linguistically, in that it involves a shift of word categories from major ones to minor ones (see \citealt[4]{Heine.2002}; \citealt[133]{Wischer.2006}). The exception is the dual membership of lexemes as verbs and nouns, which is typical, though, for Malay varieties and other western \ili{Austronesian} languages.


\section{Summary}\label{Para_5.15}

In Papuan Malay, the main criteria for defining distinct word classes are their syntactic properties, due to the lack of inflectional \isi{morphology} and the rather limited productivity of derivational patterns. Three open and a number of closed lexical classes can be distinguished. The open word classes are nouns, verbs, and adverbs. The major closed word classes are personal pronouns, interrogatives, demonstratives, locatives, numerals, quantifiers, prepositions, and conjunctions. At the same time, however, Papuan Malay has membership overlap between a number of categories, most of which involve verbs. This includes overlap between verbs and nouns which is typical of Malay varieties and other western \ili{Austronesian} languages. However, nouns, verbs, and adverbs have distinct syntactic properties which warrant their analysis as distinct word classes.



Papuan Malay nouns and verbs are distinct in terms of the following syntactic properties: (a) nouns canonically function as heads in \isi{noun} phrases and as arguments in verbal clauses; (b) verbs canonically function as predicates and have \isi{valency}; (c) nouns are negated with \textitbf{bukang} ‘\textsc{neg}’, whereas verbs are negated with \textitbf{tida}/\textitbf{tra} ‘\textsc{neg}’; (d) only nouns can be quantified with numerals and quantifiers; and (e) only verbs occur as predicates in comparative constructions, and in reciprocal constructions. Based on their syntactic properties, nouns are divided into four groups, namely common, proper, location, and direction nouns. Verbs fall into four groups, namely \isi{trivalent}, \isi{bivalent}, \isi{monovalent} dynamic and \isi{monovalent} stative verbs which have partially distinct and partially overlapping properties. The four groups of verbs can be distinguished in terms of two main criteria which also account for most of their other properties, namely their \isi{valency} and their function which is mainly predicative.



Adverbs are distinct from nouns and verbs in that adverbs, unlike nouns and verbs, (a) cannot be used predicatively; and (b) cannot modify nouns. Overall, adverbs are most closely related to verbs; some adverbs, however, are more closely linked with nouns than with verbs. Within the clause, adverbs can take different positions. The semantic effects of these positions are yet to be investigated, however.



Personal pronouns, interrogatives, demonstratives, and locatives are distinct from nouns in that (a) all four of them can modify nouns, while the opposite does not hold; and (b) in adnominal possessive constructions, personal pronouns and interrogatives only take the possessor slot while nouns also take the possessum slot. Personal pronouns, interrogatives, and demonstratives are distinct in that (a) personal pronouns express number and person, while interrogatives and demonstratives do not; (b) personal pronouns indicate definiteness, while demonstratives signal specificity; (c) only interrogatives can express \isi{indefinite} referents; and (d) only demonstratives can be stacked. Demonstratives are distinct from locatives, in that demonstratives (a) are used as independent nominals in unembedded \isi{noun} phrases while locatives always occur in prepositional phrases; (b) can take the possessor or the possessum slot in adnominal possessive constructions while locatives do no occur in these constructions; and (c) can be stacked.

