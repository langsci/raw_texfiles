\chapter[Adnominal possessive relations]{Adnominal possessive relations}
\label{Para_9}
In Papuan Malay, adnominal possessive relations between two \isi{noun} phrases are encoded with the possessive ligature \textitbf{punya} ‘\textsc{poss}’. The \isi{noun} phrase preceding the ligature (\textsc{lig}) designates the possessor (\textsc{possr}), while the \isi{noun} phrase following it expresses the possessum (\textsc{possm}), such that ``\textsc{possr}{}-\textsc{np} – \textitbf{punya} ‘\textsc{poss}’ – \textsc{possm}{}-\textsc{np}''.



The main function of adnominal possessive constructions is to denote possession of a \isi{definite} possessum. In addition, \textitbf{punya} ‘\textsc{poss}’ serves other functions in ``\textsc{possr}{}-\textsc{np} – \textitbf{punya} ‘\textsc{poss}’ – \textsc{possm}{}-\textsc{np}'' constructions. It is employed to mark and emphasize \isi{locational}, temporal, or \isi{associative} relations, to indicate beneficiary relations, or to signal speaker attitudes and evaluations. Besides, the ligature is also used in reflexive expressions. (Possession of an \isi{indefinite} possessum is not expressed with an \isi{adnominal possessive construction}, but with a \isi{two-argument existential clause} or a nominal clause; details are presented in §\ref{Para_11.4.2} and §\ref{Para_12.2}, respectively.)



The three constituents of an \isi{adnominal possessive construction} have different realizations, as illustrated in \tabref{Table_9.1}. The possessive marker can be realized with unreduced \textitbf{punya}, reduced \textitbf{pu}, clitic \textitbf{=p}, or a zero morpheme. The \isi{noun} phrases expressing the possessor and possessum can belong to different syntactic categories. The most common constituents are lexical nouns and \isi{noun} phrases. Demonstratives can also take either slot. Also very common are personal \isi{pronoun} possessors. In noncanonical possessive constructions, the possessor and possessum slots can also be filled by verbs. In addition, mid-range quantifiers, temporal adverbs and prepositional phrases can take the possessum slot. In both canonical and noncanonical possessive constructions, the possessum can be omitted.


\begin{table}
\caption{Adnominal possessive constructions}\label{Table_9.1}

\begin{tabular}{lll}
\lsptoprule

 \multicolumn{1}{c}{\textsc{possr}} & \multicolumn{1}{c}{\textsc{lig}} &  \multicolumn{1}{c}{\textsc{possm}}\\
 \midrule
Lexical nouns & \textitbf{punya} & Lexical nouns\\
Noun phrases & \textitbf{pu} & Noun phrases\\
Demonstratives & \textitbf{pu} & Demonstratives\\
Personal pronouns & \textitbf{=p} & Verbs\\
Verbs & Ø & Quantifiers\\
&  & Adverbs\\
&  & Prepositional phrases\\
&  & Ø\\
\lspbottomrule
\end{tabular}
\end{table}

Semantically, the possessor and the possessum can designate human, nonhuman animate, or inanimate referents. Overall, adnominal possessive constructions do not make a distinction between alienable and inalienable possession, with one exception. Possessive constructions with the elided possessive marker signal inalienable possession of body parts or kinship relations (see §\ref{Para_9.1.3}).



In the following sections, adnominal possessive constructions are discussed in more detail. The possessive marker \textitbf{punya} ‘\textsc{poss}’ with its different realizations is examined in §\ref{Para_9.1}. The different realizations of the possessor and possessum \isi{noun} phrases are described in §\ref{Para_9.2}. Noncanonical possessive constructions are discussed in §\ref{Para_9.3}. The main points of this chapter are summarized in §\ref{Para_9.4}. (For the uses of adnominal possessive constructions in two-argument existential clauses, see §\ref{Para_11.4.1}.)


\section{Possessive marker \textitbf{punya} ‘\textsc{poss}’}
\label{Para_9.1}
The possessive marker \textitbf{punya} ‘\textsc{poss}’ is related to the full \isi{bivalent} \isi{verb} \textitbf{punya} ‘have’ which is still used synchronically in two-argument clauses to predicate possession of an \isi{indefinite} possessum. In such clauses, the possessor is encoded by the grammatical subject (\textsc{s}) while the \isi{indefinite} possessum is the direct object (\textsc{o}) of the \isi{verb} (\textsc{v}) \textitbf{punya} ‘have’. This is illustrated in (\ref{Example_9.1}): the possessor \textitbf{sa} ‘1\textsc{sg}’ is the grammatical subject while the possessum \textitbf{rencana} ‘thought’ is the direct object of \textitbf{punya} ‘have’. Overall, however, verbal clauses with \textitbf{punya} ‘have’ are rather rare. Instead, speakers typically express possession of an \isi{indefinite} possessum with a \isi{two-argument existential clause} with \textitbf{ada} ‘exist’. This is demonstrated in (\ref{Example_9.2}): the possessor \textitbf{sa} ‘\textsc{1sg}’ is the subject while the \isi{indefinite} possessum \textitbf{ana} ‘child’ is the direct object of existential \textitbf{ada} ‘exist’. (This type of \isi{two-argument existential clause} is discussed in detail in §\ref{Para_11.4.2}.)



\begin{styleExampleTitle}
Predicative reading of \textitbf{punya} ‘have’ constructions
\end{styleExampleTitle}
\ea
\label{Example_9.1}
\glll {} {\textsc{s}}  {}  {\textsc{v}} {\textsc{o}}\\ %
{malam}  {\bluebold{saya}}  {suda}  {\bluebold{punya}}  {\bluebold{rencana}}\\
 {night}  {\textsc{1sg}}  {already}  {have}  {plan}\\
\glt ‘the night (before I go hunting) \bluebold{I} already \bluebold{have a plan}’ \textstyleExampleSource{[080919-004-NP.0002]}
\z


\begin{styleExampleTitle}
Two-argument \isi{existential clause} denoting possession
\end{styleExampleTitle}

\ea
\label{Example_9.2}
\glll \textsc{s} \textsc{v} \textsc{o}\\ %
{\bluebold{sa}}  {\bluebold{ada}}  {\bluebold{ana}},  {jadi}  {sa}  {kasi}  {untuk}  {sa}  {pu}  {sodara}\\
 {\textsc{1sg}}  {exist}  {child}  {so}  {\textsc{1sg}}  {give}  {for}  {\textsc{1sg}}  {\textsc{poss}}  {sibling}\\
\glt 
‘\bluebold{I have children}, so I gave (one) to my relative’ \textstyleExampleSource{[081006-024-CvEx.0010]}
\z


The most common function of Papuan Malay \textitbf{punya} ‘\textsc{poss}’ is that of a ligature in adnominal referential possessive constructions, that is, possessive constructions with \isi{definite} referents. Such constructions have the syntactic structure ``\textsc{possr-np} \textitbf{punya} \textsc{possm-np}''. As shown in (\ref{Example_9.3}), this type of possessive construction contrasts with the verbal constructions in (\ref{Example_9.1}) and (\ref{Example_9.2}): the possessive relation is not encoded by a two-argument clause but in a single construction which consists of two \isi{noun} phrases, which in turn functions as an argument in a clause. Hence, \textitbf{Yosina} in (\ref{Example_9.3}) is not a grammatical subject but the possessor. Likewise, \textitbf{swara} ‘voice’ is not the direct object of a \isi{verbal clause}, but a \isi{definite} possessum. The entire possessive construction in (\ref{Example_9.3}) functions as the direct object of the \isi{bivalent} \isi{verb} \textitbf{dengar} ‘hear’. The contrastive examples in (\ref{Example_9.2}) and (\ref{Example_9.3}) also illustrate the distinctions between possession of an \isi{indefinite} and a \isi{definite} possessum, respectively.



\begin{styleExampleTitle}
Adnominal reading of \textitbf{punya} ‘\textsc{poss}’ constructions
\end{styleExampleTitle}

\ea
\label{Example_9.3}
\glll {} {} {} {} \textup{\textsc{possr-np}}  \textup{\textsc{lig}}   \textup{\textsc{possm-np}}\\ %
 {bapa}  {kwatir}  {tertarik}  {dengar}  {\bluebold{Yosina}}  {\bluebold{punya}}  {\bluebold{swara}}\\
 {father}  {afraid}  {be.pulled}  {hear}  {Yosina}  {\textsc{poss}}  {voice}\\
\glt 
‘I (‘father’) was worried (and) longed to hear \bluebold{your (‘Yosina’s’) voice}’ \textstyleExampleSource{[080922-001a-CvPh.0205]}
\z



Sometimes, however, it is ambiguous whether the \textitbf{punya} construction should receive a predicative reading as in (\ref{Example_9.4a}) or an adnominal interpretation as in (\ref{Example_9.4b}), as there is no difference in intonation or stress between the utterances.



\begin{styleExampleTitle}
Predicative and adnominal readings of \textitbf{punya} ‘have/\textsc{poss}’ constructions
\end{styleExampleTitle}

\ea
\label{Example_9.4}
\ea
\label{Example_9.4a}
\gll {[de]} {[\bluebold{punya}]} {[piring} {kusus]}\\ %
   \textsc{3sg}  have/\textsc{poss}  plate  be.special\\
\glt Predicative reading: ‘he/she \bluebold{has} special plates’ \textstyleExampleSource{[081006-029-CvEx.0016]}\\
\vspace{5pt}

\ex
\label{Example_9.4b}
\gll  [\bluebold{de}  \bluebold{punya}  \bluebold{piring}]  [kusus]\\
   \textsc{3sg}  have/\textsc{poss}  plate  be.special\\
\glt Adnominal reading: ‘\bluebold{his/her plates} are special’ \textstyleExampleSource{[081006-029-CvEx.0016]}\\
\z
\z


In adnominal possessive constructions, the ligature \textitbf{punya} ‘\textsc{poss}’ has four different realizations which are discussed in the following sections: long \textitbf{punya} ‘\textsc{poss}’ and its reduced form \textitbf{pu} in §\ref{Para_9.1.1}, the clitic \textitbf{=p} ‘\textsc{poss}’ in §\ref{Para_9.1.2}, and \isi{elision} in §\ref{Para_9.1.3}. In §\ref{Para_9.1.4}, a possible \isi{grammaticalization} of the possessive marker is examined.


\subsection{\textsc{possr-np} \textitbf{punya}/\textitbf{pu} \textsc{possm-np}}
\label{Para_9.1.1}
In adnominal possessive constructions, the possessive marker is most commonly realized with the long form \textitbf{punya} ‘\textsc{poss}’ or the reduced monosyllabic form \textitbf{pu} ‘\textsc{poss}’. This reduction is independent of the syntactic or semantic properties of the possessor or possessum, as illustrated in (\ref{Example_9.5}).



Both ligature forms occur with possessors encoded by lexical nouns as in (\ref{Example_9.5a}-\ref{Example_9.5g}, by personal pronouns as in (\ref{Example_9.5h}, \ref{Example_9.5i}), or by \isi{noun} phrases as in (\ref{Example_9.5j}-\ref{Example_9.5l}). With either ligature form, the possessor can denote a human referent as in (\ref{Example_9.5a}-\ref{Example_9.5d}, \ref{Example_9.5i}-\ref{Example_9.5k}), a nonhuman animate referent as in (\ref{Example_9.5e}, \ref{Example_9.5f}, \ref{Example_9.5h} ), or an inanimate referent as in (\ref{Example_9.5g}, \ref{Example_9.5l}). Likewise, the reduction is independent of the possessum’s properties. Both markers occur with possessa encoded by nouns as in (\ref{Example_9.5b}, \ref{Example_9.5e}, \ref{Example_9.5f}, \ref{Example_9.5l}), by demonstratives as in (\ref{Example_9.5c}, \ref{Example_9.5d}), or by \isi{noun} phrases as in (\ref{Example_9.5a}, \ref{Example_9.5g}, \ref{Example_9.5i}-\ref{Example_9.5k}). With either marker, the possessum can express an inalienably possessed referent as in (\ref{Example_9.5a}, \ref{Example_9.5b}, \ref{Example_9.5h}, \ref{Example_9.5k}) or an alienably possessed referent as in (\ref{Example_9.5c}-\ref{Example_9.5g}, \ref{Example_9.5i}, \ref{Example_9.5j}, \ref{Example_9.5l}).



\begin{styleExampleNumCard}
 Adnominal possessive constructions with the long possessive marker \textitbf{punya} ‘\textsc{poss}’ and short \textitbf{pu} ‘\textsc{poss}’\footnote{Documentation: 080919-004-NP.0013, 080919-006-CvNP.0028, 080922-001a-CvPh.0141, 081006-019-Cv.0002, 081006-022-CvEx.0029, 081006-022-CvEx.0084, 081110-002-Cv.0075, 081006-024-CvEx.0016, 081011-007-Cv.0003, 081011-022-Cv.0200, 081025-006-Cv.0021, 081106-001-Ex.0007.}\\
\end{styleExampleNumCard}

\ea
\label{Example_9.5}
\setlength{\tabcolsep}{1pt}
\begin{tabular}{p{0.5cm}p{1.6cm}p{1.1cm}p{1.5cm}p{6cm}}
{ } & \textsc{possr} & \textsc{lig} & \textsc{possm} &  Possessive construction\\
\end{tabular}
\begin{xlist}

\exbox[-.8\baselineskip]{\label{Example_9.5a}
\parbox{4cm}{
 {(\textupsc{n} (\textupsc{hum})}   \textitbf{punya}   {\textupsc{np} (\textupsc{inal})\\\\
 }}
 \parbox{6cm}{
\gll \textitbf{mama}  \textitbf{punya} \textitbf{ade} \textitbf{laki{\Tilde}laki}\\
  mother  \textsc{poss}  ySb  \textsc{rdp}{\Tilde}husband\\
\glt  ‘mother’s younger brother’\\
}
} 

\smallskip

\exbox[-.8\baselineskip]{
\label{Example_9.5b}
\parbox{4cm}{
    {\textupsc{n} (\textupsc{hum})} \textitbf{pu} {\textupsc{n} (\textupsc{inal})\\\\
 }}
 \parbox{6cm}{
\gll  \textitbf{bapa} \textitbf{pu}  \textitbf{mata}\\
  father  \textsc{poss} eye\\
\glt   ‘father’s eyes’\\
}
}

\smallskip
\exbox[-.8\baselineskip]{
\label{Example_9.5c}
\parbox{4cm}{
   {\textupsc{n} (\textupsc{hum})}   \textitbf{punya}  {\textupsc{dem} (\textupsc{al})\\\\
 }}
 \parbox{6cm}{
  \gll  \textitbf{de}  \textitbf{punya}  \textitbf{ini}\\
    \textsc{3sg}  \textsc{poss}  \textsc{d.prox}\\
\glt   ‘his/her (customs)’\\
}
}

\smallskip
\exbox[-.8\baselineskip]{
\label{Example_9.5d}
\parbox{4cm}{
   {\textupsc{n} (\textupsc{hum})}  \textitbf{pu}  {\textupsc{dem} (\textupsc{al})\\\\
 }}
 \parbox{6cm}{
\gll  \textitbf{ade}  \textitbf{pu}  \textitbf{itu}\\
     ySb  \textupsc{poss} \textupsc{d.dist}\\
\glt  ‘younger sister’s (fish)’\\
}
}

\smallskip
\exbox[-.8\baselineskip]{
\label{Example_9.5e}
\parbox{4cm}{
   {\textupsc{n} (\textupsc{an})}  \textitbf{punya}  {\textupsc{n} (\textupsc{al})\\\\
 }}
 \parbox{6cm}{
\gll \textitbf{setang}  \textitbf{punya} \textitbf{kwasa}\\
    evil.spirit  \textupsc{poss} power\\
\glt  ‘force of an evil spirit’\\
}
}

\smallskip
\exbox[-.8\baselineskip]{
\label{Example_9.5f}
\parbox{4cm}{
   {\textupsc{n} (\textupsc{an})}  \textitbf{pu} {\textupsc{n} (\textupsc{al})\\\\
 }}
 \parbox{6cm}{
\gll  \textitbf{setang}  \textitbf{pu} \textitbf{pake-pake}\\
   evil.spirit  \textupsc{poss} black.magic\\
\glt  ‘an evil spirit’s black magic’\\
}
}

\smallskip
\exbox[-.8\baselineskip]{
\label{Example_9.5g}
\parbox{4cm}{
   {\textupsc{n} (\textupsc{inan})}  \textitbf{pu} {\textupsc{np} (\textupsc{al}) \\\\
 }}
 \parbox{6cm}{
\gll  \textitbf{LNG}  \textitbf{pu}  \textitbf{terpol} \textitbf{itu}\\
  LNG  \textsc{poss}  container \textsc{d.dist}\\
\glt ‘metal jerry can’\footnote{The \isi{proper noun} \textitbf{LNG} has developed from the \isi{noun} phrase ‘Liquefied Natural Gas’.}\\
}
}
% 
\smallskip
\exbox[-.8\baselineskip]{
\label{Example_9.5h}
\parbox{4cm}{
   {\textupsc{pro} (\textupsc{an})} \textitbf{punya}   {\textupsc{n} (\textupsc{inal})\\\\
 }}
 \parbox{6cm}{
\gll  \textitbf{de} \textitbf{punya} \textitbf{bulu{\Tilde}bulu}\\
    \textsc{3sg}  \textsc{poss}  \textsc{rdp}{\Tilde}body.hair\\
\glt   ‘its (the dog’s) body hair’\\
}
}

\smallskip
\exbox[-.8\baselineskip]{
\label{Example_9.5i}
\parbox{4cm}{
   {\textupsc{pro} (\textupsc{hum})} \textitbf{pu}   {\textupsc{np} (\textupsc{al})\\\\
 }}
 \parbox{6cm}{
\gll  \textitbf{de} \textitbf{pu} \textitbf{sikat} \textitbf{gigi} \textitbf{deng odol}\\
      \textsc{3sg}  \textsc{poss} toothbrush with toothpaste\\
\glt   ‘her toothbrush and toothpaste’\\
}
}

\smallskip
\exbox[-.8\baselineskip]{
\label{Example_9.5j}
\parbox{4cm}{
  {\textupsc{np} (\textupsc{hum})}  \textitbf{punya}   {\textupsc{np} (\textupsc{al})\\\\
 }}
 \parbox{8cm}{
\gll  \textitbf{orang} \textitbf{Isirawa}  \textitbf{punya,} \textitbf{apa,} \textitbf{cara} \textitbf{kawing}\\
  person  \ili{Isirawa} \textsc{poss}  what manner marry\\
\glt  ‘the \ili{Isirawa}’s, what-is-it, way of marrying’\\
}
}

\smallskip
\exbox[-.8\baselineskip]{
\label{Example_9.5k}
\parbox{4cm}{
  {\textupsc{np} (\textupsc{hum})}  \textitbf{pu}  {\textupsc{np} (\textupsc{inal})\\\\
 }}
 \parbox{6cm}{
\gll   \textitbf{mama} \textitbf{Klara}  \textitbf{pu} \textitbf{ana} \textitbf{prempuang}\\
   mother Klara \textsc{poss} child  woman\\
\glt  ‘mother Klara’s daughter’\\
}
}

\smallskip
\exbox[-.8\baselineskip]{
\label{Example_9.5l}
\parbox{4cm}{
  {\textupsc{np} (\textupsc{inan})}  \textitbf{punya}  {\textupsc{n} (\textupsc{al})\\\\
 }}
 \parbox{6cm}{
\gll  \textitbf{kebung} \textitbf{ini} \textitbf{punya}  \textitbf{hasil}\\
   garden  \textsc{d.prox} \textsc{poss} product\\
\glt   ‘this garden’s products’\\
}
}
\end{xlist}
\z

With respect to the possessive marking of personal pronouns, there are no prosodic restrictions on the use of the two possessive marker forms: either can occur with the long and the short \isi{pronoun} forms, as illustrated in Table  \ref{Table_9.2} and Table \ref{Table_9.2a}.\footnote{In the corpus there are only two unattested combinations, namely the marking with short \textitbf{pu} ‘\textsc{poss}’ of long third person plural \textitbf{dorang} ‘\textsc{3pl}’ and of short first person plural \textitbf{torang} ‘\textsc{1pl}’. The elicited examples in (\ref{Footnote_Example_9.1}) show, however, that possessive constructions with \textitbf{torang}/\textitbf{dorang} \textitbf{pu} ‘our/their’ are possible:
\vspace{-5pt}
\ea
\label{Footnote_Example_9.1}
\gll \bluebold{torang} / \bluebold{dorang} \bluebold{pu} \bluebold{ruma} ada di situ\\
\textsc{1pl} / \textsc{3pl} \textsc{poss} house exist at \textsc{l.med}\\
\glt ‘\bluebold{our/their house} is over there’ [Elicited BR111020-001.002-003]
\z
} (The \isi{pronoun} \textitbf{ko} \textsc{‘2sg’} does not have a short form.) (Pronouns are discussed in detail in \chapref{Para_6}.)


\begin{table}

\caption[Possessive marking of personal pronouns]{Possessive marking of personal pronouns\footnote{Documentation: 080916-001-CvNP.0006, 080917-008-NP.0166, 080919-004-NP.0018, 080919-004-NP.0053, 080919-004-NP.0071, 080919-004-NP.0079, 080922-001a-CvPh.0834, 080922-002-Cv.0006, 080922-005-CvEx.0004, 080922-010a-NF.0002, 080922-010a-NF.0288, 081006-022-CvEx.0043, 081006-022-CvEx.0047, 081006-029-CvEx.0015, 081011-011-Cv.0055, 081011-011-Cv.0057, 081015-005-NP.0011, 081015-005-NP.0023, 081110-001-Cv.0026, 081110-002-Cv.0015, 081110-002-Cv.0018, 081110-003-Cv.0023, 081110-008-CvHt.0058, 081110-008-CvHt.0101, 081115-001a-Cv.0275, 081115-001b-Cv.0026, 081115-001b-Cv.0026, 081115-001b-Cv.0057.}}\label{Table_9.2}

\begin{tabularx}{\textwidth}{lll}
\lsptoprule
 \multicolumn{1}{c}{Possessive construction} & \multicolumn{1}{c}{Glosses} &  \multicolumn{1}{c}{Free translation}\\
\midrule
\multicolumn{3}{l}{Possessive marking with \textitbf{punya} ‘\textsc{poss}’}\\
\multicolumn{3}{l}{Long personal \isi{pronoun} form – \textitbf{punya} ‘\textsc{poss}’}\\
\midrule
{\textitbf{saya punya sabit}} & 1\textsc{sg} \textsc{poss} sickle & ‘my sickle’\\
{\textitbf{ko punya barang}} & 2\textsc{sg} \textsc{poss} stuff & ‘your belongings’\\
{\textitbf{dia punya nama}} & 3\textsc{sg} \textsc{poss} name & ‘his name’\\
{\textitbf{kitorang punya kekurangang}} & 1\textsc{pl} \textsc{poss} shortcoming & ‘our shortcomings’\\
{\textitbf{kitong punya muka}} & 1\textsc{pl} \textsc{poss} face & ‘our faces’\\
{\textitbf{kita punya bapa}} & 1\textsc{pl} \textsc{poss} father & ‘our father’\\
{\textitbf{kamu punya otak}} & 2\textsc{pl} \textsc{poss} brain & ‘your brains’\\
{\textitbf{dorang punya kampung}} & 3\textsc{pl} \textsc{poss} village & ‘their village’\\
\midrule
\multicolumn{3}{l}{Short personal \isi{pronoun} form – \textitbf{punya} ‘\textsc{poss}’}\\
\midrule
\textitbf{sa punya nokeng} & {1\textsc{sg} \textsc{poss} stringbag} & ‘my stringbag’\\
\textitbf{de punya swami} & {3\textsc{sg} \textsc{poss} husband} & ‘her husband’\\
\textitbf{torang punya orang-tua} & {1\textsc{pl} \textsc{poss} parent} & ‘our parents’\\
\textitbf{tong punya ipar} & {1\textsc{pl} \textsc{poss} sibling-in-law} & ‘our sister in-law’\\
\textitbf{ta punya kampung} & {1\textsc{pl} \textsc{poss} village} & ‘our village’\\
\textitbf{kam punya nasip} & {2\textsc{pl} \textsc{poss} destiny} & ‘your destiny’\\
\textitbf{dong punya ruma} & {3\textsc{pl} \textsc{poss} house} & ‘their house’\\
\midrule
\multicolumn{3}{l}{Possessive marking with \textitbf{pu} ‘\textsc{poss}’}\\
\multicolumn{3}{l}{Long personal \isi{pronoun} form – \textitbf{pu} ‘\textsc{poss}’}\\
\midrule
\textitbf{saya pu hasil kebung} & {1\textsc{sg} \textsc{poss} product garden} & ‘my garden products’\\
\textitbf{ko pu kampung} & {2\textsc{sg} \textsc{poss} village} & ‘your village’\\
\textitbf{dia pu maytua} & {3\textsc{sg} \textsc{poss} wife} & ‘his wife’\\
\textitbf{kitorang pu keadaang} & {1\textsc{pl} \textsc{poss} condition} & ‘our \isi{condition}’\\
\textitbf{kitong pu kawang} & {1\textsc{pl} \textsc{poss} friend} & ‘our friend’\\
\textitbf{kita pu adat} & {1\textsc{pl} \textsc{poss} customs} & ‘our customs’\\
\textitbf{kamu pu cara hidup} & {2\textsc{pl} \textsc{poss} manner live} & ‘your ways of life’\\
\lspbottomrule
\end{tabularx}

\end{table}
\begin{table}

\caption[Possessive marking of personal pronouns continued]{Possessive marking of personal pronouns continued\footnote{Documentation: 080916-001-CvNP.0006, 080917-008-NP.0166, 080919-004-NP.0018, 080919-004-NP.0053, 080919-004-NP.0071, 080919-004-NP.0079, 080922-001a-CvPh.0834, 080922-002-Cv.0006, 080922-005-CvEx.0004, 080922-010a-NF.0002, 080922-010a-NF.0288, 081006-022-CvEx.0043, 081006-022-CvEx.0047, 081006-029-CvEx.0015, 081011-011-Cv.0055, 081011-011-Cv.0057, 081015-005-NP.0011, 081015-005-NP.0023, 081110-001-Cv.0026, 081110-002-Cv.0015, 081110-002-Cv.0018, 081110-003-Cv.0023, 081110-008-CvHt.0058, 081110-008-CvHt.0101, 081115-001a-Cv.0275, 081115-001b-Cv.0026, 081115-001b-Cv.0026, 081115-001b-Cv.0057.}}\label{Table_9.2a}
\begin{tabularx}{\textwidth}{lll}
\lsptoprule
 \multicolumn{1}{c}{Possessive construction} & \multicolumn{1}{c}{Glosses} &  \multicolumn{1}{c}{Free translation}\\
\midrule
\multicolumn{3}{l}{Short personal \isi{pronoun} form – \textitbf{pu} ‘\textsc{poss}’}\\
\midrule
\textitbf{sa pu motor} & {1\textsc{sg} \textsc{poss} motorbike} & ‘my motorbike’\\
\textitbf{de pu bahu} & {3\textsc{sg} \textsc{poss} shoulder} & ‘her shoulder’\\
\textitbf{tong pu pakeang} & {1\textsc{pl} \textsc{poss} clothing} & ‘our clothing’\\
\textitbf{ta pu orang-tua} & {1\textsc{pl} \textsc{poss} parent} & ‘our parents’\\
\textitbf{kam pu sabung} & {2\textsc{pl} \textsc{poss} soap} & ‘their soap’\\
\textitbf{dong pu jaring} & {3\textsc{pl} \textsc{poss} net} & ‘their net’\\
\lspbottomrule
\end{tabularx}

\end{table}

These examples show that the reduction of the disyllabic form \textitbf{punya} ‘\textsc{poss}’ to monosyllabic \textitbf{pu} ‘\textsc{poss}’ does not interact with the long versus reduced shape of the personal pronouns. These findings contrast with those of \citet{Donohue.2003} who found that the long \isi{pronoun} forms may not co-occur with the reduced possessive marker \textitbf{pu} ‘\textsc{poss}’ (for more details see \citealt[24–25]{Donohue.2003}).

Very occasionally, the reduced ligature takes on the form /\textstyleChCharisSIL{pum}/, /\textstyleChCharisSIL{pun}/, or /\textstyleChCharisSIL{puŋ}/ ‘\textsc{poss}’. This \isi{variation} is usually due to \isi{assimilation} to the word-initial segment of the following possessum, as illustrated in \tabref{Table_9.3}. That is, speakers realize short \textitbf{pu} ‘\textsc{poss}’ with a word-final nasal which receives its place features from the onset segment of the following prosodic word; when the following word has a vowel as onset, the nasal is typically realized as velar [\textstyleChCharisSIL{ŋ}]. (For more details on nasal place \isi{assimilation} see §\ref{Para_2.2.1}.)


\begin{table}
\caption{Assimilation of short \textitbf{pu} ‘\textsc{poss}’}\label{Table_9.3}

\begin{tabular}{lll}
\lsptoprule
 Item & Orthogr. &  Gloss\\
 \midrule

/\textstyleChCharisSIL{dɛ pu}\textstyleChCharisSILBlueBold{m}\textstyleChCharisSIL{ }\textstyleChCharisSILBlueBold{b}\textstyleChCharisSIL{apa}/ & \textitbf{de pu bapa} & ‘his/her father’\\
/\textstyleChCharisSIL{dɛ pu}\textstyleChCharisSILBlueBold{n}\textstyleChCharisSIL{ }\textstyleChCharisSILBlueBold{t}\textstyleChCharisSIL{ɛman{\Tilde}tɛmaŋ}/ & \textitbf{de pu temang{\Tilde}temang} & ‘his/her friends’\\
/\textstyleChCharisSIL{sa pu}\textstyleChCharisSILBlueBold{ŋ}\textstyleChCharisSIL{ }\textstyleChCharisSILBlueBold{k}\textstyleChCharisSIL{aka}/ & \textitbf{sa pu kaka} & ‘my older sibling’\\
/\textstyleChCharisSIL{d}ɔ\textstyleChCharisSIL{m pu}\textstyleChCharisSILBlueBold{ŋ} \textstyleChCharisSILBlueBold{a}srama/ & \textitbf{dong pu asrama} & ‘their dormitory’\\
\lspbottomrule
\end{tabular}
\end{table}

In a few cases, however, the reduced ligature takes on the form /\textstyleChCharisSIL{puŋ}/ regardless of the form of the following segment, as illustrated in (\ref{Example_9.6}) to (\ref{Example_9.8}).


\ea
\label{Example_9.6}
\gll {ada} {sa} \textup{/\textstyleChCharisSILBlueBold{puŋ}/} {\bluebold{d}usung}\\ %
 exist  1\textsc{sg}  \textsc{poss}  garden\\
\glt 
‘(over there) is \bluebold{my} garden’ \textstyleExampleSource{[081110-008-CvNP.0009]}
\z

\ea
\label{Example_9.7}
\gll   dong  \textup{/\textstyleChCharisSILBlueBold{puŋ}/}  \bluebold{p}eserta juga  macang  tra  {\ldots}\\
 \textsc{3pl}  \textsc{poss}  participant  also  variety  \textsc{neg}  \\
\glt 
‘\bluebold{their} participants also, like (they) didn’t {\ldots}’ \textstyleExampleSource{[081025-009a-Cv.0132]}
\z

\ea
\label{Example_9.8}
\gll {\ldots} {tong} \textup{/\textstyleChCharisSILBlueBold{puŋ}/} {\bluebold{c}ara} {makang} {babi} {juga}\\ %
 { }  \textsc{1pl}  \textsc{poss}  manner  eat  pig  also\\
\glt
‘[our way of eating is just like the Toraja one,] \bluebold{our} way of eating pigs also’ \textstyleExampleSource{[081014-017-CvPr.0053]}
\z


\subsection{\textsc{possr-np} \textitbf{=p} \textsc{possm-np}}
\label{Para_9.1.2}
The possessive marker can be reduced further to \textitbf{=p} ‘\textsc{poss}’, if the \isi{possessor \isi{noun} phrase} ends in a vowel, as in (\ref{Example_9.9}) to (\ref{Example_9.11}). In this case, the marker is cliticized to the possessor. In this type of reduced possessive construction, the possessor is almost always a singular personal \isi{pronoun}, such as short first person \textitbf{sa} ‘\textsc{1sg}’, second person \textitbf{ko} ‘\textsc{2sg}’ in (\ref{Example_9.9}), or short third person \textitbf{de} ‘\textsc{3sg}’ as in (\ref{Example_9.10}). The possessor may, however, also be expressed by a \isi{noun} as in (\ref{Example_9.11}), although in the corpus this example is the only one attested. Again, the same construction is used for alienable and inalienable possession.


\ea
\label{Example_9.9}
\gll {sa} {bilang,} {i,} {\bluebold{sa=p}} {\bluebold{kaka},} {de} {bilang} {\bluebold{ko=p}} {\bluebold{kaka}}?\\ %
 \textsc{1sg}  say  ugh!  \textsc{1sg=poss}  oSb  \textsc{3sg}  say  \textsc{2sg=poss}  oSb\\
\glt 
‘I said, ``ugh!, (that’s) \bluebold{my older sister}'', she said, ``\bluebold{your older sister}?''' \textstyleExampleSource{[080919-006-CvNP.0026]}
\z

\ea
\label{Example_9.10}
\gll {de} {timbul} {\bluebold{de=p}} {\bluebold{cucu}} {tanya} {dia,} {tete} {knapa}\\ %
 \textsc{3sg}  emerge  \textsc{3sg=poss}  grandchild  ask  \textsc{3sg}  grandfather  why\\
\glt 
‘(when) he (grandfather) emerged, his grandchild asked him, ``grandfather, what happened?''' \textstyleExampleSource{[081109-005-JR.0009]}
\z

\ea
\label{Example_9.11}
\gll {Fredi} {de} {pu} {\bluebold{ade=p}} {\bluebold{motor}} {\ldots}\\ %
 Fredi  \textsc{3sg}  \textsc{poss}  ySb\textsc{=poss}  motorbike  \\
\glt
‘Fredi’s \bluebold{younger brother’s motorbike} {\ldots}’ \textstyleExampleSource{[081002-001-CvNP.0058]}
\z


\subsection{\textsc{possr-np} Ø \textsc{possm-np}}
\label{Para_9.1.3}
The possessive marker can also be elided, as illustrated in (\ref{Example_9.12}) to (\ref{Example_9.16}). The \isi{elision} is limited, however, to certain semantic kinds of possession. Attested are inalienable possession of body parts, as in (\ref{Example_9.12}) and (\ref{Example_9.13}), and kinship relations, as in (\ref{Example_9.14}) and (\ref{Example_9.15}). Most commonly, the possessor is human as in (\ref{Example_9.12}) to (\ref{Example_9.15}), but it may also be animate nonhuman as in (\ref{Example_9.16}).



In \textsc{possr-possm} constructions, the possessor is usually encoded by a short personal \isi{pronoun} form, as in (\ref{Example_9.14}) to (\ref{Example_9.16}). Much less often, the possessor is expressed with a lexical \isi{noun}, such as \textitbf{bapa} ‘father’ in (\ref{Example_9.12}). Also rather infrequently, the possessor is expressed by a \isi{noun} phrase such as \textitbf{pace de} ‘the man’ in (\ref{Example_9.13}), where adnominally used \textitbf{de} ‘\textsc{3sg}’ modifies \textitbf{pace} ‘man’ (for details on the adnominal uses of the personal pronouns, see §\ref{Para_6.2}).


\ea
\label{Example_9.12}
\gll {adu,} {\bluebold{bapa}} {\bluebold{Ø}} {\bluebold{mulut}} {jahat} {skali}\\ %
 oh.no!  father { }    mouth  be.bad  very\\
\glt 
‘oh no!, \bluebold{father’s language} is very bad’ (Lit. ‘\bluebold{father’s mouth}’) \textstyleExampleSource{[080923-008-Cv.0019]}
\z

\ea
\label{Example_9.13}
\gll {\bluebold{pace}} {\bluebold{de}} {\bluebold{Ø}} {\bluebold{tangang}} {kluar} {ke} {samping}\\ %
 man  \textsc{3sg} { }    hand  go.out  to  side\\
\glt 
[About an accident:] ‘\bluebold{the man’s arm} stuck out sideways’ \textstyleExampleSource{[081108-001-JR.0003]}
\z

\ea
\label{Example_9.14}
\gll {\bluebold{de}} {\bluebold{Ø}} {\bluebold{mama}} {\bluebold{ini}} {ke} {atas}\\ %
 \textsc{3sg} { }   see  \textsc{3sg}  \textsc{poss}  wife\\
\glt 
‘\bluebold{his mother here} (went) up (there)’ \textstyleExampleSource{[080923-001-CvNP.0019]}
\z

\ea
\label{Example_9.15}
\gll {dia} {liat} {dia} {pu} {maytua} {\ldots} {ah,} {\bluebold{sa}} {\bluebold{Ø}} {\bluebold{maytua}} {cantik}\\ %
 \textsc{3sg}  see  \textsc{3sg}  \textsc{poss}  wife { }   ah!  \textsc{1sg} { }   wife  be.beautiful\\
\glt 
‘he saw his wife {\ldots} ``ah!, my wife is beautiful''' \textstyleExampleSource{[080922-010a-CvNF.0020]}
\z

\ea
\label{Example_9.16}
\gll {{langsung}} {{potong}} {dia} {{buang}} {{tali-prutnya}}\\ %
 {immediately}  {cut}  \textsc{3sg}  {throw(.away)}  {intestines:\textsc{3possr}}\\
\gll \bluebold{de}  \bluebold{Ø}  {\bluebold{tali-prut}}  {buang,}  {tinggal}  isi  saja\\
 \textsc{3sg}  { }  {intestines}  {throw(.away)}  {stay}  contents  just\\
\glt 
[About killing dogs:] ‘cut him up at once (and) throw away the intestines, (after having) thrown away \bluebold{his intestines}, just the meat remains’ \textstyleExampleSource{[081106-001-CvPr.0005]}
\z



Contrary to the possessive constructions presented in §\ref{Para_9.1.1} and §\ref{Para_9.1.2}, the data presented in (\ref{Example_9.12}) to (\ref{Example_9.16}) shows that Papuan Malay also has the option to signal inalienable possession by omitting the possessive marker.



This alienable versus inalienable distinction is also found in other \ili{Austronesian} languages of the \isi{Papuan contact zone}, whereas it is not found in Western \ili{Malayo-Poly\-ne\-sian} languages. As in other \ili{Austronesian} and \ili{Papuan languages} of this contact zone \citep[116]{Klamer.2008}, it is body parts and kinship terms that can be inalienably possessed.\footnote{\citet[116]{Klamer.2008} note that this “innovation must have occurred prior to the population of Oceania”, a conclusion that is based on \citegens{Ross.2001} hypothesis that it “is also probable that the formal distinction between alienable and inalienable possession entered Proto-\ili{Oceanic} or an immediate precursor through Papuan contact”.}


\subsection{Grammaticalization of \textitbf{punya} ‘\textsc{poss}’}
\label{Para_9.1.4}
In §\ref{Para_9.1.1} to §\ref{Para_9.1.3}, the reduction of possessive marker \textitbf{punya} ‘\textsc{poss}’ to its monosyllabic variants \textitbf{pu} or \textitbf{=p} ‘\textsc{poss}’ and its omission in \textsc{possr-possm} constructions was described.



One explanation for this reduction would be to consider it as the result of a \isi{grammaticalization} process. As \citet[719]{Bybee.2006} observes, the phonetic reduction of high-frequency words “can lead to the establishment of a new construction with its own categories” and “the grammaticization of the new construction”. One could argue that \citegen[719]{Bybee.2006} observation also applies to the high-frequency morpheme \textitbf{punya} with its variable status between a full \isi{verb} ‘have’, a clitic possessive marker, and a zero morpheme. That is, the variable status could be taken as an as-yet incomplete \isi{grammaticalization} from the independent lexical item \textitbf{punya} ‘have’ via the possessive marker \textitbf{punya} ‘\textsc{poss}’ into a clitic \textitbf{=p} ‘\textsc{poss}’ or a new possessive construction without overt marker.\footnote{One reason why \textitbf{punya} constructions are so frequent in Papuan Malay and other \ili{eastern Malay varieties}, is that unlike the \ili{western Malay varieties}, the \ili{eastern Malay varieties} do not use suffix \textitbf{-nya} ‘\textsc{3poss}’ as a marker of possessive relations, as for instance in western Malay \textitbf{tangang\-nya} ‘his/her hand’ (H. Hammarström p.c. 2013).}



In the corpus, reductions of the possessive marker to the clitic \textitbf{=p} ‘\textsc{poss}’ or a zero morpheme occur with about the same frequency. Typically, the two constructions occur when the possessor is expressed with a short singular personal \isi{pronoun}. It remains to be seen whether and to what extent over time (1) one of the constructions is going to become dominant, and (2) one or both constructions are going to occur with possessors encoded by the plural personal pronouns or common nouns. Such developments could be taken as an indication of a \isi{grammaticalization} process of the possessive marker.


\section{Realizations of \textsc{possr-np} and \textsc{possm-np}}
\label{Para_9.2}
This section discusses the different realizations of the possessor and possessum \isi{noun} phrases in adnominal possessive constructions. The syntactic categories that can take the possessor or possessum slots, together with their semantic properties are discussed in §\ref{Para_9.2.1.1}. Elision of the \isi{possessum \isi{noun} phrase} is described in §\ref{Para_9.2.1.2}, followed by a brief discussion of \isi{recursive} possessive constructions in §\ref{Para_9.2.1.3}.


\subsection[Syntactic and semantic properties]{Syntactic and semantic properties}
\label{Para_9.2.1.1}
In adnominal possessive constructions, the possessor and/or possessum can be expressed by lexical nouns as in (\ref{Example_9.17}) and (\ref{Example_9.18}), by demonstratives as in (\ref{Example_9.19}) and (\ref{Example_9.20}), or by \isi{noun} phrases as in (\ref{Example_9.21}) to (\ref{Example_9.26}). Further, the possessor can be encoded by a personal \isi{pronoun} as in (\ref{Example_9.23}) to (\ref{Example_9.26}). Semantically, the possessor and the possessum can be human as in (\ref{Example_9.17}), nonhuman animate as in (\ref{Example_9.18}), or inanimate as in (\ref{Example_9.27}), respectively.



In (\ref{Example_9.17}) and (\ref{Example_9.18}), the possessor and the possessum are expressed by lexical nouns.



\begin{styleExampleTitle}
Lexical nouns expressing the possessor / possessum
\end{styleExampleTitle}

\ea
\label{Example_9.17}
\gll {sa} {masi} {ingat} {\bluebold{bapa}} {\bluebold{pu}} {\bluebold{muka}}\\ %
 \textsc{1sg}  still  remember  father  \textsc{poss}  front\\
\glt 
‘I still remember \bluebold{father’s face}’ \textstyleExampleSource{[080922-001a-CvPh.1307]}
\z

\ea
\label{Example_9.18}
\gll {\ldots} {pake} {\bluebold{setang}} {\bluebold{punya}} {\bluebold{kwasa}}\\ %
 { }  use  evil.spirit  \textsc{poss}  power\\
\glt 
[About the power of evil spirits:] ‘[the sleeping person can’t wake up because the sorcerers are] using \bluebold{the evil spirit’s power}’ \textstyleExampleSource{[081006-022-CvEx.0084]}
\z



In (\ref{Example_9.19}) the proximal \isi{demonstrative} \textitbf{ini} ‘\textsc{d.prox}’ takes the possessor slot and in (\ref{Example_9.20}) distal \textitbf{itu} ‘\textsc{d.dist}’ takes the possessum slot.



\begin{styleExampleTitle}
Demonstratives expressing the possessor / possessum
\end{styleExampleTitle}
\ea
\label{Example_9.19}
\gll {bapa} {{masi}} {{kenal}} {kaka} {Siduas} {pu,} {masi} {kenal}\\ %
 father  {still}  {know}  oSb  Siduas  \textsc{poss}  still  know\\
\gll {\bluebold{ini}}  {\bluebold{pu}}  {\bluebold{muka}}\\
 {\textsc{d.prox}}  {\textsc{poss}}  {front}\\
\glt 
‘do you (‘father’) still know Siduas’, still know \bluebold{this (one)’s face}?’ \textstyleExampleSource{[080922-001a-CvPh.1123]}
\z

\ea
\label{Example_9.20}
\gll {ko} {ambil} {dulu} {\bluebold{ade}} {\bluebold{pu}} {\bluebold{itu}}\\ %
 \textsc{2sg}  fetch  first  ySb  \textsc{poss}  \textsc{d.dist}\\
\glt 
‘you pick (it) up first, \bluebold{that (fish) of (your) younger sister}’ (Lit. ‘\bluebold{younger sibling’s that}’) \textstyleExampleSource{[081006-019-Cv.0002]}
\z



In (\ref{Example_9.21}) to (\ref{Example_9.26}), \isi{noun} phrases take the possessor or the possessum slot (the scope of the \isi{noun} phrases is indicated with brackets). In (\ref{Example_9.21}) the possessor is encoded by a \isi{noun} phrase with a verbal modifier plus an adnominal \isi{demonstrative}, while in (\ref{Example_9.22}) the possessor is expressed by a coordinate \isi{noun} phrase.



\begin{styleExampleTitle}
Noun phrases expressing the possessor
\end{styleExampleTitle}

\ea
\label{Example_9.21}
\gll {sebut} {[[[\bluebold{orang}} {\bluebold{mati}]} {\bluebold{tu}]} {\bluebold{pu}} {[\bluebold{nama}]]} {karna} {\ldots}\\ %
 name  person  die  \textsc{d.dist}  \textsc{poss}  name  because  \\
\glt 
‘(he has) to mention \bluebold{that dead person’s name} because {\ldots}’ \textstyleExampleSource{[080923-013-CvEx.0019]}
\z

\ea
\label{Example_9.22}
\gll {itu} {ko} {pu} {[[\bluebold{ko}} {\bluebold{deng}} {\bluebold{Mateus}]} {\bluebold{pu}} {[\bluebold{tugas}]]}\\ %
 \textsc{d.dist}  \textsc{2sg}  \textsc{poss}  \textsc{2sg}  with  Mateus  \textsc{poss}  duty\\
\glt 
‘that is your, \bluebold{your and Mateus’ duty}’ \textstyleExampleSource{[081005-001-Cv.0035]}
\z


In (\ref{Example_9.23}), the possessum is encoded by a \isi{noun} phrase with an adnominally used stative \isi{verb} plus an adnominal \isi{demonstrative}. In (\ref{Example_9.24}), a \isi{noun} phrase with nominal modifier plus an adnominal \isi{demonstrative} takes the possessum slot. In (\ref{Example_9.25}) the possessum is expressed by a coordinate \isi{noun} phrase. In (\ref{Example_9.8}), repeated as (\ref{Example_9.26}), a \isi{noun} phrase with a modifying non-finite clause takes the possessum slot. The examples in (\ref{Example_9.23}) to (\ref{Example_9.26}) also illustrate that a personal \isi{pronoun} can take the possessor slot; personal pronouns do not take the possessum slot.



\begin{styleExampleTitle}
Noun phrases expressing the possessum
\end{styleExampleTitle}
\ea
\label{Example_9.23}
\gll {[[\bluebold{de}]} {{\bluebold{pu}}} {[[\bluebold{cucu}} {\bluebold{kecil}]} {\bluebold{itu}]]} {tiap} {hari} {de}\\ %
 \textsc{3sg}  {\textsc{poss}}  grandchild  be.small  \textsc{d.dist}  every  day  \textsc{3sg}\\
\gll {menangis}  {trus}\\
 {cry}  {be.continuous}\\
\glt 
‘\bluebold{that small grandchild of his}, every day he/she cries continuously’ \textstyleExampleSource{[081011-009-Cv.0055]}
\z

\ea
\label{Example_9.24}
\gll {sa} {tida} {maw} {[[\bluebold{sa}]} {\bluebold{punya}} {[[\bluebold{sodara}} {\bluebold{prempuang}]} {\bluebold{itu}]]} {mendrita}\\ %
 \textsc{1sg}  \textsc{neg}  want  \textsc{1sg}  \textsc{poss}  sibling  woman  \textsc{d.dist}  suffer\\
\glt 
‘I don’t want \bluebold{that sister of mine} to suffer’ \textstyleExampleSource{[081006-024-CvEx.0108]}
\z

\ea
\label{Example_9.25}
\gll {nanti} {[[\bluebold{de}]} {\bluebold{punya}} {[\bluebold{bapa}} {\bluebold{dengang}} {\bluebold{mama}]]} {langsung} {pergi} {\ldots}\\ %
 very.soon  \textsc{3sg}  \textsc{poss}  father  with  mother  immediately  go  \\
\glt 
‘very soon \bluebold{her father and mother} will go {\ldots}’ \textstyleExampleSource{[081110-005-CvPr.0079]}
\z

\ea
\label{Example_9.26}
\gll {\ldots} {[[\bluebold{tong}]} {\bluebold{pu}} {[\bluebold{cara}} {[\bluebold{makang}} {\bluebold{babi}]]]} {juga}\\ %
 { }  \textsc{1pl}  \textsc{poss}  manner  eat  pig  also\\
\glt 
‘[our way of eating is just like the Toraja one,] \bluebold{our way of eating pigs} also’ \textstyleExampleSource{[081014-017-CvPr.0053]}
\z



In (\ref{Example_9.17}) to (\ref{Example_9.26}) the possessor is always animate and/or human. It can, however, also be inanimate as shown in (\ref{Example_9.27}).


\begin{styleExampleTitle}
Inanimate possessor
\end{styleExampleTitle}

\ea
\label{Example_9.27}
\gll {\ldots} {\bluebold{LNG}} {\bluebold{pu}} {\bluebold{terpol}} {\bluebold{itu}} {tinggal}\\ %
 { }  liquefied.natural.gas  \textsc{poss}  container  \textsc{d.dist}  stay\\
\glt
[About the need to buy gasoline:] ‘[those jerry cans] \bluebold{that LNG jerry can} stays behind’ (Lit. ‘\bluebold{the LNG’s container}’) \textstyleExampleSource{[081110-002-Cv.0075]}
\z


\subsection[Elision of the possessum noun phrase]{Elision of the possessum noun phrase}
\label{Para_9.2.1.2}
It is also possible to omit the possessum when its identity was established earlier; this applies to inalienably as well as alienably possessed referents, as illustrated in (\ref{Example_9.28}) to (\ref{Example_9.31}). Such ‘\textsc{possessor} \textitbf{punya}’ constructions are typically used in contexts where the possessor identity is under discussion.



In `\textsc{possessor} \textitbf{punya}’ constructions, speakers most commonly employ long \textitbf{punya} ‘\textsc{poss}’, but as shown in (\ref{Example_9.30}) and (\ref{Example_9.31}), constructions with reduced \textitbf{pu} ‘\textsc{poss}’ are also possible.\footnote{In their analysis of similar possessive constructions in Classical Malay, \citet[157]{Yap.2004} conclude that Classical Malay \textitbf{(em)punya} constructions with omitted possessum denote “pronominal possessive constructions”. More specifically, \citet[7]{Yap.2007} maintains that “in such constructions \textit{(em)punya }identifies a possessee in relation to its possessor (the genitive function), while at the same time alluding to the morphologically unrealized possessee as well (the pronominal function). Consequently, possessive pronominal \textit{(em)punya }allows us to focus on the possessor, while still referring to the possessee”. It seems that this analysis is also applicable to Papuan Malay.} Elision of the possessor is unattested. Instead speakers employ a \isi{demonstrative}, as in (\ref{Example_9.19}), when the identity of the possessor has already been established.


\ea
\label{Example_9.28}
\gll {Nofi} {{tu}} {{itu}} {{bukang}} {\bluebold{bapa}} {\bluebold{Lukas}} {\bluebold{punya}} {\bluebold{Ø}} {\bluebold{mama}}  {\bluebold{Nofita}}  {\bluebold{punya}}  {\bluebold{Ø}}\\ %
 Nofi  {\textsc{d.dist}}  {\textsc{d.dist}}  {\textsc{neg}}  father  Lukas  \textsc{poss} {} {mother}  {Nofita}  {\textsc{poss}}  {} \\
\glt 
‘Nofi here, that’s not \bluebold{father Lukas’} (son nor) \bluebold{mother Nofita’s} (son)’ \textstyleExampleSource{[081006-024-CvEx.0011]}
\z

\ea
\label{Example_9.29}
\gll  {itu}  \bluebold{de}  \bluebold{punya}  \bluebold{Ø}\\
 \textsc{d.dist}  \textsc{3sg}  \textsc{poss}  \\
\glt 
‘those are \bluebold{his} (banana plants)’ \textstyleExampleSource{[081110-008-CvNP.0121]}
\z

\ea
\label{Example_9.30}
\gll {sedangkang} {Pawlus} {ini} {itu} {\bluebold{jing}} {\bluebold{pu}} {\bluebold{Ø}}\\ %
 whereas  Pawlus  \textsc{d.prox}  \textsc{d.dist}  genie  \textsc{poss}  \\
\glt 
‘whereas Pawlus here, that’s \bluebold{the genie’s} (child)’ \textstyleExampleSource{[081025-006-Cv.0276]}
\z

\ea
\label{Example_9.31}
\gll {ko} {liat} {\bluebold{Luisa}} {\bluebold{pu}} {\bluebold{Ø}} {bagus,} {suda} {kembang} {banyak}\\ %
 \textsc{2sg}  see  Luisa  \textsc{poss}   { }  be.good  already  f\isi{lowering}  many\\
\glt
‘you see \bluebold{Luisa’s} (flowers) are good, (they are) already f\isi{lowering} a lot’ \textstyleExampleSource{[081006-021-CvHt.0002]}
\z


\subsection[Recursive adnominal possessive constructions]{Recursive adnominal possessive constructions}
\label{Para_9.2.1.3}
Adnominal possessive constructions can be stacked to form \isi{recursive} possessive constructions, as illustrated in (\ref{Example_9.32}) to (\ref{Example_9.34}). Double possessive constructions, as in (\ref{Example_9.32}) and (\ref{Example_9.33}) are quite common, especially to express kinship and social relations, as in (\ref{Example_9.32}). Triple possessive constructions are also possible but extremely rare: the corpus contains only one such construction, which is presented in (\ref{Example_9.34}) in slightly modified form.


\ea
\label{Example_9.32}
\gll {kalo} {memang} {\bluebold{ko}} {\bluebold{punya}} {\bluebold{maytua}} {\bluebold{punya}} {\bluebold{waktu}} {pas} {di} {kapal} {\ldots}\\ %
 if  indeed  2\textsc{sg}  \textsc{poss}  wife  \textsc{poss}  time  precisely  at  ship  \\
\glt 
‘if indeed \bluebold{your wife’s time} (to give birth) is right then (when you’re) on the ship {\ldots}’ \textstyleExampleSource{[080922-001a-CvPh.0010]}
\z

\ea
\label{Example_9.33}
\gll {ini} {\bluebold{kaka}} {\bluebold{Natanael}} {\bluebold{pu}} {\bluebold{laki}} {\bluebold{pu}} {\bluebold{mobil}}\\ %
 \textsc{d.prox}  oSb  Natanael  \textsc{poss}  husband  \textsc{poss}  car\\
\glt 
‘this is \bluebold{sister Natanael’s husband’s car}’ \textstyleExampleSource{[081006-015-Cv.0001]}
\z

\ea
\label{Example_9.34}
\gll {de\textsubscript{i}} {{pu}} {ana} {{kawing}} {\bluebold{de}\textsubscript{i}} {\bluebold{pu}} {\bluebold{laki}} {\bluebold{punya}}\\ %
 \textsc{3sg}  {\textsc{poss}}  child  {marry.inofficially}  \textsc{3sg}  \textsc{poss}  husband  \textsc{poss}\\
\gll  {\bluebold{kaka}}  {\bluebold{prempuang}}  \bluebold{pu}  {\bluebold{ana}}\\
 {oSb}  {woman}  \textsc{poss}  {child}\\
\glt 
‘her\textsubscript{i} child (wants to) marry \bluebold{the son of her}\textsubscript{i}\bluebold{ husband’s older sister}’ \textstyleExampleSource{[Elicited BR111020.026]}\footnote{The elicited utterance in (\ref{Example_9.34}) is based on an original triple possessive construction which contains the \isi{demonstrative} \textitbf{ini} ‘\textsc{d.prox}’: \textitbf{{\ldots} de pu laki, ini, punya kaka prempuang pu ana}. In this context, \textitbf{ini} ‘\textsc{d.prox}’ functions as a \isi{placeholder} and therefore is not part of the \isi{noun} phrase \textitbf{de pu laki} ‘her husband’ (see §\ref{Para_7.1.2.6} for a discussion of the \isi{placeholder} uses of demonstratives).\\
The subscript letters indicate which personal pronouns have which referents.}
\z



As discussed in §\ref{Para_9.1.1}, the long ligature form \textitbf{punya} ‘\textsc{poss}’ and short \textitbf{pu} ‘\textsc{poss}’ are freely used in adnominal possessive constructions without any syntactic or semantic restrictions. This also applies to \isi{recursive} possessive constructions, as illustrated in (\ref{Example_9.32}) to (\ref{Example_9.34}). In terms of the attested frequencies in such constructions, however, short \textitbf{pu} ‘\textsc{poss}’ is employed more often than long \textitbf{punya} ‘\textsc{poss}’.


\section{Noncanonical adnominal possessive constructions}
\label{Para_9.3}
In addition to encoding \isi{adnominal possession}, \textitbf{punya} ‘\textsc{poss}’ (including its reduced forms) also serves other functions in possessive constructions, namely as (1) an emphatic marker that signals \isi{locational} relations or association (§\ref{Para_9.3.1}), (2) a marker of beneficiary relations (§\ref{Para_9.3.2}), or (3) an attitudinal intensifier or stance (§\ref{Para_9.3.3}). And (4), the possessive ligature is also used in reflexive construction (§\ref{Para_9.3.4}).



Syntactically, not only nouns, personal pronouns, demonstratives, or \isi{noun} phrases can take the possessor or possessum slots. In addition, these slots can be filled by verbs. Further, the possessum slot can be taken by mid-range quantifiers, temporal adverbs, or prepositional phrases. Finally, the possessum can be omitted.


\subsection{Locational relations and association}
\label{Para_9.3.1}
Cross-linguistically, one noncanonical function of possessive constructions is to signal that the possessum is “perceived to be closely related” to the possessor \citep[278]{Dixon.2010}. In Papuan Malay, this includes \isi{locational} relations, both spatial and temporal, and relations that express an association, as illustrated in (\ref{Example_9.35}) to (\ref{Example_9.41}). With this function of \textitbf{punya} ‘\textsc{poss}’, the possessive construction receives an emphatic reading; in the following examples the English translation attempts to convey this emphatic reading with the additional italicized information.



The possessive marker can signal \isi{locational} relations, or, employing \citegen[263]{Dixon.2010} terminology, relations of “orientation or location”. The \isi{locational} relations can be spatial, as in (\ref{Example_9.35}) and (\ref{Example_9.36}), or temporal, as in (\ref{Example_9.37}) to (\ref{Example_9.39}).\footnote{\citegen[263]{Dixon.2010} term “orientation/location” refers to spatial relations; temporal relations are not mentioned.}



In (\ref{Example_9.35}) and (\ref{Example_9.36}), \textitbf{pu} ‘\textsc{poss}’ marks spatial relations between the possessor and the possessum, with the possessive construction receiving an emphatic reading. In (\ref{Example_9.36}), a spatial referent, encoded with the \isi{proper noun} \textitbf{Jayapura}, takes the possessor slot. It denotes the location or source for the referent expressed by the possessum, \textitbf{dua blas orang} ‘twelve people’. In (\ref{Example_9.35}), the spatial referent, encoded in the \isi{prepositional phrase} \textitbf{di dalam itu} ‘in that inside’, takes the possessum slot.\footnote{The \isi{locative} \isi{preposition} \textitbf{di} ‘at, in’ can also be deleted (see §\ref{Para_10.1.5}) resulting in \textitbf{de pu dalam itu} ‘that inside (part) of it’.} It designates the location for the referent expressed by the pronominal possessor \textitbf{de} ‘\textsc{3sg}’.



\begin{styleExampleTitle}
Spatial \isi{locational} relations
\end{styleExampleTitle}

\ea
\label{Example_9.35}
\gll {\bluebold{Jayapura}} {\bluebold{pu}} {\bluebold{dua}} {\bluebold{blas}} {\bluebold{orang}} {yang} {lulus} {ka?}\\ %
 Jayapura  \textsc{poss}  two  teens  person  \textsc{rel}  pass(.a.test)  or\\
\glt 
‘aren’t there \bluebold{twelve people from Jayapura who graduated} (\textstyleChItalic{as opposed to other cities with fewer graduates})?’ (Lit. ‘\bluebold{Jayapura’s twelve people}’) \textstyleExampleSource{[081025-003-Cv.0311]}
\z

\ea
\label{Example_9.36}
\gll {{baru}} {{ambil}} {{bayi}} {{tu,}} {{bayi}} {yang} {\bluebold{de}} {\bluebold{pu}}\\ %
 {and.then}  {fetch}  {palm.stem}  {\textsc{d.dist}}  {palm.stem}  \textsc{rel}  \textsc{3sg}  \textsc{poss}\\
\gll \bluebold{di}  {\bluebold{dalam}}  {\bluebold{itu}}  {kang}  {kaya}  kapas  {to?}\\
 at  {inside}  {\textsc{d.dist}}  {you.know}  {like}  cotton  {right?}\\
\glt 
‘and then (he) took that palm stem, \bluebold{that inside (part) of it} (\textstyleChItalic{as opposed to other parts}), you know, is like cotton, right?’ \textstyleExampleSource{[080922-010a-CvNF.0073]}
\z



In the elicited examples in (\ref{Example_9.37}) to (\ref{Example_9.39}), the possessive marker signals temporal \isi{locational} relations. In these examples, the third person singular \isi{pronoun} \textitbf{de} ‘\textsc{3sg}’ takes the possessor slot. It designates the temporal reference point for the event under discussion. The possessum slot is taken by a temporal expression such as \textitbf{besok} ‘tomorrow’ in (\ref{Example_9.37}), \textitbf{pagi} ‘morning’ in (\ref{Example_9.38}), and \textitbf{malam} ‘night’ in (\ref{Example_9.39}). This temporal expression denotes a specific point in time relative to the temporal reference point expressed by the possessor.



\begin{styleExampleTitle}
Temporal \isi{locational} relations
\end{styleExampleTitle}

\ea
\label{Example_9.37}
\gll {\ldots} {trus} {{sa}} {{tinggal}} {{di}} {sana,} {trus} {\bluebold{de}} {\bluebold{pu}} {\bluebold{besok}}\\ %
 { }  next  {\textsc{1sg}}  {stay}  {at}  \textsc{l.dist}  next  \textsc{3sg}  \textsc{poss}  tomorrow\\
\gll  {baru}  {sa}  {kembali}  {\ldots}\\
 {and.then}  {\textsc{1sg}}  {return}  {}\\
\glt 
‘[two days ago I went to Abepura,] and then I stayed there, and then \bluebold{the} (\textstyleChItalic{very}) \bluebold{next day} only then did I return {\ldots}’ (Lit. ‘\bluebold{its tomorrow}’) \textstyleExampleSource{[Elicited BR111020.008]}
\z

\ea
\label{Example_9.38}
\gll {dong} {{kerja}} {{ruma}} {{dari}} {{pagi}} {sampe} {malam} {\bluebold{de}} {\bluebold{pu}} {\bluebold{pagi},}\\ %
 \textsc{3pl}  {work}  {house}  {from}  {morning}  until  night  \textsc{3sg}  \textsc{poss}  morning\\
\gll  {baru}  {dong}  {kasi}  {selesay}  {smua}\\
 {and.then}  {\textsc{3pl}}  {\textsc{give}}  {finish}  {all}\\
\glt 
‘they worked on the house from morning until evening, \bluebold{the} (\textstyleChItalic{very}) \bluebold{next morning} only then did they finish everything’ (Lit. ‘\bluebold{its morning}’) \textstyleExampleSource{[Elicited BR111020.009]}
\z

\ea
\label{Example_9.39}
\gll {{Petrus}} {{deng}} {Tinus} {{dong}} {{pi}} {{mandi}} {di} {pante} {tadi} {pagi,}\\ %
 {Petrus}  {with}  Tinus  {\textsc{3pl}}  {go}  {bathe}  at  coast  earlier  morning\\
\gll  \bluebold{de}  {\bluebold{pu}}  {\bluebold{malam}}  {dong}  {pi}  ke  {Jayapura}\\
 \textsc{3sg}  {\textsc{poss}}  {night}  {\textsc{3pl}}  {go}  to  {Jayapura}\\
\glt 
‘Petrus and Tinus went bathing at the beach this morning (and) \bluebold{this} (\textstyleChItalic{very}) \bluebold{evening} they went to Jayapura’ (Lit. ‘\bluebold{its night}’) \textstyleExampleSource{[Elicited BR111020.009]}
\z



Another cross-linguistically rather common function of the possessive marker it to indicate an “association” between the possessum and the possessor \citep[285]{Dixon.2010}. This also applies to Papuan Malay, as shown in (\ref{Example_9.40}) and (\ref{Example_9.41}). In (\ref{Example_9.40}), \textitbf{punya} ‘\textsc{poss}’ signals that the possessum \textitbf{tu} ‘\textsc{d.dist}’ is associated with the possessor \textitbf{lima juta} ‘five million’, giving the emphatic reading ``a minimum of five-million (\textstyleChItalic{as opposed to lower prices})''.\footnote{Alternatively, one might classify the possessive construction in (\ref{Example_9.40}) as an “‘appositive genitive’, where the two \isi{noun} phrases are equated denotatively”, adopting the terminology used by \citet[193]{Quirk.1972}.} Along similar lines, in (\ref{Example_9.41}), the ligature indicates an association between the possessum \textitbf{tu} ‘\textsc{d.dist}’ and the possessor \textitbf{tingkat propinsi} ‘provincial level’, resulting in the emphatic reading ``(a meeting at) the provincial level (\textstyleChItalic{and not at the regency level})''.



\begin{styleExampleTitle}
Association
\end{styleExampleTitle}

\ea
\label{Example_9.40}
\gll {yang} {{mahal}} {{yang}} {di} {atas} {satu} {jut}\\ %
 \textsc{rel}  {be.expensive}  {\textsc{rel}}  at  top  one  \textsc{tru}{}-million\\
\gll \bluebold{lima}  \bluebold{juta}  {\bluebold{punya}}  {\bluebold{tu}}\\
 five  million  {\textsc{poss}}  {\textsc{d.dist}}\\
\glt 
‘(traditional cloths from Sorong) which are expensive, which (cost) more than one million[\textsc{tru}], \bluebold{a minimum of five million} (\textstyleChItalic{as opposed to lower prices})’ (Lit. ‘\bluebold{that} (price) \bluebold{of five million}’) \textstyleExampleSource{[081006-029-CvEx.0009]}
\z

\ea
\label{Example_9.41}
\gll {kitong} {ikut} {ini} {\bluebold{tingkat}} {\bluebold{propinsi}} {\bluebold{punya}} {\bluebold{tu}}\\ %
 \textsc{1pl}  follow  \textsc{d.prox}  floor  province  \textsc{poss}  \textsc{d.dist}\\
\glt
‘we attended (a meeting at), what’s-its-name, \bluebold{the provincial level} (\textstyleChItalic{and not at the regency level})’ (Lit. ‘\bluebold{that} (meeting) \bluebold{of the provincial level}’) \textstyleExampleSource{[081010-001-Cv.0043]}
\z


\subsection{Beneficiary relations}
\label{Para_9.3.2}
The possessive marker \textitbf{punya} ‘\textsc{poss}’ is also used to signal beneficiary relations. Speakers employ this construction when they want to signal that the recipient is the beneficiary of a \isi{definite} theme, as discussed in §\ref{Para_11.1.3.3}. This is illustrated in (\ref{Example_9.42}) and (\ref{Example_9.43}). In the respective examples, the possessors \textitbf{mama} ‘mother’ and \textitbf{de} ‘\textsc{3sg}’ express the recipients/beneficiaries of the events expressed by the verbs \textitbf{simpang} ‘store’ and \textitbf{bli} ‘buy’, while the possessa \textitbf{makang} ‘food’ and \textitbf{alat{\Tilde}alat} ‘utensils’ denote the anticipated objects of possession or themes.


\ea
\label{Example_9.42}
\gll {mama,} {kitong} {suda} {simpang} {\bluebold{mama}} {\bluebold{punya}} {\bluebold{makang}}\\ %
 mother  \textsc{1pl}  already  store  mother  \textsc{poss}  food\\
\glt 
‘mother, we already put \bluebold{food for you} aside’ (Lit. ‘\bluebold{mama’s food}’) \textstyleExampleSource{[080924-002-Pr.0005]}
\z

\ea
\label{Example_9.43}
\gll {dong} {su} {bli} {\bluebold{de}} {\bluebold{punya}} {\bluebold{alat{\Tilde}alat}} {\bluebold{ini}}\\ %
 \textsc{3pl}  already  buy  \textsc{3sg}  \textsc{poss}  \textsc{rdp}{\Tilde}equipment  \textsc{d.prox}\\
\glt
‘they already bought \bluebold{these utensils for him}’ (Lit. ‘\bluebold{his utensils}’) \textstyleExampleSource{[080922-001a-CvPh.0558]}
\z


\subsection{Intensifying function of \textitbf{punya} ‘\textsc{poss}’}
\label{Para_9.3.3}
Another noncanonical function of possessive \textitbf{punya} ‘\textsc{poss}’ is that of an intensifier or stance that signals speaker attitudes or evaluations. The attested data suggests three different constructions in which Papuan Malay speakers use \textitbf{punya} ‘\textsc{poss}’ in such a way: constructions with (1) a nominal possessor and a \isi{quantifier} possessum (§\ref{Para_9.3.3.1}), (2) a nominal possessor and a verbal possessum (§\ref{Para_9.3.3.2}), and (3) a verbal possessor and a verbal possessum (§\ref{Para_9.3.3.3}).


\subsubsection[n{}-possr – punya – qt{}-possm constructions]{\textsc{n-possr} – \textitbf{punya} – \textsc{qt-possm} constructions}
\label{Para_9.3.3.1}
In the possessive constructions in (\ref{Example_9.44}) to (\ref{Example_9.47}), a nominal constituent takes the possessor slot while a \isi{quantifier} takes the possessum slot.



Attested in the corpus is only the one example in (\ref{Example_9.44}) in which the mid-range \isi{quantifier} \textitbf{banyak} ‘many’ takes the possessum slot. A second, elicited example is presented in (\ref{Example_9.45}). Possessive constructions with the mid-range \isi{quantifier} \textitbf{sedikit} ‘few’ are also possible, as illustrated with the elicited examples in (\ref{Example_9.46}) and (\ref{Example_9.47}). In these examples \textitbf{punya} ‘\textsc{poss}’ functions as an attitudinal intensifier, expressing speaker evaluations, such as feelings of annoyance in (\ref{Example_9.44}), of surprise in (\ref{Example_9.45}) and (\ref{Example_9.46}), or of alarm in (\ref{Example_9.47}).



\ea
\label{Example_9.44}
\gll {baru,} {mama,} {setang} {\bluebold{pu}} {\bluebold{banyak}} {di} {situ}\\ %
 and.then  mother  evil.spirit  \textsc{poss}  many  at  \textsc{l.med}\\
\glt 
‘and then, mother, (there) are \bluebold{really many }evil spirits over there’ (Lit. ‘\bluebold{many of}’) \textstyleExampleSource{[081025-006-Cv.0062]}
\z

\ea
\label{Example_9.45}
\gll {natal} {tu} {ana{\Tilde}ana} {dong} {maing} {kembang-api} {\bluebold{pu}} {\bluebold{banyak}}\\ %
 Christmas  \textsc{d.dist}  \textsc{rdp}{\Tilde}child  \textsc{3pl}  play  fire-cracker  \textsc{poss}  many\\
\glt 
‘(during) Christmas (time) the children play with \bluebold{really many} fire-crackers’ (Lit. ‘\bluebold{many of}’) \textstyleExampleSource{[Elicited BR111020.005]}
\z

\ea
\label{Example_9.46}
\gll {di} {gunung} {itu} {pohong} {\bluebold{pu}} {\bluebold{sedikit}}\\ %
 at  mountain  \textsc{d.dist}  tree  \textsc{poss}  few\\
\glt 
‘on that mountain, there are \bluebold{very few} trees’ (Lit. ‘\bluebold{few of}’) \textstyleExampleSource{[Elicited BR111020.006]}
\z

\ea
\label{Example_9.47}
\gll {tete} {de} {minum} {air} {\bluebold{pu}} {\bluebold{sedikit}}\\ %
 grandfather  \textsc{3sg}  drink  water  \textsc{poss}  few\\
\glt 
‘grandfather drinks \bluebold{very little} water’ (Lit. ‘\bluebold{few of}’) \textstyleExampleSource{[Elicited BR111020.007]}
\z



Possessive constructions with other quantifiers or with numerals taking the possessum slot are ungrammatical.


\subsubsection[n{}-possr – punya – v{}-possm constructions]{\textsc{n-possr} – \textitbf{punya} – \textsc{v-possm} constructions}
\label{Para_9.3.3.2}
In the possessive constructions in (\ref{Example_9.48}) to (\ref{Example_9.52}), a nominal constituent takes the possessor slot while a mono- or \isi{bivalent} \isi{verb} takes the possessum slot. In these constructions, speakers typically use short \textitbf{pu} ‘\textsc{poss}’ rather than long \textitbf{punya} ‘\textsc{poss}’; more investigation is needed however, to further explore these speaker preferences.



In the examples in (\ref{Example_9.48}) to (\ref{Example_9.52}), a \isi{monovalent} \isi{verb} takes the possessum slot. Again, the ligature functions as an attitudinal intensifier. In (\ref{Example_9.48}) \textitbf{pu} ‘\textsc{poss}’ adds emphasis to stative \textitbf{malas} ‘be listless’. In (\ref{Example_9.49}), \textitbf{pu} ‘\textsc{poss}’ precedes stative \textitbf{brat} ‘be heavy’, and thereby signals feelings of annoyance. Finally, in (\ref{Example_9.50}), the possessive marker precedes dynamic \textitbf{mendrita} ‘suffer’, thereby indicating negative feelings of disbelief.



\begin{styleExampleTitle}
Intensifying function of \textitbf{punya} ‘\textsc{poss}’: Preceding \isi{monovalent} verbs
\end{styleExampleTitle}

\ea
\label{Example_9.48}
\gll {dong} {{tida}} {{taw}} {umpang,} {smua} {tra} {taw} {toser,}\\ %
 \textsc{3pl}  {\textsc{neg}}  {know}  pass.ball  all  \textsc{neg}  know  pass.ball\\
\gll  {adu},  {sa}  \bluebold{pu}  {\bluebold{malas}}\\
 {oh.no!}  {\textsc{1sg}}  \textsc{poss}  {be.listless}\\
\glt 
[About playing volleyball:] ‘none of them knows (how) to pass a ball, none of them knows (how) to pass a ball, oh no!, I’m \bluebold{so very listless} (to play with them)’ (Lit. ‘\bluebold{the being listless of}’) \textstyleExampleSource{[081109-001-Cv.0127]}
\z

\ea
\label{Example_9.49}
\gll {damay,} {de} {\bluebold{pu}} {\bluebold{brat}}\\ %
 peace  \textsc{3sg}  \textsc{poss}  be.heavy\\
\glt 
‘my goodness!, he was \bluebold{so heavy}’ (Lit. ‘\bluebold{the being} \bluebold{heavy of}’) \textstyleExampleSource{[081025-009b-Cv.0041]}
\z

\ea
\label{Example_9.50}
\gll {adu,} {dong} {dua} {\bluebold{pu}} {\bluebold{mendrita}}\\ %
 oh.no!  \textsc{3pl}  two  \textsc{poss}  suffer\\
\glt 
‘oh no!, the two of them were \bluebold{suffering so much}’ (Lit. ‘\bluebold{the suffering of}’) \textstyleExampleSource{[081025-006-Cv.0059]}
\z


In (\ref{Example_9.51}) to (\ref{Example_9.52}), the possessum slot is taken by a \isi{bivalent} \isi{verb}. Again, the possessive marker has intensifying, asserting and/or evaluative function.



\begin{styleExampleTitle}
Intensifying function of \textitbf{punya} ‘\textsc{poss}’: Preceding \isi{bivalent} verbs
\end{styleExampleTitle}

\ea
\label{Example_9.51}
\gll {ka} {Sarles} {juga} {de} {\bluebold{pu}} {\bluebold{maing}} {\bluebold{pisow}}\\ %
 oSb  Sarles  also  \textsc{3sg}  \textsc{poss}  play  knife\\
\glt 
‘older brother Sarles also, he \bluebold{has a fast and smart way of playing}’ (Lit. ‘\bluebold{the knife playing of}’) \textstyleExampleSource{[081023-001-Cv.0009]}
\z

\ea
\label{Example_9.52}
\gll {baru} {nanti} {tong} {\bluebold{pu}} {\bluebold{lawang}} {deng} {siapa}\\ %
 and.then  very.soon  \textsc{1pl}  \textsc{poss}  oppose  with  who\\
\glt
‘and then later who will be \bluebold{our opponent}?’ (Lit. ‘\bluebold{the opposing of}’) \textstyleExampleSource{[081109-001-Cv.0136]}
\z


\subsubsection[v{}-possr – punya – v{}-possm constructions]{\textsc{v-possr} – \textitbf{punya} – \textsc{v-possm} constructions}
\label{Para_9.3.3.3}
In noncanonical possessive constructions, both the possessor and the possessum slot can be taken by verbs, as illustrated in (\ref{Example_9.53}) to (\ref{Example_9.56}). More specifically, a dynamic \isi{verb} takes the possessor slot, while a stative \isi{verb} takes the possessum slot. The only example attested in the corpus is (\ref{Example_9.53}), while the examples in (\ref{Example_9.54}) to (\ref{Example_9.56}) are elicited.



With its \isi{intensifying function}, \textitbf{punya} ‘\textsc{poss}’ signals an emphatic reading of both the verbal possessor and the verbal possessum, as illustrated in (\ref{Example_9.53}): \textitbf{mandi punya} ‘really bathing’ and \textitbf{punya jaw} ‘very far away (of)’.


\ea
\label{Example_9.53}
\gll {dong} {mandi} {di} {kali} {Biri,} {mm-mm,} {\bluebold{mandi}} {\bluebold{punya}} {\bluebold{jaw}} {itu}\\ %
 \textsc{3pl}  bathe  at  river  Biri  mhm  bathe  \textsc{poss}  be.far  \textsc{d.dist}\\
\glt 
[About a run-away boy:] ‘they were bathing in the Biri river, mhm, (they were) \bluebold{really bathing very far away}’ (Lit. ‘\bluebold{the being far away of the bathing}’) \textstyleExampleSource{[081025-008-Cv.0032-0033]}
\z

\ea
\label{Example_9.54}
\gll {de} {\bluebold{kerja}} {\bluebold{punya}} {\bluebold{cepat}}\\ %
 \textsc{3sg}  work  \textsc{poss}  be.fast\\
\glt 
‘he \bluebold{really worked very fast}’ (Lit. ‘\bluebold{the being fast of the working}’) \textstyleExampleSource{[Elicited BR111020.022]}
\z

\ea
\label{Example_9.55}
\gll  mama  de  \bluebold{masak}  \bluebold{punya}  \bluebold{enak}\\
 mother  \textsc{3sg}  cook  \textsc{poss}  be.pleasant\\
\glt 
‘mother \bluebold{really cooks very tastily}’ (Lit. ‘\bluebold{the being tasty of the cooking}’) \textstyleExampleSource{[Elicited BR111020.023]}
\z

\ea
\label{Example_9.56}
\gll {Marice} {deng} {Matius} {dong} {dua} {\bluebold{bicara}} {\bluebold{punya}} {\bluebold{kras}}\\ %
 Marice  with  Matius  \textsc{3pl}  two  speak  \textsc{poss}  be.harsh\\
\glt
‘the two of them Marice and Matius \bluebold{really spoke very loudly} (with each other)’ (Lit. ‘\bluebold{the being loud of the speaking}’) \textstyleExampleSource{[Elicited BR111020.024]}
\z


\subsection{\textitbf{punya} ‘\textsc{poss}’ in reflexive expressions}
\label{Para_9.3.4}
The possessive marker \textitbf{punya} ‘\textsc{poss}’ is also used to create reflexive expressions. Generally speaking, reflexives designate constructions “where subject and object refer to the same entity, explicitly [{\ldots}] or implicitly” \citep[5164]{Asher.1994}. Typically, explicit reflexive expressions are formed with a reflexive \isi{pronoun} “which refers to the same person or thing as the subject of the \isi{verb}” \citep[5165]{Asher.1994}. As Papuan Malay does not have reflexive pronouns, an alternative strategy is used. Reflexive relations are expressed with an \isi{adnominal possessive construction} where a personal \isi{pronoun} in the possessor slot and the reflexive \isi{noun} \textitbf{diri} ‘self’ in the possessum slot express the reflexive relationship between both, as illustrated with \textitbf{sa pu diri} ‘myself’ in (\ref{Example_9.57}) and \textitbf{kita punya diri} ‘ourselves’ in (\ref{Example_9.58}).


\ea
\label{Example_9.57}
\gll {{bukang}} {{sa}} {{rasa}} {{bahwa}} {{sa}} {{ini}} {sa} {banggakang}\\ %
 {\textsc{neg}}  {\textsc{1sg}}  {feel}  {that}  {\textsc{1sg}}  {\textsc{d.prox}}  \textsc{1sg}  praise\\
\gll \bluebold{sa}  {\bluebold{pu}}  {\bluebold{diri}}  {tapi}  {itu}  {yang}  {terjadi}\\
 \textsc{1sg}  {\textsc{poss}}  {self}  {but}  {\textsc{d.dist}}  {\textsc{rel}}  {happen}\\
\glt 
‘it’s not that I feel that I (\textsc{emph}), (that) I praise \bluebold{myself}, but that’s what happened’ (Lit. ‘\bluebold{the self of me}’) \textstyleExampleSource{[081110-008-CvNP.0152]}
\z

\ea
\label{Example_9.58}
\gll {kita} {rencana,} {manusia} {yang} {mengatur} {\bluebold{kita}} {\bluebold{punya}} {\bluebold{diri}}\\ %
 \textsc{1pl}  plan  human.being  \textsc{rel}  arrange  \textsc{1pl}  \textsc{poss}  self\\
\glt
‘we make plans, (it’s us) human beings who manage \bluebold{our own lives}’ (Lit. ‘\bluebold{the self of us}’) \textstyleExampleSource{[080918-001-CvNP.0032]}
\z


\section{Summary and discussion}
\label{Para_9.4}
In Papuan Malay, adnominal possessive constructions consist of two \isi{noun}   phrases linked with the possessive marker \textitbf{punya} ‘\textsc{poss}’, such that ``\textsc{possessor} \textitbf{punya} \textsc{possessum}''. In addition to signaling adnominal possessive relations between two \isi{noun} phrases, \textitbf{punya} ‘\textsc{poss}’ has a number of derived, noncanonical functions, namely as (1) an emphatic marker of \isi{locational} relations or relations of association, (2) a marker of beneficiary relations, (3) an attitudinal intensifier or stance, and (4) a ligature in reflexive constructions.



Such noncanonical functions of the possessive ligature have also been noted in other \ili{eastern Malay varieties}. Examples are its functions as a marker of beneficiary relations in \ili{Ambon Malay} \citep[164]{vanMinde.1997}, as a marker of \isi{locational} or temporal relations in \ili{Ternate Malay} \citep[52–53, 96–97]{Litamahuputty.1994}, and as an attitudinal intensifier in \ili{Manado Malay} \citep[45]{Stoel.2005}.



Two explanations have been suggested for the extended uses of the possessive marker in Malay speech varieties.



One is to propose a substratum influence of \ili{Chinese} languages. Some of the noncanonical functions of the possessive marker have long been noted for   \ili{Bazaar Malay} and have been linked to the substratum influence of \ili{Chinese}   speech varieties, namely the function of \textitbf{punya} ‘\textsc{poss}’ to link a \isi{locative} or temporal modifier or a modifying adjective in the possessor slot preceding the ligature with its head in the possessum slot (See \citealt[6–7]{Shellabear.1904}; \citealt[115]{Winstedt.1913}; \citeyear*[41]{Winstedt.1938}; \citealt{Lim.1988b}; \citealt{Bao.2009}). \citet[1, 8ff]{Yap.2007}  argues that under the influence of southern \ili{Chinese} speech varieties, the \ili{colloquial Malay} possessive marker developed into an “attitudinal intensifier” or “stance” that transforms statements into evaluative “assertions that are often laced with strong feelings, including feelings of awe, [{\ldots}] or feelings of incredulity or even annoyance”. For the different synchronic functions of \textitbf{(em)punya} in classical and \ili{colloquial Malay}, \citet[159]{Yap.2004} propose the following development or \isi{grammaticalization} path: “lexical \isi{verb} {\textgreater} genitive {\textgreater} pronominal {\textgreater} stance development”.



A second explanation proposes a \isi{grammaticalization} process of the possessive marker without any substratum influence from \ili{Chinese} varieties. \citet[2]{Gil.1999} argues that the influence of \ili{Chinese} languages does not “account for the presence of the \textitbf{punya}\textit{ }construction” in Malay varieties which have “little obvious contact with \ili{Chinese} languages”, such as \ili{Riau Indonesian} or Papuan Malay; neither does this influence “account for the choice of the specific marker \textitbf{punya}”. Instead, Gil submits that the interpretation of the \textitbf{punya} construction underwent a semantic change from predicative possessive to adnominal possessive to noncanonical possessive, such that: “thing associated with X’s having” {\textgreater} “thing associated with X” {\textgreater} “property associated with X” {(1999: 6, 8)}.



Possessive constructions with \textitbf{punya} ‘\textsc{poss}’ have a number of different realizations. The possessive marker can be represented with unreduced \textitbf{punya}, reduced \textitbf{pu}, clitic \textitbf{=p}, or a zero morpheme. There are no syntactic or semantic restrictions on the uses of the long and reduced possessive marker forms. By contrast, omission of \textitbf{punya} only occurs when the possessive construction expresses inalienable possession of body parts or kinship relations. The possessor and the possessum can be expressed with different kinds of syntactic constituents, such as lexical nouns, \isi{noun} phrases, or demonstratives. In addition, personal pronouns can also express the possessor. In noncanonical possessive constructions, verbs can also take the possessor and/or possessum slots. Further, mid-range quantifiers, temporal adverbs, and prepositional phrases can take the possessum slot. In canonical possessive constructions, the possessum can also be omitted. Semantically, the possessor and the possessum can denote human, nonhuman animate, or inanimate referents.

