\chapter[Conjunctions and constituent combining]{Conjunctions and constituent combining}
\label{Para_14}
\section{Introduction}
\label{Para_14.1}

This chapter describes how Papuan Malay combines constituents such as clauses or phrases by overt marking with conjunctions. The Papuan Malay conjunctions can be divided into two major groups, those combining same-type constituents, such as clauses with clauses, and those linking different-type constituents, such as verbs with clauses. In combining constituents, the conjunctions belong to neither of the conjuncts they combine in semantic terms. They do, however, form intonation units with the constituents they mark. Most conjunctions occur at the left periphery of the clause. Typically, an intonational break separates the \isi{conjunction} from a preceding constituent. A second strategy to combine constituents is \isi{juxtaposition} which is mentioned only briefly.

Papuan Malay has 21 conjunctions which link same-type constituents and two which combine different-type constituents. Most of the conjunctions conjoining same-type constituents link clauses with clauses. Traditionally, clause-linking conjunctions are divided into coordinating and subordinating ones: “coordinating conjunctions are those that assign equal rank to the conjoined elements” whereas “subordinating conjunctions are those that assign unequal rank to the conjoined elements, marking one of them as subordinate to the other” \citep[45]{Schachter.2007}. Modifying this terminology by employing the more general term “dependency” rather than “subordination”, \citet[46]{Haspelmath.2007c} defines the distinction between coordination and dependency as follows:
%%

\begin{quote}
In a coordination structure of the type \textstyleChItalic{A(-link-)B}, \textstyleChItalic{A}\textit{ }and \textstyleChItalic{B}\textit{ }are structurally symmetrical in some sense, whereas in a dependency structure of the type \textstyleChItalic{X(-link-)Y}, \textstyleChItalic{X}\textit{ }and \textstyleChItalic{Y}\textit{ }are not symmetrical, but either \textit{X }or \textit{Y }is the head and the other element is a dependent.
\end{quote}

According to \citet[46]{Haspelmath.2007c}, this distinction between coordination and dependency in terms of symmetry “is often thought of as a difference in the syntactic/structural relations of the elements”. As \citet[46]{Haspelmath.2007c} points out, however, “it is sometimes not evident whether a construction exhibits a coordination relation or a dependency relation”; this applies, for instance, to “languages that lack agreement and case-marking”.

The lack of a clear opposition between coordination and dependency in terms of structural relations also applies to clause combining in Papuan Malay: clauses marked with a \isi{conjunction} are not distinct from unmarked clauses in terms of their morphosyntax and word order. This is shown in (\ref{Example_14.1}) to (\ref{Example_14.3}) with purpose-marking \textitbf{supaya} ‘so that’. Omitting the \isi{conjunction} from the two purpose clauses in (\ref{Example_14.1}) leaves two grammatically complete and correct clauses: \textitbf{saya harus kas makang dia} ‘I have to give him/her food’ and \textitbf{dia kenal saya lebi} ‘he/she can know me better’.
%


\begin{styleExampleTitle}
Purpose marking \textstyleChItalBold{supaya} ‘so that’ linking two clauses
\end{styleExampleTitle}
\ea
\label{Example_14.1}
\gll   saya harus kas makang dia, \textstyleChBlueBold{supaya} dia kenal saya lebi   dekat, \textstyleChBlueBold{supaya} de bisa taw saya punya nama\\
  \textstyleChSmallCaps{1sg} have.to give eat \textstyleChSmallCaps{3sg} so.that \textstyleChSmallCaps{3sg} know \textstyleChSmallCaps{1sg} more   near so.that \textstyleChSmallCaps{3sg} be.able know \textstyleChSmallCaps{1sg} \textstyleChSmallCaps{poss} name\\
\glt ‘I have to give him/her food \textstyleChBlueBold{so that} he/she can know me better, \textstyleChBlueBold{so that} he/she can know my name’ \textstyleExampleSource{[080919-004-NP.0079]}
\z


When a \isi{conjunction} is missing an argument, the result is still a grammatically complete and correct clause. In the purpose clause in (\ref{Example_14.2}), for instance, the subject \textitbf{obat} ‘medicine’ is elided. This \isi{elision}, however, does not signify a grammaticalized gap that signals the dependent status of the purpose clause marked with \textitbf{supaya} ‘so that’. Instead, the \isi{elision} is due to the fact that speakers often omit arguments and other constituents if these can be inferred. In (\ref{Example_14.2}) the elided subject \textitbf{obat} ‘medicine’ is understood from the context.

\begin{styleExampleTitle}
Purpose clause with elided subject argument
\end{styleExampleTitle}
\ea
\label{Example_14.2}
\gll ibu itu de mo kasi obat, tapi ko harus  priksa dara, \textstyleChBlueBold{supaya} Ø harus cocok\\    
woman \textstyleChSmallCaps{d.dist} \textstyleChSmallCaps{3sg} want give medicine but \textstyleChSmallCaps{2sg} have.to  check blood so.that {}  have.to be.suitable\\
\glt ‘that lady, she wants to give (you) medicine, but you have to (get your) blood checked \textstyleChBlueBold{so that} (the medicine) fits’ \textstyleExampleSource{[080917-007-CvHt.0003]}
\z

\noindent In Papuan Malay, \isi{elision} of core arguments is not limited to clauses marked with conjunctions. It is a generalized phenomenon, as demonstrated with the reported \isi{direct speech} in (\ref{Example_14.3}) (see also §\ref{Para_11.1}). The original utterance is given in (\ref{Example_14.3a}), while in (\ref{Example_14.3b}) the elided constituents are given in brackets, such as \isi{purposive} \textitbf{supaya} ‘so that’\footnote{Alternatively, the \isi{conjunction} \textitbf{sampe} ‘until, with the result that’ could fill this slot.} or the subject of the purpose clause, \textitbf{kaki} ‘foot, leg’.


\begin{styleExampleTitle}
Elision as a generalized phenomenon
\end{styleExampleTitle}
\ea
\label{Example_14.3}
\ea
\label{Example_14.3a}
\gll   {\ldots} malam Kapolsek bilang, kalo dapat  tembak kaki pata\\ 
    {} night head.of.district.police say if get  shoot foot break\\
\vspace{5pt}
\ex
\label{Example_14.3b}
\gll
  {\ldots} malam Kapolsek bilang, kalo [kam] dapat   [dia,] tembak {[de pu]} kaki [\textstyleChBlueBold{supaya}] [kaki] pata\\
   {} night head.of.district.police say if [\textstyleChSmallCaps{2pl}] get   [\textstyleChSmallCaps{3sg}] shoot {[\textstyleChSmallCaps{3sg} \textstyleChSmallCaps{poss}]} foot [so.that] [foot] break\\
\glt [Reply to the question about who the police were looking for:] ‘[(they’re looking for Martin {\ldots},] (last) night the head of the district police said, ``if (you) get (him), shoot (his) leg (\textstyleChBlueBold{so that} it) breaks''' \textstyleExampleSource{[081011-009-Cv.0048/0050]}\\
\z
\z


\noindent This data shows that, in terms of structural relations, the opposition between coordination and dependency does not apply to purpose-marking \textitbf{supaya} ‘so that’. Neither does the distinction apply to the other clause-combining conjunctions.



Given that cross-linguistically this lack of a clear-cut opposition between coordination and dependency in terms of structural relations is not uncommon, \citet[46]{Haspelmath.2007c} suggests defining ``both coordination and dependency in semantic terms''. He also notes, however, that even the distinction on semantic grounds “is often difficult to apply” (\citeyear*[47]{Haspelmath.2007c}; see also \citealt[1–50]{Cristofaro.2005}; \citealt{Dixon.2009}).
%


This difficulty also applies to clause combining in Papuan Malay. Therefore, in discussing clause combining in Papuan Malay at this point in the current research, no attempt is being made to distinguish between coordination and dependency on semantic grounds. Instead, this chapter describes the following aspects: (1) the meaning which the different Papuan Malay conjunctions convey, (2) the position which a given \isi{conjunction} takes within its clause, and (3) the position which the clause marked with a \isi{conjunction} takes vis-à-vis the clause it is conjoined with. For lack of a better term, the clause that is not marked with a \isi{conjunction} is labeled as the ``unmarked clause'' throughout the remainder of this chapter. This label is used as a working term only for practical purposes.
%


In addition to the 21 conjunctions combining same-type constituents, Papuan Malay also has two conjunctions which link different-type constituents, namely \isi{complementizer} \textitbf{bahwa} ‘so that’ and \isi{relativizer} \textitbf{yang} ‘\textsc{rel}’. Both are subordinating conjunctions, in that they “serve to integrate a {\ldots} clause into some larger construction”, adopting  \citegen[45]{Schachter.2007} definition. Complementizer \textitbf{bahwa} ‘that’ marks a clause as an argument of the \isi{verb} as illustrated in (\ref{Example_14.4}), while \isi{relativizer} \textitbf{yang} ‘\textsc{rel}’ integrates a relative clause within a \isi{noun} phrase as demonstrated in (\ref{Example_14.5}).
%

\begin{styleExampleTitle}
Conjunctions combining different-type constituents
\end{styleExampleTitle}
\ea
\label{Example_14.4}
\gll  sa cuma taw \textstyleChBlueBold{bahwa} de ada di sini\\  
 \textstyleChSmallCaps{1sg} just know that \textstyleChSmallCaps{3sg} exist at \textstyleChSmallCaps{l.prox}\\
\glt ‘I just know \textstyleChBlueBold{that} he was here’ \textstyleExampleSource{[080922-010a-CvNF.0180]}
\z

\ea
\label{Example_14.5}
\gll  baru Iskia dia pegang sa punya lutut \textstyleChBlueBold{yang} tida baik\\  
 and.then Iskia \textstyleChSmallCaps{3sg} hold \textstyleChSmallCaps{1sg} \textstyleChSmallCaps{poss} knee \textstyleChSmallCaps{rel} \textstyleChSmallCaps{neg} be.good\\
\glt ‘and then Iskia held my knee \textstyleChBlueBold{that} is not well’ \textstyleExampleSource{[080916-001-CvNP.0003]}
\z

\newpage
Conjunctions linking same-type constituents are described in §\ref{Para_14.2} and those linking different-type constituents are discussed in §\ref{Para_14.3}. Unless mentioned otherwise, the clausal conjunctions combine clauses with same-subject coreference as well as those with a switch in reference. Juxtaposition is briefly mentioned in §\ref{Para_14.4}. The main points of this chapter are summarized in §\ref{Para_14.5}.

\section{Conjunctions combining same-type constituents}
\label{Para_14.2}
This section discusses conjunctions which combine same-type constituents. In terms of the semantic relations which they signal, the conjunctions fall into six groups, that is conjunctions marking addition (§\ref{Para_14.2.1}), alternative (§\ref{Para_14.2.2}), time and/or conditions (§\ref{Para_14.2.3}), consequence (§\ref{Para_14.2.4}), contrast (§\ref{Para_14.2.5}), and \isi{similarity} (§\ref{Para_14.2.6}).


\subsection{Addition}
\label{Para_14.2.1}
Addition-marking conjunctions combine constituents denoting events, states, or entities which are “closely linked and {\ldots} valid simultaneously”, employing \citegen[20]{Rudolph.1996}  definition.



Papuan Malay employs three addition-marking conjunctions. Most often addition is encoded with the \isi{comitative} \isi{preposition} \textitbf{dengang} ‘with’ (670 tokens); as a \isi{conjunction}, it typically conjoins \isi{noun} phrases, as discussed in §\ref{Para_14.2.1.1}. Much less often, Papuan Malay employs \isi{conjunctive} \textitbf{dang} ‘and’ (24 tokens); it typically joins two clauses, as described in §\ref{Para_14.2.1.2}. Even less often, addition is encoded with the \isi{goal-oriented} \isi{preposition} \textitbf{sama} ‘to’; as a \isi{conjunction} it links \isi{noun} phrases with human referents (8 tokens), as shown in §\ref{Para_14.2.1.3}. (For details on \isi{variation} in \isi{word class membership} see §\ref{Para_5.14}.)
%

\subsubsection[Comitative dengang ‘with’]{Comitative \textitbf{dengang} ‘with’}
\label{Para_14.2.1.1}
The \isi{comitative} \isi{preposition} \textitbf{dengang} ‘with’, with its short form \textitbf{deng}, typically conjoins \isi{noun} phrases (654 tokens). The conjoined referents can be animate as in (\ref{Example_14.6}), or inanimate as in (\ref{Example_14.7}). The fact that Papuan Malay employs the same marker for “\isi{noun} phrase \isi{conjunction} and \isi{comitative} phrases” suggests that, in terms of \citegen[1]{Stassen.2013b} typology, Papuan Malay is a  ``\textsc{with}{}-language''. Occasionally, \textitbf{deng(ang)} ‘with’ also links \isi{verb} phrases (16 tokens) as in (\ref{Example_14.8}). The linking of clauses with \isi{comitative} \textitbf{dengang} ‘with’ is unattested in the corpus. (Besides, \isi{comitative} \textitbf{dengang} ‘with’ is also used to encode inclusory \isi{conjunction} constructions, as discussed in §\ref{Para_6.1.4}; for a detailed discussion of \isi{preposition} \textitbf{dengang} ‘with’, see §\ref{Para_10.2.1}.)
%


\ea
\label{Example_14.6}
\gll {bapa} {\bluebold{dengang}} {bapa-tua} {pi} {biking} {kebung} {\ldots}\\ %
father  with  older.uncle  go  make  garden \\
\glt ‘father \bluebold{and} uncle went to work (together) in the garden {\ldots}’ \textstyleExampleSource{[080922-001a-CvPh.0629]}
\z

\ea\label{Example_14.7}
\gll {\ldots} {apa} {biologi} {\bluebold{dengang}} {apa} {astronomi} {\bluebold{dengang}} {bahasa} {Inggris}\\ %
{} what biology with  what  astronomy  with  language  England\\
\glt [About a school competition] ‘[later they’ll participate in the Olympiad contest in,] what-is-it, biology \bluebold{and}, what-is-it, astronomy \bluebold{and} English’ \textstyleExampleSource{[081115-001a-Cv.0111-0113]}
\z

\ea\label{Example_14.8}
\gll {nene} {jam} {dua} {malam} {datang} {\bluebold{deng}} {menangis}\\ %
 grandmother  hour  two  night  come  with  cry\\
\glt ‘at two o’clock in the morning grandmother came crying’ (Lit. ‘come \bluebold{with} cry’) \textstyleExampleSource{[081014-008-CvNP.0001]}
\z

\subsubsection[Conjunctive dang ‘and’]{Conjunctive \textitbf{dang} ‘and’}
\label{Para_14.2.1.2}
The \isi{conjunction} \textitbf{dang} ‘and’ typically links two clauses (168 tokens), as in (\ref{Example_14.9}). Less often, it links \isi{noun} phrases (24 tokens) as in (\ref{Example_14.10}) and (\ref{Example_14.11}), or \isi{verb} phrases (10 tokens) as in (\ref{Example_14.12}). Usually, the \isi{noun} phrases have human referents as in (\ref{Example_14.10}); coordination of inanimate referents, as in (\ref{Example_14.11}), is rare.
%

\ea
\label{Example_14.9}
\gll     de pegang de punya prahu, \textstyleChBlueBold{dang} de dayung, \textstyleChBlueBold{dang} de bilang, {\ldots}\\  
 \textstyleChSmallCaps{3sg} hold \textstyleChSmallCaps{3sg} \textstyleChSmallCaps{poss} boat and \textstyleChSmallCaps{3sg } paddle and \textstyleChSmallCaps{3sg} say\\
\glt ‘he took his boat \textstyleChBlueBold{and} he paddled \textstyleChBlueBold{and} he said, {\ldots}’ \textstyleExampleSource{[080917-008-NP.0018]}
\z

\ea\label{Example_14.10}
\gll {sa} {kas} {taw} {mama} {\bluebold{dang}} {mama-ade,} {nanti} {kam} {\ldots}\\ %
 \textsc{1sg}  give  know  mother  and  aunt  very.soon  \textsc{2pl}  \\
\glt ‘I let mother \bluebold{and} aunt know, ``later you {\ldots}''' \textstyleExampleSource{[080919-007-CvNP.0001]}
\z

\ea\label{Example_14.11}
\gll {de} {suda} {taw} {ruma} {\bluebold{dang}} {kampung}\\ %
 \textsc{3sg}  already  know  house  and  village\\
\glt ‘he already knew the house \bluebold{and} the village’ \textstyleExampleSource{[080923-006-CvNP.0002]}
\z


\ea\label{Example_14.12}
\gll {pagi} {helikopter} {turung} {\bluebold{dang}} {kembali} {ke} {Anggruk}\\ %
 morning  helicopter  descend  and  return  to  Anggruk\\
\glt ‘in the morning the helicopter came down \bluebold{and} returned to Anggruk’ \textstyleExampleSource{[081011-022-Cv.0228]}
\z

\subsubsection[Goal{}-oriented sama ‘to’]{Goal-oriented \textitbf{sama} ‘to’}
\label{Para_14.2.1.3}
The \isi{goal-oriented} \isi{preposition} \textitbf{sama} ‘to’ occasionally links \isi{noun} phrases with human referents (8 tokens), as in (\ref{Example_14.13}). The coordination of clauses or \isi{verb} phrases with \textitbf{sama} ‘to’ is unattested in the corpus. Goal-oriented \textitbf{sama} ‘to’ has trial \isi{word class membership}. That is, besides being used as a \isi{preposition} and an addition-marking \isi{conjunction}, it is also used as the stative \isi{verb} \textitbf{sama} ‘be same’ (see §\ref{Para_5.14}; see also §\ref{Para_10.2.2} for a detailed discussion of \isi{preposition} \textitbf{sama} ‘to’ and how it is distinct from \isi{comitative} \textitbf{dengang} ‘with’).\footnote{As mentioned in §\ref{Para_10.2.2}, the goal \isi{preposition} \textitbf{sama} ‘to’ is rather general in its meaning. Typically it translates with ‘to’ but depending on its context it also translates with ‘of, from, with’. For more information regarding the etymology of \textitbf{sama} ‘to’, see Footnote \ref{Footnote_10.225} in §\ref{Para_10.2.2} (p. \pageref{Footnote_10.225}).}
%

\ea
\label{Example_14.13}

\gll       {\ldots}  Aris  \textstyleChBlueBold{sama} Siduas\textsubscript{i}  deng  de\textsubscript{i}  pu  maytua,  \textstyleChBlueBold{sama} dep\textsubscript{i},   de\textsubscript{i}  punya  maytua\\
{ } Aris  to  Siduas  with  \textstyleChSmallCaps{3sg} \textstyleChSmallCaps{poss} wife  to  \textstyleChSmallCaps{3sg:poss}   \textstyleChSmallCaps{3sg}  \textstyleChSmallCaps{poss} wife\\
\glt ‘[all (of you will) be taken (on board {\ldots},)] Aris \textstyleChBlueBold{and} Siduas\textsubscript{i} and his wife\textsubscript{i}, \textstyleChBlueBold{and} his\textsubscript{i}, his\textsubscript{i} wife’ \textstyleExampleSource{[080922-001a-CvPh.0493/0497]}\footnote{The subscript letters indicate which personal pronouns have which referents.}
\z


\subsection{Alternative}
\label{Para_14.2.2}
In Papuan Malay, two conjunctions mark alternative, namely \isi{disjunctive} \textitbf{ato} ‘or’ (§\ref{Para_14.2.2.1}) and \isi{disjunctive} \textitbf{ka} ‘or’ (§\ref{Para_14.2.2.2}).


\subsubsection[Disjunctive ato ‘or’]{Disjunctive \textitbf{ato} ‘or’}
\label{Para_14.2.2.1}
Generally speaking, the notion of disjunction is defined as “a logical relationship between propositions” in the sense that “[i]f the logical disjunction of two propositions is true, then one or both of the component propositions can be true” \citep[305]{Payne.1997}.

In Papuan Malay, disjunction is marked with \textitbf{ato} ‘or’ which always occurs at the left periphery of the constituents it combines. Most often, \isi{disjunctive} \textitbf{ato} ‘or’ joins clauses, as in (\ref{Example_14.14}). Also quite often, \textitbf{ato} ‘or’ links \isi{noun} phrases as in (\ref{Example_14.15}). Only rarely \textitbf{ato} ‘or’ links prepositional phrases as in (\ref{Example_14.16}), or \isi{verb} phrases as in (\ref{Example_14.17}).


\ea
\label{Example_14.14}
\gll  kalo  saya  susa,  \textstyleChBlueBold{ato } saya  biking  acara,  nanti   dia  bantu  saya \\  
if  \textstyleChSmallCaps{1sg } be.difficult  or  \textstyleChSmallCaps{1sg } make  ceremony  very.soon   \textstyleChSmallCaps{3sg } help  \textstyleChSmallCaps{1sg}\\
\glt ‘if I have difficulties \textstyleChBlueBold{or} I make a festivity, then he’ll help me’ \textstyleExampleSource{[080919-004-NP.0065]}
\z

\ea
\label{Example_14.15}
\gll  kalo  tong  pu  uang  satu  juta,  \textstyleChBlueBold{ato}  satu  juta  lima   ratus,  tong  bisa  bakar  natal\\  
if  \textstyleChSmallCaps{1pl}  \textstyleChSmallCaps{poss}  money  one  million  or  one  million  five   hundred  \textstyleChSmallCaps{1pl}  be.able  burn  Christmas\\
\glt ‘if we had one million \textstyleChBlueBold{or} one million five hundred (thousand rupiah), we could have a Christmas party’ (Lit. ‘burn (the) Christmas (fire)’) \textstyleExampleSource{[081006-017-Cv.0016]}
\z

\ea
\label{Example_14.16}
\gll       jadi  kalo  dia,  suku  dari  situ,  dari  Masep  suda   bunu  orang  di,  a,  Karfasia,  \textstyleChBlueBold{ato}  di  Waim,  na  {\ldots}\\  
so  if  \textstyleChSmallCaps{3sg}  ethnic.group  from  \textstyleChSmallCaps{l.med}  from  Masep  already   kill  person  at  umh  Karfasia  or  at  Waim  well \\
\glt ‘so if it, the ethnic group from there, from Masep has already killed someone at, umh, Karfasia \textstyleChBlueBold{or} at Waim, well {\ldots}’ \textstyleExampleSource{[081006-027-CvEx.0002]}
\z

\ea
\label{Example_14.17}
\gll       dong  bilang,  a,  tunggu  minum  dulu,  \textstyleChBlueBold{ato}  makang  dulu\\  
\textstyleChSmallCaps{3pl}  say  ah!  wait  drink  first  or  eat  first\\
\glt ‘they said, ``ah, wait, please drink \textstyleChBlueBold{or} eat''' (Lit. ‘drink first \textstyleChBlueBold{or} eat first’) \textstyleExampleSource{[080925-003-Cv.0111]}\\
\z


\subsubsection[Disjunctive ka ‘or’]{Disjunctive \textitbf{ka} ‘or’}
\label{Para_14.2.2.2}
Disjunctive \textitbf{ka} ‘or’ signals series or sequences of alternatives. Occurring at the right periphery of a constituent, it indicates that a list of alternatives is not exhaustive. That is, a few possible options are overtly mentioned, while others are implied. To make the notion of ``non-exhaustive list of alternatives'' explicit, the \isi{conjunction} marks an \isi{interrogative} as the final enumerated constituent. Typically, \isi{disjunctive} \textitbf{ka} ‘or’ links \isi{noun} phrases, as in (\ref{Example_14.18}) and (\ref{Example_14.19}). In (\ref{Example_14.18}), the notion of a ``non-exhaustive list'' is implied, while in (\ref{Example_14.19}) it is overtly marked with \textitbf{apa ka} ‘or something else’ (literally ‘what or’). Less often, \textitbf{ka} ‘or’ combines prepositional phrases as in (\ref{Example_14.20}), or clauses as in (\ref{Example_14.21}); the linking of verbs with \textitbf{ka} ‘or’ is unattested in the corpus. Another function of \textitbf{ka} ‘or’, not discussed here, is to mark \isi{interrogative} clauses (see §\ref{Para_13.2.3}).
%

\ea
\label{Example_14.18}
\gll {\ldots} {nanti} {banjir} {\bluebold{ka},} {hujang} {\bluebold{ka},} {guntur} {\bluebold{ka}}\\ %
   {} very.soon  flooding  or  rain  or  thunder  or\\
\glt ‘[it’s not allowed to kill the snake otherwise] later (there’ll be) flooding, \bluebold{or} rain, \bluebold{or} thunder (\bluebold{or something else})’ \textstyleExampleSource{[081006-022-CvEx.0004]}
\z

\ea
\label{Example_14.19}
\gll       sa  deng  kaka  Petrus  pikir,  mungking  klapa  \textstyleChBlueBold{ka},  \textstyleChBlueBold{apa}  \textstyleChBlueBold{ka}   yang  ada  di  depang\\  
\textstyleChSmallCaps{1sg}  with  oSb  Petrus  think  maybe  coconut  or  what  or   \textstyleChSmallCaps{rel}  exist  at  front\\
\glt [About a motorbike trip:] ‘I and older brother Petrus thought, ``maybe it is a coconut \textstyleChBlueBold{or something else} that is in front (of us)''' \textstyleExampleSource{[081023-004-Cv.0002]}
\z

\ea\label{Example_14.20}
\gll {ko} {lapor} {di} {umum} {\bluebold{ka},} {di} {keuangang} {\bluebold{ka}}\\ %
 \textsc{2sg}  report  at  general  or  at  finance.affairs  or\\
\glt [About a government office:] ‘you (should) report to the general (office), \bluebold{or} the finance (office) (\bluebold{or some other} office)’ \textstyleExampleSource{[081005-001-Cv.0011]}
\z

\ea
\label{Example_14.21}
\gll       {\ldots}  waktu  ko  ada  potong  babi  \textstyleChBlueBold{ka},  potong  ikang  \textstyleChBlueBold{ka},  ato   dapat  ikang  ka  kuskus  ka,  waktu  lewat  kasi  saja\\  
{} time  \textstyleChSmallCaps{2sg}  exist  cut  pig  or  cut  fish  or  or   get  fish  or  cuscus  or  time  pass.by  give  just\\
\glt ‘[when (your) friends and relatives,] when you are carving a pig \textstyleChBlueBold{or} carving fish (\textstyleChBlueBold{or} carving \textstyleChBlueBold{something else}), or (when you) get a fish or cuscus (or something else), when (they) walk by, just share (it with them)’ \textstyleExampleSource{[080919-004-NP.0060]}
\z

\subsection{Time and/or condition}
\label{Para_14.2.3}
Papuan Malay conjunctions marking temporal relations indicate \isi{relative time}; that is, the temporal reference point is determined by the context. Providing a reference point for the events or states depicted in the unmarked clause, time-marking conjunctions signal sequence relations, \isi{anteriority}, or posteriority. Con\-dition-mark\-ing conjunctions introduce clauses which expresses conditions, while the unmarked clauses describe events or states which could come about once the conditions have been met.



In many languages, there is no distinction between \isi{conditional} ``if'' and temporal ``when'' clauses, as \citet[257]{Thompson.2007} point out. This also applies to Papuan Malay. Therefore, both types of linkings are discussed here.
%


This section describes five conjunctions: \isi{sequential} \textitbf{trus} ‘next’ (§\ref{Para_14.2.3.1}) and \textitbf{baru} ‘and then’ (§\ref{Para_14.2.3.2}), anteriority-marking \textitbf{sampe} ‘until’ (§\ref{Para_14.2.3.3}) and \textitbf{seblum} ‘before’ (§\ref{Para_14.2.3.4}), and \isi{posteriority-marking}/\isi{conditional} \textitbf{kalo} ‘when, if’ (§\ref{Para_14.2.3.5}).\footnote{Papuan Malay does not have a \isi{conjunction} that marks temporal simultaneity between two clauses. Instead speakers use the common \isi{noun} \textitbf{waktu} ‘time’ when they want to signal that the events described in each clause happened at the same time, as in (\ref{Footnote_Example_14.4}) below:
\vspace{-5pt}
\ea
\label{Footnote_Example_14.4}
\gll \bluebold{waktu} {saya} {\ldots} {tinggal} {di} {kampung} {sa} {kerja} {sperti} {laki{\Tilde}laki}\\
 {time} \textsc{1sg} {\ldots} {stay} {at} {village} \textsc{1sg} {work} {similar.to} {\textsc{rdp}{\Tilde}husband}\\
\glt ‘\bluebold{when} I {\ldots} lived in the village, I worked like a man’ (Lit. ‘\bluebold{(at that) time}’) [081014-007-Pr.0048]
\z
}
%

\subsubsection[Sequential trus ‘next’]{Sequential \textitbf{trus} ‘next’}
\label{Para_14.2.3.1}
The \isi{sequential} \isi{conjunction} \textitbf{trus} ‘next’ marks temporal relations between clauses or phra\-ses in an iconic way by organizing events in their logical and temporal order. When combining clauses, \textitbf{trus} ‘next’ always occurs in clause-initial position. The \isi{conjunction} has dual \isi{word class membership}; it is also used as the \isi{monovalent} \isi{verb} \textitbf{trus} ‘be continuous’ (see §\ref{Para_5.14}).



In terms of subject reference, an initial investigation of the attested \textitbf{trus} ‘next’ tokens in the corpus suggests the following. The \isi{conjunction} more often links clauses with a switch in reference (269 tokens), as in (\ref{Example_14.22}), than those with same-subject coreference (101 tokens). This quantitative data modifies \citegen[31]{Donohue.2003} observations that \textitbf{trus} ‘next’ “is a commonly used connective when there is same-subject coreference \isi{condition} between clauses”. Less often, \textitbf{trus} ‘next’ combines \isi{noun} phrases, as in (\ref{Example_14.23}), or prepositional phrases, as in (\ref{Example_14.24}).


\ea
	\label{Example_14.22}
	\gll       waktu  Sofia  lewat  mandi  to?  di  kamar  mandi,  \textstyleChBlueBold{trus}  Nusa   juga  lewat,  Sofia  ikat  handuk,  de  mo  lewat  masuk  ke   kamar,  \textstyleChBlueBold{trus}  Nusa  de  bicara  dia\\    
	when  Sofia  pass.by  bathe  right?  at  room  bathe  next  Nusa   also  pass.by  Sofia  tie.up  towel  \textstyleChSmallCaps{3sg}  want  pass.by  enter  to   room  next  Nusa  \textstyleChSmallCaps{3sg}  speak  \textstyleChSmallCaps{3sg}\\
	\glt ‘when Sofia passed by to bathe, right?, in the bathroom, \textstyleChBlueBold{then} Nusa also passed by, Sofia had tied (her) towel (around her waist), she wanted to pass by (and) enter the (bath)room, \textstyleChBlueBold{then} Nusa spoke to her’ \textstyleExampleSource{[081115-001a-Cv.0263]}
\z
	
	
\ea
	\label{Example_14.23}
	\gll       de  pu  potong  selesay  ambil  ubi,  \textstyleChBlueBold{trus}  daung  petatas   daung  singkong,  \textstyleChBlueBold{trus}  apa  lagi  sayur  bayam\\  
	\textstyleChSmallCaps{3sg}  \textstyleChSmallCaps{poss}  cut  finish  get  purple.yam  next  leaf  sweet.potato   leaf  cassava  next  what  again  vegetable  amaranth\\
	\glt [A recipe:] ‘(once) the cutting up (of the pig meat) is done, take purple yam, \textstyleChBlueBold{then} sweet potato leaves, cassava leaves, \textstyleChBlueBold{then}, what else, amaranth vegetables’ \textstyleExampleSource{[081014-017-CvPr.0033]}
\z
	
\ea
	\label{Example_14.24}
	\gll       {\ldots}  jalang  banyak  to?,  di  atas,  tenga,  \textstyleChBlueBold{trus}  di  laut, \textstyleChBlueBold{trus}  di  pante  sana\\  
	\textstyleChSmallCaps{}  walk  many  right?  at  top  middle  next  at  sea   then  at  coast  \textstyleChSmallCaps{l.dist}\\
	\glt ‘[I was confused (about) the road, you know,] (there) were many roads, right?, in the upper part (of the village), in the middle, \textstyleChBlueBold{and then} at the sea, \textstyleChBlueBold{and then} at the beach over there’ \textstyleExampleSource{[081025-008-Cv.0018]}
\z

\subsubsection[Sequential baru ‘and then’]{Sequential \textitbf{baru} ‘and then’}
\label{Para_14.2.3.2}
The \isi{sequential} \isi{conjunction} \textitbf{baru} ‘and then’ most commonly also marks temporal succession by ordering events in their logical and temporal sequence, as shown in (\ref{Example_14.25}). In addition, although less often, the \isi{conjunction} introduces contrast clauses, as illustrated in (\ref{Example_14.26}). The \isi{conjunction} has trial \isi{word class membership}; that is, besides being used as a \isi{conjunction}, it is also used as the stative \isi{verb} \textitbf{baru} ‘be new’ and as the temporal ad\isi{verb} \textitbf{baru} 'recently' (see §\ref{Para_5.14}; see also §\ref{Para_5.4.5} for its adverbial uses).

Typically, \textitbf{baru} ‘and then’ occurs in clause-initial position where it marks an immediate subsequent event or action, similar to \isi{sequential} \textitbf{trus} ‘next’ (§\ref{Para_14.2.3.1}). Concurrently, however, the \isi{conjunction} signals another piece of information, as shown in (\ref{Example_14.25}) (note that this example presents contiguous text). Depending on the context, the \isi{conjunction} marks noteworthy parts and/or signals a new aspect or perspective regarding the event or discourse unfolding. In this case \textitbf{baru} translates with ‘but then’, as in (\ref{Example_14.25b}) or ‘and then’ as in (\ref{Example_14.25c}). Alternatively, the \isi{conjunction} signals that the event depicted in its clause does not occur until after the event of the preceding clause. In this case, it translates with ‘only then’, as in (\ref{Example_14.25a}). In marking contrastive sequentiality, \textitbf{baru} ‘and then’ differs from \textitbf{trus} ‘next’ which indicates neutral sequentiality (see §\ref{Para_14.2.3.1}).
%


As for subject reference, an initial inspection of the \textitbf{baru} ‘and then’ tokens in the corpus suggests that the \isi{conjunction} more often links clauses with a switch in reference (524 tokens), as in (\ref{Example_14.25b}), than clauses with same-subject coreference (455 tokens), as in (\ref{Example_14.25a} and \ref{Example_14.25c}). In this respect, \textitbf{baru} ‘and then’ behaves like \textitbf{trus} ‘next’ (see §\ref{Para_14.2.3.1}).
%

\largerpage
\begin{styleExampleTitle}
Combining clauses with \textitbf{baru} ‘and then’ in clause initial position: Sequential reading
\end{styleExampleTitle}
\ea
\label{Example_14.25}
\ea
\label{Example_14.25a}

\gll  {tong} {\ldots} {jaga} {dia\textsubscript{i}} {sampe} {jam} {satu,} {\bluebold{baru}} {tong} {tidor,}\\ %
   \textsc{1pl}  {}  guard  \textsc{3sg}  until  hour  one  and.then  \textsc{1pl}  sleep\\
\glt [About a sick relative:] ‘we {\ldots} watched her until one o’clock, \bluebold{only then} did we sleep’.
\vspace{10pt}
\ex
\label{Example_14.25b}
\gll  \bluebold{baru}  Pawlus  de\textsubscript{j}  sandar  di  de\textsubscript{i}  pu  badang  begini,\\
    and.then  Pawlus  \textsc{3sg}  lean  at  \textsc{3sg}  \textsc{poss}  body  like.this\\
  \glt ‘\bluebold{but then} Pawlus was leaning against her body like this’
  \vspace{10pt}
\ex
\label{Example_14.25c}
\gll \textstyleChBlueBold{baru}  de\textsubscript{j}  kas  pata  leher  ke  bawa  di  atas de\textsubscript{i}  pu  bahu\\
    and.then  \textstyleChSmallCaps{3sg}  give  break  neck  to  bottom  at  top   \textstyleChSmallCaps{3sg}  \textstyleChSmallCaps{poss}  shoulder \\
\glt ‘\textstyleChBlueBold{and then} he bent his neck down onto her shoulder’ (Lit. ‘give to break neck’) \textstyleExampleSource{[080916-001-CvNP.0005-0006]}
\z
\z

Occasionally, the \isi{conjunction} occurs at the right periphery of a contrast clause. Summarizing what has been said before, it marks the propositional content of its clause as true despite the contents of the preceding unmarked clause. In this case, the \isi{conjunction} receives the counter-expectational reading ‘after all’, as in (\ref{Example_14.26}). As this contrast-marking function of the \isi{conjunction} is marginal, it is not further discussed in §\ref{Para_14.2.5}.

\begin{styleExampleTitle}
Combining clauses with \textitbf{baru} ‘and then’ in clause final position: Counter-expectational reading
\end{styleExampleTitle}

\ea\label{Example_14.26}

\gll sa  tra  akang  kasi  kaing,  sa  juga  dinging  stenga  mati,   ada  anging  \textstyleChBlueBold{baru}\\  
\textstyleChSmallCaps{1sg}  \textstyleChSmallCaps{neg}  will  give  cloth  \textstyleChSmallCaps{1sg}  also  be.cold  half  dead   exist  wind  and.then\\
\glt ‘I wasn’t going to give (her my) cloth, I was also half dead (from being) cold, it was windy \textstyleChBlueBold{after all}’ \textstyleExampleSource{[081025-006-Cv.0048]}
\z

\subsubsection[Anteriority{}-marking sampe ‘until’]{Anteriority-marking \textitbf{sampe} ‘until’}
\label{Para_14.2.3.3}
The \isi{conjunction} \textitbf{sampe} ‘until’ introduces a temporal clause which follows the unmarked clause. The \isi{conjunction} has trial \isi{word class membership}; that is, besides being used as a \isi{conjunction}, it is also used as the \isi{bivalent} \isi{verb} \textitbf{sampe} ‘reach’ and as the temporal \isi{preposition} \textitbf{sampe} ‘until’ (see §\ref{Para_5.14}; see also §\ref{Para_10.1.4} for its prepositional uses).

Usually, \textitbf{sampe} ‘until’ marks \isi{anteriority}. That is, it signals that the event or state of the unmarked clause occurs prior to that of the temporal clause, as shown in (\ref{Example_14.27}). Concurrently, \textitbf{sampe} ‘until’ marks temporal extent in that it indicates that the event or state of the unmarked clause continues until the event or state of the temporal clause comes about. Depending on the context, temporal \textitbf{sampe} ‘until’ can also receive a \isi{resultative} reading in the sense of ``with the result that'', as in (\ref{Example_14.28}). Given that the \isi{resultative} reading of \textitbf{sampe} ‘until’ is the derived, marginal one, this result-marking function of \textitbf{sampe} ‘until’ is not further discussed in §\ref{Para_14.2.4}.


\ea\label{Example_14.27}
\gll {\ldots} {de} {harus} {taru} {di} {mata-hari,} {\bluebold{sampe}} {de} {jadi} {papeda}\\ 
  {} \textsc{3sg}  have.to  put  at  sun  until  \textsc{3sg}  become  sagu.porridge\\
\glt [Before an ancestor had fire to heat water:] ‘[when he wanted to make sagu porridge,] he had to leave (the sago) out in the sun \bluebold{until} it turned into sagu porridge’ \textstyleExampleSource{[080922-010a-CvNF.007-0008]}
\z

\ea
\label{Example_14.28}
\gll Fredi  pu  tangang  dia  palungku  kaca,  jadi  dia  rabik,   \textstyleChBlueBold{sampe}  brapa  jahitang\\  
Fredi  \textstyleChSmallCaps{poss}  hand  \textstyleChSmallCaps{3sg}  punch  glass  so  \textstyleChSmallCaps{3sg}  tear   until  several  stitch\\
\glt [About an accident:] ‘Fredi’s hand hit glass, so it was torn \textstyleChBlueBold{with the result that} (he got) several stitches’ \textstyleExampleSource{[081006-032-Cv.0066]}
\z

\subsubsection[Anteriority{}-marking seblum ‘before’]{Anteriority-marking \textitbf{seblum} ‘before’}
\label{Para_14.2.3.4}
Anteriority-marking \textitbf{seblum} ‘before’ also introduces a temporal clause.\footnote{The \isi{conjunction} \textitbf{seblum} ‘before’ is \isi{historically derived} from the aspectual ad\isi{verb} \textitbf{blum} ‘not yet’: \textitbf{se-blum} ‘one-not.yet’ (see §\ref{Para_5.4.1}).} It indicates – similar to \textitbf{sampe} ‘until’ – that the event or state of the unmarked clause occurs prior to that of the temporal clause. Unlike \textitbf{sampe} ‘until’, however, \textitbf{seblum} ‘before’ does not signal extent. The temporal clause with \textitbf{seblum} ‘before’ can precede or follow the unmarked clause, as shown in (\ref{Example_14.29}) and (\ref{Example_14.30}), respectively. In the corpus, however, the temporal clause more often precedes the unmarked clause (21 tokens) rather than follows it (8 tokens).
%

\ea\label{Example_14.29}
\gll {de} {bilang,} {\bluebold{seblum}} {kitong} {pergi} {ke} {kota,} {kitong} {cuci} {muka} {dulu}\\ %
 \textsc{3sg}  say  before  \textsc{1pl}  go  to  city  \textsc{1pl}  wash  front  first\\
\glt ‘he said, ``\bluebold{before} we go to the city, we wash (our) faces first''' \textstyleExampleSource{[080917-008-NP.0126]}
\z

\ea\label{Example_14.30}
\gll {\ldots} {saya} {suda} {punya} {rencana} {juga,} {\bluebold{seblum}} {sa} {kluar}\\ %
  {} \textsc{1sg}  already  have  plan  also  before  \textsc{1sg} go.out\\
\glt ‘[when I hunt without taking dogs, I leave in the night,] I also already have a plan \bluebold{before} I leave’ \textstyleExampleSource{[080919-004-NP.0002]}
\z

\subsubsection[Posteriority{}-marking/{conditional} kalo ‘when, if’]{Posteriority{}-marking/{conditional} \textitbf{kalo} ‘when, if’}
\label{Para_14.2.3.5}
The \isi{conjunction} \textitbf{kalo} ‘when, if’ signals temporal relations, namely posteriority, and/or \isi{conditional} relations between two clauses. The clause it introduces always precedes the unmarked clause.



Whether \textitbf{kalo} ‘when, if’ receives a temporal reading as in (\ref{Example_14.31}) and (\ref{Example_14.32}), or a \isi{conditional} reading, as in (\ref{Example_14.33}) and (\ref{Example_14.34}), is context-dependent. Quite often, though, both interpretations are possible, as shown in (\ref{Example_14.35}). As mentioned, this lack of a “distinction between ‘if’ clauses and ‘when’ clauses” is also found in other languages; examples are “Indonesian and certain languages of Papua New Guinea” \citep[257]{Thompson.2007}.
%


When marking posteriority, \textitbf{kalo} translates with ‘when’; it signals that the event or state of the unmarked main clause occurs subsequent to that of the temporal clause, as in (\ref{Example_14.31}). When the \isi{conjunction} co-occurs with the retrospective ad\isi{verb} \textitbf{suda} ‘already’, or with its short form \textitbf{su}, it projects these events or states to the future; in this case \textitbf{kalo} translates with ‘once’. That is, in combination with \textitbf{suda} ‘already’, the \isi{conjunction} signals that the event or state of the unmarked clause will eventuate, once that of the temporal clause has come about, as in (\ref{Example_14.32}).
%
\begin{styleExampleTitle}
Combining clauses with \textitbf{kalo} ‘when/after’: Temporal reading
\end{styleExampleTitle}
\ea\label{Example_14.31}
\gll {\bluebold{kalo}} {dong} {tendang} {de} {pu} {kaki} {tu,} {dia} {pegang} {bola}\\ %
 when  \textsc{3pl}  kick  \textsc{3sg}  \textsc{poss}  foot  \textsc{d.dist}  \textsc{3sg}  hold  ball\\
\glt [About a football match:] ‘\bluebold{when} they kicked those legs of his, he grabbed the ball’ \textstyleExampleSource{[081006-014-Cv.0004]}
\z

\ea
\label{Example_14.32}
\gll       jadi  \textstyleChBlueBold{kalo}  dong  \textstyleChBlueBold{su}  tinggal  di  kota  begini,  dong   snang  tinggal,  tida  maw  pulang  ke  kampung\\  
so  if  \textstyleChSmallCaps{3pl}  already  stay  at  city  like.this  \textstyleChSmallCaps{3pl}   feel.happy(.about)  stay  \textstyleChSmallCaps{neg}  want  go.home  to  village\\
\glt ‘so \textstyleChBlueBold{once} they’ve lived in the city like this, they’re happy to stay (here), (they) don’t want to return home to the village’ \textstyleExampleSource{[080927-009-CvNP.0059]}
\z

In a different context, the \isi{conjunction} receives a \isi{conditional} reading and signals, what \citet[6]{Kaufmann.2006} calls, ``indicative \isi{conditional}'' relations or ``counterfactual \isi{conditional}'' relations. In such a context \textitbf{kalo} translates with ‘if’. An indicative \isi{conditional} relation indicates that it is possible for the \isi{condition} presented in its clause to be met. In this case the event or state of the unmarked clause will also come about, as shown in (\ref{Example_14.33}). When \isi{conditional} \textitbf{kalo} ‘if’ co-occurs with retrospective \textitbf{suda} ‘already’, the clause receives a counterfactual \isi{conditional} reading. That is, it signals that the \isi{condition} was not met in the past. If the \isi{condition} had been met, however, then the event or state of the unmarked clause would also have come about. This is illustrated in (\ref{Example_14.34}).

\begin{styleExampleTitle}
Combining clauses with \textitbf{kalo} ‘if’: Conditional reading
\end{styleExampleTitle}
\ea
\label{Example_14.33}
\gll {\bluebold{kalo}} {ko} {alpa,} {kitong} {tra} {jalang}\\ %
 if  \textsc{2sg}  be.absent  \textsc{1pl}  \textsc{neg}  walk\\
\glt [Talking to her son about an upcoming trip:] ‘\bluebold{if} you play hooky, we won’t go’ \textstyleExampleSource{[080917-003a-CvEx.0038]}
\z

\ea
\label{Example_14.34}

\gll       \textstyleChBlueBold{kalo}  sa  \textstyleChBlueBold{su}  pake  em  kaca-mata  tu, mungking  sa  su  gila\\  
if  \textstyleChSmallCaps{1sg}  already  use  uh  glasses  \textstyleChSmallCaps{d.dist}   maybe  \textstyleChSmallCaps{1sg}  already  be.crazy\\
\glt ‘\textstyleChBlueBold{if} I’d been wearing, uh, those (sun)glasses, I might already be crazy’ \textstyleExampleSource{[080919-005-Cv.0007]}
\z

\noindent Rather commonly, \textitbf{kalo} ‘when, if’ allows both a temporal and a \isi{conditional} reading, as in (\ref{Example_14.35}).

\begin{styleExampleTitle}
Combining clauses with \textitbf{kalo} ‘when, if’: Temporal and/or \isi{conditional} reading
\end{styleExampleTitle}
\ea\label{Example_14.35}
\gll {\bluebold{kalo}} {bapa} {datang,} {pluk} {bapa}\\ %
 when/if  father  come  embrace  father\\
\glt ‘\bluebold{when/if} you (‘father’) come (here), (I’ll) embrace you (‘father’)’ \textstyleExampleSource{[080922-001a-CvPh.0360]}
\z

\subsection{Consequence}
\label{Para_14.2.4}
A consequence-marking \isi{conjunction} indicates that the event or state of its clause is the outcome of an event or state depicted in the unmarked clause. Papuan Malay has five such conjunctions: \isi{resultative}/\isi{causal} \textitbf{jadi} ‘so, since’ (§\ref{Para_14.2.4.1}), \isi{purposive} \textitbf{supaya} ‘so that’ (§\ref{Para_14.2.4.2}), \isi{purposive} \textitbf{untuk} ‘for’ (§\ref{Para_14.2.4.3}), \isi{causal} \textitbf{karna} ‘because’ (§\ref{Para_14.2.4.4}), and \isi{causal} \textitbf{gara-gara} ‘because’ (§\ref{Para_14.2.4.5}). In addition, although rarely, temporal \textitbf{sampe} ‘until’ has result-marking function in the sense of ``with the result that''; given that this function is marginal, it is discussed in §\ref{Para_14.2.3.3} and not here.


\subsubsection[Resultative/{causal} jadi ‘so, since’]{Resultative/{causal} \textitbf{jadi} ‘so, since’}
\label{Para_14.2.4.1}
The \isi{resultative}/\isi{causal} \isi{conjunction} \textitbf{jadi} ‘so, since’ most often marks a \isi{resultative} relation between two clauses, as shown in (\ref{Example_14.36}). In addition, although less often, the \isi{conjunction} signals a \isi{causal} relation, as illustrated in (\ref{Example_14.37}). The \isi{conjunction} has dual \isi{word class membership}; it is also used as the \isi{bivalent} \isi{verb} \textitbf{jadi} ‘become’ (see §\ref{Para_5.14}).



Typically, \textitbf{jadi} ‘so, since’ occurs in initial position of a result clause that follows the unmarked clause. Here, the \isi{conjunction} signals that the event or state of its clause results from that of the unmarked clause, as in (\ref{Example_14.36}); hence, \textitbf{jadi} translates with ‘so’.
%

\begin{styleExampleTitle}
Combining clauses with \textstyleChItalBold{jadi} ‘so, since’: Clause-initial position
\end{styleExampleTitle}
\ea
\label{Example_14.36}
\gll       tong  tra  snang  dengang  dia,  \textstyleChBlueBold{jadi}  kitong  malas   datang  dia  pu  ruma\\  
\textstyleChSmallCaps{1pl}  \textstyleChSmallCaps{neg}  feel.happy(.about)  with  \textstyleChSmallCaps{3sg}  so  \textstyleChSmallCaps{1pl}  be.listless   come  \textstyleChSmallCaps{3sg}  \textstyleChSmallCaps{poss}  house\\
\glt ‘we don’t feel happy about her, \textstyleChBlueBold{so} we don’t want (to) come to her house’ \textstyleExampleSource{[080927-006-CvNP.0032]}
\z

Alternatively, but less often, the \isi{conjunction} occurs in clause-final position of a cause clause where it marks a \isi{causal} relation with the preceding unmarked clause, as in (\ref{Example_14.37}). In this position, the \isi{conjunction} signals that something depicted in its clause is the cause for the event or state of the unmarked clause, and that the result depicted in the unmarked clause is anticipated. Hence, \textitbf{jadi} translates with ‘since’. In that the result is expected, \isi{causal} \textitbf{jadi} ‘since’ differs from neutral causality-marking \textitbf{karna} ‘because’ (see §\ref{Para_14.2.4.4}).

\begin{styleExampleTitle}
Combining clauses with \textitbf{jadi} ‘so, since’: Clause-final position
\end{styleExampleTitle}
\ea
\label{Example_14.37}
\gll {Musa} {ini,} {e,} {de} {loyo{\Tilde}loyo} {ini,} {de} {bangung} {tidor} {\bluebold{jadi}}\\ %
 Musa  \textsc{d.prox}  uh  \textsc{3sg}  \textsc{rdp}{\Tilde}be.weak  this  \textsc{3sg}  wake.up  sleep so\\
\glt [About a small boy:] ‘Musa here, uh, right now he’s kind of weak \bluebold{since} he woke up from sleeping’ \textstyleExampleSource{[080922-001a-CvPh.1435/1437]}
\z

\subsubsection[Purposive supaya ‘so that’]{Purposive \textitbf{supaya} ‘so that’}
\label{Para_14.2.4.2}
Purposive \textitbf{supaya} ‘so that’ introduces a purpose clause which follows the unmarked clause. The \isi{conjunction} signals that the event or state of its clause is the intended outcome of the deliberate activity depicted in the unmarked clause. Most often, \textitbf{supaya} ‘so that’ introduces a purpose clause with an overt subject (96/129 tokens – 74\%), as in (\ref{Example_14.38}). Less often, the \isi{conjunction} introduces a purpose clause with elided subject (33/129 tokens – 26\%), as in (\ref{Example_14.39}).
%

\ea
\label{Example_14.38}
\gll mace  ko  sendiri  yang  ikut,  \textstyleChBlueBold{supaya}  ko  atur makangang  di  sana!\\  
wife  \textstyleChSmallCaps{2sg}  alone  \textstyleChSmallCaps{rel}  follow  so.that  \textstyleChSmallCaps{2sg}  arrange food  at  \textstyleChSmallCaps{l.dist}\\
\glt ‘you wife yourself (should) go with (them) \textstyleChBlueBold{so that} you organize the catering over there!’ (Lit. ‘(it’s) you wife yourself who {\ldots}’) \textstyleExampleSource{[081025-009a-Cv.0032]}
\z

\ea\label{Example_14.39}
\gll {e,} {angkat} {muka,} {\bluebold{supaya}} {Ø} {liat} {orang!}\\ %
 hey!  lift  front  so.that  {}  see  person\\
\glt ‘hey, lift (your) face \bluebold{so} \bluebold{that} (you) see (the other) people!’ \textstyleExampleSource{[081110-008-CvNP.0101]}
\z

\subsubsection[Purposive untuk ‘for’]{Purposive \textitbf{untuk} ‘for’}
\label{Para_14.2.4.3}
As a \isi{conjunction}, the benefactive \isi{preposition} \textitbf{untuk} ‘for’ signals a purpose relation between two clauses (for a description of \isi{preposition} \textitbf{untuk} ‘for’, see §\ref{Para_10.2.3}). Purposive \textitbf{untuk} ‘for’, like \textitbf{supaya} ‘so that’ (see §\ref{Para_14.2.4.2}), introduces a purpose clause which expresses the intended outcome of the purposeful activity depicted in the preceding unmarked clause, as shown in (\ref{Example_14.40}) and (\ref{Example_14.41}). Usually, \textitbf{untuk} ‘for’ introduces a purpose clause with an elided subject (115/163 tokens – 71\%), as shown with the second \textitbf{untuk} ‘for’ token in (\ref{Example_14.40}). Much less often the \isi{conjunction} introduces a purpose clause with an overt subject (48/163tokens – 29\%), as shown with the first \textitbf{untuk} ‘for’ token in (\ref{Example_14.40}), or as in (\ref{Example_14.41}).
%

\ea
\label{Example_14.40}
\gll       tadi  ana  bilang,  {\ldots}  bapa  dorang  siap  saja,  \textstyleChBlueBold{untuk}  kita   ke  sana  a,  sa  juga  siap,  \textstyleChBlueBold{untuk}  Ø  bawa  kamu   ke  sini\\  
earlier  child  say {}   father  \textstyleChSmallCaps{3pl}  get.ready  just  for  \textstyleChSmallCaps{1pl}   to  \textstyleChSmallCaps{l.dist}  ah!  \textstyleChSmallCaps{1sg}  also  get.ready  for  {}  bring  \textstyleChSmallCaps{2pl}   to  \textstyleChSmallCaps{l.prox}\\
\glt ‘a short while ago you (‘child’) said, ``{\ldots} father and the others are ready \textstyleChBlueBold{for} us (to move) to (\ili{Sarmi} over) there'', ah, (in that case) I’m also ready \textstyleChBlueBold{to} bring you (to \ili{Sarmi}) here’ (Lit. ‘\textstyleChBlueBold{for} (me to) bring you’) \textstyleExampleSource{[080922-001a-CvPh.1241]}
\z

\ea\label{Example_14.41}
\gll {\ldots} {tida} {bole,} {ini,} {kamu} {datang,} {\bluebold{untuk}} {kamu} {skola}\\ %
   {} \textsc{neg}  may  \textsc{d.prox}  \textsc{2pl}  come  for  \textsc{2pl}  go.to.school\\
\glt ‘[you shouldn’t hate each other, (you) shouldn’t infuriate each other,] (you) shouldn’t (do all this), what’s-its-name, you came (here) \bluebold{to} go to school’ (Lit. ‘\bluebold{for} you (to) go to school’) \textstyleExampleSource{[081115-001a-Cv.0272]}
\z


\noindent The attested data indicates that \textitbf{untuk} ‘for’ differs from \textitbf{supaya} ‘so that’ in that \textitbf{untuk} ‘for’ most often introduces purpose clauses with elided subjects. By contrast, \textitbf{supaya} ‘so that’ most often introduces purpose clauses with overt subjects.


\subsubsection[Causal karna ‘because’]{Causal \textitbf{karna} ‘because’}
\label{Para_14.2.4.4}
Causal \textitbf{karna} ‘because’ signals a neutral \isi{causal} relation between two clauses by introducing a cause clause which gives the reason for the event or state depicted in the unmarked clause. Usually the cause clause follows the unmarked clause, as in (\ref{Example_14.42}). In combination with \isi{adversative} \textitbf{tapi} ‘but’ (see §\ref{Para_14.2.5.1}), however, it can precede the unmarked clause, as in (\ref{Example_14.43}). In this case the unmarked clause is often introduced with \isi{resultative} \textitbf{jadi} ‘so’. Signaling neutral causality, \textitbf{karna} ‘because’ is distinct from \isi{causal} \textitbf{jadi} ‘since’ which marks expected results (see §\ref{Para_14.2.4.1}), and from \isi{causal} \textitbf{gara-gara} ‘because’ which marks emotive \isi{causal} relations (see §\ref{Para_14.2.4.5}).
%

\ea\label{Example_14.42}
\gll {saya} {bisa} {pulang,} {\bluebold{karna}} {sa} {su} {dapat} {babi}\\ %
 \textsc{1sg}  be.able  go.home  because  \textsc{1sg}  already  get  pig\\
\glt [Hunting a wild pig:] ‘I can return home \bluebold{because} I already got the pig’ \textstyleExampleSource{[080919-004-NP.0024]}
\z

\ea
\label{Example_14.43}
\gll       dong  memang  piara  de  di  situ,  \textstyleChBlueBold{tapi}  \textstyleChBlueBold{karna}  mama  dong   pu  {bapa-ade}  {{\ldots},}  \textstyleChBlueBold{tapi} \textstyleChBlueBold{karna}  tete  meninggal,  \textstyleChBlueBold{jadi}   dong  pu  kluarga  ini  yang  piara\\  
\textstyleChSmallCaps{3pl}  indeed  raise  \textstyleChSmallCaps{3sg}  at  \textstyleChSmallCaps{l.med}  but because  mother  \textstyleChSmallCaps{3pl}   \textstyleChSmallCaps{poss}  uncle  {}  but  because  grandfather  die  so   \textstyleChSmallCaps{3pl}  \textstyleChSmallCaps{poss}  family  \textstyleChSmallCaps{d.prox}  \textstyleChSmallCaps{rel}  raise\\
\glt ‘they took indeed care of him there, \textstyleChBlueBold{but because} the uncle of mama and her companions [umh, who’s actually the youngest offspring,] \textstyleChBlueBold{but because} grandfather died, \textstyleChBlueBold{so} (it’s) their family here who took care of him’ \textstyleExampleSource{[080919-006-CvNP.0006-0008]}
\z


\subsubsection[Causal gara{}-gara ‘because’]{Causal \textitbf{gara-gara} ‘because’}
\label{Para_14.2.4.5}
The \isi{causal} \isi{conjunction} \textitbf{gara-gara} ‘because’ indicates an emotive \isi{causal} relation between two clauses by introducing a cause clause which gives the reason for the circumstances depicted in the unmarked clause. Thereby, \textitbf{gara-gara} ‘because’ is distinct from \textitbf{karna} ‘because’ which marks neutral \isi{causal} relations.  The \isi{conjunction} has dual \isi{word class membership}; it is also used as the \isi{bivalent} \isi{verb} \textitbf{gara} ‘irritate’ (see §\ref{Para_5.14}).

Most often, the cause clause marked with \textitbf{gara-gara} ‘because’ follows the unmarked clause, as in (\ref{Example_14.44}). Alternatively, the cause clause can precede the unmarked clause. In this case, \isi{adversative} \textitbf{tapi} ‘but’ (see §\ref{Para_14.2.5.1}) precedes \textitbf{gara-gara} ‘because’, as in (\ref{Example_14.45}), in the same way as \textitbf{tapi} ‘but’ precedes \textitbf{karna} ‘because’ (see §\ref{Para_14.2.4.4}).


\ea\label{Example_14.44}
\gll {sap} {prut} {sakit,} {\bluebold{gara-gara}} {sa} {makang} {nasi}\\ %
 \textsc{1sg:poss}  stomach  be.sick  because  \textsc{1sg}  eat  cooked.rice\\
\glt ‘my stomach was sick \bluebold{because} I ate rice’ \textstyleExampleSource{[081025-009a-Cv.0046]}
\z

\ea
\label{Example_14.45}

\gll {\ldots}  \textstyleChBlueBold{tapi}  \textstyleChBlueBold{gara-gara}  Nofela  bi,  \textstyleChBlueBold{gara-gara}  Nofela  bicara deng  bapa,  bapa  pu  hati  tergrak  {\ldots}\\  
{} but  because  Nofela  {\textstyleChSmallCaps{tru}-speak}  because  Nofela  speak with  father  father  \textstyleChSmallCaps{poss}  liver  be.moved  \\
\glt [Phone conversation between a father and his daughter:] ‘[(if) I had just spoken to Siduas, maybe I wouldn’t have felt moved to come (and pick you up), right?,] \textstyleChBlueBold{but because} you (‘Nofela’) spoke[\textstyleChSmallCaps{tru}], \textstyleChBlueBold{because} you (‘Nofela’) spoke with me, my (‘father’s) heart was moved [so I’ll definitely come (and pick you up)]’ \textstyleExampleSource{[080922-001a-CvPh.1082-1083]}
\z

\subsection{Contrast}
\label{Para_14.2.5}
Contrast-marking conjunctions are cross-linguistically defined as conjunctions that signal that the events or states described in two clauses “are valid simultaneously”, but that the information given in one clause “marks a contrast to the information” given in the other clause \citep[20]{Rudolph.1996}. This section describes four Papuan Malay contrast-marking conjunctions: \isi{adversative} \textitbf{tapi} ‘but’ and \textitbf{habis} ‘after all’ (§\ref{Para_14.2.5.2} and §\ref{Para_14.2.5.1}), \isi{oppositive} \textitbf{padahal} ‘but actually’ (§\ref{Para_14.2.5.3}), and \isi{concessive} \textitbf{biar} ‘although’ (§\ref{Para_14.2.5.4}). In addition, temporal \textitbf{baru} ‘and then’ has contrast-marking function in that it signals counter-expectation in the sense of ``after all''; as this function is marginal it is discussed in §\ref{Para_14.2.3.2} and not here.


\subsubsection[Adversative tapi ‘but’]{Adversative \textitbf{tapi} ‘but’}
\label{Para_14.2.5.1}
Adversative \textitbf{tapi} ‘but’ occurs in interclausal position. It marks an \isi{adversative} contrast relation between the clause it introduces and the preceding unmarked clause, as shown in (\ref{Example_14.46}) and (\ref{Example_14.47}).
%

\ea
\label{Example_14.46}
\gll {de} {bisa} {maing} {gitar,} {\bluebold{tapi}} {de} {malu}\\ %
 \textsc{3sg}  be.able  play  guitar  but  \textsc{3sg}  feel.embarrassed(.about)\\
\glt ‘she can play the guitar \bluebold{but} she feels shy (about it)’ \textstyleExampleSource{[081014-015-Cv.0008]}
\z

\ea
\label{Example_14.47}

\gll jadi  sa  punya  bapa  kasi  saya  untuk  Iskia,  \textstyleChBlueBold{tapi}  Iskia kawing  sala,  Iskia  kawing  sa  punya  kaka\\  
so  \textstyleChSmallCaps{1sg}  \textstyleChSmallCaps{poss}  father  give  \textstyleChSmallCaps{1sg}  for  Iskia  but  Iskia marry  be.wrong  Iskia  marry  \textstyleChSmallCaps{1sg}  \textstyleChSmallCaps{poss}  oSb\\
\glt ‘so my father gave me to Iskia, \textstyleChBlueBold{but} Iskia married improperly, Iskia married my older sister’ \textstyleExampleSource{[081006-028-CvEx.0005]}
\z

\subsubsection[Adversative habis ‘after all’]{Adversative \textitbf{habis} ‘after all’}
\label{Para_14.2.5.2}
Adversative \textitbf{habis} ‘after all’ also marks an \isi{adversative} relation between two clauses. The \isi{conjunction} has dual \isi{word class membership}; it is also used as the \isi{monovalent} \isi{verb} \textitbf{habis} ‘be used up’ (see §\ref{Para_5.14}).



Introducing a contrast clause that follows the unmarked clause, \textitbf{habis} ‘after all’ summarizes what has been said before and signals that the propositional content of its clause is true in spite of the content of the preceding unmarked clause, as shown in (\ref{Example_14.48}) and (\ref{Example_14.49}). At the same time, the \isi{conjunction} signals that the interlocutor is expected to know that this content is true. Thereby \textitbf{habis} ‘after all’ is distinct from \isi{adversative} \textitbf{tapi} ‘but’ (see §\ref{Para_14.2.5.2}). Adversative \textitbf{habis} ‘after all’ is also distinct from counter-expectational \textitbf{baru} ‘after all’ which merely summarizes what has been said before (see §\ref{Para_14.2.3.2}). The exchange in (\ref{Example_14.49}) illustrates that there does not need to be an overt unmarked clause which precedes the contrast clause: speakers also use \textitbf{habis} ‘after all’ to reply to an interlocutor’s statements.


\ea\label{Example_14.48}
\gll {bilang} {bapa,} {kirim} {tong} {uang,} {\bluebold{habis}} {sa} {susa} {to?}\\ %
 say  father  send  \textsc{1pl}  money  after.all  \textsc{1sg}  difficult  right?\\
\glt ‘say (to) father, ``send us money, \bluebold{after all}, I have difficulties, right?''' \textstyleExampleSource{[080922-001a-CvPh.0866]}
\z

\ea\label{Example_14.49}
\ea\gll {{Speaker-1:}  ko  baru  masuk  klas  satu  ini?}\\ %
   {}  \textsc{2sg}  recently  enter  class  one  \textsc{d.prox}\\
\glt Speaker-1: ‘recently you got into first grade (of middle school)?’
\vspace{10pt}
\ex
\gll {Speaker-2:}  yo,  \bluebold{habis}  sa  gagal\\
  {}   yes  after.all  \textsc{1sg}  fail\\
\glt Speaker-2: ‘yes, \bluebold{after all}, I failed (the last exams)’ \textstyleExampleSource{[080922-001a-CvPh.0965-0966]}
\z
\z

\subsubsection[Oppositive padahal ‘but actually’]{Oppositive \textitbf{padahal} ‘but actually’}
\label{Para_14.2.5.3}
The \isi{conjunction} \textitbf{padahal} ‘but actually’ introduces a contrast clause, which follows the unmarked clause. Concurrent to marking contrast, the \isi{conjunction} signals that the propositional content of its clause is surprising and unexpected given the content of the unmarked clause. Thereby, \textitbf{padahal} ‘but actually’ is more \isi{oppositive} than \textitbf{tapi} ‘but’ (see §\ref{Para_14.2.5.1}). This is illustrated in (\ref{Example_14.50}) and (\ref{Example_14.51}).
%

\ea
\label{Example_14.50}
\gll ana  ini,  sa  pikir  de  suda  lewat,  \textstyleChBlueBold{padahal}  de tidor  di  atas  kayu{\Tilde}kayu\\  
child  \textstyleChSmallCaps{d.prox}  \textstyleChSmallCaps{1sg}  think  \textstyleChSmallCaps{3sg}  already  pass.by  but.actually  \textstyleChSmallCaps{3sg} sleep  at  top  \textstyleChSmallCaps{rdp}{\Tilde}wood\\
\glt ‘this child, I thought he’d already passed by, \textstyleChBlueBold{but actually} he was sleeping on top of the wood’ \textstyleExampleSource{[081013-004.Cv.0004]}
\z

\ea
\label{Example_14.51}
\gll bulang  oktober  sa  pu  alpa  cuma  dua  saja,  bayangkang, \textstyleChBlueBold{padahal}  sa  alpa  banyak\\  
month  October  \textstyleChSmallCaps{1sg}  \textstyleChSmallCaps{poss}  be.absent  just  two  just  image but.actually  \textstyleChSmallCaps{1sg}  be.absent  many\\
\glt [About the speaker’s school attendance:] ‘imagine!, in October I had just only two (official) absences, \textstyleChBlueBold{but actually} I was absent many times’ (Lit. ‘my absences were many’) \textstyleExampleSource{[081023-004-Cv.0014]}
\z

\subsubsection[Concessive biar ‘although’]{Concessive \textitbf{biar} ‘although’}
\label{Para_14.2.5.4}
Concessive \textitbf{biar} ‘although’ marks \isi{concessive} relations between two clauses. The \isi{conjunction} has dual \isi{word class membership}; it is also used as the \isi{bivalent} \isi{verb} \textitbf{biar} ‘let’ (see §\ref{Para_5.14}).

Introducing a concession clause, \textitbf{biar} ‘although’ signals that despite the event or state depicted in its clause, the event or state depicted in the unmarked clause occurred. Usually, the concession clause precedes the unmarked clause, whereby the concession is emphasized, as in (\ref{Example_14.52}). Alternatively, although less often, it can follow the unmarked clause, in which case the content of the latter clause is emphasized, as in (\ref{Example_14.53}).


\ea
\label{Example_14.52}
\gll {yo,} {\bluebold{biar}} {makangang} {tinggi,} {de} {ambil}\\ %
 yes  although  food  be.high  \textsc{3sg}  fetch\\
\glt [About a greedy child:] ‘yes, \bluebold{although} the food is (placed) high (up on a shelf), he takes (it)’ \textstyleExampleSource{[081025-006-Cv.0254]}
\z

\ea
\label{Example_14.53}

\gll {\ldots}  jangang  tinggal  di  ruma,  tida  bole,  \textstyleChBlueBold{biar}  dulu   orang-tua  dong  bilang  begini\\  
{} \textstyleChSmallCaps{neg.imp}  stay  at  house  \textstyleChSmallCaps{neg}  may  although  first parent  \textstyleChSmallCaps{3pl}  say  like.this\\
\glt ‘[so you kids have to go to school,] don’t stay home, (that’s) not allowed, \textstyleChBlueBold{although} the parents said so in the past’ \textstyleExampleSource{[081110-008-CvNP.0036]}
\z

\subsection{Similarity}
\label{Para_14.2.6}
As conjunctions, the similative prepositions \textitbf{sperti} ‘similar to’ and \textitbf{kaya} ‘like’ mark \isi{similarity} between two clauses. Introducing \isi{similarity} clauses, both signal that the event or state depicted in the unmarked clause is similar to that described in the \isi{similarity} clause. The \isi{similarity} clause always follows the unmarked clause.



Derived from their prepositional semantics, \textitbf{sperti} ‘similar to’ signals likeness in some, often implied, respect, while \textitbf{kaya} ‘like’ marks overall resemblance, as shown in (\ref{Example_14.54}) and (\ref{Example_14.55}), respectively. (See §\ref{Para_10.3.1} and §\ref{Para_10.3.2} for a detailed discussion of the prepositions \textitbf{sperti} ‘similar to’ and \textitbf{kaya} ‘like’ and their semantics.)


\ea
\label{Example_14.54}
\gll {mama} {dia} {lupa} {kamu,} {\bluebold{sperti}} {kacang} {lupa} {kulit}\\ %
 mother  \textsc{3sg}  forget  \textsc{2pl}  similar.to  bean  forget  skin\\
\glt ‘mother forgot you (in a way that is) \bluebold{similar to} a bean forgetting its skin’ \textstyleExampleSource{[080922-001a-CvPh.0932]}
\z

\ea
\label{Example_14.55}
\gll {\ldots}  tong  taputar  \textstyleChBlueBold{kaya}  kitong  ni  ana{\Tilde}ana perjalangang  yang  taputar\\  
{} \textstyleChSmallCaps{1pl}  be.turned.around  like  \textstyleChSmallCaps{1pl}  \textstyleChSmallCaps{d.prox}  \textstyleChSmallCaps{rdp}{\Tilde}child journey  \textstyleChSmallCaps{rel}  be.turned.around\\
\glt ‘[we were looking for a bathroom {\ldots}, good grief! there were no bathrooms,] we wandered around \textstyleChBlueBold{like} we here were children on a trip wandering around’ \textstyleExampleSource{[081025-009a-Cv.0059]}
\z

\section{Conjunctions combining different-type constituents}
\label{Para_14.3}
This section describes two conjunctions which combine different-type constit\-uents. Complementizer \textitbf{bahwa} ‘that’ links a clause to a \isi{bivalent} \isi{verb} (§\ref{Para_14.3.1}), while \isi{relativizer} \textitbf{yang} ‘\textsc{rel}’ integrates a relative clause within a \isi{noun} phrase (§\ref{Para_14.3.2}).


\subsection{Complementizer \textitbf{bahwa} ‘that’}
\label{Para_14.3.1}
The \isi{complementizer} \textitbf{bahwa} ‘that’ marks a clause as the complement of a \isi{verb}. Cross-linguistically, it is typically \isi{bivalent} “verbs of utterance and cognition” that take complements \citep[279]{Payne.1997}. This also applies to Papuan Malay. The corpus contains 68 complement clauses with \textitbf{bahwa} ‘that’. In 37 cases (54\%), the complement-taking \isi{verb} is \textitbf{taw} ‘know’, followed by \textitbf{bilang} ‘say’ (5 tokens), \textitbf{ceritra} ‘tell’(4 tokens), and \textitbf{liat} ‘see’ (3 tokens).

Two structural patterns are attested for complementation with \textitbf{bahwa} ‘that’. Usually, the \isi{verb} is followed by the clausal complement with \textitbf{bahwa} ‘that’ (61 tokens), as in (\ref{Example_14.56}) and (\ref{Example_14.57}). Alternatively, although much less often, the \isi{verb} is followed by an object which is followed by the clausal complement (8 tokens), as in (\ref{Example_14.58}).\footnote{Typically, speakers report speech in the form of \isi{direct speech} rather than in\isi{direct speech} as in (\ref{Example_14.58}) (see also §\ref{Para_6.2.1.1}).}
%

\begin{styleExampleTitle}
\textsc{verb} – \textitbf{bahwa} ‘that’ (\textsc{object}) – \textsc{clausal} \textsc{complement}
\end{styleExampleTitle}
\ea\label{Example_14.56}
\gll {sa} {tida} {\bluebold{taw}} {\bluebold{bahwa}} {jam} {tiga} {itu} {de} {su} {meninggal}\\ %
 \textsc{1sg}  \textsc{neg}  know  that  hour  three  \textsc{d.dist}  \textsc{3sg}  already  die\\
\glt ‘I didn’t \bluebold{know that} by three o’clock she had already died’ \textstyleExampleSource{[080917-001-CvNP.0005]}
\z

\ea
\label{Example_14.57}

\gll kalo  blum  nika  itu,  greja  \textstyleChBlueBold{bilang}  \textstyleChBlueBold{bahwa}  dong  dua blum  jadi  swami  istri\\  
if  not.yet  marry  \textstyleChSmallCaps{d.dist}  church  say  that  \textstyleChSmallCaps{3pl}  two not.yet  become  husband  wife\\
\glt ‘if (they) haven’t (officially) married yet, (then) the church \textstyleChBlueBold{says that} the two of them haven’t yet become husband and wife’ \textstyleExampleSource{[081110-006-CvEx.0196]}
\z

\ea
\label{Example_14.58}
\gll jadi  Raymon  \textstyleChBlueBold{tuntut}  \textstyleChBlueBold{sama}  \textstyleChBlueBold{kita}  to?,  \textstyleChBlueBold{sama}  \textstyleChBlueBold{kitorang} \textstyleChBlueBold{bahwa}  kamu  harus  ganti  lagi\\  
so  Raymon  demand  from  \textstyleChSmallCaps{1pl}  right?  from  \textstyleChSmallCaps{1pl} that  \textstyleChSmallCaps{2pl}  have.to  replace  also\\
\glt [About bride-price customs:] ‘so Raymon \textstyleChBlueBold{demanded from us}, right?, \textstyleChBlueBold{from us that} we also had to compensate (for that wife)’ (Lit. ‘{\ldots} \textstyleChBlueBold{from us that} you had to replace’) \textstyleExampleSource{[081006-024-CvEx.0019]}
\z


\subsection{Relativizer \textitbf{yang} ‘\textsc{rel}’}
\label{Para_14.3.2}
Relativizer \textitbf{yang} ‘\textsc{rel}’ introduces relative clauses which function as modifiers within \isi{noun} phrases (see also §\ref{Para_8.2.8}). Typically, the relative clause follows its head nominal, as in (\ref{Example_14.59}) and (\ref{Example_14.61a}). However, \textitbf{yang} ‘\textsc{rel}’ can also introduce a headless relative clause. Cross-linguistically, headless relative clauses can be used “when the head \isi{noun} is non-specific” or when “the specific reference to the head is clear” \citep[295]{Payne.1997}. This also applies to Papuan Malay. In (\ref{Example_14.60}), for instance, the head nominal is non-specific, while in (\ref{Example_14.61b}) the reference to the head is clear (“Ø” signifies the implied head nominal).

\begin{styleExampleTitle}
Relative clauses with overt head nominal and headless relative clauses
\end{styleExampleTitle}
\ea\label{Example_14.59}
\gll {kitong} {mo} {hancurkang} {\bluebold{tugu}} {\bluebold{yang}} {ada} {di} {Sarmi} {itu}\\ %
 \textsc{1pl}  want  shatter  monument  \textsc{rel}  exist  at  \ili{Sarmi}  \textsc{d.dist}\\
\glt ‘we want to destroy \bluebold{the statue that} is in \ili{Sarmi} there’ \textstyleExampleSource{[080917-008-NP.0043]}
\z

\ea\label{Example_14.60}
\gll {tong} {tra} {ke} {kampung,} {tra} {ada} {\bluebold{Ø}} {\bluebold{yang}} {jalang} {ke} {kampung}\\ %
 \textsc{1pl}  \textsc{neg}  to  village  \textsc{neg}  exist  {}  \textsc{rel}  walk  to  village\\
\glt ‘we don’t (go) to the village, there is (\bluebold{nobody}) \bluebold{who} goes to the village’ \textstyleExampleSource{[080917-003a-CvEx.0048]}
\z

\ea
\label{Example_14.61}
\ea
\label{Example_14.61a}
\gll  {Speaker-1:} {\bluebold{Nelci}} {\bluebold{itu}} {\bluebold{yang}} {mana?}\\ %
  {}   Nelci  \textsc{d.dist}  \textsc{rel}  where\\
\glt Speaker-1: ‘\bluebold{which} one is \bluebold{that} \bluebold{Nelci}?’
\vspace{10pt}
\ex
\label{Example_14.61b} 
\gll {Speaker-2:}  {\bluebold{Ø}}  \bluebold{yang}  kecil{\Tilde}kecil  {\ldots}  {\bluebold{Ø}}  \bluebold{yang}  rajing{\Tilde}rajing\\
      {} {} \textsc{rel}  \textsc{rdp}{\Tilde}be.small  {} {}    \textsc{rel}  \textsc{rdp}{\Tilde}be.diligent\\
\glt Speaker-2: ‘(\bluebold{the one}) \bluebold{who}’s kind of small {\ldots} (\bluebold{the one}) \bluebold{who}’s very diligent’ \textstyleExampleSource{[081115-001a-Cv.0285-0292]}
\z
\z

The remainder of this section describes the grammatical positions which can be relativized in Papuan Malay. The data in the corpus shows that, in terms of \citegen{Keenan.1977}  ``Accessibility Hierarchy'', Papuan Malay allows relativization on all five positions, namely:

\begin{center}
\textsc{subject} {\textgreater} \textsc{direct} \textsc{object} {\textgreater} \textsc{indirect} \textsc{object} {\textgreater} \textsc{oblique} {\textgreater} \textsc{possessor}
\end{center}

Cross-linguistically, as \citet[297, 298]{Payne.1997} points out, relativization of these positions involves two different “case recoverability strategies” which allow to identify “the role of the referent of the head \isi{noun} \textstyleChItalic{within the relative clause}”, namely the “gap strategy” or “\isi{pronoun} retention”. Both strategies are also found in Papuan Malay. Relativization of subject, direct and oblique object arguments is achieved with the gap strategy, while relativization of obliques and possessors involves \isi{pronoun} retention.

When core arguments are relativized, a gap is left. This gap, signified with ``Ø'', occurs where the relativized \isi{noun} phrase would be situated if it were expressed overtly. Relativization of the subject argument is illustrated in (\ref{Example_14.62}), and of the direct object argument of a \isi{bivalent} \isi{verb}, namely \textitbf{biking} ‘make’, in (\ref{Example_14.63}). The examples in (\ref{Example_14.64}) and (\ref{Example_14.65}) illustrate the relativization of the direct object positions in \isi{double-object} constructions; in both examples the \isi{trivalent} \isi{verb} is \textitbf{kasi} ‘give’. In (\ref{Example_14.64}), the R argument \textitbf{papeda} ‘sagu porridge’ is relativized. In (\ref{Example_14.65}), the T argument \textitbf{Efana ini} ‘this Efana’ is relativized. (Verbal clauses with \isi{bivalent} and \isi{trivalent} verbs are discussed in detail in §\ref{Para_11.1.2} and §\ref{Para_11.1.3}, respectively.)
%


\begin{styleExampleTitle}
Relativization of the subject and direct object positions
\end{styleExampleTitle}
\ea
\label{Example_14.62}
\gll tong  bagi  buat  \textstyleChBlueBold{kitorang}  \textstyleChBlueBold{yang}  {\bluebold{Ø}}  potong  itu  {{\ldots},} buat  \textstyleChBlueBold{sodara{\Tilde}sodara}  \textstyleChBlueBold{yang}  {\bluebold{Ø}}  tinggal  di  kampung\\  
\textstyleChSmallCaps{1pl}  divide  for  \textstyleChSmallCaps{1pl} \textstyleChSmallCaps{rel}  {}  cut  \textstyleChSmallCaps{d.dist} {}  for  \textstyleChSmallCaps{rdp}{\Tilde}sibling  \textstyleChSmallCaps{rel} {}    stay  at  village\\
\glt [About hunting a wild pig:] we divided (the meat) for \textstyleChBlueBold{us who} cut (it) up that day, (and) then for \textstyleChBlueBold{the relatives and friends who} live in the village’ \textstyleExampleSource{[080919-003-NP.0014]}
\z

\ea
\label{Example_14.63}
\gll saya  kas  makang  anjing  deng  \textstyleChBlueBold{papeda}  \textstyleChBlueBold{yang}  sa  pu bini  biking   {\bluebold{Ø}}  malam  untuk  anjing  dorang\\  
\textstyleChSmallCaps{1sg}  give  eat  dog  with  sagu.porridge  \textstyleChSmallCaps{rel}  \textstyleChSmallCaps{1sg}  \textstyleChSmallCaps{poss} wife  make  {}  night  for  dog  \textstyleChSmallCaps{3pl}\\
\glt ‘I fed the dogs with \textstyleChBlueBold{the sagu porridge which} my wife had prepared for the dogs in the evening’ \textstyleExampleSource{[080919-003-NP.0002]}
\z

\ea\label{Example_14.64}
\gll {\bluebold{Fitri}} {\bluebold{yang}} {de} {bapa} {kasi} {\bluebold{Ø}} {ijing} {mo} {ikut} {ke} {kampung}\\ %
 Fitri  \textsc{rel}  \textsc{3sg}  father  give  {}  permission  want  follow  to  village\\
\glt ‘(it was) \bluebold{Fitri whom} her husband gave permission to go with (us) to the village’ (Lit. ‘her (daughter’s) father {\ldots}’) \textstyleExampleSource{[080925-003-Cv.0211]}
\z

\ea\label{Example_14.65}
\gll {\bluebold{Efana}} {\bluebold{ini}} {\bluebold{yang}} {dia\textsubscript{i}} {kas} {dia\textsubscript{j}} {\bluebold{Ø}}\\ %
 Efana  \textsc{d.prox}  \textsc{rel}  \textsc{3sg}  give  \textsc{3sg}  \\
\glt [About an ancestor’s first wife:] ‘(it was) \bluebold{this Efana that} he\textsubscript{i} (‘Aris’) gave (to) him\textsubscript{j} (‘Oten’)’ \textstyleExampleSource{[080922-010a-CvNF.0062]}\footnote{The subscript letters indicate which personal pronouns have which referents.}
\z

Obliques and possessors are relativized via \isi{pronoun} retention. That is, a retained personal \isi{pronoun} explicitly marks the relativized position within the relative clause. This is illustrated with the relativization of an oblique argument in (\ref{Example_14.66}), and of a possessor in (\ref{Example_14.67}). (For a discussion of \isi{interrogative} \textitbf{mana} ‘where, which’ and its adnominal uses, see §\ref{Para_5.8.3}; for details on adnominal possessive constructions, see \chapref{Para_9}.)

\ea
\label{Example_14.66}

\gll kalo  \textstyleChBlueBold{ana}  \textstyleChBlueBold{mana}  \textstyleChBlueBold{yang}  sa  duduk  ceritra  \textstyleChBlueBold{deng}  \textstyleChBlueBold{dia}, itu  ana  itu,  de  hormat  torang\\  
if  child  where  \textstyleChSmallCaps{rel}  \textstyleChSmallCaps{1sg}  sit  tell  with  \textstyleChSmallCaps{3sg} \textstyleChSmallCaps{d.dist}  child  \textstyleChSmallCaps{d.dist}  \textstyleChSmallCaps{3sg}  respect  \textstyleChSmallCaps{1pl}\\
\glt ‘as for \textstyleChBlueBold{which kid with whom} I sit and talk, that is that kid, she respects us’ \textstyleExampleSource{[081115-001a-Cv.0282]}
\z

\ea
\label{Example_14.67}

\gll  itu  \textstyleChBlueBold{kaka}  \textstyleChBlueBold{satu}  \textstyleChBlueBold{itu}  \textstyleChBlueBold{yang}  \textstyleChBlueBold{dia}  \textstyleChBlueBold{punya}  ade  prempuang itu  tinggal  deng  Natanael  tu\\  
\textstyleChSmallCaps{d.dist}  oSb  one  \textstyleChSmallCaps{d.dist}  \textstyleChSmallCaps{rel}  \textstyleChSmallCaps{3sg}  \textstyleChSmallCaps{poss}  ySb  woman \textstyleChSmallCaps{d.dist}  stay  with  Natanael  \textstyleChSmallCaps{d.dist}\\
\glt ‘that is \textstyleChBlueBold{that one older brother whose} younger sister is staying with Natanael’ \textstyleExampleSource{[080922-001a-CvPh.0888]}
\z


\section{Juxtaposition}
\label{Para_14.4}
Juxtaposition is another strategy in Papuan Malay to link constituents, namely same-type constituents, such as \isi{noun} phrases, prepositional phrases, verbs, or clauses.

Juxtaposition of \isi{noun} phrases, as in (\ref{Example_14.68}) to (\ref{Example_14.73}), occurs considerably less often in the corpus than conjoining with a \isi{conjunction}. Most often, three, four or five \isi{noun} phrases are juxtaposed to enumerate entities, while \isi{juxtaposition} of just two \isi{noun} phrases occurs less often. These findings reflect the results of \citegen{Stassen.2000} typological study of \isi{noun} phrase \isi{conjunction} which shows that \isi{juxtaposition} is “a minor strategy” which is often used “in list-like enumerations”.\footnote{According to \citet[7–8]{Stassen.2000}, “the general trend all over the world is that zero-coordination tends to be marginalized into specific functions or is replaced altogether by overt marking strategies”. \citet[351–357]{Mithun.1988} suggests that this development is due to the global increase in bilingualism and in literacy. With respect to bilingualism, \citet[351]{Mithun.1988} observes that “an astonishing number of coordinating conjunctions have been recently borrowed into languages that previously had none”. As for the role of literacy, \citet[356]{Mithun.1988} notes that, whereas in oral language intonation suffices to signal the syntactic structure of juxtaposed constituents, written language requires the overt and “systematic specification of the precise nature of link” to disambiguate syntactic relations.}
%


Papuan Malay combines different prosodic features to indicate the structure of the juxtaposed \isi{noun} phrases: final vowel \isi{lengthening} (orthographically represented by a sequence of three vowels), slight increase in pitch of the stressed syllable (“~\'{~}~”), intonation breaks (“{\textbar}”), non-final intonation pattern with level pitch \mbox{(“ -- ”)}, and end-of-list intonation with fall pitch (“{\textbackslash}”). The enumeration structure in (\ref{Example_14.68}) is indicated with an increase in pitch, and the last item is marked off by the \isi{demonstrative} \textitbf{itu} ‘\textsc{d.dist}’. In (\ref{Example_14.69}), the enumeration is signaled with an increase in pitch as well as intonation breaks; the last item has an end-of-list intonation. In (\ref{Example_14.70}), the structure is marked with a slight increase in pitch and final vowel \isi{lengthening} of the first and third coordinands while the fourth item has an end-of-list intonation. The second and third coordinands form a compact intonation unit, separated from the first and fourth coordinands by intonation breaks. After another intonation break following the fourth coordinand, the fifth coordinand is added as an afterthought.
%

\begin{styleExampleTitle}
Juxtaposition of \isi{noun} phrases
\end{styleExampleTitle}
\ea
\label{Example_14.68}
\gll \textstyleChBlueBold{gúntur}  \textstyleChBlueBold{kílat} \textstyleChBlueBold{hújang} \textstyleChBlueBold{itu}  dia  sambar ruma  itu  sampeee\\  
thunder  lightning  rain  \textstyleChSmallCaps{d.dist}  \textstyleChSmallCaps{3sg}  strike.one.after.the.other house  \textstyleChSmallCaps{d.dist}  reach\\
\glt ‘that \textstyleChBlueBold{thunder, lightning, (and) rain}, it hit one house after the other on and on’ \textstyleExampleSource{[081006-022-CvEx.0007]}
\z


\ea
	\label{Example_14.69}
	
	\gll  \textstyleChBlueBold{káing}  {\textup\textbar}  \textstyleChBlueBold{bántal}  {\textup\textbar}  \textstyleChBlueBold{smúa}  {\textup\textbar}  \textstyleChBlueBold{tíkar}\\  
	cloth   {} pillow  {}  all  {}  plaited.mat\\
	\glt [Listing laundry items:] ‘\textstyleChBlueBold{the cloths, pillows, everything, the plaited mats}’ \textstyleExampleSource{[081025-006-Cv.0057]}
\z
	
\ea%bkm:Ref373852632
	\label{Example_14.70}
	
	\gll   kita  pake  \textstyleChBlueBold{búmbuuu}  {\textup\textbar}  \textstyleChBlueBold{fetsin}  \textstyleChBlueBold{gáraaam}  {\textup\textbar}  \textstyleChBlueBold{sere}  {\textup\textbar}  \textstyleChBlueBold{ricaaa}\\  
	\textstyleChSmallCaps{1pl}  use  spice  {}  MSG  salt  {}  lemon.grass {}   red.pepper\\
	\glt ‘we used \textstyleChBlueBold{spices, flavoring spice, salt, lemongrass, red pepper}’ \textstyleExampleSource{[080919-004-NP.0037]}
\z

\noindent Juxtaposition of prepositional phrases, verbs, or clauses is illustrated in (\ref{Example_14.71}) to (\ref{Example_14.73}). Three prepositional phrases introduced with elative \textitbf{dari} ‘from’ are juxtaposed in (\ref{Example_14.71}), three verbs in (\ref{Example_14.72}), and four clauses in (\ref{Example_14.73}) (for easier recognition the first constituent of each of the linked clauses is bolded).

\begin{styleExampleTitle}
Juxtaposition of prepositional phrases, verbs, or clauses
\end{styleExampleTitle}
\ea
\label{Example_14.71}
\gll baru  sa  punya  bapa  dia  turung  \textstyleChBlueBold{dari}  atas \textstyleChBlueBold{dari}  pedalamang  \textstyleChBlueBold{dari}  Siantoa\\  
and.then  \textstyleChSmallCaps{1sg}  \textstyleChSmallCaps{poss}  father  \textstyleChSmallCaps{3sg}  descend  from  top from  interior  from  Siantoa\\
\glt ‘and then my father came down \textstyleChBlueBold{from} the hills, \textstyleChBlueBold{from} the interior, \textstyleChBlueBold{from} Siantoa’ \textstyleExampleSource{[080927-009-CvNP.0010]}
\z

\ea\label{Example_14.72}
\gll {kepala} {desa} {mantang} {Arbais} {ada} {\bluebold{duduk}} {\bluebold{ceritra}} {\bluebold{minum}}\\ %
 head  village  former  Arbais  exist  sit  tell  drink\\
\glt ‘the former mayor of Arbais was \bluebold{sitting} (there and) \bluebold{talking} (and) \bluebold{drinking}’ \textstyleExampleSource{[081011-024-Cv.0135]}
\z

\ea
\label{Example_14.73}

\gll \textstyleChBlueBold{Oktofernus}  tra  makang,  \textstyleChBlueBold{Mateus}  tra  makang,  \textstyleChBlueBold{Wili}  tra makang,  e,  \textstyleChBlueBold{paytua}  tra  makang\\  
Oktofernus  \textstyleChSmallCaps{neg}  eat  Mateus  \textstyleChSmallCaps{neg}  eat  Wili  \textstyleChSmallCaps{neg} eat  uh  husband  \textstyleChSmallCaps{neg}  eat\\
\glt ‘\textstyleChBlueBold{Oktofernus} didn’t eat, \textstyleChBlueBold{Mateus} didn’t eat, \textstyleChBlueBold{Wili} didn’t eat, uh, (my) \textstyleChBlueBold{husband} didn’t eat’ \textstyleExampleSource{[080921-003-CvNP.0005]}
\z

\section{Summary}
\label{Para_14.5}
Papuan Malay conjunctions typically conjoin same-type constituents. Most of them combine clauses with clauses. Only two link different-type constituents, such as verbs with clauses. Typically, the conjunctions occur at the left periphery of the constituent they mark.



The 21 conjunctions linking same-type constituents are divided into six groups according to the semantic relations they signal:
%

\begin{enumerate}
\item Addition: \textitbf{dengang} ‘with’, \textitbf{dang} ‘and’, \textitbf{sama} ‘to’.
\item Alternative: \textitbf{ato} ‘or’ and \textitbf{ka} ‘or’.
\item Time and/or \isi{condition}: \textitbf{trus} ‘next’, \textitbf{baru} ‘and then’, \textitbf{sampe} ‘until’, \textitbf{seblum} ‘before’, and \textitbf{kalo} ‘when, if’.
\item Consequence: \textitbf{jadi} ‘so, since’, \textitbf{supaya} ‘so that’, \textitbf{untuk} ‘for’, \textitbf{karna} ‘because’, and \textitbf{gara-gara} ‘because’; time-marking \textitbf{sampe} also has result-mark\-ing function in the sense of ``with the result that''.
\item Contrast: \textitbf{tapi} ‘but’, \textitbf{habis} ‘after all’, \textitbf{padahal} ‘but actually’, and \textitbf{biar} ‘although’; time-marking \textitbf{baru} also marks contrast in the sense of ``after all''.
\item Similarity: \textitbf{sperti} ‘similar to’ and \textitbf{kaya} ‘like’.
\end{enumerate}

A substantial number of the conjunctions have dual \isi{word class membership}, two have trial class membership. More specifically, seven conjunctions are also used as verbs, namely \textitbf{baru} ‘and then, after all’, \textitbf{biar} ‘although’, \textitbf{habis} ‘after all’, \textitbf{jadi} ‘so, since’, \textitbf{sama} ‘to’, \textitbf{sampe} ‘until’, and \textitbf{trus} ‘next’ (see §\ref{Para_5.3}). Six conjunctions are also used as prepositions, namely \textitbf{dengang} ‘with’, \textitbf{kaya} ‘like’, \textitbf{sama} ‘to’, \textitbf{sampe} ‘until’, \textitbf{sperti} ‘similar to’, and \textitbf{untuk} ‘for’ (see §\ref{Para_5.11} and \chapref{Para_10}). One \isi{conjunction} is also used as an ad\isi{verb}, namely \textitbf{baru} ‘and.then’. Besides, alternative-marking \textitbf{ka} ‘or’ is also used to mark \isi{interrogative} clauses (see §\ref{Para_13.2.3}). Variation in \isi{word class membership} is discussed in §\ref{Para_5.14}.



The main features of the conjunctions are summarized in two tables. \tabref{Table_14.1} lists the conjunctions and the different types of constituents they link. For those linking more than one constituent type, the primary type is underlined. Empty cells signal unattested constituent combinations.
%

\begin{table}
\caption{\label{Table_14.1}Conjunctions linking same-type constituents and the constituents they combine}
\begin{tabular}{llcccc}
\lsptoprule
\multicolumn{2}{c}{\textsc{conjunctions}} & \textsc{cl-cl} & \textsc{np-np} & \textsc{pp-pp} & \arraybslash \textsc{vp-vp}\\
\midrule
\multicolumn{2}{l}{Addition} &  &  &  & \\
\midrule
& \textitbf{dengang} ‘with’ &  & \textstyleChUnderl{X} &  & \arraybslash X\\
& \textitbf{dang} ‘and’ & \textstyleChUnderl{X} & X &  & \arraybslash X\\
& \textitbf{sama} ‘to’ &  & \textstyleChUnderl{X} &  & \arraybslash X\\
\midrule
\multicolumn{2}{l}{Alternative} &  &  &  & \\
\midrule
& \textitbf{ato} ‘or’ & \textstyleChUnderl{X} & X & X & \arraybslash X\\
& \textitbf{ka} ‘or’ & X & \textstyleChUnderl{X} & X & \\
\midrule
\multicolumn{2}{l}{Time and Condition} &  &  &  & \\
\midrule
& \textitbf{trus} ‘next’ & \textstyleChUnderl{X} & X & X & \\
& \textitbf{baru} ‘and then’ & X &  &  & \\
& \textitbf{sampe} ‘until’ & X &  &  & \\
& \textitbf{seblum} ‘before’ & X &  &  & \\
& \textitbf{kalo} ‘when, if’ & X &  &  & \\
\midrule
\multicolumn{2}{l}{Consequence} &  &  &  & \\
\midrule
& \textitbf{jadi} ‘so, since’ & X &  &  & \\
& \textitbf{supaya} ‘so that’ & X &  &  & \\
& \textitbf{untuk} ‘for’ & X &  &  & \\
& \textitbf{sampe} ‘with the result that’ & X &  &  & \\
& \textitbf{karna} ‘because’ & X &  &  & \\
& \textitbf{gara-gara} ‘because’ & X &  &  & \\
\midrule
\multicolumn{2}{l}{Contrast} &  &  &  & \\
\midrule
& \textitbf{tapi} ‘but’ & X &  &  & \\
& \textitbf{habis} ‘after all’ & X &  &  & \\
& \textitbf{baru} ‘after all’ & X &  &  & \\
& \textitbf{padahal} ‘but actually’ & X &  &  & \\
& \textitbf{biar} ‘although’ & X &  &  & \\
\midrule
\multicolumn{2}{l}{Similarity} &  &  &  & \\
\midrule
& \textitbf{sperti} ‘similar to’ & X &  &  & \\
& \textitbf{kaya} ‘like' & X &  &  & \\
\lspbottomrule
\end{tabular}
\end{table}


\tabref{Table_14.2} gives an overview of the positions which the conjunctions take within the clause, and the position the clause marked with a \isi{conjunction} takes vis-à-vis the unmarked clause. Almost all conjunctions occur in clause-initial position, while only two occur in clause-final position. Typically, the clause marked with a \isi{conjunction} follows the unmarked clause; only a few conjunctions mark clauses which precede the unmarked clause. Two of the conjunctions have two functions each, which belong to different semantic groupings, namely \textitbf{baru} ‘and then, after all’ and \textitbf{sampe} ‘until, with the result that’. Both conjunctions are listed in each of the respective groupings.


\begin{table}[t]
\caption{\label{Table_14.2}\label{bkm:Ref356462326}Conjunctions linking same-type constituents and their positions}
\begin{tabular}{llccc}
\lsptoprule
\multicolumn{2}{c}{\textsc{conjunctions}} & \textsc{cl} [\textsc{cnj} \textsc{cl}] & [\textsc{cnj} \textsc{cl}] \textsc{cl} & \arraybslash \textsc{cl} [\textsc{cl} \textsc{cnj}]\\
\midrule
\multicolumn{2}{l}{Addition} &  &  & \\
\midrule
& \textitbf{dengang} ‘with’ & X &  & \\
& \textitbf{dang} ‘and’ & X &  & \\
& \textitbf{sama} ‘to’ & X &  & \\
\midrule
\multicolumn{2}{l}{Alternative} &  &  & \\
\midrule
& \textitbf{ato} ‘or’ & X &  & \\
& \textitbf{ka} ‘or’ & X &  & \\
\midrule
\multicolumn{2}{l}{Time and Condition} &  &  & \\
\midrule
& \textitbf{trus} ‘next’ & X &  & \\
& \textitbf{baru} ‘and then’ & X &  & \\
& \textitbf{sampe} ‘until’ & X &  & \\
& \textitbf{seblum} ‘before’ & X & X & \\
& \textitbf{kalo} ‘when, if’ &  & X & \\
\midrule
\multicolumn{2}{l}{Consequence} &  &  & \\
\midrule
& \textitbf{jadi} ‘so, since’ & X &  & \arraybslash X\\
& \textitbf{supaya} ‘so that’ & X &  & \\
& \textitbf{untuk} ‘for’ & X &  & \\
& \textitbf{sampe} ‘with the result that’ & X &  & \\
& \textitbf{karna} ‘because’ & X & X & \\
& \textitbf{gara-gara} ‘because’ & X &  & \\
\midrule
\multicolumn{2}{l}{Contrast} &  &  & \\
\midrule
& \textitbf{tapi} ‘but’ & X &  & \\
& \textitbf{habis} ‘after all’ & X &  & \\
& \textitbf{baru} ‘after all’ &  &  & \arraybslash X\\
& \textitbf{padahal} ‘but actually’ & X &  & \\
& \textitbf{biar} ‘although’ & X & X & \\
\midrule
\multicolumn{2}{l}{Similarity} &  &  & \\
\midrule
& \textitbf{sperti} ‘similar to’ & X &  & \\
& \textitbf{kaya} ‘like' & X &  & \\
\lspbottomrule
\end{tabular}
\end{table}

\largerpage[2]

The conjunctions combining different-type constituents discussed in this chapter are the \isi{complementizer} \textitbf{bahwa} ‘that’ and the \isi{relativizer} \textitbf{yang} ‘\textsc{rel}’. Complementizer \textitbf{bahwa} ‘that’ links a clause to a \isi{bivalent} \isi{verb}, while \isi{relativizer} \textitbf{yang} ‘\textsc{rel}’ integrates a relative clause within a \isi{noun} phrase.
