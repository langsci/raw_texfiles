\chapter{Texts}
\label{Para_B}
This appendix presents a sample of twelve texts. Included are three spontaneous conversations, one spontaneous narrative, two elicited narratives, two expositories, two hortatories, and two elicited jokes. For each text the following meta data is provided: the file name, the text type, the interlocutors, and the length (in minutes). For additional information see also §\ref{Para_1.11} and Appendix \ref{Para_C}.
%\end{styleBodyxvafter}

\section{Conversation: Playing volleyball; morning chores}
\label{Para_B.1}
\begin{tabular}{ll}
File name: & 081023-001-Cv\\
Text type: &  Conversation, spontaneous\\
Interlocutors: &  1 younger male, 2 younger females\\
Length (min.): &  4:52\\

\end{tabular}\\

\renewcommand{\Tilde}{\textasciitilde}
\ea
\upshape{Oten:}
\gll {\upshape\textsc{[up]}} {blang,} {kam} {dari} {mana?} {trus} {\upshape\textsc{[up]}} {tong} {dari} {Arbais,} {kam} {pu} {nama} {siapa{\textasciitilde}siapa?} {Herman} {de} {bilang}  de pu {nama,} {pace} {de} {tulis} {di} {kertas,} {su,} {situ} {de} ada, {de} {su} {biking} {daftar}\\
     {} {say} {\textsc{2pl}} {from} {where} {next} {} {\textsc{1pl}} {from}                  {Arbais} {\textsc{2pl}} {\textsc{poss}} {name} {\textsc{rdp}{\Tilde}who} {Herman} {\textsc{3sg}} {say} \textsc{3sg}  \textsc{poss} {name} {man} {\textsc{3sg}} {write} {at} {paper} {already} {\textsc{l.med}} {\textsc{3sg}} exist {\textsc{3sg}} {already} {make} {list}\\
     \glt 
Oten: [\textsc{up}] said, ‘where are you from?’, then [\textsc{up}], ‘we are from Arbais’, ‘what are your names?’ Herman gave his name, the man wrote (it) on a paper, that’s it, there it was!, he (the man) had already made a list
\z

\ea
\gll {su} {biking} {daftar,} {pertama} {di} {atas} {sa} {liat} {nama}  tu      Lukas {ini} {T.,} {bencong} {satu,} {Lukas} {T.} {dia,} {trus}          {suda} {spulu,} {pas} {tong,} {trus} {tamba} {kaka} {dari}  {Mamberamo} {satu,} {Agus,} {Agus} {Y}\\
{already} {make} {list} {first} {at} {top} {\textsc{1sg}} {see} {name}  \textsc{d.dist} Lukas {\textsc{d.prox}} {T.} {transvestite} {one} {Lukas} {T.} {\textsc{3sg}}  next {already} {one-tens} {precisely} {\textsc{1pl}} {next} {add} {oSb} from {Mamberamo} {one} {Agus} {Agus} {Y.} \\
\glt 
(he) had already made a list, the first one on top, I saw that name, Lukas, what’s-his-name, T., a certain transvestite, Lukas T., then (there were) already ten (names on that list), at that moment we, then add a certain older brother from (the) Mambramo (area), Agus, Agus Y.
\z


\ea
\gll {tadi}  di {pasar}  sa  ada  pegang  tangang  deng  dia, de {pake}  baju {mera}\\
{earlier}  at {market}  \textsc{1sg}  exist  hold  hand  with  \textsc{3sg}  \textsc{3sg} {use}  shirt {be.red}\\
\glt 
earlier in the market I was holding hands with him, he was wearing a red shirt
\z

\ea
\gll {Klara:} o, {\upshape [\textsc{up}]}\\
{} oh!  \\
\glt 
Klara: oh, [\textsc{up}]
\z

\ea
\gll Otend: badang besar{\Tilde}besar\\
{} body \textsc{rdp}{\Tilde}be.big\\
\glt 
Oten: (his) body is very big
\z

\ea
\gll Klara: ((laughter))\\
Klara: ((laughter))\\
\z

\ea
\gll Oten: {pace} {de} {tulis} {tong} {pu} {nama} {selesay,} {de} {bilang,}                        {besok,} {jam,} {seblum} {jam} {tiga} {kamu} {su} {ada}  di                     {sini} {untuk} {latiang,} {trus} {sa} {tanya,} {tong} {latiang} {ini}                                        mo {ke} {mana}?\\
{} {man} {\textsc{3sg}} {write} {\textsc{1pl}} {\textsc{poss}} {name} {finish} {\textsc{3sg}} {say}  {tomorrow} {hour} {before} {hour} {three} {\textsc{2pl}} {already} {exist}  at  {\textsc{l.prox}} {for} {practice} {next} {\textsc{1sg}} {ask} {\textsc{1pl}} {practice} {\textsc{d.prox}}   want {to} {where}\\    
\glt 
Oten: after the man had written down our names, he said, ‘tomorrow, o’clock, before three o’clock you’ll already be here to practice’, then I asked, ‘we (do) our very practicing to go where?’
\z

\ea
\gll   de    {blang,}    {a,}    {latiang}    {saja,}    {katanya}    {bupati}    {bilang,}   ada\\
  \textsc{3sg}    {say}    {ah!}    {practice}    {just}    {it.is.being.said}    {regent}    {say}   exist\\
\gll   {mo}    {pergi}    {maing}    {di}    {ini}    {Serui}    {ka}    {itu}    {yang}    {de}    {ada}\\
   {want}    {go}    {play}    {at}    {\textsc{d.prox}}    {Serui}    {or}    {\textsc{d.dist}}    {\textsc{rel}}    {\textsc{3sg}}    {exist}\\
\gll   {cari}    {ana{\Tilde}ana}    {untuk}    {pergi}    {maing,}    {suda,}    {baru}    {sa}    {bilang}\\
   {search}    {\textsc{rdp}{\Tilde}child}    {for}    {go}    {play}    {already}    {and.then}    {\textsc{1sg}}    {say}\\
\gll   {masi}    {bisa}    {ada}    {yang}    {masuk}    {ato}    {su}    {tra}    {ada?}\\
   {still}    {be.able}    {exist}    {\textsc{rel}}    {enter}    {or}    {already}    {\textsc{neg}}    {exist}\\
\glt
he said, ‘ah, just practice, it’s being said that the regent says that we are going to go to play maybe on, what’s-its-name, Serui (Island), that’s why he’s looking for young people to go play’, alright, and then I said, ‘can one still be included (on that list) or already not any longer?’
\z

\ea
\gll   de    {blang,}    {kalo}    {ada}    {yang}    {mo}    {masuk,}    {bisa,}    {trus}\\
  \textsc{3sg}    {say}    {if}    {exist}    {\textsc{rel}}    {want}    {enter}    {be.able}    {next}\\
\gll   {kaka}    {wa,}    {yang}    {nanti}    {kasi}    {latiang}    {itu}    {kaka}   polisi\\
   {oSb}    {\textsc{spm}}    {\textsc{rel}}    {very.soon}    {give}    {practice}    {\textsc{d.dist}}    {oSb}   police\\
\gll   {yang}    {baru{\Tilde}baru}    {deng}    {Hurki}    {jalang}    {ke}    {Jakarta}    {sana,}    {ka}\\
   {\textsc{rel}}    {\textsc{rdp}{\Tilde}recently}    {with}    {Hurki}    {walk}    {to}    {Jakarta}    {\textsc{l.dist}}    {oSb}\\
\gll   {Sarles,}    {ka}    {Sarles}    {juga,}    {de}    {pu}    {maim}    {pisow}\\
   {Sarles}    {oSb}    {Sarles}    {also}    {\textsc{3sg}}    {\textsc{poss}}    {play}    {knife}\\
\glt
he said, ‘if there is someone who wants to be included, (he/she) can (be included), then, older brother [\textsc{spm}], (the one) who will give the training, what’s-his-name, the older brother (who’s a) police (officer) who just now went to Jakarta over there together with Hurki, older brother Sarles, older brother Sarles also, he has a fast and smart way of playing’ (Lit. ‘the knife playing of’)
\z

\ea
\gll   Klara:   bola   fol\\
 {}    ball   volleyball\\
\glt
Klara: volleyball
\z

\ea
\gll   Oten:   yo,   bola   foli   ini,   tanta   Nelci\\
{}     yes   ball   volleyball   \textsc{d.prox}   aunt   Nelci\\
\glt
Oten: yes, this volleyball, [addressing Nelci] aunt Nelci
\z

\ea
\gll   Klara:   yo,   net   laki{\Tilde}laki,   tong   yang   bli,   yang   sebla\\
    {} yes   (sport.)net   \textsc{rdp}{\Tilde}husband   \textsc{1pl}   \textsc{rel}   buy   \textsc{rel}   side\\
\gll  darat   \textup{[\textsc{up}]}\\
  land    {}\\
\glt
Klara: yes, the (volleyball) net for men, (it was) us who (bought it), (the one) which is off the beach [\textsc{up}]
\z
%\todo[inline]{Check punctuation!}
\ea
\gll   Oten:   yang   sebla,   yo   sebla,   di   pinggir   kali   tu\\
{}     \textsc{rel}   side   yes   side   at   border   river   \textsc{d.dist}\\
\glt
Oten: (the one) which is off (the beach), yes, off (the beach), on the banks of that river
\z

\ea
\gll   Klara:    {itu}    {kalo}    {memang}    {bola{\Tilde}bola}    {tinggi}   kalo   smes   itu\\
 {}     {\textsc{d.dist}}    {if}    {indeed}    {\textsc{rdp}{\Tilde}ball}    {be.high}   if   smash   \textsc{d.dist}\\
\gll   {memang}    {masuk}    {kali,}    {bola{\Tilde}bola}    {terlalu}    \textup{[\textsc{up}]}\\
   {indeed}    {enter}    {river}    {\textsc{rdp}{\Tilde}ball}    {too}    {}\\
\glt
Klara: so, if indeed the balls are high, if (one) smashes them, indeed they go into the river, the balls are too [\textsc{up}]
\z
%\todo[inline]{Check punctuation!}
\ea
\gll   Oten:   sa   lompat   itu   frey,   tangang   lewat\\
 {}    \textsc{1sg}   jump   \textsc{d.dist}   be.free   hand   pass.by\\
\glt
Oten: I jump high, free (of the net), (my) hands surpass (the net)
\z

\ea
\gll   MY:   {\upshape\textsc{[up]}}\\
MY: [\textsc{up}]\\
\z

\ea
\gll   Klara:    {kemaring}    {saya,}    {Herman,}    {Maa,}    {Markus,}   siapa   ni,\\
 {}     {yesterday}    {\textsc{1sg}}    {Herman}    {\textsc{tru-}Markus}    {Markus}   who   \textsc{d.prox}\\
\gll    {Nofita,}   sa    {bilang}    {begini,}    {sa}   juga    {naik}    {frey}\\
   {Nofita}   \textsc{1sg}    {say}    {like.this}    {\textsc{1sg}}   also    {ascend}    {be.free}\\
\glt
Klara: yesterday, I, Herman, Markus[\textsc{tru}], Markus, (and) who-is-it, Nofita, I said like this, ‘I also jump free (of the net)’
\z

\ea
\gll   Oten:   Nofita,   Nofita   pu   bagi{\Tilde}bagi   tu\\
  {}   Nofita   Nofita   \textsc{poss}   \textsc{rdp}{\Tilde}divide   \textsc{d.dist}\\
\glt
Oten: Nofita, Nofita tosses well (Lit. ‘Nofita’s dividing’)
\z

\ea
\gll   Nelci:   kalo   Nofita   kena   itu   tubir\\
 {}    if   Nofita   hit   \textsc{d.dist}   steep\\
\glt
Nelci: whenever Nofita hits (the ball, it comes down in a) steep (angle)
\z

\ea
\gll    {Klara:}    {adu,}    {tong}    {maing}    {tu}    {hancur,}    {tong}    {maing}    {net}\\
   {}    {oh.no!}    {\textsc{1pl}}    {play}    {\textsc{d.dist}}    {be.shattered}    {\textsc{1pl}}    {play}    {(sport.)net}\\
\gll  sebla    {baru}    {ada,}    {a,}    {sebla}    {darat}    {tapi}    {dong}    {bilang}    {begini,}   sebla\\
  side    {and.then}    {exist}    {ah!}    {side}    {land}    {but}    {\textsc{3pl}}    {say}    {like.this}   side\\
\gll   {net}    {darat}    {tu}    {tinggi{\Tilde}tinggi}    {to?,}    {tinggi}    {itu}    {suda}\\
   {(sport.)net}    {land}    {\textsc{d.dist}}    {\textsc{rdp}{\Tilde}be.high}    {right?}    {be.high}    {\textsc{d.dist}}    {already}\\
\glt
Klara: oh no!, we did our very playing poorly, we played the net off (the beach), and then there is (one), ah, off the beach, but they talked like this, the net off the beach is very high, right?, its height is fixed
\z
%\todo[inline]{Check punctuation!}
\ea
\gll   Oten:   yo,   de\\
  {}   yes   \textsc{3sg}\\
\glt
Oten: yes, it
\z

\ea
\gll   Klara:   de   tinggi   itu   suda\\
 {}    \textsc{3sg}   be.high   \textsc{d.dist}   already\\
\glt
Klara: its height is fixed
\z

\ea
\gll   Oten:   de   pu\\
 {}  \textsc{3sg}   \textsc{poss}\\
\glt
Oten: its
\z

\ea
\gll   Klara:   pita   di   atas\\
  {}   ribbon.of.volleyball.net   at   top\\
\glt
Klara: the upper ribbon of the volleyball net
\z

\ea
\gll   Oten:   yang   pita   di   bawa   itu\\
{}     \textsc{rel}   ribbon.of.volleyball.net   at   bottom   \textsc{d.dist}\\
\glt
Oten: (its) lower ribbon
\z

\ea
\gll   Klara:   batas\\
{}     border\\
\glt
Klara: (its) height
\z

\ea
\gll    {Oten:}    {sa}   berdiri   pas   batas   ini,   angkat   tangang\\
   {}    {\textsc{1sg}}   stand   be.exact   border   \textsc{d.prox}   lift   hand\\
\gll  tapi    {\upshape\textsc{[up]}}    {lewat}\\
  but    {}    {pass.by}\\
\glt
Oten: (when) I’m standing the lower ribbon is exactly on this height, (when I) lift (my) hand [\textsc{up}]
\z

\ea
\gll    {Klara:}    {makanya}    {kalo}    {bola}    {su}    {mo}    {turung,}    {jang}\\
   {}    {for.that.reason}    {if}    {ball}    {already}    {want}    {descend}    {\textsc{neg.imp}}\\
\gll  ko    {lompat,}    {bola}    {tinggi}    {tu}    {yang}    {ko}    {lompat}   deng\\
  \textsc{2sg}    {jump}    {ball}    {be.high}    {\textsc{d.dist}}    {\textsc{rel}}    {\textsc{2sg}}    {jump}   with\\
\gll   {akang}    {to?,}    {karna}    {bola}    {turung}    {tra}    {akang}    {sampe}\\
   {it[SI]}    {right?}    {because}    {ball}    {descend}    {\textsc{neg}}    {will[SI]}    {reach}\\
\glt
Klara: so, when the ball is already coming down, don’t jump, (when) the ball is really high, you jump for it, right?, because the ball (that’s) coming down won’t hit the ground
\z

\ea
\gll   Oten:   tadi    {tong}   cara   maing   juga,   bola{\Tilde}bola   pul,\\
{}     earlier    {\textsc{1pl}}   manner   play   also   \textsc{rdp}{\Tilde}ball   pool\\
\gll  kejar,    {tangang}    {kembali}\\
  chase    {hand}    {return}\\
\glt
Oten: earlier the way we played (was) also (good in some way), we played beautifully, chasing and passing (the ball)
\z

\ea
\gll    {Klara:}    {memang,}    {baru}    {net}   de    {spang}   itu,    {mantap}\\
   {}    {indeed}    {and.then}    {(sport.)net}   \textsc{3sg}    {spank}   \textsc{d.dist}    {be.good}\\
\gll  {skali}    {to?,}    {jadi}   tong    {kemaring}    {maing}   deng    {net}   itu\\
   {very}    {right?}    {so}   \textsc{1pl}    {yesterday}    {play}   with    {(sport.)net}   \textsc{d.dist}\\
\gll  dua    {kali}    {saja}\\
  two    {time}    {just}\\
\glt
Klara: indeed, and then the net was really tight, (it was) very good, right?, so yesterday we played at that net only twice
\z
%\todo[inline]{Check punctuation!}
\ea
\gll   Wili:   sa   yang   \textup{[\textsc{up}]}\\
 {}    \textsc{1sg}   \textsc{rel}   \\
\glt
Wili: it was me who [\textsc{up}]
\z

\ea
\gll   Nelci:   siapa   yang   ganggu   \textup{[\textsc{up}]}\\
{}     who   \textsc{rel}   disturb   \\
\glt
Nelci: who was it who disturbed [\textsc{up}]
\z

\ea
\gll   Klara:   tong   maing   \textup{[\textsc{up}]}\\
  {}   \textsc{1pl}   play   \\
\glt
Klara: we were playing [\textsc{up}]
\z

\ea
\gll    {Oten:}    {lo,}    {de}    {yang}    {gara,}   ko    {jang}   mo   bilang\\
   {}    {right![SI]}    {\textsc{3sg}}    {\textsc{rel}}    {irritate}   \textsc{2sg}    {\textsc{neg.imp}}   want   say\\
\gll saya    {laing,}    {ko}    {apa?}    {siapa?}   siapa    {lu,}   siapa    {gua?}\\
  \textsc{1sg}    {again}    {\textsc{2sg}}    {what}    {who}   who    {\textsc{2sg}[JI]}   who    {\textsc{1sg}[JI]}\\
\glt
Oten: right!, it was him who irritated (you), don’t you accuse me again, who in the world do you think you are?, who are you?, who am I?\footnote{The use of the second singular person serves as a rhetorical figure of speech (“\isi{apostrophe}”) and refers to the absent person who irritated the players (see also §\ref{Para_6.2.1.1.3}).}
\z
%\todo[inline]{Check punctuation!}
\ea
\gll   Klara:    {net}    {sebla}    {kitong,}   itu   yang   langsung   tong\\
 {}     {(sport.)net}    {side}    {\textsc{1pl}}   \textsc{d.dist}   \textsc{rel}   immediately   \textsc{1pl}\\
\gll    {turung}   maing    {di}    {net}    {ini}\\
   {descend}   play    {at}    {(sport.)net}    {\textsc{d.prox}}\\
\glt
Klara: the net on the other side, we, that’s where we immediately went to play at this net
\z
%\todo[inline]{Check punctuation!}
\ea
\gll   Oten:   o!\\
  {}   oh!\\
\glt
Oten: oh!
\z

\ea
\gll   Klara:   net   prempuang   to?\\
 {}    (sport.)net   woman   right?\\
\glt
Klara: the women’s net, right?
\z

\ea
\gll   Oten:   Wili   ko   jang   gara{\Tilde}gara   tanta   dia   itu!\\
 {}  Wili   \textsc{2sg}   \textsc{neg.imp}   \textsc{rdp}{\Tilde}irritate   aunt   \textsc{3sg}   \textsc{d.dist}\\
\glt
Oten: you Wili don’t irritate that aunt!
\z

\ea
\gll   Wili:   mm-mm\\
{}     mhm\\
\glt
Wili: mhm
\z

\ea
\gll   Klara:    {Wili}    {ko}    {masuk}   suda,   ko   tadi   dengar   itu\\
{}      {Wili}    {\textsc{2sg}}    {enter}   already   \textsc{2sg}   earlier   hear   \textsc{d.dist}\\
\gll   {burung}    {itu}    {ka}    {tida?}\\
   {bird}    {\textsc{d.dist}}    {or}    {\textsc{neg}}\\
\glt
Klara: you Wili go inside!, earlier you heard, what’s-its-name, that bird or not?
\z

\ea
\gll    {Oten:}    {o,}   itu   klawar,   de   makang   ini,   mangga\\
   {}    {oh!}   \textsc{d.dist}   cave.bat   \textsc{3sg}   eat   \textsc{d.prox}   mango\\
\gll  ka,    {apa,}    {ketapang}\\
  or    {what}    {tropical-almond}\\
\glt
Oten: oh, that was a bat, it was eating, what’s-its-name, maybe mangos, what-is-it, tropical-almonds
\z

\ea
\gll   Klara:    {tida,}    {ana{\Tilde}ana}    {kecil}    {kaya}    {begini,}    {ana{\Tilde}ana}   kecil\\
 {}     {\textsc{neg}}    {\textsc{rdp}{\Tilde}child}    {be.small}    {like}    {like.this}    {\textsc{rdp}{\Tilde}child}   be.small\\
\gll   {nanti}    {de}    {bangung}    {terlambat,}    {lebi}    {bagus}   ko    {masuk}\\
   {very.soon}    {\textsc{3sg}}    {wake.up}    {be.late}    {more}    {be.good}   \textsc{2sg}    {enter}\\
 \gll tidor    {sana}    {suda}\\
  sleep    {\textsc{l.dist}}    {already}\\
\glt
Klara: no, young children like him, young children, later he’ll wake up too late, it’s better you go inside and just sleep over there
\z
%\todo[inline]{Check punctuation!}
\ea
\gll   Oten:   baru   ko?\\
 {}    and.then   \textsc{2sg}\\
\glt
Oten: and (what about) you?
\z

\ea
\gll   Klara:   ko    {jang}    {bergabung,}   sa   masi   bisa   pu\\
{}  \textsc{2sg}    {\textsc{neg.imp}}    {join}   \textsc{1sg}   still   be.able   \textsc{poss}\\
\gll {kesadarang}   sa    {bangung}    {tempo}\\
   {awareness}   \textsc{1sg}    {wake.up}    {quick}\\
\glt
Klara: don’t stay with us any longer, I have enough mindfulness, I wake up early
\z
%\todo[inline]{Check punctuation!}
\ea
\gll   Oten:   dari   tadi   siang   sa   yang   kasi   bangung   ko\\
{}     from   earlier   midday   \textsc{1sg}   \textsc{rel}   give   wake.up   \textsc{2sg}\\
\glt
Oten: earlier this noon, (it was) me who woke you up
\z

\ea
\gll   Nelci:   i,   malam   de   bangung,   e   yo   hampir\\
{}     ugh!   night   \textsc{3sg}   wake.up   uh   yes   almost\\
\glt
Nelci: ugh!, (last) night she got up, uh yes, (it was) almost
\z

\ea
\gll   Oten:   lo   hampir   siang   sa   yang   bangung   lebi   cepat\\
 {}    right![SI]   almost   day   \textsc{1sg}   \textsc{rel}   wake.up   more   be.fast\\
\glt
Oten: right, (it was) almost daylight, (it was) me who woke up earlier
\z

\ea
\gll   Klara:   em?   e?\\
{}     uh   uh\\
\glt
Klara: uh, uh
\z

\ea
\gll   Oten:   knapa   ka?\\
 {}    why   or\\
\glt
Oten: what happened?
\z

\ea
\label{Example_B.50}
\gll   Nelci:   sa   bangung   stenga   empat,   stenga   lima\\
{}   \textsc{1sg}   wake.up   half   four   half   five\\
\glt
Nelci: I got up at half past three, half past four
\z

\ea
\gll    {Klara:}    {sa}    {bangung,}    {sa}    {kluar}    {pas}    {ana}    {ini,}    {Nusa}\\
   {}    {\textsc{1sg}}    {wake.up}    {\textsc{1sg}}    {go.out}    {precisely}    {child}    {\textsc{d.prox}}    {Nusa}\\
\gll  juga    {kluar}    {dari}    {dalam,}    {de}    {kas}    {bangung}    {ana}    {ini,}   dong\\
  also    {go.out}    {from}    {inside}    {\textsc{3sg}}    {give}    {wake.up}    {child}    {\textsc{d.prox}}   \textsc{3pl}\\
\gll {dua}    {kluar}    {cuci}    {piring,}    {dong}    {dua}    {biking}    {te}    {pagi,}    {memang}\\
   {two}    {go.out}    {wash}    {plate}    {\textsc{3pl}}    {two}    {make}    {tea}    {morning}    {indeed}\\
\gll {hampir}    {siang}    {tu}    {dong}    {dua}    {yang}    {kluar}    {bangung}    {pagi}\\
   {almost}    {day}    {\textsc{d.dist}}    {\textsc{3pl}}    {two}    {\textsc{rel}}    {go.out}    {wake.up}    {morning}\\
\glt Klara: I got up, I went outside, in that moment this kid here, Nusa came outside, she woke up this kid,\footnote{\textitbf{Klara} refers to \textitbf{Nelci} (see (\ref{Example_B.50}) and (\ref{Example_B.53})).} the two of them went outside (and) washed the plates, the two of them made the morning tea, indeed (it was) almost daylight, (it was) the two of them who came outside and woke up the morning
\z

\ea
\gll   Oktofina:   e,   mama   bilang   masuk\\
 {}   hey!   mother   say   enter\\
\glt
Oktofina [addressing Wili]: hey, mother said (you should) go inside
\z

\ea
\label{Example_B.53}
\gll   Nelci:   Nusa   cuci   piring,   sa   goreng   nasi\\
  {}   Nusa   wash   plate   \textsc{1sg}   fry   cooked.rice\\
\glt
Nelci: Nusa washed the plates (and) I fried the cooked rice
\z

\ea
\gll   Wili:   de   tipu,   sa   tadi   liat   dia   tida   ada   di   dalam\\
  {}   \textsc{3sg}   cheat   \textsc{1sg}   earlier   see   \textsc{3sg}   \textsc{neg}   exist   at   inside\\
\glt
Wili: she’s deceiving (me), earlier I saw (that) she (mother) wasn’t inside
\z

\ea
\gll   Oktofina:   a,   betul,   ma   bilang\\
 {}    ah!   be.true   mother   say\\
\glt
Oktofina: ah, it’s true, mother said
\z

\ea
\gll   Oten:   siapa   yang   bla   kayu?\\
 {}    who   \textsc{rel}   split   wood\\
\glt
Oten: who (was it) who split (the fire)wood?
\z

\ea
\gll   Klara:   a,   omong   kosong,   ko   masuk   tidor   sana   suda\\
 {}    ah!   gossip[SI]   empty   \textsc{2sg}   enter   sleep   \textsc{l.dist}   already\\
\glt
Klara [addressing Wili]: ah, nonsense, you just go inside (and) sleep over there
\z

\ea
\gll   Nelci:   yo,   itu   om   siapa   ni   Hendrikus   pu   maytua\\
 {}    yes   \textsc{d.dist}   uncle   who   \textsc{d.prox}   Hendrikus   \textsc{poss}   wife\\
\glt
Nelci: yes, that was uncle, who is this, Hendrikus’ wife
\z

\ea
\gll   Oten:   hm\\
{}     pfft\\
\glt
Oten: pfft!
\z
%\todo[inline]{Check punctuation!}
\ea
\gll   Nelci:   kapang   ko   bla?\\
 {}     when   \textsc{2sg}   split\\
\glt
Nelci [addressing Oten]: when did you split (the firewood)?
\z

\ea
\gll   Klara:   RW,   RW\\
 {}  cooked.dog.meat   cooked.dog.meat\\
\glt
Klara [responding to another interlocutor]: cooked dog meat, cooked dog meat
\z

\ea
\gll   Oten:   sa   yang   bla   sore\\
 {}  \textsc{1sg}   \textsc{rel}   split   afternoon\\
\glt
Oten: (it was) me who split (the firewood) in the afternoon
\z

\ea
\gll   Klara:   RW,   tra   ada   RW\\
 {}     cooked.dog.meat   \textsc{neg}   exist   cooked.dog.meat\\
\glt
Klara: cooked dog meat, there’s no cooked dog meat
\z

\ea
\gll    {Nelci:}    {pagi,}    {bukang}   ko,    {itu}    {su}    {kemaring}    {sore}\\
   {}    {morning}    {\textsc{neg}}   \textsc{2sg}    {\textsc{d.dist}}    {already}    {yesterday}    {afternoon}\\
\gll yang    {ko}   bla,    {ini}    {pagi}    {lagi}   om    {Hendrikus}    {yang}   bla\\
  \textsc{rel}    {\textsc{2sg}}   split    {\textsc{d.prox}}    {morning}    {again}   uncle    {Hendrikus}    {\textsc{rel}}   split\\
\glt
Nelci: in the morning!!, (that) wasn’t you, that was already yesterday afternoon that you split (firewood), this morning, (it was) again uncle Hendrikus who split (the firewood)
\z
%\todo[inline]{Check punctuation!}
\section{Conversation: Buying soap; bringing gasoline to Webro}
\label{Para_B.2}
\begin{tabular}{ll}
\lsptoprule
File name: &  081110-002-Cv\\
Text type: & Conversation, spontaneous\\
Interlocutors: & 2 older males, 2 older females\\
Length (min.): & 3:55\\
\lspbottomrule
\end{tabular}
\setcounter{equation}{0}
\ea
\gll   Ida:   slamat   sore   smua\\
  {}   be.safe   afternoon   all\\
\glt
Ida: good afternoon you all
\z

\ea
\gll   Natalia:   sore,   sore\\
  {}   afternoon   afternoon\\
\glt
Natalia: afternoon, afternoon
\z

\ea
\gll   Natalia:   eh,   bagemana   ipar?   sore,   dari   Jayapura?\\
 {}    hey!   how   sibling.in-law   afternoon   from   Jayapura\\
\glt
Natalia [greeting another visitor]: hey, how is it going brother-in-law?, good afternoon!, (did you just get here) from Jayapura?
\z

\ea
\gll   MO-1:   {\upshape\textsc{[up]}}\\
MO-1: [\textsc{up}]\\
\z

\ea
\gll   Natalia:   aah,   yo!   baru   mana   tong   pu   ipar\\
{}     ah!   yes   and.then   where   \textsc{1pl}   \textsc{poss}   sibling.in-law\\
\gll   {prempuang?}\\
       {woman}\\
\glt
Natalia: ah, yes! so where is our sister-in-law?
\z
%\todo[inline]{Check punctuation}
\ea
\gll   Ida:    {ipar}   prempuang   yang   baru   lewat\\
  {}    {sibling.in-law}   woman   \textsc{rel}   recently   pass.by\\
\gll   {deng}\hspace{1cm}  {ojek}    {((laughter))}\\
   {with}\hspace{1cm}  {motorbike.taxi}    {}\\
\glt
Ida: (it’s our) sister-in-law who passed by with a motorbike taxi a short while ago ((laughter))
\z

\ea
\gll   Natalia:   ey!   baru   lewat?\\
 {}     hey!   recently   pass.by\\
\glt
Natalia: hey, did (she) pass by a short while ago?
\z

\ea
\gll   MO-1:   tadi   lewat   deng   ojek\\
 {}    earlier   pass.by   with   motorbike.taxi\\
\glt
MO-1: earlier she passed by on a motorbike taxi
\z

\ea
\gll   Natalia:   yo?\\
{}     yes\\
\glt
Natalia: yes?
\z

\ea
\gll   Ida:   de   tadi   lewat   deng   ojek\\
 {}    \textsc{3sg}   earlier   pass.by   with   motorbike.taxi\\
\glt
Ida: earlier she passed by with a motorbike taxi
\z

\ea
\gll   Natalia:   ibu,   de   su   bawa   de   pu   maytua?   ((laughter))\\
 {}    woman   \textsc{3sg}   already   bring   \textsc{3sg}   \textsc{poss}   wife   \\
\glt
Natalia: mother, did he already bring his wife? ((laughter))
\z

\ea
\gll   Ida:   tra   taw,   tanya   dia,   sa   tra   taw\\
 {}    \textsc{neg}   know   ask   \textsc{3sg}   \textsc{1sg}   \textsc{neg}   know\\
\glt
Ida: I don’t know, ask him, I don’t know
\z

\ea
\gll   MO-1:   {\upshape\textsc{[up]}}\\
MO-1: [\textsc{up}]\\
\z

\ea
\gll   MO-2:   sa   ada   lewat   deng   mobil\\
 {}    \textsc{1sg}   exist   pass.by   with   car\\
\glt
MO-2: I was passing by in a car
\z

\ea
\gll    {Natalia:}    {bahaya!,}    {((pause))}    {ko}    {punya}    {barang}    {itu}    {masi}   ada?,\\
   {}    {danger}    {}    {\textsc{2sg}}    {\textsc{poss}}    {stuff}    {\textsc{d.dist}}    {still}   exist\\
\gll   {ini}    {sa}    {mo}    {pi,}    {dong}    {ada}    {pesang,}    {sa}    {mo}    {bawa}\\
   {\textsc{d.prox}}    {\textsc{1sg}}    {want}    {go}    {\textsc{3pl}}    {exist}    {order}    {\textsc{1sg}}    {want}    {bring}\\
\gll   {titip}    {di}    {depang}    {situ,}    {bawa}    {ke}    {depang,}    {bukang}    {titip}\\
   {deposit}    {at}    {front}    {\textsc{l.med}}    {bring}    {to}    {front}    {\textsc{neg}}    {deposit}\\
\gll  tapi    {sa}    {pi}    {bawa,}    {kemaring}    {sampe}   sa    {sibuk}\\
  but    {\textsc{1sg}}    {go}    {bring}    {yesterday}    {until}   \textsc{1sg}    {be.busy}\\
\glt
Natalia: great!, ((pause)) is your stuff still (here)?, right now, I want to go, they ordered (s.th.), I want to bring (and) deposit (it) in front over there, (I want to) bring (it) to the front, not to deposit (it) but I want to go and bring (it), yesterday, (when I) arrived, I was (too) busy (to do it)
\z

\ea
\gll   Ida:    {ini}    {sa}   ada   cari,   yo,   ini,   sa   ada   cari\\
 {}     {\textsc{d.prox}}    {\textsc{1sg}}   exist   search   yes   \textsc{d.prox}   \textsc{1sg}   exist   search\\
\gll   {uang,}    {ini,}    {ojek}\\
   {money}    {\textsc{d.prox}}    {motorbike.taxi}\\
\glt
Ida: what’s-its-name, I’m looking for, yes, what’s-its-name, I’m looking for money, what’s-its-name, (for) the motorbike taxi
\z

\ea
\gll    {Natalia:}   perjalangang,   kemaring   sa   mo   bawa,   kemaring\\
   {}   journey   yesterday   \textsc{1sg}   want   bring   yesterday\\
\gll  dulu    {karna}\\
  be.prior    {because}\\
\glt
Natalia: (for your) trip, yesterday I wanted to bring (the stuff), the day before yesterday because
\z

\ea
\gll   Ida:   {\upshape\textsc{[up]}}   sabung   saja,   kam   pu   sabung   ada   di   situ\\
 {}   {}   soap   just   \textsc{2pl}   \textsc{poss}   soap   exist   at   \textsc{l.med}\\
\glt
Ida: [\textsc{up}] just (laundry) soap, your (laundry) soap is there
\z

\ea
\gll   Natalia:   damay!,   kitong   tra   ada   sabung   ini\\
 {}   peace   \textsc{1pl}   \textsc{neg}   exist   soap   \textsc{d.prox}\\
\glt
Natalia: my goodness!, we don’t have any soap right now!
\z

\ea
\gll   Ida:   yo,   suda,   kalo   begitu   tinggal   suda!\\
 {}    yes   already   if   like.that   stay   already\\
\glt
Ida: yes!, alright!, if it’s like that, no problem!
\z

\ea
\gll    {Natalia:}    {simpang,}    {sa}    {simpang}    {sratus}    {ribu}    {tu,}\\
   {}    {store}    {\textsc{1sg}}    {store/prepare}    {one:hundred}    {thousand}    {\textsc{d.dist}}\\
\gll  de    {pu}    {bapa}    {ar}    {ambil,}    {de}    {ada}    {du}\\
  \textsc{3sg}    {\textsc{poss}}    {father}    {\textsc{spm}{}-fetch}    {fetch}    {\textsc{3sg}}    {exist}    {\textsc{tru}{}-be.prior}\\
\gll   {d}    {ikut}    {platiang}    {satu}    {minggu}    {di}    {atas,}    {karna}\\
   {\textsc{tru}{}-be.prior}    {follow}    {training}    {one}    {week}    {at}    {top}    {because}\\
\gll   {tadi}    {sa}    {mo}    {cuci}    {pakeang}    {ada}    {taro}    {tinggal}   {\upshape\textsc{[up]}}\\
   {earlier}    {\textsc{1sg}}    {want}    {wash}    {use-\textsc{pat}}    {exist}    {put}    {stay}   \\
\glt
Natalia: (I) set aside, I set aside one hundred thousand, my husband\footnote{Lit. ‘her father’ (\textitbf{de} ‘\textsc{3sg}’ refers to the speaker’s daughter).} took[\textsc{spm}] took it, he was[\textsc{tru}] was[\textsc{tru}] attending a one-week training (course) up there (at the regent’s office), because earlier I wanted to wash (his) clothes, (but I) had to put it off, [\textsc{up}]
\z
%\todo[inline]{Check punctuation and text.!}
\ea
\gll   Ida:   supaya,   sa   mo   cuci   dong   dua   pu   pakeang   itu\\
 {}     so.that   \textsc{1sg}   want   wash   \textsc{3pl}   two   \textsc{poss}   use-\textsc{pat}   \textsc{d.dist}\\
\gll   {yang}\\
   {\textsc{rel}}\\
\glt
Ida: so that, I want to wash both of their clothes which
\z

\ea
\gll   Natalia:   tra   ada   ma\\
{}     \textsc{neg}   exist   mother\\
\glt
Natalia: (there) isn’t (any), mother
\z

\ea
\gll   Ida:   su   tra   ada   sabung\\
   {}  already   \textsc{neg}   exist   soap\\
\glt
Ida: alright, there’s no soap
\z

\ea
\gll    {Natalia:}    {tunggu,}    {sabar,}    {kalo}    {mo}    {sabar,}   kalo   masi\\
   {}    {wait}    {be.patient}    {if}    {want}    {be.patient}   if   still\\
\gll   {besok}   mo    {naik,}    {ini}    {bawa}    {ke}   mari,    {nanti}\\
   {tomorrow}   want    {ascend}    {\textsc{d.prox}}    {bring}    {to}   hither    {very.soon}\\
\gll  sa    {yang}    {cuci}\\
  \textsc{1sg}    {\textsc{rel}}    {wash}\\
\glt
Natalia: wait, be patient, if you want to be patient, if tomorrow (you) still want to go up (to the regent’s office and), what’s-its-name, bring (the clothes) there, I’ll wash (them)
\z

\ea
\gll   Ida:   tra   ada,   ini   suda   selesay,   jadi   besok   {\upshape\textsc{[up]}}\\
 {}  \textsc{neg}   exist   \textsc{d.prox}   already   finish   so   tomorrow   \\
\glt
Ida: no, this (meeting) is already over, so tomorrow [\textsc{up}]
\z

\ea
\gll   Natalia:   i,   kam   su   selesay?\\
 {}   ugh!   \textsc{2pl}   already   finish\\
\glt
Natalia: ugh!, you already finished?
\z

\ea
\gll   Ida:   a,   itu   bukang   apa,   hanya   penyeraang   uang\\
 {} ah!   \textsc{d.dist}   \textsc{neg}   what   only   dedication   money\\
\glt
Ida: ah, that’s not, what-is-it, (it’s) only the distribution (of) the funds
\z

\ea
\gll   Natalia:   o!\\
  {} oh!\\
\glt
Natalia: oh!
\z

\ea
\gll   Ida:   saja   {\upshape\textsc{[up]}}\\
 {}  just   \\
\glt
Ida: just [\textsc{up}]
\z

\ea
\gll   Natalia:   o,   yo\\
 {}  oh!   yes\\
\glt
Natalia: oh, yes
\z

\ea
\gll   Ida:   ibu   bupati   bicarakang   uang   ke   ibu   distrik\\
 {}    woman   regent   {speak-\textsc{app}}  money   to   woman   district\\
\glt
Ida: Ms. Regent talked to Ms. District (about the) money
\z

\ea
\gll   Natalia:   o,   begitu,   PKK\\
  {}   oh!   like.that   family.welfare.program\\
\glt
Natalia: oh it’s like that, (about) the family welfare program
\z

\ea
\gll   Ida:   yo\\
 {}    yes\\
\glt
Ida: yes
\z

\ea
\gll    {Natalia:}    {o,}    {kalo}    {begitu}    {siang,}    {tu}    {yang,}    {sa,}   siri\\
   {}    {oh!}    {if}    {like.that}    {midday}    {\textsc{d.dist}}    {\textsc{rel}}    {\textsc{1sg}}   betel.vine\\
\gll     sa    {bawa}    {ke}    {sana}    {dulu,}    {depang}   dulu,    {ini}    {su}\\
  \textsc{1sg}    {bring}    {to}    {\textsc{l.dist}}    {first}    {front}   first    {\textsc{d.prox}}    {already}\\
\gll   {mo}    {sore}    {jadi,}    {sa}    {masak}    {sayur}    {\upshape\textsc{[up]},}    {bapa}\\
   {want}    {afternoon}    {so}    {\textsc{1sg}}    {cook}    {vegetable}    {}    {father}\\
\gll  dong    {dari}    {Yawar}\\
  \textsc{3pl}    {from}    {Yawar}\\
\glt
Natalia: oh, if it’s like that, (I assume the meeting was over) at midday, that’s why, I, the betel vine I’ll bring (it) over there first, (I’ll bring it) to the front first, because now it’s already turning afternoon, I’m cooking the vegetables [\textsc{up}], the men from Yawar
\z

\ea
\gll   Ida:    {hari}    {ini}    {yo}    {suda}    {selesay,}   jadi    {ibu}    {distrik}   de\\
 {}      {day}    {\textsc{d.prox}}    {yes}    {already}    {finish}   so    {woman}    {district}   \textsc{3sg}\\
\gll   {kasi}    {kitong}    {dua}    {pu}    {uang}    {ojek}    {pulang}    {pergi}\\
   {give}    {\textsc{1pl}}    {two}    {\textsc{poss}}    {money}    {motorbike.taxi}    {go.home}    {go}\\
\glt
Ida: today, yes, (the meeting) is already over, so Ms. District gave the two of us money (for) our return fare for the motorbike taxis
\z

\ea
\gll   Natalia:   kasiang\\
  {}    pity\\
\glt
Natalia: poor thing!
\z

\ea
\gll   Ida:   jadi   sa   in   mo   ini,   ini\\
{}   so   \textsc{1sg}   \textsc{tru-d.prox}   want   \textsc{d.prox}   \textsc{d.prox}\\
\glt
Ida: so, I here[\textsc{tru}] want this (or) this (but I can’t with these limited funds)
\z

\ea
\gll    {Natalia:}    {tong}    {dua}    {tra}    {ada,}    {yang}    {pertama}    {itu}    {sa}   su\\
   {}    {\textsc{1pl}}    {two}    {\textsc{neg}}    {exist}    {\textsc{rel}}    {first}    {\textsc{d.dist}}    {\textsc{1sg}}   already\\
\gll  kasi    {dorang,}    {makanya}    {wa}    {mana}    {itu,}    {dong}    {su}\\
  give    {\textsc{3pl}}    {for.that.reason}    {\textsc{spm}}    {where}    {\textsc{d.dist}}    {\textsc{3pl}}    {already}\\
\gll {mo}    {bli}    {batu,}    {jadi}    {skarang}    {sa,}    {itu,}    {simpang}\\
   {want}    {buy}    {stone}    {so}    {now}    {\textsc{1sg}}    {\textsc{d.dist}}    {store}\\
\gll   {sratus}    {ribu}\\
   {one:hundred}    {thousand}\\
\glt
Natalia: the two of us haven’t (gotten any money left), I already gave the first (one hundred thousand) to them, that is to say [\textsc{spm}] what-is-it, they already wanted to buy stones, so now I (already), what’s-its-name, set aside one hundred thousand (rupiah)
\z

\ea
\gll   Ida:   yo   suda   kegiatang\\
  {}   yes   already   activity\\
\glt
Ida: yes, well, the activity
\z

\ea
\gll    {Natalia:}    {de}    {bapa,}    {dua}    {ratus}    {de}    {pu}   bapa,    {trus}\\
   {}    {\textsc{3sg}}    {father}    {two}    {hundred}    {\textsc{3sg}}    {\textsc{poss}}   father    {next}\\
\gll  {dep}    {bapa-ade}    {Martin}    {dia}    {bawa}    {lari}    {prempuang,}   adu,\\
   {\textsc{3sg}:\textsc{poss}}    {uncle}    {Martin}    {\textsc{3sg}}    {bring}    {run}    {woman}   oh.no!\\
\gll  in    {tong}    {lagi}    {masala}    {lagi,}    {de}    {bapa-ade}    {Martin}\\
  \textsc{d.prox}    {\textsc{1pl}}    {again}    {problem}    {again}    {\textsc{3sg}}    {uncle}    {Martin}\\
\glt
Natalia: my husband\footnote{Lit. ‘her father’ (\textitbf{de} ‘\textsc{3sg}’ refers to the speaker’s daughter).}, two hundred (thousand for) my husband, and my brother-in-law Martin\footnote{Lit. ‘her uncle’ (\textitbf{de} ‘\textsc{3sg}’ refers to the speaker’s daughter).} took a woman away (with him), oh no!, here, we are having problems again, my brother-in-law Martin
\z

\ea
\gll   Ida:   naik   motor?\\
 {}    ascend   motorbike\\
\glt
Ida: (he) took (her) on a motorbike?
\z

\ea
\gll   Natalia:   prempuang,   kapal   Papua-Lima\\
 {}    woman   ship   Papua-Lima\\
\glt
Natalia: the woman, (she came with) the Papua-Lima ship
\z

\ea
\gll   Ida:   ya   Tuhang\\
  {}   yes   God\\
\glt
Ida: oh God!
\z

\ea
\gll   Natalia:   de   bawa   prempuang   Bagayserwar\\
 {}    \textsc{3sg}   bring   woman   Bagayserwar\\
\glt
Natalia: he brought a woman (from) Bagayserwar
\z

\ea
\gll   Ida:   ya   ampung\\
{}     yes   forgiveness\\
\glt
Ida: for mercy’s sake!
\z

\ea
\gll   Natalia:   kemaring   de   pigi,   sa   pikir   mungking   de   sendiri   pigi\\
 {}    yesterday   \textsc{3sg}   go   \textsc{1sg}   think   maybe   \textsc{3sg}   alone   go\\
\glt
Natalia: yesterday he left, I thought, maybe he went by himself
\z

\ea
\gll   Ida:    {i,}    {e,}   jang   ceritra   banyak,   kasi   sayur   sa\\
{}      {ugh!,}    {hey!,}   \textsc{neg.imp}   tell   many   give   vegetable   \textsc{1sg}\\
\gll {makang,}    {sa}    {lapar}\\
   {eat}    {\textsc{1sg}}    {be.hungry}\\
\glt
Ida: ugh!, hey, don’t talk a lot, give me vegetables to eat, I’m hungry
\z

\ea
\gll   Natalia:   wa,   ko   datang   langsung   ko   lapar?\\
 {}  wow!   \textsc{2sg}   come   immediately   \textsc{2sg}   be.hungry\\
\glt
Natalia: wow!, you come (here, and) immediately you’re hungry?
\z

\ea
\gll   All:   ((laughter))\\
All: ((laughter))\\
\z

\ea
\gll   Natalia:   nasi   ada   itu,   timba   suda\\
  {}   cooked.rice   exist   \textsc{d.dist}   spoon   already\\
\glt
Natalia: the cooked rice is over there, just spoon (it)!
\z

\ea
\gll   Ida:   ah,   sa   tida   makang   nasi\\
{} ah!   \textsc{1sg}   \textsc{neg}   eat   cooked.rice\\
\glt
Ida: ah, I don’t eat rice
\z

\ea
\gll   Natalia:   habis   apa?\\
{}     after.all   what\\
\glt
Natalia: so what (do you want)?
\z

\ea
\gll   Ida:   sa   mo   makang   sayur   saja\\
    {}  \textsc{1sg}   want   eat   vegetable   just\\
\glt
Ida: I just want to eat vegetables
\z

\ea
\gll    {Natalia:}   yo,   ambil    {piring}    {suda}   di   dalam,   sa   deng   Angela\\
   {}   yes   fetch    {plate}    {already}   at   inside   \textsc{1sg}   with   Angela\\
\gll  ada    {duduk,}    {mama}    {ambil}    {piring}\\
  exist    {sit}    {mother}    {fetch}    {plate}\\
\glt
Natalia: alright, just get a plate from inside, I and Angela are sitting around, take a plate, mama
\z

\ea
\gll   Ida:   yo,   suda,   sebentar\\
  {}   yes   already   in.a.moment\\
\glt
Ida: yes, alright, (I’ll get one) in a moment
\z

\ea
\gll    {Natalia:}    {suda,}    {isi}    {sayur}    {suda,}    {masak}    {pertama}    {habis,}\\
   {}    {already}    {fill}    {vegetable}    {already}    {cook}    {first}    {be.used.up}\\
\gll  e    {bapa}    {dong}    {dari}    {Wari,}    {Aruswar}   tra   dapat,    {itu}   yang\\
  uh    {father}    {\textsc{3pl}}    {from}    {Wari}    {Aruswar}   \textsc{neg}   get    {\textsc{d.dist}}   \textsc{rel}\\
\gll  {sa}    {ada}    {masak}    {kangkung}\\
   {\textsc{1sg}}    {exist}    {cook}    {water.spinach}\\
\glt
Natalia: alright, just fill (the plate with) vegetables, (the food that I) cooked first is finished, uh, the men from Wari, Aruswar didn’t get (any of the food), that’s why I’m cooking water spinach
\z

\ea
\gll   de    {pu}    {tanta}    {dong}    {dari}    {Tarfia}    {dorang}    {ini,}   dep\\
  \textsc{3sg}    {\textsc{poss}}    {aunt}    {\textsc{3pl}}    {from}    {Tarfia}    {\textsc{3pl}}    {\textsc{d.prox}}   \textsc{3sg}:\textsc{poss}\\
\gll    {ma,}    {apa,}    {dong}    {pu}    {bapa-ade}   bli    {[Is],}    {suda,}\\
   {\textsc{tru}{}-aunt}    {what}    {\textsc{3pl}}    {\textsc{poss}}    {uncle}   buy    {}    {already}\\
\gll  {trus}    {sa}    {masak}    {nasi}    {pertama}\\
   {next}    {\textsc{1sg}}    {cook}    {cooked.rice}    {first}\\
\glt
my sister-in-law\footnote{Lit. ‘her aunt’ (\textitbf{de} ‘\textsc{3sg}’ refers to the speaker’s daughter).} and the others from Tarfia, my sister-in-law, what-is-it, their uncle bought [Is], well, then I cooked the first meal
\z

\ea
\gll   Ida:   baru   [Is]?\\
{}     and.then   \\
\glt
Ida: and then [Is]?
\z

\ea
\gll Natalia:    {mama-tua}    {de,}   e,    {kemaring}    {dia}    {kas}   taw   saya   ni,\\
 {}     {aunt}    {\textsc{3sg}}   uh    {yesterday}    {\textsc{3sg}}    {give}   know   \textsc{1sg}   \textsc{d.prox}\\
\gll    {mama-tua,}    {sa}    {masi}    {sibuk,}    {tunggu,}    {sa}    {blum}    {pigi,}\\
   {aunt}    {\textsc{1sg}}    {still}    {be.busy}    {wait}    {\textsc{1sg}}    {not.yet}    {go}\\
\gll    {sebentar}    {baru}\\
   {in.a.moment}    {and.then}\\
\glt
Natalia: the aunt, yesterday, she let me here know, aunt, I was still busy, (you) waited, I hadn’t gone yet, a moment later and then
\z

\ea
\gll   Ida:    {sebentar}   bilang   kaka   Nelci   yang   ganti   sa   pu\\
 {}     {in.a.moment}   say   oSb   Nelci   \textsc{rel}   replace   \textsc{1sg}   \textsc{poss}\\
\gll    {karong}    {lagi}\\
   {bag}    {again}\\
\glt
Ida: then tell older sister Nelci who also replaced my bag
\z

\ea
\gll   Natalia:    {e,}    {Ise}   o,    {Ise,}    {sa}    {lupa,}   kamu   bawa   pulang\\
{}      {uh}    {Ise}   oh!    {Ise}    {\textsc{1sg}}    {forget}   \textsc{2pl}   bring   go.home\\
\gll    {mama-tua}    {pu}    {cobe,}    {kam}    {bawa}    {[Is]}\\
   {aunt}    {\textsc{poss}}    {mortar}    {\textsc{2pl}}    {bring}    {}\\
\glt
Natalia [addressing her daughter Ise]: uh, Ise oh, Ise, I forgot, you return aunt’s mortar!, you return [Is]
\z

\ea
\gll   Ida:    {itu}   yang   sa    {tadi}   bilang   tu,   tadi   sa   bilang\\
{}     {\textsc{d.dist}}   \textsc{rel}   \textsc{1sg}    {earlier}   say   \textsc{d.dist}   earlier   \textsc{1sg}   say\\
\gll    {mama-tua,}    {tolong}    {karna}    {besok}\\
   {aunt}    {help}    {because}    {tomorrow}\\
\glt
Ida: that’s what I said earlier, earlier I said to aunt, ‘please, because tomorrow’
\z

\ea
\gll   Natalia:   mo   pulang,   {\upshape\textsc{[up]}}   sa   bawa\\
{}  want   go.home   {}   \textsc{1sg}   bring\\
\glt
Natalia: (I) want to go home, [\textsc{up}] I bring
\z

\ea
\gll   Ida:   skarang   kamu   kasi   terpol{\Tilde}terpol   taru   di   sini\\
 {}  now   \textsc{2pl}   give   \textsc{rdp}{\Tilde}container   put   at   \textsc{l.prox}\\
\glt
Ida: now you give (me) the jerry cans, put (them) here
\z

\ea
\gll   Natalia:   terpol   [Is],   ey,   yang   besar{\Tilde}besar   itu   jangang\\
{}   container  {}    hey!   \textsc{rel}   \textsc{rdp}{\Tilde}be.big   \textsc{d.dist}   \textsc{neg.imp}\\
\glt
Natalia: the jerry cans, [Is], hey, those big ones, don’t (take them)!
\z

\ea
\gll   Ida:   a,   yang   kecil{\Tilde}kecil\\
{}   ah!   \textsc{rel}   \textsc{rdp}{\Tilde}be.small\\
\glt
Ida: ah, (I take the ones) that are small
\z

\ea
\gll    {Natalia:}   ey,   ada   tu,   silakang,   ko   mo   bawa   pergi,   ko\\
   {}   hey!   exist   \textsc{d.dist}   please   \textsc{2sg}   want   bring   go   \textsc{2sg}\\
\gll   bawa    {duluang}\\
  bring    {be.prior-\textsc{pat}}\\
\glt
Natalia: hey, (they) are there, please, (if) you want to take (them) away, take (them) and go ahead
\z

\ea
\gll   Ida:   yo   itu   smua\\
 {}    yes   \textsc{d.dist}   all\\
\glt
Ida: yes, all (of them)
\z

\ea
\gll   Natalia:   ko   bawa   duluang\\
{}     \textsc{2sg}   bring   be.prior-\textsc{pat}\\
\glt
Natalia: take them (and) go ahead
\z

\ea
\gll   Ida:    {smua}   kasi   ke   mari,   sa   mo   bawa,   {\upshape\textsc{[up]}}   di   sana\\
 {}   {all}   give   to   hither   \textsc{1sg}   want   bring   {}    at   \textsc{l.dist}\\
\gll    {tida}    {ada}\\
   {\textsc{neg}}    {exist}\\
\glt
Ida: give all of them to (me) here, I want to take (them) [\textsc{up}], over there aren’t (any)
\z

\ea
\gll   Natalia:   sa   stembay,   sa   stembay,   ini,   bensing\\
 {}    \textsc{1sg}   stand.by.for   \textsc{1sg}   stand.by.for   \textsc{d.prox}   gasoline\\
\glt
Natalia: I stand by, I stand by (with), what’s-its-name, the gasoline
\z

\ea
\gll   Ida:   ko   stembay   bensing,   ko   bli   bensing\\
 {}  \textsc{2sg}   stand.by.for   gasoline   \textsc{2sg}   buy   gasoline\\
\glt
Ida: you stand by (with) the gasoline, you buy gasoline
\z

\ea
\gll   Natalia:   yo\\
 {} yes\\
\glt
Natalia: yes
\z

\ea
\gll   Ida:    {terpol}    {itu,}    {LNG}   pu   terpol   itu\\
{}   {container}    {\textsc{d.dist}}    {liquefied.natural.gas}   \textsc{poss}   container   \textsc{d.dist}\\
\gll    {tinggal,}    {itu}   ko    {isi}   bensing   di    {situ}\\
   {stay}    {\textsc{d.dist}}   \textsc{2sg}    {fill}   gasoline   at    {\textsc{l.med}}\\
\glt
Ida: those jerry cans, that LNG jerry can stays behind, that (metal one), you fill the gasoline in there
\z

\ea
\gll   Natalia:   yo\\
 {} yes\\
\glt
Natalia: yes
\z

\ea
\gll   Ida:   empat   liter   saja\\
{}  four   liter   just\\
\glt
Ida: just four liters
\z

\ea
\gll   Natalia:    {ey,}    {empat}    {e,}   kasiang,   mama   kampung   di\\
 {}  {hey!}    {four}    {uh}   pity   mother   village   at\\
\gll {ba}    {laut}    {mo}    {bli}\\
   {\textsc{tru}{}-bottom}    {sea}    {want}    {buy}\\
\glt
Natalia: hey, four (liters), uh, poor thing, Ms. Mayor down[\textsc{tru}] (at the) seaside wants to buy
\z

\ea
\gll   Ida:   o   yo   suda\\
 {}    oh!   yes   already\\
\glt
Ida: yes, that’s it
\z

\ea
\gll   Natalia:   sa   tra   bisa   kasi   sembarang   orang,   mama\\
{}  \textsc{1sg}   \textsc{neg}   be.able   give   any(.kind.of)   person   mother\\
\gll    {kampung}\\
   {village}\\
\glt
Natalia: I can’t give (the gasoline to just) any person, (but) Ms. Mayor
\z

\ea
\gll   Ida:   [Is]\\
Ida: [Is]\\
\z

\ea
\gll   Natalia:   Nusa   mama\\
  {}   Nusa   mother\\
\glt
Natalia: Nusa’s mother
\z

\ea
\gll   Ida:   [Is]\\
Ida: [Is]\\
\z

\ea
\gll    {Natalia:}    {kitong}    {lima}    {liter,}    {itu}    {saja,}    {yang}    {laing{\Tilde}laing}\\
   {}    {\textsc{1pl}}    {five}    {liter}    {\textsc{d.dist}}    {just}    {\textsc{rel}}    {\textsc{rdp}{\Tilde}be.different}\\
\gll {mmm,}    {sa}    {su}    {tra}    {maw,}    {sembuni}    {mati,}    {jadi}   ko   bawa\\
   {uh}    {\textsc{1sg}}    {already}    {\textsc{neg}}    {want}    {hide}    {die}    {so}   \textsc{2sg}   bring\\
\gll {laing,}    {laing}    {sa}    {tahang,}   e,    {sa}    {tahang,}\\
   {be.different}    {be.different}    {\textsc{1sg}}    {hold(.out/back)}   uh    {\textsc{1sg}}    {hold(.out/back)}\\
\gll kas    {tinggal,}    {sa}    {spulu}\\
  give    {stay}    {\textsc{1sg}}    {one-tens}\\
\glt
Natalia: we’ll (buy) five liters, that’s it, the others, uh, I already don’t want (to buy gasoline for them), hide (it) from sight, so you take some, I keep some, uh, I keep (some), leave it, I’ll (buy) ten (liters)
\z

\ea
\gll   MO-1:   [Is]\\
MO-1: [Is]\\
\z

\ea
\gll   Ida:    {[Is],}    {sa}   liat    {dulu,}    {nanti}   sa   sendiri   yang\\
{}      {}    {\textsc{1sg}}   see    {first}    {very.soon}   \textsc{1sg}   be.alone   \textsc{rel}\\
\gll {pili}    {mana}    {yang}   sa    {m}    {bawa}\\
   {choose}    {where}    {\textsc{rel}}   \textsc{1sg}    {\textsc{tru}{}-want}    {bring}\\
\glt
Ida: [Is], I’ll have a look first, then (it’ll be) me who’ll choose which (jerry can) I want[\textsc{tru}] to take
\z

\ea
\gll    {Natalia:}    {yang}    {itu,}   yang   itu   tu,    {adu}    {ini,}   ana{\Tilde}ana\\
   {}    {\textsc{rel}}    {\textsc{d.dist}}   \textsc{rel}   \textsc{d.dist}   \textsc{d.dist}    {oh.no!}    {\textsc{d.prox}}   \textsc{rdp}{\Tilde}child\\
\gll {ini}    {dong}    {tra}    {menyimpang,}    {ini}    {bapa-tua}    {kampung,}\\
   {\textsc{d.prox}}    {\textsc{3pl}}    {\textsc{neg}}    {store/prepare}    {\textsc{d.prox}}    {uncle}    {village}\\
\gll {u,}   {Arbais,}    {Arbais}    {punya}\\
  uh    {Arbais}    {Arbais}    {\textsc{poss}}\\
\glt
Natalia: that one, that one there, oh no!, what’s-its-name, these children they didn’t store (the jerry cans well), this one is (the jerry can) of uncle Mayor, umh (from) Arbais, Arbais
\z
%\todo[inline]{Check punctuation!}
\ea
\gll   Ida:   yo,   sa   tra   minta   yang   besar,   yang   kecil\\
{}   yes   \textsc{1sg}   \textsc{neg}   request   \textsc{rel}   be.big   \textsc{rel}   be.small\\
\glt
Ida: yes, I don’t ask for (the one) that is big, (I ask for the one) that is small
\z

\section{Conversation: Wanting bananas}
\label{Para_B.3}
\begin{tabular}{ll}
\lsptoprule
File name: &  081011-003-Cv\\
Text type: &  Conversation, spontaneous\\
Interlocutors: &  1 male child, 2 younger females, 2 older females\\
Length (min.): &  0:35\\
\lspbottomrule
\end{tabular}
\setcounter{equation}{0}
\ea
\gll   Fanceria:   kecil   malam   dia   menangis   pisang   goreng\\
{}  be.small   night   \textsc{3sg}   cry   banana   fry\\
\glt
Fanceria: (this) little (boy Nofi), (last) night he cried (for) fried bananas
\z

\ea
\gll   Marta:    {yo,}   dong   dua   deng   Wili   tu   biking\\
{}   {yes}   \textsc{3pl}   two   with   Wili   \textsc{d.dist}   make\\
\gll {pusing}    {mama}\\
   {be.dizzy}    {mother}\\
\glt
Marta: yes!, he and Wili there worried (their) mother
\z
%\todo[inline]{Check punctuation!}
\ea
\gll   Fanceria:   ay,   pisang   di   sana   itu   yang   mo   bli\\
{}  aw!   banana   at   \textsc{l.dist}   \textsc{d.dist}   \textsc{rel}   want   buy\\
\glt
Fanceria: aw!, (it was) the bananas (from) over there which (Nofi) wanted to buy
\z

\ea
\gll   Marta:   {\upshape\textsc{[up]}}   ni   tra   rasa   sakit,   dapat   pukul   trus\\
{}   {}  \textsc{d.prox}   \textsc{neg}   feel   be.sick   get   hit   be.continuous\\
\glt
Marta: [\textsc{up}] here doesn’t feel sick, (he) gets beaten continuously
\z

\ea
\gll   Nofi:   sa   pu   sribu\\
{}  \textsc{1sg}   \textsc{poss}   one-thousand\\
\glt
Nofi: (that’s) my one thousand (rupiah bill)
\z

\ea
\gll   Fanceria:   yo,   ini   kertas   ((laughter))\\
{}  yes   \textsc{d.prox}   paper   \\
\glt
Fanceria: yes, this is (only) paper (but not money) ((laughter))
\z

\ea
\gll   Nofi:   ko   gila   ka?\\
{}  \textsc{2sg}   be.crazy   or\\
\glt
Nofi: are you crazy?
\z

\ea
\gll   Nofita:   terlalu   nakal   ana{\Tilde}ana   di   sini\\
 {} too   be.mischievous   \textsc{rdp}{\Tilde}child   at   \textsc{l.prox}\\
\glt
Nofita: (they are) too mischievous the children here
\z

\ea
\gll   Fanceria:   a,   Nofi   {\upshape\textsc{[up]}}\\
{}  ah!   Nofi   \\
\glt
Fanceria: ah, Nofi [\textsc{up}]
\z

\ea
\gll   Marta:   [Is]\\
Marta: [Is]\\
\z

\ea
\gll   Fanceria:   mm-mm\\
{}  mhm\\
\glt
Fanceria: mhm
\z

\ea
\gll   Marta:   tida   ada   pisang   goreng,   menangis   pisang   goreng\\
{}  \textsc{neg}   exist   banana   fry   cry   banana   fry\\
\glt
Marta: (when) there aren’t (any) fried bananas, (then Nofi) cries (for) fried bananas
\z

\ea
\gll   Fanceria:   ((laughter))\\
Fanceria: ((laughter))\\
\z

\ea
\gll   Nofita:   ada   pisang   goreng,   tra   maw   makang\\
{}  exist   banana   fry   \textsc{neg}   want   eat\\
\glt
Nofita: (when) there are fried bananas, (he) doesn’t want to eat (them)
\z

\ea
\gll   Marta:   ada   pisang   goreng,   tida   maw   makang\\
 {} exist   banana   fry   \textsc{neg}   want   eat\\
\gll pisang    {goreng}\\
  banana    {fry}\\
\glt
Marta: (when) there are fried bananas, (he) doesn’t want to eat fried bananas
\z
%\todo[inline]{Check punctuation!}
\ea
\gll   Klara:   putar   balik,   ana   kecil   itu\\
{}  turn.around   turn.around   child   be.small   \textsc{d.dist}\\
\glt
Klara: (Nofi) constantly changes (his) opinion, that small child
\z

\ea
\gll   Fanceria:    {pisang}   goreng,   pisang   Sorong   sana   tu   iii,\\
  {} {banana}   fry   banana   Sorong   \textsc{l.dist}   \textsc{d.dist}   oh\\
\gll {besar{\Tilde}besar}    {manis}\\
   {\textsc{rdp}{\Tilde}be.big}    {be.sweet}\\
\glt
Fanceria: fried bananas, those bananas (from) Sorong over there, oooh, (they) are all big (and) sweet
\z %\end{styleBodyxvafter}

\section{Narrative: A drunkard in the hospital at night}
\label{Para_B.4}
\begin{tabular}{ll}
\lsptoprule
File name: &  080916-001-CvNP\\
Text type: &  Conversation, spontaneous: Personal narrative\\
Interlocutors: &  2 older females\\
Length (min.): &  2:33\\
\lspbottomrule
\end{tabular}
\setcounter{equation}{0}
\ea
\gll   Marta:    {\ldots}    {de}    {bilang,}    {mama-ade}    {bangung}    {pergi}    {makang}   di\\
{}   {}    {\textsc{3sg}}    {say}    {aunt}    {wake.up}    {go}    {eat}   at\\
\gll    {warung,}    {sa}    {bilang,}    {Tuhang}    {ini}    {jaw}    {malam}    {begini}\\
   {food.stall}    {\textsc{1sg}}    {say}    {God}    {\textsc{d.prox}}    {far}    {night}    {like.this}\\
\gll {makang}    {di}    {warung}    {ini}    {suda}    {jam}    {dua}    {malam}\\
   {eat}    {at}    {food.stall}    {\textsc{d.prox}}    {already}    {hour}    {two}    {night}\\
\glt
Marta: {\ldots} he (Pawlus) said (to me), ‘aunt get-up, go and eat at the food stall’, I said, ‘God, it’s too late at night to eat at the food stall, this is already two o’clock at night’
\z

\ea
\gll   sa    {bilang,}    {ap}    {\upshape\textsc{[up]}}    {Pawlus}    {kalo}    {ko}    {simpang}\\
  \textsc{1sg}    {say}    {\textsc{tru}{}-what}    {}    {Pawlus}    {if}    {\textsc{2sg}}    {store}\\
 \gll {musu}    {di}    {luar,}    {yo}    {suda,}    {biar}    {mama}    {mati}    {ko}    {hidup}\\
   {enemy}    {at}    {outside}    {yes}    {already}    {let}    {mother}    {die}    {\textsc{2sg}}    {live}\\
\gll  {suda,}    {de}    {bilang,}    {tida,}    {mama}    {pergi}    {makang,}    {sa}   bilang\\
   {already}    {\textsc{3sg}}    {say}    {\textsc{neg}}    {mother}    {go}    {eat}    {\textsc{1sg}}   say\\
\gll   ko    {kluar}    {pergi}    {bungkus}    {nasi}    {untuk}    {saya}\\
  \textsc{2sg}    {go.out}    {go}    {pack}    {cooked.rice}    {for}    {\textsc{1sg}}\\
\glt
I said, ‘what[\textsc{tru}] [\textsc{up}] Pawlus, if you have enemies outside, alright, let me (‘mother’) die and you just live’, he said, ‘no, you (‘mother’) go and eat’, I said, ‘you go out, go, and (get) wrapped-up rice for me’
\z

\ea
\gll    {baru}    {Iskia}    {dia}    {pegang}    {sa}    {punya}    {lutut}   yang    {tida}\\
   {and.then}    {Iskia}    {\textsc{3sg}}    {hold}    {\textsc{1sg}}    {\textsc{poss}}    {knee}   \textsc{rel}    {\textsc{neg}}\\
\gll  {baik,}    {sa}    {pu}    {lutut}    {yang}    {suda}    {sakit}    {ini,}   bekas\\
   {be.good}    {\textsc{1sg}}    {\textsc{poss}}    {knee}    {\textsc{rel}}    {already}    {be.sick}    {\textsc{d.prox}}   trace\\
\gll {ini}    {baru}    {dia}    {gepe}    {begini}    {deng}    {kuku,}   de\\
   {\textsc{d.prox}}    {and.then}    {\textsc{3sg}}    {clamp}    {like.this}    {with}    {digit.nail}   \textsc{3sg}\\
\gll kasi,    {de}    {balut}    {putar}    {sa}    {punya}    {lutut}\\
  give    {\textsc{3sg}}    {bandage}    {turn.around}    {\textsc{1sg}}    {\textsc{poss}}    {knee}\\
\glt
and then Iskia held my knee that is not well, this knee which has already been sick, this scar (is still hurting), then he clamped (it) like this, he put, he bandaged my knee
\z

\ea
\gll    {ibu}    {Marta}    {bertriak}    {sampe,}    {sa}    {bilang,}    {Tuhang}    {tolong}    {saja}   apa\\
   {woman}    {Marta}    {scream}    {reach}    {\textsc{1sg}}    {say}    {God}    {help}    {just}   what\\
\gll yang    {su}    {gigit}    {sa}    {pu}    {lutut?,}    {baru}    {dia}    {tertawa,}    {de}\\
  \textsc{rel}    {already}    {bite}    {\textsc{1sg}}    {\textsc{poss}}    {knee}    {and.then}    {\textsc{3sg}}    {laugh}    {\textsc{3sg}}\\
\gll {tertawa{\Tilde}tertawa,}    {sa}    {blang,}    {adu}    {Tuhang}   ko    {begini}    {ka?}\\
   {\textsc{rdp}{\Tilde}laugh}    {\textsc{1sg}}    {say}    {oh.no!}    {God}   \textsc{2sg}    {like.this}    {or}\\
\glt
I (‘Ms. Marta’) screamed strongly, I said, ‘God help me!, what (is it) that has bitten my knee?’ but then he laughed, he laughed intensely, I said, ‘oh God!, why does this have to happen?’ (Lit. ‘you God are like this?’)
\z

\ea
\gll    {baru}    {Pawlus}    {dia}    {mabuk}    {s}    {ini,}    {ibu}    {guru}\\
   {and.then}    {Pawlus}    {\textsc{3sg}}    {be.drunk}    {\textsc{spm}}    {\textsc{d.prox}}    {woman}    {teacher}\\
\gll {Maria}    {ini}    {kasiang,}   de    {suda}    {tidor,}    {kang}   dia\\
   {Maria}    {\textsc{d.prox}}    {love-\textsc{pat}}   \textsc{3sg}    {already}    {sleep}    {you.know}   \textsc{3sg}\\
\gll hosa    {to?,}    {tong}    {ja}    {jaga}    {dia}    {sampe}    {jam}   satu,\\
  pant    {right?}    {\textsc{1pl}}    {\textsc{tru}{}-guard}    {guard}    {\textsc{3sg}}    {until}    {hour}   one\\
\gll {baru}    {tong}    {tidor}\\
   {and.then}    {\textsc{1pl}}    {sleep}\\
\glt
and then Pawlus was drunk [\textsc{spm}], what’s-her-name, Ms. Teacher Maria here, poor thing, she was already sleeping, you know?, she has breathing difficulties, right?, we watched[\textsc{tru}] watched her until one o’clock, only then did we sleep
\z

\ea
\gll    {baru}    {Pawlus}    {de}    {sandar}    {di}    {de}    {pu}    {badang}    {begini,}\\
   {and.then}    {Pawlus}    {\textsc{3sg}}    {lean}    {at}    {\textsc{3sg}}    {\textsc{poss}}    {body}    {like.this}\\
\gll {baru}   de    {kas}    {pata}    {leher}    {ke}    {bawa}    {di}    {atas}   de\\
   {and.then}   \textsc{3sg}    {give}    {break}    {neck}    {to}    {bottom}    {at}    {top}   \textsc{3sg}\\
\gll pu    {bahu,}    {de}    {bilang,}    {adu}    {Tuhang}    {tolong,}    {ini}\\
  \textsc{poss}    {shoulder}    {\textsc{3sg}}    {say}    {oh.no!}    {God}    {help}    {\textsc{d.prox}}\\
\gll {siapa?,}    {Tuhang}    {tolong,}    {ini}    {siapa?}    {ini}    {siapa?}\\
   {who}    {God}    {help}    {\textsc{d.prox}}    {who}    {\textsc{d.prox}}    {who}\\
\glt
but then Pawlus leaned on her body like this, and then he bent his neck down onto her shoulder, she said, ‘oh God!, who is this?, God help me, who is this? who is this?’ (Lit. ‘caused his head to be broken’)
\z
%\todo[inline]{Check punctuation!}
\ea
\gll    {baru}    {de}    {su}    {tekang}    {dia}    {ke}    {bawa}    {sini,}    {hampir}\\
   {and.then}    {\textsc{3sg}}    {already}    {press}    {\textsc{3sg}}    {to}    {bottom}    {\textsc{l.prox}}    {almost}\\
\gll de    {mati,}    {mace}    {de}    {berdiri,}    {de}    {berdiri}    {sampe}    {di}   luar,\\
  \textsc{3sg}    {die}    {woman}    {\textsc{3sg}}    {stand}    {\textsc{3sg}}    {stand}    {reach}    {at}   outside\\
\gll {dia}    {lapor}    {ke}\\
   {\textsc{3sg}}    {report}    {to}\\
\glt
but he had already pressed her down, she almost died, the lady got up, she got up and went outside and reported (everything) to
\z

\ea
\gll   Efana:    {mabuk,}   tra,   macang   tida   punya   istri   saja,\\
 {}  {be.drunk}   \textsc{neg}   variety   \textsc{neg}   \textsc{poss}   wife[SI]   just\\
\gll {mabuk}    {takaroang}\\
   {be.drunk}    {be.chaotic}\\
\glt
Efana: to be drunk!, doesn’t, like he doesn’t have a wife, (getting) drunk at random (like this)!
\z

\ea
\gll    {Marta:}    {de}    {lapor}    {ke}    {suster,}    {suster}    {kluar,}    {dia}    {lapor}    {sama}\\
   {}    {\textsc{3sg}}    {report}    {to}    {female.nurse}    {female.nurse}    {go.out}    {\textsc{3sg}}    {report}    {to}\\
\gll {polisi,}    {penjagaang}    {di}    {luar,}    {tinggal}    {tunggu}    {dorang}    {dua,}   dong\\
   {police}    {guard}    {at}    {outside}    {stay}    {wait}    {\textsc{3pl}}    {two}   \textsc{3pl}\\
\gll dua    {di}    {dalam,}    {sampe}    {dong}   dua    {pu}    {kluar}    {dang}    {polisi}\\
  two    {at}    {inside}    {until}    {\textsc{3pl}}   two    {\textsc{poss}}    {go.out}    {and}    {police}\\
\gll {pegang}    {dang}    {dong}    {borgol}    {dorang}    {dua}\\
   {hold}    {and}    {\textsc{3pl}}    {handcuff}    {\textsc{3pl}}    {two}\\
\glt
Marta: she reported (everything) to the female nurse, the female nurse went outside, she reported (everything) to the police, the security outside, it remained for the two of them to wait, the two of them (who were) inside, until the two of them came out and the police got (them) and they handcuffed the two of them\footnote{One of the two detained persons is \textitbf{Pawlus}. It is unclear whether the second person is \textitbf{Iskia} or someone else.}
\z

\ea
\gll   skarang    {ada}    {di}   sel,    {masuk}    {sel}    {ada}   tidor,   siram\\
  now    {exist}    {at}   cell    {enter}    {cell}    {exist}   sleep   pour.over\\
\gll {dengang}    {air}    {baru}    {dong}    {dua}    {tidor}\\
   {with}    {water}    {and.then}    {\textsc{3pl}}    {two}    {sleep}\\
\glt
now they were in a cell, (they) went into a cell to sleep, (the police) splashed (them) with water and the two of them slept
\z

\ea
\gll   Efana:   ditahang,   dikurung\\
{}  \textsc{uv}{}-hold(.out/back)   \textsc{uv}{}-imprison\\
\glt
Efana: (they were) detained, imprisoned
\z

\ea
\gll   Marta:    {ditahangang,}    {polisi}    {kurung,}   mm-mm   tobat\\
 {}   {\textsc{uv}{}-hold(.out/back)-\textsc{pat}}    {police}    {imprison}   mhm   repent\\
 \gll  to?,   karna    {orang{\Tilde}orang}    {kejahatang}    {nakal}\\
  right?   because    {\textsc{rdp}{\Tilde}person}    {evilness}    {be.mischievous}\\
\glt
Marta: (they were) detained, the police imprisoned (them), mhm, to hell with them, right?, because (they are) bad, mischievous people
\z %\end{styleBodyxvafter}

\section{Narrative: A motorbike accident}
\label{Para_B.5}
\begin{tabular}{ll}
\lsptoprule
File name: &  081015-005-NP\\
Text type: &  Elicited text: Personal narrative\footnotemark{}\\
Interlocutors: &  2 older males, 3 older females\\
Length (min.): &  10:29\\
\lspbottomrule
\end{tabular}
\footnotetext{The previous evening, the narrator had already told the same story, but due to logistical problems, the author was not able to record the text. The next morning, however, the narrator was willing to retell her story, with the same audience being present.}
\setcounter{equation}{0}
\ea
\gll   Maria:   saya,   Martina,   Tinus,   kitong\\
{}  \textsc{1sg}   Martina   Tinus   \textsc{1pl}\\
\glt
Maria: I, Martina, Tinus, we
\z

\ea
\gll   Hurki:   kitong   tiga   orang\\
{}  \textsc{1pl}   three   person\\
\glt
Hurki: we (were) three people
\z

\ea
\gll   Maria:   tiga   orang,   tra   ada,   tra   usa\\
 {} three   person   \textsc{neg}   exist   \textsc{neg}   need.to\\
\glt
Maria: three people, no, no need (to mention that)
\z

\ea
\gll   Marta:   kitong   tiga   orang\\
{}  \textsc{1pl}   three   person\\
\glt
Marta: we (were) three people
\z

\ea
\gll    {Maria:}    {nene,}    {kitorang}   tiga   orang   ((pause)),   kitong   lari\\
   {}    {grandmother}    {\textsc{1pl}}   three   person      \textsc{1pl}   run\\
\gll ke    {mari}   sampe    {di}    {jalangang}\\
  to    {hither}   reach    {at}    {route}\\
\glt
Maria: (we) grandmothers, we were three people ((pause)), we drove (along the beach back to \ili{Sarmi}) here (until we) reached the road (Lit. ‘reached the route’)
\z

\ea
\gll   Hurki:   sampe   di   tenga   jalang\\
{}  reach   at   middle   walk\\
\glt
Hurki: (until we) reached the middle of the road
\z

\ea
\gll   Maria:    {a,}    {hssst,}   tida   bole   begitu,   itu   suda   baik\\
{}  {ah!}    {shhh!}   \textsc{neg}   may   like.that   \textsc{d.dist}   already   good\\
\gll {maksut}\hspace{1cm}    {jadi}    {((laughter))}\\
   {purpose}\hspace{1cm}  {so}    {}\\
\glt
Maria: ah, shhh!, (you) shouldn’t (correct me), that’s already good (enough), since the meaning (is already clear) ((laughter))
\z

\ea
\gll   Hurki:   adu,   sampe   di   jalangang\\
{}   oh.no!,   reach   at   journey\\
\glt
Hurki: oh boy!, (until we) reached the road (Lit. ‘reached the route’)
\z

\ea
\gll   Maria:    {ini}    {sampe}    {di}    {jalangang,}    {trus}    {tukang}    {ojek}\\
{}   {\textsc{d.prox}}    {reach}    {at}    {route}    {next}    {craftsman}    {motorbike.taxi}\\
\gll {ini}    {dia}    {tida}    {liat}    {kolam}    {ini,}    {langsung}    {dia}   tabrak\\
   {\textsc{d.prox}}    {\textsc{3sg}}    {\textsc{neg}}    {see}    {big.hole}    {\textsc{d.prox}}    {immediately}    {\textsc{3sg}}   hit.against\\
\gll {itu,}    {kolam}    {ke}    {sana,}    {langsung}    {mama}    {jatu}\\
   {\textsc{d.dist}}    {big.hole}    {to}    {\textsc{l.dist}}    {immediately}    {mother}    {fall}\\
\glt
Maria: what’s-its-name, until (we) reached the road, then this motorbike taxi driver, he didn’t see this big hole, immediately, he hit, what’s-its-name, the hole headlong, (and) immediately, I (‘mother’) fell off
\z

\ea
\gll   sa    {jatu}    {ke}    {blakang,}   Tinus    {ini}    {de}   lari    {trus,}\\
  \textsc{1sg}    {fall}    {to}    {backside}   Tinus    {\textsc{d.prox}}    {\textsc{3sg}}   run    {be.continuous}\\
\gll   {saya}    {suda}    {jatu}   di    {blakang,}   sa    {jatu}    {begini,}   langsung\\
   {\textsc{1sg}}    {already}    {fall}   at    {backside}   \textsc{1sg}    {fall}    {like.this}   immediately\\
\gll   sa    {taguling,}    {sa}    {guling{\Tilde}guling}    {di}    {situ}\\
  \textsc{1sg}    {be.rolled.over}    {\textsc{1sg}}    {\textsc{rdp}{\Tilde}roll.over}    {at}    {\textsc{l.med}}\\
\glt
I fell off backwards, Tinus here, he continued on, I had already fallen off the back (of the motorbike-taxi), as I fell, I rolled over immediately, I rolled over and over there
\z

\ea
\gll    {Tinus,}    {dorang}    {dua}    {dengang}    {Martina}    {ini,}    {dong}    {dua}    {lari}\\
   {Tinus}    {\textsc{3pl}}    {two}    {with}    {Martina}    {\textsc{d.prox}}    {\textsc{3pl}}    {two}    {run}\\
\gll {trus,}    {dong}    {dua}    {lari}    {sampe}   di    {kali,}    {baru}    {Martina}\\
   {be.continuous}    {\textsc{3pl}}    {two}    {run}    {reach}   at    {river}    {and.then}    {Martina}\\
\gll {ini}   de    {kas}    {taw}    {sama}    {tukang}    {ojek}    {ini,}\\
   {\textsc{d.prox}}   \textsc{3sg}    {give}    {know}    {to}    {craftsman}    {motorbike.taxi}    {\textsc{d.prox}}\\
\gll  de    {bilang,}    {a,}    {tukang}    {ojek,}    {itu}    {kitong}   pu\\
  \textsc{3sg}    {say}    {ah!}    {craftsman}    {motorbike.taxi}    {\textsc{d.dist}}    {\textsc{1pl}}   \textsc{poss}\\
\gll  {kawang}    {suda}    {jatu,}    {yang}    {tadi}    {kitong}    {lari}    {ke}    {mari}    {tu}\\
   {friend}    {already}    {fall}    {\textsc{rel}}    {earlier}    {\textsc{1pl}}    {run}    {to}    {hither}    {\textsc{d.dist}}\\
\glt
Tinus, he and Martina here, the two of them drove continuously, the two of them drove on all the way to the river, but then Martina here, she let this motorbike taxi driver know, she said, ‘ah, motorbike taxi driver, what’s-her-name, our friend already fell off, with whom we were driving here earlier’
\z

\ea
\gll   Nofita:   [Is]    {ko}    {liat{\Tilde}liat}    {ke}   sini,   baru   ko   ceritra,\\
   {}   {}   {\textsc{2sg}}    {\textsc{rdp}{\Tilde}see}    {to}   \textsc{l.prox}   and.then   \textsc{2sg}   tell\\
\gll ceritra,    {ko}    {ceritra}    {suda}    {\upshape\textsc{[up]}}\\
  tell    {\textsc{2sg}}    {tell}    {already}    {}\\
\glt
Nofita: [Is] you (have to) look over here, and then you tell the story, tell the story!, just tell the story! [\textsc{up}]
\z

\ea
\gll   Maria:   yo,   biar   de   juga   liat   sa   ((laughter))\\
 {} yes   let   \textsc{3sg}   also   see   \textsc{1sg}   \\
\glt
Maria: yes, (but) let her also see me\footnote{The personal \isi{pronoun} \textitbf{de} ‘\textsc{3sg}’ refers to the recording author.}
\z

\ea
\gll    {skarang}    {tukang}    {ojek}    {ini}    {de}   pulang\\
   {now}    {craftsman}    {motorbike.taxi}    {\textsc{d.prox}}    {\textsc{3sg}}   go.home\\
\gll lagi    {sampe}   di    {tempat}   yang    {dia}    {buang}    {saya}\\
  again    {reach}   at    {place}   \textsc{rel}    {\textsc{3sg}}    {discard}    {\textsc{1sg}}\\
\glt
now this motorbike taxi driver, he returned again all the way to the place where he’d thrown me off
\z

\ea
\gll   Iskia:   minta   maaf   e?,   tolong   ceritra   tu   plang{\Tilde}plang\\
 {} request   pardon   eh?   help   tell   \textsc{d.dist}   \textsc{rdp}{\Tilde}be.slow\\
\glt
Iskia: excuse me, eh?, please talk slowly
\z

\ea
\gll    {Maria:}    {de}    {buang}    {saya,}    {trus}    {dorang}    {dua}    {turung}    {dari}\\
   {}    {\textsc{3sg}}    {discard}    {\textsc{1sg}}    {next}    {\textsc{3pl}}    {two}    {descend}    {from}\\
\gll {motor,}    {dorang}    {dua}    {liat}    {sa}    {begini,}    {sa}    {su}   plaka\\
   {motorbike}    {\textsc{3pl}}    {two}    {see}    {\textsc{1sg}}    {like.this}    {\textsc{1sg}}    {already}   fall.over\\
\gll  ke    {bawa}\\
  to    {bottom}\\
\glt
Maria: he’d thrown me off, then the two of them got off the motorbike, the two of them saw me like this, I had already fallen over to the ground
\z

\ea
\gll    {dong}   dua    {bilang,}    {adu}    {kasiang,}   ko   jatu   ka?,   yo,   dorang\\
   {\textsc{3pl}}   two    {say}    {oh.no!}    {love-\textsc{pat}}   \textsc{2sg}   fall   or   yes   \textsc{3pl}\\
\gll dua    {angkat}   saya,    {trus}    {sa}   tida    {swara}\\
  two    {lift}   \textsc{1sg}    {next}    {\textsc{1sg}}   \textsc{neg}    {voice}\\
\glt
the two of them said, ‘oh no!, poor thing!, did you fall?’ ‘yes’, the two of them lifted me, and I couldn’t speak (Lit. ‘didn’t (have) a voice’)
\z

\ea
\gll    {dorang}   dua    {goyang{\Tilde}goyang}    {saya,}    {dong}   dua    {goyang{\Tilde}goyang}\\
   {\textsc{3pl}}   two    {\textsc{rdp}{\Tilde}shake}    {\textsc{1sg}}    {\textsc{3pl}}   two    {\textsc{rdp}{\Tilde}shake}\\
\gll saya,    {trus}    {sa}    {angkat}    {muka,}    {trus}    {Martina}   de   tanya\\
  \textsc{1sg}    {next}    {\textsc{1sg}}    {lift}    {front}    {next}    {Martina}   \textsc{3sg}   ask\\
\gll saya,    {mama}    {ko}   rasa    {bagemana?}\\
  \textsc{1sg}    {mother}    {\textsc{2sg}}   feel    {how}\\
\glt
the two of them shook me repeatedly, the two of them shook me repeatedly, then I lifted (my) face, then Martina asked me, ‘mother, how do you feel?’
\z

\ea
\gll    {sa}    {bilang}    {begini,}    {sa}    {pusing,}    {mata}    {saya}    {ini}    {glap,}\\
   {\textsc{1sg}}    {say}    {like.this}    {\textsc{1sg}}    {be.dizzy}    {eye}    {\textsc{1sg}}    {\textsc{d.prox}}    {be.dark}\\
\gll {trus}    {Tinus}    {ini}    {de}    {bilang}    {begini}    {sama}    {saya,}    {sa}\\
   {next}    {Tinus}    {\textsc{d.prox}}    {\textsc{3sg}}    {say}    {like.this}    {to}    {\textsc{1sg}}    {\textsc{1sg}}\\
\gll {bisa}    {bawa}    {ko}    {ke}    {Webro}    {ka?,}   trus    {sa}    {bilang}    {begini,}\\
   {be.able}    {bring}    {\textsc{2sg}}    {to}    {Webro}    {or}   next    {\textsc{1sg}}    {say}    {like.this}\\
\gll yo,    {sa}    {jatu,}    {sa}    {rasa}    {kepala}    {pusing,}    {bawa}    {saya}    {ke}   Webro\\
  yes    {\textsc{1sg}}    {fall}    {\textsc{1sg}}    {feel}    {head}    {be.dizzy}    {bring}    {\textsc{1sg}}    {to}   Webro\\
\glt
I said like this, ‘I’m dizzy, my eyes here are dark’, then Tinus here, he said to me like this, ‘can I bring you to Webro?’, then I said like this, ‘yes, I fell, my head feels dizzy, bring me to Webro’
\z

\ea
\gll   trus    {kitorang}    {tiga,}    {kitorang}   tiga   naik   di   motor,   sa   di\\
  next    {\textsc{1pl}}    {three}    {\textsc{1pl}}   three   ascend   at   motorbike   \textsc{1sg}   at\\
\gll {blakang,}    {Martina}    {di}    {tenga}\\
   {backside}    {Martina}    {at}    {middle}\\
\glt
then, we three, we three got onto the motorbike, I (was) in the back (and) Martina was in the middle
\z

\ea
\gll   trus    {tukang}    {ojek}    {ini}   de    {bawa,}    {de}   bawa\\
  next    {craftsman}    {motorbike.taxi}    {\textsc{d.prox}}   \textsc{3sg}    {bring}    {\textsc{3sg}}   bring\\
\gll {kitorang}    {menyebrang,}    {menyebrang}    {ka}    {kali,}    {menyebra}\\
   {\textsc{1pl}}    {cross}    {cross}    {\textsc{tru}{}-river}    {river}    {\textsc{tru}{}-cross}\\
\gll {menyebrang}    {kali}\\
   {cross}    {river}\\
\glt
then this motorbike taxi driver, he took, he took us (and we) crossed, crossed the river[\textsc{tru}] river, (we) crossed[\textsc{tru}] crossed the river
\z
%\todo[inline]{Check punctuation!}
\ea
\gll    {sampe}   di    {Webro}   sa    {pu}    {bapa,}   sa    {pu}    {kaka}    {dorang}\\
   {reach}   at    {Webro}   \textsc{1sg}    {\textsc{poss}}    {father}   \textsc{1sg}    {\textsc{poss}}    {oSb}    {\textsc{3pl}}\\
\gll tanya    {saya,}    {sodara{\Tilde}sodara}    {dorang,}    {knapa?,}    {ko}    {sakit}   ka?\\
  ask    {\textsc{1sg}}    {\textsc{rdp}{\Tilde}sibling}    {\textsc{3pl}}    {why}    {\textsc{2sg}}    {be.sick}   or\\
\glt
having arrived in Webro, my father (and) my older siblings asked me, (my) relatives and friends (asked me), ‘what happened? are you hurt?’
\z

\ea
\gll   sa    {bilang}    {begini,}    {ojek}    {yang}    {buang}    {saya,}    {dong}\\
  \textsc{1sg}    {say}    {like.this}    {motorbike.taxi}    {\textsc{rel}}    {discard}    {\textsc{1sg}}    {\textsc{3pl}}\\
\gll {bilang,}    {ojek}    {mana?,}   a,    {sa}    {pu}    {motor}   ini,\\
   {say}    {motorbike.taxi}    {where}   ah!    {\textsc{1sg}}    {\textsc{poss}}    {motorbike}   \textsc{d.prox}\\
\gll sa    {pu}    {tukang}    {ojek}    {yang}    {buang}    {saya,}\\
  \textsc{1sg}    {\textsc{poss}}    {craftsman}    {motorbike.taxi}    {\textsc{rel}}    {discard}    {\textsc{1sg}}\\
\gll {kurang}    {ajar,}    {kitong}    {pukul}    {dia}    {suda}\\
   {lack}    {teach}    {\textsc{1pl}}    {hit}    {\textsc{3sg}}    {already}\\
\glt
I said like this, ‘(it was) the motorbike taxi driver who threw me off’, they said, ‘which motorbike taxi?’, ‘ah, (it’s) my motorbike here, (it’s) my motorbike taxi driver who threw me off’, ‘damn him!, let us beat him up!’
\z

\ea
\gll   trus    {sa}    {bilang}    {begini,}    {jangang,}    {jangang}    {pukul}    {dia!,}\\
  next    {\textsc{1sg}}    {say}    {like.this}    {\textsc{neg.imp}}    {\textsc{neg.imp}}    {hit}    {\textsc{3sg}}\\
\gll {kasiang,}    {itu}    {manusia,}    {kamorang}    {jangang}    {pukul}   dia!,\\
   {pity}    {\textsc{d.dist}}    {human.being}    {\textsc{2pl}}    {\textsc{neg.imp}}    {hit}   \textsc{3sg}\\
\gll saya    {tida}    {mati,}    {saya}    {ada}\\
  \textsc{1sg}    {\textsc{neg}}    {die}    {\textsc{1sg}}    {exist}\\
\glt
then I said like this, ‘don’t, don’t beat him!, poor thing, he’s a human being, don’t beat him!, I’m not dead, I’m alive’ (Lit. ‘I exist’)
\z

\ea
\gll   trus    {sa}    {tidor,}    {tidor,}    {dorang}    {dua}    {pulang}    {ke}    {Waim,}    {tukang}\\
  next    {\textsc{1sg}}    {sleep}    {sleep}    {\textsc{3pl}}    {two}    {go.home}    {to}    {Waim}    {craftsman}\\
\gll {ojek}    {sama}    {Martina,}    {dong}    {dua}    {pulang}    {sendiri}   ke\\
   {motorbike.taxi}    {to}    {Martina}    {\textsc{3pl}}    {two}    {go.home}    {alone}   to\\
\gll {Waim,}    {sa}    {tinggal}    {karna}    {sa}    {rasa}    {masi}    {pusing}\\
   {Waim}    {\textsc{1sg}}    {stay}    {because}    {\textsc{1sg}}    {feel}    {still}    {be.dizzy}\\
\glt
then I slept, (I) slept, the two of them went home to Waim, the motorbike taxi driver and Martina, the two of them went home alone to Waim, I stayed (in Webro) because I still felt dizzy
\z

\ea
\gll   Nofita:   sap   badang   sakit\\
{}  \textsc{1sg}:\textsc{poss}   body   be.sick\\
\glt
Nofita: my body was hurting
\z

\ea
\gll   Maria:   badang   sakit,   saya   tidor\\
 {} body   be.sick   \textsc{1sg}   sleep\\
\glt
Maria: (my) body was hurting, I slept
\z

\ea
\gll   Nofita:   masak   air   panas\\
{}    cook   water   be.hot\\
\glt
Nofita: (they) boiled hot water
\z

\ea
\gll   Maria:   masak   air   panas\\
 {}     cook   water   be.hot\\
\glt
Maria: (they) boiled hot water
\z

\ea
\gll   Nofita:   Roni   yang   masak   air   panas\\
  {}   Roni   \textsc{rel}   cook   water   be.hot\\
\glt
Nofita: (it was) Roni who boiled hot water
\z

\ea
\gll   Maria:    {Roni,}    {ana}    {mantri}    {ini,}   de    {masak}   air   panas,\\
  {}     {Roni}    {child}    {male.nurse}    {\textsc{d.prox}}   \textsc{3sg}    {cook}   water   be.hot\\
\gll {dorang}    {tolong,}    {dorang}    {bawa}    {air,}    {dorang}    {bawa}    {daung,}\\
   {\textsc{3pl}}    {help}    {\textsc{3pl}}    {bring}    {water}    {\textsc{3pl}}    {bring}    {leaf}\\
\gll{baru}    {dorang}    {urut}    {sa}    {deng}    {itu,}   dong    {bilang,}\\
   {and.then}    {\textsc{3pl}}    {massage}    {\textsc{1sg}}    {with}    {\textsc{d.dist}}   \textsc{3pl}    {say}\\
\gll {badang}    {mana}    {yang}    {sakit?}\\
   {body}    {where}    {\textsc{rel}}    {be.sick}\\
\glt
Maria: Roni, this young male nurse, he boiled hot water, they helped, they brought water, they brought leaves, then they massaged me with those (leaves), they said, ‘which (part of your) body is hurting?’
\z

\ea
\gll    {adu,}    {sa}    {pu}    {bahu}    {sakit,}    {sa}    {pu}    {pinggang}    {sakit,}\\
   {oh.no!}    {\textsc{1sg}}    {\textsc{poss}}    {shoulder}    {be.sick}    {\textsc{1sg}}    {\textsc{poss}}    {loins}    {be.sick}\\
\gll sa    {pu}    {blakang}    {sakit,}   trus    {ana}    {mantri}    {ini,}   de\\
  \textsc{1sg}    {\textsc{poss}}    {backside}    {be.sick}   next    {child}    {male.nurse}    {\textsc{d.prox}}   \textsc{3sg}\\
\gll {urut{\Tilde}urut}    {saya,}    {de}    {pegang{\Tilde}pegang}    {di}    {bahu,}    {de}\\
   {\textsc{rdp}{\Tilde}massage}    {\textsc{1sg}}    {\textsc{3sg}}    {\textsc{rdp}{\Tilde}hold}    {at}    {shoulder}    {\textsc{3sg}}\\
\gll {pegang{\Tilde}pegang}    {blakang}\\
   {\textsc{rdp}{\Tilde}hold}    {backside}\\
\glt
‘ouch!, my shoulder is hurting, my loins are hurting, my back is hurting’, then this young male nurse, he massaged me, he massaged (my) shoulder, he massaged (my) back
\z

\ea
\gll   suda,   saya   tidor   sampe   sore,   sa   pu   laki   datang,\\
  already   \textsc{1sg}   sleep   until   afternoon   \textsc{1sg}   \textsc{poss}   husband   come\\
\gll {Lukas}\\
   {Lukas}\\
\glt
eventually I slept until the afternoon, (then) my husband came, Lukas
\z

\ea
\gll   Nofita:   sa   pu   pacar\\
{}  \textsc{1sg}   \textsc{poss}   date\\
\glt
Nofita: my lover

\z

\ea
\gll    {Maria:}    {a}    {ini}    {orang}    {Papua}    {bilang,}    {sa}    {pu}    {laki,}\\
   {}    {ah!}    {\textsc{d.prox}}    {person}    {Papua}    {say}    {\textsc{1sg}}    {\textsc{poss}}    {husband}\\
\gll sa    {pu}    {laki}    {datang,}   dia    {bilang,}    {kitong}    {dua}    {pulang,}\\
  \textsc{1sg}    {\textsc{poss}}    {husband}    {come}   \textsc{3sg}    {say}    {\textsc{1pl}}    {two}    {go.home}\\
\gll {sa}    {tanya,}   kitong    {dua}    {pulang}    {ke}    {mana?,}    {pulang}    {ke}   Waim\\
   {\textsc{1sg}}    {ask}   \textsc{1pl}    {two}    {go.home}    {to}    {where}    {go.home}    {to}   Waim\\
\glt
Maria: ah, this (is what) Papuans say ‘my husband’, my husband came, he said, ‘we two go home’, I asked, ‘where do we two go home to?’, ‘(we) go home to Waim’
\z

\ea
\gll    {trus}    {kitong}   dua    {pulang,}    {sampe}    {di}    {jalangang}    {sa}    {istirahat,}\\
   {next}    {\textsc{1pl}}   two    {go.home}    {reach}    {at}    {route}    {\textsc{1sg}}    {rest}\\
\gll de    {bilang,}    {kitong}    {dua}    {jalang}    {suda,}    {mata-hari}    {suda}\\
  \textsc{3sg}    {say}    {\textsc{1pl}}    {two}    {road}    {already}    {sun}    {already}\\
\gll {masuk,}    {nanti}    {kitong}    {dua}    {dapat}    {glap,}    {jalang}    {cepat}   suda\\
   {enter}    {very.soon}    {\textsc{1pl}}    {two}    {get}    {be.dark}    {walk}    {be.fast}   already\\
\glt
and then we two went home, on the way I rested, he said, ‘let the two of us walk (on)!, the sun is already going down, in a short while, we’ll be in the dark, walk fast already!’
\z

\ea
\gll   sa    {dengang}    {pace}    {ini}    {kitong}    {dua}    {jalang,}    {ayo,}    {kitong}   dua\\
  \textsc{1sg}    {with}    {man}    {\textsc{d.prox}}    {\textsc{1pl}}    {two}    {walk}    {come.on!}    {\textsc{1pl}}   two\\
\gll  {jalang}    {cepat,}    {kitong}    {dua}    {jalang}    {cepat,}    {kitong}    {dua}    {jalang,}\\
   {walk}    {be.fast}    {\textsc{1pl}}    {two}    {walk}    {be.fast}    {\textsc{1pl}}    {two}    {walk}\\
\gll  {sampe}    {di}    {Waim,}    {dorang{\Tilde}dorang}   di    {situ,}    {masarakat}    {dong}   datang\\
   {reach}    {at}    {Waim}    {\textsc{rdp}{\Tilde}\textsc{3pl}}   at    {\textsc{l.med}}    {community}    {\textsc{3pl}}   come\\
\glt
I and the man here, we two walked, ‘come on!, we two walk fast already!, we two walk fast already!’, the two of us walked, having arrived in Waim, all of them there, the whole community came
\z

\ea
\gll   dong    {bilang,}    {ibu}    {desa}    {ko}    {jatu}    {ka?}    {yo}   sa   jatu,\\
  \textsc{3pl}    {say}    {woman}    {village[SI]}    {\textsc{2sg}}    {fall}    {or}    {yes}   \textsc{1sg}   fall\\
\gll {knapa?}    {sa}    {jatu}    {dari}    {motor,}    {ko}    {pu}    {tulang}    {su}\\
   {why}    {\textsc{1sg}}    {fall}    {from}    {motorbike}    {\textsc{2sg}}    {\textsc{poss}}    {bone}    {already}\\
\gll {pata}    {ka?}    {tra}    {ada,}    {kosong,}    {tra}    {ada,}    {tulang}    {tra}    {pata}\\
   {break}    {or}    {\textsc{neg}}    {exist}    {be.empty}    {\textsc{neg}}    {exist}    {bone}    {\textsc{neg}}    {break}\\
\glt
they said, ‘Ms. Mayor, did you fall?’, ‘yes, I fell’, ‘what happened?’, ‘I fell off the motorbike’, ‘are your bones already broken?’, ‘no, nothing (like that), no, the bones aren’t broken’
\z

\ea
\gll    {suda,}   saya   sampe,   sa   tidor,   tidor,   sa   bangung,   suda\\
   {already}   \textsc{1sg}   reach   \textsc{1sg}   sleep   sleep   \textsc{1sg}   wake.up   already\\
\gll dong    {bilang}\\
  \textsc{3pl}    {say}\\
\glt
eventually I arrived, I slept, (I) slept, I woke up, then they said
\z

\ea
\gll   Nofita:   minum   obat\\
 {} drink   medicine\\
\glt
Nofita: take (your) medicine
\z
%\todo[inline]{Check punctuation!}
\ea
\gll   Maria:    {ko}    {minum}    {obat,}    {suda}   sa    {ambil}   sa   pu\\
{} {\textsc{2sg}}    {drink}    {medicine}    {already}   \textsc{1sg}    {fetch}   \textsc{1sg}   \textsc{poss}\\
\gll {obat,}    {tulang}    {sakit}    {punya,}    {bahu}   yang    {sakit}\\
   {medicine}    {bone}    {be.sick}    {\textsc{poss}}    {shoulder}   \textsc{rel}    {be.sick}\\
\glt
Maria: ‘take (your) medicine!’, then I took my medicine for (my) hurting bone, (it was my) shoulder which was hurting (Lit. ‘the hurting bone’s (medicine)’)
\z

\ea
\gll   sa    {minum,}    {sa}    {minum,}    {sampe}    {tenga}    {malam}    {sa}   minum\\
  \textsc{1sg}    {drink}    {\textsc{1sg}}    {drink}    {until}    {middle}    {night}    {\textsc{1sg}}   drink\\
\gll {lagi,}    {pagi}    {sa}    {bangung,}    {sa}    {makang}    {sagu,}    {makang}\\
   {again}    {morning}    {\textsc{1sg}}    {wake.up}    {\textsc{1sg}}    {eat}    {sago}    {eat}\\
\gll {kasbi,}   sa    {minum}    {lagi}\\
   {cassava}   \textsc{1sg}    {drink}    {again}\\
\glt
I took (medicine), I took (medicine), when it was the middle of the night, I took (medicine) again, in the morning I woke up, I ate sago, (I) ate cassava, I took (medicine) again
\z

\ea
\gll   trus    {sa}   tinggal   sampe   besok,   suda   sa   rasa   badang\\
  next    {\textsc{1sg}}   stay   until   tomorrow   already   \textsc{1sg}   feel   body\\
\gll {suda}    {baik}\\
   {already}    {be.good}\\
\glt
then I stayed until the next day, by then my body already felt good
\z

\ea
\gll    {baru}    {sa}    {punya}    {ana}    {ini,}    {mantri,}    {de}    {pi}    {ambil}\\
   {and.then}    {\textsc{1sg}}    {\textsc{poss}}    {child}    {\textsc{d.prox}}    {male.nurse}    {\textsc{3sg}}    {go}    {fetch}\\
\gll saya,    {kitong}    {dua}    {lari}    {deng}    {motor,}    {dengang}    {Roni,}    {sa}   pu\\
  \textsc{1sg}    {\textsc{1pl}}    {two}    {run}    {with}    {motorbike}    {with}    {Roni}    {\textsc{1sg}}   \textsc{poss}\\
\gll  {ana}    {mantri}    {di}    {Jayapura}    {ini}\\
   {child}    {male.nurse}    {at}    {Jayapura}    {\textsc{d.prox}}\\
\glt
and then, my child here, the male nurse, he came to pick me up, the two of us drove with (his) motorbike, with Roni, my young male nurse from Jayapura
\z

\ea
\gll   MO:   malam\\
 {}    night\\
\glt
[A guest arrives] MO: good evening
\z
%\todo[inline]{Check punctuation!}
\ea
\gll   Maria:    {kitorang}    {dua}    {datang}    {sampe}   di    {sini,}    {ibu}    {pendeta}\\
{}     {\textsc{1pl}}    {two}    {come}    {reach}   at    {\textsc{l.prox}}    {woman}    {pastor}\\
\gll  {ini}    {dia}    {tanya,}    {ko}    {jatu}   ka?    {yo}   sa    {jatu}    {dari}   motor,\\
   {\textsc{d.prox}}    {\textsc{3sg}}    {ask}    {\textsc{2sg}}    {fall}   or    {yes}   \textsc{1sg}    {fall}    {from}   motorbike\\
\gll  {kasiang}    {sayang}\\
   {pity}    {love}\\
\glt
the two of us came all the way here, Ms. Pastor here, she asked (me), ‘did you fall?’, ‘yes, I fell off the motorbike’, ‘poor thing, (my) dear’
\z

\ea
\gll   sa    {tinggal}    {di}    {sini,}    {sa}    {ke}    {ruma-sakit,}    {sa}    {ceritra}   sama\\
  \textsc{1sg}    {stay}    {at}    {\textsc{l.prox}}    {\textsc{1sg}}    {to}    {hospital}    {\textsc{1sg}}    {tell}   to\\
\gll {dokter,}    {dokter,}    {sa}    {jatu}    {dari}    {motor,}    {dokter}    {dorang}    {bilang}\\
   {doctor}    {doctor}    {\textsc{1sg}}    {fall}    {from}    {motorbike}    {doctor}    {\textsc{3pl}}    {say}\\
\gll {begini,}    {ko}    {jatu}    {bagemana?}\\
   {like.this}    {\textsc{2sg}}    {fall}    {how}\\
\glt
I stayed here, I went to the hospital, I talked to the doctor, ‘doctor, I fell off a motorbike’, the doctor and his companions said like this, ‘how did you fall off?’
\z

\ea
\gll   sa    {bilang,}    {sa}    {jatu}    {balik}    {begini,}    {trus}   tulang\\
  \textsc{1sg}    {say}    {\textsc{1sg}}    {fall}    {turn.around}    {like.this}    {next}   bone\\
\gll {pata,}    {sa}    {bilang,}    {tulang}    {bahu}    {yang}    {pata,}    {tulang}\\
   {break}    {\textsc{1sg}}    {say}    {bone}    {shoulder}    {\textsc{rel}}    {break}    {bone}\\
\gll {rusuk,}    {o,}    {a}    {mama}    {itu}    {hanya}    {ko}    {jatu}    {kaget}\\
   {rib}    {oh!}    {ah!}    {mother}    {\textsc{d.dist}}    {only}    {\textsc{2sg}}    {fall}    {feel.startled(.by)}\\
\glt
I said, ‘I fell backwards like this, then the bone broke’, I said, ‘(it’s my) shoulder bone which is broken, (my) ribs’, ‘oh!, ah, mother that is just because you’re under shock’
\z

\ea
\gll   sa    {bilang}    {begini,}    {adu}    {dokter,}    {ini}    {sa}    {jatu}   sengsara\\
  \textsc{1sg}    {say}    {like.this}    {oh.no!}    {doctor}    {\textsc{d.prox}}    {\textsc{1sg}}    {fall}   suffer\\
\gll {ini,}    {harus}    {tolong}    {saya,}   a    {mama,}    {sa}    {kasi}    {obat,}\\
   {\textsc{d.prox}}    {have.to}    {help}    {\textsc{1sg}}   ah!    {mother}    {\textsc{1sg}}    {give}    {medicine}\\
\gll {mama}    {minum,}    {sa}    {bilang,}    {dokter}    {trima-kasi}\\
   {mother}    {drink}    {\textsc{1sg}}    {say}    {doctor}    {thank.you}\\
\glt
I said like this, ‘oh no!, doctor, what’s-its-name, I fell really painfully, (you) have to help me’, ‘ah mother, I give (you) medicine (and) you (‘mother’) take (it)’, I said, ‘doctor, thank you’
\z

\ea
\gll    {sa}    {pulang}    {sampe}    {di}    {sini,}   sa    {bilang}    {ibu}    {pendeta,}\\
   {\textsc{1sg}}    {go.home}    {reach}    {at}    {\textsc{l.prox}}   \textsc{1sg}    {say}    {woman}    {pastor}\\
\gll {ibu}    {ko}    {kas}   sa    {air,}    {sa}    {minum}    {obat,}    {sa}   tinggal\\
   {woman}    {\textsc{2sg}}    {give}   \textsc{1sg}    {water}    {\textsc{1sg}}    {drink}    {medicine}    {\textsc{1sg}}   stay\\
\gll di    {sini}    {satu}    {minggu,}    {e,}    {dua}    {minggu,}    {baru}    {sa}    {pulang}\\
  at    {\textsc{l.prox}}    {one}    {week}    {uh}    {two}    {week}    {and.then}    {\textsc{1sg}}    {go.home}\\
\glt
I went home all the way to here, I told Ms. Pastor, ‘Madam, give me water (so that) I (can) take (my) medicine’, I stayed here for one week, uh, two weeks, only then did I return home
\z

\ea
\gll   sa    {pulang}    {ke}    {Waim}    {lagi,}    {baru}    {kitorang}    {tinggal,}   baru\\
  \textsc{1sg}    {go.home}    {to}    {Waim}    {again}    {and.then}    {\textsc{1pl}}    {stay}   and.then\\
\gll {sa}    {pu}    {masarakat}    {dong}    {tanya}    {saya,}    {ibu}    {ko}    {su}\\
   {\textsc{1sg}}    {\textsc{poss}}    {community}    {\textsc{3pl}}    {ask}    {\textsc{1sg}}    {woman}    {\textsc{2sg}}    {already}\\
\gll {sembu}    {ka?}    {sa}    {bilang,}    {sa}    {su}    {sembu,}    {trima-kasi,}\\
  {be.healed}    {or}    {\textsc{1sg}}    {say}    {\textsc{1sg}}    {already}    {be.healed}    {thank.you}\\
\gll {sampe}    {di}    {sini}\\
   {reach}    {at}    {\textsc{l.prox}}\\
\glt
I went home to Waim again, and then we stayed (there), and then my community asked me, ‘Madam, have you recovered?’, I said, ‘I’ve recovered’, thank you!, this is all (Lit. ‘reach here’)
\z %\end{styleBodyxvafter}

\section{Narrative: Pig hunting with dogs}
\label{Para_B.6}
\begin{tabular}{ll}
\lsptoprule
File name: &  080919-003-NP\\
Text type: &  Elicited text: Personal narrative\footnotemark{}\\
Interlocutors: &  1 older male, 1 older female\\
Length (min.): &  4:20\\
\lspbottomrule
\end{tabular}
\footnotetext{This narrative is one of the three personal narratives mentioned in §1.11.4.1, which the author recorded with the help of her host Sarlota Merne. Being aware of the target language variety, she was present during these elicitations and explained to the narrator that he should narrate his story in \textitbf{logat Papua} ‘Papuan speech variety’. Being one of the early recordings, the text includes quite a few instances of code-switches with Indonesian, which are marked with “[SI]”.}
\setcounter{equation}{0}
\ea
\gll   Iskia:    {jadi}    {satu}    {waktu}    {saya}    {ada}   di    {ruma,}    {malam}    {hari}   saya\\
  {}    {so}    {one}    {time}    {\textsc{1sg}}    {exist}   at    {house}    {night}    {day}   \textsc{1sg}\\
\gll {suda}    {pikir,}    {sa}    {bilang}    {sama}    {ibu,}    {besok}   sa    {bawa}\\
   {already}    {think}    {\textsc{1sg}}    {say}    {to}    {woman}    {tomorrow}   \textsc{1sg}    {bring}\\
\gll {anjing}    {cari}    {babi,}    {sa}    {snang}    {makang}    {babi}\\
   {dog}    {search}    {pig}    {\textsc{1sg}}    {feel.happy(.about)}    {eat}    {pig}\\
\glt
Iskia: so, one time I was at home, at night I had already thought, I told (my) wife, ‘tomorrow I take the dogs and look for pigs’, I like eating pig
\z

\ea
\gll   tong    {tidor}   malam    {sampe}    {pagi}    {saya}   kas    {makang}   anjing\\
  \textsc{1pl}    {sleep}   night    {until}    {morning}    {\textsc{1sg}}   give    {eat}   dog\\
\gll {deng}    {papeda}   yang    {sa}   pu    {bini}    {biking}   malam    {untuk}\\
   {with}    {sagu.porridge}   \textsc{rel}    {\textsc{1sg}}   \textsc{poss}    {wife}    {make}   night    {for}\\
\gll {anjing}    {dorang}\\
   {dog}    {\textsc{3pl}}\\
\glt
we slept through the night until morning, I fed the dogs with papeda which my wife had prepared in the evening for the dogs
\z

\ea
\gll   jadi    {pagi}    {saya}    {bangung,}    {sa}    {kasi}    {makang}   anjing,   sa\\
  so    {morning}    {\textsc{1sg}}    {wake.up}    {\textsc{1sg}}    {give}    {eat}   dog   \textsc{1sg}\\
\gll {pegang}   sa    {pu}    {parang,}    {sa}    {punya}    {jubi,}\\
   {hold}   \textsc{1sg}    {\textsc{poss}}    {short.machete}    {\textsc{1sg}}    {\textsc{poss}}    {bow.and.arrow}\\
\gll sa    {tokiang}    {pana,}   sa    {toki}    {pana}\\
  \textsc{1sg}    {\textsc{spm}{}-beat}    {arrow}   \textsc{1sg}    {beat}    {arrow}\\
\glt
so, in the morning I got up, I fed the dogs, I took my short machete, my bow and arrows, I banged[\textsc{spm}] (my) arrows, I banged my arrows
\z

\ea
\gll   Sarlota:   jubi\\
  {} bow.and.arrow\\
\glt
Sarlota: (I banged my) bow and arrows
\z

\ea
\gll   Iskia:   jubi,   anjing   ikut   saya   masuk   di   hutang\\
{} bow.and.arrow   dog   follow   \textsc{1sg}   enter   at   forest\\
\glt
Iskia: (I banged my) bow and arrows, the dogs followed me entering the forest
\z

\ea
\gll   saya    {jalang}    {sampe}   di    {blakang}    {kebung,}    {anjing}    {mulay}   gong-gong\\
  \textsc{1sg}    {walk}    {reach}   at    {backside}    {garden}    {dog}    {start}   bark(.at)\\
\gll {babi}   o,    {tida}    {lama}   lagi    {dong}    {su}    {kasi}    {berdiri}\\
   {pig}   oh!    {\textsc{neg}}    {be.long}   again    {\textsc{3pl}}    {already}    {give}    {stand}\\
\glt
I walked all the way to the back of (my) garden, the dogs start barking (because they smelt) a pig, oh, not long after that they already had (the pig) standing (still)
\z

\ea
\gll   sa    {lari}    {suda,}    {mendekati}    {babi}    {di}    {mana}    {anjing}    {dong}\\
  \textsc{1sg}    {run}    {already}    {near}    {pig}    {at}    {where}    {dog}    {\textsc{3pl}}\\
\gll {gong-gong,}    {baru}   sa    {mulay}    {pana}    {dia,}    {pana}\\
   {bark(.at)}    {and.then}   \textsc{1sg}    {start}    {bow.shoot}    {\textsc{3sg}}    {bow.shoot}\\
\gll {dengang}    {jubi,}   sa    {jubi}    {dia,}    {langsung}   babi   mati\\
   {with}    {bow.and.arrow}   \textsc{1sg}    {bow.shoot}    {\textsc{3sg}}    {immediately}   pig   die\\
\glt
I just ran closing in on the pig where the dogs were barking, then I started bow shooting (it), bow shooting (it) with (my) bow and arrows, I bow shot it, immediately the pig died
\z

\ea
\gll   wa,    {babi}    {besar}    {skali,}    {sa}    {sendiri}    {tra}    {bisa}   angkat,   sa\\
  wow!    {pig}    {be.big}    {very}    {\textsc{1sg}}    {alone}    {\textsc{neg}}    {be.able}   lift   \textsc{1sg}\\
\gll {pikir,}    {adu,}    {babi}    {ni}    {sa}    {harus}    {angkat}    {bagemana,}\\
   {think}    {oh.no!}    {pig}    {\textsc{d.prox}}    {\textsc{1sg}}    {have.to}    {lift}    {how}\\
\gll {ini}    {besar}    {ini}\\
   {\textsc{d.prox}}    {be.big}    {\textsc{d.prox}}\\
\glt
wow!, the pig was very big, I alone could not transport it, I thought, ‘oh no!, this pig, how am I going to transport (it), this (one) here is really big’
\z

\ea
\gll   tida    {lama}    {sa}   dengar   ada   swara,   orang,\\
  \textsc{neg}    {be.long}    {\textsc{1sg}}   hear   exist   voice   person\\
\gll  {baru}    {saya}    {panggil}\\
   {and.then}    {\textsc{1sg}}    {call}\\
\glt
not long after that I heard there were voices, (there were) people, and then I called (them)
\z

\ea
\gll    {mereka}    {ada}    {tiga}    {orang,}    {dorang}    {datang,}    {dengar}    {ini,}    {anjing}\\
   {\textsc{3pl}[SI]}    {exist}    {three}    {person}    {\textsc{3pl}}    {come}    {hear}    {\textsc{d.prox}}    {dog}\\
\gll {gong-gong}    {babi,}    {tapi}    {sementara,}    {karna}    {mereka}    {jaw,}   lari\\
   {bark(.at)}    {pig}    {but}    {in.meantime[SI]}    {because}    {\textsc{3pl}[SI]}    {far}   run\\
\gll mo    {pana}    {babi}    {bantu}    {sama}    {dengang}    {saya,}    {tapi}    {saya}\\
  want    {bow.shoot}    {pig}    {help}    {to}    {with}    {\textsc{1sg}}    {but}    {\textsc{1sg}}\\
\gll {suda}    {bunu,}    {pana}    {dia}    {kemuka}\\
   {already}    {kill}    {bow.shoot}    {\textsc{3sg}}    {first.before.others}\\
\glt
they were three people, they came (and) heard, what’s-its-name, the dogs barking at the pig, but in the meantime, because they were far away, (they) ran wanting to bow shoot the pig, to help me, but I had already killed (it), had bow shot (it) before the others
\z

\ea
\gll   waktu    {mereka}    {sampe}    {dekat}    {saya,}    {babi}   suda   mati   jadi\\
  time    {\textsc{3pl}[SI]}    {reach}    {near}    {\textsc{1sg}}    {pig}   already   die   so\\
\gll {tinggal}   sa    {bilang}    {saja,}    {babi}    {suda}    {mati}\\
   {stay}   \textsc{1sg}    {say}    {just}    {pig}    {already}    {die}\\
\glt
when they arrived near me the pig was already dead, so it just remained for me to say, ‘the pig is already dead’
\z

\ea
\gll   jadi    {nanti}    {kitong}    {berusaha}   pikol   ke    {ruma}   kebung,\\
  so    {very.soon}    {\textsc{1pl}}    {attempt}   shoulder   to    {house}   garden\\
\gll {baru}    {nanti}    {kita}   potong,   baru    {nanti}    {bagi}\\
   {and.then}    {very.soon}    {\textsc{1pl}}   cut   and.then    {very.soon}    {divide}\\
\glt
so later we’ll try to carry the pig on our shoulders to the garden shelter, only then we’ll cut it up, and then we’ll distribute (it)
\z
%\todo[inline]{Check punctuation!}
\ea
\gll   itu   juga,   a,    {tong}    {langsung}   ambil   itu,   pikol\\
  \textsc{d.dist}   also   ah!    {\textsc{1pl}}    {immediately}   fetch   \textsc{d.dist}   shoulder\\
\gll  itu,   babi,    {bawa}    {ke}   ruma    {kebung}\\
  \textsc{d.dist}   pig    {bring}    {to}   house    {garden}\\
\glt
right after that, ah, we took it immediately, we shouldered it, the pig, (and) carried (it) to the garden shelter
\z

\ea
\gll   tong    {potong}    {hari}    {itu,}    {tong}    {bagi}    {buat}    {kitorang}   yang\\
  \textsc{1pl}    {cut}    {day}    {\textsc{d.dist}}    {\textsc{1pl}}    {divide}    {for}    {\textsc{1pl}}   \textsc{rel}\\
\gll {potong}    {itu}    {hari,}    {kemudiang}    {buat}    {sodara{\Tilde}sodara}    {yang}\\
   {cut}    {\textsc{d.dist}}    {day}    {then[SI]}    {for}    {\textsc{rdp}{\Tilde}sibling}    {\textsc{rel}}\\
\gll  {tinggal}    {di}    {kampong,}    {kitong}    {hitung}    {ada}    {dua}    {pulu}    {satu}\\
   {stay}    {at}    {village}    {\textsc{1pl}}    {count}    {exist}    {two}    {tens}    {one}\\
\gll  {KK}   di    {sa}    {punya}    {kampung}    {itu}\\
   {household.head}   at    {\textsc{1sg}}    {\textsc{poss}}    {village}    {\textsc{d.dist}}\\
\glt
we cut (it) up that day, we divided (it) for us who cut (it) up that day, (and) then for the relatives and friends who live in the village, we counted (them), there are twenty one heads of households in that village of mine
\z

\ea
\gll   jadi,   waktu    {saya}    {potong}   babi   ini,   daging   saya\\
  so   time    {\textsc{1sg}}    {cut}   pig   \textsc{d.prox}   meat   \textsc{1sg}\\
\gll {memperkecil,}    {saya}   bagi    {juga}\\
   {make.smaller}    {\textsc{1sg}}   divide    {also}\\
\glt
so, when I cut up this pig, the meat, I cut (it) into small pieces, (and) I distributed them
\z

\ea
\gll   Sarlota:   potong   kecil{\Tilde}kecil\\
  {} cut   \textsc{rdp}{\Tilde}be.small\\
\glt
Sarlota: (I) cut (it into) small (pieces)
\z

\ea
\gll   Iskia:    {kecil{\Tilde}kecil,}    {baru}   saya   bagi   sampe   dua   pulu\\
  {} {\textsc{rdp}{\Tilde}be.small}    {and.then}   \textsc{1sg}   divide   reach   two   tens\\
\gll {bagi,}   dua   pulu   satu    {bagiang}\\
   {\textsc{tru}{}-part}   two   tens   one    {part}\\
\glt
Iskia: small (pieces), and then I divided (them) into twenty parts[\textsc{tru}], twenty one parts
\z

\ea
\gll    {waktu}   kita    {pulang,}   ta    {p,}    {empat}    {orang}   itu,\\
   {time}   \textsc{1pl}    {go.home}   \textsc{1pl}    {\textsc{tru}{}-go.home}    {four}    {person}   \textsc{d.dist}\\
\gll kita    {suda}   bawa,    {masing-masing,}    {kita}    {suda}   baku    {bagi}\\
  \textsc{1pl}    {already}   bring    {each}    {\textsc{1pl}}    {already}   \textsc{recp}    {divide}\\
\glt
when we went home, (when) we went home[\textsc{tru}], those four people, we brought (the meat) already (having been divided up), each of us, we had already divided (the meat) with each other
\z

\ea
\gll    {nanti}    {ko}    {kasi}    {sodara}    {yang}    {laing,}    {saya}   juga    {nanti}\\
   {very.soon}    {\textsc{2sg}}    {give}    {sibling}    {\textsc{rel}}    {be.different}    {\textsc{1sg}}   also    {very.soon}\\
\gll {bagi}    {so}    {sodara}    {yang}    {laing}    {suda}    {punya}    {bagiang,}\\
   {divide}    {\textsc{tru}{}-sibling}    {sibling}    {\textsc{rel}}    {be.different}    {already}    {have}    {part}\\
\gll {tinggal}    {kita}    {bawa}    {sampe}    {di}    {ruma,}    {suda}    {sore}   hari,\\
   {stay}    {\textsc{1pl}}    {bring}    {reach}    {at}    {house}    {already}    {afternoon}   day\\
\gll  kita    {bagi}    {malam}\\
  \textsc{1pl}    {divide}    {night}\\
\glt
later you give (the meat) to other friends and relatives, later I’ll also distribute (it to) other friends and relatives, (we) already have (our) share, it remains that we bring (our share home), having arrived home, it was already afternoon, we distributed (the meat until) the evening
\z

\ea
\gll    {sodara{\Tilde}sodara}    {dorang}    {mo}    {masak}   sayur,   liat   begini,\\
   {\textsc{rdp}{\Tilde}sibling}    {\textsc{3pl}}    {want}    {cook}   vegetable   see   like.this\\
\gll ta   bawa    {daging,}    {siapa}    {yang}    {dapat?}\\
  \textsc{1pl}   bring    {meat}    {who}    {\textsc{rel}}    {get}\\
\glt
the relatives and friends wanted to cook vegetables, as (they) saw that we brought (them) meat (they asked us), ‘who (is the one) who got (the pig)?’
\z

\ea
\gll    {bilang,}    {saya}    {yang}    {tadi}    {pagi}    {berburu,}    {bawa}    {anjing,}   baru\\
   {say}    {\textsc{1sg}}    {\textsc{rel}}    {earlier}    {morning}    {hunt}    {bring}    {dog}   and.then\\
\gll dapat    {babi}    {ini,}    {betulang,}    {ini}    {daging}    {yang}   saya   bawa,\\
  get    {pig}    {\textsc{d.prox}}    {chance}    {\textsc{d.prox}}    {meat}    {\textsc{rel}}   \textsc{1sg}   bring\\
\gll {antar}    {buat}    {sodara}    {dorang}\\
   {deliver}    {for}    {sibling}    {\textsc{3pl}}\\
\glt
(I) said, ‘(it was) me who went hunting this morning, (who) took the dogs and then got this pig, coincidentally, this is the meat which I brought, (which I) delivered for (my) relatives’
\z

\ea
\gll   dong    {bilang,}    {trima-kasi,}    {tong}    {mo}    {makang}    {sayur}    {malam}\\
  \textsc{3pl}    {say}    {thank.you}    {\textsc{1pl}}    {want}    {eat}    {vegetable}    {night}\\
\gll {ini,}   tapi    {ya}    {sodara}    {ko}    {bawa}    {daging,}    {kitong}    {trima-kasi,}\\
   {\textsc{d.prox}}   but    {yes}    {sibling}    {\textsc{2sg}}    {bring}    {meat}    {\textsc{1pl}}    {thank.you}\\
\gll {karna}    {kitong}    {bisa}    {masak}    {daging,}    {sodara}    {berburu}    {daging,}   babi\\
   {because}    {\textsc{1pl}}    {be.able}    {cook}    {meat}    {sibling}    {hunt}    {meat}   pig\\
\glt
they said, ‘thank you!, we were going to eat vegetables tonight, but, yes, you brother brought (us) meat, we say thank you, because (now) we can cook meat, (you) brother hunted meat, a pig’
\z

\ea
\gll   jadi    {ini}    {kehidupang}    {orang}    {Papua}    {ini}   sperti   begini,\\
  so    {\textsc{d.prox}}    {life}    {person}    {Papua}    {\textsc{d.prox}}   similar.to   like.this\\
\gll {kalo}   mo    {makang}    {babi,}    {harus}    {bawa}    {anjing}\\
   {if}   want    {eat}    {pig}    {have.to}    {bring}    {dog}\\
\glt
so, what’s-its-name, the life of (us) Papuan people here is like this: if (you) want to eat pig, (you) have to take dogs (with you)
\z

\ea
\gll    {kemudiang,}    {itu}   ceritra   waktu   kita   berburu   pake   anjing,\\
   {then[SI]}    {\textsc{d.dist}}   tell   time   \textsc{1pl}   hunt   use   dog\\
\gll ya,    {sperti}    {itu}\\
  yes    {similar.to}    {\textsc{d.dist}}\\
\glt
then, this was the story when we go hunting and use dogs, yes, it’s like that
\z %\end{styleBodyxvafter}

\section{Expository: Directions to a certain statue and tree}
\label{Para_B.7}
After having recounted a story about a certain statue in \ili{Sarmi} which was built close to a certain tree, a boy asked the narrator for directions to the statue and the tree.

\begin{tabular}{ll}
\lsptoprule
File name: &  080917-009-CvEx\\
Text type: &  Conversation, spontaneous: Expository\\
Interlocutors: &  2 male child, 1 older female\\
Length (min.): &  0:50\\
\lspbottomrule
\end{tabular}
\setcounter{equation}{0}
\ea
\gll    {Natalia:}  {\ldots}    {greja}    {sebla,}    {pokoknya}    {ruma}\\
   {}  {}    {church}    {side}    {the.main.thing.is}    {house}\\
\gll tingkat    {itu,}    {ruma-sakit}    {itu}   sebla    {itu}   ada   {\upshape\textsc{[up]}}\\
  floor    {\textsc{d.dist}}    {hospital}    {\textsc{d.dist}}   side    {\textsc{d.dist}}   exist   \\
\glt
[Reply about the directions to a certain statue:] Natalia: {\ldots} next to the church, the main landmark is the multistoried house, that hospital next to it is [\textsc{up}]
\z

\ea
\gll   Wili:   yang   Matias   de   ada   sakit   itu?\\
{}  \textsc{rel}   Matias   \textsc{3sg}   exist   be.sick   \textsc{d.dist}\\
\glt
Wili: where Matias was sick?
\z

\ea
\gll    {Natalia:}    {yo,}    {Matias}    {ada}    {sakit}    {itu,}    {liat}    {sebla}    {laut}   itu\\
   {}    {yes}    {Matias}    {exist}    {be.sick}    {\textsc{d.dist}}    {see}    {side}    {sea}   \textsc{d.dist}\\
\gll dong    {ada}    {biking}    {begini,}    {besar}    {de}    {pu}    {tugu,}\\
  \textsc{3pl}    {exist}    {make}    {like.this}    {be.big}    {\textsc{3sg}}    {\textsc{poss}}    {monument}\\
\gll {baru,}    {a,}    {dong}    {biking}    {bagus,}    {smeng}    {bagus}    {skali,}\\
   {and.then}    {ah!}    {\textsc{3pl}}    {make}    {be.good}    {cement}    {be.good}    {very}\\
\gll {nanti}    {kalo}    {ko}    {blum}    {taw}\\
   {very.soon}    {if}    {\textsc{2sg}}    {not.yet}    {know}\\
\glt
Natalia: yes, Matias was sick there, look toward the ocean, they made (the statue) like this, big is its statue, and then, ah, they built it well, (they) cemented it very well, later (you’ll see it), if you don’t know (it) yet
\z

\ea
\gll    {nanti}    {tanya}    {Matias,}    {bilang,}   Matias   ko   bawa   sa\\
   {very.soon}    {ask}    {Matias}    {say}   Matias   \textsc{2sg}   bring   \textsc{1sg}\\
\gll pergi    {liat}    {tugu}    {itu}    {ka?}\\
  go    {see}    {monument}    {\textsc{d.dist}}    {or}\\
\glt
later ask Matias, say (to him), ‘will you Matias take me to go and see that statue?’
\z

\ea
\gll   Wili:   naik   ke   atas?\\
  {} ascend   to   top\\
\glt
Wili: (to get there one has to) climb up (the hill)?
\z

\ea
\gll   Natalia:    {tra}   naik,   di   dekat   puskesmas   itu,\\
  {} {\textsc{neg}}   ascend   at   near   government.clinic   \textsc{d.dist}\\
\gll {ruma-sakit}    {situ}\\
   {hospital}    {\textsc{l.med}}\\
\glt
Natalia: (you) don’t (have to) climb, (the statue) is close to that government clinic, the hospital there
\z

\ea
\gll   Wili:   tra   liat\\
  {}   \textsc{neg}   see\\
\glt
Wili: (I) didn’t see (it)
\z

\ea
\gll    {Natalia:}    {ko}    {blum}    {liat,}    {a,}    {nanti}    {baru}    {Matias}   ka\\
   {}    {\textsc{2sg}}    {not.yet}    {see}    {ah!}    {very.soon}    {and.then}    {Matias}   or\\
\gll ato    {nanti}    {besok}    {ka,}    {deng}    {mama-ade}    {jalang,}    {baru}\\
  or    {very.soon}    {tomorrow}    {or}    {with}    {aunt}    {walk}    {and.then}\\
\gll {mama-ade}    {kas}    {tunjuk,}    {baru}    {sa}    {kas}    {tunjuk}    {pohong}   yang\\
   {aunt}    {give}    {show}    {and.then}    {\textsc{1sg}}    {give}    {show}    {tree}   \textsc{rel}\\
\gll {Matias}    {de}    {takut,}    {pohong}    {tagoyang,}    {e,}    {Ise,}    {Ise}\\
   {Matias}    {\textsc{3sg}}    {feel.afraid(.of)}    {tree}    {be.shaken}    {uh}    {Ise}    {Ise}\\
\gll {dia}    {takut}    {pohong}    {tagoyang,}    {jadi}    {de}    {menangis}\\
   {\textsc{3sg}}    {feel.afraid(.of)}    {tree}    {be.shaken}    {so}    {\textsc{3sg}}    {cry}\\
\glt
Natalia: you haven’t yet seen (the statue)?, ah, later on, maybe Matias or maybe tomorrow (you) walk (there) with me (‘aunt’), and then I (‘aunt’) will show, and then I’ll show (you) the tree which Matias was afraid of, the shaking tree, uh Ise, Ise was afraid of the shaking tree, so she cried
\z

\ea
\gll   Wili:   yang   dekat   ruma-sakit?\\
  {} \textsc{rel}   near   hospital\\
\glt
Wili: (the one) which is close to the hospital?
\z

\ea
\gll   MC:   di   mana?\\
{}  at   where\\
\glt
MC: where?
\z

\ea
\gll   Natalia:   di   dekat   ruma-sakit   sebla   laut   dulu\\
 {}  at   near   hospital   side   sea   first\\
\glt
Natalia: (it) is close to the hospital toward the ocean, in the past
\z

\ea
\gll   Wili:   yang   dekat   ada   ruma   to?\\
 {} \textsc{rel}   near   exist   house   right?\\
\glt
Wili: close by where the houses are, right?
\z

\ea
\gll   Natalia:    {mm-mm,}   ruma   di   pante,   jadi   luar   biasa\\
  {}    {mhm}   house   at   coast   so   outside   be.usual\\
\gll {((pause))}    {sampe}\\
   {}    {until}\\
\glt
Natalia: mhm, the houses along the beach, so this has been magnificent\footnote{This clause refers to the story about the statue and the tree which the narrator had told before being asked for directions.} ((pause)) until
\z

\ea
\gll   Wili:   skarang   ini\\
 {} now   \textsc{d.prox}\\
\glt
Wili: now!
\z

\ea
\gll   Natalia:    {skarang,}    {say}    {kembali,}    {pulang}    {dari}    {skola,}   sa   di\\
  {} {now}    {\textsc{1sg}}    {return}    {go.home}    {from}    {school}   \textsc{1sg}   at\\
\gll {Pante-Timur,}    {Takar,}    {ke}   sini,   itu    {Tuhang}    {buat}    {luar}\\
   {Pante-Timur}    {Takar}    {to}   \textsc{l.prox}   \textsc{d.dist}    {God}    {make}    {outside}\\
\gll  biasa    {itu}\\
  be.usual    {\textsc{d.dist}}\\
\glt
Natalia: now (after this experience with the statue and the tree), I returned (to Jayapura), (I) went home (after having finished) college, (then) I (stayed) in Pante-Timur, (in) Takar, (then I came) here, all this, God made it wonderful
\z %\end{styleBodyxvafter}

\section{Expository: Sterility}
\label{Para_B.8}
\begin{tabular}{ll}
\lsptoprule
File name: &  081006-030-CvEx\\
Text type: &  Conversation with the author: Expository\\
Interlocutors: &  1 older female\\
Length (min.): &  1:56\\
\lspbottomrule
\end{tabular}
\setcounter{equation}{0}
\ea
\gll    {Natalia:}   jadi    {ada}    {dua,}    {kalo}    {misalnya,}   kita    {suda,}    {kitong}\\
   {}   so    {exist}    {two}    {if}    {for.example}   \textsc{1pl}    {already}    {\textsc{1pl}}\\
\gll {su}    {bayar}    {mas-kawing}    {to?,}    {baru}    {prempuang}    {itu}   de\\
   {already}    {pay}    {bride.price}    {right?}    {and.then}    {woman}    {\textsc{d.dist}}   \textsc{3sg}\\
\gll tra    {hamil,}    {na,}    {mungking}    {ada}    {masala}    {menge,}\\
  \textsc{neg}    {be.pregnant}    {well}    {maybe}    {exist}    {problem}    {\textsc{tru}{}-concern[SI]}\\
\gll {dari}    {kesehatang,}    {kita}    {bisa}    {liat}    {dari}    {kesehatang,}    {a}\\
   {from}    {health}    {\textsc{1pl}}    {be.able}    {see}    {from}    {health}    {ah!}\\
\glt
Natalia: so there’re two (issues related to sterility), if, for example, we’ve already, (if) we’ve already paid the bride price, right? and (if) that woman, (if) she doesn’t get pregnant, well, maybe there is a problem regarding[\textsc{tru}], due to health (issues), we can see (that the problem of sterility) is due to a health (problem), ah! (Lit. ‘from health’)
\z

\ea
\gll    {bisa}    {juga}    {ada}    {pikirang}    {laki{\Tilde}laki}   itu   de    {mandul}\\
   {be.able}    {also}    {exist}    {thought}    {\textsc{rdp}{\Tilde}husband}   \textsc{d.dist}   \textsc{3sg}    {be.sterile}\\
\gll ato    {prempuang}    {itu}    {de}   mandul,    {makanya}   tida   ada\\
  or    {woman}    {\textsc{d.dist}}    {\textsc{3sg}}   be.sterile    {for.that.reason}   \textsc{neg}   exist\\
\gll {ana}    {sama}    {skali}\\
   {child}    {same}    {very}\\
\glt
it’s also possible that there is the thought, (that) that man, (that) he’s sterile or (that) that woman, (that) she’s sterile, for that reason there aren’t any children at all
\z

\ea
\gll   a,   nanti   liat,   tinggal,   tinggal,   tinggal\\
  ah!   very.soon   see   stay   stay   stay\\
\glt
ah, later (we’ll) see, (we’ll) wait, wait, (and) wait
\z

\ea
\gll    {kalo}    {prempuang,}    {laki{\Tilde}laki}    {itu}    {dia}    {maw}    {turungang}   to?,\\
   {if}    {woman}    {\textsc{rdp}{\Tilde}husband}    {\textsc{d.dist}}    {\textsc{3sg}}    {want}    {descendant}   right?\\
\gll {dia}    {maw}    {ada}    {ana}    {lagi,}    {orang}    {Papua}    {punya}    {kebiasaang,}\\
   {\textsc{3sg}}    {want}    {exist}    {child}    {again}    {person}    {Papua}    {\textsc{poss}}    {habit}\\
\gll a,    {dia}    {e}    {kawing}    {ini}    {harus}    {ada}   ana    {karna}\\
  ah!    {\textsc{3sg}}    {uh}    {marry.unofficially}    {\textsc{d.prox}}    {have.to}    {exist}   child    {because}\\
\gll {dia}    {harus}    {ada}    {turungang,}    {a,}    {nanti}    {laki{\Tilde}laki}    {itu}\\
   {\textsc{3sg}}    {have.to}    {exist}    {descendant}    {ah!}    {very.soon}    {\textsc{rdp}{\Tilde}husband}    {\textsc{d.dist}}\\
\gll {dia}    {kawing}    {prempuang}    {laing}\\
   {\textsc{3sg}}    {marry.unofficially}    {woman}    {be.different}\\
\glt
if that woman (or) man, (if) he/she wants offsprings, right?, (if) he/she also wants to have children, the Papuan people’s habit, ah, (when) he/she here, what’s-its-name, marries (then) there have to be children because he/she has to have offsprings, ah, (otherwise) later that man, he’ll marry a different woman
\z

\ea
\gll   a,    {de}    {kawing}    {prempuang}    {laing,}    {prempuang}\\
  ah!    {\textsc{3sg}}    {marry.unofficially}    {woman}    {be.different}    {woman}\\
\gll {itu}    {ada}   ana    {o,}    {kalo}    {begitu,}    {prempuang}    {ini}   yang\\
   {\textsc{d.dist}}    {exist}   child    {oh!}    {if}    {like.that}    {woman}    {\textsc{d.prox}}   \textsc{rel}\\
\gll {mandul,}    {prempuang}    {ini}    {tra}    {ada}   ana,    {begitu}\\
   {be.sterile}    {woman}    {\textsc{d.prox}}    {\textsc{neg}}    {exist}   child    {like.that}\\
\glt
ah, (when) he marries a different woman, (and when) that woman has children (we’ll know), ‘oh, in that case, (it’s) this (first) woman who’s sterile, this (first) woman doesn’t have children’, (it’s) like that
\z

\ea
\gll   tapi    {kalo,}    {macang}    {prempuang}    {de}    {kasi}    {tinggal}    {laki{\Tilde}laki,}\\
  but    {if}    {variety}    {woman}    {\textsc{3sg}}    {give}    {stay}    {\textsc{rdp}{\Tilde}husband}\\
\gll  {prempuang}    {de}    {kawing}    {deng}    {laki{\Tilde}laki}    {laing,}\\
   {woman}    {\textsc{3sg}}    {marry.unofficially}    {with}    {\textsc{rdp}{\Tilde}husband}    {be.different}\\
\gll {prempuang}    {itu}    {dapat}    {ana}    {o,}    {laki{\Tilde}laki}    {yang}   mandul,\\
   {woman}    {\textsc{d.dist}}    {get}    {child}    {oh!}    {\textsc{rdp}{\Tilde}husband}    {\textsc{rel}}   be.sterile\\
\gll  {kalo}    {itu}    {memang,}    {e,}    {diliat}    {dari}    {kesehatang}\\
   {if}    {\textsc{d.dist}}    {indeed}    {uh}    {\textsc{uv}{}-see}    {from}    {health}\\
\glt
but if, for example, the woman leaves (her) husband (and if) the woman marries a different man (and if) that woman has children (we’ll know), ‘oh, (it’s) the (first) man who’s sterile’, if it’s like that indeed, umh, (the issue of sterility) is due to a health (problem)
\z

\ea
\gll   Author:   yo\\
{} yes\\
\glt
Author: yes
\z

\ea
\gll    {Natalia:}    {begitu,}   tapi   kalo   kita   suda   bayar   mas-kawing,\\
   {}    {like.that}   but   if   \textsc{1pl}   already   pay   bride.price\\
\gll  kalo    {kita}   pikir    {to?}\\
  if    {\textsc{1pl}}   think    {right?}\\
\glt
Natalia: (it’s) like that, but if we’ve already paid the bride price, if we think, right?
\z

\ea
\gll   o,    {mungking}    {kitong}    {blum}    {bayar}    {mas-kawing,}   de    {tra}\\
  oh!    {maybe}    {\textsc{1pl}}    {not.yet}    {pay}    {bride.price}   \textsc{3sg}    {\textsc{neg}}\\
\gll {hamil,}    {baru}    {kitong}    {bayar}    {mas-kawing,}    {tinggal,}    {tinggal,}\\
   {be.pregnant}    {and.then}    {\textsc{1pl}}    {pay}    {bride.price}    {stay}    {stay}\\
\gll {tinggal,}    {tinggal,}    {bereskang}    {smua}    {masala}    {apa,}    {prempuang}   tra\\
   {stay}    {stay}    {clean.up}    {all}    {problem}    {what}    {woman}   \textsc{neg}\\
\gll {hamil,}    {o,}    {ini}    {prempuang}    {de}    {mandul}\\
   {be.pregnant}    {oh!}    {\textsc{d.prox}}    {woman}    {\textsc{3sg}}    {be.sterile}\\
\glt
‘oh, maybe we haven’t yet paid the bride price, (and that’s the reason why) she’s not pregnant’, but then we pay the bride price, (and we) wait, wait, wait, (and) wait, (we) settle all problems what(ever they may be, and) the woman is (still) not pregnant, (then we’ll know,) ‘oh, this is because the woman is sterile’
\z

\ea
\gll   kalo    {orang}    {yang}    {blum}    {bertobat,}    {bukang}    {hamba}    {Tuhang,}    {dia}\\
  if    {person}    {\textsc{rel}}    {not.yet}    {repent}    {\textsc{neg}}    {servant}    {God}    {\textsc{3sg}}\\
\gll {kawing}    {satu}   lagi,    {de}    {kawing}    {satu,}    {prempuang}\\
   {marry.unofficially}    {one}   again    {\textsc{3sg}}    {marry.unofficially}    {one}    {woman}\\
\gll {itu}    {ada}    {ana,}    {kawing}    {satu,}    {prempuang}    {ada}   ana,\\
   {\textsc{d.dist}}    {exist}    {child}    {marry.unofficially}    {one}    {woman}    {exist}   child\\
\gll {baru}    {o,}    {kalo}    {begitu}    {prempuang}    {ini}    {mandul}\\
   {and.then}    {oh!}    {if}    {like.that}    {woman}    {\textsc{d.prox}}    {be.sterile}\\
\glt
if someone isn’t a Christian yet (and) is not a servant of God, (if) he marries another woman, (if) he marries another (woman and) that woman has children, (if) he marries another (woman and) the woman has children, then (we’ll know), ‘oh, if it’s like that, (then) this (first) woman is sterile’ (Lit. ‘if someone hasn’t yet repented’)
\z
%\todo[inline]{Angela, please check  {\textsc{3{\textasciigrave}sg}}} 
\ea
\gll   de    {tida,}    {orang}    {Papua}    {bilang}    {[Is]}    {makanya}    {orang}   itu\\
  \textsc{3sg}    {\textsc{neg}}    {person}    {Papua}    {say}    {}    {for.that.reason}    {person}   \textsc{d.dist}\\
\gll  {tida}    {ada}    {ana,}    {mandul,}    {jadi}    {de}   pu,    {tida}   ada    {ana}\\
   {\textsc{neg}}    {exist}    {child}    {be.sterile}    {so}    {\textsc{3sg}}   \textsc{poss}    {\textsc{neg}}   exist    {child}\\
\gll {jadi}    {mandul,}    {begitu}\\
   {so}    {be.sterile}    {like.that}\\
\glt
he/she doesn’t, the Papuan people say ‘[Is]’, that is to say, that person doesn’t have children, (he/she’s) sterile, so, his/her, (he/she) doesn’t have children, so (he/she’s) sterile, (it’s) like that
\z %\end{styleBodyxvafter}

\section{Hortatory: Don’t get dirty!}
\label{Para_B.9}
\begin{tabular}{ll}
\lsptoprule
File name: &  080917-004-CvHt\\
Text type: &  Conversation, spontaneous: Hortatory\\
Interlocutors: &  2 male children,\footnotemark{} 1 older female\\
Length (min.): &  0:10\\
\lspbottomrule
\end{tabular}
\setcounter{equation}{0}
\footnotetext{The second male child did not participate in this exchange.}

\ea
\gll    {Wili:}    {Nofi}    {nanti}   ko   kejar   saya,   ko   liat,   ko   tunggu,\\
   {}    {Nofi}    {very.soon}   \textsc{2sg}   chase   \textsc{1sg}   \textsc{2sg}   see   \textsc{2sg}   wait\\
\gll tong    {dua}    {bla,}    {baru}\\
  \textsc{1pl}    {two}    {split}    {and.then}\\
\glt
Wili: Nofi, in a moment you chase (me down to the water), you observe (me), you wait, we two crack (the coconut) open, and then
\z

\ea
\gll   Nofita:    {tida}    {usa,}    {kotor}    {dang}    {ko}   nanti   kena\\
{} {\textsc{neg}}    {need.to}    {be.dirty}    {and}    {\textsc{2sg}}   very.soon   hit\\
\gll {picaang,}    {kam}    {dengar}   ato    {tida,}    {terlalu}    {nakal}\\
   {splinter}    {\textsc{2pl}}    {hear}   or    {\textsc{neg}}    {too}    {be.mischievous}\\
\glt
Nofita: don’t (go down to the beach, it’s) dirty, and later you’ll run into broken glass and cans, are you listening or not?!, (you’re) too naughty!
\z %\end{styleBodyxvafter}

\section{Hortatory: Bathe in the ocean!}
\label{Para_B.10}
\begin{tabular}{ll}
\lsptoprule
File name: &  080917-006-CvHt\\
Text type: &  Conversation, spontaneous: Hortatory\\
Interlocutors: &  3 male children,\footnotemark{} 2 older females\\
Length (min.): &  1:00\\
\lspbottomrule
\end{tabular}
\setcounter{equation}{0}
\footnotetext{The third male child did not participate in this exchange.}

\ea
\gll    {Nofita:}   kepala    {sakit}   sa   tra    {bisa}   bicara   banyak,\\
   {}   head    {be.sick}   \textsc{1sg}   \textsc{neg}    {be.able}   speak   many\\
\gll kam    {dengar{\Tilde}dengarang,}    {kam}    {cari}    {apa?}\\
  \textsc{2pl}    {\textsc{rdp}{\Tilde}hear\textsc{:pat}}    {\textsc{2pl}}    {search}    {what}\\
\glt
Nofita: (I have) a headache, I can’t talk much, you listen to me!, what are you looking for?
\z

\ea
\gll   Wili:   a,   jangang,   Nofi   mana   kitong   pu   ikang{\Tilde}ikang?\\
{}  ah!   \textsc{neg.imp}   Nofi   where   \textsc{1pl}   \textsc{poss}   \textsc{rdp}{\Tilde}fish\\
\glt
Wili: ah, don’t!, Nofi, where are our fish?
\z

\ea
\gll   Nofi:   sa   su   taru   di   ember   sini\\
{} \textsc{1sg}   already   put   at   bucket   \textsc{l.prox}\\
\glt
Nofi: I already put (the fish) in the bucket here
\z

\ea
\gll   Nofita:   kam   dua   pi   spul   badang   di   laut   sana\\
 {} \textsc{2pl}   two   go   rinse   body   at   sea   \textsc{l.dist}\\
\glt Nofita: you two go rinse (your) bodies in the ocean over there
\z

\ea
\gll   Nofi:   ada   ni\\
  {} exist   \textsc{d.prox}\\
\glt
Nofi: (the fish) are here
\z

\ea
\gll    {Nofita:}    {spul}    {badang,}    {trus}    {celana}    {cuci}    {di}    {laut,}    {baru}\\
   {}    {rinse}    {body}    {next}    {trousers}    {wash}    {at}    {sea}    {and.then}\\
\gll pake    {ke}    {mari,}    {biking}    {kotor}    {saja,}    {saya}    {stenga}   mati   cuci,\\
  use    {to}    {hither}    {make}    {be.dirty}    {just}    {\textsc{1sg}}    {half}   die   wash\\
\gll {cape}    {cuci}    {pakeang}    {juga,}    {ana{\Tilde}ana}    {ini}    {kotor{\Tilde}kotor,}\\
   {be.tired}    {wash}    {clothes}    {also}    {\textsc{rdp}{\Tilde}child}    {\textsc{d.prox}}    {\textsc{rdp}{\Tilde}be.dirty}\\
\gll {dong}    {\upshape\textsc{[up]}}    {adu}\\
   {\textsc{3pl}}    {}    {oh.no!}\\
\glt
Nofita: rinse (your) bodies, then wash (your) trousers in the ocean, and then put them on (and) come here, (they) make (all their clothes) dirty, I’m half dead (from) washing, I’m also tired of washing clothes, these kids make (their trousers) dirty, they [\textsc{up}] oh no!
\z

\ea
\gll   ey,    {kam}    {dua}    {pi}    {mandi}   di    {laut}    {suda,}    {trus}    {kam}   dua\\
  hey!    {\textsc{2pl}}    {two}    {go}    {bathe}   at    {sea}    {already}    {next}    {\textsc{2pl}}   two\\
\gll {cuci}    {celana}    {di}    {situ,}    {baru}    {pake,}    {naik,}    {tra}    {usa}\\
   {wash}    {trousers}    {at}    {\textsc{l.med}}    {and.then}    {use}    {ascend}    {\textsc{neg}}    {need.to}\\
\gll {loncat{\Tilde}loncat,}    {situ}    {ada}    {besi{\Tilde}besi}    {banyak}\\
   {\textsc{rdp}{\Tilde}jump}    {\textsc{l.med}}    {exist}    {\textsc{rdp}{\Tilde}metal}    {many}\\
\glt
hey, you two go bathe in the ocean already!, and then you two wash (your) trousers there, after that put (them) on (and) come up (to the house), don’t jump up and down, there are lots of metal pieces over there
\z

\ea
\gll   Anelia:   mm-mm,   picaang   juga   banyak\\
 {} mhm   splinter   also   many\\
\glt
Anelia: mhm, (at the beach there) are also lots of broken glass and cans (Lit. ‘the splinters are also many’)
\z

\ea
\gll   Nofita:   picaang   banyak\\
{} splinter   many\\
\glt
Nofita: (there) are lots of broken glass and cans
\z %\end{styleBodyxvafter}

\section{Joke: Drawing a monkey}
\label{Para_B.11}
\begin{tabular}{ll}
\lsptoprule
File name: &  081109-002-JR\\
Text type: &  Joke (Elicited text)\\
Interlocutors: &  2 younger males\\
Length (min.): &  0:59\\
\lspbottomrule
\end{tabular}
\setcounter{equation}{0}
\ea
\gll    {skola}    {ini}    {ibu}    {mulay}    {suru}    {ana{\Tilde}ana}    {murit}    {mulay}\\
   {school}    {\textsc{d.prox}}    {woman}    {start}    {order}    {\textsc{rdp}{\Tilde}child}    {pupil}    {start}\\
\gll {gambar}    {monyet}   di    {atas}    {pohong}    {pisang,}    {suda,}    {ibu}\\
   {draw}    {monkey}   at    {top}    {tree}    {banana}    {already}    {woman}\\
\gll mulay    {suru}    {gambar,}    {suda}    {dong}    {mulay,}    {smua}    {dong}   gambar\\
  start    {order}    {draw}    {already}    {\textsc{3pl}}    {start}    {all}    {\textsc{3pl}}   draw\\
\glt
(in) this school, Ms. (Teacher) starts ordering the school kids to start drawing a monkey on a banana tree, well, Ms. Teacher orders (them to) draw, well, they start, they all draw (a picture)
\z

\ea
\gll    {baru}    {ana}    {kecil}    {satu}    {ini}    {de}    {tra}    {gambar,}   ana\\
   {and.then}    {child}    {be.small}    {one}    {\textsc{d.prox}}    {\textsc{3sg}}    {\textsc{neg}}    {draw}   child\\
\gll murit    {satu}    {ni}    {de}    {tra}    {gambar,}    {suda,}    {begini}    {de}\\
  pupil    {one}    {\textsc{d.prox}}    {\textsc{3sg}}    {\textsc{neg}}    {draw}    {already}    {like.this}    {\textsc{3sg}}\\
\gll {gambar}    {batu,}    {trus}    {de}    {gambar}    {monyet}    {ini}   di    {bawa}\\
   {draw}    {stone}    {next}    {\textsc{3sg}}    {draw}    {monkey}    {\textsc{d.prox}}   at    {bottom}\\
\gll {pohong}    {pisang,}    {begini}    {dong}    {bawa}    {ke}    {depang}\\
   {tree}    {banana}    {like.this}    {\textsc{3pl}}    {bring}    {to}    {front}\\
\glt
but then this particular small child, he doesn’t draw, this particular school kid, he doesn’t draw, well, he draws a stone (instead), and then he draws this monkey under the banana tree, it goes on like this (and) they bring (their drawings) to the front
\z

\ea
\gll    {ibu}    {bilang,}    {ibu}    {kalo}    {toki}    {meja}    {langsung}   kumpul\\
   {woman}    {say}    {woman}    {if}    {beat}    {table}    {immediately}   gather\\
\gll ke    {depang,}    {suda}    {pace}    {de}    {pikir{\Tilde}pikir}    {sampe}    {tra}\\
  to    {front}    {already}    {man}    {\textsc{3sg}}    {\textsc{rdp}{\Tilde}think}    {until}    {\textsc{neg}}\\
\gll {jadi,}    {suda}    {begini}    {langsung}    {i}    {ibu}\\
   {become}    {already}    {like.this}    {immediately}    {\textsc{tru}{}-woman}    {woman}\\
\gll {bagi}    {meja,}    {pak!,}    {langsung}    {pace}    {gambar}    {\upshape\textsc{[up]}}    {itu,}\\
   {divide}    {table}    {bang!}    {immediately}    {man}    {draw}    {}    {\textsc{d.dist}}\\
\gll {monyet}   di    {bawa}    {pohong}    {pisang,}    {bawa}   ke    {sana}\\
   {monkey}   at    {bottom}    {tree}    {banana}    {bring}   to    {\textsc{l.dist}}\\
\glt
Ms. (Teacher) says, ‘when I (‘Ms.’) knock (on) the table, (you) bring (your pictures) together to the front immediately’, then the guy thinks on and on (but) nothing happens, as it goes on like this immediately Ms.[\textsc{tru}], Ms. (Teacher) hits the table, ‘bang!’, immediately the guy draws [\textsc{up}], what’s-its-name, a monkey under a banana tree (and) brings it to the front
\z

\ea
\gll    {ibu}    {bilang,}    {e,}    {ibu}    {priksa}    {selesay,}    {ibu}    {tanya,}\\
   {woman}    {say}    {uh}    {woman}    {check}    {finish}    {woman}    {ask}\\
\gll {ini}    {siapa}    {punya?}    {de}    {bilang,}    {ibu,}    {sa}    {punya,}\\
   {\textsc{d.prox}}    {who}    {\textsc{poss}}    {\textsc{3sg}}    {say}    {woman}    {\textsc{1sg}}    {\textsc{poss}}\\
\gll de    {tanya,}    {pace}    {maju}    {ke}    {sana,}    {ibu}    {tanya}   dia,\\
  \textsc{3sg}    {ask}    {man}    {advance}    {to}    {\textsc{l.dist}}    {woman}    {ask}   \textsc{3sg}\\
\gll {knapa}    {ko}    {gambar}    {monyet}    {di}    {bawa}    {pohong}    {pisang?}\\
   {why}    {\textsc{2sg}}    {draw}    {monkey}    {at}    {bottom}    {tree}    {banana}\\
\glt
Ms. (Teacher) says, uh, after Ms. (Teacher) has finished checking (the pictures), Ms. (Teacher) asks (them), ‘this (picture here), whose is (it)?’, he says, ‘Madam, (it’s) mine’, she asks (him), the guy comes to the front, Ms. (Teacher) asks him, ‘why did you draw the monkey under the banana tree?’
\z

\ea
\gll   de    {blang,}    {adu}    {ibu,}    {tadi}    {ibu}   toki   meja\\
  \textsc{3sg}    {say}    {oh.no!}    {woman}    {earlier}    {woman}   beat   table\\
\gll {itu}    {yang}    {monyet}   de    {jatu}    {dari}    {atas}\\
   {\textsc{d.dist}}    {\textsc{rel}}    {monkey}   \textsc{3sg}    {fall}    {from}    {top}\\
\glt
he says, ‘oh no!, Madam!, a little bit earlier you (‘Madam’) knocked on the table, that’s why the monkey fell off from the top (of the banana plant)’
\z %\end{styleBodyxvafter}

\section{Joke: Dividing three fish}
\label{Para_B.12}
\begin{tabular}{ll}
\lsptoprule
File name:  &  081109-011-JR\\
Text type:  & Joke (Elicited text)\\
Interlocutors: &  2 younger males\\
Length (min.): &  1:07\\
\lspbottomrule
\end{tabular}
\setcounter{equation}{0}
\ea
\gll   pace    {orang}    {Biak}    {dong}    {dua}    {mancing,}    {dong}    {dua}\\
  man    {person}    {Biak}    {\textsc{3pl}}    {two}    {fish.with.rod}    {\textsc{3pl}}    {two}\\
\gll {mancing,}    {mancing}    {mancing}    {mancing,}    {suda}\\
   {fish.with.rod}    {fish.with.rod}    {fish.with.rod}    {fish.with.rod}    {already}\\
\gll {dong}   dua    {dapat}    {ikang}    {ini}    {tiga}    {ekor,}    {dapat}   ikang\\
   {\textsc{3pl}}   two    {get}    {fish}    {\textsc{d.prox}}    {three}    {tail}    {get}   fish\\
\gll {tiga}    {ekor,}    {dong}    {dua}    {mulay}    {mendarat}    {ke}    {darat}\\
   {three}    {tail}    {\textsc{3pl}}    {two}    {start}    {land}    {to}    {land}\\
\glt
the two \ili{Biak} guys are fishing, they are fishing, fishing, fishing, fishing, eventually the two of them get these fish, three (of them), having gotten three fish, the two of them start landing on the shore
\z

\ea
\gll    {sampe}   di    {darat,}    {suda}    {dong}    {dua}    {mulay}    {bagi}   ikang   itu,\\
   {reach}   at    {land}    {already}    {\textsc{3pl}}    {two}    {start}    {divide}   fish   \textsc{d.dist}\\
\gll de    {mulay,}    {dep}    {kawang,}    {e,}    {mulay}    {bilang,}    {kawang}   ko\\
  \textsc{3sg}    {start}    {\textsc{3sg}:\textsc{poss}}    {friend}    {uh}    {start}    {say}    {friend}   \textsc{2sg}\\
\gll {bawa}    {satu,}   sa    {bawa}    {satu}\\
   {bring}    {one}   \textsc{1sg}    {bring}    {one}\\
\glt
having arrived on the shore, the two of them start dividing the fish, he starts to, his friend, uh, starts to say, ‘you friend take one (and) I take one’
\z

\ea
\gll   trus    {de}    {bilang,}    {i,}    {baru}    {yang}    {satu}    {ini,}   de\\
  next    {\textsc{3sg}}    {say}    {ugh!}    {and.then}    {\textsc{rel}}    {one}    {\textsc{d.prox}}   \textsc{3sg}\\
\gll {dep}    {temang}    {tu,}    {a,}    {ko}    {sala}    {bagi,}    {gabung}\\
   {\textsc{3sg}:\textsc{poss}}    {friend}    {\textsc{d.dist}}    {ah!}    {\textsc{2sg}}    {wrong}    {divide}    {join}\\
\gll {lagi,}    {temang}    {satu}    {bagi}    {lagi}\\
   {again}    {friend}    {one}    {divide}    {again}\\
\glt
then he says, ‘ugh!, but what about this one?’, he, his friend says, ‘ah, you’ve divided (the fish) incorrectly’, (they) put (the fish back) together again, (that) one friend divides them again
\z

\ea
\gll   dapat    {satu,}    {sa}    {satu,}    {yang}    {ini?,}    {de}   pu   temang   tra\\
  get    {one}    {\textsc{1sg}}    {one}    {\textsc{rel}}    {\textsc{d.prox}}    {\textsc{3sg}}   \textsc{poss}   friend   \textsc{neg}\\
\gll {trima}    {baik}    {lagi,}    {de}    {gabung}    {lagi}    {((laughter))}\\
   {receive}    {good}    {again}    {\textsc{3sg}}    {join}    {again}    {}\\
\glt
‘(you) get one, I (get) one’, ‘(and) this one?’ again his friend doesn’t accept (the result of this dividing,) well, he puts (them back) together again ((laughter))
\z

\ea
\gll    {dong}    {dua}    {bagi}    {su}    {begitu}    {trus,}   suda\\
   {\textsc{3pl}}    {two}    {divide}    {already}    {like.that}    {be.continuous}   already\\
\gll {orang}    {Ayamaru}    {datang,}    {datang}    {de}    {bilang,}\\
   {person}    {Ayamaru}    {come}    {come}    {\textsc{3sg}}    {say}\\
\gll eh,    {kam}   dua    {baku}    {melawang}    {apa?}\\
  hey!    {\textsc{2pl}}   two    {\textsc{recp}}    {fight}    {what}\\
\glt
the two of them continue dividing (the fish) just like that, eventually an Ayamaru guy comes by, having come by, he says, ‘hey, about what are you two fighting with each other?’
\z

\ea
\gll   de    {bilang,}    {om,}    {ini,}    {kitong}    {dua}    {baku}    {melawang}\\
  \textsc{3sg}    {say}    {uncle}    {\textsc{d.prox}}    {\textsc{1pl}}    {two}    {\textsc{recp}}    {fight}\\
\gll {gara-gara}    {ikang,}    {kitong}    {dua}    {bagi,}    {de}    {satu,}   sa   satu\\
   {because}    {fish}    {\textsc{1pl}}    {two}    {divide}    {\textsc{3sg}}    {one}   \textsc{1sg}   one\\
\gll {baru}    {yang}    {ini}    {nanti}    {ke}    {mana?}\\
   {and.then}    {\textsc{rel}}    {\textsc{d.prox}}    {very.soon}    {to}    {where}\\
\glt
he says, ‘uncle, what’s-its-name, the two of us are fighting each other because of the fish, we two divide (it), he (has) one (fish), I (have) one (fish), but where does this one go?’
\z

\ea
\gll   de    {bilang,}    {itu}    {yang}   masi    {tunggu}   saya,   de   pegang\\
  \textsc{3sg}    {say}    {\textsc{d.dist}}    {\textsc{3sg}}   still    {wait}   \textsc{1sg}   \textsc{3sg}   hold\\
\gll {dang}   dong    {jalang,}    {de}    {bilang,}   pas    {to?}\\
   {and}   \textsc{3pl}    {walk}    {\textsc{3sg}}    {say}   be.exact    {right?}\\
\glt
he says, ‘that (is one) which is still waiting for me’ he takes (it) and they walk (away), and he says, ‘that fits, right?’
\z

\ea
\gll   de   bilang,    {itu}    {kawang}   sa    {su}   bilang,\\
  \textsc{3sg}   say    {\textsc{d.dist}}    {friend}   \textsc{1sg}    {already}   say\\
\gll {makanya}    {skola}    {baru}    {pintar}\\
   {for.that.reason}    {go.to.school}    {and.then}    {be.clever}\\
\glt
he says, ‘friend, that’s (what) I already told (you), that’s why you should go to school, then you’ll be clever’
\z %\end{styleBodyxvafter}

%\setcounter{page}{1}
