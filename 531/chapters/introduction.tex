\begin{sloppypar}
\section{Introduction}\label{sect:introduction}
The high mountainous Hewraman region, located between Iranian and Iraqi Kurdistan, is home to a unique linguistic and cultural profile. The people of the area speak Hewramî, a member of the Gorani branch of Iranian languages, closely akin to the neighbouring Kurdish. Religiously, the region is characterised by affiliation with Sunni Islam and its Sufi orders, specifically Naghshbandi and Qaderi. Historically, the greater Hewraman region has been home to the esoteric faith of Yarsan (or Ahl-e Hagh). The Sufi masters, whom the locals call \textit{pîrs}, traditionally exerted spiritual authority in the region. Among the most venerated figures in this tradition is Pîr Shaliyar, a semi-legendary Sufi mystic who is believed to have lived around the 12th Century C.E. His legacy endures through oral narratives and some ritual practices, including an annual festival comemorating the anniversary of his marriage to the princess of Egypt (or Bokhara in some accounts), around February. The festivities in the region have some traces of pre-Islamic practices and rites in them. 

This book presents the rich cultural and belief system profile of the Hewraman region, drawing on oral accounts of the region's recent history, hagiographies, narratives concerning the greatness of Sufis and their miraculous powers, recollections of traditional life, and autobiographies.

The book is structured as follows. In this chapter, (\S\ref{sect:lang_speaker}) introduces Hewramî and its speakers and (\S\ref{sect:grammar_sketch}) outlines an overview of Hewramî grammar. Both of these sections are kept brief. Readers are encouraged to consult my detailed grammar of Hewramî \citep[][]{Mohammadirad2025} for a more comprehensive account.  \S\ref{sect:provenance} discusses the provenance of the texts, including ethics and edditing decisions. \S\ref{sect:conventions} summarises the orthographic, glossing, and translation conventions in the book. \S\ref{sect:genre} gives a brief overview of the genres and narrative styles, while \S\ref{sect:narrators}, and \S\ref{sec:texts} briefly introduce the narrators, and the text corpus, respectively. The main body of the book focuses on Part II, where texts 1–15 are presented in the following format: each text is preceded by a brief introduction and a summary of its content, followed by the text presented in parallel columns (Hewramî/English) and fully glossed in a separate section.

\section{The language and its speakers}\label{sect:lang_speaker}

Hewramî\il{Hewramî} is an Iranian language spoken in the remote mountainous region at the heart of the Kurdish\il{Kurdish}-speaking region along the western border of Iran and neighbouring areas in Iraqi Kurdistan. Hewramî is a member of the Gorani subbranch of the Iranic languages (Indo-European: Iranic: Central Iranic: Northwestern Iranic: (Adharic:) Gorani: Hewramî). 

The speakers refer to their language as \textit{Hewramî}, a term used by neighbouring Kurdish\il{Kurdish} speakers to refer to the Hewramî\il{Hewramî} vernacular. In addition, Hewramî\il{Hewramî}-speaking people generally refer to themselves as Kurds in a more socio-cultural and  sense of the term ``Kurd'', a sense which Kurds also employ to characterise Hewramî\il{Hewramî}s and Kurds alike (See also \citealt[25]{hassanpour_nationalism_1992}). 

The Hewramî-speaking population are concentrated in four regions within Hewraman \citep{MacKenzie1987Avroman}: Luhon (in the south), Tekht (in the centre), Dizlî (in the north), and Razaw (around Sarv Abad). In \citet{mahmoudveysi_hawrami_2018}, the Razaw region is further divided into the Jawero and Gawero sub-regions. Figure \ref{fig:hewmap} shows the region of Hewraman according to the mentioned divisions. Each point on the polygons indicates a locality. These geographical divisions do not necessarily correspond to linguistic subdivisions. 
 
Hewramî\il{Hewramî} varieties are traditionally divided into three major groupings: Tekht, Luhon, and Jawero. Geographically speaking, these varieties are spoken in the centre, south, and east of the greater Hewraman region, respectively. The Tekht region is linked to Jawero through a stretch of valleys, while the Luhon region is located in the western valley. 

\begin{figure}[htpb]
    \includegraphics[width=\textwidth]{figures/map-hewraman_04july.png}
    \caption{The region of Hewraman and its divisions}
    \label{fig:hewmap}
    \end{figure}
    
Within Iran, the most significant concentration of Hewramî\il{Hewramî} speakers is in the cities of Marivan, Paveh, and Sarvabad. In Iraqi Kurdistan, Hewramî\il{Hewramî} speakers can be found in cities such as Khurmal and Halabja. There are also Hewramî-speaking communities in major towns in the region, including Kermanshah and Sanandaj. In fact, Hewramî and, more generally, Goranî were once spoken over a larger territory. Over time, Goranî gave way to Kurdish \citep[][]{leezenberg_gorani_1992}. This linguistic shift has left its trace not only on the Kurdish vernaculars in the region, e.g. the Kurdish vernacular spoken in Sanandaj and the surrounding areas \citep[][]{mohammadirad_gorani_nodate,mohammadirad_zagros_nodate}, but also on the neighbouring Neo-Aramaic dialects \citep[][]{KhanMohammadirad+2024,KhanMohammadirad+2024+171+198}.

The material for this book was gathered in Hewraman Tekht, marked as an asterisk on Figure \ref{fig:hewmap}. Hewraman Tekht (Orāmān-e Takht) is the administrative centre of the Tekht region. It has a population of 5,000 people. The inhabitants of Hewraman Tekht are all Hewramî-speaking. Men are generally bilingual in Central Kurdish\il{Kurdish!Central}. Women over the age of 40 are usually monolingual in Hewramî\il{Hewramî}. The region's inhabitants also have some knowledge of Persian\il{Persian}, though it seems that competence in Persian\il{Persian} is higher among men than women. The situation for the younger generation is different since they all learn Persian\il{Persian} through schooling. It could be the case that the younger generation who has never left Hewraman does not speak any Kurdish\il{Kurdish} but knows Persian\il{Persian} through schooling. 

Due to its location within a remote, hard-to-access, high-mountainous region, many aspects of the traditional lifestyle in Hewraman have remained largely intact to date. The unique architecture of the houses made of stones, agropastoral lifestyle, man-made crafts, etc., has made this region a unique hub within the greater Kurdish region. This has resulted in the Hewraman region being registered as a cultural heritage site by UNESCO in 2021.\footnote{\url{https : / / whc . unesco . org / en / list / 1647/}} Figure \ref{fig:hewraman} illustrates the city of Hewraman Tekht. 

\begin{figure}[htp]
    \centering
    \includegraphics[width=0.70\textwidth]{figures/hewraman_tekht.jpg}
    \caption{Hewraman Tekht}
    \label{fig:hewraman}
\end{figure}
   
The inhabitants of Hewraman generally engage in an agropastoral lifestyle, which is characterised by a semi-nomadic lifestyle and vertical migration. During the warm seasons, people migrate to the highlands, which they call \textit{hewar}, where they plant cereals and tend to livestock. They return to the lowlands during the cold seasons. However, this lifestyle is waning, and urbanism is taking over in Hewraman Tekht and increasingly in the surrounding villages. 

The Hewramî people are renowned for their exceptional craftsmanship. Traditionally, men engage in woodwork and stone-work, while women specialise in making traditional men's shoes called \textit{kiɫaş}. In the absence of flat lands suitable for farming, the people of Hewraman practice gardening. Mulberries and walnuts are among the agricultural products in the region.

\section{Grammar sketch}\label{sect:grammar_sketch}
\subsection{Introduction}
Hewramî belongs to the Goranî group of Iranian languages. It has been referred to as ``the most archaic and the best preserved'' variety within the Goranî language group \citep[4]{mackenzie_dialect_1966}. Defining grammatical properties of Hewramî includes a primary phonological gender assignment system,  split-ergative alignment, two-term case system, differential argument flagging, differential argument indexing, disharmonic SOV order, phonemic stress placement, and a complex deictic system. The following description is based on the Tekht variety of Hewramî. 
\subsection{Phonology}
The vowel inventory consists of nine phonemic vowels, including four front vowels <î> [i]; <ê> [e]; <ε> [ε], <e> [ε$\sim$æ], three back vowels <û> [u]; <o> [o]; <a> [ɑ] and two central vowels <i> [ɨ] and <u> [ʊ]. Of these, [i], [e]; [ε], [ε$\sim$æ], [u], [ɑ], and [o] are long vowels and [ɨ] and [ʊ] are short vowels (see Figure \ref{Fig:Hewrami-vowels}). The vowel phonemes are additionally distinguished by quality distinctions. 

\begin{figure}[htp]
    \includegraphics[width=.55\textwidth]{figures/vowel_qualityn.png}
    \caption{Hewramî\il{Hewramî} vowel inventory}
    \label{Fig:Hewrami-vowels}
\end{figure}

The consonant inventory includes 29 phonemes, of which four are peripheral phonemes occurring either in a few loanwords or limited in their distribution within the syllable or word. These phoemes are represented within round brackets in Table \ref{cchart}. Hewramî features the lenition of post-vocalic /d/, a phenomenon shared with regional languages known as ``Zagros /d/''. The outcomes of the lenition of /d/ are either the velarised alveolar approximant [ɹˠ], represented as <đ>, or, less frequently, a palatal approximant [j], represented as <y>.

\begin{table}[htp]
\caption{Consonant inventory}
\label{cchart}\small
	\begin{tabularx}{\textwidth}{p{2cm}XXXXXXXXX}
		\lsptoprule
&{Labial} & {Labio-dental} & {Alveo-dental} & {Post-alveolar} & {Palatal} & {Velar} & {Uvular} & {Phary-ngeal} & {Glottal} \\\midrule
{Stop} & p b &  & t  d &  &  & k  ɡ& q &  & (ʔ) \\
{Affricate} &  &  &  & \makebox[2.5em][c]{t͡ʃ d͡ʒ} &  &  &  &  & \\
{Nasal} &\phantom{0 }m &  &\phantom{0 }n &  &  &\phantom{0 }(ŋ)  &  &  & \\
{Fricative} &  & f  & s  z & \makebox[2.8em][c]{ʃ ʒ} &  & x    (γ) &   & ħ  ʕ & h \\
{Tap} &  &  &\phantom{0 }ɾ &  &  &  &  &  & \\
{Trill} &  &  &\phantom{0 }r &  &  &  &  &  & \\
{Lateral} &  &  &\phantom{0 }l &\phantom{0 }(ɫ) &  &  &  &  & \\
{Approximant} &\phantom{0 }w &  &         &  &\phantom{0 }j &  & & & \\
\lspbottomrule
	\end{tabularx}
\end{table}%

Hewramî\il{Hewramî} is a language with phonemic stress\is{phonemic stress} placement: For most verbs, stress is the only mechanism to distinguish between subjunctive and indicative verbs derived from the present tense, e.g. \textit{ber\'o} `he/she takes' and \textit{b\'ero} `that he/she takes'. When used without the subjunctive prefix, subjunctive stems are marked with stress in the glossed texts. 

The most frequent syllabic patterns are VC, CV, and CVC. A rule of glottal stop insertion avoids empty onsets, which would contradict VC as being a syllabic pattern. The glottal stop insertion in this sense is a rule that shapes the actual surface form, not the underlying assumption of a syllable structure. One vowel is allowed per syllable maximally, meaning that the number of vowels is equal to the number of syllables in a word. The syllable structure allows for consonant clusters in the onset and coda, e.g. \textit{drext} `tree'; \textit{heşt} `eight'.

\subsection{Morphology}
Hewramî nouns are morphologically marked for case, number, and gender through fusional inflectional affixes. All three represent a two-way distinction: singular and plural for number, masculine and feminine for gender, and ``direct'' and ``oblique'' for case. Gender marking is limited to singular nouns. Direct and oblique are terms traditionally used in Iranian linguistics and the literature on two-term case systems, roughly equivalent to ``nominative'' and ``non-nominative'', respectively. 
The fusional case/gender/number suffixes are organised into one underlying inflectional class (see Table \ref{tab:nom-infl-under1}). The inflection of a noun is predictable from the phonological shape of the base and from its gender.
\begin{table}[htp]
    \begin{tabular}{lllllll} 
    \lsptoprule
& \textsc{sg.dir}& \textsc{sg.obl}& \textsc{pl.dir} & \textsc{pl.obl} \\
\midrule
\textsc{m}& -\O& \textit{-î} & \multirow{2}{*}{\textit{-ê}}& \multirow{2}{*}{\textit{-a}} \\
\textsc{f}& -\O& \textit{-ê} & & \\
\lspbottomrule
\end{tabular}
    \caption{Nominal inflectional suffixes--underlying forms}
    \label{tab:nom-infl-under1}
\end{table}

Hewramî has a basic phonological gender assignment system. Gender\is{gender} is assigned primarily based on the shape of the base. Nouns which, in their citation form, end in a consonant or stressed \textit{-é}, \textit{-\stackunder[-10pt]{\^{i}}{\'{}}}, \textit{-\stackunder[-10pt]{\^{u}}{\'{}}}, and \textit{-ó} are masculine\is{masculine}. In addition, a subset of nouns ending in stressed \textit{-á }are masculine\is{masculine}. On the other hand, nouns ending in \textit{-ê} (whether stressed or not) and those ending in unstressed \textit{-e}, \textit{-î}, are feminine\is{feminine}. The class of feminine\is{feminine} nouns also includes a subset in stressed \textit{-á}.

Case assignment is only applicable to 3rd person pronouns and nouns. 1st and 2nd person pronouns have lost the case distinction. Table \ref{tab:case-align} lays out the distribution of direct and oblique cases in flagging arguments of the verb.
\begin{table}[htp] 
    \begin{tabular}{lll}
    \lsptoprule
    & \textsc{direct} \textsc{case} & \textsc{oblique} \textsc{case} \\\midrule
    TAM based on present stem verbs&  S, A & O, R\\
    TAM based on past stem verbs&  S, O &  A, R  \\
    \lspbottomrule
    \end{tabular}
    \caption{Case marking of core arguments}
     \label{tab:case-align}
\end{table}

Hewramî\il{Hewramî} features a tense-based split ergative system both in argument case marking\is{case marking} (except with speech act pronouns) and verbal argument indexing. The alignment\is{alignment} system is nominative-accusative for verbs derived from the present tense and ergative-absolutive for verbs derived from the past stem. The follwoing examples illustrate ergative alignment through the noun \textit{jenî} `woman, wife'. It takes a direct form marking S (\ref{erg-abs.dir.S}) and O (\ref{erg-abs.dir.O}), and the oblique form when marking the A (\ref{erg-abs.obl.A}).

\ea
\ea
\textit{jenîm merđêne.}\\
\gll \textbf{jenî}=m merdê=ne \\
wife\textsc{.f.sg.dir=1sg:PSR} die\textsc{.pst.ptcp.f=cop.3sg.f:S} \\
\glt `My \textbf{wife} (S) died.' \label{erg-abs.dir.S}
\ex 
\textit{jenîm arđêne.}\\
\gll \textbf{jenî}=m ardê=ne \\ 
  wife\textsc{.f.sg.dir=1sg:A} take.\textsc{pst.ptcp.f=cop.3sg.f:O}\\ 
\glt `I took a \textbf{wife} (O).'  \label{erg-abs.dir.O}
\ex  
\textit{\textbf{jenê}ç kilêɫaniş nîya baxeɫeş.}\\
\gll jen(î)-ê=ç kilêɫan=iş nîya-Ø	baxeɫe=ş \\
 woman\textsc{-f.sg.obl=add} kohl\_pot.\textsc{m.sg.dir=3sg:A}	put.\textsc{pst.3sg.m:O} arms=\textsc{3sg:PSR}\\
\glt `The \textbf{woman} put the kohl pot in her arms.' \label{erg-abs.obl.A}
\z 
\z

Ergative alignment is most consistently evident in the pattern of argument indexing: past transitive clauses exhibit a specific paradigm of subject indexing (A-past)\footnote{The set of clitic pronouns indexing A-past NPs is multi-functional and additionally expresses object arguments in present-tense clauses, adpositional complements, adnominal possessors, and non-canonical subjects. For example, note the homophony between \textsc{1sg} possessive clitic in (\ref{erg-abs.dir.S}) and \textsc{1sg} A-past index in (\ref{erg-abs.dir.O}). In expressing most of these functions, clitic pronouns are mobile. They generally attach to the initial element of the VP, e.g. (\ref{erg-abs.dir.O}-\ref{erg-abs.obl.A}).}, which differs from subject indexing in intransitive clauses (S), and from object indexing in the past tense (O-past), cf. Table \ref{ex.hewramiprs.pst}. Example (\ref{ex.ergindex}) illustrates the ergative pattern of indexing.
\begin{table}[htp]
\begin{tabular}{lllllll}
\lsptoprule
& S/A-\textsc{prs}& S/O-\textsc{past}& A-\textsc{past}/O-\textsc{prs} \\\midrule
\textsc{1sg}& \textit{-û}&\textit{-a(nê)} & \textit{=im} \\
\textsc{2sg} &\textit{-î} &\textit{-î} &\textit{=it} \\
\textsc{3sg} &\textit{-o} &\textit{-\O} (\textsc{m}); \textit{-e} (\textsc{f})& \textit{=iş}\\
\textsc{1pl} &\textit{-mê} &\textit{-îmê} & \textit{=ma} \\
\textsc{2pl} &\textit{-dê}&\textit{-îdê} &\textit{=ta} \\
\textsc{3pl} &\textit{-a} &\textit{-ê} & \textit{=şa} \\
\lspbottomrule
\end{tabular} 
    \caption{Bound argument indexing}
    \label{ex.hewramiprs.pst}
\end{table}

\ea \label{ex.ergindex}
\ea
\gll wit-\textbf{ê} \\
sleep\textsc{.pst-\textbf{3pl:S}}\\
\glt `They slept.’
\ex 
\gll berđ-\textbf{ê=şa} \\
take\textsc{.pst-\textbf{3pl:O=3pl:A}} \\
\glt `They took them.’
\z
\z 

The simple verb has two verbal forms divided into \textsc{present} and \textsc{past} stems. The two-stem system is divided into two tense-based categories roughly equivalent to present and past tenses. More broadly, verbal morphological stems are organised into the following categories: present, imperative, past, resultative participle, and infinitive\is{infinitive}. The last three are formed based on the past stem. The imperative is based on the present stem. Example:
\eabox{  
\begin{tabular}{ll}
\textit{berđey} `to take’ & \\
\textup{Present stem} & \textit{ber}- \\
\textup{Imperative stem} & \textit{ber-}\\
\textup{Past stem} & \textit{berđ}- \\
\textup{Resultative participle} &\textit{berđe} \textup{(\textsc{m})}; \textit{berđê} \textup{(\textsc{f/pl})} \\
\textup{Infinitive}& \textit{berđ-ey} \\
\end{tabular}}

Verbal categories are built by present and past stems combined with inflectional person suffixes and modal prefixes. Table \ref{tab:my_label} exemplifies the inflection of the transitive verb \textit{kerđey} `do' in \textsc{1sg} across different TAM forms. 
\begin{table}[htp]
    \fittable{%
    \begin{tabular}{lll}
\lsptoprule
\textsc{tam category}&\textsc{inflection}&\textsc{gloss}\\
\midrule
Present subjunctive\is{present subjunctive} & \textit{k\'er-û}& [do.\textsc{prs.sbjv-1sg:A}] \\
Imperative & \textit{k\'er-e}&[do.\textsc{prs.imp-2sg:A}] \\
Present indicative\is{present indicative}& \textit{ker-\stackunder[-10pt]{\^{u}}{\'{}}}& [do.\textsc{prs.ind-1sg:A}] \\
Past Progressive\is{past progressive}& \textit{ker-ay ker-û}&[do.\textsc{prs-nmlz} do.\textsc{prs.ind-1sg:A}] \\
Habitual past\is{habitual past} & \textit{ker-ên-a}&[do.\textsc{prs-aug-1sg:A}] \\
Irrealis past\is{irrealis past}& \textit{ker-ên-a}&[do.\textsc{prs-aug-1sg:A}] \\
\midrule
Past perfective\is{past perfective}& \textit{kerđ-∅=im}& [do.\textsc{pst-3sg.m:O=1sg:A}] \\
Past conditional\is{past conditional}& \textit{kerđ-ε=m}&[do.\textsc{pst-cond.aug.3sg:O=1sg:A}] \\
Perfect\is{perfect} & \textit{kerđe=n=im}&[do.\textsc{pst.ptcp.m=cop.3sg.m:O=1sg:A}] \\
Perfect progressive\is{perfect progressive}& \textit{kerđ-î kerđe=n=im}&[do.\textsc{pst-nmlz} do.\textsc{pst.ptcp.m=cop.3sg.m:O=1sg:A}] \\
Irrealis perfect\is{irrealis perfect}& \textit{kerđe=b-o=m}&[do.\textsc{pst.ptcp.m}=be\textsc{.prs-3sg:O=1sg:A}] \\
Conditional perfect\is{conditional perfect}& \textit{kerđe=bî-ε=m}&[do\textsc{.pst.ptcp}=be.\textsc{pst-cond.aug.3sg:O=1sg:A}] \\
Past perfect\is{past perfect}& \textit{kerđe=b-ê=m}&[do.\textsc{pst.ptcp.m}=be\textsc{-aug.3sg:O=1sg:A}] \\
Perfect pluperfect\is{perfect pluperfect}& \textit{kerđe=bîye=n=im}&[do.\textsc{pst.ptcp.m}=be.\textsc{pst.ptcp.m=cop3sg:O=1sg:A}]\\
\lspbottomrule
    \end{tabular}}
    \caption{The inflection of \textit{kerđey} `do' in \textsc{1sg} across different TAM categories}
    \label{tab:my_label}
\end{table}


\subsection{Syntax}
Hewramî has a basic SOV word order. Yet, most of the word order configurations, e.g. genitive/noun, possessor/possessed, are head-initial, rendering it a language with disharmonic word order. In the structure of the NP, nominal heads are linked to their modifiers through a head-linking formative, generally referred to as ``\textit{izafe}'' (or ``\textit{ezafe}'') within Iranian linguistics. There are two main ezafe linkers depending on the category of the modifier: \textit{-û} is used with nominal modifiers (hence called genitive ezafe); \textit{-î} `attributive ezafe' is used with adjectival modifiers. Example (\ref{ex.np}) illustrates the use of these two linkers in the structure of the NP. Note that the nominal modifier \textit{ħemey} contains two identical formatives, \textit{-î} \textsc{m.sg.obl} and attributive ezafe \textit{-î}, whose sequence is avoided by elision.  
\ea \label{ex.np}
\textit{jenû ħemey ẍeybî hamîlê bo.} \\ 
\gll \textbf{jen(î)-û} \textbf{ħeme-î} \textbf{ẍeybî} hamîlê b-o\\ 
 \textbf{wife\textsc{-ez.gen}} \textbf{\textsc{pn-m.sg.obl.ez.attr}} invisible pregnant\textsc{.f} be\textsc{.prs.ind-3sg:S} \\ 
\glt `Hama the Invisible’s wife [, who] was pregnant.' \hfill[BP.205]
\z 

Additionally, an \textit{ezafe compound} \textit{-e} is used in the language with tightly-knit compound NPS, e.g. \textit{nan-e taz(e)-êwe} [bread\textsc{.m-ez.cmpd} fresh-\textsc{indf}] `a fresh [loaf of] bread' (see \citealt[]{Mohammadirad2025} for details).

Hewramî features differential argument flagging and differential argument indexing. Here, differential object marking and differential Agent indexing are briefly discussed. Differential object marking is applicable to the flagging of direct object arguments in present-tense constructions. The difference in case marking of direct objects is triggered by specificity and uniqueness (see \citealt[]{Mohammadirad2025} for other factors). In both (\ref{SynCaseAgr:mendir}) and  (\ref{SynCaseAgr:menobl}), \textit{pîya} `man' is plural and the direct object of a present-tense verb. The direct marking in (\ref{SynCaseAgr:mendir}) is due to a non-specific reference of `men' as possessed to the oblique marking in (\ref{SynCaseAgr:menobl}) with a definite reference.  
\ea 
\ea 
\gll \textbf{pîyε} ber-a \\ 
man.\textsc{pl.dir} take.\textsc{prs.ind-3pl:A} \\
\glt `They are taking men [to service].' \label{SynCaseAgr:mendir}
\ex 
\gll î \textbf{pîya-ya} to zan-î kamê=nê \\
\textsc{dem.prox} man-\textsc{pl.obl} \textsc{2sg} know.\textsc{prs.ind-2sg} which\textsc{.pl=cop.3pl:S}\\ 
\glt `These men, you know who they are.' \label{SynCaseAgr:menobl}
\z  
\z 

Differential A-indexing concerns past transitive clauses. Here, the clitic pronoun generally cross-references the A NP, except when the A argument is focused. The focused A argument is realised in the immediate preverbal position, which is the locus for sentence stress. Additionally, there are also effects of person and animacy on subject indexing \citep[][]{MohammadiradinreviewAindx}. Consider the difference between (\ref{ex.A-clc-complementarity1}) and (\ref{ex.flood}) in terms of A indexing. 
\ea 
\ea
\textit{pađşay desûriş dan be min.} \\ 
\gll \textbf{pađşa-î} desûr=\textbf{iş} da=n be min \\ 
 king\textsc{.m-sg.obl} order\textsc{.m=\textbf{3sg:A}} give\textsc{.pst.ptcp.m=cop.3sg.m:O} to \textsc{1sg} \\  
\glt `The king has ordered me [to do this].' \hfill[JP.\ref{JP.209}] \label{ex.A-clc-complementarity1}
\ex \label{ex.flood}
\textit{heywane awê berde} \\ 
\gll heywane \textbf{aw\stackunder[-10pt]{\^{e}}{\'{}}} berd-e \\ 
 animal\textsc{.sg.dir.f} water\textsc{.sg.obl.f} take\textsc{.pst-3sg}\textsc{.f:O} \\ 
\glt `\textbf{The flood} [Lit. water] took away the animals.' \hfill[ZB.\ref{ZB.21}] 
\z
\z

\section{Provenance}\label{sect:provenance}
The introductory text for each story in the book includes its provenance and metadata. The documentation of Hewramî began to take shape while I was working on a co-authored volume, entitled \textit{Language Contact in Sanandaj: A Study of the Impact of Iranian on Neo-Aramaic}, with Geoffrey Khan (University of Cambridge) between 2020 and 2023 \citep[][]{KhanMohammadirad+2024}. 


\subsection{Ethics}
I have visited Hewraman on a number of occasions. My first visit was in March 2016, when I conducted a pilot fieldwork study in the region. I visited the region again in June and July 2017 as part of fieldwork for my PhD dissertation. The third trip to the region was part of fieldwork for my book \textit{A grammar of Hewramî}, which took place in August 2022. During this trip, I made recordings of the texts in the book. The recordings were made using a Zoom H5 Handy Recorder, which produced audio files in WAV format. I conducted my fieldwork primarily in Hewraman Tekht, but also visited Serû Pîrî (a village located north of Hewraman) and Benen (the summer habitat of the Hewraman Tekht inhabitants, situated in the highlands). 

All participants in the research were involved on a voluntary basis. There was no community centre with whom I could coordinate the selection of texts and their inclusion in the book. The individual narrators gave me permission to record their stories and for the narratives to be used in publications. The recordings for texts were generally made in the presence of other native speakers, usually the family members of the narrators, sometimes also the neighbours. 

The narrators who recounted the narratives were all over 60 years old at the time of recording. This makes the current book special in the sense that it reflects the language use by the older generation. The narrators gave their oral consent to be recorded and for the material to be used by the researcher. In Hewramî culture, consent is often carried out orally. Written consent is used rarely and only for specific administrative purposes, and is considered a sensitive cultural issue.  

During this fieldwork, I transcribed most of the recordings and double-checked my interpretation of the recordings with my native assistants to ensure that the correct interpretation had been achieved. 

\subsection{Archived material}
The recordings have been archived on the open-access platform Zenodo (see \citealt{mohammadirad_2025_15419952}), and can be found at \url{https://zenodo.org/records/15419952}. The Zenodo repository also includes PDF files containing time-aligned transcription and translation for each recording, as well as annotated ELAN files. The recordings are available as WAV files.


\section{Conventions}\label{sect:conventions}

\subsection{Orthographic conventions}
The transcription system used in this book follows the Hawar standard Kurdish\il{Kurdish} script \citep[][]{loc_kurdish_roman}. Table \ref{tab:phon-graph} exhibits how the graphemes in the Hawar script correspond to IPA symbols. 
\begin{table}[htp]
\begin{tabular}{cc|cc}
\lsptoprule
\textsc{phoneme} & \textsc{grapheme} & \textsc{phoneme} & \textsc{grapheme} \\
\midrule
p & p & ɣ & ẍ \\
b & b & ħ & ħ \\
t & t & ʕ & ʕ \\
d & d & h & h \\
ɹˠ & đ& ɾ & r \\
k & k & r & ř \\
g & g & l & l \\
q & q & {\l} & ɫ \\
t͡ʃ & ç & w & w \\
d͡ʒ & c & j & y \\
m & m & i & î \\
n & n & e & ê \\
ŋ & ŋ & ε∼æ & e \\
f & f & ε & ε \\
s & s & ɨ & i \\
z & z & u & û \\
ʃ & ş & ʊ & u \\
ʒ & j & o & o \\
x & x & ɑ & a \\
\lspbottomrule
\end{tabular}
\caption{Phoneme-grapheme associations}
\label{tab:phon-graph}
\end{table}

In the transcription of the texts, the speakers' hesitation is represented by ellipsis (...), see (\ref{ex.goat}). Square brackets [ ] indicate audio segments which are barely audible and cases where the speech is unclear (\ref{ex.kinace}). 
\ea 
\textit{bizekê ... wêş lûlê jeno.} \\ 
\gll bize-(e)kê wê=ş lûl(e)-ê jen-o \\ 
 goat\textsc{.f-def.pl.dir} \textsc{reflx=3sg:PSR} flute\textsc{.f-obl.f} play\textsc{.prs.ind-3sg:A} \\ 
\glt `The goats ... he (Pir Shaliyar) played the flute.' \hfill[JP.\ref{JP.54}] \label{ex.goat}
\z 

\ea 
\textit{\textbf{kina[çê]} jenî her wêş îna yanene.} \\ 
\gll kina[çê] jenî her wê=ş îna-∅ yane=ne \\ 
 daughter\textsc{.f} woman\textsc{.f} just \textsc{reflx=3sg:PSR} \textsc{deic-3sg:S} house\textsc{.m=post} \\ 
\glt `The girl, [or] the woman was alone in the house.' [JH.\ref{JH.74}] \label{ex.kinace}
\z  

Phrases inserted  from other languages, such as Kurdish\il{Kurdish} and Arabic\il{Arabic}, are put with the initial letter of the language in the superscript, e.g.: 
\ea \textsuperscript{A}\textit{exî mistefa recilul kamilon}\textsuperscript{A} 
\glt ‘My brother, Mustafa, is a well-rounded man.’ [ZP.\ref{ZP.18}] \z
\ea \textsuperscript{K}\textit{sûre gîyan yekê tirî lê de}\textsuperscript{K} 
\glt ‘May I be your sacrifice, you, red face, kick it once more!' [BP.\ref{BP.172}]
\z 


\subsection{Translation conventions}
The English\il{English} translation aims to be as close as possible to the original text, even in cases where there are false starts and slips of the tongue in the narration of stories. I have chosen to translate all the narratives into the past tense, as most of them are tales or events from past times. This applies as well to the use of the present narrative to recount past events in the tales. 

It is often the case that some words and meanings are implicit or not stated in the text, translation of which is required for the proper understanding of the English\il{English} translation. In such cases, in the glossed texts, square brackets indicate missing words and intended meanings. In other words, they provide additional context for understanding the tale content (as illustrated in \ref{ex.her}). 

\newpage
\ea 
\textit{`her ta îse qisêş nekerdênê.'} \\ 
\gll her ta îse qisê=ş ne-kerdê=nê \\ 
 donkey until now talk\textsc{.pl.dir=3sg:A} \textsc{neg-}do\textsc{.pst.ptcp.pl=cop.3pl:O} \\ 
\glt `\textbf{[He said with surprise]}, ``the donkey hadn't talked until now!''' [HB.\ref{HB.46}] \label{ex.her}
\z

Round brackets, on the other hand, clarify the reference of participants in the tale. 
\ea 
\textit{serew des kero gireway.} \\ 
\gll serew des ker-o gireway \\ 
 from\_above hand\textsc{.m} do\textsc{.prs.ind-3sg:A} cry\textsc{.inf} \\  
\glt `He (Little Hama) started to cry on the roof.' \hfill[BP.\ref{BP.152}]
\z

Note that this translation convention, with square brackets to represent missing information and round brackets to clarify the intended referents, was only applied to the glossed texts. This will enable interested readers, such as linguists, to consult the glossed version of the text to see the missing information in the source language (i.e. Hewramî). On the other hand, in the parallel texts, the brackets, not their content, were removed to prevent the flow of the stories from being interrupted. This will make the parallel texts easily readable and accessible to a broader audience.  In the case of round brackets, this meant replacing the pronouns with the nominal referents in the narratives. For example, compare the translation of the following examples in the parallel text vs. glossed text. 
\ea 
\textit{mênê qeredax.}\\
\gll m-ê-nê qeredax \\ 
 \textsc{ind-}come\textsc{.prs-3pl:S} \textsc{pn} \\ 
\glt \textbf{Glossed text:} `\textbf{They (the man and his wife)} came to Qaradagh.' \\
\textbf{Parallel text:} `\textbf{The man and his wife} came to Qaradagh.' [ZB.\ref{ZB.4}]
\z 

\ea 
\textit{yewaşê ađîç maroşo mêwe.} \\ 
\gll yewaşê ađ=îç m-ar-o=ş=o m-ê=we \\ 
 then \textsc{3sg.dir.m=add} \textsc{ind-}bring\textsc{.prs-3sg:A=3sg:O=compl} \textsc{ind-}come\textsc{.prs.3sg:S=compl}\\ 
\glt \textbf{Glossed text:} `Then, \textbf{He (Heyas)} brought \textbf{her (i.e. his wife)} \textbf{[and]} came \textbf{[to Hewraman]}.'\\
\textbf{Parallel text:} `Then, \textbf{Heyas} brought \textbf{his wife} \textbf{and} came \textbf{to Hewraman}.'
[JH.\ref{JH.65}]
\z 

\subsection{Glossing conventions}

Throughout the book, I have followed Leipzig Glossing Rules as originally put forward by \citet[]{comrie2015leipzig}. Additional glossings that were absent from the Leipzig glossing list were taken from CorpAfroas's list of glossing conventions \citep{corpafroas2017}. Due to the multi-functional nature of bound person forms in Hewramî, the glossed texts contain the syntactic role of the person forms, where abbreviations in capital letters --preceded by a colon -- indicate the function of person forms: S: intransitive subject; O: transitive object; A: transitive subject; R: non-core argument; PSR: possessor; NC: non-canonical subject. Example:
\ea 
\textit{beroma yanew wêşa.} \\ 
\gll ber\textbf{-o=ma} yane-û wê\textbf{=şa} \\ 
take\textsc{.prs.ind\textbf{-3sg:A=1pl:O}} house\textsc{-ez.gen} \textsc{reflx\textbf{=3pl:PSR}} \\ 
\glt `He will invite us; he will take us to his [Lit. their] house.' \hfill[HB.\ref{HB.35}] 
\z 

In the glossed texts, the literal meaning of phrases and words, including the use of pronouns, is indicated in footnotes. This will help the readers make sense of the idiomatic language (as illustrated in \ref{ex.lit}).
\ea \label{ex.lit}
\textit{şewê padşa wêş werm wîno.} \\ 
\gll şew(e)-ê padşa wê=ş werm wîn-o \\ 
 night\textsc{.f-indf} king\textsc{.m} \textsc{reflx=3sg:PSR} sleep see\textsc{.prs.ind-3sg:A} \\ 
\glt `One night, the king had a dream.\footnote{[Lit. The king saw dream.]}'
\hfill[ZP.\ref{ZP.30}]
\z 

Object language examples are presented in two versions, an orthographic one that corresponds to actual surface realisations, and the morphologically segmented version, which contains representations of the morphemes that are closer to their assumed underlying forms. By way of example, (\ref{ex.kisunderlyin}) illustrates how the surface forms \textit{winû} `blood of' and \textit{kîseɫê} `tortoise (\textsc{f.sg.obl})' in the orthographic version can be segmented at the underlying representation. 

\ea \label{ex.kisunderlyin}
\textit{winû kîseɫê sawî be...} \\ 
\gll win(î)-û kîseɫ(î)-ê s\'aw-î be \\
 blood\textsc{.f-ez.gen} tortoise\textsc{.f-obl.f} rub\textsc{.prs.sbjv-2sg:A} to \\ 
\glt `You may rub tortoise's blood on ...'  \hfill[DG.\ref{DG.47}]
\z 

Finally, case distinction in the singular is not visible for masculine nouns ending in -î (e.g. \textit{mizgî}), and feminine nouns ending in -ê, e.g. \textit{kinaçê}, including also feminine nouns with the definie suffix \textit{-ekê}, as the direct and oblique forms are identical. In the glossed texts, these nouns are glossed without any specification for case. 

\section{Genres and narrative styles}\label{sect:genre}
Most of the texts in this volume belong to local anecdotes about the recent past and narratives on the oral history of the region. It is notable that local anecdotes, especially those linked to the Sheikhs and religious leaders in Hewraman, often incorporate mythological and magical elements, and link them to the greatness of these leaders. There is no specific term in Hewramî for this sort of narrative, which can otherwise be categorised as hagiographies. On the other hand, \textit{řaze} `myth, story' is the specific term used for folktales, e.g. Texts \ref{text:K} and \ref{text:L}. Regardless of their genre, almost all the texts in this volume contain moral lessons and the social history of the region.

Hewramî narratives are told in the form of monologues. Most texts feature the present tense as a narrative tense. In addition, the perfect may also be used as a narrative tense (see \textref{text:N}). Common stylistic devices in narratives include tail-head-linkage as a means of discourse cohesion, and the use of discourse connectors (including the verb `to rise').


\section{Narrators}\label{sect:narrators}
\begin{table}[b]
\caption{Narrators sorted by age}
\label{tab:narrators}
    \centering
	\begin{tabular}{lllll}
    \lsptoprule
        \textsc{name} &\textsc{sex} &\textsc{age}\super{a} &\textsc{home} &\textsc{text}\\
	\midrule
		Mohammad& \male  & 85  &Serûpîrî & \ref{text:B}, \ref{text:D}, \ref{text:F}\\
        Saleh&\male   &80  &Hewraman Tekht  & \ref{text:A}, \ref{text:E}, \ref{text:G}, \ref{text:M}, \ref{text:O}\\
        Amin & \male  & 75  & Benen & \ref{text:C}, \ref{text:H}, \ref{text:I}, \ref{text:J} \\
        Hanifa&\female & 70 & Hewraman Tekht & \ref{text:N}  \\
        Moʿmen& \male & 65 & Hewraman Tekht & \ref{text:K}, \ref{text:L} \\
    \lspbottomrule
	\multicolumn{5}{l}{\footnotesize \super{a}Estimated age at the time of recording.}
\end{tabular}
\end{table}

The narrators are all native speakers of Hewramî. The female narrator is monolingual in Hewramî. The other narrators exhibit some bilingualism in Kurdish\il{Kurdish}. These speakers have a weak bilingualism pattern, with Hewramî\il{Hewramî} as their dominant language and Kurdish\il{Kurdish} as less dominant. In addition, some of these narrators showed weak competence in Persian\il{Persian}. Most of the narrators live in Hewraman Tekht. This includes also Amin, the narrator of Texts \ref{text:C}, \ref{text:H}, \ref{text:I}, \ref{text:J}, who at the time of recording the tales (summar of 2022) was living in the summer habitat Benen, tending to livestock. The narrators have a living memory of the traditions and stories associated with the most recent Sufi masters in Hewraman.



\section{The texts}\label{sec:texts}\label{sect:texts}
\begin{table}[b]
\caption{Texts}
\label{tab:texts}
    \centering
    \small
\begin{tabularx}{\textwidth}{lllQr}
    \lsptoprule
        \textsc{text} & \textsc{title} & \textsc{id} &\textsc{topic} & \textsc{words}\\
	\midrule
		\ref{text:A}& \textit{zaroɫe û bizê} & ZB & Local anecdote/myth about an abandoned baby  & 410 \\
        \ref{text:B}&\textit{zaroɫe û qiřolû darî} & ZQ &Local anecdote/myth about an abandoned baby  & 390 \\
        \ref{text:C}&\textit{herbene} & HB & Local anecdote/myth about a talking donkey & 541 \\
        \ref{text:D}&\textit{peɫê merekuř} & PM  &Local anecdote/myth, grasshoppers & 372 \\
        \ref{text:E}&\textit{derde gulî} & DG & Local anecdote/myth about a man suffering from leprosy & 540 \\
        \ref{text:F}&\textit{Şêx ʕumer û Cafir san} & ŞC & Local anecdote about recent history  & 649 \\
        \ref{text:G}&\textit{duwê padşε} & DP  & Oral history, two kings claiming Hewraman & 391 \\
        \ref{text:H}&\textit{jîwayû Pîr Şelîyarî} & JP  & Oral history, hagiography  & 1,743 \\
        \ref{text:I}&\textit{zemawinew Pîr Şalîyarî } & ZP & Oral history, hagiography & 916 \\
        \ref{text:J}&\textit{babaw Pîr Şalîyarî} & BP & Oral history, hagiography  & 1,571 \\
        \ref{text:K}&\textit{kuřû şuwaney} & KŞ  & Folktale  & 796 \\
        \ref{text:L}&\textit{jîwayû Heyasî} & JH  & Folktale & 717 \\
        \ref{text:M}&\textit{řisûmatû ewsayma} & RE & Recollections of traditional life & 393 \\
        \ref{text:N}&\textit{jîwayû ewsayma} & JE & Recollections of traditional life, autobiography & 440 \\
        \ref{text:O}&\textit{jîwayû min} & JM  & autobiography & 417 \\
    \midrule
        \textbf{total}&&&&\textbf{10,286}\\
    \lspbottomrule
\end{tabularx}
\end{table}

The 15 texts in this volume are grouped according to the supposed genres in Table \ref{tab:texts}. The ID column in Table \ref{tab:texts} refers to the Text ID, as used in the Hewramî grammar and Zenodo repository. Texts \ref{text:A}-\ref{text:F} are local anecdotes where healing powers and elements of myth are linked to Sufi masters. This includes, for instance, understanding the language of animals in Texts \ref{text:C}, \ref{text:D} and \ref{text:E}, predicting the future in \textref{text:D}, and miracles occurring to Sufi's followers in Texts \ref{text:A} and \ref{text:B}. \textref{text:G} is a retelling of a historical event in Hewraman, when two chieftains claimed the Hewraman region. Texts \ref{text:H}-\ref{text:J} can be best categorised as hagiographies: Texts \ref{text:H} and \ref{text:I} contain biographies of Pir Shaliyar, and \textref{text:J} is a story about the events of his grandfather's time. These texts also include information about the social history of the region in the 11th-12th C.E. Texts \ref{text:K} and \ref{text:L} can be classified as folktales. Texts \ref{text:M} and \ref{text:N} are recollections of traditional life, and \textref{text:O} and part of \textref{text:N} are autobiographies.  
 


\end{sloppypar}



 





