This narrative recalls marriage customs in the past among Hewramî people. Traditionally, when a young man wanted a marry, his father or an elder would ask the girl's father for her hand. If the girl's family consented to the marriage, the young man would serve them for a while, during which the families of to-be-married couple would share meal and get to know each other. The chief or noblemen would also play a role, asking for gifts in exchange for giving consent to the marriage. 

The narrator reports how marriage customs have changed over time. The father's authority is lost, and chiefs no longer play a role in the community life. He also highlights how past social structures offered unpaid labour for the chiefs.