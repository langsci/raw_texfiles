This tale is about Pir Shaliyar’s grandfather. He was the chieftain of Hewraman around 11-12 C.E. At the time, the people of Hewraman paid allegiance to local rulers in Iraq. Once, the ruler in Iraq sent some officers to collect taxes from Hewraman. The officers did not return to Iraq after collecting taxes; instead, they engaged in debauchery with women. Pir Shaliyar’s grandfather ordered people to kill the officers and then go to Iraq to explain the situation to the local ruler. Once it was time to go to Iraq to report the killing, people refused to accompany Pir Shaliyar’s grandfather, and he had to go there by himself with two other people. The ruler in Iraq imprisoned him and his friends. Later, he destroyed the prison and returned to Hewraman. 
