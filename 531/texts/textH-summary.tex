Pir Shaliyar was a legendary derivish who lived in Hewraman in the 12th Century C.E. He is associated with supernatural powers, such as levitation in the story. As a child, he was raised by his uncle’s family following his parents’ death. Realising that he had special powers, the uncle sent Pir Shaliyar to study in Baghdad. After finishing his studies of Islamic jurisprudence, Pir Shaliyar returned to Hewraman and built a place for 40-day fasting called \textit{Çilexane}. The people of Hewraman only began to fully believe in him when he was able to cure the Princess of Egypt, who was both deaf and mute. Following the order from the King of Egypt, his daughter should be married to whoever cured her. Pir Shaliyar’s marriage to the princess is celebrated yearly in Hewraman and attracts many tourists. 


\begin{figure}[htp]
    \includegraphics[width=0.7\textwidth]{figures/pirshaiyar.jpg}
    \caption{Pir Shaliyar's shrine--Hewraman Tekht}
    \label{fig:shrine}
    \end{figure}



    
\begin{figure}[htp]
    \includegraphics[width=0.7\textwidth]{figures/çilexane.jpg}
    \caption{\textit{Çilexane} associated with Pir Shaliyar--Hewraman Tekht}
    \label{fig:cilexane}
    \end{figure}