\begin{Parallel}{0.47\textwidth}{0.47\textwidth}
    \ParallelLText{\noindent \textit{zemanê zemaniş bîyen. ême eger zemanê ya qeymîyêma ya zemanû wêma jenîma ardêbo, amεnê luwεnmê werû berew kabray, hawmaɫû wêma. ``babe gîyan î kinaçête'' ya tatema luwan ya mamoma ya kesûkarma. aman luwan wateniş, ``bira gîyan! min xû wêt zanî hawmaɫitna. be bîyew nebîyeym zanî. ʕerzû biraw wêm kerû î kinaçête de pî kuřîme.''}}
    \ParallelRText{\noindent In the old times ... Once, in the time of our elders or in our time, if one of us married, our elderly family members would go to the door of the the girl's father, who was the neighbour. They would say, ``Dear brother, give us this daughter of yours in marriage'' -- either our father would go to the girl's father, or our uncle, or a relative of our -- they would go and say, ``Dear brother, you know well that I am your neighbour. You know about my resources and shortages. Let me tell you, give your daughter to my son in marriage.''}
\end{Parallel}

    \vspace{.4cm}


\begin{Parallel}{0.47\textwidth}{0.47\textwidth}
    \ParallelLText{\noindent \textit{a wextî eger xway îrađeş bîyebo serû qisekê, eger xwaw peyẍemerî îrađeş bîyebo serû qisekê, ewîç waten, ``qey çêşî kero! ana nîyam la paɫaket welê be qewrû be hurmet.'' êtir kabra fers kere zemanû a anê hîtê paɫε bîyênêw çwar gezê parçe bîyenû.}}
    \ParallelRText{\noindent Then, if God was willing over the subject, if God and the Prophet were willing that the affair should happen, then, the girl's father would say, ``No problem! She is yours, I have put her on your shoes. But only treat us with appreciation and respect.'' Then, let us say, at the time, the wedding gift consisted of some pairs of shoes and four gaz\footnotemark{} of cloth.}
\end{Parallel}\footnotetext{A unit of measure equivalent to 72 cm.} 

    \vspace{.4cm} 


\begin{Parallel}{0.47\textwidth}{0.47\textwidth}
    \ParallelLText{\noindent \textit{hewramîgerî řowê luwan kariş pey kerden cîyaw xizmetîyû. řowê luwan nîştê\-nêre sere gawêşa wardenû. a wextîyekey hurêsênê hemahengîşa bîyen. bîyen saɫê ta amεnê, ta danedeso.}}
    \ParallelRText{\noindent According to Hewramî customs, he the boy would go and work for the father-in-law for service for one day. One day, the bridegroom's family would go to the bride's family, sit with the bride's family and eat a cow. Then, they were on good terms. A year would pass before the boy would come, and they would give her to him.}
\end{Parallel}

    \vspace{.4cm} 


\begin{Parallel}{0.47\textwidth}{0.47\textwidth}
    \ParallelLText{\noindent \textit{ey pîya êtir emînû yane bîyenû xizmet\-kar bîyen. haɫîta? firmanbirđar bîyen, ta sahîbû î kinaçê waço, xeber bido waço, ``fiɫane kar! luwe hardeka barewe, luwe karekem pey kere.'' a wextîyekey firmanbirđar bîyen ta jenekêş berdêne. şûnû jenekêçre a wextîyekey êtir hemahengîşa weş bîyen. qisew mise nebîyen.}}
    \ParallelRText{\noindent There was a man who was the family's confidant. He was a servant. You see? He was obedient. The girl's father would say to him and ask him, ``Do such-and-such tasks! Go and bring the flour. Go and do this task for me.'' He was at their service until he married the girl. After the marriage, the couple's families were on good terms. There was no quarrel between them.}
\end{Parallel}

    \vspace{.4cm}


\begin{Parallel}{0.47\textwidth}{0.47\textwidth}
    \ParallelLText{\noindent \textit{xeplezeřatînê nîştênêre naneşanê xeplêşa nîyεnêre. ew aman xeyr amayê weş amayê beynne bîyenû. zemanêç jenekêş berdêne êtir ço duma bîyênê têkeɫ.}}
    \ParallelRText{\noindent \textit{Xaplazaratene},\footnotemark{} they would sit together and cook a basket of xaple. They would come. There were greetings and welcoming between them When the brides were taken, the families would become one.}
\end{Parallel}\footnotetext{A bread made of corn flour.}

    \vspace{.4cm} 


\begin{Parallel}{0.47\textwidth}{0.47\textwidth}
    \ParallelLText{\noindent \textit{êtir îse zeman fařyan. haɫît bo? kabra eđaw tate, kabra melo waço, ``jenîm de!'' mebalê gêro desşo. kinaçekê qisê kero. kuřekeyç qisê kero. tate bîyen mane asawû ewsey. cûwabêş nîya xeberiş nîya ta ``tate!'' ``a!'' ``min, qewɫim dan pey fiɫane kesî. miđey mehr kerû. međey mehr kerû. to çikarenî?'' haɫîta?}}
    \ParallelRText{\noindent Now, things have changed. You see? Now consider a fellow and his parents. The fellow wouldn't go to his parents and say, ``Find a girl for me!'' He holds a mobile phone in his hand. The girl talks to the boy. The boy, too, talks to the girl. The father is no longer the authority. Then, there is no news until the girl says, ``Father!'' The father says, ``Yes!'' ``I have promised such-and-such a person. If you accept, I will marry him. If you don't accept, I will marry him. It is not your business.'' You see?}
\end{Parallel}

    \vspace{.4cm}


\begin{Parallel}{0.47\textwidth}{0.47\textwidth}
    \ParallelLText{\noindent \textit{î memliketû ême ne welê wiɫatare êtir tate bîyen çûwêw desû şuwaney. ême zemanma řisûmatû ewsayma eçîne bîyen. eçê duwê begêma bîyênê luwεnê. ``fiɫanekes!'' ``beɫê!'' ``siɫam ʕeleykum!'' ``ʕeleyke selam!''}}
    \ParallelRText{\noindent It is not common in our region, but elsewhere, the father has become like the stick in the shepherd's hand.\footnotemark{} In the old times, our customs were like this. We had two chiefs to whom the girl's family would go. The nobleman would say, ``Such and such person!'' The girl's father would say, ``Yes!'' The nobleman would say, ``Peace be upon you.'' The girl's father would say, ``And upon you be peace.''}
\end{Parallel}\footnotetext{Meaning that he has no power over whom his children will marry.}

    \vspace{.4cm} 


\begin{Parallel}{0.47\textwidth}{0.47\textwidth}
    \ParallelLText{\noindent \textit{keɫeşîrê gêrten bawşîşo qoqeqoqû epagehur. ``min kinaçêt dεne kuřû fiɫane kesî. to qibûɫ kerî ya mekerî?'' ewîç waten ``ba bizanmê.'' pîney bayê masîç bero. haɫîta? keɫeşîrêwe da duwê wero ya yerê ca qewɫ miđo. maço, ``dey luwe deş.''}}
    \ParallelRText{\noindent The girl's father would grab a rooster in his arms to present to the nobleman, while the rooster was crowing from this side to up the top of the village and say to the lord, ``I have given my daughter to the son of such-and-such person. Will you consent to it or not?'' The lord would say, ``Let me see.'' He would say this so that the girl's father would take a bowl of yoghurt to him too. You see? The nobleman would eat two, three roosters, and then would consent to the marriage. The nobleman would say to the girl's father, ``Go and give her to him.''}
\end{Parallel}

    \vspace{.4cm}


\begin{Parallel}{0.47\textwidth}{0.47\textwidth}
    \ParallelLText{\noindent \textit{zemanê ême î begêma bîyênê. î begê qibûɫ kera, a wextîyekey awyaneşa sanan. sûranew jenaşa sanan. sûrane yanê mişo barî to zeř dey ca min maçû ``erê.'' sûrane anen.}}
    \ParallelRText{\noindent Once we had these chiefs in our region. Let's say the noblemen consented to the marriage. Then, the noblemen would ask for \textit{awyana} ``money in exchange for giving consent to marriage''. The noblemen would grab \textit{surana} ``money in exchange for giving consent to marriage''. \textit{Sûrane} means that you should give me money, then I will say, ``Yes'' to the marriage. \textit{Sûrane} was that.}
\end{Parallel}

    \vspace{.4cm}


\begin{Parallel}{0.47\textwidth}{0.47\textwidth}
    \ParallelLText{\noindent \textit{î memliketû ême řisûmatû awe zemanîş eçîne bîyen. î begêma bîyênê, begê. bê a bega ême netawanma ħîç kermê. berdênmêşa lo keney bê heqû. berdênmêşa diřê keney bê heqû.}}
    \ParallelRText{\noindent The traditions of earlier times were like this in our region. We had these chiefs. Without them, we were not able to do anything. They would take us to mow fodder grass for free. They would take us to mow prickles for free.}
\end{Parallel}

    \vspace{.4cm} 


\begin{Parallel}{0.47\textwidth}{0.47\textwidth}
    \ParallelLText{\noindent \textit{luwεnmê bayê masêma berdênê bê ewû. desma gêrten sînemawe siɫam kerdey. tersεnmê werşaw. zemanêç payîz amano kewçê xelema bîyebo teẍarê. ađê amεnê tenekê girđ yo tene\-kêşa berden. pase jîwyεnê.}}
    \ParallelRText{\noindent We would go to the noblemen and take them bowls of yoghurt free of charge. We would fold our arms across our chest as a sign of respect and greet them. We were scared of them. We might have had a taghar\footnotemark{} of grains when it was autumn. The noblemen would come and each take a sack of wheat. People used to live like that.}
\end{Parallel}\footnotetext{A unit of weight equivalent to 120 kilograms.}

    \vspace{.4cm} 