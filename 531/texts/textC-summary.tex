Once, there was a lack of flour in the monastery of Sheikh Aladin, after which he asked his stableman to take his donkey with him and go to the village of Shashk to bring back some flour. On the way, the donkey keeper noticed he had no money to buy flour, nor had the Sheikh told him where to fetch flour. The donkey began to speak and told the man what to do. The donkey keeper was offended to learn that the Sheikh had told the donkey what to do, not him. After returning to the monastery with sacks of flour, he went into solitude and kept aloof. The Sheikh criticised the donkey keeper, saying that he should instead be happy since he is the only one who can understand the language of donkeys! 
