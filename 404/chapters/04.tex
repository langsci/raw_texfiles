\chapter{Turn-taking} \label{turntaking}

\section{Introduction}\label{sec:turntaking_introduction}

In this chapter, we present experimental evidence on strategies of turn-taking in German adults with and without a diagnosis of autism spectrum disorder. Turn-taking is the most fundamental skill in spoken interaction and, although it is cognitively extremely demanding for interlocutors to exchange turns in quick succession, the timing of turn-taking has been shown to be remarkably fast in previous work (see references in \sectref{turntaking_background_general} below). Rapid turn-timing has further been shown to be the preferred strategy by many different groups of speakers from varying linguistic and cultural backgrounds \citep[e.g.][]{stiversUniversalsCulturalVariation2009}. However, there is only scant empirical evidence on turn-taking in ASD and none whatsoever on turn-timing in conversations between autistic adults.

The results presented in this study constitute the first reliable quantitative evidence on the turn-taking and dialogue management strategies of individuals on the autism spectrum. Additionally, the findings on turn-taking in the CTR group constitute a major contribution to the thus-far relatively sparse empirical evidence on dialogue management in German in their own right. Data from Map Task dialogues performed by 28 speakers in disposition-matched dyads are presented. Turn-timing is investigated across the task as a whole as well as at different stages of dialogue, at both the group and the dyad level. Descriptive summary statistics, visualisation and Bayesian modelling were used to analyse results. Additionally, relative speaking times within dyads, prosodic aspects of turn-taking, and the effects of unexpectedness on turn-transitions are discussed.

For most aspects of dialogue management and when considering the dialogue as a whole, no differences between the ASD group and the CTR group were found. However, closer inspection reveals that 1) autistic dyads produced longer gaps between turns in the early stages of dialogue, 2) autistic dyads reacted differently to the introduction of matching and mismatching landmarks and 3) speaking times were less balanced within dyads for the ASD group. I will discuss the implications of these results, relate them to general theories of autism and to the notion of universal patterns of turn-timing in spoken dialogue, and furthermore compare the current findings with those from research on second-language speech.

The turn-timing analysis has been reported in \citet{wehrleTurntimingConversationsAutistic2023}, an earlier version of parts of the Match vs. Mismatch analysis in \citet{janzTurnTransitionsSpectrumConversational2019}.




	\section{Background}\label{sec:turntaking_background}
	
	Turn-taking is in essence a form of cooperative interaction. Humans engage in many temporally coordinated collaborative activities besides spoken interaction, such as manual labour, dancing or music-making \citep[see e.g.][]{hawkinsCommunicativeInteractionSpontaneous2013}. Similarly, communicative turn-taking, in either the vocal or gestural modality, is not limited to humans. Many different species from different taxa perform tightly synchronised and regulated communicative interactions. Such behaviours are sometimes referred to with the term turn-taking in studies of animals, but behaviours described as \textit{duetting} or \textit{antiphonal calling/singing} can also be seen as equivalent to or even indistinguishable from what is often defined as turn-taking. For more details on the cross-species comparison, see \citet{ravignaniInteractiveRhythmsSpecies2019}; for an overview, see \citet{pikaTakingTurnsBridging2018}; and for detailed descriptions, see e.g. \citet{takahashiEarlyDevelopmentTurntaking2016} on marmosets and \citet{frohlichUnpeelingLayersLanguage2016} on bonobos and chimpanzees.
	
	Despite these cross-species similarities, human turn-taking in conversation seems to be a particularly remarkable phenomenon because 1) it is executed with split-second, even virtuoso precision and flexibility, 2) it involves the parallel prediction, planning and production of utterances which are improvised, yet rich with meaning, and 3) it is the key means through which human language, and to a considerable extent human culture, are learned and transmitted \citep[cf.][]{schegloffInteractionInfrastructureSocial2020}.
	
	In the following section, I will first summarise the most relevant general research on turn-timing in human spoken interaction and then move to a critical discussion of previous research on turn-timing in ASD in particular.

 \subsection{General principles and patterns of turn-timing in spoken interaction}\label{turntaking_background_general}

Turn-taking is the organisation of discourse into alternating units between speakers, with the aim of ensuring that generally no more or less than one participant speaks at any one time \citep{sacksSimplestSystematicsOrganization1978}. Most turns are short and most transitions between turns consist of very short gaps between speakers \citep{levinsonTurntakingHumanCommunication2016}, which are preferred to other possible kinds of transition such as longer gaps or overlaps (two speakers talking at once).

A modal value of approximately 200 milliseconds of silence between speakers has been shown for a wide range of languages and speakers, with only slight language-specific variation \citep{heldnerPausesGapsOverlaps2010,stiversUniversalsCulturalVariation2009,weilhammerDurationalAspectsTurn2003,dingemanseTextTalkHarnessing2022}. This behaviour seems remarkably robust to individual, methodological and contextual variation -- in contrast to many other aspects of language \citep{christiansenCreatingLanguageIntegrating2016, evansMythLanguageUniversals2009, schegloffReflectionsLanguageDevelopment1989, sterponiRethinkingLanguageAutism2015}.

Effective turn-taking of precisely the kind described above is essential for the smooth flow of conversation, and human language is acquired, learned and practised almost entirely by means of these alternating exchanges of short bursts of speech. Gesture also plays an important role in the organisation of turn-taking, but as the methodology used to elicit speech for the corpus under study prevented participants from seeing each other, I will focus exclusively on spoken language in this account. For research on gesture and turn-taking in both signed and spoken languages, see e.g. \citet{mcclearyTurntakingBrazilianSign2013,hollerProcessingLanguageFacetoface2018,zellersProsodyHandGesture2016,bohusFacilitatingMultipartyDialog2010,demarchenaAtypicalitiesGestureForm2019}.

Although the smooth and rapid exchange of turns is so common in human interaction, starting with \textit{proto-conversational} turn-taking in infancy \citep{gratierEarlyDevelopmentTurntaking2015}, it is cognitively extremely demanding. The fact that interlocutors consistently manage to avoid both long silent gaps and periods of overlap between speakers is quite remarkable given that the planning of even a very short utterance takes at the very least 500 ms and often considerably longer -- 900 ms for utterances of more than two words, 1500 ms for simple sentences -- far longer in any case than the well-attested typical short gap of around 200 ms \citep{gleitmanGiveTakeEvent2007,griffinWhatEyesSay2000,schnurPlanningPhonologicalLevel2006,wesselingEarlyPreparationExperimentally2005}.

It is therefore essential for interlocutors to \emph{predict} the further content and the temporal endpoint of another speaker's turn, and to do so with a great degree of accuracy. This implies that interlocutors need to execute the cognitively highly challenging task of engaging in speech perception, planning and production in parallel. Although the interpretation of utterance-final prosodic cues is one important aspect of turn-taking, this in itself is not sufficient to enable the exchange of turns with sufficient speed and precision, as such cues appear far too late in the speech stream \citep{bogelsListenersUseIntonational2015,deruiterProjectingEndSpeaker2006,torreiraVocalReactionTimes2022}. It seems most likely instead that a two-stage process takes place. In this account, interlocutors first plan and formulate (and update) their next utterance as early as possible in reaction to the message (predicted to be) conveyed by their conversational partner. This planned utterance is then stored in a kind of mental buffer until turn-final prosodic cues are detected in the speech stream of the interlocutor, at which point speech production is initiated \citep{barthelNextSpeakersPlan2017,barthelTimingUtterancePlanning2016}.

Prediction of a conversation partner's next utterance, at all levels of language, is therefore essential for achieving rapid turn-timing, and such predictions can only be made accurately if listeners are acutely aware of and attuned to the linguistic cues produced by their interlocutor.

\subsection{Conversational turn-taking and autism}\label{turntaking_background_ASD}

Differences in social interaction and communication are crucial criteria for diagnosing ASD. Among communicative skills, pragmatic aspects (e.g.~inferring intentions and beliefs of a speaker) have conventionally been considered as particularly challenging for many autistic individuals, in contrast to more explicit aspects of language, such as syntax \citep{tager-flusbergLongitudinalStudyLanguage1990a, tager-flusbergLanguageCommunicationAutism2005a, eigstiPragmaticsMorphosyntacticDevelopment2007}. A rapid exchange of turns requires certain skills which have been described in previous research to be ``impaired" in ASD \citep{chassonSocialCompetenceImpairments2014}. These skills include mutual perspective-taking and the ability to decode another person's emotional and linguistic signals. 

Thus, it might seem plausible to assume resultant difficulties with turn-taking in the autistic population. However, the evidence is insufficient as only a small number of previous studies have investigated turn-timing in the context of ASD, and none have investigated conversations between autistic adults. Previous quantitative research on turn-timing in the context of ASD  can be summarised most succinctly as reporting a tendency for more and longer silent gaps in conversations involving autistic individuals. The relevant studies are discussed in more detail in the following paragraphs.

In the first major empirical investigation into turn-taking in autism, \citet{feldsteinChronographyInteractionsAutistic1982} report that 12 autistic adolescents and young adults (ages 14--20) produced longer pauses and shorter utterances (and therefore longer gaps) overall than controls, in line with previous, anecdotal observations reported in \citet{fayEmergingLanguageAutistic1980}. However, the generalisability of these results has to be questioned due to three key methodological issues: the age range and intellectual abilities of experimental subjects, the nature of the speech data under consideration, and the methods by which they were elicited \citep[cf.][]{griceLinguisticProsodyAutism2023}. The age range is such that at least some participants have to be assumed to be at different stages of language development, especially as this development tends to be delayed in ASD. Furthermore, no information is given on either general or verbal IQ. Finally, speech data consist of conversations between autistic subjects on the one side and either their parents or the experimenters themselves on the other side. Therefore, by the admission of the authors themselves, “the interactions...were much more like interviews than unconstrained conversations” \citep[p. 453]{feldsteinChronographyInteractionsAutistic1982}.

More recently, \citet{heemanAutismInteractionalAspects2010} investigated 26 children diagnosed with ASD who were between 4 and 8 years old. All subjects were judged to be verbal and ``high-functioning". Speech was recorded during administration of the Autism Diagnostic Observation schedule \citep[ADOS, ][]{lordAutismDiagnosticObservation2000}, a standardised diagnostic test for ASD. The authors show that autistic children produced longer gaps than age-matched children without a diagnosis of ASD. However, the age (range) of participants and the method of elicitation alone are, each in their own right, reasons enough to preclude reliable conclusions on general strategies of turn-taking in ASD \citep[for similar results on Korean see][]{choiConversationalTurnTakingTopic2013}.

The authors of \citet{warlaumontVocalInteractionDynamics2010} investigated day-long naturalistic recordings between children and their parents and found longer silences before responses to questions in the ASD compared to a CTR group \citep[see also][]{warlaumontSocialFeedbackLoop2014}.

The most recent published work of relevance \citep{ochiQuantificationSpeechSynchrony2019} is notable for featuring adult autistic participants, although they were considerably younger on average than in the sample under investigation in this book. Speech data were limited to recordings of the ADOS schedule. Similarly to all the above studies, \citet{ochiQuantificationSpeechSynchrony2019} found a clear tendency for longer silent gaps in the ASD compared to a control group.

Finally, in a meta-analysis of the literature on adult–infant turn-taking, \citet{nguyenSystematicReviewBayesian2022} confirm the overall trend for more and longer between-turn silences in conversations involving individuals on the autism spectrum.

One notable departure from this consensus can be found in the wide-ranging and influential ``anthropological perspective" put forward in \citet{ochsAutismSocialWorld2004}. The authors set out to understand autistic persons not as isolated individuals but rather as social actors with a diverse range of strengths and difficulties in relation to socio-cultural factors and expectations. Crucially, in describing a ``cline of competence" across different social domains, \citet{ochsAutismSocialWorld2004} report that in the domain of conversational turn-taking, autistic children show few difficulties and ``seem to behave qualitatively like many of the unaffected [sic] peers in their families and communities" (p. 162). They speculate that the ``local orderliness of sequences" might suit the cognitive style typical for persons on the autism spectrum. The quantitative findings on autistic adults from this work, revealing no clear overall differences in turn-timing between the ASD and the CTR group, add some support to this earlier qualitative account.










\section{Data and analysis}\label{sec:turntaking_analysis}


Speech data from 28 German-speaking adults, 14 with and 14 without a diagnosis of ASD, were analysed. Speakers were recorded in disposition-matched dyads (ASD--ASD; CTR--CTR). For further details on subjects and materials, see \chapref{sec:data}.

The data set under study contains 18332 IPUs in total (inter-pausal units; here defined as speech separated by at least 200 milliseconds of silence). For an analysis of turn-taking, not these units of speech in themselves are of primary interest, but rather the points of transition between them. The data set contains 5668 such transitions overall. There are fewer turn transitions than IPUs because most of the latter were followed by another IPU from the same speaker; i.e. separated by within-speaker pauses (see \sectref{sec:BCFP_FP_silent}) rather than between-speaker gaps.

The start and end points of all transitions were precisely labelled by hand following an automatic first-pass segmentation of recordings into silent and non-silent intervals using  \emph{Praat} (version 6.1.09) \citep{boersmaPRAATDoingPhonetics2021}. I broadly follow the methodology of \citet{levinsonTimingTurntakingIts2015a} -- which in turn builds on \citet{heldnerPausesGapsOverlaps2010} -- for the continuous analysis of turn-timing, in order to facilitate comparison of the current results to previous work. Accordingly, audible in-breaths, clicks and similar noises were counted as part of \emph{silent} intervals, rather than speech. Filled pauses such as \emph{uhm}, on the other hand, were annotated as being part of non-silent utterances. Thus, I followed the approach of essentially analysing turn-timing from a linguistic, rather than a purely acoustic perspective (which would incidentally not solve the problem of experimenters having to subjectively determine thresholds for what is considered silence).

Following \citet{levinsonTimingTurntakingIts2015a}, all turn transitions were categorised as being either \textit{gaps}, \textit{between-overlaps} or \textit{within-overlaps}; see definitions in Figure \ref{fig:Transitions}. Within-overlaps do \emph{not} in fact entail a floor transfer from one speaker to another, and did therefore not enter into the analysis of turn-timing. Distribution and characteristics of within-overlaps are instead discussed separately in \sectref{turntaking_results_signal_within}.



\begin{figure}

{\centering \includegraphics[width=1\linewidth]{images/transitions} 
	
}

\caption{Categories of turn transition \citep[adapted from][]{levinsonTimingTurntakingIts2015a}. \textit{Gaps} are silent intervals between turn transitions; \textit{between-overlaps} are turn transitions composed of overlapping speech from both interlocutors. \textit{Within-overlaps} are not true floor transfer transitions, but rather represent passages of overlapping speech which are \emph{not} followed by a change of speaker (and therefore did not enter into turn-timing analyses). SPK = Speaker.}\label{fig:Transitions}
\end{figure}

Of the 5668 transitions in the data set, 3418 were silent gaps (60.3\%), 1326 were between-overlaps (23.3\%) and 924 were within-overlaps (16.3\%).
After the exclusion of within-overlaps, 4744 transitions remained for the analysis of turn-timing. Of these, 72\% were gaps, and 28\% were (between-)overlaps.

I follow previous studies on turn-timing in analysing turn transitions using the measure of Floor Transfer Offset (FTO), in which positive values represent gaps and negative values represent overlaps between speakers. Figure \ref{fig:FTO} gives a schematic representation.



\begin{figure}

{\centering \includegraphics[width=0.8\linewidth]{images/FTO} 
	
}

\caption{Floor Transfer Offset (FTO) measurements: Overlaps are represented with negative FTO values (see left arrow for an FTO value of about -600 ms); gaps are represented with positive FTO values (see right arrow for an FTO value of about +600 ms).}\label{fig:FTO}
\end{figure}

Bayesian linear modelling was used to test for group differences in FTO values (with \textsc{dyad} as a random factor).
Further, the interaction of FTO values with part of dialogue (see \sectref{turntaking_results_FTO_group_stage} for details) was tested in order to examine whether temporal dynamics might reveal any ASD-specific patterns.

Bayesian modelling confirmed that, across groups, there was no difference in FTO between all-male, all-female and mixed dyads. Gender as a factor was therefore disregarded in the following analyses.\footnote{A Gaussian model with \textsc{floor transfer offset} as the dependent variable, \textsc{gender combination} (all-female/all-male/mixed) as a fixed factor and \textsc{dyad} as a random factor was used, and no robust differences between any of the groups was found -- more details in the accompanying \emph{OSF} repository at \url{https://osf.io/v5pn4/}.}





\section{Results}\label{sec:turntaking_results}

I will first present the overall results on turn-timing in a continuous analysis. Results are presented at the level of groups as well as dyads and include an analysis taking into account different dialogue stages. The continuous analysis is complemented by a categorical analysis of different transition types.

This is followed by an in-depth analysis examining the role of unexpectedness in turn-timing. All transitions directly following the introduction of new landmarks were compared, contrasting Matches and Mismatches.

Finally, dialogue patterns beyond turn transitions are considered with an investigation of the overall distribution of silence, single-speaker speech and overlapping speech. Further, speaking times within dyads are compared, an overview visualisation of all turns for all dyads is presented, and an exploratory analysis of the prosodic constructions used to mark turn-ends and turn-beginnings is outlined.



\subsection{Continuous analysis of turn transitions}\label{turntaking_results_FTO}

The following results are presented using the measure of Floor Transfer Offset (FTO), which allows for a continuous representation of both gaps and overlaps along the same dimension by representing gaps between speakers as positive values and overlaps as negative values.

\subsubsection{Overall results by group}\label{turntaking_results_FTO_group}


Figure \ref{fig:FTOGroup} shows turn-timing values by group. Visual inspection alone makes it clear that values are very similar across groups. Overall, the ASD group has slightly higher FTO values, with a mean of 317 ms (SD: 599) and a median of 205 ms, compared to the CTR group with a mean of 238 ms (SD: 555) and a median of 175 ms.

Assuming 100-millisecond bins, both the ASD and the CTR group have a modal FTO value of 200 ms. In this regard, the current study directly replicates a number of previous findings on turn-timing from \citet{stiversUniversalsCulturalVariation2009} onwards.
Figure \ref{fig:FTOGroupHist} in Appendix \ref{appendix:b} presents histograms using 100-millisecond bins and is directly modelled after the histograms presented in \citet{levinsonTimingTurntakingIts2015a}.



\begin{figure}

{\centering \includegraphics[width=1\linewidth]{figures/graphics-FTOGroup-1} 
	
}

\caption{Floor Transfer Offset (FTO) values by group. Positive values represent gaps; negative values represent overlaps. ASD group in blue, CTR group in green. The dotted line indicates the value of 0 ms FTO, representing no-gap-no-overlap transitions. Dashed lines indicate the values of +200 ms (expected for typical transitions) and +/-700 ms FTO (unusually long transitions).}\label{fig:FTOGroup}
\end{figure}

\subsubsubsection*{Bayesian analysis}\label{turntaking_results_FTO_group_Bayes}

A Gaussian model with \textsc{FTO} as the dependent variable, \textsc{group} (ASD/CTR) as a fixed factor and \textsc{dyad} as a random factor was used for Bayesian analysis (more details below and in the accompanying files).

Model output confirms that ASD dyads produced somewhat higher FTO values (in ms) (\(\hat{\beta}\) = 326, 95\% CI = {[}237, 414{]}) than CTR speakers (\(\hat{\beta}\) = 250, 95\% CI = {[}174, 337{]}). The group difference in the model is reported with the ASD group as the reference level. Mean \(\delta\) = -74, indicating a trend towards lower FTO values in the CTR group. However, the 95\% CI {[}-173, 25{]} includes zero by some margin and the posterior probability \(P(\delta > 0)\) = 0.9 is below the heuristic threshold of 0.95. The model therefore does not suggest a reliable difference between groups, only a trend towards higher FTO values (i.e.~longer gaps) in autistic dyads.

A model with a normal distribution was used, and weakly informative priors with a normal distribution were specified for the intercept (\(\mu\) = 0, \(\delta\) = 6000) and for the regression coefficient (\(\mu\) = 0, \(\delta\) = 1000). The default priors of the \emph{brms} package were used for the standard deviation of the likelihood function, namely a Student's \emph{t}-distribution (\(\nu\) = 3, \(\mu\) = 0, \(\delta\) = 363.2), and for the standard deviations of random effects, Student's \emph{t}-distribution (\(\nu\) = 3, \(\mu\) = 0, \(\delta\) = 363.2).

\subsubsection{Overall results by dyad}\label{sec:turntaking_results_FTO_dyad}


Figure \ref{fig:FTODyad} presents FTO values by dyad. The plot clearly shows that distributions are extremely similar across dyads. Note, for instance, that the dashed line at the 200 ms mark (indicating very short gaps) runs close to the distributional peak of all dyads from both groups.
Assuming a bin width of 100 milliseconds, 11 out of all 14 dyads produced a modal value of 200 ms (with the modes of the remaining dyads not deviating by more than 100 ms).
Mean FTO values ranged from 137 ms to 503 ms across dyads. The group-level tendency towards slightly higher FTO values in the ASD group is reflected in the fact that four out of the five highest mean FTO values were produced by ASD dyads and four out of the five lowest mean values were produced by CTR dyads.

In order to corroborate the representativeness of group-level results, it was tested whether any single dyad had a decisive influence on the group level patterns by successively omitting individual dyads and rerunning the group-level analysis, and this was found not to be the case.



\begin{figure}

{\centering \includegraphics[width=1\linewidth]{figures/graphics-FTODyad-1} 
	
}

\caption{Floor Transfer Offset (FTO) values by dyad. Positive values represent gaps; negative values represent overlaps. ASD group in blue, CTR group in green.}\label{fig:FTODyad}
\end{figure}


\subsubsection{Results by dialogue stage}\label{turntaking_results_FTO_group_stage}

Although the turn-timing behaviours of the ASD and the CTR group were quite similar overall, some clear differences between groups are revealed when we do not only consider FTO results across the dialogue as a whole, but also compare early with later dialogue stages. Detection of the first Mismatch in the first Map Task is used as a cut-off point: all dialogue preceding detection is counted as being part of the beginning of the conversation, all dialogue following detection as the remainder of the conversation (more details in \sectref{materials}, \sectref{turntaking_results_FTO_stage_corroboration} and \sectref{turntaking_results_signal_turnation}).

Figure \ref{fig:FTOGroupStage} shows FTO values by group and dialogue stage. While autistic dyads performed turn-timing essentially equivalent to that of non-autistic dyads for most of the dialogue, they did not arrive at this timing instantly. In fact, during the first few minutes of dialogue, before the first Mismatch in the Map Task was detected (2 minutes or 10\% of overall duration into the task on average), FTO values for the ASD group were far higher (mean = 511 ms; SD = 799) than in the remainder of the dialogue (mean = 299 ms; SD = 576). These values indicate considerably longer silent gaps between ASD dyads early in the task. Dyads in the CTR group show only a slight change, and in the opposite direction, with shorter gaps (and slightly more overlaps) in the beginning of the dialogue (mean = 191 ms; SD = 540) compared to the remainder (mean = 243 ms; SD = 558). This interaction signifies that the turn-timing behaviour of the CTR and the ASD group differed considerably in the beginning of conversations (\(\delta\) = 320 ms), but not at later stages (\(\delta\) = 56 ms).

\begin{figure}

{\centering \includegraphics[width=1\linewidth]{figures/graphics-FTOGroupStage-1} 

}

\caption{FTO values by group and dialogue stage. The black curve represents the beginning of the dialogue (until detection of the first Mismatch); the orange curve represents the remainder of the dialogue (after detection of the first Mismatch). Positive values represent gaps; negative values represent overlaps. ASD group on the left, CTR group on the right.}\label{fig:FTOGroupStage}
\end{figure}

Figure \ref{fig:FTODyadStage} presents FTO values by dialogue stage and dyad, with CTR dyads in the top half of the plot and ASD dyads in the bottom half. We can see that for most (but not all) CTR dyads, FTO values were essentially the same for early and later stages of dialogue. For most (but not all) ASD dyads, on the other hand, there was a lot of variability in the early stages of dialogue, mostly (but not only) in the direction of longer gaps. This variability disappeared after the initial stages, as the dyads seemed to settle into a temporally stable turn-taking style that is virtually indistinguishable from that of CTR dyads.
\begin{figure}

{\centering \includegraphics[width=1\linewidth]{figures/graphics-FTODyadStage-1} 
	

\caption{Floor Transfer Offset (FTO) values by dialogue stage and dyad. Positive values represent gaps; negative values represent overlaps. ASD dyads in the top half and outlined in blue, CTR dyads in the bottom half and outlined in green. Black curves represent the beginning of dialogue (before detection of the first Mismatch); orange curves represent the remaining dialogue (after detection of the first Mismatch).}\label{fig:FTODyadStage}
}
\end{figure}

\subsubsection{Corroboration of dialogue stage effect}\label{turntaking_results_FTO_stage_corroboration}


For the analyses reported directly above, detection (i.e. first mention) rather than resolution of the first Mismatch (i.e. the time when interlocutors finished discussing the first Mismatch and moved on to the remainder of the task) was used as a cut-off point for the early stages of dialogue. There are two main reasons for this choice. First, a detailed analysis of all turn transitions directly following the introduction of matching vs. mismatching landmarks reveals that there was a consistent and distinct reaction to the detection of the first Mismatch in both groups (in the form of longer gaps). For details see \sectref{sec:turntaking_results_mismatches}; see also \citet{janzTurnTransitionsSpectrumConversational2019}. Essentially, the first Mismatch can thus be seen as a turning point in the interaction. Before detection of the first Mismatch, participants might feel that they are expected to give their individual contribution to the solution of a known problem (i.e. draw a path on an otherwise identical map). After the first Mismatch is detected, participants might feel that they need to give a joint contribution to navigate an unknown problem (knowing that the two maps are not identical), and this difference in the conversational goal can be expected to generate a difference in the interaction.

The second reason for using detection rather than resolution is that the former is less problematic as a timestamp from a practical perspective. The time it took to resolve the first Mismatch varied widely across dyads, ranging in duration from under 10 seconds to over 5 minutes. Moreover, even determining when a Mismatch was in fact resolved can be difficult and involves a degree of subjective judgement. In contrast, the detection of the first Mismatch was in almost all cases unambiguously expressed directly in the speech of both interlocutors.

To conclusively examine the appropriateness of using detection of the first Mismatch as the cut-off point, two further analyses were performed: 1) a further analysis taking into account the three-way distinction of a) dialogue from the start of the task to the detection of the first Mismatch, b) dialogue during the discussion and up to the resolution of the first Mismatch and c) all remaining (following) dialogue, and 2) a continuous analysis of FTO values in the first 100 turn transitions.

Briefly, the analysis with a three-way distinction of dialogue stages confirms that there was a robust between-group difference only before detection, not during and after the discussion of the first Mismatch (details of statistical modelling are reported in the following section).

Finally, Figure \ref{fig:FTOStageContinuous} shows that average FTO values in the ASD group tended to continuously decrease from the start of conversations until the point when the first Mismatch was detected, strengthening the validity of using mismatch detection as a cut-off point. Note that Figure \ref{fig:FTOStageContinuous} shows only the first 100 turn transitions; dialogues contained a total of 400 transitions on average.

\begin{figure}

{\centering \includegraphics[width=1\linewidth]{figures/Fig 8} 
	
}

\caption{FTO values by turn transition and group. Positive values represent gaps; negative values represent overlaps. ASD group in blue, CTR group in green. Thin blue/green lines represent averaged FTO values by transition and group; thick lines represent fitted LOESS-smoothed curves by group, the surrounding grey shaded areas the respective standard error. The dashed vertical lines show 1) transition no. 38 (average time point for detection of first Mismatch) and 2) transition no. 90 (average time point for resolution of first Mismatch).}\label{fig:FTOStageContinuous}
\end{figure}

We can conclude that differences in turn-timing were indeed greater between groups in the early stages of dialogue compared to the remainder, independent of the specific cut-off point.

\subsubsubsection*{Bayesian analysis}\label{turntaking_results_FTO_stage_bayesian}

Bayesian modelling confirms the above description in showing that there was a clear difference in FTO between groups early on in the dialogue, but not at later stages. More details on the interaction between group (CTR vs. ASD) and dialogue stage are given below.

Group differences are reported with the ASD group as the reference level and differences between dialogue stages are reported with the beginning of the dialogue as the reference level. First, a Gaussian model with FTO as the dependent variable, the interaction \textsc{group} (ASD/CTR)*\textsc{dialogue stage} (before/after detection of the first Mismatch) as a fixed factor and \textsc{dyad} as a random factor was used. For the comparison of FTO values between groups for only the beginning of the dialogue (i.e. all transitions up to detection of the first Mismatch), the Bayesian model shows a mean \(\delta\) of -322 (milliseconds) with a 95\% CI of {[}-462, -138{]} and a posterior probability \(P(\delta > 0)\) = 1. The model therefore provides unambiguous evidence for the observation that autistic dyads produced considerably longer silent gaps between turn transitions than non-autistic dyads in the early stages of dialogue. For the remainder of the dialogue, mean \(\delta\) is -45 (milliseconds) with a 95\% CI of {[}-150, 60{]} and a posterior probability \(P(\delta > 0)\) = 0.77. The low posterior probability and the 95\% CI including zero by a large margin strongly suggest that there was no difference between the turn-timing of autistic and non-autistic dyads in the later stages of dialogue.

In a three-way distinction of dialogue stages, we can then focus on turn transitions which take place during discussion of the first Mismatch. The relevant model (with the three-way distinction \textsc{before/during/after discussion of the first Mismatch}, otherwise identical to the model described directly above) shows that there is no robust group difference for this epoch, expressed through a mean \(\delta\) of -98 with a 95\% CI {[}-228, 31{]} and a posterior probability \(P(\delta > 0)\) = 0.9. While this indicates a clear trend towards shorter FTO values in non-autistic dyads (in line with the overall trend) during discussion of the first Mismatch, the inclusion of zero in the 95\% CI and the relatively low posterior probability suggest that this is not a reliable difference between groups.



\subsection{Categorical analysis of turn transitions}\label{turntaking_results_categorical}

For another perspective on turn-timing results, the continuous FTO results as presented above were divided into five different classes. Any gaps or overlaps with an absolute duration of less than 100 ms were categorised as \textit{smooth transitions}. Gaps or overlaps with an absolute duration of or exceeding 700 ms were categorised as \textit{long gaps/overlaps} and the remaining transitions with an absolute duration of 100--699 ms were categorised as \textit{short gaps/overlaps}.

The cut-off point at 700 ms was inferred from previous work showing that gaps of 700 ms or longer are perceived as unusual by listeners. This judgement seems to stand in a causal relationship with the corresponding listener expectation that long gaps of this kind will be followed by repair initiations or non-affiliating responses (such as negative answers to yes-no questions), an expectation borne out by production data \citep{kendrickIntersectionTurntakingRepair2015, kendrickTimingConstructionPreference2015a, robertsIdentifyingTemporalThreshold2013, schegloffPreferenceSelfcorrectionOrganization1977}.

In the following, I will discuss results from this categorical perspective in detail in those cases where it is informative beyond what we have already learned from considering FTO values in a continuous analysis (in \sectref {turntaking_results_FTO}).

Considering the dialogue as a whole, the most obvious finding remains that there is no clear difference between groups; see Figure \ref{fig:TurnsCategoricalGroupPic}. Both groups have very similar proportions of \textit{smooth} transitions (such with absolute FTO values under 100 ms; ASD: 17\%, CTR: 18.5\%). The most relevant finding may be that the ASD group produced a slightly higher proportion of unusually long gaps (\geq 700 ms; ASD: 17.8\%, CTR: 14.1\%). This difference is not very large, but as discussed above, listeners are very sensitive to unusually long transitions, and so pattern may nevertheless be perceived as noteworthy by conversational partners (or outside observers) and therefore contribute to overall subjective impressions of diverging conversational behaviour.

\begin{figure}

\includegraphics[width=1\linewidth]{images/Categorical_Transitions_5way_labelled} \hfill{}

\caption{Stacked bar charts by group showing proportions of different transition types. ASD group on top, CTR group below. Transition proportions on the x-axis: long overlap transitions (FTO ≤ -700 ms) in black, overlaps (FTO -699 ms -- -100 ms) in dark purple, very short (\textit{smooth}) transitions (FTO -99 -- 99 ms) in light purple, gaps (FTO 100 ms -- 699 ms) in orange and long gaps (FTO ≥ 700 ms) in yellow.}\label{fig:TurnsCategoricalGroupPic}
\end{figure}

A dyad-specific analysis shows that the group-level analysis accurately represents the behaviour of all dyads.
The finding that there was a trend towards more long gaps in the ASD group is supported by the observation that four out of the five dyads with the highest long-gap proportions were autistic dyads, all with at least 19\% long gaps -- although the dyad with the single highest long-gap proportion was a CTR dyad (M11\_M12, with 28.8\% long gaps). Conversely, autistic dyads produced the three lowest proportions of long overlaps. It is also interesting to note that four out of the five lowest smooth transition proportions were produced by autistic dyads. Figure \ref{fig:TurnsCategoricalDyad} in Appendix \ref{appendix:b} shows bar charts for all dyads.

Considering different stages of dialogue from a categorical perspective further corroborates results from the continuous FTO analysis: autistic dyads clearly differed in their turn-timing from control dyads only in the earliest stages of dialogue, after which they achieved a rhythm of turn exchanges equivalent to that of the CTR group. This can be visualised most vividly by only showing the proportions of long-gap transitions (\geq 700 ms FTO); see Figure \ref{fig:LongGapDotLine}. In the beginning of the dialogue (before detection of the first Mismatch), the proportion of long-gap transitions was more than twice as high for ASD dyads (29.1\%) compared to CTR dyads (11.9\%), but for the remaining dialogue there was practically no difference between groups (ASD: 16.8\%; CTR: 14.4\%). Not shown in Figure \ref{fig:LongGapDotLine} is the fact that the reduction of long-gap transitions in the ASD group over time is mirrored by an increase for the same group in the proportion of smooth transitions (such with absolute values \textless{} 100 ms), with an increase from 11\% to 18\%.

\begin{figure}

{\centering \includegraphics[width=0.7\linewidth]{figures/graphics-LongGapDotLine-1} 
	
}

\caption{Proportions of long gap transitions (FTO ≥ 700 ms) by group and dialogue stage. Early stage of dialogue (before detection of first Mismatch) on the left, later stages of dialogue on the right. ASD group in blue, CTR group in green. Note that the y-axis is truncated at 50\%.}\label{fig:LongGapDotLine}
\end{figure}

\subsection{Effects of unexpectedness: Matching and mismatching landmarks}\label{sec:turntaking_results_mismatches}

In the preceding sections, different parts of dialogue were analysed by comparing the earliest stages -- up to the first Mismatch -- with the remainder of conversations. In this section, we will take a closer look at a subset of the data in only considering transitions directly following the introduction of new landmarks. The main interest here lies in comparing the effects of mismatching vs.~matching landmarks. Mismatches are conceived of as a proxy for any unexpected events in social interaction, which many autistic speakers are said to struggle with.

To my knowledge, no other line of work has addressed the question of how matching vs.~mismatching landmarks in the Map Task paradigm affect turn-taking behaviour (in any group of subjects).

\subsubsection{Context, predictions and limitations}\label{turntaking_results_mismatches_background}


As participants in a Map Task cannot in any way be assumed to expect discrepancies between the two interlocutors' maps, mismatching landmarks should force participants to interrupt and discard their existing planning and adapt to unforeseen circumstances. Difficulties in dealing with situations involving change or unpredictable events are a typical clinical characteristic of ASD \citep[p. 50, Criterion B2]{americanpsychiatricassociationDiagnosticStatisticalManual2013}. Recent phenomenological research on the subjective experience of time has corroborated that some autistic individuals not only evince reduced flexibility in planning, but also report a fear of unexpected events and interruptions in pre-planned time \citep{vogelInterruptedTimeExperience2019}.

As Mismatches are unexpected and atypical events within the task-oriented dialogue elicited through Map Tasks, we can hypothesise that they will cue repair initiations or similar responses. As discussed above, such responses have been associated with preceding extended gap transitions between speakers (\geq 700 ms). Therefore, gaps are predicted to generally be longer for Mismatches than for Matches.

Regarding the difference between Matches and Mismatches across groups, two alternative predictions can be considered:

\begin{enumerate}
\def\labelenumi{\arabic{enumi})}
\item
It could be the case that turn-transitions differ more between Matches and Mismatches for the ASD as compared to the CTR group. This prediction follows the argument that individuals on the autism spectrum might be generally more sensitive to disturbances from unexpected events and might therefore also be affected more strongly by such events in the form of mismatching landmarks.
\item
Alternatively, it could be the case that turn-transitions differ less between Matches and Mismatches for the ASD group compared to the control group. This prediction follows the argument that all new information, even in the form of expectable, matching landmarks, might be experienced as a kind of interruption for individuals on the autism spectrum, thus potentially levelling the playing field in the sense that Mismatches do not constitute a marked departure from difficulties that are already routinely experienced in conversation.
\end{enumerate}

Although we are particularly interested in the effects of Mismatches as compared to Matches, the very nature of the Map Task means that there will always be considerably less data available for Mismatches. Not only is the effect of unexpectedness greatly reduced with each subsequent occurrence of a Mismatch, as will be seen, but the task would also be increasingly difficult to complete with the addition of more Mismatches -- and some dyads struggle to complete the task with only the classic set-up involving two Mismatches. Therefore, an increase in mismatching landmarks would make breakdowns in conversation far more likely, yielding conversational data less representative of natural spoken interaction.

\hspace*{-1.5pt}For these reasons, it is not feasible to analyse equal amounts of data for Matches and Mismatches. This might partly explain why there is no previous work investigating the effects of Mismatches on dialogue management in Map Tasks. As such, it is worth keeping in mind that this analysis might best be conceptualised as a qualitative case study.
For the same reasons, i.e.~paucity of (comparable) data, I will limit myself in this section to a purely descriptive account and forego Bayesian analysis as performed in other sections. Effect size estimation using Cohen's \(d\) \citep{cohenStatisticalPowerAnalysis1988} will be reported for the main results in order to give a clearer idea of differences between groups and types of landmark.

\subsubsection{Data}\label{turntaking_results_mismatches_data}


All turn transitions following utterances in which a new landmark was introduced entered into analysis, except in the rare cases where more than one landmark was introduced within the same interpausal unit. In such cases, only the landmark that was mentioned last, at the end of the respective utterance, was included.

As this analysis is focussed on effects of unexpectedness, I will concentrate mainly on the first out of the two Map Tasks that each dyad completed. It will in fact be shown that effects of unexpectedness are already drastically diminished once the very first Mismatch (on the first set of maps) has been introduced.

The total number of turn transitions produced following the introduction of landmarks was 166 (123 after Matches, 43 after Mismatches). Note again the limited amount of data and, as a result, the inescapably qualitative character of this part of the analysis.



\subsubsection{Results}\label{turntaking_results_mismatches_results}



I will first present a continuous and categorical analysis of turn-timing and then specifically consider 1) differences between the first and subsequent mismatching landmarks in the task as well as 2) differences between the first and the second Map Task.

\subsubsubsection*{Continuous analysis}\label{turntaking_results_mismatches_results_continuous}

A continuous FTO analysis shows that the CTR and the ASD group produced very similar turn-timing following Mismatches, in the form of long gaps around 700 ms. Note that this is a considerably higher mean FTO value than we have seen in the overall results (\sectref{turntaking_results_FTO_group}). Following Matches, however, the groups showed divergent behaviour, with the ASD group producing longer FTO values overall than the CTR group.

CTR speakers evinced turn-timing following Matches representative of typical values, as documented in the preceding sections and in previous studies. The mean FTO value for turn transitions after introduction of Matches in the CTR group was 101 ms (SD = 490). The respective density curve for the CTR data (grey curve in the right panel of Figure \ref{fig:MismatchFTO}) shows a leptokurtic distribution with few extreme values and slightly more gaps than overlaps in the overall distribution.

ASD speakers on the other hand produced an unusually high mean FTO value of 433 ms (SD = 863) in transitions following the introduction of Matches. Although the median FTO value and the overall shape are similar to that of the CTR group, we can see a platykurtic distribution skewed towards the right, indicating more and longer gaps, which account for the difference in mean values (grey curve in the left panel of Figure \ref{fig:MismatchFTO}). A comparison of FTO values following Matches between groups using Cohen's \(d\) reveals an effect size of 0.51 (medium effect size) \citep{cohenStatisticalPowerAnalysis1988,sawilowskyNewEffectSize2009}.



\begin{figure}

{\centering \includegraphics[width=1\linewidth]{figures/graphics-MismatchFTO-1} 
	
}

\caption{Density plot of FTO values in milliseconds, by group and Match/Mismatch. Negative values represent overlaps; positive values represent gaps. ASD group on the left, control group on the right. FTO values on the x-axis, density on the y-axis. FTO values for matching landmarks are represented in grey; FTO values for mismatching landmarks are presented in yellow.}\label{fig:MismatchFTO}
\end{figure}

Following the introduction of Mismatches, the ASD and the CTR group behaved in a strikingly similar way, producing median FTO values of approximately 700 ms (ASD: 745ms; CTR: 724ms) and means of around 900 ms (ASD: 920 ms (SD = 805); CTR: 857 ms (SD = 744); see yellow density curves in Figure \ref{fig:MismatchFTO}). These FTO values represent long gaps, such as typically occur in situations involving repair initiations due to misunderstanding or disagreement.

Therefore, differences in turn transitions between the two groups were negligible where mismatching landmarks occurred. A comparison of FTO values following Mismatches between groups using Cohen's \(d\) reveals an effect size of 0.08 (negligible effect size).

Within groups, FTO values differed more between Matches and Mismatches for the CTR group (Cohen's \(d\) = 1.36; large effect size) than for the ASD group (Cohen's \(d\) = 0.58; medium effect size), as represented by the degree of overlap between the grey and yellow curves in Figure \ref{fig:MismatchFTO}. In other words, there was a difference in turn-timing following Matches and Mismatches for both groups, but this difference was less pronounced in the ASD compared to the CTR group.

Analyses at the dyad-level largely confirm these patterns, although there was a very high degree of variability in the ASD group.

\subsubsubsection*{Categorical analysis}\label{turntaking_results_mismatches_results_categorical}

A categorical perspective can be particularly useful for the specific case of matching vs.~mismatching landmarks. Here, the same categorisation of transition types as in \sectref{turntaking_results_categorical} is used -- dividing into long overlaps, overlaps, smooth transitions, gaps and long gaps -- but, importantly, a \textit{No Response} category is added.

As the name suggests, this category is used whenever a new landmark was introduced in the Map Task by the instruction giver, but not verbally acknowledged by the instruction follower. In other words, no floor transfer took place (and no response token was produced). These cases are treated (and visualised) as being conceptually adjacent to (very) long gaps. In essence, silence in such cases was simply maintained by the instruction follower for such a long time (1711 ms on average in the current data set) that the instruction giver eventually felt entitled (or obliged) to self-select for the next turn, thereby precluding any kind of turn transition between speakers. The expected silent gap between speakers following a turn-relevance place was in these cases effectively replaced with a period of intra-speaker silence \citep[cf.][]{sacksSimplestSystematicsOrganization1978}.

Although the following results are presented as proportions of all transitions, it is important to keep in mind the very limited sample size for this subset of the data (absolute numbers are reported to add a sense of scale and context).




\begin{figure}

{\centering \includegraphics[width=1\linewidth]{figures/graphics-MismatchCategorical-1} 
	
}

\caption{Stacked bar charts by group showing proportions of different types of turn transition. Values for matching landmarks in panel A (top), values for mismatching landmarks in panel B (bottom). Both panels contain two bars, with proportions for ASD participants on top and proportions for CTR participants below.
	Transition proportions on the x-axis: long overlap transitions (FTO ≤ -700 ms) in black, overlaps (FTO -699 ms -- -100 ms) in dark purple, very short (\textit{smooth}) transitions (FTO -99 -- 99ms) in light purple, gaps (FTO 100 ms -- 699 ms) in red, long gaps (FTO ≥ 700 ms) in orange and non-responses (no verbal reaction to mention of landmark) in beige.\\}\label{fig:MismatchCategorical}
\end{figure}

As can be seen in Figure \ref{fig:MismatchCategorical}, No Response cases (represented in beige) were less common in the CTR group. In fact, there is only one such instance following the introduction of matching landmarks (1.4\%), and none at all following the introduction of mismatching landmarks.
In the ASD group, fewer instances of newly introduced landmarks were verbally acknowledged. There were six cases of non-response following Matches (9.8\%) and, strikingly, two instances following Mismatches (8.7\%) as well.

Concerning the remaining transition types, we can see that following Matches (Panel A in Figure \ref{fig:MismatchCategorical}), ASD dyads produced more long gaps (32.8\%; n = 20) than CTR dyads (7.2\%; n = 5). Conversely, ASD dyads produced fewer overlaps and smooth (very short) transitions than control dyads. These differences were much less evident following Mismatches (Panel B in Figure \ref{fig:MismatchCategorical}), with both groups producing a similarly large proportion of long gaps, very few overlaps and no smooth (very short) transitions whatsoever. 

\subsubsubsection*{Comparison of first vs. second Mismatch and Map Task}\label{turntaking_results_mismatches_results_order}

To further test the assumption that unexpectedness played a role in these results, we can compare transition times following utterances introducing the first vs.~the second Mismatch within a map as well as in the first vs.~in the second Map Task within a dialogue.

First, transitions following the first and the second Mismatch within the first Map Task will be compared. Although the distributions for both the ASD and the CTR group peak at around 700 ms following mentions of either the first or the second Mismatch, the distributions for the first Mismatch are more variable and skewed considerably towards longer gaps (reflected in an across-groups mean value of 1228 ms; values were nearly identical across groups) compared with the second Mismatch (mean = 887 ms), as shown in Figure \ref{fig:MismatchFirst}. This analysis also makes it clear that there were virtually no overlapping transitions after the introduction of Mismatches (in Map 1). These results seems to confirm the assumption that effects of unexpectedness will be diminished with the introduction of subsequent Mismatches.



\begin{figure}

{\centering \includegraphics[width=1\linewidth]{figures/graphics-MismatchFirst-1} 
	
}

\caption{Density plots of FTO values by group and order of Mismatch. First Mismatch in the dialogue in black, second Mismatch in orange. ASD group on the left, CTR group on the right.}\label{fig:MismatchFirst}
\end{figure}

Zooming out and considering differences between the first and the second Map Task within a recorded dialogue, we can see that, following completion of the first task, the effects associated with unexpectedness diminished. In other words, transitions following Mismatches had shorter FTO values in Map 2 compared to Map 1 overall, with some interesting group differences.

For the ASD group, FTO values in Map 2 were similar following Matches (mean = 350 ms; SD = 586) and Mismatches (mean = 480 ms; SD = 630). The effect size of this difference is small (Cohen's \(d\) = 0.21), smaller still than for the same comparison in Map 1 (\(d\) = 0.58; see \sectref{turntaking_results_mismatches_results_continuous}).

For the CTR group, there was a greater difference than in the ASD group between FTO values following Matches (mean = 273 ms; SD = 304) and Mismatches (mean = 622 ms; SD = 605) in Map 2 (as for Map 1). The effect size of this difference is medium (Cohen's \(d\) = 0.74). This difference is greater than for the same comparison within the ASD group, but still far smaller than for the same comparison within the CTR group in Map 1 (\(d\) = 1.36; see \sectref{turntaking_results_mismatches_results_continuous}).

In sum, turn transitions following Mismatches were shorter in Map 2 compared to Map 1, likely reflecting a diminished effect of unexpectedness and a concordant decrease in misunderstandings and the need for repair.

In the next section, we will expand our focus once again and consider the entire data set in concentrating on aspects beyond turn transitions.

\subsection{Beyond transitions: Within-overlaps, signal analysis and speaking times}\label{turntaking_results_signal}

In this section, I will first present an analysis of within-speaker overlaps (where no floor transfer to another speaker takes place), including the presence or absence of backchannels in overlap. Second, overall proportions of silence (i.e. with\-in-speaker pauses combined with between-speaker gaps) compared to overlapping and single-speaker speech will be considered, including an analysis of the distribution of overall speaking times within dyads.


\subsubsection{Within-speaker overlaps}\label{turntaking_results_signal_within}


Within-overlaps are cases where a portion of overlapping speech is not followed by the floor being transferred to another speaker (in direct contrast to between-overlaps, as laid out in \sectref{sec:turntaking_analysis} and Figure \ref{fig:Transitions}). I will briefly characterise the nature and distribution of within-overlaps across groups in this section. In short, and similarly to overall turn-timing behaviour, results were comparable for autistic and non-autistic dyads.

Within-overlaps were typically very short, with an almost identical average duration across groups. The mean within-overlap duration for the ASD group was 380 ms (SD = 290) and the median was 314 ms. The mean within-overlap duration for the control group was 382 ms (SD = 279) and the median was 300 ms.

Not coincidentally, these durations almost exactly equal the mean duration of backchannels in the data set (378 ms; SD = 158), as 71.6\% of ASD within-overlaps and 70\% of CTR within-overlaps contained backchannels (and often consisted solely of a single backchannel). Overlaps containing backchannels can be considered \textit{principled} or \textit{sanctioned}. In other words, such overlaps don't constitute true interruptions, as backchannels are listener signals encouraging the interlocutor to hold the floor, rather than being the start of a competing turn by the interlocutor (see \chapref{backchannels} for an in-depth analysis of backchannelling behaviour). Conversely, then, 28.4\% of ASD within-overlaps and 30\% of CTR within-overlaps can be considered true interruptions, of a kind that was not resolved by a floor transfer.

Although the group-level pattern reflects the behaviour of individual dyads accurately overall, it is interesting to note that the three dyads with the largest proportions of unprincipled overlaps (i.e. not containing backchannels) were dyads from the ASD group (\geq 60\% unprincipled overlaps in each case).

Around 50\% of \emph{between-overlaps}, which were part of the FTO analyses in the pre\-ced\-ing sections, also contained, or consisted solely of, one or multiple back\-chan\-nels (ASD: 53.1\%; CTR: 52.1\%). It was ascertained that excluding these back\-chan\-nel-containing overlaps does not change the results from the FTO analysis in any meaningful way, and therefore overlaps containing backchannels were included in the analysis of turn transitions reported above.


\subsubsection{Overall signal: Silence, overlap and single-speaker speech}\label{turntaking_results_signal_signal}

In the following, all IPUs and the silent spaces between them are considered (and not only turn transitions, as in the preceding sections).

Both groups of speakers produced virtually identical proportions of silence, overlaps and single-speaker speech. In both cases, almost three quarters of dialogue were taken up by speech from a single speaker (ASD: 72.5\%; CTR: 73.1\%), almost one quarter consisted of silence within or between speakers (ASD: 24\%; CTR: 22.2\%) and the small remainder was made up of overlapping speech from both interlocutors (ASD: 3.5\%; CTR: 4.7\%).

These results are remarkable for their great consistency not only across groups, but also across dyads (as shown in Figure \ref{fig:SignalDyad} in Appendix \ref{appendix:b}). This finding adds to the extensive evidence in favour of the assertion made (prior to availability of any extensive quantitative evidence) in \citet{sacksSimplestSystematicsOrganization1978} that ``{[}i{]}t has become obvious that, overwhelmingly, one party talks at a time'' (p.~699).

Considering the overall amount of speech material, we can observe that speakers from the CTR group produced almost exactly twice as many IPUs (12121) as those from the ASD group (6211). This is mainly, but not entirely, due to the fact that ASD dyads were on average considerably quicker to complete the Map Tasks (mean time to completion: 14 minutes 40 seconds) than CTR dyads (mean: 26 minutes). Dialogue durations ranged from 9 minutes (ASD dyad M04\_M05) to 49 minutes (CTR dyad M09\_M10). Dyads from the ASD group accounted for the four shortest dialogues, while dyads from the CTR group accounted for four out of the five longest dialogues.

Although average dialogue durations were subject to a high degree of by-dyad variability in both groups, Bayesian modelling confirms a robust difference between groups. Linear regression with a log-normal distribution was used for the measure of overall dialogue duration in seconds (\(\delta\) = 566; 95\% CI {[}51, 1149{]}; \(P(\delta > 0)\) = 0.97; details in the accompanying files).

The other main reason why there are more IPUs in the CTR group (in total and per minute of dialogue) is that in CTR dialogues, dyads produced more turn transitions. In other words, dialogues between non-autistic individuals entailed more frequent switches between the roles of listener and speaker (i.e.~floor transfers per minute of dialogue).

Correspondingly, CTR dyads produced shorter utterances overall. IPUs produced by one speaker at a time (as opposed to speech from both interlocutors at once) had a mean duration of 1318 ms (SD = 1277) in the CTR group and 1424 ms (SD = 1369) in the ASD group (note the high degree of variability). Furthermore, portions of overlapping speech tended to be longer in the CTR group, but silences tended to be longer in the ASD group. Effect sizes are very small in both cases, however, and are unlikely to indicate any truly meaningful or generalisable differences.

Taking speaker roles (instruction giver/follower) into account allows us to observe that, rather unsurprisingly, instruction givers had about twice as much speaking time (52.1\%) as instruction followers (24.1\%) on average. This pattern is consistent across groups as well as dyads and can, to some extent, be explained by the fact that IPUs were longer for instruction givers (mean = 1526 ms; SD = 1366) than for instruction followers (mean = 1089 ms; SD = 1170).




\subsubsection{Speaking time within dyads}\label{turntaking_results_signal_dominance}


In this section, I will introduce a measure for relative speaking times within dyads. This score simply indicates whether and to what extent one speaker with\-in a dyad spoke more than the other. For instance, if 70\% of a dialogue consists of single-speaker speech, in a perfectly balanced dyad speech from each interlocutor would account for one half of that (or 35\% of the total), resulting in a score of 0. In other words, in such a perfectly balanced dyad, speech from each of the two participants would have the same overall duration. The lower the score, the more balanced the contributions from the two interlocutors. That is, if one person spoke more than the other, say with speech by Speaker A taking up 30\% of overall dialogue duration and speech by Speaker B taking up 40\% of overall duration, the resulting score would be 10 (40 - 30).

As can be seen in Figure \ref{fig:Dominance}, there was a clear tendency for autistic dyads to be less well-balanced in terms of speaking time. In the CTR group, four out of seven CTR dyads had almost perfectly equal speaking times (less than 1\% difference in overall speaking time for the lowest three), whereas in the ASD group, only one out of seven dyads had a score of less than 5. In line with the other analyses presented in this work, this pattern does, however, not signify that there was a clear line that could be drawn between the behaviour of the two groups, due to pervasive dyad-specific effects. For instance, one ASD dyad had very balanced speaking times and one CTR dyad was one of the least balanced overall. The average score across groups was 10 for the ASD group and 5 for the CTR group.


\begin{figure}

{\centering \includegraphics[width=1\linewidth]{figures/dominance_new-1.png} 
	
}

\caption{Speaking times within dyads. The lower the score, the more balanced the speaking times of the interlocutors within one dyad. ASD group in blue, CTR group in green.}\label{fig:Dominance}
\end{figure}

As mentioned above, instruction givers tended to have twice as much speaking time as instruction followers. Combined with the fact that, for many dyads, Map Task 2 took less time to complete than Map Task 1, it follows that one speaker was instruction giver for longer than the other in such dyads. A detailed qualitative investigation confirms that this played no important role in accounting for overall speaking times, however, as dialogue in the later stages of longer Map Tasks consisted of relatively balanced input from both speakers in most cases.




\subsubsection{Overview plots}\label{turntaking_results_signal_turnation}

Finally, I will present visualisations of the conversational data underlying all the different findings discussed in this book, such as turn-timing and speaker contributions, but also length of task and further aspects discussed in later chapters such as backchannels and filled pauses, in one single plot per dyad.

A \emph{Praat} script was used to separately plot all IPUs for both speakers within each dyad, with special annotations for backchannels and filled pauses and highlighting time frames for detection, discussion and resolution of the first Mismatch. The script was adapted from the method used and discussed in \citet{sbrannaQuantifyingL2Interactional2021}, which in turn has many commonalities with the visualisation techniques used in \citet{trouvainExploringSequencesSpeech2013,campbellApproachesConversationalSpeech2007}; see \citet{cangemiContentfreeSpeechActivity2023} for a further application and extension of this approach in the context of schizophrenia.

I will present and briefly describe two example plots below. The rest of the plots can be found in the folder ``turnation" of the accompanying repository at \url{https://osf.io/6vynj/} (for in-depth inspection, it is ideal to view these plots on screen as individual files).

Plots can be read like the written page in the Western tradition, i.e.~from left to right and top to bottom. Each horizontal line represents one minute. Interpausal units are represented as red and blue lines (one colour per speaker). Backchannels are marked with lighter colours (pink and cyan, respectively). Filled pauses are marked in grey. The section of the dialogue from detection to resolution of the first Mismatch is marked with a green dotted line. The white space in the middle of each dialogue shows the time between completion of Map Task 1 and start of Map Task 2 (these data did not enter into analysis).

It is rather straightforward in this depiction to identify speaker roles, types of turn transition, frequency and timing of backchannels and filled pauses as well as overall task duration and different stages of dialogue. Most quantitative results can be directly related to the overview plots in this way.

Our first example is dyad F23\_M22, from the control group. The overview plot is shown in Figure \ref{fig:TurnationF23M22}.
This dyad was chosen as it is representative of the average behaviour in the CTR group in many regards. Map Task 1 and Map Task 2 were completed within 16 minutes, with about 8 minutes spent on each task -- shorter than average for the CTR group. In both cases, the instruction giver (red in Map 1; blue in Map 2) has far more speaking time than the follower. Overall, however, speaking times were almost perfectly balanced, with a difference score of only 0.2 (percent proportional to overall dialogue duration).
The first Mismatch was detected quickly and resolved in a relatively short amount of time.
There were slightly more floor transfers per minute than average, resulting also in a slightly shorter average IPU length. The mean FTO for turn-timing was 220 ms, typical for this group of speakers as well as for results from previous studies. The rates of backchannels (pink/cyan), filled pauses (grey) and silent pauses (white spaces within turns) produced were very close to the group average.



\begin{figure}

{\centering \includegraphics[width=1\linewidth]{images/F23_M22_IPU_1O2S_turn_MISMATCH_1_48_to_3_11} 
	
}

\caption{Overview plot for dyad F23\_M22 from the CTR group. Speaker F23 in blue, speaker M22 in red. Backchannels in lighter colours (pink/cyan), filled pauses in grey. The section of dialogue from detection to full resolution of the first Mismatch is outlined in green.}\label{fig:TurnationF23M22}
\end{figure}

	
	Our second example is dyad M07\_M08, from the ASD group. The overview plot is shown in Figure \ref{fig:TurnationM07M08}.
	This dyad was chosen as it represents relatively unusual behaviour along several dimensions. Here, overall task duration was very short, with a total of only 10 minutes. Speaker M07 (blue lines) has considerably more speaking time than speaker M08 overall, taking many turns in the roles of both instructor (Map 1) and follower (Map 2). Overall, M07\_M08 was the second least well-balanced dyad in terms of speaking time within a dyad, with a score of 17. The dyad produced a relatively low number of floor transfers per minute, resulting in an unusually high mean IPU duration.
	The first Mismatch was detected quickly, but took a relatively long time to be resolved.
	Although FTO was close to the group average, the dyad produced an unusually high proportion of long gaps (\ge 700 ms).
	Dyad M07\_M08 produced by far the lowest rate of backchannels per minute (pink/cyan) and a relatively high rate of both filled pauses (grey) and silent pauses (white spaces within turns).
	
	
	
	\begin{figure}
	
	{\centering \includegraphics[width=1\linewidth]{images/M07_M08_IPU_1O2S_turn_MISMATCH_1_37_to_4_01} 
		
	}
	
	\caption{Overview plot for dyad M07\_M08 from the ASD group. Speaker M07 in blue, speaker M08 in red. Backchannels in lighter colours (pink/cyan), filled pauses in grey. The section of dialogue from detection to full resolution of the first Mismatch is outlined in green.}\label{fig:TurnationM07M08}
	\end{figure}
	
	I will refer back to some of the overview plots in the general discussion (\chapref{conclusion}), where findings from the different parts of this book are tied together in order to reveal general characteristics of conversation and intonation for each ASD dyad.


\subsection{Prosodic realisation}\label{turntaking_results_intonation}

Although the analysis of turn-timing in this book does not specifically focus on prosodic aspects, I did examine whether speakers used intonational cues to turn-ends (and beginnings) in ways that are comparable between groups and to what has been reported in previous studies. In this section, I will give a brief insight into methods and results, reserving a full account for future publications.

% \hspace*{-2.3pt}Intonational aspects of turn-timing were explored in the framework of prosodic constructions \citep{niebuhrSteppedIntonationContours2015, wardProsodicPatternsEnglish2019}, using the methodology introduced in \citet{wardNonnativeDifferencesProsodicconstruction2017} and fleshed out in \citet{wardProsodicPatternsEnglish2019}. Essentially, principal component analysis (PCA) is used to detect the most common prosodic constructions in a given data set. Prosodic constructions are made up of bundles of features along sets of parameters such as pitch height, pitch range, intensity, rhythm, and voice quality. Any frequently occurring combination of features is referred to as a dimension, and specific configurations of parameters are referred to as the loadings of that dimension.

Examples for commonly found prosodic constructions are downstep, late pitch peak, or, more closely related to the current investigation, turn-switch and back\-chan\-nelling. Dimensions are automatically extracted from a given data set along with summary plots and a list of time-stamped examples of the relevant construction. Identifying the significance and function of the dimensions provided by the automatic PCA procedure is left to the researcher. However, by inspecting the summary plots and relating them to the list of examples in the data set, as well as to previous research using the method, it is very straightforward to identify the most common constructions.

The most relevant construction for the investigation of turn-timing is what \citet{wardProsodicPatternsEnglish2019} calls the ``Basic Turn-Switch Construction" \citep[also discussed in][]{wardNonnativeDifferencesProsodicconstruction2017}. \citet{wardProsodicPatternsEnglish2019} describes this construction as follows:

\begin{quote}
About a second before ending a turn, the speaker typically produces a bundle of characteristic features, including higher pitch, narrower pitch range, and lengthening. This is followed by a region of lower pitch and increased creakiness, and then a half second later the turn end\ldots.the prosody afterwards, as the new speaker takes the turn, is also significant. As the loadings suggest, this commonly is high in pitch, and also loud, reduced and creaky (pp.~142--145).
\end{quote}

The results of PCA applied to the current data set reveal that both groups of speakers marked turn ends and beginnings in a way that is highly compatible with the above description. More importantly, there was no obvious difference between groups. None of this should come as too much of a surprise, given that 1) we have seen that all dyads in the data set under study produced typically rapid turn-timing for the vast majority of the dialogue and 2) experimental evidence suggests that such split-second precision in turn-timing is only possible when all relevant linguistic cues, including prosodic cues, are present in the signal \citep{barthelNextSpeakersPlan2017,barthelTimingUtterancePlanning2016,bogelsBrainResponseInsights2017,bogelsListenersUseIntonational2015,torreiraVocalReactionTimes2022}.

Figure \ref{fig:TurnConstruction} shows the loadings of the relevant dimension in the PCA analysis for both the CTR and the ASD group. These plots closely resemble the example given in \citet[p. 143]{wardProsodicPatternsEnglish2019}. This confirms that the present findings, for both groups, are compatible with those of previous studies. Specifically, turn-endings in the data set under study tended to be marked by falling pitch, lengthening and creakiness, while turn-beginnings tended to be marked by high pitch, high intensity and creakiness.

The plots are read such that the top half of each graph represents one speaker in a dialogue and the bottom half represents their interlocutor. Time flows from left to right on the x-axis and plots are centred at the core of the prosodic construction at 0 milliseconds -- in this case the precise moment of turn transition from one speaker to the next. For each parameter, as the relevant curve goes up beyond the central line, values are higher than average and as it goes down, values are lower than average. If we focus, for instance, on the line representing intensity, we can see that intensity drops (to silence) at the zero-millisecond mark for the speaker on top, while it rises (from silence) for the other speaker (in both the CTR and the ASD group).



\begin{figure}

{\centering \includegraphics[width=.95\linewidth]{images/turn_switch_CTR_ASD}
	
}

\caption{Loading plots for the Basic Turn-Switch Construction as revealed by principal component analysis. CTR group on top, ASD group on the bottom. See text for details.}\label{fig:TurnConstruction}
\end{figure}

These results clearly indicate that prosodic aspects of turn-taking were equivalent across groups and that the intonational marking of turn ends and beginnings conforms to what has been reported in previous studies. One shortcoming of the PCA approach is that it is not very well suited to in-depth analysis at the dyad level, or of shorter time windows (e.g.~for an analysis of the earliest stages of dialogue only). Not enough data are available in such cases to guarantee a robust analysis with reliable extraction of all relevant dimensions. Thus, it remains for future studies to investigate prosodic aspects of dyad-specific behaviour as well as differences between different stages of dialogue. From a more general perspective, the approach demonstrated here might be used to narrow the still considerable gap between quantitative and qualitative traditions in research on turn-taking (and conversation analysis more generally). Such an approach is already inherent in the methodology designed and described by \citet{wardProsodicPatternsEnglish2019}, as the decidedly quantitative computational black box that is PCA is here employed to produce a list of timestamped examples of conversational behaviour which are then to be examined and described qualitatively and in great detail. In this way, the sheer analytical power of quantitative approaches can be harnessed in order to facilitate the detection of examples best suited to an in-depth qualitative analysis. 






\section{Discussion}\label{sec:turntaking_conclusion}

To conclude this chapter, I will summarise the relevant findings, discuss their implications, situate them in the landscape of previous research, point out the limitations of the study and sketch some directions for future work.

\subsection{Summary}\label{turntaking_conclusion_summary}

An in-depth analysis of turn-timing in German-speaking adults with and without a diagnosis of autism spectrum disorder was presented. This is the first study of turn-timing in conversations between autistic adults and one of the first in-depth studies of turn-timing in German. It was found that autistic dyads behaved similarly to non-autistic dyads in many respects. For example, there was no reliable group-difference for overall FTO values (representing the timing of turns). A closer look at different stages of dialogue revealed that autistic dyads did in fact behave differently from control dyads, but only in the earliest stages of dialogue, where they produced more long gaps.

Another difference between groups was found in the realisation of turn transitions directly following the introduction of new landmarks. Both groups reacted to mismatching landmarks early in the task by producing many long gaps, reflecting the fact that such unexpected events are almost bound to lead to misunderstanding and repair. However, only the ASD group produced a similarly high proportion of long gaps following the introduction of \emph{matching} landmarks.

It is important to point out that turn-timing behaviour, as all other aspects of intonation and dialogue management discussed in this book, revealed no clear dividing line between the ASD group and the CTR group. Dyad-specific analyses have shown that around half of the autistic dyads showed behaviour within the range of CTR dyads for most of the dimensions investigated.

It was further shown that, for both groups, overlaps within speaker turns were typically very short, often consisting only of a single backchannel token, and that half of all overlaps contained backchannels. The speaking times between interlocutors within dyads tended to be less balanced in the ASD group. Finally, no group difference in the prosodic realisation of turn-ends and turn-beginnings was found.

\subsection{Implications and interpretation}\label{turntaking_conclusion_discussion}

In the following, I will discuss the implications of some of the results covered in this chapter and place them in the context of previous research. I will first cover overall turn-timing and then the subset of transitions following the introduction of new landmarks, and conclude with a brief comparison of results on turn-taking in ASD with results on turn-taking in second language speech.


\subsubsection{Turn-timing: Longer gaps in the early stages of conversation}\label{turntaking_conclusion_discussion_gaps}

The finding that differences in turn-taking between groups, in the form of longer gaps in the conversations of autistic dyads, were found only in the earliest stages of dialogue shows that autistic speaker pairs successfully established a degree of rapid turn-timing that is essentially indistinguishable from that of non-autistic dyads, but that they did not do so instantly \citep[cf.][]{levitanEntrainmentTurnTakingHumanHuman2015}. Arriving at such equivalent turn-timing behaviour appears to be literally a matter of time for dyads in the ASD group, as it seems to be independent of conversational content (here, progress in the Map Task or, more specifically, encountering and discussing the first Mismatch).

Given that listeners are very sensitive to even small differences in turn-timing \citep{kendrickTimingConstructionPreference2015a} and form personality impressions about speakers extremely rapidly \citep{mcaleerHowYouSay2014}, the overall turn-taking style of the ASD group may still be perceived as odd or unusual, at least by typically developed listeners. This holds true even though there was no robust difference between autistic and non-autistic dyads for most of the dialogue -- precisely because the relevant differences are found during the earliest stages of conversation.

These specific differences should, however, not overshadow the general finding that, at a global level, no robust differences were found in conversations between autistic as opposed to non-autistic adults. This might be considered surprising given that 1) it has been shown at length in previous work that achieving rapid and precise turn-timing is highly challenging cognitively, as it can only be achieved if speakers are able to accurately predict the communicative intentions of their interlocutor \citep{bogelsListenersUseIntonational2015,deruiterProjectingEndSpeaker2006,gleitmanGiveTakeEvent2007,wesselingEarlyPreparationExperimentally2005,barthelTimingUtterancePlanning2016}, and 2) predicting the behaviour of others is a skill that many autistic individuals seem to struggle with \citep{cannonPredictionAutismSpectrum2021}.

The current findings clearly show that at least the kinds of relatively socially motivated and skilled autistic adults investigated in this study, and at least when conversing in disposition-matched (ASD--ASD) dyads, are perfectly able to produce turn-timing of the same speed and precision as has been described for conversations between adults without a diagnosis of ASD. The related observations that, compared to the CTR group, speakers in the ASD group did not use more filled pauses (see \sectref{sec:BCFP_FP}) and produced turn-ends and beginnings with the same intonational realisation as the CTR group (see \sectref{turntaking_results_intonation}) furthermore discourage alternative explanations for equivalent turn-timing across the two groups (e.g. that although the utterances of autistic dyads were produced with the same timing, they may have differed in terms of informativeness or prosodic detail). An alternative or complementary theory would be that factors such as perspective-taking or Theory of Mind simply play less of a role in turn-taking \citep[and perhaps even ASD in general; see][]{williamsTheoryAutisticMind2021} than has previously been assumed.

The results presented here extend the numerous findings on the apparent universality of turn-timing for the first time to conversations between autistic adults. On the one hand, this strengthens the notion that turn-taking is a fundamental aspect of human interaction, and one that is apparently very similar across groups of speakers with different cognitive, cultural and linguistic backgrounds. On the other hand, the subtle differences between the CTR group and the ASD group detected by taking into account temporal dynamics suggest that similarly subtle differences between other groups of speakers may yet to be discovered. It is possible that a focus on the undeniably remarkable similarities of turn-timing across populations and contexts has overshadowed subtle differences at smaller scales, which might only be discovered with the use of more fine-grained qualitative and quantitative approaches.

\subsubsection{Transitions following expected vs. unexpected information}\label{turntaking_conclusion_discussion_mismatches}


An analysis of only the turn transitions following the introduction of new landmarks revealed that both groups produced long gaps of around 700 milliseconds following the introduction of mismatching landmarks. This is a value typical of situations involving misunderstanding, non-affiliating answers or repair initiations, which featured prominently in almost all interactions following the introduction of at least the first landmark to unexpectedly differ between maps.

A difference between groups was found only for transitions following the introduction of \emph{matching} landmarks. In these cases, no effects of misunderstanding or surprise are expected and indeed non-autistic dyads produced transitions with typical short gaps. Autistic dyads, on the other hand, produced longer gaps, meaning that in this group only, gaps were relatively long following the introduction of both \emph{matching and mismatching} landmarks. Additionally, it was shown that ASD speakers in certain cases also provided no verbal reaction to the introduction of new landmarks, even for Mismatches.

It is highly unusual not to explicitly acknowledge new -- albeit expected -- information (see the 1.5\% non-response rate following Matches for the control group). However, such verbal acknowledgement is not strictly necessary from a functional perspective. Cases such as the 10.9\% of non-responses following Matches for the ASD group do not directly prevent participants from being able to complete the Map Task by transferring the given route from one map to the other. This is not true, however, in the case of Mismatches. Here, it would seem necessary to directly address the issue at hand and to initiate a repair, or question the interlocutor's statement, in order to re-establish or strengthen common ground and be in a position to complete the task appropriately. Hence, the 8.7\% rate of non-responses in Mismatches for the ASD group is particularly striking, especially when considering that all 14 speakers in the control group addressed each single occurrence of a Mismatch explicitly and verbally (always keeping in mind the low sample size, which limits the generalisability of this finding).

The relatively frequent occurrence of long gaps and non-responses for both kinds of landmark suggests that autistic dyads treated new information, whether it was matching or mismatching across maps, as unexpected more often than control dyads, who only showed a high proportion of long gaps in response to unusual and unexpected events in the form of mismatching landmarks.

In principle, there could be many reasons for the longer between-turn gaps in the ASD group. General delays in response production have been attested for individuals on the autism spectrum in various studies on motor production, language production and language perception \citep[see e.g.][]{gernsbacherInfantToddlerOraland2008}. However, as the ASD group only differed from controls for Matches, and not for Mismatches, it seems necessary to provide more specific explanations for this difference. It could be speculated that autistic subjects simply did not describe landmarks as clearly as controls, but a cursory content analysis suggests that this is not the case. A more likely, if no less speculative, explanation may be found in relating the current results to characteristic features of subjective time experience in ASD \citep{zukauskasTemporalityAspergerSyndrome2009}.

It has been claimed that for individuals on the autism spectrum, the present is often experienced as a sequence of small, non-overlapping units or events which are planned in advance \citep[broadly in line with theories of Weak Central Coherence; ][]{frithAutismExplainingEnigma2003,happeExploringCognitivePhenotype2001}. This can result in a fear of outside events interrupting an individual's self-imposed and pre-planned temporal structure \citep{vogelInterruptedTimeExperience2019}. When unexpected outside influences do interrupt the present, the experience of time can appear discontinued and lead to what Vogel and colleagues refer to as ``interrupted time experience''. Time as experienced by most non-autistic persons, on the other hand, seems to more closely resemble a series of stretched-out and overlapping time windows with fuzzy boundaries \citep{vogelFlowStructureTime2020}. This latter representation should be far more robust to interruptions and unexpected events (for instance in the shape of new information being conveyed by an interlocutor) than the string of discrete, sequential and non-overlapping units that has been described to characterise time perception for at least some autistic persons.

Applying this perspective to the current data, it could be speculated that subjective time experience in ASD was one reason for the fact that autistic dyads reacted to the introduction of all new information (expected or unexpected) by producing longer silent gaps -- in contrast to non-autistic dyads, who produced longer gaps only directly following the unexpected introduction of \emph{mismatching} landmarks, which inherently contain an element of surprise.

\subsubsection{Comparison to turn-timing in non-native dialogues}\label{turntaking_conclusion_discussion_L2}


	As pointed out above, the turn-timing analysis presented in this work is directly comparable to only very few previously published studies. To expand the scope and provide a wider sense of context, the same analogy as in other parts of this book can be employed by comparing autistic language with bilingual, or non-native, language production.

Second language (L2) learners might be expected to produce more and longer silent gaps, as they are not only faced with the regular challenges inherent in performing rapid turn-timing, but additionally with the considerable and multi-faceted difficulties of communicating in a non-native language. This expectation is strengthened by the two adjacent findings that 1) word naming in L2 speech is delayed \citep{hanulovaWhereDoesDelay2011} and 2) the typical fast turn-timing patterns of adult conversation are not reached in \emph{first} language acquisition until around the age of 9 \citep{casillasTurnTakingTiming2016,garveyTimingTurnTaking1981}.

To my knowledge, the first published study that includes a detailed discussion of results on turn-timing in non-native speech is \citet{galacziInteractionalCompetenceProficiency2014}. The author examined conversations between 41 dyads that were recorded as part of Cambridge English Language Assessment. Cambridge English is an exam board and organisation which made major contributions to the development of the European Union's Common European Framework of Reference for Languages \citep{councilofeuropeCommonEuropeanFramework2001} and continues to perform language assessments aligned with the proficiency levels set out in this framework. The work of \citet{galacziInteractionalCompetenceProficiency2014} is conducted mostly in the tradition of qualitative Conversation Analysis. This entails an analysis of turn-timing which is rather coarse compared to the methods applied in this book. Specifically, only a categorical analysis with three distinct types of turn transitions was performed. Those categories were ``latch/overlap'' (corresponding to any negative FTO values), ``no-gap-no-overlap'' (any gaps of 500 milliseconds or less) and ``pause'' (gaps longer than 500 milliseconds). Describing gaps of 0 ms and 500 ms equally as ``no-gap'' is exemplary of the problematic simplifications inherent in this approach. \citet{galacziInteractionalCompetenceProficiency2014} reports that L2 English speakers with a lower proficiency tended to produce slightly higher proportions of ``pauses'', corresponding to longer silent intervals between speakers, than L2 speakers with higher proficiency levels.

A more recent study by \citet{sorensenEffectsNoiseL22019} tested native Danish speakers who were highly proficient in English performing a spot-the-difference task in both their L1 and their L2. The authors report that FTO values in L2 speech were either equivalent to those in L1 speech or, contrary to expectations, \emph{lower} compared to L1 speech, depending on whether conversations took place in silence or in noise.

The results from these studies on non-native turn-timing can be related to the findings described in the current work and in previous studies on turn-taking in ASD. As the participants in the current study were motivated and socially skilled autistic adults, one could really only expect to see parallels (if any) with results from highly proficient L2 learners or bilingual speakers (as opposed to less advanced learners; cf. \sectref{Conclusion_discussion_L2}). Indeed, the current results and those on highly proficient L2 speakers in both \citet{galacziInteractionalCompetenceProficiency2014} and \citet{sorensenEffectsNoiseL22019} reveal no differences in turn-timing compared to the relevant control groups (non-autistic native speakers). In contrast, results from previous work on autistic children more closely align with what was found for beginner learners of an L2 in \citet{galacziInteractionalCompetenceProficiency2014} in revealing a tendency for longer silent gaps between speakers.

Although research on turn-taking in second-language speech is still very limited in scope, comparing non-native with autistic turn-timing holds promise for future investigations, especially since more in-depth analyses of L2 turn-timing have been proposed and exemplified recently in work by e.g. \citet{sbrannaQuantifyingL2Interactional2021,sbrannaDevelopingL2Interactional2021}.

\subsection{Limitations, extensions and future directions}\label{turntaking_conclusion_discussion_limitations}



Although I believe that the thoroughness and transparency of the analysis allows us to draw certain conclusions on the basis of the experimental results with a certain degree of confidence, naturally there are many factors to limit external validity.

First, the behaviour of socially skilled German-speaking autistic adults was analysed. There are many ways in which results might differ for individuals situated at different points on the autism spectrum, of different native language backgrounds, and at different stages of development. The state of the art is such that we cannot directly compare the results at hand to any others on turn-timing in ASD. An obvious extension of the present work would therefore be replications with children and/or with adults speaking a language other than German.

Second, semi-spontaneous dialogues without eye contact between participants were elicited. A multi-modal analysis of video-recorded interactions between speakers with and without ASD could therefore add further crucial information, as gaze and gesture have been shown to play important roles in dialogue management \citep[e.g.][]{mondadaContemporaryIssuesConversation2019,auerGazeAddresseeSelection2018,mcclearyTurntakingBrazilianSign2013,hollerProcessingLanguageFacetoface2018,zellersProsodyHandGesture2016,bohusFacilitatingMultipartyDialog2010}. Recent work by \citet{demarchenaAtypicalitiesGestureForm2019} specifically shows that autistic speakers seemed to use gesture more than non-autistic speakers to regulate turn-timing. Follow-up experiments including recordings of gaze, posture and gesture are currently being conducted \citep{spaniolMultimodalSignallingInterplay2023}. Regarding the contextual constraints inherent in the Map Task, it is true that having to fulfil an unfamiliar task puts certain pressures and limitations on participants and the resulting linguistic output, and this may have affected speakers in the ASD group differently than those in the CTR group. However, the restricted set of dialogue options and reduced chance of unexpected events may in fact have suited the cognitive styles of autistic speakers more than fully free and spontaneous conversation, which would in turn make the between-group differences described in this paper all the more relevant.

Third, as the behaviour of disposition-matched dyads (ASD--ASD) was investigated here, perhaps most obvious would be an extension to also include mixed dyads (ASD--CTR). The overwhelming majority of experimental work on communication in ASD has in fact been conducted using mixed dyads only. Disposition-matched dyads (ASD--ASD) rather than mixed dyads (ASD--CTR) were recorded for two main reasons. First, there is quite simply a dramatic lack of research on communication in ASD based on data from such matched dyads. Second, investigating the behaviour of disposition-matched dyads seems to me the most promising way to gain insights into what might justifiably be called autistic communication. Analysing the behaviour of mixed dyads makes it very difficult to see beyond the patterns and potential difficulties arising from the interaction of individuals with different cognitive styles \citep{miltonDoubleEmpathyProblem2020,miltonOntologicalStatusAutism2012,williamsMutualMisUnderstanding2021,mccrackenAutisticIdentityLanguage2021}. While such insights are of great value in principle, they cannot be interpreted conclusively and appropriately unless we first have a clear picture of what characterises communication between autistic speakers.

This perspective, in the sense of a certain epistemic humility, extends to the study at hand. For instance, while we can accurately say that the ASD group tended to produce longer silent gaps between turns than the CTR group in certain parts of conversations, by no means can (or should) we claim that this behaviour is simply “wrong” or “inappropriate” in any way. Not only do we have to recognise the very likely possibility that autistic dialogue strategies diverge from those of non-autistic peers in ways that are the most appropriate and functional for this group in the given situation. We also have to acknowledge that we cannot say for sure whether long gaps, produced by any group of speakers, are appropriate or not in a given context without conducting a comprehensive qualitative analysis that takes into account the context of turn transitions. Previous work assures us, for instance, that long gaps are typical and expected in the direct context of verbal exchanges involving misunderstanding or non-alignment \citep{kendrickTimingConstructionPreference2015a,kendrickIntersectionTurntakingRepair2015,robertsIdentifyingTemporalThreshold2013}.

It is beyond the scope of this work to exhaustively analyse how many cases of long gaps were indeed produced in just such contexts for each group, but the detailed analysis of different stages of dialogue gives us a proxy for such an analysis. It was shown that gaps were longest for the ASD group before detection of the first Mismatch (whereas values were comparable throughout the dialogue for the CTR group). This makes it clear that these cases of long-gap transitions are not specifically linked to unexpected events as part of the task itself, but rather reflect the previously attested observation that people diagnosed with ASD tend to have more difficulties with, and tend to be less comfortable in, situations involving newness or uncertainty. Conversely, it was shown that dyads from the CTR and the ASD group produced equally long gaps immediately following the introduction of mismatching landmarks, but that dyads from the ASD group produced longer gaps immediately following the introduction of matching landmarks compared to the CTR group.

In essence, autistic dyads thus produced many long gaps in various situations sharing an element of novelty, while non-autistic dyads only produced long gaps in the context of particularly challenging aspects of the experimental task itself (such as the introduction of an unexpected mismatch). Generally speaking, using such long gaps may be an effective strategy for navigating challenging and unusual situations, and it is employed by both groups of speakers in the data set under study. The difference, then, lies only in the fact that this strategy was used by dyads from the ASD group in a wider variety of contexts. The best way to test such assertions experimentally would be to conduct perception tests in future studies. This would make it possible to assess and compare just how meaningful the differences in turn-timing found here are (perceived to be) for both autistic and non-autistic listeners, and to what extent such differences might influence understanding as well as character judgements.

I mentioned second-language speech as an analogous and similarly understudied area in research on turn-taking. This equally applies to communication in schizophrenia. Beyond a small number of published papers \citep{breitholtzReasoningMultipartyDialogue2020, howesDisfluenciesDialoguesPatients2017, lucariniConversationalMetricsPsychopathological2021,cangemiContentfreeSpeechActivity2023}, we do not know much about dialogue management in schizophrenia. A comparative study of turn-taking in ASD and in schizophrenia could help to shed light on differences and similarities between the two groups and, by extension, on social interaction and neurodiversity in general.

Finally, triadic instead of dyadic conversation, as investigated e.g.~in the aforementioned studies by \citet{auerGazeAddresseeSelection2018,breitholtzReasoningMultipartyDialogue2020} can serve as a highly promising extension to general findings on turn-taking in dialogue and could equally usefully be applied to the specific case of turn-taking in ASD.



