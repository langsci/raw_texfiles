\addchap{\lsAcknowledgementTitle} 

This book is a revised and improved version of the doctoral dissertation I submitted in October 2021. The original acknowledgments are reprinted below with minor (though in at least one case momentous) updates.

This work would quite simply and quite literally not have been possible without the doctoral scholarship I have had the privilege to receive. I would therefore like to thank the \emph{Studienstiftung des deutschen Volkes} (German Academic Scholarship Foundation) for their very generous and patient support, even in the face of more or less foreseeable vagaries such as parenthood and a global pandemic. I am also very grateful to the SFB 1252 at the University of Cologne. My membership in this collaborative research centre has benefitted me in a number of very different but equally important ways.

Massive thanks also to my doctoral supervisors, Martine Grice and Kai Vogeley. Martine has supported and inspired me since I first cautiously stepped into the Cologne phonetics lab and has been an extraordinarily approachable, helpful and insightful supervisor and human being. I am very grateful to Kai for lending some of his vast expertise to my project and for always being open-minded and encouraging in discussions of my research. Big thanks to Stefan Baumann for agreeing to be my third examiner, for being a great colleague and for closing the circle from leading the very first MA seminar I attended to the completion of this book.

I can't give enough thanks to Franceso Cangemi, my ``shadow supervisor'' and, dare I say it, mentor (sorry for making you sound old). Francesco has had a hand in the germination of most valuable ideas expressed in this book. More importantly, I am grateful for the many, many deep, fun, silly, serious, musical, scientific and all-of-the-above hours we shared. Big thanks, too, to the third Swiss Sister, Aviad Albert, for sharing musical adventures and fun discussions with me, but also for massively helping me with some of the more technical aspects of this book in particular.

Thanks to many other Cologne people past and current, all with undue brevity: Martina Krüger for getting me started on the autism data, Kieu-Phuong Glaser (Ha) for getting me into backchannels, and Malin Spaniol for brilliantly leading the way into the next phase of research and for generally being a great person to be around. To Alicia Janz, Harriet Hanekamp, Matthias Gauser and Anika Müller for their invaluable help with data, annotation, breakfasts, dog-sitting and more. Thanks to Christine Röhr for being an exemplary and open-minded gatekeeper and colleague. To Simona Sbranna and Eduardo Möking for keeping me in touch with and excited about the L2 world. Thanks to Juliane Zimmermann for lots of insightful and fun discussions; thanks also to her psychiatry colleagues Valeria Lucarini, Carola Bloch, Mathis Jording and David Vogel.

Outside my immediate surroundings, I want to first give special thanks to the dynamic duo of Bodo Winter and Timo Roettger. Bodo might not know this, but he's been a great ``enabler'' in the best sense for me through the years, having at various points given me encouragement and advice that has often proved invaluable in hindsight. He has also greatly contributed to the methodological fine-tuning of the backchannel and filled pause analysis and his sharp mind rarely fails to deliver some valuable insight even during the shortest of exchanges. Timo has been a great collaborator in the past and is a role model for his dedication to high-quality open science as well as to most other things that are not broken in academia. Together, Bodo and Timo have opened my eyes many years ago to the importance of statistical analysis, while also fuelling my existing passion for clear and beautiful data visualisation. Finally, they have both contributed to my shift from the NHST framework I never liked or subscribed to in the first place to embracing the Bayesian mindset (all via the strange no man's land of pure description and exploration which I am stubbornly keeping one foot in).

I would like to thank Nigel Ward for the collaboration on prosodic constructions and his many influential written works -- to be continued. Many other people have been kind enough to share their thoughts on this project with me and provided valuable feedback, including Mark Dingemanse, Riccardo Fusaroli, Elina Savino, Francisco Torreira, Jason Bishop, Gemma Williams, Loulou Kosmala, Mattias Heldner, Katharina Zahner-Ritter, Bob Ladd and Bettina Braun. Special thanks to Olcay Türk, Sasha Calhoun and Paul Warren for hosting me in Wellington.

Back in Cologne, many people have contributed to making the phonetics lab a special place over the years, in every sense and at various points. In no particular order and under no illusions of completeness, thank you to: Theo Klinker, Anna Bruggeman, Lena Pagel, Tabea Thies, Maria Lialiou, Janne Lorenzen, Jane Mertens, Christine Riek, Simon Roessig, Caterina Ventura, Esther Weitz, Mark Ellison, Anne Hermes, Doris Mücke, Constantijn Kaland, Katinka Wüllner, Noemi Furlani, Chem Vatho, Janina Kalbertodt, Mathias Stöber, Georg Sachse, Drenushë Valera-Kurteshi and Jessica di Napoli. Special thanks to the participants of the FORAUS discussion forum for autistic adults, who generously shared their lived experiences and thereby deeply informed my interpretation of the experimental results and analyses. 

Even closer to home, I am very grateful to my parents for their unwavering support, no matter how comprehensible or sensible my path may have seemed at times. Danke für alles. Thanks also to Betti and Chrissi for being excellent sisters. Flip gets a pat for outstanding moral support and swampy breath and for a brief but stellar stint as the lab lap dog.

Finally, ultimately, to Rachel, Wilbur and Ida. Wilbur, who blew my mind upon arrival and has been helping me to reassemble it in a better form ever since. You continue to inspire, challenge and delight me every day. And Ida, who has been pushing the envelope in her own delightful way, at once squishable and regal. Above it all, Rachel: you deserve endless credit for having been entirely patient and gracious throughout this long and sometimes arduous process and far beyond it. It would take me another 218 pages to produce the mere outline of a rough sketch of everything you mean to me and of how grateful I am to you. As I imagine this would be deemed inappropriate here, I will restrict myself to saying: thank you so much, for everything, always.


