\chapter{Conclusion} \label{conclusion}

\section{Summary analysis}\label{sec:Conclusion_SummaryAnalysis}

I will start this final part of the book by providing an overview of some key results in the form of a summary analysis, focussing on different dimensions of conversation and intonation in ASD and identifying patterns of dyad-specific behaviour.

This will be followed by a broader discussion of the overarching conclusions we can draw from the large number of results presented in the preceding chapters. Finally, I will attempt to provide a compact synthesis of the most important findings and finish with a brief outlook to future avenues of investigation.

Figure \ref{fig:OverTable} presents an overview of results from the key dimensions described in this book for all seven ASD dyads. The primary purpose of this overview table is not to give a comprehensive account of all the results described in preceding chapters, but rather to provide a concise and accessible illustration of the patterns that characterise the behaviour of autistic dyads in the current data set as compared to the CTR group.



\begin{figure}

\includegraphics[width=1\linewidth]{images/overview_langsci_cropped.pdf} \hfill{}

\caption{Overview table for each ASD dyad along five dimensions and eight parameters, as compared to averages from the CTR group. Black cells represent strong effects, grey cells represent moderate effects. See text for further details.}\label{fig:OverTable}
\end{figure}



\subsection{Rationale and parameters}\label{Conclusion_SummaryAnalysis_rationale}

One or two parameters were selected to represent each of the five main dimensions of dialogue management and intonation discussed in the preceding chapters. These parameters were chosen for being at once appropriately representative of the larger phenomenon at hand and conveniently representable in a summary table.

For instance, while for both backchannels and filled pauses, a column for rate of production is included (fourth and sixth column in Figure \ref{fig:OverTable}), a different second parameter was chosen for each dimension. Prosodic realisation is shown for filled pauses (seventh column), whereas diversity of BC types is shown for backchannels (fifth column). This was done for two complementary reasons. As the difference between the two filled pause types \emph{uh} and \emph{uhm} in prosodic realisation was not robust in the group comparison, it is appropriate to calculate results across types and represent them in a single column. For the same reason, choice of filled pause type is not a particularly informative parameter and is not included in the summary table in Figure \ref{fig:OverTable}. The reverse is the case for backchannels. As prosodic realisation is specific to each of the four main types of backchannel (\emph{genau} (`exactly'), \emph{ja} (`yeah/yes'), \emph{mmhm} and \emph{okay}), it would be uninformative at best and misleading at worst to present results of prosodic realisation across types in a single column (while the alternative of using four separate columns would defeat the purpose of a summary analysis). Conversely, choice of backchannel type is a highly informative parameter and is therefore represented with a dedicated column showing the diversity of backchannel types produced.

For intonation style, both relevant parameters (Wiggliness and Spaciousness) used in the two-dimensional analysis presented in \chapref{intonationstyle} are shown (first and second column). For turn-taking, only the single parameter of FTO (Floor Transfer Offset; third column), used as a measure of turn-timing, is shown. FTO suffices to represent all the relevant patterns found in the analysis of turn-taking (presented in \chapref{turntaking}). Backchannelling is represented by the parameters of rate and diversity, while production of filled pauses is represented by the parameters of rate and prosodic realisation (as presented in \chapref{backchannels}; see above for rationale). Finally, the rate of silent pauses \geq 700 milliseconds (as reported in \sectref{sec:BCFP_FP_silent}) is shown in the summary table (eighth column). This parameter complements the analyses of both turn-timing and filled pauses.

Behaviour was analysed at the level of dyads and not speakers, for all parameters. While speaker-specific analyses were used for suitable measures in the preceding chapters, this is not appropriate or even possible for all parameters presented in Figure \ref{fig:OverTable}. For instance, the rate of backchannels produced by one speaker crucially depends on how many appropriate opportunities for backchannelling the speech of the interlocutor provides, while turn-timing can only be measured in the transitional space between talk by two interlocutors.

Cells in the summary table are filled in and colour-coded to represent divergence from (average) behaviour in the CTR group. Strong effects are defined as any mean values by (ASD) dyad that fell outside the range of the mean of the CTR group \pm1 standard deviation (i.e.~outside the central 68\% of values). Such effects are represented with black cells, with ``+" or ``-" icons representing the direction of the effect.

Moderate effects (grey cells) are defined as any mean values by dyad that were higher (+) or lower (-) than the average of any single CTR dyad. If a mean value fulfilled the criteria for both strong and moderate effects (as was occasionally the case), the effect is considered to be strong. Details of the resulting cut-off values for each parameter along with summary tables can be found in the accompanying \textit{OSF} repository (\url{https://osf.io/6vynj/}).

\hspace*{-1pt}I have avoided relying on inferential statistical tests to characterise divergences in the behaviour of the ASD dyads in this summary analysis, in keeping with the emphasis on thorough, transparent description that I have followed throughout this book (see \sectref{analysis}). It was confirmed that the heuristic thresholds chosen are representative of the results from Bayesian modelling and of the in-depth dyad- and speaker-specific analyses previously performed for all parameters and dimensions.

\subsection{Identification and interpretation of dyad-specific patterns}\label{Conclusion_SummaryAnalysis_patterns}

Overall, we can observe that no column and no row in the summary table is either completely filled or completely empty and, moreover, that no two columns or rows are identical to each other. In other words, all ASD dyads diverged from average behaviour in the CTR group in their own unique way and along different parameters. While all dyads clearly diverged from CTR behaviour in some regards, none did so along all parameters.

This is simply another way of highlighting the fundamental fact that dyad-specific (or individual-specific) behaviour is a crucial aspect of spoken communication, and that we ignore this fact in favour of simplified group-level analyses at the peril of scientific integrity and descriptive accuracy. Studying a group of individuals with ASD is a particularly suitable test case for this assertion (due to the characteristically high inter-individual variability), but the principle holds for studies of non-autistic speakers (cf.~\sectref{analysis_individual} and \sectref{Conclusion_discussion_individuals}).

Considering the different dimensions of analysis (columns in Figure \ref{fig:OverTable}), we can see that the CTR and ASD groups differed most clearly in their use of backchannels (with strong effects for six out of seven dyads). The greatest similarity between groups, on the other hand, can be found in turn-timing behaviour (with only one dyad showing a (moderate) effect). The remaining dimensions fall in between these two poles, with at least moderate differences in intonation style for most dyads, mixed results for filled pauses (more differences for prosodic realisation than rate) and a (strong) difference in silent pause rate for two out of seven dyads.

As we shift our attention to dyad-specific behaviour (rows in Figure \ref{fig:OverTable}), the most immediate observation is that there is a ``lack of invariance'' within the ASD group, to borrow a term from speech perception \citep{libermanPerceptionSpeechCode1967a}. In other words, we find further evidence for a great degree of heterogeneity across the behaviour of different individuals (and dyads) diagnosed with ASD. In the following paragraphs, I will briefly describe the patterns observed for each ASD dyad. The relevant overview plots can be found in the folder ``turnation" of the accompanying \textit{OSF} repository at \url{https://osf.io/6vynj/}.

Dyad F02\_F03 (top row) produced notable differences from average CTR behaviour for two out of five dimensions and three out of eight parameters. This pair of speakers did not notably differ from the CTR average in terms of intonation style, turn-taking and silent pause rate. Regarding filled pauses, there was a moderate effect for prosodic realisation (more rising tokens, with the highest mean value out of all dyads -- all other ASD dyads produced lower or equivalent values), but not for rate.

The clearest differences were found for the dimension of backchannelling. F02\_F03 was the only ASD dyad to produce a \emph{higher} rate of backchannels (the second highest overall) than the CTR average (with a strong effect). Out of the remaining six ASD dyads, four dyads had a lower rate and two dyads had a rate comparable to the CTR average. Besides the lower rate, F02\_F03 also produced a reduced variety of different backchannel types (strong effect).

This pattern of behaviour is highly salient perceptually, as it results in dialogue containing a very high number of very similar backchannels, e.g.~``\textit{ja\ldots.ja\ldots.ja}''. Not shown in the summary table is the fact that, additionally, these productions were also not very diverse in terms of prosodic realisation, as almost all tokens (across types) in this dyad were produced with rising intonation (cf.~Figure \ref{fig:BCIntCategoricalSpeaker}). Backchannels also seemed to be produced at quite regular intervals and independent of conversational context. This last aspect remains an impressionistic observation at this point, as the exact timing of backchannels was not analysed. The overview plot (in the repository) gives an idea of the frequency and timing of backchannel tokens (particularly noticeable in the form of the light blue dots in the second half of dialogue).

The overview plot also illustrates that the speakers in dyad F02\_F03 seemed to adhere rather closely to their assigned roles of instruction follower and giver, as each half of the dialogue consists mostly of speech by the instruction giver. Instruction followers responded mostly with very short utterances (i.e.~backchannels in most cases), especially in the second half of the task. This conversational style is further reflected in the facts that 1) overall dialogue duration was very short (the second shortest overall) and 2) this dyad was the only one to essentially ignore the fact that there were mismatching landmarks between the two participants' maps (the issue was noted, but not discussed before the participants immediately moved on to a description of the next part of the map; see green outline in the overview plot).

Dyad M07\_M08 (second row) was the dyad to diverge most clearly overall from the CTR average, showing notable differences along all five dimensions and for seven out of eight parameters. The only dimension for which no notable difference was detected is turn-timing. Intonation style in this dyad was more melodic than in the CTR average, with moderate effects of both higher Wiggliness (the highest value of all dyads) and Spaciousness (the third highest value overall). Strong effects were also found for backchannelling, with a low diversity of BC types as well as a low rate of BCs produced (the lowest out of all dyads; see leftmost square in Figure \ref{fig:BCRateDyad}). Finally, strong effects were found in a higher rate of silent pauses (the highest rate overall) and filled pauses (the third highest rate overall), as well as a moderate effect in the prosodic realisation of filled pauses (more falling; with the lowest mean value out of all dyads).

Impressionistically, these divergences along different dimensions have a cumulative effect, leading to a notably unusual conversation style. For instance, a short exchange between M07 and M08 might not only contain few backchannels or none at all, but at the same time feature many silent \emph{and} filled pauses (the latter somewhat unusual in prosodic form as well). Additionally, task duration was rather short and speaking times were not well-balanced between interlocutors (with a score of 17\%; cf.~\sectref{turntaking_results_signal_dominance}). There are no obvious a priori reasons for the extent of divergence in this particular dyad. Both speakers were well within the range of individuals within the ASD group as regards age, AQ and verbal IQ.

Dyad M11\_F05 (third row) stands in direct contrast to dyad M07\_M08 (discussed directly above), as this pair of speakers produced the least amount of notable differences from the CTR average out of all ASD dyads. Behaviour was comparable to typical patterns in the CTR group for all dimensions except for both parameters characterising intonation style (moderate effects of both higher Wiggliness and higher Spaciousness). Just as for M07\_M08, there are no obvious a priori reasons to suggest why this dyad might stand out from the other dyads within the ASD group. It is interesting to note, however, that dialogue in this speaker pair actually started off in a very unusual fashion, with only minimal verbal contributions by the instruction follower in the first couple of minutes. This changed after an explicit statement by the instruction giver encouraging the interlocutor to comment on his instructions and to ask questions (around minute 2:45).

Dyad M04\_M05 (fourth row) produced notable differences from average CTR behaviour along three out of five dimensions and four out of eight parameters. These differences mostly concerned prosodic aspects. M04\_M05 was the only speaker pair to show a strong effect for differences in intonation style, with the highest overall Spaciousness value. There was also a moderate effect for higher Wiggliness, representing the second highest value out of all dyads.

M04\_M05 produced the second lowest value for prosodic realisation of filled pauses (i.e.~more falling contours; moderate effect) and the lowest diversity of backchannels (strong effect). All these differences are perceptually particularly salient because task duration was the shortest out of all recorded dyads (under 9 minutes). No differences in turn-timing were found, nor for the rates of backchannels, filled pauses and silent pauses.

Dyad M06\_F04 (fifth row) is the second dyad (along with M07\_M08) to show notable differences for all five dimensions (five out of eight parameters). This dyad stands out in particular as being the only dyad with a clear difference in turn-timing compared to average CTR behaviour, with the highest mean value of all dyads (moderate effect). The speaker pair showed moderate effects for Wiggliness (higher than CTR average) and the prosodic realisation of filled pauses (lower than CTR average). Further, strong effects were found in a higher rate of silent pauses and a lower rate of backchannels.

The cumulative effect of longer between-speaker gaps, more frequent long within-speaker pauses and fewer backchannels is perceptually quite salient and manifests as an unusual overall proportion of silence in the dialogue, visible as a comparatively large amount of white space in the overview plot  (in the repository). The overview plot also illustrates a pattern of interlocutors adhering rather strictly to their assigned roles of instruction follower and instruction giver in each Map Task.

Dyad M09A\_M10A (sixth row) was one of two dyads (along with M11\_F05) to differ from average CTR behaviour for only one single dimension (and two parameters), in this case, backchannelling. M09A\_M10A produced the second lowest rate and diversity of backchannel tokens (both strong effects). Behaviour was comparable to the CTR average for all other parameters. This dyad was notable also for having the longest dialogue duration out of all ASD dyads by a large margin (30 minutes -- 13 minutes more than the next longest dialogue, by ASD dyad M11\_F05).

Thereby, the two ASD dyads with the longest dialogue durations were also the two that only showed divergence from average CTR behaviour along one single dimension. We can only speculate about the significance (if any) of this correlation, but it is tempting to connect this to the suggestion that the combination of shorter overall dialogues in the ASD group combined with proportionately fewer backchannels and less laughter might indicate a more goal-oriented and efficient (at least from a functional perspective) conversation style (see the discussion of backchannels in \sectref{BCFP_Discussion_BC}). Conversely, longer dialogues such as by dyads M09A\_M10A and M11\_F05 might be indicative of a more explicitly other-oriented and affiliative conversation style.

Finally, dyad M02\_M03 (bottom row) showed relatively few divergences from average CTR behaviour. Specifically, the differences identified were a lower rate of backchannels and the highest filled pause rate of all dyads (both strong effects). Once again, these are clearly not orthogonal effects, but rather, their interaction can be expected to have a cumulative effect on the holistic perception of conversational style. Backchannels and filled pauses have directly contrasting functions (as discussed in \sectref{BCFP_FP_background} and \sectref {BCFP_Discussion_BCvsFP}). Backchannels are used by listeners to support the turn of the other speaker, while filled pauses are used by speakers to prolong their own turn and avoid transferring the floor. Thus the higher rate of filled pauses and lower rate of backchannels in dyad M02\_M03 add up to the (impressionistic) sense that both speakers were focussed much more on their own turns at talk than they were on encouraging their interlocutor to speak.

Fittingly, this pair of speakers adhered to the roles of instruction giver and follower comparatively strictly (see overview plot in the repository), possibly reflecting a prioritisation of orderly completion of the task over spontaneous engagement with the interlocutor (see description of dyad M09A\_M10A above for discussion and further examples).


\subsection{Limitations of the summary analysis}\label{Conclusion_SummaryAnalysis_limitations}

I already mentioned in the introduction that this summary analysis is not intended to be, and indeed cannot be, an exhaustive overview of all the communicative behaviours examined in this book. Rather, it represents my best attempt to reduce the considerable amount of complexity that is common to the manifold aspects of conversation and intonation covered in this work to an easily digestible whole.

Although a certain degree of simplification was necessary to achieve this aim, I believe that this overview still accurately represents the essence of the key findings presented in this work. I have been as transparent as possible about the fact that a number of subjective decisions were made regarding the inclusion and exclusion of various parameters as well as the setting of thresholds for what are considered moderate or strong effects. Although I have made every effort to ascertain that the specific choices made were best suited to this compact yet representative analysis, it should be self-evident that such choices are always debatable and can have a considerable impact on the outcome of any analysis \citep{corettaMultidimensionalSignalsAnalytic2023,roettgerResearcherDegreesFreedom2019}.

One corollary of concentrating the summary analysis on an easy-to-process number of dimensions was that results were considered across the entire duration of dialogues. It is thereby not possible to acknowledge some of the intriguing patterns that were found by comparing early with later stages of dialogue here. The reader is referred to the relevant sections in the preceding chapters (e.g.~\sectref{turntaking_results_FTO_group_stage} and \sectref{BCFP_BC_results_BCRate_Stage}) as well as in the concluding remarks (\sectref{Conclusion_discussion_stages}) for more details on the comparison of different dialogue stages.

One further general and potentially problematic limitation of this summary analysis is the fact that, for this specific purpose only, the CTR group was considered as a monolithic whole. This was done in order to identify group means which would serve as reference values in the comparison with ASD dyads. As discussed at length in the immediately preceding section, however, the most important message from this summary analysis concerns the supreme importance of appropriately considering and accounting for inter-individual and dyad-specific variability in the study of human behaviour. While this assertion might seem to stand in direct opposition to the method of quantifying ASD--CTR differences in this summary analysis, I submit that it is more fruitful to formulate a certain number of carefully considered generalisations \emph{along with} detailed, dyad/speaker-specific analyses than to avoid doing so as a matter of principle. In this light, the reader is explicitly encouraged to not only critically question the choices made in this work, but to also independently follow alternative paths of analysis using the data and code provided in the accompanying files.
















\section {General discussion}\label{sec:Conclusion_discussion}
\largerpage
I would like to conclude with a brief summary of the most important findings and by adding some final thoughts on possible implications as well as interpretation and contextualisation.

\subsection{Autistic persons as particularly individual individuals}\label{Conclusion_discussion_individuals}

Throughout this book, I have acknowledged the importance of individual specificity to not only my own field of study, but also related ones. I determined from the outset to focus on individual- and dyad-specific behaviour in sufficient detail to arrive at an accurate description at the group level, all the more so in the case of the ASD group. The analytical choice of emphasising transparent, in-depth description at the levels of individuals, dyads and groups supported by Bayesian modelling reflects this stance.

I emphasised throughout that overlap between the ASD and the CTR group was found for each single one out of the dozens of parameters investigated in this multi-dimensional analysis of conversation and intonation. It was further shown that group means usually do not suffice to accurately portray the underlying behaviour of the individuals and dyads within a group. Finally, considerable evidence has been amassed to further strengthen the well-established argument that individual differences play a particularly important role in ASD \citep[cf.][]{griceLinguisticProsodyAutism2023,goldbergConstructionistApproachOffers2021,wozniakDevelopmentAutismSpectrum2017}. Even in the slightly simplified summary analysis presented in \sectref{sec:Conclusion_SummaryAnalysis}, there is not a single dimension or parameter for which the behaviour of all seven autistic dyads was the same or equivalent.

Having performed in-depth analysis at the level of individuals and dyads ultimately also enables us to more confidently formulate generalisations at the group-level. These generalisations never apply equally to each autistic dyad in the current sample (or beyond), but they do give us some strong hints about robust tendencies of behaviour. This is all the more valuable in the description of a group as broad and varied as that of individuals diagnosed with ASD. I will discuss some of the most important general observations and conclusions in the following sections.

\subsection{Backchannelling as a prototype of other-oriented communicative behaviour}\label{Conclusion_discussion_backchannels}

The clearest overall difference between groups was found for backchannelling. Six out of seven autistic dyads clearly diverged from typical behaviour in the control group for this dimension. As mentioned previously, backchannels are distinguished by being a relatively implicit and decidedly pro-social communicative signal. The specific finding of, for instance, a reduced rate of backchannels might reflect a more general lack of interest in explicitly showing attention to an interlocutor by autistic individuals (but keep in mind that the group difference was very clear only for the earliest stages of dialogue; see \sectref{Conclusion_discussion_stages} below). The simple to calculate metric of backchannels per minute might thus serve as a reliable correlate of general tendencies in autistic communication.

A reduced rate of backchannels in ASD can be related to what has been described in previous studies for the behaviour of non-autistic speakers in task-oriented as opposed to free conversation. As pointed out in other parts of this book, a lower rate of backchannels, especially when combined with related findings, e.g. the fact that dialogues in the ASD group were shorter and contained less laughter, seems to point to an approach to social interaction that prioritises efficiency. In other words, the analyses reported in this book seem to reveal specific behavioural correlates of an autistic preference for goal-oriented communication. It is interesting to note that backchannels have historically often been described as a non-essential and functionally irrelevant element of language -- largely owing to an overemphasis on written language and/or monologic speech in the dominant theoretical and pedagogical approaches \citep{linellWrittenLanguageBias2004,schegloffDiscourseInteractionalAchievement1982,oconnellHistoryResearchFilled2004}. It might be no coincidence that backchannels seem to play a diminished role in the perception and communicative style of a group that often priorities explicitness and economy over small talk and purely affiliative aspects of social interaction.

Measuring the diversity of backchannel types is a little less straightforward than measuring their rate, but the relevant finding of reduced diversity in the ASD group can be promisingly linked to the other main diagnostic criterion for ASD besides difficulties in social communication, i.e.~circumscribed or inflexible behaviour.

The fact that backchannel behaviour in ASD has not been investigated in any detail to date makes it highly promising overall as an additional component in the description and assessment of autistic communication styles in future.

\subsection{Turn-timing as a fundamental and universal skill in interaction}\label{Conclusion_discussion_turns}

In direct contrast to backchannelling, hardly any consistent overall differences were found regarding the turn-timing of autistic as compared to non-autistic dyads. Only one out of seven ASD dyads showed a notable difference in global turn-timing, and even this difference was rather slight. While I have described backchannels as implicit and affiliative signals, a rapid exchange of turns is evidently essential for any functioning coordinated interaction with the speed  and complexity of spoken dialogue. Turn-taking is a fundamental aspect of social interaction and the relevant skills are not limited to the use of language. In this sense it is not entirely surprising that the current findings serve to add speakers on the autism spectrum to the many diverse groups of speakers who, despite manifold cognitive, cultural and linguistic differences, have been found to exhibit remarkably similar turn-timing behaviour.

While the experimental task of transferring a route from the map of one person to that of another without visual contact could in principle be accomplished without the production of any backchannels, it would be all but impossible to do so without the rapid exchange of spoken utterances. Additionally, any lengthy transitions containing overlapping speech or between-speaker silence are not only likely to be perceived as awkward or unusual, but would also reduce the efficiency of the communicative exchange at a purely functional level. Thus, fast and effective turn-timing does not require a decidedly social motivation in the same way that frequent backchannelling does.

\subsection{Initial differences as a reflection of effortful accommodation}\label{Conclusion_discussion_stages}

Social motivations aside, achieving the speed of turn-timing found in typical conversations between adult native speakers remains a formidable challenge and requires complex social skills, such as the accurate prediction of an interlocutor's behaviour. Predicting others becomes easier the more familiar interlocutors are with each other and the more two (or more) speakers establish a shared conversational rhythm in the sense of convergence or accommodation. I have  speculatively interpreted the observation that ASD dyads take considerably longer to achieve typically rapid turn-timing as signifying a delay in the establishment of a shared rhythm and a concordantly high degree of convergence between interlocutors.

A similar effect was observed for backchannelling. The rate of backchannels goes up rather steeply for most ASD dyads as conversations progress and thereby becomes much more similar to the values typically produced by CTR dyads. Thus, dyads from the ASD group often arrived at behaviour comparable to that of the CTR group after the first few minutes of conversation, suggesting that it was only a matter of time for autistic dyads to reach the level of coordination that is typical for conversation in the CTR group. This delay could be an indication that superficially equivalent behaviour between groups at a global level may obscure the fact that arriving at these behaviours may be more effortful for some autistic individuals. In this light, one reason why conversations in the ASD group were comparatively short and to the point -- besides any potential lack of social motivation -- would simply be increased cognitive effort.

Taking different stages of conversation into account has thus yielded some valuable insights that would otherwise have been overlooked. Paying special attention to the early stages of dialogue is particularly important since previous research has reliably shown that personality judgements and character attributions are disproportionately influenced by the first minutes and even seconds of a social interaction. This implies that although the behaviour of autistic dyads might be equivalent to that of non-autistic dyads for the majority of a conversation, diverging behaviour in the early stages of conversation may nevertheless leave indelible impressions of an unconventional communication style.

\subsection{Intonation as a global and local feature of speech}\label{Conclusion_discussion_intonation}

Differences between early and later stages of dialogue were not found for all areas of investigation. Intonation styles in particular are noteworthy for having remained stable throughout the task for almost all speakers. This suggests that intonation styles are a global property of speakers and largely robust to external factors such as context and content. It is therefore particularly relevant that a clear indication for only more melodic (or singsongy), but not more monotonous (or robotic) intonation in ASD dyads was found, adding to the mounting evidence that a melodic intonation style is characteristic of speech in ASD.

Prosodic aspects of speech and differences between groups were investigated not only at the global level of intonation styles, but also as part of the local realisation of backchannels and filled pauses (as well as turn ends and beginnings). For backchannels, it was found that the mapping of intonation contours to different lexical types of backchannel was less complex in the ASD group than in the CTR group. In essence, many ASD dyads showed a preference for rising intonation contours on all lexical types of BC, whereas most CTR dyads evinced a more specific probabilistic mapping of intonation contour to BC type. For filled pauses (the rate of which was identical across groups), realisations by ASD dyads deviated more from the expected level contour than those by CTR dyads.

Accurately employing prosody in the production of backchannels and filled pauses requires an acute understanding of the rather subtle ways in which not only one class of discourse marker (BC) differs from the other (FP), but also of the commonly preferred contours for different types \emph{within} the class of backchannels. The reduced degree of complexity shown by most ASD dyads in the prosodic realisation of BCs might conceivably be linked to the typically less restrained use of intonation at the global level, manifesting as a more melodic intonation style.

\subsection{The autistic sample as a filter on the spectrum}\label{Conclusion_discussion_HFA}

It is important to remember that all the differences between groups that have been described and all the characteristics of conversation in ASD that have  been inferred from these differences are based on a very specific and limited sample. Not only were the participants in the ASD group German-speaking, mostly male and considerably older than the average experimental subject in linguistics and psychology. Most importantly, they were far from representative of the autism spectrum as a whole. Through a largely implicit selection procedure, participants were required to be 1) verbal, 2) willing and able to visit an outpatient clinic, 3) of average or above-average intelligence and 4) willing and able to take part in an unfamiliar experiment (in an unfamiliar location and wearing head-mounted microphones). These requirements act as a narrow-band filter, leaving us with behavioural data from only one peripheral region within the entire autism spectrum.

As this places the experimental subjects in the ASD group very close to the point where individuals with and without a diagnosis of ASD (the latter represented by the CTR group in the corpus under investigation) are most likely to overlap, it is perhaps all the more remarkable that such varied differences be\-tween groups were found. Had the CTR group consisted of subjects from the stereotypical linguistics subject pool of female undergraduate students, even more or clearer differences between groups may well have been found, but it would not have been possible to attribute them specifically to a difference in autistic traits. Given that the CTR group was instead carefully matched for gender, age and verbal IQ, the patterns of behaviour that were identified as typical for the ASD group are likely to indeed be specific to an autistic communication style. We have to keep in mind, however, that the small sample size of 14 subjects, while relatively high compared to other studies on autistic communication, necessarily limits external validity.

Studying individuals from one narrow band along the autism spectrum crucially also entails not being able to make predictions about individuals from the rest of the spectrum. It is worth stating explicitly that the findings in this work cannot be expected to generalise to the majority of autistic people, many of whom might have been unwilling or unable to take part in the recordings and complete the experimental task. Even though terms such as ``high-functioning autism'' or even ``Asperger syndrome'' have, with some justification, fallen out of favour and indeed use in recent years (see \sectref{autism-spectrum}), the findings described in this book may most accurately represent the behaviour of adults matching just those descriptions. It bears repeating that I cannot and do not wish to make any claims regarding children and/or individuals on other parts of the autism spectrum on the basis of the work presented here, nor do I claim that the methodology used would be suitable for the study of communication in these groups.

\subsection{Bilingual and cross-cultural conversation as a valuable analogy}\label{Conclusion_discussion_L2}
\largerpage
Partly due to a lack of previous research into autistic communication, I have used results from research on second language learners and bilinguals to contextualise results from the group of autistic subjects under study in various parts of this book. I have tried to make it sufficiently clear that there are crucial differences between autistic and non-native speakers and that I focus on similarities between the two groups as an approximate analogy only (see e.g.~\sectref{autism-spectrum}). I do believe, however, that much could be gained in theory and practice from connecting the two fields of study.

Communication between autistic and non-autistic individuals could reasonably be seen as an analogue of cross-cultural communication. Interestingly, individuals with ASD often feel more at ease in cross-cultural social contexts, as in such situations, communicative difficulties are usually expected and tend not to be ascribed to a failing on the part of only one interlocutor \citep{hillaryNeurodiversityCrossculturalCommunication2020}. This relates directly to the concept of the double empathy problem (discussed in several parts of the book), i.e. essentially the idea that difficulties in communication between autistic and non-autistic speakers are mainly due to divergent social dispositions. This again entails that difficulties in communication are created in the interaction between two interlocutors, an idea closely related to that of a shared responsibility between native and non-native speakers proposed in e.g. \citet{derwingPuttingAccentIts2009}. Following this line of thought, there is a strong case to be made for putting the onus on non-autistic individuals to make every effort to understand and accommodate autistic styles of communication and cognition \citep[cf.][]{mccrackenAutisticIdentityLanguage2021}. 

To be able to do so, we of course first need to gain a much more accurate idea of what can be considered as truly autistic styles of conversation and intonation, which will require many more studies of disposition-matched autistic communication to be conducted in future. The present book represents my contribution to this effort. Specific results described here could also be used to fine-tune the analogy of autistic with non-native or bilingual speech. For instance, the current findings on intonation style, turn-taking and backchannelling much more closely resemble previous findings on bilinguals or \textit{near-native} learners \citep{soraceNearnativeness2003} than findings on beginner learners, as might be expected given the specific sample of the autistic population, that is, individuals from the more socially motivated and skilled end of the spectrum. 

Finally, I will point out the great potential for adapting research, teaching and training materials from second-language instruction and inter-cultural training to the study of communication between autistic and non-autistic people. The more we are able to pinpoint what characterises communication in ASD, the more we will be able to tap into the vast array of relevant resources and adapt them for the benefit of anyone who is engaged or merely interested in understanding and facilitating communication between autistic and non-autistic individuals.












\section{Outlook}\label{sec:Conclusion_conclusion}

I will close by summarising some of the most important projected extensions of the work described in this book.

First, follow-up production studies which are already under way include video recordings, making it possible to examine the contributions of gesture and gaze to conversational strategies and their interplay with the spoken modality \citep[see the pilot study in][]{spaniolMultimodalSignallingInterplay2023}.

Second, experimental settings have been extended to include fully free (but also highly structured) conversations \citep{spaniolMultimodalSignallingInterplay2023}.

Third, mixed dyads (ASD--CTR) should be included to compare the relevant interactions with what has been described here for disposition-matched (ASD--ASD/CTR--CTR) dyads. This could involve the analysis of triadic rather than dyadic interaction, which incidentally facilitates tracking the contributions of individual speakers to social interaction.  A comparison with conversation and intonation in persons with schizophrenia furthermore holds promise for future investigations \citep[cf.][]{lucariniConversationalMetricsPsychopathological2021,cangemiContentfreeSpeechActivity2023,howesDisfluenciesDialoguesPatients2017}. A comparison of autistic individuals from different age ranges or ideally even longitudinal observations throughout development would be highly valuable, if logistically challenging extensions.

\hspace*{-.9pt}Fourth, improvements can be made on specific methods of analysis used, e.g.~by including periodic energy or polynomial modelling in the analysis of prosody and by considering convergence continuously across the time course of conversations.

Fifth, perception studies can help us to critically examine and further refine the measures and findings described in this work. For instance, experiments can verify how closely the analysis of intonation styles proposed in this work matches listener impressions \citep[see a first validation in][]{wehrleEvaluatingProsodicAspects2023} and shed light on how conversational strategies specific to the ASD group are perceived and judged by both autistic and non-autistic listeners.

Finally, qualitative analyses and insights from autistic adults will be used to shape the interpretation of current and future results. The insights and experiences shared by the participants of the FORAUS discussion forum for autistic adults in Cologne, for instance, have been invaluable for understanding and contextualising the quantitative analyses reported throughout this book. I will continue the exchange with autistic informants and advocates to ensure that the lived experiences of autistic people are represented appropriately, and to inform future efforts of raising broader societal awareness of communication in ASD.

Once the findings presented in this book have been subject to further critical examination and, ideally, replication efforts in future studies, we will be better able to assess their relevance and suitability for applications in, for instance, training and diagnosis. For the time being, it is my hope that this work has added to our general understanding of conversation and intonation in ASD by shedding light on some of the most important underlying dimensions and mechanisms. Ultimately, I hope that the findings presented in this work might help to contribute to an acceptance of neurodiversity in highlighting ASD-specific communication strategies and their potential relevance for cross-neurotype communication.
