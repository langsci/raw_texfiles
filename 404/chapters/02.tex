	\chapter{Data and methods} \label{sec:data}
	
	In the following, I will provide details on the subjects that participated in this study as well as on experimental methods and materials. Further, I will present the chosen approach of combining in-depth exploratory analysis with Bayesian modelling.
	
	\section{Participants}\label{participants}

For the corpus used throughout this book, 28 monolingual native speakers of German (14 ASD, 14 CTR) were recorded performing Map Tasks (see \sectref{materials}) in homogeneous, disposition-matched dyads (7 ASD--ASD, 7 CTR--CTR). Participants from the ASD group had all been diagnosed with Asperger syndrome (ICD-10: F84.5) and were recruited in the Autism Outpatient Clinic at the Department of Psychiatry, University of Cologne (Germany). As part of a systematic assessment implemented in the clinic, diagnoses were made independently by two different specialised clinicians corresponding to ICD-10 criteria and supplemented by an extensive neuropsychological assessment.

Subjects from the ASD group were first recorded and described by \citet{krugerProsodicDecodingEncoding2018} and \citet{krugerProsodicMarkingInformation2018} (performing different tasks). Participants from the control group were recruited from the general population specifically for this study. All subjects were paid 10 EUR for participation. It was ascertained that participants had not been acquainted with each other before the start of the experiment (although some participants in the ASD group may have crossed paths in the context of the autism outpatient clinic).

Disposition-matched dyads (ASD--ASD; CTR--CTR) rather than mixed dyads (ASD--CTR) were recorded for two main reasons. First, there is a dramatic lack of research on communication in ASD based on data from matched rather than mixed dyads. Second, investigating the behaviour of disposition-matched dyads seems to us the most promising way to gain insights into what we might justifiably call autistic communication. Analysing the behaviour of mixed dyads makes it very difficult to see beyond patterns arising from the divergent behaviour of individuals with different cognitive styles. While such insights are of great value in principle, they cannot be interpreted accurately unless we first have a clear picture of what characterises communication in disposition-matched autistic dyads (see e.g.  \sectref{turntaking_conclusion_discussion_limitations} for further discussion). Indeed, recent research suggests that many social difficulties experienced by people on the autism spectrum might, in fact, be due to neurotype mismatches (arising in interactions with non-autistic people) rather than any inherent cognitive ``deficits" or ``impairments" \citep{cromptonNeurotypeMatchingNotBeing2020, morrisonOutcomesRealworldSocial2020, rifaiInvestigatingMarkersRapport2022}. This perspective reflects a growing (and perhaps overdue) broader awareness that analyses of interaction, rather than of isolated minds, should be at the core of cognitive science, linguistics and related disciplines \citep{dingemanseSingleMindednessFigureGroundReversal2023}.

All participants completed the German version of the Autism-Spectrum Quotient (AQ) questionnaire, an instrument developed by \citet{baron-cohenAutismspectrumQuotientAQ2001} to measure autistic traits in adults with ``normal" intelligence. AQ scores range from 0 to 50, with higher scores indicating more autistic traits. An AQ score of 32 or above is commonly interpreted as a clinical threshold for ASD \citep{ashwoodPredictingDiagnosisAutism2016, baron-cohenAutismspectrumQuotientAQ2001}.

All participants also completed the `Wortschatztest WST'  \citep{schmidtWortschatztestWSTBeltz1992}, a standardised, receptive German vocabulary test that exhibits a high correlation not only with verbal intelligence, but also with general intelligence \citep{satzgerLiefernHAWIERWST2002}.

Although participants from the CTR group were matched as closely as possible to the ASD group for age, verbal IQ and gender, some minor differences remained.

Participants from the ASD group were on average slightly older (mean = 44; range: 31--55) than participants from the CTR group (mean = 37; range: 29--54). However, there was extensive overlap between groups and, moreover, there is no a priori reason to assume that such a relatively small difference in this particular age range would act as a confound in group comparisons. Bayesian modelling confirmed the age difference between groups as a robust effect (with the ASD group as the reference level: mean \(\delta\) = -7.12; 95\% CI {[}-11.06, -3.22{]}; posterior probability \(P(\delta > 0)\) = 1). More information on the use of Bayesian modelling in this book can be found in \sectref{analysis}; details on the specific models used here are found in the accompanying scripts and files (see the \emph{Open Science Framework (OSF)} repository at \url{https://osf.io/6vynj/}).

Further, the ASD group had a slightly higher average verbal IQ score (mean = 118; range: 101--143) than the CTR group (mean = 106; range: 99--118). Again, there was considerable overlap between groups. There is no reason to assume that this difference should have a meaningful impact on results. Bayesian modelling confirmed the difference in verbal IQ as a robust effect (with the ASD group as the reference level: mean \(\delta\) = -12.31; 95\% CI {[}-18.7, -5.67{]}; posterior probability \(P(\delta > 0)\) = 1).

The gender ratio was similar, but not identical across groups. The ASD group contained 4 females and 10 males, whereas the CTR group contained 3 females and 11 males. This entails that dialogues took place in the ASD group between 1 all-female dyad, 2 mixed dyads and 4 all-male dyads, but in the CTR group between 3 mixed dyads and 4 all-male dyads (i.e.~no all-female dyad). Bayesian modelling confirms that these small differences between groups did not, however, have any notable effects on results in any of the areas under investigation. Details can be found in the relevant chapters.

Most importantly, there was a clear difference in AQ scores between groups, with a far higher average score in the ASD group (mean = 41.9; range = 35--46) than in the CTR group (mean = 16.1; range: 11--26) and no overlap at all between subjects from both groups. All subjects in the ASD group scored above the suggested threshold of 32 points and all subjects in the CTR group scored below the same threshold. Bayesian modelling provides unambiguous evidence for the group difference in AQ scores (with the ASD group as the reference level: mean \(\delta\) = -25.83; 95\% CI {[}-29.03, -22.67{]}; posterior probability \(P(\delta > 0)\) = 1).

Table \ref{tab:SubjectTable} shows summary statistics for gender, age, verbal IQ and AQ.



\begin{table}

\begin{center}
		\caption{\label{tab:SubjectTable}Subject information by group. SD = Standard deviation.}
		
		\begin{tabular}{lllllllll}
			\lsptoprule
			& \multicolumn{2}{c}{Gender (n)} & \multicolumn{2}{c}{Age} & \multicolumn{2}{c}{Verbal IQ} & \multicolumn{2}{c}{AQ} \\
			\cmidrule(r){2-3} \cmidrule(r){4-5} \cmidrule(r){6-7} \cmidrule(r){8-9}
			& female & male & Mean & SD & Mean & SD & Mean & SD\\
			\midrule
			ASD & 4 & 10 & 43.6 & 6.7 & 118.1 & 12.0 & 41.9 & 3.1\\
			CTR & 3 & 11 & 36.5 & 7.6 & 105.8 & 5.8 & 16.1 & 4.5\\
			\lspbottomrule
		\end{tabular}
\end{center}

\end{table}

All aspects of the study were approved by the local ethics committee of the Medical Faculty at the University of Cologne and were performed in accordance with the ethical standards laid down in the 1964 Declaration of Helsinki and its later amendments. All participants gave their written informed consent prior to participating in the experiment.


\section{Materials and procedure}\label{materials}

Map Tasks were used to elicit semi-spontaneous speech. The Map Task paradigm was introduced by \citet{andersonTeachingTalkStrategies1984} and has widely been used in speech research for over 30 years \citep[for an influential article describing a corpus of Map Task speech see][]{andersonHCRCMapTask1991}.

The Map Task paradigm was chosen for the current investigation as it provides us with predominantly spontaneous speech data that can, however, still be controlled along a number of key parameters, such as lexical items (via the names of landmarks on a map) and communicative obstacles (such as the introduction of mismatching landmarks between maps; see below for more detail). While the elicited dialogues are not fully free or spontaneous, the Map Task was determined to be a good choice in the context of comparing autistic and non-autistic dyads, since the constraints involved in the task serve to reduce a potentially particularly high degree of variability across the autism spectrum in terms of social motivation, interest in a given topic, and the adherence to social conventions.

Participants were recorded in pairs (dyads). After filling in a number of forms and the questionnaires listed in \sectref{participants}, participants received written instructions for the task and entered a recording booth. Each participant was presented with a simple map containing nine landmark items in the form of small pictures \citep[materials adapted from][]{griceMapTasksItalian2003}. Only one of the two participants (the instruction giver) had a route printed on their map. The experimental task was for the instruction follower to transfer this route to their own map by exchanging information with the instruction giver.

During this entire process, an opaque screen was placed between participants, meaning they could not establish visual contact and had to solve the task by means of verbal communication alone. The roles of instruction giver and instruction follower were assigned randomly. Upon completion of the first task, the participants received a new set of maps and their roles were switched. The task ended once the second Map Task was completed.

As participants were naive to the purpose of the study, they did not know (initially) that their maps differed in some crucial regards. In each map, some landmarks were either missing, duplicated and/or replaced with a different landmark compared to the interlocutor's map. This was the case for two landmarks per map in the experiment. Those items that differed between maps will hereafter be called Mismatches (or mismatching landmarks); items that were the same on both maps will be called Matches (or matching landmarks).

During annotation, the portion of dialogue in which the first Mismatch was discussed by participants was marked and this was used to divide all dialogues up into three epochs, i.e., before detection, during discussion, and after resolution of the first Mismatch. This was expanded to a continuous analysis or reduced to a binary distinction as appropriate. Further details can be found in the discussion of relevant findings in the following chapters (e.g.~\sectref{int_results_stage}, \sectref{turntaking_results_FTO_group_stage} and \sectref{BCFP_BC_results_BCRate_Stage}).

An example of maps used in this study is shown in Figure \ref{fig:MapTaskExample}, with Mismatches highlighted using red circles. All dyads received the same two pairs of maps.

\begin{figure}
	
	{\centering \includegraphics[width=1\linewidth]{images/Map Example} 
		
	}
	
	\caption{One pair of maps used in the study. The instruction giver's map, with a route leading from `Start‘ (top left) to `Ziel‘ (finish; bottom left), is in the left panel. Mismatches between maps are highlighted with red circles.}\label{fig:MapTaskExample}
\end{figure}


Map Task conversations were recorded in a sound-proof booth at the Department of Phonetics, University of Cologne. Two head-mounted microphones (AKG C420L) connected through an audio-interface (PreSonus AudioBox 22VSL) to a PC running Adobe Audition were used. The sample rate was 44100 Hz (16 bit). Recordings were transcribed orthographically and divided into inter-pausal units (IPUs) with a minimum pause length of 200 ms \citep{dejongChoosingThresholdSilent2013,goldman-eislerPsycholinguisticsExperimentsSpontaneous1968,choContributionSilentPauses2006}.

Only recorded dialogue from the start to the end of each task was included in all analyses in order to achieve a greater degree of comparability regarding conversational context and content. The total duration of all edited dialogues was 4 hours and 44 minutes. The mean dialogue duration was 20 minutes and 19 seconds (SD = 12'32''; for detailed information and analysis see \sectref{turntaking_results_signal_signal}). Note that far less time would have been necessary to simply complete the task at hand for most dyads. A qualitative analysis confirmed that most participants engaged in a very free mode of conversation, rather than strictly working through the two Map Tasks -- although there were some intriguing group differences in this regard, with ASD dyads seeming to lean more towards a task-oriented style of conversation (see results in the following chapters for more details).

Figure \ref{fig:MapTaskGAT} shows an example excerpt of Map Task dialogue from one of the ASD dyads, transcribed following GAT conventions \citep[see][]{couper-kuhlenSystemTranscribingTalkininteraction2011}. Phenomena that are of particular interest for the following analyses are highlighted in bold: two backchannels, one filled pause and two turn transitions (one following the introduction of a matching landmark -- `heller Diamant' (bright diamond), line 15/16 -- and one following the introduction of a mismatching landmark -- `goldene Moschee' (golden mosque), line 21/22). Note that the turn transitions highlighted here are considerably longer than average transitions between turns (cf.~\chapref{turntaking}).



\begin{figure}
	
	{\centering \includegraphics[width=0.92\linewidth]{images/GAT_pdf}
		
	}
	
	\caption{Example excerpt of a GAT transcription, with backchannels, filled pauses, and turn transitions following newly introduced landmarks highlighted in bold.}\label{fig:MapTaskGAT}
\end{figure}




	\section{Principles of analysis}\label{analysis}

This section will give details on the general principles and methods of analysis used throughout this book. Details applicable to specific measurements can be found in the relevant subchapters.


	\subsection{The importance of individual specificity}\label{analysis_individual}


One of the guiding principles in this work is a commitment to in-depth analyses appropriately accounting for inter-individual variability and dyad-specific behaviour \citep[cf.][]{bruggemanUnifyingSpeakerVariability2017, cangemiSpeakerspecificIntonationalMarking2016, cangemiListenerspecificPerceptionSpeakerspecific2015}. The importance of considering scientific data at the level of the individual (and the dyad) is not limited to this study, nor to the fields of linguistics and psychology. It is, however, made all the more critical when we aim to describe and understand the behaviour of a group of speakers as intrinsically heterogeneous as any group composed of individuals diagnosed with ASD. This point is taken up again throughout the book (see in particular \sectref{Conclusion_SummaryAnalysis_patterns} and \sectref{Conclusion_discussion_individuals}).

A large number of findings have shown evidence for a particularly high degree of heterogeneity in groups of individuals diagnosed with ASD \citep[e.g.][]{wozniakDevelopmentAutismSpectrum2017}. This heterogeneity is at the very core of what is by definition a spectrum disorder with a continuous distribution of features and properties \citep{americanpsychiatricassociationDiagnosticStatisticalManual2013}. One underlying reason for this heterogeneity is the fact that autistic people can be expected to adapt less than their non-autistic peers to certain specific and shifting cultural and linguistic conventions at any one point in time. We cannot always disentangle whether this might be due to a lack of interest or ability in specific cases, but the effect is that idiosyncratic aspects of speech and communication in ASD are likely to be magnified when held up to the current conventions of the non-autistic mainstream.

While individual-specific analysis can be seen an asset for any study of human behaviour and is particularly relevant for investigations into ASD, it becomes nothing less than a necessity when we are additionally faced with relatively small sample sizes, as has been the case in the vast majority of studies on communication in ASD. In this area of research, it is very unlikely that any individual study will reach the minimum sample size of 100 participants that has recently been claimed to be a requirement for achieving adequate statistical power (within conventional statistical frameworks) \citep{brysbaertPowerConsiderationsBilingualism2020}. While this claim is based on models from the related field of bilingualism research, it can easily be applied to autism research as well.

The problem is in fact only more acute in the case of ASD. There are dramatically fewer autistic people in the world (around 1\% of the general population) than there are bilinguals (billions). In this light, openly and deliberately conducting exploratory studies using descriptive analyses, ideally supported by Bayesian methods of statistical inference, seems to me the only reasonable and responsible course of action \citep{grieveObservationExperimentationReplication2021, tukeyWeNeedBoth1980, vasishthStatisticalSignificanceFilter2018, yarkoniGeneralizabilityCrisis2022}. Certainly, a single-minded pursuit of sufficiently diminutive p-values in the conventional framework of null-hypothesis significance testing cannot be the solution to this particular problem. The next section spells out some of the issues surrounding the conventional use of frequentist statistical inference and how suggestions for how they may be overcome through a combination of openly exploratory analyses and Bayesian modelling.

\subsection{In-depth exploration with Bayesian foundations}\label{analysis_Bayesian}


In reporting experimental results, I emphasise a fully transparent and visually rich descriptive analysis combined with applications of Bayesian modelling and inference. I aim to provide a comprehensive understanding of results first through detailed description and the extensive use of data visualisation \citep{anscombeGraphsStatisticalAnalysis1973, matejkaSameStatsDifferent2017}. Bayesian inference is used in the spirit of complementing, not superseding the descriptive, exploratory analysis that I consider to be at the heart of this work. Therefore, not all details of Bayesian modelling are reported for all analyses, but all information can be found in the accompanying \textit{OSF} repository (see below).

Given the severe lack of reliable previous research and relevant pilot data concerning the phenomena of interest in this book, the analyses  presented are necessarily exploratory rather than confirmatory in nature. In this situation, formally testing previously formulated hypotheses using frequentist methods would inherently involve an increased risk of disseminating spurious results based on Type I errors.

Although frequentist inference is still the dominant approach to statistical analysis across different scientific fields, the use of this framework, along with a predominant focus on statistical significance and confirmatory rather than exploratory studies, is associated with a number of grave and wide-ranging issues. These are often summarised under the term \textit{questionable research practices} and go far beyond the specifics of this book. The reader is referred to the growing literature that has been persuasively describing this set of problems as well as the underlying causes and suggestions for possible solutions \citep[e.g.][]{smaldinoNaturalSelectionBad2016,corettaMultidimensionalSignalsAnalytic2023,amrheinScientistsRiseStatistical2019, bishopReinFourHorsemen2019, headExtentConsequencesPhacking2015, johnMeasuringPrevalenceQuestionable2012, roettgerEmergentDataAnalysis2019, roettgerResearcherDegreesFreedom2019}.

While Bayesian inference does not in itself prevent the use of such questionable practices, it is an ideal alternative for two main reasons. First, given the limited sample size of the study at hand as well as the lack of previous research on the topic, I have deemed presenting the current results and analyses as exploratory, rather than confirmatory, as the only justifiable option (as discussed in the preceding section). I believe that it would be greatly advantageous for more research in linguistics and related fields to take this approach, rather than presenting as confirmatory work that truly is not \citep{kerrHARKingHypothesizingResults1998, murphyHARKingHowBadly2019}. Bayesian inference is particularly well suited to studies with a limited sample size, as this limitation can be directly reflected in the model output (e.g.~in the form of larger credible intervals and a lower posterior probability).

Bayesian inference gives outcomes based on the data at hand, the chosen model and the specified prior assumptions. Compared to frequentist inference, it is therefore, when properly applied, more conservative, but also more robust and transparent in ways that frequentist approaches never are and indeed cannot be, partly because they implicitly treat any given experiment as one in an infinite series of equivalent experiments \citep{gelmanBayesianWorkflow2020, lemoineMovingNoninformativePriors2019, mcelreathStatisticalRethinkingBayesian2020, winterPoissonRegressionLinguists2021a}.

Second, Bayesian inference is rapidly increasing in popularity in linguistics and many other fields. This is due in part to practical reasons. Statistical software, tutorials and packages such as the ones used in this book (detailed in the following paragraph) have made the application of Bayesian multilevel modelling increasingly straightforward and at the same time considerably more robust and flexible than the frequentist alternatives \citep{eagerMixedEffectsModels2017}. Additionally, Bayesian methods seem to be much more closely aligned with common human intuitions and ways of reasoning about the interpretation of statistical tests in general and the notion of significance in particular \citep{dienesBayesianOrthodoxStatistics2011, mcshaneStatisticalSignificanceDichotomization2017, winterPoissonRegressionLinguists2021a}.

I used Bayesian multilevel linear models implemented in the modelling language \emph{Stan} \citep{carpenterStanProbabilisticProgramming2017} via the package \emph{brms} for the statistical computing language \emph{R}, which was used in the software \emph{RStudio} \citep{burknerBrmsPackageBayesian2017, rcoreteamLanguageEnvironmentStatistical2022, rstudioteamRStudioIntegratedDevelopment2021}.

Analysis and presentation of Bayesian modelling broadly follows the example of \citet{frankeBayesianRegressionModeling2019}, but is also informed by a number of other tutorials \citep{mcelreathStatisticalRethinkingBayesian2020, vasishthBayesianDataAnalysis2018, winterPoissonRegressionLinguists2021a}.

Expected values (\(\hat{\beta}\)) under the posterior distribution and their 95\% credible intervals (CIs) are reported, along with the posterior probability that a difference \(\delta\) is greater than zero. In essence, a 95\% CI represents the range within which an effect is expected to fall with a probability of 95\%. Analyses in this book loosely follow the guideline that, if a hypothesis states that \(\delta > 0\), there is (strong) support for this hypothesis if zero is (by a reasonably clear margin) not included in the 95\% CI of \(\delta\) and the posterior \(P(\delta > 0)\) is close to one \citep[cf.][]{frankeBayesianRegressionModeling2019}. I use this guideline primarily to ensure comparability with conventional null-hypothesis significance testing and reporting practices, but consider 95\% credible intervals in and of themselves as the most relevant outcome of Bayesian modelling.

Regularising weakly informative priors were used for all models \citep{lemoineMovingNoninformativePriors2019}. Unless otherwise specified, four sampling chains ran for 4000 iterations with a warm-up period of 2000 iterations for each model, thereby yielding 4000 samples for each parameter tuple. Further details of all Bayesian models and their output can be found in the relevant sections of the respective scripts.

Besides the packages for Bayesian modelling, I made extensive use of the packages included in the \emph{tidyverse} collection for performing data import, tidying, manipulation, visualisation, and programming in this book \citep{wickhamWelcomeTidyverse2019a}.\footnote{The complete list of R packages used for analysis and visualisation is: \emph{bayesplot} \citep[Version 1.8.1;][]{gabryBayesplot2022}, \emph{brms} \citep[Version 2.15.0;][]{burknerBrmsPackageBayesian2017}, \emph{cowplot} \citep[Version 1.1.1;][]{wilke2019cowplot}, \emph{effsize} \citep[Version 0.8.1;][]{torchiano2020effsize}, \emph{ggridges} \citep[Version 0.5.3;][]{wilke2021ggridges}, \emph{tidybayes} \citep[Version 3.0.0;][]{kay2023tidy}, \emph{tidyverse} \citep[Version 1.3.1;][]{wickhamWelcomeTidyverse2019a}, and \emph{viridis} \citep[Version 0.6.1;][]{garnier2023viridis}.}

The original manuscript of this book was written within \emph{RStudio} in RMarkdown format \citep{allaire2020rmarkdown} using the package \emph{papaja} \citep{aust2020papaja}. One of the key advantages of this approach is that all code and plain text is available within one single file for each chapter of the book and can easily be accessed and examined. All accompanying files, including raw data and RMarkdown files containing code and manuscripts, can be found in the \textit{OSF} repository at \url{https://osf.io/6vynj/} (and other repositories, as specified in the relevant chapters).
