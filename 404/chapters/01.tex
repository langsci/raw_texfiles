\chapter{General introduction}\label{sec:introduction}

The overall aim in this book is to analyse the conversation strategies and intonation styles of German adults with and without a diagnosis of autism spectrum disorder (ASD) in order to arrive at a better characterisation of communicative behaviour in ASD. I provide an in-depth, multi-dimensional analysis focussing on the dimensions of intonation style, turn-taking, backchannels, filled pauses and silent pauses (along other parameters). Speakers engaged in semi-structured spontaneous conversation were recorded in two
groups of disposition-matched speaker pairs (i.e. interlocutors either both did or did not have a diagnosis of ASD).

In this first chapter, I will give a very brief overview of ASD in general and communication in ASD in particular before providing an outline of the book and anticipating some of the most important findings.

\section{The autism spectrum: Overview and terminology}\label{autism-spectrum}

The leading diagnostic manuals DSM-5 and ICD-11 describe autism spectrum disorder as a neurodevelopmental disorder that is characterised by ``deficits in social interaction and communication" as well as ``repetitive, restricted behaviours and interests" \citep{americanpsychiatricassociationDiagnosticStatisticalManual2013, worldhealthorganizationICD11InternationalStatistical2022}. The estimated prevalence of ASD is around 1\% \citep{christensenPrevalenceCharacteristicsAutism2018, elsabbaghGlobalPrevalenceAutism2012}.\footnote{Please note that while I explicitly do not follow a deficit-based view of autism in this work, I may make reference to such views in referring to the current diagnostic criteria and to descriptions in the previous literature.}

In the most recent classifications, the previously used subgroups of \emph{Asperger syndrome} and \emph{high-functioning autism} have been subsumed under the single category of autism spectrum disorder. Asperger syndrome as defined in ICD-10 (F84.5) refers to individuals with an IQ of over 70 and no delays in language acquisition and cognitive development. High-functioning autism is a term that is not actually included in either the DSM-5 or the ICD-10, but has commonly been used to refer to autistic individuals with an average or above-average IQ who, in contrast to individuals with Asperger syndrome, did experience a delay in language acquisition \citep[see][Chapter 4]{krugerProsodicDecodingEncoding2018}. As most research suggests that a reliable differentiation between autism spectrum disorder and the proposed subcategories of high-functioning autism and Asperger syndrome is indeed not possible \citep{frazierValidationProposedDSM52012, lordMultisiteStudyClinical2012}, I will be referring only to the overarching category of ASD throughout this book (even though the ICD-10 diagnosis F84.5 (Asperger syndrome) was applied at the time of diagnosis for all autistic participants in the corpus under study). 

Differences in communicative behaviour are a core characteristic of ASD, and the one that is most relevant to the work presented in this book. Although, typically for ASD, individual differences abound, some overarching trends in the use of gaze, gesture and language in ASD have been identified. Very broadly speaking, these include more differences in the social, rather than functional aspects of language \citep[see][]{krugerProsodicDecodingEncoding2018}, characteristic patterns in both the production and perception of prosody \citep[e.g.][]{mccannProsodyAutismSpectrum2003a, paulPerceptionProductionProsody2005, griceLinguisticProsodyAutism2023} and a more literal (rather than figurative) use and understanding of language \citep[e.g.][]{happeUnderstandingMindsMetaphors1995}.

It is important to note that many of these findings have been made predominantly or exclusively on the basis of data from children and adolescents \citep[see][]{krugerProsodicDecodingEncoding2018, griceLinguisticProsodyAutism2023}. As we know 1) that language skills often improve throughout early life in ASD \citep{gernsbacherLanguageSpeechAutism2016} and 2) that there is a general lack of research into communication by autistic adults, it is not always clear to what extent such findings apply to adults with ASD (including the speakers in the corpus analysed here).

I will provide detailed accounts of the aspects of communication in ASD that are most relevant to the findings presented in this book in the following chapters, along with the relevant experimental results.

Regarding terminology, I refer to e.g. \emph{autistic individuals} or people \emph{on the autism spectrum} rather than to \emph{individuals with ASD} in this book. In other words, I have chosen to use identity-first rather than person-first language. Although there have increasingly been calls for an exclusive use of identity-first language (\emph{autistic person}) in recent years, there is no complete consensus on the matter \citep{bothaDoesLanguageMatter2021,bottema-beutelAvoidingAbleistLanguage2021,dunnPersonfirstIdentityfirstLanguage2015,gernsbacherEditorialPerspectiveUse2017,vivantiAskEditorWhat2020,tepestMeaningDiagnosisDifferent2021}. I acknowledge this ambiguity and hope that those who prefer the use of person-first language can see past these matters of terminology and still benefit from the insights put forward in this book.

On a similar note, it is worth pointing out that I will draw on research into bilingual or second-language communication as a point of comparison with communication in ASD in some parts of this book. This is done mainly due to a considerable, or in some cases even complete, lack of previous research on relevant aspects of communication in ASD. The comparison with second-language speech strikes me as fruitful and well-motivated in many respects, but I am of course acutely aware of the crucial differences between non-native speakers and autistic persons in terms of developmental trajectories and neurobiology (among others). It is simply my hope that this comparison can elucidate some phenotypical similarities between the two groups and that it may even be possible to transfer some of the knowledge and resources from the well-established research fields of bilingual and cross-cultural communication to the benefit of research on autistic and cross-neurotype communication.

Having established these basic concepts and some important terminological choices, I will proceed to give an overview of the data and methods used in \chapref{sec:data} following the outline of the remaining parts of this book presented in the next section.

\section{Synopsis: Dimensions of conversation and intonation in ASD}\label{synopsis}
	
	In \chapref{intonationstyle}, \emph{intonation style} is investigated. Since the very beginnings of research into ASD, there have been contradicting descriptions of speech in ASD as being either particularly melodic or particularly monotonous. A novel methodology, designed to avoid shortcomings of previous acoustic analyses, was used to reliably quantify intonation styles in ASD. It is shown that ASD speakers in the corpus under study tended to produce a more melodic intonation style than non-autistic control (CTR) speakers, while none produced a more monotonous intonation style. It is further shown that the proposed method for quantifying intonation styles is at least equivalent to previous efforts relying on parameters such as pitch range and span and superior to accounts relying solely on mean fundamental frequency.
	
	\chapref{turntaking} is dedicated to an analysis of \emph {turn-taking}. The organisation of who speaks when in conversation is perhaps the most fundamental aspect of human communication. Previous research on a wide variety of speakers has revealed a seemingly universal preference for between-speaker transitions consisting of very short silent gaps. Research on turn-taking in ASD is very limited to date, and no studies have investigated dialogues between autistic adults. It is shown that turn-timing was very similar in the CTR and the ASD group overall, but also that autistic dyads produced unusually long silent gaps in the early stages of dialogue. Further evidence reveals that ASD dyads reacted differently to unexpected events in conversation, that speaking times were less balanced within ASD dyads, and that the prosodic realisation of turn ends and beginnings seems to be identical across groups. 
	
	A number of related phenomena are described in \chapref{backchannels}, all of which play particularly important roles in dialogue management. First, it is shown in \sectref{sec:BCFP_BC} that \emph{backchannels} (listener signals such as \textit{mmhm} or \textit{okay}), which in the context of ASD have not been investigated in any detail to date, were produced in unusual ways by ASD dyads. Compared to the CTR group, the ASD group produced a lower rate of backchannels (especially in the early stages of dialogue), used a less diverse range of different backchannel types and showed a less complex mapping of different intonation contours to different backchannel types. Second, it is shown in \sectref{sec:BCFP_FP} that \emph{filled pauses} (hesitation signals such as \textit{uhm}), contrary to most previous results, did not differ between groups in rate or choice of filled pause type (\textit{uh} or \textit{uhm}). For prosodic realisation (which had not been investigated in previous studies), it was found that ASD dyads produced fewer filled pauses with the prototypical level intonation.	Third, it is shown in \sectref{sec:BCFP_FP_silent} and \sectref{sec:BCFP_Laughter} that ASD dyads produced more long \emph{silent pauses} and a lower rate of \emph{laughter}.
	
	\chapref{conclusion} provides the \emph{conclusion} of the book, in which I first propose a summary analysis. This provides an in-depth description of the communicative behaviour of each ASD dyad, while also highlighting differences and similarities across autistic dyads compared to the CTR group, along all dimensions of conversation and intonation investigated. After summarising the most important findings, emphasising the important role of individual- and dyad-specific variability and discussing which behaviours seem to be most characteristic of communication in ASD, I end by reflecting on the external validity of the results presented and on future avenues of investigation.