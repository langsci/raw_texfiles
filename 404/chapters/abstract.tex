\addchap{\lsAbstract} 

This book provides an in-depth, multi-dimensional analysis of the conversational behaviour of German-speaking adults diagnosed with Autism Spectrum Disorder (ASD). Our investigation is focussed on intonation style, turn-taking and the use of backchannels, filled pauses and silent pauses.

We recorded speakers engaged in task-based, semi-structured conversations in two groups of disposition-matched speaker pairs (i.e.~interlocutors either both did or did not have a diagnosis of ASD). Recording disposition-matched pairs has the advantage of giving us direct insights into autistic communication, in contrast to most previous studies, which used disposition-mixed dyads (i.e.~conversations between autistic and non-autistic participants).

All analyses emphasise in-depth description at the levels of groups, dyads and individuals and are supported by Bayesian linear regression models.

Previous findings on intonation styles in ASD are contradictory, with claims ranging from a characteristically monotonous speech style on the one hand to a characteristically melodic speech style on the other. We used a novel methodology for quantifying intonation styles in terms of both the range and the time-varying dynamics of pitch. Results show that autistic speakers tended towards a more melodic intonation style compared to control speakers. Crucially, no ASD speakers showed a tendency towards more monotonous speech.

Research on turn-taking (the organisation of who speaks when in conversation) in ASD limited and usually based on the non-spontaneous speech of children and adolescents. Most previous studies found a tendency for longer silent gaps in ASD. We found no clear overall difference in turn-timing between the ASD and the control group, with both groups showing the same preference for very short silent-gap transitions that has been described for many other groups of speakers in the past. We did, however, find a clear difference between groups specifically in the earliest stages of dialogue, where ASD dyads produced considerably longer silent gaps than controls.

Backchannels (listener signals such as \emph{mmhm} or \emph{okay}) have barely been investigated in ASD to date. Our analysis shows that the ASD group 1) produced fewer backchannels per minute (particularly in the early stages of dialogue), 2) produced a less diverse range of different lexical types of backchannel and 3) showed a less complex and less flexible mapping of different intonation contours to different backchannel types.

Filled pauses (hesitation signals such as \emph{uhm} and \emph{uh}) in ASD have been the subject of a handful of previous studies, most of which claim that autistic speakers produced fewer \emph{uhm} tokens (only). In contrast, we found that filled pauses were produced at an identical rate in both groups and that there was an equivalent preference of \emph{uhm} over \emph{uh}. ASD speakers differed only in the prosodic realisation of filled pauses, producing fewer tokens with the prototypical level intonation contour. We further found that autistic speakers produced more long silent (within-speaker) pauses than controls.

We conclude this book with a summary analysis which highlights the importance of individual specificity, particularly in the ASD group, and shows that the clearest difference between groups was found for backchannel behaviour.

Taken together, these results provide new insights into conversation strategies and intonation styles in ASD. We discuss our findings in the context of previous literature, general characteristics of cognition in ASD, the importance of studying communication in interaction and implications for training and diagnosis.