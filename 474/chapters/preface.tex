\addchap{Preface}
\begin{refsection}

The book \textit{Universality of semantic frames and language-specific Bulgarian data} describes the principles of data organisation in the Bulgarian FrameNet, which has been under development for more than 20 years and has gone through various phases. Originally it was developed as an independent resource, but for about fifteen years it has been correlated with the Berkeley FrameNet, observing the following basic principles: The information in FrameNet that is relevant for the description of Bulgarian is considered language-independent (e.g. definition of frames and relations between them, definitions of frames and elements and relations between them, etc.) and automatically transferred into a structure called a superframe. For each superframe, there can be one or more Bulgarian frames in which the language-independent information is restructured, if necessary, so that it corresponds exactly to the description for the Bulgarian language. The Bulgarian verbs of communication, change, movement, contact and emotion are described in more detail, their subclasses are delineated and the similarities and differences in the semantic and syntactic description for Bulgarian and English are compared and discussed. 

%KOEVA 1
Chapter 1 \textit{Universality of semantic frames versus specificity of conceptual frames} introduces the Bulgarian FrameNet, which is based on the FrameNet and at the same time offers the possibility of encoding language-specific semantic structures, either by replicating or reconstructing existing semantic frames or by introducing new frames. An abstract representation, called \emph{superframe}, is developed to replicate language-independent information (at least for English and Bulgarian) from semantic frames. In Bulgarian FrameNet, the \emph{conceptual frames} inherit either all or part of the language-independent information from the semantic frames via the superframes and may contain additional language-specific data to represent scenarios evoked by the Bulgarian lexical units. Each conceptual frame is extended by a set of nouns that represent the lexical realisations of the frame elements corresponding to the target lexical units.

The study presents FrameNet \citep{Fillmore2003,fillmore2010}, the creation of FrameNets for other languages, the motivation for the introduction of conceptual frames and superframes that combine the semantic and conceptual frames in a \emph{multilingual network}, and the structure of the Bulgarian FrameNet, which includes Lexical, Grammatical, Frame, and Syntactic sections (valence patterns).

The general aim is to present our approach to the identification and transfer of \emph{language\hyp universal} knowledge from the FrameNet semantic frames, which is universal in the sense that it applies to both English and Bulgarian, and the definition and integration of \emph{language-specific components} of the conceptual frames for Bulgarian (compared to English).

%LESEVA STOYANOVA 2
Chapter 2 \textit{Language-independent and language-specific properties of semantic description: A case study on verbs of communication} focuses on the complex semantic description of verbs compiled from two main resources: Princeton WordNet \citep{Fellbaum1998} and its Bulgarian counterpart BulNet \citep{koeva2021} on the one hand and FrameNet on the other. The verb synsets in WordNet are assigned FrameNet frames, which represent the frame elements that denote the participants and props of the predicate. This chapter discusses the notion of universality in terms of semantic and conceptual features and relations, which enables the cross-linguistic transfer of language-independent descriptions (that apply to at least two languages). In addition, this chapter presents a case study on verbs of communication, analysing the transferability of FrameNet valency patterns from English to Bulgarian and the language-specific features to be taken into account. Special attention is paid to the differences resulting from alternative constructions of identical or similar situations. The analysis of these differences can shed light on the factors underlying the different conceptualisations and help to identify tendencies and regularities in the representation of the conceptual/argument structure.

%STOYANOVA 3
Chapter 3 \textit{The complex conceptual structure of verbs of change} discusses the class of verbs of change in WordNet \citep{Fellbaum1998} and the conceptual frames from FrameNet \citep{Baker1998} that describe them, together with the corresponding frame elements. The analysis begins with an overview of the treatment of verbs of change in theoretical studies, in particular with regard to the approaches to their classification based on the features of their semantic and conceptual description. The study establishes a link between the aspectual property of telicity and the notion of scale in relation to quantised change. Causation is discussed in terms of causative--inchoative verb pairs and the corresponding frames they evoke, which show similarities in their definitions and correspondences between the core frame elements involved and their semantic types and constraints. The frames discussed are connected by frame-to-frame relations (\textit{Inheritance} and \textit{Causativity}), which comprehensively describe the semantics of verbs of change together with the relevant aspects of the changes involved. The most frequent syntactic valency patterns associated with these verbs and their realisation in Bulgarian are presented on the basis of observations from the Bulgarian semantically annotated corpus \citep{koeva-2011-bulsemcor}.

%LESEVA 4
Chapter 4 \textit{The conceptualisation of the route: Non-directed and directed motion verbs in Bulgarian and English} provides an analysis of undirected and directed motion from the perspective of frame semantics by examining the semantic description and syntactic realisation of the frame elements of several FrameNet frames describing motion and drawing parallels between the syntactic properties of motion verbs in English and Bulgarian. The research questions include: Which frame elements are conceptually present in the semantics of verbs across motion-related frames, even if they are not obligatorily expressed; which frame elements are more prominent in the semantics of verbs evoking the same frame; what are the possible and preferred means of syntactic expression of these frame elements; what are the patterns of syntactic expression; and how are the investigated frame elements represented, as opposed to how they may remain syntactically implicit. The empirical evidence provided by the FrameNet corpus is validated using a sample of annotated Bulgarian examples. The second part of the chapter contains a case study of several representative frames from the domain of communication. The study illustrates how the semantics of the parent frame is further specialised, i.e. narrowed down, profiled, etc. in the inheriting frames and how this process is reflected in the configuration of the frame elements describing each frame and in their syntactic expression in English and Bulgarian. 

%TODOROVA 5
Chapter 5 \textit{Frame semantics and verbs of contact} provides a semantic description of verbs from WordNet that belong to the general Bulgarian vocabulary and to the semantic class of contact verbs. Taking into account the information about the semantic hierarchy from BulNet \citep{koeva2021}, WordNet and the semantic frames from FrameNet, the contact verbs are grouped into two main classes of predicates: verbs of physical contact by motion and verbs of physical contact in state. The main semantic frame for verbs of contact covers events in which two or more entities come into physical contact with each other. Semantic subgroups within the verbs of physical contact by movement and the verbs of physical contact in state, based on frame semantics, are provided, along with the description of syntactic properties and the definition of more specific selectional restrictions for each verb. The study makes statements about the internal semantic organisation of verbs within the domain of verbs with the semantic attribute physical contact.

%KUKOVA 6
Chapter 6 \textit{Frame semantics and verbs of emotion} is devoted to the presentation of the peculiarities of the general verbs of emotion together with their frames and frame element representations. This is followed by a description of the semantic class of emotion verbs and an analytical overview of the typological approaches of various authors. The aim of the study is to analyse the most general emotion verbs and to group them according to their semantic and syntactic features. Thus, within this class of verbs, five major subclasses are formed, namely: basic neutral verbs of emotion, verbs for emotional attitude, causative verbs of emotion that encode an \framename{Agent} or \framename{Stimulus} as subject, and stative or inchoative verbs formed from the causatives using the reflexive-in-form particle \textit{se}. The main focus is on five frames, namely \framename{Feeling}, \framename{Experiencer\_focused\_emotion}, \framename{Stimulate\_emotion}, \framename{Cause\_to\_experience} and \framename{Emotion\_directed}. They are all characterised in terms of the verbs they cover, their core frame elements and the possible representations they can have with regard to their syntactic realisation. The verbs that evoke the frame \framename{feeling} have neutral semantics. The verbs of the other four semantic frames denote a positive or negative emotion in their semantic structure.


%STEFANOVA TODOROVA DIMITROVA 7
Chapter 7 \textit{Basic verb vocabulary: An empirical approach to Argument Structure and word associations} deals with the results of a study on the degree of acquisition of a number of verbs in Bulgarian that are considered part of children's general vocabulary. The pilot experiment involved language tasks designed to test whether the verbs in the extracted set belong to the basic conceptual apparatus and whether this approach is suitable for making observations about the respondents' linguistic knowledge, experience and intuition regarding the use of these verbs. The analysis of the target verbs is based on the description of both semantic and conceptual frames. The pilot results are presented with regard to the semantic frames of the target verbs, the lexical entries in argument positions and the semantic and syntactic combinability.

The semantic frames can be used in an experiment to assess children's mastery of semantic conceptualisation and syntactic use of verbs from their basic vocabulary. These and a number of other applications: automatic assignment of semantic roles, automatic recognition of events in news, automatic recognition of scenes in images and videos are some of the applications in which the Bulgarian FrameNet can be used. In addition, the semantic and syntactic information in the Bulgarian FrameNet can be used for theoretical considerations, including comparative studies focussing on the modern state of the Bulgarian language and other languages for which a FrameNet has been developed.

{\sloppy\printbibliography[heading=subbibliography]}
\end{refsection}

