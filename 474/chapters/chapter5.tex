\documentclass[output=paper,colorlinks,citecolor=brown]{langscibook}
\ChapterDOI{10.5281/zenodo.15682192}
\title{Frame semantics and verbs of contact}

\author{Maria A. Todorova\orcid{0000-0001-5866-7180}\affiliation{Department of Computational Linguistics, Institute for Bulgarian Language, Bulgarian Academy of Sciences}}

\abstract{The article provides a semantic description of a group of verbs from WordNet with the semantic primitive `verbs of contact`, which belong to the common vocabulary of Bulgarian. We present the result of their annotation with conceptual frames using the information for the semantic hierarchy from BulNet \citep{koeva2021-wordnet} or WordNet and the semantic frames from FrameNet \citep{Fillmore1982, Baker1998} -- original or adapted. The predicates of contact are divided into the main classes: Verbs of physical contact via movement and Verbs of physical contact in state. Using information from the hierarchical structure of WordNet, semantic frames from FrameNet and based on the observations on the selective features and syntactic realisation of the concrete meaning, subtypes of contact verbs are offered.}

\IfFileExists{../localcommands.tex}{
 \addbibresource{../localbibliography.bib}
 % add all extra packages you need to load to this file

\usepackage{tabularx,multicol}
\usepackage{url}
\urlstyle{same}

\usepackage{listings}
\lstset{basicstyle=\ttfamily,tabsize=2,breaklines=true}

\usepackage{langsci-basic}
\usepackage{langsci-optional}
\usepackage{langsci-lgr}
\usepackage{langsci-osl}
% \usepackage{./langsci/styles/langsci-lgr}
% \usepackage{./langsci/styles/langsci-osl}
% \usepackage{langsci-gb4e}

\usepackage{tikz}
\usetikzlibrary{patterns,calc}
\pgfdeclarepatternformonly{south east lines}{\pgfqpoint{-0pt}{-0pt}}{\pgfqpoint{3pt}{3pt}}{\pgfqpoint{3pt}{3pt}}{
    \pgfsetlinewidth{0.6pt}
    \pgfpathmoveto{\pgfqpoint{0pt}{3pt}}
    \pgfpathlineto{\pgfqpoint{3pt}{0pt}}
    \pgfpathmoveto{\pgfqpoint{.2pt}{-.2pt}}
    \pgfpathlineto{\pgfqpoint{-.2pt}{.2pt}}
    \pgfpathmoveto{\pgfqpoint{3.2pt}{2.8pt}}
    \pgfpathlineto{\pgfqpoint{2.8pt}{3.2pt}}
    \pgfusepath{stroke}}
    
\usepackage{stmaryrd}
\usepackage{wasysym}
\usepackage{multirow}
\usepackage{caption}
\usepackage{subcaption}
\usepackage{mathrsfs}
\usepackage{qtree}

\usepackage{linguex}


 %pminos do not split footnotes
% \interfootnotelinepenalty=10000 %Footnote in Laporte chapters has to be split SN


%\DeclareIndexNameFormat{default}{%
%\nameparts{#1}%
%\usebibmacro{index:name}%
%{\index[names]}%
%{\namepartfamily}%
%{\namepartgiveni}%
% {}% L1
% {}% L2
%{\namepartprefix}% generates spurious space L3
%{\namepartsuffix}% generates spurious space L4
%}

%  {\DeclareIndexNameFormat{default}{%
%     \usebibmacro{index:name}{\index[names]}{#1}{#3}{#5}{#7}}}

%\DeclareIndexNameFormat{default}{%
%  \usebibmacro{index:name}{\sindex[nom]}{#1}{#3}{#5}{#7}}

%\DeclareIndexNameFormat{default}{%
%  \usebibmacro{index:name}{\sindex[person]}{#1}{#3}{#5}{#7}}
%\DeclareIndexNameFormat{default}{%
%\nameparts{#1} \usebibmacro{index:name}{\sindex[person]]}{\namepartfamily}{‌​\namepartgiven}{\nam‌​epartprefix}{\namepa‌​rtsuffix}}

%\newcommand{\smiley}{:)}

%\renewbibmacro*{index:name}[5]{%
%\usebibmacro{index:entry}{#1}%
%{\iffieldundef{usera}{}{\thefield{usera}\actualoperator}\mkbibindexname{#2}{#3}{#4}{#5}}}

% \newcommand{\noop}[1]{}

%remove for final
%\overfullrule=1mm

\newcommand{\tobi}[2]}}
\renewcommand{\S}[1]{\tobi{#1}{\textsc{*}}}

% this volume references
% puts: [this volume]
% already defined: \citetv
%\newcommand{\citepv}[1]{(\citeauthor{#1} \citeyear*{#1} [this volume])}
\newcommand{\citealtv}[1]{\citeauthor{#1} \citeyear*{#1} [this volume]}

%parentheses around example number
\newcommand{\pref}[1]{(\ref{#1})}

% in-text examples

\newcommand{\lnex}[1]{\textit{#1}} %target lang word
\newcommand{\lnlit}[1]{(lit.: `#1')} %literal reading
\newcommand{\lnlat}[1]{(#1)} % latinization
\newcommand{\lntrans}[1]{`#1'} %translation
\newcommand{\lnexl}[2]%
{\lnex{#1}{} \lnlat{#2}} % ex with latinization
\newcommand{\lnexlat}[3]{\lnex{#1}{} \lnlat{#2}{} \lntrans{#3}} % ex with latinization and tranl.

%ch01
\newcommand{\co}[1]{\mbox{\textbf{#1}}}

%ch09

\newcommand{\cyrbulg}[1]{\begin{otherlanguage*}{bulgarian}#1\end{otherlanguage*}}


%ch10
\newcommand{\nlp}{{\small NLP}}
\newcommand{\mwe}{{\small MWE}}
\newcommand{\rae}{{\small RAE}}
\newcommand{\lvc}{{\small LVC}}
\newcommand{\pos}{{\small P}o{\small S}}
%\newcommand{\todo}[1]{ \textcolor{red}{#1} }

%\renewcommand{\labelenumi}{\theenumi}
%\ainamefmt{{vv}{ll}{, ff}{, jj}} % fullname

\newcommand{\biberror}[1]{{\color{red}#1}}

\newcommand{\osenovaitem}{--~}
 %% hyphenation points for line breaks
%% Normally, automatic hyphenation in LaTeX is very good
%% If a word is mis-hyphenated, add it to this file
%%
%% add information to TeX file before \begin{document} with:
%% %% hyphenation points for line breaks
%% Normally, automatic hyphenation in LaTeX is very good
%% If a word is mis-hyphenated, add it to this file
%%
%% add information to TeX file before \begin{document} with:
%% %% hyphenation points for line breaks
%% Normally, automatic hyphenation in LaTeX is very good
%% If a word is mis-hyphenated, add it to this file
%%
%% add information to TeX file before \begin{document} with:
%% \include{localhyphenation}
\hyphenation{
    Beck-man
    Ngu-yen
    back-chan-nel
    back-chan-nels
    mo-not-o-nous
    ste-reo-typ-i-cal
}

\hyphenation{
    Beck-man
    Ngu-yen
    back-chan-nel
    back-chan-nels
    mo-not-o-nous
    ste-reo-typ-i-cal
}

\hyphenation{
    Beck-man
    Ngu-yen
    back-chan-nel
    back-chan-nels
    mo-not-o-nous
    ste-reo-typ-i-cal
}

 \boolfalse{bookcompile}
 \togglepaper[23]%%chapternumber
}{}

\begin{document}
\maketitle

\section{Introduction} 
This article aims at a conceptual description of 450 high-frequency Bulgarian verbs categorised as verbs of contact in WordNet \citep{Fellbaum:90}. Their selection is based both on their participation in the general Bulgarian lexicon and on their thematic coverage.

Verb classes are defined in the linguistic literature as coherent groups of verbs that have similar semantic properties, such as belonging to a common semantic domain \citep{Juffs:96} or similar argument realisation and semantic interpretation \citep{Fillmore:70, Levin:93, Kipper-Schuler2005}. \citet{Fillmore:70} emphasises the importance of verb classes for the organisation of the verb lexicon and the investigation of patterns of common verb behaviour as well as for the identification of grammatically relevant elements of meaning \citep [125]{Fillmore:70}. At the same time, the theory of frame semantics \citep{Fillmore1977, Ruppenhofer2016} is based on the interdependence of the lexicon and grammar of a language. It characterises the semantic and syntactic properties of predicates by relating them to semantic frames. The semantic arguments of a predicate correspond to the frame elements in frames that describe its semantics.

On this basis, we analyse the semantics of various contact verbs by combining the semantic classification of verbs in WordNet with the information from FrameNet \citep{Fillmore1982, Baker1998} to group them into conceptual frames. Conceptual frames \citep{svetla2021towards} are abstract structures that represent the syntagmatic relation between a predicate and its arguments on the one hand and the relation of a set of predicates (verbs) and sets of their arguments to the verbal and nominal synonym sets in WordNet on the other. In this way, they generalise the application of lexical information from WordNet in the semantic frames of FrameNet.

As a result, we provide a semantic and syntactic description and classification of Bulgarian contact verbs in comparison to their English equivalents. The categorisation into conceptual subtypes is based on implications about the conceptual description of the relevant verbs. The observations are based on data from the Bulgarian WordNet (BulNet) \citep{koeva2021-wordnet} and the Princeton WordNet \citep{Fellbaum:99a} and on the process of extending the Bulgarian FrameNet \citep{Koeva2010-framenet} with conceptual frames \citep{svetla2021towards}, which led to the creation of BulFrame -- a linked semantic and syntactic resource for Bulgarian \citep{KoevaDoychev:2022}.

WordNet and FrameNet are comprehensive lexical resources that provide semantic information on a variety of verb features. WordNet represents a multilingual conceptual network of synonym sets (synsets) linked by semantic relations such as hypernymy, antonymy, etc., and provides sets of semantic classes of verbs and nouns.
FrameNet represents the semantics of lexemes by means of schematic representations (frames) describing objects, situations, or events, and their components (frame elements) in the frame semantics apparatus.

The rest of the work is organised as follows: \sectref{ch5:sec:2} describes the data used in the annotation process – a set of contact verbs from WordNet and a set of semantic frames from FrameNet; \sectref{ch5:sec:3} gives an overview of the associated descriptions and classifications of the verbs under consideration; \sectref{ch5:sec:4} discusses the semantic features of contact verbs and their lexical semantic subtypes and provides examples of the annotation of contact verbs with semantic frames; \sectref{ch5:sec:5} comments on the valency patterns of the verbs studied; \sectref{ch5:sec:6} provides a brief discussion; and \sectref{ch5:sec:7} summarises the observations on the results and suggests directions for future work.

\subsection{Verbs of contact} 
The categorisation of verbs into semantic classes varies depending on the theory used and the focus of the description. There is a wide variety of descriptions and definitions of verbs of contact. In general, the element CONTACT is understood as the “conceptual core element” of a predicate \citep [47]{Juffs:96}. When analysing alternation patterns in English, \citet [125]{Fillmore1977} defines two main classes: Break verbs and Hit verbs, where Hit verbs involve (often forceful) contact with an entity without changing its state. \citet{Levin:93} defines a distinction between manner verbs and result verbs. Verbs such as \textit{hit}, which describe surface contact with an object through a forceful impact, are MANNER(/means) verbs and describe ways of potentially damaging objects \citep{Levin2015}.

The group of contact verbs in WordNet belongs to one of the 15 semantic classes in which the verbs in WordNet are grouped according to the general semantic domain to which they pertain. The group of contact verbs is defined as “verbs of touching, hitting, tying, digging” \citep{Miller1995, Fellbaum:90}. It is also the largest of them and consists of more than 2100 verbal synsets, out of all 14103 Bulgarian verbal synsets, and includes event and action verbs that share the semantic component of CONTACT or IMPACT.

This verb group forms a taxonomy structure by means of the hyponymy (troponymy) relation, which comprises a number of different manner relations \citep{Fellbaum:90}. The semantic definition of the class is fuzzy and does not really summarise the semantics of all the verbs it contains. Therefore, we try to define typical subclasses within the class by using both the syntactic and semantic information from WordNet and FrameNet.

\subsection{Related work}
Verbs of contact are a heterogeneous and overlapping semantic class, which is why they are less researched than other verb classes. They are the subject of research for the English \citep{Fillmore:70, Levin:93, Fellbaum:90}, the Chinese \citep{GaoCheng:03}, the Swedish \citep{Viberg2004}. \citet{Fillmore:70} focuses on two large classes of contact verbs, \textit{break} and \textit{hit}, whose members share elements of meaning and patterns of behaviour. A class of contact verbs has also defined by \citet{Levin:93} in her semantic classification based on the alternations that reflect the correlation between the semantics and syntactic behaviour of the verbs and the interpretation of their arguments. In particular, \citet[148-156]{Levin:93} defines a class of Verbs of contact by impact with a number of subclasses: \textit{Hit verbs}; \textit{Spank verbs}; \textit{Swat verbs}; \textit{Non-agentive verbs}. \citet{VulchanovaDekova:09} represent a corpus and an empirically derived classification of verbs of contact by impact using the Sign model formalism. Individual subtypes of the class have also been described by some authors: \textit{Physical contact verbs} \citep{Gao:01} and \textit{Hit and Spank verbs of contact by impact} \citep{GaoCheng:03}. These descriptions partly overlap with the classification of verbs in WordNet and only some of them are aligned with the semantic frames in FrameNet for some verbs of contact.

Previous work on the conceptual semantic descriptions of Bulgarian verbs includes the analysis of verbs of change \citep{StoyanovaLeseva:21a} and verbs of communication \citep{Kukova:21}, verbs of movement \citep{Kostova2010}, predicates of mental state \citep{Stamenov2021, Stamenov2022, Tisheva2021, Dzhonova2008}, and a description of the syntactic transformations of Bulgarian verbs \citep{Koeva2004, Koeva2021zv, Koeva:2022}. In \citet{Lesevaetal:20b} and \citet{Lesevaetal:2019} different levels of investigation of semantic features and selectional constraints relevant for the semantic description of Bulgarian verbs and their frame elements are analysed. As far as we know, the set of verbs denoting physical contact has not yet been described as a separate semantic class for Bulgarian or compared with their English equivalents.

\section{The data analysed}\label{ch5:sec:2}
As already mentioned, the analyses of the verbs in this work are based on the semantic descriptions and relational hierarchy of WordNet and the semantic frames of FrameNet. The combined information available in the resources enables a rich representation of the paradigmatic and syntagmatic aspects of the lexical semantics \citep{BakerFellbaum:09}. Therefore, the semi-automatic mapping of FrameNet frames to WordNet synsets described in \citet{StoyanovaLeseva:20} is used.

After the selected set of contact verbs was extracted from WordNet, it was filtered to include only verbs that belong to the general lexis of Bulgarian. The selection was made taking into account the theoretical semantic description and the typology of verb predicates that belong to the general vocabulary of the Bulgarian language \citep{KoevaDoychev:2022}. The collection was created by overlaying a set of 44000 English verbs selected according to the AoA criterion (age of acquisition) \citep{BrysbaertBiemiller:17} with a subset of verbs from the Bulgarian WordNet. The resulting verb set was additionally evaluated on the basis of (i) the correspondence with the list of so-called Base concepts;\footnote{A WordNet subset defined within the EuroWordNet and BalkaNet projects \url{http://globalwordnet.org/resources/gwa-base-concepts/}} (ii) the frequency information on the use of the verbs from the Bulgarian National Corpus \citep{Koevaetall:12}. The selection procedure, described in more detail in \citet{KoevaDoychev:2022} and \citep{Todetal:22}, resulted in a list of over 5000 verbs from the general vocabulary;\footnote{The data are available at \url{https://dcl.bas.bg/projects_list/enriching-wordnet/}} 804 of these were assigned to the semantic class verb.contact, and 486 were semi-automatically assigned to 107 unique semantic frames from FrameNet. These verbs were additionally validated and their set was filtered for frequency and thematic coverage. Verbs with metaphorical or figurative meaning and verbs expressing personal relationships and emotional contact were excluded.

The resulting set of 450 contact verbs forms the starting group for the annotation with semantic frames described in this paper. We propose a semantic description of contact verbs in Bulgarian based on their frame elements, their selectional restrictions (represented by the semantic classes of nouns in WordNet) and their syntactic realisation in a context as well as their classification.

\section{Annotation of contact verbs and assignment of semantic frames}\label{ch5:sec:3}
The annotation of Bulgarian contact verbs with semantic frames and the description of their frame elements and the relevant semantic restrictions is done with the help of the software system BulFrame, which was developed specifically for the description of conceptual frames \citep{KoevaDoychev:2022}. The semantic restrictions imposed on the arguments of the verb were matched against a particular subtree or subtrees of noun synsets in WordNet, based on previous work described in \citet{Leseva:18} and \citet{DimitrovaStefanova2019}. The annotation of the selected verbs with BulFrame includes the following steps:

\begin{itemize}
\item[(a)] Morphosyntactic information is assigned to each verb (transitive and intransitive, reflexive verbs, 3rd person verbs)

\item[(b)] Each verb is assigned a FrameNet frame (as is), a FrameNet frame that has been modified to better reflect the semantics of the respective verbs, or a newly formulated frame.

\item[(c)] For each frame element in a particular frame, an expert assigns a grammatical role (subject, object, adjunct).

\item[(d)] For each frame element in a given frame, an expert evaluates the general selectional restrictions assigned to it. The general semantic restrictions proposed in \citet{Lesevaetal:2019} and \citet{DimitrovaStefanova2019}, which describe the compatibility between the semantic classes of verbs and nouns corresponding to their arguments, are matched against the top-level noun synsets in the respective subtrees in WordNet. These restrictions, when assigned to a frame, give a first approximation of the semantic specification of the frame elements. If a general restriction is assigned, all hyponyms of the noun synsets selected in the corresponding subtrees are checked as potential candidates for the lexical realisation of the frame element.

\item[(e)] Each verb is examined individually in order to specify additional selectional restrictions from WordNet if required. Specific restrictions for the lexical realisation of the frame elements are represented as single WordNet synsets.
\end{itemize}
\section{Semantic features of verbs of contact}\label{ch5:sec:4}
In this section, the semantic characterisation of contact verbs proposed by \citet{Miller1990} and their division into subclasses with regard to the WordNet hierarchy is used in combination with the semantic information from FrameNet.

In order to group the selected contact verbs into conceptual subtypes based on both the realisation of their frame elements and their lexical semantics, we rely on the assumption that verbs with similar verb meanings share characteristic argument realisations.

\editcolor{red}{As mentioned above, the definition of conceptual subtypes is based on the conceptual frames of \citet{svetla2021towards}. A particular conceptual frame is associated with predicate(s) from a particular semantic class, and each element of the conceptual frame is associated with a set of nouns that are compatible with the predicate(s)}.

\subsection{Lexical semantic subtypes}
Being the largest class of verbs in WordNet, the set of contact verbs is well represented in the selection of Bulgarian verbs of general lexis -- almost 16\% of the total (2179 synsets, labelled verb.contact, out of 13,766 verbal synsets in WordNet). Most contact verbs are hyponyms of the following verb root synsets within the WordNet structure: fasten, attach, cover, cut and touch, resulting in a large tree structure within the set. Based on the WordNet hypernym relation, \citet [59]{Miller1990} define the following subgroups of contact verbs:

\begin{itemize}
\item[(a)] Verbs that encode force, intensity or iteration of the action (hit).

\item[(b)] Verbs of holding (grab, squeeze, pinch) and touching (paw, finger, stroke, poke).

\item[(c)] Verbs that involve an instrument or a material argument (paint).

\item[(d)] Verbs with a body part argument that indicates the type of contact action for which the body part is typically used: \textit{Shoulder} (support, carry); \textit{elbow} ‘push`; \textit{finger}, \textit{thumb} (touch, manipulate).
\end{itemize}

The semantics of these classes correspond to the semantics of some of the FrameNet frames. The set of WordNet verbs that encode force under the verb roots \textit{hit} and \textit{destroy} is described by the frames \framename{Impact} and \framename{Destroying} in \ref{4.3.2}. The WordNet verbs of holding and the verbs involving a body part, a material or an instrument correspond to the semantic frames presented in \ref{4.3.1.3} and the verbs of contact denoting displacement are presented in \ref{4.3.1.}. The State verbs for physical contact described in \ref{4.4} belong to smaller subtrees in WordNet.

\subsection{Frame semantic subtypes} 

The generalised semantic frame for verbs of contact stands for various situations in which two or more entities come into physical contact with each other. This frame is an abstract representation for semantic frames of a wide range of verbs denoting various aspects of contact and corresponding valency patterns.

The generalised features of the frame elements characterise the core and some non-core frame elements for the relevant semantic frames. The semantics of contact is encoded in the core frame elements representing the entities involved in the situation, often labelled subject (the one who initiates the contact) and object (the one who is contacted), but also in the frame elements representing an entity that mediates the contact -- instrument or connector. The frame elements of some contact verbs indicate the direction of contact, e.g. verbs that describe an entity reaching out to touch or affect another entity, such as \fename{Goal} and \fename{Path}, or the type of contact, which can be very different.

The description of the class is based on the hierarchy of frame-to-frame relations \citep{Ruppenhofer2016}, on the subdivision of verb lexis into activities (motions) and states \citep[100]{Vendler1957},  \citet[40]{Dowty1979}, as well as on the proposal of lexical decomposition, and the idea of complementary notions or semantic attributes in the organisation of meaning \citep{Lobner2011}. 


Various verbs in the domain of contact encode more than one semantic attribute and can be categorised into more than one semantic class depending on the focus of the classification. This becomes clear when analysing the semantic frames of verbs of change, verbs of motion and verbs of communication in WordNet as well as verbs in the FrameNet semantic frames of FrameNet: \framename{Мotion}, \framename{Cause\_change}, \framename{Undergo\_change}, \framename{State}, \framename{Commutative\_process}, \framename{Noncommutative\_process}. The lexical semantic domains of WordNet cannot be uniformly transferred to frame semantic domains in FrameNet. Many verbs from the lexical domain of verb.contact in WordNet are described by FrameNet frames that encode change, movement and communication, as described in the chapters \textit{The complex conceptual structure of verbs of change} and \textit{The conceptualisation of the route: Non-directed and directed motion verbs in Bulgarian and English} from this volume.

Taking this into account, we assume that the contact predicates are divided into two subgroups that combine semantic components of \textbf{Physical Contact} and \textbf{Motion} as well as \textbf{Physical Contact} and \textbf{State}. With regard to the general vocabulary domain, this work focuses only on verbs that denote direct physical contact and excludes verbs described by the frames \framename{Cause\_change}, \framename{Undergo\_change}, \framename{Commutative\_process}, \framename{Noncommutative\_process}.

The analysed frames are assigned to the contact verbs included in the selection of general Bulgarian vocabulary.  Based on the most typical frame elements in the semantic frames --\fename{Theme}, \fename{Body\_Part}, \fename{Source} \ \fename{Goal}, and \fename{Impactor} \ \fename{Impactee}, and the location and manner of contact, we consider three subclasses of Verbs of physical contact via motion -- Verbs of contact denoting displacement, Verbs of bodily contact and Verbs of contact by impact. They are additionally grouped according to the semantics of the frames they represent. Some of the frames are discussed to provide more precise constraints for the selection of frame elements.

\subsection {Verbs of physical contact via motion}
This verb class includes verbs that denote actions in which an object or entity comes into contact with another object or entity through a certain type of motion. These verbs emphasise the physical interaction that occurs as a result of a movement, and they are assigned to one of the following FrameNet frames:\footnote{We used the FrameNet data available in September 2023: \url{https://framenet.icsi.berkeley.edu/}}
\framename{Becoming\_attached}, \framename{Body\_movement},~ \framename{Breaking\_off},~ \framename{Cause\_fluidic\_motion},~ \framename{Closure}, \framename{Destroying}, \framename{Detaching},~ \framename{Dispersal},~ \framename{Filling},~ \framename{Fluidic\_motion}, \framename{Food\_gathering}, \framename{Gathering\textunderscore up}, \framename{Grinding}, \framename{Make\_noise}, \framename{Manipulate\_into\_shape}, \framename{Placing}, \framename{Removing}, \framename{Reshaping}, \framename{Undressing}, \framename{Processing\_materials}.

These frames can be additionally specified on the basis of the semantic relationship between the frame elements and their syntactic realisation. In many cases, some of the synsets that share the same FrameNet frame belong to the same (or semantically close) WordNet subtrees. In these cases, the top-level synset more or less matches the constraints for the frame, while its hyponyms may have more specific requirements.
Based on the motion types combined with contact manner, we divide verbs of contact via motion into verbs denoting displacement and verbs of contact by impact. The subclasses we provide represent the verbs and frames in the data we analysed and do not claim to cover all possible semantic domains of the class.

\subsubsection {Verbs of contact, denoting displacement} \label{4.3.1.} 

This subclass includes verbs that denote attaching, detaching, placing, removing, filling and emptying. They have common frame elements and restrictions based on the point of physical contact. Overall, these frames are about the movement of an entity \fename{Theme} directed either to a particular \fename{Place} or originating from \fename{Source}. Their core frame elements share similar general restrictions -- their \fename{Agents} are volitional; the \fename{Cause} denotes a physical entity or eventuality; the \fename{Item} is a physical object; the \fename{Goal} -- a physical entity or container; and the \fename{Connector} -- a physical entity. 
The semantics of the point of physical contact defines three main subgroups: verbs of contact on or along a surface, verbs of contact with a container, verbs of contact with a body.

\subsubsubsection{Verbs of contact on or along a surface} \label{4.3.1.1} 
Verbs of contact on or along a surface describe actions in which an object or entity comes into contact with a surface, moves over it or follows the contour of a surface. Taking into account the manner of the contact motion, they denote direct physical contact between two objects or entities, e.g. in \textit{докосвам} `touch’, \textit{държа} `hold’; slight physical contact that involves indirect physical interaction,  e.g. in \textit{чет\-кам} `brush’, \textit{ожулвам} `graze’; gentle contact, e.g. in  \textit{галя} `caress’, \textit{сгушвам се} `nuzzle’, \textit{потупвам} `pat’; or exploratory contact,  e.g. in \textit{опипвам} `probe’, \textit{натис\-кам} `poke’.

\begin{table}
\small
\begin{tabularx}{\textwidth}{ lQQQ }
\lsptoprule
Frame & Definition & Frame elements & Lexical units \\
\midrule
\framename{Attaching} & An \fename{Agent} attaches an \fename{Item} to a \fename{Goal} by manipulating a \fename{Connector} & \fename{Agent}; \fename{Goal}; \fename{Item}; \fename{Connector} & the verb root eng-30-01354673-v: \{\textit{връзвам}\}, `tie: connect, fasten, or put together two or more pieces’\footnote{The BulNet aligned with the English WordNet and other languages is available online: \url{http://dcl.bas.bg/bulnet/}} and its hyponyms \\
\midrule
\framename{Detaching} & An \fename{Agent} detaches an \fename{Item} from a \fename{Source} & \fename{Agent}; \fename{Source}; \fename{Item} & the verb root eng-30-01298668-v: \{\textit{махам}\}, `detach: cause to become detached or separated; take off’ and its hyponyms. \\
\midrule
\framename{Placing} & An \fename{Agent} places a \fename{Theme} at a location -- the \fename{Goal}, which is profiled & \fename{Agent}; \fename{Goal}; \fename{Theme} & the verb root eng-30-01249724-v: \{\textit{трия}\}, `rub:  move over something with pressure’ and its hyponyms -- \{\textit{четкам}\} `brush’; \{\textit{прекарвам}\} `gauge’; \{\textit{изтърквам}\} `scrub’; \{\textit{мажа}\} `smear’; \{\textit{стържа}\} `scrape’ etc. \\
\midrule
\framename{Removing} & An \fename{Agent} causes a \fename{Theme} to move away from a location, the \fename{Source}, which is profiled & \fename{Agent}; \fename{Source}; \fename{Theme} & the verb root eng-30-01532589-v: \{\textit{изчиствам}\}, (clean, `make clean by removing dirt, filth, or unwanted substances from’) and its hyponyms -- \{\textit{четкам}\} `brush’; \{\textit{мета}\} `sweep’; \{\textit{изпирам}\} `wash’ etc. \\
\lspbottomrule
\end{tabularx}
\caption{Verbs of contact on or along a surface}
\label{tab:chapter5handle:keytotable1}
\end{table}

These verbs often emphasise the physical interaction between the object or entity and a surface. Therefore,  the most characteristic semantics of frame element of these verbs is the ``surface'', represented by \fename{Source}, \fename{Goal}, and \fename{Connector}, where the contact takes place, or along which the movement occurs. It is a key component for understanding the spatial aspect of the action described by these verbs. \tabref{tab:chapter5handle:keytotable1} presents the frames and their frame elements within this group as well as examples of lexical units. The frame elements of frames \fename{Placing} (Example \ref{bg:ch5:diff-vb1a}) and \fename{Attaching} (Example \ref{bg:ch5:diff-vb1b}) are shown below.

\begin{exe} 
\ex \label{bg:ch5:diff-vb1} 
\begin{xlist}
\ex \label{bg:ch5:diff-vb1a}
%\settowidth \jamwidth{(bg)} 
\gll {[\textit{Жената}]}$_{\feinsub{Age}}$ \textit{\textbf{ТРИЕ}} {[\textit{масата}]}$_{\feinsub{Goal}}$ [\textit{с гъба}]$_{\feinsub{Ins}}$.
%\jambox{(bg)}
\\
Woman-DEF {is rubbing} table-DEF {with sponge}. \\
\glt `The woman wipes the table with a sponge.` 
\ex \label{bg:ch5:diff-vb1b}
%\settowidth \jamwidth{(bg)}
\gll {[\textit{Той}]}$_{\feinsub{Age}}$ \textit{\textbf{ВЪРЗАЛ}} [\textit{с връв}]$_{\feinsub{Conn}}$ {[\textit{разваления зъб}]}$_{\feinsub{Goal}}$.
%\jambox{(bg)}
\\
He tied {with string} {rotten-DEF tooth}. \\
\glt `He tied the rotten tooth with a string.` 
\ex \label{bg:ch5:diff-vb1c} %c.
%\settowidth \jamwidth{(bg)}
\gll {[\textit{Майсторът}]}$_{\feinsub{Age}}$ \textit{\textbf{ЛАКИРА}} [\textit{с лак}]$_{\feinsub{Thm}}$ {[\textit{новата маса}]}$_{\feinsub{Goal}}$.
%\jambox{(bg)}
\\
Craftsman-DEF varnished {with varnish} {new-DEF table}. \\
\glt `The craftsman varnished the new table.`
\end{xlist}
\end{exe}

FrameNet frames \framename{Filling} and \framename{Emptying} each describe two semantic situations. On the one hand, they represent the contact of a \fename{Theme} with a container (see \tabref{tab:chapter5handle:keytotable2}) and, on the other hand, they represent the covering areas with an object, several objects or a substance or its removal from the surface. We therefore divide \framename{Filling} and \framename{Emptying} between the verbs of contact on a surface and the verbs of contact with a container. One argument in favour of this is the difference in the core frame elements, which represent the surface of the contact -- \fename{Goal} and \fename{Container}. In frames \framename{Filling} and \framename{Emptying}, the frame element \fename{Theme} usually expresses that the substance is on the surface of the \fename{Goal} as shown in Example \ref{bg:ch5:diff-vb1c}. The \fename{Theme} imposes additional selectional restrictions on the frame element - the direct object in Bulgarian, as shown by the verb  eng-30-01269008-v: \{\textit{лакирам}\} ‘varnish: cover with varnish’. Its \fename{Agent} is a volitional human being, usually a qualified person, while the \fename{Theme} is a particular kind of substance, described by the synset eng-30-04521987-n: \{\textit{лак}\} `varnish’. The \fename{Goal} is eng-30-00002684-n: \{\textit{physical object}\} or eng-30-08660339-n: \{\textit{surface}\}.

\subsubsubsection{Verbs of contact with a container} \label{4.3.1.2}
This group of verbs represents the interaction of an object or entity with a container, which can be a box, bag, jar, vessel, or other object used to hold or store items. \tabref{tab:chapter5handle:keytotable2} describes the frames within this group as well as their frame elements and examples of lexical units.

\begin{table}
\small
\begin{tabularx}{\textwidth}{lQQQ}
\lsptoprule
Frame & Definition & Frame elements & Lexical units \\
\midrule
\framename{Filling} & Filling \fename{Containers} with some thing, things or substance, the \fename{Theme}. & \fename{Agent}; \fename{Container}; \fename{Theme} & the verb root
eng-30-01332730-v: \{\textit{запушвам; запуша}\} ‘fill up: fill or stop up` and its hyponyms \\
\midrule
\framename{Emptying} & An \fename{Agent} causes a \fename{Theme} to move away from a location, the \fename{Source}, which is profiled & \fename{Agent}; \fename{Source}; \fename{Theme} & the verb root eng-30-01488313-v: \{\textit{изпразвам; изпразня}\} `empty: remove` and its hyponyms \\
\lspbottomrule
\end{tabularx}
\caption{Verbs of Contact with a Container}
\label{tab:chapter5handle:keytotable2}
\end{table}

The most characteristic frame element of these verbs is the "container" or "receptacle" -- it specifies the particular container with which the contact is made. The nature and characteristics of the container lead to narrower selectional restrictions of verbs. The \fename{Agent} corresponds to the WordNet root synset eng-30-00007846-n: \{\textit{person}\}; the \fename{Theme} corresponds to the WordNet root synset eng-30-00002684-n: \{\textit{physical object}\} or to eng-30-00001740-n: \{\textit{entity}\}, and the \fename{Goal} matches the synset eng-30-03094503-n: \{\textit{container}\}
A specific valency pattern is represented from the \fename{Container} which can be  direct, or indirect object, and it is designated \fename{Goal} of motion of the \fename{Theme} which is indirect object.

\subsubsection {Verbs of bodily contact} \label{4.3.1.3}
Within this subgroup, verbs describe actions that involve various ways in which objects and body parts come into contact. This group includes the verbs from the FrameNet frame \framename{Manipulation}, which describes \textit{the manipulation of an \fename{Entity} by an \fename{Agent}, the \fename{Entity} is not deeply or permanently physically affected, nor is it overall moved from one place to another.} The most characteristic frame element of those verbs is the \fename{Body\_part\_Surface}, where the contact occurs or along which the movement takes place. Often it is not expressed explicitely, and is a part of the verb meaning, or the \fename{Agent}'s manipulation of an \fename{Entity} may be further specified as being localised to some part of the \fename{Entity}, a \fename{Locus}. The part of the Agent's body being used to manipulate the Entity may also be expressed, as shown in Example \ref{bg:ch5:diff-vb3}.

Based on the manner of the contact, verbs within this group can denote direct physical contact: \textit{докосвам} `touch`, \textit{държа} `hold`, \textit{стискам} `squeeze`; light and gentle touch: \textit{галя} `caress`, \textit{гъделичкам} `tickle`; caring or affectionate touch: \textit{прегръщам} `hug`, \textit{целувам} `kiss`; forceful or aggressive touch: \textit{удрям} `slap`, \textit{щи\-пя} `pinch`.

\begin{exe} 
\ex \label{bg:ch5:diff-vb3}
%\settowidth \jamwidth{(bg)} 
\gll 
 [{\textit{Майката}}]$_{\feinsub{Age}}$ \textit{\textbf{ДЪРЖИ}} [\textit{лъжицата}]$_{\feinsub{Ent}}$ [\textit{с ръка}]$_{\feinsub{Locus}}$.\\
Mother-DEF holds spoon-DEF {with hand}. \\ 
 %\jambox{(bg)}
\glt `The mother holds the spoon with a hand.` 
\end{exe}

Selectional restrictions: the \fename{Agent} corresponds to the WordNet root synset eng-30-00007846-n: \{\textit{person}\}; the \fename{Entity} corresponds to the WordNet root syn\-set eng-30-00002684-n: \{\textit{physical object}\} or eng-30-00001740-n: \{\textit{entity}\}. 

The core frame elements of the frame \fename{Manipulation} have more specific restrictions: Hyponyms of the synset eng-30-01216670-v:\{\textit{хващам}\} ‘hold: have or hold in one's hands or grip’ are used as shown in Example \ref{bg:ch5:diff-vb8} below.
\begin{exe} 
\ex \label{bg:ch5:diff-vb8} 
\begin{xlist}
\ex \label{bg:ch5:diff-vb8a} %a.
hyponym: \{\textit{стискам}\} ‘grasp: hold firmly’
\ex \label{bg:ch5:diff-vb8b} 
hyponym: \{\textit{притискам се}\} ‘clutch: hold firmly, usually with hands’
\ex \label{bg:ch5:diff-vb8c} 
hyponym: \{\textit{люлея}\} ‘cradle: hold gently and carefully’
\ex\label{bg:ch5:diff-vb8d} 
hyponym: \{\textit{сключвам}\} ‘interlace: hold in a locking position’
\ex \label{bg:ch5:diff-vb8e} 
hyponym: \{\textit{улавям}\} ‘trap: hold or catch as if in a trap’
\end{xlist}
\end{exe}

The restrictions for the \fename{Agent} of the root verb of the root verb and some of its hyponyms are different: For some verbs, \fename{Agent} is a volitional human being, which corresponds to the WordNet root synset eng-30-00007846-n: \{\textit{person}\} (Examples \ref{bg:ch5:diff-vb8b}, \ref{bg:ch5:diff-vb8d}), while in other cases  verbs can allow their Agent to be an animal, which corresponds to the WordNet root synset eng-30-08660339-n: \{\textit{animal}\} (Examples \ref{bg:ch5:diff-vb8a}, \ref{bg:ch5:diff-vb8b}), or \fename{Body\_part},  which corresponds to eng-30-03183080-n: \{\textit{body part}\}, as in (Example \ref{bg:ch5:diff-vb8e}).

The restrictions for the frame element \fename{Entity} are also not consistent in all discussed members of the tree. \fename{Entity} can be either an animate (Example \ref{bg:ch5:diff-vb8c}) or an inanimate physical object (Example \ref{bg:ch5:diff-vb8d}). 

\subsubsection {Verbs of contact by impact} \label{4.3.2}
Verbs of contact by impact denote a strong or forceful manner of physical contact. They include a wide range of verbs, the most typical of which are the verbs represented by the FrameNet frames\framename{Impact} and \framename{Destroying}. The frame \framename{Impact} represents \textit{an \fename{Impactor} in motion, making sudden, forcible contact with the \fename{Impactee}, or two \fename{Impactors} both move, mutually making forcible contact}. The frame \framename{Destroying} represents \textit{a \fename{Destroyer} (a conscious entity) or \fename{Cause} (an event, or an entity involved in such an event) affecting a \fename{Patient} negatively so that the \fename{Patient} no longer exists}. The core frame elements in those frames share similar general semantic characteristics, so more specific selectional restrictions can not be defined -- the \fename{Impactor} and the \fename{Impactee}, as well as \fename{Destroyer} and \fename{Patient} may be physical entities or eventualities, devices, or persons.  \editcolor{red}{The broad thematic range of frame elements of those verbs can be illustrated by the verbs belonging to the WordNet subtree stemming from eng-30-01236164-v: \{\textit{удрям}\} ‘hit: hit against; come into sudden contact with’ in Example \ref{bg:ch5:diff-vb4a}, \ref{bg:ch5:diff-vb4b}, whose semantics is represented by the FrameNet frame \framename{Impact}, and  verbs belonging to the WordNet subtree stemming from eng-30-01564144-v: \{\textit{унищожавам}\} ‘destroy: damage irreparably’ (Example \ref{bg:ch5:diff-vb4c},  \ref{bg:ch5:diff-vb4d})}.
\begin{exe}
\ex \label{bg:ch5:diff-vb4}
\begin{xlist}
\ex \label{bg:ch5:diff-vb4a}
hyponym: \{\textit{сблъсквам се}\} ‘shock: collide violently’
\ex \label{bg:ch5:diff-vb4b}
hyponym: \{\textit{разбивам се}\} ‘crash: undergo damage or destruction on impact’
\ex \label{bg:ch5:diff-vb4c}
\editcolor{red}{hyponym: \{\textit{опустошавам}\} ‘devastate: cause extensive destruction or ruin utterly’
\ex \label{bg:ch5:diff-vb4d}
hyponym: \{\textit{унищожавам}\} ‘ruin: destroy completely; damage irreparably’}
\end{xlist}
\end{exe}
\editcolor{red}{More rigid selectional restrictions on their frame elements impose some thematic groups within the class of verbs of contact via impact. Such are some of the verbs of digging which describe actions related to excavating or removing material from the ground are represented by the FrameNet frames \framename{Self\_motion}, \framename{Planting} and \framename{Mining}. They can denote fundamental excavating or digging into the ground: \textit{копая} `dig’, \textit{прокопавам} `burrow`; gardening: \textit{садя} `plant`, \textit{разкопавам} `delve’; mining and extraction of valuable resources, or materials: \textit{вадя} `extract`; burial in graves: \textit{заравям} `bury`, \textit{закопавам} `inhume`, \textit{погребвам} `entomb`. The frame \framename{Self\_motion} represents \textit{a living being -- а \fename{Self\_mover}, who moves under its own direction along a \fename{Path}}; the frame \fename{Planting} describes an \textit{\fename{Agent} who puts the \fename{Theme} into the \fename{Ground} for the purpose of growing} and the frame \framename{Mining} represents \textit{a \fename{Miner} who attempts to obtain a desirable \fename{Resource}, rocks and minerals, located in a \fename{Place} being mined, by digging or tunneling in the ground}.
These conceptual frames share the semantics of forceful contact with the ground, represented by the frame elements \fename{Place}, \fename{Ground}, \fename{Path}.}

\subsection {State verbs of physical contact}\label{4.4}
This verb class includes the verbs from the FrameNet frames: \framename{Locative\_relation}, \framename{Being\_wet}, \framename{Distributed\_position}, \framename{Posture}, \framename{Spatial\_contact}, \framename{Surrendering\_possession}, \framename{Surrounding}, \framename{Scouring}.
These frames describe an \fename{Agent} (Protagonist), \fename{Item}, \fename{Theme}, \fename{Figure}, or another entity being on, in or in contact with an area or a substance \fename{Location}. Within the FrameNet frame to frame hierarchy, most of them are subframes of the frame \framename{State}, its subordinate \framename{Locative\_relation}, and \framename{Spatial\_contact}, described by the definition: \textit{A \fename{Figure} is located in contact with a \fename{Ground} or relative to a \fename{Ground} location}.
Based on the manner of the contact, they denote spatial or location-based contact and describe how objects are situated in relation to each other: \textit{rest, place, position, situate.} \tabref{tab:chapter5handle:keytotable3} represents the frames and their frame elements within this group as well as examples of lexical units.


\begin{table}
\small
\begin{tabularx}{\textwidth}{lQQQ}
\lsptoprule
Frame & Definition & Frame elements & Lexical units \\
\midrule
\framename{Locative\_relation} & A \fename{Figure} -- an entity or event is located relative to a \fename{Ground} location. & \fename{Figure}, \fename{Ground} & eng-30-01466978-v: \{\textit{граничи}\} ‘border: lie adjacent to another or share a boundary’\\
\midrule
\framename{Posture} & An \fename{Agent} supports their \fename{Body\_part} in a particular \fename{Location}. & \fename{Agent}; \fename{Location}; \fename{Body\_part} & verbs from the WordNet subtree stemming from eng-30-01547001-v: \{\textit{лежа}\} `lie: be lying, be prostrate; be in a horizontal position` \\
\lspbottomrule
\end{tabularx}
\caption{State verbs of physical contact}
\label{tab:chapter5handle:keytotable3}
\end{table}

Selectional restrictions: Some verbs impose more specific selectional restrictions on their \fename{Agent} such as the hyponyms of eng-30-01547001-v: \{\textit{лежа}\} `lie` in Example \ref{bg:ch5:diff-vb5} below. 

\begin{exe} 
\ex \label{bg:ch5:diff-vb5}
\begin{xlist}
\ex \label{bg:ch5:diff-vb5a} hyponym: \{\textit{пека се}\} ‘sunbathe: expose one's body to the sun’

\ex\label{bg:ch5:diff-vb5b} hyponym: \{\textit{изтягам се}\} ‘sprawl: sit or lie with one's limbs spread out’

\ex\label{bg:ch5:diff-vb5c} hyponym: \{\textit{излягам се}\} ‘recumb: lean in a comfortable resting position’

\ex\label{bg:ch5:diff-vb5d} hyponym: \{\textit{покривам}\} ‘overlie: lie upon; lie on top of’

\ex\label{bg:ch5:diff-vb5e} hyponym: \{\textit{почивам}\} ‘repose: lie when dead’

\ex\label{bg:ch5:diff-vb5f} hyponym: \{\textit{припичам се}\} ‘bask: be exposed’
\end{xlist}
\end{exe} 

For some of them the \framename{Agent} can be a volitional human being corresponding to the WordNet root synset eng-30-00007846-n: \{\textit{person}\} (Examples \ref{bg:ch5:diff-vb5e}, \ref{bg:ch5:diff-vb5f}) as well as an animal (Examples \ref{bg:ch5:diff-vb5a}, \ref{bg:ch5:diff-vb5b}, \ref{bg:ch5:diff-vb5c}, \ref{bg:ch5:diff-vb5d}), which correspond to the WordNet root synset eng-30-08660339-v: \{\textit{animal}\}. The frame element \fename{Location} is an adjunct in Bulgarian and can be omitted, which is why it is not discussed here.

\section{Valency patterns} \label{ch5:sec:5}
The observations on the syntactic behaviour of the studied verbs led to the delineation of several general syntactic constructions within the group:
\begin{itemize}
\item[(a)] \textit{NP (pro-drop subject) Verb NP (direct object -- \fename{Theme}/\fename{Item}) PP (non-obliga\-tory indirect object -- to/on/over NP \fename{Goal}/\fename{Source})}. 

This syntactic structure is typical for verbs selecting a Theme as an object within the frame \framename{Emptying}, e.g \textit{разлея} `pour` -- \textit{Разля чая по масата} `She spilled the tea over the table`. 

\item[(b)] \textit{NP (pro-drop subject) Verb NP (direct object -- Goal) PP (non-obligatory indirect object -- with NP \fename{Theme}).} 

This pattern is found with verbs taking the frame element \fename{Goal} as an object within the frame \framename{Filling}, for instance \textit{ намажа} `spread` -- \textit{Намаза филията с масло} `She spread the slice with butter`.

\item[(c)] \textit{NP (pro-drop subject) Verb NP (direct object -- \fename{Container}) PP (non-obligato\-ry indirect object -- with NP \fename{Theme}).} 

This type of structure is typical for verbs selecting the frame element \fename{Container} as an object within the frame \framename{Filling}, such as \textit{натоваря} `load` -- Нато\-вариха камиона с тухлите `They loaded the truck with the bricks`. 
\end{itemize}

\section{Discussion} \label{ch5:sec:6}
The process of annotation raised some interesting questions about the lan\-guage-specific lexicalisation patterns of some Bulgarian verbs in comparison to their English counterparts. This led to the conclusion that different word formation mechanisms in Bulgarian and English, such as derivation, compounding, and conversion, as well as lexical gaps, reflect differences in the semantic structure of lexemes. 

The syntactic realisation of some frame elements differs in the two languages. The obligatoriness of the syntactic realisation depends on the point of contact between the core frame elements. The English verbs of contact that encode one of the frame elements in their morphological structure – e.g. the instrument (knife), the resultant shape (slice), the covering material (paint), the container (box, bag), etc. – have a different lexical expression in Bulgarian. Not all Bulgarian equivalents have the frame element incorporated in their word structure. For example, the English verb eng-30-01364483-v: \textit{cream} , ‘put on cream, as on one’s face or body` – has no one-word equivalent in Bulgarian and is translated with the expression \textit{намазвам с крем} `cover with cream`, where \textit{крем} `cream` is \fename{Theme}, cf. Example \ref{bg:ch5:diff-vb6}: 
\begin{exe} 
\ex \label{bg:ch5:diff-vb6}
\gll 
 [{\textit{Тя}}]$_{\feinsub{Age}}$ \textit{\textbf{НАМАЗА}} [\textit{лицето си}]$_{\feinsub{Goal}}$ [\textit{с крем}]$_{\text{Thm}}$.
% \jambox{(bg)}
\\ 
 She covered {face-DEF REFL} {with cream}.\\ 
\glt `She creamed her face.`  
\end{exe}

On the other hand, some of the Bulgarian verb hyponyms express a specific manner by means of prefixation, for example \textit{разрязвам} ‘\{\textit{cut}\}: cut into pieces`. Such predicates lexicalise a meaning component which specifies a scale of motion or state and contact, and they do not have full one-word equivalents in English. These and other similar cases have necessitated the modification of FrameNet frames or the definition of further specifications. 

\section{Conclusions and future work}\label{ch5:sec:7}
The study described in this paper provides a semantic description of some verbs from WordNet with the semantic primitive `contact verbs’, which belong to the general vocabulary of Bulgarian. Only the most frequent cases with a larger coverage in BulNet were discussed. Based on the information obtained both for the semantic hierarchy from BulNet and WordNet and for the semantic frames from FrameNet, the class of contact verbs is divided into two main classes of predicates: Verbs of physical contact via motion and Verbs of physical contact in state.
The selection of data and their description is based on the assumption that the semantic of verbs of contact involve actions or events in which two or more entities come into physical contact with each other. This verb group includes a wide range of verbs that describe various aspects of contact between subjects, objects, entities or domains. The core elements of the semantic frames representing the domain include the entities in contact – the one initiating the contact and the one being contacted; the direction of the contact – one entity reaching out to touch or affect another entity; manner of the contact – direct, light, forceful, gentle, exploratory. Semantic subsets based on frame semantics are provided within the Verbs of physical contact via motion and the Verbs of physical contact in state, together with the description of syntactic properties and the definition of more specific selectional restrictions for each verb. The work allows conclusions to be drawn about the internal semantic organisation of verbs within the domain of verbs with the semantic attribute \textit{physical contact}.
The analysis contributes to the development of a theoretically and empirically coherent approach to usage data and to the study of their specific features.

For the future, a more detailed analysis of certain subclasses within the verbs of contact is needed, focussing on their syntactic realisation and alternations, which will lead to an enrichment of WordNet and FrameNet.

Since the proposed analysis is based on multilingual resources such as WordNet and FrameNet, some of the observations may also be useful for other languages and contribute to the implementation of NLP applications aiming at automatic semantic analysis, word sense disambiguation, language understanding and generation, machine translation, etc.

\section*{Abbreviations}
\begin{multicols}{2}
\begin{tabbing}
MMM \= Agent\kill
\scshape Age \> \fename{Agent}\\
\scshape Conn \> \fename{Connector}\\
\scshape Ent \> \fename{Entity}\\
\scshape Ins \> \fename{Instrument}\\
NP \> Noun phrase\\
PP \> Prepositional phrase\\
\scshape Thm \> \fename{Theme}\\
VP \> Verb phrase
\end{tabbing}
\end{multicols}

\section*{Acknowledgements} 

This research is carried out as part of the project \emph{Enriching Semantic Network WordNet with Conceptual Frames} funded by the Bulgarian National Science Fund, Grant Agreement No. KP-06-H50/1 from 2020.


{\sloppy\printbibliography[heading=subbibliography,notkeyword=this]}
\end{document}
