\documentclass[output=paper,colorlinks,citecolor=brown]{langscibook}
\ChapterDOI{10.5281/zenodo.15682186}
\title[Language-independent and language-specific properties]{Language-independent and language-specific properties of semantic description: A case study on verbs of communication} 
\author{Svetlozara Leseva\orcid{0000-0001-8198-4555}\affiliation{Department of Computational Linguistics, Institute for Bulgarian Language, Bulgarian Academy of Sciences} and Ivelina Stoyanova\orcid{0000-0003-3771-435X}\affiliation{Department of Computational Linguistics, Institute for Bulgarian Language, Bulgarian Academy of Sciences}
}

\abstract{The study focuses on the properties of verb conceptual description in view of their linguistic universality and transferability of conceptual information across languages. Further, we present the semantic class of verbs of communication, the hierarchical organisation of frames and the corresponding frame elements. We consider the most prominent FrameNet frames evoking verbs of communication of higher frequency and make observations on the syntactic realisation of the frame elements in different valence patterns both in English and Bulgarian.}


\IfFileExists{../localcommands.tex}{
   \addbibresource{../localbibliography.bib}
   \usepackage{langsci-optional}
\usepackage{langsci-gb4e}
\usepackage{langsci-lgr}

\usepackage{listings}
\lstset{basicstyle=\ttfamily,tabsize=2,breaklines=true}

%added by author
% \usepackage{tipa}
\usepackage{multirow}
\graphicspath{{figures/}}
\usepackage{langsci-branding}

   
\newcommand{\sent}{\enumsentence}
\newcommand{\sents}{\eenumsentence}
\let\citeasnoun\citet

\renewcommand{\lsCoverTitleFont}[1]{\sffamily\addfontfeatures{Scale=MatchUppercase}\fontsize{44pt}{16mm}\selectfont #1}
  
   %% hyphenation points for line breaks
%% Normally, automatic hyphenation in LaTeX is very good
%% If a word is mis-hyphenated, add it to this file
%%
%% add information to TeX file before \begin{document} with:
%% %% hyphenation points for line breaks
%% Normally, automatic hyphenation in LaTeX is very good
%% If a word is mis-hyphenated, add it to this file
%%
%% add information to TeX file before \begin{document} with:
%% %% hyphenation points for line breaks
%% Normally, automatic hyphenation in LaTeX is very good
%% If a word is mis-hyphenated, add it to this file
%%
%% add information to TeX file before \begin{document} with:
%% \include{localhyphenation}
\hyphenation{
affri-ca-te
affri-ca-tes
an-no-tated
com-ple-ments
com-po-si-tio-na-li-ty
non-com-po-si-tio-na-li-ty
Gon-zá-lez
out-side
Ri-chárd
se-man-tics
STREU-SLE
Tie-de-mann
}
\hyphenation{
affri-ca-te
affri-ca-tes
an-no-tated
com-ple-ments
com-po-si-tio-na-li-ty
non-com-po-si-tio-na-li-ty
Gon-zá-lez
out-side
Ri-chárd
se-man-tics
STREU-SLE
Tie-de-mann
}
\hyphenation{
affri-ca-te
affri-ca-tes
an-no-tated
com-ple-ments
com-po-si-tio-na-li-ty
non-com-po-si-tio-na-li-ty
Gon-zá-lez
out-side
Ri-chárd
se-man-tics
STREU-SLE
Tie-de-mann
}
   \boolfalse{bookcompile}
   \togglepaper[23]%%chapternumber
}{}

\begin{document}
\maketitle

\section{Introduction}\label{intro} 

In this paper we focus on combining the semantic description available for verbs in different lexical semantic resources (WordNet and FrameNet) which contain complementary semantic information \citep{Baker2009}. We discuss the aspects of universality of conceptual knowledge that enable the transfer of semantic and to a lesser extent syntactic information across resources and languages. Further, we analyse the language-specific properties of the semantic and syntactic description. We illustrate our findings in a case study on verbs of communication in English and Bulgarian.


For the purposes of the study we employ: (a) the Princeton WordNet, PWN \citep{Fellbaum1998}, and the Bulgarian WordNet \citep{koeva2021-wordnet}, and (b) FrameNet \citep{Baker1998,Ruppenhofer2016}. In particular, we centre on the information included in them and how they complement each other in terms of coverage of lexical units and with respect to the semantic and syntactic features of the description. While we use resources for English and Bulgarian, the principles adopted in this work are applicable to other languages for which a wordnet aligned with PWN is developed.%\footnote{The alignment at the synset level, usually implemented by means of synset IDs shared across wordnets enables the use of the mapping between the wordnet synsets and the FrameNet frames}.

There are several other resources relevant to our study, which provide background on the approaches for the extensive language-specific description of verb classes in comparison to developing cross-lingual and multilingual lexical and semantic resources. Further, their brief review sheds light on the possibilities for combining resources aiming at comprehensive description of lexical units. The functionalities and the additional information contained in these resources are summed up below.

VerbNet \citep{Kipper-Schuler2005,Kipper2008} provides substantial coverage of the English verb inventory and defines syntactic-semantic relations in an explicit way by means of predicate-argument structures (defined as configurations of thematic roles) with one-to-one linking to the syntactic category (type of phrase) and grammatical function (subject, object, etc.) of each argument expressed in terms of a relatively small number of syntactic frames. Selectional restrictions are defined for the thematic roles assigned to a verb’s arguments; these restrictions capture the semantic/ontological class of the nouns that express the arguments. However, although the verb classes describe the syntactic behaviour of verbs, many of the traditional thematic roles employed may be too general for an exhaustive semantic description and appropriate handling of the syntax-semantics interface, while the syntactic description is often biased towards English. Moreover, the overlap (and hence, the coverage of the existing mappings) between the WordNet synsets and the VerbNet classes is not large enough to provide sufficient data for analysis.

VerbAtlas \citep{di-fabio-etal-2019-verbatlas} is a lexical-semantic resource representing the semantic description of the verb synsets in BabelNet. BabelNet is a very large, richly populated multilingual semantic network (covering more than 500 languages) integrating lexicographic and encyclopaedic knowledge from WordNet and Wikipedia \citep{navigli-ponzetto-2010-babelnet}. Each verb synset in VerbAtlas is assigned a frame corresponding to its prototypical predicate-argument structure. Obligatory components are described using 26 semantic roles and the semantic restrictions governing their compatibility (116 types). A semantic annotation API with the frames described in it is also provided with the resource. 

Predicate Matrix \citep{lopez-de-lacalle-etal-2014-predicate} is a lexical resource resulting from the integration of several sources of predicate information: FrameNet, VerbNet, PropBank and WordNet, that have been previously aligned in Semlink.\footnote{\url{https://verbs.colorado.edu/semlink/}} \citep{Palmer2009} Predicate Matrix is compiled using advanced graph-based algorithms to extend the mapping coverage between resources. Additionally, by exploiting SemLink, new role mappings are inferred among the different predicate schemas. 

%Тhe SynSemClass lexicon\footnote{https://ufal.mff.cuni.cz/synsemclass} has marked a notable effort towards combining the rich semantic description in the Vallex dictionary family with conceptual and syntactic information from external semantic resources towards the creation of a multilingual contextually-based verb lexicon. The aim of the lexicon is to provide a resource of classes of verbs that compares their semantic roles as well as their syntactic properties \citep{Uresova2020a}. In addition, each entry is linked to FrameNet, WordNet, VerbNet, OntoNotes and PropBank, as well as the Czech VALLExample 

The alignments of WordNet and FrameNet have been proposed for different languages, such as Danish \citep{pedersen-etal-2018-danish}, Dutch \citep{Horak2008TheDO}, Korean \citep{Gilardi2018LearningTA}, among others. One of the challenges in mapping resources developed according to different methodologies is the coverage of the alignment between the units represented in them. For instance, the alignment between lexical units evoking particular frames in FrameNet and corresponding verbs in synonym sets in WordNet, achieves coverage of 30.5\% \citep{Stoyanova2019}. New methods have been proposed to increase the coverage by discovering suitable literals based on semantic relations with literals already described in semantic frames \citep{Burchardt2005}.  

Combining the semantic description of verbs from different resources has been proposed by \citet{Uresova2020a,Uresova2020b}. The result is a multilingual dictionary encoding a comprehensive description of the semantic classes of verbs and the semantic roles and syntactic properties of their arguments.\footnote{\url{https://ufal.mff.cuni.cz/synsemclass}} The project is also aimed at creating an ontology of events, processes and states, and for this purpose each dictionary entry is linked to its correspondences in FrameNet, WordNet, VerbNet, Ontonotes and PropBank, as well as the Valence Dictionary of Czech Verbs \citep{Lopatkova2016}, which represents the predicate-argument structure of each verb, its semantic class and the syntactic transformations (diatheses) in which it participates.


Our work on aligning conceptual resources relies on the notion of universality. We side with the idea that the conceptual description provided in the FrameNet frames is to a considerable degree language-independent, which makes it possible for it to be transferred and/or adapted from one language to another. We map the conceptual knowledge contained in FrameNet onto the Princeton WordNet and through it, onto the Bulgarian WordNet. We then go on to examine the feasibility of transferring the valence information %and syntactic patterns 
described for English to Bulgarian and the language-specific features that need to be addressed. The combination of semantic and syntactic information is seen as a possible way of transferring knowledge across languages (especially underresourced ones) by relying on the universality of the semantic description. 

The study is organised as follows. \sectref{sec:resources} briefly presents the lexical-se-mantic resources involved in the work as well as the corpora used for extracting examples illustrating the various syntactic realisations in English and Bulgarian. \sectref{sec:mappings} discusses the mapping of FrameNet frames onto WordNet synsets with a view to the universality of conceptual description as the main principle for cross-lingual transfer. \sectref{sec:communication} offers a detailed analysis of the semantic class of verbs of communication in terms of their conceptual structure and frame elements involved in the relevant frames. This analysis serves as a case study illustrating the main principles of universality as well as the language-specific features of syntactic realisation of frames. \sectref{sec:conclusion} draws conclusions based on the analysis and gives some directions for future work.


\section{Resources}\label{sec:resources}

Below we describe in brief the lexical semantic resources used in the study, focusing on their strengths and the ways of overcoming their possible limitations through integrating the information contained in them. We also describe the corpora serving as a source of examples, the methodology for extracting suitable examples and the annotation of frame elements and their syntactic realisation.
 
\subsection{Lexical-semantic resources}

\subsubsection{WordNet}

WordNet\footnote{\url{https://wordnet.princeton.edu/}} \citep{Miller1995,Fellbaum1998} is a large lexical database that represents comprehensively conceptual and lexical knowledge in the form of a network whose nodes denote cognitive synonyms (synsets) linked by means of a number of conceptual-semantic and lexical relations such as hypernymy, meronymy, antonymy, etc. WordNet provides extensive lexical coverage; the verbs presented in it are organised in 14,103 synsets (including verb synsets specific for Bulgarian). In this work, we use both the Princeton WordNet and the Bulgarian WordNet \citep{koeva2021-wordnet}, which are aligned at the synset level by means of unique synset identifiers. 

WordNet provides the most coarsely-grained semantic division in terms of a set of language-independent semantic primitives assigned to all the nouns and verbs in the resource. The verbs fall into 15 groups, such as verb.change (verbs describing change in terms of size, temperature, intensity, etc.), verb.cognition (verbs of mental activities or processes), verb.motion (verbs of change in the spatial position), verb.communication (verbs describing communication and information exchange), etc.\footnote{The division of the nouns and verbs into WordNet lexicographic files (reflecting the semantic primitive distinction) along with short definitions of the primitives are available at: \url{https://wordnet.princeton.edu/documentation/lexnames5wn}.}

Verb synsets are interrelated and form a hierarchical structure based on a troponymy relation which represents a manner relation and is to a great degree analogous to hypernymy; for example, in \textit{talk}.v – \textit{whisper}.v the second member of the pair refers to a particular, semantically more specified, manner of performing the action referred to by the first verb \citep{Fellbaum1999a}. 


\subsubsection{FrameNet}

FrameNet\footnote{\url{https://framenet.icsi.berkeley.edu/}}  \citep{Baker1998,Baker2008} is a lexical semantic resource which couches lexical and conceptual knowledge in the apparatus of frame semantics. Frames are conceptual structures describing types of objects, situations, or events along with their components -- frame elements \citep{Baker1998,Ruppenhofer2016}. Depending on their status, frame elements may be core, peripheral or extra-thematic \citep{Ruppenhofer2016}. We deal primarily with core frame elements, which instantiate conceptually essential components of a frame, and which in their particular configuration make a frame unique and different from other frames.

FrameNet frames are organised into a hierarchical network by means of a number of hierarchical and non-hierarchical frame-to-frame relations \citep[81--84]{Ruppenhofer2016}. Here we list the hierarchical relations, which bear most relevance to the internal structure of verb classes. These are: \FrameRelation{Inheritance} – a relationship between a parent frame and a more specific (child) frame, such that the child frame elaborates the parent frame; \FrameRelation{Uses} (also called “weak inheritance”) – a relationship between two frames where the first one makes reference in a very general kind of way to the structure of a more abstract, schematic frame; \FrameRelation{Perspective} – a relation indicating that a situation viewed as neutral may be specified by means of perspectivised frames that represent different possible points-of-view on the neutral state-of-affairs; \FrameRelation{Subframe} – a relation between a complex frame referring to sequences of states and transitions, each of which can itself be separately described as a frame, and the frames denoting these states or transitions. 


\subsection{Corpora}

\subsubsection{Semantically annotated corpora: SemCor and BulSemCor}

In order to explore the syntactic expression of the verbs and their participants, we study the usage examples available in two semantically annotated corpora -- the English SemCor and the Bulgarian semantically annotated corpus, BulSemCor, both of which are annotated with WordNet senses. 
 
SemCor (current version 3.0) \citep{miller-etal-1993-semantic,miller-etal-1994-using,landes1998} is compiled by the Princeton WordNet team and covers texts excerpted from the Brown Corpus. SemCor is supplied with POS and grammatical tagging and all open-class words (both single words and multiword expressions, as well as named entities) are semantically annotated by assigning each word a unique WordNet sense (synset ID). The corpus is the largest manually annotated corpus of this kind and amounts to a total of 226,040 sense annotations.

BulSemCor \citep{koeva-2006-bulsemcor,koeva-2011-bulsemcor} is designed according to the general methodology of the original SemCor and criteria for ensuring an appropriate coverage of contemporary general lexis. In addition to open-class words, BulSemCor includes annotation of prepositions, conjunctions, particles, pronouns and interjections; for that purpose the Bulgarian WordNet has been expanded with closed-class words \citep{koeva-2011-bulsemcor}. The size of the corpus is close to 100,000 annotated units. 

The size of the two corpora is not sufficient to provide enough evidence for many of the studied verbs so examples from other corpora have also been employed. 

\subsubsection{Bulgarian-English parallel corpus}

The Bulgarian-English Sentence- and Clause-Aligned Corpus (BulEnAC)\footnote{\url{https://dcl.bas.bg/en/resources\_list/bulenac/}} \citep{Koeva-et-al2012} is a parallel corpus of aligned Bulgarian and English sentences and clauses with annotation of the syntactic relation between clauses. The corpus contains 366,865 tokens (176,397 tokens in Bulgarian and 190,468 tokens in English). %The texts in BulEnAC cover five categories: administrative texts, fiction, journalism, science, informal texts. The Bulgarian subcorpus contains 14,667 sentences (with an average sentence length of 12.02 words), while the English subcorpus includes 15,718 sentences (with an average sentence length of 12.11 words). The average number of clauses per sentence is 1.67 for Bulgarian and 1.85 for English. 

The syntactic annotation of BulEnAC involves:
a) sentence and clause splitting;
b) annotation of the type of syntactic relation (coordinate or subordinate) between clauses;
c) marking of the elements that introduce the clause: conjunctions, complementisers, and punctuation.

BulEnAC is suitable for extracting parallel sentences that illustrate the use of particular verbs evoking the frames under study. Further, it facilitates the identification of corresponding translation equivalents within aligned clauses.

\subsubsection{The Bulgarian National Corpus}

The Bulgarian National Corpus is the largest corpus for Bulgarian: it consists of a monolingual (Bulgarian) part and 47 parallel corpora and amounts to 5.4 billion words. The Bulgarian part includes about 1.2 billion words of running text distributed in 240,000 text samples. The texts in the corpus reflect the state of the Bulgarian language predominantly in its written modality from the middle of the 20th century (1945) until the present day  \citep{Koeva2012}. The search engine developed for the exploration of the corpus allows the extraction of information according to complex grammatical criteria. We use the corpus to study the syntactic expression and the validity of the valence patterns described in \sectref{sec:communication} in addition to the examples extracted from the semantically disambiguated part of the corpus (BulSemCor).

\subsection{Motivation for combining WordNet and FrameNet}

It has long been acknowledged that combining WordNet with conceptual resources such as FrameNet results in more comprehensive semantic and syntactic representation of the lexical entries \citep{Baker2009,Schneider2012,das-etal-2014-frame}, thus expanding the possible applications of the resources for the purposes of syntactic and semantic parsing. Elaborating a bit on the discussion of the strengths and shortcomings of the different kinds of lexical semantic resources offered by \citet{Shi2005}, we may point out the following motivation for putting effort into their alignment. 

FrameNet provides a rich semantic description of the predicates using schematic representations (frames) of the configurations of “participants and props” (elements corresponding to the surrounding circumstances or other supporting facets of meaning, in the sense of \cite[7]{Ruppenhofer2016}) that define the situation described. The corpus of sentences annotated with explicit and implicit frame elements supplies empirical evidence about the syntactic realisations of semantic frames that is particularly valuable not only for linguistic generalisations about the target language (English) but also as a point of departure for making observations cross-linguistically. Besides the explicit syntactic expression, the annotators have marked non-overt but conceptually present frame elements retrievable from the immediate or the more general context (so-called null instantiations). However, while formulating ontological semantic types that classify lexical units, frames and frame elements and in the latter case denote the selectional restrictions imposed on the fillers of frame elements \citep[86]{Ruppenhofer2016}, FrameNet does not explicitly define the content of these semantic types (see \sectref{universality-restrictions}, which provides the authors' suggestions regarding that). In addition, FrameNet's coverage is limited both in terms of the lexical units included in the frames (i.e. there are lexical units pertaining to a frame that are not listed in it) and in terms of the parts of the lexicon encompassed by the system of frames, i.e. there are lexical units that cannot be described properly by the existing frames. Finally, as some of the frame elements are too finely-grained, certain generalisations across frames and frame elements might be missed.

WordNet ensures vast lexical coverage of the English lexicon structured and enriched with lexical and semantic information in the form of synset glosses, usage examples, notes on the usage or grammatical specificities, and a rich network of semantic relations. However, WordNet encodes no explicit semantic information about the participants in the situations described by the predicates and only limited information about their syntactic behaviour. 

The combination of the resources requires: (i) mapping of the units that correspond to each other in the resources, i.e. discovering the counterparts of the synsets' members among the lexical units in FrameNet and linking them to the frames they evoke; (ii) expanding the mapping by discovering new candidates in WordNet to be matched to the relevant frames.  
Such mapping procedures are discussed in \sectref{sec:mappings}. The limitations stemming from the lack of appropriate frames to describe certain parts of the lexicon need to be addressed by defining new frames.

The greater granularity of the frame elements in FrameNet (as compared with VerbNet, VerbAtlas and other resources) is handled, where necessary, by applying a shallow hierarchy derived from the hierarchical organisation of the frames and the inheritance relations defined between them \citep{Litkowski-2014-framenet}. %\footnote{\url{https://www.clres.com/clr/fetax.php}} 
Consider for instance the taxonomy of frame elements \fename{Air} > \fename{Fluid} > \fename{Theme} derived from the frame hierarchy \framename{Breathing} > \framename{Fluidic motion} > \framename{Motion} built on the frame-to-frame relation of \FrameRelation{Inheritance} between the three frames. In certain contexts and for certain tasks it may be more appropriate to make reference not to the most specific \fename{Air} but to \fename{Fluid} or even to \fename{Theme}, or vice versa. The maintaining of the different levels of granularity provides a more robust semantic description that is relatively resource- and theory-independent.

While genuinely beneficial, the mutual enrichment of WordNet and FrameNet is by no means trivial, as senses of the synsets and the lexical units that may be thought as equivalent may in fact not correspond well. The use of corpus occurrence and especially the study of annotated examples help in elucidating both theoretical and pragmatic aspects of the alignment between the resources and informs the judgments made in the course of the manual validation of the automatic assignment of frames to synsets. The case study presented in \sectref{sec:communication} may be viewed as the result of such analysis.

\section{Mapping between WordNet and FrameNet based on universal principles}\label{sec:mappings} 

Both resources have shown to be sufficiently language-independent as to provide an approximation at a description across typologically distinct languages. Both models have been transferred and adapted cross-linguistically. These include coordinated attempts to build multilingual resources or link existing independent resources through projects such as EuroWordNet \citep{Vossen2004} or Global WordNet \citep{mccrae-etal-2021-globalwordnet}, as well as Multilingual FrameNet \citep{Gilardi2018LearningTA}, among others. 

Our work expands on the notion of universality and cross-lingual applicability of lexical-semantic resources by linking the resources to each other and then transferring the language-independent (semantic and conceptual) description of English verbs in WordNet onto the Bulgarian lexical units in the Bulgarian WordNet.

\subsection{Universality of semantic inheritance relations between synsets and between frames}

The two resources have been aligned automatically by employing existing mappings (\cite{Tonelli2009}, \cite{Palmer2014}, among others) with additional implemented procedures for expansion and validation \citep{Leseva2018} and later refined \citep{Stoyanova2019,Leseva2020}; these procedures involve the mapping of FrameNet frames to WordNet synsets on the basis of the inheritance of conceptual features in hypernym trees, i.e., by assigning frames from hypernyms to hyponyms where possible and implementing a number of validation procedures based on the structural properties of the two resources, primarily the relations encoded in them. This has resulted in 13,104 automatic alignments, of which over 6,000 have been validated and corrected manually in the framework of this project and previous initiatives.

\figref{fig:01} illustrates a hypernym--hyponym pair of synsets, with the appropriate FrameNet frames assigned to them, which are themselves related by means of an inheritance relation (\framename{Cooking\_creation} being an elaboration of the mother frame \framename{Intentionally\_create}).

\begin{figure}
\includegraphics[width= \textwidth]{figures/uni_fig01.png}
\caption{Frame inheritance (\framename{Intentionally\_create} $\rightarrow$ \framename{Cooking\_creation}) as reflected in the hypernym relation (\textit{make, create} $\rightarrow$ \textit{cook}).}\label{fig:01}
\end{figure} 

\subsection{Universality of selectional restrictions}\label{universality-restrictions}

Part of the FrameNet frame elements are supplied with `semantic types' %(selectional restrictions) 
defining noun classes that narrow down the set of possible nouns that may be realised in the respective positions in the semantic frame. These semantic types are to a great degree relevant cross-linguistically, as they define ontological distinctions that underlie human cognition. To the best of our knowledge,  the list of the FrameNet types and the pertaining definitions have not been made available, but their semantic content can be intuitively construed by speakers from the relevant designations, such as Sentient, Physical object, etc.). As noted in \citet[86]{Ruppenhofer2016} most ontological semantic types ``correspond directly to synset nodes of WordNet, and can be mapped onto ontologies, e.g. Cyc or the Knowledge Graph''. The FrameNet semantic types form a semantic type hierarchy, which, however, does not necessarily correspond to that of WordNet or any other resource. Most of the frame-to-frame relations enable the propagation of the ontological semantic types of the parent frame and its frame elements down to the child frame and its frame elements \citep[99]{Ruppenhofer2016} as well as to the lexical units in the respective frame \citep[86]{Ruppenhofer2016}.
%These selectional preferences, however, are not aligned with concrete lexicalised classes, although they are relatively easy to map to a part of an ontology that would make it possible to populate them with instances. 
Using a linguistic taxonomy (moreover one implemented for numerous languages such as WordNet) to describe the selectional restrictions imposed by verbs on the nouns that fill the positions of their arguments has been proposed in different frameworks \citep{agirre-martinez2002,Koeva:2010}. While the particulars differ, the general idea is the same as the one adopted in FrameNet, i.e. to represent semantic constraints in the form of taxonomically definable classes.
%they can thus be defined as (a combination of) WordNet substructures, i.e. hypernym-hyponym trees. This type of formulation is facilitated by the hierarchical and largely cross-linguistically applicable definition of the WordNet structure.

\subsection{Universal and language-specific aspects of valence frames and syntactic realisation}

Through the alignment between frames and synsets, each verb in WordNet is associated with a number of valence patterns defined for the lexical units evoking a given frame in FrameNet. While the semantic component of the description is language-independent, the syntactic component is more language-specific as the realisation of the frame elements depends on the syntactic properties of each language. Even so, we assume that the valence patterns that underlie the syntactic expression are valid cross-linguistically to a considerable degree as they are grounded in human cognition and the conceptualisation of situations. More precisely, valence patterns describe ``the semantic and syntactic combinatory possibilities'', or valences of lexical units \citep[7]{Ruppenhofer2016}. They thus refer to the co-occurrence combinations of frame elements (both core and non-core) attested for each annotated lexical unit in the FrameNet annotated corpus.

The second, more language-specific level of syntactic description consists of the \emph{syntactic categories and grammatical functions} by which a particular frame element for a given lexical unit is expressed. Even at this level, for many (related) languages one can observe similar syntactic expression especially with respect to the participants that are selected as the subject and the object. A great degree of differentiation may be found at the level of certain grammatical peculiarities and constructions -- for instance, unlike English, Bulgarian lacks \textit{-ing} and infinitive clauses, so propositional complements will be realised as finite clauses; Bulgarian has impersonal verbs and subjectless sentences and does not make use of pleonastic subjects. Of course, there may be mismatches in the syntactic categories across languages, e.g. a certain frame element may be a direct object in one language and a prepositional object in another. Languages may also differ in terms of the overtness of syntactic information, i.e. the possibility to leave an obligatory element non-explicit (null instantiations retrievable from the context or the grammatical construction); the language-specific diatheses, constructions, word order, morphosyntactic features, etc. The inventory of means that introduce certain frame elements such as prepositions, conjunctions, wh-words, etc. may also vary across languages. 
 
The linking from the semantic level of the frame elements to the syntactic level of patterns of co-occurrence and syntactic categories in FrameNet is implemented in a straightforward manner by associating each frame element with a syntactic category and possibly a grammatical function -- e.g. subject (NP.Ext) and object (NP.Obj). 

Example \ref{ex:01} shows a partial representation of the valence patterns and the syntactic realisation of the verb \textit{teach} in the FrameNet frame \framename{Education\_teaching}.

\begin{exe}
 \ex  \label{ex:01}
     \begin{xlist}
         \ex \gll \fename{Teacher} \fename{Institution}\\
NP.Ext PP[\textit{at}] \\
         \ex \gll \fename{Teacher} \fename{Student} \fename{Subject}\\
NP.Ext NP.Obj PP[\textit{about}] \\
         \ex \gll \fename{Teacher} \fename{Student} \fename{Skill}\\
NP.Ext NP.Obj Sinterrog/VPto \\
     \end{xlist}
\end{exe}
 

To sum up, even though there may be typological cross-linguistic differences in the conceptualisation and expression of situations for many language pairs, English and Bulgarian including, there are also parallels that facilitate the transfer of information across languages at the semantic and possibly at the syntactic level. Even where direct transfer of the syntactic description is not justified, the valence patterns and the syntactic realisation lattices taken from FrameNet may serve as a point of departure in the analysis of the Bulgarian syntactic data: they help establishing what is valid or invalid in Bulgarian by comparing the syntactic properties of the Bulgarian verbs to those of their English counterparts and the example sentences in the resources.   

\section{A case study: Verbs of communication}\label{sec:communication}

Below we offer an analysis of a selection of verbs of communication as an illustration of the universal principles and the language-specific features of the adopted linguistic description. 

The domain of speech act verbs and their classification have been discussed by many authors (\cite{Wierzbicka1987}, \cite[202--211]{Levin1993}, \cite{Levin-et-al1997}, \cite{UrbanRuppenhofer2001}, \cite{Boas2010}, among others), including for Bulgarian (\cite{Nitsolova2008kompl,Penchev1998,Tisheva2000,Tisheva2004vapros,Koeva2021kompl}, among others). While previous work in this area has served to inform the current state of the linguistic knowledge about the semantic and syntactic properties of communication verbs, the analysis below is based primarily on our observations on the descriptions proposed in FrameNet for English and exploring and extending them to Bulgarian. 


First, we identify the ``basic'' frame which describes the general scenario or situation characterising the domain of communication in terms of the participants and circumstances involved and the relations among them \citep[16]{Johnson2001}. This general scenario is then elaborated in various ways in more specific frames. The semantic generalisations among such frames exhibiting different levels of abstraction and specialisation are typically cast in the form of frame-to-frame relations based on the inheritance among the semantic descriptions or parts of them. 


\begin{figure} 
\includegraphics[width=0.95\textwidth]{figures/uni_fig02.png}
\caption{The hierarchical organisation of FrameNet frames describing the verbs of communication.}\label{fig:hierarchy}
\end{figure}



The hierarchical organisation of the domain of communication verbs is presented in \figref{fig:hierarchy}.


Starting from this basic, or prototypical frame, we delve into several of the frames inheriting from it in order to show what kinds of processes are involved in the semantic specialisation and how this is reflected in the semantic description. The frames are selected based on the frequency of the verbs evoking them in the annotated data or with the objective to illustrate particular aspects of the analysis. For each such frame (including the prototypical one), we consider: (i) its semantics in terms of the frame definition, constellation of core frame elements that represent the main participants in the situation, and the relations among them, (ii) the syntactic expression of the frame elements, and (iii) the specifics of their realisation in Bulgarian as compared to English. The semantic and syntactic aspects referred to in (i) and (ii) are mostly taken for granted as represented in the FrameNet annotated corpus. In presenting each frame inheriting from the prototypical one, we do discuss how the conceptualisation of the basic frame is specialised or narrowed down and how this is reflected in the number of frame elements and the relationships among them. The main burden of our work is focused on (iii), i.e. the analysis of the syntactic expression of the frame elements as attested in the corpus compiled for Bulgarian. The valence patterns emerge from the annotated examples and are thus specified independently from the English data. The same holds for the syntactic information (syntactic function and syntactic category of the expressed frame elements). The tagset of categories is adapted from the FrameNet corpus so that the notations in the two annotated datasets are unified. 

Although there may be differences in the conceptualisation of situations across languages, we expect the semantic properties of the description to be largely shared between English and Bulgarian, as it has been shown by efforts undertaken for other languages (\sectref{intro}). Based on our preliminary observations, we also expect that at least part of the valence patterns will be relevant for both languages, i.e. the frame elements that tend to be expressed and the particular configurations in which they co-occur will be similar, even allowing for cross-lingual differences (such as the fact that Bulgarian, unlike English, is a pro-drop language). We then look at the syntactic expression of the patterns in terms of the grammatical function and the syntactic categories of the core frame elements and, where relevant, the possibility for their contextual construal (null instantiations).

We take as a point of departure the lattices of the frame elements and their syntactic realisations for certain verbs and the valence patterns of frame elements as described in the annotated FrameNet examples\footnote{\url{http://framenet.icsi.berkeley.edu/}} \citep{Burchardt2008}. %The dataset for English is supplemented with
In addition, below we also use examples from SemCor in order to illustrate the applicability of the FrameNet description independently of the annotation undertaken in the FrameNet corpus.

After analysing this information for English, we go on to observe to what extent it is applicable to Bulgarian. For this purpose, we have constructed a corpus of manually annotated examples extracted from BulSemCor and, where the number of examples is not sufficient, from the Bulgarian National Corpus. 
%For each frame, we study the verbs with sufficient number of attestations in the FrameNet annotated corpus and SemCor; the examples given for the English part of the analysis are sentences from these two corpora. Bulgarian examples are from BulSemCor and where necessary, we supplement these with manually selected examples from the Bulgarian National Corpus.

Each example sentence in the English and the Bulgarian dataset is annotated as shown in Example \ref{ex:annotation}. The English dataset consists of 93 verbs (lexical units in FrameNet) to which an appropriate communication-related frame is assigned. The verbs are aligned to 72 WordNet synsets. %using the WordNet to FrameNet mappings. %as outlined in \sectref{sec:mappings}. 
Each verb is supplied with a number of examples from the FrameNet corpus illustrating its valence patterns; the dataset contains a total of 4,525 illustration examples representing 863 different valence patterns. The annotation of each sentence in the Berkeley FrameNet corpus includes explicit annotation of the target word (in this case a verb) and the syntactic realisation of the frame elements.

The Bulgarian dataset covers 112 communication verbs (including aspectual pairs) across 63 WordNet synsets. As the corpus of annotated examples for Bulgarian is still work in progress, it is considerably smaller than the one for English: it contains 890 annotated sentences representing 136 different patterns. The annotation consists in labelling the sentence components with the frame elements they realise in a way consistent with the annotation in the Berkeley FrameNet. 

\begin{exe}
\ex \label{ex:annotation}
\begin{xlist}
\ex 
\glt FrameNet description: \textit{ask}.v `say something in order to obtain an answer or some information from someone', frame: \framename{Questioning}\\
WordNet alignment: \{\textit{ask}:4\} `address a question to and expect an answer from', synset ID: eng-30-00897746-v\\
BulNet alignment: \{\textit{питам}:2, \textit{попитвам}:1, \textit{попитам}:1, \textit{запитвам}:3, \textit{запитам}:3\}, synset ID: eng-30-00897746-v
\ex An adapted example from the FrameNet corpus with the relevant pattern:\\ 
{[\textit{They}]}$_{\feinsub{Com}}$  \textit{\textbf{ASKED}} [\textit{Rubbie}]$_{\feinsub{Addr}}$ [\textit{what she ate}]$_{\feinsub{Msg}}$.\\
\textbf{{[NP.Ext]}$_{\feinsub{Com}}$ VERB {[NP.Obj]}$_{\feinsub{Addr}}$ {[Sinterrog]}$_{\feinsub{Msg}}$}
\ex An annotated example from BulSemCor with the relevant pattern:\\
\gll[\textit{Престъпникът}]$_{\feinsub{Com}}$ \textit{\textbf{ПОПИТАЛ}} [\textit{полицая}]$_{\feinsub{Addr}}$ [\textit{дали може да си купи цигари}]$_{\feinsub{Msg}}$.\\
Criminal-\textsc{def} asked policeman-\textsc{def} {whether he could buy cigarettes}.\\
\glt `The criminal asked the policeman whether he could buy cigarettes.'\\
\textbf{{[NP.Ext]}$_{\feinsub{Com}}$ {[NP.Obj]}$_{\feinsub{Addr}}$ {[Sinterrog]}$_{\feinsub{Msg}}$} 
\end{xlist}
\end{exe} 


% \subsection{Hierarchical organisation of frames for verbs of communication}


%Here we study the verbs and frames that show important features of verbs of communication, illustrate their organisation and have sufficient number of attestations in the resources in order to allow reliable generalisations regarding the usage patterns they exhibit in text.

\subsection{The prototypical frame: \framename{Communication}}

As noted by \citet[108]{Johnson2001}, the frames in the domain of communication describe “verbal communication between people and inherit structure and frame elements from the higher-level frame \framename{Communication}”.  \framename{Communication} is thus the prototypical frame that represents the basic conceptual structure of the activity of communication as a configuration of five main interacting frame elements. This basic structure will be further elaborated (narrowed down, profiled or otherwise specialised) in the frames that inherit it.\footnote{By “inherit” we mean the relationships between the more general and the more specific frames between which the following hierarchical frame-to-frame relations hold: \FrameRelation{Inheritance}, \FrameRelation{Using}, \FrameRelation{Perspectivises}, \FrameRelation{Subframe}.} 

\begin{description}[font=\normalfont]
\item[Definition of the frame \framename{Communication}:] A \fename{Communicator} conveys a \fename{Message} to an \fename{Addressee}; the \fename{Topic} and \fename{Medium} of the communication may also be expressed. 
\end{description}

As described in the definition, the \framename{Communication} frame does not itself involve specification of the method of communication (speech, writing, gesture, etc.) but only the fact of it. The frames that inherit \framename{Communication} can add elaboration to the general idea in several ways: 

\begin{enumerate}[label=(\roman*)]
\item by specifying the \fename{Medium} in a variety of ways, such as the particular language (\textit{in French, in Russian}), or the physical entity or channel, e.g. a medium, technology, form, etc. (\textit{on the radio, in a letter, through the Messenger, in writing}). 
\item by specifying the manner of verbal communication according to various criteria such as loudness (e.g. \textit{shout}.v, \textit{whisper}.v); volubility and/or mood (e.g., \textit{babble}.v, \textit{rant}.v), distinctness (e.g., \textit{slur}.v, \textit{stutter}.v, \textit{mutter}.v), among many others;
\item specialisation may also mean that the more concrete frames inherit only part of the \framename{Communication} frame elements or do not inherit them in a straightforward manner. For example, \framename{Judgment\_communication} (which inherits from \framename{Statement}, in turn inheriting from \framename{Communication} according to the \FrameRelation{Using} relation, see \figref{fig:hierarchy} above) reinterprets the frame element \fename{Message} as a judgement on an \fename{Evaluee} according to a \fename{Reason}.
\end{enumerate}

The prototypical and the inheriting frames might exhibit a different construal of the relationship between certain frame elements. For instance (as pointed out in the description of \framename{Communication}), in the frame \framename{Chatting}, the \fename{Communicator} and \fename{Addressee} alternate their roles, and are often expressed by a single, plural NP, i.e. the relationship between them is not asymmetrical but reciprocal as they participate in the situation in the same way.

Another aspect of specialisation is the inability for overt expression of all the frame elements \citep[16]{Johnson2001}. For example, the lexical units \textit{talk}.v and \textit{speak}.v in the \framename{Statement} frame (which inherits \framename{Communication} according to the \FrameRelation{Using} relation) usually block the overt expression of \fename{Message}, although its existence is implied at the conceptual level (in their meaning). This is shown by the fact that in the annotated examples available for the two verbs the frame element \fename{Topic} is much more frequently expressed than \fename{Message}, although it is dependent on it (the topic characterises the message).

Another kind of elaboration is represented by the incorporation of frame elements \citep[164--165]{Jackendoff1990} whereby a certain frame element is integrated in the meaning of a verb as a result of which this frame element is usually left unexpressed \citep[30]{Ruppenhofer2016}. In the domain of \framename{Communication} the frame \framename{Communication\_means} describes situations that specify the concrete means with the aid of which communication takes place; the various \fename{Means} are thus incorporated in the meaning of the respective verbs, e.g. \textit{fax}.v, \textit{telephone}.v, \textit{email}.v.

The frame \framename{Communication} is evoked by a small number of verbs -- \textit{communicate}.v, \textit{convey}.v, \textit{indicate}.v, \textit{share}.v. Although pertaining to the prototypical frame, these verbs are not the most frequent ones associated with the activity of communicating, which are in fact described in more elaborate frames.  


\subsubsection{Prototypical frame elements in the domain of communication}

Below we present the prototypical frame elements of the \framename{Communication} frame as defined in FrameNet.

\begin{description}[font=\normalfont]\sloppy
\item[\fename{Communicator} (Semantic type: Sentient)] The sentient entity that uses language in the written or spoken modality to convey a \fename{Message} to the \fename{Addres\-see}. 
\item[\fename{Medium}] The physical or abstract setting in which the \fename{Message} is conveyed.
\item[\fename{Message} (Semantic type: Message)] A proposition or set of propositions that the \fename{Communicator} wants the \fename{Addressee} to believe or take for granted; in other words it is the content which is communicated. 
\item[\fename{Topic}]  The subject matter to which the \fename{Message} pertains.  It is thus a property of \fename{Message} \citep[17]{Johnson2001} and as a result its syntactic expression is also predetermined by the expression of the \fename{Message}.
\item[\fename{Addressee} (Semantic type: Sentient)]  The \fename{Addressee} is typically a person or organisation, etc. that receives a \fename{Message} from the \fename{Communicator}.\footnote{In the FrameNet frame \framename{Communication} the \fename{Addressee} is specified as a non-core element. However, we consider it is nonetheless implied in all examples from the FrameNet annotated corpus and thus analyse it in the set of prototypical frame elements.}
\end{description}

In the remainder of the chapter the data in the annotated corpora that are subject to analysis are organised as follows. We first show and discuss how each of the considered frame elements is realised at the level of the individual verbs evoking a given frame (the odd-numbered tables). This kind of presentation allows us to observe the expression of each frame element for each verb and the differences among verbs in the same frame. The data shown in the pairs of odd-numbered tables enable the comparison between English and Bulgarian and help in drawing conclusions about the correspondences and differences in the syntactic realisation between the two languages. These tables, however, do not represent the configurations of frame elements that actually occur in the annotated corpora. To illustrate those, we give a summarised list of the most characteristic valence patterns for each frame (i.e. the best-represented patterns in terms of numbers of examples) and the verbs that are observed in these configurations in the two languages (the even-numbered tables). The information in the subsequent odd- and even-numbered tables is thus complementary. Due to the currently insufficient number of examples even for many English verbs, we represent the valence patterns as an aggregate of the valence patterns for all verb,\footnote{In theory, the differences among the individual verbs are lost in this way, but since we do not have at our disposal large samples of annotated data for each verb, in practice, this is not relevant as the sparseness of data prevents us from making such detailed observations.} thus obtaining what we call generalised valence patterns. These give us an overall idea of the distribution of valence patterns across verbs and a point of departure for a more in-depth evidence based analysis.\footnote{The numbers in the tables for English are based on a version of the Berkeley FrameNet obtained in XML format in 2019.}

\subsubsection{Syntactic realisation of the \framename{Communication} frame elements}

The syntactic expression of the basic configuration of frame elements in the \framename{Communication} frame is exemplified in \tabref{tbl:communication-synt}.

\begin{table}
\fittable{\begin{tabular}{l rrrrrrrrr}
\lsptoprule
 & NP.Ext & NP.Obj & PP & AVP & NI & Clause & Quote & Other & Total\\
\midrule
\textit{communicate} &  &  &  &  &  &  &  &  & \\  
\fename{Communicator} & 39 &  &  &  & 5 &  &  &  & 44\\ 
\fename{Addressee} &  &  & 27 &  & 16 &  &  & 1 & 44\\ 
\fename{Message} & 3 & 23 &  &  & 14 & 1 &  &  & 41\\ 
\fename{Topic} &  & 1 & 3 &  & 1 &  &  & 1 & 6\\ 
\fename{Medium} & 2 &  & 3 &  &  &  &  & 1 & 6\\ 

\midrule
\textit{indicate} &  &  &  &  &  &  &  &  & \\  
\fename{Communicator} & 7 &  &  &  &  &  &  &  & 7\\ 
\fename{Addressee} &  &  &  &  & 8 &  &  &  & 8\\ 
\fename{Message} &  & 3 &  &  &  & 6 &  &  & 9\\ 
\fename{Medium} & 4 &  &  &  &  &  &  &  & 4\\ 

\midrule
%\textit{signal} &  &  &  &  &  &  &  &  & \\  
%\fename{Communicator} & 2 &  &  &  &  &  &  &  & 2\\ 
%\fename{Addressee} &  &  & 1 &  &  &  &  &  & 1\\ 
%\fename{Message} &  & 1 &  &  &  & 1 &  &  & 2\\ 
% \midrule
\textit{say} &  &  &  &  &  &  &  &  & \\  
\fename{Communicator} & 5 &  &  &  & 6 &  &  &  & 11\\ 
\fename{Medium} & 5 &  &  &  & 1 &  &  &  & 6\\ 
\fename{Message} & 6 &  &  &  &  & 9 & 2 &  & 17\\ 
\fename{Topic} &  &  & 1 &  & 1 &  &  &  & 2\\ 
%\textit{share} &  &  &  &  &  &  &  &  & \\  
%\fename{Communicator} & 1 &  &  &  &  &  &  &  & 1\\ 
%\fename{Message}&  & 1 &  &  &  &  &  &  & 1\\ 
%\fename{Topic} &  &  & 1 &  &  &  &  &  & 1\\ 
\lspbottomrule
 \end{tabular}}
 \caption{Syntactic expression of the \framename{Communication} frame elements of selected FrameNet lexical units. } 
    \label{tbl:communication-synt}
 \end{table}


\fename{Communicator} is expressed as the external argument, i.e. as a subject of the respective sentence or clause; as it is a sentient entity, it is realised as an NP. In a number of cases the frame element is realised as a definite null instantiation (DNI), i.e. it is retrievable from the previous context, or as a constructional null instantiation (CNI), where it is the grammatical construction that allows it to remain non-overt, e.g. in passive or infinitive clauses, etc.

Here and below, unless the distinction is specifically relevant, we consider INIs (indefinite null instantiations), CNIs (constructional null instantiations) and DNIs (definite null instantiations) as one category -- NI (null instantiations), together with the category INC (incorporated frame element), see \citet{petruck-2019-meaning}. The null instantiations are a very interesting category that merits a separate in-depth study. In particular, they may be considered as exponents of distinct properties, may stand for different syntactic categories and constituents with different grammatical functions, and respectively -- may participate in different valence patterns. However, the distinction among them is not trivial and especially the one between DNIs and INIs may require a broader context to be interpreted accurately. In addition, as this has not been the focus of study, sufficient number of examples and broad enough context has not been provided in the Bulgarian data.\footnote{The category `Other’ encompasses examples where a frame element is otherwise expressed. Due to the limited number of such instances, we omit them here.} 

With the verbs in this frame, \fename{Message} is typically realised as an object NP, as a complement clause (Example \ref{ex:02comm:a}) or as a quote. Quotes represent the content of the \fename{Message} as directly stated by the \fename{Communicator} in his or her own words, while clauses denote it as being retold by someone (such as in reported speech). A \fename{Message} realised as an NP constitute a nominalisation which rephrases its content in a more concise way or as a generalised idea. In about a third of the examples available for \textit{communicate}.v the \fename{Message} is annotated as an indefinite null instantiation (INI). This means that the verb is used intransitively: the \fename{Message} remains syntactically unexpressed and receives a certain typical interpretation without a specific discourse referent \citep{Ruppenhofer2016}, as in Example \ref{ex:02comm:b}. The INIs correspond to the activity use of certain types of verbs where the object remains implicit \citep{VanValinLaPolla1997}. 


The FrameNet examples show that \fename{Topic} is rarely expressed, with only several instances in the FrameNet corpus even for \textit{communicate}.v. Extrapolating from examples from other sources and the definition of the frame element, we may conclude that the \fename{Topic} is usually expressed as a prepositional phrase headed by the preposition \textit{about}. An alternative way of realising the \fename{Topic} is as a modifier of a noun expressing the \fename{Message} (Example \ref{ex:02comm:c}); such cases corroborate syntactically its semantic dependence on the \fename{Message} communicated. In the absence of an overt \fename{Message}, the \fename{Topic} may be expressed as an independent phrase (Example \ref{ex:02comm:d}); this is one of the typical patterns of its realisation as attested in the more specific communication frames.

\fename{Medium} is expressed either as a prepositional phrase, or as the subject in the case of a non-overt \fename{Communicator}.

\fename{Addressee} is either realised as a prepositional phrase or is left unexpressed, although its presence is always required conceptually as every act of communication is addressed to someone. Predominantly, the non-overt \fename{Addressee} frame elements are indefinite null instantiations (INI).



\begin{exe}
\ex \label{ex:02comm}
\begin{xlist}
\ex  \label{ex:02comm:a}
{[\textit{Iranian officials}]}$_{\feinsub{Com}}$ \textit{\textbf{INDICATE}} [\textit{that Iran would honor its safeguards agreement with the IAEA}]$_{\feinsub{Msg}}$ [\_]$_{\feinsub{Addr-INI}}$.
\ex \label{ex:02comm:b}
{[\textit{They}]}$_{\feinsub{Com}}$ \textit{can easily \textbf{COMMUNICATE}} [\_]$_{\feinsub{Msg-INI}}$ [\textit{with \\one another}]$_{\feinsub{Addr}}$.
\ex \label{ex:02comm:c}
{[\textit{The letter}]}$_{\feinsub{Med}}$ \textit{\textbf{COMMUNICATED}} [\textit{nothing}]$_{\feinsub{Msg}}$ [\textit{of her \\pleasure and love}]$_{\feinsub{Top}}$.
\ex \label{ex:02comm:d}
{[\textit{I}]}$_{\feinsub{Com}}$ \textit{\textbf{COMMUNICATED}} [\textit{with the Minister}]$_{\feinsub{Addr}}$ [\textit{on \\that issue}]$_{\feinsub{Top}}$.
\end{xlist}
\end{exe}



\subsubsection{\framename{Communication} valence patterns}

\framename{Communication} valence patterns are presented in \tabref{tbl:communication-valence}.

The most common valence pattern found in the data is represented as an expressed subject NP \fename{Communicator}, an object NP \fename{Message} and an \fename{Addressee} PP. The \fename{Message} is usually expressed and when it is not -- the \fename{Topic} may be realised (Example \ref{ex:02comm}). Due to the small number of examples, this last observation is not included in the table, but it is supported by the expression of the relevant frame element in the more specific frames. 



\begin{table}
    \begin{tabular}{lrl}
\lsptoprule
         Pattern  & \#  & verbs \\\midrule
{}[NP.Ext]$_{\feinsub{Com}}$ [PP]$_{\feinsub{Addr}}$ [NP]$_{\feinsub{Msg}}$  & 11 & \textit{communicate, signal} \\
{}[NP.Ext]$_{\feinsub{Com}}$ [PP]$_{\feinsub{Addr}}$ {[\_]}$_{\feinsub{Msg-INI}}$  & 7 & \textit{communicate} \\
{}[NP.Ext]$_{\feinsub{Msg}}$ {[\_]}$_{\feinsub{Com-CNI}}$ [Clause]$_{\feinsub{Msg}}$  & 5 & \textit{say} \\
{}[NP.Ext]$_{\feinsub{Com}}$ {[\_]}$_{\feinsub{Addr-INI}}$ [NP]$_{\feinsub{Msg}}$  & 5 & \textit{communicate} \\
{}[NP.Ext]$_{\feinsub{Com}}$ {[\_]}$_{\feinsub{Addr-INI}}$ {[\_]}$_{\feinsub{Msg-INI}}$  & 4 & \textit{communicate} \\
{}[NP.Ext]$_{\feinsub{Com}}$ [Clause]$_{\feinsub{Msg}}$  & 4 & \textit{indicate, say, signal} \\
%[NP.Ext]$_{\feinsub{Med}}$  & 3 & \textit{say} \\
{}[NP.Ext]$_{\feinsub{Msg}}$ [PP]$_{\feinsub{Addr}}$ {[\_]}$_{\feinsub{Com-CNI}}$  & 3 & \textit{communicate} \\
{}[NP.Ext]$_{\feinsub{Com}}$ {[\_]}$_{\feinsub{Addr-DNI}}$ [NP]$_{\feinsub{Msg}}$  & 3 & \textit{communicate, indicate} \\
%[NP.Ext]$_{\feinsub{Com}}$ {[\_]}$_{\feinsub{Addr-DNI}}$  & 2 & \textit{indicate} \\
%[NP.Ext]$_{\feinsub{Com}}$  & 2 & \textit{say} \\
{}[NP.Ext]$_{\feinsub{Com}}$ [PP]$_{\feinsub{Addr}}$ [NP]$_{\feinsub{Msg}}$ [PP]$_{\feinsub{Top}}$  & 2 & \textit{communicate} \\
{}[NP.Ext]$_{\feinsub{Med}}$ {[\_]}$_{\feinsub{Addr-INI}}$ [Clause]$_{\feinsub{Msg}}$  & 2 & \textit{indicate} \\
\lspbottomrule
\end{tabular}
    \caption{FrameNet valence patterns of \framename{Communication} verbs, their frequency in the FrameNet corpus and the verbs they appear with.}
    \label{tbl:communication-valence}
\end{table} 



\subsubsection{Syntactic realisation of the \framename{Communication} frame in Bulgarian}

%In Bulgarian, due to the small number of examples in BulSemCor a small excerpt of parallel examples was collected for the translation pair \textit{communicate} -- \textit{съобщавам}. 

The core frame elements are expressed in a similar way as in English: the \fename{Communicator} is realised as a subject, the \fename{Message} is an NP object or more rarely (although varying from verb to verb) a complement clause or a quote; if overt, the \fename{Addressee} is expressed as a prepositional phrase. The \fename{Topic} is syntactically explicit in about 20\% of the cases and, similarly to English, is realised as either a prepositional phrase that modifies a \fename{Message} head noun (Example \ref{ex:02commbg:a}) or independently in the absence of an overt \fename{Message} (Example \ref{ex:02commbg:b}); the number of examples is too small to make definitive conclusions, but both languages support this observation. 


\begin{exe}
\ex \label{ex:02commbg}
\begin{xlist}
\ex  \label{ex:02commbg:a}
\gll {[\textit{Те}]}$_{\feinsub{Com}}$ \textit{\textbf{СЪОБЩАВАТ}} [\textit{съответната} \textit{информация}]$_{\feinsub{Msg}}$ [\textit{за} \textit{дейността} \textit{си}]$_{\feinsub{Top}}$.\\
They communicate relevant information about activity-\textsc{def} REFL.\\
\glt
`They communicate relevant information about their activity.'
\ex  \label{ex:02commbg:b}
\gll {[\textit{Те}]}$_{\feinsub{Com}}$ \textit{\textbf{СЪОБЩАВАТ}} {[\_]}$_{\feinsub{Msg-INI}}$ [\textit{за} \textit{пристигането} \textit{си} \textit{на} \textit{гарата}]$_{\feinsub{Top}}$.\\
They communicate {} about arrival-\textsc{def} REFL at station-\textsc{def}.\\
\glt `They communicate about their arrival at the station.'
\ex  \label{ex:02commbg:c}
\gll {[\textit{Те}]}$_{\feinsub{Com}}$ \textit{\textbf{СЪОБЩАВАТ}} [\textit{на} \textit{Комисията}]$_{\feinsub{Addr}}$ [\textit{текста} \textit{на} \textit{разпоредбите}]$_{\feinsub{Msg}}$.\\
They communicate to Commission-\textsc{def} text-\textsc{def} of measures-\textsc{def}.\\
\glt `They communicate to the Commission the text of the measures.'
\ex  \label{ex:02commbg:d}
\gll {[\textit{Органите}]}$_{\feinsub{Com}}$ \textit{\textbf{СЪОБЩАВАТ}} [\textit{цялата} \textit{съществена} \textit{информация}]$_{\feinsub{Msg}}$ {[\_]}$_{\feinsub{Addr}}$.\\
Authorities-\textsc{def} communicate all essential information. {}\\
\glt `The authorities communicate all essential information.'
\ex \label{ex:02commbg:e}
\gll {[\textit{Страните}]}$_{\feinsub{Com}}$ \textit{\textbf{ПОСОЧВАТ}},  [\textit{че} \textit{информацията} \textit{не може} \textit{да} \textit{бъде} \textit{резюмирана}]$_{\feinsub{Msg}}$ {[\_]}$_{\feinsub{Addr}}$.\\
Parties-\textsc{def} indicate that information-\textsc{def} cannot to be summarised. {}\\
\glt `The parties indicate that the information cannot be summarised.'
\end{xlist}
\end{exe}

The syntactic realisation of the \framename{Communication} frame elements in Bulgarian is shown in \tabref{tbl:communication-synt-bg}.\footnote{In the Bulgarian annotated data the verbs are assigned a WordNet sense, so the corresponding Princeton WordNet synset serves as an English translation equivalent. As this information is not available to the readers, henceforth we have provided translation equivalents for the Bulgarian verbs.}
 
\begin{table}
\centering\footnotesize
\begin{tabular}{l rrrrrrrrr}
\lsptoprule
 & NP.Ext & NP.Obj & PP & AVP & NI & Clause & Quote & Other & Total\\
\midrule
%\multicolumn{10}{l}{\textit{общувам} }\\  
%\fename{Communicator} & 31 &  &  &  &  &  &  &  & 31\\ 
%\fename{Message} &  &  &  &  & 32 &  &  &  & 32\\ 
%\fename{Addressee} &  &  & 28 &  & 4 &  &  &  & 32\\ 
% \midrule
\multicolumn{10}{l}{\textit{споделям\slash споделя}}\\
`share'\\
\fename{Communicator} & 14 &  &  &  &  &  &  &  & 14\\ 
\fename{Message} &  & 11 &  &  & 2 & 1 &  &  & 14\\ 
\fename{Addressee} &  &  & 12 &  & 2 &  &  &  & 14\\ 
\fename{Topic} &  &  & 1 &  &  &  &  &  & 1\\ 

\midrule
\multicolumn{10}{l}{\textit{съобщавам\slash съобщя} }\\ 
`communicate'\\
\fename{Communicator} & 29 &  &  &  &  &  &  &  & 29\\ 
\fename{Message} &  & 22 &  &  & 6 & 2 &  &  & 30\\ 
\fename{Addressee} &  &  & 22 &  & 8 &  &  &  & 30\\ 
\fename{Medium} &  &  & 1 &  &  &  &  &  & 1\\ 
\fename{Topic} &  & 1 & 6 &  &  &  &  &  & 7\\ 

\midrule
\multicolumn{10}{l}{\textit{предавам\slash предам} }\\
`convey'\\
\fename{Communicator} & 48 &  &  &  &  &  &  &  & 48\\ 
\fename{Message} & 3 & 42 &  &  & 1 & 1 & 1 &  & 48\\ 
\fename{Addressee} &  &  & 28 &  & 19 &  &  &  & 47\\ 

\lspbottomrule
 \end{tabular}
 \caption{Syntactic expression of the \framename{Communication} frame elements in Bulgarian.} 
    \label{tbl:communication-synt-bg}
 \end{table}


The valence patterns in Bulgarian (\tabref{tbl:communication-valence-bg}) show similar preferences for the co-occurrence of frame elements; with both \fename{Message} and \fename{Addressee} expressed (Example \ref{ex:02commbg:c}) or with a realised \fename{Message} and a non-overt \fename{Addressee} (Example \ref{ex:02commbg:d}).

\begin{table}
\begin{tabularx}{\textwidth}{lrQ}
\lsptoprule
         Pattern  & \#  & verbs \\\midrule
{[NP.Ext]}$_{\feinsub{Com}}$ {[NP.Obj]}$_{\feinsub{Msg}}$ {[PP]}$_{\feinsub{Addr}}$ & 50 & \textit{предавам\slash предам, споделям\slash споделя, съобщавам\slash съобщя}\\
%{[NP.Ext]}$_{\feinsub{Com}}$ {[PP]}$_{\feinsub{Addr}}$ {[\_]}$_{\feinsub{Msg-INI}}$ & 27 & \textit{общувам}
%, предавам\slash предам
%\\
{[NP.Ext]}$_{\feinsub{Com}}$ {[NP.Obj]}$_{\feinsub{Msg}}$ {[\_]}$_{\feinsub{Addr-INI}}$ & 13 & \textit{предавам\slash предам, съобщавам\slash съобщя}\\
{[NP.Ext]}$_{\feinsub{Com}}$ {[NP.Obj]}$_{\feinsub{Msg}}$  {[\_]}$_{\feinsub{Addr-DNI}}$ & 9 & \textit{споделям\slash споделя, предавам\slash предам}\\
{[NP.Ext]}$_{\feinsub{Com}}$ {[PP]}$_{\feinsub{Addr}}$ {[PP]}$_{\feinsub{Top}}$   {[\_]}$_{\feinsub{Msg-INI}}$ & 4 & \textit{споделям\slash споделя, съобщавам\slash съобщя}\\
{[NP.Ext]}$_{\feinsub{Com}}$ {[PP]}$_{\feinsub{Addr}}$ {[\_]}$_{\feinsub{Msg-DNI}}$ & 3 & \textit{споделям\slash споделя, съобщавам\slash съобщя}\\
%{[NP.Ext]}$_{\feinsub{Com}}$ {[\_]}$_{\feinsub{Addr-DNI}}$ {[\_]}$_{\feinsub{Msg-INI}}$ & 3 & \textit{общувам}\\
{[NP.Ext]}$_{\feinsub{Com}}$ {[Clause]}$_{\feinsub{Msg}}$  {[\_]}$_{\feinsub{Addr-INI}}$ & 2 & \textit{съобщавам\slash съобщя}\\
%{[NP.Ext]}$_{\feinsub{Com}}$ {[PP]}$_{\feinsub{Addr}}$ {[Quote]}$_{\feinsub{Msg}}$ & 1 & \textit{предавам\slash предам}\\
%{[NP.Ext]}$_{\feinsub{Com}}$ {[PP]}$_{\feinsub{Top}}$ {[\_]}$_{\feinsub{Addr-INI}}$ {[\_]}$_{\feinsub{Msg-INI}}$ & 1 & \textit{съобщавам\slash съобщя}\\
%{[NP.Ext]}$_{\feinsub{Com}}$ {[NP.Obj]}$_{\feinsub{Msg}}$ {[PP]}$_{\feinsub{Addr}}$ {[PP]}$_{\feinsub{MEDIUM}}$ & 1 & \textit{съобщавам\slash съобщя}\\
%{[NP.Ext]}$_{\feinsub{Com}}$ {[NP.Obj]}$_{\feinsub{Msg}}$ {[PP]}$_{\feinsub{Top}}$ {[\_]}$_{\feinsub{Addr-INI}}$ & 1 & \textit{съобщавам\slash съобщя}\\
{[NP.Ext]}$_{\feinsub{Com}}$ {[NP.Obj]}$_{\feinsub{Msg}}$ {[PP]}$_{\feinsub{Addr}}$ {[PP]}$_{\feinsub{Top}}$ & 1 & \textit{съобщавам\slash съобщя}\\
\lspbottomrule
\end{tabularx}
    \caption{FrameNet valence patterns of \framename{Communication} verbs, their frequency in the Bulgarian dataset and the verbs they appear with. 
    English translation equivalents: \textit{предавам\slash предам} `convey', \textit{споделям\slash споделя} `share', \textit{съобщавам\slash съобщя} `communicate'.}
    \label{tbl:communication-valence-bg}
\end{table} 

In the Bulgarian data we have found only rare instances where there is an expressed \fename{Addressee} with non-overt \fename{Message} or \fename{Topic}, but this observation needs further corroboration from the data for this frame as well as for other related frames.

The \fename{Message} can also be expressed as a quote or a clausal complement; however, as Bulgarian lacks infinitives and \textit{-ing} clauses, clausal complements are realised as finite clauses (Example \ref{ex:02commbg:e}). 




\subsection{Frame \framename{Communication\_manner}}

\begin{description}[font=\normalfont]\sloppy
\item[Definition of the frame \framename{Communication\_manner}:] The words in this frame describe \fename{Manners} of verbal communication. Core frame elements: \fename{Speaker}, \fename{Message}, \fename{Topic}, \fename{Addressee}.
\end{description}

The \fename{Speaker} is a specific type of \fename{Communicator} who uses his or her voice to produce the \fename{Message}. Thus, apart from being a sentient being, it needs to be able to produce speech, e.g. is typically a person (Example \ref{ex:03manner:a}). The type of communication involves characteristics of individual organisms, so organisations are not typically realised as \fename{Speakers}, but groups of people can be (Example \ref{ex:03manner:b}).  


In particular, the verbs in the \framename{Communication\_manner} frame describe various manners of speaking or vocalising whereby a \fename{Speaker} conveys a \fename{Message} to the \fename{Addressee}. The focus is on the specifics of the articulation or vocalisation such as clarity, speed, loudness, etc. Thus, the \fename{Manner} of the communication is incorporated in the lexical meaning of the verb, e.g. \textit{whisper}.v `speak very softly using one's breath', \textit{babble}.v `talk rapidly and continuously', etc.; the \fename{Manner} can appear overtly when expressing additional manner meaning than the one incorporated by the verb (Example \ref{ex:03manner:e}). 

The \fename{Medium} of communication is peripheral to the conceptualisation of the frame and thus has a non-core status. 

The remaining core frame elements, i.e. \fename{Message} and \fename{Topic}, have the same specifics as in the \framename{Communication} frame.

\subsubsection{Syntactic realisation of the \framename{Communication\_manner} frame elements}
\largerpage

The syntactic expression of the basic configuration of frame elements in the \framename{Communication\_manner} frame is similar to the one in the \framename{Communication} frame, but there are differences that we point out below. Like \fename{Communicator}, the \fename{Speaker} is the external argument and is realised as the subject, which, under some contextually or constructionally grounded circumstances can be left implicit. 

Similarly to the same frame element in the \framename{Communication} frame, the \fename{Message} can be expressed as a subordinate clause (Example \ref{ex:03manner:c}), a quoted expression (Example \ref{ex:03manner:d}), or an NP object  that generalises over the type of information (Example \ref{ex:03manner:a}). In some cases the \fename{Message} can be unexpressed (Example \ref{ex:03manner:e}).

The \fename{Topic} is typically expressed as a prepositional phrase complement headed by `about’ (Example \ref{ex:03manner:f}).
%; this trend on a smaller scale due to the smaller number of examples is visible in the \framename{Communication} frame, where other types of expression were found as well; 
%In quite a few of the examples, this frame element is annotated as a null instantiation, since (as shown above) it is dependent on the \fename{Message} and is usually left implicit in its presence. 
An alternative type of pattern is for it to be left implicit (a null instantiation), especially in the presence of a \fename{Message}. As shown above, the two frame elements co-occur overtly primarily as an NP and a PP, where the \fename{Topic} PP should be treated as a modifier of the \fename{Message} NP (Example \ref{ex:03manner:g}).

The \fename{Addressee} is typically left non-overt but is always implied; otherwise it is expressed as a prepositional phrase (Examples \ref{ex:03manner:a}, \ref{ex:03manner:d}, \ref{ex:03manner:e}).

Among the verbs in this frame, certain differences may also be found. For instance, \textit{rave}.v and \textit{rant}.v tend to express overtly the \fename{Topic} more often than the \fename{Message} as compared with the purely manner verbs, which give preference to the \fename{Message} itself.


\begin{exe}
\ex \label{ex:03manner}
\begin{xlist}
\ex  \label{ex:03manner:a}
\glt {[\textit{Ann}]}$_{\feinsub{Com}}$ \textit{\textbf{WHISPERED}} [\textit{the question}]$_{\feinsub{Msg}}$   {[\textit{to Harry}]}$_{\feinsub{Addr}}$. 
\ex  \label{ex:03manner:b}
{[\textit{The crowd}]}$_{\feinsub{Com}}$ \textit{\textbf{CHANTED}} [\textit{my name}]$_{\feinsub{Msg}}$ {[\_]}$_{\feinsub{Addr-INI}}$.
\ex  \label{ex:03manner:c}
{[\textit{He}]}$_{\feinsub{Com}}$ \textit{\textbf{MUMBLED}} [\textit{that he was in a state of shock}]$_{\feinsub{Msg}}$   {[\_]}$_{\feinsub{Addr-INI}}$.
\ex  \label{ex:03manner:d}
{[\textit{`Change of plan,'}]}$_{\feinsub{Msg}}$ [\textit{Peter}]$_{\feinsub{Com}}$ \textit{\textbf{SHOUTED OUT}} {[\textit{to Kelly}]}$_{\feinsub{Addr}}$.
\ex  \label{ex:03manner:e}
{[\textit{I}]}$_{\feinsub{Com}}$ \textit{was \textbf{SINGING}}  [\_]$_{\feinsub{Msg-INI}}$ [\textit{happily}]$_{\feinsub{Manr}}$  
{[\textit{to myself}]}$_{\feinsub{Addr}}$.
\ex  \label{ex:03manner:f}
{[\textit{He}]}$_{\feinsub{Com}}$ \textit{was \textbf{RAVING}}  [\_]$_{\feinsub{Msg-INI}}$  
{[\textit{about Armageddon}]}$_{\feinsub{Top}}$ {[\_]}$_{\feinsub{Addr-INI}}$.
\ex  \label{ex:03manner:g}
{[\textit{He}]}$_{\feinsub{Com}}$ \textit{\textbf{MUMBLED}} [\textit{something}]$_{\feinsub{Msg}}$  
{[\textit{about something or other}]}$_{\feinsub{Top}}$.
\end{xlist}
\end{exe}

The specifics of the syntactic expression of the basic configuration of frame elements in the \framename{Communication\_manner} frame is exemplified in \tabref{tbl:communication-manner-synt}.

\begin{table}
\centering\footnotesize
\begin{tabular}{l rrrrrrrrr}
\lsptoprule
 & NP.Ext & NP.Obj & PP & AVP & NI & Clause & Quote & Other & Total\\ 
\midrule
%\multicolumn{10}{l}{\textit{babble} } \\  
%\fename{Speaker} & 38  &  &  &  &  &  &  &  & 38\\ 
%\fename{Addressee} &  &  & 2  &  & 36  &  &  &  & 38\\ 
%\fename{Topic} &  &  & 10  &  & 18  &  &  &  & 28\\ 
%\fename{Message} &  & 5  &  &  &  &  & 6  &  & 11\\ 
% \midrule
%\multicolumn{10}{l}{\textit{chant} } \\  
%\fename{Speaker} & 38  &  & 3  &  & 7  &  &  &  & 48\\ 
%\fename{Addressee} &  &  & 5  &  & 43  &  &  &  & 48\\ 
%\fename{Message} & 10  & 18  & 1  &  &  & 1  & 9  &  & 39\\ 
%\fename{Topic} &  &  &  &  & 9  &  &  &  & 9\\ 
% \midrule
%\multicolumn{10}{l}{\textit{mumble} } \\  
%\fename{Speaker} & 62  &  &  &  &  &  &  &  & 62\\ 
%\fename{Addressee} &  &  & 11  &  & 51  &  &  &  & 62\\ 
%\fename{Topic} &  &  & 11  &  & 12  &  &  &  & 23\\ 
%\fename{Message} &  & 16  &  &  &  & 6  & 22  &  & 44\\ 
% \midrule
\multicolumn{10}{l}{\textit{mutter} } \\  
\fename{Speaker} & 89  &  &  &  &  &  &  &  & 89\\ 
\fename{Addressee} &  &  & 21  &  & 68  &  &  &  & 89\\ 
\fename{Topic} &  &  & 20  &  & 14  &  &  &  & 34\\ 
\fename{Message} &  & 32  &  &  &  & 7  & 26  &  & 65\\ 

\midrule
%\multicolumn{10}{l}{\textit{rant} } \\  
%\fename{Speaker} & 26  &  &  &  &  &  &  &  & 26\\ 
%\fename{Addressee} &  &  & 4  &  & 22  &  &  &  & 26\\ 
%\fename{Topic} &  &  & 2  &  & 14  &  &  &  & 16\\ 
%\fename{Message} &  & 1  &  &  &  & 1  & 8  &  & 10\\ 
% \midrule
\multicolumn{10}{l}{\textit{rave} } \\  
\fename{Speaker} & 27  &  &  &  &  &  &  &  & 27\\ 
\fename{Addressee} &  &  & 3  &  & 24  &  &  &  & 27\\ 
\fename{Topic} &  & 3  & 13  &  & 8  &  &  &  & 24\\ 
\fename{Message} &  & 1  & 1  &  &  &  & 3  &  & 5\\ 

\midrule
\multicolumn{10}{l}{\textit{shout} } \\  
\fename{Addressee} & 3  &  & 38  &  & 83  &  &  &  & 124\\ 
\fename{Speaker} & 116  &  &  &  & 5  &  &  &  & 121\\ 
\fename{Topic} &  &  & 3  &  & 34  &  &  &  & 37\\ 
\fename{Message} & 2  & 38  & 5  &  &  & 15  & 26  &  & 86\\ 

\midrule
\multicolumn{10}{l}{\textit{sing} } \\  
\fename{Speaker} & 59  &  & 6  &  & 2  &  &  &  & 67\\ 
\fename{Addressee} &  &  & 8  &  & 58  &  &  &  & 66\\ 
\fename{Message} & 8  & 33  &  &  & 24  &  & 2  &  & 67\\ 
\fename{Topic} &  &  & 14  &  &  &  &  & 1 & 15\\ 

\midrule
\multicolumn{10}{l}{\textit{whisper} } \\  
\fename{Speaker} & 47  &  &  &  & 5  &  &  & 1 & 53\\ 
\fename{Addressee} &  &  & 16  &  & 37  &  &  &  & 53\\ 
\fename{Message} & 6  & 14  &  &  &  & 8  & 9  &  & 37\\ 
\fename{Topic} &  &  & 7  &  & 12  &  &  &  & 19\\ 
\lspbottomrule
 \end{tabular}
 \caption{Syntactic expression of the \framename{Communication\_manner} frame elements in selected FrameNet lexical units. } 
    \label{tbl:communication-manner-synt}
 \end{table}

 \subsubsection{\framename{Communication\_manner} valence patterns}

\tabref{tbl:communication-manner-valence} shows the prevalent valence patterns found with the verbs evoking the \framename{Communication\_manner} frame in the FrameNet annotated corpus.

The most frequent patterns include a canonical expression of the \fename{Speaker} as a subject NP and a \fename{Message} realised as either a direct quote, an object NP or a clausal complement. The valence patterns also show most frequently a non-overt or less often an expressed \fename{Addressee}.

\begin{table}
\begin{tabularx}{\textwidth}{lrQ}
\lsptoprule
         Pattern  & \#  & verbs \\
\midrule
{[NP.Ext]}$_{\feinsub{Spkr}}$ {[\_]}$_{\feinsub{Addr-INI}}$ {[Quote]}$_{\feinsub{Msg}}$  & 166 & \textit{rant, chant, slur, stutter, stammer, babble, chatter, rave, mumble, mutter, whisper, sing, shout}\\
% \midrule
{[NP.Ext]}$_{\feinsub{Spkr}}$ {[\_]}$_{\feinsub{Addr-INI}}$ {[\_]}$_{\feinsub{Top-INI}}$  & 156 & \textit{rant, chant, slur, stutter, stammer, babble, chatter, rave, mumble, mutter, whisper, shout}\\
% \midrule
{[NP.Ext]}$_{\feinsub{Spkr}}$ {[\_]}$_{\feinsub{Addr-INI}}$ {[NP.Obj]}$_{\feinsub{Msg}}$  & 146 & \textit{rant, chant, slur, stutter, stammer, babble, chatter, mumble, mutter, whisper, sing, shout}\\
% \midrule
{[NP.Ext]}$_{\feinsub{Spkr}}$ {[\_]}$_{\feinsub{Addr-INI}}$ {[PP]}$_{\feinsub{Top}}$  & 70 & \textit{rant, babble, chatter, rave, mumble, mutter, whisper, shout}\\
% \midrule
{[NP.Ext]}$_{\feinsub{Spkr}}$ {[PP]}$_{\feinsub{Addr}}$ {[\_]}$_{\feinsub{Top-INI}}$  & 48 & \textit{rant, chant, babble, chatter, rave, mumble, mutter, whisper, shout}\\
% \midrule
{[NP.Ext]}$_{\feinsub{Spkr}}$ {[\_]}$_{\feinsub{Addr-INI}}$ {[Clause]}$_{\feinsub{Msg}}$  & 41 & \textit{rant, chant, mumble, mutter, stutter, stammer, whisper, shout}\\
% \midrule
{[NP.Ext]}$_{\feinsub{Spkr}}$ {[PP]}$_{\feinsub{Addr}}$ {[NP.Obj]}$_{\feinsub{Msg}}$  & 34 & \textit{mumble, mutter, stutter, whisper, sing, shout}\\
% \midrule
{[NP.Ext]}$_{\feinsub{Spkr}}$ {[PP]}$_{\feinsub{Addr}}$ {[Quote]}$_{\feinsub{Msg}}$  & 31 & \textit{rant, chant, mumble, mutter, whisper, shout}\\
% \midrule
{[NP.Ext]}$_{\feinsub{Spkr}}$ {[\_]}$_{\feinsub{Addr-INI}}$ {[NP.Obj]}$_{\feinsub{Msg}}$ {[PP]}$_{\feinsub{Top}}$  & 21 & \textit{mumble, mutter, stammer, babble, sing, shout}\\
% \midrule
%{[NP.Ext]}$_{\feinsub{Msg}}$ {[\_]}$_{\feinsub{Addr-INI}}$ {[\_]}$_{\feinsub{Spkr-CNI}}$  & 17 & \textit{chant, whisper, sing, shout}\\
% \midrule
%{[NP.Ext]}$_{\feinsub{Spkr}}$ {[\_]}$_{\feinsub{Addr-INI}}$ {[\_]}$_{\feinsub{Msg-INI}}$ {[PP]}$_{\feinsub{Top}}$  & 11 & \textit{sing}\\
% \midrule
%{[NP.Ext]}$_{\feinsub{Msg}}$ {[\_]}$_{\feinsub{Addr-INI}}$ {[PP]}$_{\feinsub{Spkr}}$  & 9 & \textit{chant, sing}\\
% \midrule
%{[NP.Ext]}$_{\feinsub{Spkr}}$ {[\_]}$_{\feinsub{Addr-INI}}$ {[PP]}$_{\feinsub{Msg}}$  & 8 & \textit{chant, rave, shout}\\
% \midrule
%{[NP.Ext]}$_{\feinsub{Spkr}}$ {[\_]}$_{\feinsub{Addr-INI}}$ {[\_]}$_{\feinsub{Msg-INI}}$  & 7 & \textit{sing}\\
% \midrule
%{[NP.Ext]}$_{\feinsub{Spkr}}$ {[PP]}$_{\feinsub{Addr}}$ {[Clause]}$_{\feinsub{Msg}}$  & 6 & \textit{mouth, whisper, shout}\\
% \midrule
%{[NP.Ext]}$_{\feinsub{Spkr}}$ {[PP]}$_{\feinsub{Addr}}$ {[PP]}$_{\feinsub{Top}}$  & 5 & \textit{natter, chatter, whisper}\\
% \midrule
%{[NP.Ext]}$_{\feinsub{Spkr}}$ {[PP]}$_{\feinsub{Addr}}$ {[\_]}$_{\feinsub{Msg-INI}}$  & 5 & \textit{sing}\\
% \midrule
%{[NP.Ext]}$_{\feinsub{Spkr}}$ {[\_]}$_{\feinsub{Addr-INI}}$ {[NP.Obj]}$_{\feinsub{Top}}$  & 3 & \textit{chatter, rave}\\
% \midrule
%{[NP.Ext]}$_{\feinsub{Addr}}$ {[PP]}$_{\feinsub{Addr}}$ {[\_]}$_{\feinsub{Spkr-CNI}}$ {[\_]}$_{\feinsub{Top-INI}}$  & 3 & \textit{shout}\\
% \midrule
%{[NP.Ext]}$_{\feinsub{Spkr}}$ {[PP]}$_{\feinsub{Addr}}$ {[NP.Obj]}$_{\feinsub{Msg}}$ {[PP]}$_{\feinsub{Top}}$  & 2 & \textit{mutter, whisper}\\
% \midrule
%{[NP.Ext]}$_{\feinsub{Spkr}}$ {[\_]}$_{\feinsub{Addr-INI}}$ {[NP.Obj]}$_{\feinsub{Msg}}$ {[Quote]}$_{\feinsub{Msg}}$  & 2 & \textit{stammer, rave}\\
% \midrule
%{[NP.Ext]}$_{\feinsub{Spkr}}$ {[NP.Obj]}$_{\feinsub{Msg}}$  & 2 & \textit{mouth, sing}\\
\lspbottomrule
\end{tabularx}
\caption{FrameNet valence patterns of \framename{Communication\_manner} verbs, their frequency in the FrameNet corpus and the verbs they appear with.}
\label{tbl:communication-manner-valence}
\end{table} 


\subsubsection{Syntactic realisation of the \framename{Communication\_manner} frame in Bulgarian}

In a similar manner, in Bulgarian the \fename{Speaker} is realised as the external subject NP, while the \fename{Message} is expressed as a direct quote (Example \ref{ex:03mannerbg:a}), a finite complement clause (Example \ref{ex:03mannerbg:b}) or an NP Object (Example \ref{ex:03mannerbg:c}). The \fename{Topic} and the \fename{Addressee} are expressed as prepositional complements (Example \ref{ex:03mannerbg:d}).

\begin{exe}
\ex \label{ex:03mannerbg}
\begin{xlist}
\ex  \label{ex:03mannerbg:a}
\gll {[\_]}$_{\feinsub{Spkr-DNI}}$ \textit{Едва} \textit{\textbf{ПРОШЕПВАМ}}: [– \textit{За} \textit{какво} \textit{става дума}?]$_{\feinsub{Msg}}$ [\_]$_{\feinsub{Addr-INI}}$ \\
{} Hardly whisper.1sg: – About what {take place word}? {}
\\
\glt `I hardly whisper: – What is it about?'
\ex  \label{ex:03mannerbg:b}
\gll {[\_]}$_{\feinsub{Spkr-DNI}}$ \textit{\textbf{ПРОМЪРМОРВАМ}}, [\textit{че} \textit{отчаяно} \textit{искам} \textit{да} \textit{си} \textit{го} \textit{върна}]$_{\feinsub{Msg}}$ [\_]$_{\feinsub{Addr-INI}}$.\\
{} Mutter.1sg that desperately want.1sg to REFL it {get back}. {}\\
\glt `I mutter that I desperately want to get it back.'
\ex  \label{ex:03mannerbg:c}
\gll {[\textit{Аз}]}$_{\feinsub{Spkr}}$ \textit{\textbf{ИЗМЪНКАХ}} [\textit{някакъв} \textit{отговор}]$_{\feinsub{Msg}}$ [\_]$_{\feinsub{Addr}}$. \\
I stammered some reply. {}
\\
%\glt `I stammered some stupid reply.'
\ex  \label{ex:03mannerbg:d}
\gll {[\textit{Тя}]}$_{\feinsub{Spkr}}$ \textit{\textbf{ДРЪНКА}} [\textit{на} \textit{всички}]$_{\feinsub{Addr}}$ [\textit{за} \textit{мен}]$_{\feinsub{Top}}$.\\
She babbles to everyone about me.
\\
%\glt `She babbles about me to all.'
\ex  \label{ex:03mannerbg:e}
\gll {[\textit{Капитанът}]}$_{\feinsub{Spkr}}$ \textit{продължи} \textit{да} \textit{\textbf{КРЕЩИ}} [\textit{заповедите} \textit{си}]$_{\feinsub{Msg}}$
[\textit{за} \textit{разни} \textit{платна} \textit{и} \textit{въжета}]$_{\feinsub{Top}}$% [\_]$_{\feinsub{Addr}}$
. \\
Captain-\textsc{def} continued to shout orders REFL about some sails and ropes.\\
\glt `The captain continued shouting his orders about sails and ropes.'
\ex  \label{ex:03mannerbg:f}
\gll {[– \textit{Здравейте}]}$_{\feinsub{Msg}}$ – \textit{\textbf{ИЗМЪНКВАМ}} [\textit{аз}]$_{\feinsub{Spkr}}$ [\_]$_{\feinsub{Addr}}$  \textit{нерешително}.\\
{– Hello} – mumble I {} hesitantly.
\\ %tuka njmaa prevod, dali taka sme iskali
%\glt `– Hello, – I mumbled hesitantly.'
\end{xlist}
\end{exe}



\begin{table}
\footnotesize
\begin{tabular}{l rrrrrrrrr}
\lsptoprule
 & NP.Ext & NP.Obj & PP & AVP & NI & Clause & Quote & Other & Total\\
\midrule
%\multicolumn{10}{l}{\textit{смотолевям\slash смотолевя} }\\  
%\fename{Message} &  &  &  &  &  &  & 4 &  & 4\\ 
%\fename{Addressee} &  &  &  &  & 4 &  &  &  & 4\\ 
%\fename{Speaker} & 4 &  &  &  &  &  &  &  & 4\\ 
% \midrule
%\multicolumn{10}{l}{\textit{заеквам\slash заекна} }\\  
%\fename{Message} &  &  &  &  &  &  & 4 &  & 4\\ 
%\fename{Addressee} &  &  &  &  & 5 &  &  &  & 5\\ 
%\fename{Speaker} & 5 &  &  &  &  &  &  &  & 5\\ 
% \midrule
%\multicolumn{10}{l}{\textit{шушукам} }\\  
%\fename{Message} &  & 1 &  &  &  &  &  &  & 1\\ 
%\fename{Addressee} &  &  & 2 &  & 2 &  &  &  & 4\\ 
%\fename{Speaker} & 4 &  &  &  &  &  &  &  & 4\\ 
% \midrule
\multicolumn{10}{l}{\textit{шепна} }\\
`whisper'\\
\fename{Message} &  & 3 &  &  & 1 &  & 3 &  & 7\\ 
\fename{Addressee} &  &  & 6 &  & 2 &  &  &  & 8\\ 
\fename{Speaker} & 8 &  &  &  &  &  &  &  & 8\\ 

\midrule
\multicolumn{10}{l}{\textit{промърморвам\slash промърморя} }\\
\multicolumn{10}{l}{`mumble, mutter'}\\
\fename{Message} &  & 2 &  &  &  &  & 8 & 3 & 13\\ 
\fename{Addressee} &  &  & 3 &  & 10 &  &  &  & 13\\ 
\fename{Medium} &  &  & 1 &  &  &  &  &  & 1\\ 
\fename{Speaker} & 13 &  &  &  &  &  &  &  & 13\\ 

\midrule
%\multicolumn{10}{l}{\textit{мърморя} }\\  
%\fename{Message} &  & 3 &  &  &  &  &  &  & 3\\ 
%\fename{Addressee} &  &  & 2 &  & 2 &  &  &  & 4\\ 
%\fename{Speaker} & 4 &  &  &  &  &  &  &  & 4\\ 
% \midrule
%\multicolumn{10}{l}{\textit{бъбря} }\\  
%\fename{Message} &  &  &  &  &  &  & 2 &  & 2\\ 
%\fename{Addressee} &  &  & 2 &  & 2 &  &  &  & 4\\ 
%\fename{Speaker} & 4 &  &  &  &  &  &  &  & 4\\ 
% \midrule
\multicolumn{10}{l}{\textit{викам} }\\
`shout'\\
\fename{Message} &  & 3 &  &  & 1 &  & 23 & 4 & 31\\ 
\fename{Addressee} &  &  & 2 &  & 33 &  &  &  & 35\\ 
\fename{Medium} &  &  & 1 &  &  &  &  &  & 1\\ 
\fename{Speaker} & 35 &  &  &  &  &  &  &  & 35\\ 

\midrule
%\multicolumn{10}{l}{\textit{крещя} }\\  
%\fename{Message} &  & 3 &  &  &  &  & 3 &  & 6\\ 
%\fename{Addressee} &  &  & 1 &  & 7 &  &  &  & 8\\ 
%\fename{Speaker} & 8 &  &  &  &  &  &  &  & 8\\ 
% \midrule
\multicolumn{10}{l}{\textit{прошепвам\slash прошепна} }\\
`whisper'\\
\fename{Message} &  & 8 &  &  &  &  & 7 & 1 & 16\\ 
\fename{Addressee} &  &  & 6 &  & 10 &  &  &  & 16\\ 
\fename{Speaker} & 17 &  &  &  &  &  &  &  & 17\\ 

%\multicolumn{10}{l}{\textit{пошушвам\slash пошушна} }\\  
%\fename{Message} &  &  &  &  &  &  & 3 &  & 3\\ 
%\fename{Addressee} &  &  & 1 &  & 3 &  &  &  & 4\\ 
%\fename{Speaker} & 4 &  &  &  &  &  &  &  & 4\\ 
% \midrule
%\multicolumn{10}{l}{\textit{дрънкам} }\\  
%\fename{Message} &  & 2 &  &  & 1 &  &  & 2 & 5\\ 
%\fename{Addressee} &  &  & 2 &  & 7 &  &  &  & 9\\ 
%\fename{Topic} &  &  & 4 &  &  &  &  &  & 4\\ 
%\fename{Speaker} & 9 &  &  &  &  &  &  &  & 9\\ 
% \midrule
\lspbottomrule
 \end{tabular}
 \caption{Syntactic expression of the \framename{Communication\_manner} frame elements in Bulgarian.} 
    \label{tbl:communication-manner-synt-bg}
 \end{table}
 
\tabref{tbl:communication-manner-synt-bg} shows a selection of verbs in Bulgarian evoking the frame \framename{Communication\_manner}, while
\tabref{tbl:communication-manner-valence-bg} presents the most frequent valence patterns. The syntactic realisation is similar to English: strong preference for the overt expression of the \fename{Message} either together with the \fename{Addressee} or in its absence; realising the \fename{Topic} most often either in the absence of (Example \ref{ex:03mannerbg:d}) or as a modifier to the \fename{Message} (Example \ref{ex:03mannerbg:e}).

We can also note that at least for some manner verbs such as \textit{мърморя} `mumble, mutter’, \textit{мънкам} `stutter’, there is a marked trend of expressing the \fename{Message} as a quote rather than as a complement clause. 



\begin{table}
\begin{tabularx}{\textwidth}{lrQ}
\lsptoprule
         Pattern  & \#  & verbs \\\midrule
{[NP.Ext]}$_{\feinsub{Spkr}}$ {[Quote]}$_{\feinsub{Msg}}$ {[\_]}$_{\feinsub{Addr-INI}}$ & 48 & \textit{бъбря, викам, заеквам\slash заекна, крещя, пошушвам\slash пошушна, промърморвам\slash промърморя, прошепвам\slash прошепна, смотолевям\slash смотолевя, шепна}\\
{[NP.Ext]}$_{\feinsub{Spkr}}$ {[NP.Obj]}$_{\feinsub{Msg}}$ {[\_]}$_{\feinsub{Addr-INI}}$ & 13 & \textit{бърборя, викам, дрънкам, крещя, мърморя, прошепвам\slash прошепна, шепна, промърморвам\slash промърморя}\\
{[NP.Ext]}$_{\feinsub{Spkr}}$ {[\_]}$_{\feinsub{Addr-INI}}$ & 12 & \textit{бъбря, викам, дрънкам, заеквам\slash заекна, крещя, мърморя, шушукам}\\
{[NP.Ext]}$_{\feinsub{Spkr}}$ {[NP.Obj]}$_{\feinsub{Msg}}$ {[PP]}$_{\feinsub{Addr}}$ & 11 & \textit{викам, дрънкам, крещя, мърморя, прошепвам\slash прошепна, шепна, шушукам}\\
{[NP.Ext]}$_{\feinsub{Spkr}}$ {[PP]}$_{\feinsub{Addr}}$ {[Quote]}$_{\feinsub{Msg}}$ & 10 & \textit{бъбря, викам, прошепвам\slash прошепна, шепна, шушна, промърморвам\slash промърморя}\\
{[NP.Ext]}$_{\feinsub{Spkr}}$ {[Clause-that]}$_{\feinsub{Msg}}$ {[\_]}$_{\feinsub{Addr-INI}}$ & 6 & \textit{викам, дрънкам, шушна, промърморвам\slash промърморя}\\
%{[NP.Ext]}$_{\feinsub{Spkr}}$ {[PP]}$_{\feinsub{Addr}}$ & 5 & \textit{бъбря, бърборя, пошушвам\slash пошушна, шепна, шушукам}\\
%{[NP.Ext]}$_{\feinsub{Spkr}}$ {[\_]}$_{\feinsub{Addr-INI}}$ {[\_]}$_{\feinsub{Msg-INI}}$ & 2 & \textit{викам, дрънкам}\\
%{[NP.Ext]}$_{\feinsub{Spkr}}$ {[NP.Obj]}$_{\feinsub{Msg}}$ {[PP]}$_{\feinsub{MEDIUM}}$ {[\_]}$_{\feinsub{Addr-INI}}$ & 2 & \textit{викам, промърморвам\slash промърморя}\\
{[NP.Ext]}$_{\feinsub{Spkr}}$ {[PP]}$_{\feinsub{Top}}$ {[\_]}$_{\feinsub{Addr-INI}}$ & 2 & \textit{дрънкам}\\
\lspbottomrule
\end{tabularx}
    \caption{FrameNet valence patterns of \framename{Communication\_manner} verbs, their frequency in the Bulgarian dataset and the verbs they appear with.    
    English translation equivalents: \textit{бъбря} `babble, prattle', \textit{викам} `shout', \textit{дрънкам} `rattle, jabber', \textit{заеквам\slash заекна} `stammer, stutter', \textit{крещя} `shout, yell', \textit{мърморя, промърморвам\slash промърморя} `mumble, mutter', \textit{пошушвам\slash пошушна}, \textit{прошепвам\slash прошепна}, \textit{шепна}, \textit{шушна}, \textit{шушукам} `whisper', \textit{смотолевям\slash смотолевя} `mumble, falter'.} 
    \label{tbl:communication-manner-valence-bg}
\end{table} 


\subsection{Frame \framename{Statement}}
\largerpage[2]
\begin{description}[font=\normalfont]
\item[Definition of the frame \framename{Statement}:] A \fename{Speaker} addresses a \fename{Message} to some \fename{Addressee} using language. Instead of (or in addition to) a \fename{Speaker}, a \fename{Medium} may also be mentioned. Likewise, a \fename{Topic} may be stated instead of a \fename{Message}. Core frame elements: \fename{Speaker}, \fename{Message}, \fename{Medium}, \fename{Topic}; Non-core: \fename{Addressee}.
\end{description}

This frame represents the greatest number of verbs of speech, including many general lexis verbs such as \textit{say}.v, \textit{state}.v, \textit{declare}.v, \textit{speak}.v, \textit{report}.v, \textit{note}.v, etc. 

\subsubsection{Syntactic realisation of the \framename{Statement} frame elements}

The frame \framename{Statement} is an elaboration of the prototypical frame \framename{Communication} which specifies verbs for communication involving language. This is reflected by the fact that the \fename{Communicator} is conceptualised as the more specific \fename{Speaker}, which denotes the person who produces the message. Likewise, this The frame element \fename{Speaker} is realised as the external NP.

The \fename{Message} is  typically expressed either as a subordinate clause, an NP object, or a direct quote that represents the content being conveyed (Example \ref{ex:04statement:a}, \ref{ex:04statement:b}, \ref{ex:04statement:c}, respectively). There is a range of preferred realisations of the \fename{Message} with the different verbs in this frame: some of them have a stronger tendency to take a complement subordinate clause (e.g., \textit{claim}.v, \textit{suggest}.v, \textit{note}.v), while others show preference for an NP object (e.g., \textit{profess}.v, \textit{reiterate}.v, \textit{relate}.v) or a quote (e.g., \textit{exclaim}.v); in some cases the three realisations are equally likely (e.g., \textit{caution}.v). 

The \fename{Topic} is typically expressed as a prepositional phrase headed by different prepositions depending on the verb, e.g. (\textit{speak about him, speak of him, preach of heaven, comment on the protests, comment upon the economic conditions}), a trend inherited from the \framename{Communication} frame. Similarly to the frames discussed above, usually either the \fename{Message} or the \fename{Topic} is expressed; as expected, they may also occur together in a phrase (Example \ref{ex:04statement:b}), where the \fename{Topic} is syntactically dependent on the \fename{Message}. In addition, some verbs co-occur more readily with a \fename{Topic} rather than with a \fename{Message}, e.g. \textit{explain}.v (Example \ref{ex:04statement:d}).

As a peripheral frame element the \fename{Addressee} is often left non-overt although implied. When present, it is expressed as a prepositional phrase most frequently with the preposition `to' (Example \ref{ex:04statement:d}). In some cases it may be realised as an indirect object (Example \ref{ex:04statement:e}).


\begin{exe}
\ex \label{ex:04statement}
\begin{xlist}
\ex  \label{ex:04statement:a}
{[\textit{North} \textit{Korea}]}$_{\feinsub{Spkr}}$ \textit{\textbf{CLAIMED}} [\textit{it} \textit{had} \textit{no} \textit{intention} \textit{of} \textit{producing} \textit{nuclear} \textit{weapons}]$_{\feinsub{Msg}}$.
\ex  \label{ex:04statement:b}
{[\textit{He}]}$_{\feinsub{Spkr}}$ \textit{\textbf{SAID}} [\textit{little}]$_{\feinsub{Msg}}$ [\textit{about} \textit{the} \textit{case}]$_{\feinsub{Top}}$.
\ex  \label{ex:04statement:c}
{[\textit{He}]}$_{\feinsub{Spkr}}$ \textit{\textbf{ADDED}}: [`\textit{Eldorado is a brave venture}']$_{\feinsub{Msg}}$.
\ex  \label{ex:04statement:d}
{[\textit{Doc}]}$_{\feinsub{Spkr}}$ \textit{\textbf{EXPLAINED}} [\textit{the} \textit{injuries}]$_{\feinsub{Msg}}$ [\textit{to} \textit{the} \textit{police}]$_{\feinsub{Addr}}$.
\ex  \label{ex:04statement:e}
{[\textit{The} \textit{agency}]}$_{\feinsub{Spkr}}$ \textit{\textbf{WROTE}} [\textit{me}]$_{\feinsub{Addr}}$ [\textit{that} \textit{you} \textit{had}  \textit{moved}]$_{\feinsub{Msg}}$.
\ex  \label{ex:04statement:f}
{[\textit{The} \textit{letter}]}$_{\feinsub{Med}}$ \textit{\textbf{ALLEGED}} [\textit{serious} \textit{breaches} \textit{of} \textit{the} \textit{law}]$_{\feinsub{Msg}}$.
\end{xlist}
\end{exe}


The various specific configuration of frame elements as expressed by verbs in the \framename{Statement} frame are shown in \tabref{tbl:statement-synt}.

\largerpage
\begin{table}
\footnotesize
\begin{tabular}{l rrrrrrrrr}
\lsptoprule
 & NP.Ext & NP.Obj & PP & AVP & NI & Clause & Quote & Other & Total\\ 

\midrule
% \multicolumn{10}{l}{\textit{add} } \\  
%\fename{Speaker} & 24  &  &  &  &  &  &  & 1 & 25\\ 
%\fename{Addressee} &  &  & 1  &  &  &  &  &  & 1\\ 
%\fename{Message} & 1  & 6  &  &  & 1  & 7  & 10  &  & 25\\ 
%\fename{Medium} & 1  &  &  &  &  &  &  &  & 1\\ 
%\fename{Topic} &  &  & 3  &  &  &  &  &  & 3\\ 
% \midrule
\multicolumn{10}{l}{\textit{announce} } \\  
\fename{Speaker} & 44  &  & 3  &  & 5  &  &  & 1 & 53\\ 
\fename{Addressee} &  &  & 6  &  &  &  &  & 1 & 7\\ 
\fename{Message} & 8  & 20  &  &  &  & 24  & 6  &  & 58\\ 
\fename{Medium} & 3  &  & 2  &  &  &  &  &  & 5\\ 

\midrule
%\multicolumn{10}{l}{\textit{assert} } \\  
%\fename{Speaker} & 24  &  &  &  & 3  &  &  &  & 27\\ 
%\fename{Message} & 3  & 3  & 1  &  &  & 18  & 5  &  & 30\\ 
%\fename{Topic} &  &  & 1  &  &  &  &  &  & 1\\ 
%\fename{Medium} &  &  & 1  &  &  &  &  &  & 1\\ 
% \midrule
%\multicolumn{10}{l}{\textit{caution} } \\  
%\fename{Speaker} & 38  &  & 6  &  & 1  &  &  &  & 45\\ 
%\fename{Addressee} & 7  & 18  &  &  &  &  &  &  & 25\\ 
%\fename{Message} &  & 1  & 11  &  & 6  & 11  & 10  &  & 39\\ 
%\fename{Topic} &  &  & 7  &  &  &  &  &  & 7\\ 
% \midrule
%\multicolumn{10}{l}{\textit{comment} } \\  
%\fename{Speaker} & 26  &  & 2  &  & 3  &  &  &  & 31\\ 
%\fename{Addressee} &  &  & 4  &  &  &  &  &  & 4\\ 
%\fename{Message} &  &  & 1  & 1  &  & 4  & 6  &  & 12\\ 
%\fename{Topic} & 5  &  & 16  &  & 16  &  &  &  & 37\\ 
%\fename{Medium} & 1  &  & 3  &  &  &  &  &  & 4\\ 
% \midrule
\multicolumn{10}{l}{\textit{declare} } \\  
\fename{Speaker} & 58  &  &  &  & 7  &  &  &  & 65\\ 
\fename{Addressee} &  &  & 7  &  &  &  &  &  & 7\\ 
\fename{Message} & 7  & 32  & 6  &  &  & 17  & 15  & 7 & 84\\ 
%\fename{Medium} & 2  &  & 1  &  &  &  &  &  & 3\\ 
%\fename{Topic} &  &  & 1  &  &  &  &  & 1 & 2\\ 

\midrule
%\multicolumn{10}{l}{\textit{explain} } \\  
%\fename{Speaker} & 34  &  & 2  &  & 1  &  &  &  & 37\\ 
%\fename{Addressee} &  &  & 15  &  &  &  &  &  & 15\\ 
%\fename{Medium} & 1  &  & 4  &  &  &  &  &  & 5\\ 
%\fename{Topic} & 3  & 14  & 6  &  & 2  & 4  &  &  & 29\\ 
%\fename{Message} &  & 2  &  &  &  & 6  & 1  &  & 9\\ 
% \midrule
%\multicolumn{10}{l}{\textit{insist} } \\  
%\fename{Speaker} & 19  &  & 2  &  &  &  &  &  & 21\\ 
%\fename{Addressee} &  &  & 2  &  &  &  &  &  & 2\\ 
%\fename{Message} & 2  &  & 16  &  &  & 3  & 2  &  & 23\\ 
% \midrule
%\multicolumn{10}{l}{\textit{mention} } \\  
%\fename{Speaker} & 19  &  &  &  & 1  &  &  &  & 20\\ 
%\fename{Addressee} &  &  & 5  &  &  &  &  &  & 5\\ 
%\fename{Message} & 2  & 3  &  & 1  &  & 12  & 3  &  & 21\\ 
%\fename{Medium} & 1  &  & 4  &  &  &  &  &  & 5\\ 
%\fename{Topic} & 1  & 1  &  &  &  &  &  &  & 2\\ 
%
% \midrule
%\multicolumn{10}{l}{\textit{propose} } \\  
%\fename{Speaker} & 23  &  & 2  &  & 3  &  &  &  & 28\\ 
%\fename{Addressee} &  &  & 3  &  &  &  &  &  & 3\\ 
%\fename{Message} & 5  & 7  & 1  &  &  & 18  & 1  &  & 32\\ 
%\fename{Medium} & 3  &  & 1  &  &  &  &  &  & 4\\ 
%
% \midrule
%\multicolumn{10}{l}{\textit{recount} } \\  
%\fename{Speaker} & 31  &  & 1  &  & 5  &  &  &  & 37\\ 
%\fename{Addressee} &  &  & 8  &  &  &  &  &  & 8\\ 
%\fename{Message} & 6  & 23  &  &  &  & 7  & 3  &  & 39\\ 
%\fename{Medium} & 2  &  & 4  &  &  &  &  &  & 6\\ 
%\fename{Topic} &  &  & 1  &  &  &  &  &  & 1\\ 
%
% \midrule
%\multicolumn{10}{l}{\textit{reiterate} } \\  
%\fename{Speaker} & 24  &  & 2  &  & 1  &  &  &  & 27\\ 
%\fename{Addressee} &  &  & 1  &  &  &  &  &  & 1\\ 
%\fename{Message} & 2  & 17  &  &  &  & 3  & 4  &  & 26\\ 
%\fename{Medium} &  &  & 1  &  &  &  &  &  & 1\\ 
%\fename{Topic} & 1  &  &  &  &  &  &  &  & 1\\ 
%
% \midrule
%\multicolumn{10}{l}{\textit{relate} } \\  
%\fename{Speaker} & 24  &  & 1  &  &  &  &  &  & 25\\ 
%\fename{Addressee} &  &  & 1  &  &  &  &  &  & 1\\ 
%\fename{Message} & 1  & 14  &  & 1  &  & 7  & 2  &  & 25\\ 
%\fename{Topic} &  &  & 1  &  &  &  &  &  & 1\\ 
%
% \midrule
%\multicolumn{10}{l}{\textit{remark} } \\  
%\fename{Speaker} & 27  &  &  &  &  &  &  &  & 27\\ 
%\fename{Addressee} &  &  & 3  &  &  &  &  &  & 3\\ 
%\fename{Message} &  &  &  & 3  &  & 6  & 10  &  & 19\\ 
%\fename{Topic} &  &  & 12  &  &  &  &  &  & 12\\ 
%\fename{Medium} & 1  &  & 1  &  &  &  &  &  & 2\\ 
% \midrule
\multicolumn{10}{l}{\textit{report} } \\  
\fename{Speaker} & 54  &  & 1  &  & 19  &  &  &  & 74\\ 
\fename{Addressee} &  &  & 8  &  &  &  &  &  & 8\\ 
\fename{Message} & 19  & 20  & 2  & 1  & 1  & 44  & 2  &  & 89\\ 
\fename{Medium} & 9  &  & 5  &  & 1  &  &  & 1 & 16\\ 
\fename{Topic} & 2  &  & 5  &  & 1  &  &  &  & 8\\ 

\midrule
\multicolumn{10}{l}{\textit{say} } \\  
\fename{Message} & 14  & 22  & 1  & 4  & 2  & 49  & 33  &  & 125\\ 
\fename{Addressee} &  &  & 8  &  &  &  &  &  & 8\\ 
\fename{Speaker} & 90  & 1  &  &  & 14  &  &  &  & 105\\ 
\fename{Medium} & 9  &  & 10  &  & 1  &  &  &  & 20\\ 
\fename{Topic} &  &  & 10  &  & 1  &  &  & 1 & 12\\ 

\midrule
\multicolumn{10}{l}{\textit{state} } \\  
\fename{Speaker} & 38  &  &  &  &  &  &  &  & 38\\ 
\fename{Addressee} &  &  & 3  &  &  &  &  &  & 3\\ 
\fename{Message} & 3  & 8  & 2  &  &  & 19  & 13  &  & 45\\ 
\fename{Medium} & 3  &  & 1  &  & 3  &  &  &  & 7\\ 

\midrule
\multicolumn{10}{l}{\textit{suggest} } \\  
\fename{Speaker} & 27  &  & 2  &  & 4  &  &  &  & 33\\ 
\fename{Addressee} &  &  & 5  &  &  &  &  &  & 5\\ 
\fename{Message} & 3  & 5  &  & 3  &  & 21  & 5  &  & 37\\ 
\fename{Medium} & 4  &  & 4  &  &  &  &  &  & 8\\ 

\midrule
\multicolumn{10}{l}{\textit{talk} } \\  
\fename{Speaker} & 32  &  & 1  &  & 3  &  &  &  & 36\\ 
\fename{Topic} & 3  &  & 29  & 2  &  & 2  &  &  & 36\\ 
%\fename{Medium} & 1  &  &  &  &  &  &  &  & 1\\ 
\fename{Message} & 1  & 3  &  &  &  &  &  &  & 4\\ 

\midrule
\multicolumn{10}{l}{\textit{write} } \\  
\fename{Speaker} & 42  &  &  &  & 1  &  &  &  & 43\\ 
\fename{Addressee} &  & 2  & 4  &  &  & 1  &  &  & 7\\ 
\fename{Message} & 1  & 5  &  &  & 2  & 10  & 13  &  & 31\\ 
\fename{Topic} & 1  &  & 22  &  &  &  &  &  & 23\\ 
\fename{Medium} & 1  &  & 8  &  &  &  &  &  & 9\\ 

\lspbottomrule
%\multicolumn{10}{l}{\textit{confirm} } \\  
%\fename{Speaker} & 22  &  &  &  & 3  &  &  &  & 25\\ 
%\fename{Addressee} &  &  & 3  &  &  &  &  &  & 3\\ 
%\fename{Message} & 3  & 5  &  &  & 1  & 17  &  &  & 26\\ 
%\fename{Medium} & 1  &  & 3  &  &  &  &  &  & 4\\ 
% \midrule
%\multicolumn{10}{l}{\textit{note} } \\  
%\fename{Speaker} & 28  &  &  &  & 1  &  &  &  & 29\\ 
%\fename{Message} &  & 5  &  &  & 1  & 26  &  &  & 32\\ 
%\fename{Medium} & 5  &  & 4  &  &  &  &  &  & 9\\ 
%\fename{Topic} & 1  & 1  &  &  &  &  &  &  & 2\\ 
% \midrule
 \end{tabular}
 \caption{Syntactic expression of the \framename{Statement} frame elements in selected FrameNet lexical units.}
    \label{tbl:statement-synt}
 \end{table}


\subsubsection{\framename{Statement} valence patterns}

The prevalent valence patterns for verbs in the FrameNet frame \framename{Statement} are shown in \tabref{tbl:statement-valence}. The most typical ones include the canonical expression of the \fename{Speaker} as the external NP and the \fename{Message} as a subordinate clause, an object NP, or a quote.

Alternatively, the \fename{Medium} may occupy the position of the external argument with an implied generalised reading of the \fename{Speaker} which is left unexpressed (Example \ref{ex:04statement:f}). 
Similarly to many of the frames describing verbs of communication, instead of the \fename{Message} the \fename{Topic} may be realised, most often as a prepositional phrase.

The patterns involving the expression of an \fename{Addressee} are quite infrequent.
%The fact that talk and speak do allow the expression of the role Topic shows that the role Message is conceptually present, because Topic is a property of Messages.


\begin{table}[t]
    \centering\footnotesize
    \begin{tabularx}{\textwidth}{ lrQ }
\lsptoprule
         Pattern  & \#  & verbs \\
\midrule
{[NP.Ext]}$_{\feinsub{Spkr}}$ {[Clause]}$_{\feinsub{Msg}}$  & 281 & \textit{explain, note, declare, maintain, remark, mention, conjecture, reiterate, assert, preach, claim, attest, state, caution, write, add, allege, exclaim, say, suggest, insist, propose, announce, confirm, acknowledge, proclaim, reaffirm, report, pronounce}\\
{[NP.Ext]}$_{\feinsub{Spkr}}$ {[NP.Obj]}$_{\feinsub{Msg}}$  & 191 & \textit{explain, note, declare, tell, conjecture, reiterate, assert, preach, claim, speak, talk, state, caution, write, add, allege, exclaim, say, suggest, propose, announce, confirm, acknowledge, refute, proclaim, reaffirm, report}\\
{[NP.Ext]}$_{\feinsub{Spkr}}$ {[Quote]}$_{\feinsub{Msg}}$  & 143 & \textit{explain, gloat, declare, remark, observe, mention, reiterate, hazard, assert, preach, speak, attest, state, caution, write, add, allege, exclaim, say, pout, suggest, insist, propose, announce, proclaim, reaffirm, report}\\
{[NP.Ext]}$_{\feinsub{Spkr}}$ {[PP]}$_{\feinsub{Top}}$  & 83 & \textit{explain, gloat, preach, report, comment, remark, speak, talk, write}\\
{[NP.Ext]}$_{\feinsub{Medium}}$ {[Clause]}$_{\feinsub{Msg}}$  & 39 & \textit{note, declare, allege, say, suggest, propose, announce, confirm, acknowledge, proclaim, report, claim, state}\\
{[NP.Ext]}$_{\feinsub{Spkr}}$ {[PP]}$_{\feinsub{Addr}}$ {[NP.Obj]}$_{\feinsub{Msg}}$  & 28 & \textit{reiterate, declare, report, say, speak, state, suggest, propose, announce, mention}\\
{[NP.Ext]}$_{\feinsub{Spkr}}$ {[PP]}$_{\feinsub{Msg}}$  & 28 & \textit{profess, declare, preach, say, speak, describe, insist, caution}\\
%{[NP.Ext]}$_{\feinsub{Msg}}$ {[\_]}$_{\feinsub{Spkr-CNI}}$  & 28 & \textit{declare, allege, say, suggest, propose, announce, confirm, reiterate, assert, proclaim, preach, report, speak}\\
{[NP.Ext]}$_{\feinsub{Spkr}}$ {[PP]}$_{\feinsub{Addr}}$ {[Clause]}$_{\feinsub{Msg}}$  & 25 & \textit{add, explain, declare, allege, suggest, insist, propose, announce, mention, confirm, preach}\\
%{[NP.Ext]}$_{\feinsub{Msg}}$ {[Clause]}$_{\feinsub{Msg}}$ {[\_]}$_{\feinsub{Spkr-CNI}}$  & 21 & \textit{assert, allege, report, say}\\
{[NP.Ext]}$_{\feinsub{Spkr}}$ {[PP]}$_{\feinsub{Medium}}$ {[Clause]}$_{\feinsub{Msg}}$  & 20 & \textit{explain, note, acknowledge, allege, claim, say, state, suggest, write, mention}\\
{[NP.Ext]}$_{\feinsub{Medium}}$ {[NP.Obj]}$_{\feinsub{Msg}}$  & 20 & \textit{explain, note, proclaim, tell, allege, reaffirm, say, state, propose, announce, mention}\\
%{[NP.Ext]}$_{\feinsub{Msg}}$ {[PP]}$_{\feinsub{Spkr}}$  & 18 & \textit{profess, suggest, propose, announce, reiterate, preach, reaffirm, report, speak, talk, attest}\\\midrule
%{[NP.Ext]}$_{\feinsub{Spkr}}$ {[AVP]}$_{\feinsub{Msg}}$  & 16 & \textit{proclaim, report, remark, say, attest, suggest, mention}\\\midrule
%{[NP.Ext]}$_{\feinsub{Spkr}}$ {[NP.Obj]}$_{\feinsub{Top}}$  & 16 & \textit{explain, acknowledge, mention}\\\midrule
%{[NP.Ext]}$_{\feinsub{Spkr}}$ {[NP.Obj]}$_{\feinsub{Msg}}$ {[PP]}$_{\feinsub{Top}}$  & 13 & \textit{add, declare, allege, say}\\\midrule
%{[NP.Ext]}$_{\feinsub{Spkr}}$ {[PP]}$_{\feinsub{Addr}}$ {[Quote]}$_{\feinsub{Msg}}$  & 12 & \textit{declare, proclaim, exclaim, remark, say, insist, announce, mention}\\\midrule
%{[NP.Ext]}$_{\feinsub{Spkr}}$ {[PP]}$_{\feinsub{Addr}}$ {[PP]}$_{\feinsub{Top}}$  & 11 & \textit{explain, gloat, preach, remark, speak, write}\\\midrule
%{[NP.Ext]}$_{\feinsub{Spkr}}$  & 10 & \textit{add, maintain, allege, report, claim, speak, state}\\\midrule
%{[NP.Ext]}$_{\feinsub{Msg}}$ {[PP]}$_{\feinsub{MEDIUM}}$ {[\_]}$_{\feinsub{Spkr-CNI}}$  & 10 & \textit{confirm, assert, report, say, describe}\\\midrule
%{[NP.Ext]}$_{\feinsub{MEDIUM}}$ {[PP]}$_{\feinsub{Top}}$  & 7 & \textit{report, remark, talk, write}\\\midrule
%{[NP.Ext]}$_{\feinsub{Spkr}}$ {[PP]}$_{\feinsub{MEDIUM}}$ {[NP.Obj]}$_{\feinsub{Msg}}$  & 7 & \textit{confirm, reiterate, reaffirm, say, suggest, mention}\\\midrule
%{[NP.Ext]}$_{\feinsub{Spkr}}$ {[\_]}$_{\feinsub{Msg-DNI}}$  & 7 & \textit{note, say, speak, write}\\\midrule
%{[NP.Ext]}$_{\feinsub{Spkr}}$ {[NP.Obj]}$_{\feinsub{Msg}}$ {[Clause]}$_{\feinsub{Msg}}$  & 7 & \textit{declare, say, announce}\\\midrule
%{[NP.Ext]}$_{\feinsub{Spkr}}$ {[NP.Obj]}$_{\feinsub{Addr}}$ {[PP]}$_{\feinsub{Top}}$  & 6 & \textit{caution}\\\midrule
%{[NP.Ext]}$_{\feinsub{Spkr}}$ {[Quote]}$_{\feinsub{Msg}}$ {[Clause]}$_{\feinsub{Msg}}$  & 6 & \textit{assert, state, write}\\\midrule
%{[NP.Ext]}$_{\feinsub{Spkr}}$ {[NP.Obj]}$_{\feinsub{Addr}}$ {[Clause]}$_{\feinsub{Msg}}$  & 6 & \textit{address, tell, caution, write}\\\midrule
%{[NP.Ext]}$_{\feinsub{Spkr}}$ {[NP.Obj]}$_{\feinsub{Addr}}$ {[NP.Dep]}$_{\feinsub{Msg}}$  & 5 & \textit{address, tell}\\\midrule
%{[NP.Ext]}$_{\feinsub{Top}}$ {[\_]}$_{\feinsub{Spkr-CNI}}$ {[PP]}$_{\feinsub{Top}}$  & 5 & \textit{comment, talk}\\\midrule
%{[NP.Ext]}$_{\feinsub{Msg}}$ {[\_]}$_{\feinsub{Spkr-CNI}}$ {[PP]}$_{\feinsub{Top}}$  & 5 & \textit{say, speak}\\\midrule
%{[NP.Ext]}$_{\feinsub{Spkr}}$ {[Clause]}$_{\feinsub{Top}}$  & 5 & \textit{explain, talk}\\\midrule
%{[NP.Ext]}$_{\feinsub{Spkr}}$ {[PP]}$_{\feinsub{MEDIUM}}$ {[PP]}$_{\feinsub{Top}}$  & 5 & \textit{comment, write}\\\midrule
%{[NP.Ext]}$_{\feinsub{Spkr}}$ {[NP.Obj]}$_{\feinsub{Addr}}$ {[Quote]}$_{\feinsub{Msg}}$  & 5 & \textit{address, caution}\\\midrule
%{[NP.Ext]}$_{\feinsub{Spkr}}$ {[\_]}$_{\feinsub{Top-DNI}}$  & 5 & \textit{gloat, comment}\\\midrule
%{[NP.Ext]}$_{\feinsub{Spkr}}$ {[NP.Obj]}$_{\feinsub{Addr}}$ {[PP]}$_{\feinsub{Msg}}$  & 5 & \textit{address, caution}\\\midrule
\lspbottomrule
    \end{tabularx}
    \caption{FrameNet valence patterns of \framename{Statement} verbs, their frequency in the FrameNet corpus and the verbs they appear with.}
    \label{tbl:statement-valence}
\end{table} 


\subsubsection{Syntactic realisation of the \framename{Statement} frame in Bulgarian}
\largerpage
The syntactic realisation of the frame element configurations in Bulgarian closely resembles that in English. The \fename{Speaker} is usually realised as the external NP and can be a person, a group or an organisation (Example \ref{ex:04statementbg:a}, \ref{ex:04statementbg:b}). In some cases the \fename{Medium} can take the position of the external argument (Example \ref{ex:04statementbg:c}). 

The \fename{Message} is either a finite clause %introduced with conjunctions \textit{че} `that', \textit{да} `to' or an interrogative word 
(Example \ref{ex:04statementbg:a}), an object NP (Example \ref{ex:04statementbg:b}) or a direct quote (Example \ref{ex:04statementbg:f}). The \fename{Topic} rarely occurs together with the \fename{Message}, and it is usually a modifier of the \fename{Message} (Example \ref{ex:04statementbg:d}). The non-core \fename{Addressee} is mostly optional and is realised as a prepositional complement (Example \ref{ex:04statementbg:b}).



\begin{exe}
\ex \label{ex:04statementbg}
\begin{xlist}
\ex  \label{ex:04statementbg:a}
\gll {[\textit{Панайотов}]}$_{\feinsub{Spkr}}$ \textit{\textbf{ДОБАВИ}}, [\textit{че} \textit{лидер} \textit{на} \textit{бъдещата} \textit{партия} \textit{ще} \textit{е} \textit{Симеон}]$_{\feinsub{Msg}}$.
\\
Panayotov added that leader of future-\textsc{def} party will be Simeon.
\\
\glt `Panayotov added that Simeon will be the leader of the future party.'
\ex  \label{ex:04statementbg:b}
 \gll {[}\textit{Кредитните} \textit{институции}{]}$_{\feinsub{Spkr}}$ \textit{\textbf{ДЕКЛАРИРАХА}} [\textit{пред} \textit{властите}]$_{\feinsub{Addr}}$ [\textit{нарасналите} \textit{печалби}]$_{\feinsub{Msg}}$. 
\\
Credit institutions declared to authorities-\textsc{def} increased-\textsc{def} profits.
\\
\glt `Credit institutions declared increased profits to the authorities.'
\ex  \label{ex:04statementbg:c}
\gll {[}\textit{Неофициалните} \textit{статистики} \textit{за} {\textit{1999 г.}}{]}$_{\feinsub{Med}}$ \textit{\textbf{СОЧАТ}} [\textit{5000} \textit{посетители}]$_{\feinsub{Msg}}$.
\\
Unofficial-\textsc{def} statistics for 1999 report 5000 visitors.
\\
\glt `The unofficial statistics for 1999 state 5,000 visitors.'
\ex  \label{ex:04statementbg:d}
\gll {[\textit{Тези} \textit{лица}]}$_{\feinsub{Spkr}}$ \textit{\textbf{ИЗКАЗВАТ}} [\textit{пред} \textit{нас}]$_{\feinsub{Addr}}$ [\textit{неприятни} \textit{истини}]$_{\feinsub{Msg}}$ [\textit{за} \textit{смъртните} \textit{ни} \textit{врагове}]$_{\feinsub{Top}}$.
\\
{These persons} state to us unpleasant truths about mortal-\textsc{def} our enemies.
\\
\glt `These people state to us unpleasant truths about our mortal enemies.'
\newpage
\ex  \label{ex:04statementbg:e}
\gll {[В \textit{интервюто}]}$_{\feinsub{Med}}$ [\textit{Симеон}]$_{\feinsub{Spkr}}$ \textit{\textbf{ОБЯВИ}} [\textit{промяна} \textit{на} \textit{политическата} \textit{посока}]$_{\feinsub{Msg}}$.
\\
{In interview-\textsc{def}} Simeon announced change of political-\textsc{def} direction.
\\
\glt `In the interview Simeon announced a change in the political direction.'
\ex  \label{ex:04statementbg:f}
\gll {[}-- \textit{Тя} \textit{го} \textit{каза} \textit{просто} \textit{така}{]}$_{\feinsub{Msg}}$ -- \textit{\textbf{ДОБАВИ}} [\textit{Джени}]$_{\feinsub{Spkr}}$.
\\
--  She it said just so -- added Jenny.
\\
\glt `-- She said it just like that -- added Jenny.'
\end{xlist}
\end{exe}


\tabref{tbl:statement-synt-bg} shows some of the most frequent verbs in Bulgarian evoking the frame \framename{Statement}. The Bulgarian examples show similar patterns to the realisation of frame elements of the examples in the English dataset.

\tabref{tbl:statement-valence-bg} presents the most frequent valence patterns typical of the verbs evoking the \framename{Statement} frame in Bulgarian. Like in English, the most preferred realisations involve a subject \fename{Speaker} and a \fename{Message} expressed as an object NP, a clause or a quote.


\begin{table}
\centering\footnotesize
\begin{tabular}{l rrrrrrrrr}
\lsptoprule
 & NP.Ext & NP.Obj & PP & AVP & NI & Clause & Quote & Other & Total\\
\midrule
\multicolumn{10}{l}{\textit{обявявам\slash обявя} }\\
`announce'\\
\fename{Speaker} & 17 &  &  &  & 1 &  &  &  & 18\\ 
\fename{Message} &  & 4 &  &  &  & 12 & 1 & 1 & 18\\ 

\midrule
\multicolumn{10}{l}{\textit{твърдя} }\\ 
`claim'\\
\fename{Speaker} & 11 &  &  &  & 1 &  &  &  & 12\\ 
\fename{Message} &  &  &  &  &  & 10 & 2 &  & 12\\ 

\midrule
%\multicolumn{10}{l}{\textit{предлагам\slash предложа} }\\  
%\fename{Communicator} & 8 &  &  &  &  &  &  &  & 8\\ 
%\fename{Message} &  & 1 &  &  &  & 7 &  &  & 8\\ 
%\fename{Addressee} &  &  & 3 &  &  &  &  &  & 3\\ 
% \midrule
%\multicolumn{10}{l}{\textit{посочвам\slash посоча} }\\  
%\fename{Communicator} & 5 &  &  &  & 2 &  &  &  & 7\\ 
%\fename{Message} & 1 & 3 &  &  &  & 3 &  &  & 7\\ 
% \midrule
\multicolumn{10}{l}{\textit{коментирам} }\\  
`comment'\\
\fename{Speaker} & 8 &  &  &  &  &  &  &  & 8\\ 
\fename{Message} &  & 4 &  &  &  & 2 & 1 & 1 & 8\\ 

\midrule
%\multicolumn{10}{l}{\textit{оповестявам\slash оповестя} }\\  
%\fename{Communicator} & 1 &  &  &  &  &  &  &  & 1\\ 
%\fename{Message} &  & 1 &  &  &  &  &  &  & 1\\ 
% \midrule
%\multicolumn{10}{l}{\textit{отбелязвам\slash отбележа} }\\  
%\fename{Communicator} & 6 &  &  &  & 1 &  &  &  & 7\\ 
%\fename{Message} &  &  &  &  &  & 5 & 2 &  & 7\\ 
% \midrule
\multicolumn{10}{l}{\textit{добавям\slash добавя} }\\  
`add'\\
\fename{Speaker} & 10 &  &  &  &  &  &  &  & 10\\ 
\fename{Message} &  &  &  &  &  & 5 & 5 &  & 10\\ 

\midrule
\multicolumn{10}{l}{\textit{съобщавам\slash съобщя} }\\  
`announce'\\
\fename{Speaker} & 10 &  &  &  & 1 &  &  &  & 11\\ 
\fename{Message} &  & 4 &  &  &  & 2 & 5 &  & 11\\ 
\fename{Addressee} &  &  & 1 &  &  &  &  &  & 1\\ 

\midrule
\multicolumn{10}{l}{\textit{казвам\slash кажа} }\\
`say'\\
\fename{Speaker} & 47 &  &  &  & 1 &  &  &  & 48\\ 
\fename{Message} & 1 & 10 &  &  &  & 18 & 19 &  & 48\\ 
\fename{Addressee} &  &  & 4 &  &  &  &  &  & 4\\ 

\midrule
%\multicolumn{10}{l}{\textit{повтарям\slash повторя} }\\  
%\fename{Communicator} & 3 &  &  &  &  &  &  &  & 3\\ 
%\fename{Message} &  & 2 &  &  &  &  & 1 &  & 3\\ 
% \midrule
\multicolumn{10}{l}{\textit{обяснявам\slash обясня} }\\
`explain'\\
\fename{Speaker} & 14 &  &  &  & 2 &  &  &  & 16\\ 
\fename{Message} & 1 & 2 &  &  & 1 & 5 & 6 & 1 & 16\\ 
\fename{Addressee} &  &  & 6 &  &  &  &  &  & 6\\ 

\midrule
\multicolumn{10}{l}{\textit{заявявам\slash заявя} }\\
`state'\\
\fename{Speaker} & 17 &  &  &  &  &  &  &  & 17\\ 
\fename{Message} &  &  &  &  &  & 10 & 7 &  & 17\\ 
\fename{Addressee} &  &  & 4 &  &  &  &  &  & 4\\ 

\lspbottomrule
%\multicolumn{10}{l}{\textit{пиша} }\\  
%\fename{Communicator} & 6 &  &  &  &  &  &  &  & 6\\ 
%\fename{Message} &  &  &  &  &  & 1 & 5 &  & 6\\ 
%\fename{Topic} &  &  & 1 &  &  &  &  &  & 1\\ 
% \midrule
 \end{tabular}
 \caption{Syntactic expression of the \framename{Statement} frame elements in Bulgarian lexical units.  } 
    \label{tbl:statement-synt-bg}
 \end{table}

  \begin{table}
    \centering\footnotesize
    \begin{tabularx}{\textwidth}{ lrQ }
\lsptoprule
         Pattern  & \#  & verbs \\
\midrule
{[NP.Ext]}$_{\feinsub{Spkr}}$ {[Clause]}$_{\feinsub{Msg}}$ & 67 & \textit{добавям\slash добавя, заявявам\slash заявя, казвам\slash кажа, коментирам, обявявам\slash обявя, обяснявам\slash обясня, отбелязвам\slash отбележа, пиша, посочвам\slash посоча, предлагам\slash предложа, твърдя}\\

{[NP.Ext]}$_{\feinsub{Spkr}}$ {[Quote]}$_{\feinsub{Msg}}$ & 48 & \textit{добавям\slash добавя, заявявам\slash заявя, казвам\slash кажа, коментирам, обявявам\slash обявя, обяснявам\slash обясня, отбелязвам\slash отбележа, пиша, повтарям\slash повторя, съобщавам\slash съобщя, твърдя}\\

{[NP.Ext]}$_{\feinsub{Spkr}}$ {[NP.Obj]}$_{\feinsub{Msg}}$ & 29 & \textit{казвам\slash кажа, коментирам, обявявам\slash обявя, оповестявам\slash оповестя, повтарям\slash повторя, посочвам\slash посоча, предлагам\slash предложа, съобщавам\slash съобщя}\\

{[NP.Ext]}$_{\feinsub{Spkr}}$ {[Clause]}$_{\feinsub{Msg}}$  {[PP]}$_{\feinsub{Addr}}$ & 9 & \textit{заявявам\slash заявя, обяснявам\slash обясня, предлагам\slash предложа, съобщавам\slash съобщя}\\

{[NP.Ext]}$_{\feinsub{Spkr}}$ {[PP]}$_{\feinsub{Addr}}$  {[Quote]}$_{\feinsub{Msg}}$ & 5 & \textit{заявявам\slash заявя, казвам\slash кажа}\\

%{[Clause]}$_{\feinsub{Msg}}$ {[\_]}$_{\feinsub{Com-INI}}$ & 4 & \textit{отбелязвам\slash отбележа, посочвам\slash посоча, съобщавам\slash съобщя, твърдя}\\

%{[NP.Ext]}$_{\feinsub{Msg}}$ {[\_]}$_{\feinsub{Com-INI}}$ & 2 & \textit{казвам\slash кажа, посочвам\slash посоча}\\

{[NP.Ext]}$_{\feinsub{Spkr}}$ {[NP.Obj]}$_{\feinsub{Msg}}$   {[PP]}$_{\feinsub{Addr}}$ & 2 & \textit{обяснявам\slash обясня}\\

%{[NP.Ext]}$_{\feinsub{Msg}}$ {[PP]}$_{\feinsub{Addr}}$ {[\_]}$_{\feinsub{Com-INI}}$ & 1 & \textit{обяснявам\slash обясня}\\

%{[Sinterrog]}$_{\feinsub{Msg}}$ {[\_]}$_{\feinsub{Com-INI}}$ & 1 & \textit{обяснявам\slash обясня}\\

%{[NP.Ext]}$_{\feinsub{Com}}$ {[PP]}$_{\feinsub{Addr}}$ {[\_]}$_{\feinsub{Msg-INI}}$ & 1 & \textit{обяснявам\slash обясня}\\

%{[NP.Ext]}$_{\feinsub{Com}}$ {[Sinterrog]}$_{\feinsub{Msg}}$ & 1 & \textit{коментирам}\\

%{[NP.Ext]}$_{\feinsub{Com}}$ {[PP]}$_{\feinsub{Top}}$ {[Quote]}$_{\feinsub{Msg}}$ & 1 & \textit{пиша}\\

%{[Clause.Ext]}$_{\feinsub{Msg}}$ {[\_]}$_{\feinsub{Com-INI}}$ & 1 & \textit{обявявам\slash обявя}\\
\lspbottomrule
    \end{tabularx}
    \caption{FrameNet valence patterns of \framename{Statement} verbs, their frequency in the Bulgarian dataset and the verbs they appear with.
     English translation equivalents: \textit{добавям\slash добавя} `add', \textit{заявявам\slash заявя} `state', \textit{казвам\slash кажа} `say', \textit{коментирам} `comment', \textit{обявявам\slash обявя, оповестявам\slash оповестя, съобщавам\slash съобщя} `announce', \textit{обяснявам, обясня} `explain', \textit{отбелязвам\slash отбележа} `note', \textit{пиша} `write', \textit{повтарям\slash повторя} `reiterate', \textit{посочвам\slash посоча} `state', \textit{предлагам\slash предложа} `suggest'.}
    \label{tbl:statement-valence-bg}
\end{table} 


\subsection{Frame \framename{Telling}}

\begin{description}[font=\normalfont]
\item[The definition of the \framename{Telling} frame is:] A \fename{Speaker} addresses an \fename{Addressee} with  a \fename{Message}, which may be indirectly referred to as a \fename{Topic}. Instead of (or in addition to) a \fename{Speaker}, a \fename{Medium} may also be mentioned. Core frame elements: \fename{Speaker}, \fename{Addressee}, \fename{Message}, \fename{Medium}, \fename{Topic}.
\end{description}

The frame \framename{Telling} is evoked by a small number of frequently occurring verbs such as \textit{tell}.v, \textit{advise}.v, \textit{inform}.v, \textit{notify}.v, etc. The frame inherits from \framename{Statement} and its specialisation consists in the fact that it describes speech acts directed to a specific \fename{Addressee}. As a result this frame element is promoted to core status and with most verbs (\textit{inform}.v, \textit{advise}.v, \textit{confide}.v, \textit{notify}.v) is favoured for the direct object position. 

\subsubsection{Syntactic realisation of the \framename{Telling} frame elements}

The frame elements generally have the same characteristics as the ones in the \framename{Statement} frame from which they are inherited. The \fename{Speaker} usually takes the position of the external NP (Example \ref{ex:05tell:a}). Most often the \fename{Addressee} is expressed as an NP object (Example \ref{ex:05tell:b}) or in the case of \textit{tell}.v as an indirect object NP or a PP. %Both the \fename{Speaker} and the \fename{Addressee} are sentient beings able to produce and perceive language, respectively. They can also be groups or organisations (Example \ref{ex:05tell:b}).

\largerpage
The \fename{Message} is most often realised as a prepositional phrase, a subordinate clause or a quote (Example \ref{ex:05tell:b}, \ref{ex:05tell:c}, \ref{ex:05tell:a}, respectively). It may also take the position of an NP object, while the \fename{Addressee} is represented by a PP (Example \ref{ex:05tell:d}), a pattern which is actually favoured by the verb \textit{confide}.v. %A limited number of verbs, such as \textit{tell}, allow both the \fename{Message} and the \fename{Addressee} to assume the position of direct objects (Example \ref{ex:05tell:e}). 
Instead of the \fename{Message}, its \fename{Topic} may be realised as a prepositional phrase (Example \ref{ex:05tell:f}). 

\begin{exe}
\ex \label{ex:05tell}
\begin{xlist}
\ex  \label{ex:05tell:a}
\glt {[}`\textit{Take your bag and go},'{]}$_{\feinsub{Msg}}$ [\textit{Jake}]$_{\feinsub{Spkr}}$ \textit{\textbf{TOLD}} [\textit{her}]$_{\feinsub{Addr}}$.
\ex  \label{ex:05tell:b}
\glt {[}\textit{The police}{]}$_{\feinsub{Spkr}}$ \textit{didn't \textbf{INFORM}} [\textit{the British Consulate}]$_{\feinsub{Addr}}$ \newline [\textit{about his disappearance}]$_{\feinsub{Msg}}$. 
 \ex  \label{ex:05tell:c}
\glt {[}\textit{We}{]} \textit{have \textbf{NOTIFIED}} [\textit{Benoit}]$_{\feinsub{Addr}}$ [\textit{that Tweed is wanted}]$_{\feinsub{Msg}}$.
 \ex  \label{ex:05tell:d}
{[}\textit{She}{]}$_{\feinsub{Spkr}}$ \textit{\textbf{CONFIDED}} [\textit{her sadness}]$_{\feinsub{Msg}}$ [\textit{in Beth}]$_{\feinsub{Addr}}$.
 %\ex  \label{ex:05tell:e}
%\glt {[}\textit{I}{]}$_{\feinsub{Spkr}}$ \textit{told} [\textit{him}]$_{\feinsub{Addr}}$  [\textit{a bedtime story}]$_{\feinsub{Msg}}$.
 \ex  \label{ex:05tell:f}
\glt {[}\textit{He}{]}$_{\feinsub{Spkr}}$ \textit{will \textbf{ADVISE}} [\textit{you}]$_{\feinsub{Addr}}$ [\textit{on the inheritance tax}]$_{\feinsub{Top}}$.
\end{xlist}
\end{exe}



The various specific configurations of frame elements as expressed by verbs in the \framename{Telling} frame are shown in \tabref{tbl:telling-synt}.


\begin{table}
\centering\footnotesize
\begin{tabular}{l rrrrrrrrr}
\lsptoprule
 & NP.Ext & NP.Obj & PP & AVP & NI & Clause & Quote & Other & Total\\ 

\midrule
 \multicolumn{10}{l}{\textit{tell} } \\  
\fename{Speaker} & 90  & 1  & 9  &  & 14  &  &  &  & 114\\ 
\fename{Addressee} & 18  & 59  & 3  & 1  & 36  & 1  &  & 2 & 120\\ 
\fename{Topic} &  & 3  & 31  &  &  & 4  &  & 1 & 39\\ 
\fename{Message} & 5  & 11  & 9  & 3  & 13  & 35  & 6  & 8 & 90\\ 
\fename{Medium} & 10  &  & 2  &  &  &  &  &  & 12\\ 

\midrule
\multicolumn{10}{l}{\textit{inform} } \\  
\fename{Speaker} & 39  &  &  &  & 8  &  &  &  & 47\\ 
\fename{Addressee} & 8  & 37  &  &  & 2  &  &  &  & 47\\ 
\fename{Message} &  &  & 10  &  & 7  & 20  & 6  &  & 43\\ 
\fename{Medium} &  &  & 3  &  &  &  &  &  & 3\\ 
\fename{Topic} &  &  & 4  &  &  &  &  &  & 4\\ 

\midrule
\multicolumn{10}{l}{\textit{advise} } \\  
\fename{Speaker} & 59  &  & 1  &  & 6  &  &  &  & 66\\ 
\fename{Addressee} & 8  & 31  & 1  &  & 27  &  &  &  & 67\\ 
\fename{Message} &  & 3  & 7  &  &  & 29  & 8  &  & 47\\ 
\fename{Topic} &  &  & 19  &  & 1  &  &  &  & 20\\ 

\midrule
%\multicolumn{10}{l}{\textit{assure} } \\  
%\fename{Speaker} & 11  &  & 4  &  & 5  &  &  &  & 20\\ 
%\fename{Addressee} & 9  & 6  &  &  & 5  &  &  &  & 20\\ 
%\fename{Message} &  &  &  &  & 1  & 11  & 6  & 1 & 19\\ 
%\fename{Topic} &  &  &  &  & 1  &  &  &  & 1\\ 
% \midrule
\multicolumn{10}{l}{\textit{confide} } \\  
\fename{Speaker} & 45  &  &  &  & 1  &  &  &  & 46\\ 
\fename{Addressee} &  &  & 23  &  & 23  &  &  &  & 46\\ 
\fename{Message} & 1  & 23  &  &  & 4  & 14  & 4  &  & 46\\ 
\fename{Medium} &  &  & 1  &  &  &  &  &  & 1\\ 

\lspbottomrule
%\multicolumn{10}{l}{\textit{notify} } \\  
%\fename{Speaker} & 22  &  & 1  &  & 7  &  &  &  & 30\\ 
%\fename{Addressee} & 6  & 18  & 4  &  & 2  &  &  &  & 30\\ 
%\fename{Message} & 2  & 4  & 4  &  & 12  & 7  &  &  & 29\\ 
%\fename{Medium} &  &  & 5  &  &  &  &  &  & 5\\ 
%\fename{Topic} &  &  & 1  &  &  &  &  &  & 1\\ 
% \midrule
 \end{tabular}
 \caption{Syntactic expression of the \framename{Telling} frame elements in selected FrameNet lexical units. } 
    \label{tbl:telling-synt}
 \end{table}



\subsubsection{\framename{Telling} valence patterns}

The prevalent valence patterns for the verbs in the FrameNet frame \framename{Telling} are illustrated in \tabref{tbl:telling-valence}. These include the prototypical expression of the \fename{Speaker} as the external NP, usually with a direct object \fename{Addressee}, which may be left implicit and/or a \fename{Message} realised as a subordinate clause, a prepositional phrase or a quote; the \fename{Message} may also be implicit. A PP \fename{Topic} may co-occur with the \fename{Addressee} but usually not with the \fename{Message}.


\begin{table}
    \centering\footnotesize
    \begin{tabularx}{\textwidth}{ lrQ }
\lsptoprule
         Pattern  & \#  & verbs \\
\midrule
{[NP.Ext]}$_{\feinsub{Spkr}}$ {[NP.Obj]}$_{\feinsub{Addr}}$ {[Clause]}$_{\feinsub{Msg}}$  & 53 & \textit{inform, advise, tell, assure, notify}\\

{[NP.Ext]}$_{\feinsub{Spkr}}$ {[NP.Obj]}$_{\feinsub{Addr}}$ {[PP]}$_{\feinsub{Top}}$  & 30 & \textit{apprise, inform, advise, tell, notify}\\

{[NP.Ext]}$_{\feinsub{Spkr}}$ {[\_]}$_{\feinsub{Addr-DNI}}$ {[Clause]}$_{\feinsub{Msg}}$  & 26 & \textit{advise, confide, tell, assure}\\

{[NP.Ext]}$_{\feinsub{Spkr}}$ {[NP.Obj]}$_{\feinsub{Addr}}$ {[\_]}$_{\feinsub{Msg-DNI}}$  & 20 & \textit{inform, tell, assure, notify}\\

{[NP.Ext]}$_{\feinsub{Spkr}}$ {[NP.Obj]}$_{\feinsub{Addr}}$ {[PP]}$_{\feinsub{Msg}}$  & 20 & \textit{inform, advise, tell, notify}\\

{[NP.Ext]}$_{\feinsub{Spkr}}$ {[\_]}$_{\feinsub{Addr-DNI}}$ {[PP]}$_{\feinsub{Top}}$  & 17 & \textit{advise, tell}\\

{[NP.Ext]}$_{\feinsub{Spkr}}$ {[\_]}$_{\feinsub{Addr-DNI}}$ {[NP.Obj]}$_{\feinsub{Msg}}$  & 16 & \textit{advise, confide, tell}\\

{[NP.Ext]}$_{\feinsub{Spkr}}$ {[PP]}$_{\feinsub{Addr}}$ {[NP.Obj]}$_{\feinsub{Msg}}$  & 16 & \textit{advise, confide, tell, notify}\\

{[NP.Ext]}$_{\feinsub{Spkr}}$ {[\_]}$_{\feinsub{Addr-DNI}}$ {[Quote]}$_{\feinsub{Msg}}$  & 14 & \textit{advise, confide, assure}\\

{[NP.Ext]}$_{\feinsub{Spkr}}$ {[NP.Obj]}$_{\feinsub{Addr}}$ {[Quote]}$_{\feinsub{Msg}}$  & 11 & \textit{inform, tell, assure}\\

%{[NP.Ext]}$_{\feinsub{Spkr}}$ {[PP]}$_{\feinsub{Addr}}$ {[Clause]}$_{\feinsub{Msg}}$  & 4 & \textit{confide}\\

%{[NP.Ext]}$_{\feinsub{Spkr}}$ {[PP]}$_{\feinsub{Addr}}$ {[\_]}$_{\feinsub{Msg-INI}}$  & 4 & \textit{confide}\\

%{[NP.Ext]}$_{\feinsub{MEDIUM}}$ {[NP.Obj]}$_{\feinsub{Addr}}$ {[Clause]}$_{\feinsub{Msg}}$  & 4 & \textit{tell}\\

%{[NP.Ext]}$_{\feinsub{Spkr}}$ {[\_]}$_{\feinsub{Addr-DNI}}$ {[PP]}$_{\feinsub{Msg}}$  & 4 & \textit{advise, tell}\\

%{[NP.Ext]}$_{\feinsub{Spkr}}$ {[NP.Obj]}$_{\feinsub{Addr}}$ {[NP.Dep]}$_{\feinsub{Msg}}$  & 3 & \textit{tell, assure}\\
\lspbottomrule
\end{tabularx}
    \caption{FrameNet valence patterns of \framename{Telling} verbs, their frequency in the FrameNet corpus and the verbs they appear with.}
    \label{tbl:telling-valence}
\end{table} 



\subsubsection{Syntactic realisation and patterns in Bulgarian}
\largerpage
In a similar manner, in Bulgarian the \fename{Speaker} is realised as the external subject NP, while the \fename{Message} takes the position of an object NP, a subordinate clause or a quote (Example \ref{ex:05tellbg:a}, \ref{ex:05tellbg:b}, \ref{ex:05tellbg:c}). 

With some of the verbs in this frame, such as \textit{казвам, съобщавам} `tell, let know' the \fename{Addressee} assumes the position of the indirect object as the receiver to whom the message is directed (Example \ref{ex:05tellbg:b}), while with verbs such as \textit{уведомявам} `notify, inform', \textit{информирам, осведомявам} `inform' it is realised as an NP object (Example \ref{ex:05tellbg:d}); the \fename{Addressee} may also be null instantiated %considered as generalised public or audience and can be an indefinite null instantiation 
(Example \ref{ex:05tellbg:e}). 

\largerpage
\begin{exe}
\ex \label{ex:05tellbg}
\begin{xlist}
\ex  \label{ex:05tellbg:a}
\gll  {[}\_{]}$_{\feinsub{Spkr-DNI}}$ \textit{Искам} \textit{да} [\textit{ви}]$_{\feinsub{Addr}}$ \textit{\textbf{СЪОБЩЯ}} [\textit{една} \textit{тъжна} \textit{вест}]$_{\feinsub{Msg}}$. \\
{} Want.1sg to you.2pl-DAT tell one sad news.
\\
\glt `I want to tell you some sad news.'
\ex  \label{ex:05tellbg:b}
\gll {[}\textit{Всеки} \textit{българин}{]}$_{\feinsub{Spkr}}$ \textit{ще} [\textit{ти}]$_{\feinsub{Addr}}$ \textit{\textbf{КАЖЕ}} [\textit{каквото} \textit{е} \textit{чул} \textit{от} \textit{майка} \textit{си}]$_{\feinsub{Msg}}$. \\
Every Bulgarian will you.2sg-DAT tell whatever has heard from mother  REFL.
\\
\glt `Every Bulgarian will tell you whatever he has heard from his mother.'
\ex  \label{ex:05tellbg:c}
\gll  {[}\textit{Не} \textit{са} \textit{намерили} \textit{Санса}{]}$_{\feinsub{Msg}}$ -- \textit{учтиво} [\textit{го}]$_{\feinsub{Addr}}$ \textit{\textbf{УВЕДОМИ}} [\textit{чичо} \textit{му}]$_{\feinsub{Spkr}}$.\\
Not have found Sansa -- politely him informed uncle his.
\\
\glt `They have not found Sansa -- his uncle informed him politely.'
\ex  \label{ex:05tellbg:d}
\gll  {[}\_{]}$_{\feinsub{Spkr-DNI}}$ \textit{Трябва} \textit{да} \textit{\textbf{ОСВЕДОМЯ}} [\textit{читателя}]$_{\feinsub{Addr}}$ [\textit{за} \textit{тайната} \textit{интрига}]$_{\feinsub{Top}}$.
\\
{} Need.1sg to inform reader-\textsc{def} about secret-\textsc{def} plot.
\\
\glt `I need to inform the reader about the secret plot.'
\ex  \label{ex:05tellbg:e}
\gll {[}\textit{Пенсионерите} \textit{да} \textit{избягват} \textit{навалиците}{]}$_{\feinsub{Msg}}$, \textit{\textbf{СЪВЕТВА}}  [\textit{г-жа} \textit{Ненова}]$_{\feinsub{Spkr}}$ [\_]$_{\feinsub{Addr-INI}}$.\\
Elderly-\textsc{def} to avoid crowds,  advises  Mrs Nenova. {}
\\
\glt `The elderly should avoid crowds, Mrs Nenova advises.' 
\end{xlist}
\end{exe}

\begin{table}
\centering\footnotesize
\begin{tabular}{l rrrrrrrrr}
\lsptoprule
 & NP.Ext & NP.Obj & PP & AVP & NI & Clause & Quote & Other & Total\\ 

\midrule
\multicolumn{10}{l}{\textit{уверявам\slash уверя} }\\
`assure'\\
\fename{Message} &  &  & 1 &  &  & 24 & 6 &  & 31\\ 
\fename{Addressee} &  & 31 &  &  &  &  &  &  & 31\\ 
\fename{Speaker} & 31 &  &  &  &  &  &  &  & 31\\ 

\midrule
%\multicolumn{10}{l}{\textit{осведомявам\slash осведомя} }\\  
%\fename{Message} &  &  &  &  & 2 & 2 &  &  & 4\\ 
%\fename{Addressee} &  & 4 &  &  &  &  &  &  & 4\\ 
%\fename{Topic} &  &  & 2 &  &  &  &  &  & 2\\ 
%\fename{Speaker} & 4 &  &  &  &  &  &  &  & 4\\ 
% \midrule
\multicolumn{10}{l}{\textit{съобщавам\slash съобщя} }\\
`tell, let know'\\
\fename{Message} &  & 3 &  &  &  & 2 &  &  & 5\\ 
\fename{Addressee} &  &  & 5 &  &  &  &  &  & 5\\ 
\fename{Speaker} & 5 &  &  &  &  &  &  &  & 5\\ 

\midrule
\multicolumn{10}{l}{\textit{уведомявам\slash уведомя} }\\ 
`inform, notify'\\
\fename{Message} &  &  &  &  & 5 & 5 & 3 &  & 13\\ 
\fename{Addressee} & 1 & 15 &  &  &  &  &  &  & 16\\ 
\fename{Topic} &  &  & 4 &  &  &  &  &  & 4\\ 
\fename{Speaker} & 15 &  &  &  & 1 &  &  &  & 16\\ 

\midrule
\multicolumn{10}{l}{\textit{казвам\slash кажа} }\\  
`tell'\\
\fename{Message} &  & 11 &  &  &  & 15 & 6 &  & 32\\ 
\fename{Addressee} &  &  & 32 &  &  &  &  &  & 32\\ 
\fename{Speaker} & 32 &  &  &  &  &  &  &  & 32\\ 

\lspbottomrule
%\multicolumn{10}{l}{\textit{посъветвам} }\\  
%\fename{Message} &  &  &  &  &  & 3 & 2 &  & 5\\ 
%\fename{Addressee} &  & 3 &  &  &  &  &  &  & 3\\ 
%\fename{Speaker} & 5 &  &  &  &  &  &  &  & 5\\ 
% \midrule
 \end{tabular}
 \caption{Syntactic expression of the \framename{Telling} frame elements in Bulgarian. } 
    \label{tbl:telling-synt-bg}
 \end{table}

\tabref{tbl:telling-synt-bg} presents the most frequent verbs in Bulgarian evoking the frame \framename{Telling}, while
\tabref{tbl:telling-valence-bg} shows the typical valence patterns. The \fename{Message} and the \fename{Addressee} tend to co-occur syntactically, while the \fename{Topic} is expressed more rarely.


\begin{table}
    \centering\footnotesize
    \begin{tabularx}{\textwidth}{ lr Q}
\lsptoprule
         Pattern  & \#  & verbs \\
\midrule
{[NP.Ext]}$_{\feinsub{Spkr}}$ {[Clause]}$_{\feinsub{Msg}}$ {[NP.Obj]}$_{\feinsub{Addr}}$ & 32 & \textit{осведомявам\slash осведомя, уверявам\slash уверя, уведомявам\slash уведомя}\\

{[NP.Ext]}$_{\feinsub{Spkr}}$ {[Clause]}$_{\feinsub{Msg}}$ {[PP]}$_{\feinsub{Addr}}$ & 15 & \textit{казвам\slash кажа, съобщавам\slash съобщя}\\

{[NP.Ext]}$_{\feinsub{Spkr}}$ {[NP.Obj]}$_{\feinsub{Msg}}$ {[PP]}$_{\feinsub{Addr}}$ & 14 & \textit{казвам\slash кажа, съобщавам\slash съобщя}\\

{[NP.Ext]}$_{\feinsub{Spkr}}$ {[NP.Obj]}$_{\feinsub{Addr}}$ {[Quote]}$_{\feinsub{Msg}}$ & 9 & \textit{уверявам\slash уверя, уведомявам\slash уведомя}\\

{[NP.Ext]}$_{\feinsub{Spkr}}$ {[PP]}$_{\feinsub{Addr}}$ {[Quote]}$_{\feinsub{Msg}}$ & 6 & \textit{казвам\slash кажа}\\

{[NP.Ext]}$_{\feinsub{Spkr}}$ {[NP.Obj]}$_{\feinsub{Addr}}$ {[PP]}$_{\feinsub{Top}}$  {[\_]}$_{\feinsub{Msg-INI}}$ & 5 & \textit{осведомявам\slash осведомя, уведомявам\slash уведомя}\\

{[NP.Ext]}$_{\feinsub{Spkr}}$ {[Quote]}$_{\feinsub{Msg}}$  {[\_]}$_{\feinsub{Addr-INI}}$ & 4 & \textit{информирам, посъветвам}\\

%{[NP.Ext]}$_{\feinsub{Spkr}}$ {[NP.Obj]}$_{\feinsub{Addr}}$ {[]}$_{\feinsub{Msg}}$ & 3 & \textit{уведомявам\slash уведомя}\\
\lspbottomrule
    \end{tabularx}
    \caption{FrameNet valence patterns of the frame \framename{Telling}, their frequency in the Bulgarian dataset and the verbs they appear with.
     English translation equivalents: \textit{информирам, осведомявам\slash осведомя} `inform', \textit{казвам, съобщавам\slash съобщя} `tell, let know', \textit{посъветвам} `advise', \textit{уведомявам\slash уведомя} `notify', \textit{уверявам\slash уверя} `assure'.}
    \label{tbl:telling-valence-bg}
\end{table} 



\subsection{Frame \framename{Judgment\_communication}}

\begin{description}[font=\normalfont]
\item[Definition of the frame \framename{Judgment\_communication}:] A \fename{Communicator} communicates a judgement of an \fename{Evaluee} to an \fename{Addressee}. The judgement may be positive (e.g. \textit{praise}.v) or negative (e.g. \textit{criticise}.v). Core frame elements: \fename{Communicator}, \fename{Expressor}, \fename{Reason}, \fename{Medium}, \fename{Topic}, \fename{Evaluee}; Non-core: \fename{Addressee}.
\end{description}

The frame \framename{Judgment\_communication} inherits from both the \framename{Statement} and the \framename{Judgment} frame (weak inheritance through the \FrameRelation{Uses} frame-to-frame relation). Verbs included in this frame concern acts of speech which also convey judgement on a certain topic, the \fename{Evaluee}. The frame elaborates on the frame \framename{Statement} most notably in the interpretation of the \fename{Message} as a judgement on a complex state-of-affairs concerning an additional participant, represented by the frame element \fename{Evaluee}. The \fename{Evaluee} can be a person, an object, an action or any topic (Example \ref{ex:06judgment:a}, \ref{ex:06judgment:b}, \ref{ex:06judgment:f}). The judgement may be positive, e.g. \textit{praise}.v, \textit{commend}.v, \textit{acclaim}.v, or negative, e.g. \textit{criticise}.v, \textit{condemn}.v, \textit{denounce}.v; its value is encoded by the verb. In addition, the frame element \fename{Reason} denotes the argumentation for the judgement. The \fename{Addressee} is a non-core frame element, reflecting the fact that the judgement regarding the \fename{Evaluee} may but need not be intended for another participant. %the \fename{Addressee}, or it can be expressed in general terms and hence null instantiated \fename{Addressee} (Example \ref{ex:06judgment:a}, \ref{ex:06judgment:d}).

\subsubsection{Syntactic realisation of the \framename{Judgment\_communication} frame elements}

%The syntactic expression of the  configuration of core frame elements in the frame \framename{Judgment\_communication} resembles the realisation of the verbs in the frame \framename{Statement} which it inherits. 

The frame \framename{Judgment\_communication} specifies the more general frame element \fename{Communicator} rather than inheriting the \fename{Speaker} from the \framename{Statement} frame. The reason for this is that the frame also includes verbs which represent communication acts that are more general or complex than speech acts, e.g. \textit{belittle}.v, \textit{ridicule}.v. 

The \fename{Communicator} is usually realised as the external argument and can be represented by a person, a group or an organisation  (Example \ref{ex:06judgment:a}, \ref{ex:06judgment:b}).  

%In this frame the message inherent in any communication act is reconceptualised as a construction involving the \fename{Evaluee}  and the \fename{Reason} for the judgement. 
The \fename{Evaluee} is most often expressed in the position of the NP direct object (Example \ref{ex:06judgment:a}, \ref{ex:06judgment:b}, \ref{ex:06judgment:c}), while the \fename{Reason} can be a prepositional phrase headed by prepositions such as \textit{for, of, as} (Example \ref{ex:06judgment:c}, \ref{ex:06judgment:d}, \ref{ex:06judgment:f}). Instead of the \fename{Reason}, a \fename{Topic} can be present (Example \ref{ex:06judgment:e}).

The \fename{Addressee}, whenever overt, is expressed as a prepositional phrase (Example \ref{ex:06judgment:b}). 

The \fename{Expressor} is rare with verbs evoking this frame and usually represents a body part or an action performed by the \fename{Communicator} in order to convey the judgment (Example \ref{ex:06judgment:g}). 

\begin{exe}
\ex \label{ex:06judgment}
\begin{xlist}
\ex  \label{ex:06judgment:a}
{[}\textit{Frank}{]}$_{\feinsub{Com}}$ \textit{\textbf{RIDICULED}} [\textit{everything}]$_{\feinsub{Eval}}$.
\ex  \label{ex:06judgment:b}
{[}\textit{Jon}{]}$_{\feinsub{Com}}$ \textit{\textbf{BELITTLED}} [\textit{Madie}]$_{\feinsub{Eval}}$ {[}\textit{to her colleagues}{]}$_{\feinsub{Addr}}$.
\ex  \label{ex:06judgment:c}
{[}\textit{Georgia}{]}$_{\feinsub{Com}}$ \textit{has \textbf{ACCUSED}} [\textit{Russian troops}]$_{\feinsub{Eval}}$ [\textit{of backing separa\-tists}]$_{\feinsub{Reas}}$.
\ex  \label{ex:06judgment:d}
{[}\textit{I}{]}$_{\feinsub{Com}}$ \textit{have \textbf{PRAISED}} [\textit{her}]$_{\feinsub{Eval}}$ [\textit{for her work}]$_{\feinsub{Reas}}$.
\ex  \label{ex:06judgment:e}
{[}\textit{He}{]}$_{\feinsub{Com}}$ \textit{\textbf{CRITICISED}} [\textit{the president}]$_{\feinsub{Eval}}$ [\textit{over his decision to go to \linebreak war}]$_{\feinsub{Top}}$.
\ex  \label{ex:06judgment:f}
{[}\textit{The conservatives}{]}$_{\feinsub{Com}}$ \textit{\textbf{DENOUNCED}} {[}\textit{the proposed reforms}{]}$_{\feinsub{Eval}}$ [\textit{as an attempt to distract voters}]$_{\feinsub{Reas}}$.
\ex  \label{ex:06judgment:g}
{[}\textit{His glance}{]}$_{\feinsub{Exr}}$ \textit{\textbf{DENIGRATED}} [\textit{her attempt at humour}]$_{\feinsub{Eval}}$.
\end{xlist}
\end{exe}

\tabref{tbl:judgment-synt} shows some of the frequent verbs of the frame and the realisation of their frame elements.

\begin{table}
\centering\footnotesize
\begin{tabular}{l rrrrrrrrr}
\lsptoprule
 & NP.Ext & NP.Obj & PP & AVP & NI & Clause & Quote & Other & Total\\ 

\midrule
%\multicolumn{10}{l}{\textit{accuse} } \\  
%\fename{Communicator} & 32  &  & 7  & 1  & 16  &  &  &  & 56\\ 
%\fename{Evaluee} & 24  & 32  &  &  &  &  &  &  & 56\\ 
%\fename{Reason} &  &  & 51  &  & 5  &  &  &  & 56\\ 
%\fename{Medium} &  &  & 1  &  &  &  &  &  & 1\\ 
% \midrule
%\multicolumn{10}{l}{\textit{commend} } \\  
%\fename{Communicator} & 29  &  & 3  &  & 5  &  &  &  & 37\\ 
%\fename{Addressee} &  &  & 3  &  &  &  &  &  & 3\\ 
%\fename{Evaluee} & 6  & 22  &  &  & 3  &  &  & 6 & 37\\ 
%\fename{Reason} & 2  & 7  & 21  &  & 5  &  &  &  & 35\\ 
% \midrule
\multicolumn{10}{l}{\textit{condemn} } \\  
\fename{Communicator} & 105  &  & 21  &  & 11  &  &  &  & 137\\ 
\fename{Evaluee} & 32  & 103  & 1  &  & 2  &    &    &  & 138\\ 
\fename{Medium} & 1  &  & 5  &  &  &  &  &  & 6\\ 
\fename{Reason} &  & 4  & 44  &  & 90  &   &  &  & 138\\ 

\midrule
\multicolumn{10}{l}{\textit{criticize} } \\  
\fename{Communicator} & 88  &  & 15  &  & 47  &  &  &  & 150\\ 
\fename{Addressee} &  &  & 1  &  &  &  &  &  & 1\\ 
\fename{Evaluee} & 74  & 71  &  &  & 9  &    &    & 1 & 155\\ 
\fename{Reason} &  & 4  & 87  &  & 59  &    &  &  & 150\\ 
\fename{Topic} &  & 1  & 3  &  &  &  &  &  & 4\\ 
\fename{Medium} & 5  &  & 3  &  &  &  &  &  & 8\\ 

\midrule
%\multicolumn{10}{l}{\textit{denounce} } \\  
%\fename{Communicator} & 45  &  & 16  &  & 12  &  &  &  & 73\\ 
%\fename{Addressee} &  &  & 4  &  &  &  &  &  & 4\\ 
%\fename{Evaluee} & 29  & 38  &  &  &  & 5  & 2  &  & 74\\ 
%\fename{Reason} &  &  & 39  & 1  & 24  & 2  &  &  & 66\\ 
%\fename{Medium} &  &  & 5  &  &  &  &  &  & 5\\ 
% \midrule
\multicolumn{10}{l}{\textit{praise} } \\  
\fename{Communicator} & 50  &  & 12  &  & 18  &  &  &  & 80\\ 
\fename{Evaluee} & 27  & 49  &  &  &  &    &    & 4 & 80\\ 
\fename{Reason} & 3  & 2  & 38  &  & 34  &    &  &  & 77\\ 
\fename{Medium} &  &  & 5  &  &  &  &  &  & 5\\ 

\midrule
\multicolumn{10}{l}{\textit{ridicule} } \\  
\fename{Communicator} & 14  &  & 16  &  & 16  &  &  &  & 46\\ 
\fename{Evaluee} & 38  & 9  &  &  &  &    &  &  & 47\\ 
\fename{Medium} & 1  &  & 2  &  &  &  &  &  & 3\\ 
\fename{Reason} &  & 1  & 13  &  & 33  &    &  &  & 47\\ 

\lspbottomrule
%\multicolumn{10}{l}{\textit{castigate} } \\  
%\fename{Evaluee} & 16  & 36  &  &  &  &  &  &  & 52\\ 
%\fename{Communicator} & 35  &  & 5  &  & 11  &  &  & 1 & 52\\ 
%\fename{Medium} &  &  & 1  &  &  &  &  &  & 1\\ 
%\fename{Reason} &  &  & 38  &  & 14  &  &  &  & 52\\ 
% \midrule
 \end{tabular}
 \caption{Syntactic expression of the \framename{Judgment\_communication} frame elements in selected FrameNet lexical units. } 
    \label{tbl:judgment-synt}
 \end{table}


\subsubsection{\framename{Judgment\_communication} valence patterns}

The valence patterns characteristic for verbs in the FrameNet frame \framename{Judgment\_\linebreak communication} are presented in \tabref{tbl:judgment-valence}. The most common ones involve a \fename{Communicator} as the external argument, a direct object NP \fename{Evaluee}, and an  either overtly expressed or implicit \fename{Reason} or much more rarely a \fename{Topic}.

\begin{table}
    \centering\footnotesize
    \begin{tabularx}{\textwidth}{ lrQ }
\lsptoprule
         Pattern  & \#  & verbs \\
\midrule
{[NP.Ext]}$_{\feinsub{Com}}$ {[NP.Obj]}$_{\feinsub{Eval}}$ {[PP]}$_{\feinsub{Reas}}$  & 263 & \textit{accuse, deprecate, denigrate, censure, castigate, condemn, ridicule, commend, belittle, denounce, praise, damn, criticize, execrate, mock}\\

{[NP.Ext]}$_{\feinsub{Com}}$ {[NP.Obj]}$_{\feinsub{Eval}}$   {[\_]}$_{\feinsub{Reas-DNI}}$  & 138 & \textit{accuse, deprecate, denigrate, censure, ridicule, commend, castigate, acclaim, belittle, condemn, denounce, praise, damn, criticize}\\

%{[NP.Ext]}$_{\feinsub{Com}}$ {[Clause]}$_{\feinsub{Eval}}$ {[\_]}$_{\feinsub{Reas-DNI}}$  & 38 & \textit{damn, deprecate, denigrate, ridicule, condemn, denounce, praise}\\

{[NP.Ext]}$_{\feinsub{Com}}$ {[NP.Obj]}$_{\feinsub{Eval}}$ {[\_]}$_{\feinsub{Reas-INI}}$  & 25 & \textit{criticize, denigrate, mock, castigate, condemn, denounce}\\

%{[NP.Ext]}$_{\feinsub{Com}}$ {[NP.Obj]}$_{\feinsub{Eval}}$ \newline {[Clause]}$_{\feinsub{Reas}}$  & 17 & \textit{criticize, denigrate, ridicule, condemn, denounce, praise}\\

%{[NP.Ext]}$_{\feinsub{Com}}$ {[\_]}$_{\feinsub{Eval-DNI}}$ {[\_]}$_{\feinsub{Reas-DNI}}$  & 15 & \textit{remonstrate, scoff}\\

%{[NP.Ext]}$_{\feinsub{Com}}$ {[NP.Obj]}$_{\feinsub{Eval}}$  & 14 & \textit{extol, commend, denounce, praise}\\

%{[NP.Ext]}$_{\feinsub{Com}}$ {[2nd]}$_{\feinsub{Eval}}$ {[NP.Obj]}$_{\feinsub{Reas}}$  & 13 & \textit{damn, criticize, denigrate, censure, commend, praise}\\

%{[NP.Ext]}$_{\feinsub{Com}}$ {[Clause]}$_{\feinsub{Eval}}$ {[\_]}$_{\feinsub{Reas-INI}}$  & 9 & \textit{criticize}\\

{[NP.Ext]}$_{\feinsub{Com}}$ {[NP.Obj]}$_{\feinsub{Eval}}$ {[PP]}$_{\feinsub{Top}}$  & 7 & \textit{slam, charge, criticize}\\
\lspbottomrule
    \end{tabularx}
    \caption{FrameNet valence patterns of \framename{Judgment\_communication} verbs, their frequency in the FrameNet corpus and the verbs they appear with.}
    \label{tbl:judgment-valence}
\end{table} 
 

\subsubsection{Syntactic realisation of the frame \framename{Judgement\_communication} in Bulgarian}

%The syntactic realisation of the verbs from the \framename{Judgement\_communication} frame in Bulgarian contains the same core frame elements. %and exhibits more varied valence patterns. 

The \fename{Communicator} is expressed as the external NP (Example \ref{ex:06judgmentbg:a}). The \fename{Evaluee} can be any concrete or abstract entity, quality, property, etc., whose properties are being evaluated, and is usually realised as the direct NP object (Example \ref{ex:06judgmentbg:a}, \ref{ex:06judgmentbg:b}) or as a prepositional phrase for a limited number of verbs such as \textit{подигравам се} `mock, ridicule' in (Example \ref{ex:06judgmentbg:e}). 

The \fename{Reason} is expressed as a prepositional phrase with a range of prepositions such as \textit{за, в, на} (Example \ref{ex:06judgmentbg:c}, \ref{ex:06judgmentbg:d}, \ref{ex:06judgmentbg:f}), or more rarely as a clause (Example \ref{ex:06judgmentbg:b}) or a direct quote (Example \ref{ex:06judgmentbg:g}). In some cases the \fename{Evaluee} and the \fename{Reason} can be expressed jointly (Example \ref{ex:06judgmentbg:f}).% na en syshto jointly i direct

The \fename{Addressee} is rarely expressed and is realised as a prepositional phrase (Example \ref{ex:06judgmentbg:g}).


\begin{exe}
\ex \label{ex:06judgmentbg}
\begin{xlist}
\ex  \label{ex:06judgmentbg:a}
\gll {[}\textit{Нашето} \textit{посолство}{]}$_{\feinsub{Com}}$ \textit{\textbf{ОСЪДИ}} [\textit{разрушаването} \textit{на} \textit{храма} \textit{в} \textit{Скопие}]$_{\feinsub{Eval}}$ {[}\_{]}$_{\feinsub{Reas-INI}}$. 
\\
Our embassy condemned destruction-\textsc{def} of church-\textsc{def} in Skopje. {}
 \\
 \glt `Our embassy condemned the destruction of the church in Skopje.'
\ex  \label{ex:06judgmentbg:b}
\gll {[}\_{]}$_{\feinsub{Com-DNI}}$ \textit{Не} \textit{могат} \textit{да} [\textit{ме}]$_{\feinsub{Eval}}$ \textit{\textbf{ОБВИНЯВАТ}}, [\textit{че} \textit{съм} \textit{ги} \textit{оскърбил}]$_{\feinsub{Reas}}$.
 \\
 {} Not can.3pl to me accuse that have them offended.
 \\
 \glt `They cannot accuse me of offending them.'
\ex  \label{ex:06judgmentbg:c}
\gll {[}\textit{България}{]}$_{\feinsub{Com}}$ [\textit{ни}]$_{\feinsub{Eval}}$ \textit{\textbf{ПРОКЛИНА}} [\textit{за} \textit{нещастията} \textit{си}]$_{\feinsub{Reas}}$.
 \\
Bulgaria us condemns for misfortunes-\textsc{def}  REFL.
 \\
 \glt `Bulgaria condemns us for its misfortunes.'
\ex  \label{ex:06judgmentbg:d}
\gll {[}\_{]}$_{\feinsub{Com-DNI}}$ \textit{\textbf{ОБВИНЯВАШЕ}} [\textit{ме}]$_{\feinsub{Eval}}$ [\textit{в} \textit{коравосърдечие}]$_{\feinsub{Reas}}$.
 \\
{} Accused me in cold-heartedness. {}
 \\
 \glt `He/she accused me of cold-heartedness.'
\ex  \label{ex:06judgmentbg:e}
\gll {[}\textit{Ти}{]}$_{\feinsub{Com}}$ \textit{\textbf{ПОДИГРАВАШ}} \textit{ли} \textit{\textbf{СЕ}} [\textit{с} \textit{мен}]$_{\feinsub{Eval}}$?
 \\
 You mock QST REFL with me?
 \\
 \glt `Are you mocking me?'
\ex  \label{ex:06judgmentbg:f}
\gll {[}\textit{Мускетарите}{]}$_{\feinsub{Com}}$ \textit{се} \textit{\textbf{ПОДИГРАВАХА}} [\textit{на} \textit{кривите} \textit{му} \textit{крака}]$_{\feinsub{Eval+Reas}}$.
 \\
Musketeers-\textsc{def}  REFL ridiculed for bow-\textsc{def} his legs.
 \\
 \glt `The musketeers ridiculed him for his bow legs.'
\ex  \label{ex:06judgmentbg:g}
\gll {[}--\textit{Много} \textit{е} \textit{наблюдателна}{]}$_{\feinsub{Reas}}$ -- \textit{\textbf{ПОХВАЛИ}} [\textit{я}]$_{\feinsub{Eval}}$ [\textit{той}]$_{\feinsub{Com}}$ [\textit{на} \textit{другите}]$_{\feinsub{Addr}}$.
 \\
 --Very is.3sg observant -- praised her he to others-\textsc{def}.
 \\
 \glt `-- She is very observant -- he praised her to the others.'
\end{xlist}
\end{exe}

\begin{table}
\centering\footnotesize
\begin{tabular}{l rrrrrrrrr}
\lsptoprule
 & NP.Ext & NP.Obj & PP & AVP & NI & Clause & Quote & Other & Total\\
\midrule
%\multicolumn{10}{l}{\textit{присмивам се} }\\  
%\fename{Communicator} & 6 &  &  &  &  &  &  &  & 6\\ 
%\fename{Evaluee} &  &  & 6 &  &  &  &  &  & 6\\ 
%\fename{Reason} &  &  & 1 &  &  &  &  &  & 1\\ 
% \midrule
\multicolumn{10}{l}{\textit{похвалвам\slash похваля} }\\  
`praise'\\
\fename{Communicator} & 16 &  &  &  &  &  &  &  & 16\\ 
\fename{Evaluee} &  & 15 &  &  & 1 &  &  &  & 16\\ 
\fename{Reason} &  &  & 2 &  &  &  &  &  & 2\\ 

\midrule
%\multicolumn{10}{l}{\textit{критикувам} }\\  
%\fename{Communicator} & 5 &  &  &  &  &  &  &  & 5\\ 
%\fename{Evaluee} &  & 5 &  &  &  &  &  &  & 5\\ 
% \midrule
%\multicolumn{10}{l}{\textit{хваля} }\\  
%\fename{Communicator} & 8 &  &  &  &  &  &  &  & 8\\ 
%\fename{Evaluee} &  & 8 &  &  &  &  &  &  & 8\\ 
%\fename{Reason} &  &  &  &  &  & 1 &  &  & 1\\ 
% \midrule
%\multicolumn{10}{l}{\textit{анатемосвам\slash анатемосам} }\\  
%\fename{Communicator} & 2 &  &  &  &  &  &  &  & 2\\ 
%\fename{Evaluee} &  & 1 &  &  & 1 &  &  &  & 2\\ 
% \midrule
%\multicolumn{10}{l}{\textit{отричам\slash отрека} }\\  
%\fename{Communicator} & 3 &  &  &  &  &  &  &  & 3\\ 
%\fename{Evaluee} &  & 3 &  &  &  &  &  &  & 3\\ 
% \midrule
%\multicolumn{10}{l}{\textit{проклинам\slash прокълна} }\\  
%\fename{Communicator} & 4 &  &  &  &  &  &  &  & 4\\ 
%\fename{Evaluee} &  & 4 &  &  &  &  &  &  & 4\\ 
%\fename{Reason} &  &  & 1 &  &  &  &  &  & 1\\ 
% \midrule
\multicolumn{10}{l}{\textit{обвинявам\slash обвиня} }\\  
`blame'\\
\fename{Communicator} & 12 &  &  &  &  &  &  &  & 12\\ 
\fename{Evaluee} &  & 11 &  &  & 1 &  &  &  & 12\\ 
\fename{Reason} &  &  & 5 & 1 &  & 2 &  &  & 8\\ 

\midrule
%\multicolumn{10}{l}{\textit{възхвалявам\slash възхваля} }\\  
%\fename{Communicator} & 3 &  &  &  &  &  &  &  & 3\\ 
%\fename{Evaluee} &  & 3 &  &  &  &  &  &  & 3\\ 
%\fename{Reason} &  &  &  &  &  &  &  & 1 & 1\\ 
% \midrule
\multicolumn{10}{l}{\textit{подигравам се\slash подиграя се} }\\  
`mock, ridicule'\\
\fename{Communicator} & 15 &  &  &  &  &  &  &  & 15\\ 
\fename{Medium} &  &  & 1 &  &  &  &  &  & 1\\ 
\fename{Evaluee} &  &   & 13 &  & 2 &  &  &  & 15\\ 
\fename{Reason} &  &   & 2 &  & 1 &  &  &  & 3\\ 

\lspbottomrule
 \end{tabular}
 \caption{Syntactic expression of the \framename{Judgment\_communication} frame elements in Bulgarian.} 
    \label{tbl:judgment-synt-bg}
 \end{table}
 
\subsubsection{\framename{Judgment\_communication} valence patterns in Bulgarian}

 The valence patterns for the Bulgarian verbs in this frame are presented in \tabref{tbl:judgment-valence-bg}. Similarly to English, the most typical ones involve the expression of the \fename{Communicator} and the \fename{Evaulee} and possibly the \fename{Reason}; in the dataset there have not been cases of \fename{Topic}.
 
\begin{table}[t]
\small
\begin{tabularx}{\textwidth}{ lrQ }
\lsptoprule
Pattern  & \#  & verbs \\
\midrule
{[NP.Ext]}$_{\feinsub{Com}}$ {[NP.Obj]}$_{\feinsub{Eval}}$ \newline {[\_]}$_{\feinsub{Reas-DNI/INI}}$  & 57 & \textit{величая, виня, възхвалявам\slash възхваля, иронизирам, клеветя, критикувам, кълна, обвинявам\slash обвиня, омаловажавам\slash омаловажа, осъждам\slash осъдя, отричам\slash отрека, подценявам\slash подценя, порицавам\slash порицая, похвалвам\slash похваля, прославям\slash прославя, славя, хваля}\\

{[NP.Ext]}$_{\feinsub{Com}}$ {[PP]}$_{\feinsub{Eval}}$  {[\_]}$_{\feinsub{Reas-DNI/INI}}$& 28 & \textit{гавря се, заяждам се\slash заям се, подигравам се\slash подиграя се, присмивам се\slash присмея се}\\

{[NP.Ext]}$_{\feinsub{Com}}$ {[NP.Obj]}$_{\feinsub{Eval}}$   {[PP]}$_{\feinsub{Reas}}$ & 12 & \textit{заклеймявам\slash заклеймя, иронизирам, обвинявам\slash обвиня, подигравам\slash подиграя, порицавам\slash порицая, похвалвам\slash похваля, проклинам\slash прокълна}\\

%{[NP.Ext]}$_{\feinsub{Com}}$ {[\_]}$_{\feinsub{Eval-INI}}$ & 8 & \textit{похвалвам\slash похваля, подигравам\slash подиграя, обвинявам\slash обвиня, заяждам се\slash заям се, заклеймявам\slash заклеймя}\\

%{[NP.Ext]}$_{\feinsub{Com}}$ {[Clause]}$_{\feinsub{Reas}}$ {[NP.Obj]}$_{\feinsub{Eval}}$ & 3 & \textit{обвинявам\slash обвиня, хваля}\\

%{[NP.Ext]}$_{\feinsub{Com}}$ {[PP]}$_{\feinsub{Eval}}$ {[PP]}$_{\feinsub{Reas}}$ & 2 & \textit{подигравам се, присмивам се}\\

%{[NP.Ext]}$_{\feinsub{Com}}$ {[AVP]}$_{\feinsub{Reas}}$ {[NP.Obj]}$_{\feinsub{Eval}}$ & 2 & \textit{виня, обвинявам\slash обвиня}\\
\lspbottomrule
    \end{tabularx}
    \caption{FrameNet valence patterns of \framename{Judgment\_communication} verbs, their frequency in the Bulgarian dataset and the verbs they appear with.
     English translation equivalents: \textit{величая, възхвалявам\slash възхваля} `extol', \textit{виня, обвинявам\slash обвиня} `blame', \textit{гавря се} `deride', \textit{заклеймявам\slash заклеймя} `condemn', \textit{заяждам се\slash заям се} `criticise',  \textit{иронизирам} `ironise', \textit{клеветя} `denigrate', \textit{критикувам} `criticise', \textit{кълна} `damn', \textit{омаловажавам\slash омаловажа} `belittle', \textit{осъждам\slash осъдя} `judge', \textit{отричам\slash отрека} `denounce', \textit{подценявам\slash подценя} `disparage', \textit{подигравам се\slash подиграя се, присмивам се\slash присмея се} `mock, ridicule', \textit{порицавам\slash порицая} `castigate', \textit{похвалвам\slash похваля, хваля} `commend, praise' \textit{прославям\slash прославя, славя} `laud'.}
    \label{tbl:judgment-valence-bg}
\end{table} 


\newpage
\subsection{Frame \framename{Questioning}}
\begin{description}[font=\normalfont]
\item[Definition of the frame \framename{Questioning}:] A \fename{Speaker} asking an \fename{Addressee} a question, which represents the \fename{Message}, calling for a reply. Core frame elements: \fename{Speaker}, \fename{Message}, \fename{Addressee}, \fename{Topic}.
\end{description}

%This frame inherits \framename{Communication} through weak inheritance (via the relation Using). This is shown by the different configuration of the relations among the core frame elements. The main difference as compared with many of the related frames, including \framename{Communication}, is that -- as the definition shows -- the speech activity is directed not to the content of the proposition, i.e. the \fename{Message}, in this case -- a question requiring a reply, but to the \fename{Addressee}, who needs to provide the reply.

\subsubsection{Syntactic realisation of the \framename{Questioning} frame elements}

The semantic specification of the core frame elements is similar to those in the other related frames. As questioning is a purposeful action, the \fename{Speaker} is necessarily a person or an organisation. The \fename{Speaker} is the external argument projected as a subject NP. 

The central role of the \fename{Addressee} is reflected in the fact that it is a frame element that is typically expressed as the direct object NP (except for \textit{inquire}.v and some uses of \textit{ask}.v where it can be expressed as a prepositional complement headed by \textit{of}).

Except for a small number of occurrences with the same verbs, i.e. \textit{inquire}.v and \textit{ask}.v, where it takes the direct object position, \fename{Message} is typically expressed as a direct quote or an embedded question. 

The \fename{Topic} is either expressed as a prepositional complement or remains implied but non-overt syntactically. 

The verbs evoking the frame \framename{Questioning} are divided along two lines:\\ (i) whether they tend to express the \fename{Message} over the \fename{Topic} or vice versa; \\
(ii) whether they tend to leave the \fename{Addressee} unexpressed if it is understood from the context or not.
\largerpage
\begin{sloppypar}
With respect to the first criterion, the valence patterns for the verbs in the frame clearly show that the \fename{Message} and the \fename{Topic} rarely co-occur. %as opposed to what was shown for other frames. 
Out of the verbs listed in this frame, \textit{grill}.v, \textit{interrogate}.v, \textit{question}.v and \textit{quiz}.v strongly favour the \fename{Topic} (Example \ref{ex07question:a}, \ref{ex07question:b}), with much rarer occurrences of the \fename{Message}, usually in the form of a direct quotation (Example \ref{ex07question:c}); at least in the annotated corpus the two frame elements do not co-occur with these verbs. %This means that these verbs prefer to refer to the subject matter of the question rather than to its content. 
\end{sloppypar}
 
The remaining verbs: \textit{ask}.v, \textit{inquire}.v, \textit{query}.v tend to express the content of the question, i.e. the \fename{Message} rather than its subject matter, the \fename{Topic}, but \fename{Topics} do occur. Besides, the two frame elements can co-occur provided that the \fename{Message} is not realised by a clause, compare (Example \ref{ex07question:d} and Example \ref{ex07question:e}) or a quote. With the verb \textit{inquire}.v, the \fename{Message} may be realised not only as a clause or a quote but also (though rarely) as a prepositional complement (Example \ref{ex07question:f}). In addition, both \textit{inquire}.v and \textit{ask}.v allow the \fename{Message} to be expressed as an object NP (Example \ref{ex07question:d}, \ref{ex07question:h}). This pattern is typical of \textit{ask}.v and rare for \textit{inquire}.v. In such cases the \fename{Addressee} is expressed as an indirect (Example \ref{ex07question:h}) or a prepositional object (see Example \ref{ex07question:i}, which is a rephrase of Example \ref{ex07question:h}).

As regards the second distinction, the same verbs that favour \fename{Topics} over \fename{Messages} -- \textit{grill}.v, \textit{interrogate}.v, \textit{question}.v and \textit{quiz}.v -- show preference to expressing the \fename{Addressee} as an object NP, rather than leaving it implicit (Example \ref{ex07question:j}). As shown in \tabref{tbl:questioning-valence}, they tend to realise the \fename{Addressee} together with the \fename{Topic} (expressed as a PP headed by \textit{about}). When the \fename{Message} is expressed, the \fename{Addressee} is often left out.

%The second group is further divided -- on the one hand \textit{inquire}, \textit{query} prefer to leave the \fename{Addressee} implicit, while \textit{ask} shows both patterns with a prevalence of the DNI interpretation. If the \fename{Addressee} of \textit{ask} is non-overt, the \fename{Message} may be a quote, an embedded clause or more rarely -- an object NP (Example \ref{ex07question:k}).



\begin{exe} 
\ex \label{ex07question}
\begin{xlist}
\ex \label{ex07question:a}
 {[}\textit{Journalists}{]}$_{\feinsub{Spkr}}$ \textit{\textbf{GRILLED}} [\textit{Mr. Major}]$_{\feinsub{Addr}}$ [\textit{about \\Maastricht}]$_{\feinsub{Top}}$.
\ex  \label{ex07question:b}
{[}\textit{She}{]}$_{\feinsub{Spkr}}$ \textit{\textbf{QUESTIONED}} [\textit{him}]$_{\feinsub{Addr}}$ [\textit{about his aspirations}]$_{\feinsub{Top}}$.
\ex  \label{ex07question:c}
{[}\textit{I}{]}$_{\feinsub{Spkr}}$ \textit{\textbf{QUIZZED}} [\textit{him}]$_{\feinsub{Addr}}$: [\textit{``Who are you?''}]$_{\feinsub{Msg}}$.
\ex  \label{ex07question:d}
{[}\textit{You}{]}$_{\feinsub{Spkr}}$ \textit{\textbf{ASK}} [\textit{many questions}]$_{\feinsub{Msg}}$ [\textit{about her}]$_{\feinsub{Top}}$ \\ {[\_]}$_{\feinsub{Addr-DNI}}$.
\ex  \label{ex07question:e}
{[}\textit{The clerk}{]}$_{\feinsub{Spkr}}$ \textit{\textbf{INQUIRED}} [\textit{\_}]$_{\feinsub{Addr-DNI}}$ [\textit{if it would be \\cash}]$_{\feinsub{Msg}}$.
\ex  \label{ex07question:f}
{[}\textit{He}{]}$_{\feinsub{Spkr}}$ \textit{\textbf{INQUIRED}} [\_]$_{\feinsub{Addr-DNI}}$ [\textit{as to their where-\\abouts}]$_{\feinsub{Msg}}$.
\ex  \label{ex07question:g}
{[}\textit{I}{]}$_{\feinsub{Spkr}}$ \textit{did not \textbf{INQUIRE}} [\textit{the reason}]$_{\feinsub{Msg}}$ [\_]$_{\feinsub{Addr-DNI}}$.
\ex  \label{ex07question:h}
{[}\textit{They}{]}$_{\feinsub{Spkr}}$ \textit{\textbf{ASKED}} [\textit{the newcomer}]$_{\feinsub{Addr}}$ [\textit{his name}]$_{\feinsub{Msg}}$.
\ex  \label{ex07question:i}
{[}\textit{They}{]}$_{\feinsub{Spkr}}$ \textit{\textbf{ASKED}} [\textit{the name}]$_{\feinsub{Msg}}$ [\textit{of the newcomer}]$_{\feinsub{Addr}}$.
\ex  \label{ex07question:j}
{[}\textit{They}{]}$_{\feinsub{Spkr}}$ \textit{\textbf{QUESTIONED}} [\textit{the convict}]$_{\feinsub{Addr}}$ [\textit{about the \\money}]$_{\feinsub{Top}}$.
\ex  \label{ex07question:k}
{[}``\textit{Why not?''}{]}$_{\feinsub{Msg}}$ \textit{\textbf{QUERIED}} [\textit{she}]$_{\feinsub{Spkr}}$ {[\_]}$_{\feinsub{Addr-DNI}}$.
\end{xlist}
\end{exe}

\tabref{tbl:questioning-synt} shows some of the frequent verbs of the frame and the realisation of their frame elements.

\begin{table}
\centering\footnotesize
\begin{tabular}{l rrrrrrrrr}
\lsptoprule
 & NP.Ext & NP.Obj & PP & AVP & NI & Clause & Quote & Other & Total\\ 

\midrule
%\multicolumn{10}{l}{\textit{grill} } \\  
%\fename{Speaker} & 22  &  & 5  &  & 5  &  &  &  & 32\\ 
%\fename{Addressee} & 10  & 22  &  &  &  &  &  &  & 32\\ 
%\fename{Message} &  &  & 1  &  &  &  &  &  & 1\\ 
%\fename{Topic} &  &  & 14  &  & 17  &  &  &  & 31\\ 
% \midrule
\multicolumn{10}{l}{\textit{inquire} } \\  
\fename{Speaker} & 37  &  &  &  &  &  &  &  & 37\\ 
\fename{Addressee} &  &  & 5  &  & 32  &  &  &  & 37\\ 
\fename{Message} &  & 3  & 1  &  &  & 5  & 18  &  & 27\\ 
\fename{Topic} &  &  & 10  &  &  &  &  &  & 10\\ 

\midrule
%\multicolumn{10}{l}{\textit{interrogate} } \\  
%\fename{Speaker} & 12  &  & 7  &  & 5  &  &  &  & 24\\ 
%\fename{Addressee} & 12  & 12  &  &  &  &  &  &  & 24\\ 
%\fename{Topic} &  &  & 3  &  & 21  &  &  &  & 24\\ 
% \midrule
%\multicolumn{10}{l}{\textit{query} } \\  
%\fename{Speaker} & 23  &  &  &  &  &  &  &  & 23\\ 
%\fename{Addressee} &  & 1  &  &  & 22  &  &  &  & 23\\ 
%\fename{Message} &  &  &  &  &  & 1  & 21  &  & 22\\ 
%\fename{Topic} &  &  & 1  &  &  &  &  & 1 & 2\\ 
% \midrule
\multicolumn{10}{l}{\textit{question} } \\  
\fename{Speaker} & 34  &  & 4  &  & 9  &  &  &  & 47\\ 
\fename{Addressee} & 13  & 29  &  &  & 5  &  &  &  & 47\\ 
\fename{Message} &  &  &  &  &  &  & 5  &  & 5\\ 
\fename{Topic} &  &  & 25  &  & 17  &  &  &  & 42\\ 

\midrule
%\multicolumn{10}{l}{\textit{quiz} } \\  
%\fename{Speaker} & 22  &  & 4  &  & 2  &  &  &  & 28\\ 
%\fename{Addressee} & 6  & 21  &  &  & 1  &  &  &  & 28\\ 
%\fename{Message} &  &  &  &  &  &  & 2  &  & 2\\ 
%\fename{Topic} &  &  & 16  &  & 9  &  &  & 1 & 26\\ 
% \midrule
\multicolumn{10}{l}{\textit{ask} } \\  
\fename{Speaker} & 68  &  &  &  & 8  &  &  &  & 76\\ 
\fename{Addressee} & 7  & 27  &  &  & 35  &  &  &  & 69\\ 
\fename{Message} & 2  & 7  & 3  &  & 5  & 26  & 18  &  & 61\\ 
\fename{Topic} &  & 4  & 8  &  &  & 1  &  & 1 & 14\\ 

\lspbottomrule
 \end{tabular}
 \caption{Syntactic expression of the \framename{Questioning} frame elements in selected FrameNet lexical units. } 
    \label{tbl:questioning-synt}
 \end{table}


\subsubsection{\framename{Questioning} valence patterns}

The valence patterns (\tabref{tbl:questioning-valence}) show the tendency outlined above: the preference for expressing the \fename{Addressee} together with the \fename{Topic} or to leave it non-overt when the focus is on the \fename{Message} (i.e. it is syntactically expressed).

\begin{table}
    \centering\footnotesize
    \begin{tabularx}{\textwidth}{ lrQ }
\lsptoprule
         Pattern  & \#  & verbs \\
\midrule
{[NP.Ext]}$_{\feinsub{Spkr}}$ {[\_]}$_{\feinsub{Addr-DNI}}$ {[Quote]}$_{\feinsub{Msg}}$  & 55 & \textit{quiz, inquire, question, query, ask}\\
{[NP.Ext]}$_{\feinsub{Spkr}}$ {[NP.Obj]}$_{\feinsub{Addr}}$ {[PP]}$_{\feinsub{Top}}$  & 48 & \textit{quiz, interrogate, question, ask, grill}\\
{[NP.Ext]}$_{\feinsub{Spkr}}$ {[NP.Obj]}$_{\feinsub{Addr}}$ {[\_]}$_{\feinsub{Top-DNI/INI}}$  & 38 & \textit{quiz, grill, interrogate, question}\\
{[NP.Ext]}$_{\feinsub{Spkr}}$ {[NP.Obj]}$_{\feinsub{Addr}}$ {[Clause]}$_{\feinsub{Msg}}$  & 13 & \textit{ask}\\
{[NP.Ext]}$_{\feinsub{Spkr}}$ {[\_]}$_{\feinsub{Addr-DNI}}$ {[PP]}$_{\feinsub{Top}}$  & 12 & \textit{inquire, ask}\\
{[NP.Ext]}$_{\feinsub{Spkr}}$ {[\_]}$_{\feinsub{Addr-DNI}}$ {[Clause]}$_{\feinsub{Msg}}$  & 10 & \textit{inquire, query, ask}\\
%{[NP.Ext]}$_{\feinsub{Addr}}$ {[\_]}$_{\feinsub{Spkr-CNI}}$ {[\_]}$_{\feinsub{Top-DNI}}$  & 9 & \textit{interrogate, question}\\
%{[NP.Ext]}$_{\feinsub{Addr}}$ {[PP]}$_{\feinsub{Spkr}}$ {[\_]}$_{\feinsub{Top-INI}}$  & 5 & \textit{quiz, grill}\\
\lspbottomrule
    \end{tabularx}
    \caption{FrameNet valence patterns of \framename{Questioning} verbs, their frequency in the FrameNet corpus and the verbs they appear with.}
    \label{tbl:questioning-valence}
\end{table} 

\subsubsection{Syntactic realisation of \framename{Questioning} in Bulgarian}


Most of the Bulgarian counterparts are derived from the basic \framename{Questioning} verb \textit{питам} `ask' -- \textit{попитвам}, \textit{запитвам} `ask', \textit{разпитвам} `ask, question, grill', \textit{пре-\newline питвам} `quiz, query'. Typically, either the \fename{Message} or the \fename{Topic} is expressed (Example \ref{ex:07questionbg:a}, \ref{ex:07questionbg:b}). The two may co-occur only if the \fename{Message} is nominalised, usually by means of any of a small inventory of pronouns such as \textit{нещо} `something, anything', \textit{нищo} `nothing', \textit{това} `this, that' or some other expressions %or another pronoun nominalising a proposition and some other words 
(Example \ref{ex:07questionbg:c}). If the \fename{Message} is expressed otherwise, most often as a quote or an embedded clause, the two frame elements typically do not co-occur. The \fename{Topic} is expressed as a PP headed by the prepositions \textit{за} or \textit{относно} `about, regarding', while the \fename{Addressee} occupies the direct object position -- NP.Obj (Examples \ref{ex:07questionbg:a}, \ref{ex:07questionbg:b}, \ref{ex:07questionbg:c}).


The predominant valence patterns in Bulgarian are similar (\tabref{tbl:questioning-valence-bg}), although the data show that the \fename{Addressee} co-occurs more frequently with \fename{Message} (Example \ref{ex:07questionbg:d}) than in English.

\begin{exe}
\ex \label{ex:07questionbg}
\begin{xlist}
\ex \label{ex:07questionbg:a}
\gll {[}-- \textit{Какво} \textit{мислиш}?{]}$_{\feinsub{Msg}}$  -- \textit{\textbf{ПОПИТА}} [\textit{я}]$_{\feinsub{Addr}}$  [\textit{той}]$_{\feinsub{Spkr}}$.\\
-- What think.2sg? -- asked her he.
 \\
 \glt `-- What do you think? -- he asked her.'
\ex \label{ex:07questionbg:b}
 \gll  {[}\textit{Тя}{]}$_{\feinsub{Spkr}}$  [\textit{го}]$_{\feinsub{Addr}}$  \textit{\textbf{РАЗПИТВА}} [\_]$_{\feinsub{Msg}}$   [\textit{за} \textit{личния} \textit{му} \textit{живот}]$_{\feinsub{Top}}$.\\
  She him inquires {} about personal-\textsc{def}  his life.
  \\
 \glt `She inquires him about his personal life.'
\ex \label{ex:07questionbg:c}
\gll {[}\_{]}$_{\feinsub{Spkr}}$  \textit{Ще} [\textit{те}]$_{\feinsub{Addr}}$  \textit{\textbf{ПОПИТАМ}} [\textit{нещо}]$_{\feinsub{Msg}}$   [\textit{за} \textit{Арон}]$_{\feinsub{Top}}$.\\
{}  Will you ask.1sg something about Aaron.
 \\
 \glt `I will ask you something about Aaron.'
\ex \label{ex:07questionbg:d}
\gll {[}\textit{Престъпникът}{]}$_{\feinsub{Spkr}}$  \textit{\textbf{ПОПИТАЛ}} [\textit{полицая}]$_{\feinsub{Addr}}$  [\textit{дали} \textit{може} \textit{да} \textit{си} \textit{купи} \textit{цигари}]$_{\feinsub{Msg}}$.
 \\
 Criminal-\textsc{def} asked policeman-\textsc{def} whether can to  himself buy cigarettes.
 \\
 \glt `The criminal asked the policeman whether he could buy cigarettes.'
\end{xlist}
\end{exe}

The most frequent verbs and the syntactic realisation of the  frame elements of \framename{Questioning} is shown in \tabref{tbl:questioning-synt-bg}.

\begin{table}
\centering\footnotesize
\begin{tabular}{l rrrrrrrrr}
\lsptoprule
 & NP.Ext & NP.Obj & PP & AVP & NI & Clause & Quote & Other & Total\\ 

\midrule
\multicolumn{10}{l}{\textit{попитвам\slash попитам} }\\  
`ask'\\
\fename{Message} &  & 1 &  &  & 1 &   20& & 7 & 29\\ 
\fename{Addressee} &  & 34 &  &  &  &  &  &  & 34\\ 
\fename{Topic} &  &  & 2 &  &  &  &  &  & 2\\ 
\fename{Speaker} & 57 &  &  &  &  &  &  &  & 57\\ 

\midrule
\multicolumn{10}{l}{\textit{запитвам\slash запитам} }\\  
`ask, question' \\
\fename{Message} &  &  &  &  &  &   6& & 14 & 20\\ 
\fename{Addressee} &  & 16 &  &  &  &  &  &  & 16\\ 
\fename{Topic} &  &  & 3 &  &  &  &  &  & 3\\ 
\fename{Speaker} & 23 &  &  &  &  &  &  &  & 23\\ 

\midrule
\multicolumn{10}{l}{\textit{питам} }\\
`ask'\\
\fename{Message} & 1 & 1 &  &  & 2 &   6& & 15 & 25\\ 
\fename{Addressee} &  & 16 &  &  & 3 &  &  &  & 19\\ 
\fename{Topic} &  &  & 5 &  &  &  &  &  & 5\\ 
\fename{Speaker} & 29 &  &  &  & 1 &  &  &  & 30\\ 

\lspbottomrule
%\multicolumn{10}{l}{\textit{разпитвам\slash разпитам} }\\  
%\fename{Message} &  &  &  &  & 1 &  &  &  & 1\\ 
%\fename{Addressee} &  & 14 &  &  & 2 &  &  &  & 16\\ 
%\fename{Topic} &  &  & 4 &  &  &  &  &  & 4\\ 
%\fename{Speaker} & 17 &  &  &  &  &  &  &  & 17\\ 
% \midrule
 \end{tabular}
 \caption{Syntactic expression of the \framename{Questioning} frame elements in Bulgarian. } 
    \label{tbl:questioning-synt-bg}
 \end{table}

\begin{table}
    \centering\footnotesize
    \begin{tabularx}{\textwidth}{ lrQ }
\lsptoprule
         Pattern  & \#  & verbs \\
\midrule
{[NP.Ext]}$_{\feinsub{Spkr}}$ {[NP.Obj]}$_{\feinsub{Addr}}$ {[Sinterrog]}$_{\feinsub{Msg}}$ & 25 & \textit{питам, запитвам\slash запитам, попитвам\slash попитам}\\

{[NP.Ext]}$_{\feinsub{Spkr}}$ {[NP.Obj]}$_{\feinsub{Addr}}$ {[\_]}$_{\feinsub{Msg-INI}}$ & 25 & \textit{разпитвам\slash разпитам}\\

{[NP.Ext]}$_{\feinsub{Spkr}}$ {[Quote]}$_{\feinsub{Msg}}$ {[\_]}$_{\feinsub{Addr-INI}}$ & 20 & \textit{питам, запитвам\slash запитам, попитвам\slash попитам, поинтересувам се}\\

{[NP.Ext]}$_{\feinsub{Spkr}}$ {[NP.Obj]}$_{\feinsub{Addr}}$ {[Quote]}$_{\feinsub{Msg}}$ & 14 & \textit{питам, запитвам\slash запитам, попитвам\slash попитам}\\

{[NP.Ext]}$_{\feinsub{Spkr}}$ {[Sinterrog]}$_{\feinsub{Msg}}$ {[\_]}$_{\feinsub{Addr}}$ & 13 & \textit{интересувам се, питам, запитвам\slash запитам, попитвам\slash попитам, полюбопитствам}\\

%{[NP.Ext]}$_{\feinsub{Spkr}}$ {[]}$_{\feinsub{Msg}}$ & 11 & \textit{попитвам\slash попитам, разпитвам\slash разпитам}\\

{[NP.Ext]}$_{\feinsub{Spkr}}$ {[NP.Obj]}$_{\feinsub{Addr}}$ {[PP]}$_{\feinsub{Top}}$ & 9 & \textit{питам, запитвам\slash запитам, попитвам\slash попитам, разпитвам\slash разпитам}\\

%{[NP.Ext]}$_{\feinsub{Spkr}}$ {[PP]}$_{\feinsub{Top}}$ {[\_]}$_{\feinsub{Addr-INI}}$ & 6 & \textit{питам, поинтересувам се, разпитвам\slash разпитам}\\

%{[NP.Ext]}$_{\feinsub{Spkr}}$ {[NP.Obj]}$_{\feinsub{Addr}}$ {[\_]}$_{\feinsub{Msg-INI}}$ & 3 & \textit{попитвам\slash попитам, разпитвам\slash разпитам, питам}\\

%{[NP.Ext]}$_{\feinsub{Spkr}}$ {[NP.Obj]}$_{\feinsub{Addr}}$ & 3 & \textit{разпитвам\slash разпитам, питам}\\
\lspbottomrule
    \end{tabularx}
    \caption{FrameNet valence patterns of the frame \framename{Questioning}, their frequency in the Bulgarian dataset and the verbs they appear with.
    English translation equivalents: \textit{питам, запитвам\slash запитам, попитвам\slash попитам} `ask, question', \textit{интересувам се\slash поинтересувам се} `inquire', \textit{разпитвам\slash разпитам} `question, grill, interrogate'.}
    \label{tbl:questioning-valence-bg}
\end{table} 


\subsection{Frame \framename{Communication\_response}}  

\begin{description}[font=\normalfont]
\item[Definition of the frame \framename{Communication\_response}:] A \fename{Speaker} communicates a reply or response, a \fename{Message}, to some prior communication or action, the \fename{Trigger}. Core frame elements: \fename{Speaker}, \fename{Message}, \fename{Trigger}, \fename{Addressee}, \fename{Topic}.
\end{description}

The \framename{Communication\_response} frame inherits from the frame \framename{Communication}. It elaborates on the prototypical frame by introducing a new frame element, the \fename{Trigger}, which requires a response, expressed as the \fename{Message}. %is in fact a response to the \fename{Trigger}.

\subsubsection{Syntactic realisation of \framename{Communication\_response} frame elements}


The \fename{Speaker} inherits the frame element \fename{Communicator} which exhibits the same characteristics and behaviour as in the other frames in the domain, and is realised most often as an external NP. 

The \fename{Trigger} is the prior communication or action to which a response is given. It can be implicit, or overtly expressed either as an NP object or as a prepositional complement (Examples \ref{ex:08response:a}, \ref{ex:08response:b}).

The \fename{Message} is not necessarily expressed when the \fename{Trigger} is present (Examples \ref{ex:08response:a}, \ref{ex:08response:b}). When the \fename{Message} is realised, it predominantly takes the form of an embedded clause (Example \ref{ex:08response:c}) or a direct quote (Example \ref{ex:08response:d}).

Although rarely, the \fename{Trigger} and the \fename{Message} may co-occur (Example \ref{ex:08response:g}).

The \fename{Addressee} is the person to whom the response is directed. When expressed, it occurs as a prepositional phrase introduced by the preposition `to’ (Example \ref{ex:08response:e}) or as an indirect object (Example \ref{ex:08response:f}).

The \fename{Topic} is possible but rare with verbs from this frame.

\begin{exe}
\ex \label{ex:08response}
\begin{xlist}
\ex \label{ex:08response:a}
{[}\textit{Sue}{]}$_{\feinsub{Com}}$  \textit{\textbf{ANSWERED}} [\textit{the question}]$_{\feinsub{Trig}}$.
\ex \label{ex:08response:b}
{[}\textit{The US}{]}$_{\feinsub{Com}}$  \textit{has not \textbf{RESPONDED}} [\textit{to the offer}]$_{\feinsub{Trig}}$.
\ex \label{ex:08response:c}
{[}\textit{Blanche}{]}$_{\feinsub{Com}}$  \textit{\textbf{RESPONDED}} [\textit{that the police were talking to \newline everyone}]$_{\feinsub{Msg}}$.
\ex \label{ex:08response:d}
{[}`\textit{Does it matter}?']$_{\feinsub{Msg}}$  [\textit{she}]$_{\feinsub{Com}}$  \textit{\textbf{COUNTERED} defeatedly}.
\ex \label{ex:08response:e}
{[}\textit{Sue}{]}$_{\feinsub{Com}}$  \textit{\textbf{RESPONDED}} [\textit{to Bob}]$_{\feinsub{Addr}}$  \textit{immediately}. 
\ex \label{ex:08response:f}
{[}\textit{The senator}{]}$_{\feinsub{Com}}$  \textit{took the floor to \textbf{ANSWER}} [\textit{critics of the \\deal}]$_{\feinsub{Addr}}$. 
\ex \label{ex:08response:g}
{[}`\textit{Does it matter}?']$_{\feinsub{Msg}}$ \textit{\textbf{REPLIED}} [\textit{she}]$_{\feinsub{Com}}$   [\textit{to his question}]$_{\feinsub{Trig}}$. 
\end{xlist}
\end{exe}

\tabref{tbl:response-synt} shows the syntactic realisations of verbs evoking the frame \framename{Communication\_response}.

\begin{table}
\centering\footnotesize
\begin{tabular}{l rrrrrrrrr}
\lsptoprule
 & NP.Ext & NP.Obj & PP & AVP & NI & Clause & Quote & Other & Total\\ 

\midrule
\multicolumn{10}{l}{\textit{answer} } \\  
\fename{Speaker} & 31  &  & 1  &  & 2  &  &  &  & 34\\ 
\fename{Addressee} &  & 1  &  &  & 30  &  &  & 3 & 34\\ 
\fename{Message} &  & 2  & 4  &  & 21  & 2  & 5  &  & 34\\ 
\fename{Trigger} & 3  & 11  &  &  & 18  & 1  &  &  & 33\\ 

\midrule
\multicolumn{10}{l}{\textit{reply} } \\  
\fename{Speaker} & 69  &  &  &  &  &  &  &  & 69\\ 
\fename{Addressee} &  &  & 4  &  & 65  &  &  &  & 69\\ 
\fename{Message} &  &  & 4  &  & 21  & 11  & 27  & 6 & 69\\ 
\fename{Trigger} &  &  & 13  &  & 56  &  &  &  & 69\\ 

\midrule
\multicolumn{10}{l}{\textit{respond} } \\  
\fename{Speaker} & 25  &  &  &  &  &  &  &  & 25\\ 
\fename{Addressee} &  &  & 1  &  & 23  &  &  &  & 24\\ 
\fename{Message} &  &  & 3  & 1  & 4  & 4  & 13  &  & 25\\ 
\fename{Trigger} &  &  & 1  &  & 24  &  &  &  & 25\\ 

\midrule
\multicolumn{10}{l}{\textit{retort} } \\  
\fename{Speaker} & 46  &  &  &  &  &  &  &  & 46\\ 
\fename{Addressee} &  &  & 1  &  & 45  &  &  &  & 46\\ 
\fename{Message} &  & 2  & 1  &  & 2  & 21  & 23  &  & 49\\ 
\fename{Trigger} &  &  & 1  &  & 45  &  &  &  & 46\\ 

\lspbottomrule
 \end{tabular}
 \caption{Syntactic expression of the \framename{Communication\_response} frame elements for selected FrameNet lexical units. } 
    \label{tbl:response-synt}
 \end{table}

\subsubsection{\framename{Communication\_response} valence patterns}

\tabref{tbl:response-valence} illustrates the valence patterns that characterise the verbs in the frame \framename{Communication\_response}. The most frequent pattern has the \fename{Message} realised as a direct quote, followed by the pattern with an embedded clause or a PP. The \fename{Trigger} is expressed in fewer instances and in such cases the \fename{Addressee} and the \fename{Message} remain non-overt.


\begin{table}
    \centering\footnotesize
    \begin{tabularx}{\textwidth}{ lrQ }
\lsptoprule
         Pattern  & \#  & verbs \\
\midrule
{[NP.Ext]}$_{\feinsub{Spkr}}$ {[\_]}$_{\feinsub{Addr-DNI}}$ {[Quote]}$_{\feinsub{Msg}}$  {[\_]}$_{\feinsub{Trig-DNI}}$  & 83 & \textit{answer, rejoin, counter, reply, respond, retort}\\

{[NP.Ext]}$_{\feinsub{Spkr}}$ {[\_]}$_{\feinsub{Addr-DNI}}$ {[Clause]}$_{\feinsub{Msg}}$   {[\_]}$_{\feinsub{Trig-DNI}}$  & 34 & \textit{answer, rejoin, counter, reply, respond, retort}\\

%{[NP.Ext]}$_{\feinsub{Spkr}}$ {[\_]}$_{\feinsub{Addr-DNI}}$ {[\_]}$_{\feinsub{Msg-INI}}$ {[\_]}$_{\feinsub{Trigger-DNI}}$  & 19 & \textit{answer, reply, respond, retort}\\

{[NP.Ext]}$_{\feinsub{Spkr}}$ {[\_]}$_{\feinsub{Addr-DNI}}$ {[PP]}$_{\feinsub{Msg}}$ {[\_]}$_{\feinsub{Trig-DNI}}$  & 14 & \textit{answer, counter, reply, respond}\\

{[NP.Ext]}$_{\feinsub{Spkr}}$ {[\_]}$_{\feinsub{Addr-DNI}}$ {[\_]}$_{\feinsub{Msg-INI}}$ {[PP]}$_{\feinsub{Trig}}$  & 10 & \textit{reply}\\

%{[NP.Ext]}$_{\feinsub{Spkr}}$ {[\_]}$_{\feinsub{Addr-DNI}}$ {[2nd]}$_{\feinsub{Msg}}$ {[\_]}$_{\feinsub{Trigger-DNI}}$  & 10 & \textit{counter, reply}\\

{[NP.Ext]}$_{\feinsub{Spkr}}$ {[\_]}$_{\feinsub{Addr-DNI}}$ {[\_]}$_{\feinsub{Msg-INI}}$ {[NP.Obj]}$_{\feinsub{Trig}}$  & 7 & \textit{answer}\\
\lspbottomrule
    \end{tabularx}
    \caption{FrameNet valence patterns of \framename{Communication\_response} verbs, their frequency in the FrameNet corpus and the verbs they appear with.}
    \label{tbl:response-valence}
\end{table} 

\subsubsection{Syntactic realisation of \framename{Communication\_response} frame in Bulgarian}

In Bulgarian the syntactic realisation of the frame is similar to English. The \fename{Message} most often appears as an embedded clause (Example \ref{ex:08responsebg:a}) or  as a direct quote (Example \ref{ex:08responsebg:b}), and in some cases as a direct object (Example \ref{ex:08responsebg:d}) or a prepositional phrase (Example \ref{ex:08responsebg:c}). The \fename{Trigger} is realised as a prepositional phrase (Example \ref{ex:08responsebg:e}). %The \fename{Trigger} can co-occur with the \fename{Message} (Example \ref{ex:08responsebg:c}).

\begin{exe}
\ex \label{ex:08responsebg}
\begin{xlist} 
\ex \label{ex:08responsebg:a}
\gll {[}\textit{Той}{]}$_\fename{Spkr}$ [\textit{ми}]$_\fename{Addr}$ \textit{\textbf{ОТГОВОРИ}}, [\textit{че} \textit{няма} \textit{да} \textit{отиде}]$_\fename{Msg}$. 
\\
He me answered that not to go. 
\\
\glt `He answered me that he won't go.'
\ex \label{ex:08responsebg:b}
\gll {[}\textit{Студентът}{]}$_\fename{Spkr}$ \textit{\textbf{ОТГОВОРИЛ}}: [-- \textit{Професоре}, \textit{забравих}!]$_\fename{Msg}$
\\
Student-\textsc{def} responded: -- Professor, forgot.1sg!
\\
\glt `The student responded: -- Professor, I forgot!'
\ex \label{ex:08responsebg:c}
\gll {[}\textit{Мнозинството}{]}$_\fename{Spkr}$ \textit{\textbf{ОТВРЪЩА}} [\textit{с} \textit{надменни} \textit{приказки}]$_\fename{Msg}$ [\textit{за} \textit{своята} \textit{безалтернативност}]$_\fename{Top}$.
\\
Majority-\textsc{def} answers with arrogant words about their-REFL {lack of prospects}].
\\
\glt `The majority answers with arrogant words about their lack of prospects.'
\ex \label{ex:08responsebg:d}
\gll {[}\textit{Той}{]}$_\fename{Spkr}$ \textit{не} \textit{\textbf{ОТГОВОРИ}} [\textit{нищо}]$_\fename{Msg}$. 
\\
He not responded nothing. 
\\
\glt `He did not respond anything.'
\ex \label{ex:08responsebg:e}
\gll {[}\textit{На} \textit{този} \textit{въпрос}{]}$_\fename{Trig}$ \textit{ще} \textit{\textbf{ОТГОВОРИ}} [\textit{министър}-\textit{председателят}]$_\fename{Spkr}$. 
\\
To this question will answer {prime minister}-\textsc{def}.
\\
\glt `The prime minister will answer this question.'
\end{xlist}
\end{exe}

The most frequent verbs evoking the frame \framename{Communication\_response} and the realisation of their frame elements are shown in \tabref{tbl:response-synt-bg}. The associated valence patterns are presented in \tabref{tbl:response-valence-bg}.

\begin{table}
\centering\footnotesize
\begin{tabular}{l rrrrrrrrr}
\lsptoprule
 & NP.Ext & NP.Obj & PP & AVP & NI & Clause & Quote & Other & Total\\ 

\midrule
\multicolumn{10}{l}{\textit{отвръщам\slash отвърна} }\\  
`reply'\\
\fename{Trigger} &  &  & 2 &  & 4 &  &  &  & 6\\ 
\fename{Message} &  & 2 &  &  & 5 &  & 23 & 5 & 35\\ 
\fename{Addressee} &  &  & 14 &  & 21 &  &  &  & 35\\ 
\fename{Manner} &  &  & 2 &  &  &  &  & 3 & 5\\ 
\fename{Speaker} & 35 &  &  &  &  &  &  &  & 35\\ 

\midrule
\multicolumn{10}{l}{\textit{отговарям\slash отговоря} }\\  
`answer, reply'\\
\fename{Trigger} &  &  & 15 &  & 15 &  &  &  & 30\\ 
\fename{Message} &  & 3 &  &  & 26 &  & 21 & 16 & 66\\ 
\fename{Addressee} &  &  & 23 &  & 43 &  &  &  & 66\\ 
\fename{Medium} &  &  & 7 &  &  &  &  &  & 7\\ 
\fename{Manner} &  &  & 2 &  &  &  &  & 12 & 14\\ 
\fename{Speaker} & 66 &  &  &  &  &  &  &  & 66\\ 

\lspbottomrule
 \end{tabular}
 \caption{Syntactic expression of the \framename{Communication\_response} frame elements in Bulgarian. } 
    \label{tbl:response-synt-bg}
 \end{table}

\begin{table}
    \centering\footnotesize
    \begin{tabularx}{\textwidth}{ lrQ }
\lsptoprule
         Pattern  & \#  & verbs \\
\midrule
{[NP.Ext]}$_{\feinsub{Spkr}}$ {[Quote]}$_{\feinsub{Msg}}$ {[\_]}$_{\feinsub{Addr-DNI}}$ {[\_]}$_{\feinsub{Trig-DNI}}$ & 22 & \textit{контрирам, отвръщам\slash отвърна, отговарям\slash отговоря}\\

{[NP.Ext]}$_{\feinsub{Spkr}}$ {[PP]}$_{\feinsub{Addr}}$ {[Quote]}$_{\feinsub{Msg}}$ {[\_]}$_{\feinsub{Trig-DNI}}$ & 12 & \textit{отвръщам\slash отвърна, отговарям\slash отговоря}\\

{[NP.Ext]}$_{\feinsub{Spkr}}$ {[Clause]}$_{\feinsub{Msg}}$ {[\_]}$_{\feinsub{Addr-DNI}}$ {[\_]}$_{\feinsub{Trig-DNI}}$ & 10 & \textit{отвръщам\slash отвърна, отговарям\slash отговоря}\\

{[NP.Ext]}$_{\feinsub{Spkr}}$ {[Clause]}$_{\feinsub{Msg}}$ {[PP]}$_{\feinsub{Addr}}$ {[\_]}$_{\feinsub{Trig-DNI}}$ & 9 & \textit{отвръщам\slash отвърна, отговарям\slash отговоря}\\

{[NP.Ext]}$_{\feinsub{Spkr}}$ {[PP]}$_{\feinsub{Trig}}$ {[\_]}$_{\feinsub{Addr-DNI}}$   {[\_]}$_{\feinsub{Msg-INI}}$ & 8 & \textit{отвръщам\slash отвърна, отговарям\slash отговоря}\\

%{[NP.Ext]}$_{\feinsub{Spkr}}$ {[AdvP]}$_{\feinsub{MANNER}}$ {[Quote]}$_{\feinsub{Msg}}$ {[\_]}$_{\feinsub{Addr-DNI}}$ {[\_]}$_{\feinsub{Trigger-DNI}}$ & 5 & \textit{отговарям\slash отговоря, отвръщам\slash отвърна}\\

%{[NP.Ext]}$_{\feinsub{Spkr}}$ {[PP]}$_{\feinsub{Addr}}$ {[\_]}$_{\feinsub{Msg-INI}}$ & 5 & \textit{отвръщам\slash отвърна, отговарям\slash отговоря}\\

%{[NP.Ext]}$_{\feinsub{Spkr}}$ {[NP.Obj]}$_{\feinsub{Msg}}$ {[\_]}$_{\feinsub{Addr-DNI}}$ & 4 & \textit{отговарям\slash отговоря, отвръщам\slash отвърна}\\

%{[NP.Ext]}$_{\feinsub{Spkr}}$ {[AdvP]}$_{\feinsub{MANNER}}$ {[PP]}$_{\feinsub{Trigger}}$ {[\_]}$_{\feinsub{Addr-DNI}}$ {[\_]}$_{\feinsub{Msg-INI}}$ & 3 & \textit{отговарям\slash отговоря}\\

%{[NP.Ext]}$_{\feinsub{Spkr}}$ {[PP]}$_{\feinsub{MEDIUM}}$ {[\_]}$_{\feinsub{Addr-DNI}}$ {[\_]}$_{\feinsub{Msg-INI}}$ {[\_]}$_{\feinsub{Trigger-DNI}}$ & 2 & \textit{отговарям\slash отговоря}\\

%{[NP.Ext]}$_{\feinsub{Spkr}}$ {[AdvP]}$_{\feinsub{MANNER}}$ {[PP]}$_{\feinsub{Addr}}$ {[Quote]}$_{\feinsub{Msg}}$ {[\_]}$_{\feinsub{Trigger-DNI}}$ & 2 & \textit{отговарям\slash отговоря}\\
\lspbottomrule
\end{tabularx}
\caption{FrameNet valence patterns of the frame \framename{Communication\_response}, their frequency in the Bulgarian dataset and the verbs they appear with.
English translation equivalents: \textit{контрирам} `counter', \textit{отвръщам\slash отвърна, отговарям\slash отговоря} `answer, reply, counter, retort'.}
\label{tbl:response-valence-bg}
\end{table} 




\section{Conclusions}\label{sec:conclusion}

In this paper we have discussed thе universal features of the conceptual description of verbs which is transferable across languages. We illustrate our analysis with examples from the class of verbs of communication with a view to their use in English and Bulgarian.


The universality of the semantic relations of inheritance (from a more generalised to a more specialised entity) underlies the hierarchical organisation of both the FrameNet frames and the WordNet synsets. The configuration of frame elements describing the behaviour of verbs evoking a particular frame are also language-independent, as well as the semantic restrictions determining their selection. Moreover, we have shown that the principles of syntactic realisation of the frame elements as represented by the generalised valence patterns are also valid to a large degree across different languages. For Bulgarian and English we have established substantial correspondence in both the valence patterns and the syntactic categories and grammatical functions by which frame elements are expressed.

Further, we have outlined some basic language-specific properties of the syntactic realisation of semantic frames and their corresponding frame elements. 
In some cases the two languages give different preference to the overt expression of particular frame elements. For example, the \fename{Topic} is more frequent in English and rarely expressed with Bulgarian communication verbs (e.g., evoking the frames \framename{Statement} and \framename{Communication\_manner}). %and in Bulgarian it most often occurs as a modifier to \fename{Message}. 
We also observe differences in the syntactic realisation of particular frame elements due to the distinct syntactic properties of the two languages. For example, Bulgarian lacks infinitives and \textit{-ing} clauses, so clausal complements expressing the frame element \fename{Message} are finite clauses. Differences at the syntactic level between Bulgarian and English are also found between verbs considered as translation equivalents (belonging to corresponding synsets in Bulgarian and English). For example, with the verb \textit{ridicule} (evoking the frame \framename{Judgment\_communication}) the \fename{Evaluee} is expressed predominantly as a direct object, while the Bulgarian verb \textit{подигравам се} realises it as an indirect object due to the fact that reflexive verbs do not take a direct object.

The analysis confirms the assumption that a large part of a verb’s semantic valency and syntactic behaviour is predictable from its lexical meaning and the semantic class it belongs to. The various semantic classifications of verbs focus on different semantic and/or syntactic properties, relying mostly on theoretical analysis or expert intuition rather than on authentic corpus data. A study based on corpus analysis and statistical observations on the frequency of valence patterns could provide more reliable evidence for the behaviour of verbs, in particular in view of cross-linguistic studies. Moreover, this will confirm the validity of the cross-linguistic analysis and the universality of semantic and syntactic features.

%The cross-linguistic analysis of conceptual structure, as well as the developed resources, could be applied to boost annotation with FrameNet frames (e.g., FATE -- FrameNet-Annotated Textual Entailment, \cite{Burchardt2008}).

In our work on describing the conceptual and syntactic properties of Bulgarian verbs, we have found the applicability of the conceptual description encoded in the FrameNet frames to be to a great extent language-independent and transferrable cross-linguistically, even if in some cases adjustments may be necessary. Given the fact that the alignment between equivalent senses in the wordnets developed for different languages is ensured by means of shared identification numbers with the original Princeton WordNet, the conceptual information from FrameNet is mappable across languages via WordNet.\footnote{For a list of existing wordnets in the world, see \url{http://globalwordnet.org/resources/wordnets-in-the-world/}.}

\section*{Abbreviations}

\begin{tabularx}{.5\textwidth}{lQ}
\scshape Addr & \fename{Addressee}\\
\scshape Auth & \fename{Author}\\
BulEnAC & Bulgarian-English \\ & Sentence- and Clause- \\ & Aligned Corpus\\
BulSemCor & Semantically annotated \\ & corpus for Bulgarian\\
CNI & Constructional null \\ & instantiation\\
\scshape Com & \fename{Communicator}\\
\scshape Cont & \fename{Content}\\
DNI & Definite null \\ & instantiation\\
\scshape Eval & \fename{Evaluee}\\
\scshape Exr & \fename{Expressor}\\
\end{tabularx}%
\begin{tabularx}{.49\textwidth}{lQ}
INI & Indefinite null instantiation\\
\scshape Manr & \fename{Manner}\\
\scshape Med & \fename{Medium}\\
\scshape Msg & \fename{Message}\\
N or n & Noun\\
NP & Noun phrase\\
PP & Prepositional phrase\\
PWN & Princeton WordNet\\
\scshape Reas & \fename{Reason}\\
SemCor & Semantically annotated \\ & corpus for English\\
\scshape Spkr & \fename{Speaker}\\
\scshape Top & \fename{Topic}\\
\scshape Trig & \fename{Trigger}\\
V or v & Verb
\end{tabularx}


\section*{Acknowledgements}

This research is carried out as part of the project \emph{Enriching Semantic Network WordNet with Conceptual Frames} funded by the Bulgarian National Science Fund, Grant Agreement No. KP-06-H50/1 from 2020.


{\sloppy\printbibliography[heading=subbibliography,notkeyword=this]}
\end{document}
