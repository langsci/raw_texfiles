\documentclass[output=paper,colorlinks,citecolor=brown]{langscibook}
\ChapterDOI{10.5281/zenodo.15682190}

\author{Svetlozara Leseva\orcid{0000-0001-8198-4555}\affiliation{Department of Computational Linguistics, Institute for Bulgarian Language, Bulgarian Academy of Sciences}}

\title[The conceptualisation of the route: Non-directed and directed motion]{The conceptualisation of the route: Non-directed and directed motion verbs in Bulgarian and English} 

\abstract{This chapter offers an analysis of non-directed and directed motion verbs from a frame semantics perspective through exploring the semantic description and syntactic realisation of the frame elements of several semantic frames in FrameNet. The study is focused on the conceptualisation and syntactic expression of the elements of the route along which motion occurs: \fename{Goal} (the final part of the route), \fename{Source} (the initial part of the route) and \fename{Path} (the middle part of the route) in English and Bulgarian by studying the syntactic expression of the corresponding frame elements in FrameNet. The research questions explored in the chapter deal with the prominent aspects in the semantics of the verbs evoking a particular semantic frame, the syntactic expression of the relevant frame elements, syntactic explicitness and implicitness. The empirical evidence provided by the FrameNet corpus is compared with a sample of annotated Bulgarian examples. The observations made throughout the chapter are brought in the perspective of linguistic hypotheses put forward in the literature: in particular, the goal-over-source hypothesis and the proposal that motion verbs tend to co-occur with expressions that align with the part of the trajectory of motion that is most prominent in their semantics.  
}

\IfFileExists{../localcommands.tex}{
   \addbibresource{../localbibliography.bib}
   % add all extra packages you need to load to this file

\usepackage{tabularx,multicol}
\usepackage{url}
\urlstyle{same}

\usepackage{listings}
\lstset{basicstyle=\ttfamily,tabsize=2,breaklines=true}

\usepackage{langsci-basic}
\usepackage{langsci-optional}
\usepackage{langsci-lgr}
\usepackage{langsci-osl}
% \usepackage{./langsci/styles/langsci-lgr}
% \usepackage{./langsci/styles/langsci-osl}
% \usepackage{langsci-gb4e}

\usepackage{tikz}
\usetikzlibrary{patterns,calc}
\pgfdeclarepatternformonly{south east lines}{\pgfqpoint{-0pt}{-0pt}}{\pgfqpoint{3pt}{3pt}}{\pgfqpoint{3pt}{3pt}}{
    \pgfsetlinewidth{0.6pt}
    \pgfpathmoveto{\pgfqpoint{0pt}{3pt}}
    \pgfpathlineto{\pgfqpoint{3pt}{0pt}}
    \pgfpathmoveto{\pgfqpoint{.2pt}{-.2pt}}
    \pgfpathlineto{\pgfqpoint{-.2pt}{.2pt}}
    \pgfpathmoveto{\pgfqpoint{3.2pt}{2.8pt}}
    \pgfpathlineto{\pgfqpoint{2.8pt}{3.2pt}}
    \pgfusepath{stroke}}
    
\usepackage{stmaryrd}
\usepackage{wasysym}
\usepackage{multirow}
\usepackage{caption}
\usepackage{subcaption}
\usepackage{mathrsfs}
\usepackage{qtree}

\usepackage{linguex}


   %pminos do not split footnotes
% \interfootnotelinepenalty=10000 %Footnote in Laporte chapters has to be split SN


%\DeclareIndexNameFormat{default}{%
%\nameparts{#1}%
%\usebibmacro{index:name}%
%{\index[names]}%
%{\namepartfamily}%
%{\namepartgiveni}%
% {}% L1
% {}% L2
%{\namepartprefix}% generates spurious space L3
%{\namepartsuffix}% generates spurious space L4
%}

%  {\DeclareIndexNameFormat{default}{%
%     \usebibmacro{index:name}{\index[names]}{#1}{#3}{#5}{#7}}}

%\DeclareIndexNameFormat{default}{%
%  \usebibmacro{index:name}{\sindex[nom]}{#1}{#3}{#5}{#7}}

%\DeclareIndexNameFormat{default}{%
%  \usebibmacro{index:name}{\sindex[person]}{#1}{#3}{#5}{#7}}
%\DeclareIndexNameFormat{default}{%
%\nameparts{#1} \usebibmacro{index:name}{\sindex[person]]}{\namepartfamily}{‌​\namepartgiven}{\nam‌​epartprefix}{\namepa‌​rtsuffix}}

%\newcommand{\smiley}{:)}

%\renewbibmacro*{index:name}[5]{%
%\usebibmacro{index:entry}{#1}%
%{\iffieldundef{usera}{}{\thefield{usera}\actualoperator}\mkbibindexname{#2}{#3}{#4}{#5}}}

% \newcommand{\noop}[1]{}

%remove for final
%\overfullrule=1mm

\newcommand{\tobi}[2]}}
\renewcommand{\S}[1]{\tobi{#1}{\textsc{*}}}

% this volume references
% puts: [this volume]
% already defined: \citetv
%\newcommand{\citepv}[1]{(\citeauthor{#1} \citeyear*{#1} [this volume])}
\newcommand{\citealtv}[1]{\citeauthor{#1} \citeyear*{#1} [this volume]}

%parentheses around example number
\newcommand{\pref}[1]{(\ref{#1})}

% in-text examples

\newcommand{\lnex}[1]{\textit{#1}} %target lang word
\newcommand{\lnlit}[1]{(lit.: `#1')} %literal reading
\newcommand{\lnlat}[1]{(#1)} % latinization
\newcommand{\lntrans}[1]{`#1'} %translation
\newcommand{\lnexl}[2]%
{\lnex{#1}{} \lnlat{#2}} % ex with latinization
\newcommand{\lnexlat}[3]{\lnex{#1}{} \lnlat{#2}{} \lntrans{#3}} % ex with latinization and tranl.

%ch01
\newcommand{\co}[1]{\mbox{\textbf{#1}}}

%ch09

\newcommand{\cyrbulg}[1]{\begin{otherlanguage*}{bulgarian}#1\end{otherlanguage*}}


%ch10
\newcommand{\nlp}{{\small NLP}}
\newcommand{\mwe}{{\small MWE}}
\newcommand{\rae}{{\small RAE}}
\newcommand{\lvc}{{\small LVC}}
\newcommand{\pos}{{\small P}o{\small S}}
%\newcommand{\todo}[1]{ \textcolor{red}{#1} }

%\renewcommand{\labelenumi}{\theenumi}
%\ainamefmt{{vv}{ll}{, ff}{, jj}} % fullname

\newcommand{\biberror}[1]{{\color{red}#1}}

\newcommand{\osenovaitem}{--~}
   %% hyphenation points for line breaks
%% Normally, automatic hyphenation in LaTeX is very good
%% If a word is mis-hyphenated, add it to this file
%%
%% add information to TeX file before \begin{document} with:
%% %% hyphenation points for line breaks
%% Normally, automatic hyphenation in LaTeX is very good
%% If a word is mis-hyphenated, add it to this file
%%
%% add information to TeX file before \begin{document} with:
%% %% hyphenation points for line breaks
%% Normally, automatic hyphenation in LaTeX is very good
%% If a word is mis-hyphenated, add it to this file
%%
%% add information to TeX file before \begin{document} with:
%% \include{localhyphenation}
\hyphenation{
    Beck-man
    Ngu-yen
    back-chan-nel
    back-chan-nels
    mo-not-o-nous
    ste-reo-typ-i-cal
}

\hyphenation{
    Beck-man
    Ngu-yen
    back-chan-nel
    back-chan-nels
    mo-not-o-nous
    ste-reo-typ-i-cal
}

\hyphenation{
    Beck-man
    Ngu-yen
    back-chan-nel
    back-chan-nels
    mo-not-o-nous
    ste-reo-typ-i-cal
}

   \boolfalse{bookcompile}
   \togglepaper[23]%%chapternumber
}{}

\usepackage{longtable}

\begin{document}
\maketitle

\section{Introduction}\label{ch4:intro}
  
This chapter deals with the semantic and syntactic description of motion verbs in Bulgarian (as compared with English) with respect to: their semantics as described in terms of semantic frames; the conceptualisation of parts of the trajectory of motion and the corresponding frame elements; the syntactic realisation of the major frame elements as reflected in corpora. 

The study is based on the description of verbs in FrameNet \citep{Baker1998} as lexical units evoking particular frames, defined themselves as schematic representations of situations in terms of the configurations of participants and props that constitute their meaning. The syntactic description will be focused on the main patterns of syntactic expression of the most essential frame elements in and across the selected motion frames. The proposed account aims at capturing the semantic and syntactic properties of Bulgarian verbs of motion against a more universal background.

%The theoretical grounding of these frames originates in the theory of Frame Semantics \citep{Fillmore2003,Ruppenhofer2016}. In addition, semantic frames specify the syntactic valence patterns associated with lexical units that belong to the relevant frame on the basis of corpus annotation. 

The analysed verbs are selected from the Bulgarian WordNet \citep{koeva2021-wordnet}, which were associated with FrameNet frames \citep{LesevaStoyanova2020} and further aligned with verbs in other resources where possible \citep{2022-Linked-Resources-towards-}. As a result, the verb synsets in the Bulgarian WordNet are mapped to FrameNet frames (one frame per synset), making it possible for observations to be made on the basis of the semantic representation available for English verbs. The FrameNet model has been widely adopted for building similar descriptions of the lexis of a number of typologically diverse languages -- German \citep{burchardt-etal-2006-salsa}, Dutch \citep{Vossen2018}, Danish \citep{pedersen-etal-2018-danish}, Swedish \citep{Borin-Lars2010-110368}, Latvian \citep{Gruzitis-EtAl:2018:IFNW},  French \citep{candito-etal-2014-developing}, Spanish \citep{Subirats+2009+135+162}, Brazilian Portuguese \citep{TimponiTorrent2018Chapter4T}, Chinese \citep{You2005BuildingCF}, Japanese \citep{Ohara2004TheJF}, Korean \citep{DBLP:conf/i-semantics/HahmKWWSKPHC14}, among others. For a more comprehensive description of the existing framenets and the Multilingual FrameNet annotation initiative,\footnote{\url{https://www.globalframenet.org/}} cf. \citet{Gilardi2018LearningTA}. FrameNet’s theoretical framework has been adopted for Bulgarian and extended into a model accounting for language-specific features, including verb aspect, semantic and syntactic diatheses and syntactic alternations. The concept was implemented in the development of the Bulgarian FrameNet \citep{Koeva2008-framenet, Koeva2010-framenet,svetla2021towards}.

This chapter will specifically address: (i) those aspects in the semantics of the verbs evoking the studied frames that are cast as any of the frame elements describing the motion of an entity along a trajectory; (ii) the syntactic expression of the relevant frame elements and the conditions predetermining their syntactic explicitness or implicitness. The empirical evidence provided by the examples in the FrameNet corpus will be studied against a sample of annotated Bulgarian examples, thus testing the cross-lingual validity of the theoretical and practical observations and drawing parallels or distinctions where appropriate. The observations made throughout the chapter will be analysed from the perspective of linguistic hypotheses that have been put forward in the literature: in particular: the goal-over-source hypothesis and the proposal that motion verbs tend to co-occur with expressions that align with the most prominent aspect of the trajectory of motion encoded in their semantics.

\section{Motion verbs}\label{motion-verbs}

The semantic representation of motion verbs has been the focus of a multitude of studies. One of the major distinctions in the verb lexis is the one between manner and result, which are usually viewed as complementary notions, i.e. verbs lexicalise either one or the other \citep{Levin2015}. In the domain of motion this differentiation criterion takes the form of a distinction between the expression (and possibly the conflation) of manner and path. It has been extensively studied by \citet{Talmy1985,Talmy1991,Talmy2000}, who offered a typology %of the so-called verb-framed and satellite-framed languages, 
characterising languages in terms of the lexicalisation patterns of motion events: in satellite-framed languages, verbs usually encode manner, while the path of the movement is encoded outside the verb (base) by satellites such as adverbial particles (but also prepositions and prefixes); in verb-framed languages, the path is expressed by the verbs, and manner is either omitted or realised by means of an adverbial expression. The discovery of finer typological distinctions across languages with respect to motion expressions has led to the refinement of the original Talmian typology in the works of a number of authors (\cite{Aske1989,Slobin1996b,Papafragou2002,Ibarretxe-Antunano2004,Slobin2004,Filipovic2007,Beavers2010,Croft2010}, among others). The interest in the elements that make up the trajectory, or path, of the motion (\cite[162]{Jackendoff1983}, \cite[57]{Talmy1985}, \cite[275]{Lakoff1987}) has been reflected in numerous studies on the lexical encoding and syntactic expression of these elements in co-occurrence with the verb %(e.g. \fename{Goal}-profiling verbs tend to co-occur with \fename{Goal}-PPs, \fename{Source}-profiling verbs with \fename{Source}-PPs) 
(\cite{Rohde2001,Rakhilina2004,StefanowitschRohde2004,Cristobal2010,Kopecka2010}, to mention but a few). A related line of research has been the study of the bias with respect to the expression of one path element over another in and across languages (\cite{Ikegami1987,DirvenVerspoor1998,StefanowitschRohde2004,WalchliZuniga2006,Verkerk2017}, among others).

%The aspects of motion verbs such as 
The distinction between manner and path of motion and the expression and profiling of different sections of the path, have been the prime focus of many other studies. For instance, \citet{Viberg2015} proposes a verb typology with respect to the expression of the endpoint of motion in Swedish in comparison with Eng-\newline lish, German, French and Finnish. In her study \citet{Kopecka2010} explores lexicalisation patterns of manner of motion verbs in Polish, while \citet{Lozinska2018} delves into the expression of path and manner in Polish and Russian in contrastive terms. \citet{Taremaa2017,Taremaa2021} has explored motion verbs in Estonian, focusing on the expression of source, goal, path, location and direction with both manner of motion verbs and source- and goal-profiling verbs. 

Various authors have previously adopted the FrameNet approach in the analy-\newline sis of motion verbs. \citet{Viberg2008} proposes a study of Swedish verbs of motion in a vehicle; the verbs have been analysed from a cross-linguistic perspective and with respect to their encoding in FrameNet. \citet{Cristobal2010} provides a detailed analysis of \framename{Arriving} verbs in English and Spanish. \citet{Imani2020} study a selection of manner of motion verbs in English and contrast them with their counterparts in Persian. 

A number of studies in these lines of research have been dedicated to Bulgarian motion verbs. \citet{tchizmarova:2015} analyses several verbs with respect to the way they divide the space of linear motion, including the co-occurrence with directional phrases. %, in particular: the source-and-path oriented \textit{отивам} (`go'), the path-and-goal oriented \textit{идвам, дойда} (`come'), the source-oriented \textit{заминавам} and \textit{тръгвам}, the path-oriented \textit{ходя, вървя} (`walk') and the goal-oriented \textit{пристигам} (`arrive'). 
\citet{Lindsey2011} and \citet{speed:2015} explore the preference for and distribution of manner and path verbs in Bulgarian in contrast with other Balkan and Slavic (Balkan and non-Balkan) languages and come to the conclusion that, as suggested for Modern Greek, Bulgarian does not conform to one of the two Talmian typological patterns of conflating motion. In her work \citet{Pantcheva-2007,Pantcheva2007} centres on prefixation involving directional prefixes in Bulgarian and how this process affects event structure and syntactic structure, as part of a cross-linguistic study on directional expressions \citep{Pantcheva2011}.

A small number of FrameNet-based studies dealing with Bulgarian motion verbs have also been published, usually focusing on a small selection of predicates and their description in FrameNet, possibly supported by corpus data. For instance, \citet{2010-Formal-Description-of-Som} offer an analysis of Bulgarian verbs of non-directed motion, while
%while \citep{nedelcheva:2018} explore corpus data excerpted for the verbs \textit{enter} and \textit{go into}. \
\citet{nestorova:2009} discusses several transitive verbs involving the relocation of masses of people (\textit{populate} verbs).

This chapter's contribution lies in delivering an analysis of a selection of a non-directed and directed motion verbs in Bulgarian as compared with their English counterparts implemented through the adoption of the descriptive devices developed within the Berkeley FrameNet project and applying them to Bulgarian. The proposed methodology provides a solid foundation for cross-linguistic study of the semantic and syntactic properties of verbs.

\section{The organisation of FrameNet}

\subsection{Semantic frames and frame elements}

FrameNet \citep{Baker1998,Baker2008} is a lexical resource which couches lexical and conceptual knowledge in %a framework originating in 
the theory of Frame Semantics \citep{Johnson2001,Fillmore2003:ch4,Ruppenhofer2016}. A semantic frame is a ``script-like structure of inferences, linked by linguistic convention to the meanings of linguistic units -- in our case, lexical items. Each frame identifies a set of frame elements (FEs) -- participants and props in the frame. A frame semantic description of a lexical item identifies the frames which underlie a given meaning and specifies the ways in which FEs, and constellations of FEs, are realised in structures headed by the word” \citep[9]{Johnson2001}. Each frame in FrameNet is represented by means of a definition that describes schematically the situation and the way in which at least the most essential FEs are involved in it. Each FE is also supplied with a definition that further clarifies its semantics and its interaction with other FEs. Frame elements have different status reflecting their role in the description of a given semantic frame: core, peripheral or extra-thematic \citep[19--20]{Ruppenhofer2016}. A core FE is ``one that instantiates a conceptually necessary component of a frame, while making the frame unique and different from other frames” \citep[23]{Ruppenhofer2016}. Peripheral FEs make reference to notions such as Time, Place, Manner, Means, Degree, etc. Extra-thematic FEs characterise an event against a backdrop of another state of affairs, either of an actual event or state of the same type (e.g. the FE \fename{Iteration}), or by evoking a larger frame within which the reported state of affairs is embedded. A frame in FrameNet is linked to the meanings of a set of linguistic items, called lexical units (LUs). Each LU is thus a pairing of a word and a meaning whose conceptual semantics is represented by the frame (so that the LU is said to evoke the relevant frame). Below, reference will be made mainly to core FEs as the ones that are most essential to the description of the different frames.

The observations presented below are based on the data in the Berkeley FrameNet requested in 2017. For the sake of consistency, in the course of this work the data have been checked against the online version of the resource.\footnote{The official Berkeley FrameNet has migrated to: \url{http://berkeleyfn.framenetbr.ufjf.br/}. The online searchable database is available for browsing at \url{https://framenet2.icsi.berkeley.edu/}.} 


\subsection{Frame-to-frame relations}

FrameNet frames are organised in a network by means of a number of hierarchical and non-hierarchical frame-to-frame relations \citep[81--84]{Ruppenhofer2016}. Four of them denote hierarchical relationships that bear relevance to the internal organisation of a particular semantic domain of the lexis and will be discussed below. \FrameRelation{Inheritance} is a relation between a more general (parent) frame and a more specific (child) frame where ``each semantic fact about the parent must correspond to an equally specific or more specific fact about the child'' \citep[81–82]{Ruppenhofer2016}, i.e. there should be a strict correspondence between entities, frame elements, frame relations and semantic characteristics in the parent and the child frame \citep{Petruck2015}. Examples of this relation in the context of the studied domain are represented by the frames \framename{Self\_motion}, \framename{Fluidic\_motion}, etc. (see Fig. \ref{ch4:fig:01}, p. \pageref{ch4:fig:01}), which share the main configuration of frame elements defined for the parent frame \framename{Motion}, but in addition provide a further specification of the \fename{Theme} as an entity moving under its own power and will, i.e. a \fename{Self\_\linebreak mover} (in \framename{Self\_motion}), or as a \fename{Fluid} (in \framename{Fluidic\_motion}).
\FrameRelation{Using}, also defined as weak \FrameRelation{Inheritance} \citep{Petruck2015}, is a relation between a parent frame and a child frame in which only some of the FEs in the parent have a corresponding entity in the child, and if such exist, they are more specific \citep{Petruck2012}. In the studied domain, an instance of such a relation exists between \framename{Motion} and its child \framename{Operate\_vehicle}. Like \framename{Motion}, the more specific frame describes the translational motion of a \fename{Theme} from a \fename{Source} to a \fename{Goal} along a \fename{Path}, but elaborates on it by introducing further frame elements: an \fename{Agent}, who controls the movement, and a \fename{Carrier}, which is the actual object carrying the \fename{Theme}. \FrameRelation{Perspective} is a relation where a more abstract situation viewed as neutral may be specified by means of perspectivised semantic frames that represent ``different possible points-of-view on the neutral frame'' \citep[82]{Ruppenhofer2016}. For instance, the frames \framename{Operate\_vehicle} and \framename{Ride\_vehicle} perspectivise different facets of the idea of moving by means of a vehicle described in \framename{Using\_\linebreak vehicle} according to the involvement of a person, who is being transported, as either the driver/operator or as a passenger. \FrameRelation{Subframe} captures the relationship between a complex frame referring to ``sequences of states and transitions, each of which can itself be separately described as a frame'' and the frames denoting these states or transitions \citep[83–84]{Ruppenhofer2016}. For example, the frames \framename{Arriving} and \framename{Departing} are defined as subframes of \framename{Traversing}, as they describe the initial and the final stage of the translational movement that results in a moving entity's change of location.

A comprehensive description of all the frame-to-frame relations with more examples is provided in \citet{Ruppenhofer2016}.

\section{English and Bulgarian data employed in the study}

\subsection{FrameNet and WordNet as a source for the inventory of motion verbs}

The inventory of English verbs and their semantic and syntactic description used in the study is directly derived from the description of the lexical units in the studied semantic frames in the Berkeley FrameNet, as well as the lattices summarising the valence patterns attested in the FrameNet corpus, including the particular syntactic realisation of the FEs in terms of their syntactic category and syntactic function. The corpus is also used as a source for the examples illustrating the realisation of the English verbs.\footnote{For brevity some of the examples throughout the paper will be adapted.}

%in FATE (FrameNet-Annotated Textual Entailment)\footnote{http://framenet.icsi.berkeley.edu/} \citep{Burchardt2008}. The dataset for English is supplemented with examples from SemCor in order to illustrate the usage of particular verb senses and verb literals.

%After analysing this information for English, we turn to observe to what extent it is applicable for Bulgarian. For the purposes of the study, we attempt to construct a similar resource for Bulgarian with manually selected and validated examples from BulSemCor and additionally, where the number of examples is not sufficient, from the Bulgarian National Corpus.
%Taking as a point of departure the semantic analysis for the English verbs, I verify the validity of the observations for Bulgarian, focusing on the parallels and differences, where relevant. 

The semantic frames are adopted from the Berkeley FrameNet without changes, but where relevant, comments regarding the set of frame elements are made. The Bulgarian verbs are studied independently but in comparison with their Eng-\newline lish counterparts, taking as a point of departure the relevant motion frames and the valence patterns described in the Berkeley FrameNet. This approach has been adopted to facilitate the description of the motion verbs in the Bulgarian FrameNet whose original concept was laid out in \citet{Koeva2008-framenet, Koeva2010-framenet} and further elaborated in \citet{svetla2021towards}, as well as in Chapter 1, this volume. The Bulgarian FrameNet is implemented within BulFrameNet \citep{koeva-doychev-2022-ontology}, a dedicated web-based system allowing the comprehensive description of the semantic and syntactic properties of verbs. The study of the valence patterns of the motion verbs and the syntactic expression of their semantic participants presented below was undertaken specifically as part of the work on the Bulgarian FrameNet. 

The set of Bulgarian motion verbs discussed in the chapter is extracted from the Bulgarian WordNet, a lexical-semantic net modelled on the Princeton WordNet (PWN). PWN \citep{Miller1995,Fellbaum1998} is a large lexical database for English that comprehensively represents conceptual and lexical knowledge in the form of a network whose nodes denote cognitive synonyms (synsets) connected through a number of conceptual-semantic and lexical relations such as hypernymy, meronymy, antonymy, etc. The synsets in the Bulgarian Wordnet  have been developed by translation and adaptation of the PWN counterparts, and the corresponding synsets in the two wordnets are related to each other through unique interlingual identifiers (which also provide links to the respective synsets in all other wordnets that support them). Thus, the lexical and conceptual knowledge is aligned cross-linguistically. In the course of its creation the Bulgarian WordNet has been expanded so as to cover all the synsets included in PWN (117,659 in total) by means of automatic translation followed by manual editing and enrichment (currently 85,954 synsets have been manually validated). The Bulgarian WordNet includes language-specific lexicalisations (synsets with no correspondence in PWN) as well as synsets describing closed-class words: prepositions, conjunctions, pronouns, particles, interjections; as a result it has amounted to 121,282 synsets altogether. It thus provides substantial coverage of the language's lexis, including verbs (forming a total of 14,103 synsets). In addition, BulNet has developed a number of language-specific characteristics, including notation of verb aspect. The current state of the Bulgarian WordNet is detailed in \citet{koeva2021-wordnet}.\footnote{The Bulgarian WordNet may be browsed at: \url{dcl.bas.bg/bulnet/}.} 

WordNet and FrameNet were aligned automatically using several previous mappings coupled with additional procedures for expansion and validation. In particular, the following were employed: (i) direct mappings provided within FrameNet \citep{Baker2009}, eXtendedWordFrameNet \citep{Laparra2010} and MapNet \citep{Tonelli2009}, supplemented with (ii) indirect mapping through VerbNet \citep{Palmer2009,Palmer2014}. This resulted in 4,306 unique WordNet synsets to FrameNet frame mappings, a coverage of 30.5\% of the verb synsets \citep[110]{LesevaStoyanova-CIT:2020}. A number of procedures inspired by ideas proposed in \citet{lopez-de-lacalle-etal-2014-predicate} and especially in \citet{Burchardt2005} were implemented towards the  improvement and extension of the mapping coverage. These procedures, described in \citet{Leseva2018} and further refined in \citet{StoyanovaLeseva2019,LesevaStoyanova2020}, are specifically based on exploring the structural properties of the two resources, such as: (i) the assumption that as verbs in a synset denote the same or very similar meaning, they are likely to evoke the same semantic frame; (ii) the hierarchical relational structure of the two resources based on the notion of inheritance from a more general to a more specific synset or frame. As a result, in general, more specific concepts should be associated with the frame of their hypernym(s) or with more specific frames elaborating on (and possibly inheriting from) this frame, although various divergences occur in practice. Part of the other relations among frames and among synsets were also cast as validation procedures. The main mapping mechanism involved: (i) manual assignment of semantic frames to root verb synset to ensure greater accuracy at the highest hierarchical level and to reduce error propagation down the tree; (ii) automatic assignment of the hypernym’s frame onto hyponyms which were not previously mapped; and (iii) verification and improvement of the assignments by applying the validation procedures. In this way, the coverage of the automatic mapping has been gradually increased to 94\% \citep[115–116]{LesevaStoyanova-CIT:2020}. Due to various peculiarities of the structure of WordNet or lack of appropriate frames in FrameNet (as part of the lexis has not yet been described by frames), the automatic assignment has been undergoing manual validation, so far covering almost 50\% of the mapping (over 6,000 synsets).  
 
The FrameNet-to-WordNet alignment together with the alignment between the Princeton WordNet and the Bulgarian WordNet has enabled the association of Bulgarian verbs with a FrameNet semantic description. This possibility is founded on the assumption that although the construal of the semantic description of situations across languages (as well as across resources) often differs, the major semantic aspects represent shared conceptual knowledge. Such an assumption underlies the development of both wordnets and framenets for other languages besides English, as well as the Global FrameNet initiative and Shared Annotation Task  (cf. \sectref{ch4:intro}). The genealogical and typological similarities between English and Bulgarian have also made it possible to base the syntactic description of the Bulgarian verbs of motion on the one provided for their English counterparts in the Berkeley FrameNet. Similar ideas have been pursued by other authors who have adopted a FrameNet-oriented approach to the semantic and syntactic analysis for languages other than English (cf. \sectref{motion-verbs}). The analysis below has been specifically informed by previous work on Bulgarian change \citep{StoyanovaLeseva:21a} and stative verbs \citep{LesevaSoyanova:2022}.

The English and the Bulgarian verbs included in the analysis are members of synsets that have been assigned one of several FrameNet frames belonging to the motion domain. In order to be selected, they had to meet the following requirements: (i) pertain to the general lexis; (ii) have a corresponding LU in FrameNet with a sufficient number of annotated sentences (20+). This means that synsets such as \{walk:1\} `use one's feet to advance; advance by steps' and \{run:34\} `move fast by using one's feet, with one foot off the ground at any given time', which have as correspondences the LUs \textit{walk.v} and \textit{run.v} in the \framename{Self\_motion} frame are included in the analysis, while ones such as \{lollop:1\} `walk clumsily and with a bounce' and \{hare:1\} `run quickly, like a hare' are not. These requirements have been adopted for the following reasons: general-lexis verbs are more likely to be represented in BulSemCor (see \sectref{annotated-data}), hence more Bulgarian examples would be available for them; the representation in the FrameNet corpus provides the pool of examples for English.


\subsection{Annotated examples}\label{annotated-data}

The statistics and analysis for English are based on the annotated sentences available for the respective verbs in the Berkeley FrameNet. 

The examples of the usage of the selected Bulgarian verbs are extracted from BulSemCor \citep{koeva-2006-bulsemcor,koeva-2011-bulsemcor} -- a 100,000-word corpus designed according to the overall methodology of the English SemCor \citep{miller-etal-1993-semantic,miller-etal-1994-using,landes1998}, further adapted by using criteria for ensuring an appropriate coverage of contemporary Bulgarian general lexis. As BulSemCor is manually annotated with wordnet senses, it provides disambiguated examples for the studied verbs. Where the number of examples is not sufficient, they have been supplemented with sentences from the Bulgarian National Corpus, a corpus of 1.2 billion words of running Bulgarian text distributed in 240,000 text samples spanning the second half of the 20th century and the beginning of the 21st century \citep{Koeva2012}. As the corpus is not sense-disambiguated, the examples have been selected manually so as to correspond to the studied senses. 

The Bulgarian examples extracted from the different corpora have been annotated so that the sentence components that syntactically realise the core frame elements related to motion are explicitly marked in a similar fashion to the annotation in the Berkeley FrameNet corpus. The selection covers 893 annotated clauses or sentences distributed as follows across the selected semantic frames: \framename{Motion} -- 149; \framename{Self\_motion} -- 262; \framename{Arriving} -- 182; \framename{Departing} -- 178; \framename{Traversing} -- 122.

%which represents the contemporary state of the language and is the largest corpus of Bulgarian, amounting to more than 1.2 billion words for Bulgarian (5.4 billion with its parallel corpora. 

\section{The domain of Motion}

\subsection{Organising semantic domains}

As suggested by \citet[16]{Johnson2001}, the lexicon pertaining to a semantic domain is hierarchically organised in a number of semantic frames of different abstraction and specialisation related through the frame-to-frame relations that capture semantic generalisations existing across frames. Thus, as pointed out in the work cited, for many semantic domains, there is one general frame that describes the common aspects of the more specific frames. It may be posited that at the conceptual level all (or most) frames in a domain share this basic structure consisting of a configuration of FEs that defines the distinctive meaning of the domain. The mechanisms that organise such a part of the lexis involve various changes in this prototypical structure that reflect the various ways in which specialisation within the domain occurs: (i) not all frames allow the overt expression of all FEs -- some of them may be blocked from overt expression, although they are conceptually necessary and implied in the meaning of the lexical units; (ii) more specific frames may exclude some FEs or demote them to non-core status; for instance, in the \fename{Goal}-profiling \framename{Arriving} and the \fename{Source}-profiling \framename{Departing} frames the FEs describing the remaining parts of the route are regarded as non-core; (iii) more specific frames may further narrow down the semantic properties of one or more of the FEs as compared with their counterparts in the more general frame (for instance, impose stricter selectional restrictions on the expressions realising the FEs): e.g. the moving entity is defined as the FE \fename{Fluid} in the \framename{Fluidic\_motion} frame, and as \fename{Mass\_theme} in the \framename{Mass\_motion} frame which both inherit from the \framename{Motion} frame (Fig. \ref{ch4:fig:01}, p. \pageref{ch4:fig:01}); (iv) more specific frames may include other FEs besides the ones describing the general frame, may change perspective, incorporate or profile a certain FE. An example of a semantic elaboration resulting in the introduction of a new FE is the specification of the vehicle which ``holds or conveys'' the traveller (the FE \fename{Mode\_of\_transportation}) in the \framename{Travel} frame (Fig. \ref{ch4:fig:01}).

The observations below are based on the theoretical and practical motivations described in \citet{Johnson2001}, \citet{Petruck2012}, \citet{Petruck2015}, \citet{Ruppenhofer2016} and the definitions, comments and frame-to-frame relations in FrameNet.

\subsection{General organisation of the domain of motion}

The lexis denoting movement is most broadly divided between translational and non-translational (or self-contained) motion with respect to some background or location. %\cite[35–26]{Talmy!!!}. 
Based on the definitions in FrameNet,\footnote{\url{http://framenet2.icsi.berkeley.edu/fnReports/data/frameIndex.xml?frame=Motion}} in the first case a moving entity typically starts at some location, moves through space along a trajectory and ends up in another location. Non-translational motion\footnote{\url{https://framenet2.icsi.berkeley.edu/fnReports/data/frameIndex.xml?frame=Moving_in_place}} involves the movement of an entity or parts of it with respect to some fixed location or landmark, without undergoing motion in space or without a significant alteration of configuration or shape. Translational motion is most broadly described by the non-lexicalised frame \framename{Motion\_scenario}, which is further perspectivised by several frames, two of which, \framename{Motion} and \framename{Traversing}, form the core of the translational motion domain. Non-translational movement is described by the \framename{Moving\textunderscore in\textunderscore place} frame and its causative counterpart \framename{Cause\_to\_move\_in\_place}, which are evoked by LUs such as \textit{rock, shake, twirl}, e.g. \textit{The earth shook} vs. \textit{He shook the remote control}. In what follows below, the focus will be on translational motion.

Another major division in the domain of motion is between (i) self-induced motion that a moving entity undergoes on its own -- under its own physical power, due to some internal cause, physical forces, features of the relief, etc., and (ii) caused motion that is brought about by an external participant that may be an animate, volitional \fename{Agent} or a non-animate \fename{Cause}. The prototypical semantic frame that organises the lexis of this type of translational motion is \framename{Cause\_motion}, which is indirectly integrated in the \framename{Motion\_scenario} through its causative relation to \framename{Motion} (i.e. \framename{Cause\_motion} Is\_causative\_of \framename{Motion}). The frames related to \framename{Cause\_motion} include \framename{Bringing} (e.g. \textit{bring, carry, transport}), \framename{Placing} (e.g. \textit{bottle, load, pocket}), \framename{Filling} (e.g. \textit{fill, flood}), \framename{Removing} (e.g. \textit{extract, remove}), \framename{Emptying} (e.g. \textit{empty, purge}), as well as some frames (e.g. \framename{Cause\_fluidic\_motion}) that have counterparts in the non-causative domain (\framename{Fluidic\_motion}). %In the remaining part of the chapter I will analyse frames from the \framename{Motion} domain.

%As defined in the Motion frame, specialisation of the vocabulary in this domain may also involve ``assumptions about the shape-properties, etc., of any of the places involved (insert, extract)”. Such assumptions are reflected in the subdomain of caused motion in frames such as Placing (bottle, enclose, put), Removing (clear, remove, strip).

As suggested in the description of the \framename{Motion} frame, a complex area in the vocabulary of motion is the depiction of the relation of \fename{Vehicles} to the moving entity and other participants. In the cases where the moving entity cannot be expressed, the LUs denoting the motion of vehicles are treated as evoking the \framename{Self\_motion} frame and the vehicles are annotated as \fename{Self\_movers}. When the \fename{Vehicle} is profiled as being operated by a \fename{Driver}, the relevant LUs evoke the frame \framename{Operate\_vehicle}; the \fename{Driver} may be construed very generally as the one controlling the vehicle, as attested by verbs such as \textit{bicycle, canoe, paddle, skate}, along with verbs involving special qualifications or skills such as \textit{drive, fly, sail, taxi}. The situation where the moving entities are passengers that are transported by means of the \fename{Vehicle} which is not under their control, is described by the \framename{Ride\_vehicle} frame (\textit{bus, hitchhike, ride, sail}).

Another type of elaboration in the motion domain described in the definition of the frame or reflected in the frame-to-frame relational structure refers to properties of the manner of motion, which basically stem from prominent features of the moving entity. One such feature is the requirement for the moving entity to be a living being whose body moves on its own, using its own energy, as in \framename{Self\_motion} (e.g. \textit{jog, limp, run, walk}), and semantic frames inheriting from it such as \framename{Cotheme} (\textit{accompany, lead, track}) and \framename{Travel} (\textit{journey, tour, voyage}). Further salient distinctions based on the types of entities involved in the motion and the specific manner of motion typical of them is reflected in the definition of several frames such as: \framename{Fluidic\_motion} (e.g. \textit{cascade, ooze, stream}), describing the motion of liquids; \framename{Mass\_motion} (e.g. \textit{crowd, swarm, throng}), which refers to the motion of a collective of individuals (a \fename{Mass\_theme}) as one entity; \framename{Motion\_noise} (e.g. \textit{buzz, roar, whir}), specified according to the type of noise the moving entity produces; \framename{Light\_movement} (e.g. \textit{gleam, shine, twinkle}), describing the emission and movement of light, etc. 

Another facet of motion has to do with the feature of directionality, which is lexically encoded in some LUs (e.g. \textit{descend, fall, rise} in the \framename{Motion\_directional} frame). Directed motion is also described in semantic frames that profile parts of the path along which an entity moves, such as its initial (\framename{Departing}) or final stage (\framename{Arriving}). 

In the remaining part of the chapter the analysis will be focused specifically on non-directed and directed motion verbs as represеnted by semantic frames such as \framename{Motion} and \framename{Self\_motion} on the one hand, and \framename{Traversing}, \framename{Arriving} and \framename{Departing} on the other, drawing parallels between the semantics and syntactic expression of the relevant frame elements.

The overall organisation of the domain of self-induced translational motion is partially represented in \figref{ch4:fig:01}.

\begin{figure}
\includegraphics[width=0.9\textwidth]{figures/Motion1.jpg}
\caption{The Motion hierarchy: green lines denote the relation of \FrameRelation{Inheritance}; red lines show the relation \FrameRelation{Perspectivises}; blue lines correspond to the \FrameRelation{Subframe of} relation; magenta lines denote the relation \FrameRelation{Using}.}\label{ch4:fig:01}
\end{figure} 


\subsection{\framename{Motion}}

\subsubsection{Semantic description of the \framename{Motion} frame}

%Below I will present in brief the prototypical frame of the semantic domain, i.e. \framename{Motion}, and the main criteria that further organise the verb lexis pertaining to the domain. Ideally, the principles of organisation are reflected in the relational structure of the relevant subsection of FrameNet (i.e. through frame-to-frame relations), although the relation may not always be direct. In outlining these criteria, I use the description proposed in the Motion frame and the frame-to-frame relations.

The \framename{Motion} frame describes the changing of spatial location of a \fename{Theme} understood in the classical sense of Gruber (\cite[27--31]{Gruber1965}, \cite[29]{Jackendoff1972}) as an entity that moves. More precisely, the LUs in this frame involve the translational motion of entities that are either not (capable of) moving under their own power or are underspecified for this feature. %in the lexical meaning of the relevant LUs: 
Therefore, many of the definitions of the motion frames for which this property is relevant note that the \fename{Theme} is frequently a living being moving on its own but need not be. Generalising over FrameNet examples such as the following, one can infer that the motion may be induced by various factors: (i) an outside force: [\textit{The black dust}]$_{\feinsub{Thm}}$ \textit{began \textbf{BLOWING OFF}} [\textit{the tailings lake}]$_{\feinsub{Src}}$; %(FN\footnote{The examples marked as (FN) are taken from the relevant frame in FrameNet.})
(ii) the \fename{Theme}’s own momentum:  \textit{It} \textit{fell on the floor and [\textit{\_}]$_{\feinsub{Thm}}$ \textit{\textbf{ROLLED}} [\textit{towards Uncle Mick’s feet}}]$_{\feinsub{Goal}}$; (iii) some internal process: [\textit{Tears}]$_{\feinsub{Thm}}$ \textit{\textbf{ROLLED}} [\textit{down my cheeks}]$_{\feinsub{Path}}$, etc., but it is represented with respect to the involvement of the \fename{Theme} in it, regardless of the cause that has brought it about. Volitional or self-directed motion is elaborated in some of the frames inheriting from \framename{Motion}, \framename{Self\_motion} and its descendants in particular. The remaining core FEs of the \framename{Motion} frame describe various elements or properties of the path\footnote{When used with a capital letter, \fename{Path} would mean the frame element; in small letters, path would be used in its accepted meaning in the literature, i.e. the medial part of the route traversed by a \textbf{Figure} (\cite[26]{Fillmore1971},  \cite[275]{Lakoff1987}, among  others). The term “route” will be used instead of “path” to refer to the line of movement that comprises all the three elements: \fename{Source}, \fename{Path} and \fename{Goal}.} that the moving entity moves along. 

\begin{description}[font=\normalfont]
\item[Definition of the \framename{Motion} frame:] Some entity, the \fename{Theme}, starts out in one place (\fename{Source}), and ends up in some other place (\fename{Goal}), having covered some space between the two (\fename{Path}). Alternatively, the \fename{Area} or \fename{Direction} in which the \fename{Theme} moves or the \fename{Distance} covered may be mentioned.\footnote{The frame definitions are taken from FrameNet: \url{https://framenet.icsi.berkeley.edu/framenet_search}.}

\item[Core FEs in the \framename{Motion} frame:] \fename{Theme}, \fename{Source}, \fename{Goal}, \fename{Path}, \fename{Area}, \fename{Direction}, \fename{Distance}.
\end{description}

The \fename{Theme}\footnote{The definitions of the FEs are taken from the description of the \framename{Motion} frame, with further elaboration on their semantic specification informed by the annotated examples studied in the paper.} is defined as ``the entity that changes location''. An important feature of this FE noted in the FrameNet description is that it need not be a \fename{Self\_mover}, that is, it need not be capable of moving on its own and by its own power or will.
%\footnote{The definitions of FEs are based on the definitions across related frames so as to provide a consistent and more comprehensive description.} is the entity that changes location either by moving on its own and/or under its own power or by being moved, carried, etc. by another entity or force. An important feature of this FE is that it may but need not be capable of moving on its own and by its own power or will (which distinguishes it from \fename{Self\_mover}). Its semantic specification includes \textbf{animate beings} and \textbf{physical objects}.
 Its semantic specification includes animate beings, physical objects, substances, etc.

The \fename{Source} is ``the location the \fename{Theme} occupies initially before its change of location''. It may refer to geological and other natural formations, geographical points, celestial bodies; physical objects, including man-made structures, such as buildings, constructions, facilities, other objects occupying space, etc.


%a location or an entity occupying space that serves as the starting point or landmark where the moving entity is at before it starts changing location. 
%Its semantic specification spans various \textbf{locations}, including geological and other natural formations, geographical points, celestial bodies; \textbf{physical objects}, including man-made structures, such as buildings, constructions, facilities, etc., but also any other object occupying space.

The \fename{Goal} is ``the location the \fename{Theme} ends up in'' as a result of the motion. It has the same semantic specification as the \fename{Source}.

The \fename{Path} refers to ``(a part of) the ground over which the \fename{Theme} travels or to a landmark by which the \fename{Theme} travels''. Its semantic specification encompasses locations, including geological and other natural formations, geographical points, celestial bodies; physical objects, including man-made structures and other objects occupying space that may be construed as having extent along which the motion takes place; extents of various media, such as water, air, etc. 


%is any description of a trajectory of motion which is neither a \fename{Source} nor a \fename{Goal} and more specifically refers to (a part of) the ground over which the moving entity travels or to a landmark by which it travels.


The \fename{Area} identifies the setting where ``the \fename{Theme}'s movement takes place without a specified \fename{Path}''. A notable consequence of the lack of a single linear trajectory is that the \fename{Area} cannot co-occur with \fename{Source}, \fename{Goal} and \fename{Path}, i.e. it is defined in an ‘excludes’ relation to each of them as well as to the FEs \fename{Distance} and \fename{Direction} which provide additional details referring to the translational motion in space. 
Like \fename{Path}, the semantic specification of \fename{Area} refers to locations, physical objects, other objects occupying space, various media, such as water, air, etc., which, however, may be construed as comprising some spatial expanse in, over or around which the motion takes place in an irregular fashion.

The \fename{Direction} %indicates the general spatial orientation %of the motion of the moving entity along a line from the deictic centre towards a (possibly implicit) reference point that is neither the \fename{Goal} of the posture change nor a landmark along the way of the moving part of the body; it characterises the \fename{Path} of motion according to some system of orientation; often \fename{Direction} is defined with reference to the canonical orientation of the moving entity, or the orientation imposed by an implicit observer. 
indicates ``motion along a line from the deitic center towards a reference point (which may be implicit) that is neither the \fename{Goal} of the posture change nor a landmark along the way of the moving part of the body. Often \fename{Direction} is defined with reference to the canonical orientation of the Protagonist, or the orientation imposed by an implicit observer''. %For some semantic frames it may be further specified, e.g. with \framename{Motion\_directional} it is the direction of motion of the \fename{Theme}, which is most often imposed by gravity). 
The semantic specification of this FE includes directions, such as compass points (north, east, south, west), body relative directions (left, right, back, front, backward, forward, up, down), coordinates, etc.

The \fename{Distance} encompasses expressions that characterise ``the extent of the motion'' covered by the \fename{Theme}. Its semantic specification includes distance denotations expressed either in various systems of measurement or as relative distances (farther, closer), etc.  

The basic configuration of the core FEs of the \framename{Motion} frame and the interaction among them determines the overall semantic specification of the prototypical notion of motion, which is subject to various modifications and elaborations in the more specific motion frames.

The syntactic expression of the semantic configuration of the \framename{Motion} frame will be discussed in terms of: (i) the (typical) syntactic projections of each core FE, in particular its syntactic (phrasal) category and grammatical relation; (ii) the most frequent valence patterns, i.e. the various frequent combinations in which the core FEs co-occur in the annotated FrameNet corpus. For English, both types of data are extracted from the summaries provided for each LU in FrameNet; the Bulgarian counterparts are analysed in comparison with the descriptions available for English and tested against the corpus of annotated examples created for Bulgarian (see \sectref{annotated-data} above).

\subsubsection{Verbs evoking the \framename{Motion} frame}

Тhe \framename{Motion} frame is evoked by a couple of basic verbs of inherently directed motion \citep{Levin1993}, such as \textit{come} and \textit{go}, as well as by verbs of non-directed motion. Within the second class, some predicates, such as \textit{move} and \textit{travel}, describe the general idea of moving through space, while others, for instance \textit{blow}, \textit{drift}, \textit{float}, \textit{circle}, \textit{roll}, denote various types of manner of motion; part of these verbs, e.g. \textit{meander}, \textit{spiral}, \textit{weave}, \textit{wind}, \textit{zigzag}, involve complex trajectories. 


\subsubsection{Syntactic realisation of the frame elements in the \framename{Motion} frame}

\tabref{tab:4:motion-synt} below illustrates the syntactic realisation of several verbs evoking the \framename{Motion} frame, chosen according to the following criteria: (i) having a sufficient number of attestations in the FrameNet corpus, thus allowing for more reliable observations; (ii) illustrating distinct syntactic patterns with respect to the expression of the FEs denoting the different parts or features of the route of movement.

The \fename{Theme} is typically projected in the subject position and the remaining core FEs are expressed primarily as prepositional (PP) or, more rarely, as adverbial phrases (marked as AVP). Some of the verbs also allow object NPs, especially as a realisation of \fename{Path}: [\textit{She}]$_{\feinsub{Thm}}$ \textit{\uppercase{\textbf{circles}}} [\textit{the taxi}]$_{\feinsub{Path}}$. In addition, the core FEs expressing elements or aspects of the route, may be conceptually present but left syntactically non-overt if they are known or retrievable from the previous text (definite null instantiations, DNIs) or implied from a broader context but without a referent in the previous text (indefinite null instantiations, INIs) or if the grammatical construction requires them to be left non-explicit (constructional null instantiations, CNIs), cf. \citep[28--30]{Ruppenhofer2016}. 

Several preliminary observations are relevant at this point. Usually, the route is conceived as a tripartite spatial extent consisting of an initial part, \fename{Source}, a medial part, \fename{Path}\footnote{Various names have been applied to this part; here I adopt the name of the relevant FrameNet frame element.}, and a final part, \fename{Goal} (\cite[162]{Jackendoff1983}, \cite[57]{Talmy1985}, \cite[275]{Lakoff1987}). As mentioned in \sectref{motion-verbs}, studies on the co-occurrence of directional phrases with motion verbs have shown that verbs tend to express the element of the route that is most prominent in their semantics. It has been convincingly demonstrated for many languages that \fename{Goal}-oriented verbs tend to co-occur with \fename{Goal} phrases (\cite[22--24]{Rakhilina2004}, \cite[255-257]{StefanowitschRohde2004}, \cite[174--178]{Taremaa2017}, among others), and \fename{Source}-oriented verbs co-occur with \fename{Source} phrases (\cite[22--23]{Rakhilina2004}, \cite[255-257]{StefanowitschRohde2004}, \cite[160--164]{Taremaa2017}), see also the analyses proposed in studies such as the ones by \citet{Cristobal2010,Kopecka2010}. %\citep{Cristobal2010,Kopecka2010}.  %(Example \ref{ex:01} a).

In addition, it has been posited that there is a marked cross-lingual asymmetry with respect to the expression of the \fename{Source} and the \fename{Goal}, known as the goal-over-source principle. This proposition suggests that \fename{Goals} are expressed more frequently, using more fine-grained linguistic devices than \fename{Sources} (\cite{Ikegami1987,WalchliZuniga2006,Verkerk2017}, among others). One of the explanations for this peculiarity offered in the literature is that \fename{Goals} bear higher information value with respect to the complete conceptualisation of motion \citep[249]{StefanowitschRohde2004}; for an extensive overview of the discussion, see \citet{Verkerk2017}. However, as noted above, the preference for one type of phrase over another depends on the semantics of the verb, specifically, whether the verb conceptualises the motion in terms of a route or not \citep{StefanowitschRohde2004}. 

Unlike \fename{Source}- and \fename{Goal}-phrases, \fename{Path}-phrases %characterise the motion through space without 
do not express directionality %are inherently non-directional 
(see also \cite[31]{Pantcheva2011}). Considering that many of the verbs in the \framename{Motion} frame describe non-directed motion, the \fename{Path} should be the most prominent phase of motion inherent in their semantics and hence will be favoured for syntactic expression, unless some other aspect of motion turns out to be more prominent. Respectively, we should expect that the inherently directed-motion verbs -- \textit{go} and \textit{come} -- favour \fename{Goal}-phrases. \tabref{tab:4:motion-synt} confirms these expectations, which are further corroborated by the co-occurrence patterns in \tabref{tab:4:motion-valence-framenet}: the ones involving \fename{Paths} are the most frequent among the top ranking patterns (Column 1) and have the greatest number of occurrences (Column 2) across the greatest number of verbs (Column 3).

Looking at individual verbs, a couple of trends may be noted with respect to the prevalence of expression of the route-related FEs (i.e. all the core FEs, excluding the \fename{Theme}) (\tabref{tab:4:motion-synt}).

\fename{Path}:~ Several~ verbs~ show~ marked~ preference~ for~ expressing syntactically \fename{Paths} over any other route-related FEs. These include \textit{move} as well as a number of manner of motion verbs: \textit{weave}, %and a number of other, less represented\footnote{By representation I mean the number of annotated examples in the FrameNet corpus.} verbs: 
\textit{circle}, \textit{glide}, \textit{meander}, \textit{wind}, \textit{zigzag}. %\textit{Move} specifies neither manner, nor directed motion. The latter fact accounts for the modest number of expressed \fename{Goals} and \fename{Sources}. By comparison, in half of the annotated sentences the verb co-occurs with a syntactically expressed \fename{Path}. Additionally, in a substantial number of cases (21) the \fename{Path} is considered to be understood in a general context, although not having a specific discourse referent (the so-called indefinite null instantiation) \citep[28-29]{Ruppenhofer2016}, and is thus conceptually present but syntactically non-overt. 
Among them \textit{glide} denotes qualitative features of the movement (effortlessness) and the contact with the surface along which the motion takes place. \textit{Weave}, \textit{wind}, \textit{meander} and \textit{zigzag}\footnote{There are other verbs, e.g. \textit{undulate} and \textit{spiral}, that possibly behave in a like manner, but the number of occurrences is too small to make a judgment.} describe complex vacillating or snake-like movement along a more or less linear route or general direction, while \textit{circle} refers to a circular trajectory. In all these cases the \fename{Path} -- including its form, landmarks, etc. --  is the default spatial dimension according to which the movement is characterised.
%with reference to its relation to the objects of the surrounding locale (e.g. \textit{across the valley, along the river}), the form or extent of the route, the interaction of the movement with the route (e.g. \textit{weave, zigzag}), etc.

The second most prominent aspect of motion with the verbs in the \framename{Motion} frame is the end-point of the route, the \fename{Goal}. It is usually less frequent than \fename{Path}, to the exception of the inherently directed motion verb \textit{go} (cf. also \citet[253--254]{StefanowitschRohde2004}, for which the prevalence of \fename{Goal}- over \fename{Path}-phrases is roughly 4:1. 

%\begin{table}
%\centering
{\small
\begin{longtable}{l ccccccccc}
\caption{\label{tab:4:motion-synt}Syntactic expression of the \framename{Motion} FEs in FrameNet}\\
\lsptoprule
  & NP.Ext & NP.Obj & PP & AVP & NI & Clause & Other & Total\\ \midrule \endfirsthead
\midrule
  & NP.Ext & NP.Obj & PP & AVP & NI & Clause & Other & Total\\ \midrule \endhead
\multicolumn{9}{l}{\textit{move} } \\*
\fename{Theme} & 67  &  &  &  &  &  &  & 67\\*
\fename{Area} &  &  & 2  &  &  &  &  & 2\\*
\fename{Source} &  &  & 5  & 2  &  &  &  & 7\\*
\fename{Path} &  & 3  & 30  & 5  & 21  &  &  & 59\\*
\fename{Goal} &  &  & 5  &  & 1  &  &  & 6\\*
\fename{Direction} &  &  &  & 1  &  &  &  & 1\\*
\fename{Distance} &  &  &  &  &  &  & 1 & 1\\\midrule
\multicolumn{9}{l}{\textit{go} } \\*
\fename{Theme} & 64  &  &  &  & 1  &  &  & 65\\*
\fename{Area} &  & 1  & 1  &  &  &  &  & 2\\*
\fename{Source} &  &  & 1  & 1  & 3  &  &  & 5\\*
\fename{Path} &  &  & 8  & 2  & 1  &  &  & 11\\*
\fename{Goal} & 1  & 2  & 30  & 8  & 6  & 1  &  & 48\\*
\fename{Direction} &  &  & 1  & 5  &  &  & 1 & 7\\*
\fename{Distance} &  &  & 1  & 1  & 1  &  &  & 3\\\midrule
\multicolumn{9}{l}{\textit{drift} } \\*
\fename{Theme} & 39  &  &  &  &  &  &  & 39\\*
\fename{Area} &  &  & 2  &  &  &  &  & 2\\*
\fename{Source} &  &  & 9  &  &  &  &  & 9\\*
\fename{Path} &  &  & 15  & 4  & 4  &  &  & 23\\*
\fename{Goal} &  &  & 8  & 1  &  &  & 1 & 10\\*
\fename{Distance} &  &  &  & 1  &  &  & 1 & 2\\
\midrule
%\multicolumn{9}{l}{\textit{glide} } \\
%\fename{Theme} & 12  &  &  &  &  &  & 1 & 13\\
%\fename{Path} &  &  & 7  & 2  & 1  &  &  & 10\\
%\fename{Source} &  &  & 1  & 2  &  &  &  & 3\\
%\midrule
%\multicolumn{9}{l}{\textit{blow} } \\
%\fename{Theme} & 21  &  &  &  &  &  & 7 & 28\\
%\fename{Area} &  & 1  & 1  & 1  &  &  &  & 3\\
%\fename{Path} &  &  & 6  & 2  & 3  &  &  & 11\\
%\fename{Source} &  &  & 7  & 1  &  &  &  & 8\\
%\fename{Goal} &  & 1  & 6  & 1  &  &  &  & 8\\
%\midrule
\multicolumn{9}{l}{\textit{float} } \\*
\fename{Theme} & 43  &  &  &  &  &  &  & 43\\*
\fename{Area} &  &  & 13  & 1  &  &  &  & 14\\*
\fename{Source} &  &  & 4  & 1  &  &  &  & 5\\*
\fename{Path} &  &  & 13  & 3  & 2  &  &  & 18\\*
\fename{Goal} &  &  & 8  &  &  &  &  & 8\\*
\fename{Distance} &  &  &  & 1  &  &  &  & 1\\\midrule
%\multicolumn{9}{l}{\textit{coast} } \\
%\fename{Theme} & 6  &  &  &  &  &  &  & 6\\
%\fename{Path} &  & 2  & 2  & 1  & 1  &  &  & 6\\
%\fename{Source} &  & 1  &  &  &  &  &  & 1\\
%\midrule
\multicolumn{9}{l}{\textit{roll} } \\*
\fename{Theme} & 31  &  &  &  &  &  &  & 31\\*
\fename{Area} &  &  & 1  &  &  &  &  & 1\\*
\fename{Source} &  &  & 1  & 1  &  &  &  & 2\\*
\fename{Path} &  & 3  & 17  & 2  &  &  &  & 22\\*
\fename{Goal} &  &  & 9  & 1  &  & 1  &  & 11\\\midrule
%\multicolumn{9}{l}{\textit{soar} } \\
%\fename{Theme} & 8  &  &  &  &  &  &  & 8\\
%\fename{Area} &  &  &  & 1  &  &  &  & 1\\
%\fename{Path} &  & 1  &  & 1  & 2  &  &  & 4\\
%\fename{Goal} &  &  & 3  &  &  &  &  & 3\\
%\midrule
%\multicolumn{9}{l}{\textit{fly} } \\
%\fename{Theme} & 11  &  &  &  &  &  &  & 11\\
%\fename{Area} &  &  & 1  &  & 1  &  &  & 2\\
%\fename{Distance} &  &  &  & 1  &  &  & 1 & 2\\
%\fename{Path} &  &  & 6  &  & 1  &  &  & 7\\
%\fename{Goal} &  &  & 2  &  &  &  &  & 2\\
%\midrule
\multicolumn{9}{l}{\textit{slide} } \\*
\fename{Theme} & 26  &  &  &  &  &  &  & 26\\*
\fename{Area} &  &  & 3  &  &  &  &  & 3\\*
\fename{Source} &  &  & 1  &  &  &  & 1 & 2\\*
\fename{Path} &  &  & 9  &  &  &  &  & 9\\*
\fename{Goal} &  &  & 10  &  &  &  &  & 10\\*
\fename{Direction} &  &  & 1  & 5  &  &  &  & 6\\\midrule
\multicolumn{9}{l}{\textit{swerve} } \\*
\fename{Theme} & 27  &  &  &  &  &  &  & 27\\*
\fename{Area} &  &  & 2  &  & 4  &  &  & 6\\*
\fename{Source} &  &  & 4  &  &  &  &  & 4\\*
\fename{Path} &  &  & 10  &  &  &  &  & 10\\*
\fename{Goal} &  &  & 2  &  &  &  &  & 2\\*
\fename{Direction} &  &  & 5  & 2  &  &  &  & 7\\\midrule
%\multicolumn{9}{l}{\textit{snake} } \\
%\fename{Theme} & 11  &  &  &  &  &  &  & 11\\
%\fename{Area} &  &  & 1  &  & 1  &  &  & 2\\
%\fename{Direction} &  &  & 2  & 1  &  &  &  & 3\\
%\fename{Path} &  &  & 2  &  &  &  &  & 2\\
%\fename{Source} &  &  & 5  & 1  &  &  &  & 6\\
%\fename{Goal} &  &  & 1  &  &  &  &  & 1\\
%\midrule
%\multicolumn{9}{l}{\textit{meander} } \\
%\fename{Theme} & 11  &  &  &  &  &  &  & 11\\
%\fename{Area} &  &  &  & 1  & 2  &  &  & 3\\
%\fename{Goal} &  &  & 3  &  &  & 1  &  & 4\\
%\fename{Path} &  &  & 7  &  &  &  &  & 7\\
%\midrule
%\multicolumn{9}{l}{\textit{undulate} } \\
%\fename{Theme} & 1  &  &  &  &  &  &  & 1\\
%\fename{Path} &  &  & 1  &  &  &  &  & 1\\
%\midrule
\multicolumn{9}{l}{\textit{weave} } \\*
\fename{Theme} & 27  &  &  &  &  &  &  & 27\\*
\fename{Area} &  &  & 3  &  &  &  &  & 3\\*
\fename{Source} &  &  & 1  &  &  &  &  & 1\\*
\fename{Path} &  & 1  & 20  &  & 1  &  &  & 22\\*
\fename{Goal} &  &  & 1  &  &  &  &  & 1\\*
\fename{Direction} &  &  & 4  &  &  &  &  & 4\\
%\multicolumn{9}{l}{\textit{wind} } \\
%\fename{Theme} & 8  &  &  &  &  &  &  & 8\\
%\fename{Direction} &  &  & 2  &  &  &  &  & 2\\
%\fename{Path} &  &  & 8  &  &  &  &  & 8\\
%\midrule
%\multicolumn{9}{l}{\textit{zigzag} } \\
%\fename{Theme} & 13  &  &  &  &  &  &  & 13\\
%\fename{Area} &  &  & 1  & 1  &  &  &  & 2\\
%\fename{Goal} &  &  &  & 1  &  &  &  & 1\\
%\fename{Path} &  &  & 6  &  & 4  &  &  & 10\\
%\fename{Source} &  &  & 2  &  &  &  &  & 2\\
%\midrule
%\multicolumn{9}{l}{\textit{circle} } \\
%%\fename{Theme} & 16  &  &  &  &  &  &  & 16\\
%\fename{Area} &  &  & 1  & 1  &  &  &  & 2\\
%\fename{Distance} &  &  &  & 2  &  &  &  & 2\\
%\fename{Path} &  & 8  & 1  &  & 6  &  &  & 15\\
%\fename{Direction} &  &  &  & 2  &  &  &  & 2\\
%\midrule
%\multicolumn{9}{l}{\textit{spiral} } \\
%\midrule
%\multicolumn{9}{l}{\textit{swing} } \\
%\fename{Theme} & 3  &  &  &  &  &  &  & 3\\
%\fename{Area} &  &  &  & 1  &  &  &  & 1\\
%\fename{Direction} &  &  & 1  &  &  &  &  & 1\\
%\fename{Path} &  &  & 1  &  &  &  &  & 1\\
%\fename{Goal} &  &  & 1  &  &  &  &  & 1\\
%\fename{Source} &  &  & 1  &  &  &  &  & 1\\
%\midrule
%\multicolumn{9}{l}{\textit{travel} } \\
%\fename{Theme} & 2  &  &  &  &  &  &  & 2\\
%\fename{Goal} &  &  & 1  &  &  &  &  & 1\\
%\fename{Source} &  &  & 1  &  &  &  &  & 1\\
%\midrule
%\multicolumn{9}{l}{\textit{come} } \\
%\fename{Theme} & 5  &  &  &  &  &  &  & 5\\
%\fename{Direction} &  &  & 2  &  &  &  &  & 2\\
%\fename{Distance} &  &  &  & 1  &  &  &  & 1\\
%\fename{Goal} &  &  & 1  &  &  &  &  & 1\\
%\fename{Source} &  &  & 1  &  &  &  &  & 1\\
\lspbottomrule
\end{longtable}
  }
%\end{table}

%The predominant valence patterns in Bulgarian are similar, although it seems from the data that the \textbf{Addressee} co-occurs readily with \textbf{Messages} (last example above).


The other verbs tend to express either the \fename{Path} or another aspect of the route, usually with prevalence of the former.

\fename{Path} or \fename{Goal}: The preference of either \fename{Path}- or \fename{Goal}-phrases (but not any other type of phrase) is typical of the verb \textit{roll}: the \fename{Path}-expressions outnumber the \fename{Goal}-expressions by two to one. 

\fename{Path}, \fename{Goal} or \fename{Source}: \textit{Blow} and \textit{drift} %and \textit{float} 
exhibit preference for either \fename{Path}- or \fename{Goal}-phrases but also tend to express the \fename{Source} more often than most of the remaining verbs evoking the \framename{Motion} frame. However, with \textit{drift}, \fename{Path} is the predominant type of expression, with \fename{Goals} and \fename{Sources} being half as few, while with \textit{blow} the three parts of the route are represented equally. 

The three other motion FEs -- \fename{Direction}, \fename{Distance} and \fename{Area} -- are not usually discussed separately in the literature. By virtue of its definition, \fename{Area} occurs in competition with the elements of the route. The rationale is that it describes motion encompassing an expanse that is not construed in terms of a discreet trajectory. \fename{Areas} are not equally represented across manner of motion verbs as a whole, but are typical for some of them. \fename{Direction} and \fename{Distance} are represented by just a few examples across different verbs and have much poorer inventories. They still do need to be considered as separate FEs, as (i) some verbs incorporate them (e.g. \textit{descend}, \textit{rise} incorporate \fename{Direction}), and (ii) they may be independently expressed syntactically (Example \ref{ex01}). 


%\begin{exe} \label{ex:01} 
%\ex  \label{bg:diff-vb}
%    \settowidth \jamwidth{(bg)} 
%\gll {[\ ]}$_{\text{THEME-CNI}}$ \textit{вървете} [\textit{в} \textit{комуникационната} %\textit{зала}]$_{\feinsub{Goal}}$. \\

\begin{exe}
\ex  \label{ex01}
  %  \settowidth \jamwidth{(en)} 
[\textit{The storm}]$_{\feinsub{Thm}}$ \textit{\textbf{was MOVING}} [\textit{north}]$_{\feinsub{Dir}}$ [\textit{along the coast}]$_{\feinsub{Path}}$.
 %\jambox{(en)}
\end{exe}

\fename{Path}, \fename{Direction} or \fename{Area}: This pattern is represented by the verb \textit{swerve}, which describes motion involving a complex route characterised by an abrupt change in direction from an imaginary straight line or course. Respectively, it tends to co-occur with \fename{Path}-expressions as well as with ones denoting the newly assumed \fename{Direction}. As this kind of motion may encompass a broader spatial region, the FE \fename{Area} is also more frequently expressed than with other verbs. 

\fename{Path}, \fename{Area} or \fename{Goal}: The verb \textit{float} denotes a manner of motion which is brought about by the movement of a fluid. As this type of motion tends to encompass an expanse of the medium where it takes place, \fename{Area}-expressions are much more typical than with the rest of the verbs evoking the frame -- almost on a par with \fename{Paths} and more than \fename{Goals}.

\fename{Path}, \fename{Goal} or \fename{Direction}: This pattern is exemplified by the verb \textit{slide}. While it is expected to co-occur with \fename{Path} (like \textit{glide}), the verb also shows a tendency to express directionality either by means of \fename{Goal}-phrases, which in the data are represented on par with \fename{Paths}, or by means of the FE \fename{Direction}. 


\subsubsection{FrameNet valence patterns}

\tabref{tab:4:motion-valence-framenet} sums up the most frequent valence patterns represented among verbs evoking the \framename{Motion} frame, understood as combinations of FEs which co-occur syntactically, including null instantiations.\footnote{Non-core FEs are not considered in the analysis.}

The patterns corroborate the prominence of the \fename{Path} FE expressed predominantly as a prepositional phrase, followed by indefinite null instantiations (INIs), noun phrases and adverbial phrases. The second most frequent pattern involves the \fename{Goal}, followed by  \fename{Area}- and \fename{Source}-phrases. It is also notable that the simultaneous expression of two route-related FEs is much rarer.

\begin{table}
   \begin{tabularx}{\textwidth}{lrQ} 
   \lsptoprule
   Pattern & \# & Verbs \\ \midrule\relax
{[NP.Ext]}$_{\feinsub{Thm}}$ {[PP]}$_{\feinsub{Path}}$  & 134 & \textit{move, meander, go, roll, snake, float, undulate, zigzag, coast, fly, slide, swerve, glide, blow, circle, weave, drift, wind} \\ \relax
{[NP.Ext]}$_{\feinsub{Thm}}$ {[PP]}$_{\feinsub{Goal}}$  & 60 & \textit{move, fly, slide, meander, go, roll, soar, swerve, come, blow, float, drift}\\ 
{[NP.Ext]}$_{\feinsub{Thm}}$ {[\_]}$_{\feinsub{Path-INI}}$  & 40 & \textit{coast, move, go, soar, glide, blow, float, circle, weave, drift, zigzag}\\ 
{[NP.Ext]}$_{\feinsub{Thm}}$ {[PP]}$_{\feinsub{Area}}$  & 29 & \textit{move, fly, slide, go, roll, snake, swerve, blow, float, weave, drift}\\ 
{[NP.Ext]}$_{\feinsub{Thm}}$ {[PP]}$_{\feinsub{Src}}$  & 28 & \textit{move, slide, snake, swerve, come, glide, blow, float, drift, zigzag}\\ 
{[NP.Ext]}$_{\feinsub{Thm}}$ {[NP.Obj]}$_{\feinsub{Path}}$  & 14 & \textit{coast, move, roll, soar, circle, weave}\\ 
{[NP.Ext]}$_{\feinsub{Thm}}$ {[PP]}$_{\feinsub{Dir}}$  & 11 & \textit{swerve, come, weave}\\ 
{[NP.Ext]}$_{\feinsub{Thm}}$ {[AVP]}$_{\feinsub{Path}}$  & 11 & \textit{move, soar, glide, blow, float, drift}\\ 
{[NP.Ext]}$_{\feinsub{Thm}}$ {[AVP]}$_{\feinsub{Goal}}$  & 9 & \textit{go, roll, blow}\\ 
\lspbottomrule
\end{tabularx}
    \caption{FrameNet valence patterns of \framename{Motion} verbs}
    \label{tab:4:motion-valence-framenet}
\end{table} 

\subsubsection{Syntactic realisation of \framename{Motion} verbs in Bulgarian}

The list of Bulgarian verbs evoking the \framename{Motion} frame includes the Bulgarian counterparts of the verbs considered above. In particular, it features (i) a couple of verbs of directed motion which belong to the central part of the motion lexis -- \textit{идвам} `come’, \textit{отивам} `go’ and their perfective aspect counterparts\footnote{For brevity only the imperfective members of aspectual pairs will be listed in the text. The annotated examples include verbs of both aspects, where such exist.} (on the deictic aspects of these verbs, cf.  \cite{Nitsolova1984,Trifonova1982,Stanisheva1985}, among others), and (ii) a number of non-directed motion verbs, predominantly ones describing various manners of motion, such as \textit{вия се} `wind’, `weave’, \textit{духам} `blow’, \textit{летя} `fly’, \textit{лъкатуша} `meander’, \textit{нося се} `drift’, `float’, \textit{плувам}, \textit{плавам} `float’, \textit{кръжа}, \textit{обикалям} `circle’, \textit{търкалям се} `roll’, etc., as well as ones denoting the general idea of moving through space, such as \textit{движа се} `move’, `locomote’ and \textit{пътувам} `travel’.

A selection of corpus examples has been collected for several of them (verbs having correspondences among the English predicates represented in \tabref{tab:4:motion-synt}), and annotated with the core FEs: \textit{вия се} `wind’, `weave’, \textit{движа се} `move’, \textit{нося се} `drift’, `float’, \textit{отивам} `go’, \textit{търкалям се} `roll’. Although on a smaller scale, the results, shown in \tabref{tab:4:motion-synt-bg}, are consistent with the observations on the FrameNet corpus. In particular, \textit{отивам} shows a very strong preference for \fename{Goal}-phrases similarly to \textit{go} (Example \ref{ex:02}), while the rest of the verbs (Examples \ref{ex:03}--\ref{ex:06}) favour \fename{Paths}, with different proportions of other FEs, in particular \fename{Areas} for \textit{нося се} `float’, `drift’ and \fename{Goals} for \textit{търкалям се} `roll’.

\begin{exe}
\ex \label{ex:02} 
  %  \settowidth \jamwidth{(bg)} 
\gll \emph{[Те]}$_{\feinsub{Thm}}$ \textit{\textbf{ОТИВАТ}} \textit{право} [\textit{в} \textit{печатницата}]$_{\feinsub{Goal}}$. \\
They go-PRS.3PL straight to printer's-DEF. \\
\glt `They are going straight to the printer's.' 
\ex \label{ex:03} 
\gll [\textit{Кучетата}]$_{\feinsub{Thm}}$ {\textit{\textbf{СЕ}} \textit{\textbf{ДВИЖАТ}}} [\textit{по} \textit{мекия} \textit{сняг}]$_{\feinsub{Path}}$. \\
Dogs-DEF move-PRS.3PL across soft-DEF snow. \\
\glt `The dogs are moving across the soft snow.' 
\ex \label{ex:04} 
\gll [\textit{Топката}]$_{\feinsub{Thm}}$ {\textit{\textbf{СЕ}} \textit{\textbf{ТЪРКАЛЯ}}} [\textit{по} \textit{тревата}]$_{\feinsub{Path}}$. \\
Ball-DEF roll-PRS.3PL across grass-DEF. \\
\glt `The ball is rolling across the grass.' 
\ex \label{ex:05} 
\gll [\textit{Колата}]$_{\feinsub{Thm}}$ {\textit{\textbf{СЕ}} \textit{\textbf{ВИЕШЕ}}} [\textit{по} \textit{завоите}]$_{\feinsub{Path}}$. \\
Car-DEF wind-PST.3SG along turns-DEF. \\
\glt `The car was winding along the turns of the road.'
%\gll [\textit{Колата}]$_{\feinsub{Thm}}$ {\textit{се} \textit{виеше}} [\textit{по} \textit{завоите}]$_{\feinsub{Path}}$ [\textit{до} \textit{тях}]$_{\feinsub{Goal}}$. \\
%[Shadow-DEF]$_{\feinsub{Thm}}$ slid [across snow-DEF]$_{\feinsub{Path}}$ [to them]$_{\feinsub{Goal}}$. \\
\ex \label{ex:06} 
\gll [\textit{Туфа водорасли}]$_{\feinsub{Thm}}$ {\textit{\textbf{СЕ}} \textit{\textbf{НОСИ}}} [\textit{във} \textit{водата}]$_{\feinsub{Area}}$. \\
Clump-INDF seaweed float-PRS.3SG on water-DEF. \\
\glt `A clump of seaweed is floating on the water.' 
%\ex \label{ex:07} 
%\gll [\textit{От} \textit{Центъра}]$_{\feinsub{Src}}$ \textit{духаха} [\textit{облаци}]$_{\feinsub{Thm}}$. \\
%[From Centre-DEF]$_{\feinsub{Src}}$ blew [clouds]$_{\feinsub{Thm}}$. \\
%\glt `Clouds blew from the Centre.' 
%\ex \label{ex:08} 
%\gll [\textit{Градът}]$_{\feinsub{Thm}}$ \textit{плава} [\textit{по} \textit{езерото}]$_{\feinsub{Area}}$. \\ 
%[City-DEF]$_{\feinsub{Thm}}$ floats [on lake-DEF]$_{\feinsub{Area}}$. \\
%\glt `The city floats on the lake.' 
% \jambox{(bg)}
\end{exe}

{\small
\begin{longtable}{l ccccccccc}   
\caption{\label{tab:4:motion-synt-bg}Syntactic expression of the \framename{Motion} FEs in Bulgarian}\\
 \lsptoprule
  & NP.Ext & NP.Obj & PP & AVP & NI & Clause & Other & Total\\ \midrule \endfirsthead
  \midrule
  & NP.Ext & NP.Obj & PP & AVP & NI & Clause & Other & Total\\ \midrule \endhead
\multicolumn{9}{l}{\textit{вия се} `wind’, `weave’ }\\*
\fename{Theme} & 26 &  &  &  &  &  &  & 26\\*
\fename{Area} &  &  & 5 &  &  &  &  & 5\\*
\fename{Source} &  &  & 4 &  &  &  &  & 4\\*
\fename{Path} &  &  & 10 &  &  &  &  & 10\\*
\fename{Goal} &  &  & 3 &  &  &  &  & 3\\*
\fename{Direction} &  &  &  & 1 &  &  &  & 1\\
%\fename{Manner} &  &  & 1 & 1 &  &  &  & 2\\ 
 \midrule
\multicolumn{9}{l}{\textit{нося се} `float’, `drift’}\\*
\fename{Theme} & 32 &  &  &  &  &  &  & 32\\*
\fename{Area} &  &  & 9 &  &  &  &  & 9\\*
\fename{Source} &  &  & 5 & 1 &  &  &  & 6\\*
\fename{Path} &  &  & 8 & 1 &  &  &  & 9\\*
\fename{Goal} &  &  & 6 &  &  &  &  & 6\\*
\fename{Direction} &  &  & 2 & 1 &  &  &  & 3\\
 \midrule
\multicolumn{9}{l}{\textit{движа се} `move’ }\\*
\fename{Theme} & 31 &  &  &  &  &  &  & 31\\*
\fename{Area} &  &  & 1 &  &  &  &  & 1\\*
\fename{Source} &  &  & 1 &  &  &  &  & 1\\*
\fename{Path} &  &  & 18 & 1 &  &  &  & 19\\*
\fename{Goal} &  &  & 2 &  &  &  &  & 2\\*
\fename{Direction} &  &  & 3 &  &  &  &  & 3\\
%\fename{Manner} &  &  & 2 & 3 &  &  &  & 5\\*
\midrule
\multicolumn{9}{l}{\textit{търкалям се} `roll’}\\*
\fename{Theme} & 30 &  &  &  &  &  &  & 30\\*
\fename{Source} &  &  & 2 &  &  &  &  & 2\\*
\fename{Path} &  &  & 19 &  &  &  &  & 19\\*
\fename{Goal} &  &  & 7 &  &  &  &  & 7\\*
\fename{Direction} &  &  &  & 2 &  &  &  & 2\\
%\fename{Manner} &  &  &  & 2 &  &  &  & 2\\ 
 \midrule
%\multicolumn{9}{l}{\textit{търкулна се} }\\*
%\fename{Path} &  &  & 2 &  &  &  &  & 2\\*
%\fename{Goal} &  &  & 1 &  &  &  &  & 1\\*
%\fename{Theme} & 4 &  &  &  &  &  &  & 4\\ 
% \midrule
%\multicolumn{9}{l}{\textit{въртя се} }\\*
%\fename{Path} &  &  & 1 &  &  &  &  & 1\\*
%\fename{Theme} & 1 &  &  &  &  &  &  & 1\\ 
% \midrule
\multicolumn{9}{l}{\textit{отивам\slash отида} `go’ }\\*
\fename{Theme} & 28 &  &  &  &  &  &  & 28\\*
\fename{Source} &  &  & 1 &  &  &  &  & 1\\*
\fename{Path} &  &  & 2 &  &  &  &  & 2\\*
\fename{Goal} &  &  & 15 & 4 & 1 &  &  & 20\\*
\fename{Direction} &  &  &  & 3 &  &  &  & 3\\*
%\fename{Manner} &  &  &  & 1 &  &  &  & 1\\*
\fename{Distance} &  &  &  & 2 &  &  &  & 2\\*
 \lspbottomrule
\end{longtable}
}

\subsubsection{Valence patterns in the Bulgarian dataset}

The valence patterns in the Bulgarian dataset, represented in \tabref{tab:4:motion-valence-bg}, show similar results to the ones in the FrameNet corpus: in particular, a prevalence of patterns exhibiting PP \fename{Paths}, followed by a more modest representation of \fename{Goals} and \fename{Areas}. Among the several top valence patterns, combinations of \fename{Sources} and \fename{Goals} are also found.

\begin{table}
   \begin{tabularx}{\textwidth}{lrQ}
  \lsptoprule
   Pattern & \# & Verbs \\ \midrule
{[NP.Ext]}$_{\feinsub{Thm}}$ {[PP]}$_{\feinsub{Path}}$ & 55 & \textit{вия се, въртя се, движа се, нося се, търкалям се, отивам\slash отида}\\ 
{[NP.Ext]}$_{\feinsub{Thm}}$ {[PP]}$_{\feinsub{Goal}}$ & 25 & \textit{вия се, движа се, нося се, търкалям се, отивам\slash отида}\\ 
{[NP.Ext]}$_{\feinsub{Thm}}$ {[PP]}$_{\feinsub{Area}}$ & 15 & \textit{вия се, движа се, нося се}\\ 
{[NP.Ext]}$_{\feinsub{Thm}}$ {[\_]}$_{\feinsub{Path-INI}}$& 10 & \textit{вия се, движа се, нося се, отивам\slash отида, търкалям се}\\ 
{[NP.Ext]}$_{\feinsub{Thm}}$ {[PP]}$_{\feinsub{Src}}$ & 6 & \textit{вия се, нося се}\\ 
{[NP.Ext]}$_{\feinsub{Thm}}$ {[AVP]}$_{\feinsub{Dir}}$ & 5 & \textit{нося се, търкалям се, отивам\slash отида}\\ 
{[NP.Ext]}$_{\feinsub{Thm}}$ {[PP]}$_{\feinsub{Goal}}$ {[PP]}$_{\feinsub{Src}}$ & 5 & \textit{движа се, нося се, търкалям се, отивам\slash отида}\\ 
{[NP.Ext]}$_{\feinsub{Thm}}$ {[PP]}$_{\feinsub{Dir}}$ & 4 & \textit{движа се, нося се}\\ 
{[NP.Ext]}$_{\feinsub{Thm}}$ {[AVP]}$_{\feinsub{Goal}}$ & 4 & \textit{отивам\slash отида}\\ 
%{[NP.Ext]}$_{\feinsub{Thm}}$ {[AVP]}$_{\text{MANNER}}$ & 4 & \textit{движа се, търкалям се}\\ 
%{[NP.Ext]}$_{\feinsub{Thm}}$ {[PP]}$_{\text{MANNER}}$ & 3 & \textit{вия се, движа се}\\ 
%{[NP.Ext]}$_{\feinsub{Thm}}$ {[AVP]}$_{\feinsub{Distance}}$ & 2 & \textit{отивам\slash отида}\\ 
%{[NP.Ext]}$_{\feinsub{Thm}}$ {[AVP]}$_{\text{MANNER}}$ {[PP]}$_{\feinsub{Path}}$ & 2 & \textit{търкалям се, отивам\slash отида}\\ 
%{[NP.Ext]}$_{\feinsub{Thm}}$ {[AVP]}$_{\feinsub{Path}}$ & 2 & \textit{движа се, нося се}\\ 
%{[NP.Ext]}$_{\feinsub{Thm}}$ {[AVP]}$_{\feinsub{Src}}$ & 1 & \textit{нося се}\\ 
%{[NP.Ext]}$_{\feinsub{Thm}}$ {[PP]}$_{\feinsub{Dir}}$ {[PP]}$_{\feinsub{Goal}}$ {[PP]}$_{\feinsub{Src}}$ & 1 & \textit{нося се}\\ 
%{[NP.Ext]}$_{\feinsub{Thm}}$ {[AVP]}$_{\feinsub{Dir}}$ {[PP]}$_{\feinsub{Path}}$ & 1 & \textit{търкалям се}\\ 
%{[NP.Ext]}$_{\feinsub{Thm}}$ {[AVP]}$_{\text{MANNER}}$ {[PP]}$_{\feinsub{Goal}}$ {[PP]}$_{\feinsub{Src}}$ & 1 & \textit{вия се}\\ 
%{[NP.Ext]}$_{\feinsub{Thm}}$ {[\_]}$_{\feinsub{Goal-DNI}}$ & 1 & \textit{отивам\slash отида}\\ 
%{[NP.Ext]}$_{\feinsub{Thm}}$ {[AVP]}$_{\feinsub{Dir}}$ {[PP]}$_{\feinsub{Goal}}$ & 1 & \textit{вия се}\\ 
\lspbottomrule
\end{tabularx}
    \caption{FrameNet valence patterns of \framename{Motion} verbs in Bulgarian}
    \label{tab:4:motion-valence-bg}
\end{table}

%\section{Frames in the Motion domain}

\subsection{\framename{Self\_motion}}

\framename{Self\_motion} is an elaboration of the \framename{Motion} frame (and related to it by means of an \textbf{Inheritance} relation) which involves a similar configuration of core FEs with some further restrictions. 

\subsubsection{Semantic description of the \framename{Self\_Motion} frame}

Frame definition: The \fename{Self\_mover}, a living being, moves under its own direction along a \fename{Path}. Alternatively or in addition to \fename{Path}, an \fename{Area}, \fename{Direction}, \fename{Source}, or \fename{Goal} for the movement may be mentioned.

The most important distinction and the one that primarily motivates the differentiation between \framename{Motion} and \framename{Self\_motion} is the capability of the \fename{Self\_mover} to change location by exercising their own will and power by the coordinated movement of their bodies,\footnote{\url{https://framenet2.icsi.berkeley.edu/fnReports/data/frameIndex.xml?frame=Self_motion}} which is not a necessity with the \framename{Motion} \fename{Theme}. By metaphorical extension, \fename{Self\_movers} may be self-directed entities such as vehicles. %Self\_motion inherits both the frames Motion and Intentionally\_act, which also points to the agent-like quality of its main participant. 
The remaining core FEs in this frame are the ones defining the elements and aspects of the route of movement.\footnote{\fename{Distance} is not defined as a core FE, but will be treated on a par with its equivalent in the mother frame.} %(\fename{Source}, \fename{Goal}, \fename{Path}), \fename{Direction} and \fename{Distance}

Core~ FEs~ in~ the~ \framename{Self\_motion}~ frame:~ \fename{Self\_mover},~ \fename{Source},~ \fename{Goal}, \fename{Path}, \fename{Area}, \fename{Direction}, \fename{Distance}.
\fename{Self\_mover} is the entity (living being or a vehicle) that changes location under its own power and direction. Its semantic specification includes \textbf{animate beings} and \textbf{vehicles}. The remaining core FEs have the same semantic specification as their counterparts in the \framename{Motion} frame from which they are inherited. 
%: \fename{Source}, \fename{Goal}, \fename{Path}, \fename{Area}, \fename{Direction}, \fename{Distance}.

%The various manners of motion specified by the verbs in this frame characterise  

%\subsubsubsection*{\textbf{Syntactic properties of the \framename{Self\_motion} frame}}

\subsubsection{Verbs evoking the \framename{Self\_motion} frame}

Unlike its parent frame, \framename{Self\_motion} prototypically describes individuals capable of applying their own will and bodies to perform the motion. The verbs thus encode various aspects of motion impossible for inanimate beings. These involve modes of motion: (i) characteristic of different organisms, e.g. \textit{fly}, \textit{swim}, \textit{crawl}, \textit{slither}, \textit{walk}, etc.; (ii) requiring different configuration of the body: \textit{slouch}, \textit{shoulder}; (iii) (lack of ) purposefulness: \textit{roam}, \textit{saunter}, \textit{wander}; (iv) intent: \textit{prowl}, \textit{hike}, \textit{hitchhike}; (v) different kinds of steps, speed, weight or force applied: \textit{mince}, \textit{scurry}, \textit{shuffle}, \textit{plod}, \textit{trample}, \textit{run}, \textit{jog}, \textit{hop}, etc. 

\subsubsection{Syntactic realisation of the frame elements in the \framename{Self\_motion} frame}

The expression of the core FEs according to syntactic categories and syntactic function is similar to those of the corresponding FEs in the \framename{Motion} frame. The \fename{Self\_mover} is realised as the external argument; the remaining core FEs are typically realised as prepositional or adverbial phrases. %\fename{Distance} may also be expressed by measurement NPs. 

%The syntactic realisations of the various locative aspects of motion shows that they are not equal with respect to the verbs’ preference for their expression. 

\tabref{tab:4:self-motion-synt} illustrates the syntactic expression of the core FEs for several English verbs with the highest number of attestations in the FrameNet corpus. The verbs evoking these semantic frames further extend the observations made for the \framename{Motion} frame with respect to the tendency for the various verbs to co-occur with motion expressions. Overall, the \fename{Path} is the prevalent FE to be expressed, followed by \fename{Goals}, \fename{Areas} and \fename{Sources} in descending order (see the valence patterns in \tabref{tab:4:self-motion-valence-framenet}).

\begin{description}[font=\normalfont]
\item[\fename{Path}:] Several verbs exhibit a strong preference for \fename{Paths} over any other core FE: \textit{amble}, \textit{drive}, \textit{make}, \textit{plod}.
\item[\fename{Path or Goal}:] Verbs that show preference to co-occur with either of these FEs can be further distinguished into two patterns.

\item[The first one is \fename{Path} > \fename{Goal}:] In this case, the examples with \fename{Path} show prevalence, amounting to around or even more than half of the examples, and \fename{Goals} usually account for a quarter to a third, rarely more, see \textit{hop} in \tabref{tab:4:self-motion-synt}. This pattern is further illustrated by \textit{hurry}, \textit{jog}, \textit{limp}, \textit{lumber}, \textit{lunge}, \textit{lurch}, \textit{proceed}, \textit{skip}, \textit{stagger}, \textit{stroll}, \textit{stumble}, \textit{swagger}, \textit{totter}, \textit{trot}, \textit{trumble}, \textit{trek}.

\item[\fename{Path} = \fename{Goal}:] With the second pattern, there is no marked preference for one FE over the other, as exemplified by \textit{walk} (\tabref{tab:4:self-motion-synt}). Other verbs which pattern in a similar way are: \textit{barge}, %\textit{bustle},
\textit{clamber}, \textit{dash}, \textit{head}, \textit{hasten}, \textit{pad}, \textit{romp}, \textit{sidle}, \textit{toddle}, \textit{wade}.
       
\item[\fename{Path}, \fename{Goal} or \fename{Source}:] This pattern is distinguished from the second subgroup of the previous one empirically on the basis of the greater ratio of \fename{Sources} against the overall number of examples for each of the verbs. The verbs in this group tend to co-occur with expressions denoting any of the three parts of the route more consistently than the remaining verbs evoking the \framename{Self\_motion} fra\-me. As already shown in the \framename{Motion} frame, the frequency of each of these FEs is not equal across verbs. In this group one finds that \fename{Paths} account for half to up to two-thirds of the examples, \fename{Goals} -- for a quarter to a third of the examples, \fename{Sources} -- usually for a fifth to a quarter of the examples, as illustrated by \textit{crawl} (\tabref{tab:4:self-motion-synt}), \textit{creep}, \textit{dart}, \textit{march}, \textit{saunter}, \textit{scamper}, \textit{scramble}, \textit{shuffle}, \textit{spring}, \textit{sprint}, \textit{stride}, \textit{trudge}. Another variation is represented by the verbs \textit{lope}, \textit{leap}, \textit{jump}, where \fename{Paths} account for half or more of the examples, and \fename{Goals} and \fename{Sources} are on a par, about one third of the instances. %\textit{stomp} - on a par all of them
%\textit{scuttle}, \textit{dart}

\item[\fename{Paths} = \fename{Goals} or \fename{Source}:] This pattern shows no marked difference between \fename{Paths} and \fename{Goals} with a weaker preference for \fename{Sources}: \textit{climb} (\tabref{tab:4:self-motion-synt}), \textit{rush}, \textit{scuttle}.

\item[\fename{Goal}:] A couple of verbs, such as \textit{file} and \textit{pounce} show marked preference for \fename{Goal}-expressions over all other motion-related FEs. 

\item[\fename{Goal} or \fename{Path}:] These verbs tend to co-occur with both \fename{Goals} and \fename{Paths} with a prevalence of the former (about a half of the examples) to the latter (around a third of the examples): \textit{steal}, \textit{run} (\tabref{tab:4:self-motion-synt}).
							
\item[\fename{Goal}, \fename{Path} or \fename{Source}:] This combination is exemplified by verbs such as \textit{troop}, \textit{sneak} (\tabref{tab:4:self-motion-synt}), \textit{stalk}. The \fename{Goals} amount to half or more of the instances, while the \fename{Paths} and \fename{Sources} are fewer: around one-third of the examples for \fename{Path} and a quarter for \fename{Source} with \textit{troop}, and equally distributed between the two FEs for \textit{sneak} and \textit{stalk}. %
\end{description}

The FE \fename{Area} usually alternates with expressions denoting one or another element or aspect of the route of a moving entity and as a whole accounts for far fewer cases than \fename{Paths} and \fename{Goals} in the frame. For some verbs, however, it is either the preferred motion expression or is much more frequent than with most verbs. This characteristic is typical of verbs that describe motion that encompasses or spreads over a larger region or expanse.

\begin{description}[font=\normalfont]
\item[\fename{Area}:] The verbs \textit{traipse} and \textit{skulk} show a much more marked preference for \fename{Areas} than for other motion-related FEs: half of the instances for \textit{traipse}, two-thirds for \textit{skulk}.

\item[\fename{Area} or \fename{Path}:] Other verbs tend to co-occur with either \fename{Areas} or \fename{Paths} with a prevalence of the former (half or more of the examples) to the latter (around one-third of the examples): \textit{prance}, \textit{prowl}, \textit{roam} (\tabref{tab:4:self-motion-synt}).
%area - 1/2 or more %path - around a third

\item[\fename{Path} or \fename{Area}:] The opposite is observed with \textit{strut} and \textit{flit} where \fename{Paths} are preferred (between half and two-thirds of the examples) to \fename{Areas} (a quarter of the examples).
											
\item[\fename{Path}, \fename{Area} or \fename{Goal}:] This pattern shows prevalence of \fename{Paths} (with half or more of the instances), a substantial number (a quarter to one-third) of \fename{Areas} and a smaller number (one-sixth to one-fifth of the examples) of \fename{Goals}: \textit{dance}, \textit{pace}, \textit{swim} (\tabref{tab:4:self-motion-synt}), \textit{tread}, \textit{tramp}. In the case of \textit{fly} the number of \fename{Areas} and \fename{Goals} is equal.

%\textit{bustle} % path, goal 1/3; area 1/5
            
\item[\fename{Path}, \fename{Goal}, \fename{Area} or \fename{Source}:] This pattern shows prevalence of \fename{Paths} (ar\-ound half of the examples), with various distributions (between one-fifth and one-third) of the other three FEs: \textit{scurry}, \textit{slither}, \textit{waddle}, \textit{wander} (\tabref{tab:4:self-motion-synt}).
\end{description}

A couple of verbs, such as \textit{flounce} and \textit{storm}, show preference to \fename{Sources} over other motion-related FEs. 

%\fename{Direction}

   
%stagger, skip					
            
%\begin{table}
%\centering
{\footnotesize
\begin{longtable}{l ccccccccc}
 \caption{\label{tab:4:self-motion-synt}Syntactic expression of the \framename{Self\_motion} FEs in FrameNet}
    \\
    \lsptoprule
& NP.Ext & NP.Obj & PP & AVP & NI & Clause & Other & Total\\ \midrule\endfirsthead
\midrule
& NP.Ext & NP.Obj & PP & AVP & NI & Clause & Other & Total\\ \midrule\endhead
\multicolumn{9}{l}{\textit{climb} } \\*
\fename{Self\_mover} & 115  &  &  &  &  &  &  & 115\\*
\fename{Area} &  & 1  & 2  &  & 2  &  &  & 5\\*
\fename{Source} &  &  & 21  &  &  &  &  & 21\\*
\fename{Path} &  & 1  & 46  & 3  & 4  &  &  & 54\\*
\fename{Goal} &  &  & 59  & 1  &  &  &  & 60\\
\midrule
\multicolumn{9}{l}{\textit{crawl} } \\*
\fename{Self\_mover} & 140  &  &  &  &  &  &  & 140\\*
\fename{Area} &  &  & 18  &  & 4  &  &  & 22\\*
\fename{Source} &  &  & 23  & 6  &  &  &  & 29\\*
\fename{Path} &  &  & 58  & 7  & 9  &  & 1 & 75\\*
\fename{Goal} &  &  & 31  & 4  &  & 1  &  & 36\\
\midrule
%\multicolumn{9}{l}{\textit{dance} } \\		
%\fename{Self\_mover} & 80  &  &  &  &  &  &  & 80\\		
%\fename{Area} &  &  & 23  & 3  & 1  &  & 1 & 28\\		
%\fename{Source} &  &  & 8  & 4  &  &  &  & 12\\		
%\fename{Path} &  &  & 46  &  &  &  &  & 46\\		
%\fename{Goal} &  &  & 15  &  &  &  &  & 15\\		
%\midrule
%\multicolumn{9}{l}{\textit{dart} } \\		
%\fename{Self\_mover} & 72  &  &  &  &  &  &  & 72\\		
%\fename{Area} &  &  & 12  &  & 1  &  &  & 13\\		
%\fename{Goal} &  &  & 25  & 2  &  &  &  & 27\\		
%\fename{Source} &  &  & 20  & 4  &  &  &  & 24\\		
%\fename{Path} &  &  & 26  & 6  &  &  &  & 32\\		
%\midrule
%\multicolumn{9}{l}{\textit{dash} } \\		
%\fename{Self\_mover} & 62  &  &  &  &  &  &  & 62\\		
%\fename{Area} &  &  & 8  &  &  &  &  & 8\\		
%\fename{Path} &  &  & 29  & 2  &  &  &  & 31\\		
%\fename{Goal} &  &  & 25  & 4  &  &  &  & 29\\		
%\fename{Source} &  &  & 12  & 1  &  &  &  & 13\\		
%\midrule
%\multicolumn{9}{l}{\textit{hobble} } \\		
%\fename{Self\_mover} & 73  &  &  &  &  &  &  & 73\\		
%\fename{Area} &  &  & 15  &  &  &  & 1 & 16\\		
%\fename{Goal} &  &  & 26  & 3  &  &  & 1 & 30\\		
%\fename{Path} &  &  & 21  &  & 4  &  &  & 25\\		
%\fename{Source} &  &  & 8  & 3  &  &  &  & 11\\		
%\midrule
\multicolumn{9}{l}{\textit{hop} } \\*
\fename{Self\_mover} & 103  &  &  &  &  &  &  & 103\\*
\fename{Area} &  &  & 14  &  & 1  &  &  & 15\\*
\fename{Source} &  &  & 9  & 2  &  &  &  & 11\\*
\fename{Path} &  &  & 50  & 8  & 2  &  & 2 & 62\\*
\fename{Goal} &  &  & 28  & 4  &  &  &  & 32\\
\midrule
\multicolumn{9}{l}{\textit{hurry} } \\		
\fename{Self\_mover} & 74  &  &  &  &  &  &  & 74\\		
\fename{Source} &  &  & 10  & 2  &  &  &  & 12\\
\fename{Path} &  &  & 41  & 7  & 2  &  &  & 50\\		
\fename{Goal} &  &  & 28  & 8  &  &  &  & 36\\		
\midrule
%\multicolumn{9}{l}{\textit{leap} } \\		
%\fename{Self\_mover} & 85  &  &  &  &  & 1  &  & 86\\		
%\fename{Area} &  &  & 8  &  & 2  &  &  & 10\\		
%\fename{Source} &  &  & 22  &  &  &  &  & 22\\		
%\fename{Path} &  &  & 26  & 4  & 6  &  & 7 & 43\\		
%\fename{Goal} &  &  & 21  & 1  &  &  & 2 & 24\\		
%\midrule
%\multicolumn{9}{l}{\textit{limp} } \\		
%\fename{Self\_mover} & 64  &  &  &  &  &  &  & 64\\		
%\fename{Area} &  &  & 7  &  & 1  &  &  & 8\\		
%\fename{Goal} &  &  & 16  & 4  &  &  & 1 & 21\\		
%\fename{Path} &  &  & 22  & 2  & 7  &  &  & 31\\		
%\fename{Source} &  &  & 6  & 2  &  &  &  & 8\\		
%\midrule
%\multicolumn{9}{l}{\textit{march} } \\		
%\fename{Self\_mover} & 96  &  & 1  &  &  &  &  & 97\\		
%\fename{Area} &  &  & 5  &  & 1  &  &  & 6\\		
%\fename{Goal} &  &  & 27  & 1  &  &  & 2 & 30\\		
%\fename{Path} & 1  &  & 47  & 9  & 9  &  & 2 & 68\\		
%\fename{Source} &  &  & 17  & 3  &  &  &  & 20\\		
%\midrule
%\multicolumn{9}{l}{\textit{plod} } \\		
%\fename{Self\_mover} & 79  &  &  &  &  & 3  &  & 82\\		
%\fename{Area} &  &  & 6  &  &  &  &  & 6\\		
%\fename{Path} &  &  & 45  & 6  & 16  &  & 1 & 68\\		
%\fename{Source} &  &  & 3  & 1  &  &  &  & 4\\		
%\fename{Goal} &  &  & 9  & 3  &  &  &  & 12\\		
%\midrule
\multicolumn{9}{l}{\textit{roam} } \\*
\fename{Self\_mover} & 66  &  &  &  &  &  &  & 66\\*
\fename{Area} &  & 13 & 26  & 1  & 5  &  & & 45\\*
\fename{Source} &  &  & 1  &  &  &  &  & 1\\*
\fename{Path} &  &  & 13  & 2  & 3  &  &  & 18\\*
\fename{Goal} &  &  & 2  & 2  &  &  &  & 4\\
\midrule
\multicolumn{9}{l}{\textit{run} } \\*
\fename{Self\_mover} & 64  &  &  &  &  &  &  & 64\\*
\fename{Area} &  &  & 1  & 3  & 2  &  &  & 6\\*
\fename{Source} &  &  & 3  & 1  & 2  &  &  & 6\\*
\fename{Path} &  &  & 16  &  & 3  &  &  & 19\\*
\fename{Goal} &  &  & 16  & 3  & 9  &  &  & 28\\*
\fename{Direction} &  &  & 4  & 2  & 2  &  &  & 8\\
\midrule
%\multicolumn{9}{l}{\textit{rush} } \\		
%\fename{Self\_mover} & 146  &  &  &  &  &  &  & 146\\		
%\fename{Area} &  &  & 9  &  &  &  &  & 9\\		
%\fename{Path} &  &  & 60  & 10  & 3  &  &  & 73\\		
%\fename{Goal} &  &  & 54  & 14  &  &  &  & 68\\		
%\fename{Source} &  &  & 24  & 4  &  &  &  & 28\\		
%\fename{Direction} &  &  & 1  & 1  &  &  &  & 2\\		
%\midrule
%\multicolumn{9}{l}{\textit{scamper} } \\		
%\fename{Self\_mover} & 72  &  &  &  &  &  &  & 72\\		
%\fename{Area} &  &  & 10  &  &  &  &  & 10\\		
%\fename{Path} &  &  & 27  & 2  & 6  &  & 2 & 37\\		
%\fename{Goal} &  &  & 23  & 2  &  &  &  & 25\\		
%\fename{Source} &  &  & 11  & 4  &  &  &  & 15\\		
%\midrule
%\multicolumn{9}{l}{\textit{scramble} } \\		
%\fename{Self\_mover} & 86  &  &  &  & 1  &  &  & 87\\		
%\fename{Area} &  &  & 2  & 1  &  &  &  & 3\\		
%\fename{Path} & 1  & 3  & 33  &  & 16  &  &  & 53\\		
%\fename{Source} &  &  & 15  & 1  &  &  &  & 16\\		
%\fename{Goal} &  & 1  & 21  & 4  &  &  &  & 26\\		
%\midrule
%\multicolumn{9}{l}{\textit{scurry} } \\		
%\fename{Self\_mover} & 90  &  &  &  &  &  &  & 90\\		
%\fename{Area} &  &  & 19  &  &  &  &  & 19\\		
%\fename{Path} &  &  & 34  & 2  & 8  &  &  & 44\\		
%\fename{Goal} &  &  & 22  & 1  &  &  &  & 23\\		
%\fename{Source} &  &  & 10  & 6  &  &  &  & 16\\		
%\midrule
%\multicolumn{9}{l}{\textit{scuttle} } \\		
%\fename{Self\_mover} & 72  &  &  &  &  &  &  & 72\\		
%\fename{Area} &  &  & 7  &  &  &  &  & 7\\		
%\fename{Path} &  &  & 23  & 2  & 6  &  &  & 31\\		
%\fename{Source} &  &  & 18  & 3  &  &  &  & 21\\		
%\fename{Goal} &  &  & 25  & 2  &  &  &  & 27\\		
%\midrule
%\multicolumn{9}{l}{\textit{shuffle} } \\		
%\fename{Self\_mover} & 91  &  &  &  &  &  &  & 91\\		
%\fename{Area} &  &  & 9  &  &  &  &  & 9\\		
%\fename{Path} &  &  & 32  & 6  & 7  &  &  & 45\\		
%\fename{Goal} &  &  & 28  & 2  &  &  &  & 30\\		
%\fename{Source} &  &  & 15  & 4  &  &  &  & 19\\		
%\midrule
%\multicolumn{9}{l}{\textit{skip} } \\		
%\fename{Self\_mover} & 82  &  &  &  &  &  &  & 82\\		
%\fename{Area} &  &  & 8  &  &  &  &  & 8\\		
%\fename{Path} &  &  & 53  & 1  & 2  &  & 1 & 57\\		
%\fename{Goal} &  &  & 19  & 3  &  &  &  & 22\\		
%\fename{Source} &  &  & 7  & 6  &  &  &  & 13\\		
%\midrule
\multicolumn{9}{l}{\textit{sneak} } \\*
\fename{Self\_mover} & 68  &  &  &  &  &  &  & 68\\*
\fename{Area} &  &  & 2  &  &  &  &  & 2\\*
\fename{Source} &  &  & 17  & 4  &  &  &  & 21\\*
\fename{Path} &  &  & 20  &  &  &  &  & 20\\*
\fename{Goal} &  & 1  & 37  & 6  &  &  &  & 44\\
\midrule
%\multicolumn{9}{l}{\textit{sprint} } \\		
%\fename{Self\_mover} & 75  &  &  &  &  &  &  & 75\\		
%\fename{Path} &  &  & 45  & 2  & 6  & 1  &  & 54\\		
%\fename{Source} &  &  & 15  & 2  &  &  &  & 17\\		
%\fename{Goal} &  & 1  & 14  & 2  &  &  &  & 17\\		
%\midrule
%\multicolumn{9}{l}{\textit{stagger} } \\		
%\fename{Self\_mover} & 135  &  &  &  &  &  &  & 135\\		
%\fename{Area} &  &  & 12  & 4  &  &  &  & 16\\		
%\fename{Path} &  &  & 29  & 14  & 29  &  &  & 72\\		
%\fename{Goal} &  &  & 44  & 4  &  & 1  &  & 49\\		
%\fename{Source} &  &  & 20  & 2  &  &  &  & 22\\		
%\midrule
%\multicolumn{9}{l}{\textit{step} } \\		
%\fename{Self\_mover} & 153  &  &  &  &  &  &  & 153\\		
%\fename{Area} &  &  & 4  &  &  &  &  & 4\\		
%\fename{Path} &  & 1  & 54  & 12  & 1  & 2  &  & 70\\		
%\fename{Direction} &  &  &  & 1  &  &  &  & 1\\		
%\fename{Source} &  &  & 38  & 4  &  &  &  & 42\\		
%\fename{Goal} &  & 2  & 74  & 8  & 1  &  &  & 85\\		
%\midrule
%\multicolumn{9}{l}{\textit{stride} } \\		
%\fename{Self\_mover} & 113  &  &  &  &  &  &  & 113\\		
%\fename{Area} &  &  & 7  &  &  &  &  & 7\\		
%\fename{Path} &  &  & 66  & 9  & 8  &  &  & 83\\		
%\fename{Source} &  &  & 20  & 6  &  &  &  & 26\\		
%\fename{Goal} &  & 1  & 29  & 2  &  &  &  & 32\\		
%\midrule
%\multicolumn{9}{l}{\textit{stroll} } \\		
%\fename{Self\_mover} & 86  &  &  &  &  &  &  & 86\\		
%\fename{Area} &  &  & 12  &  &  &  &  & 12\\		
%\fename{Path} &  & 2  & 48  & 3  & 1  &  &  & 54\\		
%\fename{Goal} &  & 1  & 18  & 3  &  &  &  & 22\\		
%\fename{Source} &  &  & 7  & 1  &  &  &  & 8\\		
%\midrule
%\multicolumn{9}{l}{\textit{stumble} } \\		
%\fename{Self\_mover} & 69  &  & 1  &  &  &  &  & 70\\		
%\fename{Area} &  &  & 3  & 1  &  &  &  & 4\\		
%\fename{Goal} &  & 1  & 19  & 1  &  &  &  & 21\\		
%\fename{Source} &  &  & 8  & 2  &  &  &  & 10\\		
%\fename{Path} &  & 2  & 35  & 2  & 10  &  &  & 49\\		
%\midrule
\multicolumn{9}{l}{\textit{swim} } \\*
\fename{Self\_mover} & 259  &  &  &  & 1  &  &  & 260\\*
\fename{Area} & 1  & 2  & 54  & 3  & 1  &  &  & 61\\*
\fename{Source} &  &  & 19  & 14  &  &  &  & 33\\*
\fename{Path} &  & 5  & 95  & 25  & 42  &  &  & 167\\*
\fename{Goal} &  & 1  & 45  & 3  & 1  & 1  &  & 51\\*
\fename{Direction} &  &  &  & 1  &  &  &  & 1\\
\midrule
\multicolumn{9}{l}{\textit{walk} } \\*
\fename{Self\_mover} & 102  &  &  &  &  &  &  & 102\\*
\fename{Area} &  &  & 9  & 2  & 4  &  & 1 & 16\\*
\fename{Source} &  &  & 17  &  & 1  &  &  & 18\\*
\fename{Path} &  & 2  & 36  &  & 6  &  & 1 & 45\\*
\fename{Goal} &  &  & 29  & 2  & 1  & 3  & 7 & 42\\*
\fename{Direction} &  &  & 3  & 3  & 1  &  &  & 7\\
\midrule
%\multicolumn{9}{l}{\textit{tramp} } \\		
%\fename{Self\_mover} & 79  &  &  &  &  &  &  & 79\\		
%\fename{Area} &  & 1  & 15  &  & 1  &  & 6 & 23\\		
%\fename{Path} &  &  & 43  & 2  & 4  &  &  & 49\\		
%\fename{Goal} &  &  & 13  & 2  &  & 1  &  & 16\\		
%\fename{Source} &  &  & 5  & 3  &  &  &  & 8\\		
%\midrule
\multicolumn{9}{l}{\textit{wander} } \\*
\fename{Self\_mover} & 81  &  &  &  &  &  &  & 81\\*
\fename{Area} &  &  & 17  &  & 3  &  &  & 20\\*
\fename{Source} &  &  & 12  & 5  &  &  &  & 17\\*
\fename{Path} &  &  & 33  & 4  & 4  &  &  & 41\\*
\fename{Goal} &  &  & 27  & 2  &  &  &  & 29\\
\lspbottomrule
%\multicolumn{9}{l}{\textit{head} } \\		
%\fename{Self\_mover} & 60  &  &  &  &  &  &  & 60\\		
%\fename{Goal} &  & 1  & 32  & 2  &  &  &  & 35\\		
%\fename{Path} &  & 6  & 26  & 7  &  &  &  & 39\\		
%\fename{Direction} &  &  & 1  & 1  &  &  &  & 2\\		
%\fename{Source} &  &  & 4  &  &  &  &  & 4\\		
%\midrule
\end{longtable}
 }
%\end{table}


\subsubsection{FrameNet valence patterns}
\begin{sloppypar}
The valence patterns exhibited in the \framename{Motion} frame are confirmed on a larger scale by \framename{Self\_motion}, in particular the prevalence of \fename{Path}-expressions over \fename{Goals}, \fename{Areas} and \fename{Sources} in descending order. It is worth noting that the number of the second most frequent pattern as compared with the most frequent one is higher than for \framename{Motion} verbs (66\% and 45\%, respectively) i.e. \fename{Goal} expressions are found more frequently as compared to \fename{Path} expressions with \framename{Self\_motion} verbs. In addition, the most frequent patterns involving two motion-related FEs are \fename{Goal} + \fename{Path}  and \fename{Goal} + \fename{Source} representing about 19\% and 11\% of the number of the most frequent pattern; this ratio is much greater than for \framename{Motion}, where the pattern \fename{Goal} + \fename{Path} amounts to 6\% of the most frequent one. An interesting hypothesis to test on this amount of data would be whether this observation ties with animacy and/or agentivity.
\end{sloppypar}

\begin{table}
\small
\begin{tabularx}{\textwidth}{lrQ} 
\lsptoprule
 Pattern & \# & Verbs \\ \midrule
{[NP.Ext]}$_{\feinsub{SMov}}$ {[PP]}$_{\feinsub{Path}}$  & 1576 & \textit{stumble, mince, lurch, frolic, stride, climb, tramp, scurry, trip, stalk, rip, burrow, strut, roam, dance, prowl, jump%, straggle, tiptoe, proceed, amble%,pace, taxi, canter, hike, venture, promenade, wander, drive, flit, gambol, sidle, trudge, hasten, hop, skim, skip, stagger, dart, sashay, waltz, file, waddle, leap, caper, storm, slink, totter, sleepwalk, march, fly, swing, slip, shuffle, slalom, hobble, sprint, lope, run, lunge, stroll, bustle, slosh, swagger, crawl, scramble, spring, pad, tread, trek, edge, romp, trundle, sneak, scuttle, vault, barge, creep, dash, swim, traipse, wriggle, jog, prance, clamber, slog, sail, steal, meander, scamper, scoot, toddle, troop, parade, saunter, head, limp, trot, skulk, stomp, make, lumber, slither, plod, bound, clomp, rush, step, hurry, wade, flounce, walk
}\\

{[NP.Ext]}$_{\feinsub{SMov}}$ {[PP]}$_{\feinsub{Goal}}$  & 1035 & \textit{stumble, mince, lurch, stride, climb, tramp, scurry, trip, stalk, rip, burrow, strut, roam, press%, dance, prowl, jump, straggle, tiptoe, proceed, amble, pace, taxi, canter, hike, venture, wander, drive, flit, sidle, trudge, hasten, hop, skip, pounce, stagger, dart, waltz, file, waddle, leap, storm, slink, totter, sleepwalk, march, fly, swing, shuffle, hobble, sprint, lope, run, lunge, stroll, bustle, swagger, crawl, scramble, spring, pad, tread, trek, edge, romp, trundle, sneak, scuttle, vault, barge, creep, mosey, dash, swim, traipse, wriggle, repair, jog, clamber, sail, steal, meander, scamper, scoot, toddle, troop, saunter, head, limp, trot, skulk, stomp, lumber, slither, plod, bound, rush, step, hurry, wade, flounce, walk
}\\

{[NP.Ext]}$_{\feinsub{SMov}}$ {[PP]}$_{\feinsub{Area}}$  & 599 & \textit{stumble, hobble, mince, lurch, lope, frolic, stroll, bustle, stride, swagger, crawl, scramble, climb, spring, tramp%, pad%, tread, trek, scurry, trip, romp, stalk, rip, trundle, strut, roam, sneak, dance, prowl, scuttle, jump, tiptoe, barge, amble, pace, creep, canter, hike, mosey, wander, dash, swim, traipse, flit, gambol, bop, jog, prance, clamber, slog, trudge, gallivant, steal, scamper, hop, toddle, skip, parade, stagger, saunter, dart, limp, trot, waddle, leap, skulk, stomp, lumber, slither, plod, bound, slink, clomp, totter, rush, sleepwalk, march, fly, swing, step, shuffle, wade, flounce, walk
}
\\

{[NP.Ext]}$_{\feinsub{SMov}}$ {[PP]}$_{\feinsub{Src}}$  & 415 & \textit{stumble, hobble, mince, lurch, sprint, lope, run, lunge, stroll, bustle, stride, slosh, swagger, crawl, scramble%, climb%, spring, tramp, pad, scurry, trip, romp, stalk, trundle, strut, roam, sneak, dance, scuttle, vault, jump, straggle, tiptoe, barge, proceed, creep, venture, mosey, wander, dash, swim, traipse, flit, sidle, prance, clamber, trudge, sail, steal, hasten, scamper, scoot, hop, toddle, skip, troop, pounce, stagger, saunter, dart, limp, waltz, file, trot, waddle, leap, skulk, stomp, storm, lumber, slither, plod, bound, slink, totter, rush, march, fly, swing, step, hurry, shuffle, wade, flounce, walk
}\\

{[NP.Ext]}$_{\feinsub{SMov}}$   {[\_]}$_{\feinsub{Path-INI}}$  & 375 & \textit{stumble, hobble, lurch, sprint, run, lunge, stroll, stride, swagger, crawl, scramble, climb, spring, tramp, pad%, tread, trek, scurry, romp, trundle, strut, roam, prowl, scuttle, straggle, tiptoe, proceed, amble, pace, taxi, canter, venture, promenade, wander, drive, swim, traipse, gambol, jog, prance, clamber, slog, trudge, sail, hasten, scamper, scoot, hop, toddle, skip, troop, stagger, saunter, limp, trot, waddle, leap, skulk, lumber, slither, plod, slink, totter, rush, sleepwalk, march, fly, step, hurry, shuffle, wade, walk
}\\

{[NP.Ext]}$_{\feinsub{SMov}}$ {[PP]}$_{\feinsub{Goal}}$ {[PP]}$_{\feinsub{Path}}$  & 297 & \textit{stumble, hobble, mince, lurch, sprint, lope, stroll, stride, swagger, crawl, scramble, climb, spring, tramp, pad%, tread%, trek, scurry, stalk, sneak, dance, prowl, scuttle, vault, tiptoe, barge, proceed, amble, creep, hike, wander, dash, swim, sidle, jog, clamber, slog, trudge, steal, hasten, meander, scamper, hop, toddle, skip, troop, pounce, stagger, saunter, head, dart, limp, waltz, trot, waddle, leap, stomp, storm, lumber, slither, plod, totter, rush, march, fly, swing, step, hurry, shuffle, wade, walk
}\\

{[NP.Ext]}$_{\feinsub{SMov}}$ {[AVP]}$_{\feinsub{Path}}$  & 187 & \textit{stumble, lurch, frolic, lunge, stroll, bustle, stride, swagger, crawl, climb, spring, tread, trek, scurry, trip, trundle, strut%, roam, prowl, scuttle, jump, straggle, tiptoe, pace, creep, wander, dash, drive, swim, wriggle, clamber, slog, trudge, hasten, scamper, hop, stagger, saunter, slop, head, dart, limp, trot, waddle, leap, lumber, slither, plod, bound, slink, totter, rush, march, step, hurry, shuffle, wade
}\\

{[NP.Ext]}$_{\feinsub{SMov}}$ {[PP]}$_{\feinsub{Goal}}$ {[PP]}$_{\feinsub{Src}}$  & 175 & \textit{stumble, hobble, lurch, sprint, lope, lunge, stroll, bustle, stride, crawl, scramble, climb, spring, pad, trek, edge%, scurry%, trip, romp, trundle, strut, sneak, dance, scuttle, jump, tiptoe, proceed, pace, creep, venture, wander, dash, swim, traipse, clamber, trudge, sail, hasten, scamper, scoot, hop, toddle, troop, stagger, saunter, head, dart, sashay, waltz, file, trot, waddle, leap, stomp, storm, lumber, slither, plod, bound, slink, totter, rush, march, swing, step, hurry, shuffle, wade, walk
}\\

{[NP.Ext]}$_{\feinsub{SMov}}$ {[AVP]}$_{\feinsub{Goal}}$  & 149 & \textit{stumble, hobble, lurch, sprint, lope, run, lunge, stroll, bustle, crawl, scramble, climb, spring, tramp, pad, edge%, scurry%, stalk, roam, sneak, scuttle, jump, tiptoe, pace, creep, hike, venture, wander, dash, swim, repair, jog, clamber, trudge, steal, hasten, scamper, scoot, hop, skip, troop, stagger, saunter, head, dart, limp, waltz, leap, skulk, stomp, storm, lumber, slither, plod, bound, rush, march, swing, step, hurry, shuffle, wade, walk
}\\

{[NP.Ext]}$_{\feinsub{SMov}}$ {[AVP]}$_{\feinsub{Src}}$  & 113 & \textit{stumble, hobble, sprint, lope, stroll, bustle, stride, swagger, crawl, scramble, tramp, pad, scurry, romp, stalk, sneak%, dance, scuttle, tiptoe, amble, pace, taxi, creep, whisk, venture, wander, drive, swim, traipse, flit, trudge, steal, hasten, scamper, hop, skip, stagger, saunter, dart, sashay, limp, waltz, trot, waddle, stomp, lumber, slither, slink, totter, rush, march, fly, step, hurry, shuffle, flounce
}\\

{[NP.Ext]}$_{\feinsub{SMov}}$ {[PP]}$_{\feinsub{Path}}$ {[PP]}$_{\feinsub{Path}}$  & 107 & \textit{stumble, hobble, lurch, sprint, lope, lunge, stroll, bustle, stride, crawl, scramble, climb, spring, tramp, pad, trek%, scurry%, trundle, strut, dance, scuttle, tiptoe, proceed, amble, pace, creep, wander, dash, swim, jog, clamber, trudge, skip, stagger, saunter, head, dart, sashay, trot, waddle, leap, lumber, slither, plod, totter, rush, march, hurry, shuffle, wade
}\\
%
{[NP.Ext]}$_{\feinsub{SMov}}$ {[PP]}$_{\feinsub{Path}}$ {[PP]}$_{\feinsub{Src}}$  & 96 & \textit{jog, prance, clamber, trudge, lurch, scamper, sprint, hop, lope, troop, stagger, stroll, bustle, stride, swagger, crawl, climb%, spring, head, tramp, trot, waddle, leap, dance, prowl, scuttle, jump, straggle, tiptoe, barge, lumber, slither, bound, creep, slink, rush, march, wander, step, dash, shuffle, swim, walk
%
}\\ 
\lspbottomrule
%{[NP.Ext]}$_{\feinsub{SMov}}$ {[\_]}$_{\text{AREA-INI}}$  & 50 & \textit{prance, lurch, hop, parade, crawl, climb, dart, tramp, limp, waddle, burrow, leap, roam, dance, prowl, caper, amble%, pace, canter, march, promenade, wander, drive, swim, walk
%}\\ \midrule
%{[NP.Ext]}$_{\feinsub{SMov}}$ {[NP.Dep]}$_{\feinsub{Area}}$  & 34 & \textit{tramp, tread, trudge, trot, pace, roam, drive, dance, prowl}\\ \midrule
%{[NP.Ext]}$_{\feinsub{SMov}}$ {[PP]}$_{\feinsub{Goal}}$ {[AVP]}$_{\feinsub{Path}}$  & 34 & \textit{wriggle, slither, sail, steal, lurch, creep, lunge, stagger, stride, totter, rush, march, crawl, climb, head, waltz, romp%, wander, trundle, burrow, step, wade, swim, jump
%}\\ \midrule
%{[NP.Ext]}$_{\feinsub{SMov}}$ {[PP]}$_{\feinsub{Area}}$ {[PP]}$_{\feinsub{Area}}$  & 33 & \textit{straggle, slither, trudge, hop, creep, skip, frolic, stroll, crawl, dart, wander, trundle, skulk, strut, dash, roam, wade, dance, flounce, swim, jump, traipse}\\ \midrule
%{[NP.Ext]}$_{\feinsub{SMov}}$ {[AVP]}$_{\feinsub{Path}}$ {[PP]}$_{\feinsub{Path}}$  & 32 & \textit{lurch, scamper, sprint, hop, lunge, stagger, bustle, stride, crawl, spring, head, dart, tramp, trek, trip, trot, leap, strut, prowl, barge, proceed, plod, pace, rush, march, wander}\\ \midrule
%{[NP.Ext]}$_{\feinsub{SMov}}$ {[PP]}$_{\feinsub{Path}}$ {[AVP]}$_{\feinsub{Src}}$  & 31 & \textit{stumble, hobble, clamber, trudge, run, stride, saunter, crawl, dart, tramp, scurry, waddle, trundle, stomp, sneak, scuttle%, make, tiptoe, slither, amble, pace, fly, wander, dash, shuffle
%}\\ \midrule
%{[NP.Ext]}$_{\feinsub{SMov}}$ {[NP.Dep]}$_{\feinsub{Path}}$  & 26 & \textit{tiptoe, tread, trek, promenade, trot, scamper, leap, skip, wade, make, vault, walk}\\ \midrule
%{[NP.Ext]}$_{\feinsub{SMov}}$ {[PP]}$_{\feinsub{Goal}}$ {[AVP]}$_{\feinsub{Src}}$  & 25 & \textit{wriggle, hobble, sidle, plod, scamper, skip, slink, rush, march, venture, stalk, wander, skulk, step, hurry, sneak, drive, dance, swim}\\ \midrule
%{[NP.Ext]}$_{\feinsub{SMov}}$ {[NP.Obj]}$_{\feinsub{Path}}$  & 25 & \textit{stumble, sidle, skim, stroll, swagger, scramble, climb, spring, head, step, make, swim, walk}\\ \midrule
%{[NP.Ext]}$_{\feinsub{SMov}}$  & 22 & \textit{spring, take to the air, run, lunge, stroll, walk}\\ \midrule
%{[NP.Ext]}$_{\feinsub{SMov}}$ {[PP]}$_{\feinsub{Goal}}$ {[PP]}$_{\feinsub{Path}}$ {[PP]}$_{\feinsub{Src}}$  & 21 & \textit{clamber, lurch, hop, stroll, stride, rush, climb, spring, tramp, fly, scurry, step, dash, hurry, shuffle, sneak, dance, swim, walk}\\ \midrule
%{[NP.Ext]}$_{\feinsub{SMov}}$ {[AVP]}$_{\feinsub{Area}}$  & 21 & \textit{stumble, clamber, lurch, run, stagger, scramble, slop, fly, roam, dance, prowl, swim, walk}\\ \midrule
%{[NP.Ext]}$_{\feinsub{SMov}}$ {[AVP]}$_{\feinsub{Path}}$ {[PP]}$_{\feinsub{Src}}$  & 20 & \textit{wriggle, storm, slither, pace, sprint, creep, troop, stagger, rush, saunter, march, head, dart, tramp, step, swim, jump}\\ \midrule
%{[NP.Ext]}$_{\feinsub{SMov}}$ {[AVP]}$_{\feinsub{Dir}}$  & 18 & \textit{head, edge, file, shrink, amble, rip, run, press, walk}\\ \midrule
%{[NP.Ext]}$_{\feinsub{SMov}}$ {[AVP]}$_{\feinsub{Goal}}$ {[PP]}$_{\feinsub{Src}}$  & 13 & \textit{dart, pad, jog, trot, slink, troop, dash, sneak, stride, rush, scuttle, swim} 
%\\ \midrule							 \midrule
\end{tabularx}
\caption{FrameNet valence patterns of \framename{Self\_Motion} verbs}
\label{tab:4:self-motion-valence-framenet}
\end{table} 

\subsubsection{Verbs evoking the \framename{Self\_motion} frame in Bulgarian}

Many of the Bulgarian verbs that evoke the \framename{Self\_motion} frame are manner of motion simplex verbs. This aligns with the fact that, overall, \fename{Self\_motion} describes the idea of movement without profiling any of the route-related aspects of motion. Other verbs, such as \textit{втурвам се} `rush’, \textit{отправям се} `head’, `make’, \textit{спускам се} `dash’, `dart’, \textit{налитам}, \textit{хвърлям се} `barge’, etc., involve directed motion or the initial phase of motion rather than manner. Such verbs usually come in aspectual pairs. 

\subsubsection{Syntactic realisation of \framename{Self\_motion} verbs in Bulgarian}

\tabref{tab:4:selfmotionbg} shows the results for several frequent Bulgarian verbs with correspondence in \tabref{tab:4:self-motion-synt}. For the English data  \fename{Goals} are found in competition with \fename{Paths} and other motion-related FEs, either in fewer numbers, but still well-represented across many verbs, or in greater numbers than the other FEs co-occurring with the respective predicates. The most notable difference found in the Bulgarian sample is the lower frequency of \fename{Goals} as compared with the data in the FrameNet corpus.

The verbs \textit{бродя} `roam’, `wander’, \textit{вървя} `walk’, \textit{катеря се} `climb’, \textit{плувам} `swim’, \textit{пълзя} `crawl’, \textit{тичам} `run’ (Examples \ref{ex:09}--\ref{ex:14}) all show a lower occurrence of \fename{Goals}, whereas in English \textit{climb} and \textit{walk} co-occur equally with both FEs and \textit{run} shows preference for \fename{Goals} over \fename{Paths}.  

\begin{exe}
\ex \label{ex:09}
  %  \settowidth \jamwidth{(bg)} 
\gll [\textit{Ноа}]$_{\feinsub{SMov}}$ \textit{\textbf{ТИЧАШЕ}} [\textit{към} \textit{него}]$_{\feinsub{Goal}}$. \\
Noah run-PST.3SG towards him. \\
\glt `Noah was running towards him.' 
%\fename{Goal}: 
%[\textit{Той}]$_{\feinsub{SMov}}$ \textit{тичаше} [\textit{към} \textit{нас}]$_{\feinsub{Goal}}$\\
%{[\textit{He}]}$_{\text{SELF-MOVER}}$ \textit{was} \textit{running} [\textit{towards} \textit{us}]$_{\feinsub{Goal}}$.
\ex \label{ex:10}
%\fename{Path}: 
\gll [\textit{Той}]$_{\feinsub{SMov}}$ \textit{\textbf{ТИЧАШЕ}} [\textit{по} \textit{дългия} \textit{коридор}]$_{\feinsub{Path}}$.\\
He run-PST.3SG down long-DEF hall. \\
\glt `He was running down the long hall.' 
%ю\fename{Area}: 
%\fename{Direction}: 
\ex \label{ex:11}
\gll [\textit{Той}]$_{\feinsub{SMov}}$ {\textit{\textbf{СЕ}} \textit{\textbf{КАТЕРИ}}} 
 [\textit{по} \textit{хълма}]$_{\feinsub{Path}}$.\\
He climb-PRS.3SG up hill-DEF. \\
\glt `He is climbing up the hill.'
\ex \label{ex:12}
%\fename{Source}: 
\gll [\textit{Те}]$_{\feinsub{SMov}}$ \textit{\textbf{БРОДЕХА}} [\textit{по} \textit{коридорите}]$_{\feinsub{Path}}$.\\
They wander-PST.3PL along corridors-DEF. \\
\glt `They wandered along the corridors.' 
\ex \label{ex:13}
\gll [\textit{Назгулите}]$_{\feinsub{SMov}}$ \textit{\textbf{БРОДЯТ}} [\textit{по} \textit{земята}]$_{\feinsub{Area}}$.\\
Nazgul-DEF roam-PRS.3PL across earth-DEF. \\
\glt `The Nazgul roam the earth.'
\ex \label{ex:14}
%\fename{Distance}: 
\gll [\textit{Той}]$_{\feinsub{SMov}}$ \textit{\textbf{ПЛУВАШЕ}} [\textit{из} \textit{бурното} \textit{море}]$_{\feinsub{Area}}$. \\
He swim-PST.3SG across stormy-DEF sea. \\
\glt `He was swimming in the stormy seas.'
%\ex
%\gll [\textit{Фен}]$_{\feinsub{SMov}}$ \textit{тича} [\textit{6 км}]$_{\feinsub{Distance}}$ \textit{след} \textit{автобуса}.\\
%[Fan-INDF]$_{\feinsub{SMov}}$ ran [6km]$_{\feinsub{Distance}}$ behind bus-DEF. \\
%\glt `A fan ran 6 km behind the bus.'
%\end{xlist}
% \jambox{(bg)}
\end{exe}

This is at least partly predictable: while English manner of motion verbs express directionality by means of \fename{Goal}- or \fename{Source}-phrases or particles with a similar meaning, the corresponding Slavic (Bulgarian) simplex verbs may also derive new verbs with a directional meaning through prefixation (\cite{Beavers2010,Lindsey2011,Pantcheva-2007,Pantcheva2007,Pantcheva2011,speed:2015}, among many others). While simplex verbs can realise directionality by means of route-related phrases, the derived prefixed verbs profile the relevant aspect of the route and encode it in their lexical structure; the two types of verbs may be used interchangeably in certain contexts, but not in others. As a cursory illustration of this point, consider the verb in (Example \ref{ex:224:a}), whose directional meaning cannot be expressed by the simplex verb it is derived from; hence the expression \textit{в стаята} in (Example \ref{ex:224:b}) cannot be interpreted as the \fename{Goal} (marked by an asterisk); still, it will have an English correspondence of manner of motion verb + a directional phrase. 

\begin{exe}
\ex \label{ex:224}
\begin{xlist}
\ex  \label{ex:224:a}
  %  \settowidth \jamwidth{(bg)} 
\gll [\textit{Птицата}]$_{\feinsub{SMov}}$ \textit{\textbf{ВЛИТА}} [\textit{в} \textit{стаята}]$_{\feinsub{Goal}}$. \\
Bird-DEF fly-PRS.3SG into room-DEF. \\
\glt `The bird flies into the room.'
\ex  \label{ex:224:b}
\gll *[\textit{Птицата}]$_{\feinsub{SMov}}$ \textit{\textbf{ЛЕТИ}} [\textit{в} \textit{стаята}]$_{\feinsub{Goal}}$. \\
Bird-DEF fly-PRS.3SG into room-DEF. \\
\glt `The bird flies into the room.'
\end{xlist}
\end{exe}


{\footnotesize
\begin{longtable}{l ccccccccc}   
\caption{Syntactic expression of the \framename{Self\_motion} FEs in Bulgarian}\label{tab:4:selfmotionbg}\\
 \midrule
  & NP.Ext & NP.Obj & PP & AVP & NI & Clause & Other & Total\\ \midrule\endfirsthead
\midrule
  & NP.Ext & NP.Obj & PP & AVP & NI & Clause & Other & Total\\
  \midrule\endhead
\multicolumn{9}{l}{\textit{пълзя} `crawl’ }\\*
\fename{Self\_mover} & 40 &  &  &  &  &  &  & 40\\*
\fename{Area} &  &  & 4 & 2 &  &  &  & 6\\*
\fename{Path} &  &  & 19 & 1 &  &  &  & 20\\*
\fename{Goal} &  &  & 8 & 1 &  &  &  & 9\\*
\fename{Direction} &  &  &  & 2 &  &  &  & 2\\ 
 \midrule
%\multicolumn{9}{l}{\textit{плавам} }\\*
%\fename{Area} &  &  & 1 &  &  &  &  & 1\\*
%\fename{Self\_mover} & 2 &  &  &  &  &  &  & 2\\ 
% \midrule
\multicolumn{9}{l}{\textit{катеря се} `climb’ }\\*
\fename{Self\_mover} & 48 &  &  &  &  &  &  & 48\\*
\fename{Area} &  &  & 2 &  &  &  &  & 2\\*
\fename{Source} &  &  & 1 &  &  &  &  & 1\\* 
\fename{Path} &  &  & 22 & 1 &  &  &  & 23\\*
\fename{Goal} &  &  & 14 & 2 &  &  &  & 16\\
%\fename{Manner} &  &  &  & 1 &  &  &  & 1\\*
 \midrule
\multicolumn{9}{l}{\textit{бродя} `roam’, `wander’ }\\*
\fename{Self\_mover} & 39 &  &  &  &  &  &  & 39\\*
\fename{Area} &  &  & 18 & 2 &  &  &  & 20\\*
\fename{Source} &  &  & 1 &  &  &  &  & 1\\*
\fename{Path} &  &  & 13 &  &  &  &  & 13\\*
\fename{Goal} &  &  & 1 &  &  &  &  & 1\\
%\fename{Manner} &  &  &  & 2 &  &  &  & 2\\*
\midrule
\multicolumn{9}{l}{\textit{плувам} `swim’ }\\*
\fename{Self\_mover} & 37 &  &  &  &  &  &  & 37\\*
\fename{Area} &  &  & 9 & 3 &  &  &  & 12\\*
\fename{Path} &  &  & 6 & 1 &  &  &  & 7\\*
\fename{Goal} &  &  & 5 &  &  &  &  & 5\\
%\fename{Manner} &  &  &  & 3 &  &  &  & 3\\ 
 \midrule
\multicolumn{9}{l}{\textit{тичам} `run’ }\\*
\fename{Self\_mover} & 42 &  &  &  &  &  &  & 42\\*
\fename{Area} &  &  & 3 &  &  &  &  & 3\\*
\fename{Source} &  &  & 1 &  &  &  &  & 1\\*
\fename{Path} &  &  & 23 & 1 &  &  &  & 24\\*
\fename{Goal} &  &  & 11 & 1 &  &  &  & 12\\
%\fename{Manner} &  &  & 3 & 1 &  &  &  & 4\\*
 \midrule
\multicolumn{9}{l}{\textit{вървя} `walk’ }\\*
\fename{Self\_mover} & 40 &  &  &  &  &  &  & 40\\*
\fename{Area} &  &  &  & 1 &  &  &  & 1\\*
\fename{Path} &  &  & 16 & 2 &  &  &  & 18\\*
\fename{Goal} &  &  & 2 & 3 &  &  &  & 5\\*
\fename{Direction} &  &  & 1 & 1 &  &  &  & 2\\*
%\fename{Manner} &  &  & 2 & 4 &  &  &  & 6\\ 
 \midrule
\end{longtable}
}

The above observations are confirmed by the distribution of the patterns involving \fename{Paths}, \fename{Goals} and \fename{Areas} in the Bulgarian dataset (\tabref{tab:4:selfmotion-valence-bg}).

\begin{table}
   \begin{tabularx}{\textwidth}{lrQ} 
   \lsptoprule
   Pattern & \# & Verbs \\ \midrule
{[NP.Ext]}$_{\feinsub{SMov}}$ {[PP]}$_{\feinsub{Path}}$ & 86 & \textit{вървя, плувам, пълзя, бродя, тичам, катеря се}\\ 
{[NP.Ext]}$_{\feinsub{SMov}}$ {[\_]}$_{\feinsub{Path-INI}}$ & 45 & \textit{вървя, плувам, пълзя, бродя, тичам, катеря се}\\ 
{[NP.Ext]}$_{\feinsub{SMov}}$ {[PP]}$_{\feinsub{Area}}$ & 37 & \textit{плувам, пълзя, бродя, тичам, катеря се}\\ 
{[NP.Ext]}$_{\feinsub{SMov}}$ {[PP]}$_{\feinsub{Goal}}$ & 35 & \textit{вървя, плувам, пълзя, тичам, катеря се}\\ 
{[NP.Ext]}$_{\feinsub{SMov}}$ {[AVP]}$_{\feinsub{Area}}$ & 6 & \textit{плувам, пълзя, бродя}\\ 
{[NP.Ext]}$_{\feinsub{SMov}}$ {[AVP]}$_{\feinsub{Goal}}$ & 6 & \textit{вървя, пълзя, катеря се}\\ 
%{[NP.Ext]}$_{\feinsub{SMov}}$ {[AVP]}$_{\text{MANNER}}$ {[PP]}$_{\feinsub{Path}}$ & 5 & \textit{вървя, бродя}\\ 
{[NP.Ext]}$_{\feinsub{SMov}}$ {[AVP]}$_{\feinsub{Path}}$ & 5 & \textit{вървя, плувам, пълзя, катеря се}\\ 
%{[NP.Ext]}$_{\feinsub{SMov}}$ {[AVP]}$_{\text{MANNER}}$ & 5 & \textit{вървя, плувам, тичам, катеря се}\\ 
%{[NP.Ext]}$_{\feinsub{SMov}}$ {[PP]}$_{\text{MANNER}}$ & 3 & \textit{вървя, тичам}\\  %
%{[NP.Ext]}$_{\feinsub{SMov}}$ {[PP]}$_{\feinsub{Goal}}$ {[PP]}$_{\feinsub{Src}}$ & 3 & %\textit{бродя, тичам, катеря се}\\ 
%{[NP.Ext]}$_{\feinsub{SMov}}$ {[PP]}$_{\feinsub{Goal}}$ {[PP]}$_{\feinsub{Path}}$ & 2 & %\textit{тичам, катеря се}\\ 
%{[NP.Ext]}$_{\feinsub{SMov}}$ {[PP]}$_{\text{MANNER}}$ {[PP]}$_{\feinsub{Path}}$ & 2 & \textit{тичам}\\ 
%{[NP.Ext]}$_{\feinsub{SMov}}$ {[AVP]}$_{\feinsub{Dir}}$ & 2 & \textit{вървя, пълзя}\\ %
%{[NP.Ext]}$_{\feinsub{SMov}}$ {[AVP]}$_{\feinsub{Area}}$ {[AVP]}$_{\text{MANNER}}$ & 1 & %\textit{плувам}\\ 
%{[NP.Ext]}$_{\feinsub{SMov}}$ {[AVP]}$_{\feinsub{Path}}$ {[PP]}$_{\feinsub{Goal}}$ & 1 & \textit{тичам}\\ 
%{[NP.Ext]}$_{\feinsub{SMov}}$ {[AVP]}$_{\feinsub{Dir}}$ {[PP]}$_{\feinsub{Path}}$ & 1 & %\textit{пълзя}\\ 
%{[NP.Ext]}$_{\feinsub{SMov}}$ {[PP]}$_{\feinsub{Dir}}$ {[PP]}$_{\feinsub{Path}}$ & 1 & %\textit{вървя}\\ 
%{[NP.Ext]}$_{\feinsub{SMov}}$ {[AVP]}$_{\feinsub{Area}}$ {[PP]}$_{\feinsub{Path}}$ & 1 & \textit{вървя}\\ 
%{[NP.Ext]}$_{\feinsub{SMov}}$ {[AVP]}$_{\feinsub{Goal}}$ {[PP]}$_{\feinsub{Path}}$ & 1 & \textit{тичам}\\ 
\lspbottomrule
\end{tabularx}
    \caption{Valence patterns of \framename{Self\_motion} verbs in Bulgarian}
    \label{tab:4:selfmotion-valence-bg}
\end{table}


\subsection{\framename{Traversing}, \framename{Arriving}, \framename{Departing}}\label{traversing}

These semantic frames narrow down the idea of motion through profiling aspects of the general motion schema corresponding to elements of the route along which the moving object changes location: the initial stage of the motion corresponding to the \fename{Source}; the end-stage -- associated with the \fename{Goal}, or the middle stage -- corresponding to the \fename{Path}, cf. \citep[16]{Johnson2001}. Borrowed from \citet{Langacker1987}, profiling is understood as ``the representation of the foregrounded part of a frame, the participant, prop, phase or moment which figures centrally in the semantic interpretation of the sentence within which the frame is evoked” \citep[16]{Fillmore2001}. 

\fename{Goal}-profiling LUs (e.g., \textit{arrive}, \textit{reach}) evoke the semantic frame \framename{Arriving};  \fename{Source}\hyp profiling LUs (e.g. \textit{leave}, \textit{depart}) evoke the \framename{Departing} frame; \fename{Path}\hyp profiling LUs, such as \textit{traverse}, \textit{cross} correspond to the \framename{Traversing} frame. \framename{Arriving} and \framename{Departing} are defined as subframes of \framename{Traversing}: as such, each of them describes a state or transition in the conceptualisation of a complex situation referring to the sequence of transitions from the \fename{Source}, through the \fename{Path}, to the \fename{Goal}. \framename{Departing} and \framename{Arriving} are related to each other by means of the Precedes relation. %, which, as implemented in FrameNet, means that \framename{Departing} temporally precedes (and is prerequisite for) \framename{Arriving}.

The profiling of a given FE is associated with the fact that the respective FE is central to the meaning and is always conceptually implied even if not necessarily overtly realised. In such cases it is often retrievable from the context and is thus understood and annotated as a definite null instantiation (DNI).


\subsubsection{\framename{Arriving}}

\subsubsection{Semantic description of the \framename{Arriving} frame}

\framename{Arriving} describes directed motion towards an end point which is part of the lexical encoding of the relevant LUs: i.e. the verbs evoking the frame are \fename{Goal}-oriented verbs of inherently directed motion.   

\begin{description}[font=\normalfont]
\item[Definition of the frame \framename{Arriving}:] An object, \fename{Theme}, moves in the direction of a \fename{Goal}. The \fename{Goal} may be expressed or it may be understood from the context, but it is always implied by the verb itself. 

\item[Core frame elements:] \fename{Theme}, \fename{Goal}. The core FEs of the \framename{Arriving} frame represent a subset of the core FEs of the  \framename{Traversing} frame of which it is a subframe. The FEs share the definition and semantic properties of their correspondences in the \framename{Motion} frame. %(inherited from the \framename{Motion\_scenario} frame) 
The profiling of the \fename{Goal} results in the backgrounding or exclusion of the remaining elements that form part of the core FEs of \framename{Traversing}. \fename{Source} and \fename{Path} become peripheral, while \fename{Path\_shape}, \fename{Distance}, \fename{Direction}, \fename{Area}, as defined in FrameNet, are not conceptually present in the scenario described by this frame.
%there are examples...
\end{description}


\subsubsection{Verbs evoking the \framename{Arriving} frame}

The verbs evoking the \framename{Arriving} frame form a central part of the lexis of \fename{Goal}-directed motion: \textit{appear}, \textit{approach}, \textit{arrive}, \textit{come}, \textit{crest}, \textit{descend (on)}, \textit{enter}, \textit{get}, \textit{hit}, \textit{make it}, \textit{make}, \textit{reach}, \textit{return}, \textit{visit}.

\subsubsection{Syntactic realisation of the frame elements in the \framename{Arriving} frame}

The syntactic realisation of the frame elements in the \framename{Arriving} frame as represented in the FrameNet corpus are illustrated in \tabref{tab:4:arriving-synt}. The \fename{Theme} is projected as a subject, while depending on the verb, the \fename{Goal} may be expressed as either a prepositional or adverbial phrase -- e.g. \textit{arrive}, \textit{come}, \textit{return}, \textit{get}, \textit{make it}, or as a direct object (NP.Obj) -- e.g. \textit{approach}, \textit{enter}, \textit{reach}, \textit{visit}.

{\footnotesize
\begin{longtable}{l ccccccccc}
\caption{\label{tab:4:arriving-synt}Syntactic expression of the \framename{Arriving} FEs in FrameNet}\\
\lsptoprule
  & NP.Ext & NP.Obj & PP & AVP & NI & Clause & Other & Total\\ \midrule \endfirsthead
  \midrule
    & NP.Ext & NP.Obj & PP & AVP & NI & Clause & Other & Total\\ \midrule \endhead
\textit{approach} &&&&&&&&\\*
\fename{Theme} & 36 &  & 1 &  &   &  &  & 37 \\*
\fename{Goal} & 1 & 29 &  & 1 & 6   &  &  & 37 \\ 
\midrule
\textit{arrive} &&&&&&&&\\*
\fename{Theme} & 81 &  &  &  &    &  &  & 81 \\*
\fename{Goal} & 1 &  & 31 & 11 & 35   &  &  & 78 \\ 
\midrule
\textit{come} &&&&&& &&\\*
\fename{Theme} & 119 &  &  &  &    &  &  & 119 \\*
\fename{Goal} &  & 2 & 44 & 16 & 50 & 3 & 2 & 117 \\ 
 \midrule
\textit{enter} &&&&&& &&\\*
\fename{Theme} & 30 & 1 &  &  & 3 &  &  & 34 \\*
\fename{Goal} & 3 & 17 &  & & 10 &  & 1 & 31 \\ 
 \midrule
\textit{return} &&&&&& &&\\*
\fename{Theme} & 48 &  &  &  & 3 &  &  & 48 \\*
\fename{Goal} & & 1 & 17 & 9 & 21 & 1 & & 49 \\ 
 \midrule
\textit{visit} &&&&&& &&\\*
\fename{Theme} & 24 &  & 3 &  & 2 &  &  & 29 \\*
\fename{Goal} & 5 & 14 & 2 &  & 7 & 1 & 1 & 29 \\ 
 \midrule
\textit{reach} &&&&&& &&\\*
\fename{Theme} & 50 & 1 &  &  & 7 &  &  & 58 \\*
\fename{Goal} & 7 & 48 &  & 1 &  &  &  & 56 \\ 
 \midrule
\textit{get} &&&&&& &&\\*
\fename{Theme} & 35 &  &  &  &  &  &  & 35 \\*
\fename{Goal} & &  & 16 & 12 & 7 &  & 1 & 36 \\ 
 \midrule
\textit{make it} &&&&&& &&\\*
\fename{Theme} & 22 &  &  &  &  &  &  & 22 \\*
\fename{Goal} & & & 12 & 1 & 9 &  &  & 22 \\ 
 \lspbottomrule
 \end{longtable}
}  
%\end{table}

The possibility for leaving the \fename{Goal} non-overt as reflected in the considerable number of definite null instantiations (NIs in the table), stems from the fact that with some verbs this FE often receives a definite interpretation as the %place where the speaker is located 
deictic centre and its identity is thus implied even without previous reference. 
This is typical for \textit{come} and to a lesser degree for \textit{arrive} due to their deictic nature. In this respect they are clearly distinct from \textit{reach}, %(which does not imply the identity of the \fename{Goal} as any place, including the location of the speaker) 
 \textit{approach}, \textit{visit}, \textit{get} and \textit{make it}, which usually express the \fename{Goal}, as it need not be identical to the deictic centre. Examples (\ref{ex:222:a}, \ref{ex:222:b}) illustrate this point.

\begin{exe}
\ex \label{ex:222}
\begin{xlist}
\ex[]{\label{ex:222:a}
  %  \settowidth \jamwidth{(en)} 
[\textit{She}]$_{\feinsub{Thm}}$ \textit{\textbf{REACHED}} [\textit{Rome}]$_{\feinsub{Goal}}$ [\textit{via Assisi}]$_{\feinsub{Path}}$.} 
\ex[*]{\label{ex:222:b}
  %  \settowidth \jamwidth{(en)} 
[\textit{She}]$_{\feinsub{Thm}}$ \textit{\textbf{REACHED}}. 
  %\jambox{(en)}
}
 \end{xlist}
\end{exe}

%\begin{exe}
%\ex  \label{ex223}
  %  \settowidth \jamwidth{(en)} 
%*[\textit{She}]$_{\feinsub{Thm}}$ \textit{reached}. 
 %\jambox{(en)}
%\end{exe}

\subsubsection{FrameNet valence patterns}

In line with the above observations, syntactically implicit \fename{Goals} represent half of the aggregated number of the \fename{Goal}-phrases (\tabref{tab:4:arriving-valence-framenet}). There is a considerable number of NP \fename{Goals}, which accounts for the fact that a great deal of the verbs are transitive. In addition, AVPs are much more prominent: they make up for a third of the prepositional \fename{Goal}-phrases, while in \framename{Self\_motion} their number is 15\% of the number of \fename{Goal}-PPs.

\begin{table}
    \begin{tabularx}{\textwidth}{lrQ}
    \lsptoprule
     Pattern & \# & Verbs \\ 
     \midrule
{}[NP]$_{\feinsub{Thm}}$ [\ ]$_{\feinsub{Goal-DNI}}$&  144& \textit{appear}, \textit{approach}, \textit{arrive}, \textit{come}, \textit{enter}, \textit{return}, \textit{visit}, \textit{get}, \textit{make it} \\
{}[NP]$_{\feinsub{Thm}}$ [NP]$_{\feinsub{Goal}}$ &  126&  \textit{approach}, \textit{enter}, \textit{visit}, \textit{reach}, \textit{make}, \textit{crest}, \textit{hit} \\ 
{}[NP]$_{\feinsub{Thm}}$ [PP]$_{\feinsub{Goal}}$& 121 & \textit{arrive}, \textit{come}, \textit{return}, \textit{visit}, \textit{get}, \textit{make it}, \textit{descend (on)}, \textit{appear} \\
{}[NP]$_{\feinsub{Thm}}$ [AVP]$_{\feinsub{Goal}}$ &  46&  \textit{approach}, \textit{arrive}, \textit{come}, \textit{return}, \textit{reach}, \textit{get}, \textit{make it}\\
\lspbottomrule
%[NP]$_{\feinsub{Thm}}$ [PP/AdvP]$_{\feinsub{Path}}$& 3& влизам, идвам, навлизам \\\midrule  [NP]$_{\feinsub{Thm}}$ [PP/AdvP]$_{\feinsub{Goal}}$    \newline
 % [PP/AdvP]$_{\feinsub{Path}}$& 2& достигам, стигам \\\midrule
%[NP]$_{\feinsub{Thm}}$ [PP/AdvP]$_{\feinsub{Goal}}$  \newline
% [PP/AdvP]$_{\feinsub{Src}}$& 2& пристигам \\\midrule
%[NP]$_{\feinsub{Thm}}$ [\ ]$_{\feinsub{Goal-DNI}}$    \newline
% [PP/AdvP]$_{\feinsub{Path}}$& 1& пристигам\\\midrule
%[NP]$_{\feinsub{Thm}}$ [\ ]$_{\feinsub{Goal-DNI}}$   \newline
 %[PP/AdvP]$_{\feinsub{Src}}$& 1& идвам\\\midrule
    \end{tabularx}
    \caption{FrameNet valence patterns of \framename{Arriving} verbs}
    \label{tab:4:arriving-valence-framenet}
\end{table} 


\subsubsection{Syntactic realisation of \framename{Arriving} verbs in Bulgarian}

The basic verbs evoking the \framename{Arriving} frame form a small but central part of the lexis of directed motion: \textit{влизам} `enter’, \textit{връщам се} `return’, \textit{добирам се} `make it’, \textit{доближавам}, \textit{доближавам се} `approach’, \textit{достигам} `reach’, \textit{завръщам се} `return’, \textit{наближавам}, \textit{приближавам}, \textit{приближавам се} `approach’, \textit{идвам}, \textit{ида} `come’, \textit{пристигам} `arrive’, \textit{стигам} `reach’, \textit{посещавам} `visit’, \textit{прибирам се} `go home’. To the exception of \textit{посещавам}, which requires object NP \fename{Goals}, and \textit{доближавам}, \textit{наближавам}, \textit{приближавам}, \textit{достигам}, \textit{стигам} -- which take either an object NP or a PP\slash AVP, the rest of the verbs select a PP/AVP complement. In this respect the Bulgarian verbs differ from their English counterparts, many of which take an object \fename{Goal} complement.



\begin{longtable}{l ccccccccc}   
\caption{Syntactic expression of the \framename{Arriving} FEs in Bulgarian} 
    \label{tab:4:arriving-synt-bg}
 \\ \lsptoprule
  & NP.Ext & NP.Obj & PP & AVP & NI & Clause & Other & Total\\ \midrule \endfirsthead
  \midrule
  & NP.Ext & NP.Obj & PP & AVP & NI & Clause & Other & Total\\ \midrule \endhead
%\multicolumn{9}{l}{\textit{добера} }\\*
%\fename{Goal} &  &  & 2 &  &  &  &  & 2\\*
%\fename{Theme} & 2 &  &  &  &  &  &  & 2\\ 
% \midrule
%\multicolumn{9}{l}{\textit{ида} }\\*
%\fename{Goal} &  &  & 3 & 1 &  &  &  & 4\\*
%\fename{Theme} & 4 &  &  &  &  &  &  & 4\\ 
% \midrule
%\multicolumn{9}{l}{\textit{приближавам\slash приближа} }\\*
%\fename{Goal} &  &  & 5 &  &  &  &  & 5\\*
%\fename{Theme} & 5 &  &  &  &  &  &  & 5\\ 
% \midrule
\multicolumn{9}{l}{\textit{отивам\slash отида} `go’}\\*
\fename{Theme} & 10 &  &  &  &  &  &  & 10\\* 
\fename{Goal} &  &  & 6 & 1 & 3 &  &  & 10\\*
 \midrule
\multicolumn{9}{l}{\textit{достигам\slash достигна} `reach’}\\*
\fename{Theme} & 4 &  &  &  &  &  &  & 4\\* 
\fename{Goal} &  & 2 & 2 &  &  &  &  & 4\\
 \midrule
%\multicolumn{9}{l}{\textit{наближавам\slash наближа} }\\*
%\fename{Goal} &  & 1 &  &  &  &  &  & 1\\*
%\fename{Theme} & 1 &  &  &  &  &  &  & 1\\ 
% \midrule
\multicolumn{9}{l}{\textit{идвам\slash дойда} `come’}\\*
\fename{Theme} & 20 &  &  &  &  &  &  & 20\\*
\fename{Goal} &  &  & 11 &  & 9 &  &  & 20\\
 \midrule
\multicolumn{9}{l}{\textit{пристигам\slash пристигна} `arrive’}\\*
\fename{Theme} & 21 &  &  &  &  &  &  & 21\\* 
\fename{Goal} &  &  & 10 &  & 11 &  &  & 21\\
% \midrule
%\multicolumn{9}{l}{\textit{завърна} }\\*
%\fename{Goal} &  &  & 2 &  &  &  &  & 2\\*
%\fename{Theme} & 2 &  &  &  &  &  &  & 2\\ 
\midrule
\multicolumn{9}{l}{\textit{стигам\slash стигна} `reach’}\\*
\fename{Theme} & 15 &  &  &  &  &  &  & 15\\* 
\fename{Goal} &  & 2 & 11 & 2 &  &  &  & 15\\
 \midrule
%\multicolumn{9}{l}{\textit{посещавам\slash посетя} }\\*
%\fename{Goal} &  &  &  &  &  &  & 1 & 1\\*
%\fename{Theme} & 1 &  &  &  &  &  &  & 1\\ 
% \midrule
\multicolumn{9}{l}{\textit{влизам\slash вляза} `enter’}\\*
\fename{Theme} & 23 &  &  &  &  &  &  & 23\\*
\fename{Goal} &  &  & 13 &  & 10 &  &  & 23\\
 \midrule
%\multicolumn{9}{l}{\textit{навлизам\slash навляза} }\\*
%\fename{Goal} &  &  & 1 &  &  &  &  & 1\\*
%\fename{Theme} & 1 &  &  &  &  &  &  & 1\\ 
% \midrule
%\multicolumn{9}{l}{\textit{завръщам} }\\*
%\fename{Goal} &  &  & 2 &  &  &  &  & 2\\*
%\fename{Theme} & 2 &  &  &  &  &  &  & 2\\ 
% \midrule
%\multicolumn{9}{l}{\textit{прибирам\slash прибера} }\\*
%\fename{Goal} &  &  & 2 & 1 &  &  &  & 3\\*
%\fename{Theme} & 3 &  &  &  &  &  &  & 3\\ 
% \midrule
%\multicolumn{9}{l}{\textit{доближавам\slash доближа} }\\*
%\fename{Goal} &  & 1 & 1 &  &  &  &   & 2\\*
%\fename{Theme} & 2 &  &  &  &  &  &  & 2\\ 
% \midrule
\multicolumn{9}{l}{\textit{връщам се\slash върна се} `return’}\\*
\fename{Theme} & 14 &  &  &  &  &  &  & 14\\* 
\fename{Goal} &  &  & 7 & 4 & 3 &  &  & 14\\
 \lspbottomrule
\end{longtable}

As expected, the \fename{Goal}-PPs (Example \ref{ex:18:a}) predominate over NPs  (Example \ref{ex:18:b}) and AVPs (Example \ref{ex:18:c}) as shown in \tabref{tab:4:arriving-valence-bg}. The possibility of leaving the \fename{Goal} syntactically unexpressed if it is construable from the context (Example \ref{ex:18:d}) is underrepresented in the sample of annotated examples.

\begin{exe}
\ex \label{ex:18}
\begin{xlist}
\ex \label{ex:18:a}
\gll [\textit{Вражески} \textit{кораби}]$_{\feinsub{Thm}}$ \textit{\textbf{ИДВАТ}} [\textit{към} \textit{вас}]$_{\feinsub{Goal}}$. \\
Hostile aircraft come-PRS.3PL towards you. \\
\glt `Hostile spacecraft are coming your way.'
\ex \label{ex:18:b}
\gll \textit{Следобед} [\textit{те}]$_{\feinsub{Thm}}$ \textit{\textbf{ДОСТИГНАХА}} [\textit{брега}]$_{\feinsub{Goal}}$. \\
In-afternoon-DEF they reach-PST.3PL coast-DEF. \\
\glt `They reached the coast in the afternoon.'
\ex \label{ex:18:c}
\gll [\textit{Никой}]$_{\feinsub{Thm}}$ \textit{не} \textit{\textbf{СЕ}} \textit{\textbf{ВРЪЩА}} [\textit{тук}]$_{\feinsub{Goal}}$. \\
Nobody not REFL return-PRS.3SG-NEG here. \\
\glt `No one returns here.'
\ex \label{ex:18:d}
\gll [\textit{\ }]$_{\feinsub{Thm}}$ \textit{\textbf{ПРИСТИГАТЕ}} [\textit{\ }]$_{\feinsub{Goal}}$ \textit{точно} \textit{навреме,} \textit{докторе}! \\
{} Arrive-PRS.2PL {} just {on time}, doctor! \\
\glt `You arrive just on time, doctor!'
\end{xlist}
\end{exe}


\begin{table}
   \begin{tabularx}{\textwidth}{ lrQ } 
   \lsptoprule
   Pattern & \# & Verbs \\ \midrule
{[NP.Ext]}$_{\feinsub{Thm}}$ {[PP]}$_{\feinsub{Goal}}$ & 78 & \textit{влизам\slash вляза, връщам се\slash върна се, добирам се\slash добера се, доближавам (се)\slash доближа (се), идвам\slash дойда, достигам\slash достигна, завръщам се, завърна се, ида, навлизам\slash навляза, отивам\slash отида, прибирам се\slash прибера се, приближавам (се)\slash приближа (се), пристигам\slash пристигна, стигам\slash стигна}\\ 
{[NP.Ext]}$_{\feinsub{Thm}}$ {[\_]}$_{\feinsub{Goal}}$ & 36 & \textit{влизам\slash вляза, връщам се\slash върна се, идвам\slash дойда, отивам\slash отида}\\ 
{[NP.Ext]}$_{\feinsub{Thm}}$ {[AVP]}$_{\feinsub{Goal}}$ & 9 & \textit{връщам се\slash върна се, ида, отивам\slash отида, прибирам се\slash прибера се, стигам\slash стигна}\\ 
{[NP.Ext]}$_{\feinsub{Thm}}$ {[NP]}$_{\feinsub{Goal}}$ & 7 & \textit{доближавам\slash доближа, достигам\slash достигна, наближавам\slash наближа, посещавам\slash посетя, стигам\slash стигна}\\ 
\lspbottomrule
\end{tabularx}
    \caption{FrameNet valence patterns of \framename{Arriving} verbs in Bulgarian}
    \label{tab:4:arriving-valence-bg}
\end{table}


\subsubsection{\framename{Departing}}

\subsubsection{Semantic description of the \framename{Departing} frame}

\framename{Departing} describes directed motion away from a starting point, which is encoded in the lexical meaning of the respective LUs.

\begin{description}[font=\normalfont]
\item[Definition of the frame \framename{Departing}:] An object (the \fename{Theme}) moves away from a \fename{Source}. The \fename{Source} may be expressed or it may be understood from context, but its existence is always implied by the departing word itself.

\item[Core frame elements:] \fename{Theme}, \fename{Source}
\end{description}

Being a subframe of \framename{Traversing} that describes the other end point of translational motion, the description of the \framename{Departing} frame mirrors that of \framename{Arriving}, but the profiled FE is the \fename{Source}. The profiling results in the backgrounding of the \fename{Goal} and the \fename{Path} to peripheral FEs and the removal of the remaining route FEs present in the description of \framename{Traversing} (\fename{Path\_shape}, \fename{Distance}, \fename{Direction}, \fename{Area}) from the scenario described by \framename{Departing}.

\subsubsection{Verbs evoking the \framename{Departing} frame}

The basic verbs that evoke the \framename{Departing} frame form a central part of the lexis of \fename{Source}-oriented directed motion: \textit{decamp}, \textit{depart}, \textit{disappear}, \textit{emerge}, \textit{escape}, \textit{exit}, \textit{leave}, \textit{skedaddle}, \textit{vamoose}, \textit{vanish}.

\subsubsection{Syntactic realisation of the frame elements in the \framename{Departing} frame}

The syntactic realisation of the frame elements in the \framename{Departing} frame as represented in the FrameNet corpus examples are illustrated in \tabref{tab:4:departingsynt}. The \fename{Theme} is projected as a subject (NP.Ext), while depending on the verb the \fename{Source} may be expressed as either a prepositional or an adverbial phrase, e.g. \textit{disappear}, \textit{emerge}, \textit{vanish}, on the one hand, or as a direct object (NP.Obj), on the other: \textit{depart}, \textit{escape}, \textit{exit}, \textit{leave}.

Unlike \framename{Arriving} predicates, which show a distinct preference to either NP or PP/AVP \fename{Goals}, the FrameNet data for \framename{Departing} point to different distribution of NP and PP \fename{Sources} across the verbs (\tabref{tab:4:departingsynt}), compare \textit{depart}, where the two types of phrases are equally distributed and \textit{leave}, which favours NP.Obj.

The \framename{Departing} verbs show a similar tendency to leave the profiled element unexpressed (less prominent for the verb \textit{leave}) if it is retrievable from the wider context and/or the movement away takes place with reference to the speaker (i.e. the deictic centre). 

While some \framename{Arriving} verbs, such as \textit{arrive}, \textit{come}, \textit{get} and \textit{return} tend to express the \fename{Goal} as either a PP or an AVP, the \framename{Departing} verbs hardly opt for AVPs, at least in the FrameNet corpus. 

\begin{table}
\caption{Syntactic expression of the \framename{Departing} FEs in FrameNet} 
\label{tab:4:departingsynt}
\begin{tabular}{l cccccccc}   
\lsptoprule
  & NP.Ext & NP.Obj & PP & AVP & NI & Clause & Other & Total\\ \midrule
  %\multicolumn{9}{l}{\textit{decamp} } \\		
%\fename{Theme} & 13  &  &  &  &  &  &  & 13\\		
%\fename{Source} &  &  & 1  &  & 12  &  &  & 13\\		
%\midrule
\multicolumn{9}{l}{\textit{depart} } \\*
\fename{Theme} & 77  &  &  & 1  &  &  &  & 78\\*
\fename{Source} &  & 14  & 11  &  & 52  &  &  & 77\\		
\midrule
\multicolumn{9}{l}{\textit{disappear} } \\*
\fename{Theme} & 120  &  &  &  &  &  &  & 120\\*
\fename{Source} &  &  & 8  &  & 111  &  &  & 119\\		
\midrule
\multicolumn{9}{l}{\textit{escape} } \\*
\fename{Theme} & 16  &  &  &  &  &  &  & 16\\*
\fename{Source} &  & 4  & 2  & 1  & 9  &  &  & 16\\		
\midrule
%\multicolumn{9}{l}{\textit{vamoose} } \\		
%\midrule
\multicolumn{9}{l}{\textit{vanish} } \\*
\fename{Theme} & 69  &  & 1  &  &  &  &  & 70\\*
\fename{Source} &  &  & 12  &  & 57  &  &  & 69\\		
\midrule
%\multicolumn{9}{l}{\textit{skedaddle} } \\		
%\midrule
\multicolumn{9}{l}{\textit{exit} } \\*
\fename{Theme} & 32  &  &  &  &  &  &  & 32\\*
\fename{Source} &  & 5  & 5  &  & 21  & 1  &  & 32\\		
\midrule
\multicolumn{9}{l}{\textit{leave} } \\*
\fename{Theme} & 90  & 1  &  &  &  &  &  & 91\\*
\fename{Source} &  & 45  & 7  & 4  & 29  &  & 3 & 88\\		
\lspbottomrule
%\multicolumn{9}{l}{\textit{emerge} } \\		
%\fename{Theme} & 36  &  &  &  &  &  &  & 36\\		
%\fename{Source} &  &  & 12  &  & 24  &  &  & 36\\		
%\midrule
\end{tabular}
\end{table}

\subsubsection{FrameNet valence patterns}

While the \fename{Goal-DNIs} of the \framename{Arriving} verbs represent 33\% of the overall number of \fename{Goals}, syntactically implicit \fename{Sources} are the prevalent pattern, making up for 63\% of the aggregated number of the patterns with \fename{Source}-phrases (\tabref{tab:4:departing-valence-framenet}). In other words, judging from these data, \fename{Goal}-profiling verbs express syntactically the profiled element twice as frequently as do \fename{Source}-profiling verbs. This observation supports the goal-over-source asymmetry.

The number of the patterns with NP and PP \fename{Sources} is similar, while, as noted above, AVPs, are poorly represented (\tabref{tab:4:departing-valence-framenet}).


\begin{table}
   \begin{tabularx}{\textwidth}{ lrQ } 
   \lsptoprule
   Pattern & \# & Verbs \\ \midrule
   {[NP.Ext]}$_{\feinsub{Thm}}$ {[\_]}$_{\feinsub{Src-DNI}}$  & 312 & \textit{decamp, exit, leave, emerge, disappear, depart, escape, vanish}\\ 
{[NP.Ext]}$_{\feinsub{Thm}}$ {[NP.Obj]}$_{\feinsub{Src}}$  & 68 & \textit{exit, leave, depart, escape}\\ 
{[NP.Ext]}$_{\feinsub{Thm}}$ {[PP]}$_{\feinsub{Src}}$  & 58 & \textit{decamp, exit, leave, emerge, disappear, depart, escape, vanish}\\ 
{[NP.Ext]}$_{\feinsub{Thm}}$ {[AVP]}$_{\feinsub{Src}}$  & 5 & \textit{leave, escape}\\ 
\lspbottomrule
%{[NP.Ext]}$_{\feinsub{Thm}}$  & 3 & \textit{leave, disappear}\\ \midrule
%{[NP.Ext]}$_{\feinsub{Thm}}$ {[NP.Dep]}$_{\feinsub{Src}}$  & 3 & \textit{leave}\\ \midrule
%{[NP.Ext]}$_{\feinsub{Thm}}$ {[\_]}$_{\feinsub{Src-DNI}}$ {[PP]}$_{\feinsub{Thm}}$  & 1 & \textit{vanish}\\ \midrule
%{[NP.Ext]}$_{\feinsub{Thm}}$ {[\_]}$_{\feinsub{Src-DNI}}$ {[AVP]}$_{\feinsub{Thm}}$  & 1 & \textit{depart}\\ \midrule
%{[NP.Ext]}$_{\feinsub{Thm}}$ {[\_]}$_{\feinsub{Src-DNI}}$ {[NP.Obj]}$_{\feinsub{Thm}}$  & 1 & \textit{leave}\\ \midrule
%{[NP.Ext]}$_{\feinsub{Thm}}$ {[Clause]}$_{\feinsub{Src}}$  & 1 & \textit{exit}\\ \midrule
\end{tabularx}
    \caption{FrameNet valence patterns of \framename{Departing} verbs}
    \label{tab:4:departing-valence-framenet}
\end{table}

\subsubsection{Syntactic realisation of \framename{Departing} verbs in Bulgarian}

The Bulgarian verbs evoking the \framename{Departing} frame represent the central lexis of \fename{Source}-oriented directed motion verbs: \textit{заминавам} `depart’, \textit{избягвам} `escape’, \textit{излизам} `exit’, \textit{изчезвам} `disappear’, \textit{напускам} `leave’, \textit{отивам си} `leave’, `go home’, \textit{тръгвам} `leave’, `depart’, \textit{отдалечавам се} `move away’, etc. Most of the Bulgarian \framename{Departing} verbs take a PP or an AVP complement, with few exceptions, such as \textit{напускам} `leave’, which takes an NP.Obj complement.


\begin{table}
\caption{Syntactic expression of the \framename{Departing} FEs in Bulgarian} 
    \label{tab:4:departing-synt-bg}
\begin{tabular}{l ccccccccc}   
\lsptoprule
  & NP.Ext & NP.Obj & PP & AVP & NI & Clause & Other & Total\\ \midrule
\multicolumn{9}{l}{\textit{напускам\slash напусна} `leave’}\\*
\fename{Theme} & 39 &  &  &  &  &  &  & 39\\*
\fename{Source} &  & 38 &  &  & 1 &  &  & 39\\ 
 \midrule
\multicolumn{9}{l}{\textit{тръгвам\slash тръгна} `leave’}\\*
\fename{Theme} & 36 &  &  &  &  &  &  & 36\\*
\fename{Source} &  &  &  &  & 36 &  &  & 36\\*
\fename{Goal} &  &  & 10 & 1 &  &  &  & 11\\*
\fename{Direction} &  &  & 1 & 1 &  &  &  & 2\\
\midrule
\multicolumn{9}{l}{\textit{заминавам\slash замина} `depart’}\\*
\fename{Theme} & 40 &  &  &  &  &  &  & 40\\*
\fename{Source} &  &  & 1 & 1 & 38 &  &  & 40\\*
\fename{Path} &  &  & 1 &  &  &  &  & 1\\*
\fename{Goal} &  &  & 20 &  &  &  &  & 20\\*
\fename{Distance} &  &  & 1 &  &  &  &  & 1\\ 
 \midrule
\multicolumn{9}{l}{\textit{излизам\slash изляза} `exit’}\\*
\fename{Theme} & 39 &  &  &  &  &  &  & 39\\*
\fename{Source} &  &  & 15 & 1 & 23 &  &  & 39\\* 
\fename{Goal} &  &  & 6 & 3 &  & 2 &  & 11\\
 \lspbottomrule
\end{tabular}
\end{table}

\begin{table}
   \begin{tabularx}{\textwidth}{ lrQ } 
   \lsptoprule
   Pattern & \# & Verbs \\ \midrule
{[NP.Ext]}$_{\feinsub{Thm}}$ {[\_]}$_{\feinsub{Src-DNI}}$ & 53 & \textit{заминавам\slash замина, излизам\slash изляза, напускам\slash напусна, тръгвам\slash тръгна}\\ 
{[NP.Ext]}$_{\feinsub{Thm}}$ {[NP]}$_{\feinsub{Src}}$ & 38 & \textit{напускам\slash напусна}\\ 
{[NP.Ext]}$_{\feinsub{Thm}}$ {[PP]}$_{\feinsub{Goal}}$ {[\_]}$_{\feinsub{Src-DNI}}$ & 35 & \textit{заминавам\slash замина, излизам\slash изляза, тръгвам\slash тръгна}\\ 
{[NP.Ext]}$_{\feinsub{Thm}}$ {[PP]}$_{\feinsub{Src}}$ & 15 & \textit{заминавам\slash замина, излизам\slash изляза}\\ 
{[NP.Ext]}$_{\feinsub{Thm}}$ {[AVP]}$_{\feinsub{Goal}}$ {[\_]}$_{\feinsub{Src-DNI}}$ & 4 & \textit{излизам\slash изляза, тръгвам\slash тръгна}\\ 
%{[NP.Ext]}$_{\feinsub{Thm}}$ {[Clause]}$_{\feinsub{Goal}}$ {[\_]}$_{\feinsub{Src-DNI}}$ & 2 & \textit{излизам\slash изляза}\\ \midrule
%{[NP.Ext]}$_{\feinsub{Thm}}$ {[AVP]}$_{\feinsub{Src}}$ & 2 & \textit{заминавам\slash замина, излизам\slash изляза}\\ \midrule
%{[NP.Ext]}$_{\feinsub{Thm}}$ {[PP]}$_{\feinsub{Path}}$ {[\_]}$_{\feinsub{Src-DNI}}$ & 1 & \textit{заминавам\slash замина}\\ \midrule
%{[NP.Ext]}$_{\feinsub{Thm}}$ {[PP]}$_{\feinsub{Goal}}$ {[PP]}$_{\feinsub{Src}}$ & 1 & \textit{излизам\slash изляза}\\ \midrule
%{[NP.Ext]}$_{\feinsub{Thm}}$ {[AVP]}$_{\feinsub{Dir}}$ {[\_]}$_{\feinsub{Src-DNI}}$ & 1 & \textit{тръгвам\slash тръгна}\\ \midrule
%{[NP.Ext]}$_{\feinsub{Thm}}$ {[PP]}$_{\feinsub{Dir}}$ {[\_]}$_{\feinsub{Src-DNI}}$ & 1 & \textit{тръгвам\slash тръгна}\\ \midrule
%{[NP.Ext]}$_{\feinsub{Thm}}$ {[PP]}$_{\feinsub{Distance}}$ {[\_]}$_{\feinsub{Src-DNI}}$ & 1 & \textit{заминавам\slash замина}\\ \midrule
\lspbottomrule
\end{tabularx}
    \caption{FrameNet valence patterns of \framename{Departing} verbs in Bulgarian}
    \label{tab:4:departing-valence-bg}
\end{table}

The data in \tabref{tab:4:departing-valence-bg} support the observations that apart from NP \fename{Sources} (Example \ref{ex:19:c}), the profiled element of the \framename{Departing} frame
 (Example \ref{ex:19:a}) tends to be left out, i.e. it is usually interpreted from the previous or the general context (Example \ref{ex:19:b}).

 In addition, while the peripheral frame element \fename{Source} in the \framename{Arriving} frame is rarely expressed (in fact not present in the data), the peripheral frame element \fename{Goal} in the \framename{Departing} frame (Example \ref{ex:19:d}) was found to be quite frequently expressed and was thus annotated in the Bulgarian examples: in fact, it has as  many occurrences as the profiled FE \fename{Source} (\tabref{tab:4:departing-valence-bg}).

\begin{exe}
\ex  \label{ex:19}
\begin{xlist}
\ex  \label{ex:19:a}
\gll [\textit{\ }]$_{\feinsub{Thm}}$ \textit{Не} \textit{\textbf{ИЗЛИЗАЙ}} [\textit{от} \textit{къщи}]$_{\feinsub{Src}}$. \\
{} Not go-out-IMP.2SG out-of house-DEF.\\
\glt `Don't leave the house.'
\ex\label{ex:19:b}
\gll [\textit{Той}]$_{\feinsub{Thm}}$ \textit{\textbf{ЗАМИНА}} [\ ]$_{\feinsub{Src}}$ \textit{на} \textit{сутринта}. \\
He leave-PST.3SG {} on morning-DEF. \\
\glt `He departed on the following morning.'
\ex\label{ex:19:c}
\gll [\textit{Тя}]$_{\feinsub{Thm}}$ \textit{\textbf{НАПУСНА}} [града]$_{\feinsub{Src}}$ \textit{завинаги}. \\
She leave-PST.3SG city-DEF {for good}. \\
\glt `She left the city for good.'
\ex \label{ex:19:d}
\gll \textit{След} \textit{завършването} [\textit{той}]$_{\feinsub{Thm}}$ \textit{\textbf{ЗАМИНА}} [\ ]$_{\feinsub{Src}}$ [\textit{за} \textit{Париж}]$_{\feinsub{Goal}}$. \\
 After graduating, he leave-PST.3SG {} for Paris. \\
\glt `After his graduation he left for Paris.'
\end{xlist}
\end{exe}

Another fact that emerged from the data is that, even though \fename{Direction} and \fename{Distance} are not specified in the \framename{Arriving} and the \framename{Departing} frame, there are examples that suggest that these FEs are part of the description of the two semantic frames, even if with a peripheral status (Example \ref{ex:20} and Example \ref{ex:21}, respectively).


\begin{exe}
\ex \label{ex:20}
\begin{xlist}
\ex \label{ex:20:а}
\gll [\textit{Корабът}]$_{\feinsub{Thm}}$ \textit{\textbf{ЗАМИНАВА}} [\ ]$_{\feinsub{Src}}$ [\textit{на} \textit{юг}]$_{\feinsub{Dir}}$. \\
Ship-DEF leave-PRS.3SG  {} to south. \\
\glt `The ship leaves south.'
\ex \label{ex:20:b}
\gll [\textit{Тя}]$_{\feinsub{Thm}}$ \textit{\textbf{ЗАМИНА}} [\textit{на 3000 км}]$_{\feinsub{Dist}}$ [\textit{от} \textit{дома}]$_{\feinsub{Src}}$. \\
 She leave-PST.3SG to 3000 km from home. \\
\glt `She went (to live) 3,000 km away from home.'
\end{xlist}
\end{exe}

\begin{exe}
\ex  \label{ex:21}
\begin{xlist}
\ex  \label{ex:21:а}
\gll [\textit{Те}]$_{\feinsub{Thm}}$ \textit{\textbf{ПРИСТИГАТ}} [\ ]$_{\feinsub{Goal}}$ [\textit{от} \textit{юг}]$_{\feinsub{Dir}}$. \\
They arrive-PRS.3PL {} from south. \\
\glt `They arrive from the south.'
\ex \label{ex:21:b}
\gll [\textit{\ }]$_{\feinsub{Thm}}$ \textit{\textbf{ИДВАХА}} [\textit{тук}]$_{\feinsub{Goal}}$ [\textit{отдалече}]$_{\feinsub{Dist}}$. \\
They come-PST.3PL here {from  far away}. \\
\glt `They came here from far away.'
\end{xlist}
\end{exe}


\subsubsection{Traversing}
\subsubsection{Semantic description of the \framename{Traversing} frame}

Traversing represents the complex situation of the motion of a \fename{Theme} with respect to the different locations constituting the route.

\begin{description}[font=\normalfont]
\item[Definition of the frame \framename{Traversing}:] A \fename{Theme} changes location with respect to a salient place, which can be expressed by a \fename{Source}, \fename{Path}, \fename{Goal}, \fename{Area}, \fename{Direction}, \fename{Path\_shape}, or \fename{Distance}.
\end{description}

The frame profiles the middle section of the trajectory of motion of a moving entity, i.e. the \fename{Path}. Its core FEs include the \fename{Path} itself, as well as  elements that represent either an alternative expression of the idea of space covered by the moving entity (such as \fename{Area}) or a characteristic feature of the \fename{Path}. These features may include: \fename{Direction}, which adds the dimension of spatial orientation to the non-directional \fename{Path}; \fename{Distance}, i.e. the length or extent of the trajectory between the starting and the end point; \fename{Path}\_shape -- the form of the \fename{Path}. All of the core FEs that describe the \framename{Traversing} frame are inherited from the most abstract motion frame \framename{Motion\_scenario} which is perspectivised by \framename{Traversing}. 

\subsubsection{Verbs evoking the \framename{Traversing} frame}

As with \framename{Arriving} and \framename{Departing}, there are just a small number of mainly non-derived verbs that evoke the frame: \textit{ascend}, \textit{circle}, \textit{crisscross}, \textit{cross}, \textit{descend}, \textit{hop}, \textit{jump}, \textit{leap}, \textit{mount}, \textit{pass}, \textit{skirt}, \textit{traverse}.


\subsubsection{Syntactic realisation of the frame elements in the \framename{Traversing} frame}

\tabref{tab:4:traversing-synt} illustrates the syntactic expression for a selection of \framename{Traversing} verbs.  The \fename{Theme} is projected as the subject. Among the motion-related FEs, usually it is the profiled \fename{Path} that is expressed syntactically; its favoured realisation is either as a direct object NP, e.g. \textit{ascend}, \textit{cross}, \textit{descend}, \textit{skirt}, or as a prepositional (or adverbial) phrase, e.g. \textit{pass} and \textit{leap}. It can also be left unexpressed (DNI), although the number of unexpressed \fename{Paths} is much fewer than that of the profiled FEs of the \framename{Arriving} and the \framename{Departing} frame.

When the \fename{Area} is expressed, it may also take the place of the direct object: for most of the verbs, these are single occurrences, except for \textit{circle} and \textit{crisscross}: their semantics are consistent with motion along an irregular trajectory over an extended region, which predetermines their preference for the \fename{Area} over the \fename{Path}.  

\fename{Sources} and \fename{Goals} are expressed as prepositional or adverbial phrases; \fename{Direction}, \fename{Distance}, sometimes \fename{Area} (when not an object), although represented by just a few examples, are realised likewise. A small number of exceptions is found with \textit{descend}, where some \fename{Distance}s and \fename{Direction}s are annotated as NP objects (e.g. \textit{descended 300 m}).
\largerpage
\begin{table}
\caption{Syntactic expression of the \framename{Traversing} FEs in FrameNet} \label{tab:4:traversing-synt}\footnotesize
\begin{tabular}{l cccccccc}
 \lsptoprule
  & NP.Ext & NP.Obj & PP & AVP & NI & Clause & Other & Total\\ \midrule
\multicolumn{9}{l}{\textit{traverse} } \\*
\fename{Theme} & 13  &  & 2  &  & 2  &  &  & 17\\*
\fename{Area} &  & 1  &  &  &  &  &  & 1\\*
\fename{Source} &  &  & 2  &  &  &  &  & 2\\*
\fename{Path} & 4  & 4  & 3  &  & 4  & 1  &  & 16\\*
\fename{Goal} &  &  & 6  &  &  &  &  & 6\\*
\fename{Path\_shape} &  &  &  &  & 17  &  &  & 17\\*
\fename{Distance} &  &  &  &  &  &  & 1 & 1\\
%\midrule
%\multicolumn{9}{l}{\textit{skirt} } \\
%\fename{Theme} & 12  &  &  &  &  &  &  & 12\\
%\fename{Path} &  & 8  & 3  &  &  &  & 1 & 12\\
%\fename{Direction} &  &  &  &  &  &  & 1 & 1\\
%%\fename{Goal} &  &  & 2  &  &  &  &  & 2\\
%\midrule
%\multicolumn{9}{l}{\textit{mount} } \\
%\fename{Theme} & 5  &  &  &  &  &  &  & 5\\
%\fename{Path} &  & 3  &  &  & 2  &  &  & 5\\
%\fename{Goal} &  & 2  &  &  &  &  &  & 2\\
\midrule
\multicolumn{9}{l}{\textit{descend} } \\*
\fename{Theme} & 35  &  &  &  & 1  &  &  & 36\\*
\fename{Source} &  &  & 5  &  &  &  &  & 5\\*
\fename{Path} & 1  & 17  & 8  &  & 3  &  &  & 29\\*
\fename{Goal} &  &  & 9  &  &  &  &  & 9\\*
\fename{Path\_shape} &  &  & 1  &  &  &  &  & 1\\*
\fename{Direction} &  &  & 1  & 1  &  &  &  & 2\\*
\fename{Distance} &  & 1  &  & 1  &  &  & 1 & 3\\
\midrule
\multicolumn{9}{l}{\textit{cross} } \\*
\fename{Theme} & 53  &  & 2  &  & 2  &  &  & 57\\*
\fename{Area} &  & 1  &  &  &  &  &  & 1\\*
\fename{Source} &  &  & 4  &  &  &  &  & 4\\*
\fename{Path} & 4  & 26  & 6  & 4  & 16  &  &  & 56\\*
\fename{Goal} &  &  & 16  &  &  &  & 1 & 17\\*
\fename{Direction} &  &  & 4  & 1  &  &  &  & 5\\
\midrule
%\multicolumn{9}{l}{\textit{crisscross} } \\
%\fename{Theme} & 13  &  &  &  &  &  &  & 13\\
%\fename{Area} &  & 13  &  &  &  &  &  & 13\\
%\fename{Path\_shape} &  &  &  &  & 13  &  &  & 13\\
%\midrule
%\multicolumn{9}{l}{\textit{ascend} } \\
%\fename{Theme} & 15  &  &  &  &  &  &  & 15\\
%\fename{Path} &  & 10  & 1  &  & 2  &  &  & 13\\
%\fename{Goal} &  &  & 3  &  & 1  &  &  & 4\\
%\fename{Source} &  &  & 1  &  &  &  & 1 & 2\\
%\midrule
%\multicolumn{9}{l}{\textit{jump} } \\
%\fename{Theme} & 5  &  &  &  &  &  &  & 5\\
%\fename{Area} &  & 1  &  &  &  &  &  & 1\\
%\fename{Path} &  &  & 3  &  &  &  &  & 3\\
%\fename{Source} &  &  & 1  &  &  &  &  & 1\\
%\midrule
\multicolumn{9}{l}{\textit{pass} } \\*
\fename{Theme} & 20  &  &  &  &  &  &  & 20\\*
\fename{Area} &  & 1  &  &  &  &  &  & 1\\*
\fename{Source} &  &  & 1  &  &  &  &  & 1\\*
\fename{Path} &  & 3  & 14  &  &  &  &  & 17\\*
\fename{Direction} &  &  & 1  &  &  &  &  & 1\\
\midrule
\multicolumn{9}{l}{\textit{circle} } \\*
\fename{Theme} & 22  &  &  &  &  &  &  & 22\\*
\fename{Area} &  & 9  &  &  & 6  &  &  & 15\\*
\fename{Path} &  & 1  & 3  &  &  &  &  & 4\\*
\fename{Direction} &  &  & 1  & 1  &  &  &  & 2\\
\lspbottomrule
%\multicolumn{9}{l}{\textit{hop} } \\
%\fename{Theme} & 6  &  &  &  &  &  &  & 6\\
%\fename{Goal} &  &  & 3  &  &  &  &  & 3\\
%\fename{Path} &  &  & 3  &  &  &  &  & 3\\
%\fename{Source} &  &  & 1  &  &  &  &  & 1\\
%\midrule
%\multicolumn{9}{l}{\textit{leap} } \\
%\fename{Theme} & 12  &  &  &  &  &  &  & 12\\
%\fename{Path} &  &  & 10  &  &  &  &  & 10\\
%\fename{Goal} &  &  & 1  &  &  &  &  & 1\\
%\fename{Distance} &  &  & 1  &  &  &  &  & 1\\
%\midrule
\end{tabular}
\end{table}

\fename{Path\_shapes} are almost always implied in the semantics of the verbs but are rarely expressed (as PPs/AVPs).

\subsubsection{FrameNet valence patterns}

The most frequent valence patterns (\tabref{tab:4:traversing-valence-framenet}) show in even more explicit terms that across the different verbs evoking the frame, the non-overt realisation of the \fename{Path} is much rarer, especially when compared with the profiled elements of \framename{Traversing}'s subframes, while NPs and PPs are both well-represented, with variations across the different verbs. Another fact worth noting is that out of the remaining FEs, the \fename{Goal} is the preferred one to be expressed.

\begin{table}
   \begin{tabularx}{\textwidth}{ lrQ }
   \lsptoprule
   Pattern & \# & Verbs \\ \midrule
   {[NP.Ext]}$_{\feinsub{Thm}}$ {[NP.Obj]}$_{\feinsub{Path}}$  & 48 & \textit{descend, ascend, skirt, pass, cross, circle, mount}\\ 
{[NP.Ext]}$_{\feinsub{Thm}}$ {[PP]}$_{\feinsub{Path}}$  & 40 & \textit{descend, ascend, skirt, pass, cross, hop, leap, circle, jump}\\ 
{[NP.Ext]}$_{\feinsub{Thm}}$ {[PP]}$_{\feinsub{Goal}}$ {[NP.Obj]}$_{\feinsub{Path}}$  & 14 & \textit{descend, ascend, skirt, cross}\\ 
{[NP.Ext]}$_{\feinsub{Thm}}$ {[NP.Obj]}$_{\feinsub{Area}}$ {[\_]}$_{\feinsub{Path\_shape-INC}}$  & 13 & \textit{crisscross}\\ 
{[NP.Ext]}$_{\feinsub{Thm}}$ {[\_]}$_{\feinsub{Path-DNI}}$  & 11 & \textit{descend, ascend, cross}\\ 
{[NP.Ext]}$_{\feinsub{Thm}}$ {[NP.Obj]}$_{\feinsub{Area}}$  & 10 & \textit{pass, circle, jump}\\ 
{[NP.Ext]}$_{\feinsub{Thm}}$ {[PP]}$_{\feinsub{Src}}$  & 6 & \textit{descend, hop, jump}\\ 
{[NP.Ext]}$_{\feinsub{Thm}}$ {[\_]}$_{\feinsub{Area-DNI}}$  & 6 & \textit{circle}\\ 
{[NP.Ext]}$_{\feinsub{Thm}}$ {[PP]}$_{\feinsub{Goal}}$ {[\_]}$_{\feinsub{Path-DNI}}$  & 5 & \textit{cross}\\ \lspbottomrule
%{[NP.Ext]}$_{\feinsub{Thm}}$  & 5 & \textit{descend, pass, cross, circle}\\ \midrule
%{[NP.Ext]}$_{\feinsub{Thm}}$ {[PP]}$_{\feinsub{Goal}}$  & 4 & \textit{descend, hop, leap}\\ \midrule
%{[NP.Ext]}$_{\feinsub{Thm}}$ {[PP]}$_{\feinsub{Path}}$ {[PP]}$_{\feinsub{Src}}$  & 3 & \textit{descend, pass, cross}\\ \midrule
\end{tabularx}
    \caption{FrameNet valence patterns of \framename{Traversing} verbs}
    \label{tab:4:traversing-valence-framenet}
\end{table} 

\subsubsection{Syntactic realisation of \framename{Traversing} verbs in Bulgarian}

The central part of the Bulgarian verbs evoking the \framename{Traversing} frame includes predicates such as \textit{минавам} `pass’, \textit{кръстосвам} `crisscross’, \textit{качвам се}, \textit{качвам} `ascend’, \textit{слизам}, \textit{спускам се} `descend’, as well as several verbs produced through derivation, though not necessarily transparent in the contemporary language: \textit{изкачвам се}, \textit{изкачвам} `ascend’, \textit{обикалям}, \textit{заобикалям} `circle’, `skirt’, \textit{пресичам}, \textit{прекосявам} `cross’, `traverse’, \textit{преминавам} `pass’, `pass over’.

In addition, there are a lot of Bulgarian verbs that represent lexicalisations of \fename{Path}-profiling formed by means of prefixation primarily from manner of motion verbs, which will be discussed in the next subsection along with similarly formed \fename{Goal}-profiling and \fename{Source}-profiling verbs. 

\tabref{tab:4:traversing-synt-bg} illustrates the syntactic realisation of several Bulgarian verbs evoking the frame \framename{Traversing}. It can be noted that, like in English, for different verbs the preferred expression of the \fename{Path} may either be a direct object NP, e.g. \textit{пресичам}, \textit{прекосявам} `cross’, `traverse’ or a prepositional (or adverbial) phrase (Example \ref{ex:22:а}), e.g. \textit{пресека} `cross’. \tabref{tab:4:traversing-valence-bg} shows that some of the verbs that may be used both transitively and intransitively, favour the transitive (NP.Obj) realisation. The profiled element tends to be syntactically expressed, rather than left non-overt.

\begin{table}
\caption{Syntactic expression of the \framename{Traversing} FEs in Bulgarian} 
\label{tab:4:traversing-synt-bg}
\begin{tabular}{l cccccccc}   
  \lsptoprule
  & NP.Ext & NP.Obj & PP & AVP & NI & Clause & Other & Total\\ \midrule
\multicolumn{9}{l}{\textit{пресичам\slash пресека} `traverse’ }\\*
\fename{Theme} & 39 &  &  &  &  &  &  & 39\\* 
\fename{Path} &  & 33 & 5 &  & 1 &  &  & 39\\*
\fename{Goal} &  &  & 2 &  &  &  &  & 2\\
 \midrule
\multicolumn{9}{l}{\textit{изкачвам\slash изкача} `ascend’ }\\*
\fename{Theme} & 12 &  &  &  &  &  &  & 12\\*
\fename{Path} &  & 12 &  &  &  &  &  & 12\\*
\fename{Goal} &  &  & 4 &  &  &  &  & 4\\ 
 \midrule
\multicolumn{9}{l}{\textit{прекосявам\slash прекося} `cross’ }\\*
\fename{Theme} & 40 &  &  &  &  &  &  & 40\\* 
\fename{Path} &  & 36 & 4 &  &  &  &  & 40\\
 \midrule
\multicolumn{9}{l}{\textit{преминавам\slash премина} `pass’ }\\*
\fename{Theme} & 15 &  &  &  &  &  &  & 15\\* 
\fename{Path} &  & 4 & 11 &  &  &  &  & 15\\
 \midrule
\multicolumn{9}{l}{\textit{изкачвам се\slash изкача се} `ascend’ }\\*
\fename{Theme} & 9 &  &  &  &  &  &  & 9\\*
\fename{Path} &  &  & 4 &  & 5 &  &  & 9\\*
\fename{Goal} &  &  & 6 &  &  &  &  & 6\\ 
 \lspbottomrule
\end{tabular}
\end{table}

%\newpage

The verbs \textit{качвам}, \textit{изкачвам} `ascend’ are always transitive (Example \ref{ex:22:b}), while \textit{качвам се}, \textit{изкачвам се} `ascend’ and \textit{спускам се}, \textit{слизам} `descend’ are always intransitive (Examples \ref{ex:22:c}, \ref{ex:22:d}).

\begin{exe}
\ex  \label{ex:22}
\begin{xlist}
\ex  \label{ex:22:а}
\gll [\textit{Те}]$_{\feinsub{Thm}}$ \textit{\textbf{ПРЕСЯКОХА}} [\textit{(през)} \textit{двора}]$_{\feinsub{Path}}$. \\
They cross-PST.3PL (through) yard-DEF. \\
\glt `The boys crossed the yard.'
\ex\label{ex:22:b}
\gll [\textit{Те}]$_{\feinsub{Thm}}$ \textit{\textbf{ИЗКАЧИХА}} [\textit{планината}]$_{\feinsub{Path}}$. \\
They climb-PST.3PL mountain-DEF. \\
\glt `They climbed the mountain.'
\ex \label{ex:22:c}
\gll [\textit{Те}]$_{\feinsub{Thm}}$ {\textit{\textbf{СЕ}} \textit{\textbf{ИЗКАЧИХА}}} [\textit{по} \textit{планината}]$_{\feinsub{Path}}$. \\
They climb-REFL.PST.3PL on mountain-DEF. \\
\glt `They climbed the mountain.'
\ex \label{ex:22:d}
\gll [\textit{Те}]$_{\feinsub{Thm}}$ {\textit{\textbf{СЛИЗАТ}}} [\textit{по} \textit{стълбите}]$_{\feinsub{Path}}$. \\
They climb-PRS.3PL down stairs-DEF. \\
\glt `They descended the stairs.'
\end{xlist}
\end{exe}

%\begin{exe}
%\ex \label{ex:23}
%\begin{xlist}
%\ex \label{ex:23:а}
%\gll [\textit{Те}]$_{\feinsub{Thm}}$ \textit{прекосиха} [\textit{тревата}]$_{\feinsub{Path}}$. \\
%[They]$_{\feinsub{Thm}}$ crossed [meadow-DEF]$_{\feinsub{Path}}$. \\
%\glt `They crossed the meadow.'
%\ex \label{ex:23:b}
%\gll [\textit{\_}]$_{\feinsub{Thm}}$ \textit{Прекосих} [\textit{Колумбия} \textit{надлъж} % \textit{и} \textit{нашир}]$_{\feinsub{Path}}$. \\
%[\ ]$_{\feinsub{Thm}}$ traversed [Columbia far and wide]$_{\feinsub{Path}}$. \\
%\glt `They traversed Columbia far and wide.'
%\end{xlist}
%\end{exe}


%In reality the two FEs are difficult to distinguish in the context except with verbs whose semantics provide additional information. Such verbs are, for instance, descend and circle. Descend incorporates the FE \fename{Direction}, so it implies directional motion along a route, i.e. the FEs it takes as direct objects are usually \fename{Path}s (or other aspects of such motion, see the example of \fename{Distance} as a direct object NP above); circle involves circular motion around an expanse, which is better construed as \fename{Area}.

Although on a very small scale due to the size of the sample, the valence patterns show that \fename{Goals} are also realised syntactically (\tabref{tab:4:traversing-valence-bg} and Example \ref{ex:24:a}). \fename{Sources} (Example \ref{ex:24:c}) and \fename{Directions} (Example \ref{ex:24:b}) as well as combinations of motion-related FEs (Example \ref{ex:24:c}) are also attested as individual occurrences in the data.

\begin{table} 
   \begin{tabularx}{\textwidth}{ lrQ } 
   \lspbottomrule
   Pattern & \# & Verbs \\ \midrule
{[NP.Ext]}$_{\feinsub{Thm}}$ {[NP.Obj]}$_{\feinsub{Path}}$ & 79 & \textit{прекосявам\slash прекося, пресичам\slash пресека, преминавам\slash премина, изкачвам\slash изкача}\\ 
{[NP.Ext]}$_{\feinsub{Thm}}$ {[PP]}$_{\feinsub{Path}}$ & 23 & \textit{прекосявам\slash прекося, пресичам\slash пресека, преминавам\slash премина, изкачвам се\slash изкача се}\\ 
{[NP.Ext]}$_{\feinsub{Thm}}$ {[NP.Obj]}$_{\feinsub{Path}}$ {[PP]}$_{\feinsub{Goal}}$ & 6 & \textit{пресичам\slash пресека, изкачвам\slash изкача}\\ 
{[NP.Ext]}$_{\feinsub{Thm}}$ {[PP]}$_{\feinsub{Goal}}$ {[\_]}$_{\feinsub{Path-DNI}}$ & 5 & \textit{изкачвам се\slash изкача се}\\ 
%{[NP.Ext]}$_{\feinsub{Thm}}$ {[PP]}$_{\feinsub{Goal}}$ {[PP]}$_{\feinsub{Path}}$ & 1 & \textit{изкача се}\\ 
%{[NP.Ext]}$_{\feinsub{Thm}}$ {[\_]}$_{\feinsub{Path-DNI}}$ & 1 & \textit{пресичам\slash пресека}\\ 
\lspbottomrule
\end{tabularx}
    \caption{FrameNet valence patterns of \framename{Traversing} verbs in Bulgarian}
    \label{tab:4:traversing-valence-bg}
\end{table}
 
\begin{exe}
\ex \label{ex:24}
\begin{xlist}
\ex \label{ex:24:a}
\gll [\textit{Тя}]$_{\feinsub{Thm}}$ \textit{\textbf{ПРЕКОСИ}} [\textit{полето}]$_{\feinsub{Path}}$ [\textit{до} \textit{крепостта}]$_{\feinsub{Goal}}$. \\
She cross-PST.3SG field-DEF to fortress-DEF. \\
\glt `She crossed the field towards the fortress.'
\ex  \label{ex:24:b}
\gll [\textit{Той}]$_{\feinsub{Thm}}$ \textit{\textbf{ПРЕКОСИ}} [\textit{залата}]$_{\feinsub{Path}}$ [\textit{по} \textit{посока} \textit{на}  \textit{вратата}]$_{\feinsub{Dir}}$. \\
He cross-PST.3SG hall-DEF in direction of door-DEF. \\
\glt `He crossed the hall towards the door.'
\ex  \label{ex:24:c}
\gll [\textit{Тя}]$_{\feinsub{Thm}}$ \textit{\textbf{ПРЕСЕЧЕ}} [\textit{моста}]$_{\feinsub{Path}}$ [\textit{от} \textit{мидълсекския}]$_{\feinsub{Src}}$ [\textit{към} \textit{сърейския} \textit{бряг}]$_{\feinsub{Goal}}$. \\
Тя traverse-PST.3SG bridge-DEF from Middlesex-DEF to Surrey-DEF shore. \\
\glt `She traversed the bridge from the Middlesex to the Surrey shore.'
\end{xlist}
\end{exe}

The verb \textit{слизам} `descend’ can also co-occur with \fename{Distances} that may be expressed as measurement NPs (Example \ref{ex:25}).
%may be annotated as NP objects (on the example of the English descended 300 m). However, as spuskam se (always intransitive as transitivity is blocked by the reflexive particle se \cite{}) shows such NPs cannot be direct objects, but rather measurement NPs; even more they cannot passivise. 

\begin{exe}
\ex  \label{ex:25}
%\begin{xlist}
%\ex  \label{ex:25:а}
\gll [\textit{Те}]$_{\feinsub{Thm}}$ {\textit{\textbf{СЕ}} \textit{\textbf{СПУСКАТ}}} [\textit{300} \textit{м}]$_{\feinsub{Dist}}$. \\
They {climb-PRS.3PL down} 300 m. \\
\glt `They descend 300 m.'
%\end{xlist}
\end{exe}


\subsubsection{Derivation of directional motion verbs}\label{prefixes}

It has been well-established in the literature that part of the verbal prefixes in the Slavic languages yield (resultative) prefixed verbs when attached to unprefixed (simplex) verbs (\cite{Beavers2010,Pantcheva-2007,Pantcheva2007,Pantcheva2011, Palmer2009,SpencerZaretskaya1998,Svenonius2005}, among many others), see also \citet[178--184]{VanValinLaPolla1997} for other languages. Regardless of the theoretical framework adopted and the specifics of the treatment of such verbs, the mechanism involves a verb with a simple internal (event, lexical semantic, logical) structure to which a prefix is attached so as to add a resultative subevent, thus producing a verb describing a more complex eventuality. 

A typical example in the domain of motion is the prefixation of manner of motion verbs using directional prefixes, which, depending on the prefix, leads to the formation of \fename{Goal}-profiled, \fename{Source}-profiled or \fename{Path}-profiled predicates. As noted earlier, besides the verbs discussed in the previous sections, most of which are underived verbs with a primary directional motion meaning, there are a number of prefixed predicates derived mainly from simplex manner of motion verbs (belonging themselves to frames such as \framename{Motion}, \framename{Self\_motion}, \framename{Fluidic\_motion}, among others), which also evoke the frames \framename{Traversing}, \framename{Arriving} and \framename{Departing}, and possibly other frames profiling the elements of the route of motion. 

\begin{table}
\footnotesize
\begin{tabularx}{\textwidth}{ p{\widthof{\textbf{Self\_motion }}} QQQ} 
\lsptoprule
\framename{Self\_motion} & \fename{Source}-profiled & \fename{Goal}-profiled & \fename{Path}-profiled \\
\midrule
\textit{летя} `fly’ & \textit{отлитам} `fly away’, \textit{излитам}, \textit{политам} `fly off’, `take off’ & \textit{долитам} `fly (up) to’, \textit{влитам} `fly into’ & \textit{прелитам} `fly over’ \\
\midrule
\textit{хвърча} `fly’ & \textit{отхвърчавам} `fly away’, \textit{изхвърчавам} `fly off’ & \textit{дохвърчавам} `fly (up) to’ & \textit{прехвърчавам} `fly over’ \\
\midrule
\textit{бягам} `run’ & \textit{избягвам} `run away’ & \textit{добягвам} `run (up) to’ (dialect) & \textit{пробягвам}, \textit{пребягвам} `run’, `cover distance by running'

\textit{пребягвам} `run across’ \\
\midrule
\textit{тичам} `run’ & \textit{изтичвам} `run out of’ & \textit{дотичвам} `run (up) to’ & \textit{претичвам} `cover distance by running' \\
\midrule
\textit{пълзя} `crawl’ & \textit{изпълзявам} `crawl out’ & \textit{допълзявам} `crawl (up) to’

\textit{пропълзявам} `crawl in’, `crawl onto’

\textit{впълзявам} `crawl into’ & \textit{препълзявам}, \textit{пропълзявам} `crawl across’, \textit{пропълзявам} `cover distance by crawling' \\
\midrule
\textit{скачам} `jump’ & \textit{изскачам} `jump out’ & \textit{доскачам} `jump (up) to’ & \textit{прескачам} `jump’, `pass over’ \\
\midrule
\textit{плувам} `swim’ & \textit{изплувам} `swim up’, `swim to the surface’ & \textit{доплувам} `swim (up) to’

\textit{вплувам} `swim into’ & \textit{преплувам} `swim across’

\textit{проплувам} `cover distance by swimming' \\
\midrule
\textit{нижа се} `file’ & \textit{изнизвам се} `file out’ &  &  \\
\midrule
\textit{газя} `wade’ &  & \textit{догазвам} `wade (up) to’ & \textit{изгазвам}, \textit{прегазвам} `pass through some substance by wading'

\textit{изгазвам} `cross by wading' \\
\midrule
\textit{танцувам} `dance’ &  & \textit{дотанцувам} `dance (up) to’ &  \\
\midrule
\textit{клатушкам се} `totter’ &  & \textit{доклатушквам се} `totter (up) to’ &  \\
\midrule
\textit{куцам}, \textit{куцукам} `limp’ &  & \textit{докуцвам}, \textit{докуцуквам} `limp (up) to’ &  \\
\lspbottomrule
\end{tabularx}
\caption{Prefixal derivation of directed motion verbs from manner of motion verbs in Bulgarian} 
\label{tab:4:big-table-of-prefixed-verbs}
\end{table}

\tabref{tab:4:big-table-of-prefixed-verbs} shows the productivity of this pattern. The inventory of verbs evoking semantic frames profiling different elements of the route, is much richer than in English, where similar meanings may be encoded either by manner of motion verbs which have developed a more complex event structure and meaning (Example \ref{ex:26:a}) or by means of certain syntactic constructions (Example \ref{ex:26:b}). %The consequence for the FrameNet organisation is that senses such as fly the Atlantic should be accommodated in one way or another. The Bulgarian verbs without correspondences in English which evoke the respective frames need to be included.

\begin{exe}
\ex  \label{ex:26}
\begin{xlist}
\ex  \label{ex:26:a}
[\textit{He}]$_{\feinsub{SMov}}$ \textit{was the first to \textbf{FLY}} [\textit{the Atlantic}]$_{\feinsub{Path}}$.
\ex  \label{ex:26:b}
[\textit{He}]$_{\feinsub{SMov}}$ \textit{\textbf{LIMPED}} [\textit{to the store}]$_{\feinsub{Goal}}$.
\end{xlist}
\end{exe}

\section{Conclusions}

%I discussed the principles of the internal organisation of the lexis of motion and how these principles project into a system of related frames. This was followed by a more detailed analysis of the definition of several related frames in terms of how frame-to-frame relations are implemented and how inheritance and other frame relations yield certain configurations of FEs (partly) inherited from more general frames; the focus was on showing what specialisations too place in the more specific frames.

In this chapter particular attention has been paid to the expression of the FEs that define the elements of the route traversed (\fename{Source}, \fename{Goal}, \fename{Path}) or region covered (\fename{Area}) by the moving entity and prominent aspects of the route such as the \fename{Distance} it spans, the \fename{Direction} it takes or the form it has (\fename{Path\_shape}).

I showed and commented on the semantic specification, syntactic expression and valence patterns typical of manner of motion and directed motion verbs by analysing the examples in the FrameNet corpus and expanding the observations to Bulgarian examples. 

Manner of motion verbs tend to express the \fename{Path} over the \fename{Goal} and especially over the \fename{Source}, but the particular distribution of the various patterns varies across verbs. \fename{Path} is especially prominent where complex notions of motion or trajectory are involved. 
%They also show preference for the peripheral FE Manner.

The data corroborate the observations made in the literature, that all other things being equal, there is a bias for expressing \fename{Goals} over \fename{Sources}, a tendency which has been studied for many typologically distinct languages. In particular, if the verbs do not profile a particular aspect of the route, they tend to express \fename{Goals} over \fename{Sources}, the intuition being that motion through space involves getting to some place, even with manner of motion verbs, and that, in this respect, the end point of the motion is a more salient feature than the starting point. 

Across verbs that profile a particular aspect of the route, the profiled FE is the one that tends to be expressed, i.e. \fename{Source}-profiling verbs co-occur more frequently with \fename{Source} expressions than verbs that do not profile this FE, \fename{Goal}-profiling verbs co-occur with \fename{Goal} expressions. While these two aspects have been of primary interest in the linguistic literature, similar observations may be made for \fename{Path} and to a lesser extent for \fename{Area} (as the examples are fewer), judging from the data. 

\fename{Distances} and \fename{Directions} are rarely expressed and at least in some cases they show to be syntactically, as well as semantically dependent on the \fename{Path}, as they represent elaborations on certain aspects of it (deictic or geographical orientation or the length of the route covered). 

Other elements of the route may be expressed besides or instead of the profiled one. %\fename{Goal}-profiling (\framename{Arriving}) verbs and 
\fename{Source}-profiling (\framename{Departing}) verbs tend to realise \fename{Goals} or \fename{Paths}, but the preference for one over the other varies across verbs and the examples are not always definitive. \fename{Path}-profiling verbs tend to favour \fename{Goals} over \fename{Sources}. In addition, the following was observed in the Bulgarian data: peripheral \fename{Goals} may be expressed on a par with profiled \fename{Sources}. Each of these observations warrants further investigation, especially with respect to the frequency and means of expression (including the available inventories) of various FE combinations within and across verbs and frames.

While only marked in passing, the productivity of prefixal derivation as a mechanism of deriving directed motion verbs from other motion verbs, especially from manner verbs, in Bulgarian (and other Slavic and non-Slavic languages) points to the need for these verbs to be systematically addressed within the FrameNet structure. This may also result in the definition of frame-to-frame relations that account for this systematicity. 

\section*{Abbreviations}
\begin{multicols}{2}
\begin{tabbing}
MMMM \= Adverbial phrase\kill
AVP \> Adverbial phrase\\
CNI \> Constructional null \\ \> instantiation\\
DEF \> Definite form\\
\scshape Dir \> \fename{Direction}\\
\scshape Dist \> \fename{Distance}\\
DNI \> Definite null instantiation\\
FE \> frame element\\
IMPF \> Imperfective aspect\\
INDF \> Indefinite form\\
INI \> Indefinite null instantiation\\
LU \> Lexical unit\\
NEG \> Negative form\\
NP \> Noun phrase\\
NP.Ext \> Subject NP\\
NP.Obj \> Object NP\\
PL \> Plural\\
PP \> Prepositional phrase\\
PRS \> Present tense\\
PST \> Past tense\\
PWN \> Princeton WordNet\\
SG \> Singular\\
\scshape SMov \> \fename{Self\_mover}\\
\scshape Src \> \fename{Source}\\
\scshape Thm \> \fename{Theme}
\end{tabbing}
\end{multicols}

\section*{Acknowledgements}

This research is carried out as part of the project \emph{Enriching Semantic Network WordNet with Conceptual Frames} funded by the Bulgarian National Science Fund, Grant Agreement No. KP-06-H50/1 from 2020.

{\sloppy\printbibliography[heading=subbibliography,notkeyword=this]}
\end{document}

