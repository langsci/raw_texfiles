\documentclass[output=paper,colorlinks,citecolor=brown]{langscibook}
\ChapterDOI{10.5281/zenodo.15682188}
\author{Ivelina Stoyanova\orcid{0000-0003-3771-435X}\affiliation{Department of Computational Linguistics, Institute for Bulgarian Language, Bulgarian Academy of Sciences}
}

\title{The complex conceptual structure of verbs of change}

\abstract{The study focuses on verbs of change and their description in two large lexical-semantic resources (WordNet and FrameNet) with a view to Bulgarian. The synonym sets (or synsets) encoding verbs of change in WordNet and their assigned semantic frames from FrameNet are studied in order to obtain a hierarchical organisation of causative and non-causative (inchoative) semantic frames aiming at a classification of the verbs with respect to their semantic and syntactic features. The main objective is to uncover the systemic semantic relations in each of the resources and employ them for the purpose of the comprehensive conceptual description of verbs of change. The work can contribute to the enrichment of WordNet with conceptual information and can support the study of semantic specialisation of verb meanings.}



\IfFileExists{../localcommands.tex}{
   \addbibresource{../localbibliography.bib}
   % add all extra packages you need to load to this file

\usepackage{tabularx,multicol}
\usepackage{url}
\urlstyle{same}

\usepackage{listings}
\lstset{basicstyle=\ttfamily,tabsize=2,breaklines=true}

\usepackage{langsci-basic}
\usepackage{langsci-optional}
\usepackage{langsci-lgr}
\usepackage{langsci-osl}
% \usepackage{./langsci/styles/langsci-lgr}
% \usepackage{./langsci/styles/langsci-osl}
% \usepackage{langsci-gb4e}

\usepackage{tikz}
\usetikzlibrary{patterns,calc}
\pgfdeclarepatternformonly{south east lines}{\pgfqpoint{-0pt}{-0pt}}{\pgfqpoint{3pt}{3pt}}{\pgfqpoint{3pt}{3pt}}{
    \pgfsetlinewidth{0.6pt}
    \pgfpathmoveto{\pgfqpoint{0pt}{3pt}}
    \pgfpathlineto{\pgfqpoint{3pt}{0pt}}
    \pgfpathmoveto{\pgfqpoint{.2pt}{-.2pt}}
    \pgfpathlineto{\pgfqpoint{-.2pt}{.2pt}}
    \pgfpathmoveto{\pgfqpoint{3.2pt}{2.8pt}}
    \pgfpathlineto{\pgfqpoint{2.8pt}{3.2pt}}
    \pgfusepath{stroke}}
    
\usepackage{stmaryrd}
\usepackage{wasysym}
\usepackage{multirow}
\usepackage{caption}
\usepackage{subcaption}
\usepackage{mathrsfs}
\usepackage{qtree}

\usepackage{linguex}


   %pminos do not split footnotes
% \interfootnotelinepenalty=10000 %Footnote in Laporte chapters has to be split SN


%\DeclareIndexNameFormat{default}{%
%\nameparts{#1}%
%\usebibmacro{index:name}%
%{\index[names]}%
%{\namepartfamily}%
%{\namepartgiveni}%
% {}% L1
% {}% L2
%{\namepartprefix}% generates spurious space L3
%{\namepartsuffix}% generates spurious space L4
%}

%  {\DeclareIndexNameFormat{default}{%
%     \usebibmacro{index:name}{\index[names]}{#1}{#3}{#5}{#7}}}

%\DeclareIndexNameFormat{default}{%
%  \usebibmacro{index:name}{\sindex[nom]}{#1}{#3}{#5}{#7}}

%\DeclareIndexNameFormat{default}{%
%  \usebibmacro{index:name}{\sindex[person]}{#1}{#3}{#5}{#7}}
%\DeclareIndexNameFormat{default}{%
%\nameparts{#1} \usebibmacro{index:name}{\sindex[person]]}{\namepartfamily}{‌​\namepartgiven}{\nam‌​epartprefix}{\namepa‌​rtsuffix}}

%\newcommand{\smiley}{:)}

%\renewbibmacro*{index:name}[5]{%
%\usebibmacro{index:entry}{#1}%
%{\iffieldundef{usera}{}{\thefield{usera}\actualoperator}\mkbibindexname{#2}{#3}{#4}{#5}}}

% \newcommand{\noop}[1]{}

%remove for final
%\overfullrule=1mm

\newcommand{\tobi}[2]}}
\renewcommand{\S}[1]{\tobi{#1}{\textsc{*}}}

% this volume references
% puts: [this volume]
% already defined: \citetv
%\newcommand{\citepv}[1]{(\citeauthor{#1} \citeyear*{#1} [this volume])}
\newcommand{\citealtv}[1]{\citeauthor{#1} \citeyear*{#1} [this volume]}

%parentheses around example number
\newcommand{\pref}[1]{(\ref{#1})}

% in-text examples

\newcommand{\lnex}[1]{\textit{#1}} %target lang word
\newcommand{\lnlit}[1]{(lit.: `#1')} %literal reading
\newcommand{\lnlat}[1]{(#1)} % latinization
\newcommand{\lntrans}[1]{`#1'} %translation
\newcommand{\lnexl}[2]%
{\lnex{#1}{} \lnlat{#2}} % ex with latinization
\newcommand{\lnexlat}[3]{\lnex{#1}{} \lnlat{#2}{} \lntrans{#3}} % ex with latinization and tranl.

%ch01
\newcommand{\co}[1]{\mbox{\textbf{#1}}}

%ch09

\newcommand{\cyrbulg}[1]{\begin{otherlanguage*}{bulgarian}#1\end{otherlanguage*}}


%ch10
\newcommand{\nlp}{{\small NLP}}
\newcommand{\mwe}{{\small MWE}}
\newcommand{\rae}{{\small RAE}}
\newcommand{\lvc}{{\small LVC}}
\newcommand{\pos}{{\small P}o{\small S}}
%\newcommand{\todo}[1]{ \textcolor{red}{#1} }

%\renewcommand{\labelenumi}{\theenumi}
%\ainamefmt{{vv}{ll}{, ff}{, jj}} % fullname

\newcommand{\biberror}[1]{{\color{red}#1}}

\newcommand{\osenovaitem}{--~}
   %% hyphenation points for line breaks
%% Normally, automatic hyphenation in LaTeX is very good
%% If a word is mis-hyphenated, add it to this file
%%
%% add information to TeX file before \begin{document} with:
%% %% hyphenation points for line breaks
%% Normally, automatic hyphenation in LaTeX is very good
%% If a word is mis-hyphenated, add it to this file
%%
%% add information to TeX file before \begin{document} with:
%% %% hyphenation points for line breaks
%% Normally, automatic hyphenation in LaTeX is very good
%% If a word is mis-hyphenated, add it to this file
%%
%% add information to TeX file before \begin{document} with:
%% \include{localhyphenation}
\hyphenation{
    Beck-man
    Ngu-yen
    back-chan-nel
    back-chan-nels
    mo-not-o-nous
    ste-reo-typ-i-cal
}

\hyphenation{
    Beck-man
    Ngu-yen
    back-chan-nel
    back-chan-nels
    mo-not-o-nous
    ste-reo-typ-i-cal
}

\hyphenation{
    Beck-man
    Ngu-yen
    back-chan-nel
    back-chan-nels
    mo-not-o-nous
    ste-reo-typ-i-cal
}

   \boolfalse{bookcompile}
   \togglepaper[23]%%chapternumber
}{}

\begin{document}

\maketitle

\section{Introduction}

Research in the field of conceptual description of verbs develops in several directions. On the one hand, there are theoretical studies on the classification of predicates and their argument structure, and, on the other hand, applied works on the compilation of language resources and computer applications for semantic and syntactic analysis, which, in turn, contribute to developing methods and applications for language understanding and generation, machine translation, etc.

The observations are focused on the verbs in Princeton WordNet \citep{Fellbaum1998}, as well as in the Bulgarian WordNet \citep{Koeva2010-wordnet,koeva2021-wordnet}, and the corresponding semantic frames from FrameNet \citep{Baker1998} that describe them, along with the semantic relations between the frames and their frame elements. 

The main objective of this study is to offer a description framework and, based on that, to propose a shallow classification within the semantic class of verbs of change using a set of lexical, semantic and syntactic features. The description of verbs of change is based on several key semantic features -- type of change (causal or internalised), scalability of the property of change (quantised or non-quantised change), and the frame elements describing the change. The study relies on previous well-established semantic classifications of verbs, and \restructured{the contribution is to offer a classification scheme that covers the verbs in WordNet, exploring its relational structure and semantic description. The classification offered is based on FrameNet frames which group together verbs with similar conceptual properties and syntactic behaviour.}

The analysis also aims at defining the class of change predicates within established classifications of predicates, in particular the classification of \citet{Levin1993} and other classifications stemming from it, by discussing some key features determining the semantic and syntactic properties of verbs. One relevant aspect of the analysis of verbs of change is the relation of causativity between pairs of senses (synonym sets, or synsets in WordNet's structure) denoting a causative and a non-causative/inchoative (internal) change, and between the frames in FrameNet that describe the relevant semantics. Causative and inchoative verbs and the respective frames they evoke exhibit similarities in the definitions and the correspondence between the core frame elements involved, their semantic type and the restrictions on their realisation. The typology of the complex semantics of causative and inchoative verbs outlines the specific features of the semantic classes of the verbs of change and sheds light on their syntactic expression. The analysis also attempts to point out the types of change and the aspects of the lexical meaning of the verbs that block the inchoative or the causative interpretation.

Another aspect of the analysis focuses on the distinction between quantised and  non-quantised change as an inherent feature of the verb's semantic interpretation. Verbs of non-quantised change (such as \textit{die, be born, fall, arrive}, and many verbs of inherently directed motion) are telic, entailing a specific terminal state. On the other hand, verbs of  quantised change (such as \textit{bend, boil, freeze}) can have either a telic or an atelic reading depending on whether the \fename{Theme} is explicitly presented as incremental. The distinction between quantised and non-quantised change is important for the analysis of the syntactic behaviour of verbs both in terms of their grammatical features and the realisation of their frame elements.

The result of the work is a system of verb classes evoking a system of frames, related to each other by means of frame-to-frame relations, that describe the semantics of the verbs of change along with the relevant aspects of the change involved and the specifics of their conceptual and syntactic structure. \added{The set of semantic frames from FrameNet with their frame-to-frame relations provide a classification scheme which is applied with the purpose to classify the verbs of change in WordNet and the Bulgarian WordNet. Although the semantic frames and their description with the relevant frame elements are adopted from FrameNet, they are additionally organised and aligned in order to reflect the structure of the semantic class of verbs of change as represented in WordNet with respect to their lexical and semantic properties.} Further, the syntactic valence patterns associated with these verbs are discussed briefly.

The structure of the paper is as follows. \sectref{sec:related} shows an overview of relevant theoretical studies focused on verb semantics and their conceptual description. \sectref{ch3:sec:resources} briefly presents the resources employed in the study -- WordNet (Princeton WordNet and Bulgarian WordNet) and FrameNet, as well as the sources of usage examples -- the SemCor and BulSemCor semantically annotated corpora. \sectref{sec:semproperties} focuses on some of the key properties relevant for the description and the classification of the verbs of change -- telicity (\sectref{sec:telicity}), causativity (\sectref{sec:causative}), and the hierarchy of frame elements in their semantic frames (\sectref{sec:roles}). \sectref{sec:classification} discusses an approach towards the semantic classification of the verbs of change with a view to their semantic properties and syntactic realisation. Finally, some conclusions are drawn in \sectref{sec:conclusions}.

\section{Related works}\label{sec:related}

\added{The objective of the literature review is two-fold: to outline the place of the class of verbs of change within known classifications of verb predicates, and to summarise the discussion on the set of lexical, semantic and syntactic properties that determine their realisation.}

The classification of predicates and the semantic relations between predicates and their arguments have been studied within various theoretical approaches, based in general on their syntactic properties and behaviour (\cite{Levin1993,Pinker1989}, among others), the thematic structure \citep{VanValinLaPolla1997} or the semantics of frames \citep{Fillmore1982}. In particular, verbal argument realisation behaviour within and across languages has been studied actively in the last two decades (\cite{RappaportHovavLevin2010,Boas2010,BeaversKoontz-Garboden2012,Levin2015,Dux2020}, among others).

\restructured{Although semantic roles and their hierarchies have been researched and applied extensively for the purposes of lexical semantic representation (most prominently, \cite{fillmore1968case}, \cite{Jackendoff1972}, among others), the role-based approaches for representing verb syntactic behaviour have shown many shortcomings discussed in a number of studies (\cite[553--559]{dowty1991thematic}, \cite[38--44]{LevinRappaportHovav2005},  \cite[711]{Fillmore2012}, \cite[28--29]{Dux2020}).}

\restructured{This gave rise to event-structural approaches to study of the syntactic behaviour of verbs based on the situations they describe, motivated by both the verb's lexical meaning and its realisation in terms of grammatical features and valence combinations.}

With respect to analysing the situations described by the verbs, as a point of departure, the present study adopts Vendler's aspectual classes of activities, achievements, accomplishments and states \citep{Vendler1957,Vendler1967}. Vendler based this taxonomy on the analysis of English verbs but its validity has been confirmed across many different languages.

\citet[37--132]{Dowty1979} relates word meaning to Vendler's aspectual classes by means of lexical decompositions which aim to reflect the logical structures of sentences. However, the classes defined by decompositions do not translate uniformly into the classes defined in aspectual terms \citep[16--20]{RappaportHovav2008}. The decompositions show that some verbs can be classified in more than one aspectual class depending on the use. For example, some verbs can be both activities and accomplishments (e.g. \textit{walk\slash walk to the store}) and some others -- both  states and achievements (e.g. the ambiguity of many mental state verbs such as \textit{recognise, understand, know}). As a consequence, the aspectual classes are considered at the VP level rather than at the lexical level, which means that the aspectual properties are expressed in a complex lexical, morpho-syntactic and valence-related way by the verb and its arguments. 

\added{These approaches to representing the structure of the situations described by the verbs are taken into account in the present study in order to outline the place of the class of verbs of change in comparison to other verb classes, with respect to their lexical, semantic and grammatical properties which determine their behaviour. We also consider the logical structure of the predicates under observation in connection with the main participants in the described situation as represented by the configuration of frame elements in the semantic frame evoked by the verb.}

\citet{Fillmore1970} focuses on verbs of change and their semantic and syntactic behaviour by analysing extensively the use of verbs like \textit{hit} and \textit{break}. He discusses three uses of the verb \textit{break} (see Example \ref{ex1}) and only two uses of the verb \textit{hit} (see Example \ref{ex2}). While senses in Examples \ref{ex1b} and \ref{ex1c} relate to \ref{ex2a} and \ref{ex2b}, respectively, the verb \textit{hit} does not have an intransitive use like the verb \textit{break} in Example \ref{ex1a}.

\begin{exe}
 \ex \label{ex1}
     \begin{xlist}
         \ex[]{\label{ex1a}\textit{The stick broke.}}
         \ex[]{\label{ex1b}\textit{John broke the stick (with a rock).}}
         \ex[]{\label{ex1c}\textit{The rock broke the stick.}}
         \ex[*]{\label{ex1d}\textit{The rock broke the stick with a hammer.}}
     \end{xlist}

 \ex \label{ex2}
     \begin{xlist}
         \ex[]{\label{ex2a}\textit{John hit the tree (with a rock).}}
         \ex[]{\label{ex2b}\textit{A rock hit the tree.}}
         \ex[*]{\label{ex2c}\textit{A rock hit the tree with a stick.}}
     \end{xlist}
\end{exe}

Furthermore, \citet[131--132]{Fillmore1970} states that while \textit{break} is a change-of-state predicate, the verb \textit{hit} does not necessarily involve change, and classifies it as a surface contact predicate (see Example \ref{ex3}).

 \begin{exe}
 \ex \label{ex3}
     \begin{xlist}
         \ex[]{\label{ex3a}\textit{I hit the window with a hammer; nothing happened to the window.}   }
         \ex[*]{\label{ex3b}\textit{I broke the window with a hammer; nothing happened to the window.}}
     \end{xlist}
\end{exe}


The two classes demonstrate different syntactic behaviour, in particular with respect to derived adjectives -- passive construction alternation and the possessed body part as object or as a place alternation, which is valid only for the verb \textit{hit}. In summary, \citet[135--137]{Fillmore1970} outlines the properties of change-of-state predicates in contrast to surface contact predicates, or, more generally, other classes of predicates that although implying change (as a consequence) do not necessarily involve change as part of their lexical meaning.

\citet{Levin1993} offers a comprehensive classification of English verbs based on their syntactic properties and their participation in particular argument alternations. Levin continues the line of analysis regarding \textit{break} and \textit{hit} by considering four classes of verbs and offering a number of tests to identify pure change-of-state predicates:

\begin{enumerate}[label=(\alph*)]
\item Break Verbs (\textit{break, crack, rip, shatter, snap});
\item Cut Verbs (\textit{cut, hack, saw, scratch, slash});
\item Touch Verbs (\textit{pat, stroke, tickle, touch}); and
\item Hit Verbs (\textit{bash, hit, kick, pound, tap, whack}).
\end{enumerate}

One test offered by \citet{Levin1993} involves (zero-)derivative nouns, which in the case of \textit{break} and \textit{cut} refer to the result of the action, while the zero-derived nominals from \textit{hit} and \textit{touch} do not allow this interpretation and refer only to the action itself. Thus, \textit{break} and \textit{cut} are both considered as verbs of causing a change of state since the nominalisation shows a result (terminal state) included in its semantics. On the other hand, since \textit{hit} or \textit{touch} are not change-of-state verbs (although they might be associated with a change as a consequence), these verbs are not found in the causative/inchoative alternation.

\citet[16--20]{RappaportHovav2008} outlines the aspectually relevant properties encoded in the meanings of verbs and the grammatical reflexes of these properties. The author looks at the relationship between the aspectual property of telicity and the notion of scale which in the event structure can be interpreted as a measure, incremental theme, quantity criterion, etc. The conclusion drawn by \citet{RappaportHovav2008} is that verbs lexically encoding a scalar change cannot be classified  either as activities, accomplishments or achievements.

\citet[278--279]{RappaportHovavLevin2005} argue that argument expression is not determined purely by the aspectual properties of the sentence but is also motivated by certain lexical features of the verb. In particular, the authors analyse verbs of change to show that certain alternations can only be explained by the lexicalised property\footnote{\added{Here we assume the following definition of the term \textit{to lexicalise}: `to represent (a set of semantic features) by a lexical item'; in particular, a lexicalised property, as used by \citet[273]{RappaportHovavLevin2005}, is a semantic property that has become an integral part of the verb's lexical meaning as opposed to being expressed by morphological means or by the valency configuration of the verb. For example, the semantic property `temperature' as the property of change is lexicalised in verbs such as \textit{warm} or \textit{freeze}.}} of the verbs, namely change of state, and whether it can be incremental or instantaneous. 

The study of argument structure of verbs and the properties of their arguments suggests that argument structures might be derivable to a large extent from the meaning of words and by combining similar verbs into classes with similar syntactic behaviour (\cite[1--3]{Levin1993}, \cite[4--7]{Pinker1989}, \cite[11--12]{Chomsky1986}). It is therefore necessary to present the linguistically relevant aspects of word meaning and to formulate the principles for deriving syntactic properties from word meaning. To this end, we employ semantic frames that are assigned to synsets, allowing us to study both semantically related words (via WordNet relations) and their corresponding conceptual descriptions (from FrameNet frames and the relations between them).

\largerpage
With respect to the classification of verbs of change, \citet[66--78]{Dowty1979} considers several semantic classes and places them within the two aspectual classes of achievements and accomplishments.

\begin{description}[font=\normalfont]\sloppy
\item[\emph{Achievements} (may be coextensive with inchoative):] Change of physical state (for absolute states) -- intransitives (\textit{melt, freeze, die, be born, molt, ignite, explode, collapse}) or two-place phrasal (\textit{turn into N, turn to N, become Adj}); Change of physical state (degree state) -- intransitive (\textit{darken, warm, cool, sink, improve}) or phrasal (\textit{become} Adj); Aspectual complement verbs -- infinite complement (\textit{begin, start, cease}), gerundive complement (\textit{stop, resume, begin, start}) or event nominal as subject (\textit{end, stop, resume, start, begin}); Possessive (\textit{acquire, receive, get, lose)}; Change of state of consciousness (\textit{awaken, fall asleep}).

\item[\emph{Accomplishments}:] Locatives -- transitive verbs involving enclosure (\textit{hide, cover, box, uncover, create}), two-place phrasals derived from activity verbs with locative result state (\textit{drive, carry, push}), or two-place phrasals not derived from activity verbs (\textit{put, place, set NP into NP}), transitive with extent (\textit{carry, push, drive NP a mile}); Intransitives that are not locatives (\textit{shape up, grow up}); Transitive verbs of creation (accusativus effectivus) -- derived from activities (\textit{draw, knit, dig}) or not derived from activities (\textit{make, build, create, construct, erect}); Transitive verbs of destruction (\textit{destroy, obliterate, raze, melt, erase, eat}); Transitive change of state (\textit{kill, petrify, marry NP to NP, cook, paint}).
\end{description}

\citet{Levin1993} offers a more detailed classification of English verbs which lies at the base of the present study. The classification relies on the verbs' general semantics and the diatheses in which they participate. Within the class of so-called “pure” change-of-state verbs, Levin distinguishes between several semantic subclasses: verbs of change of physical integrity (break verbs, 45.1)\footnote{\added{Here and below, when referring to Levin's verb classes, when appropriate, the relevant number of the class will be presented as per the classification of \citet{Levin1993}.}};  verbs of change of shape  without breaking the physical integrity (bend verbs, 45.2); verbs for heat treatment (cooking verbs, 45.3); verbs of change on a scale (calibratable change of state, 45.6), e.g. \textit{increase, decrease}; verbs of entity-specific change of state (45.5), which combine predicates denoting a change characteristic of certain entities, e.g. \textit{bloom, rust, erode}; other change-of-state verbs (45.4), united by their similar syntactic behaviour, including conversion and verbs derived from adjectives or nouns, e.g. \textit{clean, narrow, blunt, soften, flatten, decentralise, caramelise}.

Some of the other categories, although implying a change, exhibit a more complex semantic structure, thus they require separate detailed analysis. Additional attention is needed to handle verbs of creation and transformation (26), destroy verbs (44), verbs of killing (42), verbs of ingesting (39) which involve an agent and a patient that undergoes a change of state -- comes to existence or ceases to exist.

In addition, within the verbs of change class we also include several more groups of verbs. Firstly, we take into account verbs denoting externally inflicted physical change (verbs of cutting, 21), change of (body) position (roll verbs, 51.3.1) and change of location (verbs of putting, 9; verbs of removing, 10; verbs of sending and carrying, 11), as well as verbs denoting change in the psychological or emotional state (amuse type psych-verbs, 31.1) and verbs of change of possession (give verbs, 13.1, contribute verbs, 13.2, verbs of obtaining, 13.5, verbs of exchange, 13.6) which are always transitive and require an agent. Secondly, we consider verbs that are inherently intransitive and do not allow a causative counterpart such as verbs of appearance, disappearance and occurrence (48).

In a refined and enriched form, with integrated additional classes, the classification of \citet{Levin1993} becomes the basis of the organisation of verbs in VerbNet \citep{Kipper-Schuler2005}, a lexical-semantic resource presenting a shallow hierarchy of verb classes in English, the semantic roles describing the argument structure of the predicates of each class, the selectional restrictions of the arguments, their syntactic realisation, the diatheses in which the verbs participate, etc.

\citet{Levin2010}, following research by \citet{KennedyLevin2008}, \citet{Krifka1998}, \citet{RappaportHovav2008}, \citet{Beavers2008}, proposed a unified approach to the determination of telicity by considering three verb classes: incremental theme verbs with respect to the extent of the argument (e.g., volume, area, height, etc.); true change-of-state verbs with an argument exhibiting a gradable property; and inherently directed motion verbs for which the path of motion is a measurable feature. Further, to integrate the classes together with respect to the treatment of telicity, \citet[1--2]{Levin2010} introduces the feature \fename{Scale} and considers it as an integral part of the verb conceptual structure and semantics. The three types of scales are then distinguished: (a) \emph{extent scales} characterising verbs with incremental theme; (b) \emph{property scales} most often found with change-of-state verbs; and (c) \emph{path\slash spacious scales} most often found with inherently directed motion verbs.

Extent scales are not lexicalised in verbs but concern the \fename{Theme} in their conceptual structure. Property scales can be fully lexicalised in verbs: it may be a binary scale (transitions from one state to another, e.g. \textit{break, crack, die}), but can also involve multiple stages or degrees (e.g. \textit{cool, age, deepen}); a change-of-state verb may specify all components of such a scale (e.g., \textit{redden}) or only some of them (e.g., \textit{darken}). Path scales denote change of place through motion and may be lexicalised in some motion verbs (e.g., \textit{ascend},  \textit{descend}).

\citet[2]{Levin2010} points out that bounded verbs of scalar change have inherent telic interpretation; non-scalar verbs (e.g., \textit{walk, run}, etc. -- verbs for habitual activities) can also achieve telic interpretation when used in particular context. An interesting observation is that scalar verbs rarely lexicalise a manner while non-scalar verbs do.


\citet{VanValin2005} also considers verbs denoting change of state to belong to the classes of achievements and accomplishments due to the fact that change implies an inherent terminal point, therefore it entails telicity, and only these two classes are telic as compared to states and activities which are atelic. \citet[33]{VanValin2005}, following previous research \citep{Vendler1967,Dowty1979}, states the four key semantic features which determine the scope of the classes: [$\pm$ static], [$\pm$ dynamic], [$\pm$ telic] and [$\pm$ punctual]. Punctuality distinguishes achievements (which are punctual) from accomplishments (which are non-punctual). The logical structure of the predicates denoting change \citep[42--49]{VanValin2005} are as follows (where BECOME and INGR are operators and the ingressive operator INGR denotes reaching the implied resulting state in the change-of-state verbs):

\begin{description}
\sloppy
\item [Achievement:] INGR \textbf{predicate$\prime$} (x) or (x, y), or INGR \textbf{do$\prime$} (x, [\textbf{predicate$\prime$} (x) or (x, y)]).

\item [Accomplishment:] BECOME \textbf{predicate$\prime$} (x) or (x, y), or BECOME \textbf{do$\prime$} (x, [\textbf{predicate$\prime$} (x) or (x, y)]).

\item [Active accomplishment:] \textbf{do$\prime$} (x, [\textbf{predicate$\prime$$_1$}  (x, (y))]) \& INGR \textbf{predicate$\prime$}$_2$  (z, x) or (y).

\item [Causative:] α CAUSE β, where α, β are logical structures of any type.
\end{description}

Active accomplishments are composed of an activity (\textbf{do’}) and reaching a terminal point in a result state (INGR operator). However, the activity verb in itself does not entail the terminal point in its lexical meaning; it occurs only in the more complex logical structure of the verb phrase, e.g. \textit{I ran to the park} (I ran and reached the park). In their logical structure, active accomplishments differ from accomplishments, which are represented as a process leading to a change of state (the operator BECOME is decomposed into PROC+INGR, \cite[44]{VanValin2005}), e.g. \textit{The snow melted}. Thus, active accomplishments differ from activities as they involve reaching a terminal point, and are closer to achievements rather than accomplishments in terms of the logical representation of the change (operator INGR rather than BECOME).


In summary, we adopt Vendler’s classification of activities, achievements, accomplishments and states with the relevant features and tests to distinguish between them. Verbs of change within this classification fall into the categories of achievements and accomplishments as they express a transition from one state to another, with an inherent end point.

The literature review shows that in the last decades much effort has been invested into the semantic classification of verbs. The overview of classifications points to the main features underlying the syntactic behaviour of verbs of change. In addition, the review leads to the conclusion that the syntactic realisation of the verb is not determined fully by its lexical meaning; however, there are clearly features which are derivable solely by its semantics (its meaning and its belonging to a particular semantic class).

The well-known classifications presented here are taken into account in order to offer a comprehensive description of the most frequent verb subclasses within the class of verbs of change as represented in Princeton WordNet. The classification of \citet{Levin1993} as one of the most extensive with respect to verbs of change is (partially) aligned with the classification proposed in the current study based on FrameNet frames (\sectref{sec:classification}).

The review also outlines the key lexical, semantic and grammatical features which are taken into account in the analysis: telicity, quantised change, causativity and inchoativity, frame elements and their configurations in frames evoked by the verbs.

The review of relevant studies on verbs of change primarily focused on English verbs, their properties and classification, helps to outline the scope of the study and to establish the background for the study of Bulgarian verbs by applying cross-linguistic analysis and transfer of information employing the cross-linguistic potential of WordNet and FrameNet.

\section{Resources}\label{ch3:sec:resources}
\largerpage
The description of verb semantics and the grouping of verbs into semantically and syntactically homogeneous classes takes different directions depending on the adopted theoretical approach and the degree of detail of the description in the different resources.
The focus is on the representation of verbs of change in the hierarchical structure of Princeton WordNet and the Bulgarian WordNet and the semantic frames that describe them, along with the semantic relations between the frames. The study of the systematic semantic relations in each of the resources, as well as the characteristics determining the relationship between their basic units (synsets and semantic frames) and the relations between them, supports the enrichment of synsets in WordNet with conceptual information. Moreover, this will support the study of the degree of semantic specialisation of verb meanings and the granularity of conceptual description (using frames).

The present study is based on information from two main semantic resources -- Princeton WordNet (WordNet) and the Bulgarian WordNet (BulNet), and FrameNet. Since the semantic features discussed are relevant at synset level and are largely applicable to both English and Bulgarian, we consider the two wordnets as one collective resource.

\subsection{Verbs of change in WordNet}\label{sec:wordnet}

Given the diversity in the semantics of change-of-state verbs, the most comprehensive information about the set of these verbs is provided in the lexical-semantic network WordNet \citep{Miller1995,Fellbaum1998}, which represents the lexicon in the form of a network of synonym sets (synsets) interconnected by semantic, lexical and other relations. One of the main relations building the structure of WordNet is hypernymy (and its opposite relation -- hyponymy), by which the vocabulary of a given semantic field is organised into a tree, the beginning (root) of which is the most general or abstract concept of the corresponding field.

WordNet, as well as its Bulgarian counterpart BulNet \citep{Koeva2006,koeva2021-wordnet}, is the main resource used in the study. The semantic description of verb predicates in WordNet also includes their classification into general semantic classes based on assigned semantic primitives \citep{MillerFellbaum2007}, e.g. verbs of motion, verbs of emotion, verbs of communication, verbs of change, etc. Verbs of change in WordNet are a large and diverse class. In general, they include:

\begin{enumerate}[label=(\alph*)]
\item Causative verbs of change covering the tree of synsets rooted at eng-30-00126264-v: \{\textit{change}:1; \textit{alter}:1; \textit{modify}:3\}, `cause to change; make different; cause a transformation' (regardless of the semantic primitive).
\item Non-causative verbs of change covering the tree of synsets rooted at eng-30-00109660-v: \{\textit{change}:2\}, `undergo a change; become different in essence; losing one's or its original nature’ (regardless of the semantic primitive).
\item Verbs lying outside of the two WordNet trees but assigned the semantic primitive verb.change, together with their hyponyms.
\end{enumerate}

Around half of the verbs falling under (a), (b) and (c) are labelled with the semantic primitive verb.change \citep{LesevaStoyanova2021change} while others have distinct semantics expressed by a different semantic primitive. More specific details on semantic classes of verbs in WordNet will be discussed in \sectref{sec:classification}.


\subsection{Frames related to change in FrameNet}

FrameNet \citep{Baker1998} is a system of semantic frames with their frame elements. Frames are schematic descriptions of the conceptual structure of situations through actors, circumstances, and other conceptual roles presented as frame elements. The frames are organised in a system by means of a number of relations -- hierarchical (\FrameRelation{Inheritance}, \FrameRelation{Use}, \FrameRelation{Subframe}, etc.) and other types of relations (for example, \FrameRelation{Causation}).


There have been efforts to construct a FrameNet-based resource for Bulgarian. \citet{Koeva2010-framenet} discusses the properties of the resource BulFrameNet -- a corpus-based lexicon that provides an exhaustive account of the semantic and syntactic combinatory properties of Bulgarian verbs. \citet{KoevaDoychev:2022} presents BulFrame -- a web-based system for the extensive description of verbs using semantic frames offering a unified theoretical model for the formal presentation of frames and frame elements. The framework expands on FrameNet semantic frames by defining the sets of nouns that can be combined with a given verb. This is achieved by an ontological representation of noun semantic classes.


WordNet and FrameNet are automatically mapped \citep{LesevaStoyanova2020}, with synsets in WordNet being assigned semantic frames from FrameNet. Over 5,000 of the frames assigned to synsets have been manually validated \citep{LesevaStoyanova2020}. Verb literals in a synset have a common meaning, which implies a common semantic frame. This makes assigned frames valid regardless of the language and allows the transfer of frame information from one language to another. \restructured{This is in line with previous observations that the argument structure depends to a large degree on the lexical meaning of words (\cite[1--3]{Levin1993}), which suggests that verbs with similar meaning would exhibit the same or similar valency patterns.}

Lexical units in FrameNet, in particular verbs, are grouped in semantic frames based on common semantics, formalised through a common set of participants and circumstances (frame elements) and the relations between them \citep{Fillmore1982,Fillmore1985,Fillmore2003,FillmoreBaker2009,Ruppenhofer2016} with valence patterns inductively derived from corpus evidence. \restructured{The underlying principle of organisation is semantic similarity in terms of the situation described by the verb, which may lead to grouping together verbs with different syntactic behaviour in terms of alternations as shown by \citet{BakerRuppenhofer2002} in their comparative analysis between Levin's classes and FrameNet frames.}


A number of parallels can be drawn between the classes proposed by \citet{Levin1993} and the frames in FrameNet. Thus, for example, verbs of change of physical integrity (45.1) largely correspond to the \framename{Cause\_to\_fragment} and \framename{Breaking\_\linebreak apart} frames, verbs of change of shape (45.2) correspond to the \framename{Reshaping} frame, cooking verbs (45.3) -- to the frames \framename{Apply\_heat} and \framename{Cooking\_creation}, verbs of calibratable change (45.6) – to the frame \framename{Cause\_change\_of\_position\_\linebreak on\_a\_scale} and \framename{Change\_position\_on\_a\_scale}, etc. Many of the verbs of entity-specific change (45.5) and the remaining verbs of change of state (45.4), which do not fall into the listed categories and which constitute the majority of verbs of change, correspond to a number of other frames. Thus, for example, many predicates formed from adjectives, depending on the changing feature which they lexicalise, correspond to specific frames: \textit{dry} (causative and inchoative) -- \framename{Cause\_to\_be\_dry} and \framename{Become\_dry}; \textit{cool, chill, warm, heat} (causative and inchoative)  -- \framename{Cause\_temperature\_change} and \framename{Change\_of\_temperature}; \textit{blacken, redden, darken} -- \framename{Cause\_change\_of\_colour} and \framename{Change\_of\_colour}, etc. Large groups of transitive verbs belong to frames that inherit the frame \framename{Transitive\_action}, i.e. represent more specific situations that have the same or a more specific set of frame elements.

Frames that cover specific verb subclasses within the class of verbs of change will be discussed in \sectref{sec:classification}, together with their frame elements and examples for their use in English and Bulgarian.

\subsection{Usage examples}

Usage examples in \sectref{sec:classification} illustrating the use of verbs of change and their behaviour are mostly drawn from the SemCor and BulSemCor -- both annotated with WordNet senses.

SemCor \citep{miller-etal-1993-semantic,miller-etal-1994-using,landes1998} is a manually annotated corpus developed by the Princeton WordNet team. Open-class single words and multiword expressions are assigned unique WordNet senses. The corpus contains a total of 226,040 sense annotations.

BulSemCor \citep{koeva-2006-bulsemcor,koeva-2011-bulsemcor,koeva2012bulsemcor} follows the methodology of SemCor and aims to ensure good coverage of varied general lexis. In addition to open-class words, in BulSemCor closed-class words (preposition, conjunctions, particles) are also annotated. The size of BulSemCor is close to 100,000 annotated units.

The use of SemCor and BulSemCor is motivated by the fact that verbs are annotated with word senses, so they are uniquely referred to a particular WordNet synset which allows the straightforward extraction of comparable examples in English and Bulgarian.

\section{Properties determining the syntactic realisation of verbs of change}\label{sec:semproperties}

\subsection{Telicity and quantised change}\label{sec:telicity}

\citet{RappaportHovav2008} discusses at length the relationship between the aspectual property of telicity and the notion of scale. Before that, \citet[88--90]{Dowty1979} talks about degree achievements and \citet{Hay1999} consider the classes of verbs of quantised and non-quantised change which exhibit different syntactic behaviour based on the scalar structure of the base adjective (representing the property of change). With verbs of quantised change the progressive does not entail the perfect (which means that performing the action does not entail the result), whereas with verbs of non-quantised change it does (e.g., \textit{The soldier is dying}  $\nRightarrow$ \textit{the soldier has died}; \textit{The child is growing} $\Rightarrow$ \textit{the child has grown}). The verbs of non-quantised change are telic, entailing a specific terminal state. In this class fall verbs like \textit{die}, \textit{be born}, \textit{fall}, \textit{arrive}, and many verbs of inherently directed motion \citep{Levin1993}.

For verbs of change of state, the entity associated with the change is considered as an incremental theme \added{(\cite{dowty1991thematic}, \cite{RappaportHovavLevin2005})}, which limits their argument projection options.
When the verb describes a specified degree of change on the scale, it is telic, and when it describes an unspecified degree of change, it is atelic. The objects of both traditional incremental theme verbs have a scale for measuring the event's progress and determine the telicity of its sentence. If the object is quantised, as is the change on the given scale, the sentence is telic. If the object is not quantised, the scale lacks a specified endpoint, the change on this scale is unspecified, and the sentence is atelic.
This comes to show that although argument expression is determined by the verb’s meaning, it is also influenced by specific aspectual properties at lexical or syntactic level.


Telic verbs are incompatible with some adverbials such as \textit{completely} and \textit{not completely}, given that they have a specific final state as one of their entailments (\textit{The soldier has died \#completely\slash \#but not completely}). \citet[129]{Hay1999} state that for verbs of non-quantised change telicity is implied when the adjectival base of a deadjectival verb (e.g., \textit{empty, straighten} etc.) includes a gradable property in its semantics. Such verbs are telic (e.g., \textit{He is emptying the bathtub} $\nRightarrow$ \textit{He has emptied the bathtub}) but are compatible with \textit{completely\slash not completely} (\textit{He has emptied the bathtub completely\slash but not completely}).

The telicity of a sentence with a change-of-state verb depends on the nature of the scalar property lexicalised in the verb and the scale along which this property changes. For example verbs entailing temperature change are associated with a temperature scale, and a sentence with such verb is telic if the change in temperature is specified. If it is not, it is atelic (\textit{The water warmed to room temperature} vs. \textit{The water warms slowly}).


Telicity can also arise based on a conventional (not lexicalised but assumed) property of the undergoer (usually the frame element \fename{Theme}). Verbs like \textit{read, write}, and verbs of consumption allow their direct object to be interpreted as an incremental theme. When the quantised object is expressed, they may appear either as telic or atelic (see Example \ref{ex4}). \added{The same can also be observed  with some verbs of change (Examples \ref{ex4e}, \ref{ex4f}).}

 \begin{exe}
 \ex \label{ex4}
     \begin{xlist}
         \ex\label{ex4a} \textit{He \textbf{READ} for hours.} (atelic)
         \ex\label{ex4b} \textit{He \textbf{READ} the book for hours.} (atelic)
         \ex\label{ex4c} \textit{He \textbf{READ} the book.} (telic)
	   \ex \label{ex4d}\textit{He \textbf{READ} the book in a day.} (telic)
    \added{
         \ex\label{ex4e} \textit{He \textbf{BUILT} the house for years.} (atelic)
         \ex\label{ex4f} \textit{He \textbf{BUILT} the house in five years.} (telic)
         }
     \end{xlist}
\end{exe}

The syntactic behaviour of verbs of change of location can also be analysed in terms of the incremental theme. Some verbs allow for different interpretations and take different incremental property\slash entity (e.g., locative alternation verbs) as their direct object (Example \ref{ex5}).

 \begin{exe}
 \ex \label{ex5}
     \begin{xlist}
         \ex \label{ex5a} \textit{I \textbf{SPRAYED} the whole can of paint onto the wall.}
         \ex \label{ex5b} \textit{I \textbf{SPRAYED} the whole wall with paint.}
     \end{xlist}
\end{exe}

In Bulgarian and other Slavic languages, the verb aspect is a lexical category and the perfective and imperfective verbs are considered different words with different lexical meanings derived as a result of word formation. The properties of verb aspect have been studied extensively \citep{andreychin1944,Ivanchev1971,Nitsolova2008,kutsarov2007,koeva2011,charalozova2021}.

Example \ref{ex6} shows uses of the verbs \textit{готвя} `cook' (imperf.) and \textit{сготвя} `cook' (perf.), the latter derived from the former using prefixation. Relevant to the study of the semantic properties of verbs of change is the fact that perfective verbs are always telic (Examples \ref{ex6c}, \ref{ex6d}) and the limitations in their interpretation stemming from that (e.g., Example \ref{ex6e}). Ambiguity at the lexical level between the telic and atelic interpretation of verbs can only occur with imperfective verbs (Example  \ref{ex6a}, \ref{ex6b} vs. \ref{ex6b1}).

 \begin{exe}
 \ex \label{ex6}
 \begin{xlist}
 \ex \label{ex6a}
\gll \textit{Той} \textit{\textbf{ГОТВИ}} \textit{часове} \textit{наред}.\\
He cooks hours {in sequence}.  \\
\glt `He cooks for hours on end.' (\textit{готвя} `cook', imperf.; atelic) 
 \ex \label{ex6b}
\gll \textit{Той} \textit{\textbf{ГОТВИ}} \textit{ястието} \textit{часове} \textit{наред}.\\
He cooks dish-DEF hours {in sequence}.  \\
\glt `He cooks the dish for hours.' (\textit{готвя} `cook', imperf.; atelic)\footnote{Cooking a dish for hours (for a long time, etc.) does not entail reaching the terminal point `The dish is cooked'; thus the interpretation is atelic.}
 \ex \label{ex6b1}
\gll \textit{Той} \textit{\textbf{ГОТВИ}} \textit{това} \textit{ястие} \textit{за} \textit{един} \textit{час}.\\
He cooks this dish for one hour.  \\
\glt `It takes him one hour to cook this dish.' (\textit{готвя} `cook', imperf.; telic)\footnote{Here, the terminal point `The dish is cooked' is reached in some fixed time (one hour); thus the interpretation is telic.}
  \ex \label{ex6c}
\gll \textit{Той} \textit{\textbf{СГОТВИ}} \textit{ястието}.\\
He cooked dish-DEF.\\
\glt `He cooked the dish.' (\textit{сготвя} `cook a complete dish', perf.; always telic)
  \ex \label{ex6d}
\gll \textit{Той} \textit{\textbf{СГОТВИ}} \textit{ястието} \textit{за} \textit{един} \textit{час}.\\
He cooked dish-DEF for one hour.\\
\glt `He cooked the dish in an hour.' (\textit{сготвя} `cook a complete dish', perf.; always telic)
  \ex \label{ex6e}
\gll *\textit{Той} \textit{\textbf{СГОТВИ}} \textit{ястието} \textit{часове} \textit{наред}.\\
\ He cooked dish-DEF hours {in sequence}.\\
\glt *`He (completely) cooked the dish for hours.' (\textit{сготвя} `cook a complete dish', perf.; always telic)
\end{xlist}
\end{exe}



\subsection{Inchoative and causative verbs of change}\label{sec:causative}


An inchoative and causative verb pair is a pair of verbs that express the same change of state and differ only in that the situation described by the causative verb involves an agent or a cause participant who evokes the change, whereas the inchoative verb describes the situation as an internalised change of the entity (Example \ref{ex7}).\footnote{Here, we assume two distinct verbs -- the inchoative verb \text{change} `undergo a change; become different in essence; losing one's or its original nature' (WordNet Synset ID: eng-30-00109660-v) and the causative verb \text{change} `cause to change; make different; cause a transformation' (WordNet Synset ID: eng-30-00126264-v).}

 \begin{exe}
 \ex \label{ex7}
     \begin{xlist}
         \ex \label{ex7a} (inchoative) \textit{The stick broke.}
         \ex \label{ex7b} (causative) \textit{The girl broke the stick.}
     \end{xlist}
\end{exe}


\citet[45]{VanValin2005} presents the following logical structure of the causative verb: `α CAUSE β, where α, β are logical structures of any type'. In particular, verbs of change are marked by the operators INGR -- ingressive operator denoting the transition from an initial to a new state, and BECOME -- operator which shows a process of transition into a new state, both pointing to a resulting terminal state in the verb’s semantics (Example \ref{ex12}).

\begin{exe}
\ex \label{ex12} Inchoatives
\ea \textit{The window shattered.} \\ 
     INGR \textbf{shattered$\prime$} (window)\\
\ex \textit{The balloon popped.}\\
    INGR \textbf{popped$\prime$} (balloon)\\
\ex \textit{The snow melted.}\\ 
    BECOME \textbf{melted$\prime$} (snow)\\
\ex \textit{Mary learned French.}\\ 
    BECOME \textbf{know$\prime$} (Mary, French)\\
\z
\ex \label{ex12a} Causatives
\ea 
\textit{The dog scared the boy.}\\
{[}\textbf{do$\prime$} (dog, ∅){]} CAUSE [INGR \textbf{feel$\prime$} (boy, [\textbf{afraid$\prime$}])]\footnote{The original logical structure in \citet{VanValin2005} is  [\textbf{do$^\prime$} (dog, ∅)] CAUSE [\textbf{feel$^\prime$} (boy, [\textbf{afraid$^\prime$}])] and the operator INGR here is added for conformity to denote reaching the final state.}
\ex \textit{Max melted the ice.}\\
{[}\textbf{do$\prime$} (Max, ∅){]} CAUSE [BECOME \textbf{melted$\prime$} (ice)]
\ex \textit{The cat popped the balloon.}\\
{[}\textbf{do$\prime$} (cat, ∅){]} CAUSE [INGR \textbf{popped$\prime$} (balloon)]
\z
\end{exe}


In his study on deep lexical semantics linking lexical meaning to underlying abstract core theories via lexical decompositions, \citet{Hobbs2008,Hobbs2014}, following on the work of \citet{Fillmore1988}, considers causal complex (as opposed to cause only) as a composite structure which includes all the states and events that have to happen or hold in order for the effect to occur. In view of this, \citet[190]{Hobbs2008} considers FrameNet to reflect these principles, describing frames as “axiomatic characterizations of abstract situations”, and presents causation in terms of a complex structure involving multiple participants and conditions \citep[184--188]{Hobbs2008}. This approach allows for more thorough analysis of causativity as well as the relation between the properties of the situations described by the inchoative and the causative verbs in the pair.

\added{
\citet[90--98]{LevinRappaportHovav1995} argue that with respect to causativity, the verb's behaviour is determined by its lexical meaning and depends on whether the verb lexicalises an internally caused event or an externally caused event. An externally caused event implies an external cause with immediate control over the event, such as with verbs of change of state, which are usually transitive but also allow intransitive use where the cause is not explicit (e.g., \textit{He opened the door -- The door opened}). In contrast, an internally caused event occurs due to some inherent properties of the entity participating in the situation, without an external cause, and thus these verbs are typically intransitive and do not enter the causativity alternation.
}


\restructured{For English verbs causativity related to verbs of change is analysed with a view to the causative/inchoative alternation. \citet[10, 30]{Levin1993} argues that this alternation is  sensitive and only applies to pure change-of-state verbs, roughly covering verbs of change of state and change of position/location. Verbs that imply a change of state only indirectly in their meaning, for example as a consequence of the situation described by the verb, are not found in the causative/inchoative alternation. These include some verbs that are only used transitively (in a causative meaning), such as verbs of change of possession (\textit{loan, rent, lend, refund, donate, transfer}), verbs of contact by impact (\textit{hit, bang, beat, kick}), verbs of cutting (\textit{cut, chop, slice, shed}), etc., as well as some verbs that are only used intransitively (in an inchoative meaning), such as verbs of appearance, disappearance and occurrence (\textit{appear, disappear, occur, arise, emerge, erupt, happen, expire}) and verbs of entity-specific change of state (\textit{blister, bloom, blossom, decay, deteriorate, erode, ferment, flower, germinate, rust, sprout, swell}).} Psych-verbs (verbs of psychological state change) rarely participate in the causative/inchoative alternation in English (\textit{amuse, madden, puzzle, sadden, sicken, worry}), but this is a typical alternation in other languages such as French, Italian, and Russian \citep[30]{Levin1993} as well as Bulgarian (see Example \ref{ex8}).

 \begin{exe}
 \ex \label{ex8}
     \begin{xlist}
         \ex \label{ex8a}(inchoative) \\
         \gll \textit{Детето} \textit{\textbf{СЕ ЗАРАДВА}} (\textit{на} \textit{подаръка}). \\
		Kid-DEF {became happy} (at gift-DEF). \\
		\glt `The kid became happy (because of the gift).'
\newpage
         \ex \label{ex8b}(causative) \\
         \gll \textit{Майката} \textit{\textbf{ЗАРАДВА}} \textit{детето} (\textit{с} \textit{подарък}). \\
		Mother-DEF {made happy} kid-DEF (with gift). \\
		\glt `The mum made the kid happy (with a gift).'
         \ex \label{ex8c}(causative) \\
         \gll \textit{Подаръкът} \textit{\textbf{ЗАРАДВА}} \textit{детето}. \\
		Gift-DEF {made happy} kid. \\
		\glt `The gift made the kid happy.'
     \end{xlist}
\end{exe}


As stated by \citet[352--353]{Melcuk1967} and \citet[89]{Haspelmath1993}, causative verbs, on purely semantic grounds, are considered to be derived from inchoative verbs (also seen in the logical structures in Examples \ref{ex12} and \ref{ex12a} above). Semantically, \textit{A melts} is a simpler structure than its causative \textit{B causes A to melt}. However, it has been pointed out that in Russian (and the same is valid for Bulgarian), considering derivation in terms of form, there are predominantly examples of the reverse derivation where the inchoative is formally derived from the causative (\cite[89]{Haspelmath1993}, and Example \ref{ex9}). \citet[103--106]{Haspelmath1993} argues that the likelihood of spontaneous vs. caused events is the main factor determining the direction of derivation in inchoative/causative verb pairs. \restructured{While the direction of formal derivation and the investigation of the meaning formation is beyond the scope of the current analysis, we discuss the causative/inchoative pairs of verbs and the correspondence of their meanings in view of the semantic and derivational relation they exhibit. In some cases, with a view to Bulgarian, the morphological means to derive counterparts in the causative/inchoative pair are also relevant in the analysis of the lexical meaning of verbs of change.}

 \begin{exe}
 \ex \label{ex9}
     \begin{xlist}
         \ex \label{ex9a} (causative) RU: \textit{расплавить} /rasplavit'/ `melt (tr.)' \\
		BG: \textit{топя} /topya/ `melt (tr.)'
         \ex \label{ex9b} (inchoative) RU: \textit{расплавиться} /rasplavit'sja/ `melt (intr.)' \\
		BG: \textit{топя се} /topya se/ `melt (intr.)'
     \end{xlist}
\end{exe}

The inchoative member of an inchoative/causative verb pair is semantically similar to the passive of the causative but the difference is in that there is neither an explicit or implicit agent or cause as the situation is described with respect to the internalised change of the entity.

Some causative verbs exhibit agent-oriented meaning components and in that case they don’t have an inchoative counterpart since the inchoative member implies the absence of an agent. Such component might be the instrument or means of performing an action -- e.g., the verb \textit{cut} implies a sharp instrument even if not explicitly stated in the sentence, so it does not enter the causative/inchoative alternation, as opposed to the verb \textit{tear} (Example \ref{ex10}).

 \begin{exe}
 \ex \label{ex10}
     \begin{xlist}
         \ex[]{\label{ex10a}\textit{I cut the sheet (with scissors).}}
         \ex[*]{\label{ex10b}\textit{The sheet cuts.}}
         \ex[]{\label{ex10c}\textit{I tore my dress (on the fence).}}
         \ex[]{\label{ex10d}\textit{My dress tore (on the fence).}}
     \end{xlist}
\end{exe}

 Similar verbs that imply agent-oriented components are verbs like \textit{wash, execute, tie}, in contrast to verbs such as \textit{clean, kill} (\textit{kill/die} counterparts), \textit{untie}. In Bulgarian such verbs although resembling the form of verbs with the reflexive particle \textit{se} in Examples \ref{ex8} and \ref{ex9}, when they are perfective or derived from perfective, have only a passive interpretation (Example  \ref{ex11c}, \ref{ex11d}). In the case when the verbs are imperfective, alongside the passive interpretation, they can also have a stative interpretation (Example \ref{ex11b} -- describing the property of the verb to be cut).

 \begin{exe}
 \ex \label{ex11}
     \begin{xlist}
         \ex \label{ex11a}
         \gll \textit{\textbf{НАРЯЗВАМ}} \textit{листа} \textit{на} \textit{ленти} (\textit{с} \textit{ножица}). \\
Cut.1sg sheet-DEF into strips (with scissors). \\
\glt `I cut the sheet into strips (with scissors).'
         \ex \label{ex11b}
         \gll \textit{Листът} \textit{\textbf{СЕ РЕЖЕ}} (\textit{лесно}). \\
Sheet-DEF {cuts} (easily). \\
\glt `The sheet cuts (easily).' (stative interpretation of the imprerf. verb \textit{режа})
         \ex \label{ex11c}
         \gll ?\textit{Листът} \textit{\textbf{СЕ НАРЯЗВА}} (\textit{лесно}). \\
?Sheet-DEF {cuts} (easily). \\
\glt ? `The sheet cuts (easily).' (stative interpretation of the imprerf. verb \textit{нарязвам} which is derived from the perf. \textit{нарежа} is blocked; the construction is perceived as passive)
         \ex \label{ex11d}
         \gll \textit{Листът} \textit{\textbf{СЕ НАРЯЗВА}} \textit{на} \textit{ленти} (\textit{с} \textit{ножица}). \\
Sheet-DEF {is cut} into strips (with scissors). \\
\glt `The sheet is cut into strips (with scissors).' (passive interpretation)
     \end{xlist}
\end{exe}

\added{The observations on morphological vs. lexical ambiguity are aimed at demonstrating the scope of the complex nature of the class of verbs of change and the influence of their semantic and morphosyntactic properties in determining their behaviour and interpretation in context. In particular, we would like to distinguish cases of passive with reflexive article \textit{se} from cases of the use of independent inchoative lexical units. Moreover, further research is needed into the derivational means in Bulgarian involved in filling the gaps in possible causative/inchoative pairs within the class of verbs of change. This line of analysis would be beneficial for enriching WordNet and FrameNet both in terms of lexical coverage and in terms of relational structure.}


\subsection{\restructured{Structure of the situation described by verbs of change}} \label{sec:roles}

\restructured{The analysis of the situations described by the verbs of change adopted here relies on Frame Semantics \citep{Fillmore1982,Fillmore1985,FillmoreBaker2009}. With respect to causativity and the analysis of the situation, to some extend the generalised semantic roles of Role and Reference Grammar \citep{VanValin1993,VanValinLaPolla1997}, and their hierarchical organisation, are also relevant. However, here we focus on the frames evoked by verbs of change and consider the more detailed, fine-grained and verb class specific participants in the situation, as represented by the frame elements within each frame. The most characteristic core frame elements with regards to the verbs of change and their description will be discussed. Moreover, the syntactic realisation of the frame elements and the valence configurations they enter is also considered since it is relevant for the classification of the verbs of change and their use in the inchoative and the causative scenario.}

Pure change-of-state verbs such as \textit{break, tear, bend}, in both their transitive and intransitive uses, express a change of state (plus a notion of cause when transitive). \restructured{Very often, in its inchoative variant, a verb of change of state has one core frame element, denoting the entity undergoing the change of state, which is realised as the external NP.} As already discussed above, the verb includes in its semantics the notion of the property that changes. This property then serves as a restriction on the type of the entity that can take this position (Example \ref{ex13}).

\begin{exe}
 \ex \label{ex13}
       \begin{xlist}
         \ex Physical entity (flexible). \\
         \textit{The stick bends, The girl bends, *The parliament bends.}
         \ex  Physical entity (body, substance). \\
         \textit{The water froze, The ground froze, *The religion froze.}
\ex Physical entity (growing). \\
\textit{The child grows, The city grows, *The rocks grow.}
\ex Physical entity (fluids).\\
\textit{The water drained, The blood drained, *The rocks drained.}
\ex Sentient entity. \\
\gll \textit{Момичето} \textit{се развесели.} \\
Girl-DEF {became happy}.\\
\glt `The girl became happy.'\\[0.2em]
\gll \textit{Селото} \textit{се развесели.} \\
Village-DEF {became happy}.\\
\glt `The village became happy.'\\[0.2em]
\gll *\textit{Дървото} \textit{се развесели.} \\
Tree-DEF {became happy}.\\
\glt *`The tree became happy.'
     \end{xlist}
\end{exe}

The causative variant of the verb requires one more participant, namely the \fename{Agent}.  In general, the \fename{Agent} can be expressed by a sentient entity or by a non-sentient \fename{Cause} that evokes the change; \added{they appear as frame elements in the respective causative semantic frame in FrameNet and are usually realised as the external NP in the sentence} (Example \ref{ex141}). The meaning of some verbs inherently involves agent-specific components, for example an \fename{Instrument} or \fename{Means}. As already mentioned above, such verbs require the existence of an \fename{Agent} that uses this \fename{Instrument}\slash \fename{Means} to bring about a change of state in the \fename{Patient} (hence such verbs do not have an inchoative counterpart). In the causative realisation, either the \fename{Agent}, the \fename{Cause} or the \fename{Instrument}\slash \fename{Means} can take the position of the subject (Example \ref{ex14}).

\added{
\begin{exe}
 \ex \label{ex141} A pair of corresponding inchoative and causative frames (\framename{Breaking\_apart} and \framename{Cause\_to\_fragment}).\newline
Frame: \framename{Breaking\_apart} \newline
{\begin{tabular}{llll}
\hlblue{Whole} &  
\hlcyan{Pieces} &&
\end{tabular}
} \\[0.1cm]
{[}\textit{The river}]$_{\feinsub{Who}}$ \textit{\textbf{SPLITS}} [\textit{into two streams}]$_{\feinsub{Pie}}$.\\[0.2cm]
Frame: \framename{Cause\_to\_fragment} \newline
{\begin{tabular}{llll}
\hlblue{Whole\_patient} &
\hlcyan{Pieces} & \hlred{Agent} & \hlteal{Cause}
\end{tabular}
} \\[0.1cm]
{[}\textit{She}]$_{\feinsub{Age}}$ \textit{\textbf{SMASHED}} [\textit{the plate}]$_{\feinsub{WhoPat}}$ [\textit{into little pieces}]$_{\feinsub{Pie}}$. 
\end{exe}
}
 
\begin{exe}
 \ex \label{ex14}
       \begin{xlist}
         \ex \label{ex14a}{[}\textit{The man}{]}$_{\feinsub{Age}}$ \textit{\textbf{CUT} the bread with a knife.}
         \ex\label{ex14b} {[}\textit{The knife}{]}$_{\feinsub{Ins}}$ \textit{\textbf{CUT} the bread.}
\ex \label{ex14c}With this interview {[}\textit{the president}{]}$_{\feinsub{Age}}$ \textit{\textbf{ANNOUNCED} his resignation.}
\ex\label{ex14d}{[}\textit{The TV channel}{]}$_{\feinsub{Medium}}$ \textit{\textbf{ANNOUNCED} the president's resignation.}
\ex\label{ex14e}{[}\textit{The wind}{]}$_{\feinsub{Cse}}$ \textit{\textbf{BROKE} the branch.}
\ex\label{ex14f}{[}\textit{I}{]}$_{\feinsub{Age}}$ \textit{\textbf{DISTRACTED} him with my questions.}
\ex\label{ex14g}{[}\textit{My questions}{]}$_{\feinsub{Mns}}$ \textit{\textbf{DISTRACTED} him.}
     \end{xlist}
\end{exe}

The changing quantised property can also be concretised in terms of its degree on the scale, as seen in \sectref{sec:telicity} (Example \ref{ex131}). It is interesting to analyse the adverbials expressing the position of the value on the scale up to the terminal point (reaching a terminal value or state).

\begin{exe}
 \ex \label{ex131}
       \begin{xlist}
         \ex\label{ex131a}
         \gll \textit{\textbf{ИЗПРАЗНИХ}} \textit{ваната} \textit{наполовина.}\\
         Emptied.1sg bath-DEF halfway.\\
         \glt `I emptied the bathtub halfway.'
         \ex \label{ex131b}
       \gll   \textit{\textbf{ЗАТОПЛИХ}} \textit{водата} \textit{до} \textit{40 градуса.}\\
         Warmed.1sg water-DEF {to} {40 degrees}.\\
       \glt `I warmed the water up to 40 degrees.'
\ex \label{ex131c}
\gll \textit{Децата} \textit{\textbf{СТИГНАХА}} \textit{почти} \textit{до} \textit{парка.}\\
Kids-DEF reached almost to park-DEF.\\
\glt `The kids went almost as far as the park.'
\ex \label{ex131d}
\gll\textit{Той} \textit{\textbf{РАЗСМЯ}} \textit{публиката} \textit{до} \textit{сълзи}. \\
He {made laugh} audience-DEF until tears.\\
\glt `He made the audience laugh to tears.'
     \end{xlist}
\end{exe}


Other frame elements that appear in the frames evoked by verbs of change include:
\begin{itemize}
\item \fename{Goal}: realised in the logical configuration INGR/BECOME be-at/in/on (x, y) where y is the \fename{Goal}  (Example \ref{ex15a}).
\item \fename{Source}: appearing in a possible configuration INGR/BECOME NOT have (x, y) or TERMINATE be-at/in/on (x, y$_0$) \& INGR/BECOME be-at/in/on (x, y)  (Example \ref{ex15b}).
\item \fename{Recipient}: may be defined as the possessor argument in a configuration such as INGR/BECOME have (x,y) where the \fename{Recipient} is y  (Example \ref{ex15c}).
\end{itemize}

\begin{exe}
           \ex  \label{ex15a} \textit{The river level \textbf{RISES}} [\textit{up to 10 m}]$_{\feinsub{Goal}}$ \textit{in the spring}. \\
           \textit{We \textbf{LIFT} the load} [\textit{to the 10th floor}]$_{\feinsub{Goal}}$.
         \ex  \label{ex15b}  \textit{The tea \textbf{SPILLED}} [\textit{from the cup}]$_{\feinsub{Src}}$.\\
         \textit{Sam \textbf{POURED} hot water} [\textit{from the teapot}]$_{\feinsub{Src}}$.
\ex  \label{ex15c}  \textit{She \textbf{LENT} her bicycle} [\textit{to Sam}]$_{\feinsub{Rec}}$. \\
\textit{We \textbf{CONTRIBUTED} our paycheck} [\textit{to the foundation}]$_{\feinsub{Rec}}$.
\end{exe}

\restructured{
By assigning frames to the synsets in WordNet, we aim at defining semantic classes of verbs based on similar lexical semantics, but more importantly, that evoke the same or similar (related) frames which exhibit similar configurations of frame elements. Further, annotated corpus examples as presented in \sectref{sec:classification} provide material for the comparative analysis of verbs within the same frame that exhibit similar or different valence patterns and syntactic realisation of their frame elements. Moreover, these analyses can be extended to a cross-language level in an attempt to analyse the semantic and syntactic properties determining the verb class behaviour for various languages.
}

The set of frame elements characterising the evoked frames also participate in a (shallow) hierarchical structure determined by the inheritance relations between the frames. For example, the frame element \fename{Patient} in the frame \framename{Transitive\_action} corresponds to the frame element \fename{Dryee} in the more specific frame \framename{Cause\_to\_become\_dry} (inheriting from \framename{Transitive\_action}). The frame element \fename{Dryee} is a concretisation of \fename{Patient} -- a surface or an entire entity which is able to retain water inside and/or out and consecutively, to become dry.

The inheritance and correspondence between the more general and the more specific frame elements is demonstrated in \sectref{sec:classification}.

\section{Towards a classification of verbs of change with respect to their semantic properties and conceptual structure} \label{sec:classification}

The number and type of frame elements and their corresponding semantic restrictions are determined to a large degree by the lexical meaning of the verb, so it is theoretically founded to consider groups of semantically related verbs (verbs with a common hypernym) and to base the semantic classification of verbs of change on WordNet.


The class of verbs of change includes groups of predicates such as: (i) change in the degree of an inherent quality or property along a scale (\sectref{sec:physicalchangescale}); (ii) change in the integrity of an object (\sectref{sec:breaking}); (iii) change-of-state verbs involving the (momentous) transition into a new state, e.g. a change in the mode or form of existence (\sectref{sec:transition}); (iv) change involving creation or transformation (\sectref{sec:creation}); (v) changes in the conditions of an entity as a result of an outside influence (such as various manners of treating, adjusting, etc.), including those involving the movement of something (putting, removing, etc.). This last group falls outside the scope of the present study.

Each subclass is characterised by a specific property of change which is either scalable (the property can be interpreted as a measure, incremental theme, quantity criterion, etc.) or momentous (the property describes transitioning into a new state). The property is lexicalised; in most cases the final (resulting) value is also lexicalised (e.g., \textit{freeze, dry, vanish}) and thus has no syntactic realisation in the sentence. 

Verbs with a more general meaning can realise specific frame elements denoting the change (frame elements that are not incorporated in the verb's meaning), e.g. \fename{Initial\_category}\slash 
\fename{Final\_category} in the frame \framename{Cause\_change}, \fename{Prior\_\linebreak state}\slash \fename{Post\_state} in the frame \framename{Progression}, etc.

The two hypernym trees in WordNet containing the largest number of verbs of change are eng-30-00126264-v: \{\textit{change}:2\} and eng-30-00109660-v: \{\textit{change}:1\}, with causative and non-causative meanings, respectively. These two trees are relatively symmetrical, as many of the hyponyms in one tree have counterparts in the other, established by the relation `causes'.

There are also some causative/inchoative synset pairs for which no `causes’ relation has been defined. Moreover, there are also cases where the causative and inchoative meaning of a verb is encoded in the same synset (e.g., eng-30-00280532-v: \{\textit{blacken}; \textit{melanise}; \textit{nigrify}; \textit{black}\} `make or become black'). These are cases that ideally will be treated separately as they should be assigned different frames.

Correspondence in the organisation of the two WordNet trees is also reflected in the frames assigned to the synsets in them. Two interframe relations denoting causation and inchoativity are encoded in FrameNet: \textbf{Causative of} and \textbf{Inchoative of}. However, these relations are not consistently reflected in FrameNet either. Additional work was done to correlate the pairs of synsets and the pairs of frames via the relation of causation based on similarities in definitions, matching of semantic classes, and correspondence between the frame elements \citep{LesevaStoyanova2021change}.


Among the hyponyms within the two trees of causative and inchoative verbs of change there is a range of semantic classes, the most frequent of which are: verb.contact (verbs for physical contact), verb.possession (verbs of possession), verb.body (verbs of bodily processes, dressing, etc.), verb.social (verbs representing social relations and interaction), verb.motion (verbs of motion), etc. They show a narrowing and specialisation of meaning and can serve as a basis for the formation of semantic subclasses within the class of verbs of change. Most of these classes fall into the description of the Aktionsart classes pertaining to the verbs of change and their logical structure \citep[42--49]{VanValin2005} and can also be aligned to Levin's classes \citep[240--247]{Levin1993}.


The beginning of the structure of frames in FrameNet, describing a significant part of the causative verbs of change, is the abstract frame \framename{Transitive\_action}, which refers to situations in which an `\fename{Agent} or \fename{Causer} affects a \fename{Patient}'. The frames that inherit it represent instantiations of it, the most general of which is the frame \framename{Cause\_change}, which is assigned to the root of the causative tree, the synonym set eng-30-00126264-v: \{\textit{change}:2\}. It is also assigned to other verbs whose semantics is adequately described by the frame or no other, more specific frame has been found when applying the automatic frame assignment procedure. Its direct inchoative counterpart is the frame \framename{Undergo\_change} assigned to the synset eng-30-00109660-v: \{\textit{change}:1\}. In the course of the manual verification, where possible, new, more specific frames describing the semantics of a group of synsets have been assigned and some new frames have also been defined to cover larger sets of synsets \citep{StoyanovaLeseva2019,LesevaStoyanova2021change}.

This section presents the linking of: (1) the general semantic classes of verbs of change; (2) their corresponding (sub)classes in Levin's classification \citep{Levin1993} which is more often not a direct match but a many-to-many mapping; (3) a shallow hierarchy of FrameNet frames evoked by verbs of these semantic classes with their corresponding frame elements (only some characteristic frames are discussed); (4) inchoative -- causative correspondences between frames when such correspondence exists; (5) a typical WordNet synset that often presents a (sub)tree of synsets (its hyponyms) evoking the given frame. Further, the examples demonstrate comparable realisations of some typical valence patterns in Bulgarian and English.

However, there have been studies pointing out the limitations of FrameNet especially with respect to the comprehensive coverage of verb semantics and its granularity. \citet{Rosca2013} discusses entity-specific change-of-state verbs (45.5 in Levin's classification): (i) verbs which describe an increase in size (e.g. \textit{bloom, blossom, flower, germinate, sprout, swell}); (ii) verbs that describe a negative, destructive change affecting the integrity of an entity (e.g. \textit{burn, corrode, decay, deteriorate, erode, rust, rot}); (iii) the verb \textit{ferment} not related to a measurable increase\slash decrease in the values characterising the \fename{Theme}. These verbs do not take part in a causative -- inchoative opposition. \removed{The weaknesses of the FrameNet representation and its representativeness are partially overcome by a set of strategies undertaken to improve the data both in WordNet and FrameNet (see \sectref{sec:discussion}).}

Some large semantic classes, although considered in general to belong to the class of verbs representing change, are not covered here as they require a separate detailed analysis. Such is the case of the semantic class of motion verbs in WordNet which comprises a large group of verbs referring to change in spatial location of an object to a \fename{Goal} and/or from a given \fename{Source}. Further, outside of the analysis fall verbs denoting physical contact between two objects which results in a change of their properties. These verbs are traditionally classified as Verbs of Putting and Verbs of Removing \citep[111--131]{Levin1993}, to which the frames \framename{Placing} and \framename{Removing} are assigned. Closely related to them are also the frames \framename{Filling} and \framename{Emptying}, where the action is directed to filling the location with the objects or emptying it of them, respectively. Another class that lies beyond the scope of the present study are verbs of emotional and psychological change and verbs of change of possession.

The theoretical analysis of the verb subclasses within the class of verbs of change is supplemented with observations on 30 FrameNet frames related to verbs of change which are assigned to over 100 WordNet synsets. In addition, the presentation of verb classes and the discussion on the syntactic realisations of verbs of change and their frame elements 
rely on 3,482 sentences with a total of 9,048 annotations in English extracted from FrameNet examples and SemCor, and 415 sentences with 866 annotated frame element realisations in Bulgarian extracted from BulSemCor and manually annotated.

To facilitate visualisation of the corresponding elements of the frame description, in the tables below which present the frames covering the subclasses of verbs of change, the corresponding frame elements are presented in the same colour. The sign “>” in front of a frame name shows that it inherits from the frame directly above it in the table.

\subsection{Verbs of change in the physical parameters of the object}\label{sec:physicalchangescale}

A group of verbs lexicalise change in a physical or some other property of the \fename{Theme} which is a physical object or abstract entity: density or consistency, physical phase; temperature; volume, size or extent; colour; change in appearance or characteristics, etc. (\tabref{table1a}).


\begin{table}
\small
\begin{tabularx}{\textwidth}{QQ}
\lsptoprule
Inchoative & Causative \\\midrule
\rowcolor{lightgray}\multicolumn{2}{l}{Frames denoting change of the physical parameters of the object}\\  
\rowcolor{lightgray}\multicolumn{2}{l}{Bend Verbs (45.2), some of Verbs of Entity-Specific Change of State (45.5)} \\ \midrule
\framename{Go\_into\_shape} \newline
{\begin{tabular}{ll}
\hlblue{Theme} &
\end{tabular}
} \newline\newline eng-30-00374135-v: \{freeze:6\}, \newline `change to ice'
& \framename{Reshaping} \newline
{\begin{tabular}{ll}
\hlblue{Patient} & \hlred{Deformer} \\
\hlcyan{Configuration} & \hlteal{Cause}
\end{tabular}
} \newline eng-30-00142191-v: \{shape:2; form:2\}, `give shape or form to'\\ \midrule
\framename{Change\_posture}\newline
{\begin{tabular}{ll}
\hlblue{Protagonist} &
\end{tabular}
} \newline eng-30-01983771-v: \{change posture:1\}, `undergo a change in bodily posture'&\\ \midrule
\framename{Change\_of\_phase} \newline
{\begin{tabular}{ll}
\hlblue{Patient} &
\end{tabular}
}\newline  eng-30-00374135-v: \{freeze:6\}, `change to ice'
& \framename{Cause\_change\_of\_phase}\newline
{\begin{tabular}{lll}
\hlblue{Patient} & \hlred{Agent}  & \hlteal{Cause}
\end{tabular}
} \newline eng-30-00375865-v: \{freeze:7\}, `cause to freeze'
\\ \midrule
%Change\_of\_consistency\newline
%{\begin{tabular}{ll}
%\hlblue{Patient} &
%\end{tabular}
%} \newline
%&Cause\_change\_of\_consistency\newline
%{\begin{tabular}{ll}
%\hlblue{Patient} & \hlred{Agent} \\
% & \hlteal{Cause}
%\end{tabular}
%} \newline\\ \midrule
\framename{Expansion}\newline
{\begin{tabular}{ll}
\hlblue{Item} &
\end{tabular}
} \newline  eng-30-00157844-v: \{enlarge:2\}, `become larger or bigger'
&\framename{Cause\_expansion}\newline
{\begin{tabular}{lll}
\hlblue{Item} & \hlred{Agent}  & \hlteal{Cause}
\end{tabular}
} \newline  eng-30-00154778-v: \{enlarge:1\}, `make larger' \\\midrule
%Change\_of\_appearance\newline
%{\begin{tabular}{ll}
%\hlblue{Patient} &   \\
%\hllime{Body\_part} &
%\end{tabular}
%} \newline
%&Cause\_change\_of\_appearance\newline
%{\begin{tabular}{ll}
%\hlblue{Patient} & \hlred{Agent} \\
%\hllime{Body\_part} & \hlteal{Cause}
%\end{tabular}
%} \newline\\ \midrule
\framename{Becoming\_dry}\newline
{\begin{tabular}{ll}
\hlblue{Item} &
\end{tabular}
} \newline  eng-30-00219403-v: \{dry:2; dry out:2\}, `become dry or drier'
&\framename{Cause\_to\_be\_dry}\newline
{\begin{tabular}{lll}
\hlblue{Dryee} & \hlred{Agent}   & \hlteal{Cause}
\end{tabular}
} \newline eng-30-00218475-v: \{dry:1; dry out:1\} `remove the moisture, make dry' \\ \midrule
&Cause\_to\_be\_wet\newline
{\begin{tabular}{lll}
\hlblue{Patient} & \hlred{Agent}  & \hlteal{Cause}\\
\hllime{Liquid} &&
\end{tabular}
} \newline eng-30-00214951-v: \{wet:2\}, `cause to become wet'\\ \midrule
%&Cause\_to\_be\_sharp \newline
%{\begin{tabular}{ll}
%\hlblue{Patient} & \hlred{Agent} \\
% & \hlteal{Cause}
%\end{tabular}
%} \newline\\ \midrule
\framename{Corroding}\newline
{\begin{tabular}{ll}
\hlblue{Patient} &
\end{tabular}
} \newline eng-30-00273963-v: \{corrode:1; rust:2\}, `become destroyed by water, air, or a corrosive such as an acid'
& \\ 
\lspbottomrule
\end{tabularx}
\caption{Frames for verbs of change in the physical parameters.}
\label{table1a}
\end{table}

A large part of the verbs in this group are derived from adjectives expressing the relevant properties underlying the definition of the corresponding frames. Some of these frames are not included in the original frame system of FrameNet, but were defined in the course of subsequent work on FrameNet frames and their assignment to the synsets in BulNet \citep{LesevaStoyanova2021change}.

Examples \ref{ex201a}--\ref{ex201b} show corresponding realisations of common valence patterns for the inchoative and the causative verbs, respectively. As seen from example \ref{ex201a}, although the frames concern physical characteristics of objects, they can receive metaphorical use as well (increasing in size of non-living entity, e.g. shadows, example \ref{ex201a}) or be applied to abstract entities (expand activity, example \ref{ex201b}). The causative frame \framename{Cause\_expansion} requires an additional frame element, \fename{Agent} or \fename{Cause}.


 \begin{exe}
 \ex \label{ex201}
 \begin{xlist}
\ex \label{ex201a} \framename{Expansion}: [NP.Ext]$_{\feinsub{Item}}$ \\
\gll [\textit{Дългите} \textit{сенки}]$_{\feinsub{Item}}$ \textit{\textbf{РАСТАТ}} \textit{по} \textit{пясъка} \textit{и} \textit{доближават} \textit{към} \textit{нас}.  (BulSemCor)\\
[Long-DEF shadows]$_{\feinsub{Item}}$ grow on sand-DEF and approach to us. {}\\
\glt `The long shadows grow on the sand and approach us.'
\ex \label{ex201b}  \framename{Cause\_expansion}: [NP.Ext]$_{\feinsub{Age}}$ [NP.Obj]$_{\feinsub{Item}}$  \\
\gll [\textit{Софийският} \textit{университет}]$_{\feinsub{Age}}$ \textit{постоянно} \textit{\textbf{РАЗШИРЯВА}} [\textit{участието си} \textit{в} \textit{културния} \textit{живот}]$_{\feinsub{Item}}$.  (BulSemCor)\\
[{Sofia-DEF} {University}]$_{\feinsub{Age}}$ constantly expands [participation-DEF in cultural-DEF life]$_{\feinsub{Item}}$. {}\\
\glt `Sofia University constantly expands its participation in cultural life.'
 \end{xlist}
 \end{exe}



\subsection{Verbs denoting change in integrity}\label{sec:breaking}


Some of the verbs are related to a (reversible or irreversible) change in the physical integrity of an object (\tabref{table1}). This group includes the class of verbs of breaking. The reason they are considered separately is the fact that they mostly involve a momentous change in state rather than a change on a scale like the verbs describing change of physical characteristics above.

The pair of examples \ref{ex20a}--\ref{ex20b} illustrate the use of verbs in this class with their core frame elements and typical valence patterns. The frame \framename{Cause\_to\_fragment} inherits from the frame \framename{Transitive\_action}, and also  (weakly) inherits from the frame \framename{Destroying} (through the \FrameRelation{Uses} relation).

\begin{table}
\begin{tabularx}{\textwidth}{QQ}
\lsptoprule
Inchoative & Causative\\\midrule
\rowcolor{lightgray}  \multicolumn{2}{l}{Frames denoting (reversible or irreversible) change in integrity}\\
\rowcolor{lightgray}  \multicolumn{2}{l}{Break Verbs (45.1), Destroy Verbs (44)} \\ \midrule
& \framename{Destroying} \newline
{\begin{tabular}{ll}
\hlblue{Patient} & \hlred{Destroyer} \\
& \hlteal{Cause}
\end{tabular}
} \newline
eng-30-01619929-v: \{destroy:4; destruct:1\}, `do away with, cause the destruction or undoing of' \\ \midrule
\framename{Breaking\_apart} \newline
{\begin{tabular}{ll}
\hlblue{Whole} &   \\
\hlcyan{Pieces} &
\end{tabular}
} \newline
eng-30-00334186-v: \{break:12; separate:3; split up:1; fall apart:1; come apart:1\}, `become separated into pieces or fragments'
& \quad  > \framename{Cause\_to\_fragment} \newline
{\begin{tabular}{ll}
\hlblue{Whole\_patient} & \hlred{Agent}   \\
\hlcyan{Pieces} & \hlteal{Cause}
\end{tabular}
} \newline
eng-30-00334996-v: \{break:13\} `destroy the integrity of; usually by force; cause to separate into pieces or fragments', \\
\lspbottomrule
\end{tabularx}
\caption{Frames for verbs of change in (physical) integrity}
\label{table1}
\end{table}


 \begin{exe}
 \ex \label{ex20}
 \begin{xlist}
\ex \label{ex20a} \framename{Breaking\_apart}: [NP.Ext]$_{\feinsub{Who}}$  [PP]$_{\feinsub{Pie}}$\\
\gll [\textit{Моливът}]$_{\feinsub{Who}}$ \textit{се} \textit{беше} \textit{\textbf{СЧУПИЛ}} [\textit{на} \textit{две}]$_{\feinsub{Pie}}$.  (BulSemCor)\\
Pencil-DEF REFL has broken into two. {}\\
\glt `The pencil has broken into two.'
\ex \label{ex20b}  \framename{Cause\_to\_fragment}: [NP.Ext]$_{\feinsub{Age}}$ [NP.Obj]$_{\feinsub{WhoPat}}$ [PP]$_{\feinsub{Pie-INI}}$  \\
\gll [\textit{Той}]$_{\feinsub{Age}}$ \textit{\textbf{ИЗПОТРОШИ}} [\textit{всички} \textit{прозорци}]$_{\feinsub{WhoPat}}$ [\_]$_{\feinsub{Pie-INI}}$.  \\
He broke all windows. {} {}\\ 
(BulSemCor) \\
\glt `He broke all windows.'
 \end{xlist}
 \end{exe}

\subsection{Verbs denoting transition into a new state}\label{sec:transition}

The verbs in this class are predominantly inchoative and include transition into a new state or phase in the development according to some known model concerning the development or the functioning of the \fename{Entity} (\tabref{table2}), or the existence of the \fename{Entity} (\tabref{table2a}).

Examples \ref{ex21a} and  \ref{ex21b} show the use of the frame \framename{Transition\_to\_a\_state} with the \fename{Entity} realised as the external element. The new state can be expressed by an adjective, a noun phrase or an adverbial expression. The frame is applicable to both momentous transition  (Example \ref{ex21a}) and gradual (scalable) change, e.g. \textit{become sensitive} -- \textit{become more sensitive} (Example \ref{ex21b}).

\begin{table}
\begin{tabularx}{\textwidth}{QQ}
\lsptoprule
Inchoative & Causative \\\midrule
\rowcolor{lightgray}\multicolumn{2}{l}{ Frames denoting transition into a new state according to a known model}\\
\rowcolor{lightgray}\multicolumn{2}{l}{Verbs of Entity-Specific Change of State (45.5)
} \\ \midrule
\framename{Progression} \newline
{\begin{tabular}{ll}
\hlblue{Entity} &  \\
\hlviolet{Prior\_state} &\\
\hlcyan{Post\_state}&
\end{tabular}
} \newline eng-30-00094460-v: \{grow:1; develop:1; produce:1; get:4; acquire:1\}, `come to have or undergo a change of (physical features and attributes)' &   \\ \midrule
\framename{Transition\_to\_a\_state} \newline
{\begin{tabular}{ll}
\hlblue{Entity} &  \\
\hlcyan{Final\_category} or&  \\
\hlcyan{Final\_quality} or&  \\
\hlcyan{Final\_situation}
\end{tabular}
} \newline eng-30-00149583-v: \{become:1; go:1; get:7\}, `enter or assume a certain state or condition' &   \\ \lspbottomrule
\end{tabularx}
\caption{Frames for verbs of transitioning into a new state according to a known model}
\label{table2}
\end{table}


\begin{table}
\begin{tabularx}{\textwidth}{QQ}
\lsptoprule
Inchoative & Causative \\\midrule
\rowcolor{lightgray} \multicolumn{2}{l}{Frames denoting coming into existence and ceasing to exist}\\
\rowcolor{lightgray} \multicolumn{2}{l}{Verbs of Appearance, Disappearance, and Occurrence (48)} \\ \midrule
\framename{Coming\_to\_be}\newline
{\begin{tabular}{ll}
\hlblue{Entity} &
\end{tabular}
} \newline eng-30-00425071-v: \{appear:2; come along:2\}, `come into being or existence, or appear on the scene' &  \\ \midrule
\quad > \framename{Being\_born} \newline
{\begin{tabular}{ll}
\hlblue{Child} &
\end{tabular}
} \newline & \framename{Giving\_birth}\newline
{\begin{tabular}{ll}
\hlblue{Child} &  \hlred{Mother} \\
&  \hlpurple{Father} \\
&  \hlpink{Parents}
\end{tabular}
} \newline \\ \midrule
\framename{Ceasing\_to\_be} \newline
{\begin{tabular}{ll}
\hlblue{Entity} &
\end{tabular}
} \newline eng-30-02156546-v: \{vanish:5; disappear:4; go away:4\}, `become invisible or unnoticeable' & \\ \midrule
\quad  > \framename{Death} \newline
{\begin{tabular}{ll}
\hlblue{Protagonist} &
\end{tabular}
} \newline& \framename{Killing}\newline
{\begin{tabular}{ll}
\hlblue{Victim} &  \hlred{Killer} \\
& \hlteal{Cause} \\
&  \hlorange{Instrument} \\
&  \hlpink{Means}
\end{tabular}
} \newline\\ 
\lsptoprule
\end{tabularx}
\caption{Frames for verbs of appearing and disappearing}
\label{table2a}
\end{table}


 \begin{exe}
 \ex
 \begin{xlist}
\ex \label{ex21a} \framename{Transition\_to\_a\_state}: [NP.Ext]$_{\feinsub{Ent}}$ \\
\gll \textit{Властта} \textit{не} \textit{искала} [\textit{примерът} \textit{им}]$_{\feinsub{Ent}}$ \textit{да} \textit{\textbf{СТАНЕ}} [\textit{заразителен}]$_{\feinsub{Finq}}$.  (BulSemCor)\\
Authorities-DEF {did not} want example-DEF their to become contagious. {}\\
\glt `The authorities did not want their example to become popular.'
\ex \label{ex21b}
\gll [\_]$_{\feinsub{Ent-DNI}}$ \textit{\textbf{СТАВАШ}} [\textit{по-чувствителен} \textit{на} \textit{външно} \textit{влияние}]$_{\feinsub{Finq}}$. \\
{} Become.2sg {more sensitive} to external influence. {}\\
\glt `You become more sensitive to outside influence.'
 \end{xlist}
 \end{exe}




\subsection{Verbs of creation and transformation}\label{sec:creation}


The verbs described by the listed frames involve various types of manipulation or modification that lead to the creation of the \fename{Entity} (\tabref{table3}): creating objects out of materials; building an object out of components; heat treatment and cooking, or its transformation (\tabref{table3a}).

\begin{table}
\begin{tabularx}{\textwidth}{QQ}
\lsptoprule
Inchoative & Causative \\\midrule
\rowcolor{lightgray}\multicolumn{2}{l}{Frames denoting creation}\\
\rowcolor{lightgray}\multicolumn{2}{l}{Verbs of Creation and Transformation (26), Cooking Verbs (45.3)
} \\ \midrule
  & \framename{Creating} \newline
{\begin{tabular}{ll}
\hlblue{Created\_entity} &  \hlred{Creator}
\end{tabular}
} \newline eng-30-01654628-v: \{construct:4; build:6; make:27\}, `make by combining materials and parts' \\ \midrule
 &  \quad > \framename{Intentionally\_create} \newline
{\begin{tabular}{ll}
\hlblue{Created\_entity} &  \hlred{Creator}
\end{tabular}
} \newline \\ \midrule
 &  \qquad >{}> \framename{Building} \newline
{\begin{tabular}{ll}
\hlblue{Created\_entity} &  \hlred{Agent}\\
\hlviolet{Components} &
\end{tabular}
} \newline \\ \midrule
% &  \qquad >{}> Create\_physical\_artwork\newline
%{\begin{tabular}{ll}
%\hlblue{Representation} &  \hlred{Creator}
%\end{tabular}
%} \newline\\ \midrule
 &  \qquad >{}> \framename{Manufacturing}\newline
{\begin{tabular}{ll}
\hlblue{Product} &  \hlred{Producer} \\
  &  \hlteal{Factory}
\end{tabular}
} \newline eng-30-01621555-v: \{produce:2; make:22; create:2\}, `create or manufacture a man-made product'\\ \midrule
% &  \qquad >{}> Text\_creation\\ \midrule
 &  \qquad >{}> \framename{Cooking\_creation} \newline
{\begin{tabular}{ll}
\hlblue{Produced\_food} &  \hlred{Cook}
\end{tabular}
} \newline eng-30-01664172-v: \{cook:3; fix:15; ready:4; make:28; prepare:5\}, `prepare for eating by applying heat' \\ \midrule
\framename{Absorb\_heat}\newline
{\begin{tabular}{ll}
\hlblue{Entity} &  \\
\hlorange{Container} & \\
\hlbrown{Heat\_source} &
\end{tabular}
} \newline eng-30-00375021-v: \{boil:3\}, `come to the boiling point and change from a liquid to vapor'
& \framename{Apply\_heat}\newline
{\begin{tabular}{lp{2.5cm}}
\hlblue{Food} & \hlred{Cook}\\
\hlorange{Container}  & \\
\multicolumn{2}{l}{\hlbrown{Heat\_instrument} }
\end{tabular}
} \newline eng-30-00328128-v: \{boil:1\}, `immerse or be immersed in a boiling liquid, often for cooking purposes'\\ 
\lspbottomrule
\end{tabularx}
\caption{Frames for verbs of creation }
\label{table3}
\end{table}



\begin{table}
\begin{tabularx}{\textwidth}{QQ}
\lsptoprule
Inchoative & Causative \\\midrule
\rowcolor{lightgray}\multicolumn{2}{l}{Frames denoting transformation}\\
\rowcolor{lightgray}\multicolumn{2}{l}{Verbs of Creation and Transformation (26)
} \\ \midrule
\framename{Undergo\_transformation} \newline
{\begin{tabular}{ll}
\hlblue{Entity} & \\
\hlviolet{Initial\_category}& \\
\hlcyan{Final\_category} &
\end{tabular}
} \newline
&\framename{Cause\_change} \newline
{\begin{tabular}{ll}
\hlblue{Entity} & \hlred{Agent}\\
\hlviolet{Initial\_category} &\\
\hlcyan{Final\_category}& \\
\hlviolet{Initial\_value} & \\
\hlcyan{Final\_value}&
\end{tabular}
} \newline
 \\ \lspbottomrule
\end{tabularx}
\caption{Frames for verbs of transformation}
\label{table3a}
\end{table}


Predominantly the verbs of creation in this group imply agentivity, and thus have no inchoative counterparts. Verbs denoting transformation have both an inchoative and causative counterpart. Here transformation is assumed to lead to a categorically new \fename{Entity} (starting as one category and ending in a new category of object) rather than it entering a new state or phase in its development as verbs in \sectref{sec:transition}.

While inheriting from \framename{Intentional\-ly\_create} the semantic frame \framename{Cooking\_creation} also inherits (weakly) from \framename{Apply\_heat}. As a consequence, these fra-\newline mes exhibit more complex frame-to-frame relations. This is why the causative -- inchoative pair of frames \framename{Apply\_heat} -- \framename{Absorb\_heat} are also included in \tabref{table3}.

Examples \ref{ex22a} and \ref{ex22b} show the inheritance between the frames \framename{Intentionally\_create} and \framename{Cooking\_creation} with the inheritance between the frame elements, which become more concrete and specific. While the frame \framename{Intentionally\_create} allows for both abstract and concrete \fename{Created\_entity}, the frame \framename{Cooking\_creation} implies that the entity is food or similar edible product of human activity (cooking).

 \begin{exe}
 \ex
 \begin{xlist}
\ex \label{ex22a} \framename{Intentionally\_create}: [NP.Ext]$_{\feinsub{Creator}}$ [NP.Obj]$_{\feinsub{CrEnt}}$ \\
\gll [\textit{Президентът}]$_{\feinsub{Creator}}$ \textit{\textbf{СЪСТАВЯ}} [\textit{служебно} \textit{правителство}]$_{\feinsub{CrEnt}}$.  (BulSemCor)\\
President-DEF forms caretaker government. {}\\
\glt `The president forms a caretaker government.'
\ex \label{ex22b}  \framename{Cooking\_creation}: [NP.Ext]$_{\feinsub{Cook}}$ [NP.Obj]$_{\feinsub{Food}}$ \\
\gll \textit{Вкъщи} [\textit{тя}]$_{\feinsub{Cook}}$ \textit{си} \textit{\textbf{НАПРАВИ}} [\textit{кафе}]$_{\feinsub{Food}}$.   \\
{At home} she herself made coffee. \\
\glt `At home she made herself a coffee.'
 \end{xlist}
 \end{exe}

%\subsection{Verbs of change of location}\label{sec:locative}



%\begin{table}
%\begin{tabular}{|p{5.7cm}|p{5.7cm}|}
%\midrule
%\textbf{Inchoative} & \textbf{Causative} \\\midrule
%\rowcolor{lightgray}  \multicolumn{2}{|p{12cm}|}{\centering Frames denoting change of location
%  \newline
%Verbs of Inherently Directed Motion (51.1), Leave Verbs (5.2),
%Verbs of Putting (9), Verbs of Removing (10)
%} \\ \midrule
%Motion  &Cause\_motion  \\ \midrule
%\quad > Motion\_directional  &  \quad >  Placing \\ \midrule
%\quad > Motion\_directional  &  \quad > Bringing\\ \midrule
%\quad > Fluidic\_motion &  \quad > Cause\_fluidic\_motion \\ \midrule
%\quad > Departing &   \\ \midrule
%\qquad >{}> Quitting\_a\_place &  \\ \midrule
% &  \quad > Removing\\ \midrule
%\quad > Motion\_directional  & Filling \\
% & Emptying
% \\ \midrule
%\end{tabular}
%\caption{Verbs of change of location}
%\label{table4}
%\end{table}


%\subsection{Verbs of change of possession}\label{sec:possession}

%Verbs of change of possession are of the kind `x causes y to have z', in which x is the Agent, y is the Recipient, and z is the Theme. They imply agentivity and thus only appear with causative frames.

%\begin{table}
%\begin{tabular}{|p{5.7cm}|p{5.7cm}|}
%\midrule
%\textbf{Inchoative} & \textbf{Causative} \\\midrule
%\rowcolor{lightgray}  \multicolumn{2}{|p{12cm}|}{\centering Frames %denoting change of possession
%  \newline
%Verbs of Change of Possession (13)
%} \\ \midrule
% & \framename{Getting} \\ \midrule
% &  \quad > \framename{Receiving} \\ \midrule
% &  \qquad  >{}> \framename{Borrowing}\\ \midrule
% &  \quad > \framename{Taking} \\ \midrule
%\end{tabular}
%\caption{Frames for verbs of change of possession}
%\label{table5}
%\end{table}



%\subsection{Verbs of psychological change}\label{sec:phycho}

%Verbs of change of psychological state require a sentient object and the verb describes the object entering a new psychological state.

%\begin{table}
%\begin{tabular}{|p{5.7cm}|p{5.7cm}|}
%\midrule
%\textbf{Inchoative} & \textbf{Causative} \\\midrule
%\rowcolor{lightgray}  \multicolumn{2}{|p{12cm}|}{\centering Frames denoting psychological change
%  \newline
%Amuse Verbs (31.1)
%} \\ \midrule
%Experiencer\_focused\_emotion &Getting  \\ \midrule
 %&  \quad > Cause\_emotion
 %\\ \midrule
%\end{tabular}
%\caption{Verbs of psychological change}
%\label{table6}
%\end{table}


\subsection{Discussion}\label{sec:discussion}

Observations on the data show some discrepancies in the information represented in different resources. The lack of conformity can be due to various factors: 

\begin{enumerate}[label=(\alph*)]
\item Some frames describe verbs in which the change is necessarily caused by an external agent (\fename{Agent} or \fename{Cause}) and cannot take place spontaneously (e.g., frame \framename{Creating}), thus the frame is only causative and does not enter a causative\slash inchoative frame pair.

\item Some gaps are due to systemic factors and the imbalance between the causative and inchoative verbs where a certain verb does not have a counterpart (blocked by the meaning of the verb, for example, the inchoative frame \framename{Coroding} does not have a causative counterpart).

\item The absence of certain frames in the frame structure might be due to the incompleteness in the frame system.  For example, \restructured{frames such as \framename{Becoming\_wet}, \framename{Becoming\_harp}, etc. would fulfill gaps in the inchoative as correspondences of existing frames \framename{Cause\_to\_be\_wet} and \framename{Cause\_to\_be\_sharp}. However, in some cases there is only a limited number of lexical units and introducing a new frame might be impractical.}
\end{enumerate}

\added{In some cases, introducing new frames in order to ensure a ballanced representation in resources might be helpful for language with a well-developed class of specific verbs, while not necessarily so for another language where this class of verbs is not present. For example, in the Princeton WordNet there is a large set of verbs for cooking based on the \fename{Manner}, such as \textit{simmer, braise, coddle}, or \fename{Heating\_instrument} involved, such as \textit{griddle, barbecue, microwave}, etc. (direct or indirect hyponyms of the synset eng-30-00322847-v: \{\textit{cook}:1\}, `transform and make suitable for consumption by heating'). These are without lexicalisation in Bulgarian and a detailed set of frames profiling either the \fename{Manner} or the \fename{Heating\_instrument} may not be needed.}

The analysis of the available asymmetries sheds light on the semantic features that can give rise to a comprehensive classification of the verbs of change and gives evidence for the formulation of new frames relevant to the Bulgarian language.

% (e.g., eng-30-01605692-v: \{sop:3; soak through:1\} `be or become thoroughly soaked or saturated with a liquid')

A set of principles can be derived for the consistent semantic description of verbs of change through semantic frames: (i) parallelisation of corresponding areas of the lexical hierarchy (such as causative and inchoative verbs of change) in WordNet and linking them with appropriate relations where necessary; (ii) detection of inconsistencies and gaps in the hierarchical structure in each of the two resources, such as frames which are not defined but whose existence is predicted by the general structure of FrameNet (e.g., causative or inchoative correlates of existing frames); frames that adequately reflect the level of specialisation and concretisation of meanings; frames describing parts of the vocabulary not yet covered, etc.

The significant alignment between Levin's syntactically oriented classification, the hierarchical organisation of the verbs of change in the lexical-semantic network WordNet, and the corresponding system of frames in FrameNet point out to the relevance of the identified key semantic features.
Furthermore, observations on Bulgarian data, as well as the examples presented in \sectref{sec:classification}, show that valence patterns from English are at least to some degree applicable to Bulgarian, and possibly to other languages. The valence patterns considered here cover the most typical (with high frequency in the annotated examples) realisation of the configurations of core frame elements for each frame. \removed{This supports the hypothesis that conceptual information is to a large degree language-independent and transferable across languages.} The particular syntactic patterns in the realisation of frame elements exhibit language-specific features and require procedures for validation.

\section{Conclusion}\label{sec:conclusions}

The argument expression possibilities of verbs of change appear to be determined by their lexicalised property -- the property subject to change, together with a set of semantic features (such as the type of change -- causative or inchoative, quantised or non-quantised, as well as the restrictions on the participants in the situation, etc.). Although argument expression cannot be handled by purely aspectual non-lexical theories of argument projection, verb aspect still plays a key role in the realisation of arguments and the alternations that the verb enters.

The current study outlines some of the main specific features of the verbs denoting change and does not aim at completeness and extensive coverage of all semantic classes. As is evident from the data, the class of verbs of change covers a wide range of semantically diverse verbs. More detailed analysis is required to uncover the specific features of certain of its subclasses and to be able to fully describe their syntactic realisation and alternations.


A contribution of the present study is the proposed classification of verbs of change in WordNet, which goes beyond classifying lexical units (single verbs) to classifying synsets (sets of verbs with the same lexical meaning). \restructured{This is in line with the assumption that lexical meaning and semantic features determine the syntactic behaviour of verbs. On the other hand, differences in the realisation of verbs from the same WordNet synset lead to the conclusion that additional valence description adds on the semantics of the verb in order to define its syntactic behaviour. Thus, a verb-specific level of representation of syntactic patterns is needed in order to obtain a more comprehensive description of the verb classes.}

Employing semantic frames in order to present the semantic and conceptual description of synsets in WordNet facilitates the parallel study of both semantically related words (via WordNet relations) and their corresponding conceptual descriptions (from FrameNet frames). Frame-to-frame relations provide means for aligning lexical-semantic relations to conceptual relations and studying the main features that influence the syntactic realisation of frame elements across different frames.

Research in this direction can also contribute to enriching both WordNet and FrameNet and improving their structure to accommodate the complex semantic structure of verbs denoting change.


Further, as already discussed, WordNet also facilitates cross-linguistic analyses and transfer of information from one language to another and can be used as means to expand conceptual and semantic resources for less-resourced languages such as Bulgarian.


\section*{Abbreviations}

\begin{multicols}{2}
\begin{tabbing}
MMMMM \= Agent\kill
\scshape Age \> \fename{Agent}\\
CNI \> Constructional null \\ \> instantiation\\
\scshape CrEnt \> \fename{Created\_entity}\\
\scshape Cse \> \fename{Cause}\\
DNI \> Definite null instantiation\\
\scshape Ent \> \fename{Entity}\\
\scshape Finq \> \fename{Final\_quality}\\
\scshape Food \> \fename{Produced\_food}\\
INI \> Indefinite null \\ \> instantiation\\
\scshape Ins \> \fename{Instrument}\\
\scshape Mns \> \fename{Means}\\
N or n \> Noun\\
NP \> Noun phrase\\
\scshape Pat \> \fename{Patient}\\
\scshape Pie \> \fename{Pieces}\\
PP \> Prepositional phrase\\
PWN \> Princeton WordNet\\
\scshape Rec \> \fename{Recipient}\\
\scshape Src \> \fename{Source}\\
V or v \> Verb\\
\scshape Who \> \fename{Whole}\\
\scshape WhoPat \> \fename{Whole\_patient}
\end{tabbing}
\end{multicols}

\section*{Acknowledgements}
\largerpage
This research is carried out as part of the project \emph{Enriching Semantic Network WordNet with Conceptual Frames} funded by the Bulgarian National Science Fund, Grant Agreement No. KP-06-H50/1 from 2020.

%\section*{Contributions}
%John Doe contributed to conceptualization, methodology, and validation.
%Jane Doe contributed to the writing of the original draft, review, and editing.

{\sloppy\printbibliography[heading=subbibliography,notkeyword=this]}
\end{document}

