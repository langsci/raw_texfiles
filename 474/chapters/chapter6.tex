\documentclass[output=paper,colorlinks,citecolor=brown]{langscibook}
\ChapterDOI{10.5281/zenodo.15682194}
\author{Hristina Kukova\orcid{0000-0001-9938-5462}\affiliation{Department of Computational Linguistics, Institute for Bulgarian Language, Bulgarian Academy of Sciences}}

\title{Frame semantics and verbs of emotion}

\abstract{The intersection of lexical semantics and syntax has been an important area of linguistics for some time. Verbs as the core of the lexicon are key to exploring the interaction between syntax and semantics and to understanding the nature of the lexicon. The study focuses on verbs of emotion in the Bulgarian language and their frame semantics. An overview of theoretical and empirical observations forms the general aim of the study. Neutral, positive and negative verbs of emotion are discussed and the results are summarised. The analysis is based on the semantic and partly morphological information of the lexical units from the WordNet (\cite{Fellbaum1998}) as well as on the semantic and syntactic features with which the investigated emotion verbs are represented in the FrameNet (\cite{Baker1998}, \cite{Ruppenhofer2016}). Five semantic frames are documented, which were selected due to their high frequency and the wide variety of lexical units they are evoked by. The description includes grammatical features of the lexical units, semantic and syntactic restrictions that verbs impose on the frame elements, and the assignment of the frame elements to a WordNet noun synset or subtree that reflects the realisation of the frame elements in context. The status of the frame elements, which is essential for the realisation of a lexical unit, is retrieved from FrameNet.}




\IfFileExists{../localcommands.tex}{
  \addbibresource{../localbibliography.bib} 
   \usepackage{langsci-optional}
\usepackage{langsci-gb4e}
\usepackage{langsci-lgr}

\usepackage{listings}
\lstset{basicstyle=\ttfamily,tabsize=2,breaklines=true}

%added by author
% \usepackage{tipa}
\usepackage{multirow}
\graphicspath{{figures/}}
\usepackage{langsci-branding}

   
\newcommand{\sent}{\enumsentence}
\newcommand{\sents}{\eenumsentence}
\let\citeasnoun\citet

\renewcommand{\lsCoverTitleFont}[1]{\sffamily\addfontfeatures{Scale=MatchUppercase}\fontsize{44pt}{16mm}\selectfont #1}
  
   %% hyphenation points for line breaks
%% Normally, automatic hyphenation in LaTeX is very good
%% If a word is mis-hyphenated, add it to this file
%%
%% add information to TeX file before \begin{document} with:
%% %% hyphenation points for line breaks
%% Normally, automatic hyphenation in LaTeX is very good
%% If a word is mis-hyphenated, add it to this file
%%
%% add information to TeX file before \begin{document} with:
%% %% hyphenation points for line breaks
%% Normally, automatic hyphenation in LaTeX is very good
%% If a word is mis-hyphenated, add it to this file
%%
%% add information to TeX file before \begin{document} with:
%% \include{localhyphenation}
\hyphenation{
affri-ca-te
affri-ca-tes
an-no-tated
com-ple-ments
com-po-si-tio-na-li-ty
non-com-po-si-tio-na-li-ty
Gon-zá-lez
out-side
Ri-chárd
se-man-tics
STREU-SLE
Tie-de-mann
}
\hyphenation{
affri-ca-te
affri-ca-tes
an-no-tated
com-ple-ments
com-po-si-tio-na-li-ty
non-com-po-si-tio-na-li-ty
Gon-zá-lez
out-side
Ri-chárd
se-man-tics
STREU-SLE
Tie-de-mann
}
\hyphenation{
affri-ca-te
affri-ca-tes
an-no-tated
com-ple-ments
com-po-si-tio-na-li-ty
non-com-po-si-tio-na-li-ty
Gon-zá-lez
out-side
Ri-chárd
se-man-tics
STREU-SLE
Tie-de-mann
}
   \boolfalse{bookcompile}
   \togglepaper[23]%%chapternumber
}{}

\begin{document}
\maketitle

\section{Introduction} 
The aim of this study is to present the emotion verbs of the Bulgarian vocabulary. We apply the methodology of frame semantics to outline different constructions in which verbs of emotion are involved. We also use the BulNet semantic network to extract their characteristic meanings. Therefore, the verbs under investigation are presented in specific WordNet synsets containing lexical and morphological information. We then describe each predicate within the semantic frame it evokes, together with its frame elements (FEs) and their selectional restrictions, which are expressed in terms of specific WordNet synsets or subtrees. We assume that the lower levels (hyponyms) of the selected subtree can also occupy the FE position.

In this study, we adopt a usage-based approach and provide evidence for the importance of context in semantic analysis and frame profiling. The analysis of the corpus data contributes to the development of a theoretically and empirically coherent approach to describing the semantic and syntactic features of verb classes.

The main aims of this study are: (i) to systematise the main theoretical findings on emotion verbs; (ii) to analyse semantic frames and their frame elements; (iii) to demonstrate how syntactic realisations can be predicted by lexical semantics within a given verb class; (iv) to highlight the importance of the interaction between semantics (lexical-semantic properties) and syntax (syntactic behaviour).
 

We rely on \citegen{Levin1993} study, which categorises verbs of psychological state into four major subclasses based on both intuitive semantic grouping and participation in valency alternations. We consider the transitivity / intransitivity of the verbs \textit{желая} `wish' / \textit{страхувам се} `fear' and the possibility of taking the \fename{Experiencer} as a grammatical subject -- \textit{обичам} `love', or object -- \textit{харесва ми} `appeal to' in a sentence to further subdivide them. This division is reflected in a verb’s evoking
  \framename{Experiencer\_focused\_emotion} -- \textit{завиждам} `envy', or \framename{Stimulate\_emotion} and \framename{Cause\_to\_experience}  semantic frames -- \textit{изненадвам} `surprise', \textit{дразня} `annoy'. 
 

If we take \textit{съжалявам} `regret' as an example of a verb that evokes the \framename{Contrition} frame, we can see that in most cases the position of the \fename{Experiencer} is occupied in context  by the synsets eng-30-00007846-n: \{\textit{човек}\}  `person' or eng-30-07950920-n: \{\textit{социална група}\} `social group'. The \fename{Action} FE ``marks expressions that indicate a prior action that the \fename{Experiencer} has come to feel bad about'' and can be encoded both as a PP (with the prepositions \textit{за} `for' and \textit{заради} `because of') or as a clause (with the help of the conjunctions \textit{че} `that', \textit{задето} `for' and the interrogative pronouns \textit{как} `how', \textit{къде} `where', \textit{какво} `what', \textit{кой} `who', \textit{колко} `how much/many'. The prepositions \textit{за} `for' and \textit{заради}  `because of' in turn take an object, which can vary between the following synsets: eng-30-00029378-n: \{\textit{събитие}\} `event', eng-30-00030358-n: \{\textit{действие}\} `act',  eng-30-00037396-n: \{\textit{действие}\} `action', or eng-30-05770926-n:  \{\textit{умствена дейност}\} `thought process'. We take into consideration the possible selectional restrictions a verb imposes on its frame elements and group verbs further into subclasses.
 

The study is based on corpus data; unless otherwise stated, the examples are taken from the Bulgarian National Corpus \citep{Koeva2012}.

The rest of the paper is organised as follows.
Section \ref{ch6:sec:2} deals with the notion of conceptual frame and its preconditions. The resources used are explained in detail. Section \ref{ch6:sec:3} outlines previous studies and motivation. The focus is on the description of the class of verbs of emotion and the different criteria for categorisation. In the same section and throughout the paper, the differences in classification systems serve as a basis for distinguishing between subclasses of verbs of emotion. Section \ref{ch6:sec:4} gives an overview of the linguistic descriptions of the Bulgarian verbs of emotion and their special features. Section \ref{ch6:sec:5} deals with the semantic features of verbs of emotion in Bulgarian. It includes descriptions of different semantic frames and their frame elements. Section \ref{ch6:sec:6} summarises the results of this study and concludes the paper.

 

\section{Resources} \label{ch6:sec:2}


WordNet is a lexical-semantic network suitable for machine processing that was originally developed at Princeton University by a team led by George Miller (\cite{Miller1995}, \cite{Fellbaum1998}). The Bulgarian version of WordNet -- BulNet -- contains more than 100,000 synsets (\cite{koeva2021a}).

Although BulNet was used to represent the semantic and paradigmatic features of the predicates, the most important resource for their ``semantic and syntactic combination possibilities is FrameNet \citep[7]{Ruppenhofer2016}. FrameNet was launched in 1997 under the guidance of Charles Fillmore (\cite{Baker1998}) and is essential for both theoretical linguistic research and practical natural language processing.

Semantic frames represent the conceptual structure of an event or object and its participants.
Frame elements can be regarded as semantic roles. They can be core and non-core elements, the former being essential for the realisation of the respective semantic frame, while the latter are mostly descriptive (in terms of time, place, etc.). Lexical units are lemmas that describe a situation (frame). Each meaning of a word is encoded as a separate lexical unit and evokes a different semantic frame.

As Koeva and Doychev state, ``a Conceptual frame defines a unique set of syntagmatic relations between verb synsets representing the frame and noun synsets expressing the frame elements'' (\cite[203]{koeva-doychev-2022-ontology}). Based on the information we extract from the WordNet about a verb meaning and the syntactic and semantic restrictions it imposes on its FrameNet frame elements, we create a grid of possible combinations. All analysed verbs are considered separately in each sense, and their frame elements can be an NP, PP, AdvP, AccCl (obligatory accusative clitic), DatCl (obligatory dative clitic) or a clause element (S or small clause). We use the web-based system BulFrame to create and visualise conceptual frames \citep{koeva-doychev-2022-ontology}.
 
In order to provide an exhaustive description of the Bulgarian verbal lexical units, the following information was used:

\begin{itemize}
\item[(a)] From FrameNet: core and non-core frame elements, their semantic types, the sets of verbal lexical units associated with a given semantic frame and the Inheritance relation between frames.

\item[(b)] From WordNet: hypernym-hyponym relations, which organise synsets for nouns and verbs in hypernym trees, and the semantic classes to which these synsets belong.
 \end{itemize}

The web-based system BulFrame, developed at the Department of Computational Linguistics of the Bulgarian Academy of Sciences, is used to create, edit, view and review the conceptual frames \citep{koeva-doychev-2022-ontology}.

Most of the language material was taken from the Bulgarian National Corpus (\cite{Koeva2012}), which was created at the Institute for Bulgarian Language ``Prof. Lyubomir Andreychin''. The Bulgarian National Corpus consists of a monolingual part containing 240,000 texts or 1.2 billion words and 47 parallel corpora.

\section{Previous studies and motivation} \label{ch6:sec:3}


\subsection{Methodology}


S. Koeva points out the need for a formal description of syntagmatic relations in WordNet (\cite{Fellbaum1998}, \cite{koeva2021a}) by introducing the notion of conceptual frame to define a set of verbs that have unique syntagmatic relations to nouns (\cite [182]{svetla2021towards}). Leseva et al. have also explored the possibility of integrating data from WordNet, FrameNet and VerbNet and proposed a system of semantic relations that reflects thematic relations between predicates and their potential arguments in the context of WordNet' (\cite{leseva2018integrating}). Our approach, which is based on frame semantics (\cite{Baker1998}, \cite{Koeva2010FN}), combines both the abstract syntactic level and the projection of semantic relations onto the corresponding frame elements.

Since frame analysis is very sensitive and error-prone, decision-making is delegated to human experts. To facilitate the process, we have chosen the following procedure (described in detail in the chapter \textit{Universality of semantic frames versus specificity of conceptual frames} in this volume).


\begin{description}
\item[Step 1:] We select the relevant verb meaning (literal) that evokes a particular fra-me from a set of synsets.
\item[Step 2:] We check whether all core frame elements of the frame are relevant for Bulgarian and/or whether additional frame elements should be included. We can either choose from the existing FEs where appropriate or insert a completely new one and give it a name.
\item[Step 3:] We define the possible selectional restrictions by (a) choosing from a list of noun synsets for NPs; (b) specifying the prepositions for PPs; (c) specifying the conjunctions that can introduce the dependent clauses.
\end{description}

Frame-semantic analysis with its flexibility and versatility can contribute to a number of NLP tasks and applications and to improving language understanding.
\largerpage[2]
 
\begin{itemize}
\item[(i)] Frame semantics provides a framework for semantic role labelling, i.e. identifying and labelling the different roles that entities play in a sentence. This process is crucial for tasks such as question answering, information extraction and text understanding.

\item[(ii)] Sentiment analysis. Frame semantics helps to better understand and analyse the emotions and attitudes expressed in a text. By capturing the semantic frames associated with the sentiment, sentiment analysis models can understand the implicit information in a more nuanced way.

\item[(iii)] Text classification. By considering semantic frames and their associated meanings, models can identify the implicit information in context and capture the intended meaning of a text, leading to more accurate and nuanced text classification.

\item[(iv)] Machine translation. Frame semantics helps to transfer meaning from one language to another by capturing the semantic frames and their semantic roles. This approach goes beyond word-to-word translation and ensures that the intended meaning of the source sentence is preserved in the target language, resulting in more accurate translations. In addition, metrics based on frame semantics, e.g. \citep{czulo2019designing}, have been proposed for machine translation evaluation, e.g. \citep{czulo2019designing}.

\item[(v)] Information retrieval and question answering. Frame semantics helps to improve search engine results and question-answering systems. By understanding the frames and semantic roles in queries and documents, these systems can retrieve more relevant information and provide accurate answers by matching semantic frames and roles.

\item[(vi)] Building knowledge graphs. Frame semantics is useful in building knowledge graphs by identifying the relationships between entities based on semantic frames and their FEs. It helps in organising and representing the structured knowledge from an unstructured text and contributes to tasks such as knowledge extraction and knowledge representation.
 \end{itemize}

Frame semantics also plays a crucial role in corpus research. It provides a framework for analysing and understanding the meaning and structure of texts within a given corpus. It can influence corpus research in various ways:

\begin{itemize}
\item[(vii)] Semantic analysis. By identifying and labelling semantic frames and their FEs, corpus studies can uncover patterns and relationships between entities, actions and events, leading to a deeper understanding of the underlying semantics within the corpus.

\item[(viii)] Semantic annotation. Frame semantics provides a systematic approach for annotating corpora with semantic information. Corpus studies can use frame-based annotation schemes to label frames and their FEs in texts, which enables more detailed analysis and in turn facilitates the development of machine learning models for various NLP tasks.

\item[(ix)] Comparative studies. Frame semantics enables comparative studies of different corpora or subsets within a corpus. Researchers can analyse variations in the use of frames in different genres, time periods or languages and find out how meaning and semantic structure differ in distant contexts. This helps to analyse linguistic and cultural differences, diachronic changes and genre-specific semantic patterns.

\item[(x)] Semantic similarity and clustering. By applying frame semantics to corpus studies, researchers can measure semantic similarity and cluster texts based on their frame-based representations. This facilitates tasks such as document clustering, topic modelling and information retrieval, where a better understanding of the semantic relationships between texts is essential.

\item[(xi)] Corpus-based lexical semantics. Frame semantics helps with corpus-based studies of lexical semantics. By analysing lexical items in the context of semantic frames and their FEs, corpus studies can uncover the nuances and contextual meanings associated with words, leading to the identification of polysemy, homonymy and semantic shifts within the corpus.

\item[(xii)] Corpus-based frame compilation. Corpus studies contribute to the compilation of frame databases or resources. By analysing large corpora, researchers can identify recurring semantic frames, frame-triggering lexical units and their roles, which serve as valuable data for building or extending frame resources.
 \end{itemize}

Overall, frame semantics provides a rich representation of the meaning and structure of language that enables NLP models to gain a deeper understanding of texts and perform a variety of tasks more effectively. It also provides a theoretical and practical basis for corpus studies, allowing researchers to delve deeper into the semantics of texts, compare different corpora, uncover patterns and improve our understanding of language structure and meaning within a given corpus.
 

\subsection{Verbs of emotion}

Emotions can be defined as experiences or states triggered off by a certain event, situation, action, other people, our thoughts, expectations and plans (\cite[155]{belaj2011cognitive}). In view of this phenomenon we attempt to relate the complexity of the syntax of emotions to the variety of their semantics as demonstrated in Section 1. 
 

In one recently published psychological encyclopedic manual (\cite[218]{Strickland2000}) emotions are defined as ``a reaction, both psychological and physical, subjectively experienced as strong feelings, many of which prepare the body for immediate action. In contrast to moods, which are generally longer lasting, emotions are transitory, with relatively well-defined beginnings and endings. They also have valence, meaning that they are either positive or negative. Subjectively, emotions are experienced as passive phenomena. Even though it is possible to exert a measure of control over one’s emotions, they are not initiated – they happen to people.''

 

As far as the linguistic field is concerned, there have been published a number of studies dealing with the description of emotion words, starting with  \citet{anna1971kocha}, \citet[57]{wierzbicka1972semantic}, \citet[142]{wierzbicka1980lingua}, \citet{wierzbicka1986human} and \citet{иорданская1970попытка}, \citet{iordanskaja1973tentative}, \citet{iordanskaja1986russian}. Wierzbicka was the first to observe that unlike other language groups, Slavic languages tend to use verbs to speak of emotions, which holds true for the Bulgarian language as well. Her early works include attempts to formalise emotions, defining emotion words in natural language and referring to typical situations that evoke particular emotional states. Both Wierzbicka and Iordanskaja put forward the concept of evaluation of the situation by X for the description of emotion words in linguistic semantics.  \citet{зализняк1983семантика}, \citet{зализняк1985функциональная} deals with what she calls ``predicates of internal state'', establishing the distinction of the assertion and presupposition in their definitions. \citet{Lakoff1987} and  \citet{kovecses1988language} pay attention to the uses of emotion expressions and metaphors in a given language, in order to describe a conceptual model of the corresponding emotion -- as it is perceived and expressed in actual speech.


Another widely disputed issue throughout the studies of verbs of emotion and specifically among Slavic authors is the verbs’ reflexivity or mediality. The most influential account of the Slavic verbs under discussion is offered by \citet{wierzbicka1988semantics}, \citet{wierzbicka1995everyday}. The author states that these verbs in Russian and Polish with -sja and \textit{-się} respectively are reflexive forms on the basis that they indicate ``emotions to which people `give themselves' almost voluntarily and which they outwardly express'' (\cite[253]{wierzbicka1988semantics}). As the author claims, expressing emotions by reflexive verbs implies that they are ``treated not as arising by themselves but by the speaker’s conscious thoughts about the event'' \citep[22]{wierzbicka1995everyday}. Moreover, she outlines the syntactic distinction between voluntary (with \fename{Experiencer} in nominative and the \textit{-sja} verb), involuntary (with dative \fename{Experiencer} and an adverbial predicative) and neutral emotion (with nominative \fename{Experiencer} and an adjectival predicative) (\cite[253--254]{wierzbicka1988semantics}). A. Bedkowska-Kopzcyk challenges Wierzbicka’s views and considers this type of verbs in Slovene middle voice verbs (\cite{bkedkowska2014verbs}).


As far as Bulgarian language is concerned, the particle \textit{се} can be involved in rather complex relations between words and constructions. It can represent both a word-forming and a morphological marker and can bear a passive (Example \ref{ch6:ex1a}), a medial (Example \ref{ch6:ex1b}) or a reflexive (Example \ref{ch6:ex1c}) meaning \citep[100--103]{tisheva2022положителни}. 

\begin{exe} 
\ex  \label{ch6:ex1} 
\begin{xlist}
\ex \label{ch6:ex1a} 
\gll
\textit{Пациент-ът} \textit{не} \textit{трябва} \textit{да} \textit{\uppercase{\textbf{се безпокои-Ø}}} \textit{(от никого)}. \\
patient-DEF.M not should to {REFL disturb-3.SG.PRS} {(by nobody)}\\
\glt `The patient should not be disturbed (by anybody).' 
\ex \label{ch6:ex1b} 
\gll
\textit{Пациент-ът} \textit{не} \textit{трябва} \textit{да} \textit{\uppercase{\textbf{се безпокои-Ø}}}. {(да изпитва безпокойство)} \\
patient-DEF not should to {REFL disturb-3.SG.PRS} {(to experience worry)}\\
\glt `The patient should not worry. (experience worries)'
\ex \label{ch6:ex1c} 
\gll
\textit{Син-ът} \textit{ми} \textit{вече} \textit{\uppercase{\textbf{се МИЕ-Ø}}} \textit{сам-Ø}. \\
son-DEF my already {REFL wash-3.SG.PRS} alone-M.SG\\
\glt `My son can already wash himself on his own.'
\end{xlist}
\end{exe}

 \citet[76]{tisheva2022syntactic} also address this polemical issue in their research on syntactic characteristics of emotion predicates\footnote{The authors explore both verbs and other constructions, based on adjectives, adverbs or nouns (predicatives).} in Bulgarian. According to the authors ``\textit{se} is a marker for middle voice construction and does not indicate reflexivity, it occupies the direct object position and those verbs could have only PP or a complement clause as their second argument.''



Since the current study focuses on the semantic and syntactic features of the verbs under discussion together with their possible complements as imposed by the verb sense, we will not deal with this particular aspect of the verb description. As in most cases the verbs used with and without reflexive  \textit{се} involve literals from different WordNet synsets, they will have different meanings and, respectively, heterogeneous frame elements’ restrictions. 


A large number of studies have been carried out on different language material in the last 20 years involving emotion verbs, their organisation in FrameNet and their semantic specifications. 



Taking emotion concepts as a basis, Ruppenhofer describes the evolution and the development of FrameNet analyses over time due to application-oriented goals. Taking into account different linguistic theories and approaches (dimensional, categorical, meaning-oriented, etc.), the author illustrates how fine-gra\-ined distinctions of lexical units lead to formulating new semantic frames or dividing one frame into two (\cite{ruppenhofer2018treatment}). The explanation of the steps and motivation underlying the conceptualisation and the development of the frame organisation holds a specific value for emotion frame descriptions and their detailed understanding. Thus, the \framename{Experiencer\_Subj} and \framename{Experiencer\_Obj} verbs were initially grouped by valence criteria whereas in the latest version the semantics of the verbs is also considered. 

\citet{subirats2003surprise} compare the Spanish lexical units with those of English in order to work out the similarities and differences in the lexicalisation patterns of the two languages. They use the annotation of Spanish verbs with the help of FrameNet frames to summarise the different syntactic realisations.
Since the Bulgarian grammatical and syntactic realisation has many more similarities with Spanish than with English, it was particularly useful to learn about their experience.\footnote{It shows closeness in agreement, the formation of questions, negation, the use of prepositions and, above all, word order.}


\citet{subirats2004spanish} presented the Spanish FrameNet and the web application that processes bilingual information and facilitates the comparison of the semantic structures of two lexicons.

\citet{ghazi2015detecting} make an attempt to automatically recognise the emotion \fename{Stimulus}. They assemble a dataset with manually labelled emotion stimuli and then apply sequential learning methods to a complementary dataset that does not contain labelled stimuli.

All of these studies form the basis for our research and have influenced the observations we will make in the central part of this chapter, in which we will examine the nature of emotion categorisation and the way it is formally reflected in grammatical and semantic structure, particularly in emotion-verb complement constraints.

\subsection{Classifications}

The typological description of emotion verbs has also proved interesting for various authors in different studies. In this section we give a brief overview of their approaches.
  
Based on emotion words in general, \citet{kovecses2003metaphor} proposes a division into expressive and descriptive emotion words, whereby he categorises emotionally charged comments and expressions of agreement and disagreement in the first group, while in the second group he categorises terms that denote a specific emotional experience. \citet [115]{tisheva2021наблюдения} also distinguishes between the lexical and grammatical means for the emotional attitude of the speaker/writer on the one hand and the naming of emotional states, relationships or evaluations on the other. In view of this subdivision, we will only deal with the descriptive emotion words in the following.

\citet[115]{tisheva2021наблюдения} claims that duration is a fundamental concept to draw the line between emotions and feelings. According to the author, ``emotions are spontaneous reactions to certain internal or external stimuli, while feelings are more permanent and enduring and always involve an evaluation of the object to which they are directed''.

Most linguistic classifications are based on the above-mentioned psychological aspects of emotions and divide them into positive and negative emotions depending on their basic tone. \citet{scherer2005emotions} recognises three characteristic features of emotions, namely: intensity, duration and the ability to evoke a reaction, and creates a typology of affective phenomena as presented below:
 
\begin{itemize}
\item[(a)] emotion: a relatively brief response to an external or internal \fename{Stimulus}  event, e.g. \textit{angry}, \textit{sad}, \textit{joyful}, \textit{fearful}, \textit{ashamed}, \textit{desperate}, 

\item[(b)] mood: a diffuse affect state characterised by low intensity but relatively long duration, often without apparent cause, e.g. \textit{cheerful}, \textit{gloomy}, \textit{depres-sed}, 

\item[(c)] interpersonal stance: affective stance taken toward another person in a specific interaction, e.g. \textit{distant}, \textit{warm}, \textit{supportive}, \textit{contemptuous},

\item[(d)] attitude: relatively enduring, effectively colored beliefs, preferences, and predispositions towards objects or persons, e.g. \textit{liking}, \textit{hating}, \textit{desiring}, 

\item[(e)] personality traits: emotionally laden, stable personality dispositions and behavior tendencies, typical for a person, e.g. \textit{nervous}, \textit{hostile}, \textit{jealous}, \textit{envious}. 
 \end{itemize}

\citet{ляшевская2011онтологические}, on the other hand, classify verbs of emotion on the basis of their semantic structure and the consistency of the verbal operational functors contained in each meaning. Thus, they categorise the verbs in question into five different groups: Event, Feeling, Attitude, State and Feature.

In the present study, we will not focus so much on the semantic differentiation, but rather on the syntactic realisation of the verbs and  the semantic specificity of their FEs, which plays a crucial role in the frame-semantic analysis.

In terms of their grammatical features, \citet{johnson1989language} refer to two types of emotion verbs (they also speak of emotion nouns and emotion adjectives): \textbf{emotional relations}, e.g. \textit{to love}, \textit{to fear}, and \textbf{causatives}, e.g. \textit{to annoy}, \textit{to frighten}. This observation is consistent with the two types frequently described in the linguistic literature. Syntactic structures in which the \fename{Experiencer} is the subject encode the emotional relation verb class, while the structures in which the \fename{Experiencer} is encoded as the grammatical object denote the causative verb class. The former are known across languages as \textbf{Subject-}\fename{Experiencer} verbs (SE), while the latter are known as \textbf{Object-}\fename{Experiencer} verbs (OE) \citep{dowty1991thematic,levin2005argument}. \citet{Fellbaum1999a} follows this line of linguistic description by saying that emotion predicates ``fall into two grammatically distinct classes: those whose subject is the animate \fename{Experiencer} and whose object (if any) is the \fename{Source} (\textit{fear, miss, adore, love, despise}); and those whose object is the animate \fename{Experiencer} and whose subject is the \fename{Source} (\textit{amuse, charm, encourage, anger})''.

The main subdivision in the Slavic languages follows the definition of the two groups of emotion verbs based on the syntactic expression of the \fename{Experiencer} as subject or direct or indirect object \citep [55]{Croft1993CaseMA}, \citep [21]{ovsjannikova2013encoding}, \citep [75]{tisheva2022syntactic}.

Based on these observations, three main subtypes are generally distinguished for Slavic languages: (i) SE verbs (Example \ref{ch6:ex2a}), (ii) OE verbs with the \fename{Experiencer} in the accusative case (Example \ref{ch6:ex2b}), and (iii) OE verbs with the \fename{Experiencer} in the dative case (Example \ref{ch6:ex2c}). This fact has been maintained by a number of Slavic linguists: for Russian – \citet{sonnenhauser2010event}, for Polish – \citet{bialy2005polish} and \citet{rozwadowska2007various}, for Bulgarian – \citet{slabakova1996bulgarian}, among others.


\begin{exe} 
\ex  \label{ch6:ex2} 
\begin{xlist}
\ex\label{ch6:ex2a} 
\gll \textit{Аз} \textit{наистина} \textit{\uppercase{\textbf{харесвам}}} \textit{вампир-и-те.} \\
I really like-1.SG.PRS vimpire-PL.DEF\\
\glt `I really  like  vampires.'
\ex\label{ch6:ex2b} 
\gll \textit{Тази} \textit{постоянна} \textit{светлина} \textit{почва} \textit{да} \textit{ме} \textit{{\textbf{ДРАЗНИ-Ø}}.} \\
this-F.SG constant-F.SG light start-3.SG.PRS to I-ACC annoy-3.SG.PRS\\
\glt `This constant light is starting to  annoy me.'
\ex\label{ch6:ex2c} 
\gll \textit{Мисля,} \textit{че} \textit{това} \textit{му} \textit{\uppercase{\textbf{харесва-Ø}}.} \\
think-1.SG.PRS that it he-DAT appeal-3.SG.PRS\\
\glt `I think he likes it.'
\end{xlist}
\end{exe}



Some authors also observe the possibility of forming diathetic verb pairs in which the \fename{Stimulus}-subject verb is transitive, while its counterpart \fename{Experien\-cer}-subject is an intransitive reflexive verb marked with a reflexive pronoun or the suffix \citep [121]{ovsjannikova2020instrumental}. \citet{koevaсистема} introduces the system of diatheses and alternations for Bulgarian.


\section{Bulgarian verbs of emotion} \label{ch6:sec:4}

The Bulgarian verbs of emotion, traditionally considered part of the larger psychological class of verbs, form an intriguing set. In her 2008 study, \citet [265]{ницолова2008проблематика} proposes to consider verbs such as \textit{обичам} `love', \textit{мразя} `hate', \textit{нена\-виждам} `detest' and others as ``mental predicates for emotional attitude''. \citet[62--63]{koeva2019complements} further subdivides them into predicates for emotional reaction or evaluation, (i) which are expressed by verbs \textit{харесвам} `like', \textit{съжалявам} `regret', \textit{радвам се} `be glad', \textit{страхувам се} `fear', \textit{тревожа се} `worry' or (ii) constructions like \textit{благодарен съм} `be grateful', \textit{яд ме е} `be mad', \textit{срам ме е} `be ashamed', \textit{тъжно ми е} `be sad'.

When looking at the argument structure of verbs and predicative expressions for emotions in the Bulgarian language, \citet{dineva2000} states that there are four types, namely: (i) one-argument constructions, realising only an \fename{Experiencer}, such as \textit{вълнувам се} `be excited', \textit{тъжно ми е} `be sad', \textit{страх ме е} `be afraid', \textit{спокоен съм} `be calm'; (ii) two-argument structures, expressing the \fename{Experiencer} and the \fename{Stimulus} with causative verbs, such as \textit{радвам} `rejoice', \textit{натъ\-жaвам} `sadden', \textit{ядосвам} `make angry', \textit{изненадвам} `surprise', or (iii) the \fename{Experiencer} and the Object with verbs for attitude such as \textit{обичам} `love', \textit{уважавам} `respect', \textit{харесвам} `like', \textit{ценя} `appreciate', \textit{обожавам} `adore' and (iv) verbs with three arguments -- an \fename{Experiencer} and alternating arguments, expressing the \fename{Stimulus} and the \fename{Object} \textit{Книгата ми харесва.} (I like the book.)  – \textit{Харесвам книга\-та.} ({The book appeals to me.})

\citet [102]{tisheva2022положителни} divides the verbs of emotion into two groups based on the semantic role of the subject in the sentence: subject\hyp Stimulus verbs and subject\hyp Experiencer verbs, which thus form conversive pairs (\textit{безпокоя – безпокоя се} `wor\-ry', \textit{радвам -- радвам се} `rejoice', \textit{обиждам – обиждам се} `insult' and so on). The state verbs of emotion involved in these oppositions are reflexive in form and therefore intransitive.  As a rule, the emotion state verbs with \textit{се} take the \fename{Stimulus} as PP, while the causatives encode the \fename{Stimulus} as NP. Tisheva notes that not all emotion verbs fall into these pairs. A number of authors state that verbs such as \textit{боя се} `be afraid', \textit{наслаждавам се} `enjoy', \textit{страхувам се} `fear' are not used without the reflexive \textit{се}, while \textit{тъжа} `grieve', \textit{тъгувам} `sorrow', \textit{тържествувам} `triumph' do not have a counterpart with \textit{се} \citep [24]{коева1996класификация}, \citep [232]{Nitsolova2008}, \citep[101--102]{tisheva2022положителни}. This is one of the reasons why the common semantic model comprising subject and object of emotion cannot be expressed with a universal structural equivalence drawn between the causatives and \textit{се}-verbs. 


\citet [394]{tisheva2022syntactic} note that these conversive pairs can represent one and the same situation and have two identical valences, although they are occupied by different actants. Causative predicates transfer the semantic role of the \fename{Experiencer} to the direct object, while stative predicates attribute it to the subject. According to the authors, state verbs of emotion in Bulgarian are considered primary predicates and causative predicates are considered semantically derived predicates, folowing the Van Valin and LaPolla's classification of predicates \citep{VanValinLaPolla1997}.

\citet [70]{Stamenov2021} categorises the Bulgarian verbs for internal psych experiences into 12 groups based on the semantic roles that each of them requires. In addition to verbs of emotion, Stamenov's structural classification also includes verbs of mentality and perception. Of the 12 types outlined by the author, we have singled out 7 that contain verbs of emotion (at least one) and are relevant for the purpose of our study:

\begin{itemize}
\item[(i)] Intransitive verbs whose lexical meaning expresses the inseparable unity of the \fename{Actor} and the \fename{Experiencer}: \textit{копнея} `crave', \textit{тъжа} `grieve';

\item[(ii)] Transitive verbs with and \fename{Experiencer} and \fename{Stimulus} or \fename{Object}: \textit{оби\-чам} `love', \textit{мразя} `hate', \textit{обожавам} `adore', \textit{харесвам} `like';

\item[(iii)] Verbs, expressing the semantic structure of a \fename{Stimulus} and a specific effect (on the \fename{Experiencer}) with a predicate of the CAUSE + DEVERBAL NOUN type: \textit{възторгвам} / \textit{възторгвам се} `enrapture' / `go into raptures' –\textit{ предизвиквам} / \textit{преживявам възторг} `cause / experience rapture', \textit{възхи\-щавам} / \textit{възхищавам се} `cause admiration' / `admire', \textit{вълнувам} / \textit{вълну\-вам се} `excite' / `experience excitement';

\item[(iv)] Verbs with the possibility of attaching clitics – \textit{домъчнява (ми)} `start to feel unhappy', \textit{причернява (ми)} `start to feel unwell' – \textit{Четенето ми доскуча\-ва} (\textit{Reading makes me bored}) / \textit{Доскучавам на Петьо с въпросите си} (\textit{I'm boring Petyo with my questions});

\item[(v)] Verbs for ambient inner state with obligatory accusative \fename{Experiencer}: \textit{дострашава ме} `start to feel scared', \textit{доядява ме} `start to feel angry'.

\item[(vi)] Verbs for inner psychological state with obligatory dative \fename{Experiencer}, which allow for a second indirect object \fename{Stimulus} / \fename{Theme}: \textit{дожалява ми} `start to feel pity', \textit{докривява ми} `start to feel sad';

\item[(vii)] Reflexiva tantum verbs, in which the verb action is directed back to its subject \fename{Experiencer}: \textit{любувам се} `revel', \textit{срамувам се} `feel ashamed', \textit{страху\-вам се} `fear';
\end{itemize}

The author also points out that the different meanings of the verbs in his classification can be categorised into different groups.

In \sectref{ch6:sec:5} we analyse five of these subclasses of verbs, taking into account their frequency and distribution in Bulgarian. First, we analyse the top hypernyms of the general emotion verbs \{\textit{изпитвам, чувствам}\} / \{\textit{feel, experience}\}. Then we analyse the general transitive verbs that express an emotion and an attitude, such as \textit{обичам} `love', \textit{мразя} `hate', \textit{харесвам} `like'. We also deal with causatives \textit{веселя} `rejoice', \textit{радвам} `gladden', \textit{плаша} `scare', which are subdivided on the basis of the entity that evokes the emotion -- either an \fename{Agent} or a \fename{Stimulus}. And finally, we introduce the stative and inchoative verbs, which are the causative verbs’ middle-voice counterparts formally expressed with the verb and the reflexive \textit{се} -- \textit{веселя се} `rejoice (oneself)', \textit{радвам се} `gladden (oneself)', \textit{плаша се} `scare (oneself)'. We used the WordNet definitions where it was necessary to distinguish between different senses.



\section{Frames and semantic features}\label{ch6:sec:5}

The \fename{Experiencer} and the \fename{Stimulus} are the two obligatory participants in an emotion event. The \fename{Experiencer} is at the centre of much research and is known to necessarily involve a sentient participant -- usually a human or an animate being. Much less attention has been paid to the behaviour and syntactic expression of the \fename{Stimulus}, as its multifaceted nature can hardly be specified. For the predicates under consideration in our work the \fename{Stimulus}  affects the \fename{Experiencer}, changing the emotions he or she experiences. This general scenario determines the emotion verbs and specifies both the possible syntactic structures within a sentence and the morphological and semantic restrictions imposed on the situation participants. 

In the present study, we will look at some of the most common verbs of emotion. The study is based on their semantic frame representation, which builds on the FrameNet and WordNet structures. For this reason, the different subclasses of emotion verbs will presented below with brief definitions taken from FrameNet when appropriate and slightly modified when not.
 
We will outline first the core frame elements within the \framename{Emotions} frame as most of the semantic frames under study inherit them by virtue of the relations between frames. The definition of the \framename{Emotions} non-lexical frame is that ``An \fename{Experiencer} has a particular emotional \fename{State}, which may be described in terms of a specific \fename{Stimulus} that provokes it, or a \fename{Topic} which categorises the kind of \fename{Stimulus}. Rather than expressing the \fename{Experiencer} directly, it may (metonymically) have in its place a particular \fename{Event} (with participants who are \fename{Experiencers} of the emotion) or an \fename{Expressor} (a body-part of gesture which would give an indication of the \fename{Experiencer}'s state to an external observer)''. Both the core and the non-core Emotion frame elements are presented in \tabref{tab:emotionfe}.
 

We will consider five main semantic frames that demonstrate the syntactic specificity of five subclasses of emotion verbs. First, we will analyse the top hy\-pernyms of the emotion verbs  {\textit{изпитвам}:1, \textit{чувствам}:1} ({\textit{feel}:8, \textit{experience}:3}), which are represented by the \framename{Feelings} frame  (\sectref{ch6:sec:feeling}). In \sectref{ch6:sec:expfocem} we will deal with the \framename{Experienced\_focused\_emotion} frame, which is comprised of transitive verbs, expressing attitude. Thirdly, we will explore the \framename{Cause\_to\_experience} and \framename{Stimulate\_emotion} frames, which represent causative verbs of emotion, having  an \fename{Agent} or a \fename{Stimulus} as a subject (\sectref{ch6:sec:stemcaex}). And finally, in \sectref{ch6:sec:emdir}, we will examine the \framename{Emotion\_directed} frame, which represents the stative and inchoative verbs, formed from the causatives from \sectref{ch6:sec:stemcaex} and the reflexive \textit{се}.

\begin{table}
\begin{subtable}{\textwidth}
\caption{Core frame elements}
\begin{tabularx}{\textwidth}{lQ}
\lsptoprule
 \fename{Event} & The occasion or happening in which \fename{Experiencers} in a certain emotional state participate. \\
 \fename{Experiencer}  & The person or sentient entity who experiences or feels the emotion. \\ 
 \fename{Expressor} & It marks expressions that indicate body part, gesture or other expression of the \fename{Experiencer}  that reflects his or her emotional state. \\
 \fename{State} & The abstract noun that describes a more lasting experience by the \fename{Experiencer}. \\
 \fename{Stimulus}  & The person, event, or state of affairs that evokes the emotional response in the \fename{Experiencer}. \\ 
 \fename{Topic} & The general area in which the emotion occurs. It indicates a range of possible \fename{Stimulus}. \\ \lspbottomrule
\end{tabularx}
\end{subtable}\medskip\\
\begin{subtable}{\textwidth}
\caption{Non-core frame elements}
\begin{tabularx}{\textwidth}{lQ}
\lsptoprule
 \fename{Degree} & The extent to which the \fename{Experiencer}'s emotion deviates from the norm for the emotion. \\
 \fename{Empathy-target} & The individual or individuals with which the \fename{Experiencer}  identifies emotionally and thus shares their emotional response. \\
 \fename{Reason}\slash \fename{Explanation} & The \fename{Explanation} is the explanation for why the \fename{Stimulus}  evokes a certain emotional response. \\
 \fename{Manner} & Any description of the way in which the \fename{Experiencer}  experiences the \fename{Stimulus}  which is not covered by more specific FEs. Manner may also describe a state of the \fename{Experiencer}  that affects the details of the emotional experience. \\
 \fename{Parameter} &  A domain in which the \fename{Experiencer}  experiences the \fename{Stimulus}. \\
\lspbottomrule
\end{tabularx}
\end{subtable}
\caption{The \framename{Emotion} frame elements.}
\label{tab:emotionfe}
\end{table}

\subsection{\framename{Feeling}}\label{ch6:sec:feeling}

\begin{description}[font=\normalfont]
\item[{Definition}:] In this frame an \fename{Experiencer} experiences an \fename{Emotion} or is in an \fename{Emotional\_state}. There can also be an \fename{Evaluation} of the internal experiential state.
\end{description}
The verbs \textit{чувствам} `feel'  and \textit{изпитвам} `experience' typically evoke this semantic frame. The only synset that contains those verbs and is marked by the semantic prime \textit{verb.emotion} is presented in Example \ref{ch6:ex3} and is illustrated in Example \ref{ch6:ex:04}:

\begin{exe}  
\ex  \label{ch6:ex3}
\begin{xlist}
\ex %a. 
BG \{\textit{изпитвам}; \textit{изпитам}; \textit{чувствам}; \textit{почувствам}; \textit{преживея}; \textit{прежи\-вявам}; \textit{осезавам}\} (`изживявам емоционално състояние или афект (по отношение на някого или нещо)’)
\ex %b. 
EN \{\textit{feel}; \textit{experience}\} (`undergo an emotional sensation or be in a particular state of mind’)
\end{xlist}
\end{exe}

\begin{exe}
\ex   \label{ch6:ex:04} 
\gll \textit{Започна-ли} \textit{да} \textit{ЧУВСТВА-Т} \textit{любов} \textit{към} \textit{венерианк-и-те} \textit{и} \textit{същ-а-та} \textit{загриженост}, \textit{както} \textit{и} \textit{към} \textit{себе си}. \\
Start-3.PL.PST.INFR to FEEL-3.PL.PRS love towards Venusian-PL-DEF.PL and same-F-DEF.F concern as also towards oneself \\
\glt `\textit{They started to feel love towards the Venusian women and the same concern they felt towards themselves}.' 

\end{exe}

 

The core frame elements are \fename{Emotion}, \fename{Emotional\_state}, \fename{Evaluation}, and \fename{Experiencer}. The observations on Bulgarian material show that the transitive verbs (\textit{чувствам} `feel', \textit{изпитвам} `experience' only take an \fename{Experiencer} and \fename{Emotion} FEs, whereas intransitive ones (\textit{чувствам се} `feel (oneself)') encode an \fename{Experiencer} and an \fename{Emotional\_state} or \fename{Evaluation} (in rare cases). 
 

\fename{\textbf{Emotion}} -- the \fename{Emotion} is the feeling that the \fename{Experiencer} experiences. There are a lot more examples with \textit{изпитвам} `experience' than with \textit{чувствам} `feel' with a direct object position filled with a hyponym of the \{emotion\} synset (Examples \ref{ch6:ex:05} and \ref{ch6:ex:06}). 

\begin{exe} 
\ex  \label{ch6:ex:05} 
  %  \settowidth \jamwidth{(bg)}  
\gll \textit{Често} \textit{\textbf{ИЗПИТВА-М}} [\textit{завист}]$_{\feinsub{Emot}}$ \textit{към} \textit{човешк-и-те} \textit{съществ-а}. \\
often feel-1.SG.PRS envy to human-PL-DEF.PL being-PL \\  %\jambox{(bg)}
\glt `I often feel envious of human beings.' 
\end{exe}

\begin{exe} 
\ex  \label{ch6:ex:06} 
   % \settowidth \jamwidth{(bg)}  
\gll \textit{Елейн} \textit{изобщо} \textit{не} \textit{\textbf{ИЗПИТВА-ШЕ}} [\textit{гордост}]$_{\feinsub{Emot}}$. \\ 
Eleyn {at all} not {feel-3.SG.IPFV} pride \\  %\jambox{(bg)}
\glt `Elaine felt no pride at all.' 
\end{exe}

\fename{\textbf{Experiencer}} -- the \fename{Experiencer} experiences the \fename{Emotion} or is in the \fename{Emotional\_state}. The position of the \fename{Experiencer} is generally occupied by a literal belonging to the eng-30-00007846-n: \{\textit{person}\} synset or its hyponyms. It is expressed in a sentence by an NP, which functions as a subject. It can also be used metaphorically with a part of the body, usually the \textit{heart}, which has the potential to serve as an expressor of one’s feelings (Example \ref{ch6:ex:07}).

\begin{exe} 
\ex  \label{ch6:ex:07} 
  %  \settowidth \jamwidth{(bg)} 
\gll [...] \textit{и} \textit{да} \emph{каж-е} \textit{онова}, \textit{кое-то} \textit{\textbf{ЧУВСТВА-Ø}}  [\textit{сърце-то} \textit{ѝ}]$_{\feinsub{Body\_part}}$.\\
{}[...] and to tell-3.SG.PRS that-N which-DEF.N feel-3.SG.PRS heart-DEF.N her \\ % \jambox{(bg)}
\glt `[...] and to say what her heart feels.' 
\end{exe}

The \textit{se} counterpart of \textit{чувствам} `feel' – \textit{чувствам се} `feel (oneself)' is reflexive concerning its form and, respectively, intransitive. It does not take a direct object and, therefore, does not encode an \fename{Emotion}. In order to realise its meaning, it needs the other core frame element -- \fename{Emotional\_state}.


\fename{\textbf{Emotional\_state}} -- the \fename{Emotional\_state} is the state the \fename{Experiencer} is in. The \fename{Emotional\_state} can be expressed by an adjective/participle as in Examples \ref{ch6:ex:08} and \ref{ch6:ex:09} describing the \fename{Experiencer} or by an adverb (Example \ref{ch6:ex:10}), indicating the manner in which the \fename{Experiencer} feels.

\begin{exe} 
\ex  \label{ch6:ex:08} 
%    \settowidth \jamwidth{(bg)} 
\gll \textit{\textbf{ЧУВСТВА-М СЕ}}  [\textit{свободен-Ø}]$_{\feinsub{Emos}}$. \\ 
feel-1.SG.PRS  free-SG.M \\ % \jambox{(bg)}
\glt `I feel free.' 
\end{exe}

\begin{exe} 
\ex  \label{ch6:ex:09} 
    \settowidth \jamwidth{(bg)} 
\gll \textit{В} \textit{момент-а} \textit{тя} \textit{\textbf{СЕ ЧУВСТВА-ШЕ}}  [\textit{предад-ен-а}]$_{\feinsub{Emos}}$.   \\ 
in {moment-DEF.M} she feel-3.SG.IPFV betray-PTCP-F \\ % \jambox{(bg)}
\glt `Right now she felt betrayed.' 
\end{exe}

\begin{exe} 
\ex  \label{ch6:ex:10} 
 %   \settowidth \jamwidth{(bg)} 
\gll \textit{Дали} \textit{ще} \textit{\textbf{СЕ ЧУВСТВА-Ш}}  [\textit{отвратително}]$_{\feinsub{Emos}}$, \textit{може}.   \\ 
whether will feel-2.SG.PRS disgustingly, maybe \\ % \jambox{(bg)}
\glt `Will you feel disgusted, maybe.' 
\end{exe}

\subsection{\framename{Experiencer\_focused\_emotion}}\label{ch6:sec:expfocem}

\begin{description}[font=\normalfont]
\item[Definition:] The words in this frame describe an \fename{Experiencer}'s emotions with respect to some \fename{Content}. Although the \fename{Content} may refer to an actual, current state of affairs, quite often it refers to a general situation which causes the emotion.
\end{description}

\framename{Experiencer\_focused\_emotion} is a semantic frame that encodes the \fename{Experiencer} as a subject and the \fename{Content} as a direct object and is well-represented in Bulgarian. This semantic frame encompasses verbs like \textit{харесвам} `like', \textit{обичам} `love', \textit{мразя} `hate', \textit{ненавиждам} `detest', \textit{обожавам} `adore', \textit{съжалявам} `feel sorry', \textit{презирам} `despise', among others. As \citet[117]{tisheva2021наблюдения} specifies ``unlike the usage of some mental predicates (\cite[264]{ницолова2008проблематика}), with the verb \textit{обичам} `love' the negation does not affect the choice of lexical elements that can occupy the syntactic positions, but only the interpretation of the meaning of the whole sentence''. This observation can be spread over the verbs comprising this subclass with the exception of \textit{ненавиждам} `detest', which bears negation within its structure and does not allow for a second negative element. 
 


The core frame elements within the \framename{Experiencer\_focused\_emotion} in FrameNet are the \fename{Experiencer}, the \fename{Content}, the \fename{Event} and the \fename{Topic}. For the sake of the description of Bulgarian verbs we will use a modified frame, taking into consideration only the \fename{Experiencer} and the \fename{Content} as the other two core frame elements are generally combined with other parts of speech. The \fename{Event} is generally expressed by noun phrases and the \fename{Topic}, which gives additional information, was not found in the Bulgarian corpus examples with the verbs under discussion. That is why the latter two core FE will not be discussed here. 
 

We will consider firstly the two main meanings of the verb \textit{обичам} `love' as reflected in the BulNet lexical-semantic resource (Examples \ref{ch6:ex:11} and \ref{ch6:ex:12}). They both encode the \fename{Content} as a direct object or as a \textit{да}-clause. We believe that the other verbs from the group follow the same syntactic constructions.

\begin{exe} 
\ex  \label{ch6:ex:11} 
\begin{xlist}
\ex %a. 
BG \{\textit{обичам}\} (`изпитвам силна привързаност и симпатии към ня\-кого или свързаност с и удоволствие от нещо')\\
\textit{Тя обича шефа си и работи усърдно за него}; \textit{Обичам френската кухня}.
\ex %b. 
EN \{\textit{love}\} (`have a great affection or liking for’)\\
\textit{She loves her boss and works hard for him}; \textit{I love French cuisine}.
\end{xlist}
\end{exe}\textit{}

\begin{exe} 
\ex  \label{ch6:ex:12} 
\begin{xlist}
\ex %a. 
BG \{\textit{обичам}\} (`харесвам много, изпитвам удоволствие от нещо’)\\ \textit{Обичам да готвя}. 
\ex %b. 
EN \{\textit{love}\} (`get pleasure from’)\\
\textit{I love cooking}.
\end{xlist}
\end{exe}

\fename{\textbf{Experiencer}} -- The \fename{Experiencer} experiences the emotion or other internal state. The \fename{Experiencer} FE position is generally filled with a subtree of \{{\textit{person}}\} (Example \ref{ch6:ex:13}), but can also be encoded as \{{\textit{animal}}\} and its hyponyms (Example \ref{ch6:ex:14}). 

\begin{exe} 
\ex  \label{ch6:ex:13} 
  %  \settowidth \jamwidth{(bg)} 
\gll [\textit{Победител-ят}]$_{\feinsub{Exp}}$ \textit{\textbf{ОБИЧА-ШЕ}}  \textit{да} \textit{вкарва} \textit{гол-ове}.   \\ 
winner-DEF.M love-3.SG.IPFV to score goal-PL \\  %\jambox{(bg)}
\glt `The winner loved to score goals.' 
\end{exe}

\begin{exe}  
\ex  \label{ch6:ex:14}
 %   \settowidth \jamwidth{(bg)} 
\gll [\textit{Куче-то}]$_{\feinsub{Exp}}$ \textit{\textbf{ОБИЧА-Ø}}  \textit{стариц-и-те}.   \\ 
dog-DEF.N love-3.SG.PRS {old woman-PL-DEF.PL} \\ 
%\jambox{(bg)}
\glt `The dog loves old women.' 
\end{exe}

The subject \fename{Experiencer} is usually expressed by a singular noun, as shown in Examples \ref{ch6:ex:13} and \ref{ch6:ex:14}. If the subject \fename{Experiencer} is in the plural, it denotes a specific group that acts as the collective subject of the emotion, as shown in the Example \ref{ch6:ex:15}. 

\begin{exe} 
\ex  \label{ch6:ex:15} 
 %   \settowidth \jamwidth{(bg)} 
\gll [\textit{Магьосниц-и-те}]$_{\feinsub{Exp}}$ \textit{\textbf{ОБОЖАВА-Т}}  \textit{неразкри-ти-те} \textit{тайн-и}.   \\ 
magician-PL-DEF.PL adore-3.PL.PRS undiscover-PTCP.PL-DEF.PL secret-PL \\ % \jambox{(bg)}
\glt `Magicians adore undiscovered secrets.' 
\end{exe}

In addition, the \fename{Experiencer}'s position is often occupied by \{\textit{душа}\} `soul' or \{\textit{сърце}\} `heart'  synsets together with expressive modifiers or quantifiers to reveal the point where the feeling is concentrated (Example \ref{ch6:ex:16}). 

\begin{exe}
\ex   \label{ch6:ex:16} 
%    \settowidth \jamwidth{(bg)} 
\gll [\textit{Душа-та} \textit{ми}]$_{\feinsub{Exp}}$ {\textit{до болка}} \textit{те} \textit{\textbf{ОБИЧА-Ø}}.   \\ 
soul-DEF.F my {to pain} {you-ACC} {love-3.SG.PRS}
\\ % \jambox{(bg)}
\glt `My soul loves you painfully.' 
\end{exe}

Examples with metonymic shifts of the type \textit{Barcelona loves partying} (meaning the people of Barcelona) actually show that a great number of lexical units can possibly occupy a certain position if they can express the same semantic role. Our aim is to outline the syntactic regularities and that is why such occasional examples lie beyond the scope of this study. 



\textbf{\fename{Content}} -- \fename{Content} is what the \fename{Experiencer}'s feelings or experiences are directed towards or based upon. The \fename{Content} differs from a \fename{Stimulus}  because the \fename{Content} is not construed as being directly responsible for causing the emotion. The \fename{Content} FE is commonly expressed by a noun of the \{\textit{person}\} or \{\textit{animal}\} subtrees, as is shown in Example \ref{ch6:ex:17}, but the position of this FE can generally be occupied by any \{\textit{entity}\} hyponym, alluding to a specific human being (e.g. one's voice, as shown in Examples \ref{ch6:ex:18} and \ref{ch6:ex:19}). 


\begin{exe} 
\ex  \label{ch6:ex:17} 
%    \settowidth \jamwidth{(bg)} 
\gll \textit{Искрено} {\textit{да \textbf{ОБИЧА-Ø}}} [\textit{родител-я}]$_{\feinsub{Cont}}$   [...]  \\ 
{sincerely} {to love-3.SG.PRS} {parent-DEF.M} [...]
\\ % \jambox{(bg)}
\glt `To sincerely love the parent [...]'

\ex   \label{ch6:ex:18} 
 %   \settowidth \jamwidth{(bg)} 
\gll  {\textit{\textbf{ОБИЧА-М}}} [\textit{звук-а} \textit{на} \textit{глас-а} \textit{ти}]$_{\feinsub{Cont}}$!  \\ 
{love-1.SG.PRS} {sound-DEF.M} {of} {voice-DEF.M} {your}
\\ % \jambox{(bg)}
\glt `I love the sound of your voice!'
  
\ex   \label{ch6:ex:19} 
%    \settowidth \jamwidth{(bg)} 
\gll \textit{Библиотекар-ят} {\textit{\textbf{ОБИЧА-ШЕ}}} [\textit{театър-а}]$_{\feinsub{Cont}}$.  \\ 
{librarian-DEF.M} {love-3.SG.IPFV} theatre-DEF.M
\\ % \jambox{(bg)}
\glt `The librarian loved theatre.'
\end{exe}

There are examples in the corpus where the \fename{Content} is also conveyed metaphorically as in Example \ref{ch6:ex:20}, where the glass actually symbolises the \fename{Experiencer}’s attitude towards drinking.


\begin{exe} 
\ex  \label{ch6:ex:20} 
 %   \settowidth \jamwidth{(bg)} 
\gll \textit{Чичо Тошко} {\textit{\textbf{ОБИЧА-ШЕ}}} [\textit{чашка-та}]$_{\feinsub{Cont}}$ [...] \\ 
{uncle Toshko} {love-3.SG.IPFV} {glass-DEF.F}  [...]
\\ % \jambox{(bg)}
\glt `Uncle Toshko loved to drink [...]'
\end{exe}

The \fename{Content} of the emotion can also be expressed in Bulgarian by a subordinate clause, introduced by the conjunction \textit{да} (Example \ref{ch6:ex:21}), interrogative pronoun \textit{как} (Example \ref{ch6:ex:22}) and relative pronoun \textit{когато} (Example \ref{ch6:ex:23}). When the object position is filled with a clause, there is no structural dependency between the arguments of the predicates in the main and the subordinate clause. 

\begin{exe} 
\ex  \label{ch6:ex:21} 
%    \settowidth \jamwidth{(bg)} 
\gll \textit{Повече} \textit{ѝ} \textit{\textbf{ХАРЕСВА-ШЕ}} [\textit{да прочете-Ø}]$_{\feinsub{Cont}}$ [...] \\ 
{More} {she-DAT} {LIKE-3.SG.IPFV} {to read-3.SG.PRS} [...]
\\ % \jambox{(bg)}
\glt `She liked more to read [...]'

\ex   \label{ch6:ex:22} 
 %   \settowidth \jamwidth{(bg)} 
\gll  \textit{И} \textit{\textbf{МРАЗЕ-ШЕ}} [\textit{как} \textit{я}
\textit{гледа-ш}]$_{\feinsub{Cont}}$ [...] \\ 
{and}  {hate-3.SG.IPFV} {how} {she-ACC} {look\_at}-2.SG.PRS [...]
\\  %\jambox{(bg)}
\glt `And she hated how you looked at her [...]'
 
\ex   \label{ch6:ex:23} 
 %   \settowidth \jamwidth{(bg)} 
\gll  \textit{Хора-та} \textit{\textbf{ОБИЧА-Т}} [\textit{когато} \textit{някой} \textit{се нужда-е} \textit{от} \textit{свобода}]$_{\feinsub{Cont}}$ [...] \\ 
{person-PL.DEF} {love-3.PL.PRS} when somebody need-3.SG.PRS of freedom  [...]
\\ % \jambox{(bg)}
\glt `People like it when someone needs freedom [...]'
\end{exe} 

This usage should be distinguished from the one where \textit{когато}-clause is used for conflicting circumstances as in Example \ref{ch6:ex:24}. The \fename{Content} position in this sentence is filled with a direct object accusative pronoun \textit{ме} `me'.

\begin{exe} 
\ex  \label{ch6:ex:24} 
 %   \settowidth \jamwidth{(bg)} 
\gll [...] \textit{как} {\textit{може-ш}} {да} [{\textit{ме}}]$_{\feinsub{Cont}}$ \textit{\textbf{ОБИЧА-Ш}}, \textit{когато} \textit{едва} \textit{снощи} \textit{се срещнахме}  [...] \\ 
{}[...] {how} {can-2.SG.PRS} {to} {I-ACC} {love-2.SG.PRS}, {when} only last\_night {REFL met-1.PL.PST} [...]
\\ % \jambox{(bg)}
\glt `[...] how can you love me when we met only last night [...]'
\end{exe} 

English verbs show similar usage to Example \ref{ch6:ex:23} when projecting the non-core frame element \fename{Circumstances} with the help of a finite  \textit{wh}-complement, which is typically preceded by a pronominal object (Example \ref{ch6:ex:25}).


\begin{exe}  
\ex  \label{ch6:ex:25}
\textit{I \textbf{HATE} it when you do that}.
\end{exe}

An adjunct \textit{когато}-clause is also used in Example \ref{ch6:ex:26}, as \textit{обичам} `love' takes a complement \textit{да}-clause,  which occupies the position of the \fename{Content}. The proximity or remoteness of a phrase/clause and the verb does not affect the logical structure of the sentence.

\begin{exe} 
\ex   \label{ch6:ex:26}
%    \settowidth \jamwidth{(bg)} 
\gll \textit{Не} {\textit{\textbf{ОБИЧА-М}}}, {\textit{когато}} {\textit{ѝ}}  {\textit{говор-я}} {\textit{колко}} {\textit{много}} {\textit{я}} {\textit{обича-м}}, [\textit{тя да мълчи-Ø}]$_{\feinsub{Cont}}$. \\ 
{not} {love-1.SG.PRS} {when} {she-DAT} {speak-1.SG.PRS} {how} {much} {she-ACC} {love-1.SG.PRS}, {she to be silent-3.SG.PRS}
\\% \jambox{(bg)}
\glt `I don't like her keeping silent when I tell her how much I love her.'
\end{exe} 

Finally, we are going to examine a more specific sense of the verb \textit{обичам} `love' as in Example \ref{ch6:ex:27}, as it demonstrates high frequency of usage. 

\begin{exe} 
\ex   \label{ch6:ex:27}
\begin{xlist}
\ex  \label{ch6:ex:27a} %a. 
BG \{\textit{обичам}\} (`влюбен съм, изпитвам любов към някого’) \\
\textit{Тя искрено обичаше съпруга си}. 
\ex  \label{ch6:ex:27b}%b. 
EN \{\textit{love}\} (`be enamored or in love with’)\\
\textit{She loves her husband deeply}.
\end{xlist}
\end{exe}

In this particular meaning of the verb, the positions of the\fename{Experiencer} and \fename{Content} are semantically restricted to the synset \{\textit{person}:1\} and its hyponyms. Furthermore, as \citet [124]{tisheva2021наблюдения} states, they should reflect a single individual, so that both FEs should be expressed by singular nouns. If expressed with a plural form, the FEs consider a collective image of a specific group. The use of the definite form in singular or plural usually indicates a generic use. The Example \ref{ch6:ex:28} comes from the work of Tisheva.
 
\begin{exe} 
\ex   \label{ch6:ex:28}
  %  \settowidth \jamwidth{(bg)} 
\gll \textit{Майка-та} {\textit{\textbf{ОБИЧА-Ø}}} {\textit{дец-а-та}} {\textit{си}}.\\ 
{mother-DEF.F} {love-3.SG.PRS} {child-PL-DEF.PL} {REFL-POSS}
\\  %\jambox{(bg)}
\glt `A mother loves her children.'
\end{exe}


\subsection{\framename{Stimulate\_emotion} and \framename{Cause\_to\_experience}}\label{ch6:sec:stemcaex}

We will consider these two semantic frames and the lexical units that evoke them in parallel, since they show great similarities in terms of sentence structure and situation participants and differ only with respect to one of the frame elements. Both frames denote two core frame elements that are expressed conventionally.

\framename{Stimulate\_emotion}’s definition is ``Some phenomenon (the \fename{Stimulus}) provokes a particular emotion in an \fename{Experiencer}.'' Its core frame elements are an \fename{Experiencer} and a \fename{Stimulus}, defined as follows:

\begin{description}[font=\normalfont]
\item[\textbf{\fename{Experiencer}}:] the \fename{Experiencer} reacts emotionally or psychologically to the \fename{Stimulus}.
\item[\textbf{\fename{Stimulus}}:] the \fename{Stimulus} is the event or entity which brings about the emotional or psychological state of the \fename{Experiencer}.
\end{description}

Within the \framename{Cause\_to\_experience} frame an \fename{Experiencer} and an \fename{Agent} can be pointed out as core frame elements and the definition of the frame is ``An \fename{Agent} intentionally seeks to bring about an internal mental or emotional state in the \fename{Experiencer}''. 

\begin{description}[font=\normalfont]
\item[\textbf{\fename{Agent}}:] the \fename{Agent} is an external argument of the target word and purposefully arouses an emotional state.
\item[\textbf{\fename{Experiencer}}:] the \fename{Experiencer} is the person the \fename{Agent} causes to have a particular emotional state.
\end{description}

The semantic and syntactic restrictions of the frame element \fename{Experiencer} are identical for both semantic frames. It is an \textit{animate being} (Example \ref{ch6:ex:29}), but most of the time the position is represented by a \{\textit{person}\} NP.
\begin{exe} 
\ex  \label{ch6:ex:29} 
 %   \settowidth \jamwidth{(bg)} 
\gll \textit{Изведнъж} {\textit{\textbf{СТРЯСКА-МЕ}}} [\textit{заек}]$_{\feinsub{Exp}}$ [...] \\ 
{suddenly} {startle-1.PL.PRS} {rabbit} [...]
\\%  \jambox{(bg)}
\glt `Suddenly, we startle a rabbit [...]'
\end{exe}

An interesting case are the examples with the explicit presence of the \fename{Stimulus} of the emotion and an unexpressed \fename{Experiencer} (Example \ref{ch6:ex:30}). In her study of the predicative construction \textit{it is known}, \citet [175]{ницолова2001значение} notes that ``the place of \fename{Experiencer} in the semantic structure is actually occupied by a variety of epistemic subjects. The set of epistemic subjects includes at least the speaker himself, who also wants to include the hearer''. This observation can also be applied to the unexpressed \fename{Experiencer} of the causative predicates of emotion: the object is present in the semantic structure of the predicate and represents a plurality of individuals.

\begin{exe} 
\ex  \label{ch6:ex:30} 
  %  \settowidth \jamwidth{(bg)} 
\gll [\textit{Москва}]$_{\feinsub{Age}}$  {\textit{\textbf{ПЛАШИ-Ø}}}, \textit{че} \textit{ще} \textit{разкрие-Ø} \textit{истина-та}  [...] \\ 
{Moscow} {threaten-3.SG.PRS} {that} {will} {reveal-3.SG.PRS} {truth-DEF.F} [...]
\\% \jambox{(bg)}
\glt `Moscow threatens to reveal the truth [...]'
\end{exe}

Both semantic frames can be evoked by verbs such as \textit{ужасявам} `terrify', \textit{пла\-ша} `scare', \textit{разстройвам} `upset', \textit{веселя} `rejoice', \textit{радвам} `gladden', \textit{успокоявам} `comfort', \textit{вълнувам} `excite', \textit{забавлявам} `entertain', \textit{стряскам} `startle', which are causative and, correspondingly, transitive. The position of the direct object is taken by the \fename{Experiencer}, while the subject can be either animate or inanimate. If the source of the emotion is animate, it receives an agent-like interpretation and refers to the frame \framename{Cause\_to\_experience}; however, if it is inanimate, it is projected as the \fename{Stimulus} of the emotion and belongs to the frame \framename{Stimulate\_emotion}.

Within the frame \framename{Cause\_to experience} the \fename{Agent} can only be presented with the synset {\textit{person}} or its hyponyms. The frame \framename{Stimulate\_emotion} can encode all \{\textit{entity}\} hyponyms in its subject position, with the exception of \textit{person}. In addition, the \fename{Stimulus} can also be encoded as a clause. As \citet [18]{коева2021към} notes, the complementiser in Bulgarian is represented by the conjunction or conjunction-like words such as \textit{че}, \textit{да}, \textit{как} and \textit{дето}. This applies to predicates of emotion whose complement clauses representing the \fename{Stimulus} are generally introduced by one of these complementisers.
Some of the verbs allow all types of clauses, while some verbs in the corpus show no use with some of them. \tabref{tab:distributioncausative} shows the distribution of possible conjunctions with the predicates as represented in the Bulgarian National Corpus.\footnote{We have documented the results of the corpus-based search, although the values do not always match our linguistic intuition.}

\begin{table}
    \begin{tabular}{l *4{c}} 
    \lsptoprule
         verbs& \textit{че}~`\textit{that}' & \textit{да}~`\textit{to}'&  \textit{как}~`\textit{how}' & \textit{дето}~`\textit{as/for/that}'{\footnote{According to the Dictionary of Bulgarian Language \textit{дето} is a conjunction formed by an adverb or a relative pronoun. It has a variety of functions in a sentence, that is why more than one possible translation is presented in the table.}}\\ 
         \midrule
        \textit{ужасявам} `\textit{terrify}'&  +&  +&  −& −\\
        \textit{плаша} `\textit{scare}'&  +&  +&  −& +\\
        \textit{разстройвам} `\textit{upset}'&  −&  −&  −& −\\
        \textit{веселя} `\textit{rejoice}'&  −&  −&  −& −\\
        \textit{радвам} `\textit{gladden}'&  +&  +&  +& +\\
        \textit{успокоявам} `\textit{comfort}'&  +&  +&  −& −\\
        \textit{вълнувам} `\textit{excite}'&  +&  −&  +& −\\
        \textit{забавлявам} `\textit{entertain}'&  +&  +&  −& −\\
        \textit{стряскам} `\textit{startle}'&  −&  −&  −& −\\
    \lspbottomrule
    \end{tabular}
    \caption{The distribution of causative verbs and possible complementisers.}
    \label{tab:distributioncausative}
\end{table}

The results from the corpus search show that \textit{веселя} `rejoice', \textit{разстройвам} `upset' and \textit{стряскам} `startle' can have only a NP in the subject position. \tabref{tab:distributioncausative} shows that \textit{радвам} `gladden' is the only verb that allows for all four conjunctions. \textit{Ужасявам} `terrify', \textit{успокоявам} `comfort' and \textit{забавлявам} `entertain' can have \textit{че}- and \textit{да}-clauses (Example \ref{ch6:ex:31}), but do not show usage with the other two complementisers. \textit{Плаша} `scare' can be used with \textit{че}-, \textit{да}- and \textit{детo}-constructions in the subject position, and \textit{вълнувам} `excite' – with \textit{чe} and \textit{как} complementisers (Example \ref{ch6:ex:32}).


\begin{exe}  
\ex  \label{ch6:ex:31}
  %  \settowidth \jamwidth{(bg)}
\gll \textit{Винаги} {\textit{го}} {\textit{\textbf{ЗАБАВЛЯВА-ШЕ}}}, [{\textit{че}} {\textit{араб-и-те}} {\textit{им}} {\textit{вярва-ха}}]$_{\feinsub{Stim}}$. \\ 
{always}  {he-ACC} {entertain-3.SG.IPFV} {that} {Arab-PL-DEF.PL} {they-DAT} {believe-3.PL.IPFV}
\\ % \jambox{(bg)}
\glt `It always entertained him that the Arabs believed them.'
  
\ex  \label{ch6:ex:32}
%    \settowidth \jamwidth{(bg)} 
\gll \textit{Изобщо} {\textit{не}} {\textit{ме}} {\textit{\textbf{ВЪЛНУВА-Ø}}} [{\textit{как}} {\textit{изглежда-Ø}}]$_{\feinsub{Stim}}$! \\
{at all} {not}  {I-ACC} {CARE-3.SG.PRS} {how} {look like-3.PL.PRS}
\\ % \jambox{(bg)}
\glt `I don't care at all what it looks like!'
\end{exe}

The \fename{Stimulus}  clause can also be introduced with the intensifying modifier \textit{колко} `how much / many' as in Example \ref{ch6:ex:33}. 

\begin{exe} 
\ex  \label{ch6:ex:33} 
 %   \settowidth \jamwidth{(bg)} 
\gll \textit{\textbf{УЖАСЯВА-Ø}} {\textit{ме}}  [{\textit{колко}} {\textit{е}} {\textit{сериозен-Ø}}]$_{\feinsub{Stim}}$. \\ 
{terrify-3.SG.PRS} {I-ACC} {how much} {be-3.SG.PRS} {serious-M.SG}
\\  %\jambox{(bg)}
\glt `It terrifies me how serious he is.'
\end{exe}

In addition to the subject clauses (Examples \ref{ch6:ex:31} and \ref{ch6:ex:33}), the frame element \fename{Stimulus} can also be introduced by a \textit{c}-PP (Examples \ref{ch6:ex:34}). In this case, the subordinate clause applies to the PP and not to the verb.

\begin{exe} 
\ex  \label{ch6:ex:34} 
  %  \settowidth \jamwidth{(bg)} 
\gll \textit{Неведнъж} \textit{я} {\textit{беше \textbf{СТРЯСКА-Л}}} [{\textit{с}} {\textit{това}}, {\textit{което}} {\textit{знае-ше}}]$_{\feinsub{Stim}}$. \\ 
{not\_once} {she-ACC} {startle-3.SG.PLUSQ} {with} {this} {which} {know-3.SG.IPFV}
\\  %\jambox{(bg)}
\glt `Not once had he startled her with what he knew.'
\end{exe}

\subsection{\framename{Emotion\_directed}}\label{ch6:sec:emdir}

The frame \framename{Emotion\_directed} includes stative and inchoative subject\hyp\fename{Experiencer} psych verbs, which are characterised by the reflexive-by-form \textit{се} and a middle-voice use. It comprises of verbs such as \textit{ужасявам се} `feel/become terrified', \textit{плаша се} `fear',\textit{ разстройвам се} `feel/become upset', \textit{веселя се} `rejoice', \textit{радвам се} `be glad', \textit{успокоявам се} `calm down', \textit{вълнувам се} `be excited', \textit{забав\-лявам се} `entertain', \textit{стряскам се} `be startled' and others. We consider the above verbs to be the \textit{се} counterparts of the verbs we analysed in \sectref{ch6:sec:stemcaex}.

 
\begin{description}[font=\normalfont]
\item[Definition:] this frame describes an \fename{Experiencer} who is feeling or experiencing a particular emotional response to a \fename{Stimulus} or about a \fename{Topic}. There can also be \fename{Circumstances} FE under which the response occurs or a \fename{Reason} why the \fename{Stimulus} evokes the particular response in the \fename{Experiencer}.
\end{description}

The core frame elements are \fename{Event}, \fename{Experiencer}, \fename{Expressor}, \fename{Reason}, \fename{State}, \fename{Stimulus}, \fename{Topic}. We will slightly modify this semantic frame for the Bulgarian verbs by excluding the FEs \fename{Event}, \fename{Expressor} and \fename{State}, as we do not describe adjectives or nouns that evoke the semantic frame.

\fename{\textbf{Experiencer}} -- The \fename{Experiencer} is the person or sentient entity that experiences or feels the emotions. 

We found no examples in the corpus with complement clauses in subject position. When a subject is explicitly present in the sentence, the syntactic realisations of \fename{Experiencer} consist mainly of noun phrases of the subtree \textit{person}. There are rare cases with animate non-persons, which in this case belong to the subtree \textit{animate being} (Example \ref{ch6:ex:35}). As we have already mentioned, Bulgarian as a pro-drop language allows the subject position to be empty.

\begin{exe} 
\ex  \label{ch6:ex:35} 
 %   \settowidth \jamwidth{(bg)} 
\gll \textit{Животн-и-те} {\textit{\textbf{СЕ ПЛАШЕ-ХА}}}, {\textit{но}} {\textit{назад}} {\textit{не}} {\textit{може-хме}} {\textit{да}} {\textit{се върне-м}}. \\ 
{animal-PL-DEF.PL} {scare-3.PL.IPFV} {but} {back} {not} {can-1.PL.IPFV} {to} {go back-1.PL.PRS}
\\ % \jambox{(bg)}
\glt `The animals were scared, but we could not go back.'
\end{exe}

Metonymic transfers make it possible for non-animate objects to take the position of the subject, although few cases illustrated that type in the corpus (Examples \ref{ch6:ex:36} and \ref{ch6:ex:37}).

\begin{exe} 
\ex  \label{ch6:ex:36} 
 %   \settowidth \jamwidth{(bg)} 
\gll \textit{Как} \textit{\textbf{СЕ ВЕСЕЛИ-Ø}} [{\textit{град-ът}}]$_{\feinsub{Exp}}$? \\ 
{how} {rejoice-3.SG.PRS} {city-DEF.M}
\\ % \jambox{(bg)}
\glt `How does the city have fun?'
\ex  \label{ch6:ex:37}  
   % \settowidth \jamwidth{(bg)} 
\gll [...] \textit{търсене-то} \textit{беше} \textit{прекрат-ен-о} \textit{и} [{\textit{село-то}}]$_{\feinsub{Exp}}$ \textit{\textbf{СЕ УСПОКОИ-Ø}}. \\ 
{}[...] {search-DEF} {was} {call off-PTCP-N} {and} {village-DEF.N} {calm down-3.SG.PST}
\\ % \jambox{(bg)}
\glt `[...] the search was called off and the village calmed down.'
\end{exe}

\fename{\textbf{Stimulus}} -- The \fename{Stimulus} is the person, event, or state of affairs (excluding \fename{Reason}) that evokes the emotional response in the \fename{Experiencer}. As the last example (Example \ref{ch6:ex:37}) shows, the \fename{Stimulus} of the emotion can be syntactically unexpressed. When this element of the emotional scenario is expressed, it is traditionally projected into a subordinate clause. 

As in \sectref{ch6:sec:stemcaex} we checked all the possible combinations of verbs and complementisers and present them in Table 3.\footnote{We have documented the results from the corpus-based search, although the values are not always consistent with our own intuition
.} The differences from causative verbs’ usage are encircled.

\begin{table}
    \begin{tabular}{ l *4{c} } 
    \lsptoprule
         verbs& \textit{че} &  \textit{да} &  \textit{как} & \textit{дето} \\
              & `\textit{that}' & `\textit{to}' & `\textit{how}' & `\textit{as/for/that}' \\ 
         \midrule
        \textit{ужасявам се} `\textit{feel terrified}'&  +&  +&  ⊕& −\\ 
        \textit{плаша се} `\textit{fear}'&  +&  +&  ⊕& ⊖\\ 
        \textit{разстройвам се} `\textit{feel upset}'&  −&  −&  −& −\\ 
        \textit{веселя се} `\textit{rejoice}'&  ⊕&  −&  −& −\\ 
        \textit{радвам се} `\textit{be glad}'&  +&  +&  +& +\\ 
        \textit{успокоявам се} `\textit{calm down}'&  +&  ⊖&  −& −\\ 
        \textit{вълнувам се} `\textit{be excited}'&  +&  ⊕&  +& −\\ 
        \textit{забавлявам се} `\textit{entertain}'&  +&  +&  −& −\\ 
        \textit{стряскам се} `\textit{be startled}'&  ⊕&  −&  −& −\\ 
      \lspbottomrule
    \end{tabular}
    \caption{The distribution of \textit{се}-verbs and possible complementisers.}
    \label{tab:distributionse}
\end{table}

The verbs \textit{ужасявам се} `feel/become terrified', \textit{плаша се} `fear', \textit{вълнувам се} `be excited' show examples with the first three complement types as marked in \tabref{tab:distributionse} (Example \ref{ch6:ex:38}). \textit{Разстройвам се} `feel/become upset' does not take subordinate clauses of any type. \textit{Веселя се} `rejoice', \textit{успокоявам се} `calm down' and \textit{стряскам се} `be startled' allow clause complements with \textit{че} only (Example \ref{ch6:ex:39}). The compatibility of \textit{радвам се} `be glad' with subordinate conjunctions is significantly wider -- it can be used with all four of them, according to our empirical material (Example \ref{ch6:ex:40}). And finally, \textit{забавлявам се} `entertain' can be used with the two most frequent conjunctions \textit{че} and \textit{да} (Example \ref{ch6:ex:41}), but not with the other two.

\begin{exe} 
\ex  \label{ch6:ex:38} 
  %  \settowidth \jamwidth{(bg)} 
\gll \textit{\textbf{УЖАСЯВА-Ø СЕ}} [{\textit{да}} {\textit{е}} {\textit{далеч}} {\textit{от}} {\textit{теб}}]$_{\feinsub{Stim}}$. \\ 
{be terrified-3.SG.PRS} {to} {be-3.SG.PRS} {far} {from} {you-ACC}
\\ % \jambox{(bg)}
\glt `(He) is terrified of being away from you.'

\ex \label{ch6:ex:39} 
   % \settowidth \jamwidth{(bg)} 
\gll [...] \textit{а} \textit{пи-хме} \textit{и} \textit{слага-хме} \textit{трапез-и} \textit{да} \textit{\textbf{СЕ ВЕСЕЛИ-М}}, [\textit{че} \textit{те} \textit{си отидо-ха}]$_{\feinsub{Stim}}$. \\ 
{}[...] {but} {drink-1.PL.PST} {and} {set-1.PL.PST} {table-PL} {to} {rejoice-1.PL.PRS} {that} {they} {go away-3.PL.PST}
\\  %\jambox{(bg)}
\glt `[...] but drank and set tables to rejoice that they had gone.'

\ex   \label{ch6:ex:40}
  %%%  \settowidth \jamwidth{(bg)} 
\gll [...] \textit{че} \textit{я} \textit{обича-Ø} \textit{много} \textit{и} \textit{\textbf{СЕ РАДВА-Ø}}  [\textit{дето} \textit{всичко} \textit{свърши-Ø}]$_{\feinsub{Stim}}$ [...] \\ 
{}[...] {that} {she-ACC} {love-3.SG.PRS} {much} {and} {be glad-3.SG.PRS} {that} {everything} {end-3.SG.PST} [...]
\\ % \jambox{(bg)}
\glt `[...] that he loves her very much and is glad that everything ended [...]'

\ex  \label{ch6:ex:41}  
 %   \settowidth \jamwidth{(bg)} 
\gll \textit{И} \textit{тя} \textit{\textbf{СЕ ЗАБАВЛЯВА-ШЕ}}  [\textit{да} \textit{ме} \textit{учи-Ø}]$_{\feinsub{Stim}}$. \\ 
{and} {she} {entertain-3.SG.IPFV} {to} {I-ACC} {teach-3.SG.PRS}
\\  %\jambox{(bg)}
\glt `And she had fun teaching me.'
\end{exe}

In the corpus, there are occasional cases in which the complement clause is introduced with the intensifier \textit{колко} `how much / many' (Example \ref{ch6:ex:42}).

\begin{exe}
\ex   \label{ch6:ex:42} 
  %  \settowidth \jamwidth{(bg)} 
\gll [...] \textit{и} \textit{\textbf{СЕ РАДВА-ХА}}  [\textit{колко} \textit{хубав-Ø} \textit{обещава-Ø} \textit{да бъде} \textit{ден-ят}]$_{\feinsub{Stim}}$. \\ 
{}[...] {and} {rejoice-3.PL.IPFV} {how\_much} {fine-M.SG} {promise-3.SG.PRS} {to be-3.SG.PRS} {day-DEF.M}
\\ % \jambox{(bg)}
\glt `[...] and rejoiced at how fine the day promised to be.'
\end{exe}


In addition to a subordinate clause, the \fename{Stimulus} can also be introduced by a dative clitic argument (Example \ref{ch6:ex:43}) or a \textit{на}-, \textit{за}-, \textit{от}- or \textit{с}-PP (Example \ref{ch6:ex:44}).

\begin{exe} 
\ex  \label{ch6:ex:43} 
 %   \settowidth \jamwidth{(bg)} 
\gll [...] \textit{да} [\textit{му}]$_{\feinsub{Stim}}$ \textit{\textbf{СЕ РАДВА-М}}   \textit{скришом}.\\ 
{}[...] {to} {he-DAT} {rejoice-1.SG.PRS} {secretly}.
\\ % \jambox{(bg)}
\glt `[...] and enjoyed him secretly.'

\ex \label{ch6:ex:44} 
 %   \settowidth \jamwidth{(bg)} 
\gll \textit{Обикновено} \textit{\textbf{СЕ УСПОКОЯВА-МЕ}} [\textit{с} \textit{известн-ия} \textit{факт}]$_{\feinsub{Stim}}$ [...].\\ 
{usually} {calm down-1.PL.PRS} {with} {known-DEF.M} {fact} [...].
\\ % \jambox{(bg)}
\glt `We usually calm down at the well-known fact [...].'
\end{exe}

Negative-emotion verbs --~ \textit{ужасявам се} `feel / become terrified', \textit{плаша се} `fear', \textit{pазстройвам се} `feel / become upset', tend to take an \textit{от}-PP, while positive ones prefer \textit{на}- or \textit{за}-PPs. 



The \textit{с}-PP appears a lot more frequently when denoting another individual or individuals, who the \fename{Experiencer} shares emotional response with. Within the frame structure it is marked as an \fename{Empathy\_target} and is a non-core frame element (Example \ref{ch6:ex:45}). 


\begin{exe} 
\ex   \label{ch6:ex:45}
 %   \settowidth \jamwidth{(bg)} 
\gll \textit{Върв-и} \textit{да} \textit{\textbf{СЕ ЗАБАВЛЯВА-Ш}} [\textit{с} \textit{Едуард}]$_{\feinsub{EmpT}}$.\\ 
{go-SG.IMP} {to} {entertain-2.SG.PRS} {with} {Edward}
\\ % \jambox{(bg)}
\glt `Go have fun with Edward.'
\end{exe}


Another possible syntactic construction within this semantic frame is that both the \fename{Stimulus} and the \fename{Reason} appear together in one sentence. In these cases, the \fename{Stimulus} is expressed by a PP and the \fename{Reason} by a complement clause. Koeva points out that in these cases an internal left dislocation is observed -- an argument from the subordinate clause can appear in object position with the main predicate. It can also be expressed explicitly in the subordinate clause and is coreferent with the object in the main clause. No such examples were found in the corpus, but there are some on the internet (Example \ref{ch6:ex:46}).

\begin{exe}  
\ex  \label{ch6:ex:46}
  %  \settowidth \jamwidth{(bg)} 
\gll \textit{\textbf{РАДВА-М СЕ}} [\textit{на} \textit{дец-а-та}]$_{\feinsub{Stim}}$ [\textit{че} \textit{ходя-т} \textit{на} \textit{училище} \textit{с} \textit{удоволствие}]$_{\feinsub{Reas}}$.\\ 
{be glad-1.SG.PRS}  {to} {child-PL-DEF.PL} {that} {go-3.PL.PRS} {to} {school} {with} {pleasure}
\\ % \jambox{(bg)}
\glt `I am happy for the children that they attend school with pleasure.'
\end{exe}


 


\section{Conclusions} \label{ch6:sec:6}

This study is devoted to the representation of the semantic and syntactic behaviour of verbs of emotion and their arguments. %The resources used were described in \sectref{ch6:sec:2}, where their main characteristics and structures were discussed. The methodology of the work was outlined in \sectref{ch6:sec:3}, followed by a description of the class of emotion verbs and different typological approaches dealing with them.

A number of the most common emotion verbs were selected for the study and their semantic frames were discussed. The main focus was on five semantic frames, namely \framename{Feeling}, \framename{Experiencer\_focused\_emotion}, 
\framename{Cause\_to\_experience},
\framename{Stimulate\_emotion} and \framename{Emotion\_directed}.

A smaller number of semantic frames (e.g. \framename{Worry}, \framename{Fear}, \framename{Emotion\_heat} and others), which comprise fewer lexical units, were not considered in our study and will be analysed in the future.

All semantic frames investigated were characterised in terms of the lexical units which evoke them; their core frame elements and the possible representations they may have in terms of their syntactic and semantic expression. The frame \framename{Feeling} was presented with its core frame elements \fename{Experiencer}, \fename{Emotion}, \fename{Emotional\_state} and \fename{Evaluation}. It was found that the transitive verbs encode an \fename{Emotion} as a direct object, while the intransitive \textit{чувствам се} `feel (oneself)' includes the \fename{Emotional\_state} or \fename{Evaluation} in the sentence. The \framename{Experiencer\_focused\_emotion} was slightly modified with regard to the description of the Bulgarian verbs and the \fename{Content} and \fename{Experiencer} were adopted as core frame elements. Various options for the encoding of \fename{Content} were presented. The semantic frames \framename{Stimulate\_emotion} and \framename{Cause\_to\_experience} had similar characteristics: they both contain causative verbs and have two core frame elements, one of which is the \fename{Experiencer}. The second core frame element is semantically expressed as \fename{Stimulus} in the first semantic frame and as \fename{Agent} in the second. The potential conjunctions, interrogative and relative pronouns that can introduce a frame element were searched for in the corpus, and the results were presented as examples and listed in a table for clarification. The frame \framename{Emotion\_directed} comprises the middle-voice equivalents of the stative and inchoative verbs evoking \framename{Stimulate\_emotion} and \framename{Cause\_to\_experience} frames.
 
Lexical units of the frame \framename{Feeling} are neutral with respect to the emotion they denote, and their complements express the positive or negative connotation. The verbs themselves carry the semantics of a positive or negative emotion within the other four semantic frames discussed above.

To summarise, each semantic frame consists of a collection of frame elements that represent the semantic components or roles associated with it. The role of each frame element within a particular semantic frame is crucial for the accurate representation of the semantic structure and frame conceptualisation. FEs help to capture the relations, roles and interactions between the different participants and components within a semantic frame. They provide a detailed representation of the conceptual content of a frame and enable a more precise and nuanced linguistic analysis and understanding. The semantic analysis of the frame elements of the frame \framename{Emotions} and its five analysed subframes enables a prediction of the arguments of a semantic frame with respect to the specified linguistic constraints. The corresponding facets of the scenario represented for each semantic frame are a set of possible values from an inverted tree or subtree of WordNet. Sorting possible semantic components of words into groups of common semantic type (hypernyms) is in contrast to analysing the semantic argument structure of sentences based on specific words.

This in-depth analysis and manual approach to assigning semantic and syntactic information to the core frame elements provides new insights and a deeper understanding of the syntactic behaviour of verbs and their environment. Although the manual review and selection is quite time-consuming, one of the strengths of the method is that it involves precise alignment of data from different resources which are quite asymmetric for automatic alignment.


 

 
\section*{Abbreviations}
\begin{multicols}{2}
\begin{tabbing}
MMMM \= Obligatory\kill
 AccCl \> Obligatory accusative clitic \\
  \textsc{Age} \> Agent \\
  \textsc{Cont} \> Content\\
 AdvP \> Adverbial phrase \\
 DatCl \> Obligatory dative clitic \\
\scshape Emos \> \fename{Emotional\_state}\\
\scshape Emot \> \fename{Emotion}\\
\scshape EmpT \> \fename{Empathy\_target}\\
\scshape Exp \> \fename{Experiencer}\\
 FE \> Frame element  \\
 NP \> Noun phrase \\
 PP \> Prepositional phrase \\
\scshape Reas \> \fename{Reason}\\
 S \> Subordinate clause \\
\scshape Stim \> \fename{Stimulus}\\
\end{tabbing}
\end{multicols}




\section*{Acknowledgements}

This research is carried out as part of the project \emph{Enriching Semantic Network WordNet with Conceptual Frames} funded by the Bulgarian National Science Fund, Grant Agreement No. KP-06-H50/1 from 2020.



{\sloppy\printbibliography[heading=subbibliography,notkeyword=this]}
\end{document}
