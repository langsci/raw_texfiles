%!TEX root = ../main.tex

\chapter{Introduction}\label{ch:1}
\section{Background}\label{sec:1}

\citegen{Comrie1976} definition of \isi{Aspect}\footnote{I will use the term ``\isi{Aspect}"
in this work in reference to the general concept as opposed to specific levels
that are involved.  I will use the lower case ``aspect'' to refer to specific
levels or elements of \isi{Aspect} such as \isi{grammatical aspect}, or inner and outer
aspect.} as ``the internal temporal constituency of a situation'' (p.~3) does not
begin to capture the complex nature of \isi{Aspect} as articulated in the literature.
Indeed,  ``the internal constituency of a situation'' is elaborated as a
mix of information from the verb, internal and external arguments, grammatical
aspect markers, adverbials etc.
\citep[cf.][]{Klein1994,Krifka1998,Jackendoff1996,MacDonald2008,Mourelatos1981,Ramchand2008,Rothstein2004,Tenny1994,TennyPustejovsky2000,Verkuyl1993,Verkuyl1993,Verkuyl1999}
among others.  The involvement of these elements effectively establishes \isi{Aspect}
as two domains of study, namely inner and \isi{outer aspect} (\citealt{Travis1991};
2005; \citealt{Verkuyl1993}), or situation and \isi{viewpoint aspect}
(\citealt{Smith1983,Smith1991}).

In the sections which follow I will look briefly at \isi{viewpoint aspect}
(\sectref{sec:1.1}) and \isi{inner aspect} (\sectref{sec:1.2}).  Further to this, I
will look with specific regard at the case for further investigations into the
nature of the verb itself in the study of \isi{Aspect} in Caribbean English Creoles
(CECs) (Section~\sectref{sec:1.3}).  In \sectref{sec:1.4} I will look at the
compositionality of \isi{Aspect} and move into a discussion of dual aspect and the
\isi{Stative}\slash Non-stative distinction in \sectref{sec:1.5}.  In \sectref{sec:1.6} I
present a synopsis of my proposal in this study.  The aim and scope of the work
is presented in \sectref{sec:1.7} while the organisation of the work is
presented in \sectref{sec:1.8}.

\section{Outer/viewpoint Aspect}\label{sec:1.1}\largerpage
Outer aspect or \isi{viewpoint aspect} may be taken to refer to the impact of
\isi{grammatical aspect} markers in the construct of \isi{aspectual outlook}.  The most
well-known distinction in this area is that established in the \isi{Perfective} and
\isi{Imperfective} (see \citealt{Comrie1976}).  According to \citet{Comrie1976}, the
\isi{Perfective} is ``the view of a situation as a single whole, without distinction of
the separate phases that make up that situation.'' (p.~16).  By contrast, the
\isi{Imperfective} ``pays essential attention to the internal structure of the
situation.'' (p.~16).

Languages have various ways of dealing with the expression of these aspectual
viewpoints which may or may not be grammaticalized in languages.  Thus for
example, while the English language employs the use of the progressive
-\textit{ing} to express Imperfectivity, a language such as \ili{Finnish} uses case
marking to establish the difference between a \isi{Perfective} and \isi{Imperfective}
viewpoint.  
Cf. \REF{ex:1:1}\largerpage

\ea\label{ex:1:1} \ili{Finnish} \citep[8]{Comrie1976}
  \ea hän luki kirjan
  \glt `He read the book.'
  \ex hän luki kirjaa
  \glt `He was reading the book.'
  \z
\z

According to \citet{Travis2010} discussion of the \ili{Finnish} examples in
\REF{ex:1:1}, it is the partitive case in (\ref{ex:1:1}b) that results in the
\isi{Imperfective} reading as opposed to the accusative case in (\ref{ex:1:1}a) where
we get a \isi{Perfective} viewpoint (p.~2).  Case does not play a similar role in
English which, instead, employs the \isi{Progressive} marking to focus on the internal
aspect of the situation in the translation in (\ref{ex:1:1}b) whereas, the \isi{Perfective}
viewpoint is grammatically unmarked.

Semantically, \isi{viewpoint aspect} has been indicated to be subjective in that a
speaker may choose to express a particular perspective of a situation regardless
of the inherent truth properties of the situation itself. Thus as
\citet{Guéron2008} points out, expressions of \isi{viewpoint aspect} are ``unaffected
by real world experience'' (p.~1824). Based on this, in the case of the
examples in \REF{ex:1:1} above; both (\ref{ex:1:1}a) and (\ref{ex:1:1}b) may be
true at the same time dependent on ``the speaker's choice of perspective on the
situation'' \citep[479]{Smith1983}.

In my discussion of the literature on \isi{Aspect} in Caribbean English
Creoles (CECs) in \chapref{sec:2}, we will see that \isi{Imperfective} aspect in CECs
is typically marked preverbally by a particle which has been analysed as marking
\isi{Progressive}, \isi{Imperfective}, continuative or iterative depending on the author.
\isi{Perfective}, on the other hand is unmarked in CECs. It will become evident in
\chapref{ch:2} that the interaction between the notion of \isi{Stativity} and
Im\isi{perfective aspect} marking is one that underpins the controversy that has
existed in the field as it relates to the question of \isi{Stativity}. My elaboration
of this in \chapref{ch:4} is intended to bring a measure of balance to this
discussion.

\section{Compositionality}\label{sec:1.2}\largerpage
\subsection{Viewpoint aspect and inherent aspect}\label{sec:1.2.1}

Viewpoint aspect has been indicated to interact with \isi{inherent aspect} resulting
either in a modification of the viewpoint typically associated with a marker or
in a modification of the situation aspect inherently associated with the verb.
Thus, in the case of the \isi{Progressive} -\textit{ing} morpheme in English we
observe differences in interpretations as it interacts with different types of
verbs, the typically \isi{Stative} verb \textit{have} \REF{ex:1:2} as well as with the
Non-statives \textit{run} and \textit{close} in \REF{ex:1:3} and \REF{ex:1:4}:


\ea\label{ex:1:2}~\citep[adapted from][707]{Lyons1977}
  \ea She \textbf{has} a headache. (\isi{Stative})
  \ex She \textbf{is having} a headache. (Non-stative)
  \ex She \textbf{is having} one of her headaches. (Non-stative)
  \z
\z

\ea\label{ex:1:3} John \textbf{is running}. (Non-stative) \z


\ea\label{ex:1:4}
  \ea The door is \textbf{closed.} (\isi{Stative})
  \ex The door \textbf{is closing.} (Non-stative)
  \z
\z

Upon superficial examination, the \isi{Progressive} \isi{viewpoint aspect} associated with
the \--\textit{ing} morpheme appears simply to establish a processual viewpoint.
Upon closer examination however, it is noteworthy that the \isi{Stative}
interpretation typically associated with a verb like \textit{have} in \REF{ex:1:2}
shifts to one that is Non-stative analogous to that which is inherently
associated with verbs like \textit{run} and \textit{close} as shown in \REF{ex:1:3} and
\REF{ex:1:4}. This is not the only noteworthy point however as further to this,
one may analyse differences in the aspectual viewpoints that arise in each case.
In particular, I note that, the use of the \isi{Progressive} in the case of  \textit{have a
headache} as in (\ref{ex:1:2}b--c) extends the situation, establishing the situation as
on-going (cf.\ \citealt{Guéron2008}). This is similar to the interpretation that we get
in the case of  \textit{run} in \REF{ex:1:3} -- both may be taken as having occurred
and on-going in this context. In other words \textit{John is running} must be
taken to encompass the meanings: John \textsc{has run} as well as John \CONTINUES to
\RUN\@.  Similarly, She is \HAVING a headache implies that She \CONTINUES to have a
headache that arose prior to the time of the utterance.  The same is not true of
the verb \textit{close} in (\ref{ex:1:4}b) as \textit{the door is closing} does not
entail that the door has closed.  Rather, (\ref{ex:1:4}b) must be interpreted as
a Change of state\is{State!Change of} in progress.

This difference in interpretations that arises where \isi{Progressive} viewpoint
aspect interacts with different types of verbs, I analyse as directly linked to
the Event Structure (ES) that is inherently associated with particular verbs.
While generally speaking the presence of \isi{Progressive} \isi{viewpoint aspect} signals
the presence of Non-stative meaning, the nuances associated with this
interaction are far more interesting and significant than the actual presence of
Non-stative meaning. Thus, as we can see here, while the \isi{Progressive}
\textit{-ing} morpheme in the case of an inherently \isi{Stative} verb like \textit{have} is
responsible for the introduction of Non-stative meaning, this meaning is already
present in the case of a verb like \textit{run} where the \isi{Progressive} simply
establishes the situation as on-going. In the case of a verb like \textit{close} which
I analyse as a Change of state\is{State!Change of} predicate \citep[cf.][]{Pustejovsky1988} the
viewpoint established on the situation is one which focuses on the onset of the
situation within the context of a Change of state\is{State!Change of} in progress. I return to
examples such as these in \sectref{sec:5.1} of this work.

Below I will look briefly look within the compositionality of aspect at the
restriction on the combination of \isi{Progressive} aspect and \isi{Stativity} that has been cited in the literature.

\subsection{Progressive aspect and stativity}\label{sec:1.2.2}\largerpage

A restriction has also been noted between \isi{Stative} predicates, and \isi{Progressive}
\isi{viewpoint aspect} (see \citealt{Vendler1957}, \citealt{Dowty1979},
\citealt{Smith1983} etc). Regarding this, \citet{Smith1983} for example notes in
relation to the examples in \REF{ex:1:5} below that the ``same choices [in
aspectual viewpoint] may not be available for talking about every situation'' (p.
479).

\ea\label{ex:1:5}~\citep[479]{Smith1983}
  \ea[]{You know the answer.}
  \ex[*]{You are knowing the answer.}
  \z
\z

The ungrammaticality resulting from the interaction between a stative predicate
such as \textit{know} and the progressive \textit{-ing} in the case of English has
presented a case for many on the restriction between stativity and
progressivity. However, as \citet{Smith1983} points out ``things are a little
more complicated than this. Speakers sometimes make an unusual choice of aspect;
i.e., one can talk about a situation in a manner not usually associated with
it'' (p.~479). Essentially, there may be occasions where a speaker may quite
appropriately say ``I am now knowing that!"

In dealing with this interaction between \isi{viewpoint aspect} and other areas of
\isi{Aspect}, \citet{Smith1983} in her speaker-based approach, cautions that ``the
properties of an actual situation should not be confused with its presentation
in a given sentence'' (p.~480) supporting the observation in the previous
section. In \chapref{sec:5} (\sectref{sec:5.1}) I attempt to treat this issue
further where I explain my interpretation of the \isi{progressive criterion} and its
application in the model I articulate. More generally, while this area of \isi{Aspect}
will not constitute a major focus in this work, it will be highlighted in
\chapref{ch:2} as a principal area of focus in the study of \isi{Aspect} in CECs.

Below I will look briefly at the area of \isi{inner aspect}.

\section{Inner Aspect}\label{sec:1.3}\largerpage[2]

The domain of \isi{inner aspect} refers to the interaction between the verb and its
\isi{internal argument} and concerns the aspectual feature associated with whether or
not a situation has a ``distinct, definite and inherent endpoint in time''
\citep[4]{Tenny1994}.  Inner aspect has been of major interest in recent
investigations into the \isi{syntax-semantics interface} and has been characterised by
several distinctions concerned with the notion of Endpoint.  These include the
\isi{Telic}\slash \isi{Atelic} \citealt{Garey1957,Comrie1976,Smith1991, Rothstein2004},
the Bounded\slash Non-bounded \citep{Verkuyl1972,Dahl1981,Dahl1985,Jackendoff1990,Krifka1998}, 
the Culminating\slash Non-cul\-mi\-na\-ting \citep{MoensSteedman1988}, Delimited\slash Non-delimited
\citep{Mourelatos1981,Tenny1994} and Quantized\slash Non-quantized
\citep{Krifka2001, Filip2000}.

Generally speaking, these oppositions capture the difference in interpretation
that has been observed for sentences such as those in \REF{ex:1:6} below:

\ea\label{ex:1:6}
  \ea John ate.
  \ex John ate mangoes.
  \ex John ate a mango.
  \z
\z

(\ref{ex:1:6}a--b) are interpreted as lacking an Endpoint (i.e. \isi{Atelic}) due in
this case to the lack of or inability of an \isi{internal argument} which can
constrain the event to a logical endpoint \citep[cf.][]{Tenny1994}.  In
contrast, (\ref{ex:1:6}c) is interpreted as containing a logical Endpoint due to
the nature of the \isi{internal argument} \textit{a mango} which constrains the event to a
logical Endpoint. This is so as the eating event must come to an end once the
eating of \textit{a mango} is complete.\footnote{\citet{Jackendoff1996} develops and
  formalises the intuition behind the notion of  ``measuring out"
  \citep{Tenny1994} which I allude to here. He posits a representation of an
  event such that the affected object or theme is joined to a path; the telicity
  of the event depends not only on the nature of the theme but on the path.
  According to him,
  
  \begin{quote} ``[t]he position of the theme along the path is encoded as a
  function of time, so that for any arbitrary moment of time, there is a
  corresponding position [...] The theme is at the beginning of the path at the
  beginning of the event and at the end of the path at the end of the event. If
  the path has distinct segments, then the event can be divided into segments
corresponding to when the theme is on the associated parts of the path" (p. 317--318)\end{quote}}

Examples such as those in \REF{ex:1:6} which show a particular verb
appearing with different aspectual interpretations due to the
influence of the \isi{internal argument} establish the focus in aspectual
analysis as minimally on the Verb Phrase (VP) as opposed to simply the
aspectual properties of the verb. This viewpoint is summed up in 
\citet{TennyPustejovsky2000} who state that,

\begin{quote}
[i]t is now generally accepted that we must talk about the aspectual
properties of the verb phrase or the clause, rather than simply the
aspectual properties of the verb since many factors including
adverbial modification and the nature of the object noun phrase
interact with whatever aspectual properties the verb starts out
with (p.~6).
\end{quote}

This statement is made in reference to developments in the field since
\citegen{Vendler1967}; focus on the verb as a ``crucial'' factor in
aspectual interpretation and his supposed division of
verbs\footnote{Vendler's classification has left some doubt as to
  whether or not it was based on just verbs or VPs due to his
  inclusion of both in his classification.  \citet{Verkuyl1999} for
  example points to this weakness in Vendler's classification,
  stating that: one can interpret him [Vendler] very benevolently as
  acknowledging the need to analyse aspectuality at the phrase level
  but in the meantime he made it impossible by distinguishing his
  classes at the verb level (p.~96).}  into four classes: \isi{State},
\isi{Activity}, \isi{Accomplishment}, and \isi{Achievement}.  Later reinterpretations of
Vendler's verbal classes finally arriving at the basic \isi{State}\slash Non-state
distinction that we see for example in Verkuyl's (\citeyear{Verkuyl1993,Verkuyl1999}) use of
the feature [+/\textminus Change] point to a basic contribution of the verb to
\isi{Aspect}. Nevertheless, observations of the fact that a single verb may
be used to express two different aspects due to the influence of the
\isi{internal argument}, (cf. \citealt{Dowty1979,Dahl1981,Verkuyl1993,Tenny1994,MacDonald2008}, etc.) and
also that different uses may be associated with a verb through context
\citep[cf.][4]{Tenny1994} make it logical to focus on VP as opposed to the
aspectual properties of the verb.

\section{Aspect in CECs and the nature of the verb}\label{sec:1.4}

While this is accepted to be the case, the study of \isi{Aspect} in CECs may
benefit from further investigations into the nature of the verb itself
and its contribution to \isi{Aspect}.  This is due mainly to discussions
surrounding the large number of lexical items which may systematically
express contrasting aspects; this behaviour is not necessarily due to
the direct influence of internal arguments.  If we consider the
\ili{Jamaican Creole} examples\footnote{Please note that all JC\il{Jamaican Creole}
  examples, except where otherwise attributed, are from the author's
  own native speaker introspection.} in \REF{ex:1:7} and \REF{ex:1:8}
below, we will see that the JC\il{Jamaican Creole}  verb \textit{iit}
`eat' is able to express either \isi{Telicity} or Atelicity consistent with
a change in the semantic denotation of the \isi{internal argument}.  In
contrast, \textit{redi} `ready' seems to express opposing aspects
based on the structure in which it appears:

% \setcounter{enumerate}{6}
\ea\label{ex:1:7}~ JC\il{Jamaican Creole}
  \ea 
    \gll Jan \textbf{iit} mango.\\
    		John eat mango\\
    \glt `John eats mangoes.' (Habitual-\isi{Atelic})

    \ex
    \gll Jan \textbf{iit} tuu mango.\\
    		John eat two mango\\
    \glt `John ate two mangoes.' (\isi{Telic})
    \z
  \z			

  \ea\label{ex:1:8}
    \ea
    \gll Jan \textbf{redi} di pikni.\\
   		 John ready \textsc{art} child\\
    \glt `John readied the child.' (\isi{Telic})

    \ex
    \gll Di pikni \textbf{redi}.\\
    \textsc{art} child ready\\
    \glt `The child is ready.' (\isi{Atelic})
    \z
  \z 

There are varying analyses on the contrasting \isi{Telicity} that may be associated
with a verb like `eat' as shown in \REF{ex:1:7}.  First, there is a general view
that points to the contribution of the \isi{internal argument} in terms of the feature
of specificity or finiteness (see authors such as
\citealt{Garey1957,MacDonald2008,Tenny1994,Verkuyl1993,Verkuyl1999}, etc.).  Others
though accepting the general idea of a relationship between the verb and its
\isi{internal argument} as responsible for the establishment of \isi{Telicity} are divided
on how this works within a compositional framework.  Thus, for example,
authors like \citet{Jackendoff1996}, and \citet{Krifka1998} emphasise the
relationship between the verb and its object as determining \isi{Telicity} as opposed
to a particular semantic feature contributed by the object.  For such authors,
\REF{ex:1:7} is not \isi{Telic} simply because there is a specified \isi{internal argument},
but because of the intrinsic relationship that is established between a verb
like `eat' and its \isi{internal argument} whereby the \isi{internal argument} provides a path, where for each part of the event of eating a sub-portion of the object is
covered.



Focus on the relationship between the verb and its object rather than on a
particular semantic feature of the verb allows for generalisation over different
types of verbs.  Such an approach takes into consideration the fact that verbs
of motion such as \textit{carry} behave differently as it relates to the relationship
between the verb and its object (Object to Event (OTE) mapping) (see
\citealt{Krifka1998,MacDonald2008,Tenny1994}).  In another approach, 
\citet{BennettPartee2004} articulate the view that the difference in interpretation in
\REF{ex:1:2} (recalled below as \ref{ex:1:2b}) is due to the ``ambiguity of the verb [...] and not the change in
direct object'' (p.~72).  

\ea\label{ex:1:2b}~\citep[adapted from][707]{Lyons1977}
  \ea She \textbf{has} a headache. (\isi{Stative})
  \ex She \textbf{is having} a headache. (Non-stative)
  \ex She \textbf{is having} one of her headache. (Non-stative)
\z
\z

Such approaches which generally speaking may be said
to focus the semantic contribution of different elements may be contrasted with
the ``exo-skeletal'' approach articulated by \citet{Borer2005} where the focus
is on syntactic structures as providing  ``unambiguous formulas for the semantics
to interpret'' (p.~11).

\section{A note on the compositionality of Aspect}\label{sec:1.5}

In this work, I am inclined to accept an analysis which focuses on the
relationship between the verb and its \isi{internal argument} and the semantic
contribution of both these elements within the context of \isi{Aspect} as
compositional.  This viewpoint though relevant is however outside the specific
scope of the discussion that I undertake in this study where as indicated, my
focus is on the verbal component in \isi{Aspect}.  Given this, we note in the case of
\REF{ex:1:7} and \REF{ex:1:8} above that there is indeed a difference in the
\isi{Telicity} indicated by the verbs in question.  Loosely speaking, the difference
in interpretation of the examples in \REF{ex:1:7} may be attributed to a
difference in the type of \isi{internal argument}, but the same may not be said of the
examples in \REF{ex:1:8}.


In the case of the verb \textit{iit} in \REF{ex:1:7} both instances of the verb
indicate Non-stativity (i.e.: Change); whether or not an Endpoint is established
depends (in this case) on the semantics of the \isi{internal argument}.  In
(\ref{ex:1:7}a) we note that the \isi{internal argument} \textit{mango} `mangoes' is
not specified as it relates to number or what has been called finiteness
(\citet{Verkuyl1993} or the feature ``Specified Quantity of A'' or [+SQA]
(\citealt{Verkuyl1993,Verkuyl1999}; also \citealt{Krifka1998}).  The result of this
interaction between the verb \textit{iit} `eat' and a non-finite internal
argument is a predicate that is \isi{Atelic}.  In (\ref{ex:1:7}b) by contrast, the
\isi{internal argument} \textit{tuu mango} `two mangoes' is specified for number and
the result is a predicate with a logical endpoint (\isi{Telic}).  Thus the difference
in this aspectual interpretation (\isi{Telicity}) may generally speaking be attributed
to the contribution of the \isi{internal argument}.

\section{Dual aspect and the Stative/Non-stative distinction}\label{sec:1.6}

A difference in interpretation is also noted for the examples in \REF{ex:1:8}.
However, in these cases, there is a difference in the structure of the sentence.
In (\ref{ex:1:8}a) where \textit{redi} `ready' is used transitively it indicates
a Change of state\is{State!Change of} while in its intransitive use (\ref{ex:1:8}b) the default
interpretation is that of a \isi{State}.  A lexical item such as \textit{redi} `ready' falls
within the general group of items in CECs called ``property items"
(\citealt{Migge2000,Winford1993}), ``predicate adjectives" \citep{Seuren1986} or
``adjectivals"\footnote{Of these terms, I adapt that of ``property items" in an
  effort to avoid as a focal point the discussion which centres on the
  \isi{categorial status} of these items.  As I point out in \chapref{ch:3}, the major
focus in terms of these items has been their \isi{categorial status}. While I
contribute to this discussion in \chapref{ch:6} of this work, I do this from the
perspective of their \isi{aspectual status}.} (\citealt{Kouwenberg1996}; also
\citealt{Sebba1986}).  These include a range of items which to varying degrees
may express what I call ``dual aspectual" behaviour.  This is in reference to the
observation of their aspectual behaviour where in one instance they may express
the feature Change\footnote{This semantic concept is identified in \chapref{ch:4}
as the basic semantic feature within the \isi{Stative}\slash Non-stative opposition. It is
elaborated in \chapref{ch:4} as simply Change in terms of \MOTION, \CHANGEOFSTATE 
or \CONTACT or any combination of these.} but yet in another express no
Change.  Items such as \textit{sik, weeri}, \textit{redi, braad} etc. which may
be translated as either the adjective (sick, weary\slash tired, ready, broad) or
inchoative verb (\BECOME ``get" sick, weary\slash tired, ready, broad) and also
transitive verb (\CAUSE to \BECOME sick, weary\slash tired, ready, broad) have been of
interest in CECs for some time now, starting perhaps with \citet{Voorhoeve1957}.

The descriptive reality where a single item may appear in different uses raises
for many the theoretical question of the \isi{categorial status} of such items.  Are
there several lexical entries for an item based on the categories in which it
appears or can a single lexical item which allows for derivation into other
categories be posited?  The way in which this question is answered has
implications for our understanding of the overall syntactic and semantic
behaviour of such items and as such is a question to be considered carefully.
The approach that is reflected in the literature is one that presents a unified
position where ``property items" are treated as either verbs
(\citealt{Alleyne1980,Jaganauth1987,Sebba1986,Winford1993}; etc) or adjectives
\citep{Seuren1986}, or a combination of both verbs and adjectives
\citep{Kouwenberg1996}.\largerpage[-4]

Underlying this discussion of \isi{categorial status} is the question of the
\isi{aspectual status} of this group of items and the \isi{Stative}\slash Non-stative
distinction.\footnote{This distinction has been central in the
  discussion of TMA systems in Creole languages.  In particular it has
  been used to account for the observed difference in the \isi{Tense}
  interpretation of unmarked verbs in CECs where the unmarked \isi{Stative}
  verb is interpreted as present while the unmarked Non-stative verb
  is interpreted as past (cf. \citealt{Bickerton1975,Winford1993}, etc). 
  This discussion is highlighted in \chapref{ch:2}.}
  In essence, given the fact that there is a group of
items which appear in both \isi{Stative} and Non-stative use, what then is
the validity of the \isi{Stative}\slash Non-stative distinction and can this be
applied at the level of the verb?  In this work, taking a basic
``semantics prior" position where it is believed that the syntactic
behaviour of a lexical item may be predicted by its semantic
description (cf. \citealt{Dixon1977}; also \citealt{Levin1993}), I
will tackle the question of the aspectual (status) behaviour of CEC
property items from the perspective of lexico-semantic representations
of verb meaning and primitive Event Structures.

Event Structure is used here in a sense similar to that
of \citet{Pustejovsky1988,Pustejovsky1991},  in which Event Structure captures the most basic semantic information that the verb contributes to \isi{Aspect}. This in turn predicts the different syntactic uses in which a lexical item may appear.
Event Structure is ``recursively defined in syntax'' \citet[55]{Pustejovsky1991}, which means that it is affected by and redefined by the influence of other factors in the syntax.  It is in this regard for example that \citet{MacDonald2008} indicates in the case of verbs like
\textit{carry}  that ``a goal PP alters the [Event Structure] of a predicate
i.e.: it turns an activity into an accomplishment''\footnote{Note that
  in my attempt to focus on the concept of a basic contribution of the
  verb to \isi{Aspect}, I avoid the use of terms which directly include the
  interaction between the verb and other elements. Thus terms like
  \isi{Activity}, \isi{Accomplishment}, \isi{Achievement} etc, are not used in reference
  to verbs and the inherent Event Structure with which they are
  associated. Such terms are however accepted with reference to the
  interaction between the syntax and the semantics at the level of
  \isi{inner aspect}.} (p.~6).  Nevertheless, at the level of the verb
the basic opposition established in the \isi{Stative}\slash Non-stative
distinction may be seen in the definition of the notions of \isi{State}
(\isi{Stative}) on one hand and \isi{Process} and \isi{Transition} (Non-stative) on the
other (ibid, p. 56).  Regarding these, \citet{Pustejovsky1988} defines
a \isi{State} as ``an eventuality that is viewed or evaluated relative to no
other event'' (p.~22).  A \isi{Transition} is seen as ``a \isi{single eventuality}
evaluated relative to another \isi{single eventuality}'' (p.~22).  While a
\isi{Process} is ``a sequence of identical eventualities'' (p.~23)

\section{The proposal}\label{sec:1.7}

I will argue that these so called property items in CECs constitute
those items which are inherently \isi{State} (\isi{Stative}) and those which are
inherently \isi{Transition} (Non-stative).  In establishing this, I look at
not just the syntactic behaviour of these items (i.e.: whether or not
they are compatible with Non-stative use either by compatibility with
Im\isi{perfective aspect} or \isi{transitive variation}) but crucially the
\isi{semantic behaviour} that they exhibit.  I note in particular that in
Non-stative use at least three different interpretations are possible
for property items. These are:

\ea\label{ex:1:9}
  \ea \parbox[t]{\linewidth}{A \isi{change of state interpretation} within a logical opposition\footnote{The
    notion of logical opposition is elaborated in \sectref{sec:5.1.3} (see
  \citealt{Horn1989}) and Aristotle's square of opposition.} of contrariety
  (cf.\ JC\il{Jamaican Creole} \textit{di fuud a kuul} `the food is cooling', \textit{dem a kool di
  fuud} `they are cooling the food'.  This non-stative interpretation is taken
  as linking the opposition \textsc{hot}:\textsc{cold} whereby it is taken that some element of
  \textsc{hot} was the original state of the item undergoing the Change of state\is{State!Change of} (food).}
  \ex \parbox[t]{\linewidth}{A \isi{change of state interpretation} with a logical opposition of
  contradiction (cf. JC\il{Jamaican Creole} \textit{di shuuz a blak} `the shoe(s) is\slash are getting
  black' \textit{dem a blak di shuuz} `they are blackening the shoe(s)'.}
  \ex \parbox[t]{\linewidth}{An ongoing\slash processual interpretation with no change of state (cf.\ JC\il{Jamaican Creole}
  \textit{im a bad} `He is misbehaving').}
  \z
\z

I argue that items displaying the \isi{semantic behaviour} in (\ref{ex:1:9}a) are in
essence Non-stative (\isi{Transition}) predicates while those displaying behaviour
consistent with (9b and 9c) are \isi{Stative} predicates derived to express
Non-stativity (Change of state\is{State!Change of} and \isi{Process} respectively).  The latter items are
characterized by the same Event Structure as items which do not appear in
Non-stative use.  But, are distinct in that they are vulnerable to a
\isi{morphological process} that allows for the introduction of meaning components
associated with Non-stativity (i.e.: \CAUSE, \BECOME and \DO\footnote{I discuss
these as notions associated with the expression of Change in \chapref{ch:5},
\sectref{sec:5.3}.}).

An evaluation of property items from the perspective of their \isi{aspectual status}
and Event Structure projects logically into a discussion of the categorial
status of these items.  Consistent with the variation in aspectual behaviour
that is displayed by these items, I posit a diverse categorisation including
both (\isi{Stative}) adjectives and (Change of state\is{State!Change of}) verbs.  Also, consistent with
the derivation of \isi{Stative} items to express Non-stativity, I observe that base
adjectives may be derived into (Non-stative) verbs.  Likewise, Non-stative verbs
may be derived as (\isi{Stative}) adjectives.  This accounts for the Non-stative use
of JC\il{Jamaican Creole} items such as \textit{blak} `black' and \textit{red} `red' as well as
\textit{jelas} `jealous', \textit{bad} `bad', and the \isi{Stative} use of items such
as \textit{raip} `ripen', \textit{mad} `madden', \textit{sik} `sicken', etc.

In treating \isi{dual aspectual forms} in CECs, I hope to directly address the
contribution of the verb to \isi{Aspect} within the context of a compositional
approach to \isi{Aspect}, which sees \isi{Aspect} as comprising different elements apart
from the verb.  Based on this, \citegen{Bickerton1975}; \isi{Stative}\slash Non-stative
distinction is reduced to the feature Change consistent with
\citegen{Comrie1976} definition and unambiguously applied to the verb at the
\isi{lexical level}.  However, this notion of \isi{inherent aspect} as the fixed semantic
contribution of the verb is taken as a part of \isi{Aspect} which is compositional at
the syntactic level.  The accomplishment of a reconciliation between aspect as a
fixed concept at the level of the verb and \isi{Aspect} as compositional is
significant not only as it relates to the study of \isi{Aspect} in CECs but to the
study of \isi{Aspect} more generally where there has been much debate on the
compatibility of lexical approaches to \isi{Aspect} and compositional approaches (see
\citealt{Rothstein2004,Tenny1994,Verkuyl1999}; etc.).

\section{Aim and scope of this work}\label{sec:1.8}

The primary aim of this work is to provide a model for the analysis of property
items in CECs.  In so doing I will seek to:

\ea\label{ex:1:10}
  \ea\parbox[t]{\linewidth}{provide an account of the existence of lexical items which show dual
  aspectual behaviour}
  \ex\parbox[t]{\linewidth}{elucidate the \isi{Stative}\slash Non-stative distinction and how this may be applied
  in the context of dual aspectual items}
  \ex\parbox[t]{\linewidth}{lend insights into the discussion surrounding the \isi{categorial status} of
  property items based on the aspectual behaviour observed}
  \z
\z

In order to facilitate a categorisation of property items, I will examine the
aspectual behaviour of a range of these in JC\il{Jamaican Creole} for a language-specific
categorisation.  I will use \citegen{Winford1993} classification of property
items based on \citegen{Dixon1977} semantic classes as a point of departure.
However, I will attempt to provide a classification that is based on aspectual
behaviour rather than semantic concepts.  For this I will apply the standard
syntactic tests of compatibility with \isi{Progressive} aspect and transitive
alternation that have been used in the literature to test for \isi{Stativity} (see for
example \citealt{Jaganauth1987} and \citealt{Winford1993}).  These will however be
accompanied by semantic criteria linked to different event types, and on the
type of interpretation that arises where such an item appears in Non-stative
use.

Due to the variation in the behaviour of these lexical items across CECs and
even across varieties of the same Creole, it would be overly ambitious to say
the least, to attempt a classification that would account for all CECs. This is
consistent with \citet{Kouwenberg1996} who notes that ``there is too much
variation across Creole languages to attempt a single explanation with
cross-Creole validity'' (p.~9).  Nevertheless, the model that I provide which is
based on primitive event types or Event Structures and the overall aspectual
(i.e. syntactic and semantic) behaviour displayed by property items, will allow
for generalisations to be made regarding the possible behaviour that may be
observed for property items across Creoles.

The focus of this study is the contribution of the verb to \isi{Aspect}.  However,
this is taken within the context of the compositionality of \isi{Aspect} which I
acknowledge in this work.  In particular, as it relates to terminology, I
identify three separate yet interacting levels of \isi{Aspect} starting with the verb
(\isi{inherent aspect}), its interaction with arguments (\isi{inner aspect}) and its
interaction with \isi{grammatical aspect} (\isi{outer aspect}).  In this regard, this is
perhaps the first piece of work in the field of Creole studies that overtly
addresses the compositionality of \isi{Aspect} in these languages.  The major terms
may be summarised as follows:

\ea\label{ex:1:11}
  \ea {\justifying\textsc{inherent aspect}: This refers to the contribution of the verb to
  \isi{Aspect}.  It is captured in the \isi{State}\slash dynamic distinction as defined by
  \citet{Comrie1976} separating verbs which include ``necessary change" from
  those which do not.  However, while this usage has been extended to
  classifying not just verbs but phrases and propositions, I apply the
  distinction strictly to verbs that express Change from those which do
  not. Comrie's distinction is treated as analogous to the \isi{Stative}\slash Non-stative
  in Creole studies as articulated by \citet{Bickerton1975,Bickerton1981}.\par}
  \ex {\justifying\textsc{inner aspect}: This points to the level of the syntax-semantics
  interface where the verb interacts with its argument(s) and also goal
  adverbials or Prepositional Phrases (PPs). At this level the contribution of
  the verb which I identify as [+/\textminus \textsc{change}] interacts with the semantic
  contribution of the \isi{internal argument}.  Also, adverbial modifications in the
  form of a goal introduced through PP operate here in the establishment of an
  Endpoint.  The relevant distinction here is captured in the \isi{Telic}\slash \isi{Atelic}
  opposition which underlies the concepts such as \isi{Achievement} and \isi{Accomplishment}
  on one hand, and \isi{Activity} on the other.\par}
  \ex {\justifying\textsc{outer aspect}: This refers to the contribution of \isi{grammatical aspect}.
  This concerns whether a situation is viewed in its totality as a ``complete
  whole" or not (i.e.: The \isi{Perfective}\slash \isi{Imperfective} distinction,
  \citealt{Comrie1976}) or not.  Outer aspect is separated from \isi{inner aspect} and
  \isi{inherent aspect} as \isi{viewpoint aspect} which is the subjective way in which a
  situation may be viewed by a speaker \citep{Klein1994,Smith1983,Smith1991}.\par}
  \z
\z

Regarding these levels of \isi{Aspect} (\ref{ex:1:11}b--c) may be separated from
(\ref{ex:1:11}a) based on the fact that they refer to structural
levels of \isi{Aspect} and the interaction between both semantic and
syntactic information.  As indicated, the area of \isi{inherent aspect} will
be my primary focus, however the terminology outlined here is
consistent with a view of \isi{Aspect} as compositional.  In this regard,
the separation of terminology is a key element in the understanding of
the complex and multi-layered facets of aspect.

My hope is for an overall treatment of \isi{Aspect} in CECs which takes into
consideration the specific contribution of all the key elements
involved within a compositional model.  The treatment of the semantic
contribution of the verb is only one step in that direction but
crucial nonetheless as it advances the theoretical question of the
\textit{how} of compositionality.  As \citet{Dowty2006} indicates
regarding the discussion of compositionality in language, ``[this]
really should be considered ``an empirical question''.  But it is not a
yes-no question, rather it is a ``how''-question.'' (p.~5).  Indeed, an
exploration of such a central element as the verb and its contribution
to \isi{Aspect} will raise questions as to how it is that this element may
be associated with an inherent aspectual value when the aspectual
behaviour of such an item indicates \isi{aspectual flexibility}.
Nevertheless, with the assumption of compositionality, it is expected
that the specific contribution of each participating element will be
understood in an effort to analyse the different interactions that
exist between and among the various elements involved.  It is in this
regard that \citet{Verkuyl1999} points out that:

\begin{quote}
Although people in general seem to adhere to the idea of a
\isi{compositional approach}, many of them do not take the consequences
…that should be drawn: To find out which basic semantic material
underlies aspectual composition and how the composition proceeds at
higher phrasal levels (p.~16).
\end{quote}\largerpage

The work I undertake in this study should be seen as one step along
this path.  It will address issues surrounding the contribution of one
element to \isi{Aspect} in CECs, however this must be taken as a part of the
whole rather than an attempt to address \isi{Aspect} overall.

One limitation of this study is the focus on JC\il{Jamaican Creole} for a language
specific classification.  Generally speaking, this is indeed a
limitation in that descriptions of specific CEC languages are way
overdue especially in the area of \isi{Aspect}.  Nevertheless, the primary
aim of this study is the establishment of a descriptive model based
on universal properties of language to account for the behaviour of
CEC property items and in particular ``dual aspectual" items.  Rather
than a focus on whether or not an item falls within a specific
category, the main concern is an explanation of why items are able to
display the behaviour that they do.

Thus, while the study posits a language specific classification of
``property items" for JC\il{Jamaican Creole}, this must be taken as no more than an
exemplar of a classification, as even a complete classification for JC\il{Jamaican Creole}
would have to consider different varieties of JC\il{Jamaican Creole} -- something this study
does not undertake. In short, this study provides a model for a
descriptive analysis for CEC property items and hopefully property
items in general.

\section{The organisation of the work}\label{sec:1.9}\largerpage

The work is organised as follows: In \chapref{ch:2}, I will present an overview and
discussion of relevant work that has been undertaken on \isi{Aspect} in CECs.
Starting with \citet{Voorhoeve1957}, I note among the works surveyed that very
few have been concerned with \isi{Aspect} on its own but rather aspect as it relates
to \isi{Tense}.  Even then, the focus has been mainly on \isi{grammatical aspect} markers
and how these interact with different verbs. However, \citegen{Bickerton1975};
work on \ili{Guyanese Creole} (GC\il{Guyanese Creole}) which points to the \isi{Stative}\slash Non-stative distinction
as ``crucial'' in the understanding of Tense-\isi{Aspect} in Creole languages, if
nothing else triggered much debate in later works.  In particular, we see where
questions have been raised regarding the unique \isi{aspectual status} of verbal forms
in light of items which appear in contrasting aspectual uses.

\chapref{ch:3} will focus on the case of \isi{dual aspectual forms} and the issues or
problems associated with these in CECs. I point out that these items have mainly
been treated from the perspective of the question of their \isi{categorial status} and
present the work of authors such as \citet{Sebba1986}, \citet{Seuren1986} and
\citet{Kouwenberg1996} which represent different positions within this
discussion.  Regarding the question of the \isi{aspectual status} of these items and
the \isi{Stative}\slash Non-stative distinction, I revisit data from \citet{Jaganauth1987}
to highlight the conceptual problem raised by items in her analysis.  As a point
of departure for an analysis of these items, I evaluate the work of
\citet{Winford1993}, which is, to my knowledge, the most complete attempt to
treat the group of property items from the perspective of aspect.\footnote{The
  later works of \citet{Winford1997} and (\citeyear{Winford2000}) revisit this question but appear
  to lean on the basic insights from the initial (\citeyear{Winford1993}) analysis, hence the
focus on this work here.}  I highlight observational and explanatory
inadequacies of this model pointing to the need for a different model that will
account for the diversity of behaviour noted among these items.

\chapref{ch:4} is an attempt to elucidate the \isi{Stative}\slash Non-stative distinction and
its application at the \isi{lexical level}.  In this chapter I reduce the notion of
\isi{inherent aspect} to the notion of Change, which separates verbs which express
necessary Change (Non-stative) from those which do not (cf.
\citealt{Comrie1976}).  I discuss the feature Change as indicated by different
(combinations of) primitives such as what I call the initiators of Change (\CAUSE
and \DO), and the primitive associated with inchoative Change
(\BECOME)\footnote{Non-volitional Change in an \isi{internal argument}.}  (see
\citealt{Dowty1979,Carter1976,McCawley1968}). Coming out of this chapter are the
ideas that form the basis for the analysis that I present in \chapref{ch:5}.

In \chapref{ch:5}, I will present a model for the analysis of CECs property items.
I outline a treatment that accommodates the range of semantic interpretations
that may be associated with an item in Non-stative use.  Thus items are not
categorised as inherently \isi{Stative} or Non-stative based on the fact that they
appear in these uses but analysed from the perspective of an inherent Event
Structure.  I evoke \citegen{Pustejovsky1988,Pustejovsky1991} notion of a
\isi{Transition} to account for items which express a Change from one state to another
but which are interpreted relative to a logical opposition expressing a contrary
in Non-stative use. Items within this category include JC\il{Jamaican Creole} \textit{raip}
`ripe\slash become ripe\slash make ripe', \textit{wet} `wet\slash become wet\slash make wet',
\textit{sik} `ill\slash become ill\slash make ill' \textit{weeri} `weary\slash become weary\slash make
weary' among others.

I highlight items of this type as distinct from those that I label inherent
States. Included among these are those which do not appear in Non-stative use
such as JC\il{Jamaican Creole} \textit{chupid} `stupid', \textit{sluo} `slow', \textit{nais} `nice'
etc. But there are also items within this category of \isi{State} items which appear
in Non-stative use, expressing either a \isi{Process} \isi{event type} or a \isi{Transition}.
Included here are items such as JC\il{Jamaican Creole} \textit{bad} `bad\slash misbehave', \textit{jelas}
`jealous' (\isi{Process}) and items expressing Colour such as \textit{blak}
`black\slash become\slash make black' and \textit{red} `red\slash become red\slash make red'
(\isi{Transition}). With regard to these latter items, which display behaviours
similar to that of Transitions, I observe the distinction whereby they do not
appear to be linked to an overt logical opposition in the same way that inherent
Transitions are. In particular, whereas an item such as \textit{sik} `sick' may
be said to be linked to an overt logical opposition which is `well' (\textsc{sick: well})
in a relationship of contrariety, the observation for an item such as
\textit{blak} `black' or \textit{red} `red' is that the opposition is less
specific. Thus, a Change of state\is{State!Change of} arriving at these simply means that the \isi{State}
did not hold previously. What is indicated is what did not hold hence an
opposition such as \textsc{black: not black} or \textsc{red: not red}, a relationship of
contradiction.

For items such as these I present the argument that they are derived rather than
inherent Transitions. These are arrived at through the introduction of relevant
primitive components consistent with the interpretations that they allow, namely
\BECOME which accounts for the inchoative meaning and \CAUSE which accounts for
the causative meaning and the introduction of a Cause or \isi{Agent}. Similarly, in
the case of those \isi{State} items which express a \isi{Process} meaning, I argue for the
introduction of the primitive \DO which is associated with Agency.

\chapref{ch:6} is a summary of the work with a focus on the implications for the
analysis of the \isi{categorial status} of the items in question.  I observe for
property items a diverse \isi{categorial status} with the group consisting of both
(Non-stative) verbs and (\isi{Stative}) adjectives underlyingly. Consistent with the
aspectual behaviour observed for these items, those items associated with an
Event Structure of \isi{Transition} and categorised as (Non-stative) verbs are shown
to appear in \isi{verbal use} or what I posit to be derived \isi{adjectival use}. Such items
are distinguished from those that I analyse as inherently associated with a
\isi{State} Event Structure and adjectival status.  Although there are items in this
class that may be derived to appear as (Non-stative) verbs, the semantic
behaviour that these display in Non-stative use sets them apart from those that
I analyse as inherent verbs. This analysis that I present diverges from what may
be seen as the standard evaluation of these items as a monolithic group of
either verbs or adjectives. It also specifically rejects the treatment of these
items as \isi{Stative} verbs.
