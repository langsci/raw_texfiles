\chapter[Summing up: On the \isi{categorial status} of \isi{dual aspectual forms}]{Summing up: On the \isi{categorial status} of \isi{dual aspectual forms} -- The implications of an aspectual analysis of CEC property items}
\label{sec:6}\label{ch:6}

\section{Overview}\label{sec:6.0}
In this study, I have focused on the question of the unique contribution of the verb to \isi{Aspect} in CECs and argued in this regard that the verb makes a unique contribution to \isi{Aspect} in the form of the \isi{Stative}\slash Non-stative distinction and the feature [Change]. The question of the aspectual contribution of the verb in CECs is closely tied to the observation in the study of \isi{Tense} in CECs, that the interpretation of the \isi{unmarked verb} differs dependent on the type of verb involved (\isi{Stative}\slash Non-stative). In this regard, \citet{Bickerton1975} notes that the default interpretation of unmarked Statives is Present while that of Unmarked Non-stative is \isi{Past}. Such a position which holds the \isi{Stative}\slash Non-stative distinction as relevant in Tense-\isi{Aspect} interpretation is intuitively appealing but it has nonetheless been fraught with problems in Creole studies. 

As indicated (\chapref{ch:2}), there have been numerous counterexamples to Bickerton’s claims regarding the \isi{Stative}\slash Non-stative distinction. First, the restriction on the occurrence of \isi{Progressive} aspect marking with \isi{Stative} verbs has been shown to be problematic. Second, the \isi{Tense} interpretations that he predicts for \isi{Stative} as opposed to Non-stative verbs have been shown to be variable, dependent on context and other factors. However, authors such as \citet{Winford1993}, and \citet{Gooden2008} have provided some support for Bickerton’s observations but point to the \isi{Stative}\slash Non-stative distinction as “only a part of the story” \citep[307]{Gooden2008}. Both these authors indicate that the influence of other factors such as adverbials and context may account for anomalous interpretations. In this regard, \citet{Gooden2008} points out that ``in many cases discourse has been shown to be relevant in precisely those cases in which Bickertonian formulation of ``aspect” fails” (p. 307--308). This being the case, much more than the issue of the \isi{observational adequacy} of the \isi{Stative}\slash Non-stative distinction is its explanatory adequacy. The matter of items which appear in both \isi{Stative} and Non-stative use (dual aspectual items) in CECs raises this question. For this reason, items of this type have been the primary area of concern in this study. 

In addressing the question of the unique contribution of the verb to \isi{Aspect}, I identified the basic Event Structure with which a lexical item is associated as the mechanism by which an item may be labelled as either [+Change], i.e.: a \isi{Transition} or \isi{Process}, or [\textminus Change], i.e.: a \isi{State} (\citealt{Pustejovsky1988,Pustejovsky1991}). In accounting for \isi{dual aspectual forms} I looked at these items as forming part of the general group of property items in CECs and proposed a classification of this group of items based on the aspectual behaviour they display. My findings in this area revealed two distinct classes of items: Those that are inherent Transitions and associated with the feature [+Change] and those that are inherent States and associated with the feature [\textminus Change]. Note that there are no \isi{dual aspectual forms} which are inherent Processes. 

The categorisation that I propose is not just based on the syntactic behaviour of these items (i.e.: their ability to appear in Non-stative use) but also on the semantic behaviours that they display in Non-stative use. Thus I note a clear group of items which appear in Non-stative use and are inherently Non-stative (Class 1: JC\il{Jamaican Creole} \textit{raip} `ripe', \textit{wet} `wet', \textit{sik} `sick', \textit{ded} `dead', \textit{jrai} `dry', etc.) and likewise a group of items which never appear in Non-stative use and are consequently \isi{Stative} (Class 2a: JC\il{Jamaican Creole}: \textit{saaf}  `soft’, \textit{haad} `hard', \textit{chupid} `stupid', \textit{oul} `old', \textit{fat} `fat', etc.). However, the crucial point is the identification of two groups of items that appear in Non-stative use but which in my analysis are inherently \isi{Stative} (Class 2b: JC\il{Jamaican Creole} \textit{red} `red' and \textit{blak} `black' 2c: JC\il{Jamaican Creole} \textit{jelas} `jealous', \textit{bad} `bad', \textit{ruud} `rude', etc.). The recognition of these as separate classes is based on the semantic interpretations with which they are associated in Non-stative use. 

Items such as JC\il{Jamaican Creole} \textit{jelas} `jealous' and \textit{bad} `bad' when evaluated in Non-stative use, express a \isi{Process}. Thus, I analysed these as morphologically derived to express this Event Structure. The colour terms \textit{red} `red', and \textit{blak} `black' where they appear in Non-stative use express a Change of state\is{State!Change of} (\isi{Transition}). This particular class merits further investigations based on the idiomatic uses that may be associated with them in Non-stative use. Nevertheless I have argued for these as a class of derived Transitions as there is evidence of clearly acceptable non-idiomatic uses. 

The notion of logical opposition and the distinction between an opposition of contrariety and an opposition of contradiction was critical in identifying the difference between those items which are inherent Transitions and those which are derived. As I showed in \chapref{sec:5}, items that are Transitions express a Change of state\is{State!Change of} from one state to another. Consistent with their Event Structure, these are evaluated relative to their logical opposite. The difference between inherent and derived Transitions as I observed was that inherent Transitions such as, \textit{ded} `dead', \textit{jrai} `dry' \textit{raip} `ripe', \textit{sik} `sick', etc. were evaluated relative to an opposition that explicitly points to the original state of the items in question. Thus a Change of state\is{State!Change of} resulting in \textit{ded} `dead' implies an original state of \ALIVE, similarly for \textit{jrai} `dry' -- \textsc{wet}, \textit{raip} `ripe' -- \GREEN and, \textit{sik} `sick' -- \textsc{well}, etc. These oppositions are consistent with a relationship of contrariety. But derived Transitions establish a different kind of opposition in that they do not specify explicitly an original state for the items in question but rather what they were not. Thus, items expressing Colour, where permitted to appear in Non-stative use, are interpreted as the result of a Change of state process from a previous state that was for example `not \textsc{red}' or `not \textsc{black}'. The opposition indicated by these is that of contradiction as there is no clear commitment to the original state of the items in question. 

This analysis which provides an explanation of the behaviour of these items points to a sound basis for the \isi{Stative}\slash Non-stative distinction at the level of primitive Event Structure. Thus Bickerton’s observation of the centrality of the \isi{Stative}\slash Non-stative distinction in Creole studies is validated. Even more than this, however, the study feeds into the longstanding debate on the \isi{categorial status} of items of this type as either verbs or adjectives. The analysis which I have proposed indicates a group of items that are diverse in terms of their \isi{aspectual status} comprising those that are inherently Non-stative and those that are inherently \isi{Stative}. This analysis, when extended to a discussion of the \isi{categorial status} of these items, points to diversity in the category of these items as well -- a difference in this work when compared to others which have attempted to treat property items as a unified group. 

In summing up in this chapter, I look specifically at the implications that the analysis that I have presented has for the discussion of the \isi{categorial status} of the items in question. I will focus on this in \sectref{sec:6.1} and conclude in \sectref{sec:6.2} by looking at the contribution of this work to scholarship and scope for further study. 

\section{On the categorial status of dual aspectual forms}\label{sec:6.1}
With regards to the \isi{categorial status} of property items in CECs, my study points to a diverse group of items constituting verbs, adjectives and derived verbs and adjectives. Taking as a basic assumption that diverse \isi{aspectual status} translates to diverse \isi{categorial status}, what we see is Class 1 items as (Non-stative) verbs; these allow for derived \isi{adjectival use} in their \isi{Stative} appearances. In the case of Class 2 items, these are analysed as (\isi{Stative}) adjectives with some allowing for derivation as verbs. This analysis is distinct from that of previous authors who, as seen in \chapref{ch:3}, have attempted to treat this group of items as a monolithic group of either (\isi{Stative}) verbs or adjectives. The diversity of \isi{categorial status} that I posit, captures the diversity in the actual behaviour that can be observed for these items. This is consistent with the observation of \citet{Kouwenberg1996} who states that ``taking adjectivals as a single class of lexical items seems to be a misguided approach” (p.~8). Note however that even while positing a diversity in aspectual and \isi{categorial status}, at no point do I consider these forms to be \isi{Stative} verbs in JC\il{Jamaican Creole}- or other CECs, excepting perhaps the Suriname Creoles. 

In the sections below, I will deal with each Class of items that was identified in this study from the perspective of their \isi{categorial status}. As will be seen, my concern is not with whether these items are able to appear in both adjectival and \isi{verbal use} -- that they appear in these uses is generally accepted in the literature. I will take prima facie the appearance of a \isi{property item} in Non-stative use as the instantiation of a (Non-stative) verb. Likewise, I will take the appearance of a \isi{property item} in \isi{Stative} use as an instantiation of an adjective. My intention, as it was in the analysis of the \isi{aspectual status} of these items, will be to provide an account of why it is that these items can appear in such uses. I look at each Class of items in turn.

\subsection{Class 1 property items as Non-stative verbs}\label{sec:6.1.1}

Class 1 property items are those which, as indicated in \chapref{sec:5}, not only allow for both \isi{Stative} and Non-stative expressions but in Non-stative use show a Change of state\is{State!Change of} within a \isi{logical opposition of contrariety}. The example in \REF{ex:6:1} below, which features the JC\il{Jamaican Creole} item \textit{wet\slash jrai}  `wet\slash dry’ recalls example \REF{ex:5:10} from \chapref{sec:5} for ease of reference: 

\ea%1
\label{ex:6:1}
 \ea
 \gll      Di kluoz    dem \textbf{wet/jrai}.\\
 \textsc{art} clothes {\PL}        wet/dry\\
 \ea `The clothes are wet/dry.'\\
 \ex `The clothes have become wet/dry.'
 \z
 \ex
 \gll     Di kluoz      dem         a \textbf{wet/jrai} pan          di lain.\\
 \textsc{art} clothes {\PL} \textsc{asp}       wet/dry    on \textsc{art} line\\
 \glt `The clothes are drying\slash getting wet on the line.'
 \ex
 \gll Rien \textbf{wet}        di kluoz       dem pan         di lain.\\
      rain         wet \textsc{art} clothes {\PL} on \textsc{art} line\\
 \glt `Rain wet the clothes on the line.'
 \ex 
 \gll Son \textbf{jrai}         di kluoz      dem pan         di lain.\\
      sun          dry \textsc{art} clothes {\PL} on \textsc{art} line\\
 \glt `Sun dried the clothes on the line.'
 \z
\z


As seen here, items such as \textit{wet} `wet' and \textit{jrai} `dry' may express a \isi{State} as shown in (\ref{ex:6:1}a) but may also point overtly to a Change of state\is{State!Change of} in (\ref{ex:6:1}b--d). Crucially however, (\ref{ex:6:1}a--d) all entail a logical change from one \isi{State} to another. This is seen in the contrary opposition denoted by \textsc{dry} and \textsc{wet}. This Change of state\is{State!Change of} within a \isi{logical opposition of contrariety} essentially is the distinctive characteristic of Class 1 (\isi{Transition}) property items. 

Further to the \isi{semantic behaviour} of an item such as \textit{jrai} `dry', we note also its distribution where it is preceded by the aspectual marker \textit{a} in (\ref{ex:6:1}c) as opposed to (\ref{ex:6:1}a) where it appears unmarked. So far in this work, the mere ability of an item to appear with the \isi{Imperfective} \isi{aspect marker} has not held any significance outside of its function in creating a context for the evaluation of Non-stativity. In this sense, both (\ref{ex:6:1}c) and (\ref{ex:6:1}d) have the same significance and the possibility of either one of these syntactic realisations would be sufficient for the purposes of this study. However, as is evident from the discussion of works such as \citet{Winford1993,Sebba1986,Seuren1986} also \citet{Kouwenberg1996}, the syntactic behaviour of items such as \textit{jrai}  `dry’ has been a central factor in addressing their \isi{categorial status}: The acceptability of \isi{Imperfective} aspect and the possibility of transitive use provide evidence of verb status (see \chapref{ch:3}). 

In my analysis, I take into consideration the mechanism that allows a form such as \textit{jrai} `dry' to display the type of syntactic and \isi{semantic behaviour} that it does as opposed to other items which are more restricted (see \sectref{sec:6.1.2} below). This lies in the conceptual structures of these items and the \isi{event type} that is associated with them. In \sectref{sec:5.2.1}, where I discussed items displaying syntactic and semantic behaviours similar to that shown by \textit{wet} `wet', and \textit{jrai} `dry' in \REF{ex:6:1}, I discussed these as associated with an Event Structure of \isi{Transition}. I argued that a logical implication of this is the ability of these items to express both \isi{Stativity} and Non-stativity consistent with adjectival and verbal uses, respectively. This is related to the structure of the event frame associated with these items which in one instance may allow for a resultative type focus (\isi{State}\slash adjective) and in another for a focus on the Change of state\is{State!Change of} (\isi{Process}\slash verbal) meaning involved. \figref{ex:5:12} from \chapref{sec:5} is recalled here as \figref{ex:6:2} for convenience.

% % \ea%2
\begin{figure}
\caption{\isi{Transition} Event Structure \citep[56]{Pustejovsky1991}\label{ex:6:2}}
 \fbox{\parbox{5cm}{\centering
\begin{forest}
[{e$_0$\\{\footnotesize ~~~[transition]}} 
 [{e$_1$\\{[\isi{Process}]}}]
 [{e$_2$\\{[\isi{State}]}}] 
]
\end{forest}
}}
\end{figure}
% % \z

A \isi{Transition} Event Structure, as previously discussed (\sectref{sec:5.2.1}), encompasses both a \isi{State} meaning and a \isi{Process} meaning. These different appearances that are observed for an item such as \textit{jrai} `dry' may be linked to a unique classification of items of this type as (Non-stative) verbs. This is consistent with my analysis of such items as linked to a unique Event Structure that is [+Change] even though they may be used both Statively and Non-statively. In these regards, I acknowledge the instantiation of \textit{jrai} `dry' above in (\ref{ex:6:1}a) as an adjective or verb dependent on its focus: In the case of (\ref{ex:6:1}ai) the focus is on the \isi{State} (e\textsubscript{2}) and adjectival while in (\ref{ex:6:1}aii) the focus is on the \isi{Process} (e\textsubscript{1}) and verbal. From the perspective however, of syntactic use as linked to an abstract lexico-semantic specification, both the verbal and \isi{adjectival use} may be associated with a unique specification of such items as verbal and Non-stative at the conceptual level. 

Note that this analysis explains the ambiguity of interpretation that is evident in the (a) examples. While \isi{Stative} meaning does not always translate into adjectival status (cf.: SM\il{Saramaccan} \textit{satu} `to be salt' in \citealt{Kouwenberg1996}) the behaviour of items such as `wet' and `dry' in JC\il{Jamaican Creole}, point to these as derived adjectives. This is consistent with the ambiguous interpretations indicated for \REF{ex:6:3} below: 

\ea%3
\label{ex:6:3}
 \ea
 \gll    di \textbf{jrai} kluoz    dem\\
 \textsc{art}       dry   clothes {\PL}\\
 \ea \glt `the dry clothes' 
 \ex \glt `the dried clothes'
 \z
 \ex
 \gll    di    kluoz    dem we \textbf{jrai}\\
 \textsc{art} clothes {\PL} what        dry \\
 \ea `the clothes that are dry'
 \ex `the clothes that are dried'
 \z
 \z 
\z

The suggestion of a \isi{Process} that is evident in the ambiguous \isi{Stative} interpretations that are shown here, provides an argument for the status of such items in \isi{Stative} use as derived adjectives (cf. \citealt{Kouwenberg1996}). 

\citegen{Pustejovsky1991} representation of a \isi{Transition} Event Structure may be modified to reflect this analysis of items in this class from the perspective of their \isi{categorial status} (\figref{ex:6:4}).

% % \ea%4
\begin{figure}
\caption{\isi{Transition} Event Structure as reflecting a Change of state\is{State!Change of} verb\label{ex:6:4}}
\fbox{\parbox{5cm}{\centering 
\begin{forest}
 [{V$_0$ [+Change]\\\relax\scriptsize [transition]}
 [{~~e$_1$\\\bfseries V\\Change of state\is{State!Change of}}]
 [{e$_2$~~\\\bfseries A\\State}]
 ] 
\end{forest}
 }}\end{figure}
% % \z

As shown here, an abstract verb with the Event Structure of \isi{Transition} may be realised as either a verb or adjective dependent on whether the syntactic focus is on the Change of state\is{State!Change of} (e\textsubscript{1}) it encodes or the result brought about by this Change of state\is{State!Change of} (e\textsubscript{2}). Based on the inherent [+Change] verbal status that is associated with this Event Structure, the adjectival uses of items associated with this \isi{event type} must be analysed as derived. Items reflecting the behaviour of JC\il{Jamaican Creole} `wet' and `dry' may thus be analysed as inherently associated with a [+Change] verbal status and derived in its appearance as an adjective.

This analysis on some level may seem counter intuitive for authors who have been focused mainly on the syntactic appearance of such items (whether verbal or adjectival). Nonetheless, it is appealing from several angles. Firstly, from the perspective of the link between the semantic and syntactic levels; this approach establishes a connection between these two areas where a particular primitive Event Structure has consequences for the syntactic behaviour that a lexical item may display. Secondly, we are able to explain the different realisations of property items as both verbs or adjectives, and the semantic link between these realisations. In the case of items displaying the behaviour of \textit{jrai} `dry' their Event Structure establishes their different realisations as associated with a single lexical item and thus with a unique aspectual and \isi{categorial status}. 

In the section below, I will assess as well the \isi{categorial status} of Class two property items from the perspective of the aspectual analysis that I have posited for them. 

\subsection{Class 2 property items as (Stative) adjectives}\label{sec:6.1.2}

Class 2 property items are sub-divided into three classes reflecting items that are incompatible with Non-stative expression (Class 2a), those that allow for a (Non-stative) Change of state\is{State!Change of} interpretation (Class 2b) and those that allow for a (Non-stative) \isi{Process} interpretation (Class 2c). The examples below (recalled from \chapref{sec:5}) reflect this:

\ea%5
 \label{ex:6:5}
 (Class 2a -- example \REF{ex:5:14} recalled from \chapref{sec:5})\\
\ea[]{ 
\gll Di siment \textbf{haad}.\\
     art cement        hard\\
\glt `The cement is hard.'\\
 {* `The cement has hardened.'}}
 

\ex[*]{
\gll Di siment a \textbf{haad}.\\
     art cement \textsc{asp} hard\\
\glt {`The cement is hardening.'}
}

\ex[*]{
 \gll    Dem \textbf{haad} di siment.\\
\textsc{3pl}         hard  the cement\\
\glt `{They made the cement hard.'}
}
\z
\z

\ea%6
 \label{ex:6:6}
(Example \REF{ex:5:20} recalled from \chapref{sec:5})\\
\judgewidth{??}
\ea[]{
\gll    Di shuuz \textbf{blak}.\\
\textsc{art} shoes       black\\
\glt  `The shoe is black.'}

\ex[??]{
\gll     Di shuuz a \textbf{blak/red}.\\
\textsc{art} shoes \textsc{asp} black\\
\glt `The shoe is getting black.'
}

\ex[??]{ 
\gll    Dem \textbf{blak/red}         di shuuz.\\
\textsc{3pl}        black/red \textsc{art} shoes\\
\glt They are making the shoe black\slash blackening the shoe.'}
\z
\z

\ea%7
 \label{ex:6:7}
 (Class 2b -- example \REF{ex:5:22} recalled from \chapref{sec:5})\\
\ea
\gll    Di mango     dem \textbf{red}.\\
\textsc{art} mango {\PL}         red\\
\glt `The mangoes are red.'

\ex
\gll    Di   mango   dem \textbf{red} pan        di chrii.\\
\textsc{art} mango {\PL}         red on \textsc{art} tree\\
\glt `The mangoes got red on the tree.'

\ex
\gll     Di son \textbf{red}        di   mango dem  pan         di chrii.\\
\textsc{art} sun        red \textsc{art} mango {\PL} on \textsc{art} tree\\
\glt `The sun reddened the mango.'
\z
 \z
 
\ea%8
\label{ex:6:8}
(Class 2c -- examples \REF{ex:5:25} and \REF{ex:5:26} recalled from \chapref{sec:5})\\
\ea
\gll Dat           de   pikni \textbf{bad/ruud}!\\
     that \textsc{loc} child           bad/rude\\
\glt `That child is a bad\slash rude child!'

\ex
\gll Dat           de pikni           a \textbf{bad/ruud} lang taim.\\
     that \textsc{loc} child \textsc{asp}        bad/rude long time\\
\glt `That child has been misbehaving for a long time.'
\z
\z 

\ea%9
\label{ex:6:9}
\ea
\gll     Dem \textbf{jelas}.\\
 \textsc{3pl}        jealous \\
\glt `They are jealous.'

\ex
\gll      Dem         a \textbf{jelas}          mi  fi         di    kyar we          mi jraiv.\\ 
 \textsc{3pl} \textsc{asp}      jealous \textsc{1sg} for \textsc{art} car that  \textsc{1sg} drive\\
\glt `They are (being) jealous\slash envious of me because of my car.'
\z
\z

As shown in these examples, the items in this Class are diverse in the syntactic and semantic behaviours that they display. As shown in \REF{ex:6:5}, an item such as \textit{haad} `hard' in JC\il{Jamaican Creole} is resistant to any Non-stative interpretation allowing neither for \isi{Imperfective} aspect marking nor \isi{transitive variation}. Items such as `red’ and `black' in JC\il{Jamaican Creole} as discussed in \chapref{sec:5}, represent those items that may allow for a Non-stative Change of state\is{State!Change of} interpretation. As purely colour terms, these appear as only marginally acceptable in JC\il{Jamaican Creole} as indicated in \REF{ex:6:6}, but \textit{red} is fully acceptable in idiomatic use as shown by the examples in \REF{ex:6:7}. Finally, there are those items such as \textit{bad,} \textit{ruud} and \textit{jelas} which appear in Non-stative use indicating a \isi{Process}. 

In spite of the variation observed in the behaviour of these items, I have argued that these all share a common Event Structure of \isi{State} shown in \figref{ex:6:10} (\figref{ex:5:18} recalled from \chapref{sec:5}).

% % \protectedex{\ea%10
\begin{figure}
\caption{\label{ex:6:10}\isi{State} Event Structure \citep[56]{Pustejovsky1991}}
\fbox{\parbox{3cm}{\centering
\begin{forest} [S [e]] \end{forest}}}
\end{figure}
% % \z}

Based on this Event Structure, items of this type represent single eventualities that are interpreted relative to no other event. A logical implication of such an Event Structure is that items of this type will only appear in \isi{Stative} (adjectival) use. However, as we have seen, this is observationally adequate only for items displaying the behaviour of JC\il{Jamaican Creole} \textit{haad} `hard' which does not vary in \isi{Stativity}. In the case of items with behaviour similar to that of \textit{red} `red' and \textit{bad} `bad' or \textit{ruud} `rude' in \sectref{sec:5.2}, I posited the introduction of elements of meaning consistent with the primitives \CAUSE, \BECOME and \DO into their Event Structure to account for their Non-stative instantiations. Thus, while in \isi{Stative} use, they reflect an Event Structure of a pure \isi{State} (\figref{ex:6:10}); in Non-stative use, they reflect a derived Event Structure of \isi{Transition} or \isi{Process} as shown in \figref{ex:6:11}.

% % \ea%11
\begin{figure}\caption{\label{ex:6:11}\figref{ex:5:24} and \figref{ex:5:27} recalled from \sectref{sec:5.2.2}. Left: An Event Structure representing derived \isi{Transition}. Right: Event Structure of \isi{Process} (\citealt[56]{Pustejovsky1991})}
\begin{minipage}{.5\linewidth}\fbox{\parbox{5cm}{\centering%
\begin{forest}
[derived transition, s sep=1cm
 [e$_1$]
 [S,tikz={\node [draw,inner sep=0,fit to=tree,circle]{};}
 [e,]
 ]
] 
\end{forest}}}\end{minipage}%
\begin{minipage}{.5\linewidth}%
\fbox{\parbox{5cm}{\centering%
\begin{forest}
[P
 [e$_1$,no edge]
 [~~...~~,roof]
 [e$_n$, no edge]
]
\end{forest}
}}\end{minipage}
% % \z
% % \z
\end{figure}

The derivation which allows for the introduction of Change in the Event Structure of similar items has been elaborated in \sectref{sec:5.2}. Consistent with the analysis that proposes derived \isi{Transition} and \isi{Process} Event Structure for items of this type, I posit here that this process also results in the lexical conversion of these items from (\isi{Stative}) adjectives to (Non-stative) verbs. Thus, in their Non-stative instantiations, these show the behaviour of Non-stative verbs expressing either a Change of state\is{State!Change of} or a \isi{Process}. 

The differences that may be observed between these and inherently Non-sta\-tive verbs come from the fact that items of this type are derived verbs. Thus for example, I point out in terms of \isi{semantic behaviour} that those items which are derived as Transitions only express an opposition of contradiction while those that are inherent Transitions reflect an opposition of contrariety. Similarly, items which are inherently associated with the Event Structure of \isi{Process} do not also allow for \isi{Stative} uses, in contrast with property items which are derived to express this Event Structure. \citet{Kouwenberg1996} observes a similar phenomenon in relation to the \isi{categorial status} of such items, noting for Berbice Dutch\il{Berbice Dutch Creole} that ``derived verbs do not have all the properties of base verbs” (p.~35).

The proposal that Class 2 property items which allow for Non-stative interpretations are associated with derived Event Structures allows for \isi{observational adequacy} regarding the behaviour of such items and has implications for the analysis of their \isi{categorial status}. Essentially, it accounts for the variation in the appearance and aspectual denotation of these items while allowing for them to be associated with the unique Event Structure of \isi{State}. This in turn allows for the proposal that items of Class 2 are associated with an inherent adjectival status even where they may vary in their appearance as (Non-stative) verbs or (\isi{Stative}) adjectives. 


\section{Contribution to scholarship}\label{sec:6.2}

This work has attempted to bring the discussion of \isi{Aspect} in CECs to the fore. As indicated in \chapref{sec:2}, this area of study has to a large extent been treated under the umbrella of TMA where the focus has been on \isi{Tense} rather than \isi{Aspect}. The primary focus on \isi{Aspect} here is in line with the observation that \isi{Aspect} may be more basic than \isi{Tense} in Creole languages (see \citealt{Alleyne1980} for example). From this perspective, my findings may be used to strengthen what has been for the most part a very strong intuition among authors on the analysis of \isi{Tense} marking in Creoles, as being dependent on the \isi{inherent aspect} of the verb (see \citealt{Winford1993}, and discussion in \chapref{sec:2}). 

My treatment of property items from the perspective of their aspectual behaviour diverges from what may be seen as the standard assessment of the \isi{categorial status} of these items in CECs, which classifies them in \isi{predicative use} as (\isi{Stative}) verbs (cf. \citealt{Sebba1986, Alleyne1980}; \citealt{Winford1993}, etc.). It is also distinct from the contrasting position of \citet{Seuren1986}, who argues for a status of these items as adjectives, distinct from (\isi{Stative}) verbs in their ability to express causation in their variation of \isi{Stativity} (see discussion of this debate in \chapref{sec:3}, \sectref{sec:3.1}). I have shown in my discussion of the debate surrounding these items that neither of these positions is tenable. Moreover, I have argued that an ``either or” analysis of property items as verbs or adjectives, as strictly Non-statives or Statives is limited and given the diversity in their syntactic and aspectual behaviours, satisfaction of both observational and explanatory adequacy requires an analysis that lends itself to an account of the flexibility that characterizes this group of items. 

As we have seen, an \isi{event structure} (ES) approach such as I have employed in this work allows for observation and account of both syntactic and semantic phenomena as associated with these items. The traditional approaches, as discussed in Chapters~\ref{ch:2} and~\ref{ch:3} of this work, have focused on their syntactic behaviours alone for a classification of their \isi{categorial status}; instead, insights into the semantic structures associated with lexical items allows for what may be seen as a bottom up analysis. Essentially what I have done is to centralize focus on the lexico-semantics of these items into the discussion, taking a basic ``semantics-prior" position in order to explain their flexibility in behaviour at the syntactic level. With a sensitivity to the \isi{syntax-semantics interface}, where it is understood that the syntactic and semantic domains interact and impact each other, this work was able to observe more and handle more objectively the variability observed in the behaviour of property items in CECs. 

When compared to previous analyses, the one I put forward here most closely resembles that of \citet{Kouwenberg1996} in the diversity of \isi{categorial status} that she notes among property items in Berbice Dutch\il{Berbice Dutch Creole}. In her treatment, she analyses these as two classes of items, the first being assigned dual category membership as both (\isi{Process}) verbs and adjectives and the second occupying only the category of adjective. This is based on the ability of the former to allow for a processual interpretation and a restriction on the latter in this respect. However, a crucial difference between our analyses is that while she argues for the group of  `adjectivals’ as adjectives with a separation between those that allow for derivation into (Non-stative) verbs and those that do not, I argue for a distinction between a group of verbs and a group of adjectives underlyingly. Both groups may be expressed as either V or A; the former due to their inherent Event Structure and the latter due to morphological derivation. 

Most previous attempts at analying \isi{dual aspectual forms} have been subject to limitations also in that they have done so from the perspective of an account of a handful of lexical items only (e.g., \citealt{Sebba1986,Seuren1986}), whereas there is a whole group of items -- so called property items -- which merit an overall account. \citet{Winford1993} rightly addressed this entire group and provided an analysis of their behaviour from the point of view of their membership in semantic classes. However, as discussed in \chapref{ch:3}, while this allows for a separation between \isi{Stative} and Non-stative items within the large group, membership in a particular semantic class does not always predict the actual behaviour of an item. Furthermore, Winford provides no clear account of those items that seem to defy a categorisation as either \isi{Stative} or Non-stative due to their appearance in both these uses. He appears to be cognisant of the weakness in his dependence on semantic classes in his analysis of property items as his later works (\citeyear{Winford1997,Winford2001}) relied less on these semantic classes but continued focus on syntactic tests. 

Nevertheless, his intuition that there was a semantic factor involved and relevant in a categorisation of these items was no doubt justified. As I have shown, the inclusion of overarching semantic features (linked to ES) is critical in treating these items. Where authors understandably focused on the ability of these items to appear in Non-stative use and other syntactic criteria (\chapref{ch:3}) as relevant for their categorisation, my inclusion of semantic features linked to ES, allowed more clearly, I believe, for an objective categorisation of this class of items as underlyingly consisting of both verbs and adjectives.\largerpage

Firstly, there is a group of items in CECs that may be characterised as Change of state\is{State!Change of} verbs based on their ability to appear in Non-stative use and the ES by which they are characterised; these I refer to as Class 1 items. Like all other property items, they also appear in \isi{Stative} use. However, they are distinct from all others based on the fact that, they, in Non-stative use, encode what I have elaborated as a \isi{logical opposition of contrariety}. They may appear in either \isi{verbal use} (which is also Non-stative in this case) or adjectival (\isi{Stative}) use dependent on the aspect of the ES that is being expressed. A similarly complex situation exists for items that I classified as Class 2, which comprises forms with the ES ``\isi{State}" This includes three types of forms: those that I characterise as strictly \isi{Stative} in their inability to appear in Non-stative use (Class 2a), but also those that, like items in Class 1, also appear in Non-stative use expressing what I identify as a \isi{Transition} (Class 2b) and also those that in similar use express a \isi{Process} ES (Class 2c). The semantic difference between the notion of a \isi{logical opposition of contrariety} and one of contradiction facilitates my identification of Class 2b items as \isi{State} items derived to express an ES of transition in contrast items in Class 1 which are inherently transition (see discussion of this in \sectref{sec:5.1.3}).

Importantly, the semantic nuances which surround this group of items demand that careful attention be paid not just to syntactic but semantic features in an attempt at their classification. While perhaps there is necessity to conduct a more in depth investigation in this vein to facilitate a more fine-grained analysis, this study is in my estimation a step along a critical path to provide a more objective analysis of these items supported by modern theoretical models. Nevertheless, from the viewpoint of the analysis undertaken in this work, an either or analysis of property items as either verbs or adjectives or based on the simple bipartite \isi{Stative}\slash Non-stative distinction as it has been applied in Creole Studies (see \citealt{Bickerton1975,Bickerton1981}) is limited and misses completely the diversity in behaviour that may characterize lexical items. 

In conclusion, in this work, I have posited an explicit account of items that appear in \isi{dual aspectual use} in CECs -- in this way shedding new light on the \isi{Stative}\slash Non-stative distinction and its application to Creole languages. Perhaps in some small way, this analysis serves to validate and reconcile the various viewpoints that have existed on this matter in the field. Based on my analysis, the dominant viewpoint which analyses these items as \isi{Stative} verbs is untenable. In essence, the ES associated with \isi{Stative} verbs is distinct from the ES that we see expressed by members of this group of items. The opposing view that restricts them to being adjectives is also untenable: It is inconsistent with the diversity in observable behaviour of these items. The ES approach has thrown new light on this question and provided a flexible analysis that may be used to generalise over the behaviours of the entire group of items. 


\section{Scope for further study}\label{sec:6.3}

It is important, given the gap that has been noted between the study of \isi{Aspect} generally and what has been undertaken in Creole studies,\footnote{This general observation has been made previously by authors such as \citet{Winford1993,Winford1997,Winford2001} and also \citet{Dahl1993}.} that more work be done in the field which reflects the nature of \isi{Aspect} as one that is compositional. Thus, for example, where various authors have spent significant energies on the matter of \isi{grammatical aspect} (which I believe to be important), there are several other elements that are involved in the domain of \isi{Aspect} which remain unexplored in Creole studies. In the case of the contribution of the verb, this has for the most part been taken for granted with a majority of authors making reference to the existence of \isi{Stative} and Non-stative verbs. However, as I have pointed out in this study, a basic conceptual question in the application of this distinction remained unanswered, thus weakening the strength of those analyses utilising this basic distinction. 

The contribution made to \isi{Aspect} by the \isi{internal argument} and other elements such as adverbials, has been noted (see \citealt{Jaganauth1987} for example), but not reflected in our studies as far as a \isi{compositional approach} is concerned. Thus, a clear picture of the specific lexical items that contribute a particular semantic feature has not emerged in Creole studies. Consequently there is a huge gap in the awareness of the complexity of \isi{Aspect} in the works of authors in the general field as opposed to those in the field of Creole studies. This work has only managed to look at one element involved in \isi{Aspect} and attempted to account for the basic intuition associated with the \isi{Stative}\slash Non-stative distinction and the problem raised by forms which appear in both \isi{Stative} and Non-stative use. 

It is my expectation that work can be done to address the different compositional levels of \isi{Aspect} in CECs. For example, the specific contribution of the \isi{internal argument} as it regards quantification and scope as well as the role of adverbials should lend insights into how Creole languages compare to other languages that have been studied in this area. Also, given the variability in the aspectual behaviour of property items within and across Creoles, there is scope for the study of their behaviour in specific Creoles. Study along these lines will give further insights into the similarities and differences among Creoles thus strengthening the impact that we may be able to have in the context of the larger theoretical field.
