\documentclass[output=paper,russian]{langsci/langscibook}
\ChapterDOI{10.5281/zenodo.4450081}

\author{Olga Nádvorníková\affiliation{Charles University}}

\title[Stylistic normalisation, convergence \& interference in translation]{Stylistic normalisation, convergence and cross-linguistic interference in translation: The case of the Czech transgressive}

\abstract{The Czech transgressive is a non-finite verb form belonging to the cross-linguistic converb category. In contrast with other converbs (e.g. Romance \textit{gerundio} or the Russian деепричастие), the Czech transgressive has a strong stylistic mark and is very rare in contemporary language. Using a parallel (multilingual) corpus and a comparable corpus of translated and non-translated Czech, the paper investigates the differences in the frequency of the transgressive in translated and non-translated fiction and non-fiction. The data show the effect of stylistic normalisation in fiction, but not in non-fiction. The results of the potential effect of cross-linguistic interference are less conclusive, indicating that a thorough contrastive analysis of different language pairs is required first. Finally, the effect of convergence was observed neither in fiction nor in non-fiction.}
\glottocodes{czec1258}
\begin{document}

\maketitle

\section{Introduction}\label{olga:int}

The Czech transgressive is part of the cross-linguistic category of converb, i.e. \textcquote[3]{haspelmath95}{a non-finite verb form whose main function is to convey adverbial subordination}. Therefore, the Czech transgressive belongs to the same category as the Romance \textit{gerundio}, English participial adjuncts in \textit{-ing}, the Russian {деепричастие} and the Polish \textit{imiesłów przysłówkowy}, which is also reflected in translations, as illustrated in example (1):

\begin{exe}
  \ex
  \begin{xlist}
    \ex Czech\label{olga:ex1}

    \gll ``Bliju, soudruhu četaři,'' odpověděl jsem \textbf{opíraje} \textbf{se} rukou o zeď.\\
    puke.\textsc{1.sg.prs} comrade.\textsc{voc} sergeant.\textsc{voc} reply.\textsc{m.sg.pst.ptcp} be.\textsc{1sg.aux} \textbf{lean.\textsc{converb.ps.impf.m.sg}} \textsc{refl} hand.\textsc{instr} against wall.\textsc{acc}\\
    \trans `{``}I am puking, Comrade Sergeant,'' I answered, leaning with one hand against the wall.' (Milan Kundera, \textit{Žert}\slash \textit{The Joke}, 1991/1969)

    \ex English

    \enquote{Puking, Comrade Sergeant,} I replied, \textbf{leaning} against the wall with one hand. (transl. David Hamblyn; Oliver Stallybrass, 1992)

    \ex French

    Je suis en train de dégueuler, camarade sergent, expliquai-je \textbf{en m'appuyant} d'une main au mur. (transl. Marcel Aymonin, 1975)

    \ex Polish

    -- Rzygam, towarzyszu plutonowy -- odpowiedział em, \textbf{opierając się} ręką o mur. (transl. Emilia Witwicka, 1999)

    \ex Russian

    \enquote{Блюю, товарищ сержант}, -- ответил я, \textbf{опираясь} рукой о стену. (transl. Нина Шульгина, 1999)

    \ex Spanish

    \enquote{Vomito, camarada sargento}, le respondí \textbf{apoyándome} con la mano en la pared. (transl. Fernando de Valenzuela, 1994)
  \end{xlist}
\end{exe}

However, in contrast with the other converbs, the Czech transgressive has a strong stylistic mark (bookish or even archaic), is used rarely and only in written texts. 

This paper aims to investigate whether translated and non-translated Czech texts differ in the frequency of the transgressive. We assume that a higher frequency of transgressives in translations in comparison with non-translated texts may be attributed to the cross-linguistic influence (in translations from languages using converbs). The opposite result, i.e. a lower frequency of transgressives in translations than in non-translated texts, may be explained by the tendency towards normalisation. We assume also that a higher tendency to convergence may be manifested by a lower coefficient of variation of the frequency of the transgressive in translations. 

The analysis is conducted on the InterCorp parallel (multilingual) corpus; a comparable corpus of translated and non-translated Czech, named Jerome; and a large monolingual synchronic corpus of Czech (SYNv8). All the corpora are limited to fiction and non-fiction. InterCorp and Jerome are used for comparison of translated and non-translated texts; the SYNv8 corpus (restricted to non-translated texts only) serves as a reference corpus for the other two corpora under analysis.

The paper is organised as follows. \sectref{olga:tb} introduces the twofold theoretical background of the research: the morphological, semantic, syntactic and stylistic properties of the Czech transgressive in the framework of the cross-linguistic category of converbs (\sectref{olga:czt}) and the theory of special features of translated language, namely normalisation, convergence and cross-linguistic interference (\sectref{olga:trt}). After presenting the corpora used in this research (\sectref{olga:data}), we introduce the results (\sectref{olga:ana}): the analysis of the potential influence of normalisation and convergence in translations, and subsequently, the potential impact of cross-linguistic interference. In the final part of the paper, we summarise the main outcomes of the research and suggest some open questions for future research.

\section{Theoretical background}\label{olga:tb}

\subsection{The Czech transgressive}\label{olga:czt}

\subsubsection{The Czech transgressive as a converb}\label{olga:cztc}

According to \textcite[431]{nedjalkov98}, most European languages have one or two converbal forms, whereas languages outside Europe often have several converbal forms \parencite[e.g. Japanese, see][]{alppod95}.\footnote{For instance, \textcquote[431]{nedjalkov95}{the average number of converbs in languages spoken within the territory of Russia is more than seven converbal forms per language}.} According to \textcite[430]{nedjalkov98}, polyconverb languages within Europe are Basque, Finnish and Lithuanian with six converbs each. Converb-free languages in Europe are rare, for example, Maltese and Romani \parencite{nedjalkov98}.

Due to their non-finite form (see the definition in \sectref{olga:int}), converbs convey the adverbial meaning in a more condensed way than the corresponding finite subordinate adverbial clause \parencites[see][]{vachek55}{nosek64}[for Romance, for instance,][]{ceretal20}[for Czech][]{becka77}. Because of their verbal character, they are also richer in information than complete nominalisations (verbal nouns, prepositional phrases, etc.). Moreover, in contrast with coordinate finite clauses, converbs allow for hierarchisation of information; in other words, the event conveyed by the converb is considered secondary (see \ref{olga:ex1}).

Converbs may differ regarding their position in the system of the given language \parencites[see][104--106]{nedjalkov95}[433]{nedjalkov98}. Strict, canonical converbs are specialised in the converbal (adverbial) function (e.g. Russian and Czech converbs, the French \textit{gérondif} and Estonian forms in \textit{-des}, etc.). By comparison, non-strict converbs fulfil, apart from the converbal function, other functions, e.g. participial or infinitival, such as the English forms in \textit{-ing}, German \textit{-end} or Spanish and Italian \textit{-ndo} \parencites[see][425]{nedjalkov98}[for Romance, see][]{ceretal20}.

From the syntactic point of view, it should be noted that as non-finite verb forms, converbs do not have a valence position for the subject. In some languages, the subject (controller) of the converb has to be coreferential with the subject of the main clause, like in Slavic languages, including Czech \parencites[\textit{same-subject converbs}, see][425]{nedjalkov98}[153]{komarek86}\footnote{\textcite[37--45]{dvorak70}, in his diachronic study of Czech, points out that 30\% of transgressives in the 17\textsuperscript{th} century were non-coreferential.} or French \parencite[\textit{gérondif}, see][1252]{gregoo16}. In other languages, the controller of the converb may be non-coreferential, as in Portuguese, Spanish \parencite[see][111]{ceretal20} or in Modern Greek, Armenian and Albanian \parencite[425]{nedjalkov98}.

Finally, concerning the semantic interpretation of converbs, we can distinguish contextual and specialised converbs \parencite[431]{nedjalkov98}. Specialised converbs only have one or two specific meanings \parencite[e.g. Finnish has a specialised converb conveying manner, see][443]{nedjalkov98}. However, most European languages, including Czech, belong to the contextual converb type, i.e. their converbs are semantically vague, the potential meanings are numerous and given by the context \parencites[for the factors influencing the semantic interpretation of converbs, see for example][]{koenig95}[337]{koeauw90}[29--41 for Czech]{dvorak83}[and for French][]{nadvornikova12}.\footnote{\textcite[157]{moortgat78} considers the French gerund to be a \enquote{semantic chameleon} \parencite[see also][]{halmoy03}.}

The meanings conveyed by contextual converbs can be divided into two large groups: temporal (simultaneity, anteriority and posteriority) and non-temporal (manner\slash means, cause, concession, condition etc.). The meaning of simultaneity proper or concomitance/attendant circumstance is the default meaning: \enquote{if a European language has only one converb, then it is a mixed converb of contextual Simultaneity} (\cite[432]{nedjalkov98}; see a similar observation for French \textit{gérondif} in \textcites[117]{kleiber07}[19]{kleiber09}). This observation is corroborated by a diachronic trend: \textcquote[437]{nedjalkov98}{If a language moves from the group of two-converb languages to the group of mono-converb languages then the remaining converb will belong to the contextual Simultaneity type}.

Despite the aforementioned variety in the subtypes, all converbs share the basic definition as a \enquote{non-finite verb form conveying the adverbial subordination} (\cite[3]{haspelmath95}, see \sectref{olga:int}). Most European converbs share other properties, in particular, the contextual semantic interpretation and the use as a means of syntactic condensation (\textit{he said and he smiled -- he said, smiling -- he said, with a smile}). More specifically, Romance and Slavic converbs are considered typical (see \cite[45]{haspelmath95} for the former and \cite[422]{nedjalkov98} for the latter). According to \textcite{nedjalkov98}, Slavic converbs are prototypical for the category, especially the Russian converb. Germanic languages, conversely, make, except for English, \enquote{only parsimonious use of converbs} (\cite[72]{koenig95}, and a similar statement in \textcite[192]{kortmann97}). According to Kortmann, in English, \enquote{free adjuncts are far from playing a minor role} and the frequency of adverbial participial clauses is five times higher in English than in German \parencite[192]{kortmann97}. Similar differences in the use of converbs can be observed in the Slavic and Romance language families. In Slovak, the frequency of the converb is much lower than in Russian \parencite[25]{brtkova04}. Similarly, in French, the frequency of \textit{gérondif} is several times lower than the frequency of the corresponding forms in Italian, Spanish and Portuguese (1,571 instances per million words (ipm) against 4,098 ipm, 4,886 ipm and 6,939 ipm respectively, see \cite[116]{ceretal20}).

The Czech transgressive displays most of the properties considered by Nedjalkov as defining the prototypical (Slavic) converb: it is syntactically strict, i.e. it may be used only in the adverbial (converbal) function; it is formally simple, i.e. its formation involves suffixes, not adpositions; it has two converbal forms, one of which is a converb of contextual Simultaneity and the other as a converb of contextual Anteriority; and it is referentially the same-subject (co-referential). However, the Czech converb shows one important particularity: it maintains a very complex, archaistic morphology, involving adjectival agreement markers (in gender and number) (see \sectref{olga:mor}), and, as a consequence, it acquires a strong stylistic mark and is used very rarely in contemporary language (see \sectref{olga:sty}).

\subsubsection{Morphological features of the Czech transgressive}\label{olga:mor}

As mentioned above, the Czech transgressive has two main forms \parencite{karlik17}: 

\begin{itemize}
  \item the \enquote{present transgressive} (\textit{přechodník přítomný} or \textit{-c transgressive}), formed with imperfective verbs only and conveying the meaning of simultaneity.
  \item the \enquote{past transgressive} (\textit{přechodník minulý} or \textit{-š transgressive}), formed with perfective verbs only and conveying the meaning of Anteriority.
\end{itemize}

When influenced by contextual factors (see \ref{olga:cztc}), these basic meanings may acquire other nuances, such as manner, cause, condition, etc. \parencite[33]{dvorak83}. Each form (present and past) has a different set of morphemes, varying according to the type of verb stem and as the consequence of agreement with the subject (controller) of the transgressive in gender and number, as mentioned above. Table~\ref{olga:t1} summarises this complex morphology of the Czech transgressive.

\begin{table}\caption{Morphology of the Czech transgressive}\label{olga:t1}
  \begin{tabularx}{\textwidth}{Xlp{37mm}p{37mm}}
    \lsptoprule
    \multicolumn{2}{l}{\multirow{2}{35mm}{Czech transgressive forms}}  & \multicolumn{2}{c}{Form}\\
                            &                                       & Present                                               & Past\\
    \midrule
    \multicolumn{2}{l}{\textsc{m.sg}}                               & -a / -e / -ě                                          & -${\emptyset}$ / -v\\
    \multicolumn{2}{l}{\textsc{f.sg} + \textsc{n.sg}}               & -ouc / -íc                                            & -ši / -vši\\
    \multicolumn{2}{l}{\textsc{pl} (\textsc{m+f+n})}                & -ouce / -íce                                          & -še / -vše\\
    \midrule
    \multirow{2}{*}{Aspect} & Imperfective                          & \textbf{Present (Simultaneity) \textsc{conv.ps.impf}} & Simultaneity/Anteriority \textsc{conv.pt.impf}\\
                            & Perfective                            & Futurate \textsc{conv.ps.pf}                          & \textbf{Past (Anteriority) \textsc{conv.pt.pf}}\\
    \lspbottomrule
  \end{tabularx}
\end{table}


\tabref{olga:t1} demonstrates four possible combinations of aspectual and formal characteristics of the Czech transgressive. Conv.ps.impf and Conv.pt.pf are the dominant forms, mentioned by most Czech grammars \parencites[e.g.][154]{komarek86}[148--249]{cvrcek10}[or][335--337]{karetal95}.\footnote{\textcite[437]{nedjalkov98} gives for the Czech converb conveying simultaneity the endings -a/-je/-oic, which is incorrect.} For Conv.ps.impf, see \ref{olga:ex1}; for Conv.pt.pf, see \ref{olga:ex2}.

\begin{exe}
  \ex \textbf{Uslyšev}, že Švejka naznačil plukovník ordonancí u 11. kumpanie, zvolal: \enquote{Pomoz nám pánbůh.} (Jaroslav Hašek, \textit{Osudy dobrého vojáka Švejka za světové války}, 1921–1923/1996)\label{olga:ex2}
  
  hear-\textsc{conv.pt.pf.m.sg}

  `\textbf{Having heard} that the colonel marked Švejk to be the ORDONANC with the 11\textsuperscript{th} company, he yelled out: ``Help us Lord God.''' (transl. Zdenek K. Sadloň)
\end{exe}

Conv.pt.impf was already rare in Old Czech \parencite[115]{dvorak70}; in contemporary Czech, it is not in use \parencite[154]{komarek86}. Finally, the form of the futurate transgressive (Conv.ps.pf) is supposed to be used only to convey anteriority in the future, i.e. combined with the main verb in the future tense \parencites[see][126]{nedjalkov95}[154]{komarek86}. However, these forms were replaced by past converbs \parencite[Conv.pt.pf, see][261]{oktabec53} and are not in use either.

Even when limited to Conv.ps.impf and Conv.pt.pf, the morphology of the Czech converb is very complex and contrasts with the converbal systems in other Slavic languages, where the converb went through a process of adverbialisation and the forms were simplified, in particular by dropping the agreement with the subject (controller). In contemporary Czech, the only non-coreferential converbs are the grammaticalised ones: as much as in other languages \parencite[see][27--41]{haspelmath95}, Czech converbs may be reanalysed in other categories, mainly adverbs (e.g. \textit{chtě nechtě} `reluctantly') and prepositions (e.g. \textit{počínaje} `starting with' or \textit{nemluvě} `notwithstanding', see \cite[156]{komarek86}). 

Some languages maintained different sets of suffixes for the past (perfective) and the present (imperfective) forms (e.g. Russian and Polish). Other languages, for instance Slovak \parencite[487]{dvonc66}, went further in the process of simplification and use the same set of suffixes for the perfective and imperfective verbs. The same tendency can be observed in Czech dialects \parencites[55--56]{dvorak83}{michalkova63}, which demonstrates that spoken, non-standard Czech also adverbialised the transgressive.

This difference between standard literary Czech and its dialects (and other Slavic languages) is caused by a normative intervention made by grammarians during the Czech National Revival movement in the first third of the 19\textsuperscript{th} century. At that time, the use of the Czech language was limited, since the language of economic and cultural elite was German, and Czech was spoken mostly by the rural population and the poorer inhabitants of cities. Therefore, while choosing the norm for the Czech language to be resuscitated, the grammarians and lexicographers of the National Revival movement did not opt for the language of their time (the 19\textsuperscript{th} century), which was considered unprestigious and decayed, but the norm of the flourishing period of the Czech state, culture and language at the end of the 16\textsuperscript{th} century, i.e. the language that was more than 200 years old at that time. 

More importantly for our topic, the newly defined norm of standard literary Czech re-introduced in the transgressive its old complex morphology of the literary norm of the end of the 16\textsuperscript{th} century. Since the transgressive was mostly used in written texts, especially for its advantages as a means of syntactic condensation, the norm was respected. Nevertheless, the transgressive gradually ceased to be part of the internalised, unconscious competence of the speakers; its frequency was in constant decline and the form acquired its stylistic mark. 

\subsubsection{Stylistic features and frequency of the Czech transgressive}\label{olga:sty}

Extensive research into transgressives conducted by \textcite[60]{dvorak83} demonstrates the constant decrease in the frequency of this form between 1781 and 1978 (from 6.49\% of all verb forms in the period 1781–1830 to only 0.14\% in 1971–1978). He observes the decrease in the frequency of the transgressive in the 18\textsuperscript{th} century already \parencite[142]{dvorak70}, which indicates that the normative intervention during the National Revival movement may not have been the main factor triggering the decrease of the frequency of this form in Czech. Nevertheless, it is plausible to assume that in the 20\textsuperscript{th} century, the archaistic morphology and stylistic mark resulting from the normative intervention contributed considerably to the retreat of this form. The most recent grammar of Czech, published in 2010 and based on corpus data, states that the transgressive is \enquote{very rare} and that it represents less than 1\% of the verb forms in Czech \parencite[249]{cvrcek10}.

It is worth noting that there is a neat difference in the frequency of the two main forms of the transgressive: \textcite[130]{cvrkov11} indicate that the frequency of Conv.ps.impf is nowadays less than 0.1\%, but the frequency of Conv.pt.pf is even less than 0.01\% of all verb forms, which means that the ratio of the two forms is 10. A similar difference in the frequency of the present and the past transgressive was already observed by \textcite[60]{dvorak83}: 0.34\% Conv.ps.impf and only 0.04\% Conv.pt.pf of all verb forms in texts published between 1960 and 1970 (ratio 8.5). Conversely, in the period of 1781–1830, Dvořák observed 4.17\% of Conv.ps.impf and 1.39\% of Conv.pt.pf, i.e. only a ratio of 3.\footnote{Dvořák also indicates the frequencies of the two remaining forms of the trangressive in 1781--1830: 0.03\% for Conv.pt.impf and 0.9\% for Conv.ps.pf \parencite[69]{dvorak83}.} Even though the exact figures given by Dvořák for the different time spans may not be fully reliable, due to the lack of comparability of the sub-corpora under analysis, the tendency is clear: Conv.pt.pf is systematically less frequent than Conv.ps.impf.

The difference in frequency between the two main forms of the transgressive may be ascribed not only to the specific morphology of Conv.pt.pf (see \tabref{olga:t1}), but also to the differences in the meaning of the two forms and the availability of competing forms in the language. Conv.pt.pf, conveying the meaning of anteriority, is strongly concurred by other forms, especially finite subordinate clauses of a temporal or a specific adverbial meaning (e.g. the cause, as in example \ref{olga:ex2}). Conv.ps.impf, by contrast, mostly conveys a simple accompanying circumstance (see example \ref{olga:ex1} or \enquote{\textit{řekl jsem \textbf{usmívaje se}}} - \textsc{conv.ps.impf.m.sg} `I said \textbf{smiling}'), which cannot be expressed by a subordinate clause, but only by a coordinate one (\enquote{\textit{řekl jsem \textbf{a usmíval jsem se}}} `I said \textbf{and I was smiling}') or a simple SP (\enquote{\textit{řekl jsem \textbf{s úsměvem}}} `I said \textbf{with a smile}').

\textcite[281]{kortmann97} made a similar observation for most European languages: they do not explicitly encode the meaning of concomitance (by an adverbial subordinator), so this meaning is mostly conveyed by converbs or a simple juxtaposition of two finite clauses. Even though the replacement of the converb by a subordinate clause moves the form from non-finite to finite and explicates its meaning by a subordinator (see \cite{nadvornikova17}), in contrast with the coordinate clause, it maintains the adverbial subordination relation and hence the hierarchisation of events typical for converbs (see the definition in \sectref{olga:int}). As a consequence, the coordinate clause is a less obvious concurrent of the converb than a subordinate one, and the meaning of the accompanying circumstance is more likely to persist in this form than more specific adverbial meanings. Furthermore, as remarked by \textcite{nedjalkov95}, the accompanying circumstance is the most frequent meaning conveyed by converbs in general (see the same observation for Romance languages in \textcite[122]{ceretal20} and for Czech in \textcite[33]{dvorak83}).

As mentioned above, the archaistic morphology of the transgressive is also the source of its specific stylistic mark. Most Czech grammars consider the transgressive as bookish (Conv.ps.impf) or even archaic (Conv.pt.pf), and limited to the written language \parencites[154]{komarek86}[249]{cvrcek10}[337]{karetal95}. The stronger stylistic mark of Conv.pt.pf correlates with the aforementioned lower frequency. 

The bookish\slash archaistic stylistic mark of the transgressive also influences its frequency in different text registers. Most sources agree that the transgressive is typical for fiction \parencites[105]{dvorak83}[24]{becka77}[102]{cecetal97}, in particular because of its ability to convey in a condensed way the accompanying circumstance in narrative sequences and introductory clauses \parencites[107]{dvorak83}[19 and example \ref{olga:ex1}]{becka77}. The stylistic mark in fiction is also exploited in historical novels or as a means of irony or parody \parencites[102--103]{cecetal97}[154]{komarek86}. However, in fiction intended for children or young readers, transgressives are less frequent than in fiction for adults \parencite[see][90]{jeletal61}.\footnote{\textcite[23]{becka77} also mentions the potential influence of a specific author’s idiolect (e.g. the Czech author Vladislav Vančura, 1891–1942, is known for his penchant for transgressives).} In non-fiction, the transgressive is considered less frequent than in fiction and conveys more specific adverbial meanings than a simple accompanying circumstance \parencites[see][33]{dvorak83}[21]{becka77}[337]{karetal95}.\footnote{\textcite{karlik17} and \textcite{cecetal97} consider the transgressive also appropriate in highly formal, e.g. diplomatic or legal, documents.} \textcite[106 and 108]{dvorak83} points out that the transgressive is more frequent in social sciences than in natural or technical sciences. Finally, in journalistic texts, the transgressive is the least frequent, in comparison with fiction and non-fiction \parencites[106]{dvorak83}[90]{jeletal61}.

\subsection{The transgressive in translations}\label{olga:trt}

To our knowledge, only a few researchers have focused specifically on the use of transgressives in translation, apart from three rather dated studies \parencites{becka77}{dvorak72}{dvorak83}. However, the topic is occasionally addressed in contrastive studies exploring Czech equivalents of converbs.

In his quantitative study, \textcites{dvorak72}{dvorak83} analysed various Czech translations of the same source texts (four source texts in Russian, one in French and one in English). The translations were published between 1863 and 1975 and six different translations on average were analysed for each text. The results confirmed the decrease in the frequency of the transgressive observed in non-translated texts (see \sectref{olga:sty}), but the normalised frequency of transgressives was almost always higher in translations than in non-translated texts from the corresponding period. For instance, in the Czech translation of Charles Dickens' \textit{The Posthumous Papers of the Pickwick Club (chapters 1-5)} published in 1925, transgressives represent 5.5\% of all verb forms, whereas the average for the given period in non-translated texts is only 1.384\% \parencite[94]{dvorak83}.

These results suggest that in translations from languages using converbs (i.e. most European languages, see below), the effect of cross-linguistic interference (or \textit{shining through}, see \sectref{olga:trt}) may be expected and the frequency of transgressives may be higher in translated than in non-translated texts. However, other studies indicate the opposite conclusion.

First, in their contrastive research of Czech equivalents of Romance converbs, \textcites{cernad15}{ceretal20} show that in Czech translations from four Romance languages (French, Italian, Portuguese and Spanish), the transgressive represents the least frequent counterpart (from 2.0\% in translations from French to 9.6\% in translations from Portuguese), despite the presumed systemic equivalence. In comparison, the finite counterparts (coordinate and subordinate clauses) form about 70\% of the whole. \textcite[240]{malsal15} present a similar result in translations from English: the transgressive represents only 2.1\% of the Czech counterparts of English adverbial participles; the overwhelming majority of counterparts being finite verbs (73\%). Finally, in translations from Russian \parencite{kockova11}, transgressives constitute less than 1\% of the equivalents of (past) деепричастия. These results suggest that in translations, the frequency of the transgressive may be as low as in non-translations, or even lower.

Second, \textcite[26]{becka77} in his (non-quantitative) analysis of the transgressive in translations points out that translators had been warned against the use of the transgressive and that they avoid it because of its stylistic mark. Similarly, \textcite[51]{levy11} states that the frequency of the transgressive in Czech is lower in translations than in non-translated texts, because translators are over-concerned to avoid stylistically marked features. These observations indicate that, on the contrary, the frequency of transgressives may be influenced by the effect of stylistic normalisation and, therefore, be lower than in non-translated texts.

\textcite{toury95} states that translations are governed by two universal laws: the law of interference and the law of growing standardisation (or \textit{normalisation}, according to \textcites{baker93}{baker96}). Cross-linguistic interference (or \textit{shining through}, according to \textcite{teich03}) consists of transferring linguistic features of the source language into a target language \parencite[see][274--279]{toury95}. Normalisation, by contrast, may be defined as \textcquote[176--177]{baker96}{the tendency to conform to patterns and practices that are typical of the target language, even to the point of exaggerating them}.

Various studies have shown the effect of cross-linguistic interference in translation in various language pairs. For instance, \textcite{daixia11}, when analysing Chinese texts translated from English, found that passive voice is more frequent in Chinese translated from English than in non-translated Chinese texts. Similarly, \textcite{cappelle12} shows that English texts translated from French contain fewer manner-of-motion verbs than English texts translated from German. He explains this effect by the typological differences between the two source languages: German and English are satellite-framed languages, whereas French is a verb-framed language. 

As for normalisation, this is defined by the linguistic properties as well as the sociocultural norms of the target language \parencite[see][17]{lefvog13}. Alongside the explicitation, the simplification and the levelling out (convergence, \cite{laviosa02}) it is one of the specific features of translation (\enquote{translation universals}, according to \textcites{baker93}{baker96}) that is addressed the most in literature. In Chesterman's \parencite*[39]{chesterman04} terms, it can be conceived either as an S-universal, causing differences \enquote{between translations and their source texts}, or as a T-universal, giving rise to differences between translations and comparable non-translated texts in the target language. In this study, we focus on the effect of normalisation as a T-universal, by comparing non-translated Czech texts with translations in the same language. We also partially focus on the convergence (levelling out).

Normalisation as a T-universal was analysed by \textcite{deletal12}; their results confirmed the tendency to normalisation (standardisation) in translated Dutch, in comparison with non-translated texts of the same language. Similarly, \textcite{chlumska17} observed the effect of normalisation in translation in the choice of two forms of the verb \textit{say} in Czech: \textit{říci} (formal, stylistically marked form) and \textit{říct} (standard, stylistically neutral). In her corpus, translations showed a higher frequency of the latter form than non-translations, which suggests the effect of (stylistic) normalisation \parencite[65]{chlumska17}. \textcite{lapshinova18}, who compared translations from English into German (in six different text registers), observed that the effect of normalisation is sensitive to two factors: text register (the highest score of normalisation was in translations of fiction) and the translator’s proficiency (the normalisation score was higher in student translations than in professional translations).

Levelling out (or convergence, \cite{laviosa02}) is sometimes considered as a sub-type of normalisation. \textcite[177]{baker96} defines levelling out as \enquote{the tendency of translated text to gravitate around the centre of any continuum rather than move towards the fringes}. \textcite[71]{laviosa02} is more specific and points out that the convergence implies a relatively higher level of homogeneity of translated texts concerning certain linguistic features, such as lexical density, sentence length, etc. As stated by \textcite[184]{baker96} and by \textcite[104]{chlumska17}, less attention has been paid to this feature than to the other translation universals as it is more difficult to operationalise. \textcite{lapshinova15} confirmed the tendency to convergence in several translation variants in German (translated from English). \textcite[104--121]{chlumska17} analysed various potential indicators of the convergence (sentence length, type-token ratio, etc.) in Czech and observed its effect in translations of fiction but not in non-fiction.

\subsection{Hypotheses and research questions}\label{olga:hyp}

Our main research question aims to find out the differences in the frequency of transgressives in translated and in non-translated texts. Based on the theory of the interplay between the cross-linguistic interference and the normalisation in translation (see \sectref{olga:trt}), we can formulate the following hypotheses (H\textsubscript{1} and H\textsubscript{2} being in opposition):

\begin{description}
  \item[H\textsubscript{0}] Translated and non-translated texts of the same text register do not differ in the frequency of transgressives.
  \item[H\textsubscript{1}] Due to the effect of cross-linguistic interference, in translations from languages using converbs, the frequency of transgressives is higher than in non-translated texts of the same text register. Based on typological observations made in the literature (\cite{haspelmath95}, \cite{nedjalkov95}, see \sectref{olga:cztc}), we expect more transgressives in translations from Romance and Slavic languages, especially Russian, than in translations from Germanic languages (with the potential exception of English). Transgressives resulting from this interference can also be expected in translations from polyconverbal Latvian, Finnish and Japanese.
  \item[H\textsubscript{2}] Due to the effect of (stylistic) normalisation, the frequency of transgressives is lower in translations than in non-translated texts of the same text register (independently of the source language and the text register).
\end{description}

Based on the theory of convergence (see \sectref{olga:trt}), we can formulate the third hypothesis:

\begin{description}
  \item[H\textsubscript{3}] Due to the tendency of translations towards convergence (greater homogeneity), the coefficient of variation of the frequency of transgressives in translations is lower than in non-translated texts of the same text register.
\end{description}

Our second research question aims to find out what other factors influence the frequency of transgressives in translated and non-translated texts. From a strictly linguistic point of view, we expect an important difference in frequency between the two forms of the transgressive since the past form (Conv.pt.pf) is stylistically more marked than the present form (Conv.ps.impf). Among the extra-linguistic factors, we expect the greatest influences to be the date of publication of the text (the older the text, the higher the frequency) and the text register (more transgressives in fiction, exploiting its stylistic mark and the ability to convey accompanying circumstance, and fewer transgressives in non-fiction, more stylistically neutral than fiction).

\section{Data and Methods}\label{olga:data}

As mentioned in \sectref{olga:tb}, the main source of data for our research is corpora including translated texts: the comparable corpus of translated and non-translated Czech named Jerome \parencite{chlumska13}, and the InterCorp parallel (multilingual) corpus \parencite{rosetal19}. The data obtained from these corpora are confronted with the data extracted from the large monolingual synchronic corpus of Czech named SYNv8, limited to non-translated texts \parencite{kreetal19}.

All these corpora were created by the Institute of the Czech National Corpus and are freely available using the same corpus interface (KonText; \url{www.korpus.cz} and \url{http://kontext.korpus.cz}). InterCorp was annotated using the POS-tagger named Morče (see \url{http://ufal.mff.cuni.cz/morce/index.php}); SYNv8 and Jerome were annotated by a hybrid system developed in-house \parencite[combining stochastic and rule-based disambiguation, see][]{hnaetal11}.

The error rate of the transgressive POS-tagging is 3.6\% for the Conv.ps.impf and 6.8\% for the Conv.pt.pf (tested on a sample of 250 occurrences for each form).\footnote{In the corpus queries, we excluded from the analysis the most frequent grammaticalised forms of the transgressive in the two sub-corpora. The resulting queries are:\\{}
[tag="V(e|m).*" \& word!="((N|n)e)?(C|c)ht(ě|íc)|(N|n)emluvě|(P|p)očínaj(e|íc)|(K|k)onč(e|íc)|(N|n)\\
evyjímaj(e|íc)|(T|t)ak říkajíc|(S|s)oudě"] for fiction and\\{}
[tag="V(e|m).*" \& word!="((N|n)e)?(C|c)ht(ě|íc)|(N|n)emluvě|(P|p)očínaj(e|íc)|(K|k)onč(e|íc)|(N|n)\\evyjímaj(e|íc)|(T|t)ak říkajíc|(S|s)oudě|(V|v)ycházej(e|íce)|(N|n)epočítaj(e|íc)"] for non-fiction.} Even though these POS-tagging errors can influence the resulting frequencies, we consider our observations reliable, since the main information for our research is not the absolute frequencies of the transgressive, but the comparison of the relative ones.

\subsection{The Jerome comparable corpus of translated and non-translated Czech}\label{olga:je}

\textit{Jerome} \parencites{chlumska13}{chlumska17} is a monolingual corpus specifically designed for the research of translation features (in terms of T-universals, see \textcite{chesterman04} and \sectref{olga:trt}), in fiction and non-fiction. The corpus comprises translated and non-translated texts in Czech, the two subcorpora being comparable in size and other relevant factors. For instance, all the texts were published between 1992 and 2009 and the same author\slash translator may be represented by a maximum of three texts to prevent the risk of the influence of a specific idiolect.

The representation of source languages in the translation part of the corpus reflects the situation on the publishing market in the Czech Republic, where translations from English are three times more frequent (i.e. probably three times more read) than from any other language.\footnote{According to the Czech National Library statistics of translated books, available (in Czech) at \url{http://text.nkp.cz/sluzby/sluzby-pro/sluzby-pro-vydavatele/vykazy}, for more details, see \textcite[313]{cvrchl15}.} In total, it includes 22 different source languages in fiction and 15 source languages in non-fiction. The potential interference effect can be explored using a smaller balanced subcorpus including an equal amount of texts translated from 14 different languages in fiction and 6 languages in non-fiction. \tabref{olga:t2} summarises the composition of the Jerome comparable corpus:

\begin{table}\caption{Composition of the Jerome comparable corpus of translated and non-translated Czech}\label{olga:t2}
\fittable{
  \begin{tabular}{p{24mm}rrp{37mm}}
    \lsptoprule
    \textbf{Jerome corpus} & \multicolumn{2}{l}{\bfseries N\textsuperscript{o} of tokens (incl. punctuation)} & \textbf{Source languages in translation}\\
                  & translated              & non-translated                                      & \\
    \midrule
    Fiction\newline (non-balanced) & 26,617,523 & 26,551,540 & da nl en fi fr de el(new) he hu is it ja no pl pt ru sr sk sl es sv\\
    \tablevspace
    Non-fiction\newline (non-balanced) & 15,946,319 & 15,949,930 & ar en fr de el(old) hu it la pl ro ru sr sk es sv\\

    \tablevspace
    Fiction\newline (balanced) & 1,765,433 & 1,768,079 & da nl en fi fr de is it ja pl pt ru es sv\\

    \tablevspace
    Non-fiction\newline (balanced) & 774,610 & 779,288 & en fr de it pl ru\\
    \lspbottomrule
  \end{tabular}
  }
\end{table}

In the whole (non-balanced) corpus, translations from English represent 69\% of the subcorpus of translations (by the number of tokens). The other languages represented by at least 500,000 tokens are French and German (8\% each) and Russian and Polish (3\% and 2\% respectively). In non-fiction, the composition is similar: translations from English represent 55\% of the sub-corpus, followed by German (25\%) and French (8\%). The other languages usually do not exceed 1\%. 

As mentioned above, all the texts included in the Jerome corpus were published between 1992 and 2009; nevertheless, some of them were first published earlier. Since the frequency of the transgressive is highly likely to be sensitive to the date of the creation of the text (see \sectref{olga:sty}), we eliminated these texts from our corpus research (14 texts in translated fiction, 24 texts in non-translated fiction, one text in translated non-fiction and three texts in non-translated non-fiction). Thus, in translated fiction, we excluded, e.g. the Czech translation of William Faulkner´s novel \textit{The Wild Palms}, first published in 1960. In non-translated fiction, the set of eliminated texts includes not only texts first published before 1992, for instance four novels by Vladislav Vančura,\footnote{\textit{Amazonský proud, Pekař Jan Marhoul, Pole orná a válečná} and \textit{Poslední soud}.} first published in the 1920s, but also texts first published in the given interval (1992–2009), but written earlier, e.g. a posthumous edition of a novel by Lev Blatný (\textit{Servus, Ser-vá-ci}), an author who died in 1930 and the memoirs of a Habsburg Empire army officer in the Great War (\textit{Z~turecké armády do britského zajetí}).

It is very likely that, especially in translations, new editions of texts published earlier were revised and adapted by editors in publishing houses. However, our data show that the inclusion of these texts in the research may have skewed the results, if not excluded. For instance, all the aforementioned eliminated texts show the highest normalised frequencies of the transgressive in our subcorpora of fiction, which corroborates the hypothesis of the influence of the date of creation/publication of the text on the frequency of the transgressive.

Furthermore, to maintain the comparability with the InterCorp parallel corpus (see below), we limited the Jerome corpus to novels and short stories in fiction and scientific (SCI) and popular (POP) texts in non-fiction (eliminating e.g. textbooks and encyclopaedias, not included in InterCorp). For these reasons, the size of the Jerome sub-corpora introduced in the results of our research (see \sectref{olga:ana}) is smaller than that given in \tabref{olga:t2}.

\subsection{The InterCorp multilingual corpus}\label{olga:intc}

InterCorp (\url{https://wiki.korpus.cz/doku.php/cnk:intercorp:verze12}) is a large multilingual (parallel) corpus currently involving 41 languages, with Czech as pivot language \parencites{cerros12}{nadvornikova16}. The corpus is composed of the so-called core, which comprises fiction and partially non-fiction, and collections (movie subtitles, the Bible, journalistic texts, Acquis communautaire and EuroParl). Our research exploits only the core of the corpus, because, in contrast with the collections, the quality of translations is higher in the core texts and all the metadata necessary for research in translation studies is available (date of publication, source language, name and sex of the author/translator, different text sizes in tokens, etc.). 

The main advantage of the InterCorp parallel corpus, in comparison with the Jerome comparable corpus, is its larger size, i.e. the larger number of texts and different authors/translators, which reduces the risk of the influence of a specific text style or an author's\slash translator's idiolect. This is also the reason why we do not use the Jerome corpus in the interference hypothesis testing (H\textsubscript{1}), as it is limited to one to three texts per language, but instead, we use the InterCorp parallel corpus (see \sectref{olga:sty}).

However, the translated and non-translated sub-corpora of the InterCorp corpus are not comparable, neither in size nor composition. As can be seen in \tabref{olga:t3}, the size of the non-translation sub-corpora in the InterCorp corpus is quite small, which is due to the limited availability of translations from Czech into foreign languages. This issue is even more pronounced in non-fiction than in fiction. In addition, the sub-corpus of non-translations in InterCorp is not limited solely by size, but also by composition, as foreign publishing houses particularly choose texts by well-known and established authors for translations from Czech. As a consequence, the non-translation subcorpora of InterCorp are not a reliable source of data for real language use in Czech.

For this reason, the data for the comparison of translations with non-translated texts were not extracted from InterCorp, but from the largest corpus of contemporaneous Czech -- SYNv8 \parencites{kreetal19}{hnaetal14}, limited to non-translated fiction (novels and short stories) and non-fiction (scientific and popular texts). \tabref{olga:t3} demonstrates the resulting size of the sub-corpora.

\begin{table}\caption{Composition of the InterCorp parallel corpus and the SYNv8 reference corpus of Czech}\label{olga:t3}
  \begin{tabular}{llrrr}
    \lsptoprule
                                  & Corpus      & \multicolumn{2}{c}{InterCorp}                             & SYNv8\\
                                  &             & translated  & non-translated   & non-translated\\
    \midrule
    \multirow{3}{*}{Fiction}      & texts (n)   & 1,179       & 286                                         & 496\\
                                  & tokens (n)  & 107,375,278 & 19,208,622                                  & 30,527,709\\
                                  & SLs (n)     & 32          & --                                          & --\\
    \midrule
    \multirow{3}{*}{Non-fiction}  & texts (n)   & 80          & 13                                          & 650\\
                                  & tokens (n)  & 6,803,832   & 881,833                                     & 33,878,274\\
                                  & SLs (n)     & 5 (de,it,fr,en,sv) & --                                   & --\\
    \lspbottomrule
  \end{tabular}
\end{table}

\hspace*{-1mm}The overwhelming majority of texts in the InterCorp parallel corpus and SYNv8 were published after 1950, with the majority after 1980. However, some texts were first published much earlier, e.g. \textit{Osudy dobrého vojáka Švejka} by Jaroslav Hašek (1921--1923, see example \ref{olga:ex2}) and the Czech translation of \textit{The Jungle Book} by Rudyard Kipling (1911).

To maintain the comparability with the results obtained on the Jerome corpus and to reduce the influence of the date of publishing, we limited the InterCorp parallel corpus and the SYNv8 to texts (first) published after 1992. The whole corpora, including older texts, are only used to analyse the evolution of the frequency of the transgressive (see Figures \ref{olga:f1} and \ref{olga:f2}). The normalisation, convergence and cross-linguistic hypotheses are thus tested only on the texts published after 1992 (inclusive). Similar to the Jerome corpus, the resulting sub-corpora provided in \sectref{olga:ana} are smaller than those in \tabref{olga:t3}. Furthermore, as much as in the Jerome corpus, even in the sub-corpora limited to texts published after 1992, we identified and eliminated some of the texts first published or written earlier (e.g. the novel \textit{Nesmrtelnost\slash Immortality} by Milan Kundera, written in 1987–1988 and the Czech translation of \textit{Les Mots }by Jean-Paul Sartre, first published in 1967).

As for the source languages, the whole fiction/non-fiction sub-corpus of InterCorp involves 31 different source languages: Arabic (ar), Belarussian (be), Bulgarian (bg), Catalan (ca), Croatian (hr), Danish (da), Dutch (nl), English (en), Finnish (fi), French (fr), German (de), Hindi (hi), Hungarian (hu), Italian (it), Japanese (ja), Lithuanian (lt), Latvian (lv), Macedonian (mk), Norwegian (no), Polish (pl), Portuguese (pt), Romany (rn), Romanian (ro), Russian (ru), Slovak (sk), Slovene (sl), Serbian (sr), Spanish (es), Swedish (sv), Turkish (tr) and Ukrainian (uk).\footnote{Nine source languages are available in collections only and not in the core of the corpus: Greek (el), Estonian (et), Hebrew (he), Icelandic (is), Malay (ms), Maltese (mt), Albanian (sq), Chinese (zh) and Vietnamese (vi).} The most represented languages are German and English (more than 30 million tokens each, i.e. more than 10\% of the corpus each). The source languages representing between 5\% and 10\% of the corpus (i.e. more than 20 million tokens) are Polish, Spanish, Croatian and French (for detailed information, see \url{https://wiki.korpus.cz/doku.php/en:cnk:intercorp:verze12}).

In the corpus limited to texts published after 1992 (inclusive), the number of source languages is only twenty (da, de, en, es, fi, fr, hr, it, ja, lv, nl, no, pl, pt, ro, ru, sk, sl, sr, sv). Translations from English prevail (36\% of the sub-corpus), followed by German, Spanish and Swedish (see \tabref{olga:t5} for more details). It can be observed that all the languages included in this sub-corpus belong to the European area (except for Japanese) and except for Finnish and Latvian, they all belong to one of the three prevailing language families in Europe (Romance, Slavic and Germanic). The corpus thus allows to test the normalisation and convergence hypotheses (\sectref{olga:ananc}) and investigate the potential cross-linguistic interference effect (\sectref{olga:anacl}).\footnote{The number of source languages in the Jerome corpus is higher than in the InterCorp parallel corpus, because InterCorp includes only source languages for which source texts are really available in the corpus, whereas the Jerome corpus simply includes all translated texts available in Czech.}

\section{Analysis}\label{olga:ana}

Even though our main analysis focusses on the potential effects of normalisation, convergence and cross-linguistic interference in translation (see Sections \ref{olga:ananc} and \ref{olga:anacl}), we will first briefly examine the evolution of the frequency of the transgressive, in translated and non-translated texts. By doing so, we intend to verify the soundness of the limitation of the data for our analysis of the texts published after 1992 (inclusive). As mentioned above (\sectref{olga:je}), the sub-corpus of translations includes all the texts in the translated sub-corpus of InterCorp (limited to fiction and non-fiction), regardless of the date of publication or the source language. The non-translated texts are extracted from the reference corpus SYNv8.

\begin{figure}
  \includegraphics[width=\textwidth]{figures/olga/fig1}
  \caption{Normalised frequency of the transgressive in translated and non-translated fiction (InterCorp vs. SYNv8)}\label{olga:f1}
\end{figure}

As shown in Figures \ref{olga:f1} and \ref{olga:f2}, the time span for non-translated texts is larger than that of translations: the first texts in non-translated sub-corpora were published at the beginning of the 20\textsuperscript{th} century in fiction (\textit{Pověsti vyšehradské} by Popelka Biliánová, 1905) and in the 1920s in non-fiction. The first translations, by contrast, start in 1949 in fiction (Jorge Amado's novel \textit{Suor}, a translation from Portuguese) and in non-fiction (\textit{Wstęp do semantyki} by Adam Schaff, a Polish author) in 1963. Since the language of translations becomes obsolete faster than that of non-translated texts, this difference is expected.

\begin{figure}
  \includegraphics[width=\textwidth]{figures/olga/fig2}
  \caption{Normalised frequency of the transgressive in translated and non-translated non-fiction (InterCorp vs. SYNv8)}\label{olga:f2}
\end{figure}

Figures \ref{olga:f1} and \ref{olga:f2} fully confirm the tendency observed in previous research (see \sectref{olga:sty}): the normalised frequency of the transgressive constantly decreases in both translated and non-translated texts. It is worth noting that the decrease is more pronounced in fiction than in non-fiction and that the frequency of the transgressive is lower in non-fiction than in fiction. It is also necessary to point out that the actual decrease may be less dramatic than suggested by these figures, since the form of the graph is influenced by the few texts at the beginning of the observed period showing very high frequencies of the transgressive.

The data also indicate that the decrease in the frequency of the transgressive also continues after 1992, which suggests that there may be differences due to the time factor between texts within the time span of the limited corpora used in the main research. However, a further limitation of the corpus to more recent texts would have reduced the reliability of the results; hence we maintain the 1992 limit.

As for the differences between the translated and non-translated texts, Figures \ref{olga:f1} and \ref{olga:f2} suggest that they are only very slight, with a tendency to differentiation in the latest years in fiction and to a similarity in non-fiction. In what follows, we will investigate the statistical significance of these differences. 

\subsection{Normalisation and convergence in translation}\label{olga:ananc}

Tables \ref{olga:t4} and \ref{olga:t5} show the absolute (n) and normalised (f) frequencies in instances per million words of the transgressive in translated and non-translated fiction and non-fiction, and the standard deviation (SD) and the coefficient of variation (CV, (SD/f)*100) for all the subcorpora. Even though the coefficient of variation is mostly higher in non-translations than in translations, with the exception of the fiction part of the Jerome corpus, the differences are very slight. This means that the convergence hypothesis (H\textsubscript{3}, see \sectref{olga:trt}) is not confirmed by our data, and with regard to the frequency of the transgressive, translations do not show more homogeneity than non-translated texts.

\begin{table}\caption{Frequencies of the transgressive (both forms) in fiction (n = absolute frequency, f = normalised frequency in instances per million words, CV = coefficient of variation)}\label{olga:t4}
\fittable{
  \begin{tabular}{llrrrrrr}
    \lsptoprule
    \textbf{Fiction} & \textbf{corpus} & \textbf{texts} & \textbf{tokens} & \textbf{n} & \textbf{f} & \textbf{SD} & \textbf{CV}\\
    \midrule
    transl & Jerome & \raggedleft 380 & 23,301,169 & 2,538 & 108.92 & 228.23 & \textbf{209.54}\\
    non-transl & Jerome & 247 & 15,692,373 & 2,795 & \textbf{178.11} & \textbf{368.50} & 206.89\\
    transl & InterCorp & 774 & 71,063,940 & 9,268 & 130.42 & 262.94 & 201.61\\
    non-transl & SYNv8 & 328 & 20,663,102 & 3,090 & \textbf{149.54} & \textbf{343.43} & \textbf{229.66}\\
    \lspbottomrule
  \end{tabular}
  }
\end{table}

\begin{table}[p]
\caption{Frequency of the transgressive (both forms) in non-fiction (n = absolute frequency, f = normalised frequency in instances per million words, CV = coefficient of variation)}\label{olga:t5}
\fittable{
  \begin{tabular}{llrrrrrr}
    \lsptoprule
    \textbf{Non-fiction} & \textbf{corpus} & \textbf{texts} & \textbf{tokens} & \textbf{n} & \textbf{f} & \textbf{SD} & \textbf{CV}\\
    \midrule
    transl & Jerome & 221 & 15,904,500 & 754 & 47.41 & 113.73 & 239.89\\
    non-transl & Jerome & 242 & 15,719,462 & 813 & \textbf{51.72} & \textbf{126.67} & \textbf{244.91}\\
    transl & InterCorp & 78 & 6,591,970 & 720 & 109.22 & 160.85 & 147.27\\
    non-transl & SYNv8 & 592 & 30,988,911 & 3,447 & \textbf{111.23} & \textbf{166.90} & 150.05\\
    \lspbottomrule
  \end{tabular}
  }
\end{table}

As for the normalisation hypothesis, Tables \ref{olga:t4} and \ref{olga:t5} show that the normalised frequency of the transgressive is indeed higher in non-translations than in translations, regardless of the corpus (Jerome translated\slash non-translated or InterCorp\slash SYNv8) and the text register (fiction or non-fiction). However, the differences in the normalised frequency of the transgressive are statistically significant in fiction only (p<.0001), as demonstrated in Figures \ref{olga:f3}--\ref{olga:f6}. The differences observed in non\nobreakdash-fiction are not significant even at p<.05. This means that the normalisation hypothesis is confirmed in fiction, but not in non-fiction. From a methodological point of view, this result also indicates that the investigation of specific features of translation may be strongly text-type dependent.

\begin{figure}[p]
  \subfigure[Jerome translated vs. non-translated (fiction)\label{olga:f3}]{
  \includegraphics[width=0.45\textwidth]{figures/olga/fig3}
  }
  \subfigure[InterCorp vs. SYNv8 (fiction)\label{olga:f4}]{
  \includegraphics[width=0.45\textwidth]{figures/olga/fig4}
  }\\
  \subfigure[Jerome translated vs. non-translated (non-fiction)\label{olga:f5}]{
  \includegraphics[width=0.45\textwidth]{figures/olga/fig5}
  }
  \subfigure[InterCorp vs. SYNv8 (non-fiction)\label{olga:f6}]{
  \includegraphics[width=0.45\textwidth]{figures/olga/fig6}
  }
\end{figure}

The difference between the two text registers (fiction and non-fiction) regarding the tendency to normalisation may be due to various factors, especially because in fiction, translators are more likely to exploit the stylistic mark of the transgressive, which may increase their awareness about the pitfalls of this form and cause stylistic normalisation. Among the texts at the top of the frequency list of the transgressive in fiction are (in both translated and non-translated sub-corpora) texts overtly exploiting the archaistic stylistic mark of the transgressive, in particular historical novels and fantasy stories (e.g. Andrzej Sapkowski’s fantasy novel \textit{Miecz przeznaczenia} tops the list of translations in InterCorp -- 2,652.29 ipm). A similar motivation is found in translations of old source texts, first published in the 19\textsuperscript{th} century. Even if they do not belong to the category of historical novels and the translations are recent, these texts show a high normalised frequency of the transgressive (e.g. Eça de Queiros' novel \textit{A Cidade e as Serras}, 1,900.2 ipm, and two novels by Honoré de Balzac -- \textit{Le colonel Chabert} and \textit{Gobseck}, 2026.87 ipm). Especially in non-translated texts at the top of frequency lists, the transgressive is used also in order to create a humoristic or ironic effect.\footnote{For instance, at the very top of the frequency list in non-translated fiction, we find a short text by Michal Šanda (\textit{Obecní radní Stoklasné Lhoty vydraživší za 37 Kč vycpaného jezevce pro potřeby školního kabinetu} [`Municipal councillors of Stoklasná Lhota having auctioned a stuffed badger for 37 CZK for the use of the school science collection']), with 3,517.69 ipm of the normalised frequency of the transgressive.} In some texts, the transgressive reflects a specific, syntactically complex style of the author of the source text, e.g. in the translation of the novel \textit{Trans-Atlantyk} by the Polish author Witold Gombrowicz (2,196.12 ipm) and in a collection of short texts by the Belgian (French-writing) author Jean-Philippe Toussaint \textit{Autoportrait} (1,817.00 ipm). 

Nevertheless, a much more thorough analysis of the types of the use of transgressives (in translated as well as in non-translated texts) is needed. For instance, various factors triggering the use of transgressives may combine in one text,\footnote{In the \textit{Autoportrait}, for instance, the high frequency of the transgressive may be the result of a combination of complex syntax and irony in the source text (personal communication with Jovanka Šotolová, the Czech translator of the text). The age and personal style of the author (in non-translated texts) and the translator (in translations) may also come into play.} and in texts in the middle or at the bottom of the frequency list the use of the trangressive may be less motivated by its stylistic properties than by its use as a means of syntactic condensation. However, the aforementioned types extracted from the top of the frequency lists indicate that the frequency of the transgressive in fiction is probably closely related to the specific style of individual texts and authors. 

In contrast, in non-fiction, not only is the overall frequency of the transgressive lower than in fiction (see Tables \ref{olga:t4} and \ref{olga:t5}), but it appears more governed not by the specific stylistic norm of the text sub-type but by the individual style of texts and authors. Most texts containing transgressives in non-fiction sub-corpus belong to the domain of social sciences (both in translated and non-translated texts), especially philosophy and religious studies (Radim Palouš \textit{Totalismus a holismus}, 756.93 ipm or \textit{Cogitata metaphysica} by Benedict de Spinoza, 779.64 ipm), literary studies (e.g. Roland Barthes’ \textit{Mythologies}, 632.16 ipm) and history (e.g. \textit{Každodennost renesančního aristokrata} by Marie Šedivá, 717.52 ipm or Ferdinand Seibt’s \textit{Deutschland und die Tschechen}, 552.14 ipm). In technical and natural science books, by contrast, the transgressives are much less frequent or even completely absent.\footnote{This difference, already observed in previous studies (\cite[106 and 108]{dvorak83}, see \sectref{olga:sty}), may also explain the difference in the normalised frequency of the transgressive in the non-fiction sub-corpora of Jerome on the one hand, and SYNv8 on the other hand: the former is a mix of various text register sub-types, whereas the latter contains more books from the domain of humanities.}

It is important to point out that in non-fiction, the proportion of texts containing zero transgressives is higher than in fiction (one quarter of texts have no transgressive at all in the latter and one third in the former). More importantly for our topic, in both corpora (Jerome and InterCorp), more texts show zero transgressives in translations than in non-translated texts, and the maximum frequencies are higher in non-translated texts than in translations.\footnote{By contrast, in all the subcorpora, regardless of the text register or the translated\slash non-translated distinction, about a quarter of texts show the normalised frequency of the transgressive to be higher than the average of the whole sub-corpus (25\% in SYNv8 fiction and 25\% in all the other sub-corpora).}

Figures \ref{olga:f7} and \ref{olga:f8} show density plots of the normalised frequencies of the transgressive in the fiction part of InterCorp/SYNv8 (\figref{olga:f7}) and the Jerome corpus (\figref{olga:f8}) in translated and non-translated texts. It can be seen that in both corpora, even though the number of texts showing higher normalised frequencies of the transgressive is higher in non-translated texts than in translations, the differences are not extensive. Thus, the main difference between the translated and non-translated texts consists mainly in \enquote{category zero}: the number of texts containing no transgressives at all is higher in translations than in non-translated texts. This is also the main cause of the normalisation effect in translations.

\begin{figure}
  \includegraphics[width=\textwidth]{figures/olga/fig7}
  \caption{InterCorp/SYNv8 translated vs. non-translated density plot (fiction)}\label{olga:f7}
\end{figure}

\begin{figure}
  \includegraphics[width=\textwidth]{figures/olga/fig8}
  \caption{Jerome translated vs. non-translated density plot (fiction)}\label{olga:f8}
\end{figure}

Figures \ref{olga:f7} and \ref{olga:f8} suggest that if translators decide to use transgressives, they do so in a way similar to non-translated texts. However, more translators than authors of original Czech texts decide not to use transgressives at all. In the Jerome corpus, for instance, 31\% of translations do not contain any transgressive, and 13\% only one, i.e. 44\% of texts have an extremely low frequency of the transgressive. In the non-translated texts, only 20\% of texts show no transgressive and 11\% only one occurrence, i.e. only 31\% of texts without (or almost without) transgressives. These results suggest that translators could use more transgressives without being afraid to violate the norm of the target language (with respect to the style of the source text, of course). 

It is also worth noting that in fiction, normalisation and convergence are more pronounced in the past transgressive forms (Conv.pt.pf) than in the present forms (Conv.ps.impf). As expected, the frequency of the past transgressive is much lower than that of the present form (Conv.pt.pf represents 6\% of all transgressives in translations and 14\% in non-translations, see Table \ref{olga:t6}). However, the rate of the difference between translated and non-translated texts is higher in Conv.pt.pf than in Conv.ps.impf (3.24 and 1.52 respectively). The tendencies are similar in both text registers and both corpora (Jerome and InterCorp/SYNv8); therefore, we illustrate these with the numbers for the fiction part in the Jerome corpus only, in \tabref{olga:t6}.

\begin{table}\caption{Frequency of the transgressive (present and past form) in Jerome (fiction) (n = absolute frequency, f = normalised frequency in instances per million words, CV = coefficient of variation)}\label{olga:t6}
  \begin{tabularx}{\textwidth}{XXrrrr}
    \lsptoprule
    \textbf{Jerome corpus} & \textbf{form} & \textbf{n} & \textbf{f} & \textbf{SD} & \textbf{CV}\\
    \midrule
    translated      & Conv.ps.impf  & 2,376 & 108.92          & 101.97          & 93.62\\
    non-translated  & Conv.ps.impf  & 2,441 & \textbf{155.55} & \textbf{282.66} & \textbf{181.72}\\
    translated      & Conv.pt.pf    & 162   & 6.95            & 39.74           & \textbf{571.80}\\
    non-translated  & Conv.pt.pf    & 354   & \textbf{22.56}  & \textbf{118.03} & 523.18\\
    \bottomrule
  \end{tabularx}
\end{table}

All the differences between translated and non-translated texts observed in \tabref{olga:t6} are statistically significant (p<.0001), and the comparison of Figures \ref{olga:f9} and \ref{olga:f10} demonstrates that the difference is more pronounced in Conv.pt.pf (\figref{olga:f10}) than in Conv.ps.impf (\figref{olga:f9}).

\begin{figure}
  \subfigure[Jerome translated vs. non-translated (fiction) frequency of Conv.ps.impf\label{olga:f9}]{
  \includegraphics[width=0.45\textwidth]{figures/olga/fig9}
  }
  \subfigure[Jerome translated vs. non-translated (fiction) frequency of Conv.pt.pf\label{olga:f10}]{
  \includegraphics[width=0.45\textwidth]{figures/olga/fig10}
  }
\end{figure}

The greater tendency to normalisation of Conv.pt.pf is due to the more important stylistic mark of this form, in comparison with Conv.ps.impf. We can recall that Conv.ps.impf is considered bookish, whereas the Conv.pt.pf is assigned an archaistic stylistic mark. Since translators normalise, it is natural that they tend to avoid the form manifesting a stronger stylistic mark. 

\subsection{Cross-linguistic interference in translations}\label{olga:anacl}

\largerpage
Since normalisation is considered a universal phenomenon, in \sectref{olga:ananc} we analysed its potential effect in translations for all the source languages together. Conversely, cross-linguistic interference is intrinsically related to the linguistic properties of the different source languages. Concerning converbs, hypothesis H\textsubscript{1} expects more transgressives in translations from Romance and Slavic languages than in translations from Germanic languages. As stated in \sectref{olga:intc}, we conducted this analysis on the fiction part of the InterCorp parallel corpus (texts published after 1992 including), which contains a larger number of texts than the Jerome comparable corpus and the non-fiction sub-corpus of InterCorp.


\begin{table}[b]
\caption{Frequency of the transgressive (present and past form) in different source language sub-corpora of InterCorp (fiction)}\label{olga:t7}
  \begin{tabularx}{\textwidth}{XXrYYr}
    \lsptoprule
    Rank  & src.lang  & positions (n) & texts (n) & abs.fq. & ipm\\
    \midrule
    1     & pl        & 2,436,840     & 35        & 891     & 365.64\\
    2     & pt        & 1,250,080     & 16        & 398     & 318.38\\
    3     & sr        & 366,940       & 6         & 108     & 294.33\\
    4     & ro        & 372,404       & 5         & 95      & 255.10\\
    5     & es        & 8,393,499     & 101       & 1,762   & 209.92\\
    6     & fr        & 5,009,729     & 73        & 988     & 197.22\\
    7     & hr        & 1,242,178     & 19        & 209     & 168.25\\
    8     & sk        & 994,572       & 16        & 165     & 165.90\\
    9     & de        & 8,920,552     & 91        & 1,154   & 129.36\\
    10    & ru        & 1,306,704     & 11        & 154     & 117.85\\
    11    & en        & 25,810,495    & 226       & 2,597   & 100.62\\
    12    & it        & 1,044,540     & 14        & 103     & 98.61\\
    13    & fi        & 1,355,134     & 23        & 124     & 91.50\\
    14    & nl        & 1,657,687     & 23        & 151     & 91.09\\
    15    & lv        & 228,997       & 5         & 17      & 74.24\\
    16    & sl        & 835,792       & 11        & 37      & 44.27\\
    17    & da        & 1,023,334     & 9         & 44      & 43.00\\
    18    & sv        & 6,604,972     & 69        & 207     & 31.34\\
    19    & no        & 1,498,553     & 16        & 46      & 30.70\\
    20    & ja        & 710,938       & 5         & 18      & 25.32\\
    \midrule
          & \textbf{total}     & 71,063,940    & 774       & 9,268   & 130.42\\
    \lspbottomrule
  \end{tabularx}
\end{table}


\tabref{olga:t7} shows the absolute and the normalised frequencies of the transgressive (both forms together) in translations from the 20 source languages available in the sub-corpus of fiction translated into Czech in the InterCorp parallel corpus. At first sight, the results confirm the H\textsubscript{1}, since Slavic and Romance source languages are grouped at the top of the frequency list (except for Italian in Romance and Slovenian in Slavic), whereas the Germanic languages are found mostly in the lower part of the table (except for German, which is ranked 9 in the table). English, considered exceptional among the other Germanic languages, is found in the middle of the list. It is important to note that only eight source language sub-corpora show a normalised frequency higher than 140.54 ipm, i.e. the frequency in the reference non-translated fiction corpus in SYNv8 (see \tabref{olga:t4}). This confirms the tendency to normalisation observed in \sectref{olga:ananc}.

However, upon closer examination, the results introduced in \tabref{olga:t7} appear much less reliable. For instance, it is true that within the group of Romance languages, the lower frequency of the transgressive in translations from Italian may be explained by the lower frequency of the Italian converb (\textit{gerundio}, see \cite{ceretal20}) in comparison with Portuguese and Spanish (ranking second and fifth). The French \textit{gérondif}, however, is even less frequent than the Italian \textit{gerundio} (ibid.), but translations from French rank 6\textsuperscript{th}, just after Spanish. This brief observation reveals the first methodological pitfall of the analysis of the potential effect of cross-linguistic interference based only on frequencies: without understanding the \textit{valeur} of the converb in the system of the source language and without a detailed analysis of parallel concordances in the individual language pairs, all the cross-linguistic observations are potentially unreliable.

Similarly, a closer look at the group of Slavic languages reveals other discrepancies of the purely frequential approach to cross-linguistic interference. Polish, for instance, using its two converb forms extensively, is likely to be found at the top of the list, which is the case in \tabref{olga:t6}. However, the position of Russian in \tabref{olga:t6} is surprising: even though its converb is considered prototypical (see \sectref{olga:cztc}) and its two converb forms are well attested, Russian only ranks 10\textsuperscript{th}, even after Slovak, making only very limited use of its converb (see \sectref{olga:cztc} and \textcite[25]{brtkova04}). By its ranking, Russian is placed even below German, which is considered to make only \enquote{parsimonious} use of converbs (see \sectref{olga:cztc} and \textcite[72]{koenig95}). Similarly, polyconverb Finnish, Latvian and Japanese surprisingly only rank 13\textsuperscript{th}, 15\textsuperscript{th} and even 20\textsuperscript{th}.

The reliability of the results for the different language sub-corpora introduced in \tabref{olga:t7} is undermined by the same (external) factors as in the analysis of normalisation: the frequency of the transgressive may be influenced by the specific style and topic of the text, by the individual preferences of the translators, and even by the date of publication of the source text. Moreover, since the corpus is divided into 20 sub-corpora, the risk of systematic bias is higher than in the normalisation testing. For example, in the small subcorpus of translations from Slovak, we find two fantasy novels showing very high frequencies of the transgressive, which may influence the results for the whole sub-corpus, containing only 16 texts. Similarly, the normalised frequency of the transgressive in translations from Portuguese is skewed by one translation of a text first published in the 19\textsuperscript{th} century (Eça de Queiros´ novel \textit{A Cidade e as Serras}) and showing the normalised frequency of the transgressive more than 12 times higher than in the reference corpus SYNv8. In the subcorpus of translations from Romanian, it is not possible to say whether the sub-corpus reflects cross-linguistic interference or the personal preferences of the translator because all the five texts in this sub-corpus were translated by the same translator (Jiří Našinec).

\figref{olga:f11} summarises the tendencies in the frequency of the transgressive and the limitations of the reliability of the data extracted from our corpus (the confidence intervals).

\begin{figure}[t]
  \includegraphics[width=\textwidth]{figures/olga/fig11}
  \caption{Normalised frequency of the transgressive in 20 different source language sub-corpora in the InterCorp corpus (fiction)}\label{olga:f11}
\end{figure}

We can see that for Danish, Japanese, Norwegian and Serbian, the data extracted from our corpus are not reliable. The rest of the data confirm the tendencies observed in \tabref{olga:t7}, i.e. a higher frequency of transgressives in translations from Slavic and Romance languages (except for Slovenian, and partly Slovak and Russian) and a lower frequency in translations from Germanic languages.

Nevertheless, the analysis of the potential effect of the cross-linguistic interference between the converb in the source language and the Czech transgressive necessitates a thorough contrastive examination of individual language pairs. Subsequently, there needs to be a detailed analysis of the occurrences in parallel concordances, which takes into account the linguistic factors of the use of the transgressive (and its counterpart(s) in the source language), and the potential influence of the style of the text, the translators’ idiolects and other factors. 

\section{Conclusion}

The Czech transgressive is a specific case of the cross-linguistic category of converb. On the one hand, it shows most properties of the prototypical converbs: it is strict, has two forms (present and past transgressive), is referentially same-subject (i.e. coreferential with the controller of the main clause) and, as with most European converbs, its semantic interpretation is contextual (with the prevailing meaning of accompanying circumstance). On the other hand, it has an archaistic morphology, requiring agreement with the controller in number and gender and a strong stylistic mark: bookish for the present transgressive and archaistic for the past transgressive. Because of this stylistic mark, the transgressive is used rarely in contemporary language, and only in written texts.

In this study, we investigated the potential impact of these double-face characteristics of the Czech converb on translations of fiction and non-fiction in Czech. 

Our preliminary frequential analysis confirmed the constant decrease in the frequency of the transgressives in both text registers and both translated and non-translated texts during the 20\textsuperscript{th} and 21\textsuperscript{st} centuries. This observation also justified the limitation of our corpora to texts published after 1992 (inclusive). In line with expectations, the frequency analysis revealed the strong dominance of the present transgressive over the past form, which corroborates the diachronic trend suggested for Czech by \textcite{nedjalkov95}; Czech appears to be moving from a bi-converbal language to a mono-converbal one. 

The main findings of our study are the confirmation of the normalisation effect in translations of fiction (but not in those of non-fiction), the absence of convergence in translations in comparison with non-translated texts, and the necessity of a thorough contrastive analysis of converbs before investigating the potential effect of the cross-linguistic interference.

As for the normalisation, the difference in the frequency of the transgressive between translated and non-translated fiction is not extensive but is statistically significant. Of greater interest, a detailed analysis of the distribution of the frequencies revealed that this normalisation effect is caused especially by the number of texts using zero transgressives: in translations 31\%, in non-translated texts only 20\% of the texts. This means that more translators decided to avoid transgressives than the authors of the original texts. Finally, the normalisation impact is stronger in the past transgressive, showing a stronger stylistic mark, than in the present transgressive. These results suggest that if translators decided to use more transgressives -- with respect to the style of the source text, of course -- they would not violate the norm of the target language.

In non-fiction, the effect of normalisation was not observed. This text-register difference may be explained either by the overall lower frequency of the transgressive in non-fiction than in fiction or precisely by the stylistic mark of the transgressive. In fiction, the authors and translators appear to exploit this characteristic of the transgressive, e.g. the use as a means of irony or parody (mainly in non-translated texts), as the reflection of a specific, very complex style and syntax of the source text in translations or to create the archaistic effect in historical novels or in fantasy stories. This last use was also observed in translations of source texts first published in the 19\textsuperscript{th} century, even if the actual translation was recent. In non-fiction, the use of the transgressive appears to be governed not by the individual style of the text or the author, but by the norms of the text register sub-types. In line with observations in previous studies, the transgressive is more frequent in humanities (philosophy, history, literary studies, etc.) than in natural and technical sciences. Nevertheless, all these observations require a more thorough analysis of individual texts and concrete occurrences of transgressives in context.

Pertaining to the convergence hypothesis, based on the analysis of the coefficient of variation, it was observed neither in fiction nor in non-fiction. This means that translations are as heterogeneous in the frequency of the transgressive as non-translated texts. However, both in translations and in non-translated texts, the coefficient of variation is higher in the past form of the transgressive, considered archaistic, than in the present form, considered only bookish. This result indicates that the effect of convergence may vary according to the stylistic mark of the linguistic feature under investigation.

The results for the cross-linguistic hypothesis are the least conclusive. The comparison of the normalised frequency of the transgressive in twenty source language subcorpora showed a higher frequency of transgressives in translations from Slavic and Romance languages, where the converbs are considered prototypical, and a lower frequency in translations from Germanic languages, supposedly to make very limited use of converbs (except for English). However, several partial results were not consistent with the hypotheses. In the Slavic languages, for instance, translations from Slovak show a higher frequency of transgressives than translations from Russian, although converbs in Slovak are rare but abundant in Russian. Similarly, translations from French contain more transgressives than those from Italian despite the much lower frequency of the French \textit{gérondif} than the Italian \textit{gerundio}. 

These inconsistencies reveal two important pitfalls of the purely frequential analysis of the cross-linguistic interference effect in translations. First, since the use of the transgressive is intrinsically linked to its stylistic mark, the results are extremely sensitive to the composition of the different source language sub-corpora and the style of the texts they contain. Second, and more importantly, these results reveal the necessity of a thorough contrastive analysis of the different language pairs, taking into account the frequency and the \textit{valeur} of the different converbs in the language systems, and their specific uses in context.

Future research may provide not only a more fine-grained contrastive analysis of converbs in different language pairs but also a deeper understanding of the motivations of the normalisation and convergence in translation and various factors coming into play in the process of translation and the translation workflow. It is worth investigating, for instance, the potential effect of the translator’s proficiency (do experienced translators use the transgressive more than translators in the early stage of their career? What is the role of translators’ training in their attitude to the transgressive? \cite[cf.][]{lapshinova18}), the sex of the translator (preliminary results indicate female translators use transgressives less than their male colleagues; see the impact of the gender factor in \cite{magdef18}), the target audience (is there a difference between translated and non-translated literature intended for children and young readers, with regard to the use of transgressives? \cite[cf. e.g.][]{cermakova17}), and the attitude of text revisers in publishing houses to the transgressive and the impact of their interventions on its frequency in (translated as well as non-translated) texts \parencites[see also][]{bisiada17}{bisiada18}{bisiada19}{kruger18}. Only this complex approach may help to fully conceive of translation as a socially contexted behaviour and understand the norms to which the translator is supposed to adhere to.

\section*{Acknowledgements}

I would like to express my gratitude to Adrian Zasina and Martin Vavřín from the Institute of the Czech National Corpus, for providing me with the data necessary for the research, and to Tomáš Bořil from the Institute of Phonetics of the Faculty of Arts of Charles University in Prague, for the statistical analysis of my data in~R.

\hspace*{-1mm}This work was supported by the European Regional Development Fund-Project \enquote{Creativity and Adaptability as Conditions of the Success of Europe in an Interrelated World} (No. CZ.02.1.01/0.0/0.0/16\_019/0000734).

This research was also supported by the Charles University project Progres Q10, Language in the shiftings of time, space, and culture.

{\sloppy\printbibliography[heading=subbibliography,notkeyword=this]}

\end{document}
