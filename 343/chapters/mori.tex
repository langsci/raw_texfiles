\documentclass[output=paper]{langscibook}
\ChapterDOI{10.5281/zenodo.6977044}

\author{Ilaria Anghelli\orcid{}\affiliation{Centro ricerche INAIL} and Laura Mori\orcid{}\affiliation{Università degli Studi Internazionali di Roma - UNINT}}

\title[Migration in EP plenary sessions]{Migration in EP plenary sessions: Discursive strategies for the Other construction and political Self representation in Italian to Spanish interpreter-mediated texts}

\abstract{This paper deals with transfer of meaning or lack thereof in interpreting from Italian to Spanish of EP speeches delivered within the framework of Parliamentary debates during plenary sessions dealing with the phenomenon of migration. Political discourse on this topic tends to be characterised by polarity of in-group vs. out-group ideologies expressed through discursive strategies and ethnopragmatic devices that embody Self representation and the (negative/positive) construction of the Other. Our objective is to describe, through a contrastive qualitative discourse analysis, how migrants are linguistically represented (referential strategies), what qualities and traits are attributed to them (predicational strategies) and what argumentations and forms of mitigation and intensification are used to convey the political ideology toward the topic of migration. In so doing, we aim to unveil mediation strategies adopted to preserve (or conversely alter or even distort) politicians’ intentionality and to detect cues of mitigation and/or intensification of the original pragmatic intent. %%Therefore, a preliminary discourse analysis was conducted on the EP original speeches in Italian in order to pinpoint referential expressions used to designate social actors involved in international migration and the social phenomenon in itself together with the predicational strategies used to discuss on this and the organisation of the entire argumentation flow in accordance with speakers’ political pro- or anti- immigration stance. In fact, within a political environment, speakers might manifest their ideology and attitudes through their pragmalinguistic behaviour, which plays a fundamental role in building his/her political Self. Thus, beyond the locutionary aim of any political statement, interpreters of political speeches are asked to render the perlocutive dimension of the political message enacted in the original. Therefore, our main goal is that of evaluating, through a contrastive qualitative discourse analysis the mediation strategies adopted to preserve (or conversely alter or even distort) politicians’ intentionality and to detect cues of mitigation and/or intensification of the original pragmatic intent.
}

%% Abstract is too long and the footer on "how to cite" disappears

\IfFileExists{../localcommands.tex}{
  \addbibresource{../localbibliography.bib}
  % add all extra packages you need to load to this file

\usepackage{tabularx,multicol}
\usepackage{url}
\urlstyle{same}

\usepackage{listings}
\lstset{basicstyle=\ttfamily,tabsize=2,breaklines=true}

\usepackage{langsci-basic}
\usepackage{langsci-optional}
\usepackage{langsci-lgr}
\usepackage{langsci-osl}
% \usepackage{./langsci/styles/langsci-lgr}
% \usepackage{./langsci/styles/langsci-osl}
% \usepackage{langsci-gb4e}

\usepackage{tikz}
\usetikzlibrary{patterns,calc}
\pgfdeclarepatternformonly{south east lines}{\pgfqpoint{-0pt}{-0pt}}{\pgfqpoint{3pt}{3pt}}{\pgfqpoint{3pt}{3pt}}{
    \pgfsetlinewidth{0.6pt}
    \pgfpathmoveto{\pgfqpoint{0pt}{3pt}}
    \pgfpathlineto{\pgfqpoint{3pt}{0pt}}
    \pgfpathmoveto{\pgfqpoint{.2pt}{-.2pt}}
    \pgfpathlineto{\pgfqpoint{-.2pt}{.2pt}}
    \pgfpathmoveto{\pgfqpoint{3.2pt}{2.8pt}}
    \pgfpathlineto{\pgfqpoint{2.8pt}{3.2pt}}
    \pgfusepath{stroke}}
    
\usepackage{stmaryrd}
\usepackage{wasysym}
\usepackage{multirow}
\usepackage{caption}
\usepackage{subcaption}
\usepackage{mathrsfs}
\usepackage{qtree}

\usepackage{linguex}


  %pminos do not split footnotes
% \interfootnotelinepenalty=10000 %Footnote in Laporte chapters has to be split SN


%\DeclareIndexNameFormat{default}{%
%\nameparts{#1}%
%\usebibmacro{index:name}%
%{\index[names]}%
%{\namepartfamily}%
%{\namepartgiveni}%
% {}% L1
% {}% L2
%{\namepartprefix}% generates spurious space L3
%{\namepartsuffix}% generates spurious space L4
%}

%  {\DeclareIndexNameFormat{default}{%
%     \usebibmacro{index:name}{\index[names]}{#1}{#3}{#5}{#7}}}

%\DeclareIndexNameFormat{default}{%
%  \usebibmacro{index:name}{\sindex[nom]}{#1}{#3}{#5}{#7}}

%\DeclareIndexNameFormat{default}{%
%  \usebibmacro{index:name}{\sindex[person]}{#1}{#3}{#5}{#7}}
%\DeclareIndexNameFormat{default}{%
%\nameparts{#1} \usebibmacro{index:name}{\sindex[person]]}{\namepartfamily}{‌​\namepartgiven}{\nam‌​epartprefix}{\namepa‌​rtsuffix}}

%\newcommand{\smiley}{:)}

%\renewbibmacro*{index:name}[5]{%
%\usebibmacro{index:entry}{#1}%
%{\iffieldundef{usera}{}{\thefield{usera}\actualoperator}\mkbibindexname{#2}{#3}{#4}{#5}}}

% \newcommand{\noop}[1]{}

%remove for final
%\overfullrule=1mm

\newcommand{\tobi}[2]}}
\renewcommand{\S}[1]{\tobi{#1}{\textsc{*}}}

% this volume references
% puts: [this volume]
% already defined: \citetv
%\newcommand{\citepv}[1]{(\citeauthor{#1} \citeyear*{#1} [this volume])}
\newcommand{\citealtv}[1]{\citeauthor{#1} \citeyear*{#1} [this volume]}

%parentheses around example number
\newcommand{\pref}[1]{(\ref{#1})}

% in-text examples

\newcommand{\lnex}[1]{\textit{#1}} %target lang word
\newcommand{\lnlit}[1]{(lit.: `#1')} %literal reading
\newcommand{\lnlat}[1]{(#1)} % latinization
\newcommand{\lntrans}[1]{`#1'} %translation
\newcommand{\lnexl}[2]%
{\lnex{#1}{} \lnlat{#2}} % ex with latinization
\newcommand{\lnexlat}[3]{\lnex{#1}{} \lnlat{#2}{} \lntrans{#3}} % ex with latinization and tranl.

%ch01
\newcommand{\co}[1]{\mbox{\textbf{#1}}}

%ch09

\newcommand{\cyrbulg}[1]{\begin{otherlanguage*}{bulgarian}#1\end{otherlanguage*}}


%ch10
\newcommand{\nlp}{{\small NLP}}
\newcommand{\mwe}{{\small MWE}}
\newcommand{\rae}{{\small RAE}}
\newcommand{\lvc}{{\small LVC}}
\newcommand{\pos}{{\small P}o{\small S}}
%\newcommand{\todo}[1]{ \textcolor{red}{#1} }

%\renewcommand{\labelenumi}{\theenumi}
%\ainamefmt{{vv}{ll}{, ff}{, jj}} % fullname

\newcommand{\biberror}[1]{{\color{red}#1}}

\newcommand{\osenovaitem}{--~}
  %% hyphenation points for line breaks
%% Normally, automatic hyphenation in LaTeX is very good
%% If a word is mis-hyphenated, add it to this file
%%
%% add information to TeX file before \begin{document} with:
%% %% hyphenation points for line breaks
%% Normally, automatic hyphenation in LaTeX is very good
%% If a word is mis-hyphenated, add it to this file
%%
%% add information to TeX file before \begin{document} with:
%% %% hyphenation points for line breaks
%% Normally, automatic hyphenation in LaTeX is very good
%% If a word is mis-hyphenated, add it to this file
%%
%% add information to TeX file before \begin{document} with:
%% \include{localhyphenation}
\hyphenation{
    Beck-man
    Ngu-yen
    back-chan-nel
    back-chan-nels
    mo-not-o-nous
    ste-reo-typ-i-cal
}

\hyphenation{
    Beck-man
    Ngu-yen
    back-chan-nel
    back-chan-nels
    mo-not-o-nous
    ste-reo-typ-i-cal
}

\hyphenation{
    Beck-man
    Ngu-yen
    back-chan-nel
    back-chan-nels
    mo-not-o-nous
    ste-reo-typ-i-cal
}

  \togglepaper[4]%%chapternumber
}{}

\begin{document}
\maketitle
%\shorttitlerunninghead{}%%use this for an abridged title in the page headers



\section{Introduction: EU political discourse on migration}\label{sec:mori:1}
\largerpage
In van Dijk’s words “discourse analysis is not a method, but a broad, multidisciplinary field of study of the humanities and social sciences, a field that therefore should rather be called Discourse Studies” \citep[227]{VanDijk2018}. 

In political discourse politicians convey their political Self through linguistic encoding and through the enhancement or mitigation of their agency while being constantly in the process of constructing their self-image by using indexical signs and performative devices. In doing so, they express their own involvement and accountability regarding what they do, what they say and who they are. The construction of political Self tends to be affected by a polarity between in-group vs. out-group ideologies expressed through ethnopragmatic devices that embody self-representation and the (negative/positive) Other-presentation \citep{Duranti2006}. 

Political discourse may be intended as a genre, that is to say “a socially ratified way of using language in connection with a particular type of social activity” \citep[14]{Fairclough1995} and it may be interpreted by means of Critical Discourse Analysis (CDA, henceforth). As a matter of fact, CDA provides a theoretical framework to deal with “the discursively enacted or legitimated structures and strategies of dominance and resistance in social relationship of \textit{class}, \textit{gender}, \textit{ethnicity}, \textit{race}, \textit{sexual orientation}, \textit{language}, \textit{religion, age}, \textit{nationality} or \textit{world-region}” (\citealt[18]{VanDijk1995}; italics in original).

The aim of the CDA approach focuses on the analysis of the complex dialectical interplay of language and social practice in discourse, and “much work is about underlying \textit{ideologies} that play a role in the reproduction of or resistance against dominance or inequality” (\citealt[18]{VanDijk1995}; italics in original). Therefore, CDA studies have developed successful methodological techniques to detect contextual cues of ideological perspectives and speakers’ attitude toward ethical issues\footnote{For reference studies focusing on the CDA approach: \citet{Wodak1996, Wodak2001, Wodak2015}; \citet{WodakVanDijk2000}; \citet{ReisiglWodak2001}; \citet{VanDijk1995, VanDijk2001, VanDijk2006ideology}. Migration discourse analysis is discussed in \citet{RojoVanDijk1997}, \citet{WodakVanDijk2000}, and \citet{Wodak2015}.} as well as adequate heuristic tools when considering mediated discourse in institutional settings, such as EU ones, where implications related to multicultural and multilingual communication have a potentially high impact. 

This framework has also proved to be particularly useful in observing how interpreters handle conveying the evaluative and metaphorical components encoded in original political texts \citep{Boyd2016}, considering that the interpreter’s degree of participation may vary with the mode of interpreting (simultaneous or consecutive), thus creating a difficulty in assigning a stable role to him/her, in terms of addressee or side-participant \citep{Poechhacker2004}.\footnote{See \sectref{sec:mori:2} for more considerations on the interpreter’s role.}

From this perspective, in this study we are focusing on the contextual frame of Parliamentary debates which are expected to demonstrate evidence of politicians’ perspective toward a social phenomenon, such as migration, which drives opposing political ideologies. Immigration policy has a key role within the EU political framework\footnote{The legal basis for control over borders, asylum and immigration may be found in articles 77, 78, 79 and 82 of the \textit{Treaty on the functioning of the European Union} (\url{https://eur-lex.europa.eu/LexUriServ/LexUriServ.do?uri=CELEX:12012E/TXT:en:PDF}).} and the necessity for systematic international cooperation\footnote{In order to safeguard the area of freedom, security and justice Frontex, the European Border and Coast Guard Agency, is aimed at supporting at the external borders to guarantee free movement. Frontex has three strategic objectives: to reduce vulnerability of the external borders based on comprehensive situational awareness; to guarantee safe, secure and well-functioning EU borders, and to plan and maintain European Border and Coast Guard capabilities (\url{https://frontex.europa.eu/about-frontex/foreword/}). For more details see \citet[17]{DiGiambattistaEtAl2015}.} to face this phenomenon has been highlighted by Members of the European Parliament (MEPs) asking for reforms to manage migration pressure in light of the following EP standpoint:\footnote{In this regard see speeches by the following politicians: SI7, RC8, SS5.}

\begin{quote}
the EU aims to set up a balanced approach to managing regular immigration and combating irregular immigration. Proper management of migration flows entails ensuring fair treatment of third-country nationals residing legally in Member States, enhancing measures to combat irregular immigration, including trafficking and smuggling, and promoting closer cooperation with non-member countries in all fields. It is the EU’s aim to establish a uniform level of rights and obligations for regular immigrants, comparable to that for EU citizens. \citep[1]{EP2021facts}
\end{quote}

In order to face irregular immigration the European Union has signed agreements for reciprocal cooperation between EU Member States and Third Countries and some relevant directives have been enacted,\footnote{See \citet{DiGiambattistaEtAl2015} for the legal background concerning EU policy on irregular migration, management and security of external borders, asylum and legal migration.} such as the Council Directive 2001/55/EC “on minimum standards for giving temporary protection in the event of a mass influx of displaced persons and on measures promoting a balance of efforts between Member States in receiving such persons and bearing the consequences thereof”,\footnote{It is worth mentioning that intertextual references to this Directive are frequent in speeches here under examination in \sectref{sec:mori:4}: MM8, SA8, SI8, DS8, AP8, MS8, and FP12 (See Appendices~\ref{ap:mori:1} and \ref{ap:mori:2}).} Directive 2008/115/EC of the European Parliament and of the Council “on common standards and procedures in Member States for returning illegally staying third-country nationals” and Directive 2009/52/EC of the European Parliament and of the Council “providing for minimum standards on sanctions and measures against employers of illegally staying third-country nationals”.

In such a multilingual and multicultural context, European Parliamentary debates could enhance common discursive practices and, at the same time, reveal cross-speaker differences related to the socio-political background of politicians belonging to different political parties. Having in mind this pragmatic complexity our research question concerns the solutions interpreters adopt to face the mediation of ethnopragmatically-oriented choices and discursive strategies on migration that reveal the politician’s stance.

In order to answer this research question, in the following sections we are going to focus on the interpreter’s role and on the challenges for interpreters in the EP setting (\sectref{sec:mori:2}). Afterwards, we will present our methodological apparatus (\sectref{sec:mori:3}) to discuss the most relevant results with bilingual examples (original Italian; interpreted Spanish) in \sectref{sec:mori:4}. Conclusive remarks are reported in \sectref{sec:mori:5}.



\section{Mediation of EP speeches: advantages and disadvantages for interpreters and their role}\label{sec:mori:2}


During a plenary session, about 1000 interpreters are involved to cover all the EU official languages. As pointed out by \citet{Bartlomiejczyk2016}, plenary sessions constitute a more demanding setting than, for example, meetings of political groups or committee meetings, due to the quick succession of speakers and to the variety of languages spoken on the floor. Following the criteria for time allocation within the EP, a large political group may have up to five minutes, while a small group may only have one minute, thus affecting the delivery of speeches and the interpretations provided by official interpreters. In fact, as suggested by \citet{Vuorikoski2004}:

\begin{quote}
$[\ldots]$ this results in speeches that are written and recited at fast rate. Furthermore, this rule may also explain why the speeches tend to be extremely dense regarding their information content. \citep[79]{Vuorikoski2004}
\end{quote}

It would appear that the average speaking rate in a plenary session is 150 words per minute \citep{MontiEtAl2005}, while the optimal speed is around 95--120 words per minute (quoted in \citealt[52]{Bartlomiejczyk2016}). Therefore, it can be assumed that the majority of EP plenary speeches are faster than what is considered comfortable to interpret.

Moreover, \citet[69]{Marzocchi1998} and \citet[63]{Kent2009} remark that interpreters face problems related to the oral delivery of written texts, with the specific prosody related to reading aloud, the lesser redundancy, and other obstacles due to the syntactic and semantic complexity of planned, written speech, as well as the lack of fluency some MEPs may have in the \textit{institutional lingua franca} they use (usually English). It must be said that these problems are caused by the unavailability of transcripts of speeches and the linguistic behaviour of most speakers in the session, despite the efforts made to limit the difficulties of interpreting, such as the distribution of leaflets to the MEP on how to communicate through interpretation.\footnote{Among the recommendations (here summarised), delivered by Vice-President Miguel Angel Martínez during his speech on 25 March 2009: “speak at a regular speed, and not too fast, speak in your mother tongue (if possible), avoid changing language when you speak, speaking is better than reading, but if there is no alternative to reading, make sure that interpreters have the text, clearly give references to documents, articulate clearly any figure that is mentioned, explain abbreviations that you use in what you say, remember that jokes are difficult to translate. Also, when you are chairing a meeting, wait a moment before giving the floor to the next speaker so that the interpreter can finish the speech and change to the appropriate channel” \citep[55]{Bartlomiejczyk2016}.}

Finally, it seems that interpreters feel frustrated because of the lack of an effective debate and because communication among participants in the sitting is not the primary goal for speakers. In fact, according to \citet{Kent2009}:

\begin{quote}
Although described as debate the speeches given by Members during plenaries are mainly directed to consumption by home country audiences via the internet, television and radio rather than as engagement with colleagues who are in the room. \citep[57]{Kent2009}
\end{quote}

\largerpage
Nonetheless, there are some common linguistic features and fixed structures in most speeches that may turn out to be an advantage for the interpreter due to predictability and pragmatic inference. First of all, these speeches belong to the argumentative text-type and they share highly ritualised conventions that help the interpreter and allow her/him to focus on less predictable statements.\footnote{See Gile’s effort model \citep[169--170]{Gile1995} for a study on this topic.} In this regard, \citet{Vuorikoski2004} states that oral texts belonging to this genre are composed of:

\begin{itemize}
\item
an introduction, with deference to the previous speaker or greetings to the President, the Vice-president or other members;
\item
the main body, where the speaker’s stance is presented;
\item
final remarks.
\end{itemize}\clearpage

During their speech, MEPs may offer an important clue as to the opinions that they express belonging to one of the existing political groups and to the expected level of formality they usually adopt \citep[66]{Marzocchi1998}.\footnote{However, it has to be noted that “MEPs’ identity is inherently hybrid and in view of the weakness of the existing whipping system in the European Parliament, they might as well give priority to national or regional interests over the interests of their own political group.” \citep[105--108]{Beaton2007}.} This has an impact on interpreting in terms of the degree of planning of speeches, taking into account register shifts, rhetorical purposes and for handling with prosody.

In these communicative events, the interpreter assumes a fundamental social role,\footnote{See \citet{Anderson2002} for a preliminary analysis on the role of the interpreter.} that is intended as “a set of more or less normative behavioral expectations associated with a ‘social position’” \citep[147]{Poechhacker2004}. This especially refers to legal interpreters and interpreters in healthcare, domains where cultural differences and unfamiliar contexts enhance their role as facilitators, intercultural experts and visible agents.\footnote{As described in studies by \citet{LasterTaylor1994}, \citet{Barsky1996} and \citet{Angelelli2001} respectively and summarized by \citet[147--149]{Poechhacker2004}.} In this respect, \citet[149]{Poechhacker2004} reported two studies conducted by \citet{Morris1989} and \citet{Shlesinger1991} on simultaneous interpreting, showing how interpreters were responsible for omissions and stylistic changes, thus leading to a “sort of intrusiveness (as perceived by participants) or latitude (as perceived by the interpreters themselves)”.

As far as the EP setting is concerned \citet{Beaton2007} examined the impact of the simultaneous interpretation on ideology. This study provides examples of interpreter mediation and agency by defining him/her as “an additional subjective actor in heteroglot communication” \citep[271]{Beaton2007}.

More recently, researchers investigated the visibility or invisibility of the interpreter, focusing on how he/she plays an active role in the communicative event. As a matter of fact, \citet{BeatonThome2013} underlined the visible role of interpreters considering that they tend to select more neutral terms than those used in the original speeches, by making explicit what was implicit in the original. On the other hand, there are also examples where the interpreter intensifies the speaker’s ideological stance by explicitating something that was stated implicitly in the original.

On this regard \citet[128]{Bartlomiejczyk2016} highlights that studies on conference interpreting (e.g. \citealt{Diriker2004}; \citealt{Monacelli2009}) reassessed the role of the interpreter by stating the impossibility to
expect that the presence of an interpreter mediating between the speaker and the hearer(s) will not have any influence on facework in the interaction \citep[128]{Bartlomiejczyk2016}.



\section{Methodology}\label{sec:mori:3}

\begin{sloppypar}
In our paper, we focus on migration discourse related to the Parliamentary speeches that are “distinguished by genre-specific linguistic forms and/or structures and are closely linked to specific social and institutional contexts” \citep[32]{Fairclough2006}. EP speeches may be considered as belonging to a specific sub-genre\footnote{Parliamentary debates have started to be investigated in the literature by a number of scholars, but for the most part studies focused mainly on political national Parliaments rather than at supranational level. For more details see \citet[5--6]{Ilie2015}.} since Parliaments represent peculiar \textit{loci} for evaluating social use of language and discursive strategies aimed at persuading, negotiating, and/or building opinions in relation with the reference political party:
\end{sloppypar}

\begin{quote}
the discourse of parliament results in (or is the final stage of a process which results in) concrete action in the outside world, establishing regulations as to what must, may and may not be done in a given society. \citep[12]{Bayley2004}
\end{quote}

The discursive interaction within EP debates complies with a prototypical frame acknowledged by any member of the EP: a “context model” (see \citealt{VanDijk2003}) shared in accordance with the specific setting (location, time), participants (and inter-personal relations), activities and actions in which MEPs are engaged as political and institutional actors willing to affirm their political Self. More specifically, in EP plenary sessions, the~President of the European Parliament chairing the session assisted by the 14 vice-presidents opens the sitting with a speech on the current topic. During a Parliamentary debate, any speaker plays a communicative role (by expressing his/her own opinions or acting as the spokesperson of his/her party), an interaction role (opponent, enemy or ally) or a social role (based on the group, class, and the ethnicity identified with).\footnote{This could be detected by examining the shift in the use of allocutives: first person singular vs. first person plural. Similarly, the opposition in the linguistic representation of ingroupness can be analysed through deictics in the distribution between Us and Them.}


\subsection{Research goal}\label{sec:mori:3.1}

Over the last few years some empirical studies have focused on the multilingual functioning of the European Parliament (EP) generating insights into the interpreters’ role (e.g. \citealt{Bartlomiejczyk2016}; \citealt{BeatonThome2013}; \citealt{KucisMajhenic2018}). From a pragmatic perspective it is interesting to consider the filtering effect that could cause misinterpretation of the speaker’s illocutionary force, intention and attitude. In such a context, in fact, simultaneous interpreting has to comply with interactive patterns featuring plenary debates as well as with the consumption by home country audiences (see \citealt[57]{Kent2009}). In this way, it is relevant to consider both audiences, colleagues \textit{in presentia} and external public \textit{in absentia} and different speakers’ pragmatic intentions and their degree of engagement toward these two targets.

Our research goal is to evaluate the implications of oral mediation into Spanish of EP migration discourses of Italian politicians as far as how migrants are linguistically represented in EP discourses by focusing on referential strategies, what qualities and traits are attributed to them (by means of predicational strategies) and what are the argumentations and the forms of mitigation and intensification used to convey speakers’ political ideology. 

Moreover, it must also be borne in mind that political discourse about migration may be seen as a social practice thanks to which speakers act ethnopragmatically in order to build their public image in relation to a socially sensitive topic.


\subsection{The collection of data} \label{sec:mori:3.2}

The dataset was collected from a corpus based on 60 speeches delivered in European Parliament debates during plenary sessions about migration-related issues by twenty-five MEPs (15,311 tokens) and their interpretations into Spanish (16,997 tokens).\footnote{Speeches are available on the following website: \url{https://www.europarl.europa.eu/plenary/en/debates-video.html\#sidesForm}} Speeches were selected\footnote{The selection of criteria complies with the research design outlined in \citegen{Anghelli2019} MA thesis from which this study derives.} in accordance with the following external variables:

\begin{itemize}
\item
time: in a given timespan corresponding to the 8\textsuperscript{th} parliamentary term (2009--2014);\footnote{The selection of texts, collected before the end of the 9th parliamentary term, was limited to speeches in Italian where the interpreted versions were available. Basically, the speeches analysed were delivered in a three-year span 2009--2010--2011 (see Appendix~\ref{ap:mori:1}).}
\item
topic: the semantic field of migration to lead a topic-oriented research;\footnote{Italian key-words used to filter the corpus selection are the following nouns and adjectives (in singular and plural forms): \textit{migrazion*} (‘migration/s’), \textit{immigrazion*} (‘immigration/s’); \textit{fluss* migrator*} (‘migratory flow/s’), \textit{emigrazion*} (‘emigration/s’), \textit{migrant*} (‘migrant/s’), \textit{emigrant*} (‘emigrant/s’), \textit{immigrat*} (‘immigrant/s’), \textit{profug*/rifugiat*} (‘refugee/s’), \textit{richiedent* asilo} (‘asylum seeker/s’), \textit{clandestin*} (‘illegal immigrant/s’), \textit{stranier*} (‘foreigner/s’).}
\item
speakers’ political profile: speakers belonging to ALDE (Alliance of Liberals and Democrats for Europe Party); EPP (European People’s Party); S\&D (Progressive Alliance of Socialists and Democrats) and EFD (Europe of Freedom and Democracy). 
\end{itemize}

The European Parliament website only provides audio materials of the interpretations but not the transcription of the interpretations that was carried out in order to lead this current study.\footnote{Speeches here analysed are comprised in Section IA within the Corpus MULPOLDIS (Multilingual Multimodal Political Discourse) developed by the Corpus Linguistics Centre at the Università degli Studi internazionali di Roma (\url{https://www.unint.eu/it/ricerca/centri-di-ricerca/centro-di-ricerca-linguistica-su-corpora-clc/1357-corpus-mulpoldis})} More details concerning speeches dealt with in \sectref{sec:mori:4} are reported in Appendix~\ref{ap:mori:1} and \ref{ap:mori:2}.


\subsection{The theoretical framework}\label{sec:mori:3.3}

Within the CDA theoretical framework, useful methodological approaches were developed to detect contextual cues of ideological perspectives (see \citealt{Wodak1996,Wodak2001}; \citealt{ReisiglWodak2001}; \citealt{VanDijk1995, VanDijk2006ideology, VanDijk2015}) and discursive strategies of positive self- and negative Other construction. This can be seen in studies specifically devoted to dealing with the field of action\footnote{``Fields of action'' may be understood as segments of the respective societal ``reality", which contribute to constituting and shaping the ``frame" of discourse. The spatio-metaphorical distinction among different fields of action can be interpreted as a distinction among different functions or socially institutionalised aims of discursive practice” \citep[36]{ReisiglWodak2001}.} of migration such as \citet{WodakVanDijk2000} on the discursive strategies used by politicians from seven Western European countries (Austria, France, Germany, Great Britain, Holland, Italy and Spain) to refer to migrants and to the phenomenon of immigration.\footnote{See also \citet{RojoVanDijk1997} and \citet{BeatonThome2013}.} \citet{ReisiglWodak2001} pinpointed discursive strategies used to define social actors and predicate on them by conveying an overt or implicit evaluation on speaker’s attitude toward a given social category or phenomenon. As a matter of fact, findings in the analysis of migration discourse “allow[s] to conclude that much discourse about migrants and immigration seems to bear several almost universal features, throughout Europe and beyond, which can be explained by social theories about ‘Othering’ and the discursive construction of ‘the stranger’ and ‘fear of the stranger’ [\ldots]” \citep[8]{Wodak2015}.

In order to interpret our data, we decided to apply the analytical categories of the \textit{Discourse hystorical approach} (DHA, in \citealt{ReisiglWodak2009}) aimed at analysing discursive strategies for the Other-representation and the discursive construction of migration in concrete text extracts in Italian mediated into Spanish (examples in \sectref{sec:mori:4}). In particular, we are referring to five heuristic questions considered salient to DHA in \citeauthor{Wodak2015}'s (\citeyear[8]{Wodak2015}) categorisation:

\begin{itemize}\sloppy
\item How are persons, objects, phenomena/events, processes, and actions named and referred to linguistically? 
\item What characteristics, qualities and features are attributed to social actors, objects, phenomena/events and processes? 
\item What arguments are employed in the discourse in question? 
\item From what perspective are these nominations, attributions and arguments expressed?
\item Are the respective utterances articulated overtly, are they intensified or mitigated?
\end{itemize}

Therefore, our analysis on original speeches was focused on: a) referential strategies, b) predicational strategies, c) argumentative strategies.

The referential strategies used either to include, suppress, specify, genericise, depersonalise or deny the Other date back to \citegen{VanLeeuwen1995} categorisation. To realise these strategies various forms of labelling are used to name social actors and characterise them with respect to inclusion/exclusion in social events and in terms of the way they may be personally or impersonally represented and classified specifically or generically (see \citealt[145--146]{Fairclough2003}).

These strategies are considered together with predicational ones: 

\begin{quote}
Predication is the very basic process and result of linguistically assigning qualities to persons, animals, objects, events, actions and social phenomena. Through predication, persons, things, events and practices are specified and characterised with respect to quality, quantity, space, time and so on. Predications are linguistically more or less evaluative (deprecatory or appreciative), explicit or implicit and – like reference and argumentation – specific or vague/evasive. \citep[54]{ReisiglWodak2001} 
\end{quote}

Studies conducted on predicational strategies (such as \citealp{Wodak2000, Wodak2001}; \citealp{VanDijk2002}; \citealp{ReisiglWodak2001}) have mainly considered speeches concerning antisemitism, racism, nationalism or discrimination based on gender, race, religion where the Us/Them opposition emerged clearly in the pragmalinguistic representation of the Outgroup as opposed to the Ingroup.\footnote{In this regard, we can cite \citegen{VanDijk1998} theoretical concept of “ideological square” through which he encapsulates the polarisation manifested in discourse by lexical choice and other linguistic features as far as the representation of Self and Others, Us and Them are concerned.} Predications are developed by means of \textit{topoi} (such as that of numbers\footnote{See also the study on an anti-immigration leaflet conducted by \citet[118--124]{Semino2008}, where the use of numbers is consistent with the negative representation of migrants that emerges from the text.}) or metaphors and “extended metaphors”\footnote{According to \citet[25]{Semino2008} an extended metaphor -- such as football metaphor widely used in Italian political speeches -- is considered “as a particular type of cluster, where several metaphorical expressions belonging to the same semantic field or evoking the same source domain are used in proximity to one another in relation to the same element, or to elements of the same target domain”.}  that support the argumentative strategies through which the migration discourse is based on. 

In order to build argumentations and counter-argumentations referring to given social groups \citep[45]{ReisiglWodak2001}, in political discourse the use of \textit{topoi} is particularly exploited not only to discuss on a given topic but as a productive strategy to represent the commonsense reasoning typical for specific issues. Among other most frequently adopted argumentative strategies, we can cite metaphors used “when it is necessary to simplify complex issues, and to present them in vivid and potential emotional terms” \citep[124]{Semino2008} and the reporting of personal experiences related to the speaker’s private Self to direct public opinion by increasing the degree of legitimation of what is being said and, consequently, his/her reliability also by shedding a negative light on opposing past actions and on the political group through emphasis on the transformation between then and now.

The above-mentioned criteria were applied to lead our research by combining quantitative corpus-based analyses and qualitative ones as follows:

\begin{itemize}
\item
corpus-based analysis to identify referential strategies by looking for topic-related words (in terms of frequency) and qualitative analysis of selected examples and their renderings into Spanish;
\item
corpus-based analysis to highlight predicational strategies through concordancing and their comparison with mediated strategies adopted into Spanish;
\item
qualitative analysis of relevant \textit{topoi} and metaphors for the construction of migrants and the representation of migration through politicians’ argumentations.
\end{itemize}

This methodological approach allowed us to identify respectively referential, predicational and argumentative strategies adopted to express politicians’ ideologies on “ethnic topics” through discursive practices mainly based on the polarisation between the Ingroup and the Outgroup.

These results were, then, assumed as a starting point to analyse contrastively the solutions used in mediated texts in order to focus the way, and to what extent, Italian politicians’ discursive strategies interpreted into Spanish are conceived to convey pragmatic equivalence.



\section{Results and discussion} \label{sec:mori:4}

The categorisation of discursive strategies in accordance with the DHA was applied to original EP speeches in Italian in order to evaluate to what extent they are re-codified during the mediation process into Spanish.\footnote{See \citet{Anghelli2019} for detailed results on the original speeches.} In the following sub-sections, the mediation strategies adopted to comply with the fulfilment of pragmatic equivalence in terms of referential strategies (\sectref{sec:mori:4.1}), predicational strategies (\sectref{sec:mori:4.2}), argumentation strategies (\sectref{sec:mori:4.3}) are discussed.


\subsection{Mediation of referential strategies} \label{sec:mori:4.1}

Strategies used to name the social actors involved were examined as far as specification or generalisation and reference to age, race, gender, origin, and so on are concerned. It is possible to observe the distribution of referential strategies used in original discourses in Italian by referring to migrants as social actors (\figref{fig:mori:1}) or to the social phenomenon of migration (\figref{fig:mori:2}) and their interpretations in Spanish through: a) word-for-word translation (in blue), b) synonyms or re-elaborations of the originals (in grey), c) omissions (in orange).

\begin{figure}
%\includegraphics[width=\textwidth]{figures/4-1.png}
\scalebox{.8}{
\begin{tikzpicture}
            \begin{axis}[
                xbar,
                xmin=0,
                xmax=20,
                axis lines*=left,
                width=\textwidth,
                height=23cm,
                symbolic y coords={person*,migrant*,immigrati,rifugiat*,clandestini,profughi,etnonimi,gente,richiedenti asilo,delinquenti,popoli,cittadini},
                ytick=data,
                nodes near coords,
                nodes near coords align = horizontal,
                legend pos = east,
                reverse legend,
                point meta=rawx
                ]
                 \addplot+[lsDarkBlue,fill=lsDarkBlue,]
                    coordinates {
                    (20,person*)
                    (5,migrant*)
                    (0,immigrati)
                    (8,rifugiat*)
                    (2,clandestini)
                    (7,profughi)
                    (6,etnonimi)
                    (1,gente)
                    (5,richiedenti asilo)
                    (3,delinquenti)
                    (2,popoli)
                    (5,cittadini)
                };
                \addplot+[lsMidOrange,fill=lsMidOrange,]
                    coordinates {
                    (2,person*)
                    (0,migrant*)
                    (1,immigrati)
                    (1,rifugiat*)
                    (0,clandestini)
                    (0,profughi)
                    (0,etnonimi)
                    (0,gente)
                    (0,richiedenti asilo)
                    (0,delinquenti)
                    (1,popoli)
                    (1,cittadini)
                };
                \addplot+[lsLightGray,fill=lsLightGray,]
                    coordinates {
                    (6,person*)
                    (10,migrant*)
                    (14,immigrati)
                    (4,rifugiat*)
                    (7,clandestini)
                    (2,profughi)
                    (2,etnonimi)
                    (4,gente)
                    (0,richiedenti asilo)
                    (1,delinquenti)
                    (0,popoli)
                    (0,cittadini)
                };

                \legend{word-for-word translation,omissions,synonyms or re-elaborations of the originals}
            \end{axis}
        \end{tikzpicture}
        }
\caption{The mediation of referential strategies (social actors)}
\label{fig:mori:1}
\end{figure}

\begin{figure}
%\includegraphics[width=\textwidth]{figures/4-2.png}
\scalebox{.83}{
 \begin{tikzpicture}
            \begin{axis}[
                xbar,
                xmin=0,
                xmax=50,
                axis lines*=left,
                width=\textwidth,
                height=15cm,
                symbolic y coords={immigrazion*,emergenz*,problem*,fluss*,esodo,migrazion*,ondata migratoria},
                ytick=data,
                nodes near coords,
                nodes near coords align = horizontal,
                legend pos=north east,
                reverse legend,
                point meta=rawx
                ]
                 \addplot+[lsDarkBlue,fill=lsDarkBlue,]
                    coordinates {
                    (46,immigrazion*)
                    (22,emergenz*)
                    (26,problem*)
                    (15,fluss*)
                    (2,esodo)
                    (1,migrazion*)
                    (2,ondata migratoria)
                };
                \addplot+[lsMidOrange,fill=lsMidOrange,]
                     coordinates {
                    (2,immigrazion*)
                    (1,emergenz*)
                    (6,problem*)
                    (0,fluss*)
                    (0,esodo)
                    (0,migrazion*)
                    (0,ondata migratoria)
                };
                 \addplot+[lsLightGray,fill=lsLightGray,]
                    coordinates {
                    (7,immigrazion*)
                    (8,emergenz*)
                    (1,problem*)
                    (2,fluss*)
                    (2,esodo)
                    (2,migrazion*)
                    (0,ondata migratoria)
                };

                \legend{word-for-word translation,omissions,synonyms or re-elaborations of the originals}
            \end{axis}
        \end{tikzpicture}
        }
\caption{The mediation of referential strategies (phenomenon of migration).}
\label{fig:mori:2}
\end{figure}

Figures \ref{fig:mori:1} and \ref{fig:mori:2} show that word-for-word translation  is the most common strategy interpreters prefer to use, both to represent social actors and the phenomenon in itself. More specifically, this strategy allows the interpreter to convey both semantic and pragmatic equivalence, for example by maintaining the reference to the status of social actors (by translating \textit{profughi} or \textit{rifugiati} as \textit{prófugos} or \textit{refugiados}, and \textit{richiedenti asilo} as \textit{solicitantes de asilo}), the inclusion and the rights of migrants (by rendering \textit{persone,} \textit{cittadini} and \textit{popoli} into \textit{personas, ciudadanos} and \textit{pueblos}), and the reference to criminality and delinquency by using criminonyms (\textit{delinquenti} translated into \textit{criminales}). \tabref{tab:mori:1} reports the percentage of use for the word-for word strategy for the following items (\textit{refugees}, \textit{asylum seekers}, \textit{person}, \textit{citizens}, \textit{people}, \textit{criminals}) and information on the communicative events (see Appendix~\ref{ap:mori:1} and \ref{ap:mori:2} for more details).

\begin{table}
\begin{tabularx}{\textwidth}{QrXp{1.7cm}}

\lsptoprule

{\bfseries Items (number of occurrences)} & {\bfseries \makecell[rt]{Percentage of\\word-for-word\\strategy}} & {\bfseries Speakers} & {\bfseries Topics}\\
\midrule
Ciudadanos (6) & 83\% & SA (3), MEP (2) & 7, 8\\
Criminales (4) & 75\% & MBO (3) & 8, 10\\
Personas (28) & 71\% & SA (3), RA (2), CM (2), DS, SI, AP (2), MBO, RC (3), SC (2), FP (3) & 1, 3, 5, 7, 8\\
Prófugos (9) & 78\% & CF, SI (2), MBO, MB (2), SA & 7, 8, 12\\
Pueblos (3) & 67\% & RC, MM & 7, 8\\
Refugiados (13) & 64\% & SA, CC, CF, SI (4), DS & 1, 3, 5, 7, 8\\
\mbox{Solicitantes de asilo (5)} & 100\% & RB, MM, SS, RA (2) & 1, 8\\
\lspbottomrule
\end{tabularx}

\caption{Frequency of the word-for-word strategy}
\label{tab:mori:1}
\end{table}

When referring to the phenomenon of migration, interpreters tend to adopt a word-for-word translation in order to maintain the same linguistic nuances of Italian items. They mainly use lexemes conveying the difficulty to manage the phenomenon by translating \textit{emergenza} and \textit{problema}\footnote{For a study on the use of “emergency” and “problem” in the Italian press, see \citet{Orru2017}.} into \textit{emergencia} and \textit{problema}. They also refer to immigration as water-course or flood\footnote{For a study on this topic, see \citet{ReisiglWodak2001}.} by translating \textit{flusso} and \textit{ondata migratoria} into \textit{flujo} and \textit{oleada/ola migratoria/ de inmigración} as well as making use of a stereotyped metaphor of migration as an exodus: \textit{esodo >} \textit{éxodo}). Sometimes interpreters opt for more neutral solutions such as \textit{immigración} and \textit{migración} when in original texts analogous referential strategies were adopted (\textit{immigrazione} and \textit{migrazione}).

In \tabref{tab:mori:2}, the percentage of the word-for-word translation for the following items (\textit{emergency}, \textit{problem}, \textit{flow}, \textit{wave of migration}, \textit{exodus}, \textit{migration}, and \textit{immigration}) is reported.

\begin{table}
\begin{tabularx}{\textwidth}{QrXp{1.8cm}}

\lsptoprule

{\bfseries Items (number of occurrences)} & {\bfseries \makecell[rt]{Percentage of\\word-for-word\\strategy}} & {\bfseries Speakers} & {\bfseries Topics}\\
\midrule
Emergencia (32) & 71\% & SS (2), FP (3), MM (3), MB, PP, CF (2) SA, MBO (2), AP (3), CC, BM (2), RBA & 6, 7, 8, 9\\
Éxodo (4) & 50\% & SS, GP & 6, 12\\
Flujo (17) & 88\% & RB, SI (6), GLV, RBA, MM (2), GP, FP, RA, CF & 1, 2, 3, 4, 6, 7, 8, 9, 10, 12\\
Migración (3) & 33\% & SI & 8\\
Immigración (55) & 84\% & RA (9), CM (3), DS (5), RB (2), AC, MB (4), MM (6), PP, FP (3), SA (3), SI (3), RBA, CC, MBO (3), GP & 1, 2, 3, 4, 7, 8,10, 11, 12\\
Oleada migratoria (2) & 100\% & PP, MB & 7, 8\\
Problema (33) & 79\% & RA, AC (2), AP, MM (15), SI, PP (2), RC (2), FP, SA & 1, 2, 5, 7, 8, 10\\
\lspbottomrule
\end{tabularx}

\caption{Frequency of the word-for-word strategy}
\label{tab:mori:2}
\end{table}

The second strategy (synonyms or re-elaborations) is generally adopted by interpreters when a word-for-word-translation is not possible to refer to social actors or to the whole phenomenon. In these cases, synonyms, other grammatical categories or syntactical variations of the Italian forms are selected. With respect to social actors, the most frequent example concerns the re-elaboration of the Italian item \textit{clandestino}, rendered in Spanish through the adjective category (\textit{clandestino}).\footnote{In two discourses (MB8, MBO8), the Italian noun \textit{clandestino} has been rendered into Spanish as a noun, thus resulting in a calque.} For this reason, interpreters opted for the syntactic structure [N + Adj], such as \textit{emigrantes clandestinos}, \textit{embarcaciones clandestinas} and \textit{flujos clandestinos} (in MB8 exclusively).\footnote{In the mediated text MBO11 the NP \textit{inmigrantes ilegales} is adopted. It has to be noted that this is the translation proposed for any occurrence of the Italian noun clandestino in the official written translations of these speeches (\url{https://www.europarl.europa.eu/plenary/es/debates-video.html\#sidesForm}).}

There are also cases where interpreters make some grammatical variations, namely the number category by using plural nouns to stress the generalisation strategy: three times the word \textit{rifugiato} was translated as \textit{refugiados} (SI3, SI5, AP5).\footnote{We cannot avoid to remark that in one case \textit{rifugiati} (SA7) was translated as \textit{delincuentes} with a semantically severe inadequacy by using a criminonym rather than the reference to a specific legal status.} 

Some morphosyntactic changes may be observed in the following examples by adding a common noun \REF{ex:mori:1} or by removing it \REF{ex:mori:2}.

\ea\label{ex:mori:1}
\textit{Somali} (SA1) and \textit{Eritrei} (DS1) were translated respectively as \textit{habitantes de Somalia} and \textit{habitantes de Eritrea}. 
\z
\ea\label{ex:mori:2}
\textit{persone detenute} was translated as \textit{reclusos} (SI2) and \textit{persone tutelate} as \textit{protegidos} (AP5).
\z

When referring to migration as a phenomenon, we found an extended interchange of synonyms to refer to migration as a “problem” in order to maintain the original communicative intent and to present it as an emergency: from Italian \textit{emergenza} to Spanish \textit{urgencia}/\textit{urgencia humanitaria} (MM7), \textit{reto} (SS7), \textit{situación urgente} (FP7) and \textit{problema} (MB7).

In some cases interpreters preferred referring directly to social actors \REF{ex:mori:3} or using a water metaphor \REF{ex:mori:4} to refer to the phenomenon,\footnote{For an extensive list of contribution on water metaphors, see \citet[12]{Taylor2020}.} rather than a word-to-word translation:

\ea\label{ex:mori:3}
\textit{migrazione} > \textit{inmigrantes} (MBO8)
\z
\ea\label{ex:mori:4}
\textit{immigrazione} > \textit{oleada migratoria} (GLV8)\\
\textit{migrazione} > \textit{flujo} (MBO8)
\z

From our research perspective, both renderings affect the pragmatic original meaning by using a personification \REF{ex:mori:3} or the water metaphor in place of the neutral solution adopted in the original \REF{ex:mori:4}. In this last regard, \citet[3]{Taylor2020} explains that “metaphors by their very nature are not neutral” and, referring to the water metaphor in particular, \citet[269]{Marlow2015} suggests that it can be used to enhance the perception of immigrants as a source of threat.

The fact that interpreters seem to be focused on the semantic content without paying enough attention or being sufficiently aware of the pragmatic effect is even more evident in the following examples where the original referential strategies are rendered through semantic intensification \REF{ex:mori:5}, imperfect rendering \REF{ex:mori:6} and gender-specification \REF{ex:mori:7}:

\ea\label{ex:mori:5}
\textit{problema} is translated as \textit{tragedia} (MB012), thus intensifying the original meaning by adding a semantic component.
\z
\ea\label{ex:mori:6}
\textit{persone} and \textit{migranti} are translated as \textit{asilados} (AP5) and \textit{solicitantes} \textit{de asilo} (SS6) both these choices refer to the legal status that was not at all taken into consideration in the original version.
\z
\ea\label{ex:mori:7}
\textit{uomini} used to refer to human beings is rendered into mediated texts by means of \textit{hombres y mujeres} (FP8a), thus making the reference to both genders explicit.
\z


The third strategy investigated (c) concerns the use of omissions, as referred to by \citet[108]{Wadensjoe1998}, in order to detect cases of “zero rendition” and “reduced rendition”. As a matter of fact, a remarkably imperfect pragmatic correspondence in mediated discourse into Spanish emerged in instances of omissions, affecting the original pragmatic content and, consequently, the speaker’s pragmatic (and political) intent. This failure in recoding speaker’s intentionality could depend on the speed of speeches delivered by Italian speakers -- considering that they are generally read aloud and pre-planned -- and on the high information density due to the shortness of this textual sub-genre. Moreover, the difficulty in giving a complete pragmatic correspondence is caused by the nature of the context itself, where interpreters have to deal with a quick succession of speakers (see \sectref{sec:mori:2}).

The occurrences in our corpus, exemplified in \REF{ex:mori:8} and \REF{ex:mori:9}, show that omissions usually have an impact on the rendition of the interpreter, especially when the pragmatic intent of the speaker is not transmitted. The example reported in \REF{ex:mori:8}\footnote{The Italian speech from which it was extracted comprises 150 tokens uttered in just over a minute.} clearly shows the mismatch between the original and the mediated version: in the Italian phrase \textit{la nostra gente} underlines the sense of ingroupness as opposed to the Outgroup (\textit{gli immigrati}). The “reduced rendition”\footnote{A “reduced rendition” includes less explicitly expressed information than the preceding “original” utterance \citep[107]{Wadensjoe1998}} in the mediated version blurs this communicative purpose:

\ea\label{ex:mori:8}
\ea
MB2: Tutto il resto è retorica buonista che non aiuta né l'integrazione degli immigrati né tantomeno la nostra gente.
\ex
Negarlo no sirve para nada.
\z
\z

We may also find examples of original relevant expressions left untranslated, identified as cases of zero renditions, where the interpreter avoids to convey the idea of migration as a problem \REF{ex:mori:9} and, at the same time, he/she refers to efforts accomplished to face it rather than mentioning the necessity for future endeavors in accordance with the original speech:

\ea\label{ex:mori:9}
\ea
DS7: Noi sappiamo che \textbf{il} \textbf{problema} è italiano, ma è anche europeo. Occorre un considerevole sforzo finanziario perché avvenga questo e avvenga questo in un quadro di politiche coordinate (\ldots)
\ex
Se han desplegado grandes esfuerzos también por parte de la Unión Europea para que nos podamos dotar de un marco de políticas coordinadas (\ldots) \textbf{(Ø)}
\z
\z


\subsection{Mediation of predicational strategies}\label{sec:mori:4.2}

One of the selected predicational strategies used by Italian speakers to codify their political message concerns the addition of some modifiers to nouns referring to the phenomenon itself by using either NP collocates [N+Adj] or compounds such as: 

\begin{itemize}
\item
\textit{immigrazione clandestina} (RA1, DS1, RB1, MB2, RA3, RA4, MM7, DS7, SA8);
\item
\textit{immigrazione illegale} (SA1, MB11, MB11); 
\item
\textit{immigrazione irregolare} (RB1); 
\item
\textit{problema dell’immigrazione} (MM10, RA1, SA8); 
\item
\textit{emergenza immigrazione} (FP12, DS7).
\end{itemize}

In mediated discourses in Spanish, interpreters manage to express the original global meaning through word-for-word translation: the collocate \textit{immigrazione clandestina} is translated into \textit{inmigración clandestina}, opting for formal adherence.\footnote{This occurs seven times out of nine (RA1, DS1, DS1, RB1, MB1, RA3, DS2), despite the fact that \textit{inmigración ilegal} is the solution adopted in the official written translations where this is used exclusively.}

In one case, a lack of pragmatic equivalence is reported: \textit{problema} \textit{dell’immi\-gra\-zio\-ne} is translated into \textit{tema de la inmigración} (MM10), with the consequent loss of evaluation concerning immigration as a problem.

In \tabref{tab:mori:3} other noun phrases are used to predicate on the topic.

\begin{table}
\begin{tabularx}{\textwidth}{lXX}

\lsptoprule

{\bfseries Speakers} & {\bfseries NPs (original version)} & {\bfseries NPs (interpreted versions)}\\
\midrule
AC1 & non siamo in una situazione normale & no están en una situación normal\\
\multirow[t]{2}{*}{CF7} & a. questi rivolgimenti di dimensione potenzialmente epocale
& a. (este) acontecimiento de dimensión enorme\\
& b. flusso straordinario di immigrati &
b. flujo (Ø) de inmigrantes\\
MB7 & uno stravolgimento storico & un cambio histórico\\
MB8 & un’emergenza senza precedenti & una emergencia sin precedentes\\
\multirow[t]{2}{*}{MBO8} & a. questa emergenza epocale & a.esta situación de emergencia (Ø)\\
& b. emergenza anche umanitari) di carattere straordinario &
b. situación de emergencia humanitaria (Ø) también\\
MM4 & fenomeno epocale & fenómeno (Ø)\\
MM7 & sommovimento epocale & movimiento que marca época\\
SA8 & eventi straordinari & hechos extraordinarios\\
\multirow[t]{2}{*}{SI8} & a. pressione migratoria eccezionale
& a. presión migratoria excepcional\\
& b.una situazione eccezionale &
b. esa situación excepcional\\
\lspbottomrule
\end{tabularx}
\caption{Noun Phrases describing the social phenomenon.}
\label{tab:mori:3}
\end{table}

As we can see, perfectly equivalent examples of word-for-word translation are attested (SI8 a--b, SA8, AC1, MB8). As for re-elaboration (MM7, CF7 a, MB7), these cases affect the nouns within NPs more than modifiers, and the pragmatic result is a mitigation of the original meaning and, consequently, of the speaker’s attitude toward the phenomenon. In case of omissions (MBO8 a--b, MM4, CF7 b), they concern modifiers and they can be categorised as “skipping omissions”, that is “the omission of a single lexical item such as a qualifier or a short phrase which appears to be skipped over by the T and which is of minor consequence” \citep[275]{Barik1975}. Certainly, omissions cannot be considered a relevant mistake with regard to the general meaning. However, from a pragmatic point of view, they can provoke a loss of the emphasis as far as the dimension of the phenomenon is concerned. This may also be observed in \REF{ex:mori:10} where the interpreter omits the final portion of the sentence, thus affecting the semantic content and the speaker’s intent: the immigration not being a resource during troublesome times.

\ea\label{ex:mori:10}
\ea
MB2: Una strategia davvero utile che deve basarsi su alcuni punti fermi: lotta all'immigrazione clandestina lungo la frontiera sud, condivisione tra tutti gli Stati europei degli oneri a contrasto della clandestinità, politica di accordi con paesi terzi, soprattutto il riconoscimento \textbf{che} \textbf{l'immigrazione} \textbf{nel} \textbf{momento} \textbf{della} \textbf{crisi} \textbf{non} \textbf{è} \textbf{una} \textbf{risorsa}.
\ex
Hay una serie de puntos que tendrían que ser determinantes: limitar la inmigración clandestina en el Sur, llegar a acuerdos con países terceros en materia de inmigración. \textbf{(Ø)}
\z
\z

Another common predicational strategy in migration discourse concerns the use of definite or approximate numbers to quantify the extent of the phenomenon especially when referring to migrants’ arrivals: “the very first attribute applied to immigrants coming to the country is in terms of their numbers” \citep[79]{VanDijk2002}. The relevance of the phenomenon is expressed either through the use of the \textit{topos} of numbers\footnote{Occurrences of the \textit{topos} of numbers are reported in the following speeches: SI5, MB7, SI7, CF7, MBO10, DS10, MB012, MM8, RC7, RC8, FP8 b, SA1, RA1, RC8, NR4, RB1 and MB8.} \REF{ex:mori:11} or by means of general quantifications\footnote{General quantifications characterise the following speeches: CF7, MB8, DS1, MB7, SA1.} \REF{ex:mori:12}:

\ea\label{ex:mori:11}
\ea
RC8: Fra questi c'erano \textbf{4.500} \textbf{minori}, ragazzini di 12--13 anni, che condividevano quella condizione, una condizione di disagio che condividevano anche gli abitanti di Lampedusa
\ex
En fin, situaciones desastrosas. \textbf{500} \textbf{menores}, niños de doce, trece años que vivían en estas condiciones. \textbf{(Ø)}
\z
\z

From the interpreter’s side, numbers can be very difficult elements to translate, as example 11 shows.\footnote{It is the only case in the corpus where the interpreter reproduces a wrong number rather than using a more general quantification.} In this regard, \citet[19--20]{Pearl1999} states: 

\begin{quote}
The trouble with asemantic elements is that they are not part of a semantically linked chain, but just so many unconnected or loose links which cannot be inferred or anticipated from the speech flow. [\ldots] When the interpreters find the semantic flow interrupted by figures and are forced to abandon their lag, their attempts to grapple with them, as often as not unsuccessful, to take up a disproportionate amount of their time and attention, with the net result that not only are the figures themselves garbled, mangled or omitted, but the surrounding semantic material also suffers damage or omission in the confusion.
\end{quote}

In our collection of texts, the number of arrivals was represented through figures or more general quantifications (such as “dozens of” and “many”) and it has been emphasised through the recourse of temporal deixis in order to highlight the short time span in which disembarkations occurred \REF{ex:mori:12} and to the repetition of the number to enhance its scope \REF{ex:mori:13}:\footnote{It must be noted that the Spanish version is mitigated: the interpreter uses \textit{parece ser} (‘it seems to be’) rather than \textit{è} (‘is’).}

\ea\label{ex:mori:12}
\ea
NR4: La sua minuziosa contabilità è arrivata a 4.200 vittime, \textbf{18} delle quali \textbf{lo} \textbf{scorso} \textbf{marzo}: una vera ecatombe
\ex
Parece ser que hay 4.200 víctimas en, sobre todo \textbf{en} \textbf{Marzo} \textbf{de} \textbf{este} \textbf{año} ha habido \textbf{muchísimas} \textbf{víctimas.}
\z
\z

\ea\label{ex:mori:13}
\ea
RC7: Invece facciamo diventare un’emergenza umanitaria il fatto che una delle più grandi potenze mondiali, l'Italia, abbia a fronteggiare il problema di come accogliere \textbf{5.000,} \textbf{5.000} \textbf{persone}.
\ex
En cambio, lo que hacemos es convertir en una urgencia humanitaria el hecho de que una de las mayores potencias mundiales, que es Italia, tiene que hacer frente a como acoger a \textbf{(Ø)} \textbf{5.000} \textbf{personas}.
\z
\z

The above-mentioned examples clearly show that interpreters do not always succeed in translating numbers. Sometimes they may opt for a substitution with approximate quantifications,\footnote{Approximate quantifications instead of figures can be found in the following mediated-texts: MB8 and NR4.} as in the second part of the sentence in \REF{ex:mori:12}, where number 18 is replaced by the general quantification \textit{muchísimas}. In this way, the rendition lacks precision, and it affects the pragmatic effect since the number refers to individuals and the speaker had explicitly adopted a personification strategy.

Moreover, in example \REF{ex:mori:13}, the interpreter omits the repetition of the number:\footnote{Omissions can be found in mediations of RC7, MBO10 and MB7.} in this case, the omission does not alter the semantic content, but it affects the communicative equivalence since the speaker’s intent was to emphasise the phenomenon size through the repetition.

Another predicational strategy is realised through predicates and it is implicitly connected with the \textit{topos} of threat in order to highlight the importance of reducing -- or even stopping -- the arrivals or the need to take measures to face the emergency. In all cases, interpreters manage to transmit the original meaning, through a word-for-word translation or by using synonymic expressions, as we can see from the following examples:

\ea\label{ex:mori:14}
\ea
\textit{far fronte a} / \textit{affrontare} / \textit{fronteggiare} > \textit{hacer frente}\footnote{In MP8, MM12, FP12, FP7, MM7, PP7, FP8 a, SA8, CC8, MB7, RA7, SS7, MB8, MBO8, BM8, MBO10, DS2, RC7.}
\ex
\textit{bloccare} > \textit{bloquear} (RC7)
\ex
\textit{frenare} > \textit{frenar} (PP7)
\ex
\textit{porre un limite} > \textit{limitar} (RBA8)
\ex
\textit{governare} > \textit{governar} (GP12)
\ex
\textit{prevenire} > \textit{prevenir} (GLV8)
\ex
\textit{contenere} > \textit{contener} (RA1)
\ex
\textit{fermare} > \textit{poner frenos} (PP7)
\z
\z

Related to the \textit{topos} of threat some verbs, such as \textit{invadere} (‘to invade’) and the metaphoric \textit{prendere di mira} (‘to target’) can be considered since they aim to emphasise negative evaluation of these arrivals:

\ea\label{ex:mori:15}
\ea
MB8: L'Italia \textbf{è} da settimane \textbf{presa} \textbf{di} \textbf{mira} da centinaia di barconi di clandestini. Lampedusa \textbf{è} \textbf{stata} \textbf{invasa} da decine di migliaia di nordafricani che l'isola, mai e poi mai, potrebbe accogliere.
\ex
Lampedusa \textbf{ha} \textbf{sido} \textbf{invadida} por centenas de miles de norteafricanos que la isla no puede acoger. Italia, desde hace semanas, \textbf{tiene} \textbf{a} muchas embarcaciones clandestinas.
\z
\z

As we can see, the interpreter did not manage to transmit the same intent by translating \textit{invadere} as \textit{tener a}, which is a sort of mitigation. 

In \REF{ex:mori:16} a reduced rendering of the referential expression is adopted since the Outgroup is identified with migrants (rather than illegal migrants) and the different morphosyntactic structure (from passive to active) blurs the role of the Italian government in the mediated-text.

\ea\label{ex:mori:16}
\ea
MB8: I clandestini \textbf{devono} \textbf{essere} \textbf{spediti} a casa loro. 
\ex
Los inmigrantes \textbf{tienen} \textbf{que} \textbf{volver} a sus países.
\z
\z

Finally, from a semantic point of view, there are verbs expressing a prolonged action over time, such as \textit{continuare} (or \textit{permanere}),\footnote{In MB8, MB8, RBA8, MS8, SI10, SA1, RBA8, SA1.} used to declare the daily problem local people have to face and the key role Italy plays by giving shelter to them. Examples in \REF{ex:mori:17} show the way interpreters succeed in maintaining a perfect equivalence by selecting predicational strategies that show speakers’ defending and siding with the Outgroup:

\ea\label{ex:mori:17}
\ea
 MS8: (\ldots) \textbf{continuano} gli sbarchi, la gente muore in mare e si affolla in condizioni disumane sulle coste italiane e maltesi (…)
\ex
(\ldots) \textbf{siguen} \textbf{produciendose} desembarcos, la gente muere en las costas italianas y maltesas (…)
\ex
RBA8: L'Italia \textbf{continua} a fare la sua parte nell'accoglienza a questi disperati.
\ex
Italia \textbf{sigue} desempeñando su papel en la acogida que se da a estos desesperados. 
\z
\z


\subsection{Mediation of argumentative strategies}\label{sec:mori:4.3}

The preliminary pragmalinguistic analysis carried out on original interventions allowed us to identify the main argumentation strategies adopted by Italian politicians with the objective of focusing on the way interpreters manage to take them into consideration in their mediated texts.

Usually, speakers exploit argumentative strategies to create a positive representation of themselves (and therefore to discredit other political ideologies) by either showing empathy towards migrants or representing them in a negative way, explaining the reason why migration has to be contrasted.

This may be observed when referring to personal experiences and to concrete examples aimed at giving a sort of empirical proof and more accountability to what is being said:

\ea\label{ex:mori:18}
\ea
S17: \textbf{(\ldots)} \textbf{Io} \textbf{sono} \textbf{siciliano} e sono stato, \textbf{contrariamente} \textbf{a} \textbf{molti} \textbf{altri} \textbf{che} \textbf{hanno} \textbf{qui} \textbf{parlato}, nel centro di identificazione ed espulsione. Quel centro era un centro che è servito in quel momento ma che, dopo l’accordo sul trattato di amicizia, cessava di avere alcuna validità.
\ex
\textbf{Yo} \textbf{soy} \textbf{siciliano,} \textbf{(Ø)} por cierto, existía un centro de acogida que se utilizaba a la sazón, pero tras el acuerdo de amistad entre Libia e Italia ya no servía de nada ese centro de acogida.
\z
\z

In   example \REF{ex:mori:18}, the speaker aimed to stigmatise other politicians’ behaviour by referring to his own experience, being of Sicilian origin. It is to be noted that, in this speech, the interpreter does not create an equivalent version, due to the omission of this important contrast.

\begin{sloppypar}
Among other argumentative schemes used by speakers in their political speeches we may find two antithetic strategies: (a) the pro-migration argumentation; (b) the anti-migration argumentation.
\end{sloppypar}

The former (a) highlights a positive attitude towards migrants who are not referred to through a binary opposition (Them \textit{vs} Us; Outgroup \textit{vs} Ingroup), while the latter (b) enhances the ingroupness to the local community, thus strengthening ideological polarisation (\sectref{sec:mori:3}).

The first category comprises the \textit{topoi} of history, humanitarianism and “Italy as a country of migration”, the latter consists of \textit{topoi} of fear, disadvantage and the \textit{topos} of burden sharing.\footnote{See also example \REF{ex:mori:9} where the speaker refers to an Italian problem as being a European one as well.}

Being standard arguments shared by a wide range of speeches belonging to the migration domain, interpreters managed to transmit the pragmatic meaning in all occurrences. For instance, the following example shows the rendering of the \textit{topos} of “history as a teacher” into Spanish:

\ea\label{ex:mori:19}
\ea
DS1: Vogliamo sapere, signor Presidente, se la Commissione intende intervenire sulla legislazione italiana, verificare l'accordo italo-libico. (\ldots). \textbf{Non} \textbf{possiamo} \textbf{consentire} \textbf{vent'anni} \textbf{dopo} \textbf{la} \textbf{caduta} \textbf{del} \textbf{Muro} \textbf{di} \textbf{Berlino} \textbf{ad} \textbf{alcuni} \textbf{governi} \textbf{di} \textbf{alzarne} \textbf{di} \textbf{nuovi.}
\ex
Queremos saber, señor Presidente, si la Comisión intervendrá en la legislación italiana, comprobando ese acuerdo italo-libio. \textbf{No} \textbf{podemos} \textbf{permitir} \textbf{que,} \textbf{después} \textbf{de} \textbf{la} \textbf{caída} \textbf{de} \textbf{Berlín,} \textbf{veinte} \textbf{años} \textbf{después,} \textbf{algunos} \textbf{Gobiernos} \textbf{vuelvan} \textbf{a} \textbf{levantar} \textbf{otros} \textbf{muros}.
\z
\z

Similarly, in \REF{ex:mori:20} we see that the interpreter succeeds in translating the overall meaning of the original text \textit{topos} of burden sharing:

\ea\label{ex:mori:20}
\ea
DS8: (\ldots) \textbf{Domani,} \textbf{in} \textbf{Parlamento} \textbf{faremo} \textbf{la} \textbf{nostra} \textbf{parte.} \textbf{Però} \textbf{occorre} \textbf{che} \textbf{anche} \textbf{gli} \textbf{altri} \textbf{facciano} \textbf{la} \textbf{loro,} che i governi siano molto meno egoisti, che la solidarietà che serve per attivare una politica europea, beh, ci veda promotori. In questo, il suo lavoro naturalmente è al centro di questo sforzo, perché senza i governi l'Europa sarà più debole.
\ex
\textbf{Mañana} \textbf{en} \textbf{el} \textbf{Parlamento,} \textbf{vamos} \textbf{a} \textbf{hacer} \textbf{lo} \textbf{que} \textbf{nos} \textbf{toca} \textbf{hacer} \textbf{pero} \textbf{los} \textbf{demás} \textbf{también} \textbf{tienen} \textbf{que} \textbf{hacerlo}. Los gobiernos tienen que ser mucho menos egoístas, la solidariedad necesaria para aplicar una política europea. Pues, tenemos que ser nosotros los promotores de ella. Y su trabajo está en el centro de esos esfuerzos, porque, sin los gobiernos, Europa será más débil. 
\z
\z

In the following extract \REF{ex:mori:21} the interpreter successfully conveyed the global meaning though he/she failed to express the “involvement strategy”\footnote{“Strategies of involvement (see \citealt[9--35]{Tannen1989}) aim both at expressing the speakers’ inner states, attitudes and feelings or degrees of emotional interest and engagement and at emotionally and cognitively engaging the hearers in the discourse” \citep[81]{ReisiglWodak2001}.} by omitting the adverb \textit{purtroppo} (‘unfortunately’) which plays the function of pragmatic clue:

\ea\label{ex:mori:21}
\ea
SA1: \textbf{Purtroppo}, il Mediterraneo è ormai diventato un cimitero a cielo aperto e il governo Berlusconi, quindi il governo italiano, ha adottato un accordo con la Libia che consente \textbf{purtroppo} il respingimento non solo di migranti, ma consente anche il respingimento dei rifugiati che provengono da paesi dove sono in atto persecuzioni, guerre civili, come la Somalia e l'Eritrea, e nega a questi poveri disgraziati il diritto di chiedere asilo, violando così non solo tutte le norme internazionali, ma viola soprattutto la Convenzione di Ginevra.
\ex
\textbf{(Ø)} El Mar Mediterraneo se ha convertido en un cementerio a cielo abierto y el Gobierno Berlusconi, el Gobierno de Italia ha aprobado un acuerdo con Libia que \textbf{(Ø)} permite dar no acogida a los inmigrantes y refugiados que vienen de países donde hay guerra civil como Somalia, Eritrea y niega a esos pobres desgraciados el derecho de asilo, violando así no solo todas las normas internacionales, sino que procede a una violación del Convenio de Ginebra.
\z
\z

In \REF{ex:mori:21}, the interpreter did not translate the adverb \textit{purtroppo,} probably not considered to be relevant, though the Italian speaker used it twice to stress her own opinion and attitude toward the topic. By contrast, the reference to an authoritative source (\textit{Geneva Convention}) to legitimate a statement and speaker’s personal opinion was rendered in the Spanish mediated-text.

In the following example \REF{ex:mori:22} the use of metaphors to build political argumentation is worth noting: metaphors that aim at emphasising and clarifying the propositional content are often used.\footnote{See \citet{Taylor2020}. For a list of the functions of metaphors in discourse, see \citet[30--32]{Semino2008}.} In this case the reference to football metaphor as the most popular sport\footnote{In this sense, according to \citet[98]{Semino2008}: “since cultures and countries differ in their sporting preferences, different sport metaphors tend to dominate in different languages and countries.”} in Italy, gives rise to

\begin{quote}
 a new type of political discourse in the Italian context, one that abandons the traditional obscurity and replaces it with vivid and relatively simple references to domains that are likely to be accessible to a wide audience. \citep[266]{Semino1996}
\end{quote}

\ea\label{ex:mori:22}
\ea
MM9: C'è \textbf{una} \textbf{partita} \textbf{da} \textbf{giocare}, quindi. Ciò che mi colpisce è che mi sembra che in alcune circostanze \textbf{i} \textbf{giocatori} \textbf{della} \textbf{partita} degli ideali rinunciano a \textbf{giocare} \textbf{la} \textbf{partita}. (\ldots) Ora, come fa la squadra degli ideali a \textbf{vincere} \textbf{la} \textbf{partita} se \textbf{i} \textbf{nostri} \textbf{giocatori} rinunciano a tirare in porta magari perché pensano che \textbf{il} \textbf{portiere} è troppo bravo? (…) E allora, mi permetto fare questa osservazione: chi sono \textbf{i} \textbf{giocatori} \textbf{dell'attacco}? Sono le Istituzioni europee: Parlamento, Commissione e anche lei, signor Presidente Van Rompuy. (\ldots). Vi chiedo allora semplicemente: \textbf{siete} \textbf{i} \textbf{giocatori} \textbf{del} \textbf{nostro} \textbf{attacco}, \textbf{passatevi} \textbf{la} \textbf{palla,} \textbf{giocate} \textbf{all'attacco,} \textbf{fate} \textbf{goal} e come si dice in questo tipo di partite, fateci sognare.
\ex
Hay, por tanto, \textbf{una} \textbf{partida}. Lo que me surprende es que, en algunas circumstancias, parece que \textbf{los} \textbf{jugadores} \textbf{del} \textbf{partido} de los ideales renuncian a \textbf{jugar} \textbf{el} \textbf{partido}. (\ldots) Entonces, ¿Cómo puede el equipo de los ideales \textbf{ganar} \textbf{el} \textbf{partido} si \textbf{sus} \textbf{jugadores} se niegan a \textbf{tirar} \textbf{a} \textbf{gol} porque quizá piensan que \textbf{el} \textbf{portero} es demasiado bueno? (\ldots) Por ello me permito hacer este comentario. ¿Quiénes son \textbf{los} \textbf{jugadores} \textbf{al} \textbf{ataque}? Las instituciones europeas, el Parlamento, la Comisión y también usted, señor Presidente Van Rompuy. (\ldots) Solamente le pregunto: ¿\textbf{Son} \textbf{jugadores} \textbf{de} \textbf{nuestro} \textbf{ataque}? \textbf{Pues} \textbf{pásense} \textbf{la} \textbf{pelota,} \textbf{juguen} \textbf{al} \textbf{ataque,} \textbf{ganen} \textbf{el} \textbf{partido}, hágannos soñar, \textbf{marquen} \textbf{un} \textbf{gol}.
\z
\z

As for the interpreted version, in this case the interpreter uses the same football metaphor in Spanish\footnote{See \citet{Turrini2004} for an explanation of equivalent metaphors, as well as the use of different images or paraphrase in the interpreted versions of texts.} and, since football is a popular sport in Spain, the pragmatic effect is totally maintained. 

In \REF{ex:mori:23} an example the equivalent pragmatic effect of the original metaphor used in Italian is obtained through the use of figurative language:

\ea\label{ex:mori:23}
\ea
SA8: Pur in presenza dell'articolo 80 del trattato sul funzionamento dell'UE e del principio dell'equa ripartizione della solidarietà, ogni paese di fatto \textbf{tira} \textbf{acqua} \textbf{al} \textbf{proprio} \textbf{mulino} e l'atteggiamento della Francia al confine con l'Italia è inammissibile nell'attuale quadro europeo.
\ex
Aunque tenemos el artículo 80 del tratado de funcionamiento de la Unión Europea y de un reparto equitativo de la solidaridad, cada país \textbf{intenta} \textbf{barrer} \textbf{hacia} \textbf{sus} \textbf{puertas} y, de hecho, la postura de Francia en las fronteras con Italia es inadmisible.
\z
\z

In fact, in the interpreted version of example \REF{ex:mori:23}, the fixed multi-word expression \textit{barrer para casa} (‘to look after number one’) is used with an equivalent pragmatic meaning though the interpreter increases the creativity (see \citealt[21]{Semino2008}) by substituting \textit{casa} (‘house’) with \textit{puertas} (‘doors’).

The metaphor used in \REF{ex:mori:24} is rendered through a paraphrase that preserves the semantic equivalence, though not the pragmatic meaning. In fact, the Italian metaphor referring to the gruyere cheese (\textit{groviera}) and its holes (\textit{buchi}) is used to give the idea of vulnerability of Romania and Bulgaria. In this case, the same idea is rendered without using a metaphor by the interpreter, who simply refers to these countries as places used to enter in the European Union, thus causing a loss in terms of political rhetoric. Besides this, it has to be remarked upon that the metaphor used to refer to Bulgaria and Romania in the original speech was extended to the whole European Union in the mediated text, thus producing a distorted meaning.

\ea\label{ex:mori:24}
\ea
MBO11: \text{Suggerisco} la necessità di una sospensione della procedura di entrata di Bulgaria e Romania nel sistema di Schengen sulla base del principio di precauzione, anche in vista della prevedibile enorme pressione dalle frontiere esterne verso questi due paesi che \textbf{diventano} \textbf{la} \textbf{groviera,} \textbf{i} \textbf{buchi} \textbf{di} \textbf{groviera} \textbf{del} \textbf{sistema} dell'Unione europea rispetto all'entrata dei clandestini.
\ex
Por eso \textbf{pido} la suspensión del procedimiento de adhesión de Bulgaria y Rumania en Schengen, basándose en el principio de precaución y también a la vista de la presión enorme previsible que provendrá de estas fronteras exteriores hacia la Unión Europea que se \textbf{van} \textbf{a} \textbf{convertir} \textbf{en} \textbf{los} \textbf{lugares} \textbf{por} \textbf{donde} \textbf{pasen} todas esas personas que son victimas de la trata de seres humanos.
\z
\z

From our perspective what is even more remarkable in \REF{ex:mori:24} is the avoidance of an equivalent for the Italian verb \textit{suggerire} (\textit{suggerisco}) which is not used to build the speaker’s argumentation in Spanish. This provokes an imperfect equivalence in terms of “degrees of strength” because this verb has a lower illocutionary force compared with the Spanish verb \textit{pedir (pido)}. In this way, the mitigation intended to be used by the Italian speaker is not rendered into the Spanish version.



\section{Conclusive remarks}\label{sec:mori:5}

By adopting the analytical categories of the DHA developed within the framework of Critical Discourse Analysis our study focused on the mediation of discursive strategies related to migration/migrants and of ethnopragmatic devices that embody Self-representation and the (negative/positive) construction of the Other.

Analysis of migration discourse was conducted on a corpus of EP original speeches in Italian by MEPs during plenary sessions in order to pinpoint referential expressions used to designate social actors and the social phenomenon of international migration in itself. Predicational strategies used to discuss the selected referents and the way speakers organise their entire argumentation flow, in accordance with their political pro- or anti-immigration stance, were also examined. In fact, within a political environment, speakers might manifest their ideology and attitudes through their pragmalinguistic behaviour, which plays a fundamental role in building his/her political Self. Thus, beyond the locutionary aim of any political statement, interpreters of political speeches are asked to render the perlocutive dimension of the political message enacted in the original.

This preliminary analysis on 60 speeches by twenty-five politicians allowed us to uncover the most potentially interesting statements due to their ethnopragmatic value and to detect possible shifts introduced when interpreting into Spanish. Our pragmatically-oriented approach allowed us to pinpoint the main strategies used to convey speakers’ political message when discussing the social phenomenon of migration. In particular, it was possible to identify either inclusive or exclusive discursive strategies (namely referential strategies, predicational strategies and argumentative strategies) aimed at the Other construction through or a positive/negative description of the phenomenon itself and of migrants as social actors.

Our corpus was conceived in order to balance the effect of a potentially relevant variable, such as the speaker’s political group in selecting peculiar discursive strategies to reach a pragmatic intent converging toward two ideology-induced poles: neutralising attitude vs. stigmatising attitude. In order to fully understand pragmalinguistic cues it was necessary to take into consideration the national political party of MEP’s and its ideological orientation and political rhetoric on the Italian socio-cultural background.

Our data on the original speeches show that there seems to be a distinction as far as referential strategies are concerned. Namely, criminonyms (“delinquents” or “illegal migrants”), rather than neutral referential expressions, are used in the 93\% of cases by EFD members belonging to the Italian \textit{Lega Nord} party. Reference to the social condition as permanent (“immigrated”) is sligthly more attested (64\%) within EPP (from \textit{Lega Nord} and \textit{Fratelli d’Italia}) while the frequency of the variant “migrant” is perfectly balanced and it does not seem to be correlated with the speaker’s political orientation.

In order to name the social phenomenon and predicate on it the water metaphor is mainly used (73\%) by EPP and EFP exponents (from \textit{Fratelli d’Italia} and \textit{Lega Nord}). Other predicational strategies are used to support the anti-migration argumentation: the \textit{topos} of threat is overtedly adopoted in examples 15 and 16 by EFD (\textit{Lega Nord}). Differently the pro-migration argumentation develops with the topos of numbers to emphasise the personification in example 12 by ALDE (\textit{Italia dei Valori})\footnote{This political affiliation refers to the time span of speeches here considered.} and examples 11 and 13 by S\&D (\textit{Partito Democratico}), through reference to the Italian welcoming tradition in the second example reported in 17) and to the \textit{topos} of history in example 19, both by S\&D members (\textit{Partito Democatico}), as well as to the \textit{topos} of humanitarism in example 21 by ALDE (\textit{Italia dei Valori}).

By considering the European Parliament-Nation Parliament correspondence there emerges a more definite mapping that mirrors opposing ideologies that drive the politicians’ discursive strategies on this topic. This preliminary consideration clarifies the fundamental importance of the pragmatic awareness of interpreters and of any other involved in the process of recodifying political messages in terms of representation of the political Self.

Our main goal was to contrastively analyse the interpreter-mediation into Spanish of the most relevant discursive strategies (namely referential, predicational and argumentative ones) in mediated migration discourse. Data presented in \sectref{sec:mori:4} resulted from our qualitative analysis aimed at describing if and how interpreters maintain, or fail to reproduce, the pragmatic equivalence of the source political intent. As a matter of fact, interpretations into Spanish were analysed in accordance with an ethnopragmatic approach by focusing on the way interpreters are handling to transfer not only the politician’s message but also his/her attitude toward a specific topic (migration), complying with her/his political ideology, in a given situational context (see \citealt{BuelowMoeller2003}).

Our results illustrate different strategies applied by interpreters to achieve their mediation task:

\begin{itemize}\sloppy
\item word-for-word translation, whenever possible, to ensure semantic and pragmatic equivalence;
\item synonymy or re-elaboration of the original strategies in order to obtain the same perlocutive effect;
\item omissions through zero renditions or reduced renditions.
\end{itemize}

Attention was also paid to the entire argumentation flow developed in each oral text by focusing on the mediation of common anti-migration or pro-mi\-gra\-tion \textit{topoi}, the exploitation of aesthetic devices, such as metaphors, and figurative language referring to water and football, together with the practice of reporting speakers’ own experiences, thus introducing their private domain for the construction of their political Self.

In general, although our qualitative analysis put in evidence the high-degree of semantic correspondence between original discourse and interpreter-mediated one, it was possible to detect cases of partial/full loss of pragmatic equivalence during the interpreting process, thus affecting the pragmatic encoding of the speaker’s perspective and the rendering of his/her intentionality as far as a specific topic (migration) is concerned.

\begin{sloppypar}
From our research perspective, the most interesting examples are those where it is clear that interpreters focused on the semantic content without paying enough attention to (or not being enough aware of) the value of speakers’ pragmalinguistic choices or opting for mitigation strategies.\footnote{It has also to be noted that this failure in recoding speaker’s intentionality could depend on the speed of speeches uttered by Italian speakers - considered that they generally read them - and on the high information density due to the shortness of this genre.} In both cases, this might affect the representation of the political Self and, in this specific context, the rendering of his/her attitude toward ethical issues.\footnote{These results are aligned with some trends observed by \citet{Coppola2019} in her analysis of agency in interpreter-mediated speeches during bilateral (Italian-German) institutional encounters.} This emerged especially as far as the mitigation of pragmatic cues is concerned by:
\end{sloppypar}

\begin{itemize}
\item neutralising the identification strategy: \textit{topos} of numbers in examples \REF{ex:mori:12}--\REF{ex:mori:13};
\item omission of the Ingroup-Outgroup dynamics: example \REF{ex:mori:8};
\item blurring of the performative agency as for the political Self representation: example \REF{ex:mori:18};
\item change in terms of represented agency (passive-active diathesis) and indexical value expressed by the original verb: \textit{spedire via qualcuno} (‘to drive out’) in example \REF{ex:mori:16};
\item deletion of orientation markers such as \text{purtroppo} (‘unfortunately’) which complies with the politician’s involvement strategy: example \REF{ex:mori:21};
\item lack of aesthetic agency devices: example \REF{ex:mori:24}.
\end{itemize}

Intensification cues of the original pragmatic purpose because of the “intrusion” of the interpreter’s Self are also reported, thus provoking a pragmatic shift:

\begin{itemize}
\item 
metaphorisation by means of water metaphors contributing to the representation of the situation as particularly serious and dangerous, as in example \REF{ex:mori:4};\footnote{The domain of flooding (and natural disasters in general) is typical of anti-immigration, racist or xenophobic discourses \citep[88]{Semino2008}.}
\item 
addition of evaluative components: example \REF{ex:mori:5};
\item 
gender-fairness of referential expressions: example \REF{ex:mori:7}.\footnote{The interpreter’s choice toward a gender-inclusive solution -- through the explicitation of both genders (men and women) -- is not self-evident since the debate is still going on within national and supranational institutions. The issue is multi-faceted and deserves special attention when mediating oral or written texts. For some considerations on this topic from a multilingual perspective see \citet{CavagnoliMori2019}. For considerations on the use of gender-neutral language by the European Parliament refer to \citet{EP2018}.}
\end{itemize}

In conclusion our study highlights the relevance of the pragmatic dimension when mediating oral discourse, providing evidence of the way speakers are using their speech not only to deliver content but (rather) to perform linguistically their political Self and ideologies especially when socially sensitive topics such as migration are discussed. A caveat must be borne in mind: the dataset we are dealing with is limited and we cannot have access to information regarding the interpreters and institutional constraints on their output.

Cross-pollination of this research area with Interpreting Studies is undoubtedly valuable and, as far as ethnopragmatically-oriented analysis of agency and indexicality are concerned, research directions may be manifold. Furthermore, applications are also envisaged for interpreter training (see \citealt{BoydMonacelli2010}) where knowledge of theoretical models in the field of political discourse analysis could enhance students’ meta-pragmatic awareness. 

\section*{Acknowledgments}

This paper arises from the joint collaboration between the authors. For scientific purposes sections are authored as follows: Ilaria Anghelli (Sections 2, 4); Laura Mori Sections (1, 3, 5). Authors are grateful to Michael S. Boyd for fruitful discussion on the topic and they are indebted to anonymous reviewers for their insightful observations.


\appendixsection{Topics}\label{ap:mori:1}

\begin{enumerate}[label = {Topic \arabic*:}, align = left]
\item Immigration, the role of Frontex and cooperation among Member States (debate) (15-11-2009)

\item Stockholm Action Plan (debate) (18-05-2010)

\item  European Refugee Fund for the period 2008 to 2013 (amendment of Decision No 573/2007/EC) (18-05-2010)

\item Union for the Mediterranean (20-05-2010)

\item Cost of examining asylum seekers’ applications in Member States (19-01-2011)

\item State of European asylum system, after the recent decision of the European Court of Human Rights (15-02-2011)

\item Immediate EU measures in support of Italy and other Member States affected by exceptional migratory flows (15-02-2011)

\item EU response to the migration flows in North Africa and the Southern Mediterranean, in particular, in Lampedusa - Migration flows arising from instability: scope and role of EU foreign policy (04-04-2011)

\item Conclusions of the European Council meeting (24--25 March 2011) (05-04-2011)

\item Migration flows and asylum and their impact on Schengen (10-05-2011)

\item Application of the provisions of the Schengen acquis relating to the Schengen Information System in Bulgaria and Romania (07-06-2011)

\item Preparations for the European Council meeting (24 June 2011) (continuation of debate (22-06-2011)
\end{enumerate}


\appendixsection{Speaker and speech context}\label{ap:mori:2}




\begin{tabularx}{\textwidth}{llll}

\lsptoprule

{\bfseries Code} & {\bfseries Name} & {\bfseries Political party} & {\bfseries Topic}\\
\midrule
AC & Antonio Cancian & EPP & 1\\
AP & Alfredo Pallone & EPP & 8\\
BM & Barbara Matera & EPP & 8\\
CC & Carlo Casini & EPP & 8\\
CF & Carlo Fidanza & EPP & 7\\
DS & Davide Maria Sassoli & S\&D & 1, 7, 8\\
FP & Fiorello Provera & EFD & 7, 8 (x2), 12\\
GLV & Giovanni La Via & EPP & 8\\
GP & Gianni Pittella & S\&D & 12\\
MB & Mara Bizzotto & EFD & 2, 7, 8, 11\\
MBO & Mario Borghezio & EFD & 7, 8, 11, 12\\
MM & Mario Mauro & EPP & 4, 7, 8, 9, 10, 12\\
MP & Mario Pirillo & EPP & 8\\
MS & Marco Scurria & EPP & 8\\
NR & Niccolò Rinaldi & ALDE & 4\\
PP & Pier Antonio Panzeri & S\&D & 7\\
RA & Roberta Angelilli & EPP & 1, 3, 4\\
RB & Rita Borsellino & S\&D & 1\\
RBA & Raffaele Baldassare & EPP & 8\\
RC & Rosario Crocetta & S\&D & 7, 8\\
SA & Sonia Alfano & ALDE & 1,8\\
SC & Silvia Costa & S\&D & 8\\
SI & Salvatore Iacolino & EPP & 2, 3, 5, 7, 8\\
SS & Sergio Paolo Francesco Silvestris & EPP & 6, 7, 8\\
RB & Rita Borsellino & S\&D & 1\\
RBA & Raffaele Baldassare & EPP & 8\\
\lspbottomrule
\end{tabularx}

\sloppy\printbibliography[heading=subbibliography,notkeyword=this]
\end{document}
