\documentclass[output=paper]{langscibook}
\ChapterDOI{10.5281/zenodo.6977038}

\author{Bart Defrancq\orcid{}\affiliation{EQTIS, Ghent University} and Koen Plevoets\orcid{}\affiliation{EQTIS, Ghent University}}

\title[Ut interpres]{\textit{Ut interpres}: Linguistic convergence between orators and interpreters in the European Parliament}

\abstract{This paper combines a theoretical and an empirical approach to the analysis of converging linguistic features in speeches held by Members of the European Parliament and interpretations in the same Parliament. The theoretical approach seeks to determine which group has more seniority and therefore more expertise in the linguistic genre of the Parliament. The empirical analysis concentrates on key 3- and 4-grams used in speeches and interpretations to determine which group’s usage is more expert and can be considered the dominant group shaping the linguistic features of the genre. The two-pronged approach reveals that interpreters are the expert group and that for the items considered the case can be made that Members adopt interpreters’ lexical patterns. The study thus provides complementary evidence for \citegen{Poechhacker2005} idea that in an interpreter-mediated encounter all interactants influence each other’s communicative behaviour.}

\IfFileExists{../localcommands.tex}{
  \addbibresource{../localbibliography.bib}
  \usepackage{langsci-optional}
\usepackage{langsci-gb4e}
\usepackage{langsci-lgr}

\usepackage{listings}
\lstset{basicstyle=\ttfamily,tabsize=2,breaklines=true}

%added by author
% \usepackage{tipa}
\usepackage{multirow}
\graphicspath{{figures/}}
\usepackage{langsci-branding}

  
\newcommand{\sent}{\enumsentence}
\newcommand{\sents}{\eenumsentence}
\let\citeasnoun\citet

\renewcommand{\lsCoverTitleFont}[1]{\sffamily\addfontfeatures{Scale=MatchUppercase}\fontsize{44pt}{16mm}\selectfont #1}
  
  %% hyphenation points for line breaks
%% Normally, automatic hyphenation in LaTeX is very good
%% If a word is mis-hyphenated, add it to this file
%%
%% add information to TeX file before \begin{document} with:
%% %% hyphenation points for line breaks
%% Normally, automatic hyphenation in LaTeX is very good
%% If a word is mis-hyphenated, add it to this file
%%
%% add information to TeX file before \begin{document} with:
%% %% hyphenation points for line breaks
%% Normally, automatic hyphenation in LaTeX is very good
%% If a word is mis-hyphenated, add it to this file
%%
%% add information to TeX file before \begin{document} with:
%% \include{localhyphenation}
\hyphenation{
affri-ca-te
affri-ca-tes
an-no-tated
com-ple-ments
com-po-si-tio-na-li-ty
non-com-po-si-tio-na-li-ty
Gon-zá-lez
out-side
Ri-chárd
se-man-tics
STREU-SLE
Tie-de-mann
}
\hyphenation{
affri-ca-te
affri-ca-tes
an-no-tated
com-ple-ments
com-po-si-tio-na-li-ty
non-com-po-si-tio-na-li-ty
Gon-zá-lez
out-side
Ri-chárd
se-man-tics
STREU-SLE
Tie-de-mann
}
\hyphenation{
affri-ca-te
affri-ca-tes
an-no-tated
com-ple-ments
com-po-si-tio-na-li-ty
non-com-po-si-tio-na-li-ty
Gon-zá-lez
out-side
Ri-chárd
se-man-tics
STREU-SLE
Tie-de-mann
}
  \togglepaper[1]%%chapternumber
}{}

\begin{document}
\maketitle
%\shorttitlerunninghead{}%%use this for an abridged title in the page headers




\section{Introduction}

Cicero’s (46 BCE) self-reported translation method \textit{nec ut interpres, sed ut orator} (‘not as an interpreter, but as an orator’) is widely quoted in the translation literature as one of the oldest examples of a functionalist approach to translation (see for instance \citealt{Nord2013}). It does not seem to have met with the same kind of enthusiasm in Interpreting Studies. That is of course perfectly understandable, considering the negative view it carries on interpreters (although the Latin \textit{interpres} covers both translators and interpreters, as well as mediators and exegetes). In this study we will subvert Cicero’s quote and ask ourselves if there is evidence that orators speak \textit{ut interpretes}, like interpreters, and more in particular simultaneous interpreters.

Since the 1990s, the theoretical work on simultaneous interpreting has increasingly made room for functionalist thinking, albeit at a slower pace than in other areas of interpreting research. \citet{Poechhacker1994} made a first comprehensive attempt at transferring functionalist theories of translation to conference interpreting, categorising and describing its various \textit{skopoi}. Major empirical landmarks by \citet{Diriker2004} and \citet{Monacelli2009} followed, illustrating simultaneous interpreters’ agency during conference assignments. Yet, for all the progress that was made, it seems that the functionalist approach has not yet been exploited to its full potential. 

In \citegen{Poechhacker2005} interactant model of interpreting, shown in \figref{fig:defrancq:1}, interpreting is described in terms of an interaction between (at least) three participants, each coming to the interaction with their perspective on the interaction and the interactants, embedded in their socio-cultural background.

As Pöchhacker admits, the model fits situations of triadic communication best. If we were to apply the model to simultaneous interpreting in a conference, it would certainly have to include more interactants, and, crucially, more interpreters. Interaction obviously also takes place between boothmates and even with colleagues in other booths, directly or through the \textit{chef d’équipe}. This aspect of conference interpreting is clearly under-represented in the literature and has only been thoroughly investigated by \citet{Duflou2016} with respect to turn-taking.   

Ultimately, the interactant model is designed to be a framework to describe communicative behaviour. As \citet{Poechhacker2005} puts it:

\begin{quote}
The ‘interactant model of the situation’ [\ldots] seeks to show the multiple dynamic relationships which make up the communicative situation as it ‘exists’ for a given interactant and shapes his or her communicative behaviour. \citep[688]{Poechhacker2005}
\end{quote}

\begin{figure}

\includegraphics[width=\textwidth]{figures/a1UtinterpresLinguisticconvergencebetweenoratorsandinterpretersintheEuropeanParliament-img001-new.pdf}
 

\caption{Pöchhacker’s interactant model (adapted from \citealt[689]{Poechhacker2005}).}
\label{fig:defrancq:1}
\end{figure}

The communicative behaviour that has been under most scrutiny in the literature is, quite understandably, the interpreter’s, with a specific focus on communicative behaviour that runs counter various interpretations of the so-called \textit{conduit model}, i.e. the historic normative view that interpreters produce linguistic output based on linguistic input abstaining from interfering in the communication between primary participants (\citealp{Diriker2004}; \citealp{Monacelli2009}; \citealp{Bartlomiejczyk2016}). Similarly, in studies that focus on linguistic properties (see \sectref{sec:defrancq:4}), the focus has been on the interpreters’ output and how it is shaped by aspects of the communicative situation, including the other interactants. A fairly representative view in that respect is the one voiced by \citet{Bartlomiejczyk2016}, explaining that interpreters at the European Parliament (EP) acquire keywords and expressions due to prior exposure to the primary participants and to the boothmate (similar views are held by \citealt{Duflou2016} and \citealt{Henriksen2007} for the European Commission’s DG SCIC):

\begin{quote}
Secondly, the EP discourse is characterised by a large degree of repetitiveness, which concerns certain phrases that might well be described as clichés as well as keywords. One of such keywords is, for example, \textit{solidarity}, which collocates with the adjectives \textit{European} and \textit{multinational} [\ldots]. This is conducive to experienced interpreters building up a large repertoire of ready-made translation solutions, which may be worked out individually or copied from boothmates. \citep[57]{Bartlomiejczyk2016}
\end{quote}

Acquired knowledge is obviously part of an interactant’s perspective, which, in turn, is part of the communicative situation. 

Of course, what holds true for interpreters, also holds true for the other interactants. Their exposure to interpreters’ output is likely to impact their perspective and, as a result, their communicative behaviour. This dimension is, however, poorly represented in the literature. Apart from a systematic study of references to interpreters and interpreting in members of the European Parliament’s (MEP) speeches \citep{Bartlomiejczyk2017}, and a series of quality surveys (for an overview, see \citealt{Kurz2001}), the perspectives of primary participants as shaped by their interactions with interpreters in conference situations is hardly explored. 

In this paper we set out to explore precisely that dimension. We will first review the concept of linguistic convergence and some of our own studies on lexical patterns and potential linguistic convergence between Members of the European Parliament (MEPs) and interpreters in the European Parliament. This will lead us to the research questions at the end of \sectref{sec:defrancq:2}. These will focus on the potential role of interpreters in shaping the linguistic patterns of MEPs. To gain a better understanding of the EP context and to provide a theoretical answer to our research questions, we review the relevant research on MEPs in general and Dutch-speaking MEPs in particular and on EP interpreters and the Dutch booth (\sectref{sec:defrancq:3} and \ref{sec:defrancq:4} respectively). \sectref{sec:defrancq:5} presents the quantitative and qualitative methods and results of a detailed study of the n-grams or lexical bundles that were also the subject of investigation in our previous studies. \sectref{sec:defrancq:6} finally presents the conclusions.

\section{Linguistic convergence between MEPs and interpreters}\label{sec:defrancq:2}

According to various sociological and sociolinguistic theories, such as Communication Accommodation Theory (CAT; \citealt{Giles1973}; \citealt{GilesOgay2007}\todo{citation not in bib}), the theory of Discourse Communities (DC; \citealt{Swales1990}) and of Communities of Practice (CoP; \citealt{Wenger1998}), linguistic convergence is to be expected in contexts where individuals or groups frequently interact, as a way to construct group identity or as a way for one individual or group to gain social acceptance by the other. In DC and CP, linguistic convergence is theorised in terms of genre: communities develop structured linguistic repertoires or genres, made up of repeated linguistic patterns (\citealt{MillerKelley2016}), that need to be acquired by new group members. Interestingly, MEPs have been described as a Discourse Community \citep{CalzadaPerez2007} and the EP booths as Communities of Practice \citep{Duflou2016}.\footnote{Duflou considers language booths as separate communities of practice but all booths collectively as one too.} In \citet{Defrancq2018} and \citet{DefrancqPlevoetsforthcoming}, a theoretical analysis of the European Parliament as a discourse community is put forward, that not only includes the MEPs, as proposed by \citet{CalzadaPerez2007}, but also the interpreters in the EP booths.  

To test the theoretical model, an analysis of lexical patterns used by Dutch-speaking MEPs and the Dutch booth in the EP was conducted \citep{DefrancqPlevoetsforthcoming} and output from both groups was compared with the output of Dutch-speaking members of national parliaments (for an identification of the items used, see \sectref{sec:defrancq:5}). A Correspondence Analysis led us to conclude that there is indeed a degree of linguistic convergence between Dutch-speaking MEPs and the Dutch booth in the EP: members of national parliaments and the Dutch EP booth appear at the extreme ends of the linguistic spectrum, while MEPs position themselves in between. However, the case could not be made that MEPs and the Dutch booth constitute a single group from a linguistic point of view in opposition to national parliamentarians. Interestingly, it also appeared that the group of MEPs shows striking signs of internal convergence: diatopical variation (Belgian Dutch vs. Netherlandic Dutch) is considerably lower among MEPs than among members of national parliaments. MEPs thus seem to converge on the use of a hybrid variety that shares some properties with national Dutch-speaking parliamentarian registers and others with the EP’s Dutch booth.

In \citet{DefrancqPlevoetsforthcoming} we refrained from claims about the direction of the observed convergence. However, all theories of communication that account for it are based on the idea that some individuals or groups are dominant, in that their linguistic repertoire tends to be emulated by other individuals or groups and not the other way around. CAT holds that individuals and groups create, maintain or decrease social distance through linguistic, paralinguistic and non-verbal communicative strategies. Individuals or groups accommodate, i.e. shift to features that are more similar to the features of the other, in order to maximise social integration with the other individuals or groups, making the latter the dominant force in convergence. Similarly, in the theory of Discourse Communities and Communities of Practice, newcomers to the community are assumed to seek to assert their membership by proving their grasp of the community’s specific genre, the dominant or expert group being the insiders that already have knowledge of the genre. 

\citegen{Bartlomiejczyk2016} above-mentioned quote seems to prioritise MEPs as the dominant or expert group in the genre makeup of the European Parliament: interpreters are reported to acquire lexical patterns from MEPs (and from more senior interpreters), but MEPs are not reported to acquire lexical patterns from interpreters. However, the Correspondence Analysis we presented in \citet{DefrancqPlevoetsforthcoming} seems, at first sight, to give some credit to the idea that MEPs adapt to linguistic patterns used by interpreters: MEPs combined position is situated between the positions of national MPs and the EP booth. The idea is not unreasonable: Dutch-speaking MEPs are likely to listen a fair amount of time to their interpreters, to be exposed to linguistic patterns interpreters use and are therefore also likely to adopt these patterns. As a result, they might position themselves closer to the interpreters than members of national parliaments who lack that kind of exposure. It is important to note that the idea of MEPs adapting to interpreters is not incompatible with \citegen{Bartlomiejczyk2016} proposal. Bartłomiejczyk hypothesises a cross-linguistic pattern of accommodation mediated by translation, whereas our assumption relates to adaptation within one single language.  

Alternative accounts for the observed convergence are possible: it could be argued that the MEPs are most representative of the EP genre and that the booth’s outward position in the Correspondence Analysis reflects linguistic routines that are influenced by the challenging circumstances in which they produce output. As a group, MEPs present less diatopical variation, which is a clear sign of convergence taking place within that group. There is also evidence adduced by \citet{FerraresiMilicevic2017} from the Italian EP booth that suggests that interpreters’ lexical patterns are less idiomatic than those of speakers of the same language and that this could be due to source text interference, cognitive load or a combination of both. Combined, this evidence appears to contradict the idea that interpreters may be the linguistically dominant group in the EP. 

To be completely on the safe side, conclusions on linguistic convergence and the direction of convergence should be based on a longitudinal analysis of MEPs and their linguistic output. Unfortunately, the data collected for the corpus that was used for this study does not allow for such an analysis. We therefore propose to study convergence synchronically in terms of output features of different groups in the framework of linguistic theory that maps diachronic evolution to sychronic states (\sectref{sec:defrancq:5}). Accordingly, the research questions of this study are the following:

\begin{itemize}
\item What are the profiles of Dutch-speaking MEPs and the Dutch booth in the EP in terms of seniority, exposure and output in the European Parliament? Seniority is an important variable in determining who is most likely to constitute the group with most experience in the EP genre. Exposure data is required to ascertain the possibility of accommodation of output features.
\item Which group is the dominant or expert group in the EP in linguistic terms, i.e. is more likely to have shaped the features of the EP genre, while the other group is still in the process of acquiring those features?
\end{itemize}

The first question will be answered on the basis of an overview of the relevant literature on MEPs and interpreters, concentrating on the EP’s Sixth and Seventh Term, i.e. the period between 2004 and 2014. The Sixth and Seventh Terms are the ones from which most of the data that we will use to answer the second question is drawn. The second question will be answered with a combination of quantitative and qualitative methods. As the corpora used for this study do not allow for diachronic analysis, linguistic expertise will have to be interpreted synchronically and comparatively. We will assume that the non-expert group has incomplete mastery of the genre: it is therefore unlikely to have acquired all linguistic features of the genre and likely to use the acquired features to a lesser extent than the dominant or expert group. 

\section{Members of the European Parliament} \label{sec:defrancq:3}

\subsection{Seniority}\label{sec:defrancq:3.1}

The more than 700 MEPs are elected by universal suffrage according to rules laid down by the Member State’s electoral authorities. During the Sixth Term and Seventh Term, which are directly relevant to this study, as most of our data in \sectref{sec:defrancq:5} were drawn from these, the EP consisted of 732 and 736 members respectively. Electoral procedures vary across Member States. MEPs professional profiles also vary, but are predominantly situated in the legal and academic fields according to an analysis of biographies of MEPs in the Sixth Term (2004--2009) by \citet{BeauvalletMichon2010}. According to the same analysis, 81\% are university graduates and 26\% hold a PhD. Unlike in the early years of the EP, the EP mandate is for most MEPs (Sixth term: 61\%; Seventh Term: 66\%) the first electoral mandate of their political career or the first mandate beyond the local level (\citealt{BeauvalletMichon2010}; \citealt{BeauvalletEtAl2013}). \citet{BeauvalletMichon2010} conclude that the EP is a breeding ground for a new national political class as it offers most MEPs their first paid full-time job as a politician. Belgian and Dutch MEPs, who are directly relevant to our research, differ considerably: only 7\% of Dutch MEPs in 2004 had previous experience beyond the local level, compared to 42\% of the Belgian MEPs.

Roughly half (52\% in 2004) of the MEPs are newly elected with each 5-year electoral cycle and 12\% of them had left office and were replaced by newcomers before the end of the parliamentary term \citep{Whitacker2014}. This means that a sizeable number of MEPs had limited experience on the job. In the Sixth Term the average length of the EP stint was 6.6 years for the pre-2004 Member States and 6.3 and 5.7 years for Belgian and Dutch MEPs respectively (\citealt{BeauvalletMichon2010}). 

\subsection{Exposure}\label{sec:defrancq:3.2}

Even though membership of the EP does not require particular foreign language skills, linguistic competences are an asset in the EP. Among the 141 MEPs she interviewed, \citet{Wright2007} quotes several of them pointing out that MEPs who do not master English as a \textit{lingua franca} are likely to be marginalised in the political process. The EP offers simultaneous interpreting from 24 into 24 official languages during plenaries. For group and committee meetings interpretation is offered for the languages requested by participants.  

To determine to what extent MEPs are exposed to interpretation we should be able to estimate how many of the contributions to plenaries, committee and group meetings are held in languages that they are unlikely to understand. No such data are available for committee and group meetings. The literature on plenaries provides us some clues, but caution is due in interpreting the figures. The most direct source of information are corpora of EP proceedings, such as Europarl \citep{Koehn2005}. However, Europarl is built with the purpose of ensuring roughly equal numbers of data per language and does not reflect the proportions of languages actually used. One Europarl sub-corpus, extracted from the 1996--1999 plenaries by \citet{CartoniEtAl2013} reports corpus sizes for 5 of the then ten official languages which are claimed to reflect the actual language use. The three major languages, i.e. English, French and German, each account for 25 to 29\% of the data, while Dutch reaches 17\% and Spanish and Italian 14\% and 12\% respectively. These figures do not include the other languages that were official at the time (Portuguese, Greek, Finnish, Danish and Swedish) and are therefore exaggerated. In a study based on data drawn from 62 plenaries in 2006 (with 21 official languages), \citet{Cucchi2007} reports that English represents 21\% of the data: 12\% of native English and 9\% of nonnative. Other languages are not differentiated. It is important to note that the data are calculated on the basis of token counts, which does not automatically translate to speech counts. Long speeches held in one of the languages will result in higher proportions in the token count. English in particular is mostly used by the Commission representative during the plenary, who is given more speaking time than MEPs.

Only a very rough estimate can therefore be given of the amount of time Dutch-speaking MEPs will seek interpreting. Considering most of them know English well enough to do without interpretation and a fair share of them also understand French or German well enough, MEPs are likely to resort to interpretation for slightly over half of the plenary speeches. It is important to note that English speeches are not necessarily listened to directly. \citet{Wright2007} reports that on one occasion she noticed that a considerable number of MEPs put on their headphones when an Irish MEP took the floor in native English after an intervention in nonnative English by a German MEP. 

\subsection{Output}\label{sec:defrancq:3.3}

Plenary speaking time is allotted to the political groups in accordance with their numerical strength. Individual MEPs are granted speaking time at their request. Among the many factors that determine an individual’s likelihood of being allowed to hold a speech, EP seniority is one of the most significant \citep{SlapinProksch2010}. MEPs are thus likely to have spent a fair amount of time listening to their colleagues, either directly or through interpretation, before preparing and holding speeches of their own. 

\citet{Wright2007} reports an array of different attitudes among MEPs with regard to speech preparation: while many members make a point of using their own official language, some do not object to or even prefer the use of a \textit{lingua franca}, i.e. English or French, in plenaries for fear of not getting their views across through interpretation, a point also made more generally by \citet{KurzBasel2009}. It is also customary for MEPs to articulate a few words in the language of previous speakers to whom they respond. 

According to \citet{Wright2007} MEPs whose mother tongue is one of the \textit{lingua francas} (English and French) split into two sub-groups. Some members consciously adapt to the presumed needs of a nonnative audience, focusing on clear articulation and avoiding rhetorical and linguistic prowesses; others do not seem to be bothered. Among English-speaking MEPs, the latter group tends to be strictly monolingual and is reported to have trouble understanding the non-na\-tive English used in the European institutions. French-speaking MEPs of the latter sort are likely to be upset by the widespread use of English and the diminished status of French. 

MEPs’ language patterns have drawn considerable interest, not least the rhetorical and metaphorical devices put forward to construct a European identity (\citealt{DeAngelis2011}; \citealt{Flottum2013}). The availability of the written verbatim reports of MEPs speeches and their translations in particular has sparked detailed studies of specific patterns. The Europarl corpus is still by far the biggest translation corpus in terms of language scope and sheer size.

Comparing an English sub-corpus derived from Europarl (2006 plenaries) with a corpus of English TED talks, \citet{LeferGrabar2015} find that some categories of evaluative prefixes are typical of EP discourse. In particular, prefixes expressing excess or insufficiency (e.g. over-\textit{centralised}; under-\textit{represented}) are significantly more frequent in EP discourse. \citet{granger_lexical_2014} compares a bilingual (French and English) 2-million tokens’ sub-corpus of Europarl with a corpus of journal editorials, pointing at the high frequencies of lexical bundles performing EP rituals related to interaction during the plenary (e.g. thanking the President or congratulating a colleague); expressing epistemic stance (e.g. \textit{I’m delighted that, I must say that, I am sure that}) and directive stance (e.g. \textit{we want to see, we have to make sure that, we need to, we must not, we have a duty to, we have to ensure that, there is a need for/to}). Those lexical bundles are typical of MEPs’ speeches but their frequency seems to vary across languages, as the English data show higher frequencies than the French data. 

It has been pointed out that the verbatim reports are sanitised versions of MEPs’ speeches \citep{Cucchi2009} and do not accurately reflect MEPs’ linguistic patterns. Several small corpora of the spoken versions of speeches have been compiled, both with and without interpretations (\citealt{BernardiniEtAl2018}; \citealt{Cucchi2007}; \citealt{KajzerWietrzny2012}; \citealt{RussoEtAl2006}). From an analysis of the transcribed speeches (N=62) in her corpus, \citet{Cucchi2009} concludes that general extenders (\textit{e.g. and so on}) are less frequent in MEPs’ speeches than in ordinary conversation and that they are used as a way of referring to information only MEPs have access to and can complete, strengthening their institutional identity.

\section{EP interpreters}\label{sec:defrancq:4}

\subsection{Seniority}\label{sec:defrancq:4.1}

Interpreters in the European Parliament are recruited as staff interpreters or as freelancers through competitions and accreditation tests. Competitions have become increasingly rare in the last decade. Accreditation tests are organised according to need. For the Dutch booth, due to looming personnel shortages, accreditations tests have been held annually over the last 10 years, with the exception of 2020, due to the Covid-19 crisis. Since 2004 accreditation tests are organised jointly by the European Commission’s DG Interpretation (SCIC) and the European Parliament’s DG LINC \citep{Duflou2016}. Success rates are traditionally low, ranging between 20 and 30\% \citep{Duflou2016}, meaning that the influx of new interpreters is limited. For the Dutch booth, on a pool of ca. 17 staff and ca. 70 freelancers during the Sixth Term \citep{Duflou2016},\footnote{The number of staff members has decreased over the last years. Heines (p.c.), the acting booth head, confirmed that there were only 9 staff interpreters left in 2020.} only 1 to 3 new freelancers are accredited per year. This does not mean that new freelancers are immediately integrated in the EP’s workforce. \citet{Duflou2016} states that the EP prioritises freelancers on the basis of institutional loyalty in order to be able to recruit experienced interpreters offering the required language combinations: 

\begin{quote}
[N]ewly accredited interpreters are mainly recruited during peak periods, and only a few of them, depending on their language combination and the outcome of their evaluation reports, will be recruited to work for DG INTE regularly. \citep[145]{Duflou2016}
\end{quote}

Compared to the MEP workforce with a turnover of over 50\%, the interpreter pool appears much more stable. In the non-representative sample of interviewees \citet{Duflou2016} drew from the Dutch booths with DG Interpretation (SCIC) and DG LINC, 24 had started interpreting for the EU 4 years or more before the interviews that took place over the period 2007--2010. 7 of those had started before 1980. 11 had less than 4 years of experience. Based on these – admittedly – partial data, it seems nevertheless safe to assume that on the whole the pool of EP interpreters can boast substantially more collective experience in parliamentary meetings, including the plenaries, than MEPs.

\subsection{Exposure}\label{sec:defrancq:4.2}

It might seem trivial to state that interpreters are continually exposed to the output of MEPs, considering that they have to interpret it. Two caveats are nevertheless in order here: first, the Dutch booth does not cover the whole range of languages spoken in the EP. Exact figures are hard to find and vary from one plenary to another, but with an average coverage of just under 5 languages among the staff interpreters and 3.7 languages among freelancers \citep{Duflou2016}, the Dutch booth is unlikely to cover more than 10 languages in the plenary, even though it sits three interpreters. For all other languages, the Dutch booth resorts to relay interpreting, in which case they do not listen to the MEPs but tune in on other booths. Second, during the Sixth Term Belgium elected 24 MEPs, 14 of which were Dutch-speaking, the Netherlands elected 27; in the Seventh Term, the figures were the following: Belgium 22 MEPs (13 Dutch-speaking); the Netherlands 25 MEPs \citep{cvce_nombre_2022}. The group of potential speakers of Dutch consisted thus in both terms of ca. 40 members on a total of 732 to 736 MEPs, which is less than 6\%. In other words, interpreters’ potential exposure to Dutch spoken by MEPs is marginal. Moreover, it is likely that Dutch speeches are welcomed as periods of rest with little attention paid to what is said and how it is formulated. In all, the Dutch booth can be safely assumed to have only very limited exposure to Dutch-speaking MEPs.

Rather, exposure and accommodation to booth mates, especially the more senior ones, is pervasive and well-documented by \citet{Duflou2016}. Less experienced interpreters are expected to listen to more experienced colleagues and to copy their renderings to the extent that not doing so is regarded as a reason to put their competence into question. This results in the creation and maintenance over long periods of time of a joint repertoire of expressions. One DG SCIC interpreter \citet{Duflou2016} interviewed finds this “parroting”(p. 194) highly problematic. 

\subsection{Output}\label{sec:defrancq:4.3}

The linguistic patterns in EP interpretations have been investigated both in comparison with the source speeches they are based on and with the translations made of the verbatim reports of those source speeches. The available research focuses on a limited number of features: the handling of pragmatically challenging utterances, such as face-threatening acts or ideologically laden lexemes (\citealt{BeatonThome2013}; \citealt{Bartlomiejczyk2016, bartlomiejczyk_parliamentary_2020}; \citealt{MagnificoDefrancq2017}), translation universals such as simplification (\citealt{RussoEtAl2006}; \citealt{KajzerWietrzny2012}; \citealt{FerraresiMilicevic2017}; \citealt{BernardiniEtAl2016}) and explicitation (\citealt{KajzerWietrzny2012}; \citealt{DefrancqEtAl2015}); collocations and formulaic expressions (\citealt{FerraresiMilicevic2017}; \citealt{Aston2018})  and ideological homogenisation \citep{Beaton-Thome2007}. Explanatory factors for specific patterns are usually sought in the area of source language interference and heavy cognitive load interpreters experience, which is held to be conducive to simplification and even to explicitation \citep{FerraresiMilicevic2017}, and in the area of interpreters’ agency required for navigating a pragmatically complex and treacherous context. 

The use of formulaic expressions has also drawn interest. EP interpreters appear to use formulaic expressions very frequently \citep{Aston2018} due to a combination of factors. First, the context of institutionalised procedures is conducive to the use of formulaic expressions and interpreters working in this context are thus exposed to high frequencies of them \citep{Bartlomiejczyk2016}. Second, it is widely recognised that producing formulae allows interpreters to reduce cognitive load as formulae are retrieved as complete units from memory (\citealt{Gile1995}; \citealt{Setton1999}; \citealt{plevoets_cognitive_2018}). Finally, as already explained, formulaic expressions are also part of the socialisation process newly accredited interpreters go through: to blend in, they are expected to adopt expressions used by boothmates (\citealt{Bartlomiejczyk2016}; \citealt{Duflou2016}). Among items with unusually high frequencies in the English booth of the EP, \citet{Aston2018} lists the performative expressions having to do with parliamentary rituals, but also stance items, such as \textit{I think we need to, to come up with a, we need to ensure that, when it comes to the}. Incidentally but perhaps not coincidentally, there is quite some overlap with the lexical bundles \citet{granger_lexical_2014} reports as typical of the written versions of speeches held by English-speaking MEPs during plenaries. This seems to confirm the linguistic convergence reported in \sectref{sec:defrancq:2} for a different set of MEPs and interpreters.

\subsection{Intermediary conclusions}\label{sec:defrancq:4.4}

Considering the available information on MEPs and interpreters in the EP, a number of tentative conclusions can be reached about the likelihood of one group being linguistically dominant or expert. Much of the evidence is circumstantial. Hard evidence could only have been collected with observational methods over a long period of time, which is not the case here. However, it seems relatively safe to conclude that the level of expertise in the EP genre is probably higher in the booth than in the plenary room: MEPs have higher turnover rates and their sting typically does not last very long. Exposure to same-language linguistic output is likely to be far higher in the case of MEPs: they need to listen more often to their interpreters than the other way around. There is cross-linguistic quantitative and qualitative evidence that linguistic convergence takes place between MEPs and interpreters that share the same language. Given the relative expertise and the relative exposure, it seems more likely that interpreters constitute the expert group and that MEPs conform to interpreters in terms of language patterns than the other way around.   

It should be pointed out that some of these conclusions do not necessarily apply to post-2004 accession booths that practice both A-interpreting and retour. Interpreters in these booth are probably more likely to pick up patterns from MEPs who are native in the interpreters’ B languages and native MEPs are much less likely to adopt patterns from retour interpreting.

\section{Analysis of patterns}\label{sec:defrancq:5}

\subsection{Operationalisations and assumptions}\label{sec:defrancq:5.1}
\begin{sloppypar}
As explained in \sectref{sec:defrancq:2}, a diachronic process, i.e. a group acquiring specific linguistic features of a genre, will be analysed through the lens of essentially synchronic data. The corpus used is EPICG which contains on the one hand 27 Dutch speeches from the EP’s Sixth Term and 16 from the Seventh Term, spanning 4 years of plenaries (2008--2011). 33 different MEPs are included, 6 of which are included with more than one speech. The corpus also comprises 164 Dutch interpretations by an unknown number of interpreters from the Sixth and Seventh term, spanning 6 years of plenaries (2006--2011). Only for the 6 MEPs with multiple speeches would it be possible to study the adoption of certain features through time. Obviously, this is too small a sample. As for the interpreters, a lack of metadata forbids any comparable analysis.
\end{sloppypar}

This is why the corpus will be analysed as a synchronic state, rather than as a diachronic process. Translating diachronic processes to synchronic states is not uncommon in linguistics. In the area of grammaticalisation theory, synchronic variation is considered to be a “manifestation of (diachronic) change” (\citealt{Lehmann2005}). Crucial to the representation of grammaticalisation is the so-called “cline” (\citealt{HopperTraugott2003}), which represents both the diachronic evolution of single items through different stages of grammaticalisation and the relative position of multiple items in a synchronic state, including in a cross-linguistic perspective. 

A similar projection will be made here. The acquisition of expertise in the community genre can be represented as a cline (in \figref{fig:defrancq:2}), that both captures the stages individuals go through diachronically and allows us to compare individuals and groups synchronically.

\begin{figure}

%\includegraphics[width=\textwidth]{figures/a1UtinterpresLinguisticconvergencebetweenoratorsandinterpretersintheEuropeanParliament-img002.png}
\scalebox{.85}{
 	\begin{tikzpicture}
		%shapes
		\draw[lsDarkBlue, fill=lsDarkBlue] (-5,0) -- (5,0) -- (5,3);
		\node[draw, single arrow, lsDarkBlue, fill=lsDarkBlue, minimum height = 7cm, single arrow head extend=5pt] at (0,-0.5) (arrow1){};
		\node[draw, single arrow, lsDarkBlue, fill=lsDarkBlue, minimum height = 2.5cm, single arrow head extend=5pt, rotate = 90] at (5.5,1.5) (arrow2){};
		%nodes
		\node[draw, rectangle] at (-5,-0.5) (non){Non-expert};
		\node[draw, rectangle] at (0,-1.5) (ac){Acquisition};
		\node[draw, rectangle] at (5,-0.5) (exp){Expert};
		\node[draw, rectangle] at (7, 1.5) (fea){Features};
	\end{tikzpicture}
}

\caption{A cline of linguistic expertise}
\label{fig:defrancq:2}
\end{figure}

One aspect of expertise will be singled out here, namely the use of a particular set of lexical patterns that belong to the EP genre. Logically, the expert group is hypothesised to master more of these patterns than the non-expert or acquiring group and also to use those patterns more frequently than the latter. 

One significant drawback of the EP data used in this study is that the linguistic features of the EP genre cannot be determined independently from the output of both groups under study. In addition, both groups are unequally represented in the data: interpretations add up to more than 70\% of the EP data; MEPs to less than 30\%. If linguistic features were to be extracted from the sum of the two data sets, interpretations would have a significant edge over MEPs speeches in determining the features of the EP genre. 

To avoid bias, we will concentrate on the lexical patterns that both groups share as most typical of their respective outputs. Typicality will be determined through a keyness analysis and based on a comparison with the non-EP corpus, i.e. the corpus of speeches held in national parliaments. 

Crucially, the expert group is assumed to master the EP genre more completely than the non-expert group and, therefore to use a broader range of typical patterns than the non-expert group. Consequently, we hypothesise that the patterns shared by both groups will make up a modest part of the patterns typical of the expert group, but a significant portion of the patterns typical of the non-expert group. We also hypothesise that the shared patterns will be used more frequently by the expert group. 

\subsection{Keyness in the EP sub-corpora}\label{sec:defrancq:5.2}

For the purpose of the Correspondence Analysis referred to in \sectref{sec:defrancq:2}, we worked on a set of 181 3- and 4-gram types described in \citet{DefrancqPlevoetsforthcoming}. This is also the set of lexical patterns that will be studied here. The 3- and 4-grams were selected from a set of frequent 269 types drawn from the Dutch sub-corpora in EPICG \citep{BernardiniEtAl2018} and the CGNg \citep{Oostdijk2000}, i.e. a corpus of parliamentary speeches and debates in the Netherlands and Belgium, which is part of a larger corpus of spoken Dutch. The selection process is explained in \citet{DefrancqPlevoets2018}. We excluded three types of items: syntactically ill-formed items (e.g. due to repetitions of the same word), items related to EU entities that could be considered as self-references (e.g. \textit{verdrag van Lissabon} ‘Lisbon treaty’) and references to the debating context (e.g. \textit{het woord is aan} ‘has the floor’). Including the latter two categories would have artificially promoted the convergence hypothesis.

The 33 MEPs in our sample are Dutch and Dutch-speaking Belgians. They delivered the 43 Dutch speeches contained in the sub-corpus. At the time the speeches were delivered MEPs had spent on average 85 months or around one and a half terms in the EP (one term is 60 months). Experience at the time of the speech ranges from 10 months up to 177.  

The first step was to determine which 3- and 4-gram types were most typical of the Dutch speeches in the MEP sub-corpus (MEP). We therefore performed a comparative keyness analysis of these speeches with the data from the national parliaments (NAT). As our datasets are small, it is unadvisable to determine keyness based on significance tests (Likelihood ratio and Pearson chi-square test), as the results of such tests are impacted by the size of the available data \citep{Gabrielatos2018}. Gabrielatos recommends the use of \%DIFF and BIC for the comparison of frequencies in different corpora. \%DIFF yields a measure of discrepancy between the relative frequencies of items, where high scores indicate large frequency differences. We set a threshold of 250 to select the items that are most key in MEPs speeches. The threshold was randomly chosen with an aim to obtain a set of approximately 25 items. The resulting list turned out to consist of 26 items, that can be found in Appendix~\ref{ap:defrancq:1}. BIC (Bayesian Information Criterion) is an alternative way of obtaining significance scores for keyness \citep{Gabrielatos2018}. However, due to the small sizes of the sub-corpora, we found very few items that reached the significance threshold and made the choice to nevertheless proceed on the basis of the \%DIFF scores.

The same procedure was repeated for the Dutch interpretations (INT). The resulting list contained 69 items and can be found in Appendix~\ref{ap:defrancq:2}. The longer list of key items in interpretations is not surprising: as our reference corpus is the sub-corpus of national parliamentary speeches, the longer list reflects the greater discrepancy between national parliamentarians and EP interpreters, confirming the outcomes of the CA in \citet{DefrancqPlevoetsforthcoming}. 

Crucially, at this stage we needed to check how many and which of the key 3- and 4-gram types in both sets were identical. This is shown in \tabref{tab:defrancq:1} and Appendices \ref{ap:defrancq:1} and \ref{ap:defrancq:2} (items marked with an asterisk occur in both sets).

\begin{table}
\small
\begin{tabularx}{\textwidth}{rrrr}

\lsptoprule
{\bfseries \%DIFF/NAT} & \makecell[rt]{{\bfseries MEP}\\ 
{\bfseries \# (percentage shared)}} & \makecell[rt]{{\bfseries INT}\\
{\bfseries \# (percentage shared)}} & \makecell[rt]{\bfseries Shared between\\\bfseries MEP and INT}\\
\midrule
>250 & 26 (65\%) & 69 (25\%) & 17\\
\lspbottomrule
\end{tabularx}

\caption{Number of key items in the sub-corpora and their overlap; keyness with regard to national parliaments.}
\label{tab:defrancq:1}
\end{table}

Interestingly, it turns out that almost two thirds of the items that are typical of Dutch MEP speeches are also among the key items in the Dutch booth. Conversely, only a quarter of the key items in interpretation are also key in Dutch MEP speeches. In other words, it not only appears that interpreters use a broader range of key 3- and 4-grams than MEPs, but that broader range also includes a significant portion of items that are key in MEPs’ speeches. It therefore seems more likely that interpreters constitute the expert group in linguistic terms, while MEPs appear to be the group acquiring linguistic expertise. Additional support for this conclusion comes from the analysis of the nine items that are key in MEPs’ speeches, but less so in interpretations. Of those nine, eight are still more frequent in the EP interpretations than in the national parliaments, four of which obtain a \%DIFF score higher than 100. Conversely, of the 52 items reaching the keyness threshold in interpretations alone, only 17 are also more frequent in MEPs speeches than in national parliaments. In other words, all but one key items in MEPs speeches can be accounted for assuming they are adopted from interpreters, while not even half (17+17=34) of the key items in interpretations could be accounted for assuming these were adopted from MEPs. The data also contradict an alternative hypothesis in terms of interpreters’ higher likelihood to use atypical patterns due to interference or cognitive constraints: if this were the case, the presence of so many of their key patterns in MEPs’ speeches could not be accounted for.

Additionally, we compared the relative frequencies of the 17 shared items, assuming that these would be higher among the expert group. \tabref{tab:defrancq:2} shows that in all but two cases (marked with an asterisk) the relative frequencies of key n-grams are indeed higher in the interpretations than among MEPs. Cases are shown according to their keyness score in the MEPs output (not shown here).

\begin{table}
\small
\begin{tabularx}{\textwidth}{Xrr} 
\lsptoprule
& \makecell[tr]{{\bfseries INT}\\
{\bfseries Rel. Freq. /100k tokens}} & \makecell[tr]{{\bfseries MEPs}\\
{\bfseries Rel. Freq. /100k tokens}}\\
\midrule
\textit{we moeten dus}  

‘so we need to’ &  9.11 &  6.92\\
\rule{0pt}{1.2em}{\itshape de veiligheid op} 

‘the security on’ &  13.02 &  3.46\\
\rule{0pt}{1.2em}{\itshape willen danken voor}

 ‘want to thank for’ &  6.51 &  3.46\\
\rule{0pt}{1.2em}{\itshape van de veiligheid} 

‘for the security’ &  13.02 &  10.39\\
\rule{0pt}{1.2em}{\itshape de verenigde staten}$^{*}$

‘the United States’ &  9.11 &  38.08\\
\rule{0pt}{1.2em}\textit{we moeten niet} 

‘we must not’ &  6.51 &  3.46\\
\rule{0pt}{1.2em}\textit{de bestrijding van} 

‘the fight against’ &  7.81 &  6.92\\
\rule{0pt}{1.2em}{\itshape en we moeten} 

‘and we need’ &  22.13 &  10.39\\
\rule{0pt}{1.2em}\textit{ervoor zorgen dat} 

‘make sure that’ &  22.13 &  10.39\\
\rule{0pt}{1.2em}{\itshape in geval van}$^{*}$

‘in case of’ &  6.51 &  10.39\\
\rule{0pt}{1.2em}{\itshape dus we moeten} 

‘so we need to’ &  9.11 &  3.46\\
\rule{0pt}{1.2em}\textit{we moeten ook} 

‘we also need to’ &  18.22 &  3.46\\
\rule{0pt}{1.2em}\textit{om ervoor te zorgen} 

‘to make sure’ &  23.43 &  3.46\\
\rule{0pt}{1.2em}\textit{de bevoegdheden van} 

‘the competences of’ &  9.11 &  6.92\\
\rule{0pt}{1.2em}\textit{ervoor te zorgen dat} 

‘to make sure that’ &  24.73 &  6.92\\
\rule{0pt}{1.2em}{\itshape van de gegevens} 

‘of the data’ &  6.51 &  3.46\\
\rule{0pt}{1.2em}\textit{om te komen tot} 

‘to reach’ &  9.11 &  3.46\\
\lspbottomrule
\end{tabularx}

\caption{Relative frequencies of key items.}
\label{tab:defrancq:2}
\end{table}

Due caution is needed in interpreting the figures because about half of the values in the MEP column represent 1 single occurrence in absolute numbers. Nonetheless, even in most of the remaining cases interpreters are found to use key items of the MEPs speeches even more frequently than the MEPs themselves. It is therefore reasonable to conclude that the interpreting booth in the EP is the expert group in linguistic terms, while MEPs show a lower degree of linguistic expertise. This is of course completely in line with the intermediary conclusions of \sectref{sec:defrancq:4.4}. The positions of both groups can be set out against the expertise cline (\figref{fig:defrancq:3}). Interpreters are represented by the straight cross and present the most typical use of the EP genre, while the group of MEPs, represented by the diagonal cross, has less expertise in the genre.

\begin{figure}

%\includegraphics[width=\textwidth]{figures/a1UtinterpresLinguisticconvergencebetweenoratorsandinterpretersintheEuropeanParliament-img003.png}
\scalebox{.85}{
 	\begin{tikzpicture}
		%shapes
		\draw[lsDarkBlue, fill=lsDarkBlue] (-5,0) -- (5,0) -- (5,3);
		\node[draw, single arrow, lsDarkBlue, fill=lsDarkBlue, minimum height = 7cm, single arrow head extend=5pt,] at (0,-0.5) (arrow1){};
		\node[draw, single arrow, lsDarkBlue, fill=lsDarkBlue, minimum height = 2.5cm, single arrow head extend=5pt, rotate = 90] at (5.5,1.5) (arrow2){};
		\node[draw, cross out, lsMidOrange, ultra thick, minimum size = 10pt] at (0.5,1.8)(cross1){};
		\node[draw, cross out, lsMidGreen, ultra thick, minimum size = 10pt, rotate = 45] at (4.8,3)(cross2){};
		%nodes
		\node[draw, rectangle] at (-5,-0.5) (non){Non-expert};
		\node[draw, rectangle] at (0,-1.5) (ac){Acquisition};
		\node[draw, rectangle] at (5,-0.5) (exp){Expert};
		\node[draw, rectangle] at (7, 1.5) (fea){Features};
	\end{tikzpicture}
}

\caption{Positions of MEPs and interpreter on the linguistic expertise cline in the EP.}
\label{fig:defrancq:3}
\end{figure}

\subsection{Functional analysis}

A functional analysis carried out on the 17 items in \tabref{tab:defrancq:2} along the lines described in \citet{Biber2004}, reveals a number of interesting facts. Three of Biber’s categories are present: referential n-grams, stance n-grams and a discourse organiser.

The discourse organiser is \textit{in elk geval} (‘anyway’), seemingly used to refute counter-arguments as irrelevant.  

Six key n-grams are referential: \textit{de veiligheid op, van de veiligheid, de verenigde staten, de bestrijding van, de bevoegdheden van, van de gegevens}. They represent topics covered by EU legislation, such as road safety, combating terrorism, data protection, international relations and institutional competences. It should be stressed that Dutch speeches and Dutch interpretations do not necessarily come from the same plenary sessions. It is pure coincidence that some topics were covered both in sessions from which speeches were downloaded and in sessions from which interpretations were drawn.

\begin{sloppypar}
Ten key $n$-grams are stance expressions. They are exclusively attitudinal stance expressions of obligation and intention, clustering around verbs such as \textit{moeten} (‘need, have to, must’), \textit{zorgen voor} (‘make sure, ensure’), \textit{komen tot} (‘arrive at, reach’). Many of them occur with an adverbial connective or a conjunction (\textit{en, dus, ook, om, dat}).\footnote{One case (\textit{willen danken voor} ‘want to thank for’) is probably connected to the ritual of thanking the President or another MEP. If that’s the case, it should be withdrawn on the basis of the exclusion criteria mentioned in \sectref{sec:defrancq:5.1}.} The occurrence of such stance markers is plausible in a context of legislative procedure that is prescriptive in nature. What is distinctive of the EP is that a particular set of expressions is used very frequently to articulate such stance and that interpretation appears instrumental in promoting those expressions, including in MEPs speeches. 
\end{sloppypar}

Interestingly, the clusters of stance expressions happen to be equivalents of some of the n-grams found to be typical of the written reports of English EP speeches and interpretations by the English booth. \tabref{tab:defrancq:3} lists the attitudinal stance markers reported in \citet{granger_lexical_2014} for the written reports and in \citet{Aston2018} for English interpretations, next to the ones from \tabref{tab:defrancq:2}. (Parts between brackets are absent from the Dutch n-grams as these are in general shorter than the ones extracted by Aston and Granger.)

\begin{table}
\begin{tabularx}{\textwidth}{XXX}

\lsptoprule
{\bfseries This study} & {\bfseries \citet{granger_lexical_2014}}

{\bfseries English-speaking MEPs} & {\bfseries \citet{Aston2018}}

{\bfseries English booth}\\
\midrule
we moeten dus, we moeten niet, en we moeten, dus we moeten, we moeten ook & we need to, we must not, we have a duty to, there is a need for/to & (as I think) we need to \\
\rule{0pt}{1.5em}ervoor zorgen dat, om ervoor te zorgen, ervoor te zorgen dat & (we) have to ensure that, (we) have to make sure that, & (we) need to ensure that\\
\rule{0pt}{1.5em}&  & when it comes to the\\
\rule{0pt}{1.5em}om te komen tot &  & to come up with a\\
\rule{0pt}{1.5em}& we want to see & \\
\lspbottomrule
\end{tabularx}
\caption{Comparison of key n-grams across studies.}
\label{tab:defrancq:3}
\end{table}

The cross-linguistic similarities clearly support the idea that the legislative purpose gears the EP genre towards expressions of intentional and deontic stance. However, it is impossible to deduce from the data presented by \citet{Aston2018} and \citet{granger_lexical_2014} which group uses the items involved the most. More research is needed on the English items to substantiate our claim that interpreters shape the linguistic features of the genre in English as well. 

\section{Conclusions}\label{sec:defrancq:6}

In this study, starting from an observation made in earlier work \citep{Defrancq2018}, we set out to determine which group, MEPs or interpreters, plays the determining role in the linguistic convergence that seems to take place in the European Parliament. Most prominent theories of socially determined linguistic change rest on the assumption that some individuals or groups adopt linguistic features typical of other, more dominant or experienced, individuals or groups. In order to find out which group was the most experienced in the EP, a two-pronged approach was taken. First, an analysis of EP seniority and potential linguistic exposure was conducted for MEPs and interpreters. It revealed that interpreters are probably more experienced in plenary dealings than MEPs, and are therefore also more likely to be experts in the EP genre and its linguistic features. Second, a detailed analysis was carried out on lexical patterns (3- and 4-grams) typical of the EP genre in Dutch, showing that, on the one hand, the Dutch booth uses a broader range of patterns with higher frequencies, and, crucially, that lexical patterns typically used by Dutch-speaking MEPs coincide to a very large extent with patterns used by the booth. Interpreters thus seem to shape aspects of the EP genre, which are to a certain extent adopted by MEPs. This supports \citegen{Poechhacker2005} interactant model in which all participants in an interpreter-mediated encounter are assumed to influence each other’s communicative behaviour. However, our study shifts the traditional focus to interpreters influencing their audience.

A qualitative analysis showed that the overlapping patterns are related to topics covered by the EP plenaries and to intentional and deontic stance adopted by MEPs. Based on cross-linguistic similarities found in \citet{granger_lexical_2014} and \citet{Aston2018}, we speculated that the stance category is promoted by the communicative purpose of the EP plenaries, i.e. produce legislation and that interpretation is instrumental in promoting a particular set of patterns to express that kind of stance, including in MEPs. It is possible that those patterns get promoted because they offer interpreters cognitive benefits: formulaic language is known to lower cognitive load. 

Some of the limitations of the study have already been touched upon: the datasets they are based on are small. Larger datasets should be (compiled and) analysed to subtantiate our claims, preferably in several languages, as there is little reason to believe that the patterns we observed are language-specific (although the situation in post-2004 booths might differ). Another much needed extension concerns the amount of exposure to speeches and interpretations outside the EP plenaries. MEPs are also exposed to interpretation in committee or political group meetings, but no data on these meetings have been collected so far. Committee meetings or political group meetings also place MEPs in a different context, with probably other types of interaction dynamics, which may also influence the linguistic features of their output. These different factors need to be explored to obtain a richer and more nuanced picture of the EP genre.



\appendixsection{Key items in MEPs speeches; keyness with regard to national parliaments.}\label{ap:defrancq:1}

\begin{table}
\begin{tabularx}{\textwidth}{ll}

\lsptoprule
{\bfseries MEP/NAT} & \\
\midrule
misdaden\_tegen\_de\_menselijkheid & in\_geval\_van$^{*}$\\
we\_moeten\_dus$^{*}$ & de\_huidige\_situatie\\
de\_veiligheid\_op$^{*}$ & in\_elk\_geval\\
willen\_danken\_voor$^{*}$ & dus\_we\_moeten$^{*}$\\
van\_de\_veiligheid$^{*}$ & we\_moeten\_ook$^{*}$\\
de\_verenigde\_staten$^{*}$ & om\_ervoor\_te\_zorgen$^{*}$\\
het\_gebied\_van & op\_deze\_manier\\
op\_het\_gebied\_van & de\_bevoegdheden\_van\\
op\_het\_gebied & ervoor\_te\_zorgen\_dat\\
we\_moeten\_niet$^{*}$ & tot\_nu\_toe\\
de\_bestrijding\_van$^{*}$ & van\_de\_gegevens$^{*}$\\
en\_we\_moeten$^{*}$ & om\_te\_komen\_tot$^{*}$\\
ervoor\_zorgen\_dat$^{*}$ & \\
\lspbottomrule
\end{tabularx}
\end{table}

\newpage
\largerpage[3]
\appendixsection{Key items in interpretations; keyness with regard to national parliaments.}\label{ap:defrancq:2}\vspace*{-5mm}

\begin{table}[H]
\begin{tabularx}{\textwidth}{ll}
\lsptoprule
{\bfseries INT/NAT} & \\
\midrule
de\_veiligheid\_op & om\_te\_komen\_tot\\
van\_de\_mensenrechten & rekening\_houden\_met\\
van\_doorslaggevend\_belang & de\_financiële\_middelen\\
we\_moeten\_inderdaad & zien\_we\_dat\\
we\_moeten\_ervoor\_zorgen & te\_zorgen\_dat\\
de\_strategie\_van & van\_de\_markt\\
veiligheid\_op\_de\_weg & wil\_ik\_ook\\
we\_moeten\_dus & de\_gevolgen\_van\\
voor\_de\_patiënten & van\_de\_gegevens\\
de\_mobiliteit\_van & in\_geval\_van\\
de\_veiligheid\_in & van\_de\_bevolking\\
van\_de\_wereldgezondheidsorganisatie & voor\_de\_toekomst\\
willen\_danken\_voor & de\_bevoegdheden\_van\\
om\_te\_voldoen & de\_verbetering\_van\\
over\_de\_veiligheid & mannen\_en\_vrouwen\\
van\_de\_volksgezondheid & dat\_weet\_u\\
om\_te\_voldoen\_aan & is\_het\_zo\_dat\\
te\_zorgen\_voor & niet\_alleen\_maar\\
moeten\_ervoor\_zorgen & de\_bescherming\_van\\
op\_de\_weg & is\_het\_zo\\
moeten\_ervoor\_zorgen\_dat & voor\_het\_feit\\
om\_ervoor\_te\_zorgen & voor\_het\_feit\_dat\\
we\_moeten\_ook & het\_hebben\_over\\
de\_afgelopen\_maanden & het\_beleid\_van\\
en\_we\_moeten & het\_gaat\_hier\\
ervoor\_zorgen\_dat & en\_we\_hebben\\
we\_moeten\_niet & ik\_wil\_ook\\
van\_de\_veiligheid & in\_de\_wereld\\
te\_voldoen\_aan & te\_maken\_met\\
ervoor\_te\_zorgen\_dat & dan\_wil\_ik\\
het\_arrest\_van & in\_verband\_met\\
dus\_we\_moeten & te\_komen\_tot\\
de\_bestrijding\_van & het\_principe\_van\\
ervoor\_te\_zorgen & de\_verenigde\_staten\\
en\_wij\_willen & \\
\lspbottomrule
\end{tabularx}\vspace*{-1cm}
\end{table}

\clearpage
\sloppy\printbibliography[heading=subbibliography,notkeyword=this]

\end{document}
