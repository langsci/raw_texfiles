\chapter{Overview}\label{chapter:conclusion}

\section{Summary}

In the broad sense, I have investigated the nature of the relation between the lexical (linear) and the syntactic (hierarchical) structure in an approach to grammatical representations that keeps up with ongoing work on structuralization of the semantics of lexical items. The results discussed here contribute to the picture that has been getting clearer and clearer for over ten years now which shows that the three descriptive domains -- morphology, lexical semantics, and syntax -- form a single module of grammar as they operate on the same class of features, like [person], [place], [proximal], [definite], etc.
\par
Such a scenario has two immediate consequences for our understanding of the interface between syntax and the lexicon.  One is that morphological structures come out as linear realizations of syn-sem representations \is{linearization} which are seamless with respect to the grammatical \isi{feature}s. In other words, a \isi{morpheme} does not have any more or any fewer features than a syntactic tree it lexicalizes. \is{spell-out} The other one is that a lexicon of a language stores syntactic subtrees paired with their exponents (a view that implies that there is no such thing as a pre-syntactic lexical storage). Following the research program outlined in \citeauthor{Starke2009} (\citeyear{Starke2009,StarkeLA}), both these consequences have been discussed for a few empirical domains in recent years and, in the broad sense, this contribution merely adds up to the growing body of work produced in a similar vein. \is{Nanosyntax}
\par
In the narrow sense, I have investigated a spell-out procedure whereby an ordered set of grammatical operations facilitates the  lexicalization of syntactic structures in a way that allows us to predict exactly (i) how many morphemes a given sequence of syntactic heads is going to be realized by and (ii) what positions these morphemes are going to take (``pre-'' vs. ``post-'' placement). Specifically, I have examined an  alternation in the domain of \ili{Slavic} verbs which exhibits a  reduction in the number of affixes on the root and considered prospects to derive this \isi{reduction} by adding \isi{subextraction} to the existing list of spell-out driven movements, an option that I compared to deriving the reduction with \isi{backtracking}.
\par 
Next, I have argued that we can resolve a morphological \isi{containment} problem found in certain \ili{Slavic} \isi{paradigm}s that cover \isi{syncretism}s with declarative \isi{complementizer}s by, on the one hand, extending the sequence of syntactic heads \is{fseq} and, on the other, by simplifying its underlying geometry to a singleton projection line. In other words, in order to be able to derive the attested patterns of morphological containment and syncretisms that conform to the \isi{*ABA} generalization, polymorphemic structures must be represented as singleton syntactic projection lines whose partition into more geometrically complex trees is exclusively a result of the application of the \isi{spell-out algorithm}. This rules out any syntactic representation of morphological forms as underlying geometrically complex tree structures beyond the single projection line.
\par
Such a description of polymorphemic forms effectively allows us to represent two- and three-dimensional morphological paradigms as a de facto one-dimensional space, a sequence of syntactic projections. This reduction makes correct predictions about syncretic alignment of \is{syncretism} morphemes forming subclasses of pronominal categories in \ili{Latvian}.

\section{Loose ends}\largerpage

Despite these results, there are at least two significant gaps in the analyses considered here that remain to be closed in future work.
\par
The first one concerns spell-out driven \isi{subextraction}. The inclusion of subextraction in the list of spell-out driven movements can in principle reduce \is{reduction} the amount of affixes observed in an alternation. However, it remains to be figured out if the so-called deep extractions are also permissible operations in the spell-out procedure. Likewise, the material discussed here does not reveal how subextraction should be ordered with respect to successive-cyclic movement and complement movement in the algorithm. \is{spell-out algorithm} That is, it remains unclear if attempting spell-out by moving the smallest possible piece of structure is ordered before or after attempting spell-out by moving the node that has been formed at the previous cycle. The first option suggests that subextraction is the first option in the algorithm, the second one suggests the opposite.
\par
The other missing piece concerns the representation of multi-dimensional morphological paradigms as singleton projection lines in syntax. In an approach that adopts the Superset Principle defined as in \ref{superset} in Chapter \ref{chapter:nanosyntax}, \cite{CahaPantcheva2012} explored the representation of two-dimensional paradigms based on mono-morphemic forms as singleton sequences of heads. In this work, this hypothesis has been illustrated to hold also for polymorphemic forms that form two- and three-dimensional paradigms in \ili{Latvian}. However, its extension to any $n$-dimensional \isi{paradigm}s remains only a conjecture at this point.
