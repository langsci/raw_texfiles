\author{Grit Nickel}
\title{Nominale Flexionsmorphologie in den ostoberdeutschen Dialekten Bayerns}
\renewcommand{\lsISBNdigital}{978-3-96110-414-7}
\renewcommand{\lsISBNhardcover}{978-3-98554-072-3}
\BookDOI{10.5281/zenodo.8094348}
% \typesetter{}
\proofreader{Amy Amoakuh}
\lsCoverTitleSizes{40pt}{14mm}
\renewcommand{\lsID}{363}
\renewcommand{\lsSeries}{ogl}
\renewcommand{\lsSeriesNumber}{6}
\dedication{Meinen Eltern}

\renewcommand{\lsImpressumExtra}
  {%
  \vskip.5\baselineskip
  Diese Arbeit wurde an der Sprach- und Literaturwissenschaftlichen Fakultät der Katholischen Universität Eichstätt-Ingolstadt als Dissertation angenommen. Referent der mündlichen Prüfung am 18. Mai 2021 war Prof. Dr. Sebastian Kürschner (Eichstätt), Koreferent war Prof. Dr. Alexander Werth (Passau).\medskip\\
  }
  
\BackBody{Diese Arbeit fokussiert die nominale Flexionsmorphologie der ostoberdeutschen Dialekte in ihrer Systematik. Dialekte sind insbesondere für Fragen zum morphologischen Wandel relevant, da sie im Vergleich zum Standard gesprochensprachlichen Wandel besser repräsentieren. Gleichzeitig weisen Dialekte spezifischen Wandel in Phonologie und an der Schnittstelle von Phonologie und Morphologie auf. Die kontrastive Studie dialektaler Flexionsverfahren unter varianten phonologischen Voraussetzungen kann hier zeigen, wo die formale Varianz phonologisch bedingt ist, wo sie das Ergebnis genuin morphologischer Prozesse ist und wo beide Ebenen interagieren. Damit verbindet die Studie die synchrone, diachrone und areale Perspektive.

Mit dem Ziel, die Spezifika und Gemeinsamkeiten der nominalen Numerus- und Kasusflexion für die drei Teilräume des Ostoberdeutschen (Ostfränkisch, Nord- und Mittelbairisch) in ihrer Systematik kontrastiv darzustellen, wurde für 37 Ortsdialekte und die syntaktische Einheit aus Definitartikel und Substantiv Datenmaterial des Forschungsprojekts Bayerischer Sprachatlas ausgewertet. Der erste Teil der Datenauswertung fokussiert die Formenbildung des Substantivs, wobei das Ziel der Untersuchung nicht nur in einer Inventarisierung der einzelnen (evtl. dialektraumspezifischen) Markierungsstrategien für Numerus und Kasus besteht, sondern in der Erfassung des Systems. Im Zentrum des zweiten Teils stehen die Struktur der dialektalen Deklinationsklassensysteme und die Frage, inwiefern Deklinationsklassen diachron und synchron zu außerflexivischer Konditionierung tendieren (z.B. durch semantische oder phonotaktisch-prosodische Faktoren). Der dritte Teil der Datenauswertung behandelt schließlich den morphosyntaktischen Kontext und die Frage, wo Numerus und Kasus in der Nominalphrase markiert werden und inwiefern die Markierung durch morphologische oder syntaktische Mittel oder durch Disambiguierung im semantisch-pragmatischen Kontext erfolgt. Abschließend erfolgt eine Diskussion der Ergebnisse vor dem Hintergrund von Grammatikmodellen, die morphologischen Wandel, Sprachgebrauch und Kognition fokussieren.}






