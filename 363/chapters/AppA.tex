% \addchap{Anhang: Ausgewertetes Datenmaterial}
% \label{app:A}
\chapter{Fragebuch der Erhebungen des \textit{Bayerischen Sprachatlas}}
% \label{app:A1}
Die Liste enthält die 271 Items, die in der Nominativ-Singular- und Pluralform und -- falls vorhanden -- in der Diminutivform, im Akkusativ oder Dativ (Singular und/oder Plural) für die Datenanalyse aufbereitet und ausgewertet wurden. In den entsprechenden Spalten sind die Fragen (i.\,d.\,R. in der Formulierung der BSA-Fragebücher, um sie über \textit{BayDat} auffindbar zu machen) nach Kasus sortiert aufgelistet; die Kasusangabe bezieht sich auf den standardsprachlichen Fragekontext im BSA-Fragebuch (zur Problematik siehe \sectref{sec:7.2}). Als erstes erscheint die Leitform, durch Schrägstriche getrennt werden mögliche Heteronyme genannt. Wurde ein Item nur in einzelnen Teilprojekten, nicht aber im gesamten UG abgefragt, ist die Kurzbezeichnung des entsprechenden Teilprojekts in eckigen Klammern notiert.

\small{%
\begin{xltabular}{\textwidth}{>{\raggedright\arraybackslash}p{.15\textwidth}QQQ}
\lsptoprule Lemma & Nominativ & Akkusativ & Dativ\\\midrule\endfirsthead
\midrule Lemma & Nominativ & Akkusativ & Dativ\\\midrule\endhead
\endfoot\lspbottomrule\endlastfoot
Achse & Achse -- Achsen &  & \\
Acker & Acker -- Äcker &  & \\
Ader & Ader -- Adern

Äderchen -- Äderchen (Pl.) &  & \\
Ähre & Ähre -- Ähren &  & \\
Ameise & Ameise -- Ameisen &  & \\
Apfel & Apfel -- Äpfelchen -- Äpfel & einen saftigen Apfel & \\
Arm & der Arm -- Arm (Dim.) --Arme & Er hat (sich) den Arm gebrochen (reflexiv?) & \\
Ast & der lange Ast stört -- Die Äste sind lang, Das sind lange Äste & den langen Ast... -- die langen Äste muß man abschlagen & an den Ästen\slash Bäumen (woran hängen die Äpfel? Eins von beiden) (Dat. Pl.)\\
Auge & Das Auge -- Äuglein -- sie hat blaue Augen, Aber seine Augen sind braun &  & \\
Bach & Bach (Genus) -- Bäche &  & Dort drüben überm Bach\\
Backe [SMF] & Backen (Genus!) -- Backen (Pl.) & Er hat einen geschwollenen Backen & \\
Band & Womit wurden Garben gebunden? (Sg.) -- Womit wurden Garben gebunden? (Pl.) &  & \\
Bank & Bank (Abb.) -- Bänklein & Sie haben noch die alten Bänke in der Stube & \\
Bauch & Bauch -- Bäuche &  & \\
Bauer & Bauer -- Bauern & den Bauer(n) (Akk.Sg.), Ich habe einen Bauern gefragt & \\
Bäuerin & Bäuerin -- Bäuerinnen &  & \\
Baum & Baum, ein hoher Baum -- Bäumlein --Bäume, hohe Bäume &  & an den Ästen\slash Bäumen (woran hängen die Äpfel? Eins von beiden) (Dat. Pl.), auf den Bäumen\\
Bein & Bein -- Beine, Die Beine tun mir weh &  & Er hat es in\slash an den Beinen\slash Füßen\\
Berg [SOB] & Berg -- Heute sieht man die Berge &  & er geht auf den Berg (hinauf)\\
Bett & Bett -- Betten &  & Sie sind im Bett\\
Beule & Beule (Genus!) -- Plural dazu &  & \\
Biene\slash Imme & Biene (Genus!) -- Bienen &  & \\
Bild & Bild -- Bild (Dim.)-- Bilder &  & \\
Birke & Birke -- Birken &  & \\
Birne & Birne -- Birnen &  & \\
Blatt [SMF] & Blatt der Säge -- Blätter &  & \\
Block (Stamm) [SNiB] & Stamm -- Stämme &  & \\
Bock & ein Bock -- Böcke &  & \\
Boden & Boden (in der Stube) -- Böden &  & \\
Bohne & Bohne -- Bohnen &  & \\
Borste & Borste -- Borsten &  & \\
Breite\slash Breiting & Breite -- Breiten (Pl.) &  & \\
Bremse\slash Breme & Bremse (Viehbremse) -- Bremsen &  & \\
Brett & Brett -- Bretter &  & \\
Brücke & Brücke -- Brücklein (Pl.) -- Brücken &  & \\
Bruder & mein Bruder, ihr Bruder -- meine Brüder, ihre Brüder (ihre Singular!) & Er hat meinen Bruder gemalt & \\
Bube & Bube -- Bübchen -- Buben &  & Die kleinen Mädchen spielen gern mit anderen Mädchen, aber sie spielen nicht gerne mit den Buben\\
Buche & Buche -- Buchen &  & \\
Dach & Dach -- Dächer &  & \\
Darm & Darm -- Därme &  & Er hat´s in den Därmen\\
Daube & eine neue Daube -- neue Dauben &  & \\
Daumen & Daumen -- Daumen (Pl.) &  & \\
Docht & Docht -- Döchtlein (Dim. ist notwendig, wenn bei 3 kein Umlaut) -- Dochte &  & \\
Dorf & Dorf -- Dörfer &  & \\
Dorn & Dorn -- Dornen &  & \\
Draht & Draht -- Drähte &  & \\
Egge & Egge -- Eggen (Pl.) &  & \\
Ei\slash Gackelein & Ei -- Eier &  & \\
Eiche & Eiche -- Eichen & Die Buben sind auf die Eiche gestiegen & Er steht unter der Eiche\\
Eimer & Eimer -- Eimer (Pl.) &  & \\
Ellenbogen & Ellenbogen -- Ellenbogen (Pl.) &  & \\
Ente & Ente -- Enten &  & \\
Erbse & Erbse -- Erbsen &  & \\
Erle & Erle -- Erlen &  & \\
Esche & Esche -- Eschen &  & \\
Faden & Faden -- Fäden &  & \\
Fahne & Fahne (Genus!) -- Fahnen &  & \\
Fass & Faß -- Fässer

Fäßlein -- Fäßlein (Pl.) &  & \\
Faust & Faust -- Fäuste &  & \\
Feder & Feder -- Federn &  & \\
Feiertag & Feiertag -- Feiertage &  & \\
Feld & Feld (Bedeutung?) -- Felder &  & \\
Fenster & das Fenster -- die Fenster &  & \\
Fest & Fest -- Feste &  & \\
Fichte & Fichte -- Fichten &  & \\
Fisch & Fisch -- Fische &  & \\
Flasche & eine volle Flasche -- Die Flaschen sind gewiß voll &  & \\
Fleck [SMF] & Fleck (allg.) -- Flecke &  & \\
Fliege & Fliege (Mücke?) - Fliegen &  & \\
Floh & Floh (Genus!) -- Flöhe &  & \\
Flügel & Flügel (Sg.) -- Flügelein (Dim.Pl.) -- Flügel (Pl.) &  & \\
Föhre\slash Kiefer & Föhre -- Föhren &  & \\
Frau & Frau -- Frauen &  & Er hat es einer alten Frau gebracht\\
Frosch & Frosch -- Frösche &  & \\
Fuchs & Fuchs -- Füchse &  & \\
Furche & Furche -- Furchen &  & \\
Fuß & Fuß & Zum Laufen braucht man gute Füße & Er hat es in\slash an den Beinen\slash Füßen\\
Gabel (Besteck) & Gabel -- Gabeln &  & \\
Gabel (Landwirtschaft) & Gabel -- Gabeln &  & \\
Gang & Gang -- Gänge &  & \\
Gans & Gans -- Gänse &  & \\
Garbe & Garbe -- Garben &  & \\
Garten & Garten -- Gärtlein -- Gärten, Der Garten braucht das Gießen & Ich gehe durch den Garten & Ihr seid im Garten\\
Geiß & Geiß -- Geißen &  & \\
Gemeinde & Gemeinde -- Gemeinden &  & \\
Glas & Glas -- Gläslein -- Gläser &  & \\
Glocke &  & Wir brauchen eine neue Glocke -- Wir brauchen neue Glocken & Man tut mit allen Glocken läuten\\
Grab & Grab -- Dim. (Kindergrab!) -- Gräber &  & \\
Graben & Graben -- Gräben -- Gräb(e)lein & Da haben sie einen Graben gegraben, Wir graben einen Graben & \\
Griff & Griff -- Griffe &  & \\
Haar & Haar -- Das sind schwarze Haare &  & \\
Hafen (Gefäß)\slash Topf\slash Tiegel & Hafen (Fleischtopf zum Kochen) -- Hafen (Dim.) -- Häfen &  & \\
Haken & Haken -- Häklein -- Haken (Pl.) &  & \\
Halm & Halm -- Halme &  & \\
Hammer & Hammer -- Hämmerlein -- Hämmer &  & \\
Hand & Hand -- Händlein -- Hände & Sie hat sich die Hand verstaucht & Er faßt es mit beiden Händen an, Er hat Warzen an den Händen, Er schreibt alles mit der linken Hand\\
Hase & Hase -- Hasen (Pl.) & den Has(en) (Akk.Sg.) & \\
Haufen & Haufen -- Haufen (Pl.)

Häuflein -- Häuflein (Pl.) &  & \\
Haus & ein Haus, das alte Haus -- beide Häuser, Die Häuser gefallen mir nicht (Demonstr. Pron. betont!) & die alten Häuser hat man abgerissen & nahe bei den alten Häusern\\
Haut & Haut -- Häute &  & \\
Hecht & Hecht -- Hechte & Er hat einen Hecht gefangen (Akk.Sg.) & \\
Hemd & Hemd -- Hemden &  & \\
Herz & Herz -- Herzen (Pl.) &  & Er hat´s am Herzen\\
Hobel & Hobel -- Hobel (Pl.) &  & \\
Höhe & Höhe -- Höhen & Der Schreiner mißt die Höhe vom Schrank & Der Schreiner hat sich mit der Höhe vertan\\
Horn & Horn -- Hörnlein (z.B. bei der Ziege) -- Hörner &  & \\
Huhn\slash Henne & Huhn\slash Henne -- Hühnlein -- Hühner\slash Hennen &  & \\
Hund & ein Hund -- Hündlein -- Hunde &  & \\
Hut & Hut -- Hüte &  & \\
Hütte\slash Schupfen\slash Stadel & Freistehende landwirtschaftlich genutzte kleinere Gebäude -- Pl. &  & in einer alten Hütte\\
Jahr & ein Jahr -- Er ist vor zehn Jahren gekommen &  & \\
Joch & Joch -- Jöcher &  & \\
Kalb & Kalb -- Kälber

Kälblein -- Das sind schöne Kälblein &  & \\
Kamm & Kamm (des Hahns) -- Kämme &  & \\
Kammer & Kammer -- Kammern &  & Wir sind in der Kammer\\
Karpfen [SMF, SUF] & Karpfen -- Karpfen (Pl., nur SNOB) & Er hat einen Karpfen gefangen, Akk.Sg. -- Pl. & \\
Karren & Karren (Genus! Bedeutung!) -- Karren (Pl.) &  & \\
Katze\slash Kater & unsere Katze -- unsere Katzen &  & Mit unseren Katzen sind wir zufrieden.\\
Kern & Kern -- Kerne &  & \\
Kerze & Kerze -- Kerzen &  & \\
Kette & Kette -- Ketten (-enen?) &  & mit (den) Ketten (mit oder ohne Artikel?)\\
Kind & ein gesundes Kind -- Die Kinder spielen im Garten & Sie hat lauter böse Kinder & Welchen Kindern hast du es gegeben?\\
Kirche & Kirche -- Kirchlein -- Kirchen &  & \\
Kirchturm\slash Turm & Kirch“turm“ -- Kirch“türme“ & Ich gehe um den Turm herum. & Von\slash Auf einem hohen Turm (sieht man viel).\\
Kirsche & Kirsche -- Kirschen &  & \\
Klaue & Klauen (der Kuh) -- Klauen (Pl.) &  & \\
Kloß & Kloß (Genuß!) -- Klöße (Pl.) &  & \\
Knecht & Knecht -- Knechte &  & \\
Knie & Knie -- Knie (Pl.) &  & \\
Knopf & Knopf -- Knöpfe &  & \\
Knoten & Was macht man ins Taschentuch, um an eine bestimmte Sache erinnert zu werden? \slash Knopf \slash Knoten \slash Knipfel \slash -- Plural davon &  & \\
Kopf & Kopf -- Köpfchen/{}-lein -- Köpfe & er hat sich den Kopf angeschlagen & \\
Korb & Korb -- Körbe &  & \\
Körnlein\slash Korn [SUF] & Körnlein -- Körnlein (Pl.) &  & \\
Kropf & Kropf -- Kröpfe &  & \\
Kröte\slash Protz & Kröte -- Pl. davon &  & \\
Krug & Krug -- Krug (Dim.)-- Krüge &  & \\
Kübel & Kübel -- Kübel (Pl.) &  & \\
Kuchen & Kuchen (Sg.) -- Kuchen (Pl.) &  & \\
Kuh & Kuh -- Kühe, Das sind schöne Kühe & Die alte Kuh muß man schlachten, Die Kühe kauf' ich nicht (Dem. Pron. betont) & Man gibt´s den Kühen (das Futter) (sagt man: „in die Kühe“?), Du gibst jetzt unseren Kühen etwas zum Fressen.\\
Kummet\slash Geschirr & Kummet (Genus!) -- Pl. von Kummet oder Geschirr &  & \\
Kürbe & Kürbe -- Kürben (Pl.) &  & \\
Laib & Laib -- Laibe &  & \\
Lärche & Lärche -- Lärchen &  & \\
Latte & Latte -- Latten &  & \\
Leise (Geleise) & (Ge-)Leise (Genus) -- Leisen (Pl.) &  & \\
Loch & Loch -- Löcher &  & \\
Lohn & Lohn (Genus) -- Löhne &  & \\
Mädchen (Mädel)\slash Dirndl & Mädchen -- Mädchen (Dim.) -- Mädchen (Pl.) &  & Den kleinen Mädchen hab ich es gegeben\\
Magd\slash Dirn & Magd -- Mägde &  & \\
Magen & Magen -- Mägen &  & \\
Mann & Mann -- Männer &  & er sagt es einem Mann, Ich habe es dem alten Mann gegeben, Ich hab's einem alten Mann gesagt\\
Markt & Markt -- Märkte &  & \\
Masen\slash Wasen & Obere Schicht des Grasbodens -- Pl. &  & \\
Maul & Maul -- Mäulchen -- Mäuler &  & \\
Maus & Maus -- Mäuse &  & \\
Mücke & Mücke (Bedeutung) -- Mücken &  & \\
Mutter & Mutter -- Mütter &  & Er hat es der Mutter gesagt, Ich sag's deiner Mutter\\
Nacht & Nacht -- Nächte &  & \\
Nadel & Nadel -- Nadeln &  & \\
Nagel & Nagel -- Nägelein -- Nägel & Schlag mir einen Nagel ein, sonst hält es nicht & \\
Näherin & Näherin -- Plural davon &  & \\
Naht & Naht -- Nähte &  & \\
Name & Name -- Namen (Pl.) &  & \\
Nasenloch & Nasenloch -- Nasenlöcher &  & \\
Nest & Nest -- Nester &  & \\
Netz & Netz -- Netze &  & \\
Nudel (Ludel) & Nudel -- Nudeln &  & \\
Nuss & Nuss -- Nüsse &  & \\
Ochse & Kastriertes männliches Rind -- die Ochsen & einen Ochsen (Akkusativ!) & \\
Ofen & Ofen -- Ofen (Dim.) -- Öfen &  & \\
Ohr & Das Ohr -- Die Ohren &  & \\
Pfahl\slash Pflock & Pfahl\slash Pflocker\slash Pfahl\slash Stickel -- Plural davon &  & \\
Pflug & Pflug -- Pflüge &  & \\
Prügel & Prügel -- Prügelchen (Bedeutung!) -- Prügel (Pl.) &  & \\
Rabe\slash Krack\slash Krähe & Rabe (Krähe?) -- Raben & Akk. Sg. [SUF] & \\
Rad & Rad -- Räder

Rädlein -- Rädlein (Pl.) &  & \\
Rain & Rain -- Raine &  & \\
Ratte & Ratte -- Ratten &  & \\
Rebe & Woran wachsen die Trauben? (Spontanantwort: Rebe\slash Reben; Sg. oder Pl.?) -- Wie heißt der Sg. oder Pl. dazu? &  & \\
Reifen & Reifen -- Reifen (Pl.) &  & \\
Rippe & Rippe -- Rippen & Er hat sich eine Rippe gebrochen & \\
Rock & Rock -- Röcke &  & \\
Sack & Sack -- Säcklein -- Säcke &  & \\
Säge & Säge -- Sägen (Pl.) &  & \\
Sau & Sau -- Säue &  & \\
Säule & Säule -- Säulen &  & \\
Schaf & Schaf (Genus) -- Schafe &  & \\
Schaff & Schaff -- Schäfflein -- Pl. davon &  & \\
Schaufel & Schaufel -- Schaufeln &  & \\
Schere & Schere -- Scheren &  & \\
Schienbein & Schienbein -- Schienbeine &  & \\
Schlag & Schlag (Sg.) -- Schläglein (Bedeutung!) -- Schläge &  & \\
Schlitten & Schlitten -- Schlitten (Pl.) &  & \\
Schloss & Vorhängeschloß -- Schlösser &  & \\
Schlot & Schlot -- Schlöte &  & \\
Schlüssel & Schlüssel -- Schlüssel (Pl.) &  & \\
Schnabel & Schnabel -- Schnäbel &  & \\
Schrunde\slash Kluft & Schrunde -- Schrunden &  & \\
Schüssel & Schüssel -- Schüsseln &  & \\
Schwanz & Schwanz -- Schwänzlein -- Schwänze &  & \\
Schweif & Schweif -- Schweife &  & \\
Schwester & meine Schwester, ihre Schwester (ihre Singular!) -- meine Schwestern, ihre Schwestern (ihre 3.Pers.Plural!) &  & \\
See & See -- Seen &  & \\
Seife & Seife -- Seifen &  & \\
Seil & Seil -- Seile &  & \\
Sense & Sense -- Sensen &  & \\
Sieb\slash Reiter & Getreidesieb\slash Reiter\slash (Rodel)Sieb\slash Riesel\slash Röller -- Plural davon . &  & \\
Sohle & Sohle -- Sohlen, Beide Sohlen sind hin“ &  & \\
Sohn & Sohn -- Söhne &  & \\
Spatz\slash Sperk & Spatz -- Spatzen (Pl.) & den Spatz(en) (Akk.Sg.) & \\
Speiche & Speiche (Genus!) -- Speichen &  & \\
Sprosse [SNOB]) & Die dazwischen stehenden Sprossen -- Pl. &  & \\
Stadt & Städte & Ich gehe in die Stadt & \\
Stall & Stall -- Ställe &  & \\
Star (Starl) & Star -- Stare &  & \\
Stecken & Stecken -- Stecken (Pl.) & Einen Stecken in den Boden stecken & \\
Stein & Mark\slash Grenzstein -- Steine &  & \\
Stern & Stern -- Sterne &  & \\
Stich & Stich -- Stiche &  & \\
Stiefel & Stiefel -- Stiefel (Pl.) &  & \\
Stock & Stock -- Stöcke &  & \\
Straße & Straße -- Sträßlein -- Straßen & Da bauen sie wieder eine neue Straße (Akk.Sg.) & \\
Straßen\-graben [SNiB] & Straßengraben -- Pl. &  & \\
Strauß & Strauß -- Sträuße &  & \\
Striegel & Striegel (Womit putzt man die Kühe?) -- Striegel (Pl.) &  & \\
Strumpf & Strumpf -- Strümpfe &  & \\
Stube & Stube -- Stüblein -- Stuben (Pl.) &  & Du bist in der Stube\\
Stuhl & Stuhl -- Stuhl (Dim.)-- Stühle &  & \\
Suppe & Suppe -- Suppen &  & \\
Tag & Heute ist ein schöner Tag -- Eine Woche hat sieben Tage, Das sind lauter heiße Tage gewesen („Tage“ mit Umlaut?) & Es hat den ganzen Tag geschneit & \\
Tal [SMF, SUF] & Tal -- Tälchen -- Täler &  & \\
Tanne & Tanne -- Tannen &  & \\
Tasche & Tasche -- Taschen &  & \\
Taube & eine Taube -- Tauben (Pl.) &  & \\
Tisch & Tisch -- Tische & Ich muss den Tisch decken (SNOB) & Die Milch steht auf dem Tisch\\
Tochter & Tochter -- Töchter &  & \\
Tor & Tor -- Tore &  & \\
Trog & Trog -- Brunnentröge (Pl.) &  & \\
Tür & Tür -- Türen &  & \\
Vieh & Vieh -- Viecher (Kann Plural zu Vieh gebildet werden? Oder ist es sowieso schon Plural?) &  & \\
Vogel & Vogel -- Vögelein -- Vögel &  & \\
Wade & Wade(n) (Genus!) -- Waden (Pl.) &  & \\
Wagen & Wagen -- Wägen

Wägelein (Sg.) -- Wägelein (Pl.) &  & \\
Wald\slash Holz & Wald -- Wälder &  & \\
Wand & Wand -- Wände &  & \\
Wanne & Wanne -- Wanne (Dim.) -- Wannen &  & \\
Warze & Warze -- Warzen &  & \\
Wäscheseil\slash Seil\slash Schnur & Wäscheseil -- Wäscheseile &  & \\
Weg & Weg -- Wege &  & \\
Weib & Weib -- Weiber &  & \\
Werktag & Werktag -- Werktage &  & \\
Wespe & eine Wespe (Genus!) -- Wespen &  & \\
Wiese & Wiese -- Plural dazu &  & \\
Wirt & Wirt -- Wirte &  & \\
Woche & Woche -- Wochen &  & \\
Wolke & eine Wolke -- Es sind/hat Wolken am Himmel &  & \\
Wurm & Wurm -- Würmer &  & \\
Wurst & Wurst -- Würste &  & \\
Wurzel & Wurzel -- Wurzeln &  & \\
Zahn & Zahn -- Dim. von Zahn -- Zähne & Den faulen Zahn muß man ziehen & Er hat Probleme mit den Zähnen\\
Zaun & Allgemeine Bezeichnung des Zauns -- Plural davon &  & \\
Zecke & Blutsauger, den man nicht aus der Haut bekommt\slash Zecke\slash Holzbock\slash -- die Zecken (Pl.) & den Zeck(en) (Akk.Sg.) & \\
Zehe & Zehe (Genus!) -- Zehen &  & \\
Zopf & Zopf & Die langen Zöpfe hat sie abgeschnitten & \\
Zwetschge\slash Pflaume & Zwetschge -- Zwetschgen &  & \\
\end{xltabular}}

\chapter{Wenkersätze}
% \label{app:A2}
Für die flexionsmorphologische Analyse wurden sämtliche Wenkersätze ausgewertet, die für die untersuchten Lemmata eine Flexionsform des Substantivs im Plural und\slash oder den obliquen Kasus enthalten. Zudem wurde auch die Singularstammformbildung sowie die phonetisch-phonologische Realisierung der Nominativ-Singular-Form berücksichtigt.


\begin{xltabular}{\textwidth}{lllQ}
\lsptoprule Lemma & Numerus & Kasus & Wenkersatz\\\midrule\endfirsthead
\midrule Lemma & Numerus & Kasus & Wenkersatz\\\midrule\endhead
\endfoot\lspbottomrule\endlastfoot
Apfel & Pl. & Dat. & 26. Hinter unserm Hause stehen drei schöne Apfelbäumchen mit rothen Aepfelchen.\\
Bauer & Pl. & Nom. & 37. Die Bauern hatten fünf Ochsen und neun Kühe und zwölf Schäfchen vor das Dorf gebracht, die wollten sie verkaufen.\\
Baum & Pl. & Nom. & 26. Hinter unserm Hause stehen drei schöne Apfelbäumchen mit rothen Aepfelchen.\\
Berg & Pl. & Nom. & 29. Unsere Berge sind nicht sehr hoch, die euren sind viel höher.\\
Bett & Sg. & Dat. & 24. Als wir gestern Abend zurück kamen, da lagen die Andern schon zu Bett und waren fest am schlafen.\\
Blatt & Pl. & Nom. & 1. Im Winter fliegen die trocknen Blätter durch die Luft herum.\\
Bruder & Sg. & Nom. & 33. Sein Bruder will sich zwei schöne neue Häuser in eurem Garten bauen.\\
Bürste & Sg. & Dat. & 17. Geh, sei so gut und sag Deiner Schwester, sie sollte die Kleider für eure Mutter fertig nähen und mit der Bürste rein machen.\\
Dorf & Sg. & Akk. & 37. Die Bauern hatten fünf Ochsen und neun Kühe und zwölf Schäfchen vor das Dorf gebracht, die wollten sie verkaufen.\\
Ei & Pl. & Akk. & 7. Er ißt die Eier immer ohne Salz und Pfeffer.\\
Feld & Sg. & Dat. & 38. Die Leute sind heute alle draußen auf dem Felde und mähen/hauen.\\
Frau & Sg. & Dat. & 9. Ich bin bei der Frau gewesen und habe es ihr gesagt, und sie sagte, sie wollte es auch ihrer Tochter sagen.\\
Fuß & Pl. & Nom. & 8. Die Füße thun mir sehr weh, ich glaube, ich habe sie durchgelaufen.\\
Gans & Pl. & Nom. & 14. Mein liebes Kind, bleib hier unten stehn, die bösen Gänse beißen Dich todt.\\
Garten & Sg. & Dat. & 33. Sein Bruder will sich zwei schöne neue Häuser in eurem Garten bauen.\\
Haus & Sg. & Dat. & 15. Du hast heute am meisten gelernt und bist artig gewesen, Du darfst früher nach Hause gehn als die Andern.

26. Hinter unserm Hause stehen drei schöne Apfelbäumchen mit rothen Aepfelchen.\\
& Pl. & Akk. & 33. Sein Bruder will sich zwei schöne neue Häuser in eurem Garten bauen.\\
Herz & Sg. & Dat. & 34. Das Wort kam ihm von Herzen!\\
Hund & Sg. & Dat. & 39. Geh nur, der braune Hund thut Dir nichts.\\
Kind & Sg. & Nom. & 14. Mein liebes Kind, bleib hier unten stehn, die bösen Gänse beißen Dich todt.\\
Korb & Sg. & Akk. & 19. Wer hat mir meinen Korb mit Fleisch gestohlen?\\
Korn & Sg. & Akk. & 40. Ich bin mit den Leuten da hinten über die Wiese ins Korn gefahren.\\
Kuchen & Pl. & Nom. & 6. Das Feuer war zu stark/heiß, die Kuchen sind ja unten ganz schwarz gebrannt.\\
Kuh & Pl. & Akk. & 37. Die Bauern hatten fünf Ochsen und neun Kühe und zwölf Schäfchen vor das Dorf gebracht, die wollten sie verkaufen.\\
Mann & Sg. & Nom. & 4. Der gute alte Mann ist mit dem Pferde durch´s Eis gebrochen und in das kalte Wasser gefallen.\\
Mutter & Sg. & Akk. & 17. Geh, sei so gut und sag Deiner Schwester, sie sollte die Kleider für eure Mutter fertig nähen und mit der Bürste rein machen.\\
Nacht & Sg. & Nom. & 25. Der Schnee ist diese Nacht bei uns liegen geblieben, aber heute Morgen ist er geschmolzen.\\
Ochse & Pl. & Akk. & 37. Die Bauern hatten fünf Ochsen und neun Kühe und zwölf Schäfchen vor das Dorf gebracht, die wollten sie verkaufen.\\
Ofen & Sg. & Akk. & 3. Thu Kohlen in den Ofen, daß die Milch bald an zu kochen fängt.\\
Ohr & Pl. & Akk. & 11. Ich schlage Dich gleich mit dem Kochlöffel um die Ohren, Du Affe!\\
Pferd & Sg. & Dat. & 4. Der gute alte Mann ist mit dem Pferde durch´s Eis gebrochen und in das kalte Wasser gefallen.\\
Schaf & Pl. & Akk. & 37. Die Bauern hatten fünf Ochsen und neun Kühe und zwölf Schäfchen vor das Dorf gebracht, die wollten sie verkaufen.\\
Schwester & Sg. & Dat. & 17. Geh, sei so gut und sag Deiner Schwester, sie sollte die Kleider für eure Mutter fertig nähen und mit der Bürste rein machen.\\
Seife & Sg. & Nom. & 32 Habt ihr kein Stückchen weiße Seife für mich auf meinem Tische gefunden?\\
Tisch & Sg. & Dat. & 32 Habt ihr kein Stückchen weiße Seife für mich auf meinem Tische gefunden?\\
Tochter & Sg. & Dat. & 9. Ich bin bei der Frau gewesen und habe es ihr gesagt, und sie sagte, sie wollte es auch ihrer Tochter sagen.\\
Vogel & Pl. & Nom. & 36. Was sitzen da für Vögelchen oben auf dem Mäuerchen?\\
Wiese & Sg. & Akk. & 40. Ich bin mit den Leuten da hinten über die Wiese ins Korn gefahren.\\
Woche & Pl. & Dat. & 5. Er ist vor vier oder sechs Wochen gestorben.\\
\end{xltabular}
