\chapter{Morphologische Theoriebildung}
\label{chap:5}
Nachdem in den Kapiteln~\ref{chap:3} und \ref{chap:4} ein Forschungsüberblick über den Phänomenbereich der Untersuchung gegeben wurde, bietet \chapref{chap:5} einen ersten Zugang zum morphologietheoretischen Rahmen der Arbeit. Im Fokus der sprachtheoretischen Analyse stehen Fragen zu morphologischem Wandel, die zum einen das Verhältnis von Form und Funktion und zum anderen die mentale Repräsentation von Flexion und Flexionsklassen betreffen. Ziel ist es, über diese theoretischen Fragestellungen und durch die Anbindung an vorhandene Theorien dialektale Phänomene der Flexion und des morphologischen Wandels zu reflektieren und gleichzeitig eine empirische Fundierung der morphologischen Theorien zu leisten (siehe \chapref{chap:10}).

Im folgenden Kapitel werden die ausgewählten Ansätze zunächst kurz vorgestellt. Einen ersten Zugang zur Relation von Form und Funktion bieten die Konzepte und Vorhersagen der Natürlichen Morphologie (\sectref{sec:5.1}), in \sectref{sec:5.2} wird dieser Aspekt vor dem Hintergrund von Joan \citegen{Bybee1985b} Relevanzprinzip behandelt. In \sectref{sec:5.3} folgen die gebrauchsbasierten Grammatikansätze des Netzwerkmodells und das Schema-Modell nach \citet{Köpcke1993}. Die einzelnen Modelle werden dabei nicht in toto dargestellt, sondern es werden zentrale Konzepte eingeführt und einzelne, -- mit Blick auf die dialektalen Daten -- besonders relevante Aspekte vorgestellt.

\section{Die Natürliche Morphologie}
\label{sec:5.1}
Der Kerngedanke der Natürlichen Morphologie besteht darin, dass bestimmte morphologische Strukturen und Prozesse natürlicher sind als andere. Ein morphologischer Prozess oder eine morphologische Struktur sind nach \citet[2]{Mayerthaler1981} natürlich, „wenn er/sie a) weit verbreitet ist und/oder b) relativ früh erworben wird und/oder c) gegenüber Sprachwandel relativ resistent ist oder durch Sprachwandel häufig entsteht“. Das Konzept der Natürlichkeit steht dabei in einem umgekehrt proportionalen Verhältnis zum Konzept der Markiertheit: Je natürlicher eine Struktur ist, desto unmarkierter ist sie (siehe auch \citealt[180--182]{HarnischRowley1990}). Diese Grundidee der Natürlichen Morphologie wurde u.\,a. von \citet{Wurzel1984, Wurzel1994} weiterentwickelt. Annahme ist auch hier, dass natürlicher grammatischer Wandel (d.\,h. grammatisch initiierter Wandel) Natürlichkeitsprinzipien entspricht, dass er also in Richtung eines Abbaus von Markiertheit verläuft (\citealt[188]{Wurzel1984}, \citealt[29]{Wurzel1994}).

Der Grad der Markiertheit morphologischer Kategorien und der morphologischen Kodierung dieser Kategorien wird nach \citet[10]{Mayerthaler1981} durch Markiertheitswerte angegeben. Zentral ist dabei der Gedanke, dass diese Markiertheitswerte „in einer psychisch realen Weise“ \citep[8]{Mayerthaler1981} den Komplexitätsgrad morphologischer Strukturen abbilden. In der Natürlichen Morphologie waren mithin bereits Aspekte einer kognitiven Perspektive auf morphologische Strukturen angelegt, \citet[187]{Wurzel1984} zufolge ergeben sich die „generellen Prinzipien der grammatischen Strukturbildung [\ldots] aus den Bedingungen der Produktion, Perzeption und Speicherung sowie aus der Funktion morphologischer Strukturen“ (vgl. \citealt[175]{HarnischRowley1990}).

Unter der Prämisse von Morphologie als geschlossenem, autonomen System bilden sich, so die Voraussage der Natürlichen Morphologie, optimale morphologische Kodierungen heraus. Optimal und damit maximal natürlich ist eine Kodierung nach \citet[22]{Mayerthaler1981}, wenn sie „konstruktionell ikonisch, uniform und transparent ist“ (vgl. \citealt[56]{Wurzel1994}). Eine Kodierung ist dann transparent, wenn jeder Form genau eine Funktion entspricht, und sie ist uniform, wenn jeder Funktion genau eine Form entspricht. Die Prognose von optimalen, natürlichen Strukturen weist hier also in Richtung eines Abbaus von Allomorphien (im Sinne der Uniformität) und von Portmanteau-Morphemen (im Sinne der Transparenz). Die Idee hinter dem Prinzip des konstruktionellen Ikonismus besteht darin, dass ein semantisches „Mehr“ auch durch ein formales „Mehr“ markiert wird. Die Ikonizität einer morphologischen Kodierung wird graduell beschrieben: Additiv-segmentierbare Kodierungen (\textit{Hund} -- \textit{Hund-}\textbf{\textit{e}}) sind ikonischer als stamm\-affizierende Kodierungen (ofr. h\textbf{u}nd -- h\textbf{ü}nd, bair. h\textbf{ū}n\textbf{d} -- h\textbf{u}n\textbf{t}), Nullplurale sind nicht-ikonisch (ofr. hund -- hund). Kontra-ikonisch sind die subtraktiven Pluralformen, die in einigen deutschen Dialekten zu finden sind (z.\,B. hess. hond -- hon ‚Hund‘, siehe \sectref{sec:4.2.1}): Ein semantisches „Mehr“ wird hier durch ein formales „Weniger“ symbolisiert.

\begin{sloppypar}
Phonologie kann dabei, wie es \citet[43]{Mayerthaler1981} plakativ formuliert, „kontramorphologisch“ wirken und „Störungen“ des konstruktionellen Ikonismus verursachen. So hat etwa die Apokope des Schwa-Suffixes in weiten Teilen der deutschen Dialekte zum Wegfall eines maximal ikonischen, weil segmentierbaren Verfahrens geführt. Aus Perspektive der Natürlichen Morphologie sind hier „Natürlichkeitskonflikte“ zu beobachten zwischen dem Wandel hin zu phonologischer Natürlichkeit, die in der optimalen Artikulation und Perzeption besteht, und den Idealen morphologischer Natürlichkeit (\citealt[30]{Wurzel1984}, \citealt[43--44]{Mayerthaler1981}). Dass aber auch die Morphologie „Störungen“ des konstruktionellen Ikonismus schaffen und morphologische Distinktionen aufheben kann, zeigt das regional begrenzte Ausgleichsmuster der fem. \textit{n}{}-Klasse, das im Frühneuhochdeutschen in den obd. Dialekten zu synkretischen und damit nicht-ikonischen Singular- und Pluralparadigmen führt: Durch innerparadigmatischen Ausgleich wird das Nasalsuffix der obliquen Kasus analog in den Nom.Sg. übertragen (Nom.""Sg. \textit{zunge} > \textit{zungen} neben Pl. \textit{zungen}, siehe \sectref{sec:3.1.2}).
\end{sloppypar}

\citet{Wurzel1990} modelliert das Ineinandergreifen phonologischer und morphologischer Prozesse, das sich im Laufe der Sprachgeschichte immer wieder vollzieht, aus Perspektive der Natürlichen Morphologie exemplarisch für die Morphologisierung des \textit{i}{}-Umlauts. Der \textit{i}{}-Umlaut ist im Althochdeutschen ein zunächst rein phonologischer Prozess, der aus der assimilatorischen Palatalisierung eines velaren Vokals durch ein /i/ oder /j/ in der Folgesilbe besteht. „Zufälligerweise“ \citep[134]{Wurzel1990} ist diese phonologische Alternation an morphologische Kategorien geknüpft; motiviert ist sie indes durch den Abbau phonologischer Markiertheit. Sind die lautlichen Bedingungen der phonologischen Alternation nicht mehr transparent, werden die Alternanten von den Sprechern „als durch die Kategorien bedingt aufgefaßt, in denen sie vorkommen“ \citep[134]{Wurzel1990}. In diesem Morphologisierungsprozess spielen Prinzipien der morphologischen Natürlichkeit laut \citet[135]{Wurzel1990} zunächst keine Rolle: Die Verteilung der neuen Marker im Paradigma (hier des Kasus- und Pluralumlauts) ist -- aus flexionsmorphologischer Perspektive -- „zufällig“, die Symbolisierung der Flexionskategorien nicht uniform. An den Prozess der „(eigentlichen) Morphologisierung“ schließen sich „‚Ausgleichsprozesse‘“ an, die -- den Prinzipien der morphologischen Natürlichkeit entsprechend -- zu uniformen, systemangemessenen (d.\,h. natürlichen) Paradigmen führen. Erst diese „Ausgleichsmorphologisierung“ \citep[137]{Wurzel1990} führt im Althochdeutschen zur Tilgung des Kasusumlauts im Singularparadigma: Gen./Dat.Sg. ahd. \textit{henin} > \textit{hanin} ‚Hahn‘ (siehe ausführlicher \sectref{sec:3.1.1} und \tabref{tab:7}). Die Morphologisierung des Pluralmarkers Umlaut ist dabei, so \citet[138]{Wurzel1990}, „in keinem Fall ein ‚gezielte‘ Herausbildung von Flexionsmorphologie“, sondern durch den „Abbau von Markiertheit im Sprachsystem“ bedingt.

\citet{Wurzel1984, Wurzel1994} entwickelt das Konzept der Natürlichen Morphologie weiter, indem er die Kriterien der einzelsprachlichen „Normalität“ und der „systemdefinierenden Struktureigenschaften“ einführt.\footnote{\textrm{Nach \citet[133--134]{Mayerthaler1981} darf Systemangemessenheit (sprachspezifische „Normalität“) indes nicht mit Natürlichkeit gleichgesetzt werden, da Natürlichkeit universal sei und „in letzter Instanz“ auf der Perzeption basiere.}} Die Voraussage lautet, dass innerhalb eines Systems weniger natürliche (also weniger systemangemessene) Struktureigenschaften abgebaut werden und dabei das Prinzip der Flexionsklassenstabilität steuernd wirkt. Demnach findet Deklinationsklassenwechsel statt, wenn „die Wörter von der weniger normalen zur normaleren Flexionsklasse übertreten; systembezogene Markiertheit wird durch Klassenwechsel abgebaut“ \citep[78]{Wurzel1984}. Ergebnis dieses Wandels können dabei morphologische Strukturen sein, die nicht dem universalen Ideal optimaler morphologischer Kodierungen entsprechen; sie entsprechen aber dem systemspezifischen Ideal stabiler, einheitlicher Flexionsklassen und -systeme.

Welche Kriterien konstituieren dabei, was in einem Flexionssystem „normal“ ist? Nach \citet[84--86]{Wurzel1984} ist hier vor allem das „Gewicht“ der einzelnen Varianten prägend und wie dominierend ein Strukturmerkmal im System sind, also letztlich die hohe Typenfrequenz des Merkmals und einzelner Flexionsklassen.\footnote{\textrm{\citet[140]{Mayerthaler1981} sieht Frequenz indes als} „Epiphänomen von Natürlichkeit“, Natürlichkeit sei „viel interessanter und explanativer“ als Frequenz.} Im Hinblick auf das Spannungsfeld zwischen regulären und irregulären Strukturen integriert \citet{Harnisch1988} den Ansatz \citegen{Wurzel1984} und die sprachökonomischen Ansätze der Ökonomietheorie nach \citeua{Werner1987} und von \citegen{Bybee1985b} Netzwerkmodell (siehe \sectref{sec:5.3}). Eine Optimierung von Morphologie kann neben dem Streben nach optimaler Symbolisierung auch durch „kognitionsökonomische“ \citep[436]{Harnisch1988} Prinzipien geleitet sein, wenn die ganzheitliche Speicherung und der direkte Zugriff auf kürzere irreguläre oder suppletive Formen kognitiv vorteilhaft ist (siehe hierzu auch \citealt{Bittner1988}, \citealt{Harnisch1990} und \citealt{Ronneberger-Sibold1988}). Auch \citet[185--186]{Nübling2004} zeigt, dass irreguläre oder suppletive Formen in der Verbalmorphologie keine „Denaturalisierungsprozesse“ darstellen, sondern dass die „Analogieresistenz“ der Morphologie bei hoher Gebrauchsfrequenz (d.\,h. Tokenfrequenz) ökonomisch ist (siehe auch \citealt{Bybee1994}).

\section{Bybees Relevanzprinzip}
\label{sec:5.2}

Im Zentrum des Netzwerkmodells von \citet{Bybee1985b} steht die Frage, welche flexivischen Informationen oder Kategorien wie ausgedrückt werden und inwiefern dies in einem Zusammenhang mit „the general cognitive and psychological characteristics of human language users“ \citep[3]{Bybee1985b} steht. \citegen{Bybee1985b} Überlegungen zur morphologischen Organisation des Lexikons werden detaillierter in \sectref{sec:5.3} dargestellt. Im Folgenden werden mit Fusionierung und Allomorphie zwei Phänomene fokussiert, die in der morphologischen Theoriebildung (etwa der Natürlichen Morphologie) vor allem als Problem, nämlich als Widerspruch zum Prinzip „one function -- one form“ behandelt werden (siehe \sectref{sec:5.1}).

Mögliche lexikalische, flexivische und syntaktische Ausdrucksverfahren werden in dieser morphologischen Theorie auf einem Kontinuum des Fusionsgrades beschrieben: vom hochgradig fusionierten lexikalischen Ausdruck über derivationelle und flexivische Verfahren hin zu den wenig bis nicht-fusionierten Verfahren der freien grammatischen oder syntaktischen Ausdrücke \citep[12]{Bybee1985b}. Anhand einer synchronen, typologischen Untersuchung der Verbalmorphologie von 50 nicht verwandten Sprachen kann \citet{Bybee1985b} zeigen, dass der Ausdruck von Flexionskategorien dem Prinzip der Relevanz folgt (neben dem Prinzip der Allgemeingültigkeit):

\begin{quote}
A meaning element is \textit{relevant} to another meaning element \textit{if the semantic content of the first directly affects or modifies the semantic content of the second}. If two meaning elements are, by their content, highly relevant to one another, then it is predicted that they may have lexical or inflectional expression, but if they are irrelevant to one another, then their combination will be restricted to syntactic expression. (\citealt[13]{Bybee1985b}, Hervorhebungen im Original, GN)
\end{quote}

Die Vorhersage des Relevanzprinzips lautet, dass mit höherer Relevanz ein höherer Grad an Fusionierung zwischen zwei Einheiten einhergeht. Auch die Reihenfolge der lexikalischen oder flexivischen Ausdrücke ist durch Relevanz gesteuert: Je relevanter eine Kategorie ist, desto näher erfolgt die Kodierung am Stamm oder dringt sogar, wie etwa beim deutschen Umlautplural, in den Stamm ein. Die Vorhersagen des Relevanzprinzips können damit auch die Dativ-Plural-Formen des Deutschen (\textit{Hund-e-n}, \textit{Mütter}{}-\textit{n}) abbilden: Der Numerusausdruck steht vor dem Kasusausdruck, da der semantische Gehalt von Numerus relevanter ist als von Kasus. Im Numerusausdruck wird kodiert, auf wie viele Entitäten referiert wird, während der Kasusausdruck nur die Rolle der Entität im syntaktischen Kontext kodiert, nicht aber das Konzept des Stammes affiziert (vgl. \citealt[34]{Bybee1985b}).

Die Relevanz einer Kategorie wird also quasi ikonisch durch den Fusionsgrad und die Nähe zum Stamm ausgedrückt. Dieses Konzept einer „diagrammatischen Ikonizität“ \citep{Bybee1985a} und die höhere Relevanz von Numerus gegenüber Kasus können erklären, warum im Althochdeutschen der Kasusumlaut im Singularparadigma getilgt wurde (Gen./Dat.Sg. ahd. \textit{henin} > \textit{hanin} ‚Hahn‘), im Pluralparadigma dagegen erhalten und als stammaffizierendes Pluralverfahren funktionalisiert ist (Abschnitte~\ref{sec:3.1.1} und \ref{sec:5.1}, vgl. \citealt[2561]{Bybee1994}, \citealt[617]{DammelEtAl2010}). Diachron erweisen sich relevante Kategorien insgesamt als stabiler als weniger relevante Kategorien; ihr Ausdruck wird tendenziell am Stamm abgebaut und die Kodierung in den syntaktischen Kontext ausgelagert (vgl. \citealt[2561]{Bybee1994}). Die fortschreitende Numerusprofilierung und gleichzeitige Nivellierung des Kasusausdrucks am Substantiv, die die diachrone Entwicklung der Nominalmorphologie des Deutschen auszeichnen und die in den Dialekten des Deutschen noch weiter vorangeschritten sind, erscheinen somit durch die Relevanz der jeweiligen Kategorie gesteuert (siehe u.\,a. auch \citealt{Nübling2005}). Zugleich können \citet{DammelGillmann2014} anhand des Sonderwegs der historisch schwachen Maskulina zeigen, dass Belebtheitseffekte die Relevanz einer Kategorie (hier von Kasus) erhöhen können. Im Frühneuhochdeutschen erfolgte die Reorganisation der historischen mask. \textit{n}{}-Stämme auf einem anthropozentrischen Kontinuum von Belebtheitsmerkmalen (vgl. \citealt{Köpcke2000a}, ausführlicher \sectref{sec:3.1.2}). Die Klasse der schwachen Maskulina umfasst seitdem Denotate mit dem Merkmal [+menschlich] oder aus dem Nahbereich des Menschen (\textit{Affe} und weitere Säugetiere). Während sich bei den historischen \textit{n}{}- und \textit{ô}{}-Feminina eine Mischdeklination mit synkretischen Singular- und distinkten Pluralformen mit Nasalsuffix herausbildet (\tabref{tab:10}), bewahren die Maskulina der historischen \textit{n}{}-Deklination das Nasalsuffix im Gen./Dat./Akk.Sg. und im Pluralparadigma. Die formale Distinktion zwischen Nom.Sg. und den obliquen Kasus ist laut \citet[209--210]{DammelGillmann2014} durch das Belebtheitsmerkmal bedingt: Belebte Referenten können -- anders unbelebte Referenten -- sowohl in der semantischen Rolle als prototypisches Agens als auch als Patiens vorkommen (siehe auch \citealt[117--118]{Köpcke2000a}). Belebtheit führt hier zu einer „Relevanzerhöhung für Kasus“ (\citealt[211]{DammelGillmann2014}) gegenüber Numerus. Eine formale Differenzierung von Numerus erscheint nur im Nominativ: Nom. \textit{Mensch} -- \textit{Menschen}, aber Gen./Dat./Akk. \textit{Menschen} -- \textit{Menschen}. Doch langfristig bewahrt auch Belebtheit nicht vor einer Schwächung des Kasusausdrucks am Substantiv, so \citet[212]{DammelGillmann2014}: Bei einigen schwachen Maskulina ist im Singular eine Tendenz zur Deflexion zu beobachten (bei gleichzeitiger Profilierung des Numerusausdrucks): Gen./Dat./Akk.Sg. \textit{Menschen} > \textit{Mensch} -- \textit{Menschen} (vgl. Abschnitte~\ref{sec:4.1} und \ref{sec:5.3.2}).

Als weiteres Indiz einer hochrelevanten Kategorie und in Abgrenzung zum Prinzip „one function -- one form“ der Natürlichen Morphologie arbeiten \citet{DammelNübling2006} Allomorphie heraus: „The higher cognitive costs to memorize a high degree of allomorphy are only worthwhile for very strong and relevant categories“ (\citealt[110]{DammelNübling2006}, vgl. \citealt[368]{Kürschner2008a}). Allomorphie weist also auf die Relevanz und Stärke einer Kategorie hin und schützt vor Deflexion. Der Verlust von Allomorphie (und die Herausbildung eines sogenannten überstabilen Markers, siehe \sectref{sec:3.2}) dient zwar der Uniformität (im Sinne der Natürlichen Morphologie), gleichzeitig führt sie zu einem Abbau von Morphologie -- teilweise bis hin zur Deflexion -- und indiziert eine geringere Relevanz und Schwäche der Kategorie (\citealt[100]{DammelNübling2006}, vgl. \citealt[368]{Kürschner2008a}). Das Relevanzprinzip (und hierin der hohe Relevanzgrad der Numeruskategorie) kann damit die auch im innergermanischen Vergleich relativ hohe Anzahl von Plural\-allomorphen im Neuhochdeutschen erklären (ausführlicher hierzu \citealt{DammelEtAl2010}).

\section{Netzwerkmodell und Schema-Ansatz}
\label{sec:5.3}
Während im Zentrum der beiden bisher vorgestellten Theorien primär die Relation von Form und Funktion stand, wird mit den im Folgenden eingeführten gebrauchsbasierten Modellen nun stärker die mentale Repräsentation von grammatischem Wissen fokussiert. Anders als regelbasierte Ansätze, die Flexion (vielmehr: flexivische Prozesse) ausgehend vom sprachlichen Input (d.\,h. der Basisform und ihren strukturellen Eigenschaften) modellieren, wird Formenbildung in \citegenua{Bybee1985b} Netzwerkmodell und \citegenua{Köpcke1993} Schema-Ansatz ausgehend vom Output, d.\,h. dem Flexionsprodukt, erfasst. Pluralformen werden gebildet, indem Sprecher eine Substantivform mit vorhandenen Flexionsformen in ihrem mentalen Lexikon abgleichen und diese Muster anwenden. Formenbildung und damit „sprachliches Verhalten“ funktioniert nach diesen kognitiven Ansätzen wie „menschliches Verhalten schlechthin“ \citep[71]{Köpcke1993}: Muster stellen die Basis für Entscheidungen dar, etwa für die Klassifikation einer Wortform als Singular oder Plural (vgl. \citealt[453]{Bybee1995}, \citealt[267]{BybeeModer1983}).

Die Grundidee von gebrauchsbasierten Modellen besteht darin, dass Sprachproduktion und -perzeption die mentale Repräsentation von Sprache und gleichermaßen die Organisation des Lexikons prägen. Sprachgebrauch formt also die mentalen Repräsentationen, grammatische Strukturen und Relationen im Lexikon entstehen aus „Kategorisierungs- und Verallgemeinerungsprozessen“ (\citealt[48]{BittnerKöpcke2016}). Zentral ist hier ein nicht-modulares Verständnis von sprachlichem Wissen: Die kognitiven Strategien der Sprachnutzer bei der Speicherung und mentalen Vernetzung von Lexemen und Flexionsformen sind eben nicht identisch mit den Analysemethoden der Sprachwissenschaft und ihrem modularen Verständnis von Grammatik (\citealt[49]{BittnerKöpcke2016}). So grenzt \citet[123--125]{Bybee1988} ihre Überlegungen zur lexikalischen Organisation der Morphologie vom regelbasierten Item-and-Process-Modell im Sinne \citegen{Hockett1954} ab. \textit{Item} und \textit{process}, d.\,h. lexikalische Repräsentation und die morphologischen Regeln, sind demnach keine separaten Komponenten der Grammatik; sie stellen vielmehr ein Kontinuum dar. \citeauthor{Bybee1985b}s (u.\,a. \citeyear{Bybee1985b, Bybee1988, Bybee1995}) Netzwerkmodell geht damit nicht von einer separaten lexikalischen und morphologischen Domäne aus, sondern von einem einzigen Lexikon, in dem morphologische und morphophonologische Muster aus der immanenten Organisation des Lexikons entstehen: „Morphological structure and organisation emerge from the connections made among related stored items“ \citep[452]{Bybee1995}. Gespeichert wird im Lexikon das gesamte Spektrum von Basisformen bis hin zu flektierten Formen, wobei die Speicherung durch zwei zentrale Mechanismen flankiert wird: „the ability to form networks among stored elements of knowledge and the ability to register the frequency of individual items and patterns“ (\citealt[125]{Bybee1988}, vgl. \citealt[427]{Bybee1995}).

Die folgenden Unterkapitel fokussieren die Grundannahmen des Netzwerkmodells (\sectref{sec:5.3.1}), die Anwendung des Schema-Konzepts auf die Pluralflexion des Deutschen (\sectref{sec:5.3.2}) und die schemabasierte Perzeption und Produktion (\sectref{sec:5.3.3}).

\subsection{\textit{Lexical Strength}, \textit{lexical connections} und Schema}
\label{sec:5.3.1}
Bedingt durch die beiden Größen \textit{lexical strength} und \textit{lexical connections} ist das Lexikon im Netzwerkmodell dynamisch, es wandelt sich infolge des Sprachgebrauchs. Während sich lexikalische Verbindungen aus ähnlichen semantischen und phonologischen Merkmalen konstituieren, erhöht sich die lexikalische Stärke eines Wortes im Lexikon mit jeder Verwendung im Sprachgebrauch, d.\,h. lexikalische Stärke und Tokenfrequenz korrelieren. Je öfter ein Wort produziert oder verarbeitet wird, desto größer ist seine lexikalische Stärke; Wörter mit geringer Tokenfrequenz haben demgegenüber auch eine geringe lexikalische Stärke.

Lexikalische Stärke führt dazu, dass der Zugriff auf die entsprechende Wortform im Lexikon schneller erfolgt, da sie separat („autonom“) im Lexikon gespeichert wird \citep[131]{Bybee1988}. Hochfrequente Wörter werden zudem schneller erworben, während schwächere, niederfrequente Wörter eher später in Abhängigkeitsrelationen zu anderen (nämlich stärkeren) Wörtern erlernt und gespeichert werden (vgl. \cites[57]{Bybee1985b}[132--134]{Bybee1988}). Lexikalische Stärke bestimmt also die Richtung morphologischer Relationen, „in the sense that the weaker words are learned and stored in terms of related stronger words“ \citep[134]{Bybee1988}. Gleichzeitig erklärt das Konzept der lexikalischen Stärke, warum sich irreguläre oder suppletive Formen eher bei hochfrequenten Wörtern halten oder überhaupt erst herausbilden: Sie sind „stark“ genug, um autonom im Lexikon gespeichert zu werden, was sie wiederum vor Wandel und Abbau schützt (vgl. \citealt[428]{Bybee1995}, siehe auch \cites[]{Bybee2006}[145--146]{Bybee2010}).

Indem \citet[135]{Bybee1988} davon ausgeht, dass morphologische und morphophonologische Regeln keine separate Komponente, sondern Teil des Lexikons sind, geht sie auch davon aus, dass diese morphologischen Regeln nicht unabhängig von den lexikalischen Einheiten im Lexikon repräsentiert sind. Anstelle von Regeln nimmt \citet{Bybee1985b, Bybee1988} wiederkehrende morphologische Muster an, die terminologisch als Schemata gefasst werden (siehe auch \citealt{BybeeModer1983}, \citealt{BybeeSlobin1982}). Dabei ist die Idee grundlegend, dass die lexikalischen Verbindungen zwischen den Wörtern im Lexikon, d.\,h. identische oder ähnliche phonologische und semantische Merkmale, eine interne morphologische Analyse bewirken: „Parallel sets of phonological and semantic connections, if they are repeated across multiple sets of words, constitute morphological relations“ (\citealt[429]{Bybee1995}, vgl. \citealt[127--129]{Bybee1988}). Schemata sind damit Generalisierungen, die aus den vorhandenen semantischen und phonologischen Verbindungen im Lexikon entstehen.

Unterschieden werden nach \citet[430]{Bybee1995} zwei Typen von Schemata „corresponding to the two ways that morphologically complex forms can relate to other forms“: produktorientierte (d.\,h. outputbasierte) Schemata und quellenorientierte (inputbasierte) Schemata. Ein quellenorientiertes Schema sind etwa die regelmäßigen Vergangenheitsformen des Englischen (Typ \textit{wait} -- \textit{waited}). Das Schema ist hier eine Abstraktion der Formenbildung ausgehend von der Basisform, d.\,h. dem Input, und der abgeleiteten (morphologisch komplexen) Form. Die Anwendung auf neue Formen erfolgt mehr oder weniger regelbasiert über die Generalisierung, d.\,h. die analogische Ausweitung, der Formenbildung auf neue Basisformen: „a source word X undergoes some process Y to produce a form Z, and the process Y is well-defined“ (\citealt[255]{BybeeModer1983}, siehe auch \citealt[285]{BybeeSlobin1982}). Im Falle der regelmäßigen Vergangenheitsformen des Englischen besteht der Prozess Y in der Addition des Suffixes /d/, etwa in \textit{park} -- \textit{parked}, \textit{valet} -- \textit{valeted}.

\begin{sloppypar}
Bei produktorientierten Schemata erfolgt die Generalisierung dagegen ausgehend von der Form des Outputs, indem die abgeleiteten Formen hinsichtlich ihrer phonologischen Merkmale ähnlich sein sollten. Für die irreguläre \textit{string}/\textit{strung}{}-Klasse beispielsweise zeigen \citet[256]{BybeeModer1983}, dass das Schema durch die Merkmale Stammvokal [ʌ] und folgender Nasal oder Velarplosiv gebildet wird. Die relevanten Verknüpfungen in der mentalen Repräsentation bestehen zwischen den Outputs, d.\,h. den Vergangenheitsformen \textit{strung}, \textit{slung}, \textit{swung}, \textit{wrung} usw., und nicht -- wie bei den quellenorientierten Schemata -- zwischen Basisform \textit{wait} und abgeleiteter Form \textit{waited} (\citealt[255]{BybeeModer1983}). Die Struktur dieser outputbasierten Schemata ist dabei um einen Prototyp organisiert, die einzelnen Mitglieder des Schemas werden durch Familienähnlichkeit (im Wittgensteinschen Sinne) zusammengehalten (vgl. \citealt[430]{Bybee1995}, \citealt[256--257]{BybeeModer1983}).
\end{sloppypar}

Je spezifischer dabei die konstituierenden Merkmale eines Schemas sind, desto weniger wahrscheinlich ist es, dass das Schema auf neue Formen ausgedehnt wird. Produktivität wird also durch ein möglichst offenes, unspezifisches phonologisches Schema begünstigt (vgl. \cites[138]{Bybee1988}[452]{Bybee1995}, siehe auch \citealt{BittnerKöpcke2016}). Einen weiteren, zentralen Faktor von Produktivität stellt die Typenfrequenz dar (vgl. \citealt[232]{Bybee1995}). Während Tokenfrequenz zu einer stärker lexikalischen (d.\,h. autonomen) Speicherung morphologisch komplexer Formen führt, korreliert Typenfrequenz mit der Stärke eines Schemas, was wiederum die Produktivität bedingt: „general, widely applicable morphological schemas [\ldots] appear to be free of the lexicon in the sense that they apply readily to new forms“ (\citealt[135]{Bybee1988}, vgl. \citealt[452]{Bybee1995}). So kann \citet[191--195]{Birkenes2014} etwa für subtraktive Pluralformen in deutschen Dialekten zeigen, dass Tokenfrequenz und damit lexikalische Stärke ein zentraler Faktor für den Erhalt irregulärer Formen sind. Gleichzeitig ist das Auftreten subtraktiver Formen in den rezenten Dialekten nicht nur durch Token-, sondern auch durch Typenfrequenz bedingt: Subtraktive Plurale der Abfolgen /-nd/ und /-Vg/ weisen eine vergleichsweise hohe Typenfrequenz auf und haben im Falle der /-nd/-Subtraktionen eine beschränkte Produktivität im Hessischen entwickelt (\citealt[197--200]{Birkenes2014}, siehe auch \citealt{Birkenes2018}).

\subsection{Schemata im Deutschen}
\label{sec:5.3.2}
\citeauthor{Köpcke1988} (u.\,a. \citeyear{Köpcke1988, Köpcke1993}) knüpft an \citeauthor{Bybee1988}s (\citeyear{Bybee1985b, Bybee1988}) Netzwerkmodell an und analysiert die deutsche Pluralallomorphie hinsichtlich vorhandener Schemata. Schema wird nach \citet[72]{Köpcke1993} definiert „als eine ausdrucksseitige Gestalt, der eine spezifische Regelhaftigkeit in dem Sinne anhaftet, daß sie ein bestimmtes Konzept, hier das der Mehrzahligkeit, wiederholt ausdrucksseitig repräsentiert“. \citeauthor{Köpcke1988}s (\citeyear{Köpcke1988, Köpcke1993}) Annahme besteht darin, dass Sprachproduktion und -perzeption vor dem Hintergrund von abstrakten Schemata erfolgt, d.\,h. Sprecher/Hörer haben in ihrem Lexikon Schemata für mögliche Singular- und Pluralformen von Substantiven gespeichert. Die Wahrnehmung konkreter Wortformen erfolgt über den Abgleich von Merkmalen (Cues), die charakteristisch für die einzelnen produktorientierten (d.\,h. outputbasierten) Schemata sind \citep[322]{Köpcke1988}. Das Suffix -\textit{er} beispielsweise ist kein eindeutiges Merkmal, da es Pluralmarker (\textit{Brett} -- \textit{Bretter}), aber als sogenanntes „Pseudosuffix“ auch Teil des Singularstammes sein kann (\textit{Fieber}, \textit{Lehrer}, vgl. \citealt[111--113]{Köpcke1993}, \citealt{KöpckePanther2016}). Erst der Definitartikel, den \citet{Köpcke1988, Köpcke1993} in seine Analyse integriert, gibt den eindeutigen Hinweis zur Singular- oder Pluralinformation: \textit{die Bretter}, \textit{das Fieber}, \textit{der Lehrer}.

\begin{sloppypar}
Schemata haben hinsichtlich der assoziierten Funktionen „eine probabilistische und mehr oder weniger prototypische Struktur“ \citep[82]{Köpcke1994}. Das heißt, mit der Funktion Plural sind mehrere ausdrucksseitige Gestalten verknüpft, aber nur die Gestalt [die + \#\_\_\_-en] (zu lesen als: der Definitartikel \textit{die} und das Suffix -\textit{en)} transportiert die Pluralbedeutung zuverlässig; [die + \#\_\_\_-en] stellt damit die prototypische Repräsentation der Funktion Plural dar \citep[72]{Köpcke1993}. Ein prototypischer Singular weist demgegenüber keine Merkmale eines Plural-Schemas auf, zu denen der Definitartikel \textit{die} oder die Pluralmarker -\textit{e}, -\textit{er}, -\textit{en} und Umlaut gehören \citep[321]{Köpcke1988}. Damit ist das Schema [die + \#\_\_\_-e] weder prototypisch für Singular (\textit{die Trage}) noch für Plural (\textit{die Tage}), sondern ambig, da die Cues \textit{die} und -\textit{e} gleichermaßen Singular und Plural signalisieren (vgl. \citealt[88]{Köpcke1993}).
\end{sloppypar}

Muster konstituieren sich dabei nicht nur aus ähnlichen flexivischen Merkmalen, sondern sie können auch durch außermorphologische Faktoren, nämlich phonologische und semantische Merkmale, bedingt sein. So zeigt \citet{Köpcke1993, Köpcke1994, Köpcke2000a}, dass die semantischen Merkmale [+belebt] und [+menschlich] historisch eine entscheidende Rolle in der Reorganisation der schwachen Maskulina gespielt haben, es findet eine „Spezialisierung“ der schwachen Deklination auf Maskulina mit dem Merkmal [+belebt] statt (\citealt[369]{Kürschner2008a}, siehe \citealt{Kürschner2021} für einen innergermanischen Vergleich sowie \sectref{sec:3.1.2}). Gleichermaßen signalisiert das Schwa-Suffix in Kombination mit Umlaut (\textit{Söhne}, \textit{Gäs\-te}) auf einem anthropozentrischen Kontinuum Nähe zum Menschen, während -\textit{e} und \mbox{[−Umlaut]} Entfernung zum Menschen signalisieren (etwa \textit{Hunde}, \textit{Lachse}, \textit{Luchse}, vgl. \citealt[84]{Köpcke1994}). Hier erklärt die Idee einer prototypischen Struktur von Schemata den diachronen Wandel von Deklinationsklassen: Je größer die Entfernung eines Lexems zum Prototyp ist, desto wahrscheinlicher ist diachron ein Wechsel der Deklinationsklasse.

Mit dieser Beobachtung lässt sich auch die rezente Tendenz zum Kasusabbau im Singular bei einigen schwachen Maskulina fassen (Gen./Dat./Akk.Sg. \textit{Menschen} > \textit{Mensch}, vgl. Abschnitte~\ref{sec:4.1} und \ref{sec:5.2}). Nach \citet{Köpcke2000a, Köpcke2002} ist für diese Klasse das semantische Merkmal der Belebtheit in Kombination mit finalem Schwa zentral (etwa \textit{Bube}, \textit{Hase}), hinzu kommen die prosodischen Merkmale Silbenanzahl und Akzent. Das finale Schwa hat dabei eine vergleichsweise hohe Validität, das heißt, es signalisiert die Zughörigkeit zur schwachen Deklination relativ zuverlässig. Allerdings weist erst die Kombination mit dem semantischen Merkmal [+menschlich] mehr oder weniger eindeutig auf schwache Flexion hin; der zwischen Schema und Deklinationsklasse „assoziierte Zusammenhang ist nur hinsichtlich des prototypischen Schemas völlig verläßlich“ (\citealt[109]{Köpcke2000a}, vgl. \citealt[104]{Köpcke2002}).\footnote{Daneben müssen weitere Schemata angenommen werden, die mit schwacher Deklination assoziiert sind, etwa Fremdwörter auf -\textit{it}, -\textit{et}, -\textit{at} (\textit{Satellit}, \textit{Komet}, \textit{Automat}, vgl. \citealt[443]{Kürschner2021}, siehe auch \citealt{Thieroff2003}).}


\begin{figure}
\begin{tikzpicture}\footnotesize
	\matrix (abbau)
	        [matrix of nodes, 
	         nodes in empty cells,
	         nodes={text width=1.66cm}]
	  {
	  	& & & & & & \\
	    Mehrsilber, Penultimabetonung, finales Schwa & Zweisilber, Trochäus, finales Schwa & Zweisilber, Trochäus, finales Schwa & Einsilber & Einsilber & Einsilber\\
	  	{[+menschl.]} & {[+menschl.]} & {[+belebt]} & {[+menschl.]} & {[+belebt]} & {[−belebt]}\\
	  	\textit{Kollege} & \textit{Bube} & \textit{Hase} & \textit{Mensch} & \textit{Bär} & \textit{Stein}\\
  	  };
    \draw[right color=white, left color=black] (abbau-1-1.west) rectangle (abbau-1-6.south east);
    \draw[dotted, -{Triangle[]}] (abbau-1-6.north east) -- (abbau-1-2.north west) node [midway, above] {Abbaurichtung};
    \node[left=0.5ex of abbau-1-1, rotate=90, anchor=south east] {prototypisch schwach};
    \node[right=0.5ex of abbau-1-6, rotate=90, anchor=north east] {prototypisch stark};
\end{tikzpicture}
\caption{Prototypikalitätsskala der schwachen Maskulina nach \citet[104]{Köpcke2002}}
\label{fig:1}
\end{figure}

Die prototypische Struktur der schwachen Maskulina lässt sich anhand dieser Merkmale auf einem Kontinuum vom Prototyp (Mehrsilber mit Penultimabetonung) über die Peripherie (Einsilber mit den Merkmalen [+belebt] und [${\pm}$mensch"-lich]) bis zum Gegenpol der starken Flexion (mit dem Merkmal [−belebt]) modellieren (\figref{fig:1}). Die Tendenz zur Deflexion betrifft indes nur die „benachbarten Merkmalsbündel“ \citep[105]{Köpcke2002} der starken Flexion: Der Abbau vollzieht sich ausgehend von der Peripherie (\textit{dem Mensch}, \textit{den Bär}) in Richtung des prototypischen Kerns. Dabei, so \citet[105]{Köpcke2002}, ist der prototypische Kern der schwachen Maskulina nicht von Deflexion bedroht und flektiert (auch diachron) stabil schwach (vgl. \citealt[111]{Köpcke2000a}, \citealt{Thieroff2003}). Um den flexivischen Sonderweg der schwachen Maskulina vollständig zu erklären, muss die Beobachtung von \citet[212]{DammelGillmann2014}, dass „auch Belebtheit langfristig nicht vor Kasusabbau schützt“, daher um die phonologisch-prosodischen Merkmale ergänzt werden, die \citet{Köpcke2000a, Köpcke2002} als konstituierend für das Schema der schwachen Maskulina ausmacht. In diesem Sinne ist finales Schwa bei den Maskulina nach \citet[119]{Köpcke2000a} ein „Agentivitätsmarker“, den die schwache, nicht aber die starke Flexion aufweist.

\subsection{Schemabasierte Perzeption und Produktion}
\label{sec:5.3.3}
Die probabilistische und prototypische Struktur eines Schemas basiert nach \citet[82]{Köpcke1993} auf der Signalstärke (\textit{cue strength}), die sich aus den perzeptuellen Charakteristika der einzelnen Cues des Schemas ergibt. Zu diesen perzeptuellen Kriterien zählen nach \citeauthor{Köpcke1993} (\citeyear[82--83]{Köpcke1993}, siehe auch \citealt[315--316]{Köpcke1988}):

\begin{itemize}
\item Die „perzipierbare, also akustisch wahrnehmbare \textit{Salienz}“ (\citealt[82]{Köpcke1993}) einer morphologischen Markierung ist höher bei segmentierbaren (additiven), wortfinalen Markern wie -\textit{e}, -\textit{(e)n}, -\textit{er} als bei stammaffizierenden Markern wie dem Umlaut.
\item \textit{Typenfrequenz} berücksichtigt die Anzahl von lexikalischen Einträgen, die ein gemeinsames Merkmal, etwa den Pluralmarker -\textit{e}, -\textit{(e)n} oder -\textit{er}, haben. Die \textit{Tokenfrequenz} umfasst die Häufigkeit, mit der ein spezifisches Merkmal im Korpus auftaucht.
\item Die \textit{Signalvalidität} (\textit{cue validity)} bezieht sich bei \citet[82]{Köpcke1993} „auf die Frequenz, mit der ein bestimmtes Merkmal in der Kategorie auftaucht, die mit der Zielkategorie kontrastiert.“\footnote{Siehe auch \citet[266]{BybeeModer1983}: „A category has high cue validity if the features associated with it frequent\-ly occur with members of the category, and rarely occur with members of other categories.“} Das Suffix \textit{{}-(e)n} hat einen hohen Erkennungswert, da nur wenige Singularformen auf Reduktionssilbe \textit{{}-(e)n} enden (z.\,B \textit{Wagen}), während -\textit{e} einen geringen Erkennungswert hat; es ist gleichermaßen Cue für die fem. Singularform wie auch für die Pluralinformation (\textit{Trage} vs. \textit{Tage}).
\item \textit{Ikonizität} heißt vor dem Hintergrund des konstruktionellen Ikonismus der Natürlichen Morphologie, dass Pluralmarkierung silbenbildend ist. Nullmarkierung und rein stammaffizierende Markierungen sind demnach weniger ikonisch als silbische Suffixe wie -\textit{e} und -\textit{er}, und auch das nicht-silbische Suffix -\textit{n} in \textit{Bauern} ist schlechter perzipierbar als das silbische Suffix -\textit{en} (\textit{Bäuerinnen}).
\end{itemize}

Die perzeptuelle Wirksamkeit der Merkmale, die in einer Sprache mit den Funktionen Singular und Plural assoziiert sind, lässt sich anhand von \textit{cue strength} und den einzelnen Kriterien beschreiben und abstufen. Für das Deutsche modelliert \citet{Köpcke1988, Köpcke1993} ein Kontinuum zwischen den Polen eines idealen Singular- und eines idealen Plural-Schemas, das sich auf die Cues Silbenanzahl, Auslaut bzw. (Pseudo-)Suffix sowie Definitartikel bezieht (\figref{fig:2}).


\begin{figure}
\begin{tikzpicture}\footnotesize
	\matrix (sgpl)
	        [matrix of nodes, 
	         nodes in empty cells,
	         nodes={text width=2.25cm},
	         row 1 column 5/.style={right}]
	  {
	  	Singular & & & &Plural\\
	  	& & & & \\[0.5ex]
	    einsilbig & mehrsilbig & mehrsilbig & mehrsilbig & mehrsilbig\\
	    Plosiv-Ablaut & Reduktionssilbe -\textit{er} & Reduktionssilbe -\textit{e} & Reduktionssilbe -\textit{er} & Reduktionssilbe -\textit{e(n)}\\
	    \textit{der/das} & \textit{der/das} & \textit{die} & \textit{die} & \textit{die}\\
  	  };
    \draw[left color=black,right color=black,middle color=white]
    (sgpl-2-1.north west) rectangle (sgpl-2-5.south east);
\end{tikzpicture}
\caption{Kontinuum der Singular- und Plural-Schemata nach \textcites[332]{Köpcke1988}[88]{Köpcke1993}}
\label{fig:2}
\end{figure}

In \citegen[71]{Köpcke1993} Modell nehmen Schemata eine quasi vermittelnde Stellung zwischen regelbasiertem Item-and-Process-Modell und mor"-pho"-lo"-gi"-schen Modellen ein, die Flexive nicht als unabhängige sprachliche Zeichen, sondern Flexionsformen als Teil der lexikalischen Repräsentation eines Lexems im mentalen Lexikon erfassen. Beide Idealtypen werden nach \citet[98--100]{Köpcke1993} den psycholinguistischen Realitäten der Sprachproduktion und -perzeption nicht gerecht. In IP-Modellen sind Basisformen und Regeln im mentalen Lexikon gespeichert, was bei regulären Pluralformen ökonomisch, bei irregulären, aber hochfrequenten Formen dagegen maximal unökonomisch ist. Modelle, die eine lexikalische Speicherung morphologischer Formen annehmen, sind wiederum bei den hochfrequenten, irregulären (im Extremfall suppletiven) Formen ökonomisch, nicht aber bei regulären Ausdrücken. Schemata operieren „zwischen“ diesen beiden Idealtypen \citep[100]{Köpcke1993}.

Anders als \citeua{Bybee1985b} integriert \citet{Köpcke1993} IP-Regeln und Schemata in sein Grammatikmodell, indem er -- mit Blick auf die Kompetenz der Sprachnutzer -- von einer Koexistenz der analytischen Fähigkeiten ausgeht, die eine Segmentierung komplexer Ausdrücke ermöglichen, und von Kategorisierungen, die „auf holistische Repräsentationen“ von Wörtern und Wortformen im Lexikon „zurückzuführen sind“ \citep[215]{Köpcke1993}. In neueren Arbeiten wird die analytische Kompetenz der Sprachnutzer zudem berücksichtigt, indem sogenannte Schemata zweiter Ordnung (auch Paar-Schemata) angenommen werden (\citealt{Nesset2008}, \citealt{KöpckeEtAl2021}, \citealt{KöpckeWecker2017}, \citealt{Ronneberger-Sibold2021}, \citealt{Wecker2016}). Das heißt, im mentalen Lexikon finden sich nicht nur isolierte Schemata („Schemata erster Ordnung“), z.\,B. das Singular-Schema [die + \#\_\_\_-e], sondern das Plural-Schema [die + \#\_\_\_-en] wird als Schema zweiter Ordnung assoziiert, die paradigmatische Beziehung ist als Teil der mentalen Repräsentation abgespeichert. Anders als das outputbasierte Schema erster Ordnung basiert das Schema zweiter Ordnung auf Generalisierungen von verschiedenen Wortformen und ihren paradigmatischen Beziehungen, es ist also inputbasiert und verbindet -- als eine Art „‚Super‘-Schema“ (\citealt[85]{KöpckeWecker2017}) -- zwei Schemata erster Ordnung miteinander (siehe auch \sectref{sec:10.3.1}).

Dass die Annahme von Schemata und auch die probabilistische Interpretation der perzeptuellen Hinweisreize plausibel sind, zeigen \citet{DomahsEtAl2017} in einer Studie mit einer Patientin mit Progressiver Aphasie, deren lexikalisches Wissen, nicht aber das sprachliche Regelwissen gestört ist. Ausgehend von der Hypothese, dass schemabasiertes Wissen mehr oder weniger unabhängig von lexikalischem Wissen existiert, erweist sich das Leistungsmuster der Patientin bei Numerusentscheidungsaufgaben als „konsistent mit der Annahme einer schemabasierten Antwortstrategie“ (\citealt[227]{DomahsEtAl2017}).\footnote{Zum einen erwiesen sich die Art des Definitartikels, Silbenanzahl sowie das zusätzlich getestete Merkmal eines trochäischen Betonungsmusters als statistisch signifikante Hinweisgeber für eine Pluralentscheidung. Zum anderen konnte keiner dieser Cues isoliert das Antwortverhalten erklären; die Hinweisreize bewirkten in Kombination eine probabilistische Verarbeitung der Numerusinformation (vgl. \citealt[228]{DomahsEtAl2017}).} Bemerkenswert ist hier das Fazit, dass „die besonderen Stärken“ (\citealt[229]{DomahsEtAl2017}) des Schema-Ansatzes eher im Bereich der Perzeption von morphologischen Markierungen gesehen werden, während \citet{Köpcke1988, Köpcke2000b} in Kunstwort-Experimenten stärker die schemabasierte Produktion im Blick hatte. Dass es bei der Produktion Variation sowohl in der horizontalen (arealen) als auch in der vertikalen Dimension gibt, zeigt \citet{Ronneberger-Sibold2021}, die \citegen{Köpcke2000b} Kunstwort-Experiment mit Probanden aus dem bair. Dialektraum wiederholt hat. Im Fokus des \sectref{sec:10.3} steht indes der Aspekt der Perzeption von morphologischen Markierungen: Welche Cues werden von Sprechern genutzt, um die morphologische Information zu kodieren und wie verhalten sich diese Cues mit Blick auf die prototypische Struktur von Schemata? Und welche Hinweisreize sind mehr oder weniger informativ mit Blick auf die morphologische Funktion und wie gut sind sie perzipierbar? Während \citet{Köpcke1993} dies für die standarddeutsche Pluralallomorphie modelliert hat, steht dies für dialektale Flexionssysteme noch aus.
