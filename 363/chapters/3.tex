\chapter{Deklinationsklassen}
\label{chap:3}
Deklinationsklassen sind (wie auch der Überbegriff Flexionsklasse) Einheiten der Flexionsmorphologie, die -- anders als beispielsweise Numerus und Kasus -- keine Flexionskategorie, sondern ein Klassifizierungsverfahren bezeichnen (vgl. \citealt[21]{Kürschner2008a}). Substantive, die sich dieselben Flexive teilen, die sich im Paradigma also formal ähnlich verhalten, werden zu Deklinationsklassen zusammengefasst. Flexionsklassen sind gewissermaßen „versteckte Kategorien“ \citep[28]{Dammel2011}, da sie sich in der formalen Varianz der Allomorphe manifestieren. Einteilungen in Flexionsklassen „strukturieren Allomorphie“ \citep[20]{Dammel2011}. Deklinationsklassen werden nach dieser Definition bottom-up ausgehend von den Klassenmitgliedern zugewiesen: Substantive, die flexionsmorphologische Kategorien einheitlich realisieren, werden zu Klassen zusammengefasst. In Top-down-Definitionen werden Deklinationsklassen als „Etikett“ \citep[23]{Dammel2011}, als „some sort of flag“ \citep[65]{Aronoff1994} angesehen, das Lexemen im Lexikoneintrag zugewiesen ist.

Offen bleibt indes auch bei einer Bottom-up-Definition von Deklinationsklasse, wie offen oder restriktiv das Merkmal der „Ähnlichkeit“ im flexivischen Verhalten gefasst wird. So bietet beispielsweise \citet[323]{Carstairs-McCarthy1998} eine relativ enge Definition: „An inflectional class is a set of lexemes which share a paradigm and whose word forms are alike in respect of the realization of the morphosyntactic properties in every cell.“\footnote{Eine ähnlich restriktive Definition findet sich bei \citet[23]{Aronoff1994}: „An inflectional class is a set of lexemes whose members each select the same set of inflectional realizations.“}  Auch \citet[66--67]{Wurzel1984} nennt als Kriterium von Flexionsklassen, dass morphologische Information „in \so{formal einheitlicher Weise symbolisiert} wird“ (Hervorhebung im Original, GN). Der Nachteil eher restriktiver Definitionen von Deklinationsklassen besteht nach \citet[22]{Kürschner2008a} darin, dass Variation innerhalb und Zusammenhänge zwischen Deklinationsklassen weniger gut abgebildet werden können, wie folgendes Beispiel illustriert: Die sogenannten schwachen Maskulina\footnote{Die in der Tradition nach Jakob Grimm gebräuchlichen Termini „schwache“ und „starke“ Deklination sind arbiträr zu verstehen und „diachron motiviert“ (\citealt[1691]{Solms2004}, vgl. \citealt[283]{Nübling2008}, \citealt[79]{KleinEtAl2018}). Während in der schwachen Flexion Gen.Sg. und die Pluralformen mit -\textit{en} gebildet werden, werden in der starken Flexion weder Gen.Sg. noch die Pluralformen mit -\textit{en} markiert. In der sogenannten gemischten Deklination weist der Plural -\textit{en} auf, nicht aber Gen.Sg.} weisen in den obliquen Kasus im Singular und im gesamten Pluralparadigma das Flexiv mit \textit{{}-(e)n} auf (\textit{Löwe-n}, \textit{Bär-en}). Fremdwörter wie \textit{Hydrant} oder \textit{Automat} nutzen dieses Markierungsverfahren auch, tendieren aber zu einer synkretischen Form im Nom./Akk./Dat.Sg. (\textit{der/den/dem Automat}), nur Gen.Sg. weist eine distinkte Form auf (\textit{des Automaten}, vgl. \citealt{Thieroff2003}). Eine ähnliche Entwicklung vollzieht sich bei nativen schwachen Maskulina (\textit{den Mensch}, \textit{dem Bär}, vgl. \citealt[215]{DammelGillmann2014}). Sind hier nun mehrere Deklinationsklassen anzusetzen? Legt man \citegen{Carstairs-McCarthy1998} Definition zugrunde, dass Wortformen einer Deklinationsklasse „in jeder Zelle“ des Paradigmas formal ähnlich sind, müssen hier verschiedene Deklinationsklassen angesetzt werden.   Eine offenere Definition von Deklinationsklasse bietet \citet[140]{Enger1998}: „An inflection class is defined as a group of words that inflect in the same or similar fashion.“ Diese Definition ermöglicht es, auch variierende Paradigmen zu einer Deklinationsklasse zusammenzufassen, da offengelassen wird, wie „in the same or similar fashion“ definiert und in einem zweiten Schritt operationalisiert wird. Die Entscheidung, welche Merkmale klassenkonstituierend sind und welche nicht, ist in erster Linie abhängig von Erkenntnisinteresse, methodischen Faktoren oder auch der Forschungstradition und wird -- mit Blick auf dialektale Deklinationssysteme -- in \sectref{sec:8.1} erneut aufgegriffen (vgl. \citealt[28]{Dammel2011}, \citealt[100]{Werner1969}).

Im Zentrum des folgenden Kapitels stehen Deklinationsklassen zunächst als „inhärente Kategorisierungen für Substantive“, deren konkrete Definitionsbereiche durchaus unscharfe Ränder aufweisen \citep[23]{Kürschner2008a}. \sectref{sec:3.1} fokussiert zunächst die diachrone Perspektive mit dem Ziel, sich über den Wandel von Deklinationsklassen in der deutschen Sprachgeschichte ihrem Status im Deutschen und seinen Dialekten anzunähern (vgl. \citealt[89]{Wurzel1986}). In \sectref{sec:3.2} wird auf funktionale Aspekte eingegangen, die die „notorische Persistenz“ \citep[282]{Nübling2008} von Deklinationsklassen und die ausgeprägte Allomorphie im Nominalsystem des Deutschen erklären können.

\section[Deklinationsklassenwandel]{Deklinationsklassenwandel: Vom Indogermanischen zum Neuhochdeutschen}\label{sec:3.1}
Der kurze Überblick zur Diachronie von Deklinationsklassen orientiert sich an folgenden Leitfragen (vgl. \citealt[294--295]{Nübling2008}):

\begin{itemize}
\item Wie manifestiert sich Deklinationsklasse? Inwiefern ist die Markierung von Deklinationsklassen transparent (overt) oder verdeckt (kovert) in den verschiedenen Sprachstufen?
\item Deklinationsklassen werden im Neuhochdeutschen durch Pluralallomorphie in Kombination mit den Kasusmarkern des Gen.Sg. sichtbar; sie basieren auf diesen Numerus- und Kasusexponenten. Inwiefern gibt es diachron einen Wandel der formalen Exponenten von Deklinationsklassen?
\item Ist diachron eine Reduktion oder ein Ausbau des Deklinationsklasseninventars zu beobachten? Wie verhalten sich Types und Tokens von Deklinationsklassen zueinander?
\item Inwiefern findet Deklinationsklassenwechsel statt, d.\,h. ein Übergang von Mitgliedern einer Klasse in eine andere?
\end{itemize}

\sectref{sec:3.1.1} ist dabei stärker typologisch ausgerichtet, indem die Manifestation von Deklinationsklassen in ihrer historischen Entwicklung in den Blick genommen wird. Im Zentrum von \sectref{sec:3.1.2} steht der Wandel des Deklinationsklasseninventars in den einzelnen Sprachstufen des Deutschen.

\subsection{Struktureller Wandel des Deklinationsklassensystems}
\label{sec:3.1.1}
Das indogermanische Substantiv hatte eine dreigliedrige Struktur. Dem Stamm, bestehend aus lexikalischer Wurzel und stammbildendem Suffix, folgt ein fusioniertes Kasus/Numerus-Suffix (K/N in \tabref{tab:4}). Das stammbildende Suffix tritt im gesamten Paradigma auf und drückt semantische Distinktionen aus, darin ist es „mit Derivationssuffixen im heutigen Deutschen vergleichbar“ (\citealt[45]{KürschnerDammel2013}, vgl. \citealt[76]{Fortson2007}, \citealt[72]{Kürschner2008a}, \citealt[325]{Meier-Brügger2010}, \citealt[61--62]{Ramat1981}). Die flexivische Kodierung erfolgt durch das segmentierbare Kasus/Numerus-Suffix, z.\,B. -\textit{s} als Träger der Information des Nom.Sg. (idg. *\textit{dhogh-o-s} ‚Tag‘) oder -\textit{m} für Akk.Sg. (*\textit{dhogh-o-m}).

Das Deklinationssystem des Ur-Indogermanischen ist durch Akzent und Ablaut als den „beiden relevanten klassenbildenden Phänomenen“ \citep[329]{Meier-Brügger2010} geprägt (vgl. \citealt[73--74 und 107--110]{Fortson2007}; siehe auch den forschungsgeschichtlichen Überblick in \citealt[336--348]{Meier-Brügger2010}).\footnote{\citet[340]{Meier-Brügger2010} zufolge ist das Akzentmuster primär, das Ablautmuster ist indes „dessen unmittelbare Konsequenz“ und durch den Akzent gesteuert (vgl. ebd.: 336).}  Im Germanischen tritt ein tiefgreifender struktureller Wandel ein. Bedingt durch phonologischen Wandel (Übergang vom freien Wortakzent im Indogermanischen zum Initialakzent im Germanischen) wird das wortfinale Kasus/Numerus-Suffix reduziert, stammbildendes Derivationssuffix und Kasus/Numerus-Suffix verschmelzen zu einer Flexionsendung. Mit diesem Wandel von einer transparenten dreigliedrigen Struktur mit separatem stammbildendem Suffix zu einer zweigliedrigen Struktur aus Stamm und fusioniertem Flexiv ist die formale Basis der semantischen Distinktionen des idg. Klassensystems geschwunden (\citealt[45]{KürschnerDammel2013}). Gleichzeitig erfährt das Flexivinventar infolge der Fusionierung und lautlichen Veränderungen insgesamt eine Ausdifferenzierung (vgl. \citealt[47]{Nübling2005}, \citealt[109--110]{Werner1969}). Wenn Darstellungen die germanischen Deklinationsklassen mit den idg. (oder germ.) stammbildenden Suffixen bezeichnen, so hat dies laut \citet[73]{Kürschner2008a} daher „nur noch arbiträren Etikettencharakter“.

\begin{table}
\begin{tabular}{llll}
\lsptoprule
Nom.Sg. & idg. *\textit{dhogh-o-s} & > & germ. *\textit{dag-az}\\
Nom.Pl. & idg. *\textit{dhogh-o-es} & > & germ. *\textit{dag-os}\\
Akk.Sg. & idg. *\textit{dhogh-o-m} & > & germ. *\textit{dag-a\textsuperscript{n}}\\\tablevspace
& \begin{tikzpicture}[pin distance=2ex]\small
	\node at (0,0) [draw, pin={above:Wurzel}, minimum width=1.25cm, minimum height=0.75cm] (wurzel) {};
	\node [right of=wurzel, anchor=west, node distance=0.625cm, draw, minimum size=0.75cm, pin={above:stb. S.}, fill=black!50] (stbS) {};
	\node [right of=stbS, anchor=west, node distance=0.375cm, draw, pin={above:K/N}, minimum width=2cm, minimum height=0.75cm, pattern=grid] (KN) {};	
	\draw[decorate,decoration={brace,mirror,raise=.5ex}] (wurzel.south west)	-- (stbS.south east) node [midway,below=.5ex] (stamm) {Stamm};
	\path let \p1=(KN),\p2=(stamm) in node at (\x1,\y2) {Flexiv};
\end{tikzpicture} &  & \begin{tikzpicture}[pin distance=2ex]\small
	\node at (0,0) [draw, pin={above:Wurzel}, minimum width=2cm, minimum height=0.75cm] (wurzel) {};
	\node [right of=wurzel, anchor=west, node distance=1cm, draw, pin={above:K/N}, minimum width=2cm, minimum height=0.75cm, pattern=grid] (KN) {};
	\node [below=0.5ex of wurzel] {Stamm};
	\node [below=0.5ex of KN] {Flexiv};
\end{tikzpicture}\\
\lspbottomrule
\end{tabular}
\caption{Dreigliederige Struktur im Indogermanischen und zweigliedrige Substantivstruktur im Germanischen (nach \citealt[47]{Nübling2005} und \citealt[107]{Werner1969})}
\label{tab:4}
\end{table}

Im Germanischen sind die früher separaten stammbildenden Suffixe anders als im Indogermanischen nicht in allen Flexionsformen eines Paradigmas erhalten und werden nicht uniform realisiert. Gleichzeitig entfällt die transparente Motivierung der Klassenbildung, die im Indogermanischen durch das stammbildende Suffix als formale Markierung einer semantischen Differenzierung gegeben war. In der Folge bleibt die formale Varianz, nicht aber die semantische Motiviertheit der Deklinationsklassen erhalten (\tabref{tab:5}, vgl. \citealt[73]{Kürschner2008a}, \citealt[61--67]{Ramat1981}).

\begin{table}
\caption{System der Deklinationsklassen im Germanischen mit Anmerkungen zu Klassengröße, Produktivität und möglichen Ausgleichsprozessen (nach  \citealt{KraheMeid1969, Kürschner2008a, Nübling2008, Ramat1981})}
\label{tab:5}
\begin{subtable}{\textwidth}
\caption{Vokalische Stämme}
\small
\begin{tabularx}{\textwidth}{llllQQ}
\lsptoprule
 {DK} & \multicolumn{3}{c}{{Genus}} & {Beispiele} & {Anmerkungen}\\
\cmidrule(lr){2-4}
              & {F} & {M} & {N} & {Nom.Sg. -- Nom.Pl.} & \\
\midrule
 \textit{a}-Klasse & {}$-$ & + & + & m.*\textit{dagaz} -- *\textit{dagos} ‚Tag‘ & \multirow[t]{2}{=}{große, produktive Klasse}\\
  (idg. \textit{o}{}-Kl.)&  &  &  & n. *\textit{wurða} -- * \textit{wurðo} ‚Wolf\textit{‘} & \\
\tablevspace
 \makecell[tl]{\textit{ja}-Klasse\\(idg. \textit{yo}{}-Kl.)} & {}$-$ & + & + & m.*\textit{harjaz} -- *\textit{harjoz} ‚Heer‘ & \multirow[t]{4}{=}{Subklassen der \textit{a}{}-Klasse mit weniger Mitgliedern, aber ähnlichem Deklinationsmuster}\\
   &  &  &  & n. *\textit{kunja\textsuperscript{n}} -- *\textit{kunjo} ‚Stamm, Geschlecht‘ & \\
\tablevspace
 \makecell[tl]{\textit{wa}-Klasse\\(idg. \textit{wo}{}-Kl.)} & {}$-$ & + & + & m.*\textit{snaiwaz} -- *\textit{snaiwoz/os} ‚Schnee‘ & \\
  &  &  &  & n. *\textit{trewa\textsuperscript{n}} -- *\textit{trewo} ‚Baum‘ & \\
\tablevspace
 \makecell[tl]{\textit{ō}-Klasse\\(idg. \textit{ā}-Kl.)} & + & {}$-$ & {}$-$ & *\textit{wullo} -- *\textit{wulloz} ‚Wolle‘ & große, produktive Klasse\\
\tablevspace
 \makecell[tl]{\textit{jō}-Klasse\\(idg. \textit{ā}-Kl.)} & + & $-$ & $-$ & *\textit{siƀjō} -- *\textit{siƀjōz} ‚Sippe‘ & \multirow[t]{3}{=}{Subklassen der \textit{ō}{}-Klasse mit weniger Mitgliedern, aber ähnlichem Deklinationsmuster}\\
 \tablevspace
 \makecell[tl]{\textit{wō}-Klasse\\(idg. \textit{ā}-Kl.)} & + & {}$-$ & {}$-$ & *\textit{trewwō} -- *\textit{trewwōz} ‚Treue‘ & \\
 & & & & &\\
\tablevspace
 \multirow[t]{3}{*}{\makecell[tl]{\textit{i}-Klasse\\(idg. \textit{i}{}-Kl.)}} & + & + & (+) & f. *\textit{dēðiz} -- \textit{ðedijiz} ‚Tag‘ & \multirow[t]{3}{=}{kleinere Klasse (fast keine Neutra mehr),\footnote{Die „kärglichen Spuren“ \citep[71]{Ramat1981} der Neutra sind nur im Angelsächsischen und Altsächsischen erhalten; im Deutschen sind nur Maskulina und Feminina belegt.} Mischung mit anderen Klassen: Mask. mit \textit{a}{}-Klasse, Fem. mit \textit{ō}{}-Klasse}\\
  &  &  &  & m. *\textit{gastiz} -- *\textit{gastijiz} ‚Gast‘ & \\
  &  &  &  & n. *\textit{mari} -- (-) ‚Meer‘ & \\
  &  &  &  & &\\
  &  &  &  & &\\
\tablevspace
 \multirow[t]{3}{*}{\makecell[tl]{\textit{u}-Klasse\\(idg. \textit{u}{}-Kl.)}} & + & + & {}$-$ & f. *\textit{handuz} -- andere Kl. ‚Hand‘ & \multirow[t]{3}{=}{Feminina bereits ausgeschieden; kleine Klasse, im Abbau befindlich}\\
  &  &  &  & m. *\textit{sunuz} -- *\textit{suniwez} ‚Sohn‘ & \\
  &  &  &  & n. *\textit{fehu} -- (kein Pl.) ‚Vieh‘ & \\
  \lspbottomrule
\end{tabularx}
\end{subtable}
\end{table}

\begin{table}
\ContinuedFloat
\caption{Dreigliederige Struktur im Indogermanischen und zweigliedrige Substantivstruktur im Germanischen (nach \citealt[47]{Nübling2005} und \citealt[107]{Werner1969})}
\begin{subtable}{\textwidth}
\caption{Konsonantische Stämme}
\small
\begin{tabularx}{\textwidth}{llllQQ}
\lsptoprule
 {DK} & \multicolumn{3}{c}{{Genus}} & {Beispiele} & {Anmerkungen}\\
\cmidrule(lr){2-4}
              & {F} & {M} & {N} & {Nom.Sg. -- Nom.Pl.} & \\
\midrule
\multirow[t]{3}{*}{\makecell[tl]{\textit{n}-Klasse\\„schwache“\\Deklination}} & + & + & + & f. *\textit{tungo} -- *\textit{tungon(e)z} ‚Zunge‘ & \multirow[t]{3}{=}{große, hochproduktive Klasse, insb. Konkreta und Belebtes}\\
  &  &  &  & m. *\textit{mēnan} -- *\textit{mēnan(e)z} ‚Mond‘ & \\
  &  &  &  & n. *\textit{herton} -- *\textit{hertōno} Herz‘ & \\
  \tablevspace
 \textit{ī}-Klasse & + & {}$-$ & {}$-$ & *\textit{Þurstī\textsuperscript{n}} -- *\textit{Þurstīnez} ‚Durst‘ & neue Subklasse der \textit{n}{}-Klasse\\
 \tablevspace
 \textit{r}-Klasse  & + & + & {}$-$ & f. *\textit{dūhter} -- \textit{dūhter(i)z} ‚Tochter‘ & \multirow[t]{2}{=}{Kleinstklasse mit Verwandschafts-bezeichnungen; unproduktiv}\\
  &  &  &  & m. *\textit{brōþar} -- \textit{brōþ(e)riz} ‚Bruder‘ & \\
  \tablevspace
 \textit{nd}-Stämme  & {}$-$ & + & {}$-$ & *\textit{frijōnd} -- *\textit{frijōnd(e)z} ‚Freund‘  & kleine Klasse substantivierter Partizipien\\
 \tablevspace
 Wurzelstämme  & + & {}$-$ & {}$-$ & f. *\textit{burgs} -- *\textit{burgiz ‚}Burg‘ &\multirow[t]{2}{=}{kleine Klasse; im Germ. fast nur noch Feminina}\\
  &  &  &  & m. *\textit{mannz} -- *\textit{manniz} ‚Mann‘ & \\
\lspbottomrule
\end{tabularx}
\end{subtable}
\end{table}

Infolge der Reanalyse des stammbildenden Suffixes als Teil des Flexivs und der Verschiebung der Morphemgrenzen fallen Wurzel und Stamm formal zusammen. Der typologische Wandel von der Stamm- zur Grundformflexion ist im Germanischen indes noch nicht vollzogen, sowohl das Indogermanische als auch das Germanische sind stammflektierende Sprachen. Das Kasus/Numerus-Allomorph wird nicht -- wie beim grundformflektierenden Verfahren -- additiv an den Stamm gefügt, sondern ein Allomorph wird durch ein anderes ersetzt (vgl. \citealt[295]{Nübling2008}, \citealt[106]{Werner1969}). Im Germanischen besteht somit noch keine Opposition zwischen unmarkiertem Nom.Sg. und markiertem Nom.Pl., sondern auch das Singularparadigma weist (in stärkerem Maße als spätere Sprachstufen des Deutschen) distinkte Flexive auf (vgl. \citealt[76]{Kürschner2008a}). Der typologische Wandel von der Stamm- zur Grundformflexion setzt im Althochdeutschen ein. Mit Ausnahme der \textit{ō}{}-Deklination (Nom./Akk./Gen.Sg. \textit{gëba}, Dat.Sg. \textit{gëbu} ‚Gabe‘) weisen die übrigen starken Deklinationsklassen bereits eine unsuffigierte Grundform im Nom.Sg. auf (\tabref{tab:6}, vgl. \citealt[81]{Kürschner2008a}, anders \citealt{Harnisch1994b, Harnisch2001}).

\begin{table}
\small
\begin{tabularx}{\textwidth}{lllllll}
\lsptoprule

\multicolumn{2}{l}{germ. DK} & {mask. \textit{a}} & {neutr. \textit{a}} & {fem. \textit{ō}} & {mask. \textit{i}} & {fem. \textit{i}}\\
\multicolumn{2}{c}{} & \textit{tag}  & \textit{wort}  & \textit{gëba}  & \textit{gast}  & \textit{anst} \\
& & ‚Tag‘ & ‚Wort‘ &  ‚Gabe‘ &  ‚Gast‘ &  ‚Gunst‘\\
\midrule
{Sg.} & Nom./Akk. & \textit{tag} & \textit{wort} & \textit{gëba} & \textit{gast} & \textit{anst}\\
& Gen. & \textit{tages (-as)} & \textit{wortes} & \textit{gëba} & \textit{gastes} & \textit{ensti}\\
& Dat. & \textit{tage (-a)} & \textit{worte (-a)} & \textit{gëbu} & \textit{gaste} & \textit{ensti}\\
& Instr. & \textit{tagu, -o} & \textit{wortu} & \textit{{}--} & \textit{gestiu}>\textit{gastiu} & \textit{{}--}\\
\tablevspace
{Pl.} & Nom./Akk. & \textit{taga} & \textit{wort} & \textit{gëba} & \textit{gesti} & \textit{ensti}\\
& Gen. & \textit{tago} & \textit{worto} & \textit{gëbōno} & \textit{gesteo/-io>gesto} & \textit{ensto}\\
& Dat. & \textit{tagum} & \textit{wortum} & \textit{gëbōm} & \textit{gestim} & \textit{enstim}\\
\lspbottomrule
\end{tabularx}
\caption{Starke Deklinationsklassen des Althochdeutschen (nach \citealt{BrauneHeidermanns2018})}
\label{tab:6}
\end{table}

Kasus und Numerus werden in dieser Phase weiterhin durch ein Portmanteau-Morphem, d.\,h. durch ein fusioniertes Kasus/Numerus-Suffix, markiert. Im Althochdeutschen setzt dann die Separierung des Kasus/Numerus-Ausdrucks ein, Numerusprofilierung und Kasusnivellierung als die beiden zentralen Tendenzen des nominalmorphologischen Wandels im Deutschen werden eingeleitet (vgl. \citealt[§192a]{BrauneHeidermanns2018}). Entscheidend bei dieser Entwicklung ist ein phonologischer Wandel, der den morphologischen Wandel anstößt: der \textit{i}-Umlaut. Im Althochdeutschen führt ein zunächst rein phonologischer Prozess der Vokalassimilation zu innerparadigmatischen Alternationen zwischen palatalen und nicht-palatalen Vokalen. Ein /i/ oder /j/ in der Folgesilbe lösen eine regressive assimilatorische Palatalisierung eines hinteren Vokals aus, z.\,B. ahd. \textit{apful} ‚Apfel‘ -- \textit{epfili} ‚Äpfel‘, ahd. \textit{tag} ‚Tag‘ -- \textit{tegelih} ‚täglich‘ (vgl. \citealt[§51]{BrauneHeidermanns2018}, \citealt[229]{Wiese1987}). Die erste Phase des \textit{i}{}-Umlauts betrifft vor allem die mask. und fem. \textit{i}{}-Klassen und die neutr. \textit{iz/az}{}-Klasse, die jeweils \textit{i}{}-haltige Kasus/Numerus-Suffixe aufweisen. Strukturell sind die palatalisierten Umlautprodukte zunächst Allophone der velaren Vokale, bevor die Phonemisierung erfolgt und in einer Erweiterung des Vokalphoneminventars resultiert (vgl. \citealt[195]{Ronneberger-Sibold1990}, \citealt[120--123]{Schulze2016}, \citealt[306]{Sonderegger1979}). Zwei Prozesse, die auf die beiden ahd. Umlautphasen\footnote{Beim sogenannten Primär\-umlaut im Althochdeutschen wird nur die Palatalisierung von /a/ auch systematisch verschriftlicht. Der sogenannte Sekundär\-umlaut, das heißt die Verschriftlichung der übrigen palatalisierten Vokale, die zwar schon im Althochdeutschen eingetreten ist, wird meist erst auf das Mittelhochdeutsche datiert, vgl. ahd. \textit{hūsir} [hyːsir] > mhd. \textit{hiuser} [hyːsər]. Aktueller Forschungsstand ist nach \citet[16]{Nübling2013}, dass Primär- und Sekundär\-umlaut gleichzeitig stattgefunden haben.} folgen, leiten den Übergang vom phonologischen zum morphologischen Umlaut ein: der phonologische Prozess der Vokalreduktion in den Endsilben und der morphologische Prozess des innerparadigmatischen Ausgleichs.

Im Übergang vom Alt- zum Mittelhochdeutschen werden die Vollvokale in den Endsilben zu Schwa reduziert, das umlautauslösende /i/ verschwindet aus den Suffixen und damit auch der Auslöser eines rein phonologischen Umlauts: ahd. \textit{gesti} > \textit{geste} ‚Gäste‘. \citet[235]{Wiese1987} sieht in der Endsilbenabschwächung die Grundlage zur Reanalyse der Wortformen. Nach dem Schwund des /i/ als Auslöser der Vokalharmonie wird das „Harmoniemerkmal“ als Teil des Stammes reanalysiert und der Umlaut mit einem morphologischen Verfahren wie z.\,B. Pluralmarkierung assoziiert (siehe auch \citealt[68--69]{Wurzel1982}). \citet[60]{Wurzel1982} hingegen sieht in den im Althochdeutschen einsetzenden innerparadigmatischen Ausgleichsprozessen den Beginn einer Morphologisierung des Umlauts. Lautgesetzlich entstandene Umlaute scheiden aus dem Singularparadigma aus, obwohl das /i/ in der Folgesilbe und damit die phonologische Umgebung, die den Umlaut auslöst, erhalten bleibt, vgl. Gen./Dat.Sg. \textit{henin} > \textit{hanin} ‚Hahn‘ (\tabref{tab:7}, vgl. \citealt[71--72]{Dammel2018}, \citealt[308--309]{Sonderegger1979}).\footnote{Die Singular-Suffix-Variante -\textit{in} findet sich (neben -\textit{en}) im Oberdeutschen, sowie im Ost- und Rheinfränkischen (vgl. \citealt[1179]{Sonderegger2000}). Das \textit{i}{}-haltige Suffix bewirkte im Singular den Umlaut, der später jedoch wieder eliminiert wurde; bereits im 9. Jhd. sind die nichtumgelauteten Singularformen die Regel (vgl. \citealt[§221]{BrauneHeidermanns2018}).} Das Deklinationsparadigma von ahd. \textit{hano} ‚Hahn‘ aus der mask. \textit{n}{}-Klasse zeigt hier zudem, dass der Pluralumlaut keine Voraussetzung war, um den Kasusumlaut im Gen./Dat.Sg. zu tilgen, sondern dass die Tilgung durch fehlende Uniformität im Bereich des Stammvokalismus im Singularparadigma bedingt war (vgl. \citealt[49--50]{Nübling2005}). Im Ergebnis stehen sich umlautlose Singularformen und (bis hierhin rein lautgesetzlich entstandene) Pluralumlaute im Paradigma gegenüber. Im Übergang vom Mittelhochdeutschen zum Frühneuhochdeutschen folgt die analogische Ausweitung des Pluralumlauts; der Übergang einer phonologisch bedingten Alternation zu einem produktiven morphologischen Marker ist vollzogen. In diesem Zusammenhang ist der Umlaut auch typologisch von Bedeutung, da die flexivische Information in den Stamm eindringt und damit im Althochdeutschen der Wandel von einem bis dahin rein additiven Kasus/Numerus-Ausdruck hin zur stammaffizierenden Kodierung flexivischer Information einsetzt (vgl. \citealt[48]{Nübling2005}, \citealt[182]{Seiler2008}, \citealt[1691]{Solms2004}). Infolge der Tilgung der lautgesetzlichen Umlaute aus dem Singularparadigma ist dieses morphophonologische Verfahren für die Numerusmarkierung reserviert.

\begin{table}
\begin{tabular}{lllll}
\lsptoprule
&  & \multicolumn{2}{c}{{\textit{iz}/\textit{az}{}-Klasse}} & \makecell[tl]{{mask. \textit{n}{}-}{Deklination}}\\
\cmidrule(lr){3-4}
&  & {Vorahd.} & {Ahd.} & \\
\midrule
{Sg.} & {Nom.} & {\textit{*lamb}} & {\textit{lamb}} & \textit{hano}\\
& {Gen.} & \textit{*lamb-ir-as} & \textit{l\textbf{e}mb-ir-es} $>$ \textbf{\textit{lambes}} & \textit{h\textbf{e}nin} $>$ \textbf{\textit{hanin}}\\
& {Dat.} & {\textit{*lamb-ir-a}} & {\textit{l\textbf{e}mb-ir-e} $>$ \textbf{\textit{lambe}}} & \textit{h\textbf{e}nin} $>$ \textbf{\textit{hanin}}\\
& Akk. & \textit{*lamb} & \textit{lamb} & \textit{hanon}\\
\tablevspace
{Pl.} & {Nom.} & {\textit{*lamb-ir-u}} & {\textit{l\textbf{e}mb-ir}} & \textit{hanon}\\
& Gen. & \textit{*lamb-ir-o} & \textit{l\textbf{e}mb-ir-o} & \textit{hanōno}\\
& {Dat.} & {\textit{*lamb-ir-um}} & {\textit{l\textbf{e}mb-ir-um}} & \textit{hanōm}\\
& Akk. & \textit{*lamb-ir-u} & \textit{l\textbf{e}mb-ir} & \textit{hanon}\\
\lspbottomrule
\end{tabular}
\caption{Paradigma von ahd. \textit{lamb} (\textit{iz/az}{}-Klasse) und ahd. \textit{hano} (mask. \textit{n}{}-Deklination, vgl. \citealt{BrauneHeidermanns2018}, \citealt[84]{Kürschner2008a}, \citealt[49]{Nübling2005}, \citealt[60]{Wurzel1982})\label{tab:7}}
\end{table}

\tabref{tab:7} weist für die neutr. \textit{iz/az}{}-Klasse einen weiteren innerparadigmatischen Ausgleich aus. Im Althochdeutschen ist das frühere stammbildende Suffix \mbox{-\textit{ir}} im Singularparadigma im Nominativ/Akkusativ lautgesetzlich getilgt, im Gen./Dat.Sg. ist es erhalten. Zu diesem Zeitpunkt hat -\textit{ir} die Funktion als transparenter Deklinationsklassenmarker bereits verloren, da es nicht mehr uniform im Paradigma vorkommt. Das vormals stammbildende Suffix wird in der Folge nicht mehr als Teil des Stammes interpretiert, sondern als Pluralmarker reanalysiert und gleichzeitig im Gen./Dat.Sg. getilgt (vgl. \citealt[50]{Nübling2005}, \citealt[87--89]{Wegener2005}). Als Ergebnis der Eliminierung von Umlaut (UL) und \textit{ir}{}-Suffix aus dem Singularparadigma erscheint der Singular als unmarkierte Basiskategorie, die Pluralinformation wird formal durch UL+\textit{ir} markiert (vgl. \citealt[51]{Nübling2005}, \citealt[1691]{Solms2004}). Damit ist im Paradigma der \textit{iz/az}{}-Deklination der Übergang von der Stamm- zur Grundformflexion vollzogen, die Nominativ-Singular-Form ist formal vollständig im Plural enthalten. Gleichzeitig hat die Separierung von Kasus/Numerus-Ausdruck zur spezifischen Reihenfolge von Numerus- und Kasusmarker geführt, die auch im Neuhochdeutschen wirksam ist: Auf das Pluralallomorph \mbox{-\textit{ir}} folgt der Kasusmarker in Gen.Pl. \textit{lemb-ir-o}, Dat.Pl. \textit{lemb-ir-um} (vgl. nhd. \textit{Lämm-er-n}). Mit der Separierung des Kasus/Numerus-Ausdrucks und der Reanalyse von Umlaut und früherem Kasus/Numerus-Suffix zu Pluralmarkern setzen in der \textit{iz/az}{}-Klasse im Althochdeutschen jene Umstrukturierungen des Deklinationssystems ein, die diachron zu einer Profilierung der Numeruskategorie und zu einem Abbau der Kasusmarkierung am Substantiv führen (vgl. \citealt[195--199]{DammelGillmann2014}, \citealt[85]{Kürschner2008a}).\pagebreak

Einen weiteren Beitrag zu Numerusprofilierung und Kasusnivellierung leistet die Endsilbenabschwächung. Die vollvokalischen Flexionssuffixe des Althochdeutschen sind von diesem Wandel besonders betroffen, da die Vokalqualität als formaler Marker von flexivischen Distinktionen funktionalisiert ist (vgl. \citealt[80]{Kürschner2008a}, \citealt[1172]{Sonderegger2000}). \citet[109]{Werner1969} zählt sechs ahd. Pluralallomorphe (\textit{∅}, -\textit{a}, -\textit{e}, \textit{i}, -\textit{un}, -\textit{ir}). Die Vollvokalqualität der Suffixe wird im Übergang zum Mittelhochdeutschen zu einem uniformen Schwa reduziert, so werden beispielsweise das Allomorph -\textit{a} in ahd. Nom./Akk.Pl. \textit{tag-a} ‚Tag‘ und das Allomorph -\textit{o} im Gen.Pl. \textit{tag-o} zu Schwa (Nom./Akk./Gen.Pl. \textit{tag-e}) reduziert (\tabref{tab:8}). \citet[246]{Sonderegger1979} beziffert die Reduktion der segmentierbaren Flexive von 52 auf 16 im Mittelhochdeutschen. Im Ergebnis dieses phonologischen Prozesses erscheinen verstärkt synkretische Formen im Pluralparadigma, es entsteht „ein strukturell grundsätzlich neues Flexionssystem“ \citep[1691]{Solms2004}. Im Fall der fem. \textit{ō}{}-Klasse führt die Vokalreduktion endgültig zu einer Ablösung des stammflektierenden Verfahrens durch die Grundformflexion. Die Differenzierung von Nom.Sg. \textit{gëba} vs. Dat.Sg. \textit{gëbu} durch Kontraste im Vokalismus entfällt, im Singularparadigma erscheint ein einheitliches Schwa in der Reduktionssilbe (\textit{gebe}) und wird als Teil des Stammes reanalysiert (vgl. \citealt[87]{Kürschner2008a}).


\begin{sidewaystable}
\small
\begin{tabular}{ *{14}{l} }
\lsptoprule
\multicolumn{2}{c}{germ. DK} & \multicolumn{2}{c}{mask. \textit{a}} & \multicolumn{2}{c}{neutr. \textit{a}} & \multicolumn{2}{c}{fem. \textit{ō}} & \multicolumn{2}{c}{mask. \textit{i}} & \multicolumn{2}{c}{fem. \textit{i}} & \multicolumn{2}{c}{neutr. \textit{iz/az}}\\
\cmidrule(lr){3-4}\cmidrule(lr){5-6}\cmidrule(lr){7-8}\cmidrule(lr){9-10}\cmidrule(lr){11-12}\cmidrule(lr){13-14}
\multicolumn{2}{c}{} & \textit{tac}  & Ahd. & \textit{wort}  & Ahd. & \textit{gebe}  & Ahd. & \textit{gast}  & Ahd. & \textit{kraft}  & Ahd. & \textit{lamp}  & Ahd.\\
\multicolumn{2}{c}{} & ‚Tag‘ & & ‚Wort‘ & & ‚Gabe‘ & & ‚Gast‘ & & ‚Kraft‘ & & ‚Lamm‘ & \\\midrule
Sg. & N./A. & \textit{tac} & \textit{∅} & \textit{wort} & \textit{∅} & \textit{geb\textbf{e}} & \textit{a} & \textit{gast} & \textit{∅} & \textit{kraft} & \textit{∅} & \textit{lamp} & \textit{∅}\\
& G. & \textit{tag\textbf{es}} & \textit{es} & \textit{wort\textbf{es}} & \textit{es} & \textit{geb\textbf{e}} & \textit{a} & \textit{gast\textbf{es}} & \textit{es} & \textit{kr\textbf{e}}\textit{ft\textbf{e}} & \textit{UL+i} & \textit{lamb\textbf{es}} & \textit{es}\\
& D. & \textit{tag\textbf{e}} & \textit{e} & \textit{wort\textbf{e}} & \textit{e} & \textit{geb\textbf{e}} & \textit{u} & \textit{gast\textbf{e}} & \textit{e} & \textit{kr\textbf{e}}\textit{ft\textbf{e}} & \textit{UL+i} & \textit{lamb\textbf{e}} & \textit{e}\\
Pl. & N./A. & \textit{tag\textbf{e}} & \textit{a} & \textit{wort} & \textit{∅} & \textit{geb\textbf{e}} & \textit{a} & \textit{g\textbf{e}}\textit{st\textbf{e}} & {\textit{UL}}\textit{+i} &  \textit{kr\textbf{e}}\textit{ft\textbf{e}} & \textit{UL+i}  & \textit{l\textbf{e}}\textit{mb\textbf{er}} & {\textit{UL}}\textit{+ir}\\
& G. & \textit{tag\textbf{e}} & \textit{o} & \textit{wort\textbf{e}} & \textit{o} & \textit{geb\textbf{en}} & \textit{ōno} & \textit{g\textbf{e}}\textit{st\textbf{e}} & {\textit{UL}}\textit{+o} & \textit{kr\textbf{e}}\textit{ft\textbf{e}} & {\textit{UL}}\textit{+o} & \textit{l\textbf{e}}\textit{mb\textbf{er}} & {\textit{UL}}\textit{+iro}\\
& D. & \textit{tag\textbf{en}} & \textit{um} & \textit{wort\textbf{en}} & \textit{um} & \textit{geb\textbf{en}} & \textit{ōm} & \textit{g\textbf{e}}\textit{st\textbf{en}} & {\textit{UL}}\textit{+im} & \textit{kr\textbf{e}}\textit{ft\textbf{en}} & {\textit{UL}}\textit{+im} & \textit{l\textbf{e}}\textit{mb\textbf{ern}} & {\textit{UL}}\textit{+irum}\\
\tablevspace
\multicolumn{2}{c}{\makecell[tl]{Gen.Sg.-/\\Plural-\\flexiv}} & \multicolumn{2}{c}{ \textit{{}-(e)s/-e}} & \multicolumn{2}{c}{\textit{-(e)s/-∅}} & \multicolumn{2}{c}{\textit{∅/∅}} & \multicolumn{2}{c}{\textit{{}-(e)s/UL+e}} & \multicolumn{2}{c}{\textit{UL+e/UL+e}} & \multicolumn{2}{c}{ \textit{{}-(e)s/(UL)+er}}\\
\lspbottomrule
\end{tabular}
\caption{Starke mhd. Deklinationsklassen mit Bezug zu den germ. Deklinationsklassen und zum jeweiligen Kasus/Numerus-Allomorph im Althochdeutschen (vgl. \citealt{KleinEtAl2018}, \citealt{Kürschner2008a})\label{tab:8}}
\end{sidewaystable}

Dieser formale Wandel der Flexionssuffixe erfolgt dabei -- zumindest aus flexionsmorphologischer Perspektive -- unsystematisch. In den einzelnen Klassen fallen bedingt durch den phonologischen Prozess der Vokalreduktion unterschiedliche Suffixe zusammen und führen zu spezifischen Paradigmenausprägungen (vgl. \citealt[86]{Kürschner2008a}). Nur im Dat.Pl. führt die Uniformierung des Flexivs im Spätalthochdeutschen zu -\textit{(e)n} zu einem einheitlichen Marker in sämtlichen starken Deklinationsklassen (vgl. Dat.Pl. \textit{tage-n} ‚Tag‘, \textit{wort-en} ‚Wort‘, \textit{krefte-n} ‚Kraft‘). Damit setzt sich die Abfolge von Numerus- und Kasus-Ausdruck, die sich im Althochdeutschen bereits in der \textit{iz/az}{}-Klasse herausgebildet hat, im mhd. Deklinationssystem durch. Auf den Stamm folgt die Markierung der Nu"-me"-rus- und anschließend der Kasusinformation (vgl. \citealt[87--88]{Kürschner2008a}, \citealt[51--52]{Nübling2005}).

Die phonologisch und morphologisch bedingten Umstrukturierungen des Deklinationssystems und der morphophonologische Wandel der Allomorphe führen diachron auch zu einem Wandel der Exponenz der Deklinationsklassen. Zum einen ist die Klassenzuordnung auf Basis ihrer primär semantischen Konditionierung und infolge der Fusion von Kasus/Numerus-Flexiv und des stammbildenden Suffixes, das -- wenngleich nicht mehr uniform im Paradigma vorhanden -- auch im Althochdeutschen teilweise noch transparent ist, infolge der Vokalreduktion zu Schwa nicht mehr möglich. In der Folge manifestiert sich Deklinationsklasse durch klassenspezifische Allomorphe und Paradigmenkonstellationen, wobei sich deklinationsklassenspezifische Marker auf wenige Stellen im Paradigma zurückziehen. Durch den Abbau der Kasusmarkierung am Substantiv wird die Pluralallomorphie zum primären Unterscheidungsmerkmal von Deklinationsklassen, Kasusexponenz im Gen.Sg. ist nur noch sekundäres Merkmal (vgl. \citealt[71]{KleinEtAl2018}). In einigen germanischen Sprachen und auch in den Dialekten des Deutschen, in denen der Abbau der Kasusmarkierung noch weiter vorangeschritten ist, manifestieren sich Deklinationsklassen einzig in der Pluralallomorphie (so z.\,B. im Dänischen und Schwedischen, vgl. \citealt{DammelEtAl2010}).

\subsection{Aus- und Umbau der Deklinationsklassen}
\label{sec:3.1.2}
Im Zentrum der bisherigen Darstellung stand der Aspekt des Deklinationsklassenwandels, d.\,h. der Wandel des Ausdrucks der Kategorie Deklinationsklasse. Im Fokus des folgenden Kapitels stehen (1) das Inventar der Deklinationsklassen und (2) die (Re-)Organisation des Deklinationsklassensystems in Form von Ausbau, Schwund und Deklinationsklassenwechsel. Die zentralen Entwicklungen in den einzelnen Sprachstufen des Deutschen sind im Folgenden zusammengefasst.

Diachron findet in der Tendenz ein Abbau von Deklinationsklassen statt. Infolge der phonologisch und morphologisch bedingten Umstrukturierungen (insbesondere durch den \textit{i}{}-Umlaut) nimmt die Komplexität des Deklinationsklassensystems bis zum Althochdeutschen noch zu. Doch die Separierung des Kasus\slash Numerus-Ausdrucks, die Reanalyse als Pluralmarker und der damit einhergehende Wandel der Deklinationsklassenexponenz leiten -- neben der Endsilbenabschwächung -- den Abbau von Deklinationsklassen und damit eine Reduktion des Deklinationsklassensystems ein (vgl. \citealt[293]{DammelKürschner2018}).

Daneben führen Deklinationswechsel zur Auflösung kleinerer Klassen. Deklinationsklassenwechsel ist dabei in vielen Fällen durch verschiedene außerflexivische Faktoren, allen voran Genus und semantische Merkmale, bedingt (siehe auch \sectref{sec:3.2}). Bereits im Germanischen vollziehen sich Umstrukturierungen des Deklinationsklassensystems, die stark durch Genus gesteuert sind (vgl. \citealt[74--77]{Kürschner2008a}, \citealt[296--297]{Nübling2008}):

\begin{itemize}
\item In der fem. und mask. \textit{i}-Klasse findet eine Differenzierung der Flexion statt: Die Maskulina nehmen im Singularparadigma die Formenbildung der mask. \textit{a}{}-Klasse an, die Feminina verändern die Formenbildung nicht.
\item Die Entsprechungen der \textit{a}{}- und der \textit{ō}{}-Klasse im Indogermanischen weisen Klassenmitglieder aller drei Genera auf, im Germanischen scheiden die Feminina aus der \textit{a}{}-Klasse, Maskulina und Neutra hingegen aus der \textit{ō}-Klasse aus. Im Ergebnis stehen sich die fem. \textit{ō}-Klasse und die mask./neutr. \textit{a}-Klasse gegenüber.
\end{itemize}

Im Übergang vom germ. zum ahd. Deklinationssystem besteht infolgedessen bei den vokalischen Stämmen eine Genusschranke zwischen Femininum und Nicht-Femininum, die konsonantischen Stämme (mit Ausnahme der \textit{n}{}-Klasse) setzen sich aus Maskulina und Feminina zusammen (vgl. \tabref{tab:5}, \citealt[77]{Kürschner2008a}, \citealt[§3]{Paul1968}).

Im Althochdeutschen findet eine Reduktion der germ. Klassen statt, indem kleinere Klassen in größere übertreten, so etwa die mask. \textit{u}{}-Stämme in die \textit{i}{}-Klasse, bei den Feminina geht die Subklasse der \textit{wō}{}-Stämme in die übergeordneten \textit{ō}{}-Klasse auf (vgl. \tabref{tab:9}). Mit dem Wechsel der mask. \textit{nd}{}- und \textit{er}{}-Klasse, beide konsonantische Stämme, in die vokalische mask. \textit{a}{}-Klasse wird die aus dem Indogermanischen ererbte Einteilung in konsonantische und vokalische Stämme aufgegeben, die sich historisch aus der Form des stammbildenden Suffixes ergab, z.\,B. für die \textit{er}{}-Stämme germ. *\textit{brōþar} -- \textit{brōþ(e)riz} ‚Bruder‘ > spätahd. \textit{bruoder} -- \textit{bruodera} ‚Bruder‘ und -- bereits im älteren Althochdeutschen -- \textit{fater} -- \textit{fatera} ‚Vater‘ (vgl. \citealt[§235]{BrauneHeidermanns2018}, \citealt[75]{Kürschner2008a}). Infolgedessen besteht ein Unterscheidungsmerkmal der Formenbildung im ahd. Deklinationssystem zwischen starker und schwacher Deklination. Die schwache Deklination, die auf die germ. \textit{n}{}-Stämme zurückgeht, weist im Plural und in den obliquen Kasus auf Nasal auslautende Flexionssuffixe auf, während die Flexive der starken Deklination nicht auf Nasal auslauten und auch sonst kein gemeinsames phonologisches Merkmal aufweisen (vgl. \citealt[304--305]{Fortson2007}, \citealt[80]{Kürschner2008a}).\footnote{Eine Ausnahme bilden hier die Neutra, die im Akk.Sg. kein Nasalsuffix aufweisen: Nom.\slash Akk.Sg. ahd. \textit{ouga} -- Gen./Dat. \textit{ougen} ‚Auge‘ (\citealt[§224]{BrauneHeidermanns2018}, vgl. \citealt[85]{Kürschner2008a}).}

\begin{table}
\tabcolsep=.75\tabcolsep
\fittable{\begin{tabular}{llll}
\lsptoprule
Germanisch & Althochdeutsch & Mittelhochdeutsch & Neuhochdeutsch\\
\midrule
mask. \textit{wa} & {}-\textit{o} / -\textit{wes} / -\textit{wa} & {}-\textit{wes} / -\textit{we} & {}-\textit{s} / -\textit{(e)}\\
\cellcolor{lsLightGray}mask. \textit{a} & {}-\textit{es} / -\textit{a} & {}-\textit{es} / -(\textit{e}) & -\textit{s} / -\textit{(e)}\\
mask. \textit{nd} & -\textit{es} / -\textit{a} & -\textit{es} / -(\textit{e}) & -\textit{s} / -\textit{(e)}\\
\cellcolor{lsLightGray}mask. \textit{er} & -\textit{es} / -\textit{a} & {}-\textit{s} / UL & {}-\textit{s} / UL\\
mask. \textit{ja} & {}-\textit{i} / -\textit{es} / -\textit{e} & {}-\textit{s} / ∅ & {{}-\textit{s} / [UL (Mask.)] -\textit{(e)}}\\
\cellcolor{lsLightGray}neutr. \textit{a} & {}-\textit{es} / ∅ & {}-\textit{es} / ∅ & -\textit{s} / [UL (Mask.)] -\textit{(e)}\\
neutr. \textit{ja} & {}-\textit{i} / -\textit{es} / -\textit{i} & -\textit{es} & -\textit{s} / [UL (Mask.)] -\textit{(e)}\\
neutr. \textit{wa} & {}-\textit{o} / -\textit{wes} / -\textit{o} & {}-\textit{wes} / ∅ & -\textit{s} / [UL (Mask.)] -\textit{(e)}\\
\cellcolor{lsLightGray}fem. \textit{er} & ∅ / ∅ & ∅ / UL & ∅ / UL\\
\cellcolor{lsLightGray}mask. \textit{i} & {}-\textit{es} / UL-\textit{i} & {}-\textit{es} / UL -\textit{e} & {}-\textit{s} / UL -\textit{(e)}\\
mask. \textit{u} & -\textit{es} / UL-\textit{i} & -\textit{es} / UL -\textit{e} & -\textit{s} / UL -\textit{(e)}\\
mask. Wurzelnomina & ∅ / ∅ & -\textit{es} / UL -\textit{e} & -\textit{s} / UL -\textit{(e)}\\
\cellcolor{lsLightGray}fem. \textit{i} & (UL) -\textit{i} / (UL) -\textit{i} & ∅ / (UL) -\textit{e} & ∅ / UL -\textit{e}\\
fem. Wurzelnomina & ∅ / ∅ & ∅ / (UL) -\textit{e} & ∅ / UL -\textit{e}\\
\cellcolor{lsLightGray}fem. \textit{ō} & {}-\textit{a} /-\textit{a} / -\textit{a} < -\textit{ā} & ∅ / ∅ & ∅ / -\textit{(e)n}\\
fem. \textit{wō} & -\textit{a} /-\textit{a} / -\textit{a} < -\textit{ā}& ∅ / ∅ & ∅ / -\textit{(e)n}\\
fem. \textit{jō} & {}-\textit{ea} / -\textit{ea} / -\textit{ea} < -\textit{eā} & ∅ / ∅ & ∅ / -\textit{(e)n}\\
\cellcolor{lsLightGray}fem. \textit{n} & {}-\textit{a} / -\textit{ûn} / -\textit{ûn} & {}-\textit{n} / -\textit{n} & ∅ / -\textit{(e)n}\\
\cellcolor{lsLightGray}mask. \textit{n} & {}-\textit{o} / -\textit{en} / -\textit{on} & -\textit{n} / -\textit{n} & {}-\textit{(e)n} /-\textit{(e)n}\\
\cellcolor{lsLightGray}neutr. \textit{n} & {}-\textit{a} / -\textit{en} / -\textit{un} & -\textit{n} / -\textit{n} & {}-\textit{s} / -\textit{(e)n}\\
\cellcolor{lsLightGray}neutr. \textit{iz/az} & {}-\textit{es} / -\textit{ir} & {}-\textit{es} / UL -\textit{er} & {}-\textit{s} / UL -\textit{er}\\
\multicolumn{3}{c}{} & {}-\textit{s} / -\textit{s} (Mask, Neutr.)\\
 &  &  & ∅ / -\textit{s} (Fem.)\\
\lspbottomrule
\end{tabular}}
\caption{Formale Entwicklung der Deklinationsklassen vom Germanischen zum Neuhochdeutschen (nach \citealt[94]{Kürschner2008a})\protect\footnote{Grau hinterlegt sind die großen, in der Sprachgeschichte prägenden Klassen des germ. Deklinationsklassensystems, die den Referenzpunkt der dialektologischen Darstellung (\chapref{chap:7}) bilden; die Unterklassen (\textit{ja}/\textit{wa} in der \textit{a}{}-Klasse und \textit{jō}/\textit{wō} in der \textit{ō}{}-Klasse) und kleinere Klassen werden im Folgenden ausgeblendet. Die formalen Ausprägungen der Klassen werden jeweils in der Form \textit{Gen.Sg.\slash Nom.Pl.} dargestellt, im Althochdeutschen wird bei erhaltener Stammflexion außerdem das Allomorph des Nom.Sg. angegeben (d.\,h. \textit{Nom.Sg.\slash Gen.Sg.\slash Nom.Pl.}). Die nhd. Klasseneinteilung basiert auf der Klassifikation von \citet[94]{Kürschner2008a}, da diese die diachronen Klassenbewegungen und die Reorganisation des Systems veranschaulicht. Im Neuhochdeutschen sind Pluralallomorphe zu Klassen zusammengefasst, die komplementär verteilt und durch die prosodische Struktur des Stammes konditioniert sind, z.\,B +/$-$ \textit{e}{}-Suffix (Notation „(\textit{e})“). Der Umlaut als Pluralmarker wird dann ohne Klammer angegeben, wenn er bei nicht-palatalem Stammvokalen („umlautbaren Stämmen“, vgl. \citealt[94, FN 27]{Kürschner2008a}) obligatorisch ist; die Notation „(UL)“ verweist auf Pluralformen mit potenziell „umlautbarem“ Stamm aber ohne Umlaut im Plural. Auch die Notation „[UL]“ verweist auf eine komplementäre Klasse mit Umlautplural.}\label{tab:9}}
\end{table}

Im Mittelhochdeutschen setzt sich die Reduktion der historischen Deklinationsklassen fort. Bedingt durch die Endsilbenabschwächung weist die schwache Deklination ein uniformes Nasalsuffix auf, die genusspezifische Differenzierung der Allomorphe durch Vollvokalismus in den Flexiven entfällt lautgesetzlich. Gleichermaßen durch die Endsilbenabschwächung bedingt geht Schwa-Suffix aus den vokalischen Allomorphen vor allem in der ahd. \textit{a}{}- und \textit{i}"=Deklination hervor (Nom.Pl. ahd. \textit{taga} > mhd. \textit{tage} ‚Tage‘, ahd. \textit{krefti} > \textit{krefte} ‚Kräfte‘). Mit der Schwa-Apokope greift erneut ein phonologischer Prozess in das Flexionssystem ein, der vor allem in den apokopierenden obd. Dialekten zu einer Reorganisation des Deklinationssystems führt (vgl. \sectref{sec:4.2.1}). Bei den historischen mask. {\textit{a}}- und {\textit{i}}{}-Klassen sind apokopierte, d.\,h. endungslose, Pluralformen im 12. Jahrhundert nur vereinzelt belegt, im 14. Jahrhundert beträgt der Anteil im Bair. nach \citet[136]{KleinEtAl2018} dagegen 65 \%. Früher und stärker tritt die Apokope bei Mehrsilbern mit Reduktionssilbe -\textit{en}, -\textit{er} und -\textit{el} ein, z.\,B. Nom.Pl. mhd. \textit{nagele} > \textit{nagel} ‚Nägel‘ (vgl.
\citealt[137]{KleinEtAl2018}).

Der Abbau von Nullpluralen, der in einigen wenigen ahd. Deklinationsklassen vorkommt, erfolgt im Mittelhochdeutschen „zunächst noch zaghaft“ (\citealt[135]{KleinEtAl2018}, vgl. \citealt[201]{DammelGillmann2014}, \citealt[89]{Kürschner2008a}, \citealt[1544]{WegeraSolms2000}):

\begin{itemize}
\item Die neutr. \textit{a}{}-Klasse (Typ \textit{wort} -- \textit{wort}) wechselt in das Flexionsmuster der historischen \textit{iz/az-}Klasse\footnote{\citet[153]{KleinEtAl2018} führen bereits für den ahd. Abrogans Belege von historischen neutr. \textit{a}{}-Stämmen an, die das \textit{ir}{}-Flexiv der \textit{iz/az}{}-Klasse angenommen haben: ahd. \textit{feld(h)ir} ‚Felder‘, ahd. \textit{harir} ‚Haare‘ (vgl. \citealt[293]{Nübling2008}). Damit war das historische stammbildende Suffix -\textit{ir}{}- zu diesem Zeitpunkt als Pluralmarker reanalysiert und produktiv.}  (nhd. \textit{Wörter}) oder nimmt das \textit{e}{}-Suffix an (nhd. \textit{Worte}), das bisher ein spezifisches Verfahren der Maskulina war. Die analoge Ausweitung des \textit{e}{}-Suffixes auf die neutr. \textit{a}{}-Klasse vollzieht sich vor allem im Mitteldeutschen, im obd. Sprachraum wirkt die einsetzende Schwa-Apokope der Ausbreitung des \textit{e}{}-Suffixes entgegen.
\item Die fem. \textit{er}{}-Stämme \textit{Tochter} und \textit{Mutter}, die im Althochdeutschen Nullplural haben, nehmen den Umlautplural an. Die mask. \textit{er}{}-Stämme, die im Althochdeutschen in die mask. \textit{a}{}-Klasse gewechselt waren, nehmen ebenfalls den Umlautplural an, sodass die germ. {\textit{r}}\textit{{}-Klasse}, die nur Verwandtschaftsbezeichnungen umfasst, im Mittelhochdeutschen ein weitgehend einheitliches Pluralmarkierungsverfahren aufweist: mhd. \textit{muoater} -- \textit{müeter} ‚Mutter‘, \textit{bruoder} -- \textit{brüeder} ‚Bruder‘, \textit{vater} -- \textit{veter} ‚\textit{Vater}‘ (zu Konkurrenzformen vgl. \citealt[137--138]{KleinEtAl2018}).
\end{itemize}

Auch die \textit{ō}{}-Deklination ist im Mittelhochdeutschen bereits im Abbau begriffen, da Numerussynkretismen abgebaut werden und die meisten Feminina in die schwache Deklination übergehen (vgl. \citealt[89]{Kürschner2008a}). Die Mitglieder der \textit{ō}{}-Klasse sind immer zweisilbig und enden auf Reduktionssilbe: auf Schwa (Typ \textit{gebe} -- \textit{gebe} ‚Gabe‘) oder auf -\textit{el} (Typ \textit{nâdel} -- \textit{nâdel} ‚Nadel‘). Im Frühneuhochdeutschen vollzieht sich der Zusammenfall der Flexionsmuster von fem. \textit{\textit{n}}\textit{{}-Klasse} und fem. \textit{\textit{ō}}\textit{{}-Klasse} zu einer spezifisch fem. Deklinationsklasse und Paradigmenkonstellation (\tabref{tab:10}). Indem das Pluralparadigma ein uniformes Nasalsuffix aufweist, das Nasalsuffix der schwachen Deklination im Singularparadigma hingegen getilgt ist, ist in dieser Deklinationsklasse die Nivellierung von Kasus vollzogen (vgl. \citealt[204--206]{DammelGillmann2014}, \citealt[110]{KleinEtAl2018}). Infolgedessen unterscheidet sich das Flexionsmuster der historischen fem. \textit{n}{}-Stämme von dem der mask. \textit{n}{}-Stämme,\footnote{Vgl. auch \sectref{sec:4.1} zur Flexion der sogenannten „gemischten“ Deklination der Maskulina.} deren oblique Kasusformen im Singular mit Nasalsuffix markiert werden; hier hat sich Deklinationsklasse Genus „untergeordnet“ (\citealt[303]{Nübling2008}, vgl. \citealt{Bittner1994, Bittner2000}).


\begin{table}
\small
\begin{tabular}{llll @{\,}c@{\,} llll @{\,}c@{\,} ll|ll}
\lsptoprule
 & \multicolumn{2}{c}{\textit{n}-Klasse} &  &  &  &  &  &  &  & & \multicolumn{2}{c}{\textit{ō}-Klasse} & \\
\cmidrule(lr){2-3}\cmidrule(lr){12-13}
 & {Sg.} & {Pl.} &  & ${\rightarrow}$ &  & Sg. & {Pl.} &  & ${\leftarrow}$ &  & \multicolumn{1}{l}{Sg.} & {Pl.} & \\
\midrule
N & \multicolumn{1}{l|}{\textit{zunge}} & \textit{zungen} & N &  & N & \textit{zunge} & \textit{zungen} & N &  & N & \textit{gebe} &  & N\\ \cline{2-2}
A &  &  & A &  & A & \textit{gebe} & \textit{geben} & A &  & A &  &  & A\\ \cline{13-13}
G &  &  & G &  & G &  &  & G &  & G &  & \textit{geben} & G\\
D &  &  & D &  & D &  &  & D &  & D &  &  & D\\
\lspbottomrule
\end{tabular}
\caption{Fusion der fem. \textit{n}{}- und \textit{ō}{}-Klasse im Frühneuhochdeutschen (vgl. \citealt[303]{Nübling2008}, \citealt[78]{KleinEtAl2018})}
\label{tab:10}
\end{table}

Ebenfalls im Frühneuhochdeutschen vollzieht sich bei den Feminina in den obd. Dialekten, insbesondere im Bair., ein regional begrenztes Ausgleichsmuster. In der fem. \textit{n}{}-Klasse wird das Nasalsuffix der obliquen Kasus analog in den Nom.Sg. übertragen (Nom.Sg. \textit{zunge} > \textit{zungen}). \citet[83]{KleinEtAl2018} interpretieren diesen innerparadigmatischen Ausgleich als indirekt entstandene Verschiebung der Morphemgrenze zwischen Stamm und Flexiv: \textit{zungen-∅} ‚Zunge‘ (vgl. Abschnitte~\ref{sec:7.1.3.1} und \ref{sec:8.2.3}). Auch die fem. \textit{ō}{}-Stämme gehen einen obd. Sonderweg, indem das -\textit{(e)n}{}-Suffix des Gen./Dat.Pl. nach dem Muster der fem. \textit{n}{}-Klasse auf den Nom./Akk.Pl. übertragen wird; einem unmarkierten Sg. (\textit{gebe}) steht der markierte Pl. (\textit{geben}) in allen Kasus gegenüber (\citealt[83]{KleinEtAl2018}). Damit besteht eine Forschungsfrage für die dialektgeographische Analyse, inwiefern in den untersuchten oobd. Dialekten auch synchron zwei unterschiedliche fem. Deklinationsmuster, nämlich (1) ein synkretisches Flexionsparadigma des Typs Sg./Pl. \textit{zungen-∅} und (2) ein numerusdistinktes Paradigma des Typs Sg. \textit{gebe} (bzw. apokopiertes \textit{geb}) \textit{--} Pl. \textit{geben}, erhalten sind (neben den rezenten Entsprechungen der historischen \textit{i}{}-Klasse, Typ \textit{kraft} -- \textit{krefte}).

\citet[169--170]{Nübling2016} erklärt die Reorganisation des Deklinationsklassensystems im Mittelhochdeutschen durch die „neue, numerusprofilierende Funktion“ von Deklinationsklassen. Abgebaut wurden Klassen mit Nullplural, und auch die Entstehung der sogenannten Mischdeklination der Feminina aus historischer \textit{n}{}- und \textit{ō}{}-Deklination führt zu einer Profilierung der Numerusinformation bei gleichzeitiger Kasusnivellierung. Im frnhd. Deklinationssystem setzt sich der Prozess der Nivellierung der Kasusflexive fort (vor allem im Dat.Sg. durch Apokope des \textit{e}{}-Suffixes), Deklinationsklasse manifestiert sich infolgedessen primär durch die Pluralallomorphie (vgl. \citealt[91]{Kürschner2008a}, \citealt[1542--1543]{WegeraSolms2000}). Die Distribution der Pluralallomorphe wird erneut restrukturiert, als das apokopierte \textit{e}{}-Suffix ausgehend vom Ostmitteldeutschen restituiert wird (vgl. \citealt[202--209]{DammelGillmann2014}, \citealt{Köpcke1994, Köpcke2000a}, \citealt[116--122]{Kürschner2008a}, \citealt[304--307]{Nübling2008}, \citealt[324--35]{Paul1968}, \citealt[1544]{WegeraSolms2000}):

\begin{itemize}\sloppy
\item Historische mask. \textit{a}{}-Stämme wie \textit{Stab} und \textit{Hals} hatten nach Abfall des \textit{e}{}-Suffixes die Deklination der historischen \textit{i}{}-Deklination (Typ mhd. \textit{gast} -- \textit{geste}) übernommen: mhd. Pl. \textit{hals-(e)} > mhd./frnhd. \textit{Häls-(e)}. Bedingt durch die Restitution des Schwa-Suffixes wird das Pluralmarkierungsverfahren UL+\textit{e} im Schriftdeutschen auf andere Maskulina ausgeweitet (gleichzeitig wurde dieses Pluralverfahren für die Feminina geschlossen, die fem. \textit{i}{}-Klasse ist damit nicht mehr produktiv). In den apokopierenden Dialekten ist der analoge Umlautplural (d.\,h. ein rein stammaffizierendes Verfahren) bei Maskulina stärker erhalten. Zudem zeigt \citet{Köpcke1994}, dass der Wechsel historischer \textit{a}{}-Stämme in das UL+\textit{e}{}-Verfahren durch deren Semantik bedingt war. Maskulina, die entweder Menschen, Menschengruppen oder Säugetiere denotieren, nehmen den UL+\textit{e}{}-Plural an, daneben wechseln auch menschliche oder tierische Körperteile in die \textit{i}{}-Deklination, und zwar, wie die Beispiele \textit{Rumpf} und \textit{Schopf} zeigen, bis ins jüngere Neuhochdeutsche hinein. Im Neuhochdeutschen besteht nach \citet[84]{Köpcke1994} bei einsilbigen Maskulina damit eine semantische Verteilung der Pluralallomorphe: Schwa-Suffix markiert Distanz zum Menschen, während UL+\textit{e} Nähe zum Menschen signalisiert.
\item Bei Mehrsilbern auf Reduktionssilbe \textit{-\textit{en}}, \textit{-\textit{er} }\textit{und} \textit{-\textit{el}} wird das \textit{e}{}-Suffix nicht restituiert, z.\,B. beim mask. \textit{i}{}-Stamm \textit{Apfel} (mhd. Pl. \textit{epfel-e} > \textit{epfel} > nhd. \textit{Äpfel}) und beim mask. \textit{a}{}-Stamm \textit{Finger} (Pl. mhd. \textit{fingere} > mhd. \textit{finger} > nhd. \textit{Finger}). In der Folge erscheinen Mehrsilber der historischen \textit{i}"=Deklination (Typ \textit{epfele} > \textit{Äpfel}) mit Umlautplural, Mehrsilber des Typs \textit{fingere} > \textit{Finger} behalten Nullplural (siehe auch \sectref{sec:3.2} zur prosodischen Konditionierung von Pluralallomorphen).

\item Die historischen mask. \textit{n}{}-Stämme werden im Spätmittelhochdeutschen und Frühneuhochdeutschen reduziert, indem die Klasse ihrer Semantik nach auf einem anthropozentrischen Kontinuum von Be"-lebt"-heits"-merk"-ma"-len reorganisiert wird (vgl. \citealt{Köpcke2000a} sowie \citealt{Kürschner2021} für einen Vergleich nah verwandter germanischer Sprachen). Historische \textit{n}{}-Stämme mit dem Merkmal [+menschlich] oder aus dem Nahbereich des Menschen (z.\,B. \textit{Affe} und andere Säugetiere) bleiben in der schwachen Deklination, belebte Denotate, die Tiere (v.\,a. Vögel) bezeichnen, wechseln (nach Apokope des Schwa in der Reduktionssilbe des Singulars) in die UL+\textit{e}{}-Deklination der historischen \textit{i}{}-Deklination: mhd. \textit{storch(e)} -- \textit{storchen} > nhd. \textit{Storch} -- \textit{Störche}, mhd. \textit{han(e)} -- \textit{hanen} > nhd. \textit{Hahn} -- \textit{Hähne}. Belebte Denotate, die auf der anthropozentrisch strukturierten Skala noch weiter vom Menschen entfernt sind (Vögel, Fische, Reptilien) wechseln das Genus: mhd. mask. \textit{snecke} > nhd. fem. \textit{Schnecke} (vgl. \citealt[§55]{Paul1968}). Bei historischen \textit{n}"=Stämmen mit unbelebtem Denotat wird das Nasalsuffix in den Nom.Sg. übertragen. Zum Teil behalten diese Zweisilber den Nullplural (mhd. \textit{balke} -- \textit{balken} > nhd. \textit{Balken} -- \textit{Balken}), zum Teil übernehmen sie analogisch den Umlautplural der mehrsilbigen \textit{i}{}-Stämme: frnhd. \textit{gart(e)} -- \textit{garten} > nhd. \textit{Garten} -- \textit{Gärten}.
\end{itemize}

Die Vorlage der nhd. Pluralbildung für Zweisilber auf -\textit{en}, -\textit{er} und -\textit{el}, die mehrsilbigen \textit{i}{}-Stämme (Typ mhd. \textit{epfele} > \textit{Äpfel}) und die mehrsilbigen \textit{a}{}-Stämme (Typ mhd. \textit{fingere} > \textit{Finger}), waren jeweils kleine Gruppen innerhalb der jeweiligen Deklinationsklassen. Auch die \textit{iz/a}z-Klasse im Althochdeutschen umfasste weniger als ein Dutzend Mitglieder, wurde infolge der Ausdehnung des Pluralmusters UL+\textit{er} auf weitere Neutra und Maskulina aber zu einer Klasse mit ca. 100 hochfrequenten Mitgliedern, während die im Germanischen große und produktive Klasse der neutr. \textit{a}{}-Stämme abgebaut wurde (vgl. \citealt[87]{Wegener2005}). Hohe Typenfrequenz ist damit ein Faktor, nicht aber eine notwendige oder hinreichende Bedingung für Deklinationsklassenwandel oder -erhalt (vgl. \citealt[293]{Nübling2008}). Vielmehr ist die Restrukturierung des Deklinationsklassensystems diachron durch andere Prinzipien bedingt. So ist das System, das sich im Mittelhochdeutschen und Frühneuhochdeutschen herausbildet, stark durch Genus gesteuert (z.\,B. in Form eines spezifisch fem. Paradigmas des Typs \textit{Zunge} -- \textit{Zungen}) und durch klassenspezifische semantische Merkmale (Belebtheit bei den historischen \textit{a}{}- und \textit{n}{}-Maskulina, Verwandtschaftsbezeichnungen bei den historischen \textit{r}{}-Stämmen). Bereits im Mittel- und Frühneuhochdeutschen beginnt sich die Präferenz für zweisilbige, trochäische Pluralformen bei Simplizia herauszubilden, die dann im Neuhochdeutschen die Distribution der Pluralallomorphe steuert. So bleiben durch Schwa-Apokope entstandene zweisilbige Nullplurale (mhd. \textit{finger(e)} > nhd. \textit{Finger}) erhalten oder nehmen analogen Umlautplural an (mhd. \textit{nagel(e)} > nhd. \textit{Nägel}), Einsilber mit Nullplural (v.\,a. neutr. \textit{a}-Stämme) wechseln die Deklinationsklasse hin zu einer distinkten (zweisilbigen) Pluralmarkierung (vgl. \citealt[105--106 und 123]{Kürschner2008a}).

\begin{sloppypar}
Gleichzeitig zeigt der Überblick über die Entwicklung der Deklinationsklassen und hierin vor allem das Mittel- und Frühneuhochdeutsche, dass (1) phonologische Prozesse wie die Apokope in den Dialekten des Hochdeutschen unterschiedlich wirken und zu spezifischen Reorganisationen des Deklinationssystems führen, und dass (2) auch morphologische Umstrukturierungen (hier bei den historischen fem. \textit{n}{}- und \textit{ō}{}-Stämmen) dialektspezifische Ausprägungen aufweisen, weshalb eine kontrastive Untersuchung dialektaler Deklinationsklassensysteme „ein äußerst lohnendes, bislang brachliegendes Forschungsprojekt“ \citep[312]{Nübling2008} darstellt.
\end{sloppypar}

\section{Zur „notorischen Persistenz“ und Funktionalität von Deklinationsklassen}
\label{sec:3.2}
Der historische Überblick in \sectref{sec:3.1} hat zwei zentrale Entwicklungen von Deklinationsklassen offengelegt: (1) Die formale Exponenz von Deklinationsklassen wandelt sich von einem semantisch basierten System mit transparenter dreigliedriger Struktur hin zu einer „verdeckten“ Kategorie, die sich synchron primär in der Pluralallomorphie und nur sekundär in der Kasusmarkierung manifestiert. (2) Diachron werden die historischen, d.\,h. die aus dem Indogermanischen ererbten Deklinationsklassen ab dem Althochdeutschen reduziert. Gleichzeitig findet ein Ausbau des vorhandenen Klasseninventars statt: durch Mischung der historischen fem. \textit{n}{}- und \textit{ō}{}-Klasse zu einer neuen, spezifisch fem. Klasse und durch Entstehung zwei neuer Klassen mit \textit{s}{}-Plural im Neuhochdeutschen und der gemischten Maskulina und Neutra (vgl. \tabref{tab:9}). Die Forschungsfrage, die an diese beiden Entwicklungen anschließt, ist vor allem die nach der Funktionalität von Deklinationsklassen. Warum weisen Deklinationsklassen diachron eine solch „notorische Persistenz“ \citep[28]{Nübling2008} auf? Warum leistet sich das Deutsche (auch im Vergleich zu einigen anderen germanischen Sprachen) ein hohes Maß an Allomorphie? In den vorgestellten Forschungsarbeiten wird Deklinationsklassen dabei keine singuläre Funktion zugewiesen, sondern es werden vielmehr verschiedene Bereiche identifiziert, in denen Deklinationsklassen diachron Funktionalität entwickelt haben und die ihren Erhalt lizenzieren.

\citet{DammelNübling2006} und \citet[236--239]{Kürschner2008a} zeigen, dass Allomorphie ein Indikator einer hochrelevanten Kategorie ist und diese diachron vor Abbau schützt. Dies relativiert \citegen{Wurzel1984} Annahme, dass sogenannte „überstabile“ Marker eine Kategorie stabilisieren und sich morphologischem Abbau am längsten widersetzen. Überstabile Marker werden nach \citet[139]{Wurzel1984} von einer stabilen Flexionsklasse auf eine andere übertragen, ohne dass sich dabei deren „Identität“ als eigene Klasse verändert. Mit jeder Ausweitung auf weitere Flexionsklassen erhöht sich der Stabilitätsgrad des Markers \citet[139]{Wurzel1984} zufolge in einer Art „‚Lawineneffekt‘“; am Ende der Entwicklung stünde ein klassenübergreifender, uniformer Marker, wie es beispielsweise im Neuhochdeutschen im Dat.Pl. (\textit{den Hunde-n}) der Fall ist. \citet{DammelNübling2006} und \citet{Kürschner2008a} zeigen indes, dass überstabile Marker und der gleichzeitige Verlust von Allomorphie eher ein Anzeichen von „categorial weakness“ sind und zu einer weiteren Schwächung der Flexionskategorie („category weakening“, \citealt[110]{DammelNübling2006}) führen. So liegt etwa im Dänischen, Schwedischen und Niederländischen diachron eine Uniformierung des Kasusausdrucks vor, dies geht aber mit einem Abbau der Kategorie Kasus einher \citep[238]{Kürschner2008a}. Allomorphie als „counterpart of superstable markers“ (\citealt[99]{DammelNübling2006}) schützt somit vor Deflexion.

Pluralallomorphie stärkt nach \citet{Nübling2008, Nübling2016} die Numeruskategorie außerdem, indem Deklinationsklasse und Genus interagieren und funktional „eine Symbiose“ \citep[167]{Nübling2016} eingehen. Deklinationsklasse manifestiert sich primär in der Pluralallomorphie, Genus wird nur im Singular durch Kongruenz in der Nominalphrase sichtbar. Auf diese Weise ergänzen sich beide Klassifikationssysteme komplementär und stärken die Numeruskategorie „von beiden Seiten her“ \citep[309]{Nübling2008}: den Singular syntagmatisch durch Genuskongruenz, den Plural paradigmatisch durch Pluralallomorphie. \citet{Nübling2016} sieht in dieser Interaktion von Genus und Deklinationsklasse eine Neufunktionalisierung der beiden desemantisierten Klassifikationsprinzipien, die „heute primär dazu da sind, Numerus zu markieren“ (\citealt[155]{Nübling2016}, vgl. \citealt[206--208]{DammelGillmann2014}, \citealt[364--365]{Kürschner2008a}, \citealt{KürschnerNübling2011}).

Eine weitere Schnittstelle zwischen Genus und Deklinationsklasse besteht darin, dass Genus ein Steuerungsprinzip der Deklinationsklassenzugehörigkeit und von Deklinationsklassenwandel ist. Deklinationsklassen sind in diesem Sinne „nicht grundsätzlich idiosynkratisch“ \citep[28]{Kürschner2008a}, sondern an verschiedene außerflexivische Faktoren gekoppelt, die wiederum selbst Wandelprozessen unterliegen (vgl. \citealt[91]{Wurzel1986}).\footnote{\citet[91]{Wurzel1986} fasst phonologische, syntaktische und semantische Faktoren, die Deklinationsklassenzugehörigkeit „motivieren“ können, unter den Begriff der „außermorphologischen Eigenschaften“ zusammen. Da zu diesen Faktoren auch morphologische Faktoren wie Derivationssuffixe zählen (vgl. \citealt[60]{Kürschner2008a}), wird der Terminus „außerflexivisch“ verwendet. \citet[284--285]{Nübling2008} kritisiert zudem Wurzels Terminologie einer „Flexionsklassenmotivierung“, da Motivation Funktionalität impliziere; vielmehr handele es sich um „Konditionierung“.} Außerflexivische Konditionierung stellt eine „Memorierungshilfe“ \citep[285]{Nübling2008}, nicht aber die Funktion von Deklinationsklassen dar (vgl. \citealt[79]{Bittner1994}, \citealt[137]{Harnisch1987}). Außerflexivische Konditionierung ist \citet[311]{DammelKürschner2018} zufolge auch deshalb kognitiv vorteilhaft und entspricht „the pattern-seeking nature of human cognition“, da jeweils Prinzipien funktionalisiert sind, die bereits im Flexionssystem vorhanden waren. Folgende kurze, diachrone Übersicht fasst die Konditionierungsfaktoren und den Wandel der Konditionierung im Deutschen zusammen:

\begin{itemize}\sloppy
\item \textit{Genus} ist mit dem Wandel von overten zu koverten Deklinationsklassen bereits im Germanischen primäres Konditionierungsprinzip, das den Umbau des Deklinationssystems hin zu einer Opposition zwischen Femininum und Nicht-Femininum bei den vokalischen Stämmen steuert. Im Laufe der deutschen Sprachgeschichte bleibt Genus „durchweg systemprägend als Konditionierungsfaktor erhalten“ (\citealt[142]{Kürschner2008a}, vgl. \citealt{Duke2005}). Im Althochdeutschen ist im Bereich der Pluralallomorphie tendenziell eine Genusschranke zwischen Neutra und Nicht-Neutra festzustellen, im Frühneuhochdeutschen tritt ein Wandel zu einer Opposition zwischen Femininum und Nicht-Femininum ein (vgl. \citealt[97--100 und 108--116]{Kürschner2008a}, \citealt[159--165]{Nübling2016}).
\item \textit{Semantische Merkmale} konditionieren Deklinationsklassen und Deklinationsklassenwandel auf einer sekundären, dem Steuerungsprinzip Genus untergeordneten Ebene. Prägend sind diachron die Distinktionen [${\pm}$konkret], [${\pm}$belebt], bei belebten Denotaten ergänzt durch weitere Distinktionen wie [${\pm}$menschlich] auf einem anthropozentrischen Kontinuum, Nähe zum Menschen (hierin v.\,a. Körperteile) und die kleine Klasse von Verwandtschaftsbezeichnungen (vgl. \citealt{Köpcke1994, Köpcke2000a}, \citealt[100--104 und 116--122]{Kürschner2008a} sowie \sectref{sec:8.3.2} zu einer ausführlicheren Einführung semantischer Distinktionen). Darstellungen zu dialektalen Deklinationsklassen zeigen, dass daneben auch semantische Distinktionen hinsichtlich Kollektivität vs. Individuiertheit steuernd wirken können (vgl. \citealt[170]{Rowley1997}, \citealt[442--443]{Schirmunski1962}). Semantische Konditionierung spielt zudem, wie \citet[49--50]{VerslootAdamczyk2018} für die nordseegermanischen Varietäten nachweisen können, sowohl beim Erhalt als auch bei der analogischen Ausdehnung irregulärer Pluralformen eine zentrale Rolle.
\item \textit{Prosodische Konditionierung}, wie sie für das nhd. Deklinationssystem prägend ist, bildet sich im Mittelhochdeutschen und Frühneuhochdeutschen heraus. Präferiert werden zweisilbige, trochäische Flexionsformen, d.\,h. die prosodische Konditionierung erfolgt durch Anforderungen an die Form des Produkts („Output“) des flexivischen Verfahrens (vgl. \citealt[123--130]{Kürschner2008a}). Die im Mittelhochdeutschen sich bereits abzeichnende Outputkonditionierung steuert in der Folge u.\,a. die Verteilung der Pluralallomorphe. Bei den historischen mask. \textit{a}{}-Stämmen führt die Outputbedingung zu einer komplementären Verteilung von Schwa-Suffix nach einsilbigen Stämmen (nhd. \textit{Tag} -- \textit{Tage}) und Nullplural bei Zweisilbern mit Reduktionssilbe (nhd. \textit{Finger} -- \textit{Finger}). Bei der fusionierten fem. Mischdeklination ist die komplementäre Verteilung von -\textit{(e)n} in nhd. \textit{Frau} -- \textit{Frau-en} und \textit{Gabe} -- \textit{Gabe-n} ebenfalls durch die Silbenzahl des Singularstammes bedingt (vgl. \sectref{sec:4.1}).
\item \textit{Auslautkonditionierung} (auch phonotaktische Konditionierung) wirkt im Althochdeutschen auf einer der Genuskonditionierung untergeordneten Ebene, indem der Auslaut der Grundform Klassenzugehörigkeit steuert (vgl. \citealt[107]{Kürschner2008a}). Als Ergebnis der Endsilbenreduktion zu uniformem Schwa entfällt dieses Konditionierungsmerkmal bei den vokalisch auslautenden Stämmen. Infolge der Restrukturierung der historischen \textit{n}{}-Klassen im Mittelhochdeutschen signalisiert Schwa-Reduktionssilbe im Singular nur noch bei Neutra und Maskulina schwache Deklination (bei Maskulina in Kombination mit dem Merkmal [+menschlich], vgl. \citealt{Köpcke2000a}, \citealt[107]{Kürschner2008a}). Bei den Feminina umfasst die fem. \textit{i}{}-Klasse im Althochdeutschen alle konsonantisch auslautenden Feminina, während vokalisch auslautende Feminina zur \textit{ō}{}- oder \textit{n}{}-Klasse gehören. Mit der Reorganisation der fem. Deklinationsklassen im Frühneuhochdeutschen zu einer fem. Mischdeklination neben der historischen \textit{i}{}-Klasse, die synchron nicht mehr produktiv ist, ergibt sich eine prosodische und phonotaktische Konditionierung. Feminina der Mischdeklination sind prototypischerweise trochäische Zweisilber auf Schwa, \citet[128]{Köpcke1993} zufolge ist Schwa-Reduktionssilbe im Singular hier sogar „Genusindikator“, während die historische \textit{i}{}-Klasse Einsilber mit möglichst komplexer, meist auf /t/ auslautender Koda und hinterem Stammvokal sowie hoher Tokenfrequenz umfasst, z.\,B. \textit{Kraft} -- \textit{Kräfte}, daneben \textit{Kuh} -- \textit{Kühe, Maus -- Mäuse} (vgl. \citealt[124--128]{Köpcke1993}, \citealt[110]{Kürschner2008a}, \citealt[302--304]{Nübling2008}).
\item \textit{Morphologische Konditionierung} greift bereits im Althochdeutschen, indem Derivationen mit dem Suffix -\textit{il} (ahd. \textit{leffil} ‚Löffel‘, \textit{sluȝȝil} ‚Schlüssel‘, \textit{gurtil} ‚Gürtel‘) der mask. \textit{a}{}-Deklination, Diminutiva der neutr. \textit{a}{}-Deklination und movierte Substantive der fem. \textit{jō}{}-Klasse angehören (vgl. \citealt[§194, §196, §211]{BrauneHeidermanns2018}). Da die einzelnen Derivationen jeweils dasselbe Genus aufweisen, wirkt hier die übergeordnete Konditionierung durch Genus, ausgenommen sind aber die im Althochdeutschen genusunabhängig auftretenden Suffixe -\textit{nis} und -\textit{sal}, die im Mittelhochdeutschen eine vom Derivationssuffix gesteuerte Pluralmarkierung ausbilden (\citealt[104 und 122--123]{Kürschner2008a}).
\end{itemize}

Konditionierungswandel vollzieht sich auf der Ebene einzelner Deklinationsklassen: Um produktiv und offen für weitere Klassenmitglieder zu bleiben (und nicht, wie beispielsweise die historische \textit{r}{}-Klasse der Verwandtschaftsbezeichnungen oder die fem. \textit{i}{}-Klasse, homogen, aber geschlossen zu sein), wird die Basis für Homogenität entzogen, indem sich die Konditionierungsstrukturen der Klasse wandeln \citep[340--342]{Kürschner2008a}. Der Vergleich von verschiedenen germanischen Sprachen zeigt, dass Konditionierungswandel in Richtung eines Abbaus von Komplexität weist (vgl. \citealt[355--358]{Kürschner2008a}). Ausgehend von einem strukturalistischen Verständnis von Sprache als System von Zeichen im Sinne \citegen{Saussure1916} bezieht sich Komplexität hier auf das Verhältnis von formaler (signifiantbasierter) und signifiébasierter Konditionierung (ausführlicher hierzu \citealt{DammelKürschner2008}). Formale Konditionierungsfaktoren, die sich auf ausdrucksseitige Eigenschaften des Stammes beziehen (d.\,h. prosodische oder phonotaktische Eigenschaften sowie Derivationssuffixe), sind transparenter als signifiébasierte Konditionierung, die sich auf semantische Merkmale und Genus als inhärentes Merkmal des Stammes bezieht. Den höchsten Grad an Komplexität weist lexikalische Konditionierung auf, die eine irreguläre Form der Konditionierung darstellt und mit jedem Lexem gelernt werden muss (vgl. \citealt[310--314]{Kürschner2008a}, \citealt[474--476]{Neef2000}, \citealt{VerslootAdamczyk2018}). Diachron findet ein Abbau der Komplexität des primären Konditionierungsprinzips statt, indem weitere Konditionierungsfaktoren zur genusbasierten Konditionierung des Germanischen hinzukommen, im Falle des Deutschen formale Konditionierungsfaktoren und semantische Konditionierung. Damit leistet sich das Deutsche ein vergleichsweise komplexes System interagierender Konditionierungsprinzipien.
