\part{Dialektgeografischer Teil}
\label{part:II}
\noindent Die folgende Darstellung der nominalen Flexionsmorphologie in den oobd. Dialekten Bayerns knüpft an die in Teil~\ref{part:I} aufgezeigten Forschungsbefunde und "=fra\-gen zu dialektalen Deklinationsklassensystemen und zur Numerus- und Kasusmarkierung an. Ziel ist es, das Deklinationsklassensystem dieser Dialekte kontrastiv darzustellen (und zwar im interdialektalen Vergleich und in der diachronen Entwicklung) und gleichzeitig auf der Formenebene dialektale und eventuell dialektraumspezifische Kodierungsverfahren zu beschreiben. Teil~\ref{part:II} ist dabei spiegelverkehrt zu Teil~\ref{part:I} aufgebaut: Ausgehend von Markierungsstrategien und konkreten Markern wird das Deklinationsklassensystem der oobd. Dialekte analysiert. Grundlegend ist hierbei das Verständnis von Flexion als grammatischem Schnittstellenphänomen. Das Leitthema der folgenden Kapitel lautet daher -- in Anlehnung an \citet[13]{Harnisch1987} -- \textit{Where’s} \textit{morphology?}


Ziel ist es, die Flexion innerhalb des Sprachsystems der jeweiligen Ortsdialekte darzustellen und aufzuzeigen, wo Morphologie mit anderen Systemebenen interagiert (und zwar mit Blick auf Synchronie und Diachronie) und wo Morphologie unabhängig stattfindet. Im Einzelnen ist die Untersuchung dabei durch folgende Forschungsfragen motiviert:

\begin{itemize}
\item Flexion zwischen \textit{Phonologie} \textit{und} \textit{Morphologie}: Inwiefern sind die Markierungsverfahren und einzelnen Marker in den rezenten oobd. Dialekten das Ergebnis dialektspezifischer phonologischer Prozesse (etwa von Apokope oder Vokaldehnung)? Inwiefern bedingen sich hier phonologischer und morphologischer Wandel? Und in welchem Maße werden phonologische Alternationen morphologisiert und produktiv?
\item Wo löst \textit{morphologischer} \textit{Wandel} (z.\,B. in Form von innerparadigmatischen Ausgleichsprozessen) weiteren morphologischen Wandel aus, wie es beispielsweise im Bair. mit den „potenzierten“ Pluralen und im Niederdeutschen mit dem \textit{s}{}-Plural bei Nasalsuffix im Nom.Sg. geschehen ist (siehe \sectref{sec:4.2.1})?
\item Flexion zwischen \textit{Morphologie} \textit{und} \textit{Syntax}: Wo und in welchem Umfang wird die Numerus- und die Kasusinformation in der Nominalphrase markiert, wenn sie am Substantiv abgebaut ist? Inwiefern werden Synkretismen bei Artikelformen durch formale Kodierung am Substantiv kompensiert?
\item Flexion zwischen \textit{Morphologie} \textit{und} \textit{semantisch-pragmatischem} \textit{Kontext}: In welchem Maß ist die formale Kodierung flexivischer Information von der Eindeutigkeit des Kontexts abhängig? Wird die Numerus- oder Kasusinformation disambiguiert, wenn der Kontext ambig ist? Wie „statisch“ oder variabel ist also Flexion im Äußerungskontext?
\end{itemize}

Die Darstellung der Formenbildung ist zunächst -- so weit wie möglich -- theorieneutral, eine Diskussion der dialektalen Flexionsmorphologie vor dem Hintergrund verschiedener morphologischer Theoriebildungen erfolgt erst in \chapref{chap:10}. Allerdings liegen der deskriptiven Analyse die theoretischen Vorannahmen einer prozessorientierten Morphologie zugrunde, die etwa bei Nullpluralen von einem morphologischen Verfahren ausgeht (vgl. \chapref{chap:4}).


Das Ziel besteht also nicht darin, dialektale Flexionsmorphologie vor dem Hintergrund eines regelbasierten Ansatzes zu modellieren, wie etwa \citet{Harnisch1987} im Rahmen der Natürlichen generativen Morphologie. Stattdessen werden die synchronen morphologischen Strukturen, die in den Dialektdaten zu finden sind, erfasst und diachron eingeordnet. Hierin folgt die Untersuchung der Pionierarbeit von \citet{Rowley1997}, der die nominale Flexion der ofr. und nordbair. Dialekte Nordostbayerns synchron-deskriptiv darstellt und gleichzeitig „diachron erklärend sein will“ \citep[2]{Rowley1997}. Methodisch knüpft die vorliegende Untersuchung dabei an \citegen{Rowley1997} Arbeit an und weitet die Analyse zur flexionsmorphologischen Sprachgeografie auf weitere Dialekträume, Analyseaspekte und Datenmaterial aus.
