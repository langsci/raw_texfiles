\chapter{Numerus- und Kasusflexion des Deutschen und seiner Dialekte}
\label{chap:4}
Im Zentrum des zweiten Teils des Forschungsüberblicks steht die Formenbildung der beiden nominalen Flexionskategorien Numerus und Kasus und damit die Frage: Welche formalen Mittel werden genutzt, um flexivische Informationen zu kodieren? In \citegen[10]{Wurzel2000} Definition schlägt sich in der Flexionsmorphologie „das Verhältnis von Form und Funktion im Zeichenverhältnis von Marker (Ausdruck) und Kategorie (Inhalt) innerhalb der Grenzen des Wortes nieder“. Die ausdrucksseitige Realisierung wird im Folgenden auf zwei Ebenen analysiert: als morphologisches Verfahren im Sinne einer prozessorientierten Morphologie und als konkreter Marker (d.\,h. auf der Ebene des Allomorphs). Unterschieden werden die Verfahren additive Markierung (Typ \textit{Hund} -- \textit{Hund-}\textbf{\textit{e}}), stammaffizierende (auch modifikatorische) Markierung (ofr. h\textbf{u}nd -- h\textbf{ü}nd, bair. h\textbf{ū}n\textbf{d} -- h\textbf{u}n\textbf{t}), subtraktive Markierung (ofr.-hess. hōn\textbf{d} -- hö\textbf{n}) und Nullmarkierung (ofr. hund -- hund). Während bei der additiven Markierung die flexivische Information durch ein eigenes Segment ausgedrückt wird, im Fall von \textit{Hund-e} durch das Pluralallomorph -\textit{e}, erfolgt die Kodierung bei stammaffizierender und subtraktiver Markierung „indirekt“ und in Relation zur Grundform im Paradigma \citep[11]{Wurzel2000}. Mit der Nullmarkierung wird schließlich ein Verfahren beschrieben, bei dem die flexivische Information nicht am Wort selbst kodiert ist; im Flexionsparadigma erscheinen Synkretismen.

Die morphologischen Kategorien Numerus und Kasus, die durch diese Verfahren kodiert werden, repräsentieren verschiedene grammatische Konzepte und Funktionen. Kasus markiert syntaktische Relationen und spezifiziert als adverbaler Kasus semantische Rollen im Satz. Numerus markiert die Unterscheidung zwischen einem vs. mehreren Referenten, im Deutschen repräsentiert durch die Kategorienausprägungen Singular und Plural. \citet[62--66]{Wurzel1984} differenziert in diesem Zusammenhang das semantische Basiskonzept (Ein- vs. Mehrzahligkeit) von dem grammatischen Basiskonzept (Singularität vs. Pluralität), das das „Verfahren zur Versprachlichung des semantischen Basiskonzepts“ ist. Damit bilden grammatische Basiskonzepte die formale Ebene einer grammatischen Kategorie ab, sie sind ein „Bündel von Umordnungsrelationen, Schaltstellen zwischen Semantik und Morphosyntax“ \citep[62]{Wurzel1984}. Sind sie nicht grammatikalisiert (wie z.\,B. der Dual im Standarddeutschen), werden sie lexikalisch oder in Form von Wortbildungen realisiert: \textit{zwei}/\textit{beide}/\textit{ein Paar Hörner}, \textit{Gehörn}.

Wenn im Folgenden von Numerusmarkierung gesprochen wird, dann bezieht sich dies auf die Markierung der Pluralinformation. Seit dem typologischen Wandel von der Grundformflexion zur Stammflexion ist der Singular unmarkiert, Numerusflexion erfolgt in Form einer Markierung der Pluralinformation (vgl. \sectref{sec:3.1.1}).\footnote{\citet{Harnisch1994b, Harnisch2001} nimmt indes auch synchron Stammflexion an. In nativen Substantiven wie \textit{Ros-e} -- \textit{Ros-en} und \textit{Tropf-en} -- \textit{Tropf-en-∅} stellen -\textit{e} und -\textit{en} im Singular stammerweiternde Suffixe dar. Grundlage dieser Analyse ist neben dem Flexionsparadigma auch das „Wortfamilien-Paradigma“ \citep{Harnisch2001}, das Wortbildungen wie \textit{Rös-chen}, \textit{Tröpf-chen} einschließt (Beispiele aus \citealt{Harnisch1994b}).}  Das folgende Kapitel gibt eine konzise Übersicht der Markierungsverfahren und die Distribution der konkreten Marker (d.\,h. Allomorphe) von Numerus und Kasus im Neuhochdeutschen (\sectref{sec:4.1}). \sectref{sec:4.2} bietet einen Überblick des Forschungsstandes zu Distribution und Produktivität von Pluralmarkierungsverfahren und zur Kasusmarkierung in den Dialekten des Deutschen, der auch die phonologischen Voraussetzungen und damit die Diachronie in den Blick nimmt. \sectref{sec:4.3} berücksichtigt schließlich die Kodierung von Numerus- und Kasusinformation in Nominalphrase und Satzkontext.

\section{Numerus{}- und Kasusmarkierung im Neuhochdeutschen}
\label{sec:4.1}
Infolge des Abbaus der Kasusflexion am Substantiv, der im Laufe des Alt- und Mittelhochdeutschen durch phonologischen und morphologischen Wandel erfolgte, ist die Markierung der Kasusinformation auf wenige Positionen im Paradigma reduziert. Im Dat.Pl. erscheint in der starken Deklination das Suffix -\textit{n} (Dat.Pl. \textit{Gäste-n}, \textit{Lämmer-n}), die schwache und gemischte Deklination hat im gesamten Pluralparadigma eine synkretische Form mit \textit{(e)n}{}-Pluralallomorph, sodass die Dativ-Plural-Form in allen Paradigmen (mit Ausnahme der \textit{s}{}-Plural-Klasse) ein uniformes Nasalsuffix aufweist (vgl. \citealt[91]{Kürschner2008a}). Nur die schwachen Maskulina weisen ein Flexiv in allen obliquen Kasus des Singularparadigmas auf (Nom.Sg. \textit{Bär} -- Gen./Akk./Dat.Sg. \textit{Bären} -- Pl. \textit{Bären}). Allerdings besteht auch hier synchron eine Tendenz zur Deflexion oder zumindest zum sprachlichen Zweifelsfall (Akk./Dat.Sg. \textit{Bären} > \textit{Bär,} Gen.Sg. \textit{Bärs} neben \textit{Bären(s)}), und zwar fast ausschließlich bei Maskulina, die nicht auf Schwa-Reduktionssilbe enden (\citealt{Thieroff2003}, vgl. \citealt[212]{DammelGillmann2014}). In der gemischten Deklination ist Kasusmarkierung im Akk./Dat.Sg. abgebaut, die gemischten Maskulina und Neutra flektieren im Gen.Sg. mit \textit{s}{}-Suffix (ausführlicher \citealt[§36--37]{Paul1968}, \citealt{Ronneberger-Sibold2018}). Im Dat.Sg. ist das Schwa-Suffix der starken Neutra und Maskulina (Dat.Sg. \textit{dem Mann-e}, \textit{im Haus-e}) weitestgehend geschwunden. Der Gen.Sg. wird bei starken Maskulina und Neutra durch das Suffix -\textit{(e)s} markiert und ist im Neuhochdeutschen neben der Pluralallomorphie das einzige Merkmal der Deklinationsklassenexponenz. \citet{AckermannZimmer2017} zeigen indes, dass bei „peripheren Substantiven“ (Fremd- und Kurzwörter, bestimmte Eigennamentypen) auch hier eine Tendenz zur Deflexion besteht, die funktional durch morphologische Schemakonstanz, d.\,h. durch ein Streben nach einer Schonung des Wortkörpers, erklärt werden kann (z.\,B. \textit{am Fuße des Himalaya(s)}, \textit{Literatur des Barock(s)}). Die Feminina der starken und gemischten Deklination weisen in keiner Klasse ein Kasusflexiv im Singular auf, Dat.Pl. ist nur bei den starken Feminina durch \textit{n}{}-Suffix markiert. Somit ergeben sich im nhd. Deklinationssystem Synkretismen des Typs Nom./Akk./Dat. im Singularparadigma aller starken Klassen, im Plural weisen alle Klassen Nom./Akk./Gen.-Synkretismus auf (vgl. \citealt[95]{Kürschner2008a}). Infolgedessen trägt der Artikel die „‚Hauptlast‘ der Kasusdifferenzierung“ (\citealt[140]{Wiese2000}, vgl. \sectref{sec:4.3}).

\begin{sloppypar}
\tabref{tab:11} bietet einen Überblick der nhd. Deklinationsklassen und ihrer Distribution in den drei Genera. In dieser Einteilung sind Pluralallomorphe danach differenziert, ob sie als rein additive Verfahren oder in Kombination mit dem stammaffizierenden Verfahren Umlaut realisiert werden. Die Verteilung der besetzten Zellen veranschaulicht, dass es im Neuhochdeutschen eine Schranke [±Femininum] gibt. Die Spezifik der fem. Deklination besteht dabei nicht in einem genustypischen Pluralmarkierungsverfahren, sondern in der Nullmarkierung des Gen.Sg., die keine der mask. oder neutr. Klassen aufweist.
\end{sloppypar}

\begin{table}
\small
\begin{tabularx}{\textwidth}{QQQQQQQ}
\lsptoprule
DK Gen.Sg.\slash Nom.Pl. & \multicolumn{2}{c}{Feminina} & \multicolumn{2}{c}{Maskulina} & \multicolumn{2}{c}{Neutra}\\\midrule
\textit{(e)n} / \textbf{\textit{(e)n}} & \multicolumn{2}{c}{} & \textit{Bär -- Bär}\textbf{\textit{en}} & \textit{Affe -- Affe}\textbf{\textit{n}} & \multicolumn{2}{c}{}\\
\midrule
\textit{s} / \textbf{\textit{(e)n}} & \multicolumn{2}{c}{} & \textit{See -- See}\textbf{\textit{n}} &  & {\textit{Ohr -- Ohr}\textbf{\textit{en}}} & \textit{Auge -- Auge}\textbf{\textit{n}}\\
\tablevspace
∅ / \textbf{\textit{(e)n}} & {\textit{Frau -- Frau}\textbf{\textit{en}}} & \textit{Gabe -- Gabe}\textbf{\textit{n}} & \multicolumn{2}{c}{} & \multicolumn{2}{c}{}\\
\midrule
∅ / \textbf{UL+\textit{e}} & {\textit{Wurst -- W}\textbf{\textit{ü}}\textit{rst}\textbf{\textit{e}}} &  & \multicolumn{2}{c}{} & \multicolumn{2}{c}{}\\
\tablevspace
\textit{(e)s} / \textbf{UL+\textit{e}} & \multicolumn{2}{c}{} & \textit{Gast -- G}\textbf{\textit{ä}}\textit{st}\textbf{\textit{e}} &  & \multicolumn{2}{c}{}\\
\tablevspace
\textit{(e)s} / \textbf{\textit{e}} & \multicolumn{2}{c}{} & \textit{Tag -- Tag}\textbf{\textit{e}} &  & {\textit{Jahr -- Jahr}\textbf{\textit{e}}} & \\
\tablevspace
\textit{(e)s} / \textbf{UL+\textit{er}} & \multicolumn{2}{c}{} & \textit{Mann --
M}\textbf{\textit{ä}}\textit{nn}\textbf{\textit{er}} &  & {\textit{Lamm --
L}\textbf{\textit{ä}}\textit{mm}\textbf{\textit{er}}} & \\
\tablevspace
\textit{(e)s} / \textbf{\textit{er}} & \multicolumn{2}{c}{} & \textit{Leib -- Leib}\textbf{\textit{er}} &  & {\textit{Licht -- Licht}\textbf{\textit{er}}} & \\
\tablevspace
\textit{s} / \textbf{UL} & \multicolumn{2}{c}{} &  & \textit{Apfel --} \textbf{\textit{Ä}}\textit{pfel} & \multicolumn{2}{c}{}\\
\tablevspace
\textit{s} / \textbf{∅} & \multicolumn{2}{c}{} &  & \textit{Finger -- Finger} &  & \textit{Messer -- Messer}\\
\midrule
\textit{s} / \textbf{\textit{s}} & \multicolumn{2}{c}{} & \textit{Akku -- Akku}\textbf{\textit{s}} &  & {\textit{Gnu -- Gnu}\textbf{\textit{s}}} & \\
\tablevspace
∅ / \textbf{\textit{s}} & {\textit{Pizza -- Pizza}\textbf{\textit{s}}} &  & \multicolumn{2}{c}{} & \multicolumn{2}{c}{}\\
\lspbottomrule
\end{tabularx}
\caption{Übersicht der nhd. Deklinationsklassen (DK) mit Beispielen (Sg. -- Pl.) nach Genera aufgeteilt (in den Genusspalten differenziert ist die prosodische Struktur des Stammes: links Einsilber bzw. Stämme mit betonter Finalsilbe, rechts Zweisilber auf Reduktionssilbe) und Pluralmarkierungstypen differenziert (additiv, Null, UL, additiv+UL)}
\label{tab:11}
\end{table}

Außerdem werden in 	\tabref{tab:11} additives Schwa-Suffix und Nullmarkierung separat aufgeführt, die in \tabref{tab:9} noch zusammengefasst waren, um die lineare Entwicklung der Klassen aufzuzeigen. Diese reduzierte Form der Deklinationsklasseneinteilung spiegelt wider, dass beide Allomorphe (ebenso wie die Allomorphe -\textit{en} und -\textit{n}) im nhd. Flexionssystem komplementär verteilt sind. Die Distribution von Schwa-Suffix und Null, von -\textit{en} und -\textit{n} ist im Neuhochdeutschen durch ein outputorientiertes Prinzip gesteuert: Pluralformen sind mindestens zweisilbig, trochäisch und enden auf Reduktionssilbe (vgl. \citealt[71]{Bittner1994}, \citealt[600--601]{DammelEtAl2010}, \citealt[479--480]{Neef2000}). Die Outputstruktur betonte Silbe+Schwa-Silbe als „minimum structure for plural nouns“ \citep[62]{Wiese1996} wird entweder durch ein silbisches Suffix geschaffen (\textit{Frau-en}, \textit{Tag-e}) oder durch ein unsilbisches Suffix oder Null gewahrt (\textit{Auge-n}, \textit{Finger}). \citet{Wiese2009} argumentiert sogar, dass Schwa keinen Suffix-Status hat und keine morphologische Einheit darstellt, sondern allein durch die prosodische Regel eines Reduktionssilbenplurals gesteuert ist („There is no need to assume an additional suffix \textit{{}-e}; prosody does the job“, \citealt[144]{Wiese2009}).

Unterhalb des prosodischen Prinzips eines Reduktionssilbenplurals steuert Genus die Distribution der Allomorphe. Nur zweisilbige Maskulina und Neutra auf -\textit{el}, -\textit{en,} {}-\textit{er} nehmen Nullplural an (Typ mask. \textit{Finger}, neutr. \textit{Messer}), reiner Umlautplural ist ein Spezifikum der Maskulina (Typ \textit{Äpfel}).\footnote{Die einzigen Feminina, die reinen Umlautplural aufweisen, sind die Verwandtschaftsbezeichnungen \textit{Mutter} und \textit{Tochter}.} Zweisilbige Feminina auf Reduktionssilbe markieren den Plural dagegen immer additiv durch unsilbisches -\textit{n} (\textit{Gabe-n}, \textit{Gabel-n}, \textit{Kammer-n}). Silbisches -\textit{en} erscheint bei einsilbigen Singularformen oder bei betonter Finalsilbe (z.\,B. \textit{Idee}, \textit{Figur}, \textit{Fabrik}, vgl. \citealt[107]{Wiese1996}). Diese Klasse mit \textit{(e)n}{}-Allomorph ist mit einer Typenfrequenz von \mbox{97\,\%} die prototypische Klasse der Feminina \citep[46]{Pavlov1995}. Die starke fem. Klasse UL+\textit{e} umfasst synchron nur noch 40 einsilbige Mitglieder, daneben markieren Derivationen auf -\textit{sal} (\textit{Trübsal}) und -\textit{nis} (\textit{Finsternis}) den Plural mit Schwa-Suffix \citep[124]{Köpcke1993}.

Silbische Pluralmarkierungsverfahren, die nur Maskulina und Neutra (also [$-$femininum]) nutzen, sind Schwa-Suffix ohne Umlautalternanz bei den Maskulina (mit Ausnahme der oben genannten fem. Derivationen auf -\textit{sal},-\textit{nis}) sowie (UL+)\textit{er}{}-Suffix bei Maskulina und Neutra (vgl. \citealt[138]{Wiese2009}). Neutra haben dabei kein „exklusives Flexionsverhalten, nur ein inklusives“ \citep[299]{Nübling2008}, da sie -- anders als Maskulina -- die Verfahren UL+\textit{e} und reinen Umlautplural nicht nutzen (mit der Ausnahme von \textit{Floß} und \textit{Kloster}). Maskulina weisen damit das ausdifferenzierteste Inventar an Pluralmarkierungsverfahren auf, Umlaut ist hier zudem stärker funktionalisiert und -- anders als bei Feminina und Neutra -- „selbstständig distinktiv“ (\citealt[200--201]{Ronneberger-Sibold1990}; vgl. \citealt[29]{Nübling2013}). Insgesamt weisen die Verfahren (UL)+\textit{e}, (UL)+\textit{er} sowie (UL)-∅ unterschiedliche „Grade der (Un-)Abhängigkeit von Umlaut und Flexiv“ \citep[66]{Dammel2018} auf. Nicht selbstständig ist der Umlaut in Kombination mit dem \textit{er}{}-Suffix, da die Vokalalternation bei velarem Stammvokal obligatorisch eintritt. Umgekehrt ist eine Funktionalisierung des \textit{er}{}-Suffixes ohne Umlaut ein „Loch im Pluralsystem“ \citep[69]{Dammel2018} des Neuhochdeutschen; in den Dialekten des Deutschen ist zumindest teilweise eine Funktionalisierung des \textit{er}{}-Suffixes eingetreten (vgl. \sectref{sec:4.2.1}).

Die prosodische Struktur von Pluralformen aus betonter und Reduktionssilbe ist beim \textit{s}{}-Plural aufgehoben. Das unsilbische -\textit{s} bewahrt die Struktur des Stammes, v.\,a. im Fremdwortbereich ist es „sowohl ein Notplural als auch ein Transparenzplural“ (\citealt{Wegener2003}, vgl. \citealt[151]{AckermannZimmer2017}, \citealt[152--156]{Köpcke1993}, \citealt{Wegener1999}, \citealt[137--138]{Wiese1996}). Wenngleich der \textit{s}{}-Plural „unabhängig von der alten Klassenstruktur“ \citep[94]{Kürschner2008a} hinzukommt, wird in den \textit{s}{}-Plural-Klassen die vorhandene Genusschranke fortgesetzt: Feminina weisen kein Genitiv-Singular-Flexiv auf, Maskulina und Neutra nutzen das \textit{s}{}-Suffix (vgl. \citealt[299]{Nübling2008}).

\section{Die Entwicklung in den Dialekten des Deutschen}
\label{sec:4.2}
Numerusprofilierung und Kasusnivellierung sind auch in den deutschen Dialekten die zentralen Tendenzen nominalmorphologischen Wandels; in den einzelnen Varietäten finden sich aber spezifische Ausgleichsprozesse und Entwicklungen im Bereich der Numerus- und Kasusmorphologie. Den Ausgangspunkt der rezenten dialektalen Deklinationssysteme bildet das mhd. System mit den Pluralallomorphen (UL+)-\textit{e}, (UL+)-\textit{er}, -\textit{(e)n}, Null und Umlaut, da dieses diachrone „Archisystem“ (mit Ausnahme der höchstalem. Walser Dialekte) in allen deutschen Dialekten historisch zu finden war \citep[1196]{Dingeldein1983}. Es sind phonologische Prozesse wie die Schwa-Apokope, die Elision des finalen /n/ in der Reduktionssilbe -\textit{en}\footnote{Der phonologische Prozess hinter vokalisch realisierten Reduktionssilben -\textit{en} wird in der Forschungsliteratur auch als \textit{n}{}-Apokope oder als Vokalisierung (etwa bei \citealt{Rowley1997}) gefasst. Im Folgenden wird der Terminus Apokope in der engeren Definition als Schwa-Apokope verwendet (vgl. \citealt[132]{Birkenes2014}). Vokalische Realisierung der Reduktionssilbe -\textit{en} wird als Tilgung (Elision) klassifiziert (vgl. \citealt{Dingeldein1983}, \citealt{Kranzmayer1956}, \citealt{Schirmunski1962}, \citealt{SMF4} sowie \sectref{sec:7.1.2.3.2} zur Elision des wortfinalen /n/ in betonter Silbe).} sowie weitere Prozesse im Vokalismus und Konsonantismus, die einerseits zu dialektspezifischen Distributionen der additiven Marker und anderseits zu dialektspezifischen morphophonologischen Verfahren führen. Daraus ergibt sich Variation in der ausdrucksseitigen Realisierung der Pluralinformation: (1) in Bezug auf ein diachrones Archisystem, (2) in Bezug auf ein synchrones Diasystem und (3) im interdialektalen Vergleich in der horizontalen (d.\,h. arealen) Dimension (vgl. \citealt[806]{Rabanus2010}). Für die horizontale Variation von Markierungsverfahren (additiv, stammaffizierend, subtraktiv, Null) und der konkreten Marker in synchronen Dialektsystemen wird in der Literatur teilweise das Konzept der Heteromorphie vorgeschlagen, um die „Relation von Morphemalternanten \textit{unterschiedlicher} Systeme“ in Abgrenzung zur Allomorphie als „Morphemalternanten in \textit{einem} System“ adäquat zu fassen (\citealt[129]{Koch2006}, Hervorhebung im Original, vgl. \citealt[130--131]{Girnth2006}). Die folgende Zusammenschau zur Numerus- und Kasusflexion in den deutschen Dialekten basiert hauptsächlich auf den Darstellungen von \citet{Dingeldein1983} und \citeauthor{Schirmunski1962} (\citeyear{Schirmunski1962}, vgl. auch \citealt[622--623]{Rabanus2019}).

\subsection{Numerus}
\label{sec:4.2.1}
Das Schwa-Suffix der mask. \textit{a}{}- und \textit{i}{}-Deklination entfällt infolge der Apokope in den hd. Dialekten (mit Ausnahme des Ostmitteldeutschen) und in weiten Teilen des Niederdeutschen (zur Arealität der Apokope vgl. \citealt[50--56]{Birkenes2014}). Teilweise führt dies zu Nullpluralen, teilweise erscheinen stammaffizierende Markierungen, die lautgesetzliche Alternationen von Singular- und Pluralstämmen vor Apokope des Schwa-Suffixes konservieren (vgl. \citealt[63--64]{Nübling2005}, \citealt[624--625]{Rabanus2019}, \citealt[183]{Seiler2008}):

\begin{itemize}
\item Alternationen zwischen Kurzvokal im Singular und Langvokal im Plural als Folge von Vokaldehnung in offener Silbe bei zweisilbigen Pluralformen und erhaltener Kürze in geschlossener Silbe der einsilbigen Singularformen im Niederdeutschen: nd. \textit{dax} -- \textit{dǭȝ̣} ‚Tag‘ \citep[187]{Schirmunski1962}
\item Alternationen der Akzentkontur in den mittelfränkischen Dialekten („rheinische Schärfung“): mittelfr. \textit{štēn} -- \textit{štē:n} ‚Stein‘ \citep[174]{Schirmunski1962}
\item Alternationen zwischen Langvokal im Singular und Kurzvokal im Plural infolge von Einsilberdehnung und erhaltener Kürze bei zweisilbigen Pluralformen im Ofr., im Bair. korrelieren zusätzlich Vokallänge und Lenis-Fortis-Obstruent: ofr. \textit{flēk} -- \textit{flek} ‚Fleck‘, bair. \textit{flēg} -- \textit{flek} ‚Fleck‘ \citep[417--418]{Schirmunski1962}
\item Alternation zwischen auslautendem Plosiv im Singular und spirantisierter Variante oder Rho\-tazismus im historischen Inlaut der Pluralform: hess. \textit{bęrg} -- \textit{bęrx} ‚Berg‘, hess. \textit{šret} -- \textit{šrer} ‚Schritt‘ \citep[1198]{Dingeldein1983}
\item Subtraktive Pluralformen durch Konsonantenassimilation in intervokalischer Stellung im Nieder- und Mitteldeutschen, im Südwesten Thüringens variiert zudem die Vokalqualität: hess. \textit{hond} -- \textit{hon} ‚Hund‘, thüring. \textit{hoind} -- \textit{hon} ‚Hund‘ \citep[417--418]{Schirmunski1962}.
\end{itemize}

\begin{sloppypar}
Diese Auswahl stammaffizierender Markierungen zeigt, dass dieser Pluralmarkierungstypus innerparadigmatische Alternationen des Vokalismus (Qualität und Quantität) und des stammauslautenden Konsonantismus umfasst und diese teilweise auch kombiniert erscheinen. In \citegen[64]{Nübling2005} Worten liefert „die Phonologie gewissermaßen die Mutanten“ und die Morphologie nimmt „die Selektion“ vor. Offenbleiben muss dabei vorerst, inwiefern die einzelnen Muster als morphologische Marker funktionalisiert und damit auch produktiv sind, hier also tatsächlich eine Selektion durch Morphologie erfolgt ist. Voraussetzung für die Morphologisierung ist, dass die Alternation nicht mehr phonologisch bedingt, sondern „ausschließlich auf Grund von grammatischen Kontextbedingungen“ \citep[57]{Wurzel1982} funktionalisiert ist (vgl. \citealt[195]{Seiler2008}). Aufgabe einer Untersuchung von dialektalen Flexionssystemen wäre es damit nicht nur, ein Inventar der konkreten Realisierungsmuster stammaffizierender Markierung aufzustellen, sondern idealiter auch den Status als morphologischer Marker oder als phonologisch bedingte Alternation festzustellen. So kann \citet{Birkenes2014} in seiner Untersuchung dialektologisch-grammatischer Literatur zeigen, dass subtraktive Plurale lexikalisierte morphophonologische Alternationen darstellen, dass aber einzelne, höher frequente Subtraktionstypen (/nd/- und /Vg/-Abfolgen) teilweise produktiv werden können (vgl. \citealt[94--96, 197--199]{Birkenes2014}). Entscheidend für Aussagen zur Produktivität und für Vorhersagen zu möglichem Sprachwandel ist hier die Datenlage (vgl. \citealt[208]{Birkenes2014}).
\end{sloppypar}

Dass der Umlaut morphologisiert und vor allem in den obd. Dialekten ein produktiver Pluralmarker ist, zeigen die analogen Umlautplurale, die sich infolge der Schwa-Apokope bei den Maskulina finden. \citet[418]{Schirmunski1962} zufolge besteht die Tendenz eines Umlautplurals bei fast allen Maskulina mit umlautfähigem (d.\,h. velarem) Stammvokal, laut \citet[1086]{Lüssy1983} ist er „zum allgemeinsten Bildungsmittel des Plurals“ in den hd. Dialekten geworden. Analoge Umlautplurale finden sich im Oberdeutschen regelmäßig bei zweisilbigen Maskulina auf -\textit{el}, -\textit{en}, {}-\textit{er}, darunter auch bei historischen \textit{n}{}-Stämme wie mhd. \textit{name} > obd. \textit{näme} ‚Namen‘, mhd. \textit{boge} > \textit{bögen} ‚Bogen‘, daneben \textit{höbel} ‚Hobel‘, \textit{hämmer} ‚Hammer‘ u.\,a. Als Folge der Schwa-Apokope ist der Umlautplural in den obd. Dialekten -- anders als im Standard und den omd. Dialekten -- zudem auch bei Einsilbern (der historischen mask. und fem. \textit{i}{}-Deklination) als rein stammaffizierendes Verfahren produktiv: ofr. \textit{hunt} -- \textit{hint} ‚Hund‘, südbair. \textit{roux} -- \textit{röix} ‚Rauch‘.\largerpage[2]

Dass Produktivität das Ergebnis einer dialektspezifischen Interaktion von Phonologie und Morphologie ist, zeigen exemplarisch die Studien von \citet{Nübling2006} und \citet{DammelDenkler2017} zum Flexionssystem des Luxemburgischen, das sich auf Basis der moselfränkischen Dialekte entwickelt und den Status einer Nationalsprache hat (vgl. \citealt{Gilles2019}). Charakteristisch für das Luxemburgische ist die „Demotivierung und Arbitrarisierung“ \citep[118]{Nübling2006} der Umlautrelation zwischen Basis- und Umlautvokal. Während im Neuhochdeutschen eine transparente und vorhersagbare Relation zwischen Velar- und Palatalvokal besteht, ist diese 1:1-Relation im Luxemburgischen hinsichtlich Vokalqualität und -quantität durchbrochen (vgl. \citealt[116--119]{Nübling2006}). Damit ähnelt der Umlaut einem Ablautverfahren, das durch die Morphologie „akzeptiert und nicht durch Analogie beseitigt“ \citep[119]{Nübling2006} wurde.{\interfootnotelinepenalty=10000\footnote{Im Moselfränkischen stellt \citet[135]{Girnth2006} indes einen „Rückgang der ‚Arbitrarisierung‘ des Umlauts“ fest; im moselfränkischen Regiolekt wird der Umlaut (anders als im Basisdialekt) „wieder vorhersagbar“.}} Umlautplurale sind im Luxemburgischen weiterhin produktiv und markieren beispielsweise Anglizismen (\textit{Clubb} -- \textit{Clibb} ‚Club‘) und Gallizismen (\textit{Tirang} -- \textit{Tiräng} ‚Schublade‘).

Ein zweites produktives Verfahren stellen im Luxemburgischen \textit{er}{}-Plurale dar. Bemerkenswert ist hier, dass es -- anders als im Neuhochdeutschen -- eine Auflösung der Kopplung von UL+\textit{er} gegeben zu haben scheint (\citealt[105]{DammelDenkler2017}). Historisch war UL+\textit{er} ein produktives Verfahren, doch der Wechsel zahlreicher nicht-umlautfähiger Maskulina und Neutra der historischen \textit{a}{}- und \textit{i}{}-Deklination in das Verfahren führte zu einer „kritischen Masse“ (\citealt[107]{DammelDenkler2017}) ohne Vokalalternanz. Dies und die Arbitrarisierung der Umlautrelationen machten die Kopplung von Umlaut und \textit{er}{}-Suffix unvorhersagbar und führten zu einer Dissoziation von Umlaut und -\textit{er}. Ganz im Sinne eines Dialektlabors zeigt der Vergleich des Luxemburgischen und des Westfälischen, wo \textit{er}{}-Plurale im Vergleich zu anderen additiven Verfahren (-\textit{s} und -\textit{e}) nur wenig produktiv sind, welcher Faktor die notwendige Bedingung für die höhere Produktivität des \textit{er}{}-Plurals war: die Schwa-Apokope, die im Luxemburgischen durchgeführt wurde, im Westfälischen aber nicht (vgl. \citealt[108]{DammelDenkler2017}, vgl. \citealt{Dammel2018}).

Insgesamt ergeben sich in den deutschen Dialekten im Bereich der additiven Pluralmarkierung folgende Tendenzen zu Distribution und Produktivität:\largerpage

\begin{itemize}
\item Das \textit{\textit{er}}\textit{{}-Suffix} ist (in Kombination mit Umlaut) das spezifische Pluralbildungsverfahren der Neutra. Vor allem in den hd. Dialekten (wenn auch weniger im Ostmitteldeutschen) hat hier eine -- im Vergleich zum Standard -- stärkere analoge Ausdehnung auf Neutra der historischen \textit{a}{}-Deklination stattgefunden (allen voran bei \textit{better} ‚Betten‘, \textit{hemder} ‚Hemden‘, \textit{stücker} ‚Stücken‘), im Niederdeutschen ist analoges -\textit{er} dagegen weniger belegt. Daneben findet sich analoges -\textit{er} bei Maskulina im Bair. und Ostmitteldeutschen, z.\,b. thür. \textit{šdegər} ‚Stöcke‘, \textit{baimər} ‚Bäume‘ \citep[420]{Schirmunski1962}.
\item Das \textit{\textit{(e)n}}\textit{{}-Suffix} ist auch im Dialekt das spezifische Markierungsverfahren der Feminina. Im Westmitteldeutschen und Westoberdeutschen, teilweise auch im Schlesischen ist Elision des finalen /n/ erfolgt, weshalb hier das Heteromorph Schwa-Suffix erscheint. In weiten Teilen des Oberdeutschen, im Rheinfränkischen und im Westniederdeutschen ist das Flexiv der obliquen Kasus auf den Nom.Sg. ausgedehnt worden. Infolge dieses morphologischen Ausgleichs erscheint Nullplural: westfäl. Sg./Pl. \textit{brügən} ‚Brücke‘, ober\-alem. Sg./Pl. \textit{aššä} ‚Asche‘ \citep[431]{Schirmunski1962}. In den oobd. Dialekten gibt es teilweise „sekundäre Pluralkennzeichen“ \citep[418]{Schirmunski1962} durch Umlaut (ofr. \textit{brukn} -- \textit{brikn} ‚Brücke‘) oder als sogenannte potenzierte Endung: bair. \textit{tsūŋɒ} -- \textit{tsūŋɒn} ‚Zunge‘ \citep[431]{Schirmunski1962}. Bei Maskulina und Neutra findet sich Nasalsuffix (bzw. das Heteromorph Schwa) in der schwachen Deklination, in den hd. Dialekten außerdem bei Ein- und Zweisilbern mit Auslaut auf Liquid (im Bair. insbesondere bei Diminutiva auf Diminutivsuffix -\textit{el}): hess. \textit{jōrn} ‚Jahre‘, hess. \textit{šbīlə} ‚Spiele‘, nordbair. \textit{telərn} ‚Teller‘, bair. \textit{šdaŋəln} ‚Stängchen‘ \citep[1199]{Dingeldein1983}. Im Niederdeutschen erscheint Nasalsuffix (neben -\textit{s}) als „Ersatz“ für apokopiertes Schwa-Suffix: brandenb. \textit{apəln} ‚Äpfel‘ \citep[1200]{Dingeldein1983}.
\item Das \textit{\textit{s}}\textit{{}-Suffix} ist im Niederdeutschen ausgesprochen frequent und produktiv: bei Zweisilbern auf -\textit{el}, -\textit{en}, -\textit{er} (hier ist es auch an der nd. Grenze, vor allem in den omd. Dialekten belegt), bei Diminutiva auf -\textit{ken}, bei Einsilbern (hierin v.\,a. Personenbezeichnungen) sowie bei einigen Feminina (insbesondere Verwandtschaftsbezeichnungen), z.\,B. westfäl. \textit{appəls} ‚Äpfel‘, \textit{lākəns} ‚Laken‘, \textit{knexts} ‚Knechte‘, mecklenburg. \textit{dērns} ‚Dirnen‘ \citep[423--424]{Schirmunski1962}. Bei historisch schwachen Maskulina findet sich v.\,a. im Westfälischen das „hybride Morph“ \citep[1200]{Dingeldein1983} -\textit{ens}: westfäl. \textit{ochse} -- \textit{ochsens} ‚Ochse‘, \textit{hāzə} -- \textit{hāzəns} ‚Hase‘ \citep[424]{Schirmunski1962}. In anderen nd. Dialekten ist -\textit{ens} nur nach Ausweitung des Nasalsuffixes im Nom.Sg. belegt.
\end{itemize}

Mit dem genusunabhängigen \textit{s}{}-Plural weist das Niederdeutsche ein produktives additives Verfahren auf, Nullplurale sind in diesem Dialektraum in der Folge weniger frequent als in den hd. Dialekten. Insgesamt ist der Status des Nullplurals in den deutschen Dialekten \citet[64]{Nübling2005} zufolge „erklärungsbedürftig“. Einerseits werden Nullplurale im Zuge der Numerusprofilierung diachron auch in Dialekten abgebaut, doch gleichzeitig entstehen Nullmarkierungen als Folge von Apokope und von innerparadigmatischem Ausgleich, wenn eine phonologisch bedingte Alternation abgebaut und eben nicht von der Morphologie „selektiert“ wird. Daneben erscheinen Nullplurale durch Übertragung eines Flexivs aus den obliquen Kasus in den Nom.Sg. (etwa bei den historischen \textit{n}{}-Feminina) und dem Ausbleiben einer „sekundären“ Pluralmarkierung. Nullplurale können auch das Ergebnis von jüngerem Dialektwandel sein. So berichtet beispielsweise \citet[§34k8]{Kranzmayer1956} von einem Abbau von Vokalquantitätskontrasten in der jüngeren Wiener Generation zugunsten des Kurzvokals, in der Folge erscheinen synkretische Singular- und Pluralformen. \citet{Girnth2006} stellt für das Moselfränkische einen Dialektwandel von den subtraktiven Pluralformen der älteren Sprechergeneration hin zu modulatorischen Markierungen und Nullmarkierungen bei der jüngeren Generation fest, wobei Nullplurale das sich am schnellsten verbreitende Verfahren des Moselfränkischen sind.

\subsection{Kasus}
\label{sec:4.2.2}
Im Bereich der Kasusflexion ist der Abbau der formalen Markierung am Substantiv noch weiter vorangeschritten. Kasusnivellierung ist in den Dialekten das Ergebnis von Reduktionsprozessen der Kasusflexive und von Synkretismen infolge des Zusammenfalls einzelner Kasus (vgl. \citealt[623]{Rabanus2019}, \citealt[432]{Schirmunski1962}). Die deutschen Dialekte weisen nur noch Zwei- oder Drei-Kasus-Systeme auf, da der adnominale und adverbale Genitiv in den Dialekten weitestehend geschwunden ist (ausführlicher Koß 1983, \citealt[433--437]{Schirmunski1962} sowie \citealt{Shrier1965}). Deklinationsklasse manifestiert sich nur noch in der Pluralallomorphie, es gilt „Flexionsklasse = Pluralklasse“ \citep[313]{Nübling2008}.

Bei den schwachen Maskulina und Neutra ist das \textit{en}{}-Suffix in den obliquen Kasus des Singulars mehrheitlich im Niederdeutschen und in Teilen des Hochdeutschen erhalten; wo es abgebaut ist, steht eine synkretische Singularform einer mit -\textit{en} markierten Pluralform gegenüber \citep[440]{Schirmunski1962}. Das \textit{e}{}-Suffix des Dat.Sg. der starken Maskulina und Neutra ist im Ostmitteldeutschen und in Teilen des Niederdeutschen bewahrt, in den apokopierenden Dialekten ist die formale Markierung des Dat.Sg. am Substantiv geschwunden. \citet{Lindgren1953} stellt in seiner Untersuchung der funktionalen Faktoren der Apokope fest, dass Schwa-Suffix in der Funktion als Kasusmarker eher abgebaut wurde als Schwa in der Funktion als Numerus- oder Genusmarker. Morphologie hemmt hier -- ganz im Sinne von Numerusprofilierung und Kasusnivellierung -- den phonologischen Prozess der Apokope (\citealt[214]{Lindgren1953}, vgl. \citealt[51--52 und 134]{Birkenes2014}, \citealt[159--160]{Schirmunski1962}). Dieselben phonologischen Prozesse, die zu morphophonologischen Alternationen zwischen einsilbigen Singular- und zweisilbigen Pluralformen führen, sind auch bei den (vor Apokope) zweisilbigen Dativ-Singular-Formen zu beobachten, darunter Vokalquantitätskontraste und mittelfr. Tonhöhenakzente (ripuar. Nom.Sg. \textit{velt} -- Dat.Sg. \textit{om vē:l} ‚Feld‘) und subtraktive Formen (südthür. Nom.Sg. \textit{lānd} -- Dat.Sg. \textit{lan} ‚Land‘, \citealt{Schirmunski1962}: 438). Weitgehend erhalten ist das uniforme Nasalsuffix des Dat.Pl., das regional durch elidiertes finales /n/ vokalisch realisiert wird. In Teilen des Niederdeutschen sind Dativ und Akkusativ zusammengefallen, eine distinkte Dativ-Plural-Form ist nur relikthaft in einigen präpositionalen Konstruktionen erhalten. Auch im Bair. gibt es teilweise einen Zusammenfall von Dativ und Akkusativ, doch wird hier ein sogenannter „Kraftdativ“ durch eine „potenziertes“ Suffix -\textit{αn} markiert (mittelbair. Nom.Sg./Pl. \textit{haufm} -- Dat.Pl. \textit{haufmɒn} ‚Haufen‘), im Nordbair. (daneben auch im Ostmitteldeutschen) konkurrieren einfaches Nasalsuffix und „potenzierter“ Dat.Pl.: Nom./Akk.Pl. \textit{tōx} -- Dat.Pl. \textit{tōŋ} oder \textit{tōŋɒ̤\textsuperscript{n}} ‚Tag‘ \citep[441--442]{Schirmunski1962}. In Teilen des Oberdeutschen (im Elsässischen, Alemannischen und teilweise im Bair.) werden Dativ-Nominalphrasen mit einer einleitenden Präposition \textit{in} oder \textit{an}, nach \citet{Seiler2003} einem „Dativmarker“ realisiert: \textit{in/an der Mutter sagen} (vgl. \citealt[438--440]{Schirmunski1962}, \citealt[436]{Shrier1965}).

Damit scheint es trotz einer generellen Tendenz zur Kasusnivellierung in einzelnen Dialekten Strategien zur Disambiguierung von direktem und indirektem Objekt zu geben: in Form einer morphologischen (d.\,h. flexivischen) Markierung als „potenzierte“ Dativ-Plural-Endung im Bair. und teilweise im Ostmitteldeutschen oder morphosyntaktisch als präpositionale Dativmarkierung in Teilen des Oberdeutschen. Kasusmarkierung ist insgesamt stärker in die Domäne der Syntax gerückt. \citet[80]{Nübling2005} weist daher zu Recht die Nominalphrase als Forschungsdesiderat aus, da noch weitestgehend offen ist, inwiefern die Substantivbegleiter den Kategorienabbau am Substantiv tatsächlich kompensieren. \citegen{Shrier1965} Studie zeigt, dass Konstellationen und Grade von Synkretismen und distinkten Formen in Artikel- und Pronominalformen im Raum variieren, doch braucht es hier Detailuntersuchungen zur Funktionalität der einzelnen Einheiten im jeweiligen Flexionssystem. Einen Beleg für die „Tragfähigkeit“ reduzierter Flexionssysteme bietet \citet{Harnisch1984} mit einem Neutralisierungsphänomen zur Genusdistinktion aus Nordostbayern. In der starken Adjektivflexion markieren im Nordbair. die Suffixe -\textit{a} Maskulinum und -\textit{e} Femininum, im Ofr. markieren -\textit{e} Maskulinum und -\textit{a} Femininum, im ofr.-nordbair. Grenzgebiet erscheint eine „kontaminierte“, synkretische Form: -\textit{a} steht gleichermaßen für Maskulinum und Femininum (\textit{a blinda} ‚eine/ein blinde(r)‘). Die inverse Distribution der Suffixe im Ofr. und im Nordbair. führt im Übergangsgebiet zur „Neutralisierung einer -- man hätte gemeint -- unverzichtbaren Opposition“ \citep[89]{Harnisch1984}, doch zeigt dieser „Testfall“: „Existenter Synkretismus deutet darauf hin, daß der Abbau morphologischer Differenzierung ohne Schaden möglich war.“

\section{Numerus- und Kasuskodierung in Nominalphrase und Satzkontext}\label{sec:4.3}\largerpage
Eine Markierung von Numerus- und Kasusinformation erfolgt im Deutschen nicht nur am Substantiv selbst, sondern über Kongruenz auch im syntaktischen Kontext. In der Nominalphrase kongruieren Artikelformen und kongruenzfähige Attribute mit dem Substantiv in den Kategorien Genus, Kasus und Numerus, im Satzkontext kongruieren Subjektphrase und finites Verb in der Kategorie Numerus (und Person bei Pronomina). Kongruenz markiert damit einerseits die syntaktischen Beziehungen zwischen einzelnen Konstituenten, andererseits ist sie Hinweisgeber in der Dekodierung von Numerus- und Kasusinformation. \citet{Ronneberger-Sibold1994, Ronneberger-Sibold2010} fokussiert diesen funktionalen Aspekt der Kongruenz, indem sie das sogenannte klammernde Verfahren als performanzorientiertes Prinzip fasst, das ein wesentliches Merkmal der immanenten Typologie des Neuhochdeutschen ist. Linker und rechter Klammerrand sind demnach Grenzsignale, sodass „der Hörer/Leser aus dem Auftreten des ersten Signals mit sehr großer Wahrscheinlichkeit schließen kann, dass der betreffende Bestandteil erst dann beendet sein wird, wenn das passende zweite Signal in der Sprechkette erscheint“ \citep[722]{Ronneberger-Sibold2010}.

In der Nominalphrase sind die Artikel die wichtigsten Träger von morphosyntaktischer Information, insbesondere zu Kasus und Genus (vgl. \citealt[123]{Szczepaniak2010}, \citealt[140]{Wiese2000}). Infolge des Abbaus der Kasusmarkierung am Substantiv und synkretischer Formen im Paradigma erfolgt die Disambiguierung der Kasusinformation erst durch definite und indefinite Artikel: Nom.Sg. \textit{der/ein Hund} -- Akk.Sg. \textit{den/einen Hund} -- Dat.Sg. \textit{dem/einem Hund}. Die Pluralinformation wird nur bei Maskulina (\textit{der} -- \textit{die}) und Neutra (\textit{das} -- \textit{die}) durch den Definitartikel disambiguiert, da bei Feminina synkretische Artikelformen vorliegen (Nom./Akk. \textit{die} -- \textit{die}). Ambiguität in der Nominalphrase ist dabei der Normalfall, es ist der „dominierende Flexionstyp der deutschen mehrgliedrigen Nominalphrase“ \citep[117]{Ronneberger-Sibold1994}. Die Ambiguität des Artikels wird erst durch das Substantiv aufgelöst: Der Definitartikel \textit{der} ist Nom.Sg.mask. (\textit{der Mann}), Gen./Dat.Sg.fem. (\textit{der Frau}) und Gen.Pl. (\textit{der Frauen/Männer/Kinder}), erst in Kombination mit dem Substantiv \textit{Mann} ist die Kodierung der Nominalphrase eindeutig hinsichtlich Numerus (Sg.), Kasus (Nom.) und Genus (mask.). Flexivische Information wird hier „diskontinuierlich“ kodiert, indem erst die Kombination mehrerer ambiger Ausdrücke im Syntagma eine numerus-/kasus-/genus-eindeutige Lesart bietet (vgl. \citealt{Szczepaniak2010} zur Diachronie). \citet[982]{Werner1979} setzt hierfür sogar einen eigenen „diskontinuierenden“ Sprachtyp an und beschreibt das Deutsche hierin als „hochgradig diskontinuierend“.

\citegen[84]{Harnisch1984} Beleg von Genussynkretismus in der starken Adjektivflexion im ofr.-nordbair. Grenzgebiet (\teuthoo{A}{α} \teuthoo{blindA}{blindα} \teuthoo{mo.2}{mōͅ} ‚ein blinder Mann‘ -- \teuthoo{A}{α} \teuthoo{blindA}{blindα} \teuthoo{fra2}{frā} ‚eine blinde Frau‘, siehe \sectref{sec:4.2.2}) deutet darauf hin, dass die diskontinuierende Kodierung in der Nominalphrase mit einem höheren Grad an Synkretismen in Artikel- und Adjektivparadigmen zunimmt. Gleichzeitig ist zu fragen, inwiefern die Kodierung am Substantiv und in der Nominalphrase interagieren -- synchron im Sprachsystem und auch mit Blick auf die diachrone Entwicklung. Inwiefern werden einzelne flexionsmorphologische Phänomene wie Kasussynkretismen infolge des Abbaus von Kasusmarkierung am Substantiv oder Nullplurale im syntaktischen Kontext kompensiert? Und inwiefern kompensiert wiederum die Morphologie Synkretismen etwa von Artikelformen? In \sectref{sec:4.2.1} wurde für Feminina, die im Bair. in Folge innerparadigmatischen Ausgleichs das Nasalsuffix auch im Nom.Sg. aufweisen, eine „potenzierte“ Endung im Plural angeführt (Typ \textit{tsūŋɒ} -- \textit{tsūŋɒn} ‚Zunge‘, \citealt[431]{Schirmunski1962}). Der Numerussynkretismus in der Nominalphrase (\textit{die} -- \textit{die}) wird hier durch eine Kodierung der Pluralinformation am Substantiv aufgefangen.

\citegen{Shrier1965} Studie zeigt, dass es verschiedene Konstellationen und Ausprägungen von Synkretismen bei Artikel- und Pronominalformen im Raum gibt. Auch \citeauthor{Rabanus2008}‘ (\citeyear{Rabanus2008}) Untersuchung zu Synkretismen und morphologischen Dis\-tinktionen im „Minimalsatz“ (in Rabanus‘ Definition die syntaktische Einheit aus transitivem Verb und den pronominal realisierten, obligatorischen Ergänzungen) zeigt Strukturräume der Pronominal- und Verbalmorphologie in den hd. Dialekten auf. Gleichzeitig wird der Blick auf die Funktionalität von Sprachsystemen erweitert: Das Ergebnis von Sprachwandel besteht demnach nie in einer Unterschreitung eines Minimums an morphologischen Distinktionen (nach Rabanus das „morphologische Minimum“), d.\,h. der Ausdruck der grammatischen Kategorien Genus, Kasus, Numerus und Person ist im Minimalsatz stets gewahrt.

\begin{sloppypar}
Um Forschungsfragen zur Schnittstelle von Morphologie und Syntax in den Dialekten systematisch untersuchen zu können, braucht es Sprachdaten, die nicht nur aus isolierten Wortformen, sondern auch aus größeren syntagmatischen Einheiten bestehen. Für das eigene Untersuchungsgebiet sind hier exemplarisch die Erhebungen des \textit{Bayerischen Sprachatlas} und hierin der Syntax-Teil im Band 7 des \textit{Sprachatlas von Mittelfranken} sowie Band 1 des \textit{Sprachatlas von Niederbayer}n zu nennen, in dem u.\,a. verschiedene morphosyntaktische Phänomene in der syntaktischen Einheit Nominalphrase behandelt werden (vgl. auch den Überblick in \citealt[40--43]{SchmidtEtAl2019}). Um aber morphosyntaktische Phänomene über die areale Dimension hinaus und aus Perspektive der Sprachverwendung untersuchen zu können, sind dialektale Gesprächs- oder auch Erhebungsdaten notwendig, die den semantisch-pragmatischen Kontext und die Informationsstruktur sys\-tematisch berücksichtigen und nach wie vor ein Desiderat darstellen.
\end{sloppypar}
