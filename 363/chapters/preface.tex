\addchap{\lsPrefaceTitle}
Bei diesem Buch handelt es sich um die überarbeitete Fassung meiner Dissertation, die im Mai 2021 an der Katholischen Universität Eichstätt-Ingolstadt angenommen wurde. An erster Stelle danke ich Sebastian Kürschner und Alexander Werth für ihre ausgezeichnete Betreuung. Ohne ihren Rat, ihre konstruktive Kritik und ihre Förderung wäre die Arbeit in dieser Form nicht möglich gewesen. Ein besonderer Dank gilt auch Mechthild Habermann, die meine Leidenschaft für Sprachwissenschaft im Studium überhaupt erst geweckt und mein linguistisches Denken geschult hat.


Alois Dicklberger, Monika Fritz-Scheuplein, Rosemarie Spannbauer-Pollmann, Karin Rädle und Elisabeth Wellner danke ich für die stets schnelle und unkomplizierte Bereitstellung der Fragebücher des \textit{Bayerischen Sprachatlas} und für die überaus nützlichen Hinweise zu ihrer Genese. Stellvertretend für alle Kolleginnen und Kollegen, die mir bei Kolloquien, Konferenzen oder Gastvorträgen wertvolle Anregungen und neue Ideen mit auf den Weg gegeben haben, danke ich Magnus Breder-Birkenes, Lars Bülow, Hanna Fischer, Thomas Fritz, Rüdiger Harnisch, Bettina Lindner-Bornemann, Simon Pröll, Anthony Rowley und Andrea Streckenbach. Der intensive Austausch mit ihnen hat dazu geführt, dass die Promotionszeit nicht nur arbeitsreich, sondern in erster Linie eine schöne und erfüllende Zeit war.


Den Reihenherausgeberinnen und -herausgebern danke ich für die Aufnahme meines Buches in die Reihe \textit{Open Germanic Linguistics}, Felix Kopecky und Se\-bas\-ti\-an Nordhoff für die hervorragende redaktionelle Betreuung und die aufwändige Anpassung der Teuthonista-Transkriptionen an Unicode. Ein weiterer und besonders herzlicher Dank gilt dem Vorstand der Internationalen Gesellschaft für Dialektologie des Deutschen, der meine Dissertation mit einem Nachwuchspreis für die beste Dissertation 2018/2022 ausgezeichnet hat. Diese Ehrung bedeutet mir auch deshalb so viel, weil die Kongresse der IGDD immer  Meilensteine im Entstehen meiner Dissertation waren.


Dank dieser Arbeit habe ich neben meiner ostmitteldeutschen Heimat eine zweite Heimat in der bayerischen Dialektologie gefunden. Allen, die mich auf diesem Weg begleitet und unterstützt haben, danke ich von ganzem Heren.\bigskip\\
\noindent Erlangen im Oktober 2022\hfill\hbox{Grit Nickel}
