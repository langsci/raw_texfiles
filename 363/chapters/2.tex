{\setpartpreamble{\bigskip\bigskip\bigskip\bigskip\noindent In diesem Teil erfolgt ein Überblick über den Stand der Forschung zur Flexionsmorphologie des Substantivs im Deutschen. Im Fokus der Darstellung stehen zwei Phänomenbereiche: (1) Deklinationsklassen als Klassifikationssystem und Phänomen „hinter“ der konkreten Formenbildung, (2) Markierungsstrategien in der Numerus- und Kasuskategorie im Neuhochdeutschen und in den deutschen Dialekten. Beide Bereiche werden in ihrer diachronen Entwicklung und aus typologischer Perspektive dargestellt und bilden die Grundlage der Untersuchung der oobd. Flexionsmorphologie in Teil \ref{part:II}.}
\part{Nominalflexion diachron und typologisch}\label{part:I}}

