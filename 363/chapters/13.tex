\chapter{Zusammenfassung und Ausblick}
\label{chap:11}
Ziel der Untersuchung war es, das flexivische System im oobd. Dialektraum zu erfassen, das heißt die flexivischen Marker und Kodierungsstrategien nicht nur als genuin morphologischen Phänomenbereich zu beschreiben, sondern auch in ihren Bezügen zur Phonologie, zur Syntax und zum semantisch-pragmatischen Kontext des Sprachgebrauchs. Hierfür wurde das nominale Flexionssystem für die syntaktische Einheit aus Definitartikel und Substantiv für 37 Tiefenbohrungspunkte im Ofr., Nord- und Mittelbair. untersucht. Die Datengrundlage bestand primär aus den Erhebungsdaten des \textit{Bayerischen Sprachatlas}, des Weiteren wurden die Daten der Wenker-Erhebungen und die Karten des \textit{Sprachatlas des Deutschen Reiches} sowie diverse grammatische Beschreibungen der untersuchten Dialekte ausgewertet. Auch wenn -- bedingt durch die Datenlage -- vor allem in den Bereichen Morphosyntax und Sprachgebrauch vereinzelt nur Forschungsperspektiven und -desiderate aufgezeigt werden konnten, so konnte in der dialektgeografischen Analyse gezeigt werden, dass der oobd. Dialektraum aus deutlich voneinander abgrenzbaren Flexionssystemen besteht, die unterschiedliche Strategien zur Kodierung morphologischer Information entwickelt haben. Dies ist insofern bemerkenswert, als die untersuchten Dialekte ähnliche phonologische und morphologische Voraussetzungen aufweisen.

Morphologische Arealität ist im UG in weiten Teilen durch Phonologie bedingt, allen voran durch die dialektspezifischen Entwicklungen der binnenhochdeutschen vs. mittelbair. Konsonantenschwächung, die vokalische Realisierung der mhd. Reduktionssilbe -\textit{en} oder den nordbair. Diphthongwechsel bei mhd. \textit{ei}. Im kontrastiven Vergleich der Dialektsysteme und der verschiedenen stammaffizierenden Verfahren war festzustellen, dass es in einzelnen untersuchten Dialekten eine Tendenz zu stark lexikalisierten innerparadigmatischen Alternationen gibt, die nicht aus synchronen phonologischen Regeln abzuleiten sind. Exemplarisch sind hier die starke Ausdifferenzierung der dialektalen Umlautentsprechungen im Nordbair. zu nennen, die Kontraste zwischen elidiertem und erhaltenem Konsonanten in Teilen des Mittelbair. (Typ \teuthoo{bo2}{bō} -- \teuthoo{ba4x}{bạx} ‚Bach‘) oder die Alternationen von epenthetischem Schwa und elidiertem Nasal im westlichen Ofr. (Typ \teuthoo{do2ArE}{dōαrə} -- \teuthoo{do2AnA}{dōαnα} ‚Dorn‘).

Die Analyse der Formenbildung erfolgte dabei unter der Prämisse, dass zunächst alle innerparadigmatischen Alternationen morphologisches Symbolisierungspotenzial haben. Dies wurde durch die Daten einerseits insofern bestätigt, als einzelne morphophonologische Alternationen teilweise morphologische Produktivität entwickelt haben (etwa Lenis-Fortis- oder Vokalquantitätskontraste). Gleichzeitig konnte im Rahmen der sprachtheoretischen Analyse durch Einbeziehung von \citegen{Köpcke1993} Schema-Ansatz gezeigt werden, dass diese Sicht auf Perzeption und Produktion sprachlicher Formen kognitiv plausibel ist. Die Kodierung und Dekodierung flexivischer Informationen erfolgt über verschiedene mehr oder weniger signalstarke Hinweisreize, die in Kombination die Interpretation der Numerusinformation ermöglichen. So wurde für die innerparadigmatischen Lenis-Fortis-Kontraste gezeigt, dass sie nicht nur ein „Doppelleben“ \citep{Seiler2008} zwischen Phonologie und Morphologie führen, sondern auch auf der phonetischen Ebene Variation aufweisen und innerhalb eines phonetischen Kontinuums realisiert werden. Mögliche Uneindeutigkeiten der Lenis-Fortis-Konsonanz werden in der schemabasierten Verarbeitung der Formen kompensiert, indem weniger saliente Cues in Kombination mit salienteren, valideren Cues (etwa dem Definitartikel) perzipiert wird. In dieser Interpretation haben auch synchron phonologisch vorhersagbare Alternationen wie die Spirantisierungen von intervokalischem /b/ morphologisches Kodierungspotenzial, da sie die flexivische Information mitindizieren.

\begin{sloppypar}
Neben den morphophonologischen Alternationen, die im Falle der historischen \textit{i}{}- und \textit{a}{}-Deklination lautgesetzlich, teilweise aber auch durch analogischen Deklinationsklassenwechsel entstanden sind, besteht ein weiteres spezifisches Merkmal der untersuchten Dialekte in den produktiven prosodisch-phonotaktischen Input- und Outputbedingungen, die sich im Bereich der additiven Markierung in einem Teil der bair. Dialekte herausgebildet haben. Hier zeigt sich wiederum, dass ähnliche Voraussetzungen nicht unbedingt zu ähnlichen Strukturen in den Flexionssystemen führen müssen. Infolge des Umbaus der typenfrequenten historischen fem. \textit{n}{}-Deklination finden sich im Ofr. und im Bair. gleichermaßen Feminina des Typs \teuthoo{das\#n}{dašn} -- \teuthoo{das\#n}{dašn} ‚Tasche‘, und in beiden Dialekträumen finden sich vokalische vs. konsonantische Realisierungen der Reduktionssilbe mhd. -\textit{en}. Doch nur in Teilen des Bair. hat sich für Feminina des Typs \teuthoo{das\#n}{dašn} oder \teuthoo{biEkA}{biəkα} ‚Birke‘ eine numerusdistinkte, formale Markierung herausgebildet, während im Ofr. und nördlichen Nordbair. Numerussynkretismen erhalten sind. Indem die prosodische Inputbedingung einer Tiefschwa-Reduktionssilbe und die Pluralmarkierung mit Nasalsuffix auch auf Teile der Maskulina ausgeweitet wurde (Typ \teuthoo{haofA}{haofα} -- \teuthoo{haofAn}{haofαn} ‚Haufen‘), hat sich in diesem Dialektgebiet eine stärkere Formalisierung der Konditionierung der fem. und mask. Deklinationsklassen vollzogen. In der Folge ist in den rezenten bair. Dialekten ein weites Spektrum zwischen lexikalisierten morphophonologischen Alternationen einerseits und andererseits stärker formalisierten additiven Markierungen festzustellen. Ausgehend von dieser sekundären morphologischen Markierung der Feminina ergeben sich für das UG zwei unterschiedliche Typen von Flexionssystemen: Die einen tolerieren und bewahren Numerussynkretismen, während sich in anderen flexivische Kompensationsstrategien entwickelt haben. Da, wo genuin morphologische Entwicklungen ins Spiel kommen, ergeben sich auch distinkte Strukturen im Raum.
\end{sloppypar}

\begin{sloppypar}
Der diachrone Blick auf die dialektalen Deklinationsklassensysteme ergibt, dass die historischen Klassen zum Teil gewahrt werden (sich die Numerusexponenz durch lautgesetzlichen phonologischen Wandel im interdialektalen Vergleich aber ausdifferenziert, siehe \sectref{sec:8.2}), zum Teil Deklinationsklassenwechsel stattfinden oder neue dialektale Klassen entstehen (beispielsweise die bair. Klasse mit kumulativer Pluralmarkierung aus Umlaut und Nasalsuffix, Typ \teuthoo{o2bvl@}{ōbvl̥} -- \teuthoo{epfl@n@}{epf‌l̥n̥} ‚Apfel‘, siehe \sectref{sec:8.1}). Gleichzeitig zeigt etwa die Reorganisation der historischen \textit{iz/az}{}-Klasse, wie phonologischer und morphologischer Wandel ineinandergreifen und zu Restrukturierungen des Deklinationssystems führen: Lautgesetzlicher Wandel des Stammvokalismus führt teilweise zu umlautlosen Pluralformen, in der Folge ist Umlaut bei den rezenten Entsprechungen der \textit{iz/az}{}-Klasse und auch bei Klassenwechslern -- anders als im Neuhochdeutschen -- nicht grundsätzlich konkomitant (ausführlicher \sectref{sec:8.2.2}).
\end{sloppypar}

\begin{sloppypar}
Genus erweist sich auch in den untersuchten Dialekten als ein zentraler Steuerungsfaktor der Deklinationsklassenzugehörigkeit (\sectref{sec:8.3.1}). Phonologisch bedingte Unterschiede im Suffixinventar und Restrukturierungen des Deklinationssystems haben in den untersuchten Dialekten zu zwei unterschiedlichen Genuskonstellationen geführt: Werden mhd. -\textit{en} und mhd. -\textit{er} auch im rezenten Flexionssystem unterschieden, besteht in der Tendenz eine Genusschranke zwischen Neutrum und Nicht-Neutrum. Sind mhd. -\textit{en} und mhd. -\textit{er} in Folge der Vokalisierung von mhd. -\textit{en} in bestimmten phonologischen Kontexten dagegen zusammengefallen, ergibt sich keine eindeutige Opposition zwischen den Genera. Diese spezifisch bair. Genuskonstellation ist eine Folgeerscheinung der formalen Steuerung von Nasal- und Tiefschwa-Suffix bei den Feminina: Feminina markieren den Plural additiv mit beiden Allomorphen und nehmen so eine Art vermittelnde Stellung zwischen dem Nasalsuffix der Maskulina und dem Tiefschwa-Suffix der Neutra ein. Der phonologisch bedingte Zusammenfall von mhd. -\textit{en} und mhd. -\textit{er} und die stärkere Formalisierung der Deklinationsklassensteuerung haben hier diachron zur Schwächung der Genusopposition geführt.
\end{sloppypar}

Unterhalb der Genusebene weisen in einigen der untersuchten Dialekte zudem die semantischen Merkmale Kollektivität und Belebtheit eine steuernde Wirkung auf (\sectref{sec:8.3.2}). Für Teile des Nordbair. konnte etwa gezeigt werden, dass das semantische Merkmal [+belebt] zu Klassenwechsel in die schwache Deklination führt und dass Semantik (neben Genus) ein verlässlicher Hinweisgeber auf die schwache Deklination ist.

\begin{sloppypar}
Als weiteres spezifisches Phänomen des untersuchten Dialektraums ist das grammatikalisierte Prinzip einer fakultativen Markierung hervorzuheben, die sich bei den \textit{n}{}-erweiterten Feminina, aber auch bei den distinkten Da"-tiv-""Plu"-ral-""Mar"-kie"-run"-gen des nördlichen Nordbair. und Ofr. herausgebildet hat (\sectref{sec:9.2}). Bemerkenswert an den formalen, teilweise fakultativen Markierungen bei Da"-tiv-""Plu"-ral-""No"-mi"-nal"-phra"-sen ist, dass es weniger die Kasus-, sondern in erster Linie die Pluralinformation ist, die am Substantiv kodiert wird. Hier erweist sich -- aus Perspektive des Relevanzprinzips -- Numerus als relevante Kategorie.
\end{sloppypar}

Insgesamt ist mit Blick auf die fakultativen Markierungen (aber nicht nur) festzuhalten, dass Formenvarianz und Variabilität ein integraler und funktionalisierter Bestandtteil dialektaler Flexionssysteme sind (dies gilt auch für variierende Singularstammformen der historischen \textit{n}{}- und \textit{ô}{}-Feminina, z.\,B. \teuthoo{s\#dros}{šdros}\slash\teuthoo{s\#drosn}{šdrosn} -- \mbox{\teuthoo{s\#drosn}{šdrosn}} ‚Straße‘). Flexionsmorphologie stellt daher weniger eine geschlossene Systemebene in einem modularen Grammatikmodell dar, sondern die Kodierung flexivischer Information ist stark durch se"-man"-tisch-""prag"-ma"-ti"-sche Kontextbedingungen geprägt und erfolgt nicht ausschließlich durch morphologische Mittel. Da die Beobachtungen zur Variabilität flexivischer Kodierung in der Untersuchung nur anhand von Fallbeispielen gemacht werden konnten, braucht es hier eine systematische Erweiterung der Datenbasis, um die flexivische Informationsstruktur sowie die syntaktischen und semantisch-pragmatischen Geltungsbedingungen der verschiedenen Kodierungstechniken differenzieren zu können.\largerpage

In diesem Punkt sind auch die diskutierten Ansätze morphologischer Theoriebildung an Grenzen gestoßen. In keinem der Modelle ist der Abbau der Pluralmarkierung ausgerechnet bei den Feminina vorgesehen; trotzdem kann auch bei Flexionssystemen mit einem hohen Anteil an Numerussynkretismen von stabilen Systemen ausgegangen werden. Die Dialekte erweisen sich hier als pragmatischer im Umgang mit flexivischer Kodierung, als dies die Theorien ausgehend von der spezifischen Normativität und Kodifizierung der Standardsprache modellieren. Das Ergebnis der empirischen Überprüfung der ausgewählten Ansätze lautet, dass morphologische Theorien die Variabilität in der Kodierung flexivischer Information und kommunikative Strategien noch stärker in den Blick nehmen müssen, um die Perzeption der Numerusinformation (neben der Produktion) adäquat modellieren zu können.
