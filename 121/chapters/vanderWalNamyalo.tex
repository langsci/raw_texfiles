\documentclass[output=paper]{langsci/langscibook} 
\title{The interaction of two focus marking strategies in Luganda} 
\author{% 
Jenneke van der Wal \affiliation{University of Cambridge}\lastand 
 Saudah Namyalo \affiliation{Makerere University}
}
\chapterDOI{10.17169/langsci.b121.490} %will be filled in at production
 

\abstract{
The Bantu language Luganda (Uganda, JE15) has three morphosyntactic strategies to express focus on a nominal referent. This paper examines the detailed interpretational properties of two of these strategies: the preverbal focus construction and augmentless nouns. Based on various focus tests and effects of the combination of the two strategies, we propose that they express identificational focus and exclusive focus, respectively. This detailed and systematic testing refines our understanding of the expression of focus in Luganda and focus types in general.
} 

\maketitle
\begin{document}
 
 
%

% Keywords: \ili{Luganda}, focus, exclusivity, identification, \ili{Bantu}, augment


\section{Introduction: \ili{Bantu} basics and focus}\label{sec:vanderwal:1}

\ili{Luganda} is one of approximately 500 \ili{Bantu} languages \citep[4]{NursePhilippson2003}, spoken as a native language and lingua franca in Uganda.\footnote{Data come from joint fieldwork by the authors in August 2014 in Mpigi, Masaka and Kampala, unless otherwise indicated.}{} As is common in \ili{Bantu}, the canonical word order is SVO, the language has extensive verbal morphology including subject and object indexing, and nouns are divided into noun classes based on their prefixes and their concord forms and agreement patterns in the phrase and clause.\footnote{Noun classes are indicated in glosses by numbers; see further the abbreviations section at the end of the paper.}{} In \ili{Luganda}, noun stems are preceded by a noun class prefix as well as an augment or initial vowel: in \REF{ex:vanderwal:1}, the noun \textit{omusota} ‘snake’ is in class 3 and has the prefix \textit{mu} plus the augment \textit{o}. 

\ea\label{ex:vanderwal:1}
\gll O-mu-sóta  gw-áá-kútte      e-mmése.\\
     \textsc{3a}-\textsc{3px}-snake  \textsc{3sm}-\textsc{past}-catch.\textsc{perf}  \textsc{9a}-9.rat\\
\glt ‘A/The snake caught a/the rat.’
\z

The augment is important in the expression of focus, which is the subject of this paper. Specifically, we address three research questions: 
(i) the morphosyntactic strategies used in \ili{Luganda} to express focus; 
(ii) the type of focus that these strategies express; 
and (iii) how different types of focus can be identified in elicitation.

In the field of information structure, many different terms have been used to refer to \textit{focus} and many subtypes of focus have been proposed.\footnote{We do not consider the scope or size of focus in this paper (e.g. Lambrecht's 1994 distinction between argument vs. predicate vs. sentence focus).}{} When we refer to \textit{focus} in this paper, we take a by now widely accepted semantic definition of focus proposed in Rooth’s (1985; 1992; 1996) Alternative Semantics. This states that focus “indicates the presence of alternatives that are relevant for the interpretation of linguistic expressions” \citep[6]{Krifka2007}. A set of alternatives for the object ‘rat’ in \REF{ex:vanderwal:1} could be, for example, \{owl, rabbit, fish,….\}. The triggering of a set of alternatives is a unified core function of focus, and the various types of focus can be seen as the outcomes of additional pragmatic and semantic factors \citep{ZimmermannOnea2011}. Focus can thus be underspecified in its exact interpretation, the only semantically consistent part being the presence of a set of alternatives. The precise interpretation of the focus can differ, however, depending on the context in which it occurs, allowing \textit{completive}, \textit{replacing}, \textit{selecting} etc. interpretations, which differ in their “communicative point” \citep[281]{Dik1997}. Such potential interpretations are not the core point of the current paper, as we first want to establish the semantic type of focus of the \ili{Luganda} focus strategies. 

Semantic types of focus not only trigger a set of alternatives, but also operate on that set of alternatives. This can result in a scalar, exhaustive or exclusive reading, and can have a truth-conditional effect. Exhaustive and exclusive readings are important for the current paper. Exclusive focus means that there is at least \textit{some} referent in the set of alternatives to which the predicate does not apply, e.g. the snake caught a rat but not a fish (and potentially not other prey either but we do not know). Exhaustive focus means that for \textit{all} alternative referents the predicate does not hold, i.e. the predicate is true only for the focused referent, e.g. the snake caught a rat and nothing else. These semantic types of focus can thus be said to be more specific than merely triggering alternatives.

A separate type of focus is associated with a presupposition (for example of existence or exhaustivity). An example is the English \textit{it}-cleft, where the relative clause forms the presupposition and the focus identifies the referent for which the proposition is true. This is what we mean by identificational focus, but it is crucially not necessarily exhaustive, contra to what \citet{Kiss1998} claims (see also \citealt{OneaBeaver2011,ByramWashburn2013,DestruelEtAl2015}). For further discussion of semantic and pragmatic types of focus, we refer the reader to \citet{Bazalgette2015} and \citet{Wal2016}.  

The rest of the paper is structured as follows. We briefly introduce the Immediate After Verb (IAV) effect in \sectref{sec:vanderwal:2}, before discussing the two main focus constructions under study: the Preverbal Focus Construction (PFC, \sectref{sec:vanderwal:3}) and the absence of the augment (\sectref{sec:vanderwal:4}). \sectref{sec:vanderwal:5} shows the possible interpretations in the combination of the PFC and augmentless nouns. \sectref{sec:vanderwal:6} concludes the paper.

 \section{Immediate After Verb focus position}\label{sec:vanderwal:2}

Like some other \ili{Bantu} languages (Aghem: Watters 1979; \ili{Zulu}: Buell 2009; \ili{Makhuwa}: van der Wal 2006; 2009; \ili{Matengo}: Yoneda 2011), \ili{Luganda} has a dedicated linear position for focus, the Immediate After Verb (IAV) position. Although the IAV position is not the primary concern of this paper, it is important to mention its effects here. The focus effect of the IAV position can be seen in restrictions on the placement of inherently focused content question words in \REF{ex:vanderwal:2} and \REF{ex:vanderwal:3}, which need to be in IAV position (\ref{ex:vanderwal:2b},\ref{ex:vanderwal:3b}): it is ungrammatical for a content question word like \textit{ani} ‘who’ or \textit{ki} ‘what’ to occur in other postverbal positions (\ref{ex:vanderwal:2c},\ref{ex:vanderwal:3c}). When postverbal, focused arguments as well as adverbs must occur in the IAV position.

\ea\label{ex:vanderwal:2}
\ea\label{ex:vanderwal:2a}
\gll   A-fúmb-íir-á    á-b-áana    e-m-mére.\\
       \textsc{1sm}-cook-\textsc{appl}-\textsc{fs}  \textsc{2a}-\textsc{2px}-children  \textsc{9a}-\textsc{9px}-food\\
\glt   ‘She cooks food for the children.’
\ex\label{ex:vanderwal:2b}
\gll   A-fúmb-ír-á    ání  e-m-mére?\\
       \textsc{1sm}-cook-\textsc{appl}-\textsc{fs}  who  \textsc{9a}-\textsc{9px}-food\\
\glt   ‘Who does she cook the food for?’
\ex\label{ex:vanderwal:2c}
\gll   *A-fúmb-ír-á  é-m-mére   aní?\\
       \textsc{1sm}-cook-\textsc{appl}-\textsc{fs}  \textsc{9a-9px}-food  who\\
\glt   (‘Who does she cook the food for?’)
\z
\z

\ea\label{ex:vanderwal:3}
\ea\label{ex:vanderwal:3a}
\gll   O-mu-sómésa  y-a-wá     á-b-áanà    e-m-mére.\\
       \textsc{1a-1px}-teacher  \textsc{1sm-past}-give  \textsc{2a-2px}-children  \textsc{9a}-\textsc{9px}-food\\
\glt   ‘The teacher gave the children food.’
\ex\label{ex:vanderwal:3b}
\gll   O-mu-sómésa  y-a-wá    kí  á-b-áanà?\\
       \textsc{1a-1px}-teacher  \textsc{1sm-past}-give  what  \textsc{2a-2px}-children\\
\glt   ‘What did the teacher give the children?’
\ex\label{ex:vanderwal:3c}
\gll  *O-mu-sómésa  ya-wá      à-b-áana    kí?\\
       \textsc{1a-1px}-teacher  \textsc{1sm.past-}give  \textsc{2a-2px}-children  what\\
\glt   (‘What did the teacher give the children?’)
\z
\z

Answers to content questions are also found in IAV position, with potential intervening non-focal elements in dislocated position, as in \REF{ex:vanderwal:4d}. All examples (4b-d) are felicitous answers to \REF{ex:vanderwal:4a}.

\ea\label{ex:vanderwal:4}
\ea\label{ex:vanderwal:4a}
\gll A-fúmbyé    á-tyá    e-m-púúta?\\
       \textsc{1sm}-cook.\textsc{perf}  1-how    \textsc{9a}-\textsc{9px}-Nile.perch\\
\glt ‘How has she cooked the Nile perch?’
\ex\label{ex:vanderwal:4b}
   Bulúngi.\\
\glt  ‘Well.’
\ex\label{ex:vanderwal:4c}
\gll  A-gi-fúmbyé      búlúngi.\\
       \textsc{1sm}-\textsc{9om}-cook.\textsc{perf}  well\\
\glt   ‘She has cooked it well.’
\ex\label{ex:vanderwal:4d}
\gll  (E-m-púúta)    a-gi-fúmbyé      búlúngi.\\
       \textsc{9a}-\textsc{9px}-Nile.perch  \textsc{1sm}-\textsc{9om}-cook.\textsc{perf}  well\\
\glt   ‘(The Nile perch) She has cooked it well.’
\z
\z

As shown by \citet{HymanKatamba1993}, a \ili{Luganda} noun without an augment [A] in an affirmative clause is in focus (see \sectref{sec:vanderwal:4}). If a [A] noun occurs postverbally, it can only appear in IAV, as shown in \REF{ex:vanderwal:5}: it is grammatical as the first noun after the verb \REF{ex:vanderwal:5b} but not as the second after an augmented [+A] noun \REF{ex:vanderwal:5c}.\footnote{In examples (\ref{ex:vanderwal:5}--\ref{ex:vanderwal:7}) only, the absence of the augment is indicated by an underscore placed before the noun, and small caps in the translation indicate focus. [+A] indicates that the noun has the augment, while [-A] indicates it does not.  The first linear [A] corresponds to the first linear noun in the clause and the second linear [A] to the second linear noun in the clause.} 

\ea\label{ex:vanderwal:5} \citet[228--229]{HymanKatamba1993} \\
\settowidth{\jamwidth}{[-A, +A]}
\ea\label{ex:vanderwal:5a}{
\gll Y-a-gúl-ir-a      a-b-áana    e-bí-tábó.  \\
       1\textsc{sm}-\textsc{past}-buy-\textsc{appl }   \textsc{2a-2px}-children  \textsc{8a-8px}-books\\\jambox{\textup{[+A, +A]}}
}
\glt   ‘He bought the children books.’             
\ex\label{ex:vanderwal:5b}  Yagúlírá {\longrule}-báana ebitábó.             \jambox{\textup{[-A, +A]}}
\glt   ‘He bought \textsc{the children} books.’           
\ex \label{ex:vanderwal:5c} *Yagúlira a-báana {\longrule}bitábó.            \jambox{\textup{[+A, -A]}}
\glt   (‘He bought the children \textsc{books}.’)           
\ex  \label{ex:vanderwal:5d} {Yagúlírá {\longrule}-báana {\longrule}-bitábó.}            \jambox{\textup{[-A, -A]}}
\glt  ‘He bought \textsc{the children books}.’
\z
\z

For completeness we note that when both postverbal objects are [-A], as in \REF{ex:vanderwal:5d}, the interpretation is VP focus, including the verb and both objects. This is visible in the appropriate situation for \REF{ex:vanderwal:6}, which contrasts the action with a whole other action of doing something else, and the inappropriateness of the contrast with just one object rather than the whole action in \REF{ex:vanderwal:7}.

\settowidth{\jamwidth}{[-A, +A]}
\ea\label{ex:vanderwal:6}{
\gll Y-á-gúlíddé    {\longrule}-m-mótoka    {\longrule}-mu-píira.      \\
 \textsc{1sm-past}-buy.\textsc{perf}       \textsc{9px}-car              \textsc{3px}-tyre\\\jambox{\textup{[-A, -A]}}
}
\glt‘He \textsc{bought a tyre for the car}.’ (i.e. this was the only work he did) 
\z

\settowidth{\jamwidth}{[-A, +A]}
\ea\label{ex:vanderwal:7}{
\gll \#Y-á-gúl-íddé      {\longrule}-m-mótoka    {\longrule}-mu-píira,  sí    picipici.\\
     \textsc{1sm-past}-buy.\textsc{perf   }    \textsc{9px}-car       \textsc{3px}-tyre   \textsc{neg}.\textsc{cop}  motorbike\\\jambox{\textup{[-A, -A]}}
}
\glt   ‘\textsc{He bought a tyre for the car}, not for the motorcycle.’
\z

The interpretation of focus in IAV is not restricted to one type, but is underspecified: postverbal objects can be interpreted as exclusive focus or part of the comment (see \sectref{sec:vanderwal:4}), and in locative inversion the subject can occur in IAV too. In subject inversion, either the subject is in focus, as in the question in \REF{ex:vanderwal:8a} and the narrowly focused subject in \REF{ex:vanderwal:8b}, or it is just detopicalised as part of a thetic sentence (said “out of the blue”), as in \REF{ex:vanderwal:9}:

\ea\label{ex:vanderwal:8}
\ea\label{ex:vanderwal:8a}
\gll  Mu-no    mú-súlá-mú    àni?\\
       18-\textsc{dem}  \textsc{18sm}-sleep-\textsc{18loc}  who\\
\glt   ‘Who sleeps in here?’

\ex\label{ex:vanderwal:8b}
\gll  Mu-no    mú-súlá-mú    mu-lalû.\\
       18-\textsc{dem}  1\textsc{8sm}-sleep-\textsc{18loc}  \textsc{1px}-crazy\\
\glt   ‘It’s a mad person who sleeps here.’
\z
\z

\ea\label{ex:vanderwal:9}
\gll Ku-kyalo  kw-á-f-íír-á-kó      o-mu-sájja.\\
     17-7.village  \textsc{17sm}-\textsc{past}-die-\textsc{appl}-\textsc{fs}-\textsc{17loc}  \textsc{1a-1px}-man\\
\glt ‘In the village died the man.’
\z

\section{The Preverbal Focus Construction}\label{sec:vanderwal:3}
 

A second focus strategy, the Preverbal Focus Construction (PFC), places the focused referent in the preverbal domain with an agreeing morpheme -\textit{e}\footnote{In the examples, we gloss the morpheme -\textit{e} as ‘e’, since it is as yet unclear whether it is a focus marker or a relative marker.}{} preceding the verb. Both arguments and adverbs can be focused in the PFC. In \REF{ex:vanderwal:10a}, the object \textit{emmese} ‘rat’ precedes the subject \textit{omusota} ‘snake’ and triggers class 9 agreement on \textit{e}. In \REF{ex:vanderwal:10b} the subject \textit{omusota} ‘snake’ is in focus and triggers class 3 agreement on \textit{e}. In \REF{ex:vanderwal:11}, the adverb \textit{eggulo} ‘yesterday’ appears sentence-initially and \textit{e} agrees with it in class 11.

\ea\label{ex:vanderwal:10}
\ea\label{ex:vanderwal:10a}
\gll    E-m-mése  o-mu-sóta    gy-e  gw-a-kuttê.\\
       \textsc{9a-9px}-rat  \textsc{3a}-\textsc{3px}-snake  9-e  \textsc{3sm}-\textsc{past}-catch.\textsc{prf}\\
\glt   ‘It’s a/the rat that the snake caught.’
\ex\label{ex:vanderwal:10b}
\gll    O-mu-sóta    gw-e  gw-á-lya    e-m-mése.\\
       \textsc{3a-3px}-snake  3-e  \textsc{3sm-past}-eat  \textsc{9a-9px}-rat\\
\glt   ‘It’s a/the snake that ate the rat.’
\z
\z

\ea\label{ex:vanderwal:11}
\gll E-ggulo    lw-e  y-a-limyé.\\
     11\textsc{a}-yesterday  11-e  \textsc{1sm}-\textsc{past}-dig.\textsc{perf}\\
\glt ‘It’s yesterday that he dug.’
\z

Leaving aside questions concerning the syntactic structure of the PFC,\footnote{A specific question is how far the PFC has developed from a biclausal cleft to a monoclausal focus construction, since the PFC shows characteristics of both.}{} the main concern in this paper is what type of focus the PFC expresses. Our hypothesis is that the PFC expresses identificational focus. Identification is here taken to mean that there is a presupposition that a referent exists for which the proposition holds true, and that the speaker asserts that this referent is identified as corresponding to the focused element (possibly from among various alternatives). For example, in \REF{ex:vanderwal:10b} there is a presupposition that someone ate the rat, and the one who is identified as that someone is the snake. In what follows we test this hypothesised meaning of the \ili{Luganda} PFC in four linguistic environments: content questions, indefinites, incomplete yes/no questions and with focus particles (see van der Wal 2016 for an overview of focus tests that we draw on).

\subsection{Content questions}\label{sec:vanderwal:3.1} %3.1

The identification reading is evident in the interpretation of content questions, such as ‘who did you hit?’ in \REF{ex:vanderwal:12} and \REF{ex:vanderwal:13}. If there is a presupposition that somebody was hit, then the prediction is that the answer is drawn from a non-empty set, and hence that answering ‘nobody’ is not felicitous. This is true for the question formed with the PFC as in \REF{ex:vanderwal:12}, confirming our hypothesis that the PFC contains a presupposition and expresses identificational focus. 

\ea\label{ex:vanderwal:12}
\gll Aní  gw-e  w-á-kúbyȇ?\\
     who  1-e  \textsc{2sg.sm-past}-hit.\textsc{perf}\\
\glt ‘Who is it that you hit?’\\{}
\gll \#Sí-rí-nâ.    /  Te-wá-lî.\\
     \textsc{neg.1sg.sm}-be-with   ~ \textsc{neg-16sm}-be\\
\glt ‘Nobody.’
\z

By comparison, an object content question in SVO order \REF{ex:vanderwal:13} can felicitously be answered by ‘nobody’, since there is no presupposition that anyone in fact was hit.

\ea\label{ex:vanderwal:13}
\gll W-á-kúbyé    ání?\\
     \textsc{2sg.sm-past}-hit.\textsc{perf}  who\\
\glt ‘Who did you hit?’\\{}
\gll Sí-rí-nâ.      /   Te-wá-lî.\\
     \textsc{neg.1sg.sm}-be-with  ~  \textsc{neg-16sm}-be\\
\glt ‘Nobody.’
\z

\subsection{Indefinites}\label{sec:vanderwal:3.2} %3.2

If the PFC contains a presupposition that some referent was involved in the event expressed by the verb, then the identification of that referent should add information and be specific. For example, it is odd to say ‘who he hit was someone or other’, because we already know from the presupposition contained in ‘who he hit’ that someone was hit. A non-specific interpretation is thus predicted to be infelicitous for the focused referent in the PFC. This is easily tested with the word \textit{omuntu} ‘person’, which can receive an indefinite, non-specific interpretation (‘anyone’) in SVO order, as in \REF{ex:vanderwal:14a} and \REF{ex:vanderwal:15a}, but as predicted not in the PFC (\ref{ex:vanderwal:14b}, \ref{ex:vanderwal:15b}). Instead, \textit{omuntu} must be interpreted as a specific entity or the type named by the noun (as opposed to an entity of some other type) in \REF{ex:vanderwal:14b} and \REF{ex:vanderwal:15b}.

\ea\label{ex:vanderwal:14}
\ea\label{ex:vanderwal:14a}
\gll    Y-á-kúbyé    o-mu-ntu.\\
         \textsc{1sm-past}-hit.\textsc{perf}  \textsc{1a-1px}-person\\
\glt     ‘He hit someone.’
\ex\label{ex:vanderwal:14b}
\gll    O-mu-ntu    gw-e  y-a-kubyȇ.\\
         \textsc{1a-1px}-person  1-e  \textsc{1sm-past}-hit.\textsc{perf}\\
\glt     ‘It’s a person he hit.’ 
\glt     \#‘He hit someone.’ / \#‘It is someone (non-specific) that he hit.’
\z
\z

\ea\label{ex:vanderwal:15}
\ea\label{ex:vanderwal:15a}
\gll   O-mu-ntu    a-gudde.\\
         \textsc{1a-1px}-person  1\textsc{sm}-fall.\textsc{perf}\\
\glt     ‘(A certain) Someone fell.’
\ex\label{ex:vanderwal:15b}
\gll     O-mu-ntu    y’  aa-gudde.\\
         \textsc{1a-1px}-person  1.e  \textsc{1sm}-fall.\textsc{perf}\\
\glt     ‘It’s a human being who fell.’

\glt     \#‘Someone fell.’ / \#‘It is someone (non-specific) that fell.’
\z
\z

\subsection{Incomplete ‘yes/no’ questions}\label{sec:vanderwal:3.3} %3.3

A further test involves incomplete ‘yes/no’ questions. These are questions that ask about a subset of the referents for which the sentence is true. Concretely, the test involves showing participants a picture of Thomas with a cat and a rabbit, and asking a ‘yes/no’ question on a subset of ‘cat and rabbit’, namely whether Thomas has a cat. Incomplete ‘yes/no’ questions and their answers can reveal focus effects in two ways: the choice of the answering particle (‘yes’ or ‘no’) and the choice of conjunction (‘and’ or ‘but’).\footnote{This test is modified from the Questionnaire on Information Structure \citep{SkopeteasEtAl2006}; see \citet{OneaBeaver2011} for an experimental application of this test for \ili{Hungarian}, and \citet{DestruelEtAl2015} for further crosslinguistic testing of it-clefts.}

  With regard to the answering particle, only the answer ‘yes’ is predicted to appear in an answer to an SVO question, since it is true that Thomas has a cat, regardless of whether he has other animals too. By contrast, the answer to the corresponding question in the PFC can relate not just to the content, but also to the unique identification: the answer ‘no’ here negates not that Thomas has a cat, but that Thomas has only a cat. The answer ‘no’ is thus predicted to be possible for the PFC question but not for the SVO question. This is precisely what we found: an incomplete ‘yes/no’ question in the PFC \REF{ex:vanderwal:16a} is corrected by answering ‘no’ \REF{ex:vanderwal:16b}, whereas for an SVO question \REF{ex:vanderwal:17a}, the answer is either a simple ‘yes’ \REF{ex:vanderwal:17b}, or a ‘yes but/and’ \REF{ex:vanderwal:17c}, but not a straight ‘no’. 

\newpage

\ea\label{ex:vanderwal:16}PFC question
       (picture is shown of Thomas having a cat and a rabbit)\\
\ea\label{ex:vanderwal:16a}
\gll     Kkápa  Thomas  gy-e  a-li-nâ?\\
         9.cat  1.Thomas  9-e  \textsc{1sm}-be-with\\
\glt     ‘Is it a cat that Thomas has?’
\ex\label{ex:vanderwal:16b}
\gll      Nédda,  sí    y-okkâ,  Thomas  a-li-ná    n’  á-ká-myû.\\
         no  \textsc{neg}.\textsc{cop}  9-only  1.Thomas  \textsc{1sm}-be-with  and  \textsc{12a}-\textsc{12px}-rabbit\\
\glt     ‘No, not only, Thomas also has a rabbit.’
\z
\z

 
\ea\label{ex:vanderwal:17}SVO question
       (picture is shown of Thomas having a cat and a rabbit)\\
\ea\label{ex:vanderwal:17a}
\gll   Thomas  a-li-ná  kkápâ?\\
       1.Thomas  \textsc{1sm}-be-with  9.cat\\
\glt   ‘Does Thomas have a cat?’
\ex\label{ex:vanderwal:17b}
\gll   Yee,   Thomas  a-li-ná  kkápa.\\
       yes  1.Thomas  \textsc{1sm}-be-with  9.cat\\
\glt   ‘Yes, Thomas has a cat.’
\ex\label{ex:vanderwal:17c}
\gll     Yee,  Thomas    a-li-ná    kkápa,  nayé  a-li-ná    ne  á-ká-myû.\\
       yes  1.Thomas  \textsc{1sm}-be-with  9.cat   but    \textsc{1sm}-be-with  and  \textsc{12a}-\textsc{12px}-rabbit\\
\glt   ‘Yes, Thomas has a cat, but he has a rabbit too.’
\z
\z

When the answer to the PFC question is not a straight negation as in \REF{ex:vanderwal:16b}, the prediction is that the PFC strongly prefers ‘yes but’ over ‘yes and’. This is because the identificational focus associated with the PFC suggests an exclusion of and contrast to alternative referents, whereas the ‘yes and’ answer is explicitly non-exclusive and non-contrastive. The conjunction ‘but’ is required here to highlight the identification associated with the PFC and to express the contrast with possible alternatives, as is confirmed by the data in \REF{ex:vanderwal:18}. In \REF{ex:vanderwal:18a} the yes/no question is formed by a PFC, and the answer ‘yes but’ in \REF{ex:vanderwal:18b} is acceptable, but the answer ‘yes and’ in \REF{ex:vanderwal:18c} is not. The explicit contrast induced by ‘but’ is not necessary for the SVO counterpart in \REF{ex:vanderwal:19}, since no exclusion or contrast is expected for SVO: both answers ‘yes but’ and ‘yes and’ are fine (\ref{ex:vanderwal:19b}-c).

\ea\label{ex:vanderwal:18}PFC question
     (picture is shown of Maria with a red and a yellow pepper)\\
\ea\label{ex:vanderwal:18a}
\gll     Kikáámuláli  e-ki-myúfu  Maria    ky-e  a-li-ná?\\
         7.pepper  \textsc{7a}-\textsc{7px}-red  1.Maria  7-e  \textsc{1sm}-be-with\\
\glt     ‘Is it a red pepper that Maria has?’

\ex\label{ex:vanderwal:18b}
\gll     Ye,  kyokka/naye  a-li-ná  ne  e-ky-á    ky-énvu.\\
         yes  only/but  \textsc{1sm}-be-with  also  \textsc{7}-\textsc{7}-\textsc{conn}  7-yellow\\
\glt     ‘Yes, only/but she also has a yellow one.’
\ex\label{ex:vanderwal:18c}
\gll   \#Era/ate  a-li-ná  ne  e-ky-á    ky-énvu.\\
          and/and.even  \textsc{1sm}-be-with  also  \textsc{7}-\textsc{7}-\textsc{conn}  7-yellow\\
\glt     ‘And she also has a yellow one.’
\z
\z

\ea\label{ex:vanderwal:19}
SVO question
       (picture is shown of Maria with a red and a yellow pepper)\\
\ea\label{ex:vanderwal:19a}
\gll      Maria    a-li-ná  kikáámuláli  e-ki-myúfu?\\
         1.Maria  \textsc{1sm}-be-with  7.pepper  \textsc{7a}-\textsc{7px}-red\\
\glt     ‘Does Maria have a red pepper?’
\ex\label{ex:vanderwal:19b}
\gll    Ye,  era/ate  a-li-ná  ne  e-ky-á    ky-énvu.\\
         yes  and/and.even  \textsc{1sm}-be-with  also  \textsc{7}-\textsc{7}-\textsc{conn}  7-yellow\\
\glt     ‘Yes, and she has also/even a yellow one.’
\z
\z

The first part of this test (‘yes/no’) suggests that the PFC is associated not just with identification but with exhaustivity, that is, the exclusion and negation of all possible alternatives to the set expressed by the focused phrase (‘only a cat and nothing else’). This is because a ‘no’ answer negates that Thomas \textit{exhaustively} has a cat. A question is whether this is an inherent semantic part of the PFC or rather a pragmatically implied part of meaning. A suggestive answer is found in the second test (‘and/but’). Specifically, if the PFC has an inherently exhaustive interpretation, we would predict ‘no’ to be the only appropriate answer. Instead, the fact that ‘yes, but’ is also compatible with the PFC suggests that the exhaustive interpretation is implied but not encoded. More specifically, it seems that there is a presupposition of contrast and an implicature of exhaustivity (cf. \citealt{ByramWashburn2013}). This is why there is a need for a contrasting conjunction ‘but’, but there is no truth-conditional exhaustive meaning inherently present in the PFC. 

\subsection{Focus particles ‘only’ and ‘even’}\label{sec:vanderwal:3.4}

Focus particles also reveal the identificational nature of the PFC. The particle ‘only’ shows a subset of identification, namely exhaustive identification. ‘Only’ is compatible with the PFC for objects, as in \REF{ex:vanderwal:20}, and the PFC is in fact required for subjects modified by ‘only’, as shown by the ungrammaticality of \REF{ex:vanderwal:21b} where the subject ‘only the house’ is not in the PFC. Note that ‘only’ requires the augmentless form of the noun, as further discussed in \sectref{sec:vanderwal:4}.

\ea\label{ex:vanderwal:20}
\gll   Ka-wûnga  k-okká    k-e  y-a-lyâ.{} \\
       \textsc{12px}-ugali  12-only  12-e  \textsc{1sm}-\textsc{past}-eat\\
\glt   ‘It is only ugali that he ate.’\footnote{\textit{Ugali}, a stiff porridge made from maize flour, is the staple food in East Africa.}
\z

\ea\label{ex:vanderwal:21}
\ea\label{ex:vanderwal:21a}
\gll      Nnyúmba  y-okká  y-e  y-a-yér-ebw-a.\\
         9.house  9-only  9-e  \textsc{9sm}-\textsc{past}-sweep-\textsc{pass}-\textsc{fs}\\
\glt     ‘Only the house was swept.’
\ex\label{ex:vanderwal:21b}
\gll     *Nnyúmba  y-okká    y-a-yér-ebw-a.\\
         9.house  9-only    \textsc{9sm}-\textsc{past}-sweep-\textsc{pass}-\textsc{fs}\\
\glt     (‘Only the house was swept.’)
\z
\z

The particle \textit{ne} ‘even, also’, however, is not compatible with the PFC, as shown in \REF{ex:vanderwal:22b} and \REF{ex:vanderwal:23b}. For comparison, it \textit{is} grammatical in a preposed topic position \REF{ex:vanderwal:22a}.

\ea\label{ex:vanderwal:22}
\ea\label{ex:vanderwal:22a}
\gll     Ne  á-ká-wûngá    y-a-ká-lya.\\
         even  \textsc{12a}-\textsc{12px}-ugali  \textsc{1sm}-\textsc{past}-\textsc{12om}-eat\\
\glt     ‘He ate even ugali.’
\ex \label{ex:vanderwal:22b}
\gll     *Ne  akawûnga    k-e  y-a-lyâ.\\
         even  \textsc{12a}-\textsc{12px}-ugali  12-e  \textsc{1sm}-\textsc{past}-eat\\
\glt     (‘It’s even ugali that he ate.’)
\z
\z

\ea\label{ex:vanderwal:23}
\ea\label{ex:vanderwal:23a}
\gll     Né  énnyúmba  y-a-yér-ebw-a.\\
         even  9.house  \textsc{9sm}-\textsc{past}-sweep-\textsc{pass}-\textsc{fs}\\
\glt     ‘Even the house was swept.’
\ex\label{ex:vanderwal:23b}
\gll      *Né  énnyúmba  y-e  y-a-yér-ebw-a.\\
         even  9.house  9-e  \textsc{9sm}-\textsc{past}-sweep-\textsc{pass}-\textsc{fs}\\
\glt     (‘It is even the house that was swept.’)
\z
\z

The opposite distribution of ‘only’ and ‘even’ in the PFC can be explained by the incompatibility between ‘even’ and exclusivity: ‘even’ includes all alternatives up to the furthest end of a scale of possibilities, thus excluding none of the alternatives. This suggests that the PFC is associated with exclusive focus, not just identification. This is also how \citet[52]{Walusimbi1996} characterizes the PFC: as expressing the existence of a presupposition and simultaneously an exclusive set that satisfies a missing variable in the presupposition. Alternatively, however, exclusivity may not be inherent to the PFC (as suggested in \sectref{sec:vanderwal:3.3} above), but instead be due to the pragmatic incompatibility between scalarity and identification of a referent. To illustrate, if one would want to identify ‘who ate ugali?’ the scalarity-related answer \textsuperscript{??}‘Even John ate ugali’ does not seem appropriate.

  In conclusion, the incompatibility of the PFC with the scalar particle ‘even’ is not directly expected if the PFC expresses just identificational (not inherently exclusive) focus. Further tests exploring the combination of the PFC with augmentless nouns can further elucidate our hypothesis that the PFC expresses identificational focus. We turn to these in \sectref{sec:vanderwal:4} and \sectref{sec:vanderwal:5}.

\section{The augment}\label{sec:vanderwal:4}

Apart from the IAV position and the PFC, a third linguistic strategy related to focus is the presence or absence of the augment on nouns.  \citet{HymanKatamba1990} and \citet{HymanKatamba1993} analyse the presence ([+A]: \textbf{\textit{o}}\textit{-mu-sota}) or absence ([-A]: \textit{{\longrule}-mu-sota}) of the augment as related to two licensers: negation and focus. They show that any NP following a negative verb must be [-A], as shown in \REF{ex:vanderwal:24a} and the ungrammaticality of (\ref{ex:vanderwal:24b}-d); this is a purely syntactic restriction and has no influence on information structure.

\ea\label{ex:vanderwal:24} \citet[224]{HymanKatamba1993} \\
\ea\label{ex:vanderwal:24a}
\gll    Te-bá-á-w-a      {\longrule}-b-áana    {\longrule}-bi-tábó.\\
       \textsc{neg}-\textsc{2sm}-\textsc{past}-give-\textsc{fs}       \textsc{2px}-children       \textsc{8px}-books\\
\glt   ‘They didn't give the children books.’
\ex \label{ex:vanderwal:24b}   *Tebááwa abáana ebitábó.\\
\ex \label{ex:vanderwal:24c}   *Tebááwa abáana {\longrule}-bi-tábó.\\
\ex \label{ex:vanderwal:24d}   *Tebááwa {\longrule}-báana ebitábó. \\
\z
\z

They also identify focus as a licenser of augmentless nouns: an augmentless object in an affirmative clause receives a focus reading, as shown in the difference between \REF{ex:vanderwal:25a} and \REF{ex:vanderwal:25b}.

\ea\label{ex:vanderwal:25} \citet[228]{HymanKatamba1993}\\
\ea\label{ex:vanderwal:25a}
\gll    Y-a-gúla    e-bí-tábó.\\
       \textsc{1sm}-\textsc{past}-buy  \textsc{8a}-\textsc{8px}-books\\
\glt   ‘He bought books.'
\ex\label{ex:vanderwal:25b}
\gll   Y-a-gúla    {\longrule}-bi-tábó.\\
       \textsc{1sm}-\textsc{past}-buy       \textsc{8px}-books\\
\glt   ‘He bought \textsc{books}.’
\z
\z

We generally accept their analysis of the augment in \ili{Luganda}, and extend it by studying what the precise interpretation of this focus strategy is. Our proposal is that the focus expressed by the absence of the augment is exclusive. As explained before in \sectref{sec:vanderwal:1}, by \textit{exclusive} we mean that the focused referent triggers a set of alternatives and excludes at least some of those alternatives. For example, in a sentence ‘it’s matooke that she ate’,\footnote{\textit{Matooke} is a green plantain which is traditionally steamed in banana leaves and mashed for eating. } {} the focused item \textit{matooke} triggers the existence of a set of alternative things that she may have eaten, like rice, beans, porridge, and conveys that the action of eating is true for \textit{matooke}, but not true for some (and potentially all) of the alternatives. This exclusive focus can be shown in five tests: alternative questions, correction, indefinites, focus particles, and answers to content questions.

\subsection{Alternative questions}\label{sec:vanderwal:4.1} %4.1

In answers to alternative questions, a selection needs to be made between alternatives. This means that one or more alternatives are excluded. The data show that in this case the [-A] form is obligatorily used in both the question \REF{ex:vanderwal:26a} and the answer \REF{ex:vanderwal:26c}, which can also just be a one word fragment \REF{ex:vanderwal:26d}. Example \REF{ex:vanderwal:26b} shows that nouns with the augment in object position lead to ungrammaticality of the question. 

\ea\label{ex:vanderwal:26}
\ea\label{ex:vanderwal:26a}
\gll    O-yágálá    mágí    obá  nnyamá?\\
         \textsc{2sg}.\textsc{sm}-want    6.eggs    or  9.meat \\
\glt     ‘Do you want eggs or meat?’
\ex\label{ex:vanderwal:26b}
\gll    *O-yágálá    a-mágí  obá  e-nnyama?\\
         \textsc{2sg}.\textsc{sm}-want    \textsc{6a}-6.eggs  or  \textsc{9a}-9.meat\\
\glt     (‘Do you want eggs or meat?’)
\ex\label{ex:vanderwal:26c}
\gll      N-jágálá    nnyámá / \#é-nnyámá.\\
         \textsc{1sg}.\textsc{sm}-want    9.meat / \textsc{9a}-9.meat\\
\glt     ‘It’s meat that I want’, ‘I want \textsc{meat}.’
\ex\label{ex:vanderwal:26d}
\gll     Nnyámá.\\
         9.meat\\
\glt     ‘Meat.’
\z
\z

\subsection{Overt contrast/correction}\label{sec:vanderwal:4.2} %4.2

A second test for exclusivity involves adding an overt contrast or correction to a statement. The overt contrast or correction phrase excludes the mentioned alternative(s), and thus requires the [-A] form of the object \textit{nnyúmba} ‘house’ in the correction in \REF{ex:vanderwal:27}, and the object \textit{ssúúká} ‘bedsheets’ in the contrast in \REF{ex:vanderwal:28}.\footnote{This particular example has a preverbal topical benefactive object (resumed by an object marker on the verb), which allows the theme object to be contrasted in IAV position.}

\ea\label{ex:vanderwal:27}
\ea\label{ex:vanderwal:27a}
\gll      Y-a-yéra    o-lú-ggya.\\
         \textsc{1sm}-\textsc{past}-sweep  \textsc{11a}-\textsc{11px}-yard\\
\glt     ‘She swept the yard.’
\ex\label{ex:vanderwal:27b}
\gll     Nédda,  y-a-yérá    nnyúmba / \#e-nnyúmba.\\
         no    \textsc{1sm}-\textsc{past}-sweep 9.house\\
\glt     ‘No, she swept the house.’
\z
\z

\ea\label{ex:vanderwal:28}
\gll   Máaka  a-b-áana    y-a-bá-gúl-ír-á        {ssúúká / \#e-ssúúká},  sí    bulangíti. \\
       1.Mark  \textsc{2a}-\textsc{2px}-children  \textsc{1sm}-\textsc{past}-\textsc{2om}-buy-\textsc{appl}-\textsc{fs}  10.bedsheets     \textsc{neg}.\textsc{cop}  blankets\\
\glt    ‘Mark bought the children bedsheets, not blankets.’
\z

Crucially, the [+A] form is judged as felicitous if the contrasting clause is absent, as in \REF{ex:vanderwal:29}.

\ea\label{ex:vanderwal:29}
\gll   Máaka  a-b-áaná    y-a-bá-gúl-ír-a      e-ssúúkâ.\\
       1.Mark  \textsc{2a}-\textsc{2px}-children  \textsc{1sm}-\textsc{past}-\textsc{2om}-buy-\textsc{appl}-\textsc{fs}  \textsc{10a}-10.bedsheets\\
\glt   ‘Mark bought the children bedsheets.’
\z

\subsection{Indefinites}\label{sec:vanderwal:4.3} %4.3

A third test showing exclusivity involves the indefinite noun \textit{(o)muntu} ‘person’. This can be interpreted as non-specific in its [+A] form, as in \REF{ex:vanderwal:30a}; but refers to a type of entity when in its [-A] form and when the scope of focus is the object, as in \REF{ex:vanderwal:30b}. This is because non-specific indefinites do not allow for the exclusion of alternatives (‘anyone’ includes everyone), whereas types do allow for exclusion, as indicated in the following contrastive clause (‘not a cat’).

\ea\label{ex:vanderwal:30}
\ea\label{ex:vanderwal:30a}
\gll     N-á-kúbyé      o-mu-ntu.\\
         \textsc{1sg}.\textsc{sm}-\textsc{past}-hit.\textsc{perf}  \textsc{1a}-\textsc{1px}.person\\
\glt     ‘I beat someone.’
\ex\label{ex:vanderwal:30b}
\gll     N-á-kúbyé      mu-ntú,  si    kkapa.\\
         \textsc{1sg}.\textsc{sm}-\textsc{past}-hit.\textsc{perf}  \textsc{1px}-person  \textsc{neg.cop}  9.cat\\
\glt     ‘I beat a person, not a cat.’
\z
\z

\subsection{Focus particles ‘only’ and ‘even’}\label{sec:vanderwal:4.4} %4.4

The exhaustive focus particle -\textit{okka} ‘only’ requires the noun it modifies to be [-A], which is expected considering that an exhaustive reading excludes all alternative referents.

\ea\label{ex:vanderwal:31}
\gll   Y-a-yérá    nnyúmba /  *e-nnyúmba  y-okká,  sí    ki-yûngu.\\
       \textsc{1sm}-\textsc{past}-sweep  9.house /  \textsc{9a}-9.house  9-only  \textsc{neg}.\textsc{cop}  \textsc{7px}-kitchen\\
\glt ‘She swept only the house, not the kitchen.’
\z

Equally expected is the finding that the focus particle ‘even/also’ cannot modify a [A] noun, as shown in \REF{ex:vanderwal:32}.\footnote{%
  Interestingly, negation (which normally licenses a [-A] object) does not affect this constraint:
    \ea 
      \ea 
      \gll \textup{Te-y-a-lyá}    \textup{n’}  \textup{ákáwûnga.}\\
	\textsc{neg}-\textsc{1sm}-\textsc{past}-eat  even  \textsc{12a}-\textsc{12px}-ugali\\
      \glt   ‘He didn’t even eat ugali.’ (though it’s his favourite dish)
      \ex
      \gll \textup{*Te-y-a-lyá}    \textup{né}  \textup{káwûnga.}\\
	\textsc{neg}-\textsc{1sm}-\textsc{past}-eat  even  \textsc{12px}-ugali\\
\glt	(‘He didn’t even eat ugali.’ (though it’s his favourite dish))
      \z
    \z
}

\ea\label{ex:vanderwal:32}
\ea\label{ex:vanderwal:32a}
\gll     Y-a-lyá    n’  á-ká-wûnga.\\
         \textsc{1sm}-\textsc{past}-eat  even  \textsc{12a}-\textsc{12px}-ugali\\
\glt     ‘He ate even ugali.’
\ex\label{ex:vanderwal:32b}
\gll     *Y-a-lyá    né  ká-wûnga.\\
         \textsc{1sm}-\textsc{past}-eat  even  \textsc{12px}-ugali\\
\glt     (‘He ate even ugali.’)
\z
\z

If [-A] indeed encodes exclusivity, the ungrammatical status of \REF{ex:vanderwal:32b} is explained by the fact that né ‘even, also, with’ includes all the less likely alternatives (not excluding any alternatives), and thus is incompatible with the exclusive meaning of the absence of the augment on the noun. This ungrammaticality would not be expected if the absence of the augment merely encodes focus (cf. \citealt{HymanKatamba1993}), since ‘even ugali’ is indeed the new information focus in this clause.

\subsection{Restricted interpretation of answers to content questions}\label{sec:vanderwal:4.5} %4.5

A fifth test involves the interpretation of answers to content questions. Either form of the object [+A/- A] is grammatical here, but whereas the [+A] answer is described as “neutral” \REF{ex:vanderwal:33b}, a contrast is spontaneously indicated by our consultants for the [A] form \REF{ex:vanderwal:33c}. 

\ea\label{ex:vanderwal:33}
\ea\label{ex:vanderwal:33a}
\gll     O-mu-nyá    gu-kutte    ki?\\
         \textsc{3a}-\textsc{3px}-gecko  \textsc{3sm}-catch.\textsc{perf}  what\\
\glt     ‘What has the gecko caught?’
\ex\label{ex:vanderwal:33b}
\gll    Gú-kutte    e-nsirî.\\
         \textsc{3sm}-catch.\textsc{perf}  \textsc{9a}-9mosquito\\
\glt     ‘It caught a mosquito.’
\ex\label{ex:vanderwal:33c}
\gll    Gú-kútté    nsírî.\\
         \textsc{3sm}-catch.\textsc{perf}  9mosquito\\
\glt ‘It caught a \textsc{mosquito}.’\\
Situation: “not something else”
\z
\z

\subsection{A note on the [-A] subject}\label{sec:vanderwal:4.6} %4.6

Just as a subject modified by ‘only’ is required to appear in the PFC (see \sectref{sec:vanderwal:3.4}), the subject cannot be [-A] in its canonical preverbal position \REF{ex:vanderwal:34b}, that is, the PFC is a precondition for [-A] subjects \REF{ex:vanderwal:34a}. For comparison, the [+A] subject is grammatical in the PFC \REF{ex:vanderwal:34c} as well as in SVO order \REF{ex:vanderwal:34d}.

\ea\label{ex:vanderwal:34}
\ea\label{ex:vanderwal:34a}
\gll   Mu-sómésa  y-e  y-a-mú-laba.\\
         \textsc{1px}-teacher  1-e  \textsc{1sm}-\textsc{past}-\textsc{1om}-see\\
\glt     ‘It’s the teacher that saw him.’
\ex\label{ex:vanderwal:34b}
\gll      *Mu-sómésa  y-a-mú-laba.\\
         \textsc{1px}-teacher  \textsc{1sm}-\textsc{past}-\textsc{1om}-see\\
\glt     (‘It’s the teacher that saw him.’)
\ex\label{ex:vanderwal:34c}
\gll    O-mu-sómésa  y-e  y-a-mú-laba.\\
         \textsc{1a}-\textsc{1px}-teacher  1-e  \textsc{1sm}-\textsc{past}-\textsc{1om}-see\\
\glt     ‘The teacher saw him.’
\ex\label{ex:vanderwal:34d}
\gll   O-mu-sómésa  y-a-mú-laba.\\
         \textsc{1a}-\textsc{1px}-teacher  \textsc{1sm}-\textsc{past}-\textsc{1om}-see\\
\glt     ‘The teacher saw him.’
\z
\z

The logical subject can, however, be [-A] in locative inversion. Locative inversion is a construction in which the logical subject appears in a linearly postverbal position while the preverbal position is filled by a locative expression like \textit{muno} ‘here’ in \REF{ex:vanderwal:35} or \textit{mukatale} ‘in the market’ in \REF{ex:vanderwal:36} and \REF{ex:vanderwal:37}. The function of this inversion is to detopicalise the subject, which results in a thetic interpretation when the subject is [+A] (see example \ref{ex:vanderwal:9} above), and an exclusive interpretation when it is [-A]. This exclusive interpretation can be seen in the contrast in \REF{ex:vanderwal:35}, the modification by ‘only’ in \REF{ex:vanderwal:36}, and the generic or type interpretation of the indefinite in \REF{ex:vanderwal:37}.

\ea\label{ex:vanderwal:35}
\gll   Mu-no    mú-súlá-mú    mu-lalû.\\
       18-\textsc{dem}  \textsc{18sm}-sleep-\textsc{18loc}  \textsc{1px}-crazy\\
\glt   ‘It’s a mad person who sleeps here.’ (nobody else)
\z

\ea\label{ex:vanderwal:36}
\gll   Mu-katále  mw-áá-báddé-mú      báána    b-okká.\\
       18-market  \textsc{18sm}-\textsc{past}-be.\textsc{perf}-\textsc{loc18}  2.children  2-only\\
\glt   ‘In the market were only children.’
\z

\ea\label{ex:vanderwal:37}
\gll   Mu-katále  mw-áá-báddé-mú    baantú,    si    mbwa.\\
       18-market  \textsc{18sm}-\textsc{past}-be.\textsc{perf}-\textsc{loc18}  2.people  \textsc{neg}.\textsc{cop}  10.dogs\\
\glt   ‘In the market were people, not dogs.’
\z

For completeness we mention that the agent NP in a passive can be [-A] as well, and as expected it triggers the exclusive interpretation \REF{ex:vanderwal:38}. The agent does not need a preposition ‘by’ in \ili{Luganda} (see \citealt{Pak2008}).

\ea\label{ex:vanderwal:38}
\gll   E-m-mése  y-a-l-ííbw-á      mú-sótâ.\\
       \textsc{9a}-\textsc{9px}-rat  \textsc{9sm}-\textsc{past}-eat-\textsc{pass}-\textsc{fs}  \textsc{3px}-snake\\
\glt   ‘The rat was eaten by a snake.’ (implying ‘not by a cat’)
\z

The syntactic licensing of augmentless nouns seems to be restricted to a vP-internal position (also depending on the analysis of the PFC as a cleft or not); cf. \citet{Halpert2012,Halpert2013} for this proposal for \ili{Zulu} [-A] nouns, and \citet{CarstensMletshe2015,CarstensMletshe2016} on \ili{Xhosa}. The syntactic structure will be left to one side here, since our main aim is to unravel the precise interpretation of focus strategies.

  In conclusion, the absence of the augment encodes a focus interpretation that is more specific than \citegen{HymanKatamba1993} underspecified focus, which the five tests in this section diagnose as exclusive focus. The next question is what happens when the identificational focus of the PFC and the exclusive focus of the augmentless nouns are combined.

\section{Combination of PFC and [-A]}\label{sec:vanderwal:5}
 
Identification and exclusion are very close to one another but are not identical, and the two focus marking strategies encoding them in \ili{Luganda} can be combined. On the basis of our analysis in the previous sections, we predict the following combinations, as also represented in Table 1 below:

\begin{itemize}[noitemsep]
 \item a postverbal [-A] object triggers an exclusive focus reading, excluding at least some of the alternatives;
 \item a preverbal [+A] subject or object in the PFC is identified as the referent for which the otherwise presupposed predication is true;
 \item a preverbal [-A] subject or object in the PFC marks the predicate as part of the presupposition AND excludes alternatives for the focused referent.
\end{itemize} 

\begin{table}
\caption{Predicted combinations of two focus strategies for \ili{Luganda} objects}
\begin{tabular}{cll} 
\lsptoprule
& {{SVO}} & {{PFC}}\\ 
\midrule{}
[-A] & {\mdseries exclusive} & {\mdseries exclusive identification}\\{}
[+A] & {\mdseries neutral/new info} & {\mdseries identification}\\
\lspbottomrule
\end{tabular}
\end{table}

We demonstrate the various combinations of the focus strategies and their interpretations in three example cases.

  First, we show the difference in interpretation between the [+A] and [-A] form of the subject in the PFC. As an answer to the subject question in \REF{ex:vanderwal:39}, the PFC will be used. Since it concerns subject focus, the PFC \textit{must} be used,\footnote{Locative inversion would be an alternative for subject focus, but only if there is indeed a topical locative. The point here is that the subject cannot occur in its canonical preverbal position if it is focused.} and this allows us to test the influence of [-A]. Since the only difference is the presence or absence of the augment, the difference in interpretation shows that exclusivity is \textit{only} linked to [A], not to the PFC. In both answers, there is a presupposition that there exists someone who is reading. The difference is found in the explanation that consultants provide for felicitous use of each example. When the answer has a [+A] noun in subject position \REF{ex:vanderwal:39a}, this simply identifies the one reading as a/the child, whereas when the answer has a [A] noun in subject position \REF{ex:vanderwal:39b}, this implies that there are others in the room and the child is identified to the exclusion of these alternatives.

\ea\label{ex:vanderwal:39}
{Sketched situation: I want to know who is reading in the room next door.}
\glt   Q: ‘Who is reading?’
\ea\label{ex:vanderwal:39a}
\gll    O-mw-ána  y-e  a-somâ.\\
         \textsc{1a}-\textsc{1px}-child  1-e  \textsc{1sm}-read\\
\glt     ‘It’s a/the child that is reading.’\\
Situation: There is only one person in the room.
\ex\label{ex:vanderwal:39b}
\gll    Mw-ána  y-e  a-somâ.\\
         \textsc{1px}-child  1-e  \textsc{1sm}-read\\
\glt   ‘It’s the child that is reading.’\\
Situation: There are other people in the room.
\z
\z

Second, the difference in interpretation between the two focus strategies is also detected in a focus test that \citet{Kiss2009} applies for \ili{Hungarian}. Numerals are normally interpreted as a lower boundary ‘at least this amount’ \citep{Horn1972,Levinson2000}, but in exclusive focus they refer to precisely that quantity. This is because alternative (higher) values are excluded. If the PFC is identificational but not inherently exclusive or exhaustive, we would not predict the PFC to result in the ‘exact amount’ reading. Indeed, the lower-bound reading of ‘at least one million’ is found in post-verbal position with a [+A] object, as in \REF{ex:vanderwal:40a}, and crucially does not change in the PFC, as shown in \REF{ex:vanderwal:41a}. When the object takes its [-A] form, the meaning narrows down to the precise the value given in the focused constituent: exactly one million. This is the case regardless of whether the object is in-situ after the verb \REF{ex:vanderwal:40b}, or in the PFC \REF{ex:vanderwal:41b}.

\ea\label{ex:vanderwal:40}
\ea\label{ex:vanderwal:40a}    {SVO [+A]}\\
\gll   Tw-étaaga  a-ka-kadde    ka-mú  o-ku-mála    é-nnyúmba.\\
       \textsc{1pl}.\textsc{sm}-need  \textsc{12a}-\textsc{12px}-million  12-one  \textsc{15a}-\textsc{15px}-finish  \textsc{9a}-9.house\\
\glt   ‘We need (at least) one million to finish the house.’
\ex \label{ex:vanderwal:40b}
 SVO [-A]\\
\gll Tw-é-taaga  ka-kadde  ka-mú  o-ku-mála    é-nnyúmba.\\
     \textsc{1pl}.\textsc{sm}-need  \textsc{12px}-million  12-one  \textsc{15a}-\textsc{15px}-finish  \textsc{9a}-9.house\\
\glt ‘We need (exactly) one million to finish the house.’ (we’ve calculated)
\z
\z

\ea\label{ex:vanderwal:41}
\ea\label{ex:vanderwal:41a}PFC [+A]\\
\gll A-ka-kadde  ka-mú  k-e  tw-etaagá  o-ku-mála    é-nnyúmba.\\
     \textsc{12a}-\textsc{12px}-million  12-one  12-e  \textsc{1pl}.\textsc{sm}-need  \textsc{15a}-\textsc{15px}-finish  \textsc{9a}-9.house\\
\glt ‘(About) one million we need to finish the house.’
\ex \label{ex:vanderwal:41b}
  PFC [-A]\\
\gll Ka-kadde    ka-mú  k-e  tw-etaagá  o-ku-mála    é-nnyúmba.\\
     \textsc{12px}-million  12-one  12-e  \textsc{1pl}.\textsc{sm}-need  \textsc{15a}-\textsc{15px}-finish  \textsc{9a}-9.house\\
\glt ‘(Exactly) one million we need to finish the house.’
\z
\z

These interpretations can be accounted for if the PFC merely presupposes that a certain amount of money is needed to finish the house, which is then identified to be one million (at least), while the absence of the augment encodes exclusivity, meaning that alternatives to the amount given are excluded, resulting in an exact reading.

A final example shows the same interpretations in the felicitous situations sketched for the different [+A/-A] forms provided by our consultants. The [-A] variant of the preverbal focused subject is interpreted as corrective or contrastive, i.e. excluding alternatives, as indicated for \REF{ex:vanderwal:42a}. This is not the case for the [+A] variant that serves as an identificational answer in \REF{ex:vanderwal:42b}.\footnote{Note again that the subject cannot be focused in its canonical preverbal position.}

\ea\label{ex:vanderwal:42}
\ea\label{ex:vanderwal:42a}
\gll   Aní  á-sula      mu  nnyúmba  eyo?\\
       1.who  \textsc{rel1sm}-sleep  18  9.house  9.\textsc{dem}\\
\glt   ‘Who sleeps in this house?’


\ex\label{ex:vanderwal:42b}
\gll     Mu-wála  ye  a-súlá-mû.\\
         \textsc{1px}-girl  1.e  \textsc{1sm}-sleep-\textsc{18loc}\\
\glt ‘A girl sleeps there.’\\
Situation 1: There are a boy and a girl. Which one sleeps here? (contrastive)\\
Situation 2: You expect a man to be sleeping there. (corrective)
\ex\label{ex:vanderwal:42c}
\gll    O-mu-wála  ye  a-súlá-mû.\\
         \textsc{1a}-\textsc{1px}-girl  1.e  \textsc{1sm}-sleep-\textsc{18loc}\\
\glt ‘The girl sleeps there.’\\
Situation: You know the girl.
\z
\z

Testing the combination of the PFC and augmentless nouns reinforces our analysis of the precise focus interpretation of these strategies: the PFC triggers a presupposition and identificational focus, whereas [-A] has an exclusive focus effect.

\section{Conclusion and further research}\label{sec:vanderwal:6}

Our research questions for this paper were 
(i) What morphosyntactic strategies does \ili{Luganda} use to express focus? 
(ii) What type of focus do these express? 
and (iii) How can different types of focus be identified in elicitation? We have shown that \ili{Luganda} uses the Immediate After Verb position, the Preverbal Focus Construction and augmentless nouns to express focus, and that, while the IAV position expresses an underspecified type of focus (in the sense of merely triggering a set of alternatives), the PFC expresses identificational focus and the augmentless nouns encode exclusive focus (in affirmative clauses). As regards methodology, while we acknowledge the value and necessity of spontaneous speech and corpora (especially in the area of information structure) we have shown how specific diagnostics for focus are useful, and in fact indispensable, in elicitation if one aims to establish the precise semantic and pragmatic interpretation of a certain strategy that is associated with focus. The spontaneous and elicited data should ideally go hand in hand.

This narrow study of the interpretation of focus in \ili{Luganda} triggers further questions, both for \ili{Luganda} and for the study of information structure in general. We mention three of the issues that arise.

First, we need to clarify the syntactic status of the PFC and [-A] nouns, as indicated above. The PFC shows properties of both cleft and non-cleft constructions, suggesting it is halfway in the grammaticalisation path from a cleft to a monoclausal focus construction (cf. \citealt{HeineReh1983,HarrisCampbell1995,VanderWalManiacky2015}). What should the synchronic syntactic analysis be? With respect to the augmentless nominals, their appearance seems to be restricted by position (vP-internal) and licensing by focus or negation. Could this be related to abstract Case licensing (cf. \citealt{Halpert2012,Halpert2013,CarstensMletshe2015,CarstensMletshe2016})? Does the ``size" of the nominal phrase play a role, with demonstratives also licensing [-A]?

Second, which features do we assume in the structural analysis of focus? Considering the relevance of exclusivity, more specific features than just an underspecified [focus] are clearly needed (cf. \citealt{NeelemanEtAl2009}). A further question concerns whether the morphosyntactic and interpretational effects are part of the syntax, or belong in (the interface of morphosyntax with) semantics and pragmatics.

This is related to a third point, which is the broader question of how we define focus and the pragmatic/semantic divide. The triggering of an alternative set, as proposed in Rooth’s (\citeyear*{Rooth1985,Rooth1992,Rooth1996}) Alternative Semantics, can be seen as a unified core function of focus, and the various types of focus are then the outcomes of additional pragmatic and semantic factors \citep{ZimmermannOnea2011}. Rather than having a dichotomy between identificational vs. new information (\citealt{Kiss1998}), or Kontrast vs. Rheme \citep{VallduviVilkuna1998}, \citet{Bazalgette2015} suggests a four-way distinction between types of focus, according to their semantic complexity: simple focus (triggering alternatives under Roothian semantics of focus and nothing else), focus with an implicature (triggering alternatives, and pragmatically implying exclusion or scalarity), focus with a presupposition, and focus with truth-conditional effects (triggering alternatives and operating upon them). The usefulness of this distinction is illustrated in this paper by the different linguistic strategies for expressing focus in \ili{Luganda} and their specific interpretation as identificational focus and exclusive focus.

\section*{Acknowledgements}

Jenneke’s work is part of the research project ``Rethinking Comparative Syntax", funded by the European Research Council Advanced Grant No. 269752. The collaboration on \ili{Luganda} was made possible thanks to research grants from the Alborada Trust and CAPREx. We would like to thank our language consultants in Mpigi and Masaka, as well as the native speaker linguists at Makerere University. Thanks furthermore to Larry Hyman, the editors of this volume (especially Sara Pacchiarotti and Doris Payne for their thorough reading and helpful suggestions), and the audiences at \ili{Bantu} SynPhonIS (November 2014), LingSoc (February 2015), the Fieldwork forum at Berkeley (February 2015) and ACAL 46 (March 2015) for comments and discussion.

\section*{Abbreviations and symbols}

High tones are marked by an acute accent, falling by a circumflex. Numbers in glosses refer to noun classes, but to persons when followed by \textsc{sg} or \textsc{pl}.
\medskip 

\begin{tabularx}{.45\textwidth}{lX}
* & ungrammatical\\
\# & infelicitous \\
\textsc{a} & augment \\
\textsc{appl} & applicative \\
\textsc{caus} & causative \\
\textsc{conn} & connective \\ 
\textsc{cop} & copula \\
\textsc{dem} & demonstrative \\
\textsc{fs} & final suffix \\
IAV & immediate after verb \\
\textsc{loc} & locative \\
\end{tabularx} 
\begin{tabularx}{.45\textwidth}{lX}
\textsc{neg} & negation \\
\textsc{om} & object marker \\
\textsc{pass} & passive \\
\textsc{past} & past tense \\
\textsc{perf} & perfective \\
PFC & preverbal focus construction \\
\textsc{px} & noun class prefix \\ 
\textsc{red} & reduplication \\
\textsc{rel} & relative \\
\textsc{sm} & subject marker \\
\\
\end{tabularx}
  
{\sloppy
\printbibliography[heading=subbibliography,notkeyword=this]
}
\end{document}