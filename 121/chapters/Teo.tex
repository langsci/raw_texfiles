\documentclass[output=paper]{langsci/langscibook} 
\title{The coding of identifiability in Mooré} 
\author{Amos Teo \affiliation{University of Oregon}
}
\chapterDOI{10.17169/langsci.b121.284} %will be filled in at production


\abstract{
This paper looks at the morpheme /-wã/ {\Tilde} /-ã/ in Mooré (Gur, Burkina Faso), which previous analyses (e.g. \citealt{Peterson1971,Canu1976,Nikiema1989}) call a “definite” marker. The paper aims to show that an analysis of /-wã/ {\Tilde} /-ã/ needs to consider how NPs are marked for the semantic and/or discourse-pragmatic function of referentiality and/or identifiability. By drawing on both elicited and textual data, the analysis shows that, in addition to marking identifiability, /-wã/ {\Tilde} /-ã/ is often used to mark contrastive focus. Furthermore, in some cases, /-a/ (without nasality) can still be used to mark an NP as both referential and identifiable to both the speaker and listener, but this depends on the phonological shape of the NP. The patterns suggest that one can analyze the marker /-ã/ as two morphemes: /-a/; and a separate nasal morpheme, with the later having the potential to also mark contrastive focus. However, this analysis is complicated by the phonological form of the NP. 
}

\maketitle
\begin{document}



%\textbf{Key words}: \ili{Mooré}, discourse, definiteness, focus. 


\section{Introduction}\label{sec:teo:1}

\ili{Mooré} (ISO: mos), also known as Mossi, is a \ili{Gur} language within the larger \ili{Niger-Congo} family. It is one of the main languages of Burkina Faso, with an estimated 5 million speakers in the country \citep{Lewis2013}. Estimates also give around 60,000 speakers living in neighboring countries, including Côte d’Ivoire, Benin, \ili{Togo} and Senegal.

Previous descriptions of \ili{Mooré} (e.g. \citealt{Peterson1971,canu1974,Canu1976,Nikiema1989}) point to the existence of a “definite” morpheme /wã/ that follows the noun phrase (NP) it modifies, along with its phonologically determined allomorph \mbox{/-ã/}. The latter is described as occurring after nasal consonants or when replacing the vowel in a suffix that begins with a consonant (i.e. the consonant is part of the suffix). In most analyses, the treatment of /-wã/ as a “definite” marker appears to be motivated by its translation into English as ‘the’ or French ‘le’ / ‘la’ / ‘les’. These analyses are problematic, given that the term \textit{definite} (or \textit{définisseur}) is not clearly defined in the descriptions cited above, along either structural or functional lines, nor are examples of /-wã/ {\Tilde} /-ã/ from actual discourse given in those descriptions.

This paper therefore aims to show that an analysis of /-wã/ {\Tilde} /-ã/ needs to consider how NPs are marked for the semantic / pragmatic notions of identifiability in \ili{Mooré}; I will reserve the term \textit{definite} to describe a structural or formal category (e.g. as per \citealt{DuBois1980}). By drawing on both elicited and textual data, the analysis will show that /-wã/ {\Tilde} /-ã/ is often used to mark contrastive focus in addition to identifiability, while /-a/, without nasality, can be used to mark an NP as identifiable to both the speaker and listener, at least in the speaker’s judgment of the listener’s state of mind. This suggests that one can analyze the form /-ã/ as consisting of two morphemes: /-a/; and a separate nasal morpheme, with the latter used to mark contrastive focus. However, this analysis is complicated by the phonological form of a lexical noun with its citation noun class suffix, specifically the vowel of the noun class suffix.

To illustrate the problem, consider the following elicited examples:

\ea\label{ex:teo:1}
\glll màm n\'ɛ\`ɛ b\'ʊ\'ʊʁá.{\rmfnm}\\
màm n\'ɛ-\`ɛ b\'ʊ\'ʊ-gá\\
1\textsc{sg.subj} see-\textsc{aff} goat-\textsc{cl}12  \\
\glt ‘I see a goat.’ (AT\_20140605){\rmfnm}

\z

\addtocounter{footnote}{-1}
\footnotetext{The 1st line of the examples gives a phonetic transcription of the data, while the 2nd line gives the proposed phonemic form with morpheme boundaries. Both lines are provided since some of the proposed phonemic forms differ slightly from the phonetic forms. Furthermore, if there are any errors or disagreements, future researchers can see if an error lies in the phonetic transcription or in the phonological analysis.} 
\stepcounter{footnote}
\footnotetext{The source for each example is given in parentheses with the name of the file. More discussion is still needed with regards to giving access to the recorded data.}

\ea\label{ex:teo:2}
\glll màm n\'ɛ\`ɛ b\'ʊ\'ʊʁ\'{ã}.\\
màm n\'ɛ-\`ɛ b\'ʊ\'ʊ-g-\'{ã}\\
\textsc{1sg}.\textsc{subj} see-\textsc{aff} goat-\textsc{cl12}-\textit{ã}\\
\glt ‘I see the goat.’ (that we talked about); or ‘I see \textbf{the goat}.’ (as opposed to seeing something else); or ‘I \textbf{see the goat}.’ (as opposed to not seeing the elephant) (AT\_20140605)
\z

\ea\label{ex:teo:3} 
\glll màm n\'ɛ\`ɛ wóbʁò.\\
màm n{\'ɛ}-{\`ɛ} wób-gò\\
\textsc{1sg.subj} see-\textsc{aff} elephant-\textsc{cl15}\\
\glt  ‘I see an elephant.’ (AT\_20140605)
\z

\ea\label{ex:teo:4}
\glll  màm n\'ɛ\`ɛ wóbʁà.\\
màm n\'ɛ-\`ɛ wób-g-à\\
\textsc{1sg.subj} see-\textsc{aff} elephant-\textsc{cl15-}\textit{à} \\
\glt ‘I see the elephant (that we talked about).’ (AT\_20140605)
\z

\ea\label{ex:teo:5}
\glll  màm n\'ɛ\`ɛ wóbʁ\`{ã}.\\
màm n\'ɛ-\`ɛ wób-g-\`{ã} \\
\textsc{1sg.subj} see-\textsc{aff} elephant-\textsc{cl15-}\textit{ã\'{}} \\
\glt ‘I see \textbf{the elephant}.’ (as opposed to seeing something else); or ‘I \textbf{see the elephant}.’ (as opposed to not seeing the elephant) (AT\_20140605)
\z

In \REF{ex:teo:1}–\REF{ex:teo:2}, we can see a two-way distinction between the absence vs. presence of /-ã/, which corresponds to the use of the indefinite or definite article in the English translation. However, in \REF{ex:teo:3}-\REF{ex:teo:5} we find a three-way distinction between no marker, /-a/ and /-ã/. Here, the nasal form is not clearly required for a definite reading of the NP. It will be later shown that in discourse the lack of nasality in an example like \REF{ex:teo:1} does not necessarily correspond to a non-identifiable (or indefinite) reading.

The paper is organized as follows: in \sectref{sec:teo:2}, I define the semantic / pragmatic notion of identifiability, and its relation to the structural category of definiteness. In \sectref{sec:teo:3}, I briefly look at some previous descriptions of “definiteness” in \ili{Mooré}. In \sectref{sec:teo:4}, I give a brief description of NP morphology. In \sectref{sec:teo:5}, I look at the expression of non-identifiable NPs in \ili{Mooré}; while in \sectref{sec:teo:6}, I consider the expression of identifiable NPs. In \sectref{sec:teo:7}, I consider some of the complications in analyzing how identifiability and focus are coded in \ili{Mooré}. Finally, in \sectref{sec:teo:8}, I summarize the analysis presented in this paper and propose further avenues for research.

The data for this study come from Timbwaoga Aimé Judicaël Ouermi, a male \ili{Mooré} speaker in his early 20s from Ouagadougou, Burkina Faso. It is acknowledged that future research will require the participation of more speakers of the language. The data were recorded at the University of Oregon, Eugene, over a 9 month period between 2013 and 2014. 


\section{Identifiability and Definiteness}\label{sec:teo:2}

An important distinction to make is between the formal categories of \textsc{definite / indefinite} and the semantic / pragmatic categories of \textsc{referentiality} and \textsc{identifiability}. According to \cite[280]{DuBois1980}, “[a] noun phrase is \textit{referential} when it is used to speak about an object as an object, with continuous identity over time”. The “object” in question could be physical or conceptual; specifically known or unknown; a single entity or multiple ones; and it may exist in the real world or in a hypothetical world, or “universe of discourse” as per \citet[388]{Givon2001}'s use of this term. A referential NP can function to either: (a) activate a “mental file” for a particular object; or (b) refer back to a “previously opened mental file”. This is in contrast to non-referential NPs, which are not sensitive to any previous mentions in a discourse, nor are they sensitive to any semantic distinction between singular and plural (at least in English; \citealt[210]{DuBois1980}). Non-referential Ns/NPs can appear: (a) as the modifier element in a compound; (b) in predicating expressions (denoting proper inclusion in a category); and (c) as what Du Bois refers to as “conflated objects”, where the object is non-individuated and “conflated” with the verb (similar to noun incorporation).

Once an NP is interpreted as referential, it can be interpreted as identifiable or non-identifiable. According to \citet[232]{DuBois1980}, “[i]dentification ordinarily involves singling out the particular referent intended by the speaker”. \citet[233]{DuBois1980} goes on to propose a “curiosity principle” that states, “A reference is counted as identifiable if it identifies an object close enough to satisfy the curiosity of the hearer” – though it may be more precise to think of this as what the speaker \textit{believes} is close enough to satisfy the curiosity of the hearer. The level of satisfaction seems to be based on the Gricean maxims of quantity and relevance – it is more common for people in everyday communication to only partially identify referents than to specify every single characteristic that would identify a referent as a unique entity.

The term \textit{definite} is often used to describe a referent that can be identified by the listener (as in \citealt[65]{Comrie1989}), or to describe a nominal expression denoting a referent that is presumed by the speaker to be identifiable by the listener (as in \citealt[79]{Lambrecht1994}). The reasons for following the latter option of using the term \textit{definite} to name a formal category (i.e. for particular forms), separate from the semantic or pragmatic concept of identifiability, are clear if we consider the use of the definite article \textit{the} in English. Typically, \textit{the} marks NPs that point to objects that are both referential and identifiable, as in \REF{ex:teo:6}. However, there are contexts in which \textit{the} can also mark a non-referential (and generic) NP, as in \REF{ex:teo:7}:

\ea\label{ex:teo:6}
I saw a small elephant get attacked by a lion. \textbf{The} elephant got pretty angry.
\z

\ea\label{ex:teo:7}
\textbf{The} elephant is the largest land animal on earth.
\z

Similarly, the indefinite article \textit{a / an} in English can mark a referential non-identifiable NP, as in \REF{ex:teo:8}, or a non-referential (and hence by default, non-identifiable) NP, as in \REF{ex:teo:9}.

\ea\label{ex:teo:8}
I’m looking for \textbf{a} black jacket. I think I left it here this morning. 
\z

\ea\label{ex:teo:9}
I’m looking for \textbf{a} black jacket. Do you sell any in your store? 
\z

Consequently, it is useful to reserve the terms \textit{definite} and \textit{indefinite} for structural / formal categories in a language, and to see how they align with the semantic / pragmatic functions of marking referentiality and / or identifiability, and perhaps even contrastive focus. As \citet[79]{Lambrecht1994} notes, the categories of formal definite marking and identifiable information status do not always align perfectly. We shall see that this is also the case for \ili{Mooré}.

\section{Previous descriptions of definiteness in Mooré}\label{sec:teo:3}

Most descriptions of \ili{Mooré} include a brief account of a “definite” marker /wã/. \citet[77]{Peterson1971} states that what he calls the “definitizer” has the form /wã/, which goes at the end of relative clauses and also on nouns, as in:

\ea\label{ex:teo:10}
Example as given in \citet[77]{Peterson1971}:\\
 kí w\'{ã}  [kí \'{w}\'{ã} ]\\
‘the millet’
\z

However, according to Peterson, if the marker follows an elided vowel (or a nasal consonant), the /w/ is deleted and the vowel merges with the word, as in the following examples:

\ea\label{ex:teo:11}
Examples as given in \citet[77]{Peterson1971}:\\
\gll {bòãŋgá}  {bòãŋg\'{ã}  [bw\`{ã}ŋg\'{ã}]} \\
{‘donkey’}  {‘the donkey’} \\
\gll {béngré}  {béngr\'{ã}  [béŋg\'{ə}r\'{ã}]} \\
‘bean’  {‘the bean’} \\
\z
%%check this typesetting!!

\cite[179]{canu1974} similarly posits the underlying form of the \textit{définisseur} ‘definitizer’ as /uã\={} /, which “dans le discour rapide et l’élocution relachée ... s’amalgame avec la dernièrne voyelle du nominal” (“in fast or relaxed speech ... merges with the final vowel of the noun”), as in:

\ea\label{ex:teo:12}
Examples as given in \citet[179]{canu1974}:\\
bá:gá u\={ã}  \textup{${\rightarrow}$}\textup{ /bá:g\'{ã} / [bá:ɣ\'{ã}]} \\
‘le chien’ (‘the dog’)\\

u\={o}bgò u\={ã}  \textup{${\rightarrow}$}\textup{ /u\={o}bg\`{ã} / [w\={o}bɣ\`{ã}]} \\
‘l’éléphant’ (‘the elephant’)\\

zóm u\={ã}  \textup{${\rightarrow}$}\textup{ /zóm\={ã} / [zóm\={ã}]} \\
‘la farine’ (‘the flour’)
\z

Similarly, \cite[96]{Nikiema1989} notes that: “[l]a marque du défini \textit{wã} est réduite à \textit{ã} après un mot terminé par une consonne” (“the definite marker \textit{wã} is reduced to \textit{ã} after a word ending in a consonant”); but he does not elaborate further.

Interestingly, in none of these sources is there any mention of the definitizer having the form /-a/ with no nasality. For instance, note the following example, presented earlier as \REF{ex:teo:4}:

\ea\label{ex:teo:4repeated}
\glll màm n\'ɛ\`ɛ wóbʁà.\\
 màm n\'ɛ-\`ɛ wób-g-à\\
\textsc{1sg.subj} see-\textsc{aff} elephant-\textsc{cl15-}\textit{à} \\
\glt ‘I see the elephant (that we talked about).’ (AT\_20140605)
\z
%%find out how to label this as 4 (repeated)

This leads us to the following questions that will be addressed in this paper:
 
\begin{itemize}[noitemsep]
	\item What is / are the function(s) of /-a/ on NPs? Is /-a/ a definite marker? How does it relate to the coding of referentiality and identifiability in \ili{Mooré}?
\item What, then, is the function of /-wã/ {\Tilde} /ã/ in Ouagadougou \ili{Mooré} as represented in the speech of the consultant for this work? Is this also a definite marker? Is its function different from what has been claimed in previous descriptions of \ili{Mooré}?
\end{itemize}

\section{Noun phrase structure}\label{sec:teo:4}
In order to understand NP marking, we must first look briefly at the structure of the Noun Phrase in \ili{Mooré}.\footnote{For reasons of space, we will not look at pronouns and proper nouns in this paper.} Table 1 gives a selection of noun classes and examples for each class, adopting the traditional \ili{Niger-Congo} noun class numbering system. The third column provides the citation form of the noun, which is the form given when the speaker is asked to give the \ili{Mooré} equivalent of an English or French word. For labeling convenience, I use the term \textit{citation form} to refer to such word forms, before assigning a functional label to them. The fourth column gives the form of each noun with the /-a/ suffix: in all cases, the vowel of the noun class suffix that we see in the citation form is elided. The fifth column gives the form of each noun with the other definite /-wã/ suffix, which is realized as \textit{-ã} when added to nouns with overt noun class markers in their citation form. Monosyllabic nouns which do not have an overt class marker, e.g. /kí/ ‘millet’, do not take the suffix /-a/, only the suffix /-wã/.

%\begin{tabularx}{\textwidth}{XXX>{\itshape}X}
\begin{table}[t]
  	\caption{\ili{Mooré} noun class markers and examples}
  \label{tab:teo:1} 
\begin{tabularx}{\textwidth}{lX>{\itshape}l>{\itshape}l>{\itshape}l}
\lsptoprule
Noun class & Gloss & \textup{Citation} & \textup{Form with} \textit{-a} & \textup{Form with}\textit{ (w)ã}\textup{\footnotemark{}}\\
\midrule
1 & ‘woman’ &  pág-á  &  pág-á  & pág-ã\\

 2 &‘women’ & pág-bá &  pág-b-á &  pág-b-ã\\
 
 5 & ‘rock’ &  kúg-rì &  kúg-r-á &  kúg-r-ã\\
 
6 &‘rocks’ & kúg-à &  kúg-à & kúg-{ã} \\

12 &‘goat’ & b\'ʊ\'ʊ-gá &  b\'ʊ\'ʊ-g-á &  b\'ʊ\'ʊ-g-ã\\

13 & ‘goats’ &  b\'ʊ\'ʊ-sé &  b\'ʊ\'ʊ-s-à &
\itshape b\'ʊ\'ʊ-s-ã\\

14 & ‘\textit{sagbo}’ (type of doughy food) &  ság-bó &  ság-b-á & ság-b-ã\\

15 & ‘elephant’ & wób-gò & wób-g-à & wób-g-ã\\

19 &
‘bicycle’ & wèè-fó &  wèè-f-á &  wèè-f-ã\\

21 & ‘grave’ & yáá-dò & yáá-d-à &  yáá-d-ã\\

22 & ‘alcohol’ & ráà-m  & ráà-m-à  &  ráà-m-ã\\

(no marker) & ‘millet’ & kí & - & kí-wã  \\
& ‘cart’ & ʃárétè & ʃárét-à & ʃárét-ã\\
\lspbottomrule
\end{tabularx}
\end{table}
\footnotetext{It is unclear what the tone on the final nasalized vowel is in each word. Although the consultant does not consider there to be any difference in pitch between the \textit{-a} and \textit{-wã} forms, a phonetic difference in pitch has sometimes been perceived by the researcher: specifically, a slight dip in pitch on the nasalized vowel.}

The vowel in the noun class suffix of a noun in the citation form is typically only produced at the end of a clause / sentence, as in \REF{ex:teo:13} and is elided in other contexts, as in \REF{ex:teo:14} and \REF{ex:teo:15}.

\ea\label{ex:teo:13}
\glll {à músá} n\'ɛ àlì yáá kàràmbíísì.\\
à=músá n\'ɛ àlì yà-à kàràmbíí-sì\\
\textsc{3sg.subj=pn} and \textsc{pn} \textsc{cop-aff} student-\textsc{cl13}\\
\glt ‘Moussa and Ali are students.’ (AT\_2014-02-25\_NominalPreds\_Existential, Ex.3)
\z

\ea\label{ex:teo:14}
\glll wóbs rítà t\'ɪ\`ɪsè.\\
 wób-s ríd-d-à t\`ɪ\`ɪ-sè\\
elephant-\textsc{cl13} eat-\textsc{prog-aff} tree-\textsc{cl13}\\
\glt ‘Elephants eat trees.’ (general) (AT\_2014-05-21\_nasality, Ex.1)
\z

\ea\label{ex:teo:15}
\glll wóbs rítà t\'ɪ\`ɪs dár f\'{ã}\`{ã}.\\
 wób-s ríd-d-à t\`ɪ\`ɪ-s dá-r f\'{ã}\`{ã}\\
elephant-\textsc{cl13} eat-\textsc{prog-aff} tree-\textsc{cl13} day-\textsc{cl5} all\\
\glt ‘Elephants eat trees all the time.’ (general) (AT\_2014-05-28)
\z

The syntactic role of the NP does not affect whether the final vowel of the citation form (third column of \tabref{tab:teo:1}) is produced or not: a comparison of \REF{ex:teo:14} with \REF{ex:teo:15} shows that the subject [wóbs] and the object\footnote{Here, I use the term \textit{subject} to refer to both the subject of a transitive clause and the subject of an intransitive clause, and \textit{object} to refer to the object of a transitive clause.} [t\'ɪ\`ɪs] both do not display the final vowel of their respective noun classes as long as they are not in clause-final position. However, it is possible for the final vowel to be produced in careful speech, or when listing out nouns, as in [m\`ɔ\`ɔdó] in (16).\footnote{It is possible that the final vowel is produced at the end of some kind of intonational unit, as opposed to some kind of syntactic unit.}

\ea\label{ex:teo:16}
\glll màm n\'ɛ\`ɛ m\`ɔ\`ɔdó lá t\`ɪ\`ɪsé.\\
 màm n\'ɛ-\`ɛ m\`ɔ\`ɔ-dó lá t\`ɪ\`ɪ-sé\\
1\textsc{sg.subj} see-\textsc{aff} bush-\textsc{cl21} and tree-\textsc{cl13}\\
\glt ‘I see bushes and trees.’ (AT\_2014-01-13\_WomanDonkeyCart\_Text Ex.17)
\z

It is important to note that for some noun classes, the noun with the definite /-a/ suffix is homophonous with the noun in its citation form, i.e. nouns from noun classes 1, 2, 6 and 12. However, despite the apparent homophony between some of the nouns in citation form and their /-a/ suffixed forms (fourth column in \tabref{tab:teo:1}), potential ambiguity is only an issue with such nouns in clause-final position, as with [kàràmbíígá] in \REF{ex:teo:17}. In general, it is possible to tell if a noun has the /-a/ suffix if it occurs in non-final position in a clause or sentence: the \textit{-a} suffix is not elided in words like [ráwá] in \REF{ex:teo:17} and [kàràmbíígà] in \REF{ex:teo:18} when in non-final position. Compare these with the elided form [káràmbííg] in examples \REF{ex:teo:19} and \REF{ex:teo:20}.

\ea\label{ex:teo:17}
\glll ráwá yàà káràmbíígà.\\
ráw-á yà-à kàràmbíí-gà\\
man-\textit{a} \textsc{cop-aff} student-\textsc{cl12}\\
\glt ‘The man is a student.’ (AT\_2014-02-11\_PropertyPredicates, Ex.25)
\z


\ea\label{ex:teo:18}
\glll {à músà} ká kàràmbíígà yè.\\
à=músà ká kàràmbíí-g-à yè \\
3\textsc{sg.subj=pn} \textsc{neg} student-\textsc{cl}12-\textit{a} \textsc{cfn}\\
\glt ‘Moussa is not the student (we talked about).’ (AT\_2014-03-11\_NominalPredicates, Ex.39)
\z

\ea\label{ex:teo:19}
\glll fò yáá kàràmbíìg là?\\
 fò yà-à kàràmbíí-g là?\\
2\textsc{sg.subj} \textsc{cop-aff} student-\textsc{cl}12 \textsc{q}\\
\glt ‘Are you a student?’ (AT\_2014-02-25\_NominalPreds\_Existential Ex.11)
\z

\ea\label{ex:teo:20}
\glll {à músá} ká kàràmbíìg yè.\\
 à=músá ká kàràmbíí-g yè\\
3\textsc{sg.subj=pn} \textsc{neg} student-\textsc{cl}12 \textsc{cfn}\\
\glt ‘Moussa is not a student.’ (AT\_2014-02-25\_NominalPreds\_Existential, Ex.1)
\z

In general, the /-a/ and /-wã/ forms appear only on the last element of the NP, where they replace the vowel of a class marker suffix. They also do not undergo vowel elision in non-final position in a clause. Deserving special mention are the demonstrative modifiers [káŋá] ‘this’ and [k\'{ã}\~{i}sá] ‘these’, which always end in \-[-a], even in non-clause-final position, as demonstrated by \REF{ex:teo:21}–\REF{ex:teo:23}.

\ea\label{ex:teo:21}
\glll nù(g) káŋá yàà bédrè.\\
 nù(g) kán-g-á yà-à béd-rè\\
hand \textsc{dem-cl}12-\textit{a} \textsc{cop-aff} big-\textsc{cl}5 \\
\glt ‘This hand is big.’ (KB\_20140224\_08\_Noun-phrase, Ex.2)
\z

\ea\label{ex:teo:22}
*nù(g) káŋ yàà bédrè.
\glt (‘This hand is big.’)
\z

\ea\label{ex:teo:23}
\glll kòr b\'ɛdà yììb k\'ã\~{i}sá yàà p\'ɛ\`ɛlsè.\\
 kòr b\'ɛd-à yììb k\'ã\~{i}-s-á yà-à p\'ɛ\`ɛl-sè\\
bag big-\textit{a} two \textsc{dem}-\textsc{cl13-}\textsc{def} \textsc{cop-aff} white-\textsc{cl13}\\
\glt ‘These two big bags are white.’ (KB\_20140224\_08\_Noun-phrase, Ex.29)
\z

However, in genitive and relator noun constructions, the definite /-a/ suffix is not necessarily the last element in the phrase. For example, in \REF{ex:teo:24}, [ròòdá] ‘houses’\footnote{An anonymous reviewer has pointed out that \textit{roodo / rooda} is not an acceptable plural of ‘house’, even taking some dialectal differences into account. However, this was the form produced by our consultant, which may reflect differences due to age, geography or the fact that the speaker has been living overseas for a number of years.} takes the /-a/ suffix, but [kwèlàms] ‘doors’ does not; while in \REF{ex:teo:25}, [f\'ɪlmà] takes the /-a/ suffix.

\ea\label{ex:teo:24}
\glll súg ròòdá kwèlàms yáá rààdó.\\
 sú-g ròò-d-á kwèlàm-s yà-à ràà-dó\\
thatch-\textsc{cl15} house-\textsc{cl21-}\textit{a} door-\textsc{cl13} \textsc{cop-aff} wood-\textsc{cl21}\\
\glt ‘The thatch house doors are (made of) wood.’ (AT\_2014-01-13\_Women-carrying-pots-on-head\_Text Ex.10-11)
\z

\ea\label{ex:teo:25}
\glll f\'ɪlmà s\'ɪŋrè\\
 f\'ɪlm-à s\'ɪŋ-rè\\
film-\textit{a} start-\textsc{cl5}\\
\glt ‘Aat the start of the film’ (Pear Story, Ln1)
\z

With this background on the structure of \ili{Mooré} nominals/NPs, we now turn to the discourse-based information structure status of nouns, and how those statuses are coded.

\section{Indefinite NPs}\label{sec:teo:5}

We first look at the distribution of NPs in citation form: recall that, as seen in \REF{ex:teo:26} and \REF{ex:teo:27}, these forms retain the vowel of the noun class suffix when the NP occurs in the final position in a clause or phrase (cf. [t\'ɪ\`ɪsè] in \REF{ex:teo:26}), but the final vowel is usually elided when the non-identifiable NP does not appear in final position (cf. [wóbs] and [t\'ɪ\`ɪs] in \REF{ex:teo:27}).

\ea\label{ex:teo:26}
\glll wóbs rítà t\'ɪ\`ɪsè.\\
 wób-s ríd-d-à t\`ɪ\`ɪ-sé\\
elephant-\textsc{cl13} eat-\textsc{prog-aff} tree-\textsc{cl13}\\
\glt ‘Elephants eat trees.’ (general) (AT\_2014-05-21\_nasality, Ex.1)
\z

\ea\label{ex:teo:27}
\glll wóbs rítà t\'ɪ\`ɪs dár f\'{ã}\`{ã}.\\
 wób-s ríd-d-à t\`ɪ\`ɪ-s dá-r f\'{ã}\`{ã}\\
elephant-\textsc{cl13} eat-\textsc{prog-aff} tree-\textsc{cl13} day-\textsc{cl5} all\\
\glt ‘Elephants eat trees all the time.’ (general) (AT\_2014-05-28)
\z

In semi-elicited narrative tasks, where the speaker was asked to describe what he could see in a picture or retell a story based on a video, the first mention of a referent is usually in the citation form of the noun, e.g. [ʃárétè] ‘cart’ and [rààdó] ‘wood’ in \REF{ex:teo:28}; [t\`ɪ bíísì] ‘fruit’ in .

\ea\label{ex:teo:28}
\glll màm n\'ɛ\`ɛ páʁá, bwàŋá, ʃárétè lá rààdó.\\
 màm n\'ɛ-\`ɛ pá-gá bwàn-gá ʃárétè lá ràà-dó\\
1\textsc{sg.subj} see-\textsc{aff} woman-\textsc{cl1} donkey-\textsc{cl12} cart and wood-\textsc{cl21}\\
\glt ‘I see a woman, a donkey, a cart and wood.’ (AT\_2014-01-13\_WomanDonkeyCart\_Text Ex.1)
\z

\ea\label{ex:teo:29}
\glll à y\'{ã}k\`{d} t\`ɪ bíísì.\\
 à y\'{ã}k-\`{d}{\rmfnm} t\`ɪ bíí-sì\\
3\textsc{sg.subj} pick-\textsc{prog} tree baby-\textsc{cl13}\\
\glt ‘He was picking fruit...’ (Pear Story, Ln3-a)
\z

\footnotetext{It is not clear why the speaker did not produce the suffix /-a/ ‘\textsc{aff}’ in this example.}

In these examples, the speaker is pointing out to the listener entities that have not yet been previously mentioned in the discourse. As such, these NPs denote referential, but non-identifiable entities. However, once the referents have been established in discourse, and are identifiable by the listener, subsequent mentions of the NP take the \textit{-à} suffix, as in [ʃárétà] \REF{ex:teo:30} and [t\`ɪ bíísà] in \REF{ex:teo:31}:

\ea\label{ex:teo:30}
\glll á ká zó ʃárétà yé.\\
 á ká zó ʃárét-à yé\\
 3\textsc{sg}\textsc{.subj} \textsc{neg} stand cart-\textsc{def}{\rmfnm}  \textsc{cfn}\\
\glt ‘She is not on the cart.’ (AT\_2014-01-13\_WomanDonkeyCart\_Text Ex.7)
\z
\footnotetext{There does not appear to be an overt locative marker in this sentence.}

\ea\label{ex:teo:31}
\glll t\`ɪ bíísà w\'ɛndà gwèyáàbè.\\
 t\`ɪ bíí-s-à w\'ɛn-dà gwèyáàbè\\
 tree baby-\textsc{cl13-}\textsc{def} resemble-\textsc{prog} guava\\
\glt ‘The fruits look like guava.’ (Pear Story, Ln3-b)
\z

In \cite{DuBois1980}’s terms, it appears that the non-definite NPs are used to activate a “mental file”, while the definite NPs are referring back to these “previously opened files”. Given that the citation forms of nominals are used to refer to entities that the speaker may presume are non-identifiable to the listener at that point in the discourse,\footnote{It is acknowledged that it is difficult to make claims about the mental state of the speaker and that psycholinguistic experiments may help to clarify this statement.} it would be suitable to call these the ‘indefinite’ forms of the nominals. Conversely, we might tentatively treat the /-a/ suffix as a ‘definite’ marker. More evidence of this will be provided in the following section.

We can look at other examples that support an analysis of the citation forms as indefinite NPs. The NPs in \REF{ex:teo:14} and \REF{ex:teo:15}, as well as in \REF{ex:teo:32} and \REF{ex:teo:33}, denote non-referential and non-identifiable entities, i.e. NPs that do not refer to a particular entity in the world or universe of discourse.\footnote{Although \REF{ex:teo:32} and \REF{ex:teo:33} make a number distinction, it is debatable whether the propositions behind these two generic statements actually have different semantic truth values.}

\ea\label{ex:teo:32}
\glll wóbs yàà b\'ɛdà.\\
 wób-s yà-à b\'ɛd-à\\
 elephant-\textsc{cl13} \textsc{cop-aff} big-\textsc{cl6}\\
\glt ‘Elephants are big.’ (in general) (AT\_2014-05-21\_nasality, Ex.1)
\z

\ea\label{ex:teo:33}
\glll wóbʁ yàà bédré.\\
 wób-g yà-à béd-ré\\
elephant-\textsc{cl12} \textsc{cop-aff} big-\textsc{cl5}\\
\glt ‘The elephant is a big animal.’ (Lit. ‘Elephant is big.’) (AT\_2014-05-21\_nasality, Ex.4)
\z

Another type of construction in which we find the indefinite forms of nouns are as part of predicating expressions that either (a) mark a referent as a member of a particular category, as in \REF{ex:teo:34} and \REF{ex:teo:35}; or (b) predicate some property about a referent, as in \REF{ex:teo:36}–\REF{ex:teo:39}.

\ea\label{ex:teo:34}
\glll {à músá} yàà káràmbíígà.\\
  à=músá yà-à kàràmbíí-gà\\
 \textsc{3sg.subj=pn} \textsc{cop-aff} student-\textsc{cl12}\\
\glt ‘Moussa is a student.’ (AT\_2014-02-25\_NominalPreds\_Existential Ex.1)
\z

\ea\label{ex:teo:35}
\glll {à músá} n\'ɛ àlì yáá kàràmbíísì.\\
 à=músá n\'ɛ àlì yà-à kàràmbíí-sì\\
\textsc{3sg.subj=pn} and \textsc{pn}  \textsc{cop-aff} student-\textsc{cl}13\\
\glt ‘Moussa and Ali are students.’ (AT\_2014-02-25\_NominalPreds\_Existential, Ex.3)
\z

\ea\label{ex:teo:36}
\glll ráàmà yáá tóóʁó.\\
 ráà-m-à yà-à tóó-gó\\
drink-\textsc{cl22-def} \textsc{cop-aff} bitter-\textsc{cl15}\\
\glt ‘The drink is bitter.’ (KB\_20140127\_04\_Non-verbal-clauses, Ex.5)
\z

\ea\label{ex:teo:37}
\glll màm rá yàà bédrè.\\
 màm rá yà-à béd-rè\\
\textsc{1sg.subj} \textsc{pst} \textsc{cop-aff} big-\textsc{cl5}\\
\glt ‘I was/had been big/fat.’ (MO\_20140205\_Session\_5, Ex.12)
\z

\ea\label{ex:teo:38}
\glll tónd rá yàà b\'ɛdà.\\
 tónd rá yà-à b\'ɛd-à\\
\textsc{1pl.subj} \textsc{pst} \textsc{cop-aff} big-\textsc{cl6}\\
\glt ‘We were/had been big/fat.’ (based on MO\_20140205\_Session\_5)
\z

\ea\label{ex:teo:39}
\glll súg ròòdá kwèlàms yáá rààdó.\\
 sú-g ròò-d-á kwèlàm-s yà-à ràà-dó\\
thatch-\textsc{cl15} house-\textsc{cl21}-\textsc{def} door-\textsc{cl13} \textsc{cop-aff} wood-\textsc{cl21}\\
\glt ‘The thatch house doors are (made of) wood.’ (AT\_2014-01-13\_Women-carrying-pots-on-head\_Text Ex.10-11)
\z

In \REF{ex:teo:34} and \REF{ex:teo:35} , [kàràmbíígà] and [kàràmbíísì] represent the category of ‘student’. In \REF{ex:teo:36}–\REF{ex:teo:38}, words like [tóóʁó] ‘bitter’ and [bédrè] ‘big’ semantically assign a quality to a referent. In \REF{ex:teo:39}, [rààdó] designates the material from which the referents are made, and does not refer to particular pieces of wood. Importantly, although these words denoting properties behave formally like nouns, in that they take noun class markers, they cannot take the definite suffix /-a/ in these constructions.\footnote{However, these non-referential predicates are still sensitive to a formal singular / plural distinction. They agree with the subject for number, as seen in \REF{ex:teo:34} and \REF{ex:teo:35}, as well as \REF{ex:teo:37} and \REF{ex:teo:38}.}

In summary, indefinite NPs in \ili{Mooré} are expressed by the citation form of the nominal and are used to refer to non-identifiable entities, i.e. entities that are presumed to be unidentifiable to the listener, as well as to describe properties of subjects in predicate position. In contrast, we shall see that definite NPs take the suffix /-a/, and denote entities that are presumably identifiable to both the speaker and listener.

\section{Definite NPs and contrastive focus}\label{sec:teo:6}

As mentioned in the previous section, once a referent has been established in discourse and is presumed by the speaker to be identifiable to the listener, subsequent mentions of the NP take the definite /-a/ suffix. If we look at the words for ‘bicycle’ (in bold) in the following examples, we can see that the first mention of [wéèfò]\footnote{The tones in [wéèfò] are the result of a tone sandhi rule, such that the underlying low tone on the first syllable of /wèèfó/ is realized as falling [wéèfò] when preceded by a word ending in low tone.} in the narrative is in the indefinite, but subsequent occurrances take the definite suffix /-a/. Similarly, the first mention of the ‘rock’ [kúgrì] is in the indefinite, but the next mention takes the definite suffix.

\ea\label{ex:teo:40}
\glll bííg h\'{ã} z\'ɔ \textbf{wéèfò} rà pyóʁdá t\`ɪ\`ɪʁá s\`ɛ\`ɛʁá.\\
 bíí-g h\'{ã} z\`ɔ wèè-fó rà pyóg-d-á t\`ɪ\`ɪ-g-á s\`ɛ\`ɛ-gá\\
child-\textsc{cl}12 \textsc{rel} stand bicycle-\textsc{cl}19 \textsc{pst} \textsc{pass-prog-aff} tree-\textsc{cl12-def} side-\textsc{cl12}\\
\glt ‘A kid who was sitting on a bicycle passed by the tree.’ (Pear Story, Ln22-23)
\z

\ea\label{ex:teo:41}
\glll tá \textbf{wèèfá} t\'ɔ\`ɔr t\`ɪ zùm kúgrì \textbf{wèèfá} h\'{ã} zùm kúgrà, \textbf{wéèfà} lw\'ɪ\`ɪ m\`ɛ. \\
 tɪ=á wèè-f-á t\'ɔ\`ɔ-r t\`ɪ zùm kúg-rì wèè-f-á h\'{ã} zùm kúg-r-à wèè-f-á lw\'ɪ-\`ɪ m\`ɛ\\
 \textsc{conn}=3\textsc{sg} bicycle-\textsc{cl19}-\textsc{def} front-\textsc{cl}5 \textsc{conn} sit rock-\textsc{cl}5 bicycle-\textsc{cl19}-\textsc{def} \textsc{rel} sit rock-\textsc{cl}5-\textsc{def} bicycle-\textsc{cl}19-\textsc{def} fall-\textsc{aff} \textsc{cf}\\
\glt ‘Then the front of his bicycle was on a rock. When the bicycle was on the rock, it fell down.’ (Pear Story, Ln40-44)
\z

More examples of definite forms of the noun occurring as either subject or object are given in \REF{ex:teo:42}–\REF{ex:teo:44}.

\ea\label{ex:teo:42}
\glll {à músá} n\'ɛ àlì yáá kàràmbíísà.\\
 à=músá n\'ɛ àlì yà-à kàràmbíí-s-à \\
\textsc{3sg.subj=pn} and \textsc{pn} \textsc{cop-aff} student-\textsc{cl13-def} \\
\glt ‘Moussa and Ali are the students (that the speaker and listener know about).’ (AT\_2014-06-04)
\z

\ea\label{ex:teo:43}
\glll wóbsà yáá b\'ɛdà.\\
 wób-s-à yà-à b\'ɛd-à\\
elephant-\textsc{cl13-def} \textsc{cop-aff} big-\textsc{cl6}\\
\glt ‘The elephants (the speaker and listener know about) are big.’ (AT\_2014-05-21\_nasality)
\z

\ea\label{ex:teo:44}
\glll wóbsà ríì t\'ɪ\`ɪsà.\\
 wób-s-à rí-ì t\`ɪ\`ɪ-s-à\\
elephant-\textsc{cl13-def} eat-\textsc{aff} tree-\textsc{cl13-def}\\
\glt ‘The elephants (the speaker and listener know about) ate the trees (the speaker and listener know about).’ (AT\_2014-05-28)
\z

The question therefore is if /-a/ is the definite suffix, what about /-wã/ {\Tilde} /-ã/, as claimed in previous analyses? In data from elicitation\footnote{Although examples from semi-elicited narratives would be ideal, the analysis of such data is complicated by the presence of a connective nasal morpheme that appears on the final element of clause in running speech with the meaning ‘and then’.} it appears that the use of the nasalized suffix is associated with a contrastive focus reading (indicated by underlining in examples). For example, in one reading of \REF{ex:teo:45}, the speaker presupposes that the listener believes some other entity to have eaten the trees. The nasalized suffix therefore highlights the argument ‘the elephants’ to correct this belief. However, even with the nasalized suffix, it is still possible to interpret the sentence without contrastive focus. It simply marked an entity that the speaker presumes is identifiable to the listener. Certainly, more data from naturalistic data would help to clarify this function of the nasalized suffix.

\ea\label{ex:teo:45}
\glll wóbsã{\rmfnm} ríì t\'ɪ\`ɪsà.\\
 wób-s-ã rí-ì t\`ɪ\`ɪ-s-à\\
elephant-\textsc{cl13-}\textit{ã} eat-\textsc{aff} tree-\textsc{cl13-def}\\
\glt ‘The elephants (the speaker and listener know about) ate the trees (the speaker and listener know about)’; or ‘\textbf{The elephants} (the speaker and listener know about) (not something else) ate the trees (the speaker and listener know about)’ (e.g. it wasn’t the mice that ate the trees). (AT\_2014-05-28)
\z
\footnotetext{It is unclear what the tone on this suffix is.}

In \REF{ex:teo:46}, the nasalized suffix also marks contrastive focus and either has scope over the argument ‘the elephant’ or over the entire predicate. At present, it is unclear if a reading without focus is available for this sentence and more naturalistic data is certainly needed. Nevertheless, it should still be noted that a contrastive focus reading is not available in \REF{ex:teo:47}, where [wóbʁà] does not take the nasalized suffix.


\ea\label{ex:teo:46}
\glll màm n\'ɛ\`ɛ wóbʁã.\\
 màm n\'ɛ-\`ɛ wób-g-ã\\
\textsc{1sg.subj} see-\textsc{aff} elephant-\textsc{cl15}-\textit{ã}\\
\glt ‘I see \textbf{the elephant}’ (as opposed to seeing something else); or ‘I \textbf{see the elephant}’ (as opposed to not seeing the elephant). (AT\_20140605)
\z

\ea\label{ex:teo:47}
\glll màm  n\'ɛ\`ɛ wóbʁà.\\
 màm n\'ɛ-\`ɛ wób-g-à\\
1\textsc{sg.subj} see-\textsc{aff} elephant-\textsc{cl15-def}\\
\glt ‘I see the elephant (that the speaker and listener talked about).’ (AT\_20140605)
\z

In the examples given above, the citation form noun class suffix vowel on the nouns is not /a/, e.g. /wób-gò/ ‘elephant-\textsc{cl15}’, /wób-sè/ ‘elephant-\textsc{cl13}’. In contrast, when the citation form noun class suffix vowel is /a/, e.g. /b\'ʊ\'ʊ-gá/ ‘goat-\textsc{cl12}’, there is a stronger preference for the speaker to use the nasalized /-ã/ form simply to indicate identifiability; compare \REF{ex:teo:48} and \REF{ex:teo:49}. This does not necessarily place any contrast or focus on the argument.


\ea\label{ex:teo:48}
\glll màm  n\'ɛ\`ɛ b\'ʊ\'ʊʁã.\\
màm n\'ɛ-\`ɛ b\'ʊ\'ʊ-g-ã\\
\textsc{1sg.subj} see-\textsc{aff} goat-\textsc{cl12}-\textit{ã}\\
\glt ‘I see the goat’ (that we talked about); or ‘I see \textbf{the goat}’ (as opposed to seeing something else); or ‘I \textbf{see the goat}’ (as opposed to not seeing the goat). (AT\_20140605)
\z

\ea\label{ex:teo:49}
\glll màm n\'ɛ\`ɛ b\'ʊ\'ʊʁá.\\
màm n\'ɛ-\`ɛ b\'ʊ\'ʊ-gá\\
\textsc{1sg.subj} see-\textsc{aff} goat-\textsc{cl12}\\
\glt ‘I see a goat.’ (AT\_20140605)
\z

When asked about \REF{ex:teo:49}, the consultant stated that [b\'ʊ\'ʊʁá] ‘goat’ in this example could only refer to a goat that has not been mentioned before; however, there is evidence from other examples – both from elicitation such as \REF{ex:teo:50}, and from narrative discourse such as \REF{ex:teo:51} – that even such NPs without the nasal marker can be interpreted as being identifiable.

\ea\label{ex:teo:50}
\glll {à músá} yàà {káràmbíígà /} {káràmbíígã}. \\
à=músá yà-à kàràmbíí-g-à kàràmbíí-g-ã\\
\textsc{3sg.subj=pn} \textsc{cop-aff} student-\textsc{cl12-def} student-\textsc{cl12}-\textit{ã}\\
\glt ‘Moussa is the student (we know about).’ (nasality preferred) (AT\_2014-03-11\_NominalPreds, Ex.34)
\z

\ea\label{ex:teo:51}
\glll páʁá búkà bíígá.\\
pág-á búk-à bíí-gá\\
woman-\textsc{def} carry-\textsc{aff} child-\textsc{cl12}\\
\glt ‘The woman (mentioned previously) is carrying a child.’ (AT\_2014-01-13\_WomanDonkeyCart\_Text, Ex2)
\z

Importantly, monosyllabic nouns that do not have a noun class marker in citation form (or, which have a zero-form class marker), e.g. /kí/ ‘millet’, can only be marked for identifiability with the nasalized suffix /-wã/ as in \REF{ex:teo:52} (and not with /-a/, /-ã/ or /-wa/). This nasalized form may also give the possibility of a contrastive focus reading.

\ea\label{ex:teo:52}
\glll kíwã  yáá k\'ʊdrè.\\
kí-wã  yà-à k\'ʊd-rè\\
millet-\textit{wã} \textsc{cop-aff} old-\textsc{cl5}\\
\glt ‘The millet (we know about) is old’; or ‘\textbf{The millet} (not something else) is old.’ (AT\_2014-06-04)
\z

For NPs ending in a demonstrative modifier, the nasal /-ã/ suffix occurs on the demonstrative (i.e. on the final element of the NP), where it also marks contrastive focus, as in \REF{ex:teo:53} and \REF{ex:teo:54}.

\ea\label{ex:teo:53}
\glll wób káŋ\'{ã} yàà bédré.\\
wób kán-g-\'{ã} yà-à béd-ré\\
elephant \textsc{dem-cl12}-\textit{ã} \textsc{cop-aff} big-\textsc{cl5}\\
\glt ‘\textbf{This elephant} is big.’ (e.g. correcting someone who has said ‘This mouse is big.’) (AT\_2014-05-21\_nasality, Ex.30)
\z

\ea\label{ex:teo:54}
\glll nù(g) káŋ\'{ã}  yàà bédrè.\\
nù(g) kán-g-\'{ã}  yà-à béd-rè\\
hand \textsc{dem-cl12}-\textit{ã} \textsc{cop-aff} big-\textsc{cl5}\\
\glt ‘\textbf{This hand} is the one that is big.’ (not something else) (KB\_20140224\_08\_Noun-phrase, Ex.2)
\z

The data raise two interesting points. Firstly, the marking of identifiability on NPs is dependent on the phonological shape of the head, or rather final noun in the phrase, specifically the vowel (if any) in the noun class suffix. The use of nasality may be necessary in some forms to help disambiguate the definite forms from their citation forms, or perhaps when the speaker feels the need to disambiguate the two. Secondly, the formal marking of identifiability in \ili{Mooré} is intimately linked to what we might call \textsc{focus}. Using \cite{Lambrecht1994}'s terminology, we typically find the nasalized suffixes /-wã/ {\Tilde} /-ã/ not just marking identifiability, but also \textsc{argument focus} where the subject or object is being contrasted with another possible referent, and sometimes even \textsc{predicate focus}, where the entire predicate is being contrasted with another one.


This then brings us to the question of whether the nasal feature of /-wã/ {\Tilde} /-ã/ can be considered a separate morpheme from /-a/. A preliminary analysis suggests that contrastive focus on indefinite NPs can be marked with a nasal feature (indicated in the second line of the examples just by a tilde) without the vowel /-a/. For example, in \REF{ex:teo:55} and \REF{ex:teo:56}, where there is no /a/ to carry the nasalization, the nasal feature is often heard as a nasalized schwa at the end of the NP in \REF{ex:teo:55}; or sometimes as a full syllabic nasal consonant that is phonologically bound to the following verb in \REF{ex:teo:56}.
 
\ea\label{ex:teo:55}
\glll w\'{o}bs\~{\textschwa} yá b\'ɛd-à.\\
wób-s-\textsuperscript{{\Tilde}}  yà b\'ɛd-à\\
elephant-\textsc{cl13}-\textsuperscript{{\Tilde}}  \textsc{cop} big-\textsc{cl6}\\
\glt ‘\textbf{Elephants} are big (in general).’ (e.g. correcting someone who said ‘Mice are big.’) (AT\_2014-05-21\_nasality, Ex.16)
\z
 
\ea\label{ex:teo:56}
\glll {y\'{\~{u}}ys\~{\textschwa} {{\Tilde}y\'{ũ}ysn}} dit sáʁbó.\\
y\'ũy-s-\textsuperscript{{\Tilde}}  ríd-d{\rmfnm} ság-bó\\
mouse-\textsc{cl}13-\textsuperscript{{\Tilde}} eat-\textsc{prog} dough-\textsc{cl14} \\
\glt ‘\textbf{Mice} eat \textit{sagbo} (a kind of doughy food).’ (e.g. correcting someone who said, ‘Elephants eat \textit{sagbo}.’) (AT\_2014-05-28)
\z
\footnotetext{It is not clear why the consultant did not produce the suffix /-a/ ‘\textsc{aff}’ in this example. Perhaps this is the case when focus is placed on the argument preceding the verb, and may reflect the historical origin of this construction coming from a cleft construction with a relativized clause.}

The presence of nasality without the /a/ vowel (i.e. the [\~{\schwa}] form) in some constructions suggests that /-ã/ could be analyzed as two separate morphemes: /-a/ plus nasalization.


In summary, although /-wã/ {\Tilde} /-ã/ are always found on identifiable NPs, the full range of data examined suggests that previous analyses of these suffix forms as the “definite” marker are not entirely accurate, at least for the speech of some \ili{Mooré} speakers. Rather, the nasalized forms play a role in marking contrastive focus, and for some nouns non-nasalized /-a/ is the indicator of non-contrastive identifiability. Thus, it may be more appropriate to think of the /-a/ on NPs as the definite marker (\textsc{def}), and the nasal /\textsuperscript{\Tilde}/ as a focus marker (\textsc{foc}) that gives the \textit{potential} for some sort of contrastive focus reading. These forms represent two separate structural categories that map onto different semantic and pragmatic functions depending on the construction and the context.


\section{Complications to the analysis of identifiability and focus}\label{sec:teo:7}

This study of identifiability marking of NPs is still preliminary, as more texts need to be collected and analyzed. One complication here, however, is that some non-final clauses in a chain in \ili{Mooré} are marked with a nasal connective morpheme. It is therefore not always possible to tell if the nasality on [b\'ʊ\'ʊʁã] in the first line of \REF{ex:teo:58} marks definiteness and focus on the NP, or if it marks non-finality of the clause, or both.

\ea\label{ex:teo:57}
\glll ráw hǹ ták\`{d} b\'ʊ\'ʊʁá píòʁà t\'ɪ\`ɪʁà s\'ɛ\`ɛʁà.\\
ráw hǹ ták-\`{d} b\'ʊ\'ʊ-gá píòg-à t\`ɪ\`ɪ-g-á s\`ɛ\`ɛ-gá\\
man \textsc{rel} pull-\textsc{prog} goat-\textsc{cl12} pass-\textsc{aff} tree\textsc{-cl12-def} side\textsc{-cl12}{\rmfnm}\\
\glt ‘A man who was pulling a goat passed by the tree.’ (Pear Story, Ln18-19)
\z
\footnotetext{In relator noun constructions, the relator noun comes after the head noun, but it is the head noun that takes the definite suffix /-a/.}
\ea\label{ex:teo:58}
\glll ráwá ták b\'ʊ\'ʊʁã píòʁ t\'ɪ\`ɪʁà s\'ɛ\`ɛʁà lâ lóóʁè.\\
ráw-á ták b\'ʊ\'ʊ-g-ã píòg t\`ɪ\`ɪ-g-á s\`ɛ\`ɛ-gá lâ lóóg-è\\ 
man-\textsc{def} pull goat-\textsc{cl12-}\textsc{def/conn?} pass tree\textsc{-cl12-def} side\textsc{-cl12} and leave-?\\
\glt ‘The man pulled the goat, passed by the tree and left.’ (Pear Story, Ln20)
\z

There are also examples from texts where the indefinite form of a noun is used to refer to an entity that has already been established in discourse. For example, in a story about a Rabbit and Hyena who go off in search of honey, we find in the second-to-the-last line the form [à s\'ɪ\`ɪdò] ‘his honey’, as shown in \REF{ex:teo:59}. In most of the preceding text, the honey has been referred to in the definite form [sɪɪdà]. Perhaps the overt definite marker is not necessary with possessed NPs, but this is something that needs to be checked. 


\ea\label{ex:teo:59}
\glll {...} {n d\'ɪk à} s\'ɪ\`ɪdò.\\
{...} n=d\'ɪk=à s\'ɪ\`ɪ-dò\\
{}  \textsc{conn}=take=\textsc{3sg} honey-\textsc{cl21}\\
\glt ‘... and took his honey.’ (DP\_Rabbit \& Hyena August 2013, Ln.20)
\z

Furthermore, the analysis has only looked at the coding of identifiability on NPs with common nouns as heads. Other kinds of NPs have been ignored, including NPs headed by borrowed nouns and NPs that refer to unique entities (entities that are referential and presumably identifiable to the listener). For example, [prezid\~{ɔ}] ‘the president (of Burkina Faso)’ in \REF{ex:teo:60}, is not formally coded with the definite suffix. The addition of the nasal \textit{-wã}, as in \REF{ex:teo:61}, necessarily leads to a contrastive focus interpretation of the argument.

\ea\label{ex:teo:60}
\glll prezid\~{ɔ} wád-dà=m\`ɛ. \\
prezid\~{ɔ} wád-dà=m\`ɛ \\
president come\textsc{-aff=cf}\\
\glt ‘The President (of Burkina Faso) is coming.’ (assuming one is in Burkina Faso, so it is clear which president you are talking about) (AT\_2014-06-04)
\z

\ea\label{ex:teo:61}
\glll prezid\~{ɔ}-wã wátàm\`ɛ.\\
prezid\~{ɔ}-wã wád-dà=m\`ɛ\\
president-\textsc{foc} come\textsc{-aff=cf}\\
\glt ‘\textbf{The President} is coming’ (not anyone else). (AT\_2014-06-04)
\z

As a final point, it should be mentioned that in addition to morphological marking, arguments in \ili{Mooré} can be brought into focus by syntactic means. For example, word order can be manipulated to bring an argument to the front of a clause, as in \REF{ex:teo:62}.

\ea\label{ex:teo:62}
\glll kàràmbíígà yáá máám.\\
kàràmbíí-g-à yà-à máám\\
student-\textsc{cl12-def} \textsc{cop-aff} \textsc{1sg.o}\\
\glt ‘The student (that we were talking about) is me.’ (AT\_2014-03-04\_NominalPreds, Ex.15)
\z

In addition, there is a construction containing the copula \textit{la}, which marks the subject for greater emphasis: [\textsc{subject} \textit{la} \textsc{predicate}]. This is illustrated in \REF{ex:teo:63} and \REF{ex:teo:64}. This is a more structurally marked construction, especially in \REF{ex:teo:63} where the object form of the pronoun appears preverbally, as one would normally expect it to appear post-verbally. In \REF{ex:teo:64}, the alternate \textit{kàràmbíígã} is preferred, though the form without the final nasalization is also possible. 

\ea\label{ex:teo:63}
\glll máám lá kàràmbíígà.\\
máám lá kàràmbíí-g-à\\
\textsc{1sg.o} \textsc{cop} student-\textsc{cl12-def}\\
\glt ‘\textbf{I} am the (one) student’ (not anyone else). (AT\_2014-03-04\_NominalPreds, Ex.16)
\z

\ea\label{ex:teo:64}
\glll à=músá lá {kàràmbíígà /} kàràmbíígã. \\
à=músá lá kàràmbíí-g-à kàràmbíí-g-a-\textsuperscript{\~{}} \\
\textsc{3sg.subj=pn} \textsc{cop} student-\textsc{cl12-def} student-\textsc{cl12-def-foc}\\
\glt ‘\textbf{Mo}\textbf{ussa} is the (one) student’ (not anyone else). (AT\_2014-03-11\_NominalPreds, Ex.31)
\z

\section{Summary and further research}\label{sec:teo:8}

It is hoped that this new analysis of /-a/ as a definite marker in \ili{Mooré} will serve as a useful update to previous analyses (e.g. \citealt{Peterson1971,canu1974}) that have only considered the nasal forms /-wã/ {\Tilde} /-ã/. The paper has also argued that /-ã/ could be treated as bimorphemic, with the nasal element marking (potentially contrastive) focus. However, it is more difficult to morphologically analyze \mbox{/-wã/}, which is used with monosyllabic nouns like /kí/ ‘millet’. Most importantly, it has shown that the three forms /-a/, /-ã/ and /-wã/ may share similar semantic / pragmatic functions of coding identifiability, but this depends on the underlying phonological form of the noun class suffix.

The link between identifiability and the marking of contrastive focus is particularly interesting, though perhaps not too surprising given that the marking of definiteness is also associated with speaker and listener attention. In \ili{Dagbani}, a related \ili{Gur} language, \citet[40]{Olawsky1999} reports two “definite articles”: \textit{la} and \textit{maa}, with the latter described as a “strong” article that adds “more emphasis” than the former.\footnote{\cite{KroppDakubu1991}, cited in \citet[30]{Olawsky1999}, suggests that meaning of a particle \textit{la} in \ili{Dagaare} and \ili{Gurune} appears to be related to definiteness and new information or focus.} In general, the languages of West Africa look like a fertile place to examine the interaction between morphosyntax and information structure. Previous studies of these languages include \cite{FiedlerSchwarz2005} that looks at the “non-focal” part of sentences\footnote{This appears to be similar to \cite{Lambrecht1994}'s notions of “predicate focus” and “sentence focus”.} in five \ili{Kwa} and \ili{Gur} languages, as well as \cite{Schwarz2010gur} on argument and predicate focus in four \ili{Gur} languages; neither of these include data on \ili{Mooré}.

Future studies could also investigate how syntactic focus-marking strategies interact with the use of the nasal focus morpheme. For instance, \citet[445-450]{Kabore1985} looks at a number of constructions that pertain to the marking of contrast and emphasis, including what he writes as a construction that uses /s\~{i}/ which seems to corresponds to a /h\~{\schwa}/ subordinator / relativizer in the data collected here. Indeed, further studies of identifiability and focus in \ili{Mooré} will need to consider relativization strategies.

The study of information structure calls for a move away from elicited data and demands much more work in the analysis of texts collected from a wide range of speakers across different genres. More importantly, it suggests the need for greater experimental work that considers the role of speaker and listener attention and which provides ways of describing cognitive categories \textit{independent} of the linguistic correlates that linguists traditionally use as evidence for such cognitive categories. It would seem that we are just barely beginning to scratch the surface when it comes to understanding the notions of \textit{referentiality}, \textit{identifiability} and \textit{contrastive focus} and how they might be coded by linguistic expressions.

\section*{\textbf{Acknowledgments}}

I would like to thank Timbwaoga Aimé Judicaël Ouermi for all his time and patience and Doris Payne for her advice and expertise, as well as two anonymous reviewers for this paper. I am also grateful to my fellow researchers in the 2014-2015 University of Oregon field methods class. All mistakes in the data are my own, and transcriptions of tone should be considered preliminary at best.

\section*{Abbreviations}

\begin{tabularx}{.55\textwidth}{lQ}
1,2,3 &   1st, 2nd, 3rd person    \\
\textsc{aff} & affirmative (on the first verb in a clause chain / verb in a single  clause)\\
\textsc{cf} &  affirmative clause-final marker\\
\textsc{cfn} & negative clause-final marker\\
\textsc{cl} & noun class marker\\
\textsc{conn} & connective\\
\textsc{cop} & copula\\
\textsc{def} & definite\\
\textsc{dem} & demonstrative\\
\textsc{emph} & emphatic pronoun\\
\end{tabularx}
\begin{tabularx}{.45\textwidth}{lX}
\textsc{foc} & argument focus\\
\textsc{fut} & future tense\\
\textsc{neg} & negative\\
\textsc{o} & object form (of pronoun)\\
\textsc{obj} & pronominal object suffix\\
\textsc{pl} & plural\\
\textsc{prog} & progressive\\
\textsc{pst} & past tense\\
\textsc{rel} & relativizer\\
\textsc{q} & question\\
\textsc{subj} & subject form (of pronoun)\\
\textsc{sg} & singular\\ 
\end{tabularx}

 \printbibliography[heading=subbibliography,notkeyword=this]

\end{document}