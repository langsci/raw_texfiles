\documentclass[output=paper]{langsci/langscibook} 
\title{Remote past and phonological processes in Kaonde} 
\author{%
Boniface Kawasha\affiliation{Savannah State University}
}
\chapterDOI{10.17169/langsci.b121.481} %will be filled in at production



\abstract{This paper examines the various morphophonological processes triggered by the remote past suffix -\textit{ile} and the environments in which they occur in Kaonde (Bantu L41), a language spoken in Zambia. The phenomena include vowel height harmony, partial nasal consonant harmony, complete consonant assimilation, alveolar palatalization, and imbrication. Depending on the verb structure and phonological environment, the phenomena affect either the final consonant of the verb root or base, the remote past suffixal consonant, or both the verb stem-final consonant and consonant of the suffix. The remote past suffix \textit{-ile} exhibits at least ten morphophonemic alternations. The final consonant of certain verb roots or bases also alternates between alveolar and palatalized affricate pronunciations when in contact with the high front vowel of the suffix. The imbrication process displays different patterns, and may be accompanied by segment deletion, vowel coalescence, vowel lengthening, and glide formation.
}

\maketitle
\begin{document}



% Key words: \textit{morphophonology, harmony, assimilation, palatalization, imbrication}

\section{Introduction}\label{§1:introduction.kawasha}

This paper examines the various types of morphophonological processes triggered by the remote past inflectional suffix -\textit{ile} and the environments in which they occur when the suffix combines with a verb root or base in \ili{Kaonde} (\ili{Bantu} L41; \citealt{guthrie1967}), one of the languages spoken in northwest Zambia. The processes include vowel height harmony, partial nasal consonant harmony, complete consonant assimilation, alveolar palatalization, and imbrication. Depending on the phonological properties of the verb, the phenomena affect either the final consonant of the verb root or verb base, the remote past suffix lateral consonant, or both the verb stem-final consonant and the suffix. The remote past suffix shows at least ten morphophonemic alternations which are \textit{-ile, -ile, -ele, -ine, -ene, -ishe, -eshe, -izhe, -ezhe, -inye} and -\textit{enye}.\footnote{The phonemes /ʃ/, /ʒ/, /tʃ/, /dʒ/ and /\href{http://en.wikipedia.org/wiki/Ɲ}{ɲ}/ are orthographically represented by \textit{sh, zh, ch, j,} and \textit{ny}, respectively, in \ili{Kaonde}. Morphemes are generally cited in their orthographic form, but IPA symbols are used at times to represent specific phonemic or phonetic forms.} In addition, the final consonant of a number of verb roots or bases alternates between the voiceless alveolar stop /t/ and the voiceless affricate /tʃ/, and between the voiced alveolar stop /d/, or the lateral /l/ and the voiced affricate /dʒ/ when it comes into contact with the high front vowel of the suffix. With respect to imbrication, the process displays different patterns, and may be accompanied by other processes that include segment deletion, vowel coalescence, vowel lengthening, and glide formation.

In \ili{Kaonde}, the remote past tense is expressed by means of the inflectional suffix -\textit{ile} in conjunction with the prefix tense maker \textit{-a-}. Though the morpheme is referred to as \textit{perfect} or \textit{perfective} in many \ili{Bantu} languages, it codes only the remote past tense and cannot be used to express any other meaning in \ili{Kaonde} \citep[137]{wright1977}; it is also attested with the same function in other neighboring languages, namely \ili{Chokwe}, \ili{Luvale} (\citealt{white1947}; \citeyear{white1949}; \citeyear{white196x}; \citealt{horton1949}; \citealt{yukawa1987}) and \ili{Lunda} (\citealt{fisher1984,kawasha2003}).\footnote{\citep{white196x} is a manuscript without an exact date. The perfective or perfect meaning “does not seem to fit its use in \ili{Kaonde}” \citep{wright1977}.} \citet[6]{white1947} points out that “this tense is used to denote a past action prior to yesterday” in \ili{Chokwe}, Luchazi, \ili{Lunda}, and \ili{Luvale}. The same applies to \ili{Kaonde}. Verb forms in the remote past are shown in \REF{ex:1.kawasha}.\footnote{Some of the data used in this paper are based on my own knowledge of the language, while others are drawn from various written sources. I verified them with speakers of \ili{Kaonde}.}


\ea
\label{ex:1.kawasha}
	\ea\label{ex:1a.kawasha}
\gll n-a-kay-ile \\
	\textsc{1sg-tns}-play-\textsc{rp} \\
\glt `I played.'

	\ex\label{ex:1b.kawasha}
\gll w-a-tang-ile \\
	\textsc{cl1}-\textsc{tns}-read-\textsc{rp} \\
\glt `He read.'
	\z
\z 


The sentences in \REF{ex:1.kawasha} consist of the first person singular subject prefix \textit{n-} and the \textsc{cl1} (third person) subject prefix \textit{w-} in initial position followed by the tense marker \textit{-a-} attached directly to the verb roots \textit{kay} ‘play’ and \textit{tang} ‘read’ with the remote past suffix -\textit{ile} as the last verbal element. \tabref{tab:1.kawasha} presents examples of simple verb stems on the left, and their remote past forms in the right-hand column. In the examples in the left-hand column of \tabref{tab:1.kawasha}, the verb stem is composed of the root and the final vowel \textit{-a}. On the right is the verb form in the remote past consisting of the verb root and the suffix -\textit{ile}.


\begin{table}
\begin{tabularx}{\textwidth}{XXXX}
\lsptoprule
  Stem &  &   Root-remote past & \\
\midrule
\textit{j-a} & `eat' & \textit{j-ile} & `ate' \\
\textit{fw-a} & `die' & \textit{fw-ile} & `died' \\
\textit{fik-a} & `arrive' & \textit{fik-ile} & `arrived' \\
\textit{amb-a} & `say' & \textit{amb-ile} & `said' \\
\textit{end-a} & `move' & \textit{end-ele} & `moved' \\
\textit{imb-a} & `sing' & \textit{imb-ile} & `sang' \\

\lspbottomrule
\end{tabularx}

\caption{Simple and remote past verb stems}
\label{tab:1.kawasha}

 \end{table}


This paper is organized as follows: \sectref{§2:harmony.kawasha} discusses the various phonological processes that affect either the vowel /i/ or the consonant /l/ of the remote past suffix -\textit{ile} in \ili{Kaonde}. \sectref{§3:palatalization.kawasha} examines palatalization which only affects the final consonant of the verb root or base. \sectref{§4:imbrication.kawasha} presents the imbrication process and the concomitant phonological changes it brings about. \sectref{§5:conclusion.kawasha} concludes the paper.


\section{Harmony} \label{§2:harmony.kawasha}

\ili{Kaonde} has three types of assimilatory phonological processes whereby segments become identical or more or less identical. They include vowel height harmony, partial nasal consonant harmony, and coronal consonant assimilation. They all target either the consonantal segment /l/ or the initial vowel of the suffix by spreading a feature or features rightward from the verb base to the suffix.

\subsection{Vowel height harmony}

Like most of the languages spoken in Zambia, the process of vowel height harmony involves assimilation of the suffixal vowel /i/ in \ili{Kaonde}. In other words, a verb root vowel induces assimilation in the vowel of the suffix. If the verb root contains the vowel /i/, /u/, or /a/, the first vowel of the suffix is a high front vowel. However, mid vowels /e/ and /o/ trigger a left-to-right direction of harmony targeting the initial high vowel of the remote past suffix, which assimilates completely to the height of the preceding mid vowel. It becomes mid front /e/ if it is preceded by a mid vowel. This type of height harmony occurs across an intervening consonant.

\ea
\label{ex:2.kawasha}
i → e/ mid vowel (C) \_\_\_

\z

Compare the examples in \tabref{tab:2.kawasha} and \tabref{tab:3.kawasha}. Verb roots with non-mid vowels in \tabref{tab:2.kawasha} have an inflectional suffix with an initial high front vowel /i/, while those with mid vowels in \tabref{tab:3.kawasha} contain a suffix with an initial mid vowel /e/. 

\begin{table}
\begin{tabularx}{\textwidth}{lXlX}
\lsptoprule
Root-FV & & \multicolumn{2}{l}{Root-remote past} \\
\midrule
\textit{amb-a} & ‘speak’ & \textit{amb-ile} & ‘spoke’\\
\textit{lamb-a} & ‘be smooth’ & \textit{lamb-ile} & ‘was smooth’\\
\textit{zh-a} & ‘dance’ & \textit{zh-ile} & ‘danced’ \\
\textit{languluk-a } & ‘think’ &\textit{languluk-ile} & ‘thought’ \\
\textit{itab-a } & ‘answer’ &\textit{itab-ile } & ‘answered’ \\
\textit{bijik-a } & ‘shout’ & \textit{bijik-ile } & ‘shouted’ \\
\textit{jimb-a } & ‘deceive’ & \textit{jimb-ile } & ‘deceived’ \\
\textit{imb-a } & ‘sing’ & \textit{imb-ile } & ‘sang’ \\
\textit{zhiik-a } & ‘bury’ & \textit{zhiik-ile } & ‘buried’ \\
\textit{ub-a } & ‘do’ & \textit{ub-ile } & ‘did’ \\
\textit{yuk-a } & ‘know’ &\textit{yuk-ile } & ‘knew’ \\
\lspbottomrule
\end{tabularx}

\caption{No height harmony in remote past suffix}
\label{tab:2.kawasha}

 \end{table}



\begin{table}

\begin{tabularx}{\textwidth}{lXlX}
\lsptoprule
\textit{keb-a}   & ‘look for’    & \textit{keb-ele}    & ‘looked for’    \\
\textit{fwenk-a} & ‘swim’    & \textit{fwenk-ele}  & ‘swam’  \\
\textit{kenket-a}    & ‘look around’ & \textit{kenket-ele} & ‘looked around’ \\
\textit{leet-a}  & ‘bring’   & \textit{leet-ele}   & ‘brought’   \\
\textit{nemb-a}  & ‘write’   & \textit{nemb-ele}   & ‘wrote' \\
\textit{lob-a}   & ‘catch fish’  & \textit{lob-ele}    & ‘caught fish’   \\
\textit{lomb-a}  & ‘ask’ & \textit{lomb-ele}   & ‘asked’ \\
\textit{owv-a}   & ‘wash’    & \textit{ovw-ele}    & ‘washed’    \\
\textit{sok-a}   & ‘burn’    & \textit{sok-ele}    & ‘burned’    \\
\lspbottomrule
\end{tabularx}

\caption{ Mid vowel height harmony in remote past suffix}
\label{tab:3.kawasha}

 \end{table}




\subsection{Nasal consonant harmony}

 
\ili{Kaonde}, like a many other \ili{Bantu} languages \citep{greenberg1951}, presents partial nasal consonant harmony that operates on suffixes if a verb root ends in a non-palatal or non-velar nasal /m/ or /n/. This process assimilates the segment /l/ of the suffix to an alveolar nasal by harmonizing it with the preceding nasal segment of the root in a single feature. That is, the base nasal triggers left-to-right harmony targeting the lateral consonant in suffix.\footnote{Unlike some \ili{Bantu} languages such as \ili{Kikongo}, Kiyaka, \ili{Chokwe}, etc. nasalization does not operate in \ili{Kaonde} if the input is NVC-\textit{ile} or CVNVC-\textit{ile}, where the final consonant of the verb root is not a nasal.} Notice that nasal consonants induce nasalization of underlying /l/ of the remote past suffix in \tabref{tab:4.kawasha}. 
 
Harmony does not normally operate on consonant clusters consisting of a nasal and an oral consonant. The final example in part (a) of \tabref{tab:4.kawasha} shows nasal harmony even though the consonant /v/ intervenes between the high vowel and the verb stem nasal. This could be attributed to the root ending in a vowel. 

\begin{table}
\small
\begin{tabularx}{\textwidth}{llQll}
\lsptoprule
 & Root-\textsc{fv} &  & \multicolumn{2}{l}{Root-remote past}\\
\midrule
a. & \textit{lam-a} & `keep' & \textit{lam-ine} & `kept' \\
 & \textit{kaan-a} & `refuse' & \textit{kaan-ine} & `refused' \\
 & \textit{jim-a} & `cultivate' & \textit{jim-ine} & `cultivated' \\
 & \textit{zhindam-a} & `be quiet' & \textit{zhindam-ine} & `was/were quiet' \\
 & \textit{pum-a} & `beat' & \textit{pum-ine} & `beat' \\
 & \textit{nun-a} & `become fat' & \textit{nun-ine} & `became fat' \\
 & \textit{unvw-a} & `understand, hear, feel' & \textit{unvw-ine} & `heard' \\
\tablevspace
b. & \textit{nyem-a} & `run (away from)' & \textit{nyem-ene} & `ran, ran away from' \\
 & \textit{nem-a} & `be heavy' & \textit{nem-ene} & `was/were important' \\
 & \textit{sem-a} & `give birth' & \textit{sem-ene} & `gave birth' \\
 & \textit{ten-a} & `mention' & \textit{ten-ene} & `mentioned' \\
 & \textit{tom-a} & `drink' & \textit{tom-ene} & `drank' \\
 & \textit{pon-a} & `fall' & \textit{pon-ene} & `fell' \\

\lspbottomrule
\end{tabularx}

\caption{Nasal consonant harmony in remote past suffix}
\label{tab:4.kawasha}

\end{table}

\subsection{Complete consonant assimilation}
 

In addition to vowel height harmony and nasal consonant harmony, \ili{Kaonde} also shows complete consonant assimilation across a vowel. This is the type of progressive assimilation that affects the consonant /l/ of the remote past suffix when it attaches to verb stems that contain the alveopalatal fricative consonants /ʃ/, /ʒ/. The verb can be an underived monomorphemic root or a derived causative. The segment assimilates totally to the preceding verb root-final consonant becoming \textit{sh}. Thus, the remote past suffix -\textit{ile} surfaces as [iʃe], and [iʒe]. The changes in \ili{Kaonde} are schematized in \REF{ex:3.kawasha}:

\ea
\label{ex:3.kawasha}
/il/ → [iʃ], [iʒ] after verb stems ending in /ʃ/ and /ʒ/.
\z

Some examples of lateral /l/ assimilation to alveopalatal fricative consonants are in \tabref{tab:5.kawasha}. In \tabref{tab:5.kawasha}, part a, the root-final consonant spreads rightward all its features to the suffixal consonant /l/, while in part b the consonant of the suffix assimilates completely to the causative suffix of the verb stem. 

\begin{table}[t]
\begin{tabularx}{\textwidth}{llllX}
\lsptoprule
 & Root-\textsc{fv} &  & \multicolumn{2}{l}{Root-remote past} \\
\midrule
a. & \textit{konsh-a} & `can' & \textit{konsh-esh-e} & `could' \\
 & \textit{baanzh-a} & `delay' & \textit{banzh-izh-e} & `delayed' \\
 & \textit{kozh-a} & `make sick, hurt' & \textit{kozh-ezh-e} & `made sick, hurt' \\
\tablevspace
b. & \textit{jiish-a} & `feed' & \textit{jiish-ish-e} & `fed' \\
\lspbottomrule
\end{tabularx}

\caption{Alveopalatal fricative assimilation in remote past}
\label{tab:5.kawasha}

\end{table}

This same process also affects causativized stems that are derived from a very restricted set of verb roots that still employ the now obsolete causative suffix \textit{-i}. When this morpheme attaches to a verb root ending with a lateral consonant, the segment /l/ of the root palatalizes to the voiced alveopalatal fricative. Compare the underived verb stems in the first column of \tabref{tab:6.kawasha} with their causativized verb forms in the second column.

\begin{table}
\begin{tabularx}{\textwidth}{XXXl}
\lsptoprule
  Root-\textsc{fv} &   &  \multicolumn{2}{l}{Causative Stem-\textsc{fv}}\\ 
\midrule
% {\todo[inline]{reduce line space between table header and midrule}}
\textit{jil-a} & `make cry' & \textit{jizh-a} & `make cry' \\
\textit{bweel-a} & `bring back' & \textit{bweezh-a} & `bring back' \\
\textit{tweel-a} & `enter' & \textit{tweezh-a} & `cause to enter, insert' \\

\lspbottomrule
\end{tabularx}

\caption{Palatalization of root-final /l/ in causative formation}
\label{tab:6.kawasha}

 \end{table}

With such causativized verbs, the /l/ of the remote past suffix assimilates completely to the causative consonant when the two suffixes occur together (\tabref{tab:7.kawasha}).


\begin{table}
\begin{tabularx}{\textwidth}{p{2cm}Xp{3cm}X}
\lsptoprule
\multicolumn{2}{l}{Causative verb-\textsc{fv}}  & \multicolumn{2}{l}{Causativized   verb-remote past } \\
\midrule
\textit{jizh-a} & `make cry' & \textit{jizh-izhe} & `made cry' \\
\textit{bweezh-a} & `bring back' & \textit{bwezh-ezhe} & `brought back' \\
\textit{tweezh-a} & `bring in, insert' & \textit{twezh-ezhe} & `brought in' \\

\lspbottomrule
\end{tabularx}

\caption{Alveopalatal fricative assimilation in remote past causative}
\label{tab:7.kawasha}

 \end{table}

The imbricated verb form in \REF{ex:4b.kawasha} displays this type of assimilation when it occurs in the remote past tense. 


\ea\label{ex:4.kawasha}
   \ea\label{ex:4a.kawasha} {Stem-FV}\\
mweesh-a ‘show’
    \ex\label{ex:4b.kawasha} {Remote past assimilation}\\
mwesh-eshe ‘showed'
\z
\z 

Unlike languages such as \ili{Bemba} and \ili{Tonga} which show double frication as a result of suffixation of the causative morpheme \textit{-i-} (\citealt{hyman1995,hyman2003}) in conjunction with either the applicative suffix or the perfective suffix, the phenomenon also affects certain underived verbs.\footnote{An anonymous reviewer has suggested that this may be a case of cyclic derivation induced by an underlying causative /i/. However, the \ili{Kaonde} assimilatory process affects both underived and derived verb forms, suggesting it is not due to an underlying causative.}
 
 \largerpage
The /l/ of the remote past also assimilates to a stem-final palatalized alveolar nasal consonant \textit{ny}. First, as in many \ili{Bantu} languages, palatalization of the verb stem consonant is triggered by the affixation of a second type of causative suffix \textit{-y-}, though this is not productive in \ili{Kaonde} where it only attaches to some verb stems ending in the alveolar nasal /n/ and to some extent to the bilabial nasal /m/. However, this palatalization affects only the alveolar nasal and not /m/. Assimilation of the remote past /l/ to the stem consonant is illustrated by the data in \tabref{tab:8.kawasha}, where the consonant of the remote past tense suffix -\textit{ile} turns into \textit{ny}.  
 

\begin{table}
\begin{tabularx}{\textwidth}{lXlX}
\lsptoprule
Root-\textsc{fv} &  & \multicolumn{2}{l}{Root-remote past} \\
\midrule 
\textit{kany-a} & `forbid' & \textit{kany-inye} & `forbade' \\
\textit{fweny-a} & `approach, move near' & \textit{fweny-enye} & `approached' \\
\textit{chiny-a} & `scare' & \textit{chiny-inye} & `scared' \\

\lspbottomrule
\end{tabularx}

\caption{Palatalized nasal assimilation in remote past}
\label{tab:8.kawasha}

 \end{table}


\section{Palatalization} \label{§3:palatalization.kawasha}
\largerpage
Palatalization is another phonological change that alveolar consonants are susceptible to as a result of suffixation of the remote past morpheme in \ili{Kaonde}. The suffix triggers this phenomenon by palatalizing the root/stem-final consonant (Table 9).\footnote{This applies also to the causative -\textit{ish}, applicative -\textit{il}, and neuter -\textit{ik}.} This morphophonological process changes either their place of articulation from alveolar to alveopalatal or both place and manner of articulation. Subject to palatalization are coronal consonants, namely sibilant fricatives /s/ and /z/, the voiceless stop /t/, and the lateral /l/.\footnote{Palatalization is also very productive in other languages such as \ili{Chokwe} and \ili{Luvale} spoken in the same region in  Zambia. In these languages, /t/ followed by /i/ becomes [tʃ], /nd/ changes to /dʒ/, /n/ to /\href{http://en.wikipedia.org/wiki/Ɲ}{ɲ}/, /s/ to /ʃ/, and /z/ becomes [ʒ].} 

\begin{table}[t]
\begin{tabularx}{\textwidth}{llXlX}
\lsptoprule
 & Root-\textsc{fv} &  & \multicolumn{2}{l}{Root-remote past} \\
\midrule
a. & \textit{as-a} & `shoot' & \textit{ash-ile} & `shot' \\
 & \textit{fis-a} & `hide' & \textit{fish-ile} & `hid' \\
 & \textit{sans-a} & `sprinkle' & \textit{sansh-ile} & `sprinkled' \\
\tablevspace
b. & \textit{baanz-a} & `make a fire' & \textit{baanzh-ile} & `made a fire' \\

\lspbottomrule
\end{tabularx}

\caption{Palatalization of root-final coronals}
\label{tab:9.kawasha}

 \end{table}


The verb stem-final coronal sibilant fricative consonants /s/ and /z/ change only their anterior feature from [+anterior, + coronal] to [-anterior, + coronal]. The two segments palatalize to /ʃ/ and /ʒ/, respectively, if they occur in front of the high front vowel of the remote past suffix.\footnote{In \ili{Nyamwezi} (\citealt{magangaschadeberg1992}), the alveolar fricative /s/ and the lateral /l/ palatalize to [ʃ] and [dʒ], respectively, like in \ili{Kaonde}. However, in \ili{Mashi}, if the verb stem-final consonant is an alveolar fricative /s/ or /z/, it changes to velar stop [k] and [g], respectively, preceding the applicative extension \textit{-iz} or \textit{-ez} (\citealt{bashi2008}): \textit{ku-yuus-a }\textit{\textup{‘to finish’}}\textit{\textup{→}} \textit{ku-yuuk-iz-a} ‘\textit{\textup{to finish for’.}}} Compare the root-final consonant of the infinitive verbs in the left column of \tabref{tab:9.kawasha} and those of the verb forms of the remote past in the right hand column.
 
\largerpage
The coronal voiceless stop /t/ changes in place of articulation to become an alveopalatal affricate /tʃ/. Thus, verb root/base-final consonants alternate between [+cor, +ant] and [+cor, -ant] (\tabref{tab:10.kawasha}).


\begin{table}[p]
\begin{tabularx}{\textwidth}{lXlX}
\lsptoprule
\multicolumn{2}{l}{Root/stem-\textsc{fv}}  & \multicolumn{2}{l}{Root-remote past} \\
\midrule
\textit{angat-a} & `take from s.o.' & \textit{angach-ile} & `took from s.o.' \\
\textit{it-a} & `call' & \textit{pich-ile} & `called' \\
\textit{kwat-a} & `arrest, hold' & \textit{kwach-ile} & `arrested, held' \\
\textit{pit-a} & `pass' & \textit{pich-ile} & `passed' \\
\textit{sant-a} & `thank' & \textit{sanch-ile} & `thanked' \\

\lspbottomrule
\end{tabularx}

\caption{Palatalization of root-final /t/}
\label{tab:10.kawasha}

 \end{table}




The voiced counterpart of /t/ only occurs after the alveolar nasal /n/, realized as [nd]. It palatalizes to /dʒ/. This can be seen through the alternation of the root-final consonant in \tabref{tab:11.kawasha}.

\begin{table}[p]
\begin{tabularx}{\textwidth}{lXlX}
\lsptoprule
\multicolumn{2}{l}{Root-\textsc{fv}}  & \multicolumn{2}{l}{Root-remote past} \\
\midrule
\textit{fund-a} & `teach' & \textit{funj-ile} & `taught' \\
\textit{yand-a} & `suffer' & \textit{yanj-ile} & `suffered' \\

\lspbottomrule
\end{tabularx}

\caption{Palatalization of root-final /nt/ [nd]}
\label{tab:11.kawasha}

 \end{table}

\largerpage
Also susceptible to palatalization are verb roots/bases that end in the lateral /l/ which becomes a voiced alveopalatal affricate /dʒ/ before \textit{-ile}. In short, this consonant, which varies between [l] and a central flap [ɾ] or a lateral flap [ɺ], alternates with /dʒ/. Consider mutation of the verb final consonant /l/ to the alveopalatal affricate /dʒ/ in \tabref{tab:12.kawasha}.\footnote{In \ili{Lunda} the cognates to \textit{shaala} ‘remain’ and t\textit{waala }‘take’ in \tabref{tab:12.kawasha} undergo imbrication to become \textit{sheeli} and \textit{tweeli} respectively.}
\newpage  

\begin{table}[p] 
\begin{tabularx}{\textwidth}{lXlX}
\lsptoprule
\multicolumn{2}{l}{Root/stem-\textsc{fv}}  & \multicolumn{2}{l}{Root-remote past} \\
\midrule 
\textit{bul-a} & `lack' & \textit{buj-ile} & `lacked' \\
\textit{lal-a} & `break' & \textit{laj-ile} & `broke' \\
\textit{vul-a} & `be plenty' & \textit{vuj-ile} & `was plenty' \\
\textit{vuul-a} & `undress' & \textit{vuuj-ile} & `undressed' \\
\textit{vwal-a} & `wear' & \textit{vwaj-ile} & `wore' \\
\textit{shal-a} & `remain' & \textit{shaj-ile} & `remained' \\
\textit{twal-a} & `take' & \textit{twaj-ile} & `took' \\
\textit{ingil-a} & `work' & \textit{ingij-ile} & `worked' \\
\textit{jil-a} & `cry' & \textit{jij-ile} & `cried' \\


\lspbottomrule
\end{tabularx}

\caption{Voiced palatalization of root-final /l/}
\label{tab:12.kawasha}

\end{table}

  \begin{table}[p] 
\begin{tabularx}{\textwidth}{lXlX}
\lsptoprule
\multicolumn{2}{l}{Root-applicative-\textsc{fv}}  & \multicolumn{2}{l}{Root-applicative-remote past}\\
\midrule
\textit{fw-il-a} & `die for' & \textit{fw-ij-ile} & `died for' \\
\textit{twaj-il-a} & `take for' & \textit{twaj-ij-ile} & `took for' \\
\textit{kajip-il-a} & `be angry with' & \textit{kajip-ij-ile} & `was angry with' \\
\textit{tambw-il-a} & `receive for' & \textit{tambw-ij-ile} & `received for' \\
\textit{pum-in-a} & `hit at' & \textit{pum-ij-ile} & `hit at' \\
\textit{kash-il-a} & `tie for' & \textit{kash-ij-ile} & `tied for' \\

\lspbottomrule
\end{tabularx} 

\caption{Voiced palatalization of applicative /l/}
\label{tab:13.kawasha}
\end{table}

\clearpage 

The applicative suffix -\textit{il} changes to -\textit{ij} when it precedes the remote past suffix.\footnote{Like the remote past, the causative, applicative, and perfective suffixes also condition palatalization. For example, based on the root \textit{kas} ‘tie’, \ili{Kaonde} has the intensive form \textit{kash-ish-a} ‘tie securely’, \textit{kash-il-a} ‘tie for’, and \textit{kash-ij-il-a} ‘tie completely for’.} This is illustrated by the examples in the right column of \tabref{tab:13.kawasha} where the inflectional remote past morpheme follows the applicative. The last example in \tabref{tab:13.kawasha} possesses two palatalized segments, i.e. the root-final consonant /ʃ/ and /dʒ/ of the applicative extension. 




The vowel of the remote past tense suffix does not harmonize in height with the mid vowel of the verb root if the former is preceded by an applicative extension. In other words, vowel harmony only affects the suffix closest to the root. Compare the remote past examples in the third row in the upper part of \tabref{tab:14.kawasha} with those containing a sequence of two suffixes (applicative plus remote past) in the lower part. The remote past suffix -\textit{ele} in the right-hand column harmonizes with the vowel of the verb root. If the applicative occurs immediately after the root, it assimilates in height to the root vowel. The remote past maintains its underlying form -\textit{ile}, as it is not in the immediate post-root position.
 

\begin{table}[t]
\begin{tabularx}{\textwidth}{lllll}
\lsptoprule
 & Root/Stem-\textsc{fv} &  & \multicolumn{2}{l}{Root/Stem-remote past}\\
\midrule
\multicolumn{5}{l}{harmony after root}\\
\midrule
 & \textit{lek-a} & `to leave off' & \textit{lek-ele} & `left off' \\
 & \textit{leng-a} & `to create' & \textit{leng-ele} & `created' \\
 & \textit{nemb-a} & `to write' & \textit{nemb-ele} & `wrote' \\
 & \textit{komb-a} & `to sweep' & \textit{komb-ele} & `swept' \\
\tablevspace
\multicolumn{5}{l}{no harmony after applicative}\\
\midrule
 & \textit{ketek-el-a} & `trust' & \textit{ketekej-ile} & `trusted' \\
 & \textit{lek-el-a} & `to let off' & \textit{lekej-ile} & `let off' \\
 & \textit{leng-el-a} & `to create for' & \textit{lengej-ile} & `created for' \\
 & \textit{mwek-el-a} & `appear to/at' & \textit{mwekej-ile} & `appeared to/for' \\
 & \textit{nemb-el-a} & `to write to/for' & \textit{nembej-ile} & `wrote to/for' \\
 & \textit{pemb-el-a} & `to wait for' & \textit{pembej-ile} & `waited for' \\
 & \textit{sek-el-a} & `rejoice' & \textit{sekej-ile} & `rejoiced' \\
 & \textit{komb-el-a} & `to sweep for' & \textit{kombej-ile} & `swept for' \\
 & \textit{sok-el-a} & `burn for' & \textit{sokej-ile} & `burned for' \\

\lspbottomrule
\end{tabularx}

\caption{Height harmony in remote past}
\label{tab:14.kawasha}

\end{table}




Nasal harmony in the remote past does not occur if a root carries the applicative before the remote past. In this case, neither the consonant of the applicative extension nor that of the remote past nasalize. Instead, the applicative -\textit{il} palatalizes to -\textit{ij} before the remote past suffix. 

\begin{table}[t]
\begin{tabularx}{\textwidth}{lXll}
\lsptoprule
\multicolumn{2}{l}{Root-applicative-\textsc{fv}}  & \multicolumn{2}{p{5cm}}{palatalization of applicative /l/ before remote past} \\
\midrule 
\textit{tam-in-a} & `be bad for' & \textit{tamij-ile} & `was bad for' \\
\textit{pum-in-a} & `beat/hit at' & \textit{pumij-ile} & `beat/hit at' \\
\textit{tum-in-a} & `send to' & \textit{tumij-ile} & `sent to' \\
\textit{nyem-en-a} & `run to/for' & \textit{nyemej-ile} & `ran to/for' \\
\textit{sem-en-a} & `give birth in' & \textit{sem-ej-ile} & `gave birth in' \\

\lspbottomrule
\end{tabularx}

\caption{Palatalization of applicative /l/ (without nasal harmony)}
\label{tab:15.kawasha}

\end{table}

As seen in the first column of \tabref{tab:15.kawasha}, the underlying lateral consonant of the applicative suffix of the verb stems harmonizes with the nasal consonant of the preceding the verb root. The examples in the right-hand column show the applicative /l/ (or /n/) does not change in the applicative if the remote past suffix occurs. Nevertheless, the applicative consonant /l/ close to the root palatalizes to /dʒ/. One might argue that the applicative /l/ has become /n/, which then turns into /dʒ/. 

  
It is to be noted that the remote past suffix does not trigger palatalization of the root-final consonant if the suffix is attached directly to a verb root containing a mid vowel /e/ or /o/. This behavior suggests that vowel height harmony applies before palatalization. Consider the examples in \tabref{tab:16.kawasha}, where palatalization fails to apply to verb roots containing a mid vowel. 
  


\begin{table}[t]
\begin{tabularx}{.75\textwidth}{llll}
\lsptoprule
Root-\textsc{fv} &  & \multicolumn{2}{l}{Root-remote past} \\
\midrule
\textit{end-a} & `walk' & \textit{end-ele} & `walked' \\
\textit{send-a} & `carry' & \textit{send-ele} & `carried' \\
\textit{pel-a} & `grind' & \textit{pel-ele} & `ground' \\
\textit{loot-a} & `dream' & \textit{loot-ele} & `dreamed' \\
\textit{leet-a} & `bring' & \textit{leet-ele} & `brought' \\
\textit{kos-a} & `be strong' & \textit{kos-ele} & `was strong' \\

\lspbottomrule
\end{tabularx}

\caption{No palatalization of root-final consonant after mid vowel }
\label{tab:16.kawasha}

\end{table}


However, palatalization of a stem-final /l/ takes place before the remote past suffix if that /l/ in the stem is part of the suffix or extension, even if the stem has a mid vowel. Most of the extensions have lost meanings and verbs carrying them are lexicalized. The examples in \tabref{tab:17.kawasha} show how the consonant /l/ palatalizes to an alveopalatal affricate \textit{j} before the remote past suffix.

\begin{table}
\begin{tabularx}{\textwidth}{XXXX}
\lsptoprule
Root-\textsc{fv} &  & \multicolumn{2}{l}{Stem-remote past} \\
\midrule
\textit{londel-a} & `follow' & \textit{londej-ile} & `followed' \\
\textit{imeen-a} & `stand for/at' & \textit{imeneej-ile} & `stood for/at' \\
\textit{mwekel-a} & `appear at' & \textit{mwekej-ile} & `appeared at' \\

\lspbottomrule
\end{tabularx}

\caption{Palatalization of stem-final consonant after mid vowel}
\label{tab:17.kawasha}

\end{table}


\section{Imbrication} \label{§4:imbrication.kawasha}

Imbrication is a phonological pattern in some \ili{Bantu} languages whereby the perfective or remote past suffix is overlaid and fuses with a preceding root or base. Depending on the language, the suffix either splits into two separate parts or the verb root or base loses its last consonant. Normally, suffixation of the remote past suffix -\textit{ile} to a verb root results in the elongation of the verb by at least one syllable, as exemplified by the data in \sectref{§2:harmony.kawasha}. For clarification, consider the uninflected verb forms on the left, and those with regular suffixation of the remote past morpheme on the right in \tabref{tab:18.kawasha}. 

\begin{table}
\begin{tabularx}{.75\textwidth}{lXlX}
\lsptoprule
Root-\textsc{fv} &  & \multicolumn{2}{l}{Root-remote past} \\
\midrule
\textit{sak-a} & `want' & \textit{sak-ile} & `wanted' \\
\textit{imb-a} & `sing' & \textit{imb-ile} & `sang' \\
\textit{keb-a} & `look for' & \textit{keb-ele} & `looked for' \\
\textit{pot-a} & `buy' & \textit{pot-ele} & `bought' \\
\textit{tum-a} & `send' & \textit{tum-ine} & `sent' \\

\lspbottomrule
\end{tabularx}

\caption{Simple and remote past stems}
\label{tab:18.kawasha}

\end{table}

However, this is not the case with several verbs that bear certain types of derivational suffixes, frozen suffixes, and extensions. The suffixation of the remote past morpheme to such verb bases results in a phonological phenomenon termed \textit{imbrication} \citep{bastin1983}. This process usually brings about sound changes which alter the resulting verb size without increasing its number of syllables. The concomitant effects of the imbrication process, which depend on the property of the verbal suffix or extension may include segmental deletion, segmental insertion, gliding, and vowel coalescence. Consider the examples in \tabref{tab:19.kawasha}. 

\begin{table}
\begin{tabularx}{.75\textwidth}{lXlX}
\lsptoprule
Base-\textsc{fv} &  & \multicolumn{2}{l}{Remote past} \\
\midrule
\textit{iman-a} & `stand' & \textit{im-eene} & `stood' \\
\textit{ikal-a} & `sit, stay' & \textit{ik-eele} & `sat, stayed' \\

\lspbottomrule
\end{tabularx}

\caption{Imbrication in remote past formations}
\label{tab:19.kawasha}

\end{table}

As the two examples in \tabref{tab:19.kawasha} show, the verb forms in the right column contain the same number of syllables as those on the left column despite the addition of the two-syllable remote past morpheme. This suffixation causes phonological changes in the verb, namely the fusion of the expansion (underlying) -\textit{al} and the inflectional remote past suffix -\textit{ile} into a single chunk resulting in consonantal deletion, which in turn triggers vowel coalescence between /a/ and /i/ accompanied by vowel lengthening. 

In \ili{Kaonde}, it is worthwhile noting that imbrication is not uniform as it comes in different patterns. The first pattern affects a class of verb stems of more than one syllable of the structure /…C-aC-/ or /…C-aCaC-/. These verbs change their form to /…CeeC-e/ when they attach the remote past suffix. Verbs that belong to this group contain derivational extensions ending in the lateral or alveolar nasal, such as the reciprocal \textit{-añan}, the associative -\textit{akan}, causativized verbs, and those with a frozen stative -\textit{al} or -\textit{an}.\footnote{\citet[90]{meeussen1967} points out that the extension -\textit{ad}- “appears partly as an expansion, partly as a suffix with ill-defined meaning”. Though the extensions are identifiable, the seeming roots to which they attach do not have independent meaning. This is attested in \ili{Bemba} \citep{hyman1995}.} Most of what appear to be verb roots “suffixed” with -\textit{al} do not occur on their own and have no independent meaning without /al/. The “suffix” forms a lexicalized unit with the root. When these stems attach the remote past suffix, the initial vowel of the suffix is inserted before the final consonant of the verb stem, and the remote past consonant deletes. This segmental loss results in the fusion of the verb stem low vowel /a/ and the vowel of the suffix /i/ into a single front mid vowel /e/ accompanied by compensatory lengthening. That is, the two adjacent vowels merge into a single vowel that combines properties of both source vowels. This imbrication process can be schematically described as in \REF{ex:5.kawasha}.

\ea
\label{ex:5.kawasha}
 \glll …C-VC-ile  →  …C-ViCe  →  C-eeCe  \\
  ik-al-ile {}   ik-aile  {}  ik-eele\\
 im-an-ile  {}  im-aine  {}  im-eene\\

\z

Listed in \tabref{tab:20.kawasha} are examples of verb bases with frozen suffixes, reciprocal extensions, and associative extensions.


\begin{table}[t]
\begin{tabularx}{\textwidth}{llp{25.5mm}lQ}
\lsptoprule
 & Stem-\textsc{fv} &  & Stem-remote past & \\
\midrule
a. & \textit{kankaman-a} & `tie together' & \textit{kankam-eene} & `tied together' \\
 & \textit{unvwañan-a} & `understand each other' & \textit{unvw-añeene} & `understood each other' \\
\tablevspace
b. & \textit{esekan-a} & `be equal in size' & \textit{esek-eene} & `was equal in size' \\
 & \textit{kasankan-a} & `tie together' & \textit{kasank-eene} & `tied together' \\
 & \textit{kwakan-a} & `unite' & \textit{kwatak-eene} & `united' \\
 & \textit{sambakan-a} & `meet' & \textit{sambak-eene} & `met' \\
 & \textit{tentekan-a} & `shake' & \textit{tentek-eene} & `shook' \\
\tablevspace
c. & \textit{abany-a} & `share' & \textit{abe-eny-e} & `shared' \\
 & \textit{esekany-a} & `compare' & \textit{esek-eenye} & `compared' \\
\tablevspace
d. & \textit{ikal-a} & `sit, stay' & \textit{ik-eele} & `sat, stayed' \\
 & \textit{sangalal-a} & `rejoice' & \textit{sangal-eele} & `rejoiced' \\
 & \textit{iman-a} & `stand, stop' & \textit{im-eene} & `stood, stopped' \\
 & \textit{pusan-a} & `differ' & \textit{pus-eene} & `differed' \\

\lspbottomrule
\end{tabularx}

\caption{Imbrication}
\label{tab:20.kawasha}

\end{table}


A very small set of monosyllabic verb bases of the form CV:C- or CwV:C- with the frozen suffix -\textit{an} is also subject to imbrication. However, though these verb forms imbricate, they behave slightly different than the verb stems in \tabref{tab:18.kawasha} and \tabref{tab:19.kawasha}. This is because consonant truncation does not result in fusion between the two adjacent vowels as would normally be expected in that specific context. Instead, the two juxtaposed non-identical vowels remain separate segments as they are pronounced individually. This is illustrated schematically in \REF{ex:6.kawasha}. 

\ea
\label{ex:6.kawasha}
C(w)V:C + -\textit{ile}  → CVVCV
\z

Examples of monosyllabic verb roots that have the pattern C(w)V:C- are in \tabref{tab:21.kawasha}. Observe that the sequence consisting of the root /a/ and the suffixal vowel /i/ does not induce vowel fusion. 

\begin{table}[t]
\fittable{
\begin{tabular}{lllll}
\lsptoprule
 & Stem-\textsc{fv} &  & \multicolumn{2}{l}{Stem-remote past} \\
\midrule
CV:C- & \textit{paan-a} & `share, distribute' & \textit{pa-i-ne} & `shared, distributed' \\
 & \textit{taan-a} & `meet, find' & \textit{ta-i-ne} & `met, found' \\
 & \textit{kaan-a} & `refuse' & \textit{ka-i-ne} & `refused' \\
\tablevspace
CwV:C- & \textit{fwaan-a} & `be deadly' & \textit{fwa-i-ne} & `was deadly' \\
 & \textit{swaan-a} & `inherit, succeed' & \textit{swa-i-ne} & `inherited, succeeded' \\

\lspbottomrule
\end{tabular}
}

\caption{Imbrication in monosyllabic C(w)V:C- roots}
\label{tab:21.kawasha}

\end{table}


Another group of verbs that undergoes imbrication consists of verb stems that are extended with the transitive suffix -\textit{ul }and -\textit{ulul}.\footnote{The suffix is commonly known as \textit{separative} in \ili{Bantu} studies, though it has other functions. I prefer to call it \textit{transitive}.} Before we discuss these verbs, it is important to look at their intransitive counterparts. The remote past of verb stems with the intransitive counterpart -\textit{uk} and its long form -\textit{uluk} is formed through regular suffixation, as exemplified in \tabref{tab:22.kawasha}.



\begin{table}
\begin{tabularx}{\textwidth}{XXll}
\lsptoprule
Stem-\textsc{fv} &  & \multicolumn{2}{l}{Stem-remote past} \\
\midrule
\textit{jimuk-a} & `be alert' & \textit{jimuk-ile} & `was, were alert' \\
\textit{lumbuluk-a} & `be clear' & \textit{lumbuluk-ile} & `was clear' \\
\textit{tabuk-a} & `be torn' & \textit{tabuk-ile} & `was, were torn' \\
\textit{shinkuk-a} & `open' & \textit{shinkuk-ile} & `opened' \\
\textit{chimuk-a} & `be snapped' & \textit{chimuk-ile} & `was, were snapped' \\
\textit{languluk-a} & `think' & \textit{languluk-ile} & `thought' \\
\textit{kasuluk-a} & `be united' & \textit{kasuluk-ile} & `was were united' \\
\textit{chimauk-a} & `be sewn' & \textit{chimauk-ile} & `was sewn' \\

\lspbottomrule
\end{tabularx}

\caption{Intransitive verbs with \textit{-uk} and \textit{-uluk}}
\label{tab:22.kawasha}

\end{table}



On the other hand, the remote past suffix fuses with derived transitive verbs carrying the transitive suffix -\textit{ul,} inducing imbrication and other phonological modifications. The initial vowel of the remote past suffix is placed before the final consonant of the transitive verb stem, and the lateral consonant of the inflectional tense morpheme deletes resulting in gliding of the final vowel /u/ or /o/ into /w/ due to adjacency with the front high vowel of the triggering suffix. This process is accompanied by compensatory lengthening of the front high vowel of the suffix. In addition, vowel height harmony occurs with verb root stems if the first vowel is mid-back /o/. That is, the vowel /i/ of the suffix lowers to the front mid vowel /e/ by partially assimilating a single feature from the preceding vowel. This imbrication process is schematically described in \REF{ex:7.kawasha}.

\ea
\label{ex:7.kawasha}

\glll …C-VC-ile  →  …C-Vile  →  …Cw-iile/…Cw-eele \\
tamb-ul-a {}   tamb-uile {}   tamw-iile \\
‘receive’ {} {} {} ‘received’ \\

\z

The examples in \tabref{tab:23.kawasha} show imbrication of extended verb stems that contain the transitive suffix \textit{-ul}.


\begin{table}
\begin{tabularx}{\textwidth}{llXlX}
\lsptoprule
 & Stem-\textsc{fv} &  & \multicolumn{2}{l}{Stem-remote past} \\
\midrule
a. & \textit{kasulul-a} & `untie' & \textit{kasulw-iile} & `untied' \\
 & \textit{lumbulul-a} & `explain' & \textit{lumbulw-iile} & `explained' \\
 & \textit{shinkul-a} & `open' & \textit{shinkw-iile} & `opened' \\
 & \textit{tabul-a} & `tear' & \textit{tabw-iile} & `tore' \\
 & \textit{tatul-a} & `begin' & \textit{tatw-iile} & `began' \\
\tablevspace
b. & \textit{kongol-a} & `borrow' & \textit{kongw-eele} & `borrowed' \\
 & \textit{solol-a} & `reveal' & \textit{solw-eele} & `revealed' \\
 & \textit{songol-a} & `marry' & \textit{songw-eele} & `married' \\
 & \textit{tongol-a} & `choose' & \textit{tongw-eele} & `chose' \\
\tablevspace
c. & \textit{chimun-a} & `snap' & \textit{chimw-iine} & `snapped' \\
 & \textit{jimun-a} & `alert' & \textit{jimw-iine} & `alerted' \\
 & \textit{chimaul-a} & `sew' & \textit{chimaw-iine} & `sewed' \\
 & \textit{onaun-a} & `destroy' & \textit{onaw-iine} & `destroyed’' \\

\lspbottomrule
\end{tabularx}

\caption{Imbrication in transitive verbs with \textit{-ul} and \textit{-ulul}}
\label{tab:23.kawasha}

\end{table}

\largerpage
As in most of the \ili{Bantu} languages, imbrication is also observed with the verb root \textit{món} ‘see’ even though it is monosyllabic without an extension. 

\ea\label{ex:8.kawasha}
   \ea\label{ex:8a.kawasha} Root-FV \\
   mon-a ‘see’
   
   \ex\label{ex:8b.kawasha} Remote past \\
     mw-eene ‘saw’ \\
\z
\z 

Imbrication also affects verb stems extended with the causative suffix -\textit{ish} and other verb roots ending in the voiced alveopalatal fricative /ʒ/. The imbricated verb formation does not show any sign of the remote past tense suffix on the surface. The suffixal segments -\textit{il} are completely truncated and the presence of the final vowel /e/ is the only indication that they are imbricated verb forms in the remote past. This phenomenon is also observed in isi\ili{Zulu} and \ili{Sesotho} in which certain verbs exhibit imbrication in the perfect without making “changes to the verb stem” \citep{monich2015}. Furthermore, this phenomenon does not yield an expected vowel lengthening in the derivational extension domain, as is the case of most of the verbs with irregular suffixation. The form CVC-\textit{ish-ile} changes to CVC-\textit{ish-e} with zero suffixation, as exemplified in \tabref{tab:24.kawasha}.

\begin{table}
\begin{tabularx}{\textwidth}{llXll}
\lsptoprule
 & Root-\textsc{caus}-\textsc{fv} &  &\multicolumn{2}{l}{Remote past} \\
\midrule
a. & \textit{kos-esh-a} & `make strong' & \textit{kos-eshe} & `made strong' \\
 & \textit{pos-esh-a} & `cure' & \textit{pos-eshe} & `cured' \\
 & \textit{pot-esh-a} & `sell' & \textit{pot-eshe} & `sold' \\
 & \textit{keb-esh-a} & `cause to search' & \textit{keb-eshe} & `made to search' \\
 & \textit{pulush-a} & `save' & \textit{pulw-ishe} & `saved' \\
 & \textit{unvw-ish-a} & `listen attentively' & \textit{unvw-ishe} & `listened attentively' \\
 & \textit{took-esh-a} & `whiten' & \textit{took-eshe} & `whitened' \\
\tablevspace
b. & \textit{ipwiizh-a} & `ask' & \textit{ipw-iizhe} & `asked' \\
 & \textit{kamb-izh-a} & `command' & \textit{kamb-izhe} & `commanded' \\
 & \textit{katezh-a} & `be difficult' & \textit{kat-ezhe} & `was difficult' \\
 & \textit{pitaizh-a} & `pass beyond' & \textit{pita-izhe} & `passed beyond' \\

\lspbottomrule
\end{tabularx}

\caption{Imbrication in verbs with causative \textit{-ish}}
\label{tab:24.kawasha}

\end{table}

As seen in \tabref{tab:24.kawasha}, the verb base on the left and the imbricated verb on the right have the same forms CVCVCV-, VCCGVCV- and VCGVCV-. What distinguishes verb forms in the two columns is only the final vowel \textit{-a} versus \textit{-e}.

 
Verb bases with the passive suffix -\textit{w} or -\textit{iw }imbricate as well. Some verbs with the passive morphology have developed lexicalized or non-compositional meaning, as the -\textit{w} is frozen and has no semantic content. In addition, what appears as a root does not have independent meaning. When attached to such verbs, the -\textit{il} of the remote past suffix splits from the final vowel and is inserted before the verb stem-final consonant. Thus, the passive suffix appears between the two parts of the remote past morpheme. However, since there is no segment deletion if the stem is monosyllabic, there is an increase in the number of syllables. This is also attested in many other \ili{Bantu} languages such as \ili{Bemba} (\citealt{chebanne1993,hyman1995,kula2002}). Consider the remote past verb forms in \tabref{tab:25.kawasha}.
 

\begin{table}[t]
\begin{tabularx}{\textwidth}{lXlX}
\lsptoprule
\multicolumn{2}{l}{Root-passive-\textsc{fv}}  & \multicolumn{2}{l}{Root-remote past-passive} \\
\midrule 
\textit{kas-w-a} & `be tied' & \textit{kash-il-w-e} & `was tied' \\
\textit{kelw-a} & `be late' & \textit{kel-el-w-e} & `was late' \\
\textit{kolw-a} & `be sick' & \textit{kol-el-w-e} & `was sick' \\
\textit{leng-w-a} & `be created' & \textit{leng-el-w-e} & `was created' \\
\textit{pum-w-a} & `be hit' & \textit{pum-in-w-e} & `was hit' \\
\textit{temw-a} & `love' & \textit{tem-en-w-e} & `was loved' \\

\lspbottomrule
\end{tabularx}

\caption{Imbrication in remote past passive}
\label{tab:25.kawasha}

\end{table}

Consonant deletion occurs in longer verb bases displaying the passive morphology. When the remote past suffix is inserted to the immediate left of -\textit{w}, the /l/ of the applicative or other relevant verb stem deletes, inducing either a long vowel [ii] or vowel coalescence of [ai] into [ee], depending on the property of the verb. Imbrication in the examples in \tabref{tab:26.kawasha}, part a, induces a sequence of identical long vowels [ii]; while in part b it results in vowel coalescence. The same thing is noted with verb stems already suffixed with the applicative morpheme in the passive, as shown in part c of \tabref{tab:26.kawasha}.

\begin{table} 
\begin{tabularx}{\textwidth}{llXll}
\lsptoprule
 & \multicolumn{2}{p{4cm}}{Root-extension/ applicative-passive-\textsc{fv}}  & \multicolumn{2}{p{4cm}}{~\newline Remote past passive} \\
\midrule
a. & \textit{fwainw-a} & `have to' & \textit{fwaiinw-e} & `had to' \\
 & \textit{ambiw-a} & `be told' & \textit{ambiiw-e} & `was/were told' \\
 & \textit{itabiw-a} & `be accepted' & \textit{itabiiw-e} & `was/were accepted' \\
 & \textit{paanyiiw-a} & `be given' & \textit{panyiiw-e} & `was/were given' \\
 & \textit{ubiw-a} & `be done' & \textit{ub-iiw-e} & `was done' \\
\tablevspace
b. & \textit{kankalw-a} & `be unable to' & \textit{kankeelw-e} & `was unable to' \\
\tablevspace
c. & \textit{nemb-el-w-a} & `be written to' & \textit{nemb-eel-w-e} & `was/were written to' \\
 & \textit{twaj-il-w-a} & `be taken for' & \textit{twaj-iil-w-e} & `was/were taken for' \\
\lspbottomrule
~\\\end{tabularx} 

\caption{Imbrication with vowel lengthening in remote past passive}
\label{tab:26.kawasha}

\end{table}

Certain verb bases already affected by the process also imbricate for a second time. In other words, the same verb may undergo double imbrication. The applicative extension triggers the first instance of imbrication, while the inflectional remote past suffix induces the second one (\tabref{tab:27.kawasha}).

\begin{table} 
\small
\begin{tabularx}{\textwidth}{lXlX}
\lsptoprule
\multicolumn{2}{l}{Root-applicative-passive-\textsc{fv} }  & \multicolumn{2}{l}{Applied passive in remote past} \\
\midrule
\textit{zhikw-il-w-a} & `be unearthed for' & \textit{zhikw-iil-w-e} & `was unearthed for' \\
\textit{sangw-il-w-a} & `be resurrected' & \textit{sangw-iil-w-e} & `was resurrected' \\

\lspbottomrule
\end{tabularx} 

\caption{Double imbrication in applied remote past passive}
\label{tab:27.kawasha}

\end{table}


\section{Conclusion} \label{§5:conclusion.kawasha}

This paper has discussed the phonological processes triggered by the suffixation of the remote past tense suffix to verb roots and stems. It shows that the phenomena include vowel height harmony, nasal consonant harmony, complete consonant assimilation, and vowel harmony when preceded by mid vowels in the root. Nasal consonant harmony is a process that changes the lateral consonant of the suffix to a nasal /n/ following a bilabial or alveolar nasal in the root. Consonant assimilation is a long-distance phenomenon that affects the consonant of the suffix when it is attached to verb stems ending in the alveopalatal fricatives or the palatal nasal. Palatalization affects alveolar stops and the lateral which turns into the alveopalatal affricate. Regarding imbrication, the process exhibits different patterns depending on the extension and suffix that precedes the triggering remote past suffix. 



\section*{Abbreviations}

\begin{tabularx}{.45\textwidth}{lX}
\textsc{1sg} & first person singular \\
\textsc{caus} & causative  \\
\textsc{fv} & final vowel \\
\end{tabularx}
\begin{tabularx}{.45\textwidth}{lX}
\textsc{rp} & remote past \\
\textsc{tns} & tense \\
\\
\end{tabularx}



\section*{Acknowledgements}

I am grateful to an anonymous reviewer for comments, and to Phil Mukanzu and Kelvin M. Mambwe for their help with the data. 



{\sloppy
\printbibliography[heading=subbibliography,notkeyword=this]
}

\end{document}
