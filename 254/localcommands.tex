\newcommand{\appref}[1]{Appendix \ref{#1}}
\newcommand{\fnref}[1]{Footnote \ref{#1}}
\renewcommand{\sectref}[1]{Section~\ref{#1}}
\renewcommand{\lsChapterFooterSize}{\footnotesize}

\patchcmd{\mkbibindexname}{\ifdefvoid{#3}{}{\MakeCapital{#3} }}{\ifdefvoid{#3}{}{#3 }}{}{\AtEndDocument{\typeout{mkbibindexname could not be patched.}}}

%INTRO
\newcommand{\agr}{\textsc{agree}}
\newcommand{\agrl}{\textsc{agree-link}}
\newcommand{\agrc}{\textsc{agree-copy}}

%CARSTENS LOCAL COMMANDS, D'Alessandro has the same

\renewbibmacro*{index:name}[5]{%
	\usebibmacro{index:entry}{#1}
	{\iffieldundef{usera}{}{\thefield{usera}\actualoperator}\mkbibindexname{#2}{#3}{#4}{#5}}}


% \newcommand{\noop}[1]{}

%\newcommand{\ul}{\underline{\hspace{12pt}}}

\newcommand{\sub}[1]{\ensuremath{_{\textnormal{\scriptsize #1}}}} %subscript

\newcommand{\trace}[1]{\ensuremath{\langle}#1\ensuremath{\rangle}} %trace with angled brackets

\newcommand{\sqboxEmpty}[1]{	% square boxes for the agreement
	\begin{tikzpicture}%[baseline={(a.base)}]
	\draw[line width=0.6pt,#1] (0.5\pgflinewidth,0.5\pgflinewidth) rectangle (1.2ex-0.5\pgflinewidth,1.2ex-0.5\pgflinewidth);
	\end{tikzpicture}%
}


%Boerjesson-Mueller

\newcommand{\excite}[2]{\citeauthor{#2} (\citeyear{#2}, #1)}
\newcommand{\scite}[2]{\citeauthor{#2}#1 (\citeyear{#2})}

%JM 10.01.: Die folgenden Befehle sollten eigentlich Werke von Stefan Müller als Müller, St. im Text zitieren. Gereon meinte, dass ist so Konvention bei den beiden. Leider scheint es nicht zu funktionieren und Werke von Gereon Müller und Stefan Müller werden beide nur als Müller zitiert. Das müssen wir auf jeden Fall noch klären.

\defcitealias{Stmueller2005}{M{\"u}ller,~St. (2005)}
\defcitealias{SMueller:07:pas}{M{\"u}ller,~St. (2007)}
\defcitealias{SMueller:15}{M{\"u}ller,~St. (2015)}
\defcitealias{SMueller:02}{M{\"u}ller,~St. (2002)}

\newlength{\strichlaenge}
\newcommand{\strich}[1]{\settowidth{\strichlaenge}{#1}%
	#1%
	\llap{\rule[.8ex]{\strichlaenge}{.7pt}}}

\newcommand{\Next}[1][]{\ref{\refprefix\the\numexpr\value{xnumi}+1}#1}
\newcommand{\NNext}[1][]{\ref{\refprefix\the\numexpr\value{xnumi}+2}#1}
\newcommand{\Last}[1][]{\ref{\refprefix\the\numexpr\value{xnumi}}#1}
\newcommand{\LLast}[1][]{\ref{\refprefix\the\numexpr\value{xnumi}-1}#1}

%DIERCKS ET AL

\newcommand{\circled}[1]{\begin{tikzpicture}[baseline=(word.base)]
	\node[draw, rounded corners, text height=8pt, text depth=2pt, inner sep=2pt, outer sep=0pt, use as bounding box] (word) {#1};
	\end{tikzpicture}
}

%SMITH

\newcommand{\ra}{$\rightarrow$}
\newcommand{\object}{\textsc{object}}
\newcommand{\robj}{\textsc{object$_{\textrm{$\theta$}}$}}
\newcommand{\subj}{\textsc{subject}}
\newcommand{\theme}{\textsc{theme}}
\newcommand{\patient}{\textsc{patient}}
\newcommand{\goal}{\textsc{goal}}
\newcommand{\causee}{\textsc{causee}}
\newcommand{\agree}{\textsc{agree}}
\newcommand{\topic}{\textsc{topic}}

%McFADDEN

% % % \newcommand{\T}{\textrtailt}

\renewcommand{\U}{\u{u}}
\newcommand{\alloc}{\textsc{alloc}{}\xspace}
\newcommand{\allagr}{AllAgr}

%SUNDARESAN

% IPA Commands
\renewcommand{\U}{\u{u}}
%\newcommand{\U}{\textbari}


\newcommand{\ph}{$\phi$}

% Other commands:
\newcommand{\ul}{\underline}
\newcommand{\ec}{\textsc{ec}}
\newcommand{\pro}{\textsc{pro}}
\newcommand{\itpro}{\textit{pro}}
\newcommand{\person}{\textsc{person}}
\newcommand{\num}{\textsc{number}}
\newcommand{\gender}{\textsc{gender}}
\renewcommand{\part}{\textsc{participant}}
\newcommand{\nul}{\textsc{null}}
\newcommand{\anaph}{\textsc{anaph}}
\newcommand{\anaphor}{{\scshape anaphor}}
\newcommand{\lilv}{{\itshape v}}
\newcommand{\lilvp}{{\itshape v}P}
\newcommand{\kol}{{\itshape kol}}
\newcommand{\taan}{{\itshape ta(a)n}}
\newcommand{\self}{\textsc{self}}
\newcommand{\subscr}{{\{i,*j\}}}
\newcommand{\var}{\textsc{var}}
\renewcommand{\op}{\textsc{op}}
\newcommand{\dep}{\textsc{dep}}
\newcommand{\refl}{\textsc{refl}}
\newcommand{\id}{\textsc{id}}
\newcommand{\empathy}{\textsc{empathy}}
\newcommand{\anim}{\textsc{anim}}
\newcommand{\sentience}{\textsc{sentience}}


%Table commands
\renewcommand{\top}{\lsptoprule}
\newcommand{\bottom}{\lspbottomrule}
\renewcommand{\mid}{\midrule}


% denotation brackets:
\newcommand{\den}[1]{\ensuremath{\llbracket #1 \rrbracket}}


\newenvironment{points}{%
	\begin{dinglist}{43}}{\end{dinglist}}


\newcounter{lsConnectDashTempGroup}
\NewDocumentCommand\ConnectDashTail{m O{\thelsConnectDashTempGroup}}{%read: mandatory arg #1, optional argument #2 with the current group counter as its default value.
    \edef\lsConnectDashTempPosition{#2}%\edef expands the argument, which means reading the current value of the counter.
    {\tikz[remember picture,
           anchor=base, baseline,
           inner xsep=0pt,
           inner ysep=-.5ex]\node (ConnectDashTempTail\lsConnectDashTempPosition) {\strut{}#1};}%\strut for baseline
}
\NewDocumentCommand\ConnectDashHead{s O{1ex} m O{\thelsConnectDashTempGroup}}{%read: star #1, optional argument (distance of arrow from text= std. one x-height), mand. arg. (node text), optional argument #2, the group specifier
    \edef\lsConnectDashTempPosition{#4}%
    \stepcounter{lsConnectDashTempGroup}%We have a match, let's update the group counter
    {\tikz[remember picture,
           anchor=base, baseline,
           inner xsep=0pt,
           inner ysep=-.5ex] \node (ConnectDashTempHead\lsConnectDashTempPosition) {\strut{}#3};%
     \tikz[remember picture] \draw[% we have a tail and a head, let's bring them together
                                \IfBooleanTF#1{{Triangle[]}-}{-{Triangle[]}},% Check if the starred version is used. The starred version is right->left, the normal version left->right
                                overlay,dashed] (ConnectDashTempTail\lsConnectDashTempPosition.north) -- ++(0,#2) -| (ConnectDashTempHead\lsConnectDashTempPosition.north);
    }%
}

\RenewDocumentCommand\ConnectHead{s O{1ex} m O{\thelsConnectTempGroup}}{%read: star #1, optional argument (distance of arrow from text= std. one x-height), mand. arg. (node text), optional argument #2, the group specifier
    \edef\lsConnectTempPosition{#4}%
    \stepcounter{lsConnectTempGroup}%We have a match, let's update the group counter
    {\tikz[remember picture,
           anchor=base, baseline,
           inner xsep=0pt,
           inner ysep=-.5ex] \node (ConnectTempHead\lsConnectTempPosition) {\strut{}#3};%
     \tikz[remember picture] \draw[% we have a tail and a head, let's bring them together
                                \IfBooleanTF#1{{Triangle[]}-}{-{Triangle[]}},% Check if the starred version is used. The starred version is right->left, the normal version left->right
                                overlay] (ConnectTempTail\lsConnectTempPosition.north) -- ++(0,#2) -| (ConnectTempHead\lsConnectTempPosition.north);
    }%
}
