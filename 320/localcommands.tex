%to put in a parenthesized reference, e.g. referring to a numbered example.
\newcommand{\inl}[1]{(\ref{#1})}

% \newcommand{\mycomment}[1]{\textbf{#1}}
%\newcommand{\mycomment}[1]{{#1}} %% FOR NO BOLDFACE

%commands to create { F1... } bracketing
\newcommand{\bo}{\{~}
\newcommand{\bc}{~\}}
\newcommand{\br}[1]{\bo{#1}\bc}
\newcommand{\gks}{\textsigma}
%%%===================
\newcommand{\Sec}{\textsection}

%%%% End custom infoboxes commands

\newcommand{\ee}{\vspace{.25cm}\\} %ee for "Example End"
\renewcommand{\ee}{\\}
\newcommand{\et}[1]{#1} %et for Example Title
\newcommand{\lt}{\textit} %lt for Line Title (lettered lines within examples)
% \newcommand{\slt}{\underline} %slt for Sub Line Title (under lettered lines in examples)
\newcommand{\au}{da/dp} %if needed, we can easily change this to da/dp -- this is used when noting that something embedded in a quote is something that the AUthor added.

%commands to simplify use of conjunction and disjunction and negation signs
\renewcommand{\not}{$\neg$}
\renewcommand{\and}{$\wedge\ $}
%\newcommand{\Asterisk}{\@chooseSymbol{'116}}
%shorthand commands for miscellaneous symbols
\newcommand{\zero}{$\emptyset$}

\newcommand{\ipa}[1]{\textipa{#1}}
\newcommand{\ip}[1]{\textipa{[#1]}}
\newcommand{\ips}[1]{$/$\textipa{#1}$/$}

% \renewcommand{\textipa}[1]{{\color{red!50!black}#1}}
\renewcommand{\textipa}[1]{#1}

%vowel shorthand

\newcommand{\ua}{\textturna}
\newcommand{\A}{\textscripta}
\newcommand{\sa}{\textscripta}
\newcommand{\E}{\ipa{E}} %epsilon
\newcommand{\3}{\textrevepsilon}
\newcommand{\rE}{\rt{\textrevepsilon}}%retracted reverse epsilon
\newcommand{\I}{\ipa{I}} %small cap i
\renewcommand{\ibar}{\textbari}
\renewcommand{\O}{\ipa{O}} %open o
\renewcommand{\obar}{\textbaro}
\newcommand{\U}{\ipa{U}} %lax u
\newcommand{\ubar}{\textbaru}
\newcommand{\ue}{\textschwa}
\renewcommand{\@}{\textschwa}
\newcommand{\uv}{\textturnv} %"caret"


%consonant shorthand

\newcommand{\B}{\textbeta}
\newcommand{\D}{\ipa{D}} %eth
\renewcommand{\D}{ð}
\newcommand{\dJ}{\textbardotlessj}
\newcommand{\dZ}{\textdyoghlig}
\renewcommand{\dZ}{ʤ}
\newcommand{\dz}{\textdzlig}
\renewcommand{\dz}{ʣ}
% \newcommand{\G}{\ipa{G}}
\renewcommand{\hl}{\textbeltl}
\newcommand{\Hx}{\textcrh}
\newcommand{\jn}{\textltailn}
\newcommand{\lb}{\textbeltl}
\newcommand{\rl}{\textrtaill}	 %Warlpiri retroflex lateral
\newcommand{\ly}{\textturny} %Warlpiri palatal lateral
\newcommand{\tl}{\textcrlambda}
\newcommand{\lh}{\textltilde}
\newcommand{\rn}{\textrtailn}		%Warlpiri retroflex nasal
\newcommand{\N}{\ipa{N}}
\newcommand{\R}{\textfishhookr}
\newcommand{\ur}{\textturnr}
\newcommand{\rr}{\textrtailr}	%Warlpiri retroflex flap
\newcommand{\uR}{\textinvscr}
\newcommand{\Wr}{\textturnrrtail}	%Warlpiri apical approximant (alveolar/retroflex?)
\renewcommand{\S}{\textesh}
\newcommand{\T}{\texttheta}
\newcommand{\rtt}{\textrtailt}	%Warlpiri retroflex plosive
\newcommand{\tS}{\textteshlig}
\renewcommand{\tS}{ʧ}
\newcommand{\ts}{\texttslig}
\newcommand{\Z}{\textyogh}
\newcommand{\?}{\textglotstop}
\newcommand{\tc}{\texttctclig}
% \newcommand{\C}{\textctc}
% \newcommand{\z}{\textctz}
\newcommand{\tcz}{\texttctzlig}
\newcommand{\dcz}{\textdctzlig}


\newcommand{\subdot}[1]{\textsubdot{#1}}
\newcommand{\btb}[1]{\textbottomtiebar{#1}}
\renewcommand{\d}[1]{\textsubbridge{#1}}%dentalization "subbridge" mark
\newcommand{\tr}[1]{\textretracting{#1}} %puts retracting symbol under "1"
%\renewcommand{\.}[1]{\textsubdot{#1}}

\newcommand{\textacuteumlaut[1]}{\tipaupperaccent[-.2ex]{1}%
{\tipaupperaccent[-.1ex]{4}{#1}}}
\newcommand{\textacutenasal[1]}{\tipaupperaccent[-.2ex]{1}N}

%stress diacritics
\newcommand{\rt}[1]{{#1}\textrptr} %"1" followed by a raised > pointing right
\newcommand{\ps}{\textprimstress}%primary stress marker




\renewcommand{\rightthumbsdown}{\leftthumbsdown}
\renewcommand{\rightthumbsup}{\leftthumbsup}
\newcommand{\ck}{\checkmark}


%to get superscripted (\up) and subscripted text in a smaller font.
\newcommand{\superscript}[1]{\ensuremath{^{\textrm{\footnotesize{#1}}}}}
\newcommand{\subscript}[1]{\ensuremath{_{\textrm{\footnotesize{#1}}}}}
\newcommand{\up}[1]{\ensuremath{^{\textrm{\footnotesize{#1}}}}}
\newcommand{\down}[1]{\ensuremath{_{\textrm{\footnotesize{#1}}}}}




\newcommand{\HandRight}{\leftthumbsup}
\newcommand{\HandLeft}{\leftthumbsup}

\newcommand{\WaremboriMSR}{MSR\down{[d]$\sim$[r]}}
\newcommand{\WaremboriMSC}{MSC\down{[d]$\sim$[r]}}


\newcommand\marktopleft[1]{%
    \tikz[overlay,remember picture]
        \node (marker-#1-a) at (0,1.5ex) {};%
}
\newcommand\markbottomright[1]{%
    \tikz[overlay,remember picture]
        \node (marker-#1-b) at (0,-1ex) {};%The "-1ex" provides a little extra space between the bottom of the row of text and the circle itself. This is helpful when there are diacritics below the symbols.
    \tikz[overlay,remember picture,thick,inner sep=3pt]
        \node[draw,rounded rectangle,fit=(marker-#1-a.center) (marker-#1-b.center)] {};%
}

\newcommand{\ib}{\texthtb}
\newcommand{\id}{\texthtd}

\newcommand{\XaboveYaboveZplain}[3]{\raisebox{-.9ex}{$\stackrel[{\hbox{#1}}]{\hbox{#2}}{\hbox{ #3}}$}}
\def\boxit#1{%
  \smash{\fboxsep=0pt\llap{\rlap{\fbox{\strut\makebox[#1]{}}}~}}\ignorespaces
}


\colorlet{lightviolet}{red!80!black!20!violet}
\newcommand{\dadp}[1]{\textcolor{lightviolet}{\textbf{#1}}}

\colorlet{dablue}{blue!75!red}
\newcommand{\da}[1]{\textcolor{dablue}{\textbf{#1}}}
% %\colorlet{douggreen}{blue!50!black!40!green}
\colorlet{douggreen}{blue!30!black!40!green}%{blue!65!green}
\newcommand{\pblk}[1]{\textcolor{douggreen}{\textbf{#1}}}



\makeatletter
\newcommand{\labeltext}[2]{%
  \@bsphack
  \csname phantomsection\endcsname % in case hyperref is used
  \def\@currentlabel{#1}{\label{#2}}%
  \@esphack
}

\makeatother

\renewcommand{\textalpha}{α}
\renewcommand{\U}{ʊ}

%below for Yangben
\newcommand{\latr}{*[rtr][atr]}
\newcommand{\ratr}{*[atr][rtr]}
\newcommand{\fr}{\textcolor{red}{\bf *** in the master}$\begin{bmatrix}\textrm{fr}\\\textrm{mid}\end{bmatrix}$C\down{$\emptyset$}$\begin{bmatrix}\textrm{fr}\\\textrm{mid}\end{bmatrix}$}

\newcommand{\rd}{*$\begin{bmatrix}\textrm{nonrd}\\\textrm{mid}\end{bmatrix}$$\begin{bmatrix}\textrm{rd}\\\textrm{mid}\end{bmatrix}$}
\newcommand{\Ytr}{*$\begin{bmatrix}\textrm{atr}\\\textrm{low}\end{bmatrix}$}

\newcommand{\phonotactic}{condition}
\newcommand{\phonotactics}{conditions}
\newcommand{\partition}{set}
\newcommand{\partitions}{sets}
\newcommand{\dom}{$\mathcal{D}$}

%Below are used in Warlpiri.

\newcommand{\WiCu}{*iCu} %20-5-21
\newcommand{\WiTu}{*iTu} %20-5-21
\newcommand{\WuCi}{*uCi} %20-5-21
\newcommand{\Wfr}{*]\down{\sc n}i}
\newcommand{\Wrd}{*u}
\newcommand{\WaCi}{*a\{Ci\}\down{\sc imm.fut}}
\newcommand{\Wnom}{*i]\down{\sc nom}Cu}
\renewcommand{\Wrd}{*u} %20-5-21

%Below are used especially in the Yoruba discussion; if these get changed in some way, be sure to also change the corresponding %'ed line in Yoruba_high_retr_##d# for ease of reference.
\newcommand{\Y}{Yor\`ub\'a}
\newcommand{\SY}{Standard Yor\`ub\'a}
\newcommand{\Ek}{\`Ek\`it\`i}
\newcommand{\If}{\`If\`ak\`i}
\newcommand{\Ij}{\`I\ipa{j\`{\textsubdot{e}}\textsubdot{s}\`a}}
\newcommand{\Is}{I\textsubdot{s}\textsubdot{e}kiri}
\newcommand{\Ylohi}{*[low]C\down{$\emptyset$}[high]}
\newcommand{\Ik}{\`Ik\'a\ipa{l\`{\textsubdot{e}}}}
\newcommand{\PYIG}{Proto-Yoruboid}
%\newcommand{\Mhiatr}{{*}$\begin{bmatrix}\textrm{atr}\\\textrm{high}\end{bmatrix}$C\down{$\emptyset$}[rtr]}
\newcommand{\Mhiatr}{{*}$\begin{bmatrix}\textrm{atr}\\\textrm{high}\end{bmatrix}$[rtr]}
\newcommand{\Matr}{*[atr]}
%\newcommand{\Mbkatr}{{*}$\begin{bmatrix}\textrm{atr}\\\textrm{back}\end{bmatrix}$C\down{$\emptyset$}$\begin{bmatrix}\textrm{rtr}\\\textrm{back}\end{bmatrix}$}
\newcommand{\Mbkatr}{{*}$\begin{bmatrix}\textrm{atr}\\\textrm{back}\end{bmatrix}$$\begin{bmatrix}\textrm{rtr}\\\textrm{back}\end{bmatrix}$}

\newcommand{\PolVC}{*VC\{...\}\down{\textalpha}}
\newcommand{\Identity}{\textcolor{black}{Identity Principle}}

%HIDE TEXT
% Set this =0 to hide/for "disseminate" mode, =1 to show/for "da/dp" mode
\def\showanswers{0}
\newcommand{\hide}[1]{}
% \ifnum\showanswers=1
% #1 %\vspace{\baselineskip}
% % \fancyfoot[L]{\currfilename}
% \fi
% \ifnum\showanswers=0\fancyfoot[L]{\it 2020 December version}\fi
% %\fancyfoot[L]{\it DRAFT. Contents volatile; handle with care. Please do not disseminate.}
% %\fancyfoot[L]{{\it Language Science Press} submission version}
% %\vspace{2\baselineskip} \hspace{2cm}
% %\fi
% }%seems to work, but hidden text must be between {...}
%spaces get added now; not sure why. By eliminating spaces or line breaks here, we minimise the spaces



\makeatletter
\newcommand{\tbox}{%
    \collectbox{%
        \setlength{\fboxsep}{1pt}%
        \setlength{\fboxrule}{0.8pt}
        \shadowbox{\BOXCONTENT}%
    }%
}
\makeatother


%
% %Commands for conditions and language names, etc.
% %adding versions that do not make reference to C0 here, keeping previous versions (just in case)
% %\newcommand{\latr}{*[rtr]C\down{$\emptyset$}[atr]}
% %\newcommand{\Latr}{*[rtr]\down{V} [atr]\down{V}}
% \newcommand{\latr}{*[rtr] [atr]}
% %\newcommand{\ratr}{*[atr]C\down{$\emptyset$}[rtr]}
% \newcommand{\ratr}{*[atr] [rtr]}
%
\newcommand{\czero}{...}
%
% \newcommand{\fr}{\textcolor{red}{\bf *** in the master}$\begin{bmatrix}\textrm{fr}\\\textrm{mid}\end{bmatrix}$C\down{$\emptyset$}$\begin{bmatrix}\textrm{fr}\\\textrm{mid}\end{bmatrix}$}
% %\Big[\xabovey{fr}{mid}\Big]C\down{$\emptyset$}\Big[\xabovey{fr}{mid}\Big]}
% %\newcommand{\rd}{*$\begin{bmatrix}\textrm{nonrd}\\\textrm{mid}\end{bmatrix}$C\down{$\emptyset$}$\begin{bmatrix}\textrm{rd}\\\textrm{mid}\end{bmatrix}$}
% \newcommand{\rd}{*$\begin{bmatrix}\textrm{nonrd}\\\textrm{mid}\end{bmatrix}$~$\begin{bmatrix}\textrm{rd}\\\textrm{mid}\end{bmatrix}$}
%
%
% %{*\Big[\xabovey{nonrd}{mid}\Big]C\down{$\emptyset$}\Big[\xabovey{rd}{mid}\Big]}
% \newcommand{\phonotactic}{condition}
% \newcommand{\phonotactics}{conditions}
% \newcommand{\partition}{set}
% \newcommand{\partitions}{sets}
% \newcommand{\dom}{$\mathcal{D}$}
\newcommand{\tier}{$\mathcal{F}$}
%
% %Below are used in Warlpiri.
%
% \newcommand{\WiCu}{*iCu} %20-5-21
% \newcommand{\WiTu}{*iTu} %20-5-21
% \newcommand{\WuCi}{*uCi} %20-5-21
% \newcommand{\Wfr}{*]\down{\sc n}i}
% \newcommand{\Wrd}{*u}
% \newcommand{\WaCi}{*a\{Ci\}\down{\sc imm.fut}}
% \newcommand{\WaCu}{*a\{Cu\down{{\sc cl.}\textgreek{b}}\}}
% \newcommand{\Wnom}{*i]\down{\sc nom}Cu}
% \renewcommand{\Wrd}{*u} %20-5-21
%
% %Below are used especially in the Yoruba discussion; if these get changed in some way, be sure to also change the corresponding %'ed line in Yoruba_high_retr_##d# for ease of reference.
% \newcommand{\Y}{Yor\`ub\'a}
% \newcommand{\SY}{Standard Yor\`ub\'a}
% \newcommand{\Ek}{\`Ek\`it\`i}
% \newcommand{\If}{\`If\`ak\`i}
% \newcommand{\Ij}{\`I\ipa{j\`{\textsubdot{e}}\textsubdot{s}\`a}}
% \newcommand{\Is}{I\textsubdot{s}\textsubdot{e}kiri}
% \newcommand{\Ylohi}{*[low]C\down{$\emptyset$}[high]}
% \newcommand{\Ik}{\`Ik\'a\ipa{l\`{\textsubdot{e}}}}
% \newcommand{\PYIG}{Proto-Yoruboid}
% %\newcommand{\Mhiatr}{{*}$\begin{bmatrix}\textrm{atr}\\\textrm{high}\end{bmatrix}$C\down{$\emptyset$}[rtr]}
% \newcommand{\Mhiatr}{{*}$\begin{bmatrix}\textrm{atr}\\\textrm{high}\end{bmatrix}$~[rtr]}
% \newcommand{\Matr}{*[atr]}
% %\newcommand{\Mbkatr}{{*}$\begin{bmatrix}\textrm{atr}\\\textrm{back}\end{bmatrix}$C\down{$\emptyset$}$\begin{bmatrix}\textrm{rtr}\\\textrm{back}\end{bmatrix}$}
% \newcommand{\Mbkatr}{{*}$\begin{bmatrix}\textrm{atr}\\\textrm{back}\end{bmatrix}$~$\begin{bmatrix}\textrm{rtr}\\\textrm{back}\end{bmatrix}$}
% %by accident, we have TWO ways of getting the same result for two cases: \Mahr = \Mhiatr; \Mbabr = \Mbkatr
\newcommand{\Mahr}{{*}$\begin{bmatrix}\textrm{atr}\\\textrm{high}\end{bmatrix}$~[rtr]}
\newcommand{\Mbabr}{{*}$\begin{bmatrix}\textrm{atr}\\\textrm{back}\end{bmatrix}$~$\begin{bmatrix}\textrm{rtr}\\\textrm{back}\end{bmatrix}$}


\newtcbtheorem[number within=chapter,list inside=dadpboxes]
  {dadpbox}
  {Box}
  { 
    graphical environment = tikzpicture,
    boxsep = 0pt,
    fonttitle = \normalsize\sffamily\bfseries,
    toptitle = 5mm,
    top = 5mm,
    bottom = 5mm,
    left = 5mm,
    right = 5mm,
    borderline = {0.8mm}{0pt}{\lsSeriesColor},
    sharp corners = all,
    before upper = {\hspace*{-\parindent}},
    beforeafter skip balanced = \baselineskip,
  }
  {box}
  
\DeclareTColorBox{whiteshadowbox}{O{black!12}}
  {
    breakable,
    graphical environment = tikzpicture,
    boxsep = 0pt,
    toptitle = 5mm,
    top = 5mm,
    bottom = 5mm,
    left = 5mm,
    right = 5mm,
    frame engine = path,
    frame style = {fill=#1},
    sharp corners = all
  }

\newenvironment{example}{\ea}{\z}

\newcommand{\É}{\'{ɛ}}
\newcommand{\È}{\`{ɛ}}
\newcommand{\Ó}{\'{ɔ}}
\newcommand{\Ò}{\`{ɔ}}
\newcommand{\Ú}{\'{ʊ}}
\newcommand{\Ù}{\`{ʊ}}
\newcommand{\Í}{\'{ɪ}}
\newcommand{\Ì}{\`{ɪ}}
\newcommand{\í}{\'{\.{ı}}}
\newcommand{\ì}{\`{\.{ı}}}

\forestset{
  asr/.style={s sep=0pt, 
              for descendants={inner sep=0pt, no edge}, 
              calign=child}
}
