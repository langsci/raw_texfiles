\chapter{Building a phonological grammar} \label{chapter_grammar}
\label{Yangben_assessment_section}\label{ch3}

Hearing, remembering\is{memory}, noting similarities\is{similarity} and frequencies,\is{frequency} creating categories\is{category}, generalising\is{generalisation}, and re-generalising serve not only to identify distinctive sounds (as laid out in Chapter \ref{ch2}) but also help the learner identify words and morphs within words.\is{word} As morphemic structure is acquired,\is{acquisition!morph} the learner must recognise that there can be multiple morphs with the same meaning,\is{meaning|(} that there are systematic ways of combining morphs\is{compilation} to generate new words, that forming well-formed words involves selecting among possible morphs with related meanings, and that the class of observed morphs can be expanded to enable use of as-yet-unheard morphs.\is{morph}

Concreteness continues to guide us. Learners encounter surface forms that are meaningful;\is{meaning} they therefore posit representations corresponding to pairing such forms with their meanings. Learners encounter instances where phonologically distinct forms have the same meaning; they therefore posit representations corresponding to such cases. Learners also encounter instances where a single phonological form has more than one meaning; again, the learner posits representations that transparently correspond to such occurrences. The fundamental notion is one where sound strings are meaningful, and where the sound-meaning correspondences\is{sound-meaning correspondence} are represented in the grammar.\is{grammar} Meaningful (surface) morphs are grouped into sets according to their\is{surface-to-surface} meaning.\is{acquisition!morph}\is{morph}\is{morph set}\is{meaning}

In this chapter, we therefore focus on morph sets and their interactions, beginning with characterising related morphs as sets (\Sec\ref{section_intro_morph-set}). When morphs from different sets are compiled to create words, some means is needed to select among the possible compilations.\is{compilation} In large part, this is a consequence of the well-formedness conditions\is{well-formedness condition} already  identified in order to understand the relevant categories in the language (\Sec\ref{section_choice}). We close with a discussion of systematic phonological relations among members of morph sets and how to determine which combination of morphs is appropriate to represent a given set of morphosyntactic\is{feature!morphosyntactic} features\is{morph}\is{morph set} (\Sec\ref{Yangben-MSRs-section}).

\section{Cataloguing the data: Morph sets}\label{section_intro_morph-set}
We hypothesise a learning trajectory along the following lines. As the learner becomes aware of  and stores longer sequences,\is{acquisition!early}  recurring longer chunks\is{chunk!frequency} get high frequency\is{acquisition!frequency} counts; those sequences\is{sequence!acquisition} become the learner's proto-words\is{proto-word} (\citealt{Martin+:2013}).  Refinement occurs over time as more items are acquired,\is{acquisition!morph} and stored words\is{word} become increasingly adult-like. The learner recognises that certain sound strings cooccur with particular meanings.\is{sound-meaning correspondence} We define morphs\is{morph!definition} along the lines of traditional definitions of the \is{morpheme}``morpheme'' (\citealt{Bloomfield:1933}, etc.; see discussion in \textsection\ref{morph-sets-as-building-blocks-section}): a \textit{morph} is a string of sounds associated with a meaning and/or grammatical function:\is{morph!definition} \{[sound string]\}\down{\sc meaning/function}.\footnote{This use of the term \textit{morph} is consistent with our earlier work (such as \citealt{Archangeli+:2015_Frontiers, Archangeli+:2018routledge, Archangeli+:2018Henry}). See also \citet[117]{Haspelmath:2020morph}:  ``A morph is a minimal linguistic form''.\is{morph!definition}} In addition, as we discuss below, there are many instances where more than one string of sounds corresponds to the same meaning.\is{sound-meaning correspondence} We refer to sets of morphs which share a label as \textit{morph sets}\is{morph set!definition} where a morph set could include a single morph or multiple morphs: \{[sound string]\down{1}, [sound string]\down{2}, ...\}\down{\sc meaning/function}.

\begin{dadpbox}{Morph Sets}{box-morph-sets}
Morph\is{morph set!definition} sets are defined by morphosyntactic\is{feature!morphosyntactic} features. A morph set may have as few as one member, but is not restricted to only one. Where there are multiple members, the different morphs in a set may be systematically related to each other, although this is not necessary. Systematic relations are captured by \textit{Morph Set Relations} (\ref{section_MSR_introduction});\is{Morph Set Relation} productive relations are expressed with \textit{Morph Set Conditions} (\ref{section_MSCs}).\is{Morph Set Condition} Additionally, there may be well-formedness conditions\is{well-formedness condition} that restrict a class of morphs; such conditions may be distinct from the well-formedness conditions which play a role in selecting between morph compilations\is{compilation} that make up words.
\end{dadpbox}

Morphs and morph sets\is{morph!acquisition|(}\is{morph set} are identified through their similarity\is{similarity!morphs} -- initially, similarity of both the phonological string and the semantic\is{semantics} and syntactic functions\is{syntax!syntactic function} -- and their frequency of occurrence.\is{frequency} A Yangben  learner might begin identifying words (due to phonotactics)\is{phonotactics!acquisition} such as \ipa{[kùt{\ì}m]} `dig.{\sc inf}' and \ipa{[kùtùn]} `back up (rear first).{\sc inf}', alongside \ipa{[k\`{ʊ}n\`{ɛ}ːn]} `abandon, let fall.{\sc inf}' and  \ipa{[k\`{ʊ}j\`{ɛ}k]} `rot.{\sc inf}'. Without knowing the syntactic function of these sequences, the learner might still note the similarity of how the words begin, and identify two classes, the  [kù...]~class and the \ipa{[k\`{ʊ}...]}~class. (In (\ref{Yangben_early_morph_learning}), the component words in the glosses are separated by ``.'' to denote that the learner has yet to identify  independent meanings for subparts of the word; data are from \citealt[162]{Boyd:2015}.)\is{acquisition!early}\is{acquisition!sets}\is{acquisition!morph}\is{set!acquisition}

\begin{example} \et{Early morph acquisition: two word classes, [{kù}...]-words and [k\ipa{\`{ʊ}}...]-words } \label{Yangben_early_morph_learning}
    \ea \begin{tabular}[t]{@{}ll@{}}
    \multicolumn{2}{@{}l}{[{kù}...]-words}\\
    \ipa{[kùt{\ì}m]} &`to.dig'\\                        
    \ipa{[kùtùn]} &`to.back.up.(rear.first)'           
    \end{tabular}
    \ex \begin{tabular}[t]{@{}ll@{}}
    \multicolumn{2}{@{}l}{[k\ipa{\`{ʊ}}...]-words}\\
      \ipa{[k\`{ʊ}n\`{ɛ}ːn]} &`to.abandon, to.let.fall'\\
      \ipa{[k\`{ʊ}j\`{ɛ}k]} &`to.rot'
    \end{tabular}
    \z
\end{example}

Note that these classes must be distinct if the learner has established the difference between [u] and [ʊ] prior to the onset of vocabulary development; see \textsection\ref{Yangben_categories_section}.

As syntactic\is{syntax!syntactic function} functions are identified, the learner establishes that these two classes each have a particular syntactic function, {\sc infinitive}. Now the two classes can be labelled more precisely, the [kù...]\down{\sc inf} class and the \ipa{[k\`{ʊ}...]}\down{\sc inf} class, leading to the generalisation\is{generalisation} that ``{\sc infinitive} begins with [kù] or \ipa{[k\`{ʊ}]}''.\is{acquisition!morph}

\begin{example} \et{Mid morph acquisition: two classes of {\sc infinitives}}
    \begin{tabular}{@{}l@{~~}ll@{~~}l@{}}
    \multicolumn{2}{@{}l}{{[kù...]}-{\sc infinitives}}&\multicolumn{2}{l}{[k\`{ʊ}\ipa{...}]-{\sc infinitives}}\\
    \ipa{[kùt{\ì}m]} &`dig-{\sc inf}' &\ipa{[k\`{ʊ}n\`{ɛ}ːn]} &`abandon-{\sc inf}, let.fall-{\sc inf}'\\
    \ipa{[kùtùn]} &`back.up.(rear.first)-{\sc inf}' &\ipa{[k\`{ʊ}j\`{ɛ}k]} &`rot-{\sc inf}'
    \end{tabular}
\end{example}


Similarity\is{similarity!morphs} of the initial sound sequence\is{sequence} in each class leads to separating those initial sequences into units that are independent of the rest of the word.\is{acquisition!morph}

\begin{example} \et{Mid morph acquisition: two  {\sc infinitive} morphs: \{kù\}\down{\sc inf}, \{k\ipa{\`{ʊ}}\}\down{\sc inf}}\smallskip\\
    \begin{tabular}{@{}l@{~~}ll@{~~}l@{}}
    \multicolumn{2}{@{}l}{[kù...]-{\sc infinitives}}&\multicolumn{2}{l}{[\ipa{k\`{ʊ}}]...-{\sc infinitives}}\\
    \ipa{[kù-t{\ì}m]} &`dig-{\sc inf}' &\ipa{[k\`{ʊ}-n\`{ɛ}ːn]} &`abandon-{\sc inf}, let.fall-{\sc inf}'\\
    \ipa{[kù-tùn]} &`back.up.(rear.first)-{\sc inf}' &\ipa{[k\`{ʊ}-j\`{ɛ}k]} &`rot-{\sc inf}'
    \end{tabular}
\end{example}

Once this step is reached, the learner is in position to hypothesise both \{kù\}\down{\sc inf} and \{\ipa{k\`{ʊ}}\}\down{\sc inf}, and, by the union of these two {\sc infinitive} forms, the morph set \{kù, \ipa{k\`{ʊ}}\}\down{\sc inf}. This hypothesis is supported by identifying the remainder of the forms as a verb morph, which in turn is supported by learned forms containing the same sound-meaning pairing.\is{acquisition!morph set}\is{acquisition!late}\is{sound-meaning correspondence}

\begin{example}\et{Late morph acquisition: an  {\sc infinitive} morph set, plus verb morphs}\smallskip\\
    \begin{tabular}{@{}llll@{}}
    \{kù, k\ipa{\`{ʊ}}\}\down{\sc infinitive}\\
    \ipa{\{t{\ì}m\}}\down{\sc dig}&
    \{\ipa{n\`{ɛ}ːn}\}\down{\sc abandon, let.fall}\\
    \{tùn\}\down{\sc back.up.(rear.first)}&
    \{\ipa{j\`{ɛ}k}\}\down{\sc rot}
    \end{tabular}
\end{example}

As more infinitives are learned, the hypothesis is repeatedly supported: infinitive forms begin with either [kù] or \ipa{[k\`{ʊ}]},  (\ref{Yangben-infinitives}). The stronger the evidence,\is{grammar} the more entrenched the hypothesis becomes in the grammar.\footnote{In (\ref{Yangben-infinitives}), the final suffix is the {\sc continuous}, except for one form that involves the causative, [\ipa{kù-téːɲ-{\ì}}] `drip-{\sc causative}'. We do not address suffix-related alternations\is{alternation!Yangben} here, such as the alternations in root vowels when in the context of different suffixes.}\is{acquisition}


\begin{example} \et{Yangben infinitives (\citealt[162]{Boyd:2015})} \label{Yangben-infinitives}
    \ea\relax {[atr]} roots\smallskip\\
    \begin{tabular}{@{}ll ll@{}}
       & {\sc inf}-verb-{\sc cont}	&{\sc inf}-verb	&gloss\\
       \relax [i]&\ipa{kù-t{\í}m-è}	&\ipa{kù-t{\ì}m}	&`dig'\\ %
       \relax &\ipa{kù-t{\í}ːn-è} 	&\ipa{kù-t\^{\.{i}}ːn}	&`flee in fear' \\
       \relax [e]&\ipa{kù-sèl-èn}	&\ipa{kù-sèl}	&`descend'\\
       \relax &\ipa{kù-téː\jn-{\ì}}	&\ipa{kù-têːn}	&`(make) drip'\\
       \relax [o]&\ipa{kù-p{\í}-kóf-ò} &\ipa{kù-p{\í}-kòf} &`devour' \\
       \relax &\ipa{kù-fóːk-òn} &\ipa{kù-fôːk} &`advance, go ahead' \\
       \relax [u]&\ipa{kù-tùn-è}	&\ipa{kù-tùn} &`back up (rear first)' \\
       \relax &\ipa{kù-túːn-è}	&\ipa{kù-tûːn} &`crush'\\
    \end{tabular}
    \ex\relax  {[rtr]} roots\smallskip\\      
    \begin{tabular}{@{}ll ll@{}}
    & {\sc inf}-verb-{\sc cont}	&{\sc inf}-verb	&gloss\\
    \relax [\ipa{ɪ/ɛ}]&\ipa{k\`{ʊ}-j\`{ɪ}k-à} 	&\ipa{k\`{ʊ}-j\`{ɛ}k}  &`rot'\\
    \relax &\ipa{k\`{ʊ}-j\'{ɪ}ːl-à} 	&\ipa{k\`{ʊ}-j\^{ɛ}ːl}  &`(be) slimy (food)'\\
    \relax [\ipa{ɛ}]&\ipa{k\`{ʊ}-f\`{ɛ}k-\`{ɛ}} 	&\ipa{k\`{ʊ}-f\`{ɛ}k}	&`measure' \\
    \relax &\ipa{k\`{ʊ}-n\`{ɛ}ːn-\`{ɛ}n} 	&\ipa{k\`{ʊ}-n\`{ɛ}ːn}	&`abandon, let fall' \\
    \relax [\ipa{a}]&\ipa{k\`{ʊ}-fát-à}	&\ipa{k\`{ʊ}-fàt} 		&`husk (corn); shell' \\
    \relax &\ipa{k\`{ʊ}-fáːt-à}	&\ipa{k\`{ʊ}-fâːt} 		&`carve, sharpen' \\
    \relax [\ipa{ɔ}]&\ipa{kù-s\'{ɔ}k-\ipa{\`{ɔ}}} &\ipa{k\`{ʊ}-s\`{ɔ}k} 	&`extract' \\
    \relax &\ipa{k\`{ʊ}-s\'{ɔ}ːk-\`{ɔ}} &\ipa{k\`{ʊ}-s\^{ɔ}ːk} 	&`grow (of plants)' \\
    \relax [\ipa{ʊ/ɔ}]&\ipa{kù-k\'{ʊ}t-à}	&\ipa{k\`{ʊ}-k\`{ɔ}t} &`fasten, bind' \\
    \relax &\ipa{k\`{ʊ}-p\'{ʊ}ːk-à}	&\ipa{k\`{ʊ}-p\^{ɔ}ːk} &`cook meat (wrapped in leaves)'
    \end{tabular}
    \z
\end{example}

To achieve a maximal degree of generalisation\is{generalisation}, the learner moves towards definitions of morph sets\is{acquisition!morph set} that  group together all morphs with the same syntactic or semantic properties, and that provide sets which cannot be further broken down in terms of the syntactic or semantic components -- minimal morph sets.\is{morph set!minimal}\is{morph!acquisition|)}\is{meaning|)}

\begin{dadpbox}{Minimal Morph Sets}{box-minimal-morph-sets}
While\is{morph set!minimal} morph sets are defined as the set of morphs sharing any set of morphosyntactic\is{feature!morphosyntactic} features, we use the term \textit{minimal morph set} for a set bearing a simple semantic label.\\ 

Examples of minimal morph sets include sets with single members, such as \{\ipa{dɑɡ}\}\down{\sc dog, noun}, and sets with multiple members, such as \{hɪt, hɪɾ\}\down{\sc hit, verb}, \{naɪf, naɪv\}\down{\sc knife, noun}, and \{mæn, mɛn\down{\sc pl}\}\down{\sc man}.\\

Fundamentally, the \is{acquisition}learner is concerned with chunking\is{chunk} the speech stream and identifying strings of sounds with a particular meaning\is{meaning} or \is{syntax}syntax.\is{sound-meaning correspondence}  In many instances, such chunking will result in sets that are non-minimal. As described in \citet{Bybee:1999}, ``[i]n network models, internal structure is emergent -- it is based on the network of connections built up among stored units. The stored units are pronounceable linguistic forms -- \is{word}words or \is{phrase}phrases stored as clusters of surface variants organised into clusters of related words.''\is{word} In our terms, a chunk\is{chunk} like \{hæpin\@s\}\down{\sc happiness, noun} would belong with \{\ipa{dɑɡ}\}\down{\sc dog, noun} in a non-minimal set labelled {\sc noun} -- all nouns in the language would be included in that set. Another type of non-minimal morph set\is{morph set!non-minimal} is a semantically random set of items linked only by their behaviour, such as the various sets making up the noun classes in Yangben, linked by the affixes they appear with, or the set of morphs which behave disharmonically\is{disharmony} (see box \ref{box:box-alpha-classes} on p.\ \pageref{box:box-alpha-classes} and the surrounding discussion).\\

In general, we refer to \textit{morph sets}, reserving the term \textit{minimal morph set} for contexts where the distinction is critical.\is{morph set!minimal}
\end{dadpbox}


\section{Selection among multiple morphs: Well-formedness conditions}\label{section_choice}

During production,\is{production} speakers identify meanings\is{meaning!production} that they wish to express. For the meaning α--β, this involves bringing together morph sets \{...\}\down{α} and \{...\}\down{β}. When morph sets are combined, the consequence may bring together singleton morph sets  (\{X\}\down{α}+\{A\}\down{β}), or morph sets with multiple members (\{X, Y, ...\}\down{α}+\{A, B, ...\}\down{β}), or some combination of singleton and multiple morph sets.\is{compilation} For instance, Yangben {\sc infinitive-back.up} identifies the morph sets \{kù, \ipa{k\`{ʊ}}\}\down{\sc inf} and \{tùn\}\down{\sc back.up} while {\sc infinitive-rot} identifies \{kù, \ipa{k\`{ʊ}}\}\down{\sc inf}, and \{\ipa{j\`{ɪ}k, j\`{ɛ}k}\}\down{\sc rot}.\footnote{The set \{\ipa{j\`{ɪ}k, j\`{ɛ}k}\}\down{\sc rot} may be a truncated morph set. We provide evidence in \Sec\ref{Yangben-MSRs-section}  for there being additional morphs involving tongue root alternations, and there may also be tonal variation (which we do not discuss), e.g.\ \{\ipa{s\'{ɔ}ːk, \ipa{s\^{ɔ}ːk}, sóːk}\}\down{\sc grow}.} Where morph sets contain multiple members, as  in these examples, multiple potential words are created by compiling the morphs in the logically possible ways.

\begin{dadpbox}{Morph Compilation}{box-morph-compilation}

The \textit{morph compilation} \is{compilation!definition}for two morph sets, \{...\}\down{α} and  \{...\}\down{β}, is created by combining each member of \{...\}\down{α} with each member of  \{...\}\down{β}. \\


If there are more than two morph sets to be compiled, the morph compilation is built in the same way, resulting in all possible combinations of the individual morphs from the participant morph sets. (Assessment\is{assessment!morph order} selects one  member from each morph compilation; in assessments, we consider only compilations\is{compilation} with the correct order of morphs.)\\

\begin{tabular}{|l|l|}
\hline
{\it Morph sets} &\{\{kù, \ipa{k\`{ʊ}}\}\down{\sc inf} + \{tùn\}\down{\sc back.up}\}\down{\sc infinitive-back.up}\\
{\it Morph compilation} &kù-tùn, \ipa{k\`{ʊ}}-tùn\\
{\it Output of assessment} &\{kù-tùn\}\down{\sc inf-back.up}\\\hline
\multicolumn{2}{l}{~~}\\
\hline
{\it Morph sets} &\{\{kù, \ipa{k\`{ʊ}}\}\down{\sc inf} + \{\ipa{j\`{ɪ}k, j\`{ɛ}k}\}\down{\sc rot}\}\down{\sc infinitive-rot}\\
{\it Morph compilation} &\ipa{kù-j\`{ɪ}k, kù-j\`{ɛ}k, k\`{ʊ}-j\`{ɪ}k, k\`{ʊ}-j\`{ɛ}k}\\
{\it Output of assessment} &\{\ipa{k\`{ʊ}-j\`{ɛ}k}\}\down{\sc infinitive-rot}\\
\hline
\end{tabular}\is{compilation!definition}
\end{dadpbox}


Part of the task of the speaker is to identify which morph combination to use in cases where there are multiple options, an uncertain situation.\is{uncertainty!resolution} In this section, we have two goals. First, we consider ways in which such uncertainties are resolved, focussing on  \textit{syntagmatic} and \textit{paradigmatic} well-formedness conditions\is{well-formedness condition!syntagmatic}\is{well-formedness condition!paradigmatic} guided by frequency.\is{frequency} Second, we provide the formal framework for the kinds of \textit{well-formedness conditions} that we motivate throughout this monograph. 


\subsection{Syntagmatic conditions} \label{phonotactic-selection-section-revised}
We have already introduced the key component in choosing between alternative members of a morph set, well-formedness conditions of different types ((\ref{Yangben-atr/rtr-phonotactic}) and (\ref{Yangben-final-voiceless}) in Chapter \ref{ch2}). Syntagmatic phonotactics\is{well-formedness condition!syntagmatic} of the type schematised in (\ref{phonotactics-schema-prelim})  play a central role in this kind of choice.  (See (\ref{phonotactics-schema-original}) for the final version of this schema.)\is{well-formedness condition!syntagmatic, formalised}

\begin{example} \et{Preliminary syntagmatic phonotactic schema}\is{phonotactics!syntagmatic} \label{phonotactics-schema-prelim}\smallskip\\
    \begin{tabularx}{\linewidth}{@{}lQ@{}}
        *[X][Y] & Assign a violation to a form for each  sequence of [X] followed by [Y], where [X], [Y] are phonological properties.
    \end{tabularx}
\end{example}

Phonological properties are\is{phonological properties} the labels assigned to partitioned\is{partition} sets; these may be simple, referring to sets with a single label ([low]); they may also be complex, referring to the intersection of two sets ([high, atr]); they may also be a complement class\is{complement class} (\up{$\wedge$}[coronal]).

\begin{dadpbox}{How close must two elements in a phonotactic be to each other?}{box-locality-in-conditions}
\is{locality|(}While there is no question that well-formedness conditions\is{well-formedness condition!locality} frequently hold of segments, X and Y, that are string-adjacent, there are also cases where the interacting segments are separated by some number of segments. This can be seen in cases of vowel harmony,\is{harmony!vowel} consonant harmony,\is{harmony!consonant} consonant and vowel dissimilation,\is{dissimilation} and so on. At issue is whether well-formedness conditions must be allowed to include potentially \is{locality}unbounded non-participating classes of segments (W in *XWY) or whether simple well-formedness conditions hold of ``tiers'' or ``projections'' of some kind (*XY, W excluded from consideration). Under the tier/projection approach, for example, if we specifically focus\is{focus \tier} on the class of vowels, excluding consonants, then vowel harmony\is{harmony!vowel} is governed by a {\it local} condition governing vowel sequences.\is{sequence}\is{well-formedness condition!syntagmatic} See, for example, \citet{Jardine:2016, Jardine+:2016, Gouskova+:2020} for recent approaches to how nonlocal conditions might be learned in such an approach. We assume some version of such a tier/projection model here, designating the class of segments constituting the \textit{focus} of\is{focus \tier} a well-formedness condition by \tier\ in the formal expression of a condition.\is{phonotactics!locality}\\

We assume that local effects will be easier to observe and therefore both more frequent\is{frequency!locality} and more \is{learnable}learnable (\citealt{Finley:2011, Finley:2012, McMullin:2016}); such local effects could also be due to coarticulatory\is{coarticulation}\is{articulation!coarticulation} effects that would be observable locally but not nonlocally. This, likely in combination with properties of the learner (\citealt{Hayes+:2008, Gouskova+:2020}), can derive the claim that more proximal  patterns are more likely (\citealt{Suzuki:1998, Pulleyblank:2002, Rose+:2004, Hansson:2010}).\is{acquisition!locality} We expect that the salience of a not-strictly-local class is crucial in differentiating  the focus\is{focus \tier} segments from the segments that are excluded by some well-formedness condition, with  more salient  classes being more likely to serve as the focus of some  well-formedness condition.\is{well-formedness condition!locality} For example, vowels share salient features acoustically\is{acoustics!vowels} that consonants as a class do not; certain sub-classes of consonants are distinctively salient, however, such as the class of sibilants. \\ \is{locality|)}
\end{dadpbox}

\noindent
To illustrate with Yangben, the harmony\is{harmony!Yangben conditions} conditions\is{well-formedness condition!syntagmatic} given in (\ref{Yangben-atr/rtr-phonotactic}) in Chapter \ref{ch2}, \latr\ and \ratr, determine whether [atr] or [rtr] variants of class prefixes are chosen from the relevant prefixal morph sets. Consider the example of [\ipa{kù-sèl}]\down{\sc descend} from (\ref{Yangben-infinitives}a). Here, while the prefix has the two morphs, \{\ipa{kù, k\`{ʊ}}\}, the verb stem has only one morph, \{sèl\}\down{\sc descend}. Since the Yangben harmony conditions\is{harmony!Yangben conditions}  prohibit words that mix tongue root specifications, the only well-formed prefix option  in this case is advanced [kù], shown by the \textit{assessment table} in (\ref{Yangben-descend-inf-moved}).\is{assessment}

\begin{dadpbox}{Assessment tables}{box-read-assessment-table}
\is{assessment table!how to read}

\is{assessment!formalised}\is{Emergence vs.\ OT}
The role of an \textit{assessment table} is to demonstrate that a proposed grammar serves to capture an observed phonological distribution.\is{distribution!assessment} Each table demonstrates whether a proposed ranking\is{ranking} of specific well-formedness conditions\is{well-formedness condition} appropriately identifies an attested morph compilation\is{compilation} corresponding to the desired semantic, syntactic, and morphological properties. \\

Assessment tables intentionally follow the organisation of the tableaux in Optimality Theory\is{Optimality Theory} (\citealt{Prince+:1993}), with certain crucial differences. In an assessment table, the upper left-hand cell identifies the morph sets to be compiled, either by the morph sets themselves, as in (\ref{Yangben-descend-inf-moved}), or by the morphosyntactic features,\is{feature!morphosyntactic} as in (\ref{Yangben-uproot-cause}). In either case, we include the morph sets\is{morph set!assessment table} being compiled in each example above the related assessment table, to be clear about what is being compiled.\\

The left-hand column lists the members of the morph compilation\is{compilation} -- the logically possible combinations of the morphs in the relevant morph sets; this is a finite set. Phonotactics\is{phonotactics!assessment} and other well-formedness conditions\is{well-formedness condition!assessment} motivated for the language are arrayed across the top row of the table.  The ``*'' marks where a well-formedness condition is violated by the form; ``*!'' marks a fatal violation. In (\ref{Yangben-descend-inf-moved}), the infinitive has two morphs and the verbs each have one, so there are only two possible forms to assess. \\
 
Following the Optimality Theory convention,\is{Optimality Theory} a solid line between two conditions indicates that the well-formedness condition on the left has priority over the condition on the right. Dashed lines between conditions indicates that the ranking\is{ranking!notation} of those conditions is not crucial. While we assume strict domination\is{ranking!strict domination} in our presentation throughout, the adequacy of strict domination may be an artifact of the examples we discuss. We do not discuss alternatives such as weighted conditions\is{well-formedness condition!weighted} (see, for example, \cite{Pater:2009weights, vandeWeijer:2012}), leaving for further investigation the nature of the interaction among conditions.
\end{dadpbox}\is{assessment!formalised}


\begin{example} \et{Assessment for [\ipa{kù-sèl}]\down{\sc infinitive-descend}}
    \label{Yangben-descend-inf-moved}

    {\it morph sets}: \{\ipa{kù, k\`{ʊ}}\}\down{\sc infinitive}; \{\ipa{sèl}\}\down{\sc descend}

    \begin{center}
    \renewcommand*{\arraystretch}{1.2}
    \begin{tabular}{lll | c : c}
    \hline
    \hline
    \multicolumn{3}{c|}{\{\ipa{kù, k\`{ʊ}}\}\down{\sc infinitive}-\{\ipa{sèl}\}\down{\sc descend}}	&\latr &\ratr				 \\
    \hline
    &a. &\ipa{k\`{ʊ}-sèl}	&*!&		 	  	\\
    \hline
    \rightthumbsup
    &b. &\ipa{kù-sèl}	&&	 	\\
    \hline
    \hline 
    \end{tabular}
    \end{center}
\end{example}

In parallel fashion, if the root is [rtr] (e.g.\ \ipa{[k\`{ʊ}-fàt]} `husk (corn); shell') then the only well-formed prefix option would be the [rtr] morph.

In such cases, the phonotactics\is{phonotactics!assessment} posited as the result of observations about words are reinforced in the grammar\is{grammar} by their role in assessing\is{assessment} morph compilations.\is{compilation} The skewed distribution\is{distribution!Yangben vowels} within Yangben vowel sequences\is{sequence!phonotactic} is that advanced vowels occur more frequently with adjacent advanced vowels  while retracted vowels occur more frequently\is{frequency!locality} with  adjacent  retracted vowels. This is encoded as a purely phonological well-formedness condition\is{well-formedness condition!syntagmatic} on sequences.\is{sequence!phonotactic}

\begin{dadpbox}{Emergence is not Optimality Theory in disguise}{box-EG-is-not-OT}
\is{Emergence vs.\ OT}

While assessments\is{assessment!compared to OT} owe an obvious debt to the tableaux of Optimality Theory, Emergent Phonology is  quite different from Optimality Theory.\is{Optimality Theory} Four key differences  are noted here.  
\begin{enumerate}
\item {\itshape Richness of the Base}: There is no Richness of the Base in Emergence. Inputs are morph sets which are based on observed forms -- either directly observed or generated from an observed form based on a (deduced) general pattern.\is{richness of the base} 
\item {\itshape Gen}:\is{Gen} There is no infinite candidate set in Emergence; instead there is  morph compilation\is{compilation} (see box \ref{box:box-morph-compilation}  on p.\ \pageref{box:box-morph-compilation}). That is, the forms to be assessed are the result of compiling morphs from the relevant morph sets, a finite number of compiled forms. Since there is only one morph for \{tùn\}\down{\sc back.up}, all forms with {\sc back.up} in their compilation include this morph. An imaginable form with a retracted verb root vowel, such as *[\ipa{k\`{ʊ}-t\`{ʊ}n}], cannot result from compiling the relevant morphs (since [\ipa{t\`{ʊ}n}] is not a member of the set labelled {\sc back.up}) so such a form is simply not an option.

\item {\itshape Faithfulness}: There is no role for Faithfulness conditions in Emergence. The forms to be assessed are composed from morphs that correspond to surface forms.
\item {\itshape Universality}: There is no ``universal constraint set'' in Emergence. The phonotactics\is{phonotactics!assessment} and other well-formedness conditions used  for assessing the possible compilations\is{compilation} are exactly those motivated for the language. (We reserve the term \textit{constraint} for  universal prohibitions as in Optimality Theory; we use \textit{(well-formedness) conditions}\is{well-formedness condition} when referring to the learned prohibitions within the Emergent framework.)
\end{enumerate}\is{assessment!compared to OT}
\end{dadpbox}

\subsection{Prohibitions of types}\label{section_defaults}\label{section_types}
While sequential\is{sequence!phonotactic} phonotactics are often sufficient to determine the choice between possible compilations,\is{compilation} there are also instances where additional well-form\-ed\-ness conditions are required. In Yangben, this can be illustrated by the  root alternations\is{alternation!Yangben} observed when forms with two suffixes,   {\sc causative} and {\sc agentive}, are compared with the familiar infinitive forms. 


\begin{example} \et{Yangben root alternations, {\sc causative} \& {\sc agentive} (\citealt[177]{Boyd:2015})}\\ \label{Yangben-root-allomorphs-causative-moved}
    \ea \begin{tabular}[t]{@{}l ll ll@{}}
        \relax 	&\multicolumn{2}{l}{{\sc inf}-verb}&\multicolumn{2}{l}{{\sc inf}-verb-{\sc causative}}\\
     	\relax [\ipa{u}]&\ipa{kù-sùk} &`miss, stop'	&\ipa{kù-súk-{\ì}}  &`cause to stop'\\
        \relax [\ipa{ʊ, u}]&\ipa{k\`{ʊ}-f\'{ʊ}l-à} &`flow'	&\ipa{kù-fúl-{\ì}}  &`cause to flow'\\
        \relax [\ipa{ɔ, o}]&\ipa{k\`{ʊ}-s\'{ɔ}ːk-\`{ɔ}} &`grow'	&\ipa{kù-sóːk-{\ì}}  &`germinate'\\
        \relax [\ipa{a, e}]&\ipa{k\`{ʊ}-pàl} &`uproot'	&\ipa{kù-pèl-{\ì}}  &`cause to uproot'\\
        \relax [\ipa{ɛ, e}]&\ipa{k\`{ʊ}-k\'{ɛ}t-\`{ɪ}k} &`blink'	&\ipa{kù-két-{\ì}k-èɲ-{\ì}}  &`cause to blink'\\
        \relax [\ipa{ɪ, i}]&\ipa{k\`{ʊ}-j\`{ɪ}k-à} &`boil'	&\ipa{kù-j{\ì}k-{\ì}}  &`boil over'\\
    \end{tabular}
    \ex \begin{tabular}[t]{@{}l ll ll@{}}
        & \multicolumn{2}{l}{{\sc inf}-verb}&\multicolumn{2}{l}{{\sc inf}-verb-{\sc agentive}}\\
        \relax [\ipa{e}]&\ipa{kʷ-ěp-è}&`steal'&\ipa{èŋ-ép-{\ì}}  &`robber'\\
        \relax [\ipa{ʊ, u}]&\ipa{k\`{ʊ}-s\`{ʊ}l-à} &`drink (spoon)'	&\ipa{è-sùl-{\ì}}  &`drinker'\\
        \relax [\ipa{ɔ, o}]&\ipa{k\`{ʊ}-l\'{ɔ}k-\`{ɔ}} &`fish'	&\ipa{ò-lók-{\ì}}  &`fisherman'\\
        \relax [\ipa{a, e}]&\ipa{k\`{ʊ}-tát-à} &`do sorcery'	&\ipa{è-tét-{\ì}}  &`sorcerer/ess'\\
        \relax [\ipa{ɛ, e}]&\ipa{k\`{ʊ}-f\'{ɛ}ːf-\`{ɛ}} &`watch'	&\ipa{è-féːf-{\ì}}  &`sentry'\\
    \end{tabular}
    \z
\end{example}

As shown by the first line in each of  (\ref{Yangben-root-allomorphs-causative-moved}a) and (\ref{Yangben-root-allomorphs-causative-moved}b), when the root in the infinitive is advanced, such a root appears unchanged in the causative and agentive forms (e.g., \ipa{[kù-sùk]} `miss, stop' vs.\ \ipa{[kù-súk-{\ì}]} `cause to stop'). In contrast, when the root is retracted in the infinitive, it appears as advanced in both the causative and the agentive (e.g., \ipa{[k\`{ʊ}-pàl]} `uproot' vs.\ \ipa{[kù-pèl-{\ì}]} `cause to uproot'). %\ipa{kù-só:k-ò} `grow' vs.\ \ipa{kù-só:k-ì} `germinate').
 The motivation for the consistently advanced forms seen in the causative and agentive is straightforward. The morphs for causative and agentive are nonalternating and advanced: \{\ipa{{\ì}}\}\down{\sc causative} and \{\ipa{{\ì}}\}\down{\sc agentive}; when followed by such suffixes, only the [atr] morph satisfies the tongue root phonotactics\is{phonotactics!Yangben} (Chapter \ref{ch2}, (\ref{Yangben-atr/rtr-phonotactic})).

There are three types of morph sets observed in these forms. First, we have the familiar prefix sets involving [atr] and [rtr] pairs,  \{kù, \ipa{k\`{ʊ}}\}\down{\sc infinitive}. Second, we have comparable [atr]/[rtr] morph sets involving  roots, e.g.\ \{pàl, pèl\}\down{\sc uproot}.  %\{\ipa{sò:k, só:k}\}\down{\sc grow}). 
Third, we have instances where morph sets include a single member, roots such as \{sùk\}\down{\sc miss, stop}\footnote{We abstract away from the tonal alternations seen in some cases.} and suffixes such as \{\ipa{{\ì}}\}\down{\sc causative} and \{\ipa{{\ì}}\}\down{\sc agentive}.\footnote{It is not accidental that the nonalternating forms are [atr]; see \textsection\ref{section_MSR_introduction}.}

In the words involving a nonalternating suffix, both root and prefix forms are unambiguously determined by the tongue root phonotactics.\is{phonotactics!Yangben} Consider a case such as  [\ipa{kù-pèl-{\ì}}] `cause to uproot' ({\sc infinitive-uproot-causative}):%\ipa{kù-só:k-ì}  `germinate' ({\sc infinitive-grow-cause}):

\begin{example} \et{Assessment for [\ipa{kù-pèl-{\ì}}]\down{\sc infinitive-uproot-causative}}
    \label{Yangben-uproot-cause}

    {\it morph sets}: \{\ipa{kù, k\`{ʊ}}\}\down{\sc infinitive}; \{\ipa{pàl, pèl}\}\down{\sc uproot}; \{\ipa{{\ì}}\}\down{\sc causative}

    \begin{center}
    \renewcommand*{\arraystretch}{1.2}
    \begin{tabular}{lll | c : c}
    \hline
    \hline
    \multicolumn{3}{c|}{{\sc infinitive}-{\sc uproot}-{\sc causative}}	&\latr &\ratr				 \\
    \hline
    &a. &\ipa{k\`{ʊ}-pàl-{\ì}}	&*!&		 	  	\\
    \hline
    &b. &\ipa{k\`{ʊ}-pèl-{\ì}}	&*!&	 	\\
    \hline
    &c. &\ipa{kù-pàl-{\ì}}	&*!&*!	 	\\
    \hline
    \rightthumbsup
    &d. &\ipa{kù-pèl-{\ì}}	&&	 	\\
    \hline \hline
    \end{tabular}
    \end{center}
\end{example}

There is only one morph in the causative morph set  and it is [atr],  while all other morph sets have both advanced and retracted morphs. The tongue root phonotactics are satisfied only by selection of advanced morphs throughout. 

Consider, however, the result of combining the infinitive and root morph sets without the causative, that is, combining \{\ipa{kù, k\`{ʊ}}\}\down{\sc infinitive} and \{\ipa{pàl, pèl}\}\down{\sc uproot}. In this case, there are two possible compilations that would respect the tongue root phonotactics.

\begin{example} \et{Preliminary assessment for [\ipa{k\`{ʊ}-pàl}]\down{\sc infinitive-uproot}}

{\it morph sets}: \{\ipa{kù, k\`{ʊ}}\}\down{\sc infinitive}; \{\ipa{pàl, pèl}\}\down{\sc uproot} \label{Yangben-uproot}

\begin{center}
\renewcommand*{\arraystretch}{1.2}
\begin{tabular}{lll | c : c}
\hline
\hline
\multicolumn{3}{c|}{{\sc infinitive}-{\sc uproot}}	&\latr &\ratr				 \\
\hline
?&a. &\ipa{k\`{ʊ}-pàl}	&&		 	  	\\
\hline
&b. &\ipa{k\`{ʊ}-pèl}	&*!&	 	\\
\hline
&c. &\ipa{kù-pàl}	&&*!	 	\\
\hline
?
&d. &\ipa{kù-pèl}	&&	 	\\
\hline \hline
\end{tabular}
\end{center}
\end{example}


Given an absence of relevant phonotactics, we would expect the speaker to select the most commonly observed morph -- in essence, the prediction of exemplar theory\is{Exemplar Theory} (\citealt{Bybee:2001, Pierrehumbert:2001ed, Pierrehumbert:2003PhoneticDiversity, Johnson:2007, Wedel:2007, vandeWeijer:2012}). This can be straightforwardly formalised by imposing a penalty on any morph in a morph set that is not the most frequently\is{frequency!penalty} occurring one, a lexically-based generalisation.\is{generalisation!lexical}\is{morph!frequency}\is{morph!penalty}

\begin{example} \et{Penalty on less frequent\is{frequency} morphs}\label{morph-penalty}\\
\begin{tabularx}{\linewidth}{@{}lQ@{}}
*\{morph\down{β}\}, &Assign a violation to each  morph\down{β} which is not the most frequently occurring morph in its morph set
\end{tabularx}
\end{example}

\begin{dadpbox}{Frequency}{box-frequency}
\is{frequency}

The issues relating to \textit{frequency} are multiple and complex; we only touch on them here. The frequency effect in (\ref{morph-penalty}) might be better expressed as, ``given a choice between two morphs from the same morph set, choose the one that is more frequent''. This would mean that if the most frequent morph was ruled out for some reason, there would be a preference for the next most frequent morph. We adopt the simple statement in (\ref{morph-penalty}) since it is adequate for the cases we consider. \\

See \citet{Yang:2016} for a review and for modeling of ranked frequency in language acquisition\is{acquisition!frequency} and word recognition. How learners deal with patterns of varying frequency is important, whether they reproduce frequency distributions\is{distribution!frequency} observed in the data or whether they impose regularity, creating a grammar\is{grammar!frequency} that is consistent even in the face of inconsistent data (\citealt{HudsonKam+:2005, HudsonKam+:2009}).
\end{dadpbox}

Consistent with the overall framework we are proposing, phonological generalisations\is{generalisation!phonological} will be distilled out of sets of such lexically-based generalisations.\is{generalisation!lexical} For example,  in the Yangben case, where root alternations involve retracted forms in all instances other than when followed by the agentive and causative suffixes, there will be a recurrence of prohibitions of the less frequent forms: *\{pèl\}\down{\sc uproot}, *\{\ipa{sóːk}\}\down{\sc grow}, *\{\ipa{j{\ì}k}\}\down{\sc boil}, etc. The penalised morphs share a recurrent property and so are regularised to a phonological condition: they are all advanced. From the Yangben lexically-specific conditions, a general prohibition on a particular type can be extracted, prohibiting [atr].\footnote{The discussion here reframes the concept of \textit{default}\is{default} introduced in our earlier work on Emergence, e.g.\ \citet{Archangeli+:2016mm, Archangeli+:2018routledge} and builds on the proposal of {\it priority}\is{priority} in \citet{Mascaro:2007, Bonet+:2007}, and, by generalising,\is{generalisation} goes beyond the role of \textit{frequency} used in \citet{vandeWeijer:2012}. The concept here is quite different from the role for  ``default''\is{default} in    underspecification\is{underspecification} theory (\citealt{Archangeli:1984, Archangeli:1988, Pulleyblank:1986}).\label{page-chapter_default} Another use of ``default''\is{default!Construction Grammar}\is{Construction Grammar!default} is found in  Construction Grammar phonology for constructions which place no requirements on the form of the \is{morphology!Construction Grammar}\is{Construction Grammar!morphology}morpheme being used (\citealt[142--143]{Valimaa-Blum:2011}).}\is{tongue root!Yangben phonotactic}

\begin{example} \et{Yangben tongue root phonotactic}\is{phonotactics!Yangben}\label{Yangben_tr_default}\smallskip\\
\begin{tabular}{@{}ll@{}}
*[atr], \tier: vowels, \dom: word \is{word!domain}	\\
With a focus on vowels, assign a violation to  a word for 
 each advanced\\ vowel.

\end{tabular}
\end{example}

Given (\ref{Yangben_tr_default}), an [rtr] form will be chosen from a morph set unless some other well-formedness condition\is{well-formedness condition} overrides that choice.\is{ranking!Yangben} In Yangben, the tongue root phonotactic\is{tongue root!Yangben phonotactic} conditions do just that. Note that the harmonic conditions\is{harmony!Yangben conditions} must outrank *[atr] to prevent a form like *[\ipa{k\`{ʊ}-pàl-{\ì}}] from surfacing in (\ref{Yangben-uproot-cause}). (The dashed and solid lines in (\ref{Yangben-uproot-fixed}) show unranked and ranked well-formedness conditions respectively;  see box \ref{box:box-read-assessment-table} on p.\ \pageref{box:box-read-assessment-table}.)

\begin{example} \et{Final assessment for [\ipa{k\`{ʊ}-pàl}]\down{\sc infinitive-uproot}}
\label{Yangben-uproot-fixed}

{\it morph sets}: \{\ipa{kù, k\`{ʊ}}\}\down{\sc infinitive}; \{\ipa{pàl, pèl}\}\down{\sc uproot}

\begin{center}
\renewcommand*{\arraystretch}{1.2}

\begin{tabular}{lll | c : c  | c }
\hline
\hline
\multicolumn{3}{c|}{{\sc infinitive}-{\sc uproot}}	&\latr &\ratr	&*[atr]			 \\
\hline
\rightthumbsup&a. &\ipa{k\`{ʊ}-pàl}	&&	&	 	  	\\
\hline
&b. &\ipa{k\`{ʊ}-pèl}	&*!&	 &*	\\
\hline
&c. &\ipa{kù-pàl}	&&*!	 &*	\\
\hline

&d. &\ipa{kù-pèl}	&&	&*!* 	\\
\hline \hline
\end{tabular}
\end{center}
\end{example}


\subsection{Schemas for well-formedness conditions} \label{subsection-paradigmatic-schemas}
The prohibition *[atr] is a condition penalising a {\it type} of representation.\footnote{When specific well-formedness conditions are presented, as in (\ref{Yangben_tr_default}), we specify the relevant focus (\tier)\is{focus \tier} and domain (\dom);\is{domain \dom} see box \ref{box:box-domains} on page \pageref{box:box-domains}. The schemas presented in this section show the conventions we use to formulate the three different kinds of prohibitions independently of focus\is{focus \tier} and domain.\is{domain \dom}}\is{well-formedness condition!type, formalised}

\begin{example} \et{Type condition schema: [X] is a property} \label{types-schema-original}\smallskip\\
    \begin{tabularx}{\linewidth}{@{}lQ@{}}
             *[X] 	& Assign a violation to a form for each [X], where [X] may be either morphological or phonological.\\
    \end{tabularx}
\end{example}

\is{morphology}(Morphological properties include both labels on morphological sets as well as edges\is{edge!domain} of morphological categories.\is{category} See discussion of (\ref{phonotactics-schema-prelim}) on phonological properties.)

In any representation containing at least one instance of [X], one violation is assessed\is{assessment!violations} for every instance of [X]. As seen in \textsection\ref{section_defaults}, the element [X] can be of (at least) two types: (i) [X] can be a morph (\ref{morph-penalty}); (ii) [X] can be a featural property (\ref{Yangben_tr_default}). That is, [X] may be morphological or phonological. The type condition\is{well-formedness condition!Yangben} in (\ref{Yangben_tr_default}) is an example where a phonological property is penalised, *[atr] in Yangben. 

Using morphological units in (\ref{types-schema-original}) leads to considering \is{morphology}morphological units as an option in the syntagmatic schema, introduced in (\ref{phonotactics-schema-prelim}) and revised here.\is{well-formedness condition!syntagmatic, formalised}

\begin{example} \et{Syntagmatic schema} \label{phonotactics-schema-original}\smallskip\\
    \begin{tabularx}{\linewidth}{@{}lQ@{}}
        *[X][Y] & Assign a violation to a form for each  sequence of [X] followed by [Y], where [X], [Y] may be either morphological or phonological.\\
    \end{tabularx}
\end{example}

We have seen cases where [X] and [Y] are phonologically defined in (\ref{Yangben-atr/rtr-phonotactic})  and (\ref{Yangben-final-voiceless}) in Chapter \ref{ch2} and have seen an example of the role they play in \textsection\ref{phonotactic-selection-section-revised}. In \textsection\ref{section-Yangben-morpho-phonotactics} we  discuss a Yangben case where one of the factors in a syntagmatic condition\is{well-formedness condition!syntagmatic, complex} is morphological.

\begin{dadpbox}{Formally unlimited syntagmatic well-formedness conditions}{box-unlimited-syntagmatics}
\is{well-formedness condition!syntagmatic, formalised}

We formulate syntagmatic prohibitions as involving two elements. Clearly this is the simplest type of syntagmatic prohibition, and therefore, we assume, the easiest to learn and consequently the most common. We do not rule out more complex well-formedness conditions,\is{well-formedness condition!syntagmatic, complex} e.g.\ ones involving a sequence\is{sequence!phonotactic} of three elements, though we consider that they would be harder to learn\is{acquisition!well-formedness condition} and therefore less common due to their added complexity. The basic schema in (\ref{phonotactics-schema-original}) would simply be extended to account for such cases; an example is found in our discussion in Chapter \ref{ch5} of %both Kinande\il{Kinande} (\ref{K-tone_cond}b) and 
Polish\il{Polish} (\ref{Polish_*VC}).
\end{dadpbox}

The third and final schema is for paradigmatic conditions, well-formedness conditions that prohibit overlapping properties.%\is{paradigmatic condition!formalised}

\begin{example} \et{Paradigmatic schema} \label{paradigmatic-schema-original}\smallskip\\
    \begin{tabularx}{\linewidth}{@{}lQ@{}}
        \multirow{2}{*}{{*}$\begin{bmatrix}\textrm{X}\\\textrm{Y}\end{bmatrix}$} & Assign a violation to a form for each combination of [X] and [Y], where [X], [Y] may be either morphological or phonological.\\
    \end{tabularx}
\end{example}

We illustrate a paradigmatic condition\is{well-formedness condition!paradigmatic} involving phonological properties, in \textsection\ref{section-paradigmatic-condition-example}.

\subsection{Paradigmatic featural prohibition}\label{section-paradigmatic-condition-example}

One of our central contentions is that human learners are highly sensitive to skewed distributions\is{distribution!acquisition} and that, where such distributions\is{distribution} involve language, the learner encodes them into a grammar.\is{grammar} Consider again the vowel inventory\is{inventory!Yangben} given in Chapter \ref{ch2}, \tabref{Yangben-vowels}, repeated here.

\begin{table} 
\caption{Yangben vowels\label{Yangben-vowels-repeat}}
\begin{tabular}{lcccccc}
\lsptoprule
            &\multicolumn{2}{c}{front}	&central	&\multicolumn{2}{c}{back/round}\\\midrule
advanced	&\ipa{{\í}/{\í}ː/{\ì}/{\ì}ː}	&é/\ipa{éː}/è/\ipa{èː}	&	&ó/\ipa{óː}/ò/\ipa{òː}	&ú/\ipa{úː}/ù/\ipa{ùː}\\
retracted	& \'{ɪ}/\'{ɪ}ː/\`{ɪ}/\`{ɪ}ː	&\ipa{\'{ɛ}/\'{ɛ}ː}/\ipa{ \`{ɛ}/\`{ɛ}ː} 	&~ ~ á/\ipa{áː}/à/\ipa{àː} ~ ~	&\ipa{\'{ɔ}/\'{ɔ}ː}/\ipa{\`{ɔ}/\`{ɔ}ː} &\ipa{\'{ʊ}/\'{ʊ}ː}/\ipa{\`{ʊ}/\`{ʊ}ː}\\
\lspbottomrule
\end{tabular}
\end{table}

By the kinds of considerations discussed in \textsection\ref{Yangben_categories_section} and \textsection\ref{intro-partitions-sec}, these vowels can be categorised\is{category} into long and short vowels, high-toned and low-toned vowels, front and back vowels, advanced and retracted vowels, and so on. While some of these properties are quite symmetrical, there is a marked skewing in terms of the way tongue root advancement/retraction treats vowels of different heights. While high and mid vowels all exhibit pairs of [atr] and [rtr] vowels, low vowels are consistently [rtr] -- there are no low advanced vowels. To encode this unexpected gap\is{gap} -- an extreme instance of a skewed distribution\is{distribution!gap} -- a paradigmatic condition\is{well-formedness condition!paradigmatic} prohibiting the combination of [low] and  [atr]  is motivated, a paradigmatic condition of the sort given in (\ref{paradigmatic-schema-original}).


\begin{example} \et{Tongue root redundancy in Yangben low vowels}\label{Yangben-low-atr}
\begin{tabularx}{\linewidth}{@{}lQ@{}}
\Ytr, \tier: segments, \dom: morph, word \\
With a focus on segments, assign a violation to a morph or word for each\\ low advanced segment. \is{morph!domain}\is{word!domain}

\end{tabularx}
\end{example}

This well-formedness condition\is{well-formedness condition!Yangben} holds of words in Yangben, so is very easy for the learner to identify. It also  governs well-formed lexical entries. It reflects a close to 0\% probability of encountering an advanced low vowel, a 0\% frequency\is{frequency} of occurrence -- an extreme skewing in terms of feature combinations. In essence, it means that the intersection of the [atr] partition\is{partition} and the [low] partition is null. As a condition on the well-formedness  of morphs,\is{well-formedness condition!morph} this condition means that  a putative morph combining [atr] and [low] would be ill-formed in Yangben, but that morphs with nonlow vowels, advanced or retracted, are well-formed as are morphs whose vowels are all retracted. 


\subsection{Syntagmatic conditions combining morphology and phonology}\label{section-Yangben-morpho-phonotactics}\label{section-morpho-phonotactics}\il{Yangben|(}\is{harmony!Yangben|(}

In many instances, skewed distributions\is{distribution} depend on a combination of morphological and phonological factors,\is{morpho-phonotactics|(} cases where X and Y in the schemas in (\ref{phonotactics-schema-original}) involve a mixture of \is{morphology}morphology and phonology. A commonly attested case of this type is one where specific morphs or morph sets require particular phonological properties in the phonological strings found before or after them.\footnote{While different in execution, the discussion of interactions of \is{morphology}morphology and phonology owes a debt to \citet{Ford+:1983}. This section develops and modifies the concept of \textit{selection}, introduced in our earlier work, e.g.\ \citet{Archangeli+:2018routledge}.} Our proposals here build on ``subcategorisation''\is{subcategorisation} models argued for in work such as \citet{Lieber:1980, Yu:2007-infixation, Paster:2009, Paster:2015}; we differ from such work in folding {\it subcategorisation}\is{subcategorisation} into a general class of sequential\is{sequence!phonotactic} prohibitions, where the units that legitimately occur in a sequence happen to involve both phonological and morphological elements.

In keeping with our use of a single language for illustration throughout this introduction, we continue with an illustration from Yangben.  We first show that Yangben has a further harmonic pattern, one of round harmony, motivating syntagmatic conditions governing sequences\is{sequence!phonotactic} involving round vowels. We then turn to instances where a morph surfaces that is not  the one preferred by the general harmonic conditions. Such lexically-conditioned cases require morpho-phonotactics, conditions which supersede the more general requirements of harmony.

Consider the words that involve Class 3 and Class 6 prefixes in \tabref{tab:Yangben-nonhigh-class-prefixes-revised}.\footnote{We consider a subpart of the overall pattern here, setting aside forms that involve low-vowel variants of the prefixes, e.g., \ipa{[àm-bàŋ-\'{ɔ}]} `{\sc c3}-crying' (\citealt[160]{Boyd:2015}). The forms with low vowels require an additional morph in the relevant morph sets, with a phonotactic\is{phonotactics!Yangben} requiring agreement in lowness. These effects are not problematic but are orthogonal to the issue of morpho-phonotactics under discussion.}

\begin{table} 
\caption{Class prefixes (\citealt[173--175]{Boyd:2015} except as noted)\label{tab:Yangben-nonhigh-class-prefixes-revised}}
\begin{tabular}{lllll}
\lsptoprule
Vowels  &  \multicolumn{2}{l}{Class 3}	                   &\multicolumn{2}{l}{Class 6}\\\midrule
%{[a]}  &   %\ipa{à-s\v{a}:}	&`river'                       & %\ipa{mà-nùN}	&`blood'\\
{[ɛ]}   &   \ipa{\`{ɛ}m-b\`{ɛ}s\`{ɛ}}	&`maize'           & \ipa{m\`{ɛ}-p\'{ɛ}ːn\`{ɛ}}	&`milk' \\%163, 175
{[e]}	&   \ipa{è-mèkú}	&`flesh, muscle'               & \ipa{mè-kút}	&`fat, oil'\\%170, 173
{[ɔ]}   &   \ipa{\`{ɔ}-\`{ɔ}p\`{ɪ}}	&`green mamba' p.\ 179 & \ipa{m\`{ɔ}-f\`{ɔ}\up{ɱ}f\`{ɛ}}	&`marrow'\\%173
{[o]}	&   \ipa{ò-ŋòl{\í}}	&`vine (generic)' p.\ 161      & \ipa{mò-ɲ\v{o}ː}	&`cemetery'\\%173
\lspbottomrule
\end{tabular}
\end{table}

Two harmonic properties are important in \tabref{tab:Yangben-nonhigh-class-prefixes-revised}: (i) the tongue root value of the prefix vowel matches the tongue root value of the root vowel; (ii) the prefix has a rounded back vowel if the first vowel of the root is a mid rounded back vowel; otherwise the prefix vowel is mid, front, and nonround. 

The observation concerning tongue root agreement falls out from the   phonotactics already discussed, \latr\ and \ratr.\is{tongue root}\is{phonotactics!Yangben} The observation concerning the rounded vs.\ nonround forms requires two properties. First, we assume that all else being equal, the preferred morphs for Class 3\footnote{Class 3 prefixes include morphs with and without a final nasal consonant; we do not address this pattern here.} and Class 6 are nonround;  we attribute this to a type penalty on rounded vowels, *[round],  formalised in (\ref{Yangben-nonround-default}) (following the schema in (\ref{types-schema-original})). Second, we  posit a syntagmatic condition\is{well-formedness condition!Yangben} governing round harmony (\ref{Yangben-round-phonotactic}).


\begin{example} \et{Class 3 \& Class 6} \label{c3c6-morph-sets}\smallskip\\
\begin{tabularx}{\linewidth}{@{}lQ@{}}
Morph sets: & \{\ipa{è, \`{ɛ}, ò, \`{ɔ}}\}\down{\sc class.3} \\
            & \{\ipa{mè, m\`{ɛ}, mò, m\`{ɔ}}\}\down{\sc class.6}
\end{tabularx}
\ex \et{Nonround condition} \label{Yangben-nonround-default}\smallskip\\
\relax {*[round]}, \tier: vowels, \dom: word\\\is{word!domain}
With a focus on vowels, assign a violation to a word for each round vowel. \is{morph!domain}

\ex \et{Round phonotactic}\is{phonotactics!Yangben}\label{Yangben-round-phonotactic}\smallskip\\
\rd, \tier: vowels, \dom: morph, word\\
With a focus on vowels, assign a violation to a morph or a word for each sequence of a mid unrounded vowel followed by a mid rounded vowel.
\end{example}

Two Class 6 examples with [round] stem vowels illustrate how these phonotactics\is{phonotactics!Yangben} interact, shown in the assessments in (\ref{Yangben-c6-fat}) and (\ref{Yangben-c6-cemetery}).\footnote{In both cases, the conditions governing tongue root harmony also play a role; since only \latr\ is relevant with an [atr] root vowel, \ratr\ is omitted for expositional simplicity.} In (\ref{Yangben-c6-fat}), the vowel is high so the round phonotactic (\ref{Yangben-round-phonotactic}) is irrelevant. The choice between rounded and nonround options falls to *[round] (\ref{Yangben-nonround-default}).

\begin{dadpbox}{Acquisition and type conditions}{box-*F_condition_acquisition}

%\is{acquisition!type condition}
\is{well-formedness condition!acquisition}


There are different considerations with regard to how type conditions\is{well-formedness condition!type} like *[round] might be learned, and exactly what is learned.\is{acquisition!well-formedness condition}\\

First,  it is possible that *[round] might be acquired along the lines of the acquisition of *[atr],  (\ref{morph-penalty}). A learner might identify the need for *[round] with respect to a specific morph set first, then later do the same for another morph set,\is{acquisition!morph set} leading to a generalisation over generalisations\is{generalisation} because multiple morph sets share the same restriction. However, there are only a few instances in Yangben\il{Yangben} where *[round] plays a deciding role -- only with the small number of morph sets containing both [round] and [nonround] morphs, while *[atr] is significant for virtually every verb stem.  And whether or not the learner extends this morph-based generalisation more broadly to a generic *[round], there would be no impact on functionality of the grammar and communication (though it might result in differences in, e.g., nonce word studies). \\ 

Turning  to frequency,\is{frequency}\is{acquisition!frequency}\is{frequency} we note that in terms of vowel types, there are more nonround vowels in the inventory\is{inventory!frequency} than rounded vowels. In addition, although we do not have the numbers, we think it is likely that token frequency would show a similar skewing, perhaps even more dramatically. For example, low vowels -- a member of the nonround class -- are very frequent in a number of Bantu languages (\citealt{Archangeli+:2012Smith}).\\

Hence, while acknowledging that our frequency\is{frequency} motivation for the condition\is{well-formedness condition!type} given in (\ref{Yangben-nonround-default}) is speculative at this point, we think  it  a plausible hypothesis for Yangben.\il{Yangben}
\end{dadpbox}

\begin{example}\renewcommand*{\arraystretch}{1.2}%
\et{Assessment for [\ipa{mè-kùt}]\down{\sc class.6-fat, oil}}\label{Yangben-c6-fat}\\
{\it morph sets}: \{\ipa{mè, m\`{ɛ}, mò, m\`{ɔ}}\}\down{\sc class.6}; \{\ipa{kùt}\}\down{\sc fat, oil}\smallskip\\
\begin{tabular}{lll | c : c | c  | c  }
\hline
\hline
\multicolumn{3}{c|}{\sc C6-fat, oil}	&\latr &\rd &*[round]&*[atr]\\
\hline


\rightthumbsup
&a. &\ipa{mè-kùt}	&	&	&* 	&**	  	\\
\hline
&b. &\ipa{m\`{ɛ}-kùt}			&*!   &	&* 	&* 	\\
\hline
&c. &\ipa{mò-kùt}			&	&	&**! 	&**  	\\
\hline
&d. &\ipa{m\`{ɔ}-kùt}			&*! 	& 	&**	&* 	\\
\hline
\hline 
\end{tabular}
\end{example}

In the second case, (\ref{Yangben-c6-cemetery}), the stem vowel is mid, making the round phonotactic\is{phonotactics!Yangben} critical in deciding which morph to choose from the morph compilation: both (\ref{Yangben-c6-cemetery}a,  b) are eliminated due to the round disharmony.\is{disharmony}\is{harmony!Yangben|)}

\begin{example} \et{Assessment for [\ipa{mò-ɲòː}]\down{\sc class.6-cemetery}}
\label{Yangben-c6-cemetery}

{\it morph sets}: \{\ipa{mè, m\`{ɛ}, mò, m\`{ɔ}}\}\down{\sc class.6} \{\ipa{ɲòː}\}\down{\sc cemetery}

\begin{center}
\renewcommand*{\arraystretch}{1.2}
\begin{tabular}{lll | c : c | c  : c  }
\hline
\hline
\multicolumn{3}{c|}{\sc C6-cemetery}	&\latr &\rd &*[round]&*[atr]	 \\
\hline

&a. &\ipa{mè-ɲòː}		&	&*!	&*	&** 	  	\\
\hline
&b. &\ipa{m\`{ɛ}-ɲòː}	&*!  &*!	&*	&* 	\\
\hline
\rightthumbsup
&c. &\ipa{mò-ɲòː}		&	&	&**	&**	  	\\
\hline
&d. &\ipa{m\`{ɔ}-ɲòː}	&*! 	& 	&**	&* 	\\
\hline
\hline 
\end{tabular}
\end{center}
\end{example}

So far, these examples constitute additional illustrations of type conditions in conjunction with syntagmatic conditions. However, a sporadic set of stems shows an interesting interaction between phonology and \is{morphology}morphology; examples are in (\ref{Yangben-atypical-cl-prefixes}). In these cases, certain stems appear with an unexpected form of the class prefix, either [round] where [nonround] is expected or [nonround] where [round] is expected. For each example, the unattested (but expected) forms are given in the rightmost column.\footnote{Where patterns are not completely regular, we have assumed that there is a threshold determining whether a pattern is analysed as a generalisation\is{generalisation} with exceptions or simply as a list of exceptions (\citealt{Archangeli+:2012McGill, Archangeli+:2012Smith}), perhaps along the lines of the Tolerance Principle\is{Tolerance Principle} (\citealt{Yang:2016}). Thanks to \name{Beau}{Peterson} and
\name{Ryan}{Smith}
for help in understanding the math used in the Tolerance Principle.}\il{Yangben}
 
\begin{example} \et{Atypical prefix-root pairings in Yangben (\citealt[172]{Boyd:2015})}\label{Yangben-atypical-cl-prefixes}
\ea\relax [è] is expected but does not appear\\
\begin{tabular}[t]{@{}lll@{}}
	 attested      &                       &expected\\
	\ipa{òn-dé}	&`{\sc c3}-grass.sp.'	&*\ipa{èn-dé}	\\%&172\\
\end{tabular}

\ex\relax [\ipa{\`{ɛ}}] is expected but does not appear\\
\begin{tabular}[t]{@{}lll@{}}
attested      &                       &expected\\
%\ipa{mà-té}	&`{\sc cl.6}-tree'	&*\ipa{mè-tè}\\%&171\\lowness, not rounding
\ipa{\`{ɔ}-k\'{ɛ}l}	&`{\sc c3}-mountain'	&*\ipa{\`{ɛ}-k\'{ɛ}l}	\\%&172\\
\ipa{\`{ɔ}-k\`{ɛ}n}	&`{\sc c3}-tail'	&*\ipa{\`{ɛ}-k\`{ɛ}n}	\\%&172\\
\ipa{\`{ɔ}-m\`{ɪ}ⁿd\'{ɛ}}	&`{\sc c3}-fence'	&*\ipa{\`{ɛ}-m\`{ɪ}ⁿd\'{ɛ}}	\\%&172\\
\ipa{\`{ɔ}m-b\'{ɛ}l}	&`{\sc c3}-hole'	&*\ipa{\`{ɛ}m-b\'{ɛ}l}	\\%&172\\
\end{tabular}
\ex\relax [\ipa{\`{ɔ}}] is expected but does not appear\\
\begin{tabular}[t]{@{}lll@{}}
attested      &                       &expected\\
\ipa{\`{ɛ}-t\`{ɔ}}	&`{\sc c3}-head'	&*\ipa{\`{ɔ}-t\`{ɔ}}	\\%&172\\
\ipa{\`{ɛ}-s\v{ɔ}}	&`{\sc c3}-penis'	&*\ipa{\`{ɔ}-s\v{ɔ}}	\\%&172\\
\end{tabular}
\z
\end{example}

In each of these cases, the only way to know that an atypical prefix morph is required is to hear and learn the correct form. Such cases represent a skewing in the data that is parallel to the skewings that motivate phonotactics,\is{phonotactics} but in this kind of case, we are dealing with a skewing that is specific to a particular \is{morphology}morphological environment -- in fact, in this particular case, the skewing is specific to particular  morphs. For example, given the morph set for Class 3, \{\ipa{è, \`{ɛ}, ò, \`{ɔ}}\}\down{\sc class.3} (\ref{c3c6-morph-sets}), a lexical item like {\sc head} must assign a penalty to the prefix form that ought to be selected. Similar penalties would be required for all of the  atypical stems shown in (\ref{Yangben-atypical-cl-prefixes}).

\begin{example} \et{Morph-conditioned penalties}\\ \label{morph-conditioned-pentalties_Yangben}
\begin{tabular}{@{}lll@{}}
%a.  &*[nonround]C\down{$\emptyset$}\{\ipa{dé}\}\down{\sc grass.sp.}; \dom: word\\
a.  &*[nonround] \{\ipa{dé}\}\down{\sc grass.sp.}; &\tier: vowels; \dom: word\\\is{word!domain}
&*[nonround] \{\ipa{k\'{ɛ}l}\}\down{\sc mountain}; &\tier: vowels; \dom: word\\
&*[nonround] \{\ipa{k\`{ɛ}n}\}\down{\sc tail}; &\tier: vowels; \dom: word\\
&*[nonround] \{\ipa{m\`{ɪ}ⁿd\'{ɛ}}\}\down{\sc fence}; &\tier: vowels; \dom: word\\
&*[nonround] \{\ipa{b\'{ɛ}l}\}\down{\sc hole}; &\tier: vowels; \dom: word\ee

b. &*[round] \{\ipa{t\`{ɔ}}\}\down{\sc head}; &\tier: vowels; \dom: word\\
 &*[round] \{\ipa{s\v{ɔ}}\}\down{\sc penis}; &\tier: vowels; \dom: word\\
\end{tabular}
\end{example}

Formally, these instances of lexically-imposed penalties constitute instances of the syntagmatic schema in (\ref{phonotactics-schema-original}) where the first element is phonological and the second is \is{morphology}morphological,  expressed in terms of a morphological set of stems in (\ref{Yangben-exception-conditions}).\footnote{Since the set of conditions penalising [nonround] has [nonround] vowels (\ref{morph-conditioned-pentalties_Yangben}a) while the set penalising [round] has [round] vowels (\ref{morph-conditioned-pentalties_Yangben}b), it might be possible to achieve further generalisation\is{generalisation!lexical}, uniting these two penalties as a single prohibition (still lexically conditioned): *[α round] \{α round\}\down{β}, where \{...\}\down{β} $\in$ \{{\sc grass.sp., mountain, head, penis,} ...\}; \tier: vowels; \dom: word. This further generalisation requires treating [round] and [nonround] formally as a binary feature and then invoking SPE-style variable reference (\citealt{Chomsky+:1968}). Since we do not need reference to binarity\is{binarity} elsewhere, and since this case involves a small number of lexically specific exceptions, we do not introduce binarity here.}\largerpage[2]

\begin{example} \et{Lexically conditioned penalties}\\\label{Yangben-exception-conditions}
\begin{tabular}{@{}lp{10cm}@{}}
a.  &*[nonround] \{~~~\}\down{α}, where \{~~~\}\down{α} $\in$ \{{\sc grass.sp., mountain,} ...\}\\& \tier: vowels; \dom:~word\\
b.  &*[round] \{~~~\}\down{β}, where \{~~~\}\down{β} $\in$ \{{\sc head, penis,} ...\} \\&\tier: vowels; \dom: word\\\is{word!domain}
\end{tabular}
\end{example}

Since the class of stems prohibiting a nonround prefix is phonologically and \is{morphology}morphologically arbitrary, and similarly for the class prohibiting round prefixes, these penalties can only be expressed in terms of  arbitrary sets, which we label here with α\ and β.

\begin{dadpbox}{Phonologically and morphologically arbitrary classes}{box-alpha-classes}\is{class!arbitrary}

Arbitrary \is{set!arbitrary, definition}sets are identified by distribution\is{distribution!set identification}\is{set!evidence for sets} alone: there is no independent unifying phonological or morphological property. Once the set is identified, its label is available for reference, comparable at this point to a phonological label like [high] or [voiceless] or a morphological label like {\sc verb} or {\sc class.3}.\\

In previous work (e.g.\ \citealt{Archangeli+:2015_Frontiers, Archangeli+:2015_K-tone, Archangeli+:2018routledge}), we  labelled these sorts of sets as {\it selectors}; using that terminology, these morphs  would be called the  ``[round] selectors'' ([rd]\uline{~~~}) and ``[nonround] selectors'' ([nonrd]\uline{~~~}) respectively.
\end{dadpbox}


\sloppy{Assessment of the morph compilation proceeds as shown in (\ref{Yangben-c3-head}), where the morph-specific conditions (morpho-phonotactics) must outrank\is{ranking!Yangben} the round phonotactic,\is{phonotactics!Yangben} {*}[nonrd, mid] [rd, mid]. % \begin{center}\rd.\end{center} 
The morpho-phonotactic\is{morpho-phonotactics|)} is violated in (\ref{Yangben-c3-head}c,d), since the morph \{\ipa{t\`{ɔ}}\}\down{\sc head.{\textbeta}}, a member of the arbitrary class β, is preceded by a [round] vowel, not a [nonround] vowel.\il{Yangben} }


\begin{example} \et{Assessment for [\ipa{\`{ɛ}-t\`{ɔ}}\down{{\textbeta}}]\down{\sc class.3-head}}
\label{Yangben-c3-head}

{\it morph sets}: \{\ipa{è, \`{ɛ}, ò, \`{ɔ}}\}\down{\sc class.3}; \{\ipa{t\`{ɔ}}\}\down{\sc head}\down{{\textbeta}}


\renewcommand*{\arraystretch}{1.2}
\setlength{\tabcolsep}{4pt}
\begin{tabular}{lll | c |c : c | c  : c }
\hline
\hline
\multicolumn{3}{c|}{\sc c3-head\down{β}	}&{*[rd]\{...\}\down{{\textbeta}}}&\ratr &\rd &*[round] &*[atr]\\
\hline
&a. &\ipa{è-t\`{ɔ}}\down{{\textbeta}}	&	&*!	&* 	  &* &*	\\
\hline
\rightthumbsup
&b. &\ipa{\`{ɛ}-t\`{ɔ}}\down{{\textbeta}}				&	&  	&*	&* 	\\
\hline
&c. &\ipa{ò-t\`{ɔ}}\down{{\textbeta}}				&*!	&*	&	&** &*	  	\\
\hline
&d. &\ipa{\`{ɔ}-t\`{ɔ}}\down{{\textbeta}}				&*! 	& 	&	&** 	\\
\hline
\hline 
\end{tabular}
\end{example}\il{Yangben}


\subsection{Summary} 


In this section, we have addressed the issue of how to select\is{assessment} among multiple members of a morph compilation.\is{compilation!selection} In general, the most frequently\is{frequency!assessment}\is{assessment!frequency} occurring morph would be expected to be used (the formal result of {\it type} conditions\is{well-formedness condition!type} penalising less frequent morphs (\ref{types-schema-original})) except in instances where such selection would violate some condition. We have presented the three kinds of conditions, {\it type} (\ref{types-schema-original}),\is{well-formedness condition!type} {\it syntagmatic}\is{well-formedness condition!syntagmatic} (\ref{phonotactics-schema-original}), and {\it paradigmatic}\is{well-formedness condition!paradigmatic} (\ref{paradigmatic-schema-original}), that we assume throughout this work, illustrated in this chapter with examples from Yangben. These conditions may make reference to phonologically and morphologically defined sets, the former constituting generalisations\is{generalisation} over the latter. The learner's high sensitivity to skewings\is{distribution!acquisition} in the data encountered means that learning may either begin with the identification of purely phonological asymmetries -- pure phonotactics\is{phonotactics} -- or it may begin by noting certain morph-specific properties, encoding them, and then generalising to sets exhibiting comparable behaviour.

\section{Expanding morph sets: Morph Set Relations \& Conditions}\label{Yangben-MSRs-section}
\label{Yangben-tr-alternations}
\is{Morph Set Relation}\is{well-formedness condition}



When words are created by compiling morph sets with multiple members, the resulting morph compilations\is{compilation} are assessed\is{assessment} to identify the preferred form in accordance with the conditions relevant for the language. Such word-level compilations are created by the addition of derivational or inflectional \is{morphology}morphology, by compounding, by \is{reduplication}reduplication, and so on (\citealt{Aronoff:1976, Kiparsky:1982lexical-phonology, Lieber:1992}, among many, many others); these word-formation operations are one source of productivity\is{productivity!lexicon|(} in the lexicon. 

In this section, we address a second source of productivity in the lexicon, one that is widely recognised as part of the grammar,\is{grammar!productivity|(} but that is often conceptualised in a way that excludes it from the domain of lexical productivity: productivity that results when  the phonological system creates previously unencountered morphs. In the framework we are presenting here, word formation in the sense of affixation, compounding, and so on, occurs when morph sets are compiled; we have little to say about such word-formation here though we assume some appropriate mechanism throughout.\is{compilation} The second type is the result of mechanisms that expand morph set membership. It is frequently the case -- and a hallmark of paradigms -- that the existence of one morph implies the existence of another\is{morph!productivity} -- phonologically-related -- morph. In such cases, the lexicon exhibits productivity, so that the learner may generate a large class of lexical items after exposure to a small subset of relevant forms.\is{acquisition!productivity} Such productivity in the lexicon is the topic of this section.\is{productivity!morph}



Consider, for example, the Yangben morph sets introduced above: many of these morph sets contain two members, differing only by tongue root values, [atr] and [rtr]. For example, the infinitive prefix has two forms: \{\ipa{k\`{ʊ}, kù}\}\down{\sc infinitive} (\ref{Yangben-infinitives}), and verb roots may alternate in their tongue root values, seen by comparing plain infinitives with causatives and agentives: \{\ipa{f\'{ʊ}l, fúl}\}\down{\sc flow} (\ref{Yangben-root-allomorphs-causative-moved}). Are such alternations evidence of lexical productivity? In the case of some affixes, there may be no particular reason for productively generating alternative morphs since in many cases the relevant sets are both finite and small. We could plausibly assume that the learner simply incorporates morphs into sets as they are encountered. With verbs, on the other hand, such an assumption would be highly implausible. Verbs constitute a large, potentially unlimited set. It is extremely unlikely that the learner will encounter all of the relevant forms of the verbal lexemes that are learned -- some generative\is{generative capacity} predictive power is necessary in the grammar\is{grammar} (cf. \citealt{Ford+:1983}). We account for the observed regularities with \textit{Morph Set Relations}\is{Morph Set Relation} -- which define the systematic relations between related morphs -- and \textit{Morph Set Conditions} -- which identify sub-optimal morph sets. Together, Morph Set Relations and Morph Set Conditions\is{Morph Set Condition} bring about productivity in the appropriate cases. In Yangben, then, both prefixes and verb roots show evidence of the same Morph Set Relation; this relation is necessarily productive in verb roots, perhaps productive in affixes as well.\is{grammar!productivity|)}\is{productivity!lexicon|)}

\largerpage
\begin{dadpbox}{Top-down learning}{box-topdown-learning}
\is{top-down}\is{acquisition}

When the morph compilation\is{compilation} contains more than one member, there is \is{uncertainty!resolution}uncertainty: which is the appropriate form to use in a given case? We hypothesise that this uncertainty is commonly resolved in favour of the most frequent\is{frequency} morph. \\

In cases where there are multiple morphs in a set and the learner observes that the morph used is {\it not} the most frequently occurring morph, selection based on frequency would result in an error. Hence the \is{uncertainty}uncertainty motivates seeking out a resolution, a case of top-down\is{top-down} learning. At this point, the learning can be quite focussed: the properties of the competing morph combinations direct the learner's attention\is{attention} to the types of conditions that might be relevant.\\

Similarly, the existence of a syntagmatic condition\is{well-formedness condition!syntagmatic} can lead the learner to expect multiple members of a morph set\is{productivity!morph set} differing by properties in that condition,\is{acquisition!morph set} guiding the identification of relevant Morph Set Relations.\is{Morph Set Relation}
\end{dadpbox}

\subsection{Morph Set Relations (MSRs)} \label{section_MSR_introduction}


\is{Morph Set Relation|(}
We begin our discussion by identifying an issue that arises when the learner establishes a morph set.\is{acquisition!morph set} In general, learners show evidence of a principle of {\it contrast}\is{contrast} (\citealt{Clark:1987}) or {\it mutual exclusivity}\is{mutual exclusivity} (\citealt{Markman+:1988}): an object has a single label.\footnote{Thanks to
\name{Janet F.}{Werker} for pointing us in the direction of this literature.} Hence when introduced to a novel word, the learner assumes that the word refers to something not already labelled. This general principle has been formulated in a number of ways with somewhat different implications, accounting for a range of effects discussed in a large literature (\citealt{Slobin:1973, Wexler+:1980, Pinker:1984, Clark:1987, Markman+:1988, Markman:1989, Markman:1992, Musolino:1999, Markman+:2003}, {\it inter alia}).\footnote{In the literature on morphology, \citet{Clark:1987} suggests the effect  derives ``blocking'',\is{morphology!blocking} where the presence of one form {\it blocks} the formation of a morphologically related form that would have the same meaning\is{meaning} (see, for example, \citealt{Aronoff:1976, Kiparsky:1982lexical-phonology}).} Important with respect to Emergence, there is suggestive work indicating that the effect may not be specific to word\is{word} learning (\citealt{Markman+:1988, Markman:1989, Markson+:1997, Childers+:2003, Moher+:2010, Orena+:underreview}) or even perhaps to humans (\citealt{Kaminski+:2004, Markman+:2004, Fischer+:2004}).\is{nonhuman species}
 



\is{meaning!mutual exclusivity|(}The willingness of a learner to postulate multiple morphs in a minimal morph set,\is{acquisition!morph set}\is{morph set!minimal} \{X, Y, ...\}\down{α},\footnote{For the definition of minimal morph sets, see box \ref{box:box-minimal-morph-sets} on p.\ \pageref{box:box-minimal-morph-sets}.} seems to directly contradict this principle, whether or not a learner exhibits a stage where different meanings are attributed to the distinct morphs \{X\}\down{α-i},  \{Y\}\down{α-j} that ultimately are grouped into a single set \{X, Y\}\down{α}.   Yet there is evidence suggesting that  systematicity among morphs competing for the same meaning -- i.e.\ in the same minimal morph set\is{morph set!minimal} -- facilitates an override of mutual exclusivity.\is{mutual exclusivity}

Work such as \citet{Byers-Heinlein+:2009, Kandhadai+:2017} presents evidence that bilingual\is{bilingualism} children do not exhibit mutual exclusivity the way that monolingual children do; \citet{Clark:1987} speculates that such behaviour might emerge when bilingual\is{bilingualism} children begin to systematically distinguish phonologically between the languages they are exposed to. As both \citet{Markman+:1988} and \citet{Markman:1989} observe, it is not that mutual exclusivity\is{mutual exclusivity} cannot be overridden, it is rather that the learner is biased towards respecting it and will only override the principle when presented with sufficient evidence.\is{acquisition!identity}  The evidence cited above suggests that systematic phonological differences between two labels for one object is sufficient evidence, such as the systematic differences that arise due to  learning distinct phonological systems. For our purposes, we hypothesise that the evidence of systematic phonological relations between morphs  constitutes sufficient evidence for the child to override mutual exclusivity\is{mutual exclusivity} in such cases, allowing the learner to assign multiple morphs to a single minimal morph set.\is{morph set!minimal}

{
\begin{example} \et{The Systematicity hypothesis} \label{systematicity-hypothesis}

Systematic phonological differences between morphs with the same label constitute sufficient evidence to override mutual exclusivity.\is{Systematicity Hypothesis}
\end{example}
}


By hypothesis, systematicity is the key to exempting related sets of morphs from the  mutual exclusivity\is{mutual exclusivity} requirement,  the  general expectation  that morphs in a morph set will be identical except for those systematic divergences, formalised here as the \Identity\ (\ref{identity-assumption}).\footnote{A corollary is that, while \is{idiosyncrasy}idiosyncratic differences are penalised, in the event that there is  considerable positive evidence, minimal morph sets\is{morph set!minimal} may contain distinct and disparate morphs,  as is the case with, for example, the English\il{English} morph sets for {\sc have} \{hæv, hæz, hæd\}\down{\sc have}, and for {\sc be} and  \{\ipa{æm, ɪz, ɑ\ur, w\uv z, wə\ur, biː, bɪn}\}\down{\sc be}.}

\begin{example} \et{The Identity Principle} \label{identity-assumption}\is{Identity Principle!definition}\\
Morphs in a minimal set are identical except in systematically identified ways. 
\end{example}

Let us see how \textit{Systematicity} and \textit{Identity} play out in acquiring Yangben.\is{acquisition!identity} Consider the impact of learning forms like those in (\ref{Yangben-root-allomorphs-causative}). 

\begin{example} \et{Yangben root patterns (\citealt[177]{Boyd:2015})}\\ \label{Yangben-root-allomorphs-causative}
\ea nonalternating with respect to tongue root values\smallskip\\
\begin{tabular}{@{}l@{~~}l@{~~}l@{~~}l@{~~}l@{}}
 \{ép, ěp\}\down{\sc steal} &\ipa{kʷ-ěp-è} &`steal'&\ipa{èŋ-ép-{\ì}}  &`robber'\\
 \{sùk, súk\}\down{\sc miss, stop} &\ipa{kù-sùk} &`miss, stop'	&\ipa{kù-súk-{\ì}}  &`cause to stop'\\
\end{tabular}
\ex alternating with respect to tongue root values\smallskip\\
\begin{tabular}{@{}l@{~~}l@{~~}l@{~~}l@{~~}l@{}}
 \{\ipa{f\'{ɛ}ːf, féːf}\}\down{\sc watch}&\ipa{k\`{ʊ}-f\'{ɛ}ːf-\`{ɛ}} &`watch'	&\ipa{è-féːf-{\ì}}  &`sentry' \\
 \{\ipa{f\'{ʊ}l, fúl}\}\down{\sc flow} &\ipa{k\`{ʊ}-f\'{ʊ}l-à} &`flow'	&\ipa{kù-fúl-{\ì}}  &`cause to flow'
\end{tabular}
\z
\end{example}

There are a variety of observations to encode, even when abstracting away from tonal alternations (as we do here, though we continue to mark tone and encode tonally distinct morphs as separate morphs in a morph set). One observation of importance is the existence of prefixes and suffixes, allowing the learner to isolate verb stems as well. In some cases, such as \{ép, ěp\}\down{\sc steal} and  \{sùk, súk\}\down{\sc miss, stop}, tongue root values are consistent. The morphological decomposition also reveals that meanings such as {\sc watch} can be associated with two different sound strings differing solely by tongue root features, [\ipa{f\'{ɛ}ːf}] and [\ipa{féːf}], just as {\sc flow} is associated with both [\ipa{f\'{ʊ}l}] and [\ipa{fúl}]. When sufficient such pairs are identified, a systematic relation emerges between morphs with retracted vowels and those with advanced vowels. Were there to be no systematic relation between the pairs of morphs with the same meanings, putting a pair into a single morph set would be in direct contradiction to mutual exclusivity\is{mutual exclusivity} and its derivative, the \Identity.\is{Identity Principle}



To the extent that such pairs are  systematically related, the learner must override mutual exclusivity.\is{acquisition!identity} There is such evidence in Yangben.

In the adult language, morph sets are of three types:
(i) both noun and verb morph sets where all morphs are [atr], \tabref{Yangben_3-min-sets}a;
(ii) noun morph sets where all morphs are [rtr], \tabref{Yangben_3-min-sets}b; and
(iii) verb morph sets where morphs are related, in [atr]$\sim$[rtr] pairs, \tabref{Yangben_3-min-sets}c. It is type (iii), the verb morph sets with two members, that is our focus.\footnote{The verbs in \tabref{Yangben_3-min-sets} are from (\ref{Yangben-root-allomorphs-causative}); the full forms of the nouns in \tabref{Yangben_3-min-sets} are: [\ipa{k{\ì}-tèŋé}] `waterhole'; [\ipa{k{\ì}-kújè}] `plant sp., fan'; [\ipa{{\ì}-nòn{\í}}] `bird'; [\ipa{k\`{ɪ}-p\'{ɛ}ːs\`{ɛ}}] `twins'; [\ipa{\`{ɪ}-k\'{ɔ}t\'{ɔ}}] `pipe'; [\ipa{\`{ɛ}-t\`{ɛ}n\'{ʊ}}] `shame' (\citealt[165]{Boyd:2015}). Since morph sets, not words, are represented in \tabref{Yangben_3-min-sets}, the nouns have been stripped of their class prefixes.}

\begin{table}
\caption{Three types of minimal morph sets in Yangben\label{Yangben_3-min-sets}} 
\begin{tabular}{ *{5}{l} }
\lsptoprule
	&\multicolumn{2}{l}{Nouns}  &  \multicolumn{2}{l}{Verbs}\\\midrule
a.	& \multicolumn{4}{l}{[atr] morphs}\\
	&\{tèŋé\} 	&`waterhole'	&\{ép, ěp\}	&`steal' \\%[kì-tè\Né] 	&`waterhole'	\\ 165 -- 177
	&\{kújè\}	&`plant sp., fan'&\{sùk, súk\}&`miss, stop'\\%[kì-kújè] 		&`plant sp., fan'\\  165 -- 177
	&\{\ipa{nòn{\í}}\} 	&`bird'\ee%[ì-nòní] 		&`bird'\ee

b.	&\multicolumn{4}{l}{[rtr] morphs}\\
	&\{\ipa{p\'{ɛ}ːs\`{ɛ}}\} &`twins'	 &---\\%[\ipa{kì-péːsè}] &`twins'\\
	&\{\ipa{k\'{ɔ}t\'{ɔ}}\} &`pipe'		\\%[\ipa{ì-kótó}] &`pipe'\\
	&\{\ipa{t\`{ɛ}n\'{ʊ}}\} &`shame'\ee
c. &\multicolumn{4}{l}{[atr] and [rtr] morphs}\\
	&--	&	&\{\ipa{f\'{ɛ}ːf, féːf}\} &`watch'\\
	&	&	&\{\ipa{f\'{ʊ}l, fúl}\} &`flow'\\\lspbottomrule
\end{tabular}
\end{table}

In the earliest stages of learning,\is{acquisition!morph set}\is{acquisition!early} we hypothesise that the learner posits single-morph sets, even for morphs like those in \tabref{Yangben_3-min-sets}, in accordance with mutual exclusivity.\is{mutual exclusivity} However, as learning progresses, evidence begins to accrue that, alongside the morph sets with only [atr] vowels and those with only [rtr] vowels, there is sufficient evidence with some verbs to posit minimal morph sets with both [atr] and [rtr] morphs.  Of course, learners do not directly encounter morph sets with multiple members: the words\is{word} actually encountered by the learner will have either one form or the other. To put two distinct morphs together into one morph set requires sufficient evidence that the two forms have the same meaning and function.\is{acquisition!morph set}\is{meaning!mutual exclusivity|)} Importantly for our point here, the resistance brought by mutual exclusivity\is{mutual exclusivity} decreases as the  similarity\is{similarity!morphs} between the two morphs increases. In the case of Yangben, the related morphs under discussion are completely identical except for the tongue root position.

Since learners are rampant generalisers,\is{generalisation!acquisition}  as learners identify two distinct morphs as members of the same morph set, they identify patterns (here, that the morphs are identical except for tongue root).\is{acquisition!morph set} As additional such sets are acquired,\is{acquisition!morph} the same pattern emerges repeatedly, leading to the generalisation\is{generalisation!lexical} that if a morph set has two morphs in it, the morphs are identical except for tongue root position. In the Yangben case, such sets have a second similarity: the pattern is found  in verb stems, not noun stems. 

Given the nature of the Yangben data, this is a robust generalisation, leading the  learner to posit a relation along the lines of (\ref{Yangben-rtr-root-MSR}).\is{acquisition!Morph Set Relation} Such relations are to be interpreted non-exhaustively: the claim is that morphs of the type described are attested, not that such pairs are necessarily the only morphs encountered within the relevant set. For example, there might be morphs with different tones, and so on.

\begin{whiteshadowbox}
{
\begin{example} \et{Yangben Morph Set Relation\down{\sc [tr] }} \label{Yangben-rtr-root-MSR}\is{Morph Set Relation!Yangben} \ee

{In a minimal verb morph set, there is a systematic relation between morphs with nonlow advanced vowels and morphs with retracted vowels}.

\begin{tabular}{lp{5in}}
~\\
{\it examples}	&\{\ipa{s\'{ɔ}ːk, sóːk}\}\down{\sc grow, verb}\\
&\{\ipa{pàl, pèl}\}\down{\sc uproot, verb}\ee
\end{tabular}


\begin{tabular}{llll}
{MSR\down{\sc [tr]}}:&\{$\mathcal{M}$\down{\it i}, $\mathcal{M}$\down{\it j}\}\down{\sc verb} &$\mathcal{M}$\down{\it i}: [atr] $\land$ *$\begin{bmatrix}\textrm{atr}\\\textrm{low}\end{bmatrix}$ &\ee
&&$\mathcal{M}$\down{\it j}: [rtr] \ee
\end{tabular}

\end{example} }

 \end{whiteshadowbox}
 
For completeness, the Yangben MSR\down{\sc [tr]}\is{Morph Set Relation!Yangben} includes a further criterion, that of respecting the paradigmatic prohibition on combining [atr] and [low] (\ref{Yangben-low-atr}) because of pairs like  \{\ipa{pàl, pèl}\}\down{\sc uproot} and \{\ipa{tát, tét}\}\down{\sc do.sorcery}: while there are no low advanced vowels, low vowels do have advanced counterparts -- which are nonlow.\footnote{{In a language like Yangben (where *[atr, low] (\ref{Yangben-low-atr}) is part of the grammar), were there no conjunction with this condition, then we assume that morphs with low vowels would have no advanced counterpart. This is the case in languages like \il{Fula}Fula (\citealt{Paradis:1992}) and \il{Yor\`ub\'a}\Y\ (\citealt{Archangeli+:1989}), whose vowel inventories\is{inventory!vowel} are similar to that of Yangben, and  which have  morph sets with advanced and retracted pairs -- except for morph sets with low vowels because (i) there are no low advanced vowels in these languages and (ii) there is no other advanced counterpart for the low vowels (unlike in Yangben).}} 



MSRs characterise properties of minimal morph sets\is{morph set!minimal} with more than one member. The relation between two such morphs is not only as defined in an MSR, but also as restricted by the \Identity.\is{Identity Principle} Given the \Identity, the assumption is that morphs in a set are identical; the MSR characterises only the differing properties of the related morphs, both the stipulated differences and differences that hold when specific phonotactics are invoked. 


The final point to be addressed is the role of MSRs in expanding  the learner's inventory of morph sets, that is, whether the relation characterised by an MSR is productive.\is{productivity!morph set}

\begin{dadpbox}{Morph Set Relation format}{box-MSR-formalism}\is{Morph Set Relation}


In general, we formulate a Morph Set Relation (MSR) as in (\ref{MSR-schema}). In expressing such relations, we assume the \Identity\ (\ref{identity-assumption}).\is{Identity Principle} Hence morphs are assumed to be identical to each other in all aspects not either (a) explicitly designated in the expressed relation, or (b) the result of explicitly conjoined well-formedness conditions.\is{well-formedness condition!conjoined} \\

\begin{whiteshadowbox}
{\begin{example} \et{Morph Set Relation}\label{MSR-schema}
\is{Morph Set Relation!formalised}

In a minimal morph set, there is a systematic relation between {morphs} with α\  (subject to $\mathcal{C}$\down{\it m}) and morphs with β\  (subject to $\mathcal{C}$\down{\it n}).

\begin{tabular}{lll}
~\\
{MSR}:&\{$\mathcal{M}$\down{\it i}, $\mathcal{M}$\down{\it j}\} &$\mathcal{M}$\down{\it i}: α\ ($\land$ $\mathcal{C}$\down{\it m})\\
&&$\mathcal{M}$\down{\it j}: β\ ($\land$ $\mathcal{C}$\down{\it n})
\end{tabular}

\end{example} }
 \end{whiteshadowbox}



For expository convenience, we label specific MSRs\is{Morph Set Relation!labelling} -- for example, MSR\down{\sc [tr]} in (\ref{Yangben-rtr-root-MSR}), ``{\sc [tr]}'' for ``tongue root''.\is{tongue root!Yangben Morph Set Relation} To assist in interpreting the formal statements, when giving a specific MSR for a language, we include examples illustrating the relation, as seen in (\ref{Yangben-rtr-root-MSR}).\\

A Morph Set Relation simply constitutes the expression of a relation holding between the observed morphs within a morph set. If an MSR does not define a counterpart for some segment, our assumption is that an applicable Morph Set Condition (MSC) (see \textsection\ref{section_MSCs}) would simply fail to produce an expanded set. For example, if the tongue root MSR simply related [rtr] and [atr] vowels, without reference to respecting *[atr, low] then the MSC\is{Morph Set Condition} would have no effect on a morph set containing a low vowel morph: there is no segment that only differs from a low retracted vowel (in Yangben) solely with reference to tongue root values. We crucially do {\it not} assume that the expansion of morph sets has the effect of creating unobserved segment types.\\

In many instances, there may be multiple such relations holding; for example, a morph set could exhibit independent regularities concerning both tongue root values and tone. If there is some systematic relation between the morphs within a morph set, then there is a Morph Set Relation. Whether such relations result in the augmentation of a morph set -- that is, whether they productively result in new forms -- depends on Morph Set Conditions, discussed in \textsection\ref{section_MSCs}.\is{Morph Set Relation!formalised}

\end{dadpbox}\is{Morph Set Relation|)}
 

\subsection{Morph Set Conditions (MSCs) and productivity} \label{section_MSCs}\is{Morph Set Condition|(}

\is{grammar!productivity|(}Having recognised that morphs of different shapes can have the same meaning\is{meaning!mutual exclusivity} -- giving rise to morph sets -- the learner can form generalisations about these derived structures: generalisations over generalisations.\is{generalisation} We propose therefore that productivity\is{productivity!lexicon} is triggered by a skewed distribution\is{distribution!acquisition}\is{distribution!productivity} favouring morph sets whose members are consistent with some MSR\is{Morph Set Relation}; if morph sets with a morph of type α\ typically also have a morph of type β, then a morph set with both types of morphs is favoured over a morph set with only one type. Such skewed distribution\is{distribution!morph set} leads to conditions about the structure of morph sets: the rare or non-occurring type of morph set is identified as ill-formed, or sub-optimal. A learner, on acquiring a morph\is{acquisition!morph set} which alone constitutes an ill-formed morph set, regains equilibrium in the grammar by generating the missing morph(s) in accordance with both the MSR and the \Identity;\is{Identity Principle} the result is an augmented, now well-formed, morph set.

Consider how this plays out in the case of Yangben harmony.\is{harmony!Yangben acquisition|(} As learning progresses beyond the point of identifying Yangben's MSR\down{\sc [tr]},\is{Morph Set Relation!Yangben} the learner  discovers  that, among verb roots, the number of verb morph sets with singleton [rtr] morphs diminishes as observation adds an [atr] morph to such sets. Consequently, there is a preponderance of only two types of sets, not three, along the tongue root dimension; those with [atr] morphs and those with corresponding [atr] and [rtr] morphs, expressed in (\ref{Yangben-root-generalisations}). The skewed distribution of the (well-formed) [atr]-only sets and [atr]-and-[rtr] sets vs.\ the (ill-formed) [rtr]-only sets becomes increasingly apparent as additional verbs are encountered.

\begin{example} \et{Generalisations about verb root patterns} \\ \label{Yangben-root-generalisations}
\ea \textit{nonalternating roots}: consistently [atr]
\ex \textit{alternating roots}: correspondents systematically related by MSR\down{\sc [tr]}\is{alternation!morph set}
\z
\end{example}

Relating these observations to Yangben's MSR\down{\sc [tr]},\is{Morph Set Relation!Yangben} the learner finds morph sets  with only advanced morphs,  \{$\mathcal{M}$\down{[atr]}\}, and morph sets with corresponding advanced and retracted morphs, \{$\mathcal{M}$\down{[atr]}, $\mathcal{M}$\down{[rtr]}\}.\is{acquisition!morph set} On the other hand, morph sets of the type \{$\mathcal{M}$\down{[rtr]}\}, while possible for nouns, are rare if not non-existent for verbs. This gap,\is{gap!Yangben Morph Set Condition} we propose, is codified in the Yangben grammar by a Morph Set Condition, or MSC. The Yangben MSC\down{\sc [tr]}\is{Morph Set Condition!Yangben} penalises morph sets that contain an [rtr] morph but no [atr] morph, $\mathcal{M}$\down{\it j} and $\mathcal{M}$\down{\it i} respectively in (\ref{Yangben-rtr-root-MSR}). 

On acquiring a new [rtr] morph, the logical thing for a learner to do is to posit a morph set \{$\mathcal{M}$\down{[rtr]}\} -- a set that is perfectly well-formed in the nominal system, but that is ill-formed if it is a verb, due to  (\ref{Yangben-rtr-root-MSC}): a \{$\mathcal{M}$\down{[rtr]}\}  verb morph set  is  ill-formed. This  creates tension in the grammar.\is{grammar} The learner has essentially two choices to rectify the situation, either (i) disregard the evidence of having heard the retracted morph, or (ii) accept the evidence and posit that the morph set contains another morph.\is{acquisition!morph set} The latter strategy is intrinsically limited:  the \Identity\ \is{Identity Principle}ensures that the rectifying morph is minimally different from the observed one, while the related MSR precisely defines that minimal difference -- in the Yangben case, the result is the emergence of a corresponding morph with an advanced, nonlow vowel, conforming to MSR\down{\sc [tr]}.\is{Morph Set Relation!Yangben}\is{harmony!Yangben acquisition|)}

\begin{whiteshadowbox}
{%
\begin{example} \et{Yangben Morph Set Condition\down{\sc [tr]} (MSC\down{\sc [tr]})} \ee \label{Yangben-rtr-root-MSC}

{With respect to MSR}\down{\sc [tr]}{, a minimal morph set is ill-formed if  there is a morph with a retracted vowel and there is no corresponding morph with a nonlow advanced vowel.}

\begin{tabular}{llp{4in}}
~\\
{\it examples}	&{\it observed}&{\it repaired}\\ 
&*\{\ipa{l\'{ɔ}k}\}\down{\sc fish; verb} &\{\ipa{l\'{ɔ}k, lók}\}\down{\sc fish; verb}\\
&*\{\ipa{tát}\}\down{\sc do.sorcery; verb} &\{\ipa{tát, tét}\}\down{\sc do.sorcery; verb}\ee
\end{tabular}
~\ee

\begin{tabular}{ll}
~&\\
{MSC}\down{\sc [tr]} &For $\mathcal{M}$\down{\it i}, $\mathcal{M}$\down{\it j} of MSR\down{[TR]}, *\{$\mathcal{M}$\down{\it j}, $\neg$$\mathcal{M}$\down{\it i}\} \ee
~&\\
Concrete expression:&{*\{...[rtr]..., $\neg$...$\begin{bmatrix}\textrm{atr}\\\textrm{nonlow}\end{bmatrix}$...\}}\ee
Schematic examples:&{*\{...ɛ..., $\neg$...e...\}, *\{...\U..., $\neg$...u...\}, etc.}
\end{tabular}

\end{example}}
\end{whiteshadowbox}\is{Morph Set Condition!Yangben}
 
The * in the \textit{observed} column means that a morph set is ill-formed, not that the morph itself is unattested.\largerpage[-1]
 
\begin{dadpbox}{Interpreting a Morph Set Condition}{box-MSC-Yangben-tr}
\is{Morph Set Condition!interpretation}

$\mathcal{M}$\down{\it i}, $\mathcal{M}$\down{\it j} in a Morph Set Condition are interpreted in terms of the paired Morph Set Relation\is{Morph Set Relation}; the indices on morphs in the MSC correspond to the requirements identified in the MSR.  
For expository reference, we label a specific MSC with a label identical to that of the corresponding MSR. Thus, MSC\down{\sc [tr]} corresponds to MSR\down{\sc [tr]}. {As an expository device, we also provide a concrete expression of the MSC in terms of the relevant categories (here features), and schematic examples.}
\end{dadpbox}



By provoking morph set expansion, MSCs are key to productivity:\is{productivity!Morph Set Condition} MSCs define certain morph sets as ill-formed with respect to a particular MSR; ill-formed morph sets posited on the basis of observed forms are repaired to satisfy MSC(s) in accordance with the relevant MSR(s) and the \Identity.\is{Identity Principle} The critical grammatical element is the Morph Set Condition; productivity is the consequence. \is{grammar!productivity|)}


Before moving on, we turn briefly to a general formalisation of MSCs. Since each MSC is interpreted with respect to an MSR,\is{Morph Set Relation} we begin there. Schematically, as established in \Sec\ref{section_MSR_introduction}, an MSR is motivated by observing a systematic relation between morphs within a morph set \{$\mathcal{M}$\down{\it i}, $\mathcal{M}$\down{\it j}\}. There are four logically possible types of languages with a particular MSR, as shown in \tabref{schematic-morph-set-types}, depending on whether singleton morph sets, \{$\mathcal{M}$\down{\it i}\} or \{$\mathcal{M}$\down{\it j}\}, occur in conjunction with a directly observed polymorph set containing \{$\mathcal{M}$\down{\it i}, $\mathcal{M}$\down{\it j}\} (the existence of which motivates the MSR in question).\largerpage[-1]


\begin{table} 
\caption{Well-formed minimal morph sets including \{$\mathcal{M}$\down{\it i}, $\mathcal{M}$\down{\it j}\}\label{schematic-morph-set-types}}


\begin{tabular}{l@{ }lccc}
\lsptoprule
& &\{$\mathcal{M}$\down{\it i}\} &\{$\mathcal{M}$\down{\it j}\} &\{$\mathcal{M}$\down{\it i}, $\mathcal{M}$\down{\it j}\}\\\midrule
a.&neither singleton set is well-formed&--&--&observed\\
b.&one singleton set is well-formed&observed&--&observed\\
c.&the other singleton set is well-formed&--&observed&observed \\
d.&both singleton sets are well-formed&observed&observed&observed\\
\lspbottomrule
\end{tabular}
\end{table}

While the MSR\is{Morph Set Relation} is simply a statement of a relation between $\mathcal{M}$\down{\it i} and $\mathcal{M}$\down{\it j}, the relevance of each MSR in a language varies depending on which singleton sets are -- or are not -- common. {An MSC\is{Morph Set Condition}  requires  a particular  morph to have a correspondent morph, as defined by some MSR. Requiring that $\mathcal{M}$\down{\it i} have a correspondent is consistent with \tabref{schematic-morph-set-types}a,c, where \{$\mathcal{M}$\down{\it i}\} is rare or unattested; requiring that $\mathcal{M}$\down{\it j} have a correspondent is consistent with \tabref{schematic-morph-set-types}a,b, where \{$\mathcal{M}$\down{\it j}\} is rare or unattested. Put together, three types of MSC effects are derived: {\it no singletons} (\tabref{schematic-morph-set-types}a) is the result of requiring that both $\mathcal{M}$\down{\it i} and $\mathcal{M}$\down{\it j} have correspondent morphs (as defined by the relevant MSR); {\it one singleton} (\tabref{schematic-morph-set-types}b,c) is the result of  requiring that either $\mathcal{M}$\down{\it i} or $\mathcal{M}$\down{\it j} have a correspondent, but not the other. The fourth case, {\it two singletons} (\tabref{schematic-morph-set-types}d), is the result of no related MSC, so there is no requirement that  the morph set contain corresponding morphs (as defined by the relevant MSR).} Thus, formally speaking there are two possible versions of a given MSC because the underrepresentation of \{$\mathcal{M}$\down{\it i}\} and of \{$\mathcal{M}$\down{\it j}\} can be independent of each other. %\footnote{
These possibilities are all attested. The symmetric type where no singletons are allowed, \tabref{schematic-morph-set-types}a, is found in Warembori (\textsection\ref{section_Warembori}). Yangben, as discussed in this section, is an example of the asymmetric type, \tabref{schematic-morph-set-types}b,c, while Mayak (\textsection\ref{section_Mayak_low_vowels}) and Polish (\textsection\ref{section_Polish}) are examples of the fully differentiated type, \tabref{schematic-morph-set-types}d. 

As presented here, neither Morph Set Relations\is{Morph Set Relation} nor Morph Set Conditions\is{Morph Set Condition} prohibit particular types of morphs. Rather, the MSR characterises relations among morphs within polymorph sets, while the MSC  defines certain morph sets as ill-formed as characterised by the relevant Morph Set Relation. The consequence of having both MSR and MSC in a grammar is that the grammar\is{grammar} is able to automatically generate the morphs necessary to have only well-formed  morph sets.


\begin{dadpbox}{The two Morph Set Conditions}{box-MSC-formalism}


Morph Set Conditions\is{Morph Set Condition!formalised} determine whether the morph sets of a language are ill-formed with respect to a particular Morph Set Relation\is{Morph Set Relation}, MSR\down{γ}. A language may impose no conditions, a single condition, or both conditions.\\

\begin{whiteshadowbox}
\begin{example} \et{Morph Set Conditions} \label{MSC-schema}
\ea With respect to MSR\down{γ}, a minimal morph set is ill-formed if there is an $\mathcal{M}$\down{\it i} and there is no corresponding $\mathcal{M}$\down{\it j}.\smallskip\\ %\bigskip\\
MSC\down{γ}: For $\mathcal{M}$\down{\it i}, $\mathcal{M}$\down{\it j} of MSR\down{γ}, *\{$\mathcal{M}$\down{\it i}, $\neg$$\mathcal{M}$\down{\it j}\} \medskip\\

\ex With respect to MSR\down{γ}, a minimal morph set is ill-formed if there is an $\mathcal{M}$\down{\it j} and there is no corresponding $\mathcal{M}$\down{\it i}.\smallskip\\
MSC\down{γ}: For $\mathcal{M}$\down{\it i}, $\mathcal{M}$\down{\it j} of MSR\down{γ}, *\{$\lnot$$\mathcal{M}$\down{\it i}, $\mathcal{M}$\down{\it j}\}
\z
\end{example}
\end{whiteshadowbox}
 

\begin{whiteshadowbox}
\begin{example} \et{Morph Set Relation}\is{Morph Set Relation!formalised} (repeated from (\ref{MSR-schema})) \ee

{In a minimal morph set, there is a systematic relation between morphs with α\ (subject to $\mathcal{C}$\down{\it m}) and morphs with β\ {(subject to $\mathcal{C}$\down{\it n}).}\smallskip

\begin{tabular}{lll}

{MSR}:&\{$\mathcal{M}$\down{\it i}, $\mathcal{M}$\down{\it j}\} &$\mathcal{M}$\down{\it i}: α\ ($\land$ $\mathcal{C}$\down{\it m})\\
&&$\mathcal{M}$\down{\it j}: β\ ($\land$ $\mathcal{C}$\down{\it n}) \ee
\end{tabular}
}
\end{example} 
 \end{whiteshadowbox}
 

In general, we label MSCs with the same label as the corresponding MSR (MSR\down{γ} and MSC\down{γ}) regardless of which schema(s) are part of the grammar,\is{grammar} (\ref{MSC-schema}a), (\ref{MSC-schema}b), or both. \\

Just as more than one relation (MSR) may hold of a morph set, a morph set may be ill-formed along more than one dimension. For example, a morph set might lack morphs both with respect to tongue root values and with respect to tone. In such instances of multiple relevant MSCs, a morph set would be augmented in all of the appropriate ways.\is{Morph Set Condition!formalised}
 \end{dadpbox}
\is{Morph Set Condition|)}

\subsection{Productivity}\is{productivity!lexicon|(}
The very important role of MSCs in a grammar\is{grammar} is to expand the lexicon. To see how this works, consider the ``toy lexicon''  in (\ref{Yangben-toy-lexicon-1}), simulating a stage in Yangben   acquisition that precedes identification of the  MSR\down{\sc [tr]}. (Boundaries between morphs are left in as a convenience to the reader; the learner may still be figuring some of these out.)  

\begin{example} \et{Pre-MSR Yangben toy lexicon 1: random lexical items (\citealt[177]{Boyd:2015})} \label{Yangben-toy-lexicon-1}\smallskip\\
\begin{tabular}{@{}lllll@{}}
\ipa{kù-sùk} &`{\sc inf}-miss, stop'&\ipa{kù-súk-{\ì}}  &`cause to stop'\\ %177 ... 177
\ipa{k\`{ʊ}-s\'{ɔ}ːk-\`{ɔ}} &`{\sc inf}-grow'&\ipa{kù-fúl-{\ì}}  &`cause to flow'\\% 177 ... 177
\ipa{k\`{ʊ}-k\'{ɛ}t-\`{ɪ}k} &`{\sc inf}-blink'&\ipa{kù-sóːk-{\ì}}  &`germinate'\\% 177 ... 177
\ipa{k\`{ʊ}-pàl}	&`{\sc inf}-uproot' &\ipa{kù-pèl-{\ì}}  &`cause to uproot'\\% 177 ... 177
 \ipa{k\`{ʊ}-j\`{ɪ}k-à} &`{\sc inf}-boil'&\ipa{èŋ-ép-{\ì}}  &`robber'\\% 177 ... 177
\ipa{kʷ-ěp-è}&`{\sc inf}-steal'&\ipa{è-sùl-{\ì}}  &`drinker'\\% 177 ... 177
\ipa{k\`{ʊ}-l\'{ɔ}k-\`{ɔ}} &`{\sc inf}-fish'&\ipa{ò-lók-{\ì}}  &`fisherman'\\% 177 ... 177
\ipa{k\`{ʊ}-tát-à} &`{\sc inf}-do sorcery' &\ipa{è-tét-{\ì}}  &`sorcerer/ess'\\%177 ... 177 .
\ipa{k\`{ʊ}-f\'{ɛ}ːf-\`{ɛ}} &`{\sc inf}-watch'\\ %177
\end{tabular}
\end{example}

The learner identifies semantic\is{sound-meaning correspondence} connections among some of these lexical entries, creating  morph sets with multiple members; we consider only those sets involving roots here.\is{acquisition!morph set} (We include the tonally distinct morphs for \{sùk, {súk}\}\down{\sc miss, stop} and \{ěp, ép\}\down{\sc steal} for completeness; we do not explore the Yangben tonal alternations.)

\newpage
\begin{example} \et{Pre-MSR Yangben toy lexicon 2: preliminary morph sets} \label{Yangben-toy-lexicon-2}

\ea \begin{tabular}[t]{@{}lll@{~~}ll@{~~}l@{}}
\multicolumn{5}{@{}l@{}}{\it Sets with morphs differing by tongue root}\\
\{\ipa{s\'{ɔ}ːk, sóːk}\}&\ipa{k\`{ʊ}-s\'{ɔ}ːk-\`{ɔ}} &`{\sc inf}-grow'&\ipa{kù-sóːk-{\ì}}  &`germinate'\\
\{\ipa{l\'{ɔ}k, lók}\}	&\ipa{k\`{ʊ}-l\'{ɔ}k-\`{ɔ}} &`{\sc inf}-fish'&\ipa{ò-lók-{\ì}}  &`fisherman'\ee
\end{tabular}
\ex \begin{tabular}[t]{@{}lll@{~~}ll@{~~}l@{}}
\multicolumn{5}{@{}l@{}}{\it Sets with only [atr] morphs}\\
\{ěp, ép\}		&\ipa{kʷ-ěp-è}&`{\sc inf}-steal'&\ipa{èŋ-ép-{\ì}}  &`robber'\\
\{sùk, {súk}\}		&\ipa{kù-sùk} &`{\sc inf}-miss, stop'&\ipa{kù-súk-{\ì}}  &`cause to stop'\\
\{\ipa{fúl}\}	&\ipa{kù-fúl-{\ì}}  &`cause to flow'\\
\{sùl\}			&\ipa{è-sùl-{\ì}}  &`drinker'\ee
\end{tabular}
\ex \begin{tabular}[t]{@{}lll@{~~}ll@{~~}l@{}}
\multicolumn{5}{@{}l@{}}{\it Sets with only [rtr] morphs}\\
\{\ipa{k\'{ɛ}t}\}	&\ipa{k\`{ʊ}-k\'{ɛ}t-\`{ɪ}k} &`{\sc inf}-blink'\\
\{\ipa{j\`{ɪ}k}\}	&\ipa{k\`{ʊ}-j\`{ɪ}k-à} &`{\sc inf}-boil'\\
\{\ipa{pàl}\}	&\ipa{k\`{ʊ}-pàl}	&`{\sc inf}-uproot'\\
\{\ipa{f\'{ɛ}ːf}\}	&\ipa{k\`{ʊ}-f\'{ɛ}ːf-\`{ɛ}} &`{\sc inf}-watch'\\
\{tát\}			&\ipa{k\`{ʊ}-tát-à} &`{\sc inf}-do sorcery'\\
\end{tabular}
\z
\end{example}

When enough entries are acquired (\citealt{Gerken+:2008}) to recognise the pattern relating morphs,\is{acquisition!Morph Set Relation} the learner posits the Morph Set Relation\is{Morph Set Relation} involving tongue root values, MSR\down{\sc [tr]}.\is{tongue root!Yangben Morph Set Relation} As morph set acquisition continues, the learner observes that \{[rtr]\} morph sets merge with \{[atr]\} morph sets, but that many of the \{[atr]\} morph sets do not match up with a  corresponding \{[rtr]\}.\is{acquisition!Morph Set Relation} This leads to the generalisation that  \{[rtr]\} -- a morph set with only retracted morphs -- is ill-formed because there is no advanced counterpart to the retracted morph, formalised as MSC\down{\sc [tr]}.\is{Morph Set Condition} As a consequence of this MSR/MSC pair, the learner/speaker is now able to generate some of the missing forms in the lexicon, stage 3.\is{acquisition!productivity}


\begin{example} \et{Post-MSC\down{\sc [tr]} Yangben toy lexicon 3: filling in missing morphs via MSR\down{\sc [tr]} and MSC\down{\sc [tr]}}\label{Yangben-toy-lexicon-3}
\ea \textit{Direct evidence for both [atr] and [rtr] morphs}\smallskip\\
\begin{tabular}[t]{@{}lll@{~~}ll@{~~}l@{}}
\{\ipa{s\'{ɔ}ːk, sóːk}\}&\ipa{k\`{ʊ}-s\'{ɔ}ːk-\`{ɔ}} &`{\sc inf}-grow'&\ipa{kù-sóːk-{\ì}}  &`germinate'\\
\{\ipa{l\'{ɔ}k, lók}\}	&\ipa{k\`{ʊ}-l\'{ɔ}k-\`{ɔ}} &`{\sc inf}-fish'&\ipa{ò-lók-{\ì}}  &`fisherman'\ee
\end{tabular}
\ex \textit{Indirect evidence for [rtr] morphs; MSC\down{\sc [tr]} adds [atr] morph}\smallskip\\
\begin{tabular}[t]{@{}lll@{~~}ll@{~~}l@{}}
\{\ipa{k\'{ɛ}t, két}\}	&\ipa{k\`{ʊ}-k\'{ɛ}t-\`{ɪ}k} &`{\sc inf}-blink'\\
\{\ipa{j\`{ɪ}k, j{\ì}k}\}	&\ipa{k\`{ʊ}-j\`{ɪ}k-à} &`{\sc inf}-boil'\\
\{\ipa{pàl, pèl}\}	&\ipa{k\`{ʊ}-pàl}	&`{\sc inf}-uproot'\\
\{\ipa{f\'{ɛ}ːf, féːf}\}	&\ipa{k\`{ʊ}-f\'{ɛ}ːf-\`{ɛ}} &`{\sc inf}-watch'\\
\{tát, tét\}			&\ipa{k\`{ʊ}-tát-à} &`{\sc inf}-do sorcery'\\
\end{tabular}
\ex \textit{Direct evidence for [atr] morphs (MSC\down{\sc [tr]} does not generate [rtr] morphs)}\smallskip\\
\begin{tabular}[t]{@{}lll@{~~}ll@{~~}l@{}}
\{sùk, súk\}		&\ipa{kù-sùk} &`{\sc inf}-miss, stop'&\ipa{kù-súk-{\ì}}  &`cause to stop'\\
\{\ipa{fúl}\}	&\ipa{kù-fúl-{\ì}}  &`cause to flow'\\
\{sùl\}				&\ipa{è-sùl-{\ì}}  &`drinker'\\
\{ěp, ép\}		&\ipa{kʷ-ěp-è}&`{\sc inf}-steal'&\ipa{èŋ-ép-{\ì}}  &`robber'\\
\end{tabular}
\z
\end{example}

The consequence of having MSRs coupled with MSCs in a grammar\is{grammar} is that the learner is able to hypothesise ``missing'' morphs to fill out incomplete morph sets.\is{acquisition!productivity} As new lexical items are acquired throughout the speaker's life,\is{acquisition!late} MSCs allow the speaker/learner to expand morph sets beyond the forms actually heard: the learner is able to generate other morph set members rather than waiting to actually hear those items.\footnote{Though quite different in their effects, MSCs bear a relation to the productive\is{productivity!Morph Set Condition} power of the structural change of a rule in classic generative phonology\is{generative phonology} (\citealt{Chomsky+:1968, Kenstowicz+:1979}) and the\is{Emergences vs.\ OT} role of  Gen\is{Gen} in Optimality Theory\is{Optimality Theory} (\citealt{Prince+:1993}): each of these mechanisms enables the language user to produce forms that have yet to be heard.} \label{page-section-MSR-vs-rules-or-GEN} At the same time, it is important to realise that the only morph sets that get expanded are those that are deemed ill-formed by the MSC. In Yangben for example, on hearing only a causative or agentive, the learner establishes a morph set that contains an [atr] morph, because an [atr] morph was heard.\is{acquisition!morph set} There is no MSC identifying such morph sets as ill-formed, so no further morph is generated. From an advanced morph alone it cannot be predicted whether there will be a corresponding retracted form or not. For such items, it is necessary to learn some form that does not have an advanced suffix, such as the infinitive,  to determine whether the morph set is complete or whether there is also a retracted morph. \is{productivity!Morph Set Condition}

\largerpage
When multiple morphs exist in a morph set, whether as the result of direct observation or as the the result of an MSC\is{Morph Set Condition} such as (\ref{Yangben-rtr-root-MSC}), a tension exists in the grammar due to \is{uncertainty}uncertainty: once two morphs are posited for a given morph set, when is each morph to be used? Interestingly, MSRs explicitly identify dimensions along which \is{uncertainty!resolution}uncertainty can be resolved. Thus, the learner has the potential to build on existing generalisations\is{generalisation} in a top-down\is{top-down} fashion. In Yangben, MSR\down{\sc [tr]} (\ref{Yangben-rtr-root-MSR}) identifies  systematicity in corresponding [rtr]/[atr] values in verb roots. The tension created here can be resolved by a phonotactic\is{phonotactics!acquisition} that addresses the distribution of tongue root position in words.\is{acquisition!well-formedness condition} (See \Sec\ref{section_sequential-evidence} for discussion of the relevant phonotactics.) However, if the learner were to generalise over multi-member morph sets before identifying the relevant phonotactic, the nature of the MSR\down{\sc [tr]} and MSC\down{\sc [tr]} creates tension specifically around the distribution of tongue root features in words. This tension can only be resolved by a  phonotactic governing the distribution of [atr] and [rtr] morphs in words, thereby giving focus to the learning process.\is{productivity!lexicon|)}


\section{Conclusion}
\largerpage
This extended discussion of some properties of the Yangben\il{Yangben|)}  morphophonological system has served to illustrate different aspects of  phonological knowledge under Emergence. 
We have couched much of our discussion in terms of language acquisition because, in order for a grammar\is{grammar} to exist it must be acquired.\is{acquisition} Only learnable\is{learnability} grammars become adult grammars. In this way, Emergence is learner-focussed. The learner isolates chunks\is{chunk} in the speech stream, committing sequences\is{sequence} to memory where particular sound strings are surmised to have identifiable meanings.\is{meaning} The chunks will be of varying sizes, with minimal chunks \is{morph}({\it morphs}) being combined into larger chunks\is{chunk} including \is{word}{\it words}\is{word} (which we consider) and \is{phrase}{\it phrases} (which we do not). In some cases, more than one morph can be identified for the same meaning,\is{meaning!mutual exclusivity} resulting in \is{morph set}{\it morph sets} with multiple members. 

Morphs\is{morph} and words are subject to a variety of conditions. The key property of a well-formedness condition\is{well-formedness condition} is that it encodes a skewing in the observed data. Initial acquisition\is{acquisition!early} is bottom-up\is{bottom-up} while later stages of acquisition\is{acquisition!late} may also be driven by top-down\is{top-down} considerations, using what is known (conditions on segments and sequences, existing morph sets, Morph Set Relations, and Morph Set Conditions)\is{Morph Set Relation}\is{Morph Set Condition}\is{well-formedness condition} to explicitly guide and focus learning.\is{Morph Set Condition}  The consequence is that complex patterns emerge from the interaction of phonological and lexical generalisations. 

The framework builds on a learner's ability to assess frequency distributions,\is{frequency}\is{distribution!frequency} noting asymmetries in the distribution of phonological elements and of morphological elements. Sequences\is{sequence!acquisition} of phonological and morphological elements that are infrequently encountered are penalised, while the morphological forms observed the most frequently are preferred when all else is equal.  Hence grammar construction is limited by the general cognitive\is{cognition} abilities of the learner to assess \is{frequency}frequency and similarity\is{similarity}, and to generalise.\is{generalisation} Such generalisation results in a symbolic system\is{symbolic system}, a grammar.\is{grammar} 

A central element of the Emergent model proposed here is the \is{morph set}morph set, eradicating the commonly postulated device of a single \is{underlying representation}underlying representation for each lexical item. The next chapter argues that despite underlying representations (inputs) being entrenched in phonological modelling since the structuralists,\is{structuralism} there is no conceptual evidence in support of the concept (Chapter \ref{chapter_URs}). A consequence of this approach, discussed in Chapter \ref{chapter_consequences}, is complete, simple, and concise analyses of data which, if \is{underlying representation}underlying representations are adopted, involve a significant and frequently undesirable degree of abstraction.\is{abstractness}
