\chapter{Use-conditional meaning}\label{sec:11}

\section{Introduction}\label{sec:11.1}

One of the characteristic properties of expressive meaning that we mentioned in \chapref{sec:2} was independence: expressive meaning is “independent” of descriptive meaning. For example, it would not be inconsistent for someone to disagree with the expressive content of an utterance but still agree that the speaker’s statement was true. For this reason, among others, \citet{Potts2007c} speaks of expressive meaning and descriptive meaning as belonging to separate “dimensions” of conventional meaning.


Another example of this kind of independence was noted by \citet{Grice1961}. In discussing example (\ref{ex:11.1}a), based on a cliché of the Victorian era, Grice argued that a speaker who says (\ref{ex:11.1}a) only \textsc{asserts} (\ref{ex:11.1}b). If (\ref{ex:11.1}b) is true, then (\ref{ex:11.1}a) cannot be, strictly speaking, judged to be false.



\ea \label{ex:11.1}
\ea She is poor but she is honest.\\
\ex She is poor and she is honest.  [\citealt{Grice1961}: 127]
                       \z
\z


Grice claimed that \textit{and} and \textit{but} have the same truth-conditional meaning. However, the word \textit{but} provides an additional “adversative” component of meaning, indicating that the speaker believes there to be a potential conflict, or a contrast of some sort, between poverty and honesty. This extra element of meaning (implied contrast or counter-expectation) is part of the conventional meaning of \textit{but}, and is not derived from the context of use. However, in Grice’s view it does not contribute to the truth conditions of the statement. Frege had expressed very similar views several decades earlier:


\begin{quote}
The word \textit{but} differs from \textit{and} in that with it one intimates that what follows is in contrast with what would be expected from what preceded it. Such suggestions in speech make no difference to the thought [i.e., the propositional content–PK].   (\citealt{Frege1918})
\end{quote}



Nevertheless, even if (\ref{ex:11.1}b) is true it would be perfectly natural for someone to object to (\ref{ex:11.1}a) as in \REF{ex:11.2}, claiming that the word \textit{but} has been misused. The core of this objection would not be the truth of the statement in (\ref{ex:11.1}a) but the appropriateness of the conjunction that was chosen.


\ea \label{ex:11.2}
What do you mean “but”? There is no conflict between poverty and honesty!
\z


What these arguments suggest is that the adversative meaning of \textit{but} does not affect the conditions under which a sentence will be true, but rather the conditions under which the sentence can be appropriately used. For dealing with such cases it is helpful to make a distinction between \textsc{truth-conditional meaning} vs. \textsc{use-conditional meaning}.\footnote{\citet{Gutzmann2015}. See \citet{Recanati2004} for an earlier usage of the term \textsc{use-conditional}.} The truth-conditional meaning that is asserted in (\ref{ex:11.1}a) would be equivalent to the meaning of (\ref{ex:11.1}b), while the adversative implication comes from the use-conditional meaning of \textit{but}.



In this chapter we will discuss the differences between truth-conditional meaning vs. use-conditional meaning, and illustrate various types of use-conditional expressions. We begin in \sectref{sec:11.2} with a discussion of the definition and diagnostic properties of use-conditional meaning, based on the work of Christopher Potts. We illustrate this discussion using certain types of adverbs in English which seem to contribute use-conditional rather than truth-conditional meaning. In the rest of the chapter we look at some use-conditional expressions in other languages: honorifics in {Japanese} (\sectref{sec:11.3}), politeness markers in {Korean} (\sectref{sec:11.4}), honorific pronouns and other polite register lexical choices (\sectref{sec:11.5}), and discourse particles in {German} (\sectref{sec:11.6}).


\section{Identifying use-conditional meaning}\label{sec:11.2}

In addition to \textit{but}, Grice (1975) suggests that the sentence adverb \textit{therefore} is another example of a word whose conventional meaning does not affect the truth value of a sentence:



\begin{quote}
If I say (smugly), \textit{He is an Englishman; he is, therefore, brave}, I have certainly committed myself, by virtue of the meaning of my words, to its being the case that his being brave is a consequence of (follows from) his being an Englishman. But while I have said that he is an Englishman, and said that he is brave, I do not want to say that I have said (in the favored sense [i.e. as part of the truth-conditional meaning–PK]) that it follows from his being an Englishman that he is brave…  I do not want to say that my utterance of this sentence would be, strictly speaking, false should the consequence in question fail to hold.  [\citealt{Grice1975}: 44]
\end{quote}


What Grice is saying is that the adverb \textit{therefore} does not assert that a causal relation exists between two propositions, unlike the conjunctions \textit{because} or \textit{since} (see chapter \chapref{sec:18}). Rather, \textit{therefore} indicates something about the speaker’s attitude toward the proposition that is currently being asserted, namely that it follows as a logical consequence of what has come before. The adverb \textit{nevertheless} indicates the opposite attitude, a sense of counter-expectation on the part of the speaker. But these attitudes are taken for granted and are not part of what the speaker is asserting to be true. For this reason they cannot be the basis for calling the speaker’s utterance “false”.



Several decades earlier, Frege (1918) had made similar comments about the word \textit{still}:


\begin{quote}
With the sentence \textit{Alfred has still not come} one really says ‘Alfred has not come’ and, at the same time, hints that his arrival is expected, but it is only hinted. It cannot be said that, since Alfred’s arrival is not expected, the sense of the sentence is therefore false. [\citealt{Frege1918}]
\end{quote}



Grice coined the term \textsc{conventional implicature} to refer to inferences of this type: conventional, linguistically encoded meanings which do not affect the truth value of a sentence. Grice’s choice of terminology has been the source of much confusion. Grice used the term “implicature” because he believed (like many other scholars at that time) that sentence meaning (“what is said”) could be identified with truth-conditional meaning. Under this view, any content that does not contribute to the truth conditions must be pragmatic in nature; hence the label “implicature”. As we saw in chapter 9, this assumption makes it difficult to account for cases where pragmatic inferences seem to affect the truth conditions of a sentence. Use-conditional meaning presents the opposite kind of challenge: conventional meanings which do not affect truth conditions.



In this book we will largely avoid the term “conventional implicature”, and (following Gutzmann 2015) refer to \textsc{use-conditional meaning} instead. The core, defining property of use-conditional meanings is that they do not change the conditions under which the sentence will be true, but rather the conditions under which the sentence can be appropriately used.



Grice actually said very little about “conventional implicatures”, and never developed a full-blown analysis. Recent work by Christopher Potts and others has tried to clarify Grice’s original intuition, and has greatly extended the range of expressions which are included in this category. Examples include:


\begin{enumerate}[label=\alph*.]
\item classic “conventional implicature” triggers: \textit{but, therefore, even}, etc. 
\item expressives \& epithets
\item supplements (parenthetical comments, appositions, non-restrictive relative clauses)
\item evaluative adverbs (e.g. \textit{fortunately, surprisingly})
\item speech act modifiers (e.g. \textit{frankly, confidentially})
\item evidential markers (source of information)
\item common ground indicators (new/old information, quantity hedge, mirative, etc.)
\item expressions which manage social relationships (politeness, familiarity, honorifics, etc.)

\end{enumerate}

\subsection{Diagnostic properties of use-conditional meaning}\label{sec:11.2.1}


Based on Grice’s comments about \textit{therefore}, quoted at the beginning of section \sectref{sec:11.2}, Potts formulates a definition of “conventional implicatures” (i.e., use-conditional meaning) that includes the following points: (i) use-conditional meanings are (normally) beliefs of the speaker (“I have certainly committed \textsc{myself}”), and so in a sense “speaker-oriented”; (ii) they are part of the intrinsic, conventional meaning of a given expression or construction (“by virtue of the meaning of my words”), and so are not cancellable; (iii) they do not contribute to the truth-conditional content which is the main point of the utterance.\footnote{\citet{Potts2005,Potts2012}; see also \citet[39]{Horn1997}.}



Potts uses the term \textsc{at-issue content} to refer to the main point of an utterance: the core information that is asserted in a statement or queried in a question. So in the sentence \textit{He is an Englishman; he is, therefore, brave}, Grice states that the at-issue content of the assertion is \textit{He is English and brave}. The use-conditional meaning contributed by \textit{therefore}, namely that some kind of causal relationship exists between these two properties, is not part of the at-issue content.



Based on the definition outlined above, Potts lists the following diagnostic features which allow us to distinguish use-conditional meaning from other kinds of meaning. Use-conditional meanings are:\footnote{\citet{Potts2015}; a similar list is presented for expressives in \citet{Potts2007c}.}

\begin{itemize}
\item \textsc{conventional}, i.e., semantic in nature rather than pragmatic (as we defined those terms in \chapref{sec:9}). They must be learned as part of the meaning of a given word or construction, and cannot be calculated from context.

\item \textsc{independent}: separate from and logically independent of the at-issue content.

\item \textsc{secondary}: they do not address the current “Question Under Discussion” (QUD), which defines the at-issue content, but rather are used to provide extra clarification, contextual information, editorial comments, evaluation, etc.

\item \textsc{“scopeless”:} since use-conditional meanings are not part of the at-issue content, they are typically not interpreted as falling within the scope of clausal negation, interrogative mood, etc. Often they take scope over the whole sentence even when embedded in subordinate clauses.
\end{itemize}


You may have noticed that these properties are similar to the properties of expressive meaning, which we listed in \chapref{sec:2}. This is no accident, since expressives provide a clear example of use-conditional meaning. The expressive term \textit{jerk} in example (\ref{ex:11.3}a) reflects a negative attitude toward Peterson, and this negative attitude is a belief of the speaker. The addressee does not have to calculate this negative attitude based on the context of the utterance; it comes directly from the conventional meaning of the word \textit{jerk}. It is not part of the at-issue content of the sentence, so even if an addressee does not share this negative attitude, it would normally not cause him to regard (\ref{ex:11.3}a) as being a false statement. The negative attitude is still expressed if the sentence is negated or questioned (\ref{ex:11.3}b--c).


\ea \label{ex:11.3}
\ea That jerk Peterson is the only economist on this committee.\\
\ex That jerk Peterson isn’t the only economist on this committee.\\
\ex Is that jerk Peterson the only economist on this committee?\\
                       \z
\z


 Most of these properties are also shared by presuppositions, which we discussed in \chapref{sec:3}. The distinction (or lack of distinction) between presuppositions and “conventional implicatures” has been a topic of great controversy, which is beyond the scope of the present book.\footnote{See for example  \citet{KarttunenPeters1979}, \citet{Bach1999} for two differing viewpoints on this issue.} Both \citet{Potts2005} and \citet{Gutzmann2015} argue that we should view use-conditional meaning (or “conventional implicature”) as a distinct category from presupposition. Potts points out that supplements, for example, are clearly not presupposed. They are normally informative, and typically sound unnaturally redundant if the content they express is already part of the common ground. But this is not the case for many other types of use-conditional meaning. Honorifics, for example, encode social relationships which would normally be recognized and agreed upon by both speaker and addressee. If a speaker uses an honorific in an unexpected way, the effect might be compared to a kind of presupposition failure.\footnote{\citet{Schlenker2007} and \citet{Lasersohn2007} argue that expressive meaning can be seen as a type of presupposition.}  For present purposes, however, we will leave the exact relation between presupposition and use-conditional meaning as an open question.



\subsection{Speaker-oriented adverbs}\label{sec:11.2.2}
\largerpage
In this section we will illustrate the diagnostic properties listed above by discussing two classes of sentence adverbs in English which seem to contribute use-conditional rather than truth-conditional meaning. The \textsc{evaluative adverbs (}e.g. \textit{(un)fortunately, oddly, sadly, surprisingly, inexplicably}) provide information about the speaker’s attitude toward the proposition being expressed. The \textsc{speech act adverbials} (e.g. \textit{frankly}, \textit{honestly}, \textit{seriously}, \textit{confidentially}) provide information about the manner in which the current utterance is being performed. We will use the term \textsc{speaker-oriented adverbs} as a generic term that includes both of these classes.\footnote{The label \textsc{evaluative adverbs} comes from \citet{Ernst2009}. Ernst uses the term \textsc{speaker-oriented adverbs} as to include not only evaluative adverbs and speech act adverbials, but also modal adverbs like \textit{probably}. \citet{Potts2005} uses the term \textsc{speaker-oriented adverbs} to refer to the class that I call \textsc{evaluative adverbs}.}


Let us consider some of the evidence which supports the claim that the core meanings contributed by these speaker-oriented adverbs have the properties identified with use-conditional meaning: they are conventional, independent, secondary, and scopeless.



\subsection*{Conventional}

The meanings contributed by speaker-oriented adverbs like \textit{surprisingly} or \textit{confidentially} are clearly conventional. They are the meanings that would be listed in the dictionary entries for these words. They are not calculable but must simply be learned, one word at a time.


\subsection*{Independent and Secondary}

Because speaker-oriented adverbs do not contribute to the at-issue content of the utterance, they are not \textsc{challengeable}. In other words, they cannot be the grounds for challenging the truth of an assertion, as illustrated in (\ref{ex:11.5}–\ref{ex:11.6}). In both examples, B’s first response challenges some part of the at-issue content of the sentence, and this challenge is fully acceptable. The second response (labeled B$'$) challenges the content contributed by the sentence adverb, and the result seems somewhat unnatural.



\ea \label{ex:11.5}
\begin{description}
\item[A:] \textit{Curiously}/\textit{fortunately} the mayor never asked where all the money came from.
\item[B:] That’s not true; he asked me just last week.
\item[B$'$:] \#That’s not true; he never asked, but there is nothing curious/fortunate about that.
\end{description}
\ex \label{ex:11.6}
\begin{description}
\item[A:] \textit{Frankly/confidentially}, Jones is not the best-qualified candidate for this job.
\item[B:] That’s not true; he is the only candidate who holds a relevant degree.
\item[B$'$:] \#That’s not true; he is not qualified, but you are not speaking frankly/confidentially.
\end{description}
\z


Of course, in responding to statements like these the addressee may well express disagreement with the adverbial content by saying something like: \textit{I do not consider that to be curious/fortunate/…}. The crucial issue is whether the statement \textit{That’s not true} is appropriate or felicitous in such examples. There may be some differences between speakers concerning these judgments, but the majority view seems to be that misuse of a speaker-oriented adverb is not an adequate basis for declaring a statement to be false.


These claims about speaker-oriented adverbs apply only to their use as sentence adverbs, where the speaker uses them to describe his own manner of speaking or attitude toward the current speech act. However, some of these adverbs are polysemous. A number of speech act adverbials also have a second sense in which they function as manner adverbs. In this use they describe the manner of the agent of a reported speech act.


Sentence adverbs in English occur most freely in sentence initial position, as in (\ref{ex:11.8}a) and (\ref{ex:11.9}a); but other positions are also possible (normally with the adverb set off from the rest of the sentence by pauses) as illustrated in (\ref{ex:11.8}b--d) and (\ref{ex:11.9}b--d).


\ea \label{ex:11.8}
\ea \textit{Curiously}, the mayor never asked where all the money came from.
\ex The mayor, c\textit{uriously}, never asked where all the money came from.
\ex The mayor never asked, c\textit{uriously}, where all the money came from.
\ex The mayor never asked where all the money came from, c\textit{uriously}.
       \z
\ex \label{ex:11.9}
\ea \textit{Frankly/confidentially}, Jones is not the best-qualified candidate for this job.
\ex Jones, \textit{confidentially}, is not the best-qualified candidate for this job.
\ex Jones is not, \textit{frankly}, the best-qualified candidate for this job.
\ex Jones is not the best-qualified candidate for this job, \textit{frankly}.
                       \z
\z

Manner adverbs, on the other hand, typically occur within the verb phrase as illustrated in (\ref{ex:11.10}A), at least in basic (or neutral) word order. When adverbial forms like \textit{frankly} or \textit{confidentially} are used as manner adverbs, they do contribute to the at-issue content of the sentence. We can see that this is so because the truth of an assertion can be challenged if such an adverb is misused, as illustrated in (\ref{ex:11.10}B).



\ea \label{ex:11.10}
\begin{description}
\item[A:] Jones told the committee \textit{frankly/confidentially} about his criminal record.
\item[B:] That’s not true; he told them, but he did not speak frankly/confidentially.
\end{description}
\z


A number of the evaluative adverbs are morphologically related to an adjective that takes a propositional complement. In simple sentences, the adverb and adjective can be used to paraphrase each other, as seen in (\ref{ex:11.13}--\ref{ex:11.15}).


\ea \label{ex:11.13}
\ea \textit{Fortunately}, Jones doesn’t realize how valuable this parchment is.\\
\ex It is \textit{fortunate} that Jones doesn’t realize how valuable this parchment is.
                       \z
\z

\ea \label{ex:11.14}
\ea \textit{Curiously}, the mayor never asked where all the money came from.\\
\ex It is \textit{curious} that the mayor never asked where all the money came from.
\z \z

\ea \label{ex:11.15}
\ea \textit{Oddly}, Jones never got that parchment appraised before he put it up for auction.\\
\ex It is \textit{odd} that Jones never got that parchment appraised before he put it up for auction.
                       \z
\z


However, evaluative adjectives, in contrast to the corresponding evaluative adverbs, do contribute to the at-issue content of the utterance. They can provide grounds for challenging the truth of a statement, as seen in \REF{ex:11.16}.


\ea \label{ex:11.16}
\begin{description}
\item[A:] It is \textit{curious/fortunate} that the mayor never asked where all the money came from.\\
\item[B:] That’s not true; the fact that he never asked is \{not curious at all/most unfortunate\}.
\end{description}
\z


Another important way of identifying at-issue content is that only the at-issue content of an utterance can be used to answer the current “Question Under Discussion”.\footnote{\citet{Roberts1996}, \citet{Ginzburg1996}.}  As examples (\ref{ex:11.30}--\ref{ex:11.31}) illustrate, evaluative adjectives and manner adverbs can be used to answer the current Question Under Discussion (A1), but the corresponding sentence adverbs cannot (A2).


\ea \label{ex:11.30}
\begin{description}
\item[Q:] What is your opinion of the procedures that were followed?\\
\item[A1:] It is \textit{odd} that Jones never got that parchment appraised before he put it up for auction.\\
\item[A2:] \#\textit{Oddly}, Jones never got that parchment appraised before he put it up for auction.\\	
\end{description}
\z


\ea \label{ex:11.31}
\begin{description}
\item[Q:] How openly were these issues discussed with the committee?\\
\item[A1:] Jones told the committee \textit{frankly} about his criminal record (but only guardedly about his medical problems).\\
\item[A2:] \#\textit{Frankly}, Jones told the committee about his criminal record.\\	
\end{description}
\z 


Such examples support our earlier conclusion that evaluative adjectives and manner adverbs contribute truth-conditional meaning, whereas speaker-oriented adverbs contribute use-conditional meaning.



\subsection*{Scopeless}


Further evidence for the claim that these speaker-oriented adverbs are not part of the at-issue content of an utterance comes from their behavior under negation and questioning. When a sentence containing an evaluative or speech act sentence adverb is negated or questioned, the adverb itself cannot be interpreted as part of what is being negated or questioned. For example, (\ref{ex:11.7}a) cannot mean ‘It is not fortunate that the best team won’ but only ‘It is fortunate that the best team did not win.’ Example (\ref{ex:11.7}b) cannot mean ‘Was it unfortunate that he lost the vision in that eye?’ but only ‘Did he lose the vision in that eye? If so, it was unfortunate.’ Speech act adverbials in questions like (\ref{ex:11.7}c) are not part of what is being questioned, but generally describe either the manner in which the question is being asked, or the manner in which the speaker wants the addressee to answer the question. As such examples show, evaluative and speech act sentence adverbs are not interpreted as being under the scope of sentence negation or interrogative mood.


\ea \label{ex:11.7}
\ea  … the best team \textit{fortunately} didn’t win on this occasion.\footnote{\url{http://sportwitness.ning.com/forum/topics/nextgen}} \\
\ex Was it ok or did he \textit{unfortunately} lose the vision in that eye?\footnote{\url{https://www.inspire.com/groups/preemie/discussion/rop-after-2-ops-scarring-is-pulling-the-retina-away/}} \\
\ex Is he, \textit{frankly}, combative enough? (referring to a potential presidential candidate)\footnote{\href{http://www.wbur.org/2011/12/21/romney-nh-6}{{www.wbur.org/2011/12/21/romney-nh-6}}} 
                       \z
\z


When speech act adverbials are used in their manner adverb senses, they do contribute to the at-issue content of the sentence. As a result, these adverbs can be interpreted as being part of the propositional content which is negated (\ref{ex:11.11}a) or questioned (\ref{ex:11.12}a). This contrasts with the behavior of the same forms used as sentence adverbs, which are not interpreted as being included under negation (\ref{ex:11.11}b) or questioning (\ref{ex:11.12}b).



\ea \label{ex:11.11}
\ea Jones did not inform the committee \textit{confidentially} about his criminal record; he told them in a public hearing.\\
\ex Jones, \textit{confidentially}, did not inform the committee about his criminal record.
                       \z
\z

\ea \label{ex:11.12}
\ea Did Jones tell you this \textit{confidentially}, or can we inform the other members of the committee?\\
\ex \textit{Confidentially}, did Jones tell the committee about this?
                       \z
\z



In the same way, adjectival correlates of evaluative adverbs are part of the propositional content which can be negated \REF{ex:11.17} or questioned \REF{ex:11.18}. This fact confirms our earlier claim that these adjectives do contribute to the at-issue content of the utterance.



\ea \label{ex:11.17}
It is not \textit{odd} that Jones asked for an appraisal before he bought that parchment; it seems natural under the circumstances.
\z

\ea \label{ex:11.18}
A: Was it \textit{odd} that Jones did not ask for an appraisal?\\
B. No, I think it was fairly natural under the circumstances.
\z



To summarize, we have argued that evaluative adverbs and speech act adverbials in English contribute use-conditional rather than truth-conditional meaning to the utterances in which they occur. We argued this on the grounds that they reflect the beliefs and attitudes of the speaker. They are independent of and secondary to the at-issue propositional content of the utterance. For this reason they cannot be negated or questioned, and they do not affect the truth value of a statement. But clearly the meaning that these adverbs contribute is conventional: it has to be learned, rather than being calculated from the context of use.



\section{Japanese honorifics}\label{sec:11.3}

Honorifics are grammatical markers that speakers use to show respect or deference to someone whom they consider to be higher in social status than themselves. \ili{Japanese} has two major types of honorifics. One type is used to show respect toward someone referred to in the sentence, with different forms used for subjects vs. non-subjects. We will refer to this type as \textsc{argument honorifics}. The other type is used to show respect to the addressee, and so are considered to be a mark of polite speech. This type is often referred to as “performative honorifics”, because they indicate something about the context of the current speech event, specifically the relationship between speaker and addressee. We will instead refer to this second type as \textsc{addressee honorifics}.\footnote{The term \textit{argument honorifics} is adapted from \citet{Potts2005}, who referred to this type as “argument-oriented honorifics”. \citet{Harada1976}, one of the first detailed discussions of these issues in English, refers to this type as “propositional honorifics”. Harada was the original source of the term “performative honorifics” for those which show respect to the addressee, a terminology which is now widely adopted.}


The use of an argument honorific to indicate the speaker’s respect for a person referred to in the sentence is illustrated in (\ref{ex:11.19}a), which shows respect for the referent of the subject NP (Prof. {Sasaki}). The use of an addressee honorific to indicate the speaker’s respect for the addressee is illustrated in (\ref{ex:11.19}b).


\ea \label{ex:11.19}
\ea   \gll {Sasaki}  sensei=wa  watasi=ni  koo  \textbf{o}-hanasi.\textbf{ni.nat}-ta.\\
{Sasaki}  teacher=\textsc{top}  1sg=\textsc{dat}  this.way  speak.\textsc{hon-past}\\
\glt ‘Prof. {Sasaki} told me this way.’  [\citealt{Harada1976}: 501]
\ex \gll
    Watasi=wa  sono  hito=ni  koo  hanasi-\textbf{masi}-ta.\\
\textsc{1sg}=\textsc{top}  that  man=\textsc{dat}  this.way  speak-\textsc{hon-past}\\
\glt ‘I told him (=that man) this way.’  (polite speech)   [\citealt{Harada1976}: 502]
\z \z


Argument honorifics are only allowed in sentences that refer to someone socially superior to the speaker; sentence (\ref{ex:11.20}a) is unacceptable, because no such person is referred to. But addressee honorifics are not subject to this constraint (\ref{ex:11.20}b).


\ea \label{ex:11.20} \ea  \gll *Ame=ga  \textbf{o}-huri.\textbf{ni.nat}-ta.\\
  rain=\textsc{nom}  fall.\textsc{hon-past}\\
\glt (intended: ‘It rained.’)  [\citealt{Harada1976}: 502]
\ex \gll
Ame=ga  huri-\textbf{masi}-ta.\\
rain=\textsc{nom}  fall-\textsc{hon-past}\\
\glt ‘It rained.’  (polite speech)   [\citealt{Harada1976}: 502]
\z \z


In the remainder of this section we will focus primarily on addressee honorifics. \citet{Potts2005} analyzes addressee honorifics as use-conditional items, specifically as a kind of expressive. This means that addressee honorifics do not contribute to the truth-conditional, at-issue content of the sentence. The truth conditions of (\ref{ex:11.20}b) would not be changed if the honorific marker were deleted. Misuse of the honorific (e.g. for referring to someone socially inferior), or dropping the honorific when it is expected, would not make the statement false, only rude and/or inappropriate.\footnote{Thanks to Eric Shin Doi for very helpful discussion of these issues, and for providing the examples in \REF{ex:11.21}.}



As we would predict under Potts’ proposal, the honorific meaning cannot be part of the propositional content that is negated or questioned. (\ref{ex:11.21}a--b) are felt to be just as polite as (\ref{ex:11.20}b); the element of respect is neither negated in (\ref{ex:11.21}a) nor questioned in (\ref{ex:11.21}b).


\ea \label{ex:11.21}
\ea  \gll Ame=ga                 huri-\textbf{mas-en}  desi-ta.\\
            rain=\textsc{nom}  fall-\textsc{hon-neg}  \textsc{cop-past}\\
\glt ‘It didn’t rain.’  (polite speech)

\ex
 \gll  Ame=wa  huri-\textbf{masi}-ta-ka?\\
rain=\textsc{top}  fall-\textsc{hon-past-q}\\
\glt ‘Did it rain?’  (polite speech)
\z \z


We have seen that addressee honorifics express beliefs or attitudes of the speaker. They are independent of and secondary to the at-issue propositional content of the utterance. They cannot be negated or questioned, and do not affect the truth value of a statement. Thus they clearly fit our definition of use-conditional meaning.


\section{Korean speech style markers}\label{sec:11.4}

\ili{Korean} also has the same two types of honorifics as \ili{Japanese}, argument honorifics vs. addressee honorifics.\footnote{\citet{KimSells2007}} As part of the addressee honorific system, \ili{Korean} distinguishes grammatically six levels of politeness, often referred to as \textsc{speech styles}: formal, semiformal, polite, familiar, intimate, and plain.\footnote{\citet{Martin1992}, \citet{Pak2008}, \citet{Sohn1999}} A seventh level, “super-polite”, was used for addressing kings and queens; it is now considered archaic, and is used mostly in prayers. The choice of speech style marking depends on “(i) the \textit{relationship} between speaker and addressee (e.g., intimacy, politeness), and (ii) the \textit{formality} of the situation”.\footnote{\citet{PakEtAl2013}} The uses of these styles, as described by \citet[120]{Pak2008}, are summarized in \tabref{extab:11.22}.

\begin{table}
 \caption{Use of Korean speech styles following \citet[120]{Pak2008}}

\label{extab:11.22}
\begin{tabularx}{\textwidth}{L{2.5cm}Q}
\lsptoprule
 Speech styles & Contexts of use\\
 \midrule
 Formal & used for speaking to someone to whom deference is due (e.g., one’s superior or employer, a professor, a high official, etc.); or on formal occasions such as oral news reports and public lectures\\
 \tablevspace
 Semiformal & could be used by a husband speaking to his wife, or by a younger superior speaking to an older subordinate; gradually disappearing from daily usage\\
  \tablevspace
 Polite & used by adults for speaking to adults who are not close friends or family members; to address a socially equal or superior person; or by children speaking to adults in a polite way\\
  \tablevspace
 Familiar & mostly used by male adults, for speaking to male adult friends, an adolescent, or a son-in-law\\
 \tablevspace
 Intimate (“half-talk”) & used for talking to family members or close friends\\
 \tablevspace
 Plain & used by adults for speaking to children or younger siblings, and by children among themselves; also used in written texts and newspapers\\
\lspbottomrule
\end{tabularx}
\end{table}

Speech style is marked grammatically by a verbal suffix referred to as the “sentence ender”. Since \ili{Korean} is an SOV language, the main clause verb typically occurs at the end of the sentence and hosts the sentence ender. The sentence ender is actually a portmanteau suffix which encodes three distinct grammatical features: (a) speech style (i.e. politeness); (b) “special mood” (not discussed here); and (c) sentence type (i.e. sentence mood; grammatical indicator of illocutionary force).\footnote{\citet{Sohn1999}, \citet{Pak2008}.} \ili{Korean} has an unusually rich inventory of sentence mood categories. The exact number is a topic of controversy; \citet{Sohn1999} lists four major sentence types (declarative, interrogative, imperative, and “propositive” or hortative); plus several minor types including admonitive (warning), promissive, exclamatory, and apperceptive (new or currently perceived information?). Combinations of four of the speech styles with two sentence types (declarative and imperative) are illustrated in \tabref{extab:11.23}; the sentence enders are italicized.\footnote{These examples are taken from \citet{PakEtAl2013}.}
 
 \begin{table}
\caption{Declaratives and imperatives in Korean}
\label{extab:11.23}\small
\begin{tabularx}{\textwidth}{lll}
\lsptoprule
          &   Declarative &   Imperative  \\ 
          \midrule 
  Formal & \parbox{6cm}{\gll Chayk=ul  ilk-ess-\textit{supnita}.\\
			      book=\textsc{acc}  read-\textsc{past-decl.form} \\
			 \glt ‘I read the book.’} & \parbox{5cm}{\gll  Chayk=ul  ilk-\textit{usipsio}.\\
								      book=\textsc{acc}  read-\textsc{imp.form}\\
								      \glt ‘Please read the book!’}\\

\tablevspace
  Polite &  \parbox{6cm}{\gll  Chayk=ul  ilk-ess-\textit{eyo}. \\
			  book=\textsc{acc}  read-\textsc{past-decl.pol}\\
			  \glt ‘I read the book.’ }&  \parbox{5cm}{\gll  Chayk=ul  ilk-\textit{useyyo.}\\  
									  book=\textsc{acc}  read-\textsc{imp.pol}\\
									  \glt ‘Please read the book.’}\\

\tablevspace
  Intimate  &  \parbox{6cm}{\gll Chayk=ul  ilk-ess-\textit{e}.  \\
			    book=\textsc{acc}  read-\textsc{past-decl.int} \\
			    \glt ‘I read the book.’}  &  \parbox{5cm}{\gll Chayk=ul  ilk-\textit{e}.\\
									    book=\textsc{acc}  read-\textsc{imp.int}\\ 
									    \glt ‘Read the book!’}\\

\tablevspace
Plain &  \parbox{6cm}{\gll  Chayk=ul  ilk-ess-\textit{ta}.\\
			  book=\textsc{acc}  read-\textsc{past-decl} \\
			\glt ‘I read the book.’} &  \parbox{5cm}{\gll  Chayk=ul  ilk-\textit{ela}.\\
								book=\textsc{acc}  read-\textsc{imp}\\ 
								\glt ‘Read the book!’}\\ 
								\\
\lspbottomrule
\end{tabularx}
 \end{table}


Like \ili{Japanese} honorifics, the \ili{Korean} speech style markers contribute information about the current speech act, specifically the relationship between speaker and hearer, rather than contributing to the at-issue propositional content of the utterance. Use of the wrong speech style marker in a particular situation would not cause a statement to be considered false, but would be felt to be inappropriate. A speaker who committed such an error would probably be corrected quickly and emphatically. Moreover, the information contributed by the speech style markers cannot be negated or questioned. The negative statement in (\ref{ex:11.24}b) and the question in (\ref{ex:11.24}c) are felt to be just as polite as the corresponding positive statement in (\ref{ex:11.24}a), and would be appropriate in the same range of situations.\footnote{Thanks to Shin-Ja Hwang for very helpful discussion of these issues.}


\ea \label{ex:11.24} 
\ea  \gll Pi=ka  w-ayo.\\
rain=\textsc{nom}  come-\textsc{decl.pol}\\
\glt ‘It is raining.’ (polite)

\ex \gll Pi=ka  an-w-ayo.\\
rain=\textsc{nom}  \textsc{neg}-come-\textsc{decl.pol}\\
\glt ‘It is not raining.’ (polite)

\ex \gll  Pi=ka  w-ayo?\\
rain=\textsc{nom}  come-\textsc{decl.pol}\\
\glt ‘Is it raining?’ (polite)  [\citealt{Sohn1999}: 269–270]
\z
\z

\section{Other ways of marking politeness}\label{sec:11.5}

Honorific markers and speech style markers like those discussed in the previous two sections have no at-issue descriptive content, but only a use-conditional, utterance modifying function. However, there are words in many languages which express both normal descriptive content plus a use-conditional function as a marker of politeness.



One of the most common ways across languages of showing respect or politeness to the addressee is by distinguishing polite vs. familiar forms of the second person pronoun, e.g. \textit{vous} vs. \textit{tu} in \ili{French}, \textit{Sie} vs. \textit{du} in \ili{German}, etc. \ili{Malay} has a very complex system of first and second person pronouns. The neutral first person singular form is \textit{saya}; \textit{aku} is considered more intimate, for use with friends and family members. \textit{Beta} is the first person singular form used by royalty, and \textit{patik} is the first person singular form used by commoners when addressing royalty. There is no native \ili{Malay} second person singular pronoun which is truly neutral; \textit{kamu}, \textit{awak}, and \textit{engkau} are all felt to be informal or intimate to varying degrees. The term \textit{anda} was invented as part of the standardization of Malaysian as a national language to fill this gap, but is rarely used in conversational speech. Second person pronouns tend to be avoided when addressing royalty or other highly respected people, by using titles, kin terms, etc. instead.



Lexical substitution as a means of honorification is not limited to pronouns. \ili{Balinese} and \ili{Javanese} are famous for their speech levels, or registers. In these languages, two or more forms are available for thousands of lexical items, e.g. \ili{Balinese} \textit{makita} (high) vs. \textit{edot} (low) ‘want’; \textit{sanganan} (high) vs. \textit{jaja} (low) ‘cake’.\footnote{\citet{Arka2005}.} The choice of which form to use is determined by the relative social status, caste, etc. of the speaker and addressee. \ili{Korean} and \ili{Japanese} also have suppletive forms for some words, e.g. \ili{Korean} \textit{pap} (plain) vs. \textit{cinci} (polite) ‘cooked rice, meal’. The primary meaning contributed by words of this sort is to the truth-conditional content of the sentence; their use-conditional politeness function is in a sense secondary.


\section{Discourse particles in German}\label{sec:11.6}

\ili{German} and \ili{Dutch} are well-known for their large inventories of discourse particles. These particles have been intensively studied, but their meanings are difficult to define or paraphrase. Those that occur in the “middlefield” (i.e., between the V2/Aux position and the position of clause-final verbs) have traditionally been referred to as \textit{Modalpartikeln} ‘modal particles’ in \ili{German}, although they do not express modality in the standard sense of the term.\footnote{\citet[45--46]{Palmer1986}. Another and perhaps better name for these particles is \textit{Abtönungspartikel}, which seems to suggest that the particle adds nuance or shading to the meaning of the sentence.} Some examples and a description from \citet[2013]{Zimmermann2011} are presented in \REF{ex:11.25}.


\ea \label{ex:11.25}
\ea  Max ist \textit{ja} auf See.\\
\ex Max ist \textit{doch} auf See.\\
\ex Max ist \textit{wohl} auf See.\\
‘Max is \textsc{prtcl} at sea.’
\z

\begin{quote}
The sentences in (\ref{ex:11.25}a–c) do not differ in propositional content: they all have the same truth conditions…  A difference in the choice of the particle (\textit{ja}, \textit{doch}, \textit{wohl}) leads to a difference in felicity conditions, however, such that each sentence will be appropriate in a different context. As a first approximation, (\ref{ex:11.25}a) indicates that the speaker takes the hearer to be aware of the fact that Max is at sea. In contrast, (\ref{ex:11.25}b) signals that the speaker takes the hearer not to be aware of this fact at the time of utterance. (\ref{ex:11.25}c), finally, indicates a degree of speaker uncertainty concerning the truth of the proposition expressed. In each case, the discourse particle does not contribute to the descriptive, or propositional, content of the utterance, but to its expressive content.
\end{quote}
\z


Most of the \ili{German} modal particles are homophonous with a stressed variant belonging to one of the standard parts of speech. For example, stressed \textit{ja} means ‘yes’ and stressed \textit{wohl} means ‘probably’. However, when used as particles these words are unstressed and take on a variety of meanings, many of which are difficult to paraphrase or translate. Some of the variant meanings of \textit{ja} and \textit{doch} are illustrated in (\ref{ex:11.26}--\ref{ex:11.27}).


\ea \label{ex:11.26}
\ea  Die Malerei war \textit{ja} schon immer sein Hobby.\\
\glt ‘<\textit{As you know}>, painting has always been his hobby.’

\ex  Dein Mantel ist \textit{ja} ganz schmutzig.\\
\glt ‘<\textit{Hey}> your coat is all dirty.’ (not previously known to hearer)

\ex Fritz hat \textit{ja} noch gar nicht bezahlt.\\
\glt ‘<\textit{Hey}> Fred has not paid yet.’ (newly discovered by speaker)
\footnote{Examples from \citealt{König1991,KönigEtAl1990,Waltereit2001}.}
\z \z

\ea \label{ex:11.27} \ea  A: Maria kommt mit. ‘Maria is coming with me.’\\
    B: Sie ist \textit{doch} verreist. ‘She has left, <\textit{hasn’t she}>?’
\ex  Das ist \textit{doch} der Hans! Was macht der hier?\\
‘That’s Hans over there <\textit{surprise}>! What is he doing here?’
\ex Ich war \textit{doch} letztes Jahr schon dort.\\
‘<\textit{Did you forget?}> I was here last year.’\footnote{Examples from \citet{Karagjosova2000,Grosz2010}; \url{http://en.wikipedia.org/wiki/German_modal_particle}.}
\z \z 


In the passage quoted above, \citet{Zimmermann2011} states that these particles contribute to the expressive content of the utterance rather than its descriptive, or at-issue, content; they affect the felicity conditions of the utterance, but not its truth conditions. So, for example, all of the sentences in \REF{ex:11.25} would be true if Max is in fact at sea at the time of speaking. Using the wrong particle would make the utterance infelicitous, but not false. Other authors have reached similar conclusions. \citet{Waltereit2001} states:


\begin{quote}
{}[Modal particles] modify the preparatory conditions, as they evoke a speech situation in which the desired preparatory conditions are fulfilled… Preparatory conditions describe the way the speech act fits into the social relation of speaker and addressee, and they describe how their respective interests are concerned by the act.\footnote{cf. \citet{Searle1969}.}
\end{quote}


\citet{Karagjosova2000} states that “[modal particles] indicate if and how incoming information in dialogue is processed by the interlocutors in terms of its consistency with the information or beliefs the interlocutors already have.” For example, modal particles may indicate whether a proposition has succeeded in becoming \textsc{grounded}, i.e., part of the shared assumptions (\textsc{common ground}) of the speaker and hearer. She continues:


\begin{quote}
{}[T]he meaning of [modal particles] seems not to be part of the proposition indeed and thus not part of the truth conditions of the sentence they occur in. …  [W]e conclude that \textit{doch} does not contribute to the sentence meaning but to the utterance meaning and represents thus semantically an utterance modifier rather than a sentence modifier.
\end{quote}


The hypothesis that \ili{German} modal particles function as utterance modifiers, and do not contribute to truth-conditional content, is supported by the fact that they cannot be negated, as seen in \REF{ex:11.28}. Moreover, they cannot be questioned and cannot function as the answer to a question.\footnote{This point is mentioned in most descriptions of the \ili{German} modal particles, including \citet{Bross2012} and \citet{Gutzmann2015}.}


\ea \label{ex:11.28}
Hein ist \textit{ja} nicht zuhause.\\
\glt ‘\textit{As you know}, Hein is not at home.’  [\citealt{Gutzmann2015}, sec. 7.2.2.2]\\
(cannot mean: ‘You do not know that Hein is at home.’)
\z

\section{Conclusion}\label{sec:11.7}

In this chapter we have looked at several types of expressions in various languages that seem to contribute use-conditional rather than truth-conditional meanings. The characteristic properties of such expressions are those identified by Potts in his work on “conventional implicatures”. They tend to be speaker-oriented; independent of and secondary to the at-issue, truth-conditional content of the utterance; and immune to negation and questioning.



We noted that speech act adverbials in English (e.g. \textit{frankly}, \textit{confidentially}) can function either as sentence adverbs with use-conditional meanings, or as manner adverbs with truth-conditional meanings. In future chapters we will see that similar ambiguities arise with certain conjunctions, notably \textit{because} (\chapref{sec:18}) and \textit{if} (\chapref{sec:19}). We will argue that, at least for \textit{because}, such ambiguities need not be treated as polysemy (distinct senses), but can be seen as a kind of pragmatic ambiguity: a single sense that can function on two levels, modifying the sentence meaning or the utterance meaning. In the first case, it contributes truth-conditional meaning, while in the second case it contributes use-conditional meaning.



\furtherreading{



\citet{Potts2007b,Potts2007a} and (\citeyear{Potts2012}) provide concise introductions to his analysis of conventional implicatures. \citet{Potts2007c} focuses more specifically on expressives. \citet{Scheffler2013} applies this analysis to sentence adverbs in English and \ili{German}. \citet{Gutzmann2015} presents an introduction to the idea of use-conditional meaning in chapter 2, and an analysis of the \ili{German} “modal particles” in chapter 7. \citet{Brown2015} offers a good overview of honorifics and polite speech in Korean.

}
 
\discussionexercises{

\paragraph*{A. Use the kinds of evidence discussed in this chapter to determine whether the italicized expressions in the following examples contribute truth-conditional or use-conditional meaning:} ~


\medskip 

\begin{enumerate}  %[parsep=1ex] (not working)
\item  Sir Richard Whittington, \textit{a medieval cloth merchant}, served four terms as Lord Mayor of London.

\item  Wilma \textit{probably} loves sauerkraut.

\item  \textit{Even} Priscilla loves sauerkraut.

\item  Mrs. Natasha Griggs, \textit{who served six years as MP for Darwin}, is a cancer survivor.

\item  Baxter \textit{reportedly} supported Suharto.
\end{enumerate}
}
