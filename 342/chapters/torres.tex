\documentclass[output=paper]{langscibook}
\ChapterDOI{10.5281/zenodo.6759974}
\author{Olga Torres-Hostench\affiliation{Universitat Autònoma de Barcelona}}
\title{Europe, multilingualism and machine translation}
\abstract{This chapter explains multilingualism as a foundational principle of the European Union, describing how it is put into practice and supported through language learning and translation. Taking the university campus as a case study, it argues that machine translation can be used to foster multilingualism in this context.}


\begin{document}
\AffiliationsWithoutIndexing{}
\maketitle

\section{Introduction}\largerpage


The European Union's motto “united in diversity” is said to symbolize “the essential contribution that linguistic diversity and language learning make to the European project” \citep{EuropeanCommission2021}. But European Union (EU) policy on multilingualism is mostly built upon language learning and mobility, both time-consuming activities. And human language learning presents particular challenges. After all, there is a limit to the number of languages the average EU citizen can learn. The aim in this chapter is to suggest answers to these questions by arguing that machine translation can contribute to the promotion of multilingualism in Europe and thus to European linguistic diversity.

\section{A multilingual EU}
\begin{quote}
It is … an open secret that the EU’s supposedly humane multilingualism is but an illusion. \citep[561]{House2003}
\end{quote}

ISO 639-3 is a set of codes developed by the International Organization for Standardization (ISO) that defines three-letter identifiers for all known human languages. As of 30 January 2020, the standard contained entries for 7,868 languages \citep{Wikizero2020}, around 600 of which are spoken in Europe, and 24 of which are official languages of the EU. These are: Dutch, French, German, Italian (since 1958); Danish, English (since 1973); Greek (since 1981); Portuguese, Spanish (since 1986); Finnish, Swedish (since 1995); Czech, Estonian, Hungarian, Latvian, Lithuanian, Maltese, Polish, Slovak, Slovene (since 2004); Bulgarian, Irish, Romanian (since 2007) and Croatian (since 2013).

Linguistic diversity is part of Europe’s cultural heritage. In Europe, there are languages with official status at state level, and indigenous regional and/or minority languages with different degrees of recognition. The 1998 European Charter for Regional or Minority Languages is the European convention for the protection and promotion of languages used by traditional minorities. It was reformed and strengthened by a monitoring mechanism in 2019. The Charter covers 79 languages used by 201 national minorities or linguistic groups \citep{CoE2020}. They are presented in alphabetical order in \tabref{tab:torres:1}. 



\begin{table}
\begin{tabularx}{\textwidth}{l}
\lsptoprule
\parbox{\textwidth}{
\begin{multicols}{4}
Albanian\\
Aragonese\\
Aranese\\
Armenian\\
Assyrian\\
Asturian\\
Basque\\
Beás\\
Belarusian\\
Bosnian\\
Bulgarian\\
Bunjevac\\
Catalan\\
Cornish\\
Crimean Tatar\\
Croatian\\
Cypriot Maronite \hspace*{2mm}Arabic\\
Czech\\
Danish\\
Finnish\\
Franco-Provençal\\
French\\
Frisian\\
Gagauz\\
Galician\\
German\\
Greek\\
Hungarian\\
Inari-Sami\\
Irish\\
Istro-Romanian\\
Italian\\
Karaim\\
Karelian\\
Kashub\\
Krimchak\\
Kurdish\\
Kven/Finnish\\
Ladino\\
Lemko\\
Leonese\\
Limburgish\\
Lithuanian\\
Low German\\
Lower Saxon\\
Lower Sorbian\\
Lule Sami\\
Macedonian\\
Manx Gaelic\\
Meänkieli\\
Moldovan\\
North Frisian\\
North Sami\\
Polish\\
Romani\\
Romanian\\
Romansh\\
Russian\\
Ruthenian\\
Sater Frisian\\
Scots\\
Scottish-Gaelic\\
Serbian\\
Skolt Sami\\
Slovakian\\
Slovenian\\
South Sami\\
Swedish\\
Tatar\\
Turkish\\
Ukrainian\\
Ulster Scots\\
Upper Sorbian\\
Valencian\\
Vlach\\
Welsh\\
Yenish\\
Yezidi\\
Yiddish
\end{multicols}
}\\
\lspbottomrule
\end{tabularx}
\caption{Languages covered by the European Charter for Regional or Minority Languages}
\label{tab:torres:1}
\end{table}

According to the Charter, some of these languages are to be protected in just one country, such as Skolt Sami in Finland, whereas others should be protected in several countries, such as Slovenian in Austria, Bosnia and Herzegovina, Croatia and Hungary. Beyond the Charter, there are other languages with different levels of recognition. For instance, Sardinia, an autonomous region of Italy, recognizes the Sardinian language as an official language, and Romansh Ladino, Cimbrian and Mocheno, spoken in certain communes of the mountainous North of Italy, also have local recognition. 

The European Charter for Regional or Minority Languages, however, guarantees the rights only of regional minority groups, and not of migrant groups. What’s more, the Charter has noteworthy absences, such as Breton, spoken in the North West of France, although a Breton language agency was created by the Region of Brittany in 2010 to promote daily use of the language.

\begin{sloppypar}
Multilingualism in Europe is also enhanced by immigration and mobility. There have been intra-European migrations, leading, for example, to Portuguese being spoken in Andorra and Polish in Ireland, alongside languages traditionally spoken outside the EU, such as Mandarin Chinese or Arabic. In the Multilingual Cities Project (\citealt{ExtraYagmur2005}), home language surveys amongst pupils both in primary and secondary schools were collected in Brussels, Hamburg, Lyon, Madrid, The Hague and Göteborg. The list of collected languages was the following: Romani, Turkish, Urdu, Armenian, Russian, Serbian/Croatian/Bos\-nian, Albanian, Vietnamese, Chinese, Arabic, Polish, Somali, Portuguese, Berber, Kurdish, Spanish, French, Italian, English, German. The authors of the study reached an obvious but provocative conclusion:
\end{sloppypar}

\begin{quote}
Amongst the major 20 languages in the participating cities, 10 languages are of European origin and 10 languages stem from abroad. These findings show that the traditional concept of language diversity in Europe should be reconsidered and extended. (\citealt{ExtraYagmur2005})
\end{quote}

But what is the “traditional concept” of language diversity? 


\subsection{The 24 official languages as a symbol of European linguistic diversity}\largerpage

The European Union considers its linguistic diversity a valuable asset. Article 22 of the Charter of Fundamental Rights of the European Union (The Member States, 2012) states that “[t]he Union shall respect cultural, religious and linguistic diversity.” However, member states have the exclusive right to define and recognize national and regional minority languages, and language policies are highly controversial. 


Meanwhile, the EU prides itself on standing up for language diversity through the use of the 24 official languages in the main EU institutions. From a practical point of view, this position involves a major challenge that deserves closer attention.\largerpage



For instance, in the European Parliament, parliamentary documents are published in all the official languages “as EU citizens must be able to read legislation affecting them in the language of their own country” \citep{EuropeanParliament2020} and members of the European Parliament have the right to speak and write in any of the official languages. Rule 167 of the Rules of Procedure of the European Parliament is related to languages, and specifies that: (i) all documents of Parliament shall be drawn up in the official languages; (ii) all members shall have the right to speak in Parliament in the official language of their choice; (iii) interpretation services shall be provided and (iv) the President of the Parliament shall rule on any alleged discrepancies between the different language versions \citep{EuropeanParliament2021}.



As for the citizens of the EU, according to the Treaty on the Functioning of the European Union (TFEU\footnote{The most recent, consolidated version of the TFEU is available from \url{https://eur-lex.europa.eu/legal-content/EN/TXT/PDF/?uri=CELEX:12012E/TXT\&from=EN}. Unless otherwise indicated, all urls mentioned in this chapter were more recently accessed in January 2022.}), all European citizens have the right to address the official EU institutions in any of the EU’s official languages and to receive an answer in that language. This is intended to make the EU institutions more democratic and accessible to EU citizens. Other provisions related to multilingualism in the TFEU are contained in articles 20, 24 and 342.


Some people think that 24 official languages is too many, and others that 24 official languages is not enough. Some countries try alternative approaches. For instance, Catalan, Euskara and Galician, all spoken in Spain, are considered “additional languages” by the EU  (they are co-official languages together with Spanish in their respective territories). This status means that any communication from an EU citizen in these languages has to be translated in Spain into a “procedural language” of the EU, and the answer from the EU institution will be also translated from the procedural language into the additional language. The cost of these translations is borne by Spain. 

The use of three procedural languages, English, French and German, is intended to simplify multilingual communication in the EU: given 24 official languages, the EU is faced with a total of 552 possible translation combinations, “since each language can be translated into 23 others” \citep{EuropeanParliament2020} and this would be difficult to handle for all EU documentation. For this reason, there are norms to establish which documents are translated into the other 23 languages and which are translated just into the three procedural languages. 

The European Commission's Directorate-General for Translation (DGT) translates texts for the institutions and the citizens of the EU. As of 2022 it produces more than 2.75 million translated pages per year, 91\% of which are translations from English, 2\% from French, just under 1\% from Spanish, and slightly less again from German. Other source languages combined account for around 5\% of translation activity. Of all translated documents, 63\% are translated internally by the DGT and 37\% are outsourced to external companies. Some 55\% of translations involve EU law-making, 22\% external communication and the web, 12\% communication with other EU institutions and national parliaments, 5\% correspondence with EU citizens, 4\% other official documents, and 2\% public consultation on EU policies. The translation budget for 2022 was 355 million euros, or 0.2\% of the whole EU budget \citep{DGT2022}.\largerpage[2]

\subsection{Machine translation enters the fray}

The DGT has an in-house staff of 2,000, between linguists and support staff, and works with several thousand selected external translators \citep{DGT2022}. The translations they produce, in all language combinations, are stored in the Euramis system (the EURopean Advanced Multilingual Information System), which includes, for instance, the \textit{Acquis Communautaire}, a corpus of the EU’s legislative documents in all 24 of its official languages \citep{JRC2022}. In order to increase productivity and reduce costs, the DGT has incorporated machine translation into some of its workflows, most recently using a system called eTranslation.\footnote{\url{https://ec.europa.eu/info/resources-partners/machine-translation-public-administrations-etranslation_en}} The use of eTranslation is expected to save time and money for the EU, but not only that. Eventually, as machine translation develops further, it could contribute to an increase not only in the number of documents that are translated and that otherwise would not have been considered for translation, but also, at some time in the future, to an expansion of the set of languages for which translation is available, and hence to a better reflection of European language diversity; an ideal that, without machine translation, would have been inconceivable just a few years ago. This might be the only way that indigenous or regional minority languages, as well as “non-territorial” or even immigrant minority languages, will gain representation in the EU institutions alongside official, national languages.

\subsection{What does multilingualism mean to the EU?}

Policies on multilingualism are a way of organizing the above-mentioned language diversity, and affirming its richness. In the EU, multilingualism is also seen as a means of social cohesion and worker mobility: “[l]anguage competences contribute to the mobility, employability and personal development of European citizens” \citep{CounciloftheEU2014}. 

A multilingual approach to linguistic policies aims at promoting languages not only in multilingual states but also within organizations. We may talk about multilingualism in many forms, including:

\begin{itemize}
\item  A multilingual policy, which is the policy of an organization, company or institution to use more than two languages for its internal and external communication.
\item A multilingual European Union, which means that different languages coexist in the EU.
\item A multilingual citizen, who has the capacity to use several languages.
\item A multilingual health system, which is a health system which incorporates linguistic diversity to improve health delivery to newly arrived people, for instance.
\end{itemize}

The main EU policies on multilingualism are reflected in a series of documents  \citep{EuropeanCouncil2002, EuropeanCommission2008, CounciloftheEU2008may, CounciloftheEU2008dec, CounciloftheEU2011, CounciloftheEU2014, EuropeanCouncil2017}.


In these EU policies and pronouncements on multilingualism, constant reference is made to language learning and, specifically, the “mother tongue plus two” policy, according to which citizens would learn “at least two foreign languages from a very early age” \citep{EuropeanCouncil2002}.  But would implementation of this policy make European citizens multilingual enough? From my point of view, there is a need to incorporate further elements into European policies on multilingualism.

Multilingualism is related to language policies, but that is not the full story. \citet{Cenoz2013}, who provides a wide spectrum of definitions of multilingualism, reminds us that multilingualism is multidimensional, involving, for example, the individual versus the social dimension, the proficiency versus use dimension, bilingualism versus multilingualism, etc. It can also be applied to geographical areas or social spheres. Moreover, it can be studied from different perspectives, including those of cognition, social construction, identity, language practices, multimodalities and technologies, among others. From the simplest definition in Wikipedia (“the use of more than one language, either by an individual speaker or by a group of speakers”) to the more complex multidimensional definitions provided by \citet{Cenoz2013}, in this chapter I will adopt the European Commission’s definition of multilingualism as “the ability of societies, institutions, groups and individuals to engage, on a regular basis, with more than one language in their day-to-day lives” \citep{EuropeanCommission2007}. The term \textit{engage} allows us to incorporate a useful nuance for the purposes of this book. By writing this chapter, I wish to invite readers to consider whether multilingualism can be defined from a technological perspective, adapting the above-mentioned definition used by the European Commission.

\begin{tblsframed}{Discussion topic}
Is there a “technological multilingualism”, understood as the ability of societies, institutions, groups and individuals to engage, on a regular basis, with more than one language in their day-to-day lives, through multilingual translation tools?
\end{tblsframed}

Interestingly, the above-mentioned report from the High Level Group on Multilingualism, which provided our definition of multilingualism, mentions “the potential of multilingual electronic tools as support for non-specialist users of second and third languages” \citep{EuropeanCommission2007} as a research area. Likewise, the European Commission's communication on “Multilingualism – an asset and a commitment” \citep{EuropeanCommission2014} claims that “the language gap in the EU can be narrowed through the media, new technologies and translation services”. This book aims to make a contribution precisely to this field.


\subsection{EU actions for linguistic diversity}

The EU’s webpage on linguistic diversity\footnote{\url{https://ec.europa.eu/education/policies/linguistic-diversity_en}} mentions the following initiatives to promote linguistic diversity: the European Day of Languages, Erasmus+ Mobility programmes, the European Capitals of Culture and the Creative Europe programme:

\begin{itemize}
\item Established by the Council of Europe in 2001, European Day of Languages\footnote{\url{https://edl.ecml.at}} takes place on September 26\textsuperscript{th} each year. On this day EU countries organize activities to promote linguistic diversity and the ability to speak other languages.
\item Erasmus + Mobility programmes. Between 2014 and 2020, €14.7 billion was assigned to more than 4 million mobility grants, 2 million of which were designated for university students.
\item The European Capitals of Culture is an initiative to highlight the diversity of cultures in Europe, including linguistic diversity. For instance, in 2020, European Capitals of Culture (and corresponding languages) were Rijeka (Croatian) and Galway (Irish and English).
\item Creative Europe.\footnote{\url{http://www.creativeeuropeuk.eu/funding-opportunities/literary-translation-0}} This European Commission framework programme supports the culture and audiovisual sectors, including literary translation. Specifically, it funds the translation of literary work from one European language to another.
\item Another interesting initiative to promote linguistic diversity is the “European Language Label”\footnote{\url{https://ec.europa.eu/education/initiatives/label/label_public/index.cfm}} attached to EU funded projects. Although most of these projects are oriented towards language learning, some specifically target language diversity.
\end{itemize}

\begin{tblsframed}{Discussion topic}
From the current perspective, machine translation is not present enough in the EU discourse of language diversity and multilingualism. How could machine translation be included in language learning projects?
\end{tblsframed}

Including machine translation in multilingualism initiatives would allow us to increase the number of languages European citizens could become familiar with. It would also allow citizens to approach unknown languages with curiosity and without fear, to access unfamiliar language environments more easily and to respect local languages. Moreover, MT could be used to support reading comprehension in any unlearned languages.


\subsection{Multilingualism and language learning in the EU}

According to the Council of Europe’s Language Policy Portal:\footnote{\url{https://www.coe.int/en/web/language-policy/home}}

\begin{quote}language learners/users lie at the heart of the work of the Language Policy Programme. Whatever their status, all languages are covered: foreign languages, major languages of schooling, languages spoken in the family and minority or regional languages, as well as a specific programme on the linguistic integration of migrants and refugees.\end{quote}

\hspace*{-1mm}Initiatives to foster multilingualism are many and varied, but language learning deserves closer attention, especially given the EU’s above-mentioned “mother tongue plus two” policy.

Some of the EU’s recent initiatives to improve language skills include the European Centre for Modern Languages (\url{www.ecml.at}; the Eurydice Report \citep{Eurydice2019}, which provides information on policy efforts in Europe that support the teaching of regional or minority languages in schools; the Online Linguistic Support (OLS) platform;\footnote{\url{https://erasmusplusols.eu/en/about-ols/}} the Common European Framework of Reference for Languages (CEFR); and Erasmus+ mobility programmes.

European projects funded to improve language learning deserve special attention. Methodologies, languages and countries involved vary enormously from one project to another. \tabref{tab:torres:2} lists some interesting examples.


\begin{table}
\small
\begin{tabularx}{\textwidth}{Qp{2.5cm}Q}
\lsptoprule
{Project} & {Languages} & {Specific features}\\
\midrule
\href{https://www.itongue.eu/}{{iTongue: Our Multilingual Future (2013)}} & Not specified & Paralinguistic digital tools for foreign language learning\\
\tablevspace
\href{http://medlang.eu/videos.php}{{Massive open online courses with videos for palliative clinical field and intercultural and multilingual medical communication (2014)}} & DT, EN, FR, IT, ES, RO & 20 multilingual fundamental palliative medicine procedures\\
\tablevspace
\href{https://languages4work.eu/}{{Crafting Employability Strategies for HE Students of Languages in Europe (2015).}} & Not specified & Embedding employability within language teaching\\
\tablevspace
\href{http://www.mooc2move.eu/}{{LMOOCs for university students on the move (2018)}} & FR, ES & Open educational resources for university students\\
\tablevspace
\href{http://stratapp.eu/}{{Gamifiying Academic English Skills in Higher Education: Reading Academic English App (2016).}} & Not specified & Game-based application to improve English academic reading skills of university students\\
\tablevspace
\href{http://elengua.usal.es/}{{E-LENGUA: E-Learning Novelties towards the Goal of a Universal Acquisition of Foreign and Second Languages (2015).}} & EN, AR, ES, FR, DE, IT,  PT & Best practices of the integration of digital competences into the teaching of languages\\
\tablevspace
\href{https://eulaliaproject.eu/es/}{{EULALIA: Enhancing University Language courses with an App powered by game-based Learning and tangible user Interfaces Activities (2019).}} & IT, PO, ES, MT & Inclusive learning tools based on the paradigm of Mobile Learning and Game-Based Learning methodology and the application of Tangible User Interfaces (TUIs)\\
\tablevspace
\lspbottomrule
\end{tabularx}
\caption{Examples of European projects focused on language learning}
\label{tab:torres:2}
\end{table}

Eurostat, the website for European Statistics,\footnote{\url{https://appsso.eurostat.ec.europa.eu}} provides statistics on the second and foreign languages studied by pupils at different education levels in the EU. According to Eurostat data from 2019, English was by far the most popular language at lower secondary level, studied by nearly 86.8\% of pupils, followed by French (19.4\%), German (18.3\%) and Spanish (17.5\%) \citep{Eurostat2022}.

\begin{tblsframed}{Discussion topic}
In second language learning in Europe, is “multilingual” a euphemism for “English-speaking”?
\end{tblsframed}

Another interesting question is how many students learn two or more foreign languages, as recommended by the \citet{EuropeanCouncil2002}: it is known that 89.9\% (almost 14 million) of secondary level pupils studied more than one foreign language in 2019 \citep{Eurostat2022}. Among them, more than 7 million (48.1\%) studied two or more foreign languages. 

In short, of the 600 languages spoken around Europe by more than 700 million speakers (EU and non-EU) \citep{WorldBank2020}, the majority of EU students are learning one or two out of the following four as their first or subsequent foreign language: English, French, German or Spanish. 

\subsection{Levels of attainment}

The European Survey on Language Competences \citep{EuropeanCommission2019} tested 54,000 secondary pupils (aged 14-15) in 16 educational systems, and covered the two most widely taught foreign languages in all concerned education systems. The survey tested writing, reading and listening comprehension. It did not test oral expression. The key finding of the survey was that only 42\% of tested students reached the level of independent user (B1 and B2 in the Common European Framework of Reference for Languages) in their first foreign language, and only 25\% reached this level in their second foreign language. Moreover, a large number of pupils did not even achieve the level of a basic user: 14\% failed to achieve this level for their first foreign language and 20\% failed to achieve it for their second foreign language \citep{EuropeanCommission2019}.

In the same report, there are data from a 2018 flash Eurobarometer survey among 15-30 year-olds, where 85\% of respondents stated that they wished to improve their proficiency in a language they had already learned (mainly English): 

\begin{quote}
This indicates that the survey respondents were not satisfied with the level they achieved at the end of compulsory education or they did not have a chance to maintain their level. One third of surveyed young Europeans said they were unable to study in a language other than the one they used in school (i.e. often the mother tongue). \citep[102]{EuropeanCommission2019}  
\end{quote}


%%[Warning: Draw object ignored]

\subsection{Is there a role for machine translation in language learning?}

We have already seen that language-learning efforts in the EU tend to be concentrated on a small number of large languages, and that learners do not always reach desired levels of competence in their chosen foreign languages. These circumstances suggest that further support for language learning is needed, and it behoves us to investigate whether such support could come in the form of machine translation. As neural machine translation learns faster than any foreign language learner, it could, in theory, be used to help learners read complex texts and develop more advanced written skills in their second language. They could learn how to make the most of machine translation in the second language so that they could detect and edit machine translation mistakes based on their knowledge of the second language. And while empirical studies of the use of machine translation in language classes are still thin on the ground, a small number of sources suggest interesting avenues of research. Relevant studies are discussed in \textcitetv{chapters/carre}, which includes further ideas and strategies that can be used in language learning classes. Yet others are included in the database of activities of the MultiTraiNMT project \citep{MultiTraiNMT2020}. My view is that there are many ways of using machine translation in language learning classes and there is no need to forbid its use if it is used in a conscious and critical way.

On occasion, however, there is just no time to train second language students. Indeed, in the history of machine translation there have been many occasions on which research was partially triggered by a perceived lack of people learning a particular foreign language. Cold War research into Russian-English machine translation is one such case \citep{Gordin2016}. More recently, the Japanese organizers of the Tokyo 2020 Olympic Games (actually held in 2021 due to COVID-19) realized that learning Japanese was out of the question for most foreigners and that they needed a faster approach to overcome language barriers during the Olympics. The Japanese internal affairs ministry thus allocated ¥1.38 billion to machine translation research to improve the quality of real-time speech translation technology, with the aim of covering 90\% of the language needs of the Olympic teams and tourists who, it was hoped, would go to Japan \citep{Murai2015}. The Japanese government funded the research for a specific machine translation system to be used during Tokyo Olympics and private companies were tasked with the development of devices and mobile apps to run the system. The plan was that companies would recover their investment by selling the devices and apps subscriptions to users. In this case, the introduction of machine translation was the chosen shortcut to bring multilingualism to Japan, instead of language learning.

\section{Case study: Multilingual universities}
\subsection{Internationalization and multilingualism}

The European Commission’s communication on “European higher education in the world” establishes the key priorities for the internationalization of European universities focused on mobility, digital learning, and the strengthening of strategic cooperation. Regarding languages, the communication notes that:

\begin{quote}
proficiency in English is de facto part of any internationalization strategy for learners, teachers and institutions and some Member States have introduced, or are introducing, targeted courses in English (especially at Masters level) as part of their strategy to attract talent which would otherwise not come to Europe. \citep{EuropeanCommission2013}
\end{quote}
    
\begin{sloppypar}
At the same time, multilingualism is a significant European asset: it is highly valued by international students and should be encouraged in teaching and research throughout the higher education curriculum \citep{EuropeanCommission2013}.
\end{sloppypar}

Indeed, the EU remains committed to multilingualism on university campuses: firstly, because multilingual campuses reflect European linguistic diversity; secondly, because they provide students with more mobility and employment opportunities; and thirdly, because they promote contact with different cultures and learning approaches. In a similar vein, \citet{Gao2019} lists different reasons for universities to engage in internationalization, including the fact that internationalization can help prepare students to interact with people from different cultures as a way to create cultural understanding and reduce mutual hostility between countries.  Internationalization poses challenges for universities however (ibid.), not least those related to multilingualism. Firstly, a lack of translation resources (including human resources) can prevent a university from becoming fully English-speaking. Secondly, internationalization can involve the displacement of native languages and a loss of language diversity. I see a role, however, for machine translation in counteracting these dangers. And the technology seems particularly promising, given shifting understandings of internationalization: traditionally, universities have developed plans to do “internationalization at home” (to attract foreign students), and “internationalization abroad” (to send students abroad). \citet{MittelmeierRaghuram2020} incorporate the concept of “internationalization at a distance” to develop online international distance learning models for campus-based institutions. The COVID-19 pandemic has undoubtedly given extra impetus to this third way. Technologies may change the way internationalization is conceived and machine translation is a technology that has potential to contribute to the internationalization game.

\subsection{English as Lingua Franca (ELF), Local Languages (LLs) and machine translation}

Universities’ internationalization strategies are numerous, but mobility and running English or bilingual programmes are possibly the most visible ones.  The latter strategy goes under different names, for example English as Lingua Franca (ELF), Englishization, and English-medium instruction (EMI). It has been defined as the use of “English as medium of instruction in institutions of higher education in non-English speaking countries” \citep{MultilingualHigherEducation2016}.

The decision to use English allows for attendance by international students. However, local students who are not fluent in English may feel betrayed if their local university worries more about international students than about them, especially if it is for economic reasons (as international students’ fees are higher than local students’ fees). Some universities opt to deliver MOOCs in English as a strategy to attract international students, but recent research suggests that the impact of MOOCs on international student enrolment is still minimal \citep{Zakharova2019}.

As a way to overcome the tensions between English as Lingua Franca and local languages, \citet{House2003} distinguishes between “languages for communication” and “languages for identification”. According to \citet[560]{House2003}, languages for communication are instrumental in enabling communication with others who do not speak one’s own L1. Languages for identification, on the other hand, are

\begin{quote}
 {normally local languages, and particularly an individual’s L1(s), which are likely to be the main determinants of identity, which means holding a stake in the collective linguistic-cultural capital that defines the L1 group and its members …  and the type of affective-emotive quality involved in identification. \citep[561]{House2003}}
\end{quote}

\begin{sloppypar}
Under this approach, English would be used for communication between speakers who do not share the same language and for “pockets of expertise” \citep[561]{House2003} and languages for identification would be used between same-language speakers. 
\end{sloppypar}

\citet{House2003} presents a case study involving the use of English as a medium of instruction at Hamburg University (Germany), in which she examines how English interacts with the local language, and how international students perceive, and react to this “diglossic situation” \citep[570]{House2003}. Results showed that English was not seen as being in competition with German. English was described as being in “a class of its own”, a supranational, auxiliary means of communication. In this project “there were no signs (yet) of a threat to a native language (German) and to multilingualism” \citep[574]{House2003} and international students were invited to learn German during the academic year.

\subsection{A truly multilingual university}

Current multilingual universities are universities which include EMI for international students, and/or universities in borderlands and/or areas with more than two official languages. Such universities may be multilingual for historical, political, geographical, economic, or other reasons, and finding the balance in their language policy may be a challenge for them. Neither internationalization nor multilingual policies can be improvised.  \citet[12]{Knight1994} proposed a six-phase model of the process of institutional internationalization: 

\begin{itemize}
\item Awareness of need, purpose, and benefits 
\item Commitment by all university actors 
\item Planning: Identify resources, purposes and objectives, priorities and strategies 
\item Operationalize: Develop academic activities and services 
\item Review: Assess and enhance the quality, impact, and progress and 
\item Reinforcement: Develop incentives, recognition, and rewards.
\end{itemize}

Related specifically to multilingualism, this plan would require us to answer questions like: will EMI be restricted to international students/courses? Can a university be considered multilingual if EMI is used only for international students? Will international students be considered “multilingual” if they use only English in the “multilingual” university? Is it enough for a university to have its website in its official languages and English but deliver courses in local languages? Are courses to be delivered using more than one language or will instruction take place in just one language? Could the native languages of students (which may not be the official languages of the university) be incorporated into the courses? Is the language of instruction the only parameter to identify a “multilingual university”? In which languages are training materials to be offered? Will language proficiency be assessed together with non-linguistic content? How can local students be prepared for EMI? Does being multilingual mean using EMI? Does being multilingual mean using local languages and English? If so, in what proportion? How many multilingual strategies are enough to become a multilingual university? Which languages are needed to guarantee further integration of the university student into the geographical region or local economy? Will international students live in English-only “bubbles”?

On a truly multilingual campus, universities could welcome languages from international students as well as languages from socially or culturally marginalized groups. Many local languages have been historically undervalued in academia, with doubts expressed, for example about the extent of their academic vocabulary. An extreme case in point is that of Quechua: the first doctoral thesis fully written in Quechua was defended in 2019, some 468 years after the first university was established in Peru.\footnote{\url{https://www.theguardian.com/world/2019/oct/27/peru-student-roxana-quispe-collantes-thesis-inca-language-quechua}}

The e-course \textit{Multilingualism and plurilingualism in education} developed as part of the Erasmus+ project \citet{MultilingualHigherEducation2016}, describes the language policies for different multilingual universities. For instance, at the University of Fribourg (Switzerland), there are courses offered in French and German. At the University of Helsinki (Finland), there are courses offered in Finnish, Swedish and English. At the Free University of Bozen-Bolzano, subjects are taught in German, Italian and Ladin, and in English as a lingua franca. At the University of Luxembourg, at least 20\% of all courses should be taught in the three languages of instruction – French, English and German. 


\citet[89]{Gao2019} proposes measurement dimensions and indicators to distinguish between strategic aspiration and reality when it comes to internationalization. Her proposals have been adapted below specifically to multilingualism, our field of interest.

Under the dimension of university governance, actions promoting multilingualism could include:
(i) A supportive multilingual policy framework/organiza\-tion\-al structure;
(ii) A languages office/translation services;
(iii) Machine translation infrastructure;
(iv) Multilingual presence/signage at the university;
(v) Development of multilingual awareness and skills among staff;
(vi) Budget for multilingualism initiatives; and
(vii) Monitoring/evaluation systems for multilingual performance.

From an academic perspective, actions promoting multilingualism could include:
(i) Multilingual courses (why does a course have to be in one and only one language?);
(ii) Multilingual teaching, normalizing multilingual classes;
(iii) Multilingual research and multilingual conferences;
(iv) Multilingual students in the class, involving interaction between international and local students; and
(v) Multilingual visiting scholars;
(vi) Multilingual curricula;
(vii) Multilingual research journals;
(viii) Multilingual extracurricular activities and
(ix) Cultural diversity visibility.

And last, but not least, the university could provide multilingual orientation programmes, multilingual support and multilingual libraries. 


%%[Warning: Draw object ignored]

\subsection{Ideas for using machine translation in teaching, research and administration in multilingual universities}

Multilingual universities could employ machine translation systems capable of translating as much as possible between English and local languages. One strategy might be using free online machine translation services, but the users should be aware of machine translation mistakes and be able to handle them properly, either by correcting them themselves or asking for professional post-editing services. Further information on post-editing can be found in \textcitetv{chapters/obrien}. 

Another strategy would be to develop university-customized quality machine translation resources between local languages and English, as explained further in \textcitetv{chapters/ramirez}. If resources allowed, a customized machine translation system could be shared between universities in Europe and beyond.

In the truly multilingual campuses of the future: 

\begin{itemize}
\item International students could mix with local students, and have machine translation resources available to follow the class in any local language as there would be: (i) teaching materials in different languages (assuming copyright issues have been resolved); (ii) access to multilingual glossaries and databases for specialized terminology; (iii) a recording of the class using available voice recognition, transcription and machine translation features, etc.
\end{itemize}
\begin{itemize}
\item Students would learn post-editing skills to review machine translation results, either in English and/or other languages, in order to be able to use machine translation output wisely.
\item Universities would provide post-editing services for good-quality teaching guides and teaching materials.
\item Multilingual research dissemination and multilingual publications would be encouraged, providing embedded machine translation features.
\item Rather than struggling through talks and conference papers in faltering English, visiting professors who wish to would express themselves in their native languages and translation/interpreting services would be provided, either human (if there is funding), or machine.
\item Multilingual activities would be organized on campus, in various fields such as music, theatre, cuisine, politics, literature, solidarity, social service, etc.
\end{itemize}

\begin{sloppypar}
In a truly multilingual university, local languages, English, and other languages brought by mobility students should be able to coexist. This strategy would help international students integrate into a multilingual environment. 
\end{sloppypar}

\section{Conclusion}\largerpage

This chapter has championed the idea of machine translation as a tool to foster multilingualism in Europe. As seen in the chapter, the EU has published charters, treaties and parliamentary documents promoting multilingualism as a core value in Europe that has to be fostered and preserved. However, despite all efforts and resources put into language learning, the goal of learning one’s “mother tongue plus two” is difficult to reach. On the one hand, in practice, most EU citizens are learning only English as a foreign language. On the other hand, the learning curve in language learning is long and slow. In this context, machine translation seems to offer some support to those who do not have the time or resources to keep learning more and more languages.

The chapter also explores the case of universities as small multilingual communities who can design language policies that promote multilingualism. Language policies may generate tensions on campuses for a number of reasons, but most campuses are multilingual in practice nowadays, either through the internationalization/Englishization of the university or due to the arrival of foreign students. In this context, the chapter explores the need to design language policies that acknowledge the potential of machine translation to facilitate multilingualism, without forgetting the challenges that machine translation presents, especially those related to quality and ethics. As we say in Spanish, my aim here is to \textit{abrir el melón} (literally to ‘open the melon’) of machine translation in multilingualism and language learning. Opening the melon means tackling a question that needs to be dealt with sooner or later, although nobody wants to do it because the consequences are unknown. In other words, nobody knows if the melon will be sweet enough to eat, but there is only one way to find out. Even if existing machine translation systems do not communicate the non-literal meaning of \textit{abrir el melón}, anyone reading a literal machine translation will still learn a useful Spanish metaphor. And who knows? This metaphor may even travel to new languages and cultures, as it allows a long and complex meaning to be conveyed in just three words. This is multilingualism in action.\largerpage

\printbibliography[heading=subbibliography,notkeyword=this]

\end{document}
