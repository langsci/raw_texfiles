This is a story about going to Kowo river with brother Bruno Kakut. As I came back from school one day, I sat down and asked him: ``Hey brother. How about going to Kowo river?'' Bruno agreed and I explained the plan: ``Brother, we will go. Going to Kowo river, we can stay at Widi river. Next we will stay in Udeben. And from there, we go to the estuary of the Kilok Kone river. We stay there a little while and from there we take the same route in the opposite direction. We stay for five days and come down to Udeben again for five days. After five more nights, we come to Widi river and spend only one night there and then we will go home to the village'' I laid it all out to him and my brother agreed: ``Brother, we will do like that.''

So we prepared the trip and first packed the tobacco. From Lampion we took three packs.\footnote{\textit{Lampion} and \textit{Surya} are popular brands for cigarettes in Indonesia.} From Surya we took five packs. And then the coconuts. We took five of them. Then salt, vetsin\footnote{\textit{Vetsin} = Monosodium glutamate.}, a small cooking pot, we took things like that. Then we told our parents. ``Parents, we want to go to Kowo river'', we said. ``And we will go from Widi river to the estuary of Kilok Kone river. From the estuary of Kilok Kone river, we go on like that and come back to Widi river. Eventually we will come back.'' We said that to our parents. ``Yes, sons, you can travel like that, but there is one prohibition: Do not separate! Stay together.'' That was our parents advice as we took off.

First, we went to the mouth of Widi river and stayed there for five nights. Then we packed our things in the early morning and rowed our boat upstream to Udeben. And we stayed in Udeben. But why did we go to Kowo river in the first place? It was for fishing. We needed to get fish to eat. So we ate the fish, those from Widi river. We came to stay in Udeben and we fished and we also ate them all there. We ate all we could catch for five days, then we went upstream to the estuary of Kilok Kone. As we arrived there, we tied the canoe safely and I said: ``Brother! It is difficult catching the little shrimps here, so we'll try it with shrimp traps. We need three traps. We make them and then go to put one into the water down there, put one into the water up there, and put the third one into the water in the middle. We just use the traps, think about that. It is a stony place, this is a good spot.'' Then he went to cut some sago leaves and brought them, so that I could make the traps. As I finished the first one, I showed it to him: ``You too can make one like this.'' As I was teaching him, he tried to make one: ``Brother, like this?'' - ``Yes, you go on like that, it's like the one I made here.'' But he struggled to make even one. He struggled because it was new for him, so I set one first and he set one after that. We finished preparing and then we took some termites. We chopped them quickly and cut some leaves around there. We took them into the shrimp trap, we split the termites and put them into the shrimp trap. Then we put some stones into it and put the traps into the water.

When we walked around and we saw that some small shrimps must have gone into a trap. That's why we went to take the traps out. As we took a big one out, we saw that shrimps were inside. We took them all and went to another trap. At the other trap, we pulled it out, took the shrimps and put them in a cooking pot. Directly into the cooking pot. Then we went on to another trap. There were also many little shrimps. 

There were the fishing hooks also. So as for the fishing lines, it was like that: I had one fishing line and my brother had two fishing lines. Since we only had long fishing lines, we just threw them and fished. And indeed, we caught some fish. You must know: they call me ``the fishhook'' because that is my thing. We caught those fish and bound them with a string and then put them in various places. We brought a big fish here, bound one over there, bound another one over there, bound them with nylon or with rattan, we just put them everywhere. This way we could later take one out for lunch or for dinner. That's how we keep the fish fresh.

This all happened on the first day. We still had four more days and five nights. On the fifth day, we prepared to leave. When a fish is almost dead, it is usually taken, split and stored. But some were still alive at the end of our stay. My brother said: ``Oh, about those fish! We will go to Udeben. It is not near enough to go and come back regularly for the fish.'' When we were just about to go, I said: ``Brother! Actually, these fish are enough! Let's go directly from here back to Widi river instead of staying in Udeben. We go to Widi river and we take these fish to our parent's village first'', I said. - And he said: ``Well brother, it's up to you. I ate a lot of the fish we chose to eat and I also have a headache. I feel queasy and almost vomited, so let us take the fish to our parents first.'' So we packed out things the next day and rowed the canoe back. From Widi river, we wanted to go to the village. Approaching the estuary of Inggo river, we already saw: ``Oh, there is smoke.'' Since there was smoke, he thought: ``Brother, maybe the parents came to the river.'' - ``Yes, it must be them'', I answered. We went on like that and finally came to the mouth of Widi river. As we arrive, we tied the rope of the canoe. It was done by me, since I was at the back. The other guy got up and was standing, I also stood up and the canoe almost flipped over. The canoe rocked, so I went off first. When I went off and hit the canoe with a paddle, the parents heard it up there at the house. They heard it and his mother came down. She thought this might be the children and came down to have a look. That woman came down and saw us: ``It's the children who went to Kowo river and to the Kilok Kone estuary'', she said. ``Oh! Yes. Madam, we have brought our catch. Please go tell the other women up there, then come down to get it'', he said. They agreed and did as he said. The woman up there called another woman and they both came back. It was Mrs. Monika and my mother Tekla. So they came and arranged everything. They saw that the canoe was full of fish. There was a sack full of fish. ``Oh! Son, that is enough'', they said and took some of them. 

We threw those fish that were still alive back into the river. We had already prepared strings, so we threw them in the river. They stayed exactly where we put them. Then we took the cloved fish and went home where they arranged everything. ``Oh! We cannot eat the fish all by ourselves, they are too many. But all our friends are at work. We are here at Widi river and we should share the fish with out friends'', they said. But we replied: ``Oh, mom! We came here just for a short moment, and we want to go back upstream. We were going to Udeben but we saw these fish here when we were passing, so we did not go there directly. We wanted to go to the village, then we came here to Widi river. So you have to eat the fish now, we want to go back immediately'', we said. ``You mean these fish?'', they asked. - ``Oh! That is what we brought. We want to go back.'' - ``But what about your food?'' - ''For now we have food, we can eat sago'', we said and added: ''If you have rice, maybe, please give us some of it.'' - ``There is just a little river. But then again it is only two of you, so ... okay. You can take one bag of rice'', they said and they gave it to us. ``Is that all? What about tobacco? Do you have tobacco, salt and vetsin?'' - ``Yes, we still have some.''

We brought the coconuts back without using them, so they told us to leave them at Widi river. But we just left two of them and we took tree with us. We took them to Udeben and stayed there. Additionally, we brought the shrimp traps. We had told to the parents: ``We will go for five nights.'' So we went there and stayed there to go fishing from morning until evening when it would become dark. When it was not raining, we would go down to the river and stay there. The fish would get caught in traps when they get the baits and then we would start to tie new ones. We kept on doing that until the day breaks.

As for the fish, it is usually organised in a certain way. The big ones are separated from the others. They are tied on strings. So we tied two middle sized fish at one string, three middle sized fish at another string, and so on. We tied each and every one. As for the food, there was enough of it. So nothing to worry about, and tobacco as well. There was only the two of us. We had nothing else to do, our business was catching fish, that is all. As for firewood, it was in the house where we were staying. It belonged to a man from Waewok. So no problem, the firewood was still there. There was nothing to worry about. We stayed there for five nights and when it was time, then we took the fish. Then one time we noticed: ``Oh, we brought this coconut without eating it, taking it all the way with us. We could split one'', we said. There was a coconut grater in the house. This forest house as it happens had a coconut grater. So we split the coconut, took the parts and grated it. Then we squeezed it and we did all that. The Alok fish we had caught was very big. We used to fish for that before but we never caught a fish of that kind. So now we had this very big fish. ``Brother, we have coconut milk'', he said. ``This stuff is great to cook with.'' So that's what we did. We ate so much that we spent the whole night only eating until the day broke. 

The next day we went to take the fishing lines, then we came back to the house and packed our things. Finally, we took off, untied the canoe and went to the place where we had put the fish. We untied the fish and put them in the canoe, then moved further to untie some more fish and put them in the canoe. We had to do this because the fish were put at a different places. Not only in one place. As we finished this, we quickly rowed downstream, heading to our place. We went down the Widi river, where we had told our parents to wait. We had made an appointment with them. ``They should have stayed there'', we hoped and soon we should see that we were right. They were still there, we saw the smoke already. So we rowed down the Widi river and tied the rope of the canoe. This time brother Kakut knocked at the side of the canoe. As he did it, mother knew: ``Oh, that must be the children.'' They came to stand on the top of a hill looking down. And they spotted us. ``Yes, the children are here'', she said. And we said: ``Of course, mother, we came. We had an appointment and we wouldn't miss it! That's enough, our trip is done. That's why we came down here.'' And then we said: ``It is as usual. Please come and take the fish first.'' Then we came back up to house, while the women organised everything. They collected the fish and brought them up. Some of the fish were brought to eat, but others were tied and put in the river. They said: ``We cannot eat all ourselves. The older sister and the brother-in-law live next door. Also, our grandchild lives here. Grandchild Omnumun, that is our grandchild.'' 

And that's just how it is. They live at Widi river and share everything with their relatives. Oh, it's just how it is, over there. The women are the one's who share. As for us, we do not know how to behave properly. We only know how to eat, that's it. The man would go out for food, bring it and eat it. But the parents used to say like this: ``Son, whatever things there are, be it meat or vegetables, whatever things there are. When you have plenty of it, it must be shared with friends. It is also for the friends, so we may not eat all by ourselves.'' So one must take it like that to friends, relatives, brother-in-laws, and to uncles and cousins. ``Always share like that'', the older people used to say. So we could not hesitate: ``Shall we bring them the fish or not?'' Because they were family. ``Just do it!'' That's what they said and they acted accordingly and after we finished eating, we said: ``Mom! For us, our time here is over. So can we go to the village?'' - ``No, not today. We run some errands tomorrow, you have to carry peanuts then we will go to the village'', they said. So we agreed and that's what we did.

We stayed there for the night and the next day we packed everything. As for the peanuts, they just entrusted me with them. The brother was younger then me, so he could not take them. He just carried some stones for his slingshot. That's how we would walk back to the village. I carried the peanuts on my shoulders and my younger brother decided the moment of our departure. ``Oh brother! This is a good time, can we go?'' So we went off.