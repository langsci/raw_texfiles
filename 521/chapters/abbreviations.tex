\addchap{\lsAbbreviationsTitle}
% \addchap{Abbreviations and symbols}
The category labels for abbreviations follow the Leipzig Glossing Rules.\footnote{\url{http://www.eva.mpg.de/lingua/resources/glossing-rules.php}}
\vspace{.5cm}

\begin{tabularx}{.45\textwidth}{lQ}
    ? & unknown element \\
    \textsc{1} & first person \\
    \textsc{2} & second person \\
    \textsc{3} & third person \\
    \textsc{adv} & adverbaliser \\
    \textsc{ass.pl} & associative plural \\
    \textsc{aux} & auxiliary \\
    \textsc{bi} & Bahasa Indonesia \\
    \textsc{caus} & causative \\
   \textsc{com} & comitative \\
    \textsc{conj} & conjunction \\
    \textsc{cont} & continuative \\
    \textsc{contr} & contrastive \\
    \textsc{cop} & copula \\
    \textsc{dem} & demonstrative \\
    \textsc{dir} & directive \\
    \textsc{ds} & different subject \\
    \textsc{dur} & durative \\
    \textsc{emph} & emphatic \\
    \textsc{ep} & epenthetic \\
    \textsc{epn} & epenthetic nasal \\
    \textsc{epv} & epenthetic vowel \\
    \textsc{exp} & experiencer \\
    \textsc{f} & feminine \\
    \textsc{imp} & imperative \\
    \textsc{inf} & infinitive \\
    \textsc{intj} & interjection \\
    \textsc{inv} & invisible \\
    \end{tabularx}
    \begin{tabularx}{.45\textwidth}{lQ}
    \textsc{irr} & irrealis \\
    \textsc{hes} & hesitation \\
    \textsc{lnk} & linker \\
    \textsc{lv} & light verb \\
    \textsc{m} & masculine \\
    \textsc{mod} & modal \\
    \textsc{n} & n-participle \\
    \textsc{neg} & negation \\
    \textsc{obl} & oblique \\
    \textsc{onom} & onomatopoeia \\
    \textsc{pfv} & perfective \\
    \textsc{pl.o} & plural object (=verbal number) \\
    \textsc{pn} & proper noun \\
    \textsc{poss} & possessive \\
    \textsc{prohib} & prohibitive \\
    \textsc{ptc} & particle \\
    \textsc{q} & question \\
    \textsc{quot} & quotative \\
    \textsc{rcv}& receiver \\
    \textsc{red} & reduplication \\
    \textsc{sbjv} & subjunctive \\
    \textsc{seq} & sequential in order \\
    \textsc{sg.o} & singular object (=verbal number) \\
    \textsc{sm} & serial marker \\
    \textsc{spec} & specific \\
    \textsc{ss} & same subject \\
    \textsc{vblz} & verbalizer \\
    %\textsc{unspc} & unspecific
\end{tabularx}
