% Muyu040
\ea\label{ex:text10-167}
ta ma edo ne ambo Lukas yom\\
\gll ta	ma	edo	ne	ambo	Lukas	yom\\
     and	but	\textsc{dem}.this	1\textsc{sg}	older\_brother	\textsc{pn}	\textsc{com}\\
\glt `But this one is about brother Lukas and me.'
\z

\ea\label{ex:text10-43}
oye otbop\\
\gll oye	otbop\\
     \textsc{poss}.\textsc{dem}	language\\
\glt `Its story.'
\z

\ea\label{ex:text10-1242}
tinggan Muying a amun wa tien go\\
\gll tinggan	muying	=a	amun	wa	ti-en	=go\\
     rat	kind\_of\_mouse	=\textsc{lnk}	nest	build\_nest	stay-3\textsc{sg}.\textsc{m}	=\textsc{ptc}\\
\glt `A Muying rat that made a nest to stay in and ...'
\z

\ea\label{ex:text10-1243}
amun wa tiun go\\
\gll amun	wa	ti-un	=go\\
     nest	build\_nest	stay-3\textsc{sg}.\textsc{f}	=\textsc{ptc}\\
\glt `it made a nest to stay in ...'
\z

\ea\label{ex:text10-44}
wene ninggan go kido ne tinggi tobun ye otbop\\
\gll wen-e	ningg-an	=go	kido	ne	tinggi	tob-un	ye	otbop\\
     go-\textsc{sm}	grab:\textsc{sg}.\textsc{o}-1\textsc{sg}	=\textsc{ptc}	down	1\textsc{sg}	hand	bite-3\textsc{sg}.\textsc{f}	3\textsc{sg}.\textsc{m}	story\\
\glt `then I cought it but it bit my hands, that's the story.'
\z

\ea\label{ex:text10-45}
Wolok kaba bet aliodonup a\\
\gll Wolok	kaba	=bet	alio-d-o-n-up	=a\\
     Wolok	estuary	=\textsc{obl}	stay-\textsc{dur}-\textsc{ep}-\textsc{ss}-1\textsc{pl}	=\textsc{lnk}\\
\glt `We stayed at the Wolok estuary and ...'
\z

\ea\label{ex:text10-46}
ambo Lukas odo ye kacang alumen\\
\gll ambo	Lukas	odo	ye	kacang	alum-en\\
     older\_brother	\textsc{pn}	\textsc{dem}	3\textsc{sg}.\textsc{m}	peanut(\textsc{bi})	plant:\textsc{pl}.\textsc{o}-3\textsc{sg}.\textsc{m}.\textsc{ipfv}\\
\glt `brother Lukas planted his peanuts.'
\z

\ea\label{ex:text10-47}
kacang petak talep yanop okune alumen\\
\gll kacang	petak	talep	yanop	okun-e	alum-en\\
     peanut(\textsc{bi})	patch(\textsc{bi})	big	there\_is	like\_that-\textsc{sm}	plant:\textsc{pl}.\textsc{o}-3\textsc{sg}.\textsc{m}.\textsc{ipfv}\\
\glt `It was a very big patch of peanuts that he planted.'
\z

\ea\label{ex:text10-1278}
alumogole\\
\gll alumo-gol-e\\
     plant:\textsc{pl}.\textsc{o}-\textsc{ss}.\textsc{seq}-3\textsc{sg}.\textsc{m}\\
\glt `He planted them and he ...'
\z

\ea\label{ex:text10-48}
wene wene ``kacang edo tua kele wonon yo'' enggenet\\
\gll wen-e	wen-e	kacang	edo	tua	kel-e	won-on	=yo	engge-n-e	=t\\
     go-\textsc{sm}	go-\textsc{sm}	peanut(\textsc{bi})	\textsc{dem}.this	old(\textsc{bi})	become-\textsc{sm}	go-3\textsc{sg}.\textsc{m}.\textsc{pfv}	=\textsc{quot}	say-\textsc{ss}3\textsc{sg}.\textsc{m}	=and\_then.\textsc{ss}\\
\glt `went on until ``The peanuts are becoming old already'', he said and then ...'
\z

\ea\label{ex:text10-1331}
badaen\\
\gll bada-en\\
     pull\_out-3\textsc{sg}.\textsc{m}\\
\glt `he harvested them.'
\z

\ea\label{ex:text10-49}
emba kanggon yeka ye kacang aip okune alumen\\
\gll emba	kanggon	yeka	ye	kacang	aip	okun-e	alum-en\\
     father	also	3\textsc{sg}.\textsc{refl}	3\textsc{sg}.\textsc{m}	peanut	there\_is	like\_that-\textsc{sm}	plant:\textsc{pl}.\textsc{o}-3\textsc{sg}.\textsc{m}\\
\glt `Father too, he also planted his own peanuts.'
\z

\ea\label{ex:text10-50}
kacang ambikin talep kai okune komboden\\
\gll kacang	ambikin	talep	kai	okun-e	komb-o-den\\
     peanut	ground	big	very	like\_that-\textsc{sm}	put:\textsc{sg}.\textsc{o}-3\textsc{sg}.\textsc{m}-\textsc{pfv}\\
\glt `He made a very big peanut garden.'
\z

\ea\label{ex:text10-51}
okune go ambo Lukas ye kacang odo badanmo bomane\\
\gll okun-e	=go	ambo	Lukas	ye	kacang	odo	bad-an	=mo	boma-n-e\\
     like\_that-\textsc{sm}	=\textsc{ptc}	older\_brother	\textsc{pn}	3\textsc{sg}.\textsc{m}	peanut	\textsc{dem}	pull\_out-\textsc{n}	=\textsc{adv}	walk-\textsc{ss}-3\textsc{sg}.\textsc{m}\\
\glt `It was like that and brother Lukas, his peanuts, brother Lukas harvested his peanuts and he ...'
\z

\newpage

\ea\label{ex:text10-52}
wene wene oknum adep okune kan wonoden\\
\gll wen-e	wen-e	oknum	adep	okun-e	ka-n	won-o-den\\
     go-\textsc{sm}	go-\textsc{sm}	island	like	like\_that-\textsc{sm}	put:\textsc{pl}.\textsc{o}-\textsc{n}	go-3\textsc{sg}.\textsc{m}-\textsc{pfv}\\
\glt `he made like some islands and left them.'
\z

\ea\label{ex:text10-53}
oknum adep okune kolimo kolimo kemen bomaen\\
\gll oknum	adep	okun-e	kol-i=mo	kol-i=mo	kem-en	boma-en\\
     island	like	like\_that-\textsc{sm}	leave-\textsc{inf}=always	leave-\textsc{inf}=always	do-3\textsc{sg}.\textsc{m}	walk-3\textsc{sg}.\textsc{m}\\
\glt `He made like some further islands and left them.'
\z

\ea\label{ex:text10-54}
okunegole badane nowan kelimo\\
\gll okune-gol-e	badan-e	nowan	kel-i=mo\\
     like\_that-\textsc{ss}.\textsc{seq}-3\textsc{sg}.\textsc{m}	pull\_out-\textsc{sm}	finish	finish-\textsc{inf}-always\\
\glt `He always did like that until he would finish harvesting.'
\z

\ea\label{ex:text10-55}
oknum adep awane tumbe badane nowan kelimo ogo kemen bomadon e\\
\gll oknum	adep	awane	tumb-e	badan-e	nowan	kel-i=mo	ogo	kem-en	boma-d-on	=e\\
     island	like	round	turn\_around-\textsc{sm}	pull\_out-\textsc{sm}	\textsc{neg}	become-\textsc{inf}=always	\textsc{dem}	do-3\textsc{sg}.\textsc{m}	walk-\textsc{dur}-3\textsc{sg}.\textsc{m}	=and\_then.\textsc{ds}\\
\glt `He would harvest them by going around pulling them out and forming like islands and then ...'
\z

\ea\label{ex:text10-1332}
wene wene mim mo koloden\\
\gll wen-e	wen-e	mim	mo	kol-o-den\\
     go-\textsc{sm}	go-\textsc{sm}	one	only	leave-3\textsc{sg}.\textsc{m}-\textsc{pfv}\\
\glt `until (the last) one, he left one only.'
\z

\ea\label{ex:text10-168}
oknum mim adep kacang odo badane wene oknum mim adep keloden\\
\gll oknum	mim	adep	kacang	odo	badan-e	wen-e	oknum	mim	adep	kel-o-den\\
     island	one	like	peanut	\textsc{dem}	pull\_out-\textsc{sm}	go-\textsc{sm}	island	one	like	become-3\textsc{sg}.\textsc{m}-\textsc{pfv}\\
\glt `Like an island, he harvested the peanuts and he left one just like an island.'
\z

\newpage

\ea\label{ex:text10-57}
ye odo okune badane balanon odo ne edo mene ambam anggan bomane bele kemin\\
\gll ye	odo	okun-e	badan-e	bal-an-on	odo	ne	edo	men-e	ambama-ngg-an	boma-n-e	bel-e	kem-in\\
     3\textsc{sg}.\textsc{m}	\textsc{dem}	like\_that-\textsc{sm}	pull\_out-\textsc{sm}	\textsc{aux}.\textsc{cont}-\textsc{irr}-3\textsc{sg}.\textsc{m}	\textsc{dem}	1\textsc{sg}	\textsc{dem}.this	come-\textsc{sm}	play-2/3\textsc{sg}.\textsc{o}-\textsc{n}	walk-\textsc{ep}-\textsc{sm}	\textsc{aux}.\textsc{cont}-\textsc{sm}	do-1\textsc{sg}\\
\glt `I used to come to play with him when he was harvesting.'
\z

\ea\label{ex:text10-58}
mene ambikin a batkop\\
\gll men-e	ambikin	=a	batkop\\
     come-\textsc{sm}	ground	=\textsc{lnk}	round\_piece\\
\glt `I came to the place and a clump ...'
\z

\ea\label{ex:text10-59}
oknedem batkop wani alinggamben ogo be kaneni\\
\gll oknedem	batkop	wani	alinggamb-en	ogo	b-e	kane-n-i\\
     sand	round\_piece	down	lay\_down:\textsc{pl}.\textsc{s}-3\textsc{sg}.\textsc{m}	\textsc{dem}	take:\textsc{pl}.\textsc{o}-\textsc{sm}	take:\textsc{sg}.\textsc{o}-\textsc{ss}-1\textsc{sg}\\
\glt `I took some clumps of sand from the ground and I ...'
\z

\ea\label{ex:text10-1333}
kale anbananan e\\
\gll kal-e	an-ban-an-an	=e\\
     throw:\textsc{sg}.\textsc{o}-\textsc{sm}	3\textsc{sg}.\textsc{m}.\textsc{o}-hit-\textsc{irr}-1\textsc{sg}	=\textsc{ds}.\textsc{seq}\\
\glt `threw them to hit him and then ...'
\z

\ea\label{ex:text10-60}
ta kadap kido na engge kemanon e ombe kale wenogilit\\
\gll ta	kadap	kido	n-∅-a	engg-e	kem-an-on	=e	omb-e	kal-e	weno-gol-i	=t\\
     and	answer	down	1\textsc{sg}.\textsc{o}-hit-1\textsc{pl}.\textsc{imp}	say-\textsc{sm}	\textsc{aux}-\textsc{irr}-3\textsc{sg}.\textsc{m}	=\textsc{ds}.\textsc{seq}	run\_away-\textsc{sm}	run\_away-\textsc{sm}	run\_away-\textsc{ss}.\textsc{seq}-1\textsc{sg}	=and\_then.\textsc{ss}\\
\glt `He would try to hit me back but I usually ran away and then ...'
\z

\ea\label{ex:text10-61}
ta ye timbele ye kacang wani badane balanon\\
\gll ta	ye	timbel-e	ye	kacang	wani	badan-e	bal-an-on\\
     again	3\textsc{sg}.\textsc{m}	sit-\textsc{sm}	3\textsc{sg}.\textsc{m}	peanut(\textsc{bi})	down	pull\_out-\textsc{sm}	\textsc{aux}.\textsc{cont}-\textsc{irr}-3\textsc{sg}.\textsc{m}\\
\glt `again he would sit calmly and harvest his peanuts.'
\z

\ea\label{ex:text10-1142}
ta mene kale aimo okemene bele kimin\\
\gll ta	men-e	kal-e	a-∅-i=mo	okeme-n-e	bel-e	kim-in\\
     again	come-\textsc{sm}	throw:\textsc{sg}.\textsc{o}-\textsc{sm}	3\textsc{sg}.\textsc{m}.\textsc{o}-hit-\textsc{inf}=always	do\_that-\textsc{ep}-\textsc{sm}	\textsc{aux}.\textsc{cont}-\textsc{sm}	do-1\textsc{sg}\\
\glt `And again I would come to throw (something) and hit him.'
\z

\ea\label{ex:text10-62}
okune nup alop nenggon kadap ane bele kemup\\
\gll okun-e	nup	alop	nenggon	kadap	a-Ø-n-e	bel-e	kem-up\\
     like\_that-\textsc{sm}	1\textsc{pl}	two	play	play	3\textsc{sg}.\textsc{m}-hit-\textsc{epn}-\textsc{sm}	\textsc{aux}.\textsc{cont}-\textsc{sm}	do-1\textsc{pl}\\
\glt `Both of us used to play like that.'
\z

\ea\label{ex:text10-63}
kot okemodonup wene wene a ambo ye tin oya bet\\
\gll kot	okemo-d-o-n-up	wen-e	wen-e	ambo	ye	tin	oya	=bet\\
     and\_then.\textsc{ds}	do\_that-\textsc{dur}-\textsc{ep}-\textsc{ss}-1\textsc{pl}	go-\textsc{sm}	go-\textsc{sm}	older\_brother	3\textsc{sg}.\textsc{m}	close	there	=\textsc{obl}\\
\glt `And then we were doing like that and (I) came closer to brother.'
\z

\ea\label{ex:text10-64}
``ambo kabakadanep a''\\
\gll ambo	kabaka-d-an-ep	=a\\
     older\_brother	become\_angry\_at-1\textsc{sg}.\textsc{o}-\textsc{irr}-2\textsc{sg}	=\textsc{lnk}\\
\glt `{``}Brother, please don't be angry at me.'''
\z

\ea\label{ex:text10-65}
mene tanggodin engge okune wene otbop yanop okune wene kudu wune go\\
\gll men-e	tanggo-d-in	engg-e	okun-e	wen-e	otbop	yanop	okun-e	wen-e	kudu	wun-e	=go\\
     come-\textsc{sm}	disturb-\textsc{dur}-1\textsc{sg}	say-\textsc{sm}	like\_that-\textsc{sm}	go-\textsc{sm}	language	there\_is	like\_that-\textsc{sm}	go-\textsc{sm}	lap	go\_in-\textsc{sm}	=\textsc{ptc}\\
\glt `I thought I was disturbing him and then I came closer to him and then ...'
\z

\ea\label{ex:text10-1334}
wene wene temaden got ``ih!''\\
\gll wen-e	wen-e	tem-a-den	=got	ih\\
     go-\textsc{sm}	go-\textsc{sm}	see-1\textsc{sg}-\textsc{pfv}	=and\_then.\textsc{ds}	\textsc{intj}\\
\glt `I went to see and ``Oh!'''
\z

\newpage

\ea\label{ex:text10-66}
ambikin a kacang badane wene koloden ye oknum adep eyom edo\\
\gll ambikin	=a	kacang	badan-e	wen-e	kol-o-den	ye	oknum	adep	eyom	edo\\
     soil	=\textsc{lnk}	peanut	pull\_out-\textsc{sm}	go-\textsc{sm}	become-3\textsc{sg}.\textsc{m}-\textsc{pfv}	3\textsc{sg}.\textsc{m}	island	like	in\_here	\textsc{dem}.this\\
\glt `Inside the soil where he harvested peanuts and made these little ``islands'' ...'
\z

\ea\label{ex:text10-67}
tinggan amun yanop an\\
\gll tinggan	amun	yanop	=an\\
     rat	nest	there\_is	=\textsc{cop}\\
\glt `there was a rat's nest.'
\z

\ea\label{ex:text10-68}
tinggan Muying amun\\
\gll tinggan	muying	amun\\
     rat	kind\_of\_mouse	nest\\
\glt `The nest of a Muying rat.'
\z

\ea\label{ex:text10-69}
wenoni wene tin bet kabanggodoni kelegan ga ``ih edo tinggan amun!''\\
\gll weno-n-i	wene	tin	=bet	kabanggo-d-o-n-i	keleg-an	=ga	ih	edo	tinggan	amun\\
     go-\textsc{ss}-1\textsc{sg}	go-\textsc{sm}	close	=\textsc{obl}	pay\_attention-\textsc{dur}-\textsc{ep}-\textsc{ss}-1\textsc{sg}	check-1\textsc{sg}	=\textsc{ptc}	\textsc{intj}	\textsc{dem}.this	rat	nest\\
\glt `I went on, went closer and checked it well, ``Oh, it is a rat's nest!'''
\z

\ea\label{ex:text10-70}
``tinggan Muying amun edo kili in o!''\\
\gll tinggan	muying	amun	edo	kili	=in	=o\\
     rat	kind\_of\_mouse	nest	\textsc{dem}.this	new	=\textsc{cop}	=\textsc{quot}\\
\glt `{``}This nest of a Muying rat is new!'''
\z

\ea\label{ex:text10-71}
``eyom edo tinggan yanop an'' engge yem opkogoli\\
\gll eyom	edo	tinggan	yanop	=an	engg-e	yem	opko-gol-i\\
     in\_here	\textsc{dem}.this	rat	there\_is	=\textsc{cop}	say-\textsc{sm}	quietly	think-\textsc{ss}.\textsc{seq}-1\textsc{sg}\\
\glt `{``}In here, there must be a rat'', I thought like that and ...'
\z

\ea\label{ex:text10-1176}
ambo bane kumunggan\\
\gll ambo	ban-e	kumungg-an\\
     older\_brother	call:once-\textsc{sm}	tell-1\textsc{sg}\\
\glt `I called my brother and told him.'
\z

\ea\label{ex:text10-72}
``ambo eyom edo tinggan yanop a non o'' enggan\\
\gll ambo	eyom	edo	tinggan	yanop	=a	=non	=o	engg-an\\
     older\_brother	in\_here	\textsc{dem}.this	rat	there\_is	=\textsc{lnk}	=\textsc{maybe}	=\textsc{quot}	say-1\textsc{sg}\\
\glt `{``}Brother, maybe there is a rat in here'', I said.'
\z

\ea\label{ex:text10-73}
enggan go ``ah wani ningge teme yo'' enggon\\
\gll engg-an	=go	ah	wani	ningg-e	tem-e	=yo	engg-on\\
     say-1\textsc{sg}	=\textsc{ptc}	\textsc{intj}	down	catch:\textsc{sg}.\textsc{o}-\textsc{sm}	see-2\textsc{sg}.\textsc{imp}	=\textsc{quot}	say-3\textsc{sg}.\textsc{m}\\
\glt `When I said that, he said: ``Oh, try and catch it down there!'''
\z

\ea\label{ex:text10-74}
wene wene tin bet ninggaanan nea kolanan nea?\\
\gll wen-e	wen-e	tin	=bet	ningga-an-an	nea	kol-an-an	nea\\
     go-\textsc{sm}	go-\textsc{sm}	near	=\textsc{obl}	catch:\textsc{sg}.\textsc{o}-\textsc{irr}-1\textsc{sg}	\textsc{q}	leave-\textsc{irr}-1\textsc{sg}	\textsc{q}\\
\glt `I went closer and hesitated if I should catch it or not? (lit. will I  catch it or will I refuse?)'
\z

\ea\label{ex:text10-1335}
okemodoni ``aih!''\\
\gll okemo-d-o-n-i	aih\\
     do\_that-\textsc{dur}-\textsc{epv}-\textsc{ss}-1\textsc{sg}	\textsc{intj}\\
\glt `I did (thought) like that and ``Oh!'''
\z

\ea\label{ex:text10-75}
yon bet nolona ye enggan\\
\gll yon	=bet	nolon-a	=ye	engg-an\\
     foot	=\textsc{obl}	touch:once-1\textsc{sg}.\textsc{imp}	=\textsc{quot}	say-1\textsc{sg}\\
\glt `I wanted to touch it with (my) foot.'
\z

\ea\label{ex:text10-76}
``aih okunanon balin edo tinggan yanop an'' engge tinggi bet ninggaden got\\
\gll aih	okun-an-on	balin	edo	tinggan	yanop	=an	engg-e	tinggi	=bet	ningg-a-den	=got\\
     \textsc{intj}	like\_that-\textsc{irr}-3\textsc{sg}.\textsc{m}	\textsc{neg}	\textsc{dem}.this	rat	there\_is	=\textsc{cop}	say-\textsc{sm}	hand	=\textsc{obl}	catch:\textsc{sg}.\textsc{o}-1\textsc{sg}-\textsc{pfv}	=and\_then.\textsc{ds}\\
\glt `{``}Oh, not like that, there must be a rat'', (I said) and then I tried with my hand and then ...'
\z

\ea\label{ex:text10-77}
yanam an wom odo tinggan Muying yanop\\
\gll yanam	=an	wom	odo	tinggan	muying	yanop\\
     true	=\textsc{cop}	inside	\textsc{dem}	rat	kind\_of\_mouse	there\_is\\
\glt `it was true, there was a Muying rat inside.'
\z

\ea\label{ex:text10-78}
tinggan Muying tinggan Muying mana yanop tiun\\
\gll tinggan	muying	tinggan	muying	mana	yanop	ti-un\\
     rat	kind\_of\_mouse	rat	kind\_of\_mouse	offspring	there\_is	stay-3\textsc{sg}.\textsc{f}\\
\glt `A Muying rat stayed with her offspring in there.'
\z

\ea\label{ex:text10-79}
okune tinggan Muying amun odo ninggaden got\\
\gll okun-e	tinggan	muying	amun	odo	ningg-a-den	=got\\
     like\_that-\textsc{sm}	rat	kind\_of\_mouse	nest	\textsc{dem}	catch:\textsc{sg}.\textsc{o}-1\textsc{sg}-\textsc{pfv}	=and\_then.\textsc{ds}\\
\glt `Just like that I held the nest of the rat and then ...'
\z

\ea\label{ex:text10-80}
kedo ne tinggi ya tobun go kenambun umkan wain go\\
\gll kedo	ne	tinggi	ya	tob-un	=go	kenambun	umkan	wa-in	=go\\
     then	1\textsc{sg}	hand	at	bite-3\textsc{sg}.\textsc{f}	=\textsc{ptc}	strongly	blood	\textsc{lv}-1\textsc{sg}	=\textsc{ptc}\\
\glt `it bit my hand so that I bled strongly and ...'
\z

\ea\label{ex:text10-1409}
ambo Lukas odo ne alambon belewaen\\
\gll ambo	Lukas	odo	ne	alambon	belewa-en\\
     older\_brother	\textsc{pn}	\textsc{dem}	1\textsc{sg}	laughing	make\_effort-3\textsc{sg}.\textsc{m}\\
\glt `brother Lukas laughed at me.'
\z

\ea\label{ex:text10-81}
alambon belewaen go okune aleng yanop kolo ambip wanaden\\
\gll alambon	belewa-en	=go	okun-e	aleng	yanop	kolo	ambip	wan-a-den\\
     laugh	make\_effort-3\textsc{sg}.\textsc{m}	=\textsc{ptc}	like\_that-\textsc{sm}	crying	there\_is	back	house	go-1\textsc{sg}-\textsc{pfv}\\
\glt `He laughed at me and I just went back home crying.'
\z

\ea\label{ex:text10-82}
okunupten o\\
\gll okun-up-ten	=o\\
     like\_that-1\textsc{pl}-\textsc{pfv}	=\textsc{quot}\\
\glt `We did like that.'
\z