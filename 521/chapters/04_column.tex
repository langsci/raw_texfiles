% Muyu065 
\begin{Parallel}{0.47\textwidth}{0.47\textwidth}
    \ParallelLText{\noindent \textit{Oke. Eto tana nanangmimo kemip. Tana olalimo kemip. Olaleyimo kemip yeman, olalane keman.}}
    \ParallelRText{\noindent Okay. With this story they teach children a lesson. They often tell it to the children. And now I am going to tell it to you.}
\end{Parallel}

    \vspace{.4cm}

\begin{Parallel}{0.47\textwidth}{0.47\textwidth}
    \ParallelLText{\noindent \textit{Omotom, omotom onggo ini Kati ambip, ogilan ye ka aip monok balin, enamba alip, yi tana. Enamba, yi tana alop, anyan nenggan. Belon-belon bet wene bayipten go wene taleptelepip. Yi odo inon kibinok okune: ``Ini men o, yowo yo, wene aluptem o, aluptem o, aluptem.'' Oto Aluptem yatpon kole. Wenenip a, bip, temnong etawo. Be wene men aminggi yeman.}}
    \ParallelRText{\noindent A long time ago, before the foreigners came, there were two parents who lived here in the Kati{\footnotemark} area, with their children. They had two daughters, a younger sister and an older sister, and they took care of the little ones until they were grown up. Every day, they did just like this: ``Here are a string bag and a sago bag, go to the Genemon trees!'' Because there was a grove of Genemon trees. The sisters would go there and collect Genemon vines and other stuff. They would use them to make string bags.}\footnotetext{\textit{Kati} is an alternative name for Muyu. It is used mainly by older speakers.}
\end{Parallel}

    \vspace{.4cm}

\begin{Parallel}{0.47\textwidth}{0.47\textwidth}
    \ParallelLText{\noindent \textit{Amit mim a, katuk mene kumunggip: ``Ata bon telepane kemip kole, men a, yowo onongmembime. Okune ko ata bon eya telepi yeman.'' Nimbin alop, tana-tana oto bayipten onet wene talep-talep telepen winip, suda.}}
    \ParallelRText{\noindent Sometimes people would come and tell them: ``People want to have a party, so make many string bags and sago bags. We need them for the party.'' And so they did. Okay. That's how the two girls were growing up.}
\end{Parallel}

    \vspace{.4cm}

\begin{Parallel}{0.47\textwidth}{0.47\textwidth}
    \ParallelLText{\noindent \textit{Amit mim oto, anjan nanggan, adik kakak, yi alop a, yem atep olalenip a: ``Ine oto wene, wenenup a, aluptem etawo ben bomi yeman.'' Taman kole, opnon a, opnon a amnom oto, om etawo oto, animan yitwaip. Okune ko, oto wene, wonggo wene ani yeman. Anenggakolip, iyo, okune anggipten.}}
    \ParallelRText{\noindent One day, the sisters, the younger and older sister, were making secret plans: ``Tomorrow, we will go out to take some Genemon fruit and other stuff.'' Since it was far away they had to prepare carefully. That night they cooked sago and other things. They prepared some food for the trip on the next day. Then they had dinner and went to sleep.}
\end{Parallel}

    \vspace{.4cm}

\begin{Parallel}{0.47\textwidth}{0.47\textwidth}
    \ParallelLText{\noindent \textit{Nake wene bili keto winipten. Wene, Aluptem etawo ben bomaip. Ta nimbin oto kon kewet kolo, okune, otpotom okune: Katuk awanem yeman ye otpotom okune olale bomanip okune. Ben bomanip wene, wene, wene, ``Ihh!'' Wene, aluptem yatpon nanggip. Na\-nggenip a, okune aluptem ombetmo beotip ko wene, opnon. Opnon go, katekmip oyeki! ``Keto kete wananup?'' Okune, ketonip a, opkonggaip ka okunon go suda, anggipten. Om animan ombetmo anengga kole anggipten.}}
    \ParallelRText{\noindent At the crack of dawn, they took off. They went to get Genemon fruit and other stuff. And since the women were still young, they had a conversation about typical young woman things.  They talked about their desire to marry a man while they were walking. They walked on and on and on and finally, they arrived at the Genemon grove. They picked Genemon fruit until the evening. But in the evening, they got lost. ``Now, where shall we go?'' It was like that and they were thinking about what to do but there was nothing they could do. So they decided to wait for the next day. They just ate some of the food and slept.}
\end{Parallel}

    \vspace{.4cm}

\begin{Parallel}{0.47\textwidth}{0.47\textwidth}
    \ParallelLText{\noindent \textit{Wene, wene nakon e, ta keto ambip wenem engge ambonggaip, opkonggaip ka, welen atep kelon gole, sudah. Welen atep kelon kole wenenip a, ``Ambip kim ogo wene kete bet an ko'', engge ambon boma wene wene wene, ``Ih, katuk oye!'' Yongbon aip nanggip. Yongbon ya nanggip ko, itu suda. Animan etawo oto nowan ko okune, monopni yowotkaip. Wene temip ko, im yanop an. Im yumu. Okune yumune anggen. Kole suda. ``Oni, oni!'' Yu taman ombet, ``Oni, oni! Im aip an. Mene, mene!'' Kole suda, wenenip a. Im ogo kawene anenggatip.}}
    \ParallelRText{\noindent The next day, they wanted to go home and searched for the road. But although they were trying hard, they could not find it. Okay. It was too difficult, they were just walking around: ``Where is the way to go home?'' They went on and on. ``Oh, there are people!'' They arrived at a garden. When they arrived at the garden, they felt lucky because they had no food left. They were starving. There was a pandanus, a red pandanus fruit. It was ripe already. ``Hey, sister!'', the younger girl said. ``Hey  sister! There is a red pandanus. Come here!'' So okay, they went there and climbed the tree and ate the pandanus.}
\end{Parallel}

    \vspace{.4cm}

\begin{Parallel}{0.47\textwidth}{0.47\textwidth}
    \ParallelLText{\noindent \textit{Oto katuk yena ye yongbon. Yongbon bilimbon. Kole wene anenggatip. Ma ye atiman oto, wetaen go ``Ih, okpotom katap yanop an.'' Okpotom katap yanop ko, wene temon go ``Ih, nimbin alop oto im ogo anenggatip o!'' Im anenggatip ko, okune ano tinim ogo tikpat bumbene okune, kapu aip kombon go okune. Nimbin alop oto kutulenip a, ambut eleng wenebip. Ambut eleng wenebonip a, wene, wene ap neyong onggo. Yu taman oto oyop kole yato tole, keto wunun. Ma yu oni oto, yato tola enggun go ap oto oke kombon go, ta okune monkane yato ap ma. Kanggon ogo okon. Ta monkane yato ap ma: ``Ih!'' Ap Wet neyong tolun go okune, neyong anwane kolem okune wiyokombon go, wiokombon go, suda. ``Suda, ah! Suda, niokombon gole, taman kup betmo wati wene. Ne eto keto tut wanan.'' Onggenun a, keto wenebip oto, ta okune kawut okune, tinggi ekemip ombet ta, kutulenip a tinggi, langsung wene bulu koloten. Bulu kolon go okune kawut keto, on atep a kane kemonip wenepip oye ki.}}
    \ParallelRText{\noindent But it was someone else's garden. An old garden. So while they were eating, the owner of the garden heard them: ``Oh, there are voices.'' There was much talking, so he went to see: ``Oh, these two women are eating the pandanus fruit!'' While they were eating the pandanus fruit, he made a sound swinging a bowstring and yelled at them. The two women were frightened and tried to escape in fear. They climbed up on a branch. Since the younger sister was light, she stepped on it and was out of reach. But the elder sister wanted to step on a branch as well but it broke. She quickly jumped on another tree but that one also broke. And again she jumped on another tree. It was the branch of a ``Wet'' tree, she stepped on it and when the branch broke, she fell down on the ground. ``Ouch! Oh younger sister, I fell down! You stay up there alone. I'll escape on the ground.'' She said like that and then they went away. Suddenly their hands were waving in the air. They were shocked as their hands turned into wings. As they got wings, they suddenly flapped their wings like birds and escaped.}
\end{Parallel}

    \vspace{.4cm}

\begin{Parallel}{0.47\textwidth}{0.47\textwidth}
    \ParallelLText{\noindent \textit{Wenebonip a, jadi yu taman oto wenenun a, on kelenun okune kawut, on Kitim kuluten. Ma yu oni oto, kito tut kole ``Ah suda! Ap kobi wati tonanan balin. Okune tut timbalan. Tut mo okune wanan gole.'' Wene kawut niyap kuluten oyen. Okunuten kot, kole oto enamba woyambang oto, olayimo kemip. Ege keto wini ka, eto ambanggi ka, okune enamba aip kumungge kole, keto wini mo on. Oto okune olayimo kemip. Wananip, wene okune wene, wene tut wonggo timbalanip, angganip, enamba oto, yi kat kelipten gole, ambon mene temanip. Okune atep an. Iya, anyan nanggana, anyan oto niyap kulun go, nanggan eto on Kutim kuluten. Okune atep an. Okune atep kole, eto biasa. Enamba ka, oni welet, ta yi tamana, yi tana, olaye bilimo kemip. Okune atep yeman. Enamba, woyambang, okune okpotom oto ninon kanon. Okune, olaleyimo kemip. Kole okune atep an.}}
    \ParallelRText{\noindent The younger sister had become a crowned pigeon. But the elder sister was down on the ground. She said: ``Oh, okay! I can't climb on the tree. I'll stay on the ground and walk around like that.'' It was a cassowary that she turned into. So that's what parents and grandparents usually tell us. Whether you are going out or going to work or wherever, you must tell your parents before going out. That's what they used to tell their children. They need to tell their parent so that they always can find them when they stay in they jungle for the night. It's like that. Yes, now back to the the sisters. While the older sister became a cassowary, the younger sister became a crowned pigeon. It was like that. This is normal. Parents tell this story to their children or older sisters to their younger siblings.  Parents and grandparents passed down this story from generation to generation. They usually tell it like that.}
\end{Parallel}
