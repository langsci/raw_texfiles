This is the story about the origin of the coconut tree. There once was a young man. He and his wife lived in a house, he had many dogs.
He used to hunt in the morning, he hunted pigs. No, not he himself, it was not him who shot the pig. The dogs would bite and kill it and he would bring the dead pig back home and cook it.
Then the man and his wife would eat the meat, and they would give the bones to the dogs. They always gave only the bones to the dogs.

The dogs were thinking: ``Oh come on! Just shoot it yourself, with your arrows and bow! From all that, you always give us the bones, although we were the ones who killed it! It is easy for us to kill it, then you bring it home and eat it, that's not fair!'' They thought like that and then one day they stopped helping him. They came back from the hunt and got nothing. So the man said to his wife, ``Darling, when we are going, it was like taking a walk only. It seems there are are no pigs. But it was strange. The dogs were always snuffling and growling at me. Next time, when you wait for me until it gets dark and the dogs come home alone, it might well mean that the dogs have killed me.'' She agreed and remembered his words.

Another day, he went off hunting again. His wife stayed at home and waited until it became dark. ``Oh, dogs, you are here. But where is father? Oh, father is nowhere'', she talked to them. The dogs just wagged their tails and stuck out of their tongues. ``Oh, you dogs'', she said. ``What shall this mean?''

So, they slept until the next day and when the day broke, she took the dogs and went out. ``Where is he? Where is he?'' Then they came to a river. The dogs stopped at the side of the river. They had eaten him and thrown his head into this river. Now they just stood up there on the land and were watching the river beneath. ``Come on, you dogs, that is only the river down there!'', she scolded them. ``Oh, maybe they have thrown him into the river!'', she thought and then she saw it: ``Yes, it is true!'' They had severed his head. They had threwn it into the river.

The woman took a palm tree rib and cleaned it. She took three of them and made a fork from it. She connected it to a stick and bound it with a string. Then she tried to get the head out of the water with the fork. She stabbed for the first time, but the prong broke. ``Oh, not like that'', she said. There were two prongs left now. She stabbed again but second prong also broke. Just one prong was left. She stabbed it and this time the prong went in. It went into the floating head successfully, so she could take it quickly. Then she took a rib of a palm tree and tied it and brought it back home.

She brought the head home and dug a hole next to her house. But the hole was not large, she made a small one only. Then she wrapped the head with the rib of a palm tree and put some of leaves down in the whole. She put the wrapped head on it and finally she put the soil back on it. When the hole was closed, she simply waited. She only could wait but it seemed that the head of her husband would not decay.

When she checked next time, she saw that there was a tree now exactly at the place where she had buried the head. It was like the palm tree we call Kuk Kimit. ``Here are leaves like the ones of Kuk Kimit. Is this a palm tree? Maybe I should cut it? Oh no, I cannot cut it.'' She watched it day after day, ``Oh it is not a palm tree. It is not a palm tree! This must be what they call a coconut tree'', she said. Then she took some soil and heaped it up. It became a big tree and it had several stems and some leaves. At dawn she heard a strange voice. She heard, ``Bian-bian-bian! Bian-bian-bian! Bian-bian-bian!'' It was only the word 'Bian' repeated three times and then it stopped. She looked and saw: ``Oh, there is a bird!'' This bird was sitting on the leaves of the coconut tree. It was a wild chicken. It was standing up there and making sounds. ``What was that? It said `Bian-Bian'!'' 

In the morning she prepared her needs. She took everything and she took some seeds of the coconut tree and went to the Bian river. That's where she planted them. She planted them at Bian river and then she piled some wood and made a fence. She cut some wood and bound it with strings then she waited for the coconut tree to grow up. Finally, she saw it, ``Oh, it has come out.'' It grew and indeed, there were some stems. Initially, she thought the stems were empty but they bore fruit. The tree bore fruit and not only one side but on the other side as well. She waited until they were ripe and fell down. They fell in the Bian river with a sizzling sound. Whenever a coconut fell into the river, the water took it away. And people who were living downstream saw it, ``Oh what is this?'' They took and kept it then they said: ``Oh guys, this is called coconut.'' Each time someone took out a coconut from the water, he would share it with all the others. Someone would get just one, someone would get two, other people would take three and plant them, three, four, five. They took the fruits like that. And some of them were planted and they spread out to become coconut trees.

So earlier, the woman had pierced the head where people nowadays make a hole in the coconut to get the water. They make a hole with a stick, with a small bamboo stick. They put it inside and they would call it `Yopke', that is a word from the Komoyan subdialect. You know, so it was like that. We still do it like that and we insert the bamboo and that is how we usually drink. So there are three holes. But we cannot make three of them, there is only one hole. It was like that and then what they are calling coconut has been spread. That is a story from where my mother lives. I used to listen when she told it. So it's like that, that is it.
