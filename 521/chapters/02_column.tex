% Muyu031
\begin{Parallel}{0.47\textwidth}{0.47\textwidth}
    \ParallelLText{\noindent \textit{Kelapa ambat odo. Kaduk tana, ye yom ye amban yom ambip mim tiip, ye go anon kadap. Amkali bali winimo, bali wene wene awon ko. Bukan, ye bet a, ye aip mombe kemok balin. Anon bet tomone anekombe kombanip e, ko, kane wene, mingge, yom ko yi alop ananip e, konokap ombet anon ko kayimo. Konokap ombet anon ko kayimo.
}}
    \ParallelRText{\noindent This is the story about the origin of the coconut tree. There once was a young man. He and his wife lived in a house, he had many dogs. He used to hunt in the morning, he hunted pigs. No, not he himself, it was not him who shot the pig. The dogs would bite and kill it and he would bring the dead pig back home and cook it. Then the man and his wife would eat the meat, and they would give the bones to the dogs. They always gave only the bones to the dogs.
}
\end{Parallel}

    \vspace{.4cm}

\begin{Parallel}{0.47\textwidth}{0.47\textwidth}
    \ParallelLText{\noindent \textit{``Aih moyon mok ah, kep ano tinim bet mombenepket a, ombet nup ko, ombet konokap kayimo bet a, ma nup bet wene anekombanup kot, mit anekombanup kot, kep ko kane mene animo kemodep, ah!'' engganipket, ko, wene-mene nowan keli yamo, kedo kumunggon gi, ``Enggon! Wananup ko, bomanup ko yo, awon aip balin gole. Ne ya mo ipmone, kulungot wane, okemodip kole yo, wene wene teme golo, midikanon, nowan, anon mo mananip, odep kole, anon bet a anekombe anip yo engge.'' -- ``Eih, yanam an.'' Okuneneget wenen.
}}
    \ParallelRText{\noindent The dogs were thinking: ``Oh come on! Just shoot it yourself, with your arrows and bow! From all that, you always give us the bones, although we were the ones who killed it! It is easy for us to kill it, then you bring it home and eat it, that's not fair!'' They thought like that and then one day they stopped helping him. They came back from the hunt and got nothing. So the man said to his wife, ``Darling, when we are going, it was like taking a walk only. It seems there are are no pigs. But it was strange. The dogs were always snuffling and growling at me. Next time, when you wait for me until it gets dark and the dogs come home alone, it might well mean that the dogs have killed me.'' She agreed and he went off.}
\end{Parallel}

    \vspace{.4cm}

\begin{Parallel}{0.47\textwidth}{0.47\textwidth}
    \ParallelLText{\noindent \textit{Wenenip ko wene bekmun go wene, ih midikon te. ``Ah, anon kip ko, ma emba ede? Ih emba go nowan an.'' Tokot yibi wane, ong mo ``hmm hmm'' tanggale okembip. Ong mo, ``hmm hmm.'' -- ``Aih, anon kip ko ede kede adep an ga, ih?''}}
    \ParallelRText{\noindent Another day, he went off hunting again. His wife stayed at home and waited until it became dark. ``Oh, dogs, you are here. But where is father? Oh, father is nowhere'', she talked to them. The dogs just wagged their tails and stuck out of their tongues. ``Oh, you dogs'', she said. ``What shall this mean?''}
\end{Parallel}

    \vspace{.4cm}

\begin{Parallel}{0.47\textwidth}{0.47\textwidth}
    \ParallelLText{\noindent \textit{Kot a, anggip e nakon go, nakon e kedo anon ko be wunuden. Be wenonu, wene wene wene, ``Kede bet an? Kede bet an?'' kemip ko, anon ombet wenonip ko ok kebet. Anonipket, kodolok odo kane ok kalipten. Ok kalipten got, wenenip wene, wene adi bet alonip, ok ani mo kelegadip. ``Mok anon kip ko wani odo ok an, ah! Aih mok kane ok wani kalipten'', engge wene temuden ge ``Eh! yanam an te!'' Kodolok odo wane kalipten jadi, wane kalogilip ket, ok wani kalipten.}}
    \ParallelRText{\noindent So, they slept until the next day and when the day broke, she took the dogs and went out. ``Where is he? Where is he?'' Then they came to a river. The dogs stopped at the side of the river. They had eaten him and thrown his head into this river. Now they just stood up there on the land and were watching the river below. ``Come on, you dogs, that is only the river down there!'', she scolded them. ``Oh, maybe they have thrown him into the river!'', she thought and then she saw it: ``Yes, it is true!'' They had severed his head. They had thrown it into the river.}
\end{Parallel}

    \vspace{.4cm}

\begin{Parallel}{0.47\textwidth}{0.47\textwidth}
    \ParallelLText{\noindent \textit{Kot, ami kono. Ami kono tabatkaun. Ami kono tabatkanu, alopmim aip, tabak ko alopmim aip mo tabatkaun. Okunenuget, ap tem telenuden, nong bet yenenggaun. Yenenggaguluget, kiduluk mangguden odo okoden. ``Okune balin'', engge ta mok tabak alop ege. Manggun ga, ``Ieh, ogo kanggon!'' Tabak alop ombelen te, tabak ko mim ombet mo. Manggun ga wunoden. Wunoden got, kawut kanenuget. Kedo ambulok wane, ambulok tem kombe yenengganu, kane munuden.}}
    \ParallelRText{\noindent The woman took a palm tree rib and cleaned it. She took three of them and made a fork from it. She connected it to a stick and bound it with a string. Then she tried to get the head out of the water with the fork. She stabbed for the first time, but the prong broke. ``Oh, not like that'', she said. There were two prongs left now. She stabbed again but second prong also broke. Just one prong was left. She stabbed it and this time the prong went in. It went into the floating head successfully, so she could take it quickly. Then she took a rib of a palm tree and tied it and brought it back home.}
\end{Parallel}

    \vspace{.4cm}

\begin{Parallel}{0.47\textwidth}{0.47\textwidth}
    \ParallelLText{\noindent \textit{Kane menonuget, ambip kebet. Ambip kebet bet a ambikin go nitkanuget, mok talep balin. Ambikin ko belon ekine nitkaun, okune mo. Nitkanuget, ambulok bet wombenuget, wadi go konet bet kemengganu, ambulok bet kombenu, ambikin ko, be kane amone ko, tiun. Tiun go wene odo bopnok balin an.}}
    \ParallelRText{\noindent She brought the head home and dug a hole next to her house. But the hole was not large, she made a small one only. Then she wrapped the head with the rib of a palm tree and put some leaves down in the hole. She put the wrapped head on it and finally she put the soil back on it. When the hole was closed, she simply waited. She only could wait but it seemed that the head of her husband would not decay.}
\end{Parallel}

    \vspace{.4cm}

\begin{Parallel}{0.47\textwidth}{0.47\textwidth}
    \ParallelLText{\noindent \textit{Yu wadi odo, ``Ih!'', temuden begini, temuden got a, kodolok wadi ya ayetmun odo, kodolok ayetmun odo temuden go, ``iih!'' Kuk adep an, kuk ko Kuk Kimit adep teboden. ``Kuk Kimit embit adep a! Aih, mok edo kuk an nea ka? Wanala. Aih, mok wanalanan balin.'' Kelegaun got kolo wene wene, ``Ih mok kuk balin. Ih mok kuk balin an o. Kalapa enggaip iyen te. Moyon odep kole, edo kalapa enggaip eyen'', enggun a. Ogan otbop uyot kelapa an, ma nup otbop odo, ambang a ye otbop odo kalapa enggimo kemip jadi. ``Ih kalapa an e, ah.'' Eyuk ko, ambikin bet be kane, ta om akaun. Talep kelon e, ye nolok aip kelene, embit aip kelon go, nake wenembili wedaun got a, enggaen. Wedaun got: ``Bian-bian-bian! Bian-bian-bian! Bian-bian-bian!'' Bian ye aninggo mo alopmim, ta okune alopmim mo kele kolon. Temuden got, ``Ih, on aip an!'' On ege kalapa embit yinim bet tioneget, on ko temuden ge, on sa an. Ogan otbop bet ko, ``Ayam utan o'', enggaip uyen. Nup otbop odo ``on Sa'' an. On Sa bet tioneget, olaloden. ``Midep an ka? Eih Bian-Bian enggaen ko?''
}}
    \ParallelRText{\noindent When she checked next time, she saw that there was a tree now exactly at the place where she had buried the head. It was like the palm tree we call Kuk Kimit. ``Here are leaves like the ones of Kuk Kimit. Is this a palm tree? Maybe I should cut it? Oh no, I cannot cut it.'' She watched it day after day, ``Oh it is not a palm tree. It is not a palm tree! This must be what they call a coconut tree'', she said. Then she took some soil and heaped it up. It became a big tree and it had several stems and some leaves. At dawn she heard a strange voice. She heard, ``Bian-bian-bian! Bian-bian-bian! Bian-bian-bian!'' It was only the word ``Bian'' repeated three times and then it stopped. She looked and saw: ``Oh, there is a bird!'' This bird was sitting on the leaves of the coconut tree. It was a wild chicken. It was standing up there and making sounds. ``What was that? It said `Bian-Bian'!''}
\end{Parallel}

    \vspace{.4cm}

\begin{Parallel}{0.47\textwidth}{0.47\textwidth}
    \ParallelLText{\noindent \textit{Odep kole ko nakon e, kedonu yu bet opkoun ogo, beonu wane kedo kalapa kap odo kane, Bian wunuden. Kane Bian wenenuget, wene Bian, mangguden. Wene ok Bian bet mangguden got, okuneget kedo ap wilibe, kelak onongmun. Ap bet wanenu ko, nong yenengga wane kole tiun got mene kalapa odo, nop kelen e, temuden go, ``Eih! Adi go.'' Ogan otbop bet ``Mayang o'', enggaip. Nup otbop a, ambanga ye otbop, ``Eh, namang aip teboden o'', engge kemip. Teboden kot yanam an a. Nolok yanop an. Nolok tokot an e engguden got kole, yop aip bodoben. Bodoben ogo wene ta yado atma yanop. Bodobone wene wene kelegaun go wene yumunene, nombelen. Nombelene go wene Bian kebet odo okune ``Ssss''. Kuk wilinuden onet kuk wilinuden, waum odo kalapa mo. Kalapa kok odo kuk wangga odo ma, ma, koke Bian kubunoden, Bian bet ok kane wenen. Ta Bian tiip ok tikap wani tiip ye kaduk ombet: ``Moyon mok odo medep an kai?'' Kane wene kombenip, ``Aih, enggon, edo kelapa enggaip eyen e.'' Enggenip ket, ombet okune ombet mulai: Ombet kanenipket, ``Enggon, ne aip an, kep aip an.'' Mim mo kane wini mo, alop mo kane wini mo, alopmim mo kane wene manggi mo, alopmim, kanin, anggo. Okune benbaip. Benbanipket wene alumip kot, ko, ombet wene wene, kuip! Kumun ane kaloden, kelapa odo.}}
    \ParallelRText{\noindent In the morning she prepared her needs. She took everything and she took some seeds of the coconut tree and went to the Bian river. That's where she planted them. She planted them at Bian river and then she piled some wood and made a fence. She cut some wood and bound it with strings then she waited for the coconut tree to grow up. Finally, she saw it, ``Oh, it has come out.'' It grew and indeed, there were some stems. Initially, she thought the stems were empty but they bore fruit. The tree bore fruit and not only one side but on the other side as well. She waited until they were ripe and fell down. They fell in the Bian river with a sizzling sound, like that ``Sssss''. Whenever a coconut fell into the river, the water took it away. And people who were living downstream saw it, ``Oh what is this?'' They took and kept it then they said: ``Oh guys, this is called coconut.'' Each time someone took out a coconut from the water, he would share it with all the others. Someone would get just one, someone would get two, other people would take three and plant them, three, four, five. They took the fruits like that. And some of them were planted and they spread out to become coconut trees.}
\end{Parallel}

    \vspace{.4cm}

\begin{Parallel}{0.47\textwidth}{0.47\textwidth}
    \ParallelLText{\noindent \textit{Kole mim mangguden ye kaduk odo, odo kalalang bet ok nangge kemip. Nangge bulut bet a, owet bet, owet belon bet a. Waum telene, aninggo Yopke engge kemip odo Komoyan otbop an. `Midepan' odo nup odo: ``Ok anem''. Animan kanggon anem. Kep kat a, okunen. Jadi nup ko 'Ok Anem'. Okunenupket, owet bet telenupket, animo kumup uyen. Odo ami kono amtem an. Kole konoptem ko alopmim. Tapi nanggi adep balin an. Mim mo nanggoden oyen an. Okuneneget, wene ane tanoden o kelapa enggadip ko. Odo ta ne ena, ne ena go onggo ya tiun uyen. Ombet olalanun got wedambele kemin. Okunen. Kole okunen, oyen.}}
    \ParallelRText{\noindent So earlier, the woman had pierced the head where people nowadays make a hole in the coconut to get the water. They make a hole with a stick, with a small bamboo stick. They put it inside and they would call it ``Yopke'', that is a word from the Komoyan subdialect. You know, so it was like that. We still do it like that and we insert the bamboo and that is how we usually drink. So there are three holes. But we cannot make three of them, there is only one hole. It was like that and then what they are calling coconut has been spread. That is a story from where my mother lives. I used to listen when she told it. So it's like that, that is it.}
\end{Parallel}
