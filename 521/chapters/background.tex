\section{Introduction}\label{sec:introduction}

Muyu is a Papuan language from the Ok branch of the Trans New Guinea family, located in the central lowlands of West Papua (Indonesia) close to the international border with Papua New Guinea (PNG). It is currently spoken by an estimated 2,000 speakers although the language is highly endangered due to its lack of transmission to a younger generation. 

The area where the language is spoken is characterised by dense rain forest. People nowadays settle in around 20 villages at the rivers Kao and Muy, which follow a north-south direction and flow into the Digul river near Tanah Merah. Whereas the inhabitants were traditionally scattered all over this swampy area, living in single houses or small hamlets, modern villages are a result of resettlement promoted by missionaries and government officials since the middle of the 20\textsuperscript{th} century. Most villages are accessible via road from Tanah Merah, the major city of the Boven Digul regency, within 3 to 4 hours by motorcycle.

Muyu has at least six dialects. These are: Aree, Kawiyet, Okpari, Kakaip, Kamindip, Yonggom. Dialects of Muyu vary greatly in their lexicon and sometimes in phonetic detail. Little is known, however, about morphosyntactic variation yet. In any case, speaker identities are strongly tied to the respective dialect and speakers enjoy explaining the differences to outsiders. This collection focuses exclusively on the \textit{Kawiyet} dialect spoken in the two villages Upyetetko and Kanggewot in the central Muyu area. 

The Muyu language is in close contact with Wambon, an Awyu-Dumut language that is spoken at the west side of the Kao river. Marriage between Muyu and Wambon speakers have been reported from early on and continue until today. A decent command of Wambon was probably normal for many Muyu of previous generations. However, the preferred means of communication today is Indonesian, to which both languages are rapidly losing ground. While contact with Wambon is well attested, little can be said about contact with other Ok languages, such as Ngalum in the Star Mountains to the north or Ninggerum across the international border in PNG.

The remainder of this chapter is structured as follows. \sectref{sec:muyu_and_ok_languages} briefly pre\-sents the Muyu language in its genealogical context, before I introduce the Muyu people in \sectref{sec:the_muyu_people}. \sectref{sec:about_this_collection} gives information about this collection, including the gathering and processing of the texts, before I explain some orthographic and glossing conventions in \sectref{sec:orthography_and_glossing}. The main focus will be on \sectref{sec:short_outlines_of_the_texts}, in which the content of each text is briefly summarised. Finally, \sectref{sec:typological_overview} offers a typological overview of the Muyu language.

\section{Muyu and the Ok languages}
\label{sec:muyu_and_ok_languages}

Muyu (ISO \textit{kti} for North Muyu, \textit{kts} for South Muyu)\footnote{The glottocodes are nort2916 for North Muyu and sout2940 for South Muyu.} is a Papuan language of the Ok language family, which is part of the larger group of Trans-New Guinea languages (TNG). TNG spans over large parts of mainland New Guinea and is located mainly across the highlands, but also south of it. Ok lies central within the TNG area and the appertaining languages are spoken at both sides of the international border between Indonesia and Papua New Guinea. 

Since \textcite{healeySurveyOkFamily1964}, Ok has been subdivided into two major branches: Mountain Ok and Lowland Ok. Muyu belongs to the latter. \figref{fig:Ok_tree} gives a tentative family tree for Ok. Notice that Ngalum is not definitively classified yet and may either belong to one of the two branches or represent a branch on its own.\footnote{At the moment of publishing this text collection, the author is working on a first description of Ngalum. His first impression is that the Ngalum structure is somewhat of a link between Mountain Ok and Lowland Ok. It seems to contain features from both branches. Of course, it is not clear yet what this means with regard to classification.} As can be seen, the Mountain Ok branch seems to be more diverse than Lowland Ok, comprising five languages: Mian, Faiwol, Tifal, Telefol and Bimin.

\begin{figure}
    \centering
%     \includegraphics[width=0.9\linewidth]{figures/Language_Family_Ok.png}
    \begin{forest}for tree={grow'=east,forked edges}
      [Ok family
        [Mountain Ok
            [Mian, tier=language]
            [Faiwol, tier=language]
            [Tifal, tier=language]
            [Telefol, tier=language]
            [Bimin, tier=language]
        ]
        [Ngalum, tier=language]
        [Lowland Ok
            [Muyu, tier=language]
            [Yonggom, tier=language]
            [Ninggerum, tier=language]
        ]
      ]
    \end{forest}
    \caption{Family tree of the Ok languages}
    \label{fig:Ok_tree}
\end{figure}

The Lowland Ok languages currently known are Muyu, Yonggom, and Ninggerum. However, Yonggom and Muyu seem to be two dialects of the same language (cf. \cite{christensenYongkomReferenceGrammar2013}, also the Ninati dialect in \cite{drabbeTalenDialectenVan1954}). The linguistic landscape in the area between Muyu and Ngalum to its north is still largely unknown. The varieties spoken in this rough terrain at the foothills of the Star Mountains could either be varieties of the already known languages or turn out to be separate languages in their own right.

The Ok language family is probably not a direct descendant of TNG. Genealogical links between Ok and Oksapmin, a language spoken east of the Ok languages, have been shown by \textcite{loughnaneOksapminOkStudy2011} who propose a Greater Ok grouping. Similarly, a genealogical link with the Greater Awyu languages has been proposed by \textcite{voorhoeveAsmatKamoroAwyuDumutOk2005}. However, \textcite{van-den-Heuvel:2014rp} studied this link in more detail and explained similarities between the individual languages by language contact rather than genealogically.

\section{The Muyu people}
\label{sec:the_muyu_people}

The Muyu people inhabit a swampy area of dense rain forest and traditionally lived from gardening. As a staple food, they eat sago, a starch that is won from the sago palm in a labor-intensive process of chopping, grating and/or pounding, and washing. Besides sago, Muyu grow bananas, sweet potatos, and various kinds of vegetables. The diet is occasionally supplemented with meat and fish. Today the Muyu also plant some cash crops that are sold at the market in Tanah Merah, such as peanuts.

While traditional Muyu were self-sustaining garderners, many adults nowadays move to the cities and live on a wage. Non-indigenous education is becoming increasingly important and some Muyu adolescents even attend universities in Merauke or Jayapura. The Muyu villages have electricity and most of them have access to the internet via the 4G standard at least since 2020. Smartphones are also widespread in the villages.

A comprehensive ethnographic description of the Muyu people can be found in the book-length study of \textcite{Schoorl:1957pz}. Only some key points can be summarised here: Muyu social structure focuses on the nuclear family with the father as the head of the family. Larger social groups like lineages and clans exist but seem to have little influence on an individual's everyday life. The marriage pattern is exogamous and it is expected that a woman moves to her husbands village. Children were traditionally multilingual, speaking Muyu and at least one of the surrounding languages, but identified with the language of the father. Another important institution is the avunculate, i.e. a special bond between a male child and his maternal uncle. Land tenure in the Muyu area is tied to the individual. Gardens are inherited from the father and cultivated by members of the family. As for religion, the majority of Muyu adhere to the Roman Catholic Church. The first missionary was the Dutchman Petrus Hoeber, who began his work in the Muyu area in the middle of the 1930s and is still warmly remembered today.

\section{Typological overview of the Muyu language}
\label{sec:typological_overview}

This section gives a brief summary of the findings and serves as a first orientation towards understanding the annotations in the text collection. It is a condensed version of information provided in \citet{zahrerGrammarMuyu2023}.

\subsection{Phonology}

Muyu has a medium-sized phoneme inventory. There are 9 consonant phonemes: /b, t, k, m, n, ŋ, l, w, j/. There are no fricatives and the voicing of plosives is only allophonic variation. The liquid represented by /l/ varies from speaker to speaker between [r], [l] and [ɾ]. There are 5 vowels /a, e, i, o, u/. Vowel length is virtually not contrastive with the exception of a few minimal pairs that contrast in [a] vs. [a\textlengthmark]. Some verb roots contain a vowel that harmonises in quality to the vowel of the first suffix, e.g. \textit{w\uline{i}n-i} [go-\textsc{inf}], \textit{w\uline{u}n-un} [go-3\textsc{sg}.\textsc{f}], \textit{w\uline{a}n-an-un} [go-\textsc{irr}-3\textsc{sg}.\textsc{f}]. Typical syllable structures are V, VC, CV and CVC. Consonant clusters are avoided by the language. Both disyllabic and trisyllabic words outnumber monosyllabic words. Words with more than three syllables are rare and probably due to compounding. Unlike other languages of the Ok family, Muyu does not have lexical tone. Furthermore, there seems to be no lexical stress. Prosodic features take place in larger units, e.g. falling and rising intonation to signal finality or continuation in complex sentences (see \citealt{zahrer2024exploring}).

\subsection{Word classes}

The basic distinction is between verbs and nouns. Verbs are morphologically complex and take a broad range of affixes, both for argument indexing and some aspect/mood categories. Nouns, on the other hand, are morphologically simple and mostly unmarked, with the exception of an optional plural morpheme \textit{-a} that can be added only to kin terms. Muyu also has a class of adjectives that can be defined by their post-nominal position in the noun phrase (see below: §\ref{subsec:noun_phrase_syntax}). Like nouns, adjectives are not inflected. Muyu has two genders: masculine and feminine, which are mostly found at pronouns and argument indexes at the verb. However, there is no NP-internal agreement or agreement with inanimate nouns. Closed word classes are pronouns, adverbs, directionals, postpositions, quantifiers, conjunctions, interjections and particles. The pronouns distinguish number (singular vs. plural) as well as gender (masculine vs. feminine) in the second and third person singular. Contrary to other Ok languages, Muyu does not have an inclusive/exclusive distinction in the first person pronouns.

Muyu speakers make extensive use of interjections. Most interjections are simple vowels or diphtongs in combination with a peculiar movement in pitch (often a sudden rise). These elements are very frequent, since they signal the beginning of direct speech, which is ubiquitous in Muyu narratives. For example, a quoted question would be marked like this: \textit{``Oh, where are you?'', he asked}. In dialogs with fastly alternating turns, each change of speaker is indicated with an interjection at the beginning of the turn.

\subsection{Demonstratives}

Demonstratives have various syntactic functions and therefore cannot be described as a single word class. Muyu comprises a rich inventory of more than thirty demonstratives, encoding such information as distance to the deictic center, elevation, visibility, etc. The basic system is binary and comprises many demonstrative pairs beginning with \textit{e-} 'proximal, i.e. close to the center' and \textit{o-} 'neutral, i.e. not close to the center'. The deictic center can be set relative to the context but is mostly conceptualised as the current speaker. Table \ref{tab:demonstratives} lists only the simple demonstratives together with their functions and glosses. Note that for most pairs the proximal form (\textit{e-}) has an extended gloss (e.g. `.this' or `.here') in comparison to the neutral form. Semantically, the proximal forms are more specific in Muyu (see \citealt{zahrerGrammarMuyu2023}: Ch. 5). In practice, however, the proximal/distal contrast mostly boils down to `this' vs. `that' in the English translations. 

\begin{table}
    \caption{Inventory of simple Muyu demonstratives}
    \label{tab:demonstratives}
    \begin{tabularx}{\textwidth}{XXl}
        \lsptoprule
            \textsc{forms} & \textsc{functions}  & \textsc{gloss}              \\ \midrule
            \textit{edo / odo}       & \begin{tabular}[c]{@{}l@{}}pronominal,\\ adnominal,\\ relative,\\ identificational\end{tabular} & \textsc{dem}.this / \textsc{dem} \\ 
            \midrule
            \textit{ege / ogo}       & \begin{tabular}[c]{@{}l@{}}pronominal,\\ adnominal\\ relative,\\ adjunct\end{tabular} & \textsc{dem}.here / \textsc{dem} \\ 
            \midrule
            \textit{eya / oya}       & \begin{tabular}[c]{@{}l@{}} adverbial, \\ adnominal\end{tabular} & here / there \\ 
            \midrule
            \textit{eyamin / oyamin} & adverbial & until\_here / until\_there \\
            \textit{éyom / oyom}     & adnominal & \textsc{com}.\textsc{dem}.this / \textsc{com}.\textsc{dem} \\
            \textit{eyóm}          & adverbial & in\_here \\
            \textit{eyot / oyot}     & adnominal & \textsc{contr}.\textsc{dem}.this / \textsc{contr}.\textsc{dem} \\
            \textit{eyot / oyot}     & conjunctional & \textsc{caus}.\textsc{dem}.this / \textsc{caus}.\textsc{dem} \\
            \textit{eyen$\sim$iyen / oyen}     & copula     & this\_is, here\_is / that\_is, there\_is  \\
            \textit{embet / ombet}   & oblique    & \textsc{obl}.\textsc{dem}.this / \textsc{obl}.\textsc{dem} \\
            \textit{eye$\sim$iye / oye}       & adnominal  & \textsc{poss}.\textsc{dem}.this / \textsc{poss}.\textsc{dem} \\
            \textit{ekuni / okuni}   & predicate  & like\_this / like\_that \\ 
        \lspbottomrule
    \end{tabularx}
\end{table}

Typologically interesting is the existence of a demonstrative copula pair \textit{eyen} `this is' and \textit{oyen} `that is', which are found in cleft-like constructions, as well as the demonstrative verb pair \textit{ekuni} `do/be like this' and \textit{okuni} `do/be like that', which are highly frequent and work as a coherence device to link actual and previous discourse.

\subsection{Verbal morphology}

As was stated earlier, verbs are morphologically complex. They are comprised of a root and one or several affixes. Verb roots pertain to three different conjugation classes (\textit{e-}, \textit{o-}, and \textit{a-}conjugation) that vary in inflectional properties and combinability with other morphemes. Virtually all affixes are suffixes, with the exception of one paradigm of object prefixes. Muyu has an interesting phenomenon that is found in many languages of the TNG family: While subject indexes are obligatory on all clause final verbs, object indexing is confined to only a handful of verbs. There are both, object prefixes and object suffixes, depending on the lexical item. One peculiarity is the verb for the meaning `hit/kill'. It does not have a formal verb root but fuses object prefix with subject suffix and potential other suffixes. To gloss the meaning `hit/kill', I segmented a zero morpheme ($\emptyset$) in all occurrences of this verb.
Besides argument indexing, we find bound morphemes for durative and perfective aspect, irrealis mood, and switch reference marking (see below: §\ref{subsec:clause_chaining_and_switch_reference}). Aspect is generally less salient a feature than in languages from the Mountain Ok branch (\citealt{healeySurveyOkFamily1964}). Irrealis mood denotes various meanings, among which the most important are: future events, possibility, ability, desired events, prohibited actions and habitual/ritualised events. In contrast to the morphologically marked categories mentioned above, some aspect/mood categories are formed periphrastically (see below: §\ref{subsec:complex_predicates}).

\subsection{Noun phrase syntax and postpositional phrases}\label{subsec:noun_phrase_syntax}

The Muyu NP can be described as having several slots, consisting minimally in a head noun or pronoun. Although there is a pre-head slot that is reserved for possessors and prenominal relative clauses, most noun phrases begin with their head, typically a noun. If further elements are present, we find the following order: noun $\rightarrow$ adjective $\rightarrow$ numeral/quantifier $\rightarrow$ demonstrative. Relative clauses can be either prenominal or postnominal. Their morphosyntactic marking of relative clauses varies with their position (prenominal RCs have a linker \textit{ye}, whereas postnominal RCs have a demonstrative after their verb). Muyu does not have articles but some adnominal demonstratives seem to help indicating specificity/definiteness of a referent when needed. Noun phrases can be coordinated with the conjunction \textit{yom} (e.g. \textit{ne yom Paul yom} `I and Paul') which also serves as comitative marker (e.g. \textit{ne, Paul yom} `I, together with Paul'). Disjunctive coordinantion of noun phrases (``X or Y'') is not possible in Muyu. Muyu postpositional phrases consist of a postposition and an NP as its complement. They are mostly used to denote spatial relations: \textit{adimbon yimin} `on the bridge', \textit{meja tap} `under the table', \textit{manggo kobi} `on the Manggo tree'. Muyu shows a syntactic similarity regarding postpositional phrases. When the NP inside the phrase is determined by a demonstrative, the NP is discontinuous and order becomes: noun $\rightarrow$ postposition $\rightarrow$ demonstrative, as in \textit{kudu tap eyani} [car under down.here] `under this car'. This seems to be a residue from the nominal origin of Muyu postpositions.

\subsection{Clause syntax}

The common order in intransitive clauses is SV, while transitive clauses have mostly AOV order. Objects can be shifted to clause-initial position though. Muyu clearly prefers a maximum of one overt argument, which means that subjects are often not expressed overtly in transitive clauses. Ditransitive clauses are rare in Muyu. The only clear cases are found with the verb \textit{ka-} `give', which indexes the receiver via suffix. However, three overt noun phrases in a Muyu clause are highly exceptional and only found in elicitational data. An important feature of Muyu is the marking of oblique arguments with \textit{=bet}. This clitic has many different functions. An oblique argument can denote: locations, goals of movements, sources of movements, instruments, temporal adjuncts, partitives and agentive subjects (a.k.a. optional ergative). An overview with example phrases is given in Table \ref{tab:functions_of_obliques}.

% Functions of =bet
\begin{table}
\caption{Functions of obliques marked with \textit{=bet}.}
\label{tab:functions_of_obliques}

\begin{tabularx}{\textwidth}{Xll}
    \lsptoprule
    \textsc{function} & \textsc{example} & \textsc{translation} \\ \midrule
    Location &              \textit{Kanggewot=bet} & `in Kanggewot' \\
    Goal of movement &      \textit{ok Inggo kaba=bet} & `to the estuary of the river Inggo' \\
    Source of movement &    \textit{ambip=bet} & `from the house' \\
    Instrument &            \textit{alung=bet} & `with a crowbar' \\
    Temporal &              \textit{adon alopmim=bet} & `at three o'clock' \\
    Partitive &             \textit{om=bet} & `from the sago' \\
    Agentive subject &      \textit{adi=bet} & `it was father who ...'\\
    \lspbottomrule
\end{tabularx}
\end{table}

Non-verbal clauses can have an optional copula clitic \textit{=an} or the demonstrative copula (\textit{eyen} `this is' / \textit{oyen} `that is'). If aspect or mood marking is required, the copula is replaced with the light verb \textit{keli} `become'. Existential clauses make use of an existential particle \textit{aip} or \textit{yanop}. As for negation, the clause final negation particle \textit{balin} is immediately succeeding the verb.

\subsection{Verbal number}

A typologically interesting feature is the alternation of verb stems depending on participant number (subject or object) as well as event number. Muyu comprises more than thirty verb pairs that take part in the verbal number system. Here are two examples:

\newpage
\ea\label{ex:verbal_number_1}
    Yum kane kemun.\\
    \gll yum \textbf{kan}-e kem-un \\
    banana take:\textsc{sg}.\textsc{o}-\textsc{sm} do-3\textsc{sg}.\textsc{f} \\
    \glt  `She used to take a banana.'
\z
\ea \label{ex:verbal_number_2}
    Yum be kemun. \\
    \gll yum \textbf{b}-e kem-un \\
    banana take:\textsc{pl}.\textsc{o}-\textsc{sm} do-3\textsc{sg}.\textsc{f} \\
    \glt `She used to take (several) bananas.'
\z

As can be seen from (\ref{ex:verbal_number_1}) and (\ref{ex:verbal_number_2}), the choice of verb stem selects for the intended number value of the object argument. Similar pairs determine subject number (\textit{koke-} `one X falls' vs. \textit{kombil-} `many X fall' ) and event number (\textit{bane-} `call once' vs. \textit{yongga-} `call several times'). Details on this phenomenon can be found in Chapter 8 of \textcite{zahrerGrammarMuyu2023}.

\subsection{Complex predicates}\label{subsec:complex_predicates}

Predicates that consist of more than one lexical element are abundant. There are three types: (a) light verb constructions have a nominal main element preceding a semantically impoverished light verb as in (\ref{ex:complex_pred_1}), (b) multi-verb constructions combine two or more verbs which contribute equally to the overall meaning of the predicate, such as the caused accompanied motion construction in (\ref{ex:complex_pred_2}), (c) auxiliary verb constructions combine a main verb and an auxiliary to denote certain aspect/mood categories periphrastically, such as the desiderative construction in (\ref{ex:complex_pred_3}). As can be seen from the latter two the examples, only the final verbs are fully inflected, whereas non-final verbs take a generic suffix \textit{-e} `\textsc{sm}, i.e. serial marker'.

\ea\label{ex:complex_pred_1}
    Kep olok tain. \\
    \gll kep olok ta-in \\
    2\textsc{sg}.\textsc{m} longing \textsc{lv}-1\textsc{sg} \\
    \glt `I miss you.'
\z

\ea\label{ex:complex_pred_2}
    Yum be minip. \\
    \gll yum b-e min-ip \\
    banana take:\textsc{pl}.\textsc{o}-\textsc{sm} come-2/3\textsc{pl} \\
    \glt `They brought some bananas.'
\z

\newpage
\ea\label{ex:complex_pred_3}
    Yum kanane kemun. \\
    \gll yum kan-an-e kem-un \\
    banana take:\textsc{sg}.\textsc{o}-\textsc{irr}-\textsc{sm} do-3\textsc{sg}.\textsc{f} \\
    \glt `She wants to take a banana.'
\z

As for the auxiliary constructions, Muyu expresses the following aspect/mood categories periphrastically: continuative, habitual, inchoative, desiderative and completive. The topic of complex predicates in Muyu is quite extensive. The interested reader is referred to Chapters 10 and 11 in \textcite{zahrerGrammarMuyu2023}.

\subsection{Clause chaining and switch reference}\label{subsec:clause_chaining_and_switch_reference}

Muyu discourse is often structured in complex chains of clauses which together form a complex sentence. When a clause takes part in such a chain, only the final clause can bear the full verb morphology, whereas all others have \textit{medial verb forms}. Medial verbs are marked for switch reference, i.e. they bear markers for whether the next clause has a coreferential subject argument or not. Same subjects are marked with \textit{-n} [\textsc{ss}] or \textit{-gVl} [\textsc{ss}.\textsc{seq}]\footnote{The symbol V indicates that the vowel of this morpheme harmonises in vowel quality with the vowel of the subsequent suffix.}, whereas different subjects are marked with the clitic \textit{=e} [\textsc{ds}.\textsc{seq}].\footnote{The gloss \textsc{seq} indicates that in addition to subject continuity, temporal sequentiality can be marked as well. The presence of \textsc{seq} means that the events denoted by the clauses occur one after another, whereas the absence of \textsc{seq} means that there can be a temporal overlap.} For two basic examples, see (\ref{ex:claue_chain_1}) and (\ref{ex:clause_chain_2}). There are further optional clitics and conjunctions that can be added and the topic as a whole is rather complex. The interested reader is referred to Chapter 12 in \textcite{zahrerGrammarMuyu2023}.

\ea\label{ex:claue_chain_1}
    Be kane menone, menggamben. \\
    \gll b-e kan-e meno\textbf{-n}-e mengga-mb-en\\
    take:\textsc{pl}.\textsc{o}-\textsc{sm} take:\textsc{sg}.\textsc{o}-\textsc{sm} come-\textsc{ss}-3\textsc{sg}.\textsc{m} cook\_on\_leaf\_oven-\textsc{pl}-3\textsc{sg} \\
    \glt `[Having caught many fish that day] he collected, brought, and cooked them.'
\z

\ea\label{ex:clause_chain_2}
    Kaduk kawene wipe, taun. \\
    \gll kaduk kawen-e w-$\emptyset$-ip\textbf{=e} ta-un\\
    man climb-\textsc{sm} 3\textsc{sg}.\textsc{f}-hit-2/3\textsc{pl}=\textsc{ds}.\textsc{seq} die-3\textsc{sg}.\textsc{f} \\
    \glt `[While the husband was out hunting] people came up (to the house) and beat her and then she died.'
\z

\subsection{Tail-head linkage and anaphoric verbs}

The discourse strategy commonly known as tail-head linkage is ubiquitous in monological texts from Muyu. In Muyu, two sentences can be linked by repetition. A final part of a sentence (the ``tail'') is repeated at the beginning of a new sentence (the ``head''), as seen in (\ref{ex:thl_1}).\footnote{Here I adopt the convention that the first part of the linkage is underlined, while the second part is bold.} This is mainly used to enhance continuity between two larger discourse units, often chains of clauses (see above). Morphosyntactical marking between tail and head often differs, since they are both embedded in different syntactic contexts, i.e. often a final verb in the tail and a medial verb in the head. The size of the repeated unit varies, ranging from single words to verbatim repetition of whole clauses.

\ea\label{ex:thl_1}
    Apyop odo kane \uline{kulubun}. \textbf{Kulubenu} ...\\
    \gll apyop odo kan-e kulub-un kulube-n-u (...) \\
    fruit \textsc{dem} take:\textsc{sg}.\textsc{o}-\textsc{sm} show-3\textsc{sg}.\textsc{f} show-\textsc{ss}-3\textsc{sg}.\textsc{f}\\
    \glt ‘She took the fruit and showed it. She showed it and then she...’
\z

A similar strategy to link discourse units consists in the use of demonstrative verbs \textit{ekuni} `do/be like this' and \textit{okuni} `do/be like that' as well as the less frequent pair \textit{ekemi} `do this' and \textit{okemi} `do that'. The following example is from the text about the Benem rat (\textref{text:benem_rat}, lines \ref{ex:text9-1389}-\ref{ex:text9-70}).

\ea\label{ex:anaphoric_verb}
    amun wagolet \uline{tien}. \textbf{Okune} tien go, ena edo tinggi ap ulukap ketma bet yile mene,\\
    \gll amun	wa-gol-e	=t	ti-en okun-e	ti-en	=go	ena	edo	tinggi	ap	ulukap	ketma	=bet	yil-e	men-e\\
    nest	make\_nest-\textsc{ss}.\textsc{seq}-3\textsc{sg}.\textsc{m}	=and\_then.\textsc{ss}	stay-3\textsc{sg}.\textsc{m}    like\_that-\textsc{sm}	stay-3\textsc{sg}.\textsc{m}	=\textsc{ptc}	mother	\textsc{dem}.this	hand	tree	aerial\_root	side	=\textsc{obl}	reach\_out-\textsc{sm}	come-\textsc{sm}\\
    \glt `it made a nest and stayed there. It stayed like that, so mom put her hand in the aerial root from one side...'\\
\z

The first verb of the second sentence in (\ref{ex:anaphoric_verb}) refers anaphorically back to the previous clause predicate, thereby linking the two clause chains.

\subsection{The clitic particle \textit{=ko}}

This particle deserves separate clarification since it is abundant in all Muyu speech. Unfortunately, its functions are so diverse that I did not come up with a transparent gloss yet. The texts simply gloss \textit{=ko} as `\textsc{ptc}, i.e. particle'. So the following notes should make it more transparent. Basically, \textit{=ko} has a syntactic function to signal the end of a syntactic unit. At the same time, it has an information structural function to indicate that discourse still continues. The combination of these two functions causes different pragmatic effects, depending on context and the nature of the clitic's host. For example, between two clauses it can serve as a simple clause coordinator, while at the end of an uncontinued sentence, it usually turns a statement to a yes-no question similar to English tag-questions (`... right?'). Attached to noun phrases it serves to mark the phrase boundary which can be very useful in disambiguation, e.g. \textit{kep tana} `your child / you are a child' (ambiguous) vs. \textit{kep ko tana} `You are a child' (unambiguous). In contrast to unmarked NPs, those marked with \textit{=ko} tend to have definite reference more often, e.g. \textit{om anem} `Let's eat sago' vs. \textit{om ko anem} `Let's eat the sago' (a specific amount). At the beginning of a sentence and in combination with a prosodic break, \textit{=ko} often marks topicalisation. Between verbs in complex predicates, \textit{=ko} serves to mark boundaries between two juxtaposed VPs, which has ramifications on event structure.\footnote{Two VPs can be modified with adverbs separately.} Finally, \textit{ko} can also occur as a free form and serve as a discourse marker when speakers wish to change the topic (like `okay, enough, let's move on'). Probably derived from this function is the use of a free standing \textit{ko} as a repair strategy for wrongly assigned switch reference in clause chaining.\footnote{For example, if the speaker uses a same subject marker but then plans a clause with a different subject, he can repair the chain by closing it with \textit{ko} and resume with the other subject.} For examples of all these functions and further explanation, the reader is referred to \textcite{zahrerGrammarMuyu2023}.

\section{About this collection}
\label{sec:about_this_collection}

This collection is a special edition of texts that were recorded and transcribed in the course of a larger documentation project funded by the Endangered Languages Documentation Programme.\footnote{ELDP grant ID: IGS0367} The full corpus was collected during 13 months of fieldwork in the years 2019-2024 and is fully accessible at the Endangered Languages Archive (\cite{zahrerDocumentationMuyuLowland2019}).\footnote{ELAR, available at: https://www.elararchive.org/dk0601/} In addition to the corpus, the project resulted in a comprehensive grammatical description of the Muyu language (\cite{zahrerGrammarMuyu2023}).

The collection represents a subcorpus of 10 monological narratives that were specially edited for the tenets and goals of Open Text Collections. Each text is outlined in \sectref{sec:short_outlines_of_the_texts} below. They were selected by three guiding principles: First, they should be clustered around a common topic. Two thematic groups were established by comparing the available texts: myths of origin and stories of animal encounters.\footnote{In addition, the archived corpus contains procedural texts about horticulture, personal biographies, stimulus based speech, personal memories, reflections on life in the village, and much more.} Second, English translations should be available.\footnote{At the time of this publications, not all texts in the archive are translated. Furthermore, many texts have exclusively Indonesian translations which are generally more accessible to the community.} Third, the texts should have a reasonable length. All in all, this left me with the 10 texts presented in this volume. An unfortunate consequence is that there is no gender balance in the collection. All speakers are male, which also reflects the gender imbalance in the archived material.\footnote{During my fieldtrips women were generally much more reluctant to be recorded, for reasons yet unclear to me.} The following two subsections summarise the general workflow and some details of the editing process for OTC.

\subsection{The general workflow}
\label{subsec:general_workflow}

Prior to publication, the texts were mainly used as primary data for grammatical and lexical analyses. Transcriptions and translations, as well as grammatical annotations were made entirely within ELAN. All annotations were done manually. However, these original ELAN files were rarely considered as full texts with a continuous story line from beginning to end, leading to occasional inconsistencies within one and the same file. Therefore, they had to undergo intensive editing for the publication in this collection. Thus the edition of the texts can be split into two phases, the pre-OTC phase and the OTC phase. The whole process is outlined in \figref{fig:processing_flowchart}. The two subsequent phases shall be discussed in turn.

\begin{figure}
    \centering
    \includegraphics[width=0.9\linewidth]{figures/OTC_processing_flowchart.png}
    \caption{Outline of the editing process in two phases.}
    \label{fig:processing_flowchart}
\end{figure}

For each text, the pre-OTC phase of editing began at recording. Each narrative was recorded on video with a Canon XA40 camera and a lavalier microphone attached to the speaker's shirt. The recording was then segmented in ELAN, and transcribed and translated into English or Indonesian. The goal at that time was to produce translations that were closer to the original text instead of offering perfect idiomatic translations. In this way, grammatical analysis could be facilitated. All this work was performed while in the field. The next step included the grammatical annotation of each recording, which was done over the course of several months parallel to the grammatical analysis of the collected material. In this phase, mistakes in the transcripts and translations were corrected. Additionally, the segmentation was completely revised, since now the emerging grammatical analysis would give insights about clause and sentence boundaries. The final phase of the documentation project was devoted to writing the PhD thesis in Zahrer (\citeyear{zahrerGrammarMuyu2023}). During that time, many transcripts, translations and glosses were revised, especially for parts that were used as examples in the grammar. This resulted in a corpus of very heterogenous quality and thus posed a challenge for the OTC phase of editing.

The OTC phase began as soon as I decided to submit a contribution to the Open Text Collections. Since the recordings had not been edited with publication in mind, this phase required serious effort. First, I needed to select texts that were both interesting to the prospective readers and of such quality that the editing effort remained manageable. The result was a collection of 10 texts from two broad topics: mythological narratives and stories about animal encounters. Second, each text had to undergo editing of the transcript, translation, and the glosses. Finally, a running parallel text was prepared such that for each text the original text and a more idiomatic translation could be presented side by side (in the `Parallel text' subsections of the individual chapters).

\subsection{Editing efforts}
\label{subsec:editing_efforts}

The \textbf{transcript} needed editing in several ways. The orthography was not consistent through all texts, most importantly the separation of words needed harmonisation. Then typographic symbols were carefully revised. In the original files, a comma represents a prosodic break while a full stop signals finality (roughly: falling pitch and discontinuation). However, these were only guiding principles and not strictly exercised throughout the corpus. For the purposes of the OTC edition, punctuation needed to be handled more strictly. In addition to comma and full stop, question marks are used, as well as quotation marks. Quoted speech is a frequent phenomenon in all texts I have encountered so far. Not only do Muyu speakers frequently quote speech from others, the language itself requires quotative constructions for certain grammatical functions, such as desiderative modality.\footnote{For example, instead of `He wanted to go hunting' a Muyu speaker would often say something like `Let me go hunting, he said.'} This revision was done in the ELAN files. This collection itself has two differing ways of handling punctuation. The full punctuation is visible in the parallel text sections that were prepared to guarantee readability. However, commas and full stops were removed from the interlinearised texts to follow best practices in linguistic annotation.

Another major issue was phenomena like production errors, slips of the tongue, mispronunciations, false starts, unnecessary repetitions, and the like. As is typical for spontaneous speech, the Muyu corpus is full of imperfect language production. In the editing process, such elements illustrate the sometimes conflicting goals between a linguistic corpus and an edited text collection. A full linguistic transcript needs to include all speech that was produced, regardless of acceptability, grammaticality, or similar aspects. However, the requirement for OTC was different, since the final result should consist of easily readable texts. Thus, I cleaned the texts from instances that were clearly identified as faulty production. Many of the existing repair sequences had already been identified and marked in the original translations. Another problem was the production errors that went unrecognized by the speakers themselves. For example, the occasional use of an incorrect subject index. Such cases were identified and commented upon by my language consultants in a separate note tier in the original files. In this collection, I simply replaced the erroneous elements in the transcript. The whole cleanup process was performed as carefully as possible. A decision was often not easy and if in doubt, I opted to leave the element in question unaltered in the transcript. 

The final issue of the transcript has to do with phonological processes and phonetic production. Muyu speakers tend to elide certain phonemes in fast speech. For example, \textit{tana edo} `this child' is often pronounced as [ta.na.d\textopeno]. In the original files, elided sounds are represented but the corresponding graphemes are put in parentheses: \textit{tana (e)do}. This way, the reader can identify the original element and has information about the actual pronunciation at the same time. However, information about pronunciation seemed less important for this collection and thus I removed all such parentheses, transcribing \textit{tana edo} in the OTC. As a consequence, the orthographic transcript represents words in their full forms, as if the speaker would use careful speech all the time.

The \textbf{translation} was edited to represent less the ductus of the original text and more idiomatic English instead. While a stronger similarity with the Muyu text facilitated the structural analysis of individual examples, many translations seemed inappropriate in the broader context of this collection. One serious challenge was the different word order between Muyu and English. When a Muyu sentence stretched over multiple segments, the verb would come only at the end. Verbs were occasionally moved to earlier segments if this increased the readability of the English translation. Another issue was that arguments are often left implicit in Muyu clauses. Since English mostly requires an overt NP where Muyu does not, additional NPs were added in the translations – mostly in the form of short pronouns, e.g. \textit{Tem-an} [see-1\textsc{sg}] `I saw \textbf{it}'. If the respective referent has not been mentioned for a while, a full NP was used instead to remind the reader of the referent. In this case, the supplemented NP was set in parentheses, e.g. \textit{Tem-an} [see-1\textsc{sg}] `I saw \textbf{(the fish)}'. Finally, some subject pronouns in the translation were set in parantheses if there was neither an argument index nor a free pronoun in the Muyu original. This can happen in long chains of multi-verb constructions, where the subject is mostly retrieved from context. The parentheses reminds the reader that there is no corresponding morpheme in the original text, e.g. \textit{tem-e...} [see-\textsc{sm}] `\textbf{(I)} saw it and...'. 

As for the \textbf{glosses}, the texts were far from fully glossed. On the one hand, there were missing values, for example when a certain lexeme was not identified yet.\footnote{In contrast to yet unidentified lexical morphemes, grammatical morphemes were rarely an issue, since the grammatical description has gotten considerable attention in the pre-OTC phase. However, in some rare cases, I could not make sense of a certain bound morpheme in the particular context. Such morphemes were glossed with `?'.} On the other hand, the existing glosses were not all up-to-date considering the latest analyses and insights. Therefore, all glosses needed thorough revision. Missing glosses were filled with the help of the native speaker Yakobus Woman who was available for remote fieldwork sessions via Messenger.

Glossing conventions also needed slight adaptation. For example, glosses with multiple English words used a dot as connective (e.g. `carry.on.shoulder'). The OTC version uses underscores instead (e.g. `carry\_on\_shoulder'). This way, lexical meanings can be easily distinguished from glosses with multiple grammatical meanings (e.g. `3\textsc{sg}.\textsc{m}').

The final step of the OTC phase was to prepare the \textbf{running parallel text and translation} that could be presented separated from the glossed original texts. As it turned out, this required further editing on both parts, the transcripts and the translations. The transcripts were stripped of punctuation as explained above. The translations were further condensed to more idiomatic English. The requirements for literal translation were less restricting here, since there are no matching glosses that guide choices in the translations. For example, tail-head linkage (see \sectref{sec:typological_overview}) was mostly reduced and repetitions that did not suit the ductus of English were removed. Simlarly, many clauses are linked with anaphoric verbs that originally translated to `He/She/They did like that and then...'. Such instances were paraphrased to more English-sounding constructions such as `After doing that...'. Furthermore, many cases where Muyu relies on quotative structures were paraphrased in the running translation. For example, an agreement originally stated as `{``}Yes'', they said' was simply rendered `They agreed'. Next I removed all parentheses from the translations (see above) and represented all arguments overtly. Cumbersome descriptions and insertions were also cleaned. For example, a passage originally stating `So, the wives, the two women, yes the wife, since there were two women, they made the sago' was simply paraphrased as `So his two wives made the sago'. Finally, I removed many interjections which did not seem necessary for a running translation.

The editing of the running parallel text introduced an unexpected problem. The original ELAN files contained continuous speech. The interlinearised text could just follow this flow and present line after line. In contrast, an edited written text needed segmentation into paragrahs. This required editorial choices, since there are no clear cut-off points in oral speech that would naturally translate to written paragraphs. I tried to organise each text into paragraphs depending on the content and set breaks according to indicators such as change of location, change of main participants etc. However, it is important to recognise that this is a purely subjective result for which I took the English translation as a basis. In no way should the paragraph breaks be related to any kind of structure in the recorded original speech.

\section{Orthography and glossing}
\label{sec:orthography_and_glossing}

This section deals with orthographic symbols in the transcriptions (\sectref{subsec:orthographic_symbols}) and some important conventions for the glossing tier (\sectref{subsec:glossing_conventions}). Abbreviations used in the glosses are listed in a separate section in the frontmatter.

\subsection{Orthographic representation}
\label{subsec:orthographic_symbols}

Muyu is an endangered language and rarely written by the language community. Occasional social media posts suggest that Muyu speakers tend to apply their orthographic knowledge from school where they are educated in Indonesian. It was therefore necessary to devise a Muyu orthography for the documentation project. This orthography is also used in the OTC collection.

The orthographic representation in the transcripts is straightforward.  \tabref{tab:symbols} lists all graphemes and their underlying phones as well as example words that contain the respective symbol.

\begin{table}
\caption{Orthographic representations of the sounds.}
\label{tab:symbols}
    \begin{tabularx}{.78\textwidth}{llX}
        \lsptoprule
            \textsc{grapheme} & \textsc{phone} & \textsc{example words}\\ 
        \midrule
            \textless{}a\textgreater{}  & [a]& \textit{ambip} `house', \textit{am} `rain'\\
            \textless{}e\textgreater{}  & [ɛ]& \textit{ep} `puddle', \textit{et} `eight'\\
            \textless{}i\textgreater{}  & [i]& \textit{it} `theft', \textit{ine} `tomorrow'\\
            \textless{}o\textgreater{}  & [ɔ]& \textit{otbop} `language', \textit{on} `bird'\\
            \textless{}u\textgreater{}  & [u]& \textit{up} `wind', \textit{umkan} `blood'\\ 
        \midrule
            \textless{}p\textgreater{}  & [p]& \textit{otbo\textbf{p}} `language'                                           \\
            \textless{}b\textgreater{}  & [b]& \textit{bon} `place', \textit{ben} `hand'\\
            \textless{}t\textgreater{}  & [t]& \textit{temi} `to see', \textit{tim} `louse'\\
            \textless{}d\textgreater{}  & [d]& \textit{dim} `smoke', \textit{dip} `kind of bird'\\
            \textless{}k\textgreater{}  & [k]& \textit{kun} `heavy', \textit{kabak} `ax'\\
            \textless{}m\textgreater{}  & [m]& \textit{mom} `uncle', \textit{mini} `to come'\\
            \textless{}n\textgreater{}  & [n]& \textit{nin} `snake', \textit{nong} `string'\\
            \textless{}ng\textgreater{} & [ŋ]& \textit{nong} `string',\\
            \textless{}g\textgreater{}  & [g], [ɣ]& \textit{kumu\textbf{g}ap} `moist', \textit{alalu\textbf{g}op} `song' \\
            \textless{}l\textgreater{}  & [l], [r], [ɾ] & \textit{talep} `big', \textit{belon} `small'\\
            \textless{}w\textgreater{}  & [w]& \textit{wini} `egg', \textit{kowot} `wall'\\
            \textless{}y\textgreater{}  & [j]& \textit{yongbon} `garden', \textit{ye} `he'\\
        \lspbottomrule
    \end{tabularx}
\end{table}

In addition to the symbols listed in Table \ref{tab:symbols}, the transcripts also contain \textless h\textgreater{}. This grapheme is used together with vowel graphemes to signal an interjection, e.g. \textit{ih!} or \textit{ah!} are long vocalic interjections. The \textless h\textgreater{} does not correspond to any phoneme.

Common words are mostly written in lower-case letters. In contrast, upper-case letters were used for proper nouns, such as human poper names (e.g. \textit{Bruno}), names of villages (e.g. \textit{Awin}) or rivers (e.g. \textit{Kowo}), and also for the name of species of animals and plants (e.g. \textit{Yom} = a kind of bat, \textit{Kimit} = a kind of palm tree).\footnote{Species were not further identified in this text collection. Since the documentation of Muyu did not focus on botanical and zoological knowledge, the author did not attempt any amateurish classification with Latin terms.} Additionally, upper-case letters also indicate the start of a new sentence after a full stop in the parallel texts.

Punctuation in the parallel texts roughly represents prosodic breaks in speech intonation. A comma indicates a break with non-final intonation, i.e. either rising or level pitch, whereas a full stop indicates a break with final intonation, i.e. falling pitch. Such breaks can be followed by a silence, i.e. a pause, or emerge as sudden pitch jumps in faster speech. The interlinarised texts do not contain such symbols.

Quotations and direct speech are always set between double quotations marks, both in the transcription line and the translation line. Each line in the glossed examples has its own quotation marks, even if it is part of a longer direct speech spanning over multiple lines.

The orthography also captures some morphonological processes like intervocalic lenition. For example, the onset of \textit{=ket} is softened in intervocalic position and therefore transcribed as \textit{=get}. However, this is not applied to lexical morphemes since they should be clearly identifiable in all contexts, e.g. the final consonant of \textit{tip} becomes [w] in \textit{tip=an} `It is good' but remains [p] in \textit{tip kem-ip} `You did well'. Both cases are transcribed as \textit{tip}.

\subsection{Glossing conventions}
\label{subsec:glossing_conventions}

Lexical units are segmented at morpheme breaks and glosses are offered in a separate tier for each morpheme. I largely follow the Leipzig glossing rules. Lexical meanings are glossed with corresponding English approximations, while grammatical meanings are glossed with the symbols and abbreviations listed in the frontmatter. Proper nouns (e.g. names of humans, rivers or villages) are simply glossed with `\textsc{pn}'.

If the lexical meaning of a morpheme cannot be glossed by a single English word, several words are concatenated with underscores. For example, \textit{tanggal-} is glossed `put\_tongue\_out'.

Morphemes that convey several grammatical meanings (i.e. portmanteaus) are glossed in such a way that symbols for all grammatical meanings are concatenated with dots. For example, \textit{-gol} is glossed `\textsc{ss}.\textsc{seq}' (i.e. same subject and sequential time order). Additionally, dots are used to specify the function of an auxiliary verb, i.e. \textsc{aux}.\textsc{cont} for an auxiliary that forms a continuative.

Morphemes that combine lexical and grammatical meaning are glossed in such a way that the two types of glosses are concatenated with a colon. Lexical glosses precede grammatical glosses. Within each type, the regular concatenation symbols (underscores or dots) are used. This is especially important for the feature of verbal number, where the verb stem encodes information about the number of participants or event. For example, \textit{kan-} is glossed `take:\textsc{sg}.\textsc{o}' (i.e. lexical meaning `take' and the grammatical meaning selecting for `singular objects') or \textit{yongga-} `call:several\_times' (i.e. lexical meaning `call' and the grammatical meaning of multiple events).

The recordings contain many instances of code mixing between Muyu and Indonesian. Indonesian words are glossed with the extension `(\textsc{bi})' (i.e. Bahasa Indonesia) added to the lexical meaning. For example, Indonesian \textit{kelapa} is glossed as `coconut(\textsc{bi})'

Multi-word expressions sometimes have idiomatic meanings. If the meaning of individual elements is no longer transparent or the overall meaning is too far removed from a hypothetical compositional meaning, I glossed each element with the meaning of the whole. For example:

\ea\label{ex:multi-word_expression}
bane walon \\
\gll ban-e wal-on \\
prepare-\textsc{sm} prepare-3\textsc{sg}.\textsc{m} \\
\glt `He prepared it.'
\z

Although the individual words in (\ref{ex:multi-word_expression}) exist, no individual meaning seems to fit the multi-word expression.\footnote{The stem \textit{ban-} can mean either `call someone', `open something', `grow (fungus)' or `take a part of something'. The stem \textit{wal-} regularly means `to stop, end'.}

\section{Outlines of the texts}
\label{sec:short_outlines_of_the_texts}

The texts presented here can be divided into two thematic clusters. First we have texts that are myths about the origin of some natural entity, such as a certain kinds of animals, trees or even the moon (Texts \ref{text:moon}-\ref{text:giant}). The second cluster revolves around animal encounters, such as hunting, fishing or the occasional catching of mice (Texts \ref{text:diving_for_fish}-\ref{text:muying_rat}). The outlines in this section are of medium length. For an even more concise overview, each text begins with a brief summary at the start of the respective chapter. 

\subsection*{\textref{text:moon}: The origin of the moon}

% \textit{Speaker:} Patrisius Enip \\
% \textit{Date of recording:} 30.07.2019 \\
% \textit{Location of recording:} Upyetetko village \\
% \textit{ELAR ID:} muyu007 \\
% \textit{Length}: 1239 token words \\

This story tells how the moon came into being, involving a woman and her brother who went to the forest to make sago. Over several days, they felled sago trees, prepared the sago, and ate it. Tired of only eating sago, the brother decided to catch fish using a bow and arrow. He cooked the fish and saved some for his sister, but she insisted they wait for her husband before eating the sago. The brother thought they should eat the sago with the fish since they were in the forest to find sago.

Despite her objection, he continued to hunt fish and cook them with sago pulp, but he wasn't happy. He devised a plan involving thorns, bees, and rough objects, which he wrapped into a large package and hung from a tree. He tricked his sister into standing under it, pretending there was a bird's nest he would drop. When she stood in the right spot, he dropped the package, which hurt her.

He then stuck out his tongue, making the moon come closer, and climbed onto it, becoming the moon. His sister, realizing what had happened, cried and missed him. When her husband returned, she told him the whole story. In his grief and anger, he killed her with a stone adze, cutting her into two halves, which became a beetle and a spider.

The spider's nest is an imitation of a bird's nest, reflecting her fate. The beetle comes out when the moon rises, making a sound. The husband transformed into a Kookaburra bird, crying out when the moon rises, signaling the arrival of his brother-in-law, the moon.

This ancestral story explains how the moon came to be and why certain creatures behave the way they do, highlighting themes of familial conflict and transformation.

\subsection*{\textref{text:coconut}: The origin of the coconut}

% \textit{Speaker:} Adrianus Yamuk \\
% \textit{Date of recording:} 05.08.2019 \\
% \textit{Location of recording:} Upyetetko village \\
% \textit{ELAR ID:} muyu031 \\
% \textit{Length}: 905 word tokens \\

This story narrates the origin of the coconut tree, beginning with a young man and his wife who lived in a house with many dogs. The man hunted pigs with the help of his dogs, who would catch and kill the pigs while he brought them back to cook and eat with his wife. The dogs, however, were only given the bones, which they began to resent, feeling they deserved more for their efforts.

One day, the man went hunting but did not return. The dogs came back alone, leading the wife to worry and search for him the next day. She discovered that the dogs had eaten him and thrown his head into the river. Devastated, she cleaned three palm tree ribs and fashioned them into a tool to retrieve his head from the river. She succeeded in retrieving it and buried it next to their house.

In time, a plant resembling a palm tree sprouted from the burial site. As it grew, the woman realized it was not a palm tree but a coconut tree. The tree flourished, producing leaves and fruit. One morning, she heard the sound ``Bian-Bian-Bian'' and saw a wild chicken perched on the tree. She took some seeds from the coconut tree and planted them by the Bian River, fencing the area to protect them. The tree bore many fruits, which eventually fell into the river and floated downstream.

People living downstream discovered the coconuts, took them, and planted them. The coconut trees spread widely, and people learned to pierce the coconuts to drink the water inside, a practice called \textit{yopke} in the Komoyan subdialect and \textit{ok anem} in the subdialect of the narrator. This story, often told by the narrator’s mother, explains how coconuts were discovered and spread, and it reflects the customs and language of their ancestors.

\subsection*{\textref{text:bat}: The origin of the bat}

% \textit{Speaker:} Adrianus Yamuk \\
% \textit{Date of recording:} 05.08.2019 \\
% \textit{Location of recording:} Upyetetko village \\
% \textit{ELAR ID:} muyu032 \\
% \textit{Length}: 1177 word tokens \\

This story recounts the origin of bats and their connection to sago. It begins with a man and his wife, whose brother had cut down a sago tree for them before leaving for the Awin region. The couple remained behind, often making sago and hoarding it, refusing to share with others. This selfish behavior led to the frustration of their sons.

The four brothers, tired of not getting any sago, decided to transform into bats. They gathered materials like Genemon sticks and taro bars, using them to create bat-like structures with wings, claws, and other features. The eldest brother became a Yom bat, another became a Kudumbop bat, another a Dimin bat, and the last a Tinggulut bat.

They practiced flying and perfected their bat forms. One evening, they hung themselves from the rafters of their house, staying silent and still. When their parents returned and found the sago in disarray, they were furious. As the children revealed their bat forms, they flew out of the house. The parents tried to stop them but only managed to injure the Kudumbop bat, giving it a lean waist.

The story goes on to explain the different types of bats: the Kudumbop bat makes a ``wauk-wauk" sound, the Dimin bat makes a ``peet-peet" sound, and the Tinggulut bat is small and often seen flying around in the evening. The transformation into bats was a result of the children's frustration with their parents' selfishness regarding the sago.

The narrative also includes a cautionary aspect about not eating sago before hunting bats, as it is believed to bring bad luck and make the bats fly away. In the modern era, hunting bats has become easier with foreign guns, but the traditional belief and practices still hold significance.

The story, passed down from ancestors, is a reminder of the importance of sharing and the cultural beliefs surrounding hunting and sago. It highlights the connection between human actions and the natural world, as well as the lessons taught by elders to younger generations.

\subsection*{\textref{text:sisters_birds}: Sisters becoming birds}

% \textit{Speaker:} Xaverius Kerenjop \\
% \textit{Date of recording:} 16.10.2019 \\
% \textit{Location of recording:} Merauke city \\
% \textit{ELAR ID:} muyu065 \\
% \textit{Length}: 627 word tokens \\

Long ago, in the village of Kati, before the arrival of outsiders, a family resided with their two daughters, the younger and the older. They nurtured the children until they reached maturity. One day, they were tasked with crafting string bags and sago bags for a community celebration. The sisters, pondering their futures, decided to venture into the Genemon trees to gather vegetables and other provisions for the journey. As dusk fell, they camped, feasting on their prepared food before continuing their journey at dawn.

Their conversation turned to aspirations of marriage as they traversed the path. Arriving at the grove, they diligently gathered the Genemon trees' offerings until nightfall left them disoriented. They sought a way home but found themselves lost, stumbling upon a garden with ripe red pandanus fruits. Ignoring the garden's ownership, they indulged, unaware of the owner's approach.

Upon discovery, the owner startled them with a bowstring's twang, prompting them to flee. As they escaped, the older sister's attempts to follow her sibling led to multiple branch breaks until she landed on the ground. Unable to join her sister above, she resigned herself to the earth, transforming into a cassowary.

Meanwhile, the younger sister, nimble and light, soared into the sky as a Mambruk bird. Their unusual fate became a cautionary tale passed down through generations, emphasizing the importance of heeding parental advice and the consequences of disobedience when venturing into the unknown.

\subsection*{\textref{text:giant}: A violent giant}

% \textit{Speaker:} Xaverius Kerenjop \\
% \textit{Date of recording:} 16.10.2019 \\
% \textit{Location of recording:} Merauke city \\
% \textit{ELAR ID:} muyu067 \\
% \textit{Length}: 1166 word tokens \\

Long ago, a man had two wives. One day, they went to find sago, with the man cutting down a sago tree while his wives processed it. As they worked, the man went hunting and killed a pig. Meanwhile, a giant overheard the women squeezing sago and approached them. He tricked one woman into looking into his bag, captured her, and killed her. He then did the same to the second wife, placing both bodies in his large carrying bag.

As evening fell, the man returned to find the sago unfinished and his wives missing. Carrying the pig and sago, he set off and eventually encountered the giant. The giant suggested they make a fire and sleep, each on opposite sides of a large tree. During the night, the man saw the giant devour one of his wives. Determined to seek revenge, the man cooked the pig, re-tied the bag with faeces in it instead of pork, and left it on the fire.

The next morning, while the giant slept, the man took the cooked pork, crossed the Kowo River with a small canoe, and hid it. He sat on the other side of the river eating sago and pork. The giant, upon waking, found the bag empty and realized he had been tricked. He followed the man's tracks to the riverbank and saw him across the river.

The giant tried to cross, but the man deceived him into believing there was ford. As the giant waded into the river, the water reached his chest, and he eventually sank and drowned. This story is often told to children as a lesson: those who do bad things will ultimately face consequences. The tale serves as a moral warning about the repercussions of harmful actions.

\subsection*{\textref{text:diving_for_fish}: Diving for fish}

% \textit{Speaker}: Galus Kolop \\
% \textit{Date of recording}: 06.08.2019 \\
% \textit{Location of recording}: Upyetetko village \\
% \textit{ELAR ID}: muyu035 \\
% \textit{Length}: 533 word tokens \\

In the late afternoon, a pair of friends gathered, preparing to cook young cassava leaves, when unexpectedly, Brother Kosmas arrived. He informed them of a fish he had spotted at the Nimi River the day before, igniting their curiosity. They abandoned their cooking plans and set out for the river with makeshift fishing gear. With tenong rattan strands and a tree branch, they devised a clever method to lure the fish out of its cave. Despite their efforts, Brother Kosmas and Ansel failed to find the fish after diving into the cold waters. Disheartened but determined, the narrator took another plunge and eventually spotted the elusive fish under a stone. Armed with a spear, they attempted to catch it but initially failed. However, after a second try, they successfully speared the fish and brought it ashore. Back at home, they decided to cook the fish whole and share it among themselves, honoring their friendship and the adventure they had shared.

\subsection*{\textref{text:shooting_a_cassowary}: Shooting a cassowary}

% \textit{Speaker}: Galus Kolop \\
% \textit{Date of recording}: 06.08.2019 \\
% \textit{Location of recording}: Upyetetko village \\
% \textit{ELAR ID:} muyu037 \\
% \textit{Length}: 910 word tokens \\

The story describes an adventure where the narrator, along with his friends Lukas and Agus, went to the forest near the Oga River to search for Nik fruits. They decided to split up to cover more ground. While the narrator was on his path, he saw a cassowary, and shot and killed it. He then removed a feather and placed it in his hat before heading back to find Lukas.

Lukas was setting a fish net in the river, and the narrator decided to play a prank on him by pretending he had found a dead woman instead of the cassowary. Lukas initially believed the story but soon realized it was a joke when the narrator mentioned the cassowary. They decided to call Agus, who was also searching for Nik fruits. The two went to find the cassowary while the narrator stayed behind to surprise Agus.

Once they located the cassowary, they tied it and placed it in the river to keep it fresh. They continued their search for Nik fruits but found only a few. Deciding to return, they retrieved the cassowary, plucked its feathers, and butchered it by the riverside. Each took a share of the meat.

Feeling hungry, they made a fire, cooked sago, and roasted some meat. Although one of them chose not to eat the meat, the others did. After their meal, they drank water and returned to the village. This story highlights their hunting skills, camaraderie, and playful spirit during their adventure.

\subsection*{\textref{text:fishing_trip}: A fishing trip}

% \textit{Speaker:} Patrisius Enip \\
% \textit{Date of recording:} 06.08.2019 \\
% \textit{Location of recording:} Upyetetko village \\
% \textit{ELAR ID:} muyu038 \\
% \textit{Length}: 2024 word tokens \\

This story recounts a trip to Kowo River with brother Bruno Kakut. After school, they decided to go to the Kowo River, planning to stay at the Widi River, then to Udaben, the Kilok Kone estuary, and back. They prepared supplies including tobacco, coconuts, salt, MSG, and a small cooking pot, and informed their parents of their plans. Their parents advised them to stay together.

They began their journey, staying at Widi River for five nights, then moved upstream to Udaben for five days, fishing and catching shrimp. They used shrimp traps and fishing lines, catching many fish. The journey continued to Kilok Kone estuary, where they set shrimp traps and fished. While at Kilok Kone, they caught many shrimp and fish. After five days, they decided to return to Widi River instead of staying longer at Udaben. They brought the fish back to their parents at Widi River, who were pleased with the catch. The mothers organized the fish, sharing it with the family and neighbors.

The next day, the boys prepared to return to the village, carrying peanuts and other supplies. They were reminded by their elders of the importance of sharing abundance with friends and family. They set off back to the village, ending their fishing adventure.

This story emphasizes the values of family, cooperation, and sharing within the community.

\subsection*{\textref{text:benem_rat}: The Benem rat}

% \textit{Speaker:} Yakobus Wonam \\
% \textit{Date of recording:} 07.08.2019 \\
% \textit{Location of recording:} Upyetetko village \\
% \textit{ELAR ID:} muyu039 \\
% \textit{Length}: 415 word tokens \\

This story recounts an adventure involving the narrator, their mother, Yamuk, and friends as they journeyed to the estuary of the Wolok River. They left their village well-prepared and began their trek. As they approached the estuary, the narrator’s mother decided to check a spot where a Benem rat usually nested, asking the others to continue along the road.

After a short while, the mother called out to the narrator and Yamuk, saying she had found the rat and needed help to catch and kill it. The two hurried to her location and found the rat hiding in the aerial roots of a tree. Despite her efforts, the mother couldn’t catch the rat alone and asked for their assistance.

The narrator and Yamuk each reached into the aerial roots from opposite sides. The narrator mistakenly grabbed his mother’s thumb, thinking it was the rat’s snout, and tried to break it. The mother, feeling the pain, struggled to free her hand. Realizing the mistake, the narrator finally understood he had been trying to break his mother’s thumb instead of the rat.

Throughout the confusion, the mother exclaimed in pain and clarified that the narrator was holding her hand, not the rat. The humorous yet painful misunderstanding was resolved, and the story ends with the narrator reflecting on the incident. This tale highlights the family's adventurous spirit and the close bond shared during their escapades.

\subsection*{\textref{text:muying_rat}: The Muying rat}

% \textit{Speaker:} Yakobus Wonam \\
% \textit{Date of recording:} 07.08.2022 \\
% \textit{Location of recording:} Upyetetko village \\
% \textit{ELAR ID:} muyu040 \\
% \textit{Length}: 358 word tokens \\

This story recounts an experience between the narrator and his brother Lukas involving a Muying rat. They were at the Wolok estuary where Lukas had planted a large patch of peanuts. As the peanuts matured, Lukas began harvesting them, forming small ``islands'' of harvested plants.

The narrator often played around while Lukas worked, throwing clumps of sand at him, and Lukas would playfully try to hit back. During one of these playful moments, the narrator noticed something unusual in one of the peanut ``islands.'' Upon closer inspection, he discovered it was a rat’s nest.

Excitedly, the narrator told Lukas about the nest, thinking there might be a rat inside. Lukas encouraged him to catch it. Hesitant but curious, the narrator approached the nest. As he reached into it, he was bitten by the Muying rat, causing his hand to bleed. Lukas found the situation amusing and laughed, while the narrator, in pain and upset, went back home crying.

This story highlights the playful and sometimes painful adventures of childhood, as well as the bond between the narrator and his brother Lukas.


\subsection*{Overview of the texts}

\tabref{tab:texts} provides an overview of the texts in this collection. Note that the table includes a hyperlink to the archived source files at ELAR.

\begin{xltabular}{.7\textwidth}{llXr}
    \caption{The texts in this collection.} \label{tab:texts}\\

    \lsptoprule
        \textsc{text} & \textsc{title} & \textsc{elar id} & \textsc{words}\\
    \midrule
    \endfirsthead

    \multicolumn{4}{c}%
    {\tablename\ \thetable{} -- continued from previous page}\\
    \lsptoprule
        \textsc{text} & \textsc{title} & \textsc{elar id} & \textsc{words}\\
	\midrule
    \endhead

    \hline \multicolumn{4}{r}{{Continued on next page}}\\
    \endfoot

    
    \endlastfoot

		\ref{text:moon}&The origin of the moon &\href{http://hdl.handle.net/2196/111815f4-2946-4cf0-ae86-06066f0646b8}{muyu007}&1239\\
        \ref{text:coconut}&The origin of the coconut &\href{http://hdl.handle.net/2196/7cbc7412-b750-4188-870f-3ec2d6696720}{muyu031}&905\\
        \ref{text:bat}&The origin of the bat&\href{http://hdl.handle.net/2196/7d200203-c1bf-4a23-88be-afa71ab8cb0c}{muyu032}&1177\\
        \ref{text:sisters_birds}&Sisters becoming birds&\href{http://hdl.handle.net/2196/c508d439-d46a-4ebb-9cb5-ebefc25346e2}{muyu065}&627\\
        \ref{text:giant}&A violent giant&\href{http://hdl.handle.net/2196/913b3ee6-b0cb-493d-8253-c583675fa2c5}{muyu067}&1166\\
        \ref{text:diving_for_fish}& Diving for fish&\href{http://hdl.handle.net/2196/aa8f4115-7cee-403e-83b2-7bbb345448a2}{muyu035}&533\\
        \ref{text:shooting_a_cassowary}&Shooting a cassowary&\href{http://hdl.handle.net/2196/5864b9f0-9bd9-43b4-afd6-33ad8a0bac56}{muyu037}&910\\
        \ref{text:fishing_trip}&A fishing trip&\href{http://hdl.handle.net/2196/0dfa459f-5529-431a-ad50-8cc69f62fd34}{muyu038}&2024\\
        \ref{text:benem_rat}&The Benem rat&\href{http://hdl.handle.net/2196/00ede06c-7a60-4ca0-a752-c3e84097f518}{muyu039}&415\\
        \ref{text:muying_rat}&The Muying rat&\href{http://hdl.handle.net/2196/83ca5b5d-922e-4bd9-b6bb-345dc9814ac9}{muyu040}&358\\
    \midrule    
        \textbf{total}&&&\textbf{9354}\\
    \lspbottomrule
\end{xltabular}
