\newcommand{\appref}[1]{Appendix \ref{#1}}
\newcommand{\fnref}[1]{Footnote \ref{#1}} 

\newenvironment{langscibars}{\begin{axis}[ybar,xtick=data, xticklabels from table={\mydata}{pos}, 
        width  = \textwidth,
	height = .3\textheight,
    	nodes near coords, 
	xtick=data,
	x tick label style={},  
	ymin=0,
	cycle list name=langscicolors
        ]}{\end{axis}}
        
\newcommand{\langscibar}[1]{\addplot+ table [x=i, y=#1] {\mydata};\addlegendentry{#1};}

\newcommand{\langscidata}[1]{\pgfplotstableread{#1}\mydata;}

% 
% \makeatletter
% \let\theauthor\@author %store in more accessible format
% \makeatother 
%  
% \newcommand{\lsFirstAuthorFullName}{}%temporary, will be overwritten
% \newcommand{\lsFirstAuthorFirstName}{}%temporary, will be overwritten
% \newcommand{\lsFirstAuthorLastName}{}%temporary, will be overwritten
% \newcommand{\lsFirstAuthorString}{\lsFirstAuthorLastName, \lsFirstAuthorFirstName} %can be customized in localmetadata.tex
% \newcommand{\lsNonFirstAuthorsString}{}  %default, will be overwritten iff more than one author
% \newcommand{\lsImpressionCitationAuthor}{\lsFirstAuthorString \lsNonFirstAuthorsString}
% 
% 
% \renewcommand{\and}{NONLASTAND} %expand for easier checking. Might need to be undone later on
% \renewcommand{\lastand}{LASTAND} %expand for easier checking
% 
% \IfSubStr{\theauthor}{NONLASTAND}{%2+authors
%   \renewcommand{\lsFirstAuthorFullName}{\StrBefore{\theauthor}{\and }}
%   \renewcommand{\lsFirstAuthorFirstName}{\StrBefore{\theauthor}{ }} 
%   \renewcommand{\lsFirstAuthorLastName}{\StrBetween{\theauthor}{ }{\and }}
%   \renewcommand{\lsNonFirstAuthorsString}{\and\StrBehind{\theauthor}{\and }} 
%   }{%else
%     \IfSubStr{\theauthor}{LASTAND}{%less than two authors, more than one
%     \renewcommand{\lsFirstAuthorFullName}{\StrBefore{\theauthor}{\lastand }}
%     \renewcommand{\lsFirstAuthorFirstName}{\StrBefore{\theauthor}{ }}
%     \renewcommand{\lsFirstAuthorLastName}{\StrBetween{\theauthor}{ }{\lastand }}
%     \renewcommand{\lsNonFirstAuthorsString}{\lastand\StrBehind{\theauthor}{\lastand }} 
%     }{%else exactly one author
%       \renewcommand{\lsFirstAuthorFirstName}{\StrBefore{\theauthor}{ }}
%       \renewcommand{\lsFirstAuthorLastName}{\StrBehind{\theauthor}{ }}
%       }
%     }   

% \tikzstyle{pie chart}=[
%     chart,
%     slice/.style={line cap=round, line join=round, very thick,draw=white},
%     pie title/.style={font={\bf}},
%     slice type/.style 2 args={
%         ##1/.style={fill=##2},
%         values of ##1/.style={}
%     }
% ]

% \pgfdeclarelayer{background}
% \pgfdeclarelayer{foreground}
% \pgfsetlayers{background,main,foreground}


%% Code from https://tex.stackexchange.com/a/451154
\newif\ifOut
%\Debugtrue
\Outfalse
\pgfkeys{/pgf/.cd,
Move Out/.initial=1,
Move Out=1,
Minimum Angle/.initial=0}
\newcounter{myslice}
\makeatletter
\renewcommand{\pgfpie@slice}[8]{\stepcounter{myslice}
  \pgfmathparse{0.5*(#1)+0.5*(#2)}
  \let\midangle\pgfmathresult

  \path (#8) -- ++(\midangle:#5) coordinate(O);

  \pgfmathparse{#7+#5}
  \let\radius\pgfmathresult

  \pgfmathsetmacro{\SwitchOff}{ifthenelse(#2-#1<\pgfkeysvalueof{/pgf/Minimum
  Angle},1,0)}

  % slice
  \draw[line join=round, fill=#6, \style] (O) -- ++(#1:#7) arc (#1:#2:#7) -- cycle;

  \pgfmathparse{min(((#2)-(#1)-10)/110*(-0.3),0)}
  \let\temp\pgfmathresult
  \pgfmathtruncatemacro{\itest}{ifthenelse(\pgfkeysvalueof{/pgf/Move Out}==1,1,0)}
  \ifnum\itest=1
  \pgfmathsetmacro\innerpos{((max(\temp,-0.5)+0.8)*#7)}
  \else
  \pgfmathsetmacro\innerpos{\pgfkeysvalueof{/pgf/Move Out}*#7*(1+0.2*sin(90*\themyslice))}
  \fi
  \ifthenelse{\equal{\pgfpie@text}{inside}}
  {
    % label and number together
    \path (O) -- ++(\midangle:\innerpos) node
    {\scalefont{#3}\shortstack{#4\\\beforenumber#3\afternumber}};
  }
  {
    % label
    \iflegend
    \else
    \path (O) -- ++ (\midangle:\radius)
    node[inner sep=0, \pgfpie@text=\midangle:#4]{};
    \fi

    % number
    \ifnum\SwitchOff=1
    \else
     \ifnum\itest=1
      \path (O) -- ++(\midangle:\innerpos) node
      {\scalefont{#3}\beforenumber#3\afternumber};
     \else
      \path (O) -- ++(\midangle:\innerpos)  node (pie-N-\themyslice)
      {\scalefont{#3}\beforenumber#3\afternumber};
      \draw[latex-] (\midangle:{0.85*#7}) -- (pie-N-\themyslice);
     \fi
    \fi
  }
}
\makeatother

% \newcommand{\pie}[3][]{
%     \begin{scope}[#1]
%     \pgfmathsetmacro{\curA}{90}
%     \pgfmathsetmacro{\r}{1}
%     \def\c{(0,0)}
%     \node[pie title] at (90:1.3) {#2};
%     \foreach \v/\s in{#3}{
%         \pgfmathsetmacro{\deltaA}{\v/100*360}
%         \pgfmathsetmacro{\nextA}{\curA + \deltaA}
%         \pgfmathsetmacro{\midA}{(\curA+\nextA)/2}
% 
%         \path[slice,\s] \c
%             -- +(\curA:\r)
%             arc (\curA:\nextA:\r)
%             -- cycle;
%         \pgfmathsetmacro{\d}{max((\deltaA * -(.5/50) + 1) , .5)}
% 
%         \begin{pgfonlayer}{foreground}
%         \path \c -- node[pos=\d,pie values,values of \s]{$\v\%$} +(\midA:\r);
%         \end{pgfonlayer}
%         \global\let\curA\nextA
%     }
%     \end{scope}}
% 
% \newcommand{\legend}[2][]{
%     \begin{scope}[#1]
%     \path
%         \foreach \n/\s in {#2}
%             {
%                   ++(0,-10pt) node[\s,legend box] {} +(5pt,0) node[legend label] {\n}
%             }
%     ;
%     \end{scope}
% }


\newcommand{\sourcebox}[1]{\parbox{.8\textwidth}{\tiny\raggedright#1}}