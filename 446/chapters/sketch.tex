\chapter{Grammatical overview}\label{ChSketch}

This chapter serves an introductory sketch of the Mandan language and its grammar. The principle goal of this chapter is to provide a succinct highlight of the major aspects of the grammar presented in subsequent chapters without having to consume all the prose therein. One purpose of this book is to be useful to both Mandan community members and to linguists, and it is worthwhile to prime both sets of potential readers to ease their understanding of the data presented throughout. Rather than just provide summaries of the ensuing chapters, this chapter groups major linguistic domains together and provides either the most fundamental information to know about within that domain or the most common grammatical processes that might not be immediately obvious when looking at examples throughout this book.

\section{Sound inventory}\label{CHSk1}

The topic of the phonetics and phonology of Mandan is discussed at length throughout Chapter \ref{chapter2}. This section provides an inventory of the salient sounds found in Mandan and their orthographic representation within the present work.

\subsection{Consonants}\label{CHSk1.1}

Mandan has 10 phonemic consonants. We can see an inventory of these consonants below, represented in IPA and orthographic representations in angled brackets where IPA and orthography differ.

\begin{table}
\caption{Consonant inventory} \label{ChSKconsonantinventory}
\begin{tabular}{lccccc}
\lsptoprule
          & Bilabial & Alveolar & Postalveolar & Velar & Glottal \\
\midrule
Plosive   & p        & t        &               & k     & ʔ $\langle$'$\rangle$    \\ 
Fricative &          & s        & ʃ $\langle$sh$\rangle$          & x     & h       \\ 
Sonorant  & w        & ɾ $\langle$r$\rangle$         &          	&       &         \\ \lspbottomrule
\end{tabular}
\end{table}

The actual number of consonants differs between several scholars, with some arguing for 10, while others analyze Mandan as having as many as 13 (\citealt[2]{kennard1936}, \citealt[14]{hollow1970}, \citealt[6]{mixco1997a}). The work presented in this book assumes a 10-consonant analysis.\footnote{See \sectref{inventory} for a discussion of why other analyses are ruled out.}

There are no voiced obstruents in Mandan. Instead, the only voiced consonants are /w/ and /ɾ/. There are other allophonic manifestations of the consonants above in \tabref{ChSKconsonantinventory}, but sound alternations are discussed in \sectref{CHSk2} below. Mandan lacks the different stop and fricative series found in other Siouan languages, e.g., Osage phonemically distinguishes between plain stops, post-aspirated stops, pre-aspriated stops, glottalized stops, and voiced stops \citep[25]{quintero1997}.

\subsection{Vowels}\label{CHSk1.2}

Mandan retains the Proto-Siouan vowel system \citep[367]{rankinetalnd}, in that it has five oral vowels of two lengths and three nasal vowels of two lengths. We see an inventory of phonemic vowels in Mandan below, where the vowels are represented in IPA and noting that each vowel below also has a long counterpart.

\begin{figure}
\subfigure[oral vowels\label{CHSkoralvowelchart}]{
\begin{vowel}
	\putcvowel[l]{i}{1}
	\putcvowel[r]{u}{8}
	\putcvowel[l]{e}{2}
	\putcvowel[r]{o}{7}
	\putcvowel[l]{a}{4}
	\end{vowel}
}
\subfigure[nasal vowels\label{CHSknasalvowelchart}]{
\begin{vowel}
	\putcvowel[l]{ĩ}{1}
	\putcvowel[r]{ũ}{8}
	\putcvowel[l]{ã}{4}
\end{vowel}
}
\caption{Mandan vowels}
\end{figure}

There are no nasal mid vowels in Mandan, so there is not a perfect symmetry between the oral vowels and the nasal vowels. Mandan is like most other Siouan languages in this respect \citep[106]{parksrankin2001}, though there are languages within the family with fewer phonemic nasal vowels, e.g., Omaha--Ponca has only a front nasal and a non-front nasal vowel of varying realization \citep[12]{Koontz1984}, while Crow and Hidatsa have no nasal vowels whatsoever \citep[19]{grimm2012}.

\newpage
Mandan has a single diphthong /au̯/, though it appears only in one lexical item, \textit{háu} [ˈhau̯] `hello, yes.' This term is likely a loanword, given its prolific use across Indigenous languages of the Great Plains and northern Eastern Woodlands.\footnote{See \sectref{subsectiondiphthongs} for additional description of the origin of this diphthong in Mandan and its phonetic characteristics.}

\subsection{Orthography}\label{CHSk1.3}

Various orthographies have been used to represent Mandan since \citet{kipp1852} first wrote down a small Mandan lexicon. The orthography used in this book stems from the one used in \citet{kasak2019}, which itself was based on an earlier version of the conventions used by the Nueta Language Initiative during the winter of 2015.%\footnote{See \sectref{SecOrthography} for an elaboration on the choices that went into chosing the orthography used here.}

There is a large overlap between the Americanist Phonetic Alphabet (APA) and the orthography of Mandan. Mandan consonants are generally identical to their corresponding APA symbol, e.g., the voiceless velar fricative [x] is $\langle$x$\rangle$. The voiceless postalveolar fricative [ʃ] is represented by the digraph $\langle$sh$\rangle$, given the familiarity most Mandan community members have with English orthographic conventions. The glottal stop is represented by an apostrophe throughout this book.

The orthographic representation of vowels likewise conforms to their APA symbols. Nasality is marked on vowels with an underhook rather than a tilde, e.g., high front unrounded nasal vowel [ĩ] is orthographically $\langle$į$\rangle$. When a nasal vowel is preceded by a [m] or [n], then there is no marking of nasality on the vowel, since such vowels will be inherently nasal. Earlier orthographies such as the one used in \citet{hollow1973a,hollow1973b} made all nasal vowels orthographically explicit, but marking nasality on vowels following [m] or [n] is considered redundant. Long vowels are represented as double-vowel digraphs, e.g., long high back rounded oral vowel [uː] is written as $\langle$uu$\rangle$. An acute accent indicates a primary stress on a syllable. Where primary stress falls on long vowels, the first vowel in the digraph receives the acute accent, e.g., [ˈkaːɾe], the negative imperative marker `don't', is written as $\langle${káare}$\rangle$.

A lengthier discussion of the Mandan orthography used throughout this work can be see in \sectref{SecOrthography}. It bears mentioning that the use of the present orthography does not imply that other orthographic representations of Mandan are less accurate or somehow not as good as the one used throughout this book. I encourage any Mandan community members who have been using a different orthography to continue to work with the system that they have already developed or have learned.

\section{Morphophonology}\label{CHSk2}

Mandan is an agglutinative language, where numerous formatives can accrete onto a stem. The combination of formative often results in the juxtaposition of phonemes that are disallowed by the phonotactics of the Mandan language.

\subsection{Consonantal lenition and syncope}\label{CHSk2.1}

The phonotactic constraints of Mandan prohibit sequences of the same consonant appearing in surface representations. Supralaryngeal stops like /t/ and /k/ become fricatives when followed by an identical segment. This lenition serves to dissimilate pseudogeminate sequences, as we see in (\ref{CHSkEx1}) and (\ref{CHSkEx2}) below.\footnote{See \sectref{geminatedissimilation} for further discussion on the topic of pseudogemination dissimilation.}

\begin{exe}
    \item\label{CHSkEx1} Lenition of /t+t/ to [st]
        \begin{xlist}
                \item\label{CHSkEx1a} /ũʔt/ `far away' + /=taː/ locative enclitic $\to$ \\{[}ˈũʔstaː] `long ago, far away'
                \item\label{CHSkEx1b} /i-ɾu-hĩt/ `tan hide by hand' + /=ta/ male imperative $\to$\\
                {[}ˈiɾuhĩsta] `soften that hide!'
                \item\label{CHSkEx1c} /huː/ `come here' + /=ɾĩt=ta/ second person plural plus imperative $\to$\\
                {[}ˈhuːnĩsta] `come here, you all!'
        \end{xlist}
    \item\label{CHSkEx2} Lenition of /k+k/ to [hk]
        \begin{xlist}
            \item\label{CHSkEx2a} /suk/ `exit' + /=ka/ habitual enclitic $\to$\\
            {[}ˈsuhka] `to always exit'
            \item\label{CHSkEx2b} /kok/ `pronghorn' /kɾE=s/ third person plural plus definite article $\to$\\
            {[}ˈkohkᵉɾes] `the pronghorns'\footnote{The /E/ represents the ablaut vowel in Mandan, which is phonetically realized as either [e] or [a], depending on the surrounding morphology. See \sectref{ablaut} for an explanation of how ablaut is triggered in Mandan.}\textsuperscript{,~}\footnote{The superscript vowel here represents an excrescent vowel. See \sectref{dorseyslaw} for more information on excrescent vowels and Dorsey's Law in Mandan.}
            \item\label{CHSkEx2c} /i-sek/ `make, do' + /=ki/ conditional enclitic $\to$\\
            {[}ˈisehki] `if he did it'
        \end{xlist}
\end{exe}

Identical fricative sequences delete one segment to avoid pseudogemination, and the sonorants never occur as codas, so sonorants will never occur in environments that will trigger lenition or syncope. We can see the simplification of these clusters with sequences of identical fricatives in (\ref{CHSkEx3}) below.\footnote{See \sectref{SecPhonotactics} for more a more detailed explanation of Mandan phonotactics.}

\begin{exe}
    \item\label{CHSkEx3} Syncope of fricative + fricative sequences

    \begin{xlist}
        \item\label{CHSkEx3a} /wɾĩs/ `horse' + /se/ `red' $\to$\\
        {[}ˈm\textsuperscript{ĩ}nĩse] `a red horse'
        \item\label{CHSkEx3b} /wɾãʃ/ `tobacco' + /ʃi/ `be good' $\to$\\
        {[}ˈm\textsuperscript{ã}nãʃi] `good tobacco'
        \item\label{CHSkEx3c} /waːx/ `cottonwood' + /xwãh/ `be small' $\to$\\
        {[}ˈwaːxᵃmãh] `a small cottonwood'
        \item\label{CHSkEx3d} /kuh/ `come back here' + /hɾE/ causative $\to$\\
        {[}ˈkuhᵉɾe] `to make someone come back here'
    \end{xlist}
\end{exe}

These two morphophonological strategies for avoiding a violation on the constraint against pseudogeminates are incredibly productive and can be seen in word-building of different types, e.g., affixation, cliticization, compounding, \textit{inter alia}.

\subsection{Epenthesis and vocalic syncope}\label{CHSk2.2}

Mandan has a single diphthong, seen only in the word \textit{háu} /ˈhau̯/ `hello, yes.' In all other contexts, a vowel may never appear adjascent to another within the domain of the same word. In situations where one vowel abuts another, an epenthetic consonant is inserted to avoid a constraint against hiatus.

There are two varieties of epenthesis in Mandan. One variety occurs within the domain of a word, i.e., between affixes and a root or between words that comprise a compound. This first epenthesis inserts a glottal stop [ʔ] between adjoining vowels. The other variety occurs when enclitics are added to the right edge of a stem. This second epenthesis inserts a flap [ɾ] when a long vowel abuts another vowel at an enclitic boundary. We can see examples of these two epentheses in (\ref{CHSkEpenthesis1}) and (\ref{CHSkEpenthesis2}) below.\footnote{See \sectref{epentheticprocesses} for more information about these differing epentheses.}

\begin{exe}
    \item\label{CHSkEpenthesis1} /ʔ/ epenthesis within a word

    \begin{xlist}
        \item\label{CHSkEpenthesis1a} /ka-/ `by force' instrumental + /ux/ `be broken' $\to$\\
        {[}kaˈʔux] `to smash something apart'
        \item\label{CHSkEpenthesis1b} /waː-/ nominalizer + /iɾĩː/ `grow' $\to$\\
        {[}ˈwaːʔĩnĩː] `a plant, a growing thing'
        \item\label{CHSkEpenthesis1c} /i-/ instrumental preverb + /ah/ `be covered' $\to$\\
        {[}ˈiʔah] `skin, shell'
    \end{xlist}
    \item\label{CHSkEpenthesis2} /ɾ/ epenthesis at an enclitic boundary

    \begin{xlist}
        \item\label{CHSkEpenthesis2a} /huː/ `come here' + /=oʔʃ/ male indicative enclitic $\to$\\
        {[}ˈhuːɾoʔʃ] `he/she came here'
        \item\label{CHSkEpenthesis2b} /hĩː/ `drink' + /=oʔɾã/ female interrogative enclitic $\to$\\
        {[}ˈhĩːɾoʔnã] `did he/she drink it?'
        \item\label{CHSkEpenthesis2c} /haː/ `cloud, sky' + /=E/ stem vowel complementizer $\to$\\
        {[}ˈhaːɾe] `cloud, sky'
    \end{xlist}
\end{exe}

A third strategy for resolving hiatus in Mandan involves the syncope of the second vowel in sequences where an enclitic beginning with a short vowel is added to a short vowel-final stem. We can see examples of this strategy for hiatus resolution in (\ref{CHSkVowelSyncope}) below.

\begin{exe}
    \item\label{CHSkVowelSyncope} Vocalic syncope

    \begin{xlist}
        \item\label{CHSkVowelSyncopeA} /ʃi/ `be good' + /=oʔɾe/ female indicative enclitic $\to$\\
        {[}ˈʃiʔʃ] `it is good'
        \item\label{CHSkVowelSyncopeB} /hi/ `arrive there' + /=ak/ different-subject switch-reference marker $\to$\\
        {[}ˈhik] `he/she arrived and...'
        \item\label{CHSkVowelSyncopeC} /hũ/ `many' + /oʔʃa/ male interrogative enclitic $\to$\\
        {[}ˈhũʔʃa] `is there a lot?'
    \end{xlist}
\end{exe}

These three strategies occur frequently, given that the agglutinating nature of Mandan morphology often creates conditions where illicit vowel plus vowel sequences would otherwise appear.

\subsection{Nasal harmony}\label{CHSk2.3}
\largerpage
Even though the nasal consonants [m] and [n] are very common in Mandan, they only occur due to nasal assimilation of a following nasal vowel. Nasal vowels spread their nasality onto sonorant consonants like /w/ and /ɾ/, turning them into [m] and [n], respectively. The nasality spreads leftward from the vowel onto the sonorant preceding it, illustrated below in (\ref{CHSkNasal1}) by an underline in the nasal vowel in the surface form and the affected sonorant.

\begin{exe}
    \item\label{CHSkNasal1} Nasal assimilation from vowel to consonant

    \begin{xlist}
        \item\label{CHSkNasal1a} /ɾãːka/ $\to$ [ˈ\uline{nãː}ka] `be new'
        \item\label{CHSkNasal1b} /wĩːh/ $\to$ [ˈ\uline{mĩː}h] `woman'
        \item\label{CHSkNasal1c} /ɾo-ɾĩʔ/ $\to$ [ⁿdoˈ\uline{nĩ}ʔ] `to shoot us'
        \item\label{CHSkNasal1d} /o-ɾãp/ $\to$ [ˈo\uline{nã}p] `to find something'
    \end{xlist}
\end{exe}

Nasal assimilation in Mandan is not restricted to the local environment. In the examples in (\ref{CHSkNasal1}) above, nasality spreads leftward from a nasal vowel onto a single segment. However, nasal assimilation can continue to spread leftward as long as there are no blocking mechanisms. Nasality is blocked by obstruents, mid vowels, word boundaries, and preverbs.\footnote{See \sectref{nasalharmony} for a more articulated analysis of nasal harmony in Mandan and why preverbs act as a blocking mechanism for nasal assimilation.} Otherwise, nasalization can spread into another syllable, as we see in (\ref{CHSkNasal2a}) and (\ref{CHSkNasal2b}), or even across many syllables to the left edge of the word, as we see in (\ref{CHSkNasal2c}) below.

\begin{exe}
    \item\label{CHSkNasal2} Long-distance nasal assimilation

    \begin{xlist}
        \item\label{CHSkNasal2a} \glll \textnormal{[ˈik\uline{ãm\textsuperscript{ĩ}nĩ}]}\\
        /i-ka-wɾĩ/\\
        pv.ins-ins.frce-\textnormal{be.twisted}\\
        \glt `to twist something up'
        
        \item\label{CHSkNasal2b} \glll \textnormal{[\uline{mãˈnãː}teʔʃ]}\\
        /wa-ɾãːtE=oʔʃ/\\
        1a-\textnormal{stand.up}=ind.m\\
        \glt `I stood up'

        \item\label{CHSkNasal2c} \glll \textnormal{[ˈ\uline{mãːmãnãnũː}nĩx\textsuperscript{ĩ}nĩstoʔʃ]}\\
        /waː-wa-ɾa-ɾũː=ɾĩx=ɾĩt=t=oʔʃ/\\
        neg-unsp-2a-\textnormal{steal}=neg=2pl=pot=ind.m\\
        \glt `thou shalt not commit adultery' (lit. `you should not steal anyone')
    \end{xlist}
\end{exe}

Nasality is unable to spread leftward across an enclitic boundary, as an enclitic boundary functions as a word boundary, per \citet[269]{kasak2019}.\footnote{See \citet{kasak2019} for a theoretical analysis for why word boundaries, preverbs, and enclitics are all blocking mechanisms that arise from the same root cause based in the morphological system of Mandan.} Otherwise, nasal harmony is blocked featurally by mid vowels, as we see in (\ref{CHSkNasal1c}), and obstruents, as we see in (\ref{CHSkNasal2a}).

\section{Allomorphy}\label{CHSk3}

In addition to the morphophonological processes described above in \sectref{CHSk2}, The grammatical system of Mandan can result in an impressive array of morphological alternations. The syllable shape of a stem can affect which formative is used, including some formatives that have different phonological shapes. The purpose of this section in this chapter is to provide a summary of some of the morphological alternations that are not conditioned by phonotactics, as we have seen in \sectref{CHSk2} above, but are conditioned by the morphology itself. While the morphology here is explained in greater detail throughout Chapter \ref{chapter3} and Chapter \ref{chapter4}, this section serves as a preview of the various manifestations of certain common morphology to prime the reader to be able to read the longer glosses that appear throughout this book.

\largerpage
\subsection{Subject and object prefixes}\label{CHSk3.1}
Person-marking prefixes have the widest variety in terms of phonological shape. We can see that each person marker has several allomorphs, with the singular person markers having the greatest degree of alternate forms. These prefixes are divided into two classes: active and stative. Active prefixes correspond to the subject that is semantically the agent, while stative generally corresponds to any non-agent arguments.\footnote{There is a general pattern where the active set of prefixes is used to mark agents and the stative set of prefixes is used to mark non-agents, but there are verbs that select for active marking in their subjects that do not have a semantic agent, e.g., \textit{s\'{ı̨}h} `be strong' is an active verb, despite the fact its subject is non-agentive and should therefore be a stative verb like \textit{hą́ska} `be tall, long.' See Chapter \ref{chapter3} for a more elaborate explanation about the active--stative alignment in Mandan.} The formative that is the default form appears first in all the following examples.

\begin{exe}
    \item\label{CHSkEx1A} First person singular active marker

    \begin{xlist}
        \item\label{CHSkEx1Aa} {/wa-/}: default
        \item\label{CHSkEx1Ab} {/we-/}: before stems beginning with a sonorant plus /e/ or /eː/
        \item\label{CHSkEx1Ac} {/wʔ-/}: before vowel-initial stems
        \item\label{CHSkEx1Ad} {/w-/}: before a second person marker
    \end{xlist}

    \item\label{CHSkEx1S} First person singular stative marker

    \begin{xlist}
        \item\label{CHSkEx1Sa} {/wã-/}: default
        \item\label{CHSkEx1Sb} {/wĩ-/}: less common, mostly used before reflexive markers\footnote{See \sectref{wiPprefix} for more explanation as to when this allomorph is used in the corpus.}
        \item\label{CHSkEx1Sc} {/wʔ\~-/}: before vowel-initial stems
        \item\label{CHSkEx1Sd} {/w-/}: before a second person marker
    \end{xlist}

    \item\label{CHSkEx2A} Second person active marker

    \begin{xlist}
        \item\label{CHSkEx2Aa} /ɾa-/: default
        \item\label{CHSkEx2Ab} /ɾe-/: before stems beginning with a sonorant plus /e/ or /eː/
        \item\label{CHSkEx2Ac} /ɾʔ-/: before vowel-initial stems
        \item\label{CHSkEx2Sd} /ɾã-/: after a first person singular marker

    \end{xlist}

    \item\label{CHSkEx2S} Second person stative marker
        \begin{xlist}
        \item\label{CHSkEx2Sa} /ɾĩ-/: default
        \item\label{CHSkEx2Sb} /ɾʔ\~-/: before a vowel-initial stem
        \item\label{CHSkEx2Sc} /ɾũ-/: after a first person plural marker
        \end{xlist}
    \item\label{CHSk1APl} First person plural active marker
    \begin{xlist}
        \item\label{CHSk1APla} /ɾũ-/: default
        \item\label{CHSk1APlb} /ɾV-/: before short vowel-initial stems
        \item\label{CHSk1APlc} /ɾ-/: before long vowel-initial stems
    \end{xlist}
    \item\label{CHSk1SPl} First person plural stative marker
    \begin{xlist}
        \item\label{CHSk1SPla} /ɾo-/: default
        \item\label{CHSk1SPlb} /ɾV-/: before short vowel-initial stems
        \item\label{CHSk1SPlc} /ɾ-/: before long vowel-initial stems    
        \item\label{CHSk1SPld} /ɾũ-/: before reflexive markers
    \end{xlist}
\end{exe}

The allomorphs listed above are not derived from synchronous phonological rules. There are diachronic explanations for these alternations, but those explanations are discussed more fully throughout \sectref{Ch2InflectionalPrefixes}.

\subsection{Enclitics}\label{CHSk3.2}

Mandan features a rich system of enclitics that carry aspect, mood, number, and other features. In \sectref{SecPhrasalMorphology}, I examine each of the enclitics and their usages within the corpus, but the information here highlights the allomorphs whose alternations are not conditioned purely by phonological rules.

\begin{exe}
    \item\label{CHSkExNEG} Negative enclitic
        \begin{xlist}
            \item\label{CHSkExNEGa} /=ɾĩx/: default
            \item\label{CHSkExNEGb} /=xi/: after stems ending in a short vowel
        \end{xlist}
    \item\label{CHSkExPOT} Potential mood enclitic
        \begin{xlist}
            \item\label{CHSkExPOTa} /=kt/: default
            \item\label{CHSkExPOTb} /=t/: after stems that end in consonants other than /ʔ/
            \item\label{CHSkExPOTc} /=kte/: before the different-subject switch-reference marker /=ak/
            \item\label{CHSkExPOTd} /=kti/: before the conditional enclitic /=ki/
        \end{xlist}
\end{exe}

The enclitics above represent the number of formatives that have alternative realizations based on their phonological or morpho-syntactic contexts. The variation seen in the phonetic shaped of other Mandan enclitics is due to the epenthesis and syncope processes described above in \sectref{CHSk2.2} and in greater detail in \sectref{epentheticprocesses}.

%\subsection{Ablaut vowel}\label{CHSk3.3}

\section{Templatic morphology}\label{CHSk4}
\largerpage
One of the hallmarks of the morphological system of Mandan -- and of other Siouan languages -- is that words are built according to a template. \citet[112]{manovaaronoff2010} define templatic morphology as a system whereby affixation occurs in an unmotivated by prescribed manner, i.e., an affix occurs within a specific slot, regardless of any other phonological, syntactic, or semantic motivation.\footnote{I argue in \citet[339]{kasak2019} that templatic morphology does have motivation for the ordering of affixes, only that the motivation comes from the morphology of a language itself. I do not press this point in the present work, as my aim is to provide a descriptive grammar of the Mandan language rather than to use it to back up claims about linguistic theory.} Each affix must appear in a specific order with respect to other affixes. We can see a template for verbs in Mandan below in \tabref{prefixfieldmandanSketch}.

\begin{table}
\caption{Verbal prefix field in Mandan}\label{prefixfieldmandanSketch}

\fittable{\scshape \begin{tabular}{llllllllllll}
    \lsptoprule
    11  & 10  & 9      & 8    & 7   & 6          & 5      	& 4		& 3			& 2     & 1     & 0    \\
    \midrule
    rel & neg & unsp   & 1pl & pv.irr & pv.loc   & 1sg 		& 2sg		& suus		& iter  & ins & stem \\
    ~   & ~   & ~ & ~    & ~   & pv.ins 	 & ~      	& 2pl		& mid			& incp   & ~     & ~    \\
    ~   & ~   & ~      & ~    & ~   & pv.tr 	 & ~      	& ~		& recp		& ~  & ~     & ~    \\
    \lspbottomrule
    \end{tabular}}
\end{table}

Under this template, a first person singular argument is marked on a verb before any second person argument. The ordering of these affixes with respect to one another does not depend on any syntactic or semantic relationship these arguments have within the clause; their ordering is the same because that is what the template in Mandan proscribes. We can see the template in action in (\ref{ExSkTemplate}) below, where the first person singular prefix in slot 5 precedes the second person prefix in slot 4 despite the fact that there is a difference in the subject versus the direct object.

\begin{exe}
    \item\label{ExSkTemplate} Example of first person singular before second person affixation

    \begin{xlist}
        \item\label{ExSkTemplateA} \glll minihé'sh\\
        w-rį-hE=o'sh\\
        1a-2s-\textnormal{see}=ind.m\\
        \glt `I see you.'
        \item\label{ExSkTemplateB} \glll manahé'sh\\
        w-rą-hE=o'sh\\
        1s-2a-\textnormal{see}=ind.m\\
        \glt `You see me.'    \end{xlist}
\end{exe}

The order of prefixes in (\ref{ExSkTemplateA}) matches the subject--object--verb order of sentences in Mandan, but the order of prefixes in (\ref{ExSkTemplateB}) is reversed, i.e., the marker that encodes who the object is precedes the marker that encodes who the subject is. It is ungrammatical to rearrange the order of prefixes in (\ref{ExSkTemplate}) above, which is true of any other prefix in Mandan. There is no enclitic template, however, as the ordering of enclitics reflects the underlying semantic relationship. The farther away from the base an enclitic is, the wider its scope over the entire proposition.\footnote{See \citet[319{ff}]{kasak2019} for more on the theoretical underpinnings of this analysis.}

%Nouns in Mandan have a much smaller template, given that there is less morphology that is dedicated to nouns. There are two prefixes 

\section{Phrasal structure}\label{CHSk5}

Like all languages, words in Mandan must appear within the context of a phrase. This section is dedicated to providing an overview of the kinds of phrase structures found in Mandan. The overall structure of phrases in Mandan is relatively unsurprising from a typological point of view in that Mandan shares many structural similarities with other languages with a default subject--object--verb sentence order, e.g., Mandan features postpositions after noun phrases instead of prepositions \citep[56]{croft2003}.

\subsection{Noun phrases}\label{CHSk5.1}

%\subsubsection{Noun phrases}\label{CHSk5.1.1}

The noun  invariably appears as the initial element in a noun phrase in Mandan. The noun, as head of the noun phrase, is the most salient element within its domain, and all adjunct material appears after it. Stative verbs used in an adjectival manner will always follow the nouns they describe. Likewise, determiners and the topic marker will be found after any adjunct material. We can see an example of a fully articulated noun phrase, complete with an adjunct stative verb, in the example below. 

In (\ref{SkExNP}), we see the noun \textit{minísweerut} `dog' at the leftmost edge of the noun phrase. The adjunct \textit{xí'h} `to be old' follow the noun it modifies. All determiners then encliticize onto the right edge of the rightmost element within the noun phrase, which is the stative verb \textit{xí'h} in this case. The definite article =\textit{s} appears as the determiner closest to the head of the noun phrase. This article encodes for a specific dog in the context of the narrative. This element is followed by the distal  demonstrative =\textit{ee},  which specifies the physical proximity of the noun phrase in relation to the present interlocutors. Finally, the topic marker =\textit{na} indicates that the speaker wishes to emphasize the salience of this who noun phrase in the context of the discourse.

\begin{exe}
    \item\label{SkExNP} Example noun phrase

    \glll minísweerut xí'hseena\\
    wrįs\#wee\#rut xi'h=s=ee=rą\\
    \textnormal{horse}\#\textnormal{feces}\#\textnormal{eat} \textnormal{be.old}=def=dem.dist=top\\
    \glt `the old dog' \citep[189]{hollow1973a}
\end{exe}

The one exception to the rule that adjunct materials must follow the head noun of a phrase is in constructions expressing possession. In such instances, the possessor precedes the possessee. In (\ref{SkExNP2}) below, we can see that the possessor \textit{Kóoxą'te Míihs} `Corn Woman' comes before the possessee \textit{tasúkseena} `her child.'

\begin{exe}
    \item\label{SkExNP2} Example of a noun phrase with a possessor

    \glll Kóoxą'te Míihs tasúkseena\\
    kooxą'tE wįįh=s ta-suk=s=ee=rą\\
    \textnormal{corn} \textnormal{woman}=def al-\textnormal{child}=def=dem.dist=top\\
    \glt `The Corn Woman's child' \citep[112]{hollow1973a}
    
\end{exe}

It is ungrammatical to switch the order of the possessor and possessee in Mandan. The ordering between the two noun phrases is the only indicator as to the semantic role one has to the other. Some languages, like English, have multiple constructions to express possessor--possessee relationships, e.g., \textit{the writer's pen} versus \textit{the pen of the writer}. Mandan does not have any such alternative structures; the possessor must precede the possessee. Further discussion of noun phrases can be found in \sectref{Ch5NounPhrases}, where other aspects of noun phrases and noun phrase structure are described in greater detail.

\subsection{Postpositional phrases}\label{CHSk5.2}

Mandan conforms to the typological generalization that languages with a default subject--object--verb sentence structure will have postpositions instead of preposition. Postpositions will always appear immediately after the noun phrase over which they have semantic scope. We can see an example of this word order below in (\ref{ChSkPP}), where the postposition \textit{ų́ųpa} `with' follows the noun phase \textit{Kóoxą'te Míihs} `Corn Woman.' This position at the rightmost edge of a noun phrase is the only place where a postposition can occur in Mandan.

\begin{exe}
    \item\label{ChSkPP} Example of a postpositional phrase

    \glll Kóoxą'te Míihs ų́ųpa\\
    kooxą'tE wįįh=s ųųpa\\
    \textnormal{corn} \textnormal{woman}=def \textnormal{with}\\
    \glt `with Corn Woman' \citep[112]{hollow1973a}
\end{exe}

Postpositions behave similarly to verbs in that they take person marking when expressing the relationship with a first or second person entity. There are no free pronouns in Mandan, so pronominal marking will appear on the postposition itself. We can see an example of a postposition with person marking in (\ref{ExSkPP2}) below.

\begin{exe}
    \item\label{ExSkPP2} Example of a postpositional phrase with person marking

    \glll nú'pa\\
        r'-ųųpa\\
        2s-\textnormal{with}\\
        \glt `with you' \citep[224]{hollow1973b}
\end{exe}

Postpositional phrases are often used as adjuncts within a clause. A more detailed description of postpositional phrases appears in \sectref{Ch5PostpositionalPhrasesSec}.

\subsection{Verb phrases}\label{CHSk5.3}

The verb phrase in Mandan is characterized by the verb being in the rightmost position within its domain. Verbs in Mandan are the primary locus of agreement marking, resulting in constructions where the information for a whole sentence might be encoded entirely on the verb alone. We can see this feature of Mandan in (\ref{ExSkVP1}) below, where there are no overt nominal elements present in the sentence, and the verb bears many inflectional formatives that provide enough context so as to express a complete thought.

\begin{exe}
    \item\label{ExSkVP1} Example of a sentence consisting of only a verb

    \glll Máamananuunixinisto'sh.\\
        waa-wa-ra-rųų=rįx=rįt=t=o'sh\\
        neg-unsp-2a-\textnormal{abduct}=neg=2pl=pot=ind.m\\
        \glt `thou shalt not commit adultery' (lit. `you all should not run off with anyone') \citep[22]{hollow1970}
\end{exe}

Auxiliary verbs in Mandan come after the lexical verbs over which they have semantic scope. One common construction that requires an auxiliary verb with the lexical verb is benefactive constructions. The verb \textit{kų́'} `give' is used after a lexical verb indicates that the subject is doing something for someone else. We can see an example of a verb phrase containing an auxiliary verb in (\ref{ExSkVP2}) below.



\begin{exe}
    \item\label{ExSkVP2} Example of a verb phrase with an auxiliary verb

    \glll Wahará minikú'kto'sh.\\
    wa-hrE w-rį-kų'=kt=o'sh\\
    1a-caus 1a-2s-\textnormal{give}=pot=ind.m\\
    \glt `I will do it for you.' \citep[138]{hollow1973a}
\end{exe}

Mandan requires that auxiliary verbs and the lexical verbs they modify bear subject marking, as we see above in (\ref{ExSkVP2}). This behavior contrasts with languages like English, where auxiliary verbs bear all person marking and the lexical verb becomes non-finite, e.g., \textit{he is going home} has person and tense marking only on the auxiliary \textit{is}, while the lexical verb appears with the progressive participle suffix -\textit{ing} and cannot take person or tense marking.

Verbal morphology plays a major role in Mandan grammar. For that reason, an entire chapter of this book is devoted to looking at the morphology of verbs, i.e., Chapter \ref{chapter3}. A more elaborate discussion of the structure of verb phrases can be found in \sectref{Ch5Verbs}.

\section{Sentence structure}\label{CHSk6}

Mandan employs a default subject--object--verb clause order. \citet{dryer2013} finds that a plurality of languages follow this same pattern, so Mandan is typologically unmarked in this respect. We can see an example of a sentence with this SOV word order below. In (\ref{ExSkSS1}), we see the subject \textit{numá'k} `man' in the initial position, followed by the direct object \textit{minísą't} `that horse.' The final element is the verb \textit{waká'ro'sh} `he possesses it.'

\begin{exe}
    \item\label{ExSkSS1} Example of subject--object--verb word order in Mandan

    \glll Numá'k minísą't waká'ro'sh.\\
    ruwą'k wrįs=ą't wa-ka'=o'sh\\
    \textnormal{man} \textnormal{horse}=dem.anap unsp-\textnormal{possess}=ind.m\\
    \glt `The man has a horse.' \citep[32]{hollow1976}
\end{exe}

This sentence order is unchanged when the illocutionary force of a proposition is not indicative. Some languages, such as English, can move the verb to an initial position to indicate that the sentence is interrogative, e.g., \textit{she is tall} versus \textit{is she tall?} Mandan has no such change in word order to indicate any kind of illocutionary information, because illocutionary information is encoded in specific verbal morphology. The information in (\ref{ExSkSS1}) above appears below, rendered as a question. This subject--object--verb word order remains consistent, as we can see in the example in (\ref{SkSOV}) below.

\begin{exe}
\item\label{SkSOV} SOV word order in questions

    \glll Numá'k minísą't waká'ro'sha?\\
    ruwą'k wrįs=ą't wa-ka'=o'sha\\
    \textnormal{man} \textnormal{horse}=dem.anap unsp-\textnormal{possess}=int.m\\
    \glt `Does the man have a horse?'
    
\end{exe}

Ditransitive constructions, where there are three obligatory arguments, the subject is still sentence-initial. The primary distinction in word order from those seen in transitive or intransitive constructions is that the direct object does not follow the subject; the other argument always precedes the direct object. The direct object, therefore, must appear directly before the verb. We can see an example of a sentence involving a ditransitive verb \textit{íkų'te} `to throw', which must naturally take an argument as its subject, but also two additional arguments as the direct object (i.e., that which is thrown) and the destination (i.e., where the direct object is thrown).

\largerpage
In (\ref{SkSOV2}) below, the subject, \textit{Kinúma'kshi} `Royal Chief', is the person who is doing the throwing and therefore is the first element in the sentence. The subject is displayed in small caps. The direct object, \textit{istámi'} `his eyes', is what Royal Chief is throwing and is therefore the element closest to the verb. This direct object is underlined. The destination, \textit{skiskíka kaxtékseet} `towards the willow bunch', is where Royal Chief is throwing the eyes to, and this element appears immediately before the direct object. The destination is displayed in bold. The verb continues to appear in its default position word-finally.

\begin{exe}
    \item\label{SkSOV2} Ordering arguments in ditransitive constructions

    \glll \textsc{Kinúma'kshi} \textbf{skiskíka} \textbf{kaxtékseet} \uline{istámi'} ~ ~ ~ ~ ~ ~ íkų'teoomako'sh.\\
    ki-ruwą'k\#shi skiskika kaxtek=s=ee=t istawį' ~ ~ ~ ~ ~ ~ i-kų'tE=oowąk=o'sh\\
    mid-\textnormal{man}\#\textnormal{be.good} \textnormal{willow} \textnormal{bunch}=def=dem.dist=loc \textnormal{eye} ~ ~ ~ ~ ~ ~ pv.dir-\textnormal{throw}=narr=ind.m\\
    \glt `\textsc{Royal Chief} threw \uline{his eyes} \textbf{to the willow bunch}.' \citep[34]{hollow1973a}
\end{exe}

Adjunct material, such as adverbial elements have a more liberal distribution throughout the sentence structure. Given that this chapter aims to provide a basic sketch of Mandan, an explanation on the nuances of where adverbials go in a sentence is outlined elsewhere in this book. Sentence structure for simplex clauses is discussed in greater length in \sectref{Ch5Clauses}.

\section{Topicalization}\label{CHSk7}

Mandan adheres to a basic SOV word order for sentences. However, Mandan rigorously flouts this default order in order to place some narrative emphasis on some element that the speaker wishes to highlight for the listener. As such, it is not uncommon for arguments in Mandan to appear outside of the expected SOV word order. In such cases, the change in the default word order is motivated by pragmatic choices on the speaker's part to topicalize a salient element. Typical topicalization in Mandan involves moving some element to the leftmost edge of the sentence. 

In (\ref{SkTopic1}) below, we see an instance where the indirect object \textit{numá'kaaki} `people' appears at the leftmost edge of the sentence, despite the fact that its prototypical position within the sentence should be immediately before the direct object \textit{wáa'ąąwe} `everything'. The speaker is highlighting who the addressee is talking to and bringing their attention to the fact that this element is particularly salient.

\begin{exe}

    \item\label{SkTopic1} Topicalized indirect object
    
    \glll \textbf{Numá'kaaki} tashkák wáa'ąąwe rakų́' kina'ka'na?\\
    ruwą'k-aaki tashka=ak waa-ąąwe ra-kų' kirą'=ka=o'rą\\
    \textnormal{person}-coll \textnormal{how}=ds nom-\textnormal{all} 2a-\textnormal{give} \textnormal{tell}=hab=int.f\\
    \glt `\textbf{To people} why do you always tell everything?' \citep[213]{hollow1973a}
\end{exe}

Another kind of topicalization in Mandan involves postposing an element at the very end of the sentence, i.e., right dislocation. Elements that are right dislocated serve as reminders or clarifying topics to something alluded to already in the discourse. We can see an example of right dislocation below in (\ref{SkExRightDislocation}), where the right dislocated element \textit{máa'ąk} \textit{kú'sht} \textit{ó'harani} `from underneath the earth', is the location from where buffalo had emerged. This postpositional phrase serves to clarify or emphasize the postpositional phrase \textit{óo} \textit{ó'harani} `from there' that was mentioned earlier in the sentence.

\newpage


\begin{exe}
    \item\label{SkExRightDislocation} Example of right dislocation

            \glll \uline{óo} \uline{ó'harani} pt\'{ı̨}įtkushkeres ą́ąwe ~ ~ ~ ~ ~ ~ ~ ~ ~ ~ ~ ~ ~ ~ ~ ~ ~ súhkereroomako'sh, \textbf{máa'ąk} \textbf{kú'sht} \textbf{ó'harani}.\\
        oo o'\#hrE=rį ptįį\#tkush=krE=s ąąwe ~ ~ ~ ~ ~ ~ ~ ~ ~ ~ ~ ~ ~ ~ ~ ~ ~  suk=krE=oowąk=o'sh waa'ąk ku'sh=t o'\#hrE=rį\\
        dem.mid \textnormal{be}\#caus=ss \textnormal{buffalo}\#\textnormal{be.true}=3pl=def \textnormal{all} ~ ~ ~ ~ ~ ~ ~ ~ ~ ~ ~ ~ ~ ~ ~ ~ ~ \textnormal{exit}=3pl=narr=ind.m \textnormal{earth} \textnormal{inside}=loc \textnormal{be}\#caus=ss\\
        \glt `All the buffalo came \uline{from there}, \textbf{from underneath the earth}.' \citep[114]{hollow1973b}
\end{exe}

Additional discussion of topicalization and topic marking in Mandan and what elements can be topicalized can be found in \sectref{SubSecTopics}.\footnote{See also \citet{kasak2022} for additional discussion of topic marking in Mandan, as well as how prosody is used to focalize elements within a Mandan sentence.}

\section{Multiclausal structure}\label{CHSk8}

Like all languages, Mandan has different approaches to expressing relationships between clauses. The corpus contains a preponderance of data that feature sentences with multiple clauses combined to express complex ideas. One noteworthy feature of Mandan grammar is the stark lack of phrasal or clausal coordinators; the most common relationship between clauses is that of adjunction. This section outlines the two most common types of clausal adjuncts that are rooted in a matrix clause. Only matrix clauses can bear illocutionary force and therefore any verb that bears such markers must inherently be the matrix clause. The presence of allocutivity marking is therefore a diagnostic for independent clausehood in Mandan. Allocutive agreement markers are discussed more fully in \sectref{Ch3SubSubSecAllocutive} and \sectref{Ch5IllocutionaryForce}.

\subsection{Switch-reference}\label{SkSR}

Switch-reference is a morpho-syntactic system whereby some formative indicates that the subject of one clause is either the same or different from that of an adjacent clause \citet[ix]{haimanmunro1983}. In Mandan, the enclitic complementizers =\textit{ni} and =\textit{ak} appear in clause-final position to indicate that there is a switch in what subject is being referenced.

The =\textit{ni} enclitic indicates that the subject of its clause is the same as the one of the following clause. The dependent clause \textit{są́ąka róonapi{ni}} `we found a few' features the same-subject switch-reference marker, which signals to the listener that the following clause will have the same subject. The same subject is confirmed in \textit{nukeréeho'sh} `we went back [home].' The same-subject switch-reference marker is shown in bold in (\ref{SkSS1}) below.

\begin{exe}
    \item\label{SkSS1} Example of same-subject switch-reference construction

    \glll Są́ąka róonapi\textbf{ni} nukeréeho'sh.\\
    sąąka rV-o-rąp=\textbf{rį} rų-k-rEEh=o'sh\\
    \textnormal{be.few} 1a.pl-pv.loc-\textnormal{find}=\textbf{ss} 1a.pl-vert-\textnormal{go.there}=ind.m\\
    \glt `We found a few and we went home' \citep[470]{hollow1970}
\end{exe}

The different-subject switch-reference marker =\textit{ak} similarly indicates a relationship with the following clause, only that the subject of the clause bearing =\textit{ak} does not have the same subject as the following clause. The dependent clause \textit{kihkaráarak} `he looked around' bears =\textit{ak}, signaling that the subject of the independent clause \textit{mí'xteseena xką́hereka'eeheero'sh} `the big rock was chasing him, so they say' must have a different subject. The different-subject switch-reference marker is shown in bold in (\ref{SkDS1}) below.

\begin{exe}
    \item\label{SkDS1} Example of different-subject switch-reference construction

    \glll Kihkaráar\textbf{ak} mí'xteseena ~ ~ ~ ~ ~ ~ ~ ~ ~ ~ ~ ~ ~ ~ ~ xką́hereka'eeheero'sh.\\
    kihkraa=\textbf{ak} wį'\#xtE=s=ee=rą ~ ~ ~ ~ ~ ~ ~ ~ ~ ~ ~ ~ ~ ~ ~ xkąh\#hrE=ka'eehee=o'sh\\
    \textnormal{look.for}=\textbf{ds} \textnormal{stone}\#\textnormal{be.big}=def=dem.dist=top ~ ~ ~ ~ ~ ~ ~ ~ ~ ~ ~ ~ ~ ~ ~ \textnormal{move.away}\#caus=quot=ind.m\\
    \glt `He looked around and that big rock was chasing him, so they say.' \citep[18]{hollow1973a}
\end{exe}

The fact that overt nominal elements in Mandan can be omited from the discourse can render such constructions ambiguous without the presence of switch-reference marking. For example, in the sentence (\ref{SkDS1}) above, the independent clause alone could be interpreted as `the big rock was chasing him' or `he was chasing the big rock', given the fact that subjects are frequently dropped when speaking. The =\textit{ak} in the preceding clause clarifies the context of whether `that big rock' is the subject or the direct object, given that both `he' and `that big rock' are third person singular arguments and there would be no morphology on the verb to disambiguate subjecthood.


See \sectref{Ch5switchreference} for a more thorough discussion of switch-reference marking in Mandan and the syntactic structure of clauses in a switch-reference relationship. The transcribed narrative in \sectref{Ch6EyeJuggler} further illustrates ways in which switch-reference is an integral part of Mandan multiclausal syntax and its overall narrative structure.

\subsection{Serial verb constructions}\label{SkSVC}

Switch-reference represents the most typical morpho-syntactic process whereby two or more clauses are combined in Mandan. The other common method by which two clauses will establish a relationship between themselves is through serial verb constructions. These serial verb constructions are used to express actions or states that are treated as part of the same event. 

Serial verb constructions, being a single complex predicate, must necessarily share the same subject among each of the verbs involved. The structure of these serial verb constructions involve one or more verbs that are inflected with subject and object prefixes where relevant, plus an encliticized stem vowel complementizer /=E/. Furthermore, each serialized verb that bears the stem vowel complementizer also has a null enclitic /=$\varnothing$/, which is the continuous aspect marker. The continuous aspect marker triggers ablaut on the stem vowel. The final verb in a serial verb relationship does not bear this complementizer. 

We can see an example of a serial verb construction in the data below in (\ref{SkSVC1}). The serial verb construction begins with the serialized verb \textit{áaraaha} `bringing him with him', which is an action that happens within the same eventuality as the proposition \textit{shą́ątaa íkaaxe'sh} `he landed on the other side.' The morphological manifestation of serialization appears in bold below.

\begin{exe}
    \item\label{SkSVC1} Example of a serial verb construction

    \glll Áaraah\textbf{a} shą́ątaa íkaaxe'sh.\\
    aa-rEEh=\textbf{E}=\textbf{$\varnothing$} shąą=taa i-kaaxE=o'sh\\
    pv.tr-\textnormal{go.there}=\textbf{sv}=\textbf{cont} \textnormal{be.across}=loc pv.dir-\textnormal{decend}=ind.m\\
    \glt `He landed on the other side, taking him with him.' \citep[5]{hollow1973b}

\end{exe}

The serial verb constructions in Mandan permit intervening material between the serialized verbs. This pattern of forming a serial verb construction differs from other languages with serial verb constructions where a serialized verb must have another verb immediately following it. As such, serial verb constructions in Mandan can be analyzed as a sequence of verb phrases. A longer discussion of serial verb constructions in Mandan appears in \sectref{Ch5SerialVerb}. There are also instances of serial verb constructions throughout the narrative ``Eye Juggler'' found in \sectref{Ch6EyeJuggler},
