\addchap{\lsPrefaceTitle}

First and foremost, I want to expressly state that I profess no personal ownership over the Mandan data or any aspect of the Mandan language presented here and welcome all Mandan community members to make free and unimpeded use of any aspect of the materials present here as they see fit. I expect no financial compensation from any Mandan people to use any part of the present book for the purpose of creating language learning materials, composing songs or prayers, or engaging in any cultural repatriation practices. The data and the language in general does not belong to me; it belongs to the Mandan people. \textit{Núu'etaa numá'kaaki ą́ąwe wakápusa wakų́'ro'sh. Káni ínuma'kaakis tóo'irooro'sh.}
 
This book is the culmination of work that began almost a decade and a half ago in a computer lab on the north side of Chicago. Back then, I was an M.A. student at Northeastern Illinois University in the spring of 2010. I was teaching Russian and Spanish at the high school level during the day in the Hermosa neighborhood of Chicago, and I was working on my Master's degree in Linguistics in the evenings. I initially chose a degree in Linguistics so I could branch out into teaching ESL classes to further my career prospects as a high school teacher. I had also believed that a degree in Linguistics would improve my ability to teach my Russian and Spanish students to learn other languages in a more efficient manner. However, I quickly realized that Theoretical Linguistics was too captivating a subject to ignore, though my eventual M.A. thesis was very much in the realm of Applied Linguistics and Language Policy.  As I was working towards gathering data for my thesis, Dr. John Boyle -- my then-professor and now-friend -- was teaching Translations and Linguistic Analysis: Mandan as an upper-level course. My initial thought was that it would be a great elective credit, but it wound up shaping the trajectory of the rest of my life.

My first interests in Linguistics revolved around Indo-European languages, especially Old Church Slavonic and Classical Latin, given that I had been teaching Russian and Spanish at the high school level in Chicago. I quickly realized that Indo-European Linguistics was a field that had an enormous amount of previous literature to consume before I could adequately wade into that field, but Siouan Linguistics was still full of possibilities and new things to say. That seminar on Mandan at NEIU really opened my eyes to how much about language I did not know and reinforced my desire to continue onto a more advanced degree. I produced a lengthy paper for that Mandan seminar course on vertitivity and motion verbs in Mandan, and that paper became a writing sample for my Ph.D. program applications.

My subsequent move to Yale for my Ph.D. led me to increasingly learn again and again that the Mandan language was itself worthy of extensive study on its own, and that I could happily leave analyses of Old Church Slavonic clitic orderings and the use of subjunctive in Latin to the professionals; Mandan had enough unanswered questions that any course I took could be used to look at some issue of Mandan grammar. These issues led to my dissertation, from which the present book stems. 

For the past decade, I have thoroughly enjoyed going to conferences to present on topics related to the grammar of Mandan. The annual Siouan and Caddoan Languages Conference has been somewhere I have made good friends and colleagues, and where I have acquired a fuller understanding of how Mandan does and does not work based on looking at its distant relatives across the Siouan language family. The classes that I teach here at the University of Oklahoma are also filled with examples from Mandan. I have used the allocutive agreement markers in my Sociolinguistics course to talk about how different languages treat speaker and listener gender in discourse; I have used instances of nasal harmony in my Phonology course to illustrate how long-distance phonological processes work; and I have used recordings of L1 Mandan speakers from the 1960s speaking English to show how one's L1 affects their L2 in my Second Language Acquisition Theory course. The Mandan language has helped my students become better linguists in the same way that -- I think -- it made \textit{me} a better linguist.

The Mandan language has demonstrated that its complexities are worthy a numerous tomes, and herein I offer up my attempt to make good on this statement. I have expanded my interest into other Siouan languages through my interest in the Mandan language, and I acknowledge that my academic career would never be where it is at without the Mandan language being the locus of my theoretical linguistic studies. I again want to make it clear that any aspect of this book is free and available to be used in any way by members of the Mandan community and I welcome any requests for the materials I used to create the corpus built herein.