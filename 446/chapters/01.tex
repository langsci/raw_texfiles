\chapter{Introduction}\label{chapter1}
\largerpage
Mandan [ISO: mhq] is a \ili{Siouan} language spoken in northwestern North Dakota on and near the Fort Berthold Indian Reservation. With a no remaining L1 speakers and with between one and two dozen remaining heritage or L2 speakers, Mandan is critically endangered. Mandan also possesses some typologically rare phenomena, such as {allocutive agreement} similar to \ili{Basque}, where a sentence-final verb must bear some marking that denotes whether the listener is male or female. Indeed, verbs in Mandan can display a large amount of complex morphology. In particular, Mandan employs a high degree of affixation on its verbs, and has often been described as having {templatic morphology}, as have \ili{Siouan languages} in general \citep{rankinetal2003}. 

Mandan has not been as well-studied as its cousins \ili{Lakota} (e.g., \citealt{roodtaylor1996} and \citealt{ullrichblackbear2016}) or \ili{Crow} (e.g., \citealt{wallace1993} and \citealt{graczyk2007}), though Mandan has been described in very compact grammars that do not exceed 40 pages (e.g., \citealt{kennard1936} and \citealt{mixco1997a}). \citet{kasak2019} presents a more in-depth analysis of certain aspects of Mandan grammar, especially its sounds system and its verbal morphology, but the emphasis of that work is to present these aspects of its grammar through a particular theoretical lens. This book is an attempt at presenting a formal description of Mandan grammar without certain theoretical assumptions, i.e., the purpose of this book is to provide a grammar of the Mandan language that is as comprehensive as possible and one whose goal it is to portray the data hereafter in as theoretically neutral a light as possible. The need for a grammar that is able to be accessed by scholars and community members stems from the fact that most published work on Mandan has been written almost exclusively in underlying representation per certain authors' interpretation of the phonology of Mandan (i.e., \citealt{hollow1970} and \citealt{mixco1997a}), and those works present a challenge to members of the Mandan community who  might otherwise wish to utilize them for personal reasons. This book contributes to empirical literature by organizing data from existing Mandan corpora in order to facilitate its use by linguists, but it is also written so that members of the Mandan community will be able to find an explanation for how words and sentences are built in their language.

In this chapter, I present a historical sketch on the Mandan people in \S\ref{background} to contextualize how their language became endangered and the conditions imposed upon them to further complicate their ability to perpetuate their cultural and linguistic heritage on the Northern Plains that dates back at least a millennium. Next, I talk about Mandan's place within the Siouan language family in \S\ref{genetics}. I elaborate on the work previous scholars have done on the language in \S\ref{previousresearch} and what I have done in my own fieldwork in \S\ref{fieldwork}, and I end the chapter by giving an overview of the organization of the rest of this book in \S\ref{organization}.

\section{Background on the Mandan people}\label{background}
\largerpage
In this section, I provide a background of the Mandan people and the circumstances behind the massive drop in the population of Mandan speakers, which has ultimately forced any future research on the language to be done through extant corpora and recordings instead of fieldwork with fluent speakers to seek judgments and record novel data. The lack of fluent L1 speakers of the language restricts the future viability of exploring the grammar of Mandan more fully and accentuates the need to document as much of the language as possible while heritage speakers are still able to contribute their own insight to our understanding of Mandan. It is necessary to understand the context behind the interaction between the Mandan people and outside groups and individuals, as a number of factors have contributed to the state of the language today and the shift to languages other than Mandan, such as English and \ili{Hidatsa}.

\subsection{Overview}

According to tradition, the Mandan have several creation stories that explain the origin of their people and how they came to live on the Upper Missouri. In \citeapos{hollow1973a} narratives, Mmes. Otter Sage and Annie Eagle both tell variations on how \textit{Kinúma'kshi} `Royal Chief'\footnote{The name of this figure often is rendered in English as `Old Man Coyote', and stories involving his deeds and travails are often called coyote stories, which traditionally should be told only during the winter, according to consultants.} and \textit{Numá'k Máxana} `Lone Man' create the world and populate it with beings that look like them, as well as other beings. In a different explanation on the origin of the peopling of the world comes from the telling of Mr. Mark Mato\footnote{Mr. Mato assisted both Bowers and Kennard with their translations of Mandan narratives and traditional accounts. He was an L1 speaker of Mandan and his father was Bear-on-the-Flat, who was a principal consultant of Densmore, Bowers, and Kennard and whose Mandan name was \textit{Ópshiitaa Mató}. Mr. Mato's surname, \textit{mató} `bear', appears as Mahto on his death certificate, though his registration from the Carlisle Indian School spells his surname as Mato, as does he in all recorded materials cataloged at the Carlisle Indian School Digital Resource Center: \href{http://carlisleindian.dickinson.edu/people/mato-mark}{http://carlisleindian.dickinson.edu/people/mato-mark}.} (1886--1964), who relates a version of this second creation story to Kennard where people emerged from beneath the earth near the mouth of the Mississippi River, climbing upwards on a grape vine until that vine broke, leaving a portion of humanity below the ground and those on the surface departed from the crack in the earth. Those people migrated north towards the Heart River and then camped at Devil's Lake in North Dakota.

While this book focuses linguistic attention on one of the two varieties of Mandan to make it into the twentieth century, \citet[24]{bowers1950} cites Crows Heart\footnote{A more accurate translation of his name is `Raven Heart', or \textit{Kéekanatka}. He is always referred to by the English translation of his Mandan name, though his death certificate lists him as Paul Crows Heart.} (1856--1951) and Scattercorn\footnote{Her name in Mandan is \textit{Wóopįte}, literally `something that has been scattered all over the place.' Her death certificate identifies her simply as Mrs. Holding Eagle, without any first name.} (1858--1940), who say that there were at one time five bands\footnote{The use of the term `band' here refers to a subdivision of the Mandan people that is based on linguistic and/or political differences between other Mandan groups. This distinction differs from that of a clan, which centers around the biological or social kinship one has, apart from the polity with which one associates. For example, it is possible for members of the same clan to come from different bands and members of the same band may have different clan affiliations. Additional information on Mandan social organization appears in \citet{bowers1950}.} of Mandan: the Nuitadi,\footnote{Also called Nueta, which is \textit{Núu'etaa} or \textit{Núu'etaare} in the orthography used within this book. Their name means `our people.'} the Nuptadi,\footnote{Also called Rupta or Nupta, which is \textit{Rúpta} or \textit{Rúptaare} in the present orthography. Though \citet[25]{bowers1950} gives no definition for this band's name, consultants tell me it means either `two voices', or `ones who came second', because they formerly lived apart from the \textit{Núu'etaa} until an attack from the Lakota drove them away from their village in 1792 \citep[116]{bowers1950}.} the Istope, the Mananare,\footnote{Their name means `those who quarrel', and is written \textit{Máananaare} in the orthography used here.} and the Awikaxa.\footnote{This group's name is also spelled Awigaxa. \citet[25]{bowers1950} gives no definition for this name, but it does resemble \textit{Ą́ąwe kaxé}, a name that one of my consultants gives for all Mandan, which he says means `something everyone has.'} \citeauthor{bowers1950} also states that not all his consultants agreed that the Mananare were a band unto themselves, and this was in actuality just the term used to describe any group who left a village due to some disagreement. \citeauthor{bowers1950}' consultants state that there were three dialectal differences among the Mandan: Nuitadi, Nuptadi, and Awikaxa. After the first smallpox epidemic in 1782, the Awikaxa were absorbed by the Nuitari, leaving only two varieties to survive the next smallpox outbreak a half-century later in 1857.

One complication in describing the Mandan is that the Mandan traditionally have had no autonym for their people as a whole. When Prince \citet{maximilian1839} lived among the Mandan, he wrote that they called themselves \textit{Númangkake},\footnote{In the modern Mandan orthography, this word would be \textit{numá'kaaki}.} meaning `people'. When the artist and adventurer George \citet[260]{catlin1844} lived with the Mandan between 1834 and 1835, just before the looming smallpox outbreak, \citeauthor{catlin1844} states that the Mandan called themselves \textit{See-pohs-ka-nu-mah-ka-kee} `the people of the pheasants.' However, `the people of the pheasants' is not an accurate translation, as \textit{sípųųshka numá'kaaki} means `prairie chicken people.' Prairie Chicken is one of the original thirteen clans of the Mandan, though only four clans survived into the twentieth century \citep[30]{bowers1950}.\footnote{The remaining Mandan clans are named the \textit{Waxíhkina'} `Tells Bad Stories' or `Bad News', the \textit{Tamísik}, whose clan name is never given a translation, and the \textit{Ípųųxka numá'kaaki} `Speckled Eagle People.' The Speckled Eagle People were later absorbed by the Prairie Chicken clan.} It is very likely that Catlin's consultant(s) proffered their clan affiliation rather than the name of their ethnicity.

Due to their location within the North American continent, the Mandan did not directly interact with European settlers until possibly the eighteenth century. Pierre Gaultier de Varennes, sieur de la Vérendrye, is considered to be the first European to make contact with the Mandan in 1738 with the help of his \ili{Assiniboine} guides. He recorded that the Assiniboine refer to the Mandan as \textit{Mantannes},\footnote{The etymology of this term is not clear, though it is worth noting that cognates for the word for `Mandan' is found in all other \ili{Dakotan languages}, with Lakota and \ili{Dakota} varying between \textit{miwátani} or \textit{mawátani}, and \textit{mayádąna} in modern Assiniboine \citep{parksdemallie2002}. If these words have some literal meaning (as most Siouan words are wont to), then one possible meaning might be gleaned from the Assiniboine form. The word \textit{mayá} `river bank' is shared between Lakota, Dakota, and Assiniboine, and the -\textit{da} looks to be a reflex of the Proto-Dakotan locative marker *-ta, with -\textit{na} being a distal marker. Thus, the meaning of the Assiniboine term would be `people who at the river bank over there.' This interpretation lines up with what the Hidatsa call the Mandan, \textit{Aróxbagua} `people at the confluence [of the Heart and Missouri]', and what the Crow call them, \textit{Assahkashí} `people at the river's edge.'} though his \ili{Cree} guides had earlier referred to them as \textit{Ouachipouennes} `Sioux who go underground' or by the \ili{French} term \textit{caserniers} `quartermasters' \citep[213]{mapp2013}.\footnote{The term \textit{Ouachipouennes} does not appear to be a Cree term, since the \textit{-pouennes} resembles the \ili{Ojibwa} \textit{bwaan} `Sioux' more than the Cree \textit{pwâta} `Sioux'. Furthermore, \textit{ouachi-} resembles the Ojibwa \textit{waazhi} `cave', versus the Cree \textit{wâti} `cave', making it more likely that de Vérendrye had conflated the Cree and Objibwa and called them both `Cree'. The modern word for Mandan in Cree is \textit{kâ-otasiskîwikamikowak} `those who have earth (clay) lodges' (\name{Arok}{Wolvengrey} p.c., \name{Kees}{van Kolmeschate} p.c.). The term `cave Sioux' does not appear in either modern Cree or Ojibwa, though it is possible that it is an epithet used in the past, owing to the fact that the Mandan differed from the neighboring Algonquian and Dakotan peoples by living in earth lodges rather than tipis or other dwellings.}

It is possible that the Mandan people made contact with Europeans as early as 1689, when the French aristocrat-adventurer Louis Armand de Lom d'Arce Lahontan, Baron de Lahontan, met a people he called the \textit{Essanape}, who were the enemies of the \textit{Eokoros} he had met some sixty leagues south. The \textit{Eokoros} are likely the \ili{Arikara}, which would make the \textit{Essanape} likely contenders for being the Mandan \citep[28]{fenn2015}. In particular, it is possible that \textit{Essanape} is not the name of whole body of Mandan, but the Istopa\footnote{The name literally means `tattoo' in Mandan, and is spelled \textit{Istópe} in this orthography.} band of the Mandan.

The Mandan themselves have competing accounts of how they first encountered Europeans. One elder told me that the Mandan word \textit{mashí} `white person' came from the word \textit{shí} `good', since the first European-Americans they saw were members of the military, whose uniforms were very impressive so the Mandan said that they ``looked good,'' which then became lexicalized to \textit{mashí}. Another elder informed me that \textit{mashí} is short for \textit{mashí'na} `generous', since early traders made a habit of bringing gifts when entering villages. This etymology seemingly conflicts with a similar word for whites in Lakota, \textit{wašíču}, whose folk etymology holds that it means `one who takes the best part [of the meat]' after the story of the first time the Lakota encountered a white person, who being brought back to their village and given food, grabbed the best part of the meat and ran away.\footnote{Older Mandan sources state that the word for `white person' is actually \textit{washí} \citep{kipp1852}. Sentence-initially, /w/ is often pronounced as [m] in Mandan, and the older term \textit{washí} may have eventually been reanalyzed as \textit{mashí}. The term \textit{washí} furthermore suggests that this word may be a borrowing from Lakota. Mandan certainly could have clipped the final syllable of \textit{wašíču} to get \textit{washí}, which eventually became \textit{mashí}. This hypothesis is complicated by the cognate in Hidatsa \textit{mashíi}$\sim$\textit{washíi} `white person', which features a long vowel of unclear origin. Hidatsa could have innovated this length to avoid confusion with \textit{mashí}$\sim$\textit{washí} `blanket', which ends in a short vowel.}

In more recent times, Núeta or Nu'eta has generally been the term used to describe all Mandan, regardless of which band one belongs to, though some individuals resist this and prefer to identify by their own band.\footnote{The Hidatsa similarly are named for their largest band, the \textit{Hiráaca}, though there is currently no contention over using this as a cover term for that group in English or Hidatsa.} Throughout this book, I will simply use the exonymic term `Mandan' instead of attempting to use an autonym, given the lack of consensus over what members of this ethnic and linguistic community wish to call themselves.

Today there are no L1 speakers of Mandan, given the fact that the last L1 speaker of the \textit{Núu'etaa} variety of Mandan passed away on December 9, 2016 at his residence in Twin Buttes, ND. There are several heritage speakers between Twin Buttes, Mandaree, and New Town, and one of these heritage speakers grew up with the \textit{Rúptaa} variety of Mandan. Much of the old dialectal differences have been leveled due to the population loss that occurred after the Smallpox Epidemic in 1837. \citet{carter1991a} examines the only published source of grammatical and lexical differences between Mandan dialects: the \citet{maximilian1839} wordlist. These grammatical differences will be further discussed at a later point in this book, but some lexical differences remain even into the present. The \textit{Núu'etaa} variety is the prevalent one found in previous literature and represents the vast majority of the data presented herein. I note any non-\textit{Núu'etaa} forms within the present work, but \textit{Núu'etaa} remains the de facto standard speech variety.

There is an urgent need for documentation and sharing of linguistic materials kept around Fort Berthold and at other locations off the reservation. There is currently a coordinated reservation-wide push for revitalization under the auspices of the MHA Nation itself. From 2013 until 2017, the Tribal Council had been awarding the Language Conservancy a contract to produce pedagogical materials for all three languages on the reservation and put on a two-week long summer institute. I was affiliated with the Language Conservancy and taught at the inaugural summer institute and each subsequent one until the summer of 2016. In Twin Buttes, the Nueta Language Initiative works with residents and local elementary school children to produce materials and lessons to pass on the language. With the last L1 speaker having passed away, the Mandan language is in a precarious situation and is desperately in need of additional study, not just for the purposes of examining the several typologically rare features it has, but also for the sake of current and future Mandan people who may wish to experience this aspect of their cultural heritage. 

\subsection{900 CE to 1851 CE}

The Mandan and their ancestors have lived near the Middle Missouri River since at least 900--1000 CE. %Indeed, the ancestors of the Mandan may have been the among the first permanent inhabitants of that region. By the end of the Woodlands period, a multivariate principal components analysis of 860 crania from the Dakotas, Iowa, Kansas, Missouri, and Nebraska reveals a strong biological continuity in native population from the initial Paleo-Indian settlement around 10,000 BP to the end of the Woodlands period around 900 CE \citep[146]{key1982}. Changes in this biological continuity after 900 CE overlap with the collapse of population centers in the Midwest around that time and an influx of migration from groups from the south, such as the Caddoan-speaking Arikara. 
\citet[203]{lehmer1971} notes that historic Mandan material culture represents a direct continuation of the older Middle Missouri Tradition. The Middle Missouri Tradition is the cultural complex found within the Missouri River Valley and the adjacent prairie and plains from the confluence of the Missouri and Cheyenne Rivers in central South Dakota to the confluence of the Knife and Missouri Rivers in western North Dakota \citep[84]{will1906}.  The Middle Missouri Tradition is distinct from that of neighboring Central Plains traditions in its style of pottery, design of domiciles, composition of fortifications, and manner of burials (\citealt[202]{lehmer1971}; \citealt[10]{johnson2007}). Further archaeological evidence supports the notion of uninterrupted habitation of the region of the Middle Missouri River, with distinct archaeological evidence directly attributed to the Mandan (\citealt[97]{lehmer1971}; \citealt[109]{johnson2007}). The map in \figref{mandanmigration} highlights the major movements of the bulk of ancestral Mandan people(s) since the end of the Woodlands period until the era during which they made contact with Europeans.\footnote{This map is adapted from one in \citet[5]{fenn2015}, using the Wikimedia Commons map \href{https://commons.wikimedia.org/wiki/File:Mississippiriver-new-01.png}{File:Mississippiriver-new-01.png} by user Shannon1 as a base.}

\begin{figure}[t]
\caption{Mandan migrations}
\label{mandanmigration}
\includegraphics[width=0.90\textwidth]{figures/mandanmigrations2.png}
\end{figure}

The confluence of the Heart and Missouri Rivers in North Dakota is often considered to be the homeland of the Mandan, though they had occupied lands farther down the Missouri River in the past. This area is highlighted on the map in \figref{mandanmigration}. \citet[26]{lehmer1971} states that sites in southern Minnesota and northern Iowa form a cultural continuum with sites in central South Dakota that archeologists attribute to the Mandan and their ancestors. The migration of the Caddoan-speaking Arikara from the central Plains onto the Missouri River during the thirteenth century resulted in the Mandan gradually moving farther upstream.\footnote{The confluence of the Heart River and the Missouri is approximately at the center of the region highlighted as the Ancestral Mandan settlement area in the map above.} This re-settlement up the Missouri brought the Mandan to the Heart River in North Dakota and into close contact with the Hidatsa, with whom they developed close relations -- with occasional disputes -- that have lasted into the present day.

It is not clear whether this relocation upstream was voluntary on the part of the Mandan, or if the movement of additional peoples onto the upper Missouri triggered the northern migration of the Mandan.\footnote{\citeapos{fenn2015} map in \figref{mandanmigration} does not reflect the migration of various groups of Lakota, Dakota, and Nakota who moved out of the Great Lakes region in the 16th century due to pressure from the Ojibwa and Cree, who had moved to the region from the east and had procured guns through the French via the fur trade \citep[170]{riggs1893}. The Cheyenne likewise had to abandon their lands in the Great Lakes due to conflict with neighboring peoples and pass through the Missouri Valley, coming into contact with the Mandan \citep[18]{moore1999}. The coming of these peoples into the region could also been a factor in the gradual progression northward of the Mandan people.} The infamous Crow Creek site contains the aftermath of a brutal massacre that took place around 1350 CE that could have been spurred on by the drive to take the arable land upon which that settlement sat. The identity of the villagers is thought to be that of a group related to or ancestral to the Arikara. The belligerents were almost certainly groups ancestral to the Mandan, given that these two groups were relatively alone in the area until the arrival of newer groups from the upper Midwest (e.g., the Cheyenne and the Lakota) almost three centuries later \citep{zimmerman1980,fenn2015}.

Within the Heart River area, the Mandan did not necessarily inhabit a single site continuously throughout the time frame between 1350 and the era before relocation to the Fort Berthold Indian Reservation in 1870. There is evidence from both archeology and first-hand accounts of the Mandan migrating short distances to be closer to supplies of lumber or moving to avoid a hostile group, such as the Assiniboine or the Lakota. \citet[104]{allen1814a} cites a November 21, 1804 journal entry by Meriwether Lewis on this subject, with additional points of clarification presented in footnotes:

\begin{quotation}

The villages near which we [the Corps of Discovery] are established are five in number, and are the residence of three distinct nations: the Mandans, the Ahnahaways,\footnote{i.e., the Awaxaawi band of Hidatsa, whose name means something similar to `branching land.'} and the Minnetarees.\footnote{i.e., the Awadixaa band of Hidatsa (whose name means `short village') and the Hidatsa proper (whose meaning is opaque, but folk etymology states that it is derived from a variety of willow) are collectively called \textit{Minítaari} `water crossers' by the Mandan, due to the story of the Mandan and Hidatsa's first meeting, where the Hidatsa had crossed the Missouri River in bullboats to greet the Mandan.} This history of the Mandans, as we received it from our interpreters and from the chiefs themselves, and as it is attested by existing monuments, illustrates more than that of any other nation the unsteady movements and the tottering fortunes of the American nations. Within the recollection of living witnesses, the Mandan were settled forty years ago in nine villages, the ruins of which we passed about eighty miles below, and situated seven on the west and two on the east side of the Missouri. The two finding themselves wasting away before the small-pox and the Sioux,\footnote{i.e., the Lakota (also known as Tetons or \textit{Thítȟuŋwaŋ} [ˈtʰi.tˣũ.wã] `prairie dwellers', and Yankton Dakota or \textit{Iháŋktȟuŋwaŋ} [i.ˈhã.ktˣũ.wã] `those dwelling at the end').} united into one village, and moved up the river opposite the Ricara.\footnote{i.e., the Arikara (also known as Ree or \textit{Sáhniš} [ˈsah.niʃ] `people').} These same causes reduced the remaining seven to five villages, till at length they emigrated in a body to the Ricara nation, where they formed themselves into two villages, and joined those of their countrymen who had gone before them. In their new residence they were still insecure, and at length the three villages ascended the Missouri to their present position.

\end{quotation}

The Mandan split their time between summer villages and winter villages. The winter village would serve as their home for a quarter of the year; they would move into the lowlands near their summer villages. Such areas would have forestation or topography that would serve to block the cold winter winds of the Plains. Their summer villages, however, were more permanent and were chosen for their defensibility and the fecundity of the land for agriculture. A summer village depended on a reliable source of lumber for the construction of new earth lodges and to fuel the hearth fires at the center of each lodge. Once the supply of lumber had been exhausted, the village would have to move elsewhere. A village could normally last one or two generations before the local supply of wood had been depleted. However, On-a-Slant Village, located a few minutes south of modern Bismarck, ND at the confluence of the Heart and Missouri Rivers, had been occupied since the last half of the sixteenth century and was only abandoned near the beginning of the eighteenth century due to population collapse caused by European diseases \citep[118]{fenn2015}.

The catastrophic effect that illnesses like smallpox and the measles had upon the Mandan cannot be overstated. The village chief of Mitutanka\footnote{This name is also spelled Mih-Tutta-Hang-Kush by \citet{maximilian1839}, which is \textit{Mí'tųųtaahąhkas} /wį'\#ti\#ųųtaahąk=ka=s/ `the East Village' in the new orthography.} was known as Shehek Shote,\footnote{His name is also spelled Shekehe Shote or Shahaka, which is \textit{Shehékshot} `White Coyote' in the orthography used here.} who related to Meriwether Lewis that he was born in On-a-Slant Village, which was the smallest of the nine Mandan villages at the time, having a mere eighty-six earth lodges and approximately one thousand inhabitants. Shehek Shote's description of On-a-Slant before the 1781 smallpox outbreak suggests that the total pre-pandemic population of the Mandan was between ten and fifteen thousand. One generation later, Meriwether Lewis estimates that the two Mandan villages near Fort Lincoln could raise a total of seven hundred warriors, suggesting a population of at least two thousand people, with similar numbers estimated for the Hidatsa two miles upriver \citet[131]{allen1814a}. 

Three decades after Lewis and Clark had visited the Mandan, smallpox returned to the Middle Missouri. The outbreak of 1837 nearly caused the extinction of the Mandan people. The fur trader Francis Chardon wrote that ``the small-pox had never been known in the civilized world, as it had been among the poor Mandans and other Indians.'' \citep[20]{stearn1945}. The smallpox first took hold among the Mandan in June of 1837, and by the end of August, Chardon wrote that ``the Mandan are all cut off except twenty-three young and old men'' \citep[297]{calloway2008}. The fact that Chardon focused on the number of men makes it difficult to give a precise number for the Mandan who survived the devastation of the  1837. \citet[223]{fenn2015} cites various sources that give conflicting accounts of how many Mandan survived the smallpox outbreak, but the number was certainly no more than three hundred and possibly as low as near thirty.

This population crisis severely affected the economic and political position of the Mandan on the Middle Missouri. By 1851, the two Mandan villages visted by Lewis and Clark, Mittutaka and Ruptare\footnote{The name of this village is synonymous with the band of Mandan that occupied it: \textit{Rúptaare}, called the `two voices' or `the ones who came second' by other Mandan. In the \textit{Núu'etaare} variety, they may call this dialect \textit{Núptaare}. It is varyingly referred to in the literature as Rupta, Nupta, Ruptadi, or Nuptadi.} were still severely depopulated, with Mittutaka having between eight and twenty-one lodges occupied, depending on the source, while a fur trade said that Ruptare was occupied by just ``fourteen huts, most of them empty'' \citep[72]{kurz1937}.

The Mandan of Mittutaka moved upriver in the summer of 1845 to join with the Hidatsa, who had suffered severely from the smallpox as well. Together, these two people settled Like-a-Fishhook Village, named so for the shape of the land jutting into the Missouri River upon which they built their new homes. This was the beginning of the cohabitation between the Mandan and Hidatsa, which continues to this day. The residents of Ruptare followed the rest of the Mandan to Like-a-Fishhook 1857 following yet another smallpox outbreak, and in 1862, the Arikara joined them. These three tribes have lived with each other ever since, and are classified as a single tribe by the federal government under the name the Three Affiliated Tribes (TAT) or the Mandan, Hidatsa, Arikara (MHA) Nation.

\subsection{1851 CE to present day}\label{postlaramie}

The government included the Three Affiliated Tribes in with a group of other nations in the 1851 Treaty of Fort Laramie. The signatories included the Arapaho, Arikara, Assinboine, Cheyenne, Crow, Hidatsa, Lakota and Dakota, and Mandan. The Mandan, Hidatsa, and Arikara nations were grouped together under this treaty, wherein they were allotted 12.5 million acres that encompassed portions of what is now North and South Dakota, Montana, and Wyoming. Not long after agreeing to this treaty, hostilities resumed on the northern plains as non-Native settlers began to forcibly settle upon these lands and scarcity of bison and game caused friction between Native peoples of the region. The Arikara were forced to the opposite bank of the Missouri River after incursions from the Lakota, and the United States government did little to protect the land rights or otherwise protect the Three Affiliated Tribes, including the Mandan. 

Through a series of executive orders during the eighteenth century, the land was whittled down to just over 643 thousand acres by 1891, as shown in \figref{laramie}.\footnote{This map was created by the \citet{northdakota}, adapted from the one in \citet[193]{meyer1977}} The Mandan and other treaty signatories were promised an annuity of fifty thousand dollars for fifty years, which Congress later amended to just ten years. Settlers were permitted to pass through Mandan lands, but in the rush for gold on the West Coast, some prospectors decided to stay. The United States did not uphold its end of the treaty in preventing Americans and Europeans from homesteading on sovereign lands. Like-a-Fishhook, near Old Fort Berthold, was gradually abandoned due to high volume of white settlers and attacks by hostile bands of Dakota and Lakota \citep[119]{meyer1977}. By 1887, The U.S. government encouraged residents to move upriver to settle on allotments in and around Elbowoods, which would remain a major population center for the Mandan people and the other members of the Three Affiliated Tribes for next seven decades \citep[1]{densmore1923}.\footnote{This map was created by the \citet{northdakota}, adapted from the one in \citet[112]{meyer1977}.}


\begin{figure}[t]
\caption{Land under the Treaty of Fort Laramie and subsequent cessions}
\label{laramie}
\includegraphics[width=.75\textwidth]{figures/laramie2}
\end{figure}

This problem of European-American encroachment upon lands that were supposed to be legally closed to settlement continued even into the twentieth century, where in 1910, the tribe was forced to allow homesteaders in the northeastern quadrant of the reservation under the Act of June 1, 1910 (36 Stat. 455). The act of opening up this land to white settlers alienated another 60 thousand acres from the Three Affiliated Tribes. A map of the Fort Berthold Indian Reservation and the major population centers appears in \figref{oldfb}.\footnote{This map was created by the \citet{northdakota}, adapted from the one in \citet[193]{meyer1977}.}

\begin{figure}
\caption{Fort Berthold Indian Reservation pre-1953}
\label{oldfb}
\includegraphics[width=0.9\textwidth]{figures/oldbf2.jpg}
\end{figure}

The seasonal flooding of the Missouri River created excellent farmland in the river valley as nutrients were added to the lowlands every year. The Army Corps of Engineers, however, decided that the seasonal flooding of the Missouri was too problematic for farmers and boat traffic farther downriver, so a series of dams were constructed to prevent such floods. The result of these dams, and the Garrison Dam in particular had a strongly negative impact upon the Mandan, who had often continued to support themselves through farming throughout the reservation period. The construction of the Garrison Dam resulted in the creation of an artificial lake where the Missouri would rise up and overflow its banks. The Three Affiliated Tribes were given compensation from the Department of the Interior for the loss of 146 thousand acres, over a fifth of their total territory, but the money would not ameliorate the situation that the creation of what is now Lake Sakakawea caused: the majority of the remaining land was dry and rocky, ill-suited for farming. Furthermore, nearly every settlement on the reservation was on the Missouri River, so thousands had to move from their home in the face of the rising waters of the Missouri \citep{harper1948}. A map of the current-day Fort Berthold Indian Reservation appears in \figref{newfb}.\footnote{This map was created by the \citet{northdakota}, adapted from the one in \citet[239]{meyer1977}}

\begin{figure}
\caption{Fort Berthold Indian Reservation post-1953}
\label{newfb}
\includegraphics[width=0.9\textwidth]{figures/newfb2.jpg}
\end{figure}

Most of the Mandan people lived in Elbowoods, Charging Eagle, or Red Butte before the Dam, and despite the presence of the Missouri River, it was relatively simple for families and friends to cross the river to visit. After the Dam, Lake Sakakawea was too wide and too deep for easy crossing (\citeauthor{benson2000} p.c.). The Mandan spread out across the reservation, concentrating in the Southern Segment in what is now Twin Buttes, but also being present in Mandaree\footnote{Mandaree was originally intended to be a home for all three peoples on Fort Berthold, with its name being a blend of \textbf{\textit{Man}}dan, Hi\textbf{\textit{da}}tsa, and \textbf{\textit{Ree}}. The latter, Ree, is a term often used for the Arikara in older literature.} in the Western Segment or New Town in the Northern Segment. Where the Mandan-speaking community had been relatively concentrated in a single area beforehand, it was now mixed in with Arikara- and especially Hidatsa-speakers. 

Already a linguistic minority on the reservation around the turn of the 20th century \citep[2]{kennard1936}, many Mandan speakers intermarried with Hidatsa-speakers, resulting in language attrition as the default indigenous language on the reservation shifted gradually to Hidatsa. This shift can be attributed to the fact that at the beginning of the twentieth century, there were several larger families of mixed heritage that employed Hidatsa in the home due to Hidatsa being the language with which both parents were most comfortable (\citeauthor{birdbear2016} p.c.). Numerous Hidatsa-speaking elders on Fort Berthold are actually ethnically Mandan due to this interruption in language transmission, hastening the decline in L1 Mandan speakers in the middle part of the twentieth century. One possible reason for shifting so readily to Hidatsa from Mandan during the twentieth century might lie in the similarities between the two languages, facilitating the acquisition of Hidatsa by ethnic Mandan who were married or related to ethnic Hidatsa during the first half-century after the reservation period. \citeauthor{benson2000} (p.c.) has stated that both languages are quite different from each other, but have many grammatical constructions in common. This commonality might be due to the centuries of close interaction between the two peoples, but it also might be due to common genetic similarities in the languages themselves.


\section{Genetic relationships}\label{genetics}

The position of Mandan within the Siouan language family has long been a point of contention. \citet[16]{grimm2012} summarizes previous analyses of the relationship of Mandan to other languages as ``somewhat arbitrary.'' These past attempts to explain the relationship between Mandan and other Siouan langauges have relied on limited comparisons of vocabulary, not paying much attention to grammar or phonology that is either shared with or innovated from Proto-Siouan. This section makes the case that Mandan is most closely related to Hidatsa and Crow, and that this relationship is supported by original computational phylogenetic work done by the author. 

The most current consensus tree appears in \figref{rankintree}. Within Siouan, there are three families apart from Mandan: Missouri Valley, consisting of just Crow and Hidatsa; Ohio Valley, consisting of Biloxi, Ofo, and various forms of Virginia Siouan; and Mississippi Valley, consisting of numerous other groups like the Lakota, Omaha, and the Hoocąk.\footnote{The Hoocąk are also known as the \ili{Ho-Chunk} or Winnebago. This work opts to use their autonym, as Winnebago is an exonym from an Algonquian language (cf. \ili{Potawatomi} from \citealt{neely2019}: \textit{winbyégo} `Hoocąk' < \textit{win-} `dirty' + \textit{byék} `waters'), meaning something to the effect of `dirty water people', due either to their presence by Green Bay on Lake Michigan, which often experiences strong algal blooms or to their proximity to the muddy Fox River. The Canadian city Winnipeg shares a similar etymology, e.g., Plain Cree \textit{wînipêk} `body of muddy water.'}


\begin{figure}
\caption{Consensus tree for Siouan Proper from \citet{rankin2010}}
\label{rankintree}
\begin{tikzpicture}[sibling distance=-5pt]
\tikzset{every tree node/.style={align=center,anchor=north}}
\footnotesize

\Tree[.\textbf{Siouan} [.\textbf{Missouri}\\\textbf{Valley}\\Crow\\Hidatsa ] [.~~~~~~~~~~Mandan~~~~~~~~~~ ] [ [.\textbf{Ohio}\\\textbf{Valley} [.\textbf{Southeastern}\\\textbf{Siouan}\\Biloxi\\Ofo ] [.\textbf{Virginia}\\\textbf{Siouan}\\Monyton\\Occaneechi\\Saponi\\Tutelo ] ] [.\textbf{Mississippi}\\\textbf{Valley} [.\textbf{Dhegihan}\\Kansa\\Omaha\\Osage\\Ponca\\Quapaw ] [.\textbf{Dakotan}\\Assiniboine\\Dakota\\Lakota\\Stoney ] [.\textbf{Hoocąk-Chiwere} [.Hoocąk ] [.\textbf{Chiwere}\\Ioway\\Missouria\\Otoe ] ] ] ] ] ]


\end{tikzpicture}

\end{figure}

Given their proximity and cultural ties with the Hidatsa and Crow, Mandan has often been grouped with them as part of the Missouri Valley family. Early researchers on the Plains, such as \citet[97]{will1906}, divide the Siouan languages of the Plains into four groups based on archeological and first-hand accounts of their migrations. The Mandan, Hidatsa, and Crow were the first to move onto the Plains, followed by Hoocąk and Chiwere-speaking groups, then Dhegihan speakers, then Dakotan speakers in the seventeenth century.

\citeapos{will1906} proposal for the subdivisions within Siouan language family did not take the Ohio Valley languages into account, nor any other relationships between their proposed four-way distinction. \citet[249]{voegelin1941} groups Mandan with Hoocąk based on a single phenomenon they have in common, i.e., Dorsey's Law, where a copy vowel is inserted to break up clusters involving a sonorant \citep[923]{dorsey1885}.\footnote{I argue in \S\ref{dorseyslaw} that these Dorsey's Law vowels are not phonologically generated, but are post-phonological (i.e., phonetic) in that they are not treated as syllables for the purpose of stress assignment due to their status as excrescent (or intrusive) vowels rather than epenthetic vowels.} We can see this rule formalized in (\ref{chapter1dorseyslaw}) below.

\begin{exe}

\item\label{chapter1dorseyslaw} \textbf{Dorsey's Law}\\
/CRV\textsubscript{1}/ $\to$ [CV\textsubscript{1}RV\textsubscript{1}]\\
	\textit{Insert a copy of the following vowel between a consonant-sonorant cluster.}

\end{exe}

Grouping Mandan with Chiwere and Hoocąk on the basis of a single shared phonetic characteristic is problematic, as \citet[246]{voegelin1941} himself notes that the intrusive copy vowel found in Mandan and Hoocąk is also found in Dakota and in various Dhegihan varieties (i.e., most of the language family). After conducting a brief survey of lexical and phonological characteristics of Siouan languages, \citet{wolff1950a,wolff1950b,wolff1950c,wolff1951} likewise remarks that grouping languages by a single shared phonological feature is not especially convincing, and instead proposes a different grouping, stating that Siouan had seven divisions: 1) Crow and Hidatsa, 2) Mandan, 3) Dakota,\footnote{\citet{wolff1950a,wolff1950b,wolff1950c,wolff1951} uses Dakota as a cover term for all Dakotan languages, i.e., Lakota, Yankton Dakota, Assiniboine, Stoney, etc.)} 4) Chiwere and Hoocąk, 5) Dhegiha,\footnote{The Dhegihan group includes Omaha-Ponca, Kanza-Osage, and Quapaw.} 6) Ohio Valley Siouan, and 7) Catawba.

\citet[54]{headley1971} argues that Mandan forms a clade with Missouri Valley due to the degree of lexical similarity between those two groups. \citet[255]{rood1979} opts to place Mandan within its own branch of Siouan, while \citet{koontz1985} argues that Mandan forms a basal clade within the Mississippi Valley family. Ultimately, \citet{rankin2010} argues that the place of Mandan within Siouan is too difficult to discern due to the large amount of morphology it shares with other Siouan languages and the fact that many of the lexical similarities between it and Crow-Hidatsa could be due to contact.

More recent work in computational phylogenetics points to Mandan truly belonging with Missouri Valley \citep{kasak2015}. Making use of a suite of phylogenetic software and a database of cognates derived from \citeapos{rankin2015} \textit{Comparative Siouan Dictionary}, I created a character set of 446 lexical item coded for cognacy. The data then underwent a Bayesian maximum-likelihood analysis using BEAST \citep{drummond2007}, using a stochastic Dollo model, a lognormal relaxed clock, and a UPGMA starting tree. The resulting set of trees were then summarized into a target tree using TreeAnnotator, followed by generating this target tree using FigTree. The results firmly placed Mandan with Hidatsa and Crow, though at a deeper time depth than Hidatsa and Crow from each other. Furthermore, this analysis supports the language isolate Yuchi being genetically related to Siouan, a relationship first championed by \citet{sapir1929} and latter again by \citet{rankin1996,rankin1998} and \citet{kasak2016} since Yuchi clusters within already existing branches of the Siouan family tree.

The newly-proposed Siouan family tree appears in \ref{newtree}, where Mandan forms a basal clade within Missouri Valley, while Catawba and Yuchi form a clade with Mandan-Missouri Valley. Ohio Valley and Mississippi Valley likewise form a clade, as previously described by \citet{rankin2010}. This study was done just on lexical items, and future work should involve incorporating morphology into the character set. However, what is noteworthy about this work is that it captures the established subgroupings within Siouan, both with respect to the major families (i.e., Mississippi Valley, Ohio Valley, Missouri Valley, and Catawban), but it also captures higher-order groupings that had been discussed openly among Siouanists, such as the fact that Mississippi Valley and Ohio Valley share a large number of lexical innovations not found in Missouri Valley, Mandan, or Catawban.

\begin{figure}
\caption{Siouan family tree from \citet{kasak2015}}
\label{newtree}

\includegraphics[scale=.48]{figures/newtree.png}

\end{figure}

The tree in \figref{newtree} includes the posterior probabilities of each clade given the taxa inputted into the data (i.e., each language with data in the character set). A posterior probability is the statistical probability that proposition is true having taken some evidence into account under a Bayesian analysis. For the established subgroupings (e.g., Dakotan, Southeastern, Dhegihan, etc.), the posterior probabilities were quite high (i.e., $p$>0.95). For other expected groupings, like Lakota and Dakota, which form a dialect continuum, we see a low posterior probability that is due to the Bayesian analysis dealing with very closely-related language varieties by trying different results (i.e., trees) that do not improve the probability for forming a clade. Another confounding factor lies in the composition of the cognate set, since instances where Lakota and Dakota shared a lexical item were rarely recorded, but items that differentiate between them were regularly included. If a newer set of data were coded that includes every cognate between Lakota and Dakota that did not simply assume that Lakota and Dakota share an item unless otherwise stated, we would expect to see an extremely high posterior probability of Lakota and Dakota forming a clade, given their mutual intelligibility.

The low posterior probability for the clade including Yuchi-Catawban and Mandan-Missouri Valley could be caused by the time depth separating them or the high degree of innovation within Yuchi-Catawban. The analysis yields a tree where Catawban and Yuchi form an in-group, rather than an out-group, suggesting further work is needed to understand the correspondences between Proto-Siouan and Catawban-Yuchi. Nonetheless, the biggest takeaway from these findings is that Mandan is not an isolate within the Siouan language family, but has demonstrably closer ties to Missouri Valley languages and shares a stronger lexical affinity with Catawban and Yuchi than with Mississippi and Ohio Valley Siouan.

The purpose of building the case for the place of Mandan within Siouan serves two purposes for this work. Firstly, we can see that Mandan is not alone within the family tree, despite its uniqueness among Siouan languages in sharing so many features and lexical items with other members of the language family across so many branches. Secondly, this work places Mandan within the same subfamily as Hidatsa and Crow, which suggests further comparative grammatical study is needed between these three languages, as well as raising the possibility for investigating what a proto-language would look like between them. The argumentation in this section is relevant to the book in that if the synchronic analysis of Mandan affix ordering within a template holds, then we can use the same analysis of the templates of other Siouan languages to look at the diachronic reordering of affixes (i.e., realization constraints being reordered with respect to one another) across various branches of the language family to examine the ways in which language change can occur at the morphological level in a language family that features such a diverse array of affixes within the templates of its members.

\section{Previous research on the Mandan language}\label{previousresearch}

This section serves to examine the research on the Mandan language that has taken place up to this point in time. There are no published bibliographies of Mandan language resources of which I am aware, so the following information shall act as a bibliography of Mandan. This documentation is meant to assist in future research on the language by pointing academic and community scholars to resources on the Mandan language and where those resources are held.

The first published account of the Mandan language was by Prince \citet{maximilian1839}, who lived among the Mandan people with the Swiss artist Karl Bodmer in the years before the Smallpox Epidemic of 1837. Together, they introduced the rest of the world not only to the striking visual depiction of life among the Mandan, but also to their language. To this day, Maximilian's vocabulary with its brief grammatical sketch is the only documented source to compare lexical and grammatical differences between the \textit{Núu'etaa} and \textit{Rúptaa} varieties of Mandan.

The American trader James \citeapos{kipp1852} wordlist was published in the \citet[446]{schoolcraft1853} collection, which aimed to document the numerous indigenous languages of the United States. \citeauthor{kipp1852} lived with the Mandan for a time, even marrying a Mandan woman. While no linguist, his vocabulary consisted of nearly 350 words, ranging from plants and animals to physical actions and simple verb paradigms. \citeapos{schoolcraft1853} transcription of \citeapos{kipp1852} handwritten vocabulary list is unfortunately riddled with typographic errors, as well as a few instances of confusing Mandan data and non-Mandan data from some other list. As such, any attempts to use these data should refer back to the original handwritten list by \citeauthor{kipp1852}, which is currently stored at the Smithsonian Institution in Washington, DC. When looking at this modest lexicon, there are several words in his vocabulary that are different from the terms used today. One such example of this change appears below in (\ref{semantichorse}), where the word for `horse' went from being a descriptive compound that approximated the appearance of this animal to the compound being reduced through its frequency of use until the second member of the compound replaced the semantics of the original word \textit{miníse} `dog'.

\begin{exe}

\item\label{semantichorse} Reduced compound in Mandan: `horse'\footnote{The pre-contact Mandan word for `dog' was \textit{miníse}, which is historically derived from the unspecified argument marker \textit{wa-} and the verb \textit{inís} `be alive', i.e., `something that is living' or `animal'. But upon the introduction of the horse, \textit{miníse} became generalized to any domestic quadraped, and eventually `dog' became \textit{minís wéerut} `horse that eats feces' to distinguish it from what is now the word for horse, \textit{miníse}. Other Plains languages similarly equated horses with dogs, e.g., Lakota \textit{šúŋkawakȟáŋ} `horse' (lit. `holy dog'), though \textit{wakȟáŋ} `holy' may  be dropped so that \textit{šúŋka} can be either `dog' or `horse' in casual speech.}

	\begin{xlist}

	\item Older Mandan: \textit{ų́ųpa miníse} [lit. `elk dog'] \citep{kipp1852}

	\item Contemporary Mandan: \textit{miníse} `horse' \citep{kasak2014}

	\end{xlist}
	
% \item Semantic shift in Mandan: \textit{mashkáphskapka} [lit. `it pricks all over']

%     \begin{xlist}
    
%     \item Older Mandan: `rosehips' \citep[182]{hollow1973a}
    
%     \item Contemporary Mandan: `tomatoes' \citep[65]{hollow1976}
    
%     \end{xlist}

\end{exe}

In the twentieth century, there was a renewed interest in the people of the Plains. Musicologist Frances \citet{densmore1923} recorded over one hundred Mandan and Hidatsa songs, including the lyrics and their translations, though she was more concerned with the music itself rather than the words. Her recordings were done on wax cylinders, and due to the fact she had to power her equipment using the engine of her Model T car, the audio quality is quite poor. However, these represent the first recorded instances of Mandan in an auditory medium. Mr. \name{Ben}{Benson}\footnote{His name was originally \textit{Weróokpa} `Buffalo Bull Head', but his legal name was given to him by missionaries.} (1867--1939) was the grandfather of the last L1 Mandan speaker and one of \citeauthor{densmore1923}'s principal Mandan consultants, singing 16 of the 82 Mandan songs recorded during \citeauthor{densmore1923}'s fieldwork. The antropologist Robert \citet{lowie1913} describes meeting with several Mandan consultants in his description of the societies of the Crow, Hidatsa, and Mandan, and Mr. \name{Ben}{Benson}'s name is never mentioned. However, it is likely that Mr. Benson would have been among those consulted on the histories and functions of various moieties, given his age and social stature at the time of \citeauthor{lowie1913}'s visits to Fort Berthold. Mr. Benson's participation in this project led to him later working  beginning a nearly century-long practice of his family working with outside scholars.\footnote{It is not clear what has led to the Benson family having such a long history in working with outside scholars, though it could have to do with the fact that members of the Benson family have held at least some of the sacred Turtle Drums, which traditionally have been the holiest artifacts to the Mandan people. Other holders of a Turtle Drum, such as Mr. \name{Leon}{Little Owl}, have likewise assisted in documentation efforts, so it is possible that having such important ceremonial responsibilities has also conveyed either the desire or the responsibility to share their language. \citet[105]{bowers1950} writes that Mr. \name{Ben}{Benson} possessed two of the three sacred Turtle Drums as of 1931, and that Mrs. Scattercorn held the other. The Mandan believe that a fourth turtle drum left below the waters of the Missouri and may return one day (\citeauthor{benson2000} p.c.). At the time of his passing in 2016, Mr. \name{Edwin}{Benson}, \name{Ben}{Benson}'s grandson, was the keeper of two Turtle Drums.}

The anthropologist Edward \citeauthor{kennard1936} visited the Mandan in the summers of 1934 and 1935, working with many of the same consultants who had worked with \citeauthor{densmore1923} almost two decades earlier. While \citeauthor{kennard1936} was mostly concerned with the folklore of the Mandan people, he produced the earliest textual corpus of Mandan in the form of 302 typewritten pages containing 28 narratives in Mandan, along with free translations and some basic interlinear glossing for certain texts \citep{kennard1934}. His work with the Mandan resulted in the first published grammar of this language \citep{kennard1936}. Two of Ben Benson's grandchildren later assisted with Mandan documentation efforts: Mrs. \name{Louella}{Benson Young Bear} (1921--2008) and Mr. \name{Edwin}{Benson} (1931--2016), who was the last L1 speaker of the Mandan language. Mr. \name{Edwin}{Benson} formerly worked as the Mandan teacher for Twin Buttes School, following the retirement of his cousin, Mrs. \name{Otter}{Sage}.

Alfred \citeapos{bowers1950} ethnographic work on Mandan ceremony and social practice remains a valuable resource for information on Mandan culture due to \citeauthor{bowers1950}'s numerous consultants who had grown up before the forced assimilation imposed on Mandan families during the reservation and boarding school era. \citeauthor{bowers1950} was reputed to be a competent speaker of Mandan, being able to translate Crows Heart's autobiography from spoken Mandan into written English in 1947. He later returned to Twin Buttes, ND to back-translate it into Mandan and Hidatsa with two fluent speakers, collecting nearly 150 hours of recordings. These recordings were done in 1969 and then sent to the American Philosophical Society for archiving \citep{bowers1971}.

Robert \citeauthor{hollow1970}, one of Wallace Chafe and Terrence Kaufman's students at the University of California, Berkeley, undertook field work in the late 1960s that resulted in the first and only dictionary of the Mandan language \citep{hollow1970}. \citeauthor{hollow1970} continued to work on Mandan after completing his doctorate, recording and transcibing 24 narratives \citep{hollow1973a}, and also re-eliciting and re-transcribing \citeapos{kennard1934} narratives. Though no known audio recordings of those sessions exist, \citep{hollow1973b} re-elicited and transcribed all but four of \citeapos{kennard1934} 28 narratives. He collaborated in efforts to revitalize the Mandan language through the creation of a textbook \citep{hollow1976}, and he published translated Mandan narratives in the \textit{Earth Lodge Tales from the Upper Missouri} collection \citep{parksjoneshollow1978}. Dr. \citeauthor{hollow1970} passed away in Bismarck, ND in 1986 due to complications from cancer at the age of 41.

Mauricio \citeauthor{mixco1997a}, a classmate of \citeauthor{parks1976} and \citeauthor{hollow1970}'s from Berkeley, thereafter began working on Mandan in the summer of 1993. These efforts produced a grammar sketch \citep{mixco1997a} and an overview of Mandan's switch-reference system \citep{mixco1997b}. His Mandan fieldwork produced no other publications, as Dr. Mixco became increasingly involved with the Shoshoni Language Project at the University of Utah.

Sara \citeauthor{trechter2012} began working on Mandan in 2000 following a suggestion from the late Robert Rankin while she was a doctoral student at the University of Kansas. Dr. \citeauthor{trechter2012} continued to work on Mandan through 2012, producing pedagogical materials alongside local Mandan heritage language learners and Mr. Edwin Benson, the man who was then the last L1 Mandan speaker. These efforts culminated in two DVDs, \textit{In the Words of Our Ancestors}, which showed video footage of Mr. Benson sitting in an earth lodge in traditional regalia, telling traditional Mandan narratives in Mandan. The DVDs were accompanied by data CDs that included transcriptions and translations of those narratives so listeners can follow along \citep{trechter2012}.

The most recent work on Mandan has been by Indrek \citeauthor{park2012}, who has been working with the Nueta Language Initiative in Twin Buttes, ND. As of this writing, he still lives with the community in Twin Buttes and participates with revitalization efforts for both Mandan and Hidatsa.

This summary of existing Mandan research highlights how limited the published linguistic information on the Mandan language is, despite the major role the Mandan people played on the economy of the Upper Missouri for the past half millennium, leading up until the reservation period.

\section{Personal fieldwork and sources of data}\label{fieldwork}

This section serves to explain the conditions under which I conducted my own fieldwork and state the sources of the data used within this book. Given the extremely small number of possible consultants at the beginning of my work on Mandan, I have had to rely mostly on previous fieldwork, though I thankfully have been able to work with the last L1 Mandan speaker, Mr. Edwin Benson, in a limited capacity up until 2016. Below, I describe my fieldwork and list the sources that act as the corpus from which I draw most of the data in the present work.

\subsection{Personal fieldwork}

My own fieldwork with Mandan began in the summer of 2014, when I first traveled to the Fort Berthold Indian Reservation to meet with the lone L1 Mandan speaker and investigate the possibility of finding other speakers. This trip was partially funded by a 2014 grant from Phillips Fund for Native American Research through the American Philosophical Society. I was asked by the Language Conservancy to create pedagogical materials for Mandan, as they had just received a contract from the tribe to do so. That trip resulted in my Mandan textbook \citep{kasak2014}, the first introductory Mandan textbook since \citeapos{hollow1976} nearly four decades earlier.

Through the Language Conservancy, I returned to North Dakota in the winter of 2015 and stayed for almost three months during that same summer. In each of my visits, I found it increasingly difficult to arrange time with the last L1 Mandan speaker due to competition for his time. A local organization had begun its own work to document and attempt to revitalize Mandan, and they had already made arrangements to meet with the last speaker regularly. Due to his age and health, he was unable to meet as much as I would have preferred, so I sought out other Mandan speakers on Fort Berthold.

Though the last L1 speaker has passed, Mandan is not totally forgotten. There is a small number of heritage learners who had spoken Mandan with a parent or older relative and still remembered it. More remarkably, one of these individuals grew up speaking Mandan with a father who spoke the Nuptare (i.e., \textit{Rúptaare}) dialect, which \citet[3]{mixco1997a} cites as having died out well before the beginning of the twentieth century. \citet[1]{hollow1970} goes so far as to state that there are no data recorded on this variety, though he does give several words that consultants inform him belong to the Ruptare dialect.

Given the scarcity of speakers, the spread-out geography of the reservation, the speakers' busy schedules, and the hyperinflated cost of lodging and travel due to the oil and fracking boom happening in the Bakken region, it has been a challenge to arrange meetings to elicit new data. As such, the bulk of my analysis of Mandan has stemmed from materials collected by previous scholars. Throughout this book, though, I refer to ``contemporary speakers'' of Mandan. I use this term to include those speakers who were recorded during the 1960s and 70s, along with those recordings made with Mr. Edwin Benson after 2000. This term is intended to reflect that Mandan is not a language that has disappeared, and serves to acknowledge that the speakers who worked with me and with other researchers and how these speakers who passed away are still helping learners with their data today.

\subsection{Sources of data}

The vast majority of extant Mandan materials is derived from the fieldwork of Robert \citet{hollow1973a,hollow1973b} during the 1960s and 1970s. There are no known audio recordings from his re-elicitation of \citeapos{kennard1936} narratives, but 20 hours of recordings from his novel elicitation sessions are preserved at the North Dakota State Historical Society. In the summer of 2014, I had these reel-to-reel recordings digitized through the North Dakota State Historical Society. These recordings consist of data from three individuals: Mrs. Mattie Grinnell (1867--1975),\footnote{Mrs. Grinnell's Mandan name is \textit{Náakuhųs} `Many Roads.' Mrs. Grinnell is also noteworthy for being the last individual to receive a Civil War widow's pension, which she was granted in 1971, sixty-seven years after the passing of her first husband John Nagel, who served under the Third Regiment of the Missouri Volunteer Cavalry from 1861 to 1864 \citep{lovett1975}.} Mrs. Annie Eagle (1889--1975), and Mrs. Otter Sage (1912-1994). Given that Mrs. Grinnell was born in Like-a-Fishhook Village before the reservation period, her Mandan is especially valuable to examine, due to it being the language of daily life for several decades before settlers started to outnumber the indigenous inhabitants of the area around the Fort Berthold Indian Reservation. She was also described around the reservation as the last full-blooded Mandan at the time.

Mrs. Eagle (née Crows Heart) was a daughter of Crows Heart. She and Mrs. Otter Sage (née Holding Eagle) were also instrumental to helping \citeauthor{bowers1971} translate the materials he archived with the American Philosophical Society. Given the fact that \citet{hollow1973a,hollow1973b} transcribed his data, while \citet{bowers1971} did not, the \citeauthor{hollow1973a} materials are much more readily accessible for study. At some point, the \citeauthor{bowers1971} materials will need to be transcribed and published, but that is a task for a later date. The data present in those recordings were not included here, as both consultants produce both Mandan and Hidatsa, and a Hidatsa speaker will be needed in order to interpret side conversations and asides between the two. Both sets of narratives collected by \citeauthor{hollow1973a} total 546 pages worth of transcribed and translated Mandan and contain minimal Hidatsa data or code-switching, and as such, these narratives were selected to form the initial corpus of Mandan used within this book.

The other major source of data transcriptions for Mandan is Sara \citeapos{trechter2012} work with the Circle Eagle Project, yielding 10 hours of Mandan and 273 pages of transcriptions and translations from her work with Mr. Edwin \citet{benson2000}. \citeauthor{trechter2012} distinguishes herself from \citeauthor{hollow1970} in marking vowel length, a major phonemic features that is not present in \citeauthor{hollow1970}'s transcriptions.\footnote{\citet{hollow1970,hollow1973a,hollow1973b} and \citet{hollow1976} do mark coda glottals in transcriptions, but this marking of coda glottals is inconsistent.} 

My own personal fieldwork is also part of the data present here, comprised of two hours of recordings done with Mr. \citeauthor{benson2000}, as well as elicitations done with heritage speakers: Mrs. Delores Sand, Mr. Valerian Three Irons, and Mr. Leon Page Little Owl.\footnote{Mr. Little Owl's father, Mr. Ronald Samuel Little Owl (1941--2003), was formerly the Mandan language instructor at the tribal college. Their family speaks the \textit{Rúptaa} variety of Mandan.}

Combining all the sources above, this book makes use of 35 hours of Mandan recordings and 819 pages of transcribed Mandan narratives. A dictionary is being compiled that currently had over 500 entries, with the intention of including the lexical and morphological items present in the aforementioned sources of data.\footnote{A work-in-progress version of this dictionary was formerly available at the Mandan language website: \href{http://www.mandanlanguage.org/dictionary/}{http://www.mandanlanguage.org/dictionary/}, though the domain is not active as of this writing due to the Mandan-Hidatsa-Arikara Nation not renewing its contract with the Language Conservancy.} Future work is needed to examine other Mandan recordings that are preserved at the Philosophical Society, as well as those held at Nueta Hidatsa Sahnish College in New Town, ND.


\section{Organization of this book}\label{organization}

The over-arching aim of this book includes one major goal: the description of the phonology, morphology, syntax, and narrative structure of Mandan. In addition to the discussion of the synchronic grammar of Mandan, I make references to reconstructions of Proto-Siouan forms to connect a process in Mandan to Proto-Siouan or to other Siouan languages. Unless otherwise stated, all Proto-Siouan reconstructions come from \citet{rankin2015}.

I begin my description of Mandan with Chapter \ref{ChSketch}, where I provide a sketch of the language. This chapter provides a grammatical overview of Mandan and aims to reduce the level of complexity a reader may initially encounter when going through this book. Rather than waiting several hundred pages to find out about some major facet of the grammar of Mandan, readers will be able to get a brief overview of salient grammatical features before continuing on to subsequent chapters that explore these grammatical features in much greater detail.

In Chapter \ref{chapter2}, I identify the salient sounds present in Mandan with corroborating phonetic evidence from existing recordings. Furthermore, I discuss allophony and phonotactics, as well as word-level phonological processes, such as nasal harmony and different varieties of epenthesis at work in Mandan. Lastly, I give an account of primary stress assignment in Mandan, something that has not been described previously due to the wide variability in stress marking by previous scholars.

In Chapter \ref{chapter3}, I describe the inflectional and derivational morphology that is present on Mandan verbs. In particular, I emphasize how person and number are marked, and also identify morphology for aspect, mood, and evidentiality. The distribution of preverbs in Mandan is an important section within this description of Mandan verbs, as it is with the preverbs that we can most clearly see the separate domains of affixation due to the structure of verbs featuring preverbs. All such verbs are composites. I describe the different ways negation is marked on the verb, as well as in serial verb situations.

Nominal morphology is the subject of Chapter \ref{chapter4}. This chapter describes the distribution of inflectional and derivational affixes in Mandan. The overall structure of a noun phrase is detailed in this chapter, along with the kinds of determiners, demonstratives, and deictic markers that can appear within the noun complex. I explain the system whereby all nouns are lexically coded for either alienable or inalienable possession in Mandan. In addition to the description of nominal morphology, this chapter also gives an overview of how noun phrases interact with postpositions and quantifiers.

Chapter \ref{chapter5} goes into detail regarding the basic structure of a Mandan clause. Like all Siouan languages, Mandan has a default SOV clause structure, but it also features a system of canonical switch-reference. This switch-reference system involves different complementizer enclitics to denote that a clause either has the same subject as the following clause or if the subject is different. Mandan is an aggressively pro-drop language, so not only may subjects be omitted in discourse, but direct objects, indirect objects, and oblique objects as well. The entire noun phrase complement of a postpositional phrase may be elided as well, so switch-reference is a frequent component of a clause that involves more than one state or act. Mandan \textit{wh}-words remain \textit{in situ} within the structure and do not move to the left edge of the clause. I describe some complementizer-level enclitics that cannot co-occur with agreement enclitics, suggesting that such constructions are not finite.

The final chapter, Chapter \ref{chapter6}, provides an overview of some of the pragmatic items that occur in Mandan speech, such as interjections, filler words, and sentence connectors. The most salient component of this chapter is an interlinear gloss of a narrative by L1 speaker Mrs. Otter Sage, where discourse markers that are omitted by \citet{hollow1973a} are present to assist learners, linguists, and other interested parties to understand speech patterns of Mandan speakers in greater detail.
