\chapter{Narrative structure}\label{chapter6}

This chapter is devoted to the study of the discourse structure of Mandan. Items such as subjects, direct objects, and other semantic roles that would otherwise be clarified by the overt presence of an element in the syntax must often be inferred through context within the narrative structure of an exchange. The bulk of the Mandan corpus is derived from extemporaneous speech in the form of traditional and personal narratives, which can obscure the immediate syntactic structure of any given sentence. Each sentence, therefore, requires the context of the preceding narrative to interpret it.

The main goal of this chapter is to provide a Mandan narrative with interlinear glosses to aid in their analysis by learners, linguists, and other interested parties. Before going through this narrative, I provide an overview of discourse markers in Mandan. These discourse markers range from interjections to items used to indicate a connection or separation from previously described situations. After this overview of discourse markers, I provide an interlinear gloss of a Mandan narrative. This narrative is one that appears in \citet{hollow1973a} with a basic two-line interlinear glossing. Here, I provide elaborated four-line glossing that provides a morphological breakdown and shows the underlying lexical and grammatical formatives present in the narrative. \citet{coberly1979}, \citet{carter1991b}, and \citet{mixco1997a} all provide more nuanced glosses of the \citet{hollow1973a,hollow1973b} narratives, and I continue that tradition here.

\section{Discourse markers}\label{Ch6DiscourseMarkers}

The syntactic composition of phrases and clauses appears in Chapter \ref{chapter5}. One aspect of Mandan grammar that has heretofore been mentioned only in passing is the treatment of linguistic material that does not fit neatly into a particular syntactic position within the structure of an utterance is that of discourse markers. These discourse markers include interjections, filler words, and sentential connectors.

\subsection{Interjections}\label{Ch6Interjections}

\citet[32]{kennard1936} provides the first known account of interjections in Mandan. This list is short, though we see that the system of interjections in Mandan is likely more complex than we can glean from this information alone. Namely, we see that interjections are largely gendered in Mandan, where certain interjections are used only by men, while others are used only by women. These interjections are represented in the table below, with interjections associated with a man are denoted with the {\Male} symbol, interjections associated with a woman are denoted with the {\Female} symbol, and interjections that do not depend on the gender of a person are denoted with the {\Hermaphrodite} symbol.

\begin{table}
\centering\caption{Mandan interjections}\label{TabInterjections}
\begin{tabular}{lll}
\lsptoprule
    \textit{íshkahąą}&{\Female}&`now I remember'\\
    \textit{kahą́ąroo o's}&{\Female}&`now I remember\\
    \textit{shokíshka}&{\Male}&`now I remember'\\
    \textit{anáa}&{\Female}&expression of disgust\\
    \textit{shók}&{\Male}&expression of disgust\\
    \textit{pxók}&{\Hermaphrodite}&`how silly!'\\
    \textit{hįh\'{ı̨}hee}&{\Hermaphrodite}&exclamation of surprise or delight\\
    \textit{katáshka}&{\Hermaphrodite}&`what of it!'\\
    \lspbottomrule
\end{tabular}
\end{table}

These interjections rarely appear in the corpus, as they only appear in cited speech. Furthermore, the corpus is assembled from one-on-one style consultation sessions where a single Mandan speaker is providing narration while at least one other individual is writing down what is being said. There is an absence of interpersonal data in the discourse that is not just a direct or indirect quotation of an individual in a narrative. Furthermore, there is another interjection in the corpus that has not been listed by \citet{kennard1936} or \citet{hollow1970}; \textit{xéepa} is often translated as `wait!' or `hold on!' in \citet{hollow1973a,hollow1973b}, and it does not change for number of addressees, so it is clearly an immutable form that is just used as an attention-grabbing interjection. Likewise, \textit{hée} `hey!' is used as an attention-grabber, though it is not clear if this term is a loan from English or a native Mandan term that just happens to be phonetically similar.

On top of these interjections, we also frequently see interjections for `yes' and `no' throughout the corpus. There are at least two ways to provide an affirmative response in Mandan, with \textit{hų́ų} being most frequent and prototypical. There are several alternatives to \textit{hų́ų}, including \textit{hóo}, \textit{hą́ą}, and \textit{háu}. The interjection \textit{hóo} seems related to the general Plain exclamation \textit{ahóo}, which carries various greeting or affirming semantics in various languages of the Plains. Likewise, \textit{háu} is a general greeting in Mandan and is a borrowing that is shared with other Plains languages for greeting. The \textit{hą́ą} is noteworthy in that it is a greeting in Lakota and Dakota, but only when used by a woman to greet another woman. It is unclear if there is any pragmatic reason for using one version of `yes' over the other.

A refusal or negation in Mandan makes use of the verb \textit{mík} `be none', along with the allocutive agreement marker that matches the gender of the addressee or addressees: \textit{mikó'sh} `no' for speaking to a man, a group of men, or mixed company versus \textit{mikó're} `no' for speaking to a woman or group of women. Unlike `yes', there are no attested alternatives for `no' in Mandan. One Mandan consultant offered up a third option for a response to a yes--no question: \textit{áahahii} (Little Owl, p.c.). This interjection is uttered when the speaker does not wish to respond `yes' or `no' to a proposition, but simply wishes to acknowledge that they have heard it and are contemplating it. This interjection has not been attested in the speech of other Mandan speakers, however, and the consultant in question comes from a family who historically spoke the {Rúptaa} variety of Mandan. Whether this is a dialectal item or a more generalized one is unclear.

An interjection of greeting in Mandan is \textit{háu}, which has already been stated to be a loanword that is common to languages of the Plains. Likewise, \textit{ahóo} or \textit{hahóo} can be used to greet someone, though they are most usually used as an exclamation of gratitude. Both the latter terms are likewise words that are common to Plains languages.

\subsection{Filler words}\label{Ch6Filler}

All languages have sets of words that have no lexical meaning of their own but serve to cue the listener in on the fact that the speaker is reflecting on what they are saying or buying time to think about what is about to be said. Mandan likewise has a set of filler words that are common in recordings. \citet{trechter2012b} most faithfully records these filler words in her transcriptions of the narratives provided by Mr. Edwin Benson and Mrs. Otter Sage, while filler words are absent almost entirely from \citet{kennard1934} and \citet{hollow1973a,hollow1973b}. A list of filler words appears in \tabref{Ch6FillerWordsList}. Note that there is no apparent gender distinction between filler words in Mandan, unlike interjections. While all these filler words serve similar purposes within the discourse, they are glossed with different English translations to help approximate any distinctions that exist between them.

\begin{table}
\caption{Mandan filler words}\label{Ch6FillerWordsList}
\begin{tabular}{ll}
\lsptoprule
\textit{aa}&`oh' (thinking)\\
\textit{ee}&`eh' (displeased or sad)\\
\textit{he}&`uh'\\
\textit{hį}&`um' (continuing same idea)\\
\textit{ų'}&`sort of' (unsure or hedging)\\
\textit{ų'sh}&`so'\\
\textit{waa}&`well' (gearing up to say something)\\
\lspbottomrule
\end{tabular}
\end{table}

Throughout the narratives that \citet{trechter2012b} transcribes, almost all of these filler words have been translated with the English word `and' multiple times. The most straightforward explanation for why these filler words are sometimes translated as conjunctions is to express their function as an attempt to avoid the interruption of the utterance or turn at speaking while the speaker thinks about what to say next. These filler words are not true coordinators, as Mandan seems to lack dedicated coordinators in this sense, as previously discussed in \sectref{Ch5SubsecCoordination}.

\subsection{Narrative connectors}\label{Ch6NarrativeConnectors}

Mandan has several strategies for demonstrating that the grammatical subject of one clause is the same or different from that of a subsequent clause. These strategies have been discussed previously in \sectref{Ch5switchreference} and \sectref{Ch5SerialVerb}. However, Mandan relies on different strategies for connecting one sentence back to one that has previously been uttered. \citet[57]{mixco1997a} identifies two items, \textit{ha}- and \textit{ka}-, which he labels as a pro-verb and pro-sentence, respectively. I treat both these items as pro-verbs, which act to take the place of a whole clause or utterance.

Both \textit{ha}- and \textit{ka}- are historically derived from Proto-Siouan demonstratives. Proto-Siouan *ha is a determiner that has more proximal semantics: e.g., Hidatsa \textit{\textbf{ha}wá} `and then, so', Crow \textit{\textbf{ha}wá} ‘some, one,’ \textit{\textbf{ha}saa'haatiiriye} and `he began to step out \textbf{here}.' \citet{rankin2015} likewise analyze the -\textit{ha} in the autonym \textit{Dhegi\textbf{ha}} as being cognate, where the terms means `one who dwells \textbf{here}.' Proto-Siouan *ka has more distal semantics: e.g., Crow \textit{\textbf{ka}n} `then, already', Hidatsa \textit{\textbf{ga}}- `be there', Lakota \textit{\textbf{ká}} `that (far away)', \textit{\textbf{ga}'a} `that', Osage, \textit{\textbf{ga}} `that (out of sight)', Catawba \textit{\textbf{ka}t} `now.'

In terms of how \textit{ha}- and \textit{ka}- are used in Mandan, there is a slight distributional difference. The pro-verb \textit{ka}- is far more common within the corpus, almost always occurring with the same-subject switch-reference marker =\textit{ni}. \textit{Káni} is a general utterance linker that appears sentence-initially to connect the content of the previous sentence to what is happening in the sentence where \textit{káni} appears. While this term bears the same-subject switch-reference marker, its use does not require that the last subject of the previous sentence be the first subject within the sentence with \textit{káni}. 

Mandan uses =\textit{ni} and =\textit{ak} on adjunct clauses to indicate canonical switch-reference, but \textit{káni} is a marker of non-canonical switch-reference in the sense that the reference this term is tracking is the overall topic in question. Using \textit{káni} signifies that the topic from the previous sentence is carried over to the present sentence. We can see examples of this discourse item in (\ref{Ch6KaniEx}) below.

\begin{exe}
    \item\label{Ch6KaniEx} Examples of \textit{káni}
        \begin{xlist}
        \item\label{Ch6KaniExA}
        \glll Xópini ítiihįįks ~ ~ ~ ~ ~ ~ ~ ~ ~ ~ ~ ~ kihkų́'roomako'sh, numá'ks. \textbf{Káni} óo ~ ~ ~ ~ ~ ~ ó'harani numá'k ínupkereseena ``Hiré ~ ~ ~ ~ nu'ó'na  ą́'skanuhere'sh,'' éehekereroomako'sh.\\
        xop=rį i-tV-i-hįį=k=s ~ ~ ~ ~ ~ ~ ~ ~ ~ ~ ~ ~ ki-k-kų'=oowąk=o'sh ruwą'k=s ka=rį oo ~ ~ ~ ~ ~ ~ o'\#hrE=rį ruwą'k i-rųp=krE=s=ee=rą hire ~ ~ ~ ~ rų-o'=rą ą's=ka\#rų-hrE=o'sh ee-hE=krE=oowąk=o'sh\\
        \textnormal{smoke.up}=ss pv.poss-al-pv.ins-\textnormal{drink}=hab=def ~ ~ ~ ~ ~ ~ ~ ~ ~ ~ ~ ~ vert-suus-\textnormal{give}=narr=ind.m \textnormal{man}=def prov=ss dem.mid ~ ~ ~ ~ ~ ~ \textnormal{be}\#caus=ss \textnormal{man} pv.coll-\textnormal{two}=3pl=def=dem.dist=top \textnormal{now} ~ ~ ~ ~ 1a.pl-\textnormal{be}=top \textnormal{this.way}\#1a.pl-caus=ind.m pv-\textnormal{say}=3pl=narr=ind.m\\
        \glt `After smoking it up, he gave his pipe back to him, to the man. \textbf{And} from there, to the man the two of them said, ``Now, we are the ones who did it that way.{''}' \citep[175]{hollow1973b}

        \item\label{Ch6KaniExB} \glll Kamíxere ré ą́'ska, Kamíxe áakihąą ~ ~ ~ ~ ~ ~ mákoomako'sh. \textbf{Káni} ée ó'rak ~ ~ ~ ~ ~ ~ ~ ~ ~ ~ ~ ~ kahúukereroomako'sh.\\
        kawįx=re re ą's=ka kawįx=E aaki=hąą ~ ~ ~ ~ ~ ~ wąk=oowąk=o'sh ka=rį ee o'=ak ~ ~ ~ ~ ~ ~ ~ ~ ~ ~ ~ ~ kahuu=krE=oowąk=o'sh\\
        \textnormal{circle}=dem.prox dem.prox \textnormal{near}=hab \textnormal{circle}=sv \textnormal{above}=sim ~ ~ ~ ~ ~ ~ pos.lie=narr=ind.m prov=ss dem.dist \textnormal{be}=ds ~ ~ ~ ~ ~ ~ ~ ~ ~ ~ ~ ~ \textnormal{go.after}=3pl=narr=ind.m\\
        \glt `Circle was lying on top of it in her form as a circle. \textbf{And} that was what they were after.' \citep[107]{trechter2012b}
        \end{xlist}
\end{exe}

While \textit{ka}- most often appears with =\textit{ni}, we do see variants in the corpus with differing enclitic markings. \textit{Károotiki} is composed of the pro-verb \textit{ka}- plus the indirect evidential enclitic =\textit{oote} and the conditional complementizer =\textit{ki}.

Like \textit{káni}, this item indicates a shared topic is being carried over to this new sentence. However, the presence of the evidential and the conditional complementizer alters the semantics of \textit{ka}-. \textit{Károotiki} is used in situations where the events of the sentence where it occurs must necessarily happen as a consequence of the events of the previous sentence. This word is always translated as `and then' throughout the corpus. We can see examples of \textit{károotiki} in the data in (\ref{Ch6ExamplesKarootiki}) below.

\begin{exe}
    \item\label{Ch6ExamplesKarootiki} Examples of \textit{károotiki}

    \begin{xlist}
        \item\label{Ch6ExamplesKarootikiA} \glll Minísweerut xí'hseena kikų́'teroomako'sh, ~ ~ inák súks. \textbf{Károotiki}, ``éepeso'sh. ~ ~ ~ ~ ~ ~ MiníKEE, riréesike manakų́'ki, ~ ~ ~ ~ ~ ~ ~ ~ ~ éepeso'sh.''\\
        wrįs\#wee\#rut xi'h=s=ee=rą kikų'tE=oowąk=o'sh ~ ~ irąk suk=s ka=ooti=ki ee-pE=s=o'sh ~ ~ ~ ~ ~ ~ wį-rįk=E=$\varnothing$ ri-reesik=E w-rą-kų'=ki ~ ~ ~ ~ ~ ~ ~ ~ ~ ee-pE=s=o'sh\\
        \textnormal{horse}\#\textnormal{feces}\#\textnormal{eat} \textnormal{be.old}=def=dem.dist=top \textnormal{help}=narr=ind.m ~ ~ \textnormal{again} \textnormal{child}=def prov=evid=cond pv-\textnormal{say}.1a=def=ind.m ~ ~ ~ ~ ~ ~ 1poss-\textnormal{son}=sv=voc 2poss-\textnormal{tongue}=sv 1s-2a-\textnormal{give}=cond ~ ~ ~ ~ ~ ~ ~ ~ ~ pv-\textnormal{say}.1a=def=ind.m\\
        \glt `The old dog helped him, the child once again. \textbf{And then}, he was like, ``I said it. O my son, I said that you should give me your tongue.{''}' \citep[190]{hollow1973a}

        \item\label{Ch6ExamplesKarootikiB} \glll ``Réeharata ósuuharani réna ~ ~ ~ ~ kų́'ta,'' éeheeroomako'sh. \textbf{Károotiki} ~ ~ ~ ~ ~ ~ ~ ~ ~ ~ numá'kseena manásh ósuuharani ~ ~ ~ ~ ~ ~ kų́'roomako'sh.\\
        rEEh\#hrE=ta o-suu\#hrE=rį re=rą ~ ~ ~ ~ kų'=ta ee-hee-oowąk=o'sh ka=ooti=ki ~ ~ ~ ~ ~ ~ ~ ~ ~ ~ ruwą'k=s=ee=rą wrąsh o-suu\#hrE=rį ~ ~ ~ ~ ~ ~ kų'=oowąk=o'sh\\
        \textnormal{go.there}\#caus=imp.m pv.loc-\textnormal{be.filled}\#caus=ss dem.prox=top ~ ~ ~ ~ \textnormal{give}=imp.m pv-\textnormal{say}=narr=ind.m prov=evid=cond ~ ~ ~ ~ ~ ~ ~ ~ ~ ~ \textnormal{man}=def=dem.dist=top \textnormal{tobacco} pv.loc-\textnormal{be.filled}\#caus=ss ~ ~ ~ ~ ~ ~ \textnormal{give}=narr=ind.m\\
        \glt `He said, ``Go on, fill it and give it to this one!'' \textbf{And then}, the man filled it with tobacco and gave it to him.' \citep[175]{hollow1973b}
    \end{xlist}
\end{exe}

The pro-verb \textit{ka}- can also be used in a disjunctive sense. When the same topic is carried over from the previous sentence, but the speaker wishes to contrast some aspect of this topic, then the pro-verb takes the disjunctive =\textit{shka} or =\textit{shka'nik}. \textit{Káshka} or \textit{káshka'nik} are always translated as `but' in the corpus, but they behave more like an adverbial than a true conjunction. There does not appear to be an obvious semantic or pragmatic difference between these two items. We can see examples of these disjunctive sentence connectors in (\ref{Ch6ExKashka}) below.

\begin{exe}
    \item\label{Ch6ExKashka} Examples of \textit{káshka} and \textit{káshka'nik}

    \begin{xlist}
        \item\label{Ch6ExKashkaA} \glll Kiná'ki, ókaakuhą't; ée ~ ~ ~ ~ ~ kowóorooreena ~ ~ ~ ~ ~ ~ ~ ~ ~ ~ ~ ~ ~ ~ ~ ~ ~ ~ ~ ókaakuhą't. \textbf{Káshka}, míihe ~ ~ ~ róoxkaso'nik, wáa'oksahanash ótaa ~ ríiseką't.\\
        kirą'=ki o-k-aa-kuh=ą't ee ~ ~ ~ ~ ~ ko-wooroo=ee=rą ~ ~ ~ ~ ~ ~ ~ ~ ~ ~ ~ ~ ~ ~ ~ ~ ~ ~ ~ o-k-aa-kuh=ą't ka=shka wįįh=E ~ ~ ~ rV-o-xka=so'rįk waa-o-ksah=rąsh o=taa ~ rV-i-sek=ą't\\
        \textnormal{tell}=cond pv.irr-vert-pv.tr-\textnormal{come.here}.vert=hyp dem.dist ~ ~ ~ ~ ~ 3poss.pers-\textnormal{husband}=dem.dist=top ~ ~ ~ ~ ~ ~ ~ ~ ~ ~ ~ ~ ~ ~ ~ ~ ~ ~ ~ pv.irr-vert-pv.tr-\textnormal{come.here}.vert=hyp prov=disj \textnormal{woman}=sv ~ ~ ~ 1a.pl-pv.irr-\textnormal{be.wild}=comp.caus nom-pv.irr-\textnormal{way}=att pv.loc=loc ~ 1a.pl-pv.ins-\textnormal{make}=hyp\\
        \glt `If she talked about it, he would bring her back; that there husband of hers would bring her back. \textbf{But}, since women are untamable, we would do it anyway with sneaky ways.' \citep[80]{hollow1973a}

        \item\label{Ch6ExKashkaB} \glll Wáatishi'sh. \textbf{Káshka} wáarakina'nixo'sh.\\
        waa-ti=ishi=o'sh ka=shka waa-ra-kirą'=rįx=o'sh\\
        \textnormal{some}-\textnormal{arrive.here}=vis=ind.m prov=disj neg-2a-\textnormal{tell}=neg=ind.m\\
        \glt `Someone must have been here. \textbf{But}, you're not telling.' \citep[162]{hollow1973a}

        \item\label{Ch6ExKashkaC} \glll Wáashi waherés wáaxte íwasekak... ~ ~ ~ ~ ~ Káshka'nik, mi'óshka íwaseką't; ptamáah tóps wakírushekere'sh.\\
        waa-shi wa-hrE=s waa-xtE i-wa-sek=ak ~ ~ ~ ~ ~ ka=shka'rįk wį-oshka i-wa-sek=ą't p-ta-wąąh top=s wa-ki-ru-shE=krE=o'sh\\
        nom-\textnormal{be.good} 1a-caus=def nom-\textnormal{be.big} pv.ins-1a-\textnormal{make}=ds ~ ~ ~ ~ ~ prov=disj 1s-emph pv.ins-1a-\textnormal{make}=hyp 1poss-al-\textnormal{arrow} \textnormal{four}=def 1a-vert-ins.hand-\textnormal{grasp}=3pl=ind.m\\
        \glt `I really did my medicine badly... \textbf{But}, I myself did it; [when] I took my four arrows back.' \citep[35]{hollow1973a}

        \item\label{Ch6ExKashkaD} \glll Róopxata. \textbf{Káshka'nik}, raróopxeki, ínihį ~ ~ ~ ~ ~ káare kishíharata.\\
        roopxE=ta ka=shka'rįk ra-roopxE=ki i-rį-hį ~ ~ ~ ~ ~ kaare ki-shih\#hrE=ta\\
        \textnormal{enter}=imp.m prov=disj 2a-\textnormal{enter}=cond pv.poss-2poss-\textnormal{hair} ~ ~ ~ ~ ~ imp.neg mid-\textnormal{be.sharp}\#caus=imp.m\\
        \glt `Go in. \textbf{But}, when you go in, don't let your hair become sharp.' \citep[4]{hollow1973b}
    \end{xlist}
\end{exe}

The other pro-verb \textit{ha}- is almost always followed by the conditional enclitic =\textit{ki} throughout the corpus. The purpose of this discourse marker is to indicate that the topic of the sentence beginning with \textit{háki} is different from the topic of the previous sentence. In this way, \textit{ha}- serves as a non-canonical switch-reference marker, signaling to the listener that there is a shift in aboutness from one sentence to the next. In the corpus, this item is regularly translated into English as `so'. We can see examples of \textit{háki} in the data in (\ref{Ch6ExamplesHaki}) below.

\begin{exe}
    \item\label{Ch6ExamplesHaki} Examples of \textit{háki}

    \begin{xlist}
        \item\label{Ch6ExamplesHakiA} \glll Hį he, Níhkasiire, rá'ts ~ ~ ~ ~ ~ ~ ~ ~ wáa'owakiniire ą́ąwe maséero'sh. \textbf{Háki}, ~ ~ ~  pt\'{ı̨}įre weréhe rusékereki, pti\'{ı̨}įre ~ ~ ~ ~ ~ ta'ą́ą súhkereki, hą́k \'{ı̨}'kapąt  ~ ~ ~ numá'kaakina rúta ómaakakara't, ~ ~ ~ ~ ~ rá'ts maséero'sh.\\
        hį he rįk=ka\#sii=E r'-at=s ~ ~ ~ ~ ~ ~ ~ ~ waa-o-wa-ki-rįį=E ąąwe wą-see=o'sh ha=ki ~ ~ ~ ptįį=E wreh=E ru-sE=krE=ki ptįį=E ~ ~ ~ ~ ~ ta'ąą suk=krE=ki hąk į'-ka-pąt ~ ~ ~ ruwą'k-aaki=rą rut=E=$\varnothing$ o-wąąkE=krE=ą't ~ ~ ~ ~ ~ r'-at=s wą-see=o'sh\\
        \textnormal{uh} \textnormal{um} \textnormal{offspring}=hab\#\textnormal{be.yellow}=sv 2poss-\textnormal{father}=def ~ ~ ~ ~ ~ ~ ~ ~ nom-pv.irr-1a-rflx-\textnormal{run}=sv \textnormal{all} 1s-\textnormal{defeat}=ind.m prov=cond ~ ~ ~  \textnormal{buffalo}=sv \textnormal{door}=sv ins.hand-\textnormal{grasp}=3pl=cond \textnormal{buffalo}=sv ~ ~ ~ ~ ~ \textnormal{how.many} \textnormal{exit}=3pl=cond pos.stnd pv.rflx-ins.frce-\textnormal{increase} ~ ~ ~ \textnormal{person}-coll=top \textnormal{eat}=sv=cont pv.irr-\textnormal{lie}.pos.aux=3pl=hyp ~ ~ ~ ~ ~ 2poss-\textnormal{father}=def 1s-\textnormal{defeat}=ind.m\\
        \glt `Um, uh, Young Calf, your father beat me every time that I raced him. \textbf{So}, if they opened the cows' doorway, if however many of those cows got out, then there would be an increase of people there eating, [which is how] your father beat me.' \citep[124]{hollow1973a}

        \item\label{Ch6ExamplesHakiB} \glll ``Ptaníishkeres ą́'te,'' éeheerak, ``\textbf{Háki}, ~ ~ ~ ~ ~ ~ nitániishkerek, ké'kaani numá'k ratóokta'shka ~ ~ ~ ~ ~ éewerehini, ké’kaahere, kikų́'minihere'sh.''\\
        p-ta-rįįshkrE=s ą't=E ee-hee=ak ha=ki ~ ~ ~ ~ ~ ~ rį-ta-rįįshkrE=ak ke'kaa=rį ruwą'k ratoo=kt=a'shka ~ ~ ~ ~ ~ ee-we-reh=rį ke'kaa\#hrE ki-kų'\#w-rį-hrE=o'sh\\
        1poss-al-\textnormal{medicine}=def dem.anap=sv pv-\textnormal{say}=ds prov=cond ~ ~ ~ ~ ~ ~ 2poss-al-\textnormal{medicine}=ds \textnormal{keep}=ss \textnormal{man} \textnormal{be.mature}=pot=able ~ ~ ~ ~ ~ pv-1a-\textnormal{think}=ss \textnormal{keep}\#caus suus-\textnormal{give}\#1a-2s-caus=ind.m\\
        \glt `{``}That is my medicine,'' he said, ``\textbf{so}, if it is your medicine and he keeps it, then I think that he is sure to grow to to be a man, causing him to keep it, so I told you to give it to him.{''}' \citep[59]{hollow1973a}
        
    \end{xlist}
\end{exe}

The pro-verb \textit{ha}- can also take the potential modal =\textit{kt} plus the different-subject switch-reference marker =\textit{ak}. It functions as a counterpart to \textit{károotiki} in that it is not merely a jumping off point into a new sentence, but it indicates that there is some aspect of the present sentence that is caused by or reliant on the topic of the previous sentence with the added caveat that the topic is pivoting to a new one. It is usually just translated as `so' or `and so' in the corpus. We can see examples of \textit{háktek} in the data in (\ref{Ch6ExHaktek}) below.

\begin{exe}
    \item\label{Ch6ExHaktek} Examples of \textit{háktek}

    \begin{xlist}
        \item\label{Ch6ExHaktekA} \glll Károotiki, súkeena minísishataa ~ ~ ~ náhka'eeheero'sh. \textbf{Háktek}, minísweeruts pt\'{ı̨}į ~ ~ \'{ı̨}xini réehak xawáakereka'ehe.\\
        ka=ooti=ki suk=ee=rą wrįs\#i-shE=taa ~ ~ ~ rąk=ka'eehee=o'sh ha=kte=ak wrįs\#wee\#rut=s ptįį ~ ~ įx=rį rEEh=ak xwaa=krE=ka'ehe\\
        prov=evid=cond \textnormal{child}=dem.dist=top \textnormal{horse}\#pv.ins-\textnormal{hold}=loc ~ ~ ~ pos.sit=quot=ind.m prov=pot=ds \textnormal{horse}\#\textnormal{feces}\#\textnormal{eat}=def \textnormal{buffalo} ~ ~ \textnormal{bark}=ss \textnormal{go.there}=ds \textnormal{be.lost}=3pl=quot\\
        \glt `And then, a child was sitting on a travois, it is said. \textbf{So}, the dog was barking at a cow and left, so they got lost, it is said.' \citep[278]{hollow1973b}

        \item\label{Ch6ExHaktekB} \glll Numá'k Máxana éeheni Kinúma'kshi ~ ~ ~ ~ ~ ~ ~ ~ ~ ~ ~ ~ ~  íkirookereroomako'sh, kotewé ~ ~ ~ ~ ~ ~ ~ ~ ~ ~ ~ óratoore. \textbf{Háktek}, ``Ą́'taaka't róo ~ ~ ~ ~ ~ wakxų́hki, ó'iraheką't,'' éeheni ~ ~ ~ ~ ~ ~ ~ ~ Kinúma'kshi kxų́hoomako'sh.\\
        ruwą'k wąxrą ee-he=rį ki-ruwą'k\#shi ~ ~ ~ ~ ~ ~ ~ ~ ~ ~ ~ ~ ~ i-ki-roo=krE=oowąk=o'sh ko-t-we ~ ~ ~ ~ ~ ~ ~ ~ ~ ~ ~ o-ratoo=E ha=kte=ak ą'taaka=ą't roo ~ ~ ~ ~ ~ wa-xųh=ki o-i-ra-hek=ą't ee-he=rį ~ ~ ~ ~ ~ ~ ~ ~ ki-ruwą'k\#shi kxųh=oowąk=o'sh\\
        \textnormal{man} \textnormal{one} pv-\textnormal{say}=ss mid-\textnormal{man}\#\textnormal{be.good} ~ ~ ~ ~ ~ ~ ~ ~ ~ ~ ~ ~ ~ pv.ins-rflx-\textnormal{talk}=3pl=narr=ind.m rel-wh-indf ~ ~ ~ ~ ~ ~ ~ ~ ~ ~ ~ pv.irr-\textnormal{be.mature}=sv prov=pot=ds \textnormal{be.alright}=hyp dem.mid ~ ~ ~ ~ ~ 1a-\textnormal{lie.down}=cond pv.irr-pv.ins-2a-\textnormal{know}=hyp pv-\textnormal{say}=ss ~ ~ ~ ~ ~ ~ ~ ~ mid-\textnormal{man}\#\textnormal{be.good} \textnormal{lie.down}=narr=ind.m\\
        \glt `First Creator and Lone man argued about it, [about] which one was older. \textbf{So}, he said ``All right, if I lay down here, you would know it', and First Creator layed down.' \citep[1]{hollow1973a}
    \end{xlist}
\end{exe}

All of the narrative connectors that have been mentioned above are purely optional from a syntactic point of view. Unlike canonical switch-reference markers, these non-canonical switch-reference markers are pragmatic in nature and do not form integral parts of the morpho-syntax of the Mandan language. Furthermore, they do not occupy a position within the syntax as rigidly as the switch-reference enclitics do, as they can be postposed at the right edge of a clause, as we see in (\ref{Ch6KaskaEnd}) below.

\begin{exe}
    \item\label{Ch6KaskaEnd} \glll Hókeena ``Kók Kí're'' éehees ~ ~ ~ ~ ~ ~ ~ ~ íwarooni éewereh shí're. Íwahekanashe ą́ąwe wáashiwaharaxi'sh, \textbf{káshka}.\\
    hok=ee=rą kok ki'=E ee-hee=s ~ ~ ~ ~ ~ ~ ~ ~ i-wa-roo=rį ee-we-reh shi=o're i-wa-hek=rąsh=E ąąwe waa-shi\#wa-hrE=xi=o'sh ka=shka\\
    \textnormal{story}=dem.dist=top \textnormal{antelope} \textnormal{pack.on.back}=sv pv-\textnormal{say}=def ~ ~ ~ ~ ~ ~ ~ ~ pv.ins-1a-\textnormal{speak}=ss pv-1a-\textnormal{want} \textnormal{be.good}=ind.f pv.ins-1a-\textnormal{know}=att=sv \textnormal{all} neg-\textnormal{be.good}\#1a-caus=neg=ind.m prov=disj\\
    \glt `The story called ``Packs Antelope'' that I want to tell is good. I won't do a good job of telling all of what I know about it, \textbf{however}.' \citep[210]{hollow1973a}
\end{exe}

These words function as pragmatic connectors between one sentence and another, but they behave like adverbials in that they appear sentence-initially most often due to being topicalized. The fact that \textit{káshka} above occurs sentence-finally is evidence that these items are adjuncts that can appear in different parts of the sentence according to how the speaker wishes to highlight or downplay the connection between the sentence bearing the narrative connector and the preceding sentence.


\section{Narrative: ``Eye Juggler''}\label{Ch6EyeJuggler}

This narrative is told Mrs. Otter Sage as recorded by Dr. Robert Hollow in Twin Buttes on July 13, 1967. Dr. Hollow was conducting fieldwork that would lead to his dissertation \citep{hollow1970}, while Mrs. Sage was one of the three principal Mandan-speaking consultants who agreed to record and translate a series of narratives. 

Mrs. Sage was a frequent source of information about the Mandan language within her own community, and she also worked with several outside scholars beyond \citet{hollow1970,hollow1973a,hollow1973b}, such as Dr. Alfred \citet{bowers1971}, who recorded over 150 hours alongside Mrs. Anna Eagle that deal with the traditional narratives and other knowledge presented in \citeapos{bowers1950} book. Mrs. Sage was also a Mandan language teacher at the school in Twin Buttes. After her passing, that position came to her relative, Mr. Edwin Benson.

The narrative below is a traditional one. ``Eye Juggler'' is about Royal Chief (\textit{Kinúma'kshi}) traveling in his role as the Coyote, who engages in different cultural \textit{faux pas} to instruct listeners on what not to do. This version of ``Eye Juggler'' involves Royal Chief encountering a group of children who have medicine that allows them to remove their eyes, throw them somewhere else, and then call them back again. Royal Chief asks for this medicine and buys it from the children for the price of four arrows. 

The boys teach Royal Chief how to do what they do, and he quickly leaves as soon as he masters the medicine, to the chagrin of the children.\footnote{The term ``medicine'' here refers to the cultural practice of treating spiritual health and physiological health as tied to one another. Medicine is a spiritual practice that can be associated with a particular ritual or ceremony, an item like a sacred bundle, or the ability to perform spectacular deeds. I acknowledge that my understanding of traditional Mandan spiritual and cultural practices does not compare to those who live these practices, so this description of medicine reflects my own understanding of the term. \citet{bowers1950} attempts to describe aspects of traditional Mandan social and spiritual practices, but his book is likewise written through the lens of someone from outside the Mandan community.} Royal Chief engages in various bouts of mischief to scare small animals until he comes upon a thicket of diamond willow trees. He throws his eyes into the trees, calls them back, and is very satisfied with himself. He then decides that this medicine is his and he should ask for his arrows back. The children sadly return his arrows, which ruins Royal Chief's ability to perform his medicine correctly, and results in a series of misfortunes. After receiving help navigating blindly around from the trees in the area, Royal Chief makes it to the river, where the water restores his eyes and he sets off traveling again.

The overall lesson here is to respect one's medicine and to follow the rules. Performing rites incorrectly or taking them for granted can incur negative consequences. The telling below is the Mandan version of the ``Eye Juggler'' narrative, a narrative that is shared by many other peoples of the Northern and Southern Plains (\citealt{lowie1909}, \citealt{thompson1966}, \textit{inter alios}).\footnote{I make no attempt to draw any conclusions about the origin of this narrative. The thematic elements present in this narrative differ from those in the version of ``Eye Juggler'' told by other peoples \citep{thompson1966}, but any comparison of those narrative and thematic features is beyond the scope of this grammar.}



\subsection{Glossed text}\label{CH6InterlinearGloss}

The interlinear glosses here appear in the same format as the majority of glosses throughout this work. The number of each gloss represents a full sentence or utterance from this narrative. For the sake of keeping count of the total number of utterances within this narrative, the numbering of examples has been reset.



\begin{exe}
\setcounter{xnumi}{0}

\item\label{EJ1} \glll Kinúma'kshi kasímika'ehesįh.\\
ki-ruwą'k\#shi ka-si=awį=ka'ehe=sįh\\
mid-\textnormal{man}\#\textnormal{be.good} incp-\textnormal{travel}=cont=quot=ints\\
\glt `Royal Chief was traveling along, it is certainly said.'

\item\label{EJ2} \glll Máatah íwokahąą kasími wáa'eroomako'sh, wáasherok.\\
wąątah i-woka=hąą ka-si=awį waa-E=oowąk=o'sh waa-shro=ak\\
\textnormal{river} pv.dir-\textnormal{edge}=ins incp-\textnormal{travel}=cont \textnormal{something}-\textnormal{hear}=narr=ind.m nom-\textnormal{shout}=ds\\
\glt `As he was traveling along the river edge, he heard something, and it was a shout.'

\item\label{EJ3} \glll Nakóxe kirúpsheroomako'sh.\\
rąkox=E ki-ru-pshe=oowąk=o'sh\\
\textnormal{ear}=sv mid-ins.hand-\textnormal{prick}=narr=ind.m\\
\glt `His ears pricked up.'

\item\label{EJ4} \glll Káni, wáa'es í'ų́'taa réehoomako'sh.\\
ka=rį waa-E=s i-ų'=taa rEEh=oowąk=o'sh\\
prov=ss nom-\textnormal{hear}=def pv.dir-\textnormal{be.close}=loc \textnormal{go.there}=narr=ind.m\\
\glt `And he went toward what he heard.'

\item\label{EJ5} \glll Kinúma'kshi kasúhki, súk hų́na miníxa ~ ~ ~ ~ ~ máakaho'sh.\\
ki-ruwą'k\#shi ka-suk=ki suk hų=rą wrįx=E ~ ~ ~ ~ ~ wąąkah=o'sh\\
mid-\textnormal{man}\#\textnormal{be.good} ins.frce-\textnormal{exit}=cond \textnormal{child} \textnormal{man}=top \textnormal{play}=sv ~ ~ ~ ~ ~ \textnormal{lie}.pos.aux.hab=ind.m\\
\glt `When Royal Chief peeked out, a lot of children were playing there.'

\item\label{EJ6} \glll Háktek, súhkereseena Kirúma'kshi ~ ~ ~ ~ ~ hékarani óshiriiha ptéhkereroomako'sh.\\
ha=kte=ak suk=krE=s=ee=rą ki-ruwą'k\#shi ~ ~ ~ ~ ~ hE=krE=rį o-shriih=E=$\varnothing$ ptEh=krE=oowąk=o'sh\\
prov=pot=ds \textnormal{child}=3pl=def=dem.dist=top mid-\textnormal{man}\#\textnormal{be.good} ~ ~ ~ ~ ~ \textnormal{see}=3pl=ss pv.loc-\textnormal{be.scattered}=sv=cont \textnormal{run}=3pl=narr=ind.m\\
\glt `So, the children saw Royal Chief and then scattered as they ran away.'

\item\label{EJ7} \glll Háktek, Kinúma'kshiseena: ``SúkiniTEE! ~ ~ ~ Káare ptáhinista! Kúhinista! ~ ~ ~ ~ ~ ~ ~ ~ ~ ~ ~  Wáa'ą'skaharaxi'sh, ~ ~ ~ ~ ~ ~ ~ ~ ~ ~ ~ ~ ~ ~ ~ ~ ~ ~ ~ kotáwaratoore húuki.''\\
ha=kte=ak ki-ruwą'k\#shi=s=ee=rą suk=rįt=E ~ ~ ~ kaare ptEh=rįt=ta kuh=rįt=ta ~ ~ ~ ~ ~ ~ ~ ~ ~ ~ ~ waa-ą's=ka\#hrE=xi=o'sh ~ ~ ~ ~ ~ ~ ~ ~ ~ ~ ~ ~ ~ ~ ~ ~ ~ ~ ~ ko-ta-wa-ratoo=E huu=ki\\
prov=pot=ds mid-\textnormal{man}\#\textnormal{be.good}=def=dem.dist=top \textnormal{child}=2pl=sv ~ ~ ~ imp.neg \textnormal{run}=2pl=imp.m \textnormal{come.here}.vert=2pl=imp.m ~ ~ ~ ~ ~ ~ ~ ~ ~ ~ ~ neg-\textnormal{be.this.way}=hab\#caus=neg=ind.m ~ ~ ~ ~ ~ ~ ~ ~ ~ ~ ~ ~ ~ ~ ~ ~ ~ ~ ~ 3poss.pers-al-unsp-\textnormal{be.mature}=sv \textnormal{come.here}=cond\\
\glt `So, Royal Chief was like, ``O children! Don't run! Come back! It is rude to act this way, especially when one's elder is coming.{''}'

\item\label{EJ8} \glll Súhkeres ą́ąwe kúhkereroomako'sh, ~ ~ ~ ~ ~ ~ ~ ~ ~ ~ Kinúma'kshi ų́'t.\\
suk=krE=s ąąwe kuh=krE=oowąk=o'sh ~ ~ ~ ~ ~ ~ ~ ~ ~ ~ ki-ruwą'k\#shi ų't\\
\textnormal{child}=3pl=def \textnormal{all} \textnormal{come.here}.vert=3pl=narr=ind.m ~ ~ ~ ~ ~ ~ ~ ~ ~ ~ mid-\textnormal{man}\#\textnormal{be.good} \textnormal{towards}\\
\glt `All the children came back to him, to Royal Chief.'

\item\label{EJ9} \glll ``Sháa nixíhkas, nuptéhą't.''\\
shaa rį-xik=ka=s rų-ptEh=ą't\\
\textnormal{always} 2s-\textnormal{be.bad}=hab=def 1a.pl-\textnormal{run}=hyp\\
\glt `{``}You are always bad, so we figured that we would run away,{''} [said the children].'

\item\label{EJ10} \glll ``Ínikxąhinisto'sh, numá'kaaki ~ ~ ~ ~ ~ ~ ~ ~ ~ ~ ~ ~ ~ ~ ~ ~ ~ ~ ~ ~ ínihekinitki.''\\
i-rį-kxąh=rįt=t=o'sh ruwą'k-aaki ~ ~ ~ ~ ~ ~ ~ ~ ~ ~ ~ ~ ~ ~ ~ ~ ~ ~ ~ ~ i-rį-hek=rįt=ki\\
pv.ins-2s-\textnormal{laugh}=2pl=pot=ind.m \textnormal{person}-coll ~ ~ ~ ~ ~ ~ ~ ~ ~ ~ ~ ~ ~ ~ ~ ~ ~ ~ ~ ~ pv.ins-2s-\textnormal{know}=2pl=cond\\
\glt `{``}They will laugh at you, if people knew about you,'' [said Royal Chief].'

\item\label{EJ11} \glll {``}`Kotáwaratoore kikaráahkerek', ~ ~ ~ ~ ~ ~ ~ ~ ~ ~ éenihenisto'sh.''\\
ko-ta-wa-ratoo=E ki-kraah=krE=ak ~ ~ ~ ~ ~ ~ ~ ~ ~ ~ ee-rį-he=rįt=t=o'sh\\
3poss.pers-al-unsp-\textnormal{be.mature}=sv suus-\textnormal{be.afraid}=3pl=ds ~ ~ ~ ~ ~ ~ ~ ~ ~ ~ pv-2s-\textnormal{say}=2pl=pot=ind.m\\
\glt `{``}They are afraid of their own uncle,'' they might say about you,{''} [said Royal Chief].'

\item\label{EJ12} \glll ``Súkinite, matwé írasekinito'sha?''\\
suk=rįt=E wa-t-we i-ra-sek=rįt=o'sha\\
\textnormal{child}=2pl=sv unsp-wh-indf pv.ins-2a-\textnormal{make}=2pl=int.m\\
\glt `[Royal Chief then asked them,] ``Children, what are you doing?{''}'

\item\label{EJ13} \glll ``Nustámi nukirúsaani ~ ~ ~ ~ ~ manátaa róokasanito'sh.''\\
rų-ista\#wį rų-k-ru-saa=rį ~ ~ ~ ~ ~ wrą=taa rV-o-ka-saa=rįt=o'sh\\
1pl.poss-\textnormal{face}\#\textnormal{orb} 1a.pl-suus-ins.hand-\textnormal{remove.meat.from.bone}=ss ~ ~ ~ ~ ~ \textnormal{tree}=loc 1a.pl-pv.loc-ins.frce-\textnormal{hang}=2pl=ind.m\\
\glt `{``}We are taking out our eyes and hanging them on a tree,'' [the children replied].'


\item\label{EJ14} \glll ``Ptúunite, wahé íwateero'sh.''\\
p-tuu=rįt=E wa-hE i-wa-tee=o'sh\\
1poss-\textnormal{sister's.child}=2pl=sv 1a-\textnormal{see} pv.ins-1a-\textnormal{die}=ind.m\\
\glt `{``}My nephews, I would like to see it,'' [Royal Chief said].'

\item\label{EJ15} \glll Háktek, súhkeres maná skikíka ~ ~ ~ ~ ~ ~ ~ ~ ~ ~ kaxteka íkisąąpakereroomako'sh.\\
ha=kte=ak suk=krE=s wrą skiski=ka ~ ~ ~ ~ ~ ~ ~ ~ ~ ~ ka-xtE=ka i-ki-sąąpa=krE=oowąk=o'sh\\
prov=pot=ds \textnormal{child}=3pl=def \textnormal{tree} \textnormal{diamond.willow}=hab ~ ~ ~ ~ ~ ~ ~ ~ ~ ~ agt-\textnormal{be.big}=hab pv.dir-mid-\textnormal{be.around}=3pl=narr=ind.m\\
\glt `So, the children went around a bunch of diamond willow trees.'

\item\label{EJ16} \glll Káni, súhkeres istámi kirúshani ~ ~ ~ ~ ~ ~ ~ ~ ~ ~ ímanaseetaa íkų'tekereroomako'sh.\\
ka=rį suk=krE=s ista\#wį ki-ru-shE=rį ~ ~ ~ ~ ~ ~ ~ ~ ~ ~ i-wrą=s=ee=taa i-kų'tE=krE=oowąk=o'sh\\
prov=ss \textnormal{child}=3pl=def \textnormal{face}\#\textnormal{orb} suus-ins.hand-\textnormal{grasp}=ss ~ ~ ~ ~ ~ ~ ~ ~ ~ ~ pv.dir-\textnormal{tree}=def=dem.dist=loc pv.dir-\textnormal{throw}=3pl=narr=ind.m\\
\glt `And then, the children took out their eyes and threw them towards the tree.'

\item\label{EJ17} \glll Máa'istami íra'shąąshi ~ ~ ~ ~ ~ ~ ~ ~ ~ ~ ~ ~ ~ ~ ~ núunihoomako'sh.\\
waa-ista\#wį i-ra'-shąąshi=$\varnothing$ ~ ~ ~ ~ ~ ~ ~ ~ ~ ~ ~ ~ ~ ~ ~ ruurįh=oowąk=o'sh\\
nom-\textnormal{face}\#\textnormal{orb} pv.ins-ins.heat-\textnormal{be.glistening}=cont ~ ~ ~ ~ ~ ~ ~ ~ ~ ~ ~ ~ ~ ~ ~ \textnormal{be}.pl.aux.dur=narr=ind.m\\
\glt `Their eyes were there, glistening.'

\item\label{EJ18} \glll Háktek, Kinúma'kshi: ``Súkinite, ~ ~ ~ ~ ~ ~ ~ ~ ~ ~ ~ ~ ~ ~ ~ ~ ~ ~ ~ ~ wáa'okipkaxanashxte'sh.''\\
ha=kte=ak ki-ruwą'k\#shi suk=rįt=E ~ ~ ~ ~ ~ ~ ~ ~ ~ ~ ~ ~ ~ ~ ~ ~ ~ ~ ~ ~ waa-o-ki-k-pax=rąsh=xtE=o'sh\\
prov=pot=ds mid-\textnormal{man}\#\textnormal{be.good} \textnormal{child}=2pl=sv ~ ~ ~ ~ ~ ~ ~ ~ ~ ~ ~ ~ ~ ~ ~ ~ ~ ~ ~ ~ nom-pv.irr-iter-mid-\textnormal{be.broken}=att=aug=ind.m\\
\glt `So, Royal Chief [said,] ``Children, that is really pretty.{''}'

\item\label{EJ19} \glll Háktek, súhkeres istámi kirúherekereroomako'sh: ``mí'stami kúha skóte; mí'stami ~ ~ ~ kúha skóte!''\\
ha=kte=ak suk=krE=s ista\#wį k-ru\#hrE=krE=oowąk=o'sh w'\~~-ista\#wį kuh=E=$\varnothing$ skot=E w'\~~-ista\#wį ~ ~ ~ kuh=E=$\varnothing$ skot=E\\
prov=pot=ds \textnormal{child}=3pl=def \textnormal{face}\#\textnormal{orb} vert-\textnormal{call}\#caus=3pl=narr=ind.m 1poss-\textnormal{face}\#\textnormal{orb} \textnormal{come.here}.vert=sv=cont \textnormal{go.plop}=sv 1poss-\textnormal{face}\#\textnormal{orb} ~ ~ ~ \textnormal{come.here}.vert=sv=cont \textnormal{go.plop}=sv\\
\glt `So, the children called their eyes back, [saying] ``My eyes go plop coming back; my eyes go plop coming back!{''}'

\item\label{EJ20} \glll Háktek, ratóore tashíxteroomako'sh.\\
ha=kte=ak ratoo=E ta-shi-xtE=oowąk=o'sh\\
prov=pot=ds \textnormal{be.mature}=sv al-\textnormal{be.good}-aug=narr=ind.m\\
\glt `So, the elder really liked it.'

\item\label{EJ21} \glll ``Ómiha makų́'nista, ~ ~ ~ ~ ~ ~ ~ ~ ~ ~ ~ ~ ~ ~ ~ ~ ~ ~ ~ ~ ~ ~ ~ ~ ~ wáa'irasekinitą't.''\\
o-wįh=E wą-kų'=rįt=ta ~ ~ ~ ~ ~ ~ ~ ~ ~ ~ ~ ~ ~ ~ ~ ~ ~ ~ ~ ~ ~ ~ ~ ~ ~ waa-i-ra-sek=rįt=ą't\\
pv.loc-\textnormal{point}=sv 1s-\textnormal{give}=2pl=imp.m ~ ~ ~ ~ ~ ~ ~ ~ ~ ~ ~ ~ ~ ~ ~ ~ ~ ~ ~ ~ ~ ~ ~ ~ ~ nom-pv.ins-2a-\textnormal{make}=2pl=dem.anap\\
\glt `[Royal Chief said,] ``Teach it to me, that thing that you did.{''}'

\item\label{EJ22} \glll Háktek, súhkereseena ``Háu,'' ~ ~ ~ ~ ~ ~ ~ ~ ~ ~ ~ ~ ~ ~ ~ éeheekereroomao'sh.\\
ha=kte=ak suk=krE=s=ee=rą hau ~ ~ ~ ~ ~ ~ ~ ~ ~ ~ ~ ~ ~ ~ ~  ee-hee=krE=oowąk=o'sh\\
prov=pot=ds \textnormal{child}=3pl=def=dem.dist=top \textnormal{yes} ~ ~ ~ ~ ~ ~ ~ ~ ~ ~ ~ ~ ~ ~ ~ pv-\textnormal{say}=3pl=narr=ind.m\\
\glt `So, the children said ``yes.{''}'

\item\label{EJ23} \glll ``Matewé róorakų'ro'sha?''\\
wa-t-we rV-o-ra-kų'=o'sha\\
unsp-wh-indf 1s.pl-pv.irr-2a-\textnormal{give}=int.m\\
\glt `{``}What will you give us for it?'' [asked the childen].'

\item\label{EJ24} \glll Kinúma'kshi: ``Ptamáah tóops minikų́'nisto'sh.''\\
ki-ruwą'k\#shi p-ta-wąą toop=s w-rį-kų'=rįt=t=o'sh\\
mid-\textnormal{man}\#\textnormal{be.good} 1poss-al-\textnormal{arrow} \textnormal{four}=def 1a-2s-\textnormal{give}=2pl=pot=ind.m\\
\glt `Royal Chief was like, ``I will give you my four arrows.{''}'

\item\label{EJ25} \glll Háktek, súhkeres nátka ~ ~ ~ ~ ~ ~ ~ ~ ~ ~ ~ ~ ~ ~ ~ shíkereroomako'sh.\\
ha=kte=ak suk=krE=s rąt=ka ~ ~ ~ ~ ~ ~ ~ ~ ~ ~ ~ ~ ~ ~ ~ shi=krE=oowąk=o'sh\\
prov=pot=ds \textnormal{child}=3pl=def \textnormal{be.middle.of}=hab ~ ~ ~ ~ ~ ~ ~ ~ ~ ~ ~ ~ ~ ~ ~ \textnormal{be.good}=3pl=narr=ind.m\\
\glt `So, the children were happy.'

\item\label{EJ26} \glll Kinúma'kshina: ``Ptúunite, ómiha ~ ~ ~ ~ makų́'nista.''\\
ki-ruwą'k\#shi=rą p-tuu=rįt=E o-wįh=E ~ ~ ~ ~ wą-kų'=rįt=ta\\
mid-\textnormal{man}\#\textnormal{be.good}=top 1poss-\textnormal{sister's.child}=2pl=sv pv.loc-\textnormal{show}=sv ~ ~ ~ ~ 1s-\textnormal{give}=2pl=imp.m\\
\glt `Royal Chief was like, ``Nephews, show me how to do it.{''}'

\item\label{EJ27} \glll Háktek, súhkereseena Kinúma'kshi ų́'t: ``Ríisehki, ó'kashka ~ ~ ~ ~ ~ ~ ~ ~ ~ ~ ~ ~ ~ ~ ~ ~ ~ ~ ~ ~ írasekto'sh.''\\
ha=kte=ak suk=krE=s=ee=rą ki-ruwą'k\#shi ų't rV-i-sek=ki o'=ka=a'shka ~ ~ ~ ~ ~ ~ ~ ~ ~ ~ ~ ~ ~ ~ ~ ~ ~ ~ ~ ~ i-ra-sek=t=o'sh\\
prov=pot=ds \textnormal{child}=3pl=def=dem.dist=top mid-\textnormal{man}\#\textnormal{be.good} \textnormal{be.near} 1a.pl-pv.ins-\textnormal{make}=cond \textnormal{be}=hab=able ~ ~ ~ ~ ~ ~ ~ ~ ~ ~ ~ ~ ~ ~ ~ ~ ~ ~ ~ ~ pv.ins-2a-\textnormal{make}=pot=ind.m\\
\glt `So, the children said to Royal Chief, ``If we do it, you should do it the exact same way.{''}'

\item\label{EJ28} \glll ``Nustámi nukirúsheki, ní'shak ~ ~ ~ ~ ~ ~ ~ ~ ~ ~ írasekto'sh.{''}\\
rų-ista\#wį rų-k-ru-shE=ki r'\~~-ishak ~ ~ ~ ~ ~ ~ ~ ~ ~ ~ i-ra-sek=t=o'sh\\
1pl.poss-\textnormal{face}\#\textnormal{orb} 1a.pl-suus-ins.hand-\textnormal{grasp}=cond 2s-pro ~ ~ ~ ~ ~ ~ ~ ~ ~ ~ pv.ins-2a-\textnormal{make}=pot=ind.m\\
\glt `[The children explained,] ``When we take our eyes, you should do it, too.{''}'

\item\label{EJ29} \glll Háktek, súhkeres istámi ą́ąwe ~ ~ ~ ~ ~ ~ ~ ~ ~ ~ ~ ~ ~ ~ ~ ~ ~ ~ ~ ~ kirushékereroomako'sh.\\
ha=kte=ak suk=krE=s ista\#wį ąąwe ~ ~ ~ ~ ~ ~ ~ ~ ~ ~ ~ ~ ~ ~ ~ ~ ~ ~ ~ ~ k-ru-shE=krE=oowąk=o'sh\\
prov=pot=ds \textnormal{child}=3pl=def  \textnormal{face}\#\textnormal{orb} \textnormal{all} ~ ~ ~ ~ ~ ~ ~ ~ ~ ~ ~ ~ ~ ~ ~ ~ ~ ~ ~ ~ suus-ins.hand-\textnormal{grasp}=3pl=narr=ind.m\\
\glt `And so, the children grabbed all of their eyeballs.'

\item\label{EJ30} \glll Káni, ímanataa wíikų'tekereroomako'sh.\\
ka=rį i-wrą=taa wV-i-kų'tE=krE=oowąk=o'sh\\
prov=ss pv.dir-\textnormal{tree}=loc unsp-pv.dir-\textnormal{throw}=3pl=narr=ind.m\\
\glt `And then, they threw them toward the tree.'

\item\label{EJ31} \glll Háktek, máa'istami ókasaara ~ ~ ~ ~ ~ ~ ~ ~ ~ ~ mákoomako'sh, manátaa.\\
ha=kte=ak waa-ista\#wį o-ka-saa=E ~ ~ ~ ~ ~ ~ ~ ~ ~ ~ wąk=oowąk=o'sh wrą=taa\\
prov=pot=ds nom-\textnormal{face}\#\textnormal{orb} pv.loc-ins.frce=\textnormal{hang}=sv ~ ~ ~ ~ ~ ~ ~ ~ ~ ~ pos.lie=narr=ind.m \textnormal{tree}=loc\\
\glt `And so, their eyes were just hanging there all over, on the tree.'

\item\label{EJ32} \glll Háktek, ų́'ka, súhkereseena: ``Ratóore, ní'shak inák éetekto'sh, wáaroo'eeheere: ~ ~ ~ ~ ~ `Mí'stami kúha skóote; mí'stami ~ ~ ~ kúha skóte!'{''}\\
ha=kte=ak ų'=ka suk=krE=s=ee=rą ratoo=E r'\~~-ishak irąk ee-tE=kt=o'sh waa-rV-o-ee-hee=E ~ ~ ~ ~ ~ w'\~~-ista\#wį kuh=E=$\varnothing$ skot=E w'\~~-ista\#wį ~ ~ ~ kuh=E=$\varnothing$ skot=E\\
prov=pot=ds \textnormal{be.near}=hab \textnormal{child}=3pl=def=dem.dist=top \textnormal{be.mature}=sv 2s-pro \textnormal{also} pv-\textnormal{say}.2a=pot=ind.m nom-1a.pl-pv.irr-pv-\textnormal{say}=sv ~ ~ ~ ~ ~ 1poss-\textnormal{face}\#\textnormal{orb} \textnormal{come.here}.vert=sv=cont \textnormal{go.plop}=sv 1poss-\textnormal{face}\#\textnormal{orb} ~ ~ ~ \textnormal{come.here}.vert=sv=cont \textnormal{go.plop}=sv\\
\glt `So, then, the children were like, ``Elder, you should also say it, what we are going to say: `My eyes go plop coming back; my eyes go plop coming back!'{''}'

\item\label{EJ33} \glll Háktek, istámikeres ą́ąwe karóopxekereroomako'sh.\\
ha=kte=ak ista\#wį=krE=s ąąwe ka-roopxE=krE=oowąk=o'sh\\
prov=pot=ds \textnormal{face}\#\textnormal{orb}=3pl=def \textnormal{all} ins.frce-\textnormal{enter}=3pl=narr=ind.m\\
\glt `So, all their eyes went back in.'

\item\label{EJ34} \glll Háktek, Kinúma'kshi tashíxteroomako'sh.\\
ha=kte=ak ki-ruwą'k\#shi ta-shi-xtE=oowąk=o'sh\\
prov=pot=ds mid-\textnormal{man}\#\textnormal{be.good} al-\textnormal{be.good}-aug=narr=ind.m\\
\glt `So, Royal Chief really liked that.'

\item\label{EJ35} \glll ``Wáa'okipkaxanashxte'sh.''\\
waa-o-ki-k-pax=rąsh=xtE=o'sh\\
nom-pv.irr-iter-mid-\textnormal{be.broken}=att=aug=ind.m\\
\glt `{``}It is really pretty,'' [said Royal Chief].'

\item\label{EJ36} \glll ``Máama'sho, íwawaaxani mí'stami manátaa ~ ~ ~ ~ ~ ~ ~ ~ ~ ówakasaani, wakirúhereki, mí'stami ~ ~ ~ ~ ~ ~ ~ karóopxekto'sh.''\\
waa-wą'sho i-wa-waaxE=rį w'\~~-ista\#wį wrą=taa ~ ~ ~ ~ ~ ~ ~ ~ ~ o-wa-ka-saa=rį wa-kru\#hrE=ki w'\~~-ista\#wį ~ ~ ~ ~ ~ ~ ~ ka-roopxE=kt=o'sh\\
nom-\textnormal{be.infrequent} pv.ins-1a-\textnormal{stop}=ss 1poss-\textnormal{face}\#\textnormal{orb} \textnormal{tree}=loc ~ ~ ~ ~ ~ ~ ~ ~ ~ pv.loc-1a-ins.frce-\textnormal{hang}=ss 1a-\textnormal{call}\#caus=cond 1poss-\textnormal{face}\#\textnormal{orb} ~ ~ ~ ~ ~ ~ ~ ins.frce-\textnormal{enter}=pot=ind.m\\
\glt `{``}Sometimes, I will stop and hang my eyes on a tree, so when I call them, my eyes will go back in,{''} [said Royal Chief].'

\item\label{EJ37} \glll Háktek, súhkereseet tamáah tóops ~ ~ ~ ~ ~ ~ ~ ká'hereroomako'sh.\\
ha=kte=ak suk=krE=s=ee=t ta-wąąh toop=s ~ ~ ~ ~ ~ ~ ~ ka'\#hrE=oowąk=o'sh\\
prov=pot=ds \textnormal{child}=3pl=def=dem.dist=loc al-\textnormal{arrow} \textnormal{four}=def ~ ~ ~ ~ ~ ~ ~ \textnormal{possess}\#caus=narr=ind.m\\
\glt `So, Royal Chief gave his four arrows to the children.'

\item\label{EJ38} \glll Háktek, ``Ptúunite, wakási ~ ~ ~ ~ ~ ~ ~ ~ waréehto'sh.''\\
ha=kte=ak p-tuu=rįt=E wa-ka-si=$\varnothing$ ~ ~ ~ ~ ~ ~ ~ ~ wa-rEEh=t=o'sh\\
prov=pot=ds 1poss-\textnormal{sister's.child}=2pl=sv 1a-inch-\textnormal{travel}=cont ~ ~ ~ ~ ~ ~ ~ ~ 1a-\textnormal{go.there}=pot=ind.m\\
\glt `So, [Royal Chief said], ``My nephews, I am going to go head out traveling.{''}'

\item\label{EJ39} \glll Súhkereseena ruką́hkereroomako'sh.\\
suk=krE=s=ee=rą rukąh=krE=oowąk=o'sh\\
\textnormal{child}=3pl=def=dem.dist=top \textnormal{forbid}=3pl=narr=ind.m\\
\glt `The children said not to.'

\item\label{EJ40} \glll ``Ratóore, wáashinuharanitak, ų́'sh ~ ~ ~ ~ ~ ~ ~ ~ ~ ~ rorárusanąhini raráahini éerereho'sh.''\\
ratoo=E waa-shi\#rų-hrE=rįt=ak ų'sh ~ ~ ~ ~ ~ ~ ~ ~ ~ ~ ro-ra-ru-srąh=rį ra-rEEh=rį ee-re-reh=o'sh\\
\textnormal{be.mature}=sv nom-\textnormal{be.good}\#1a.pl-caus=2pl=ds \textnormal{be.thus} ~ ~ ~ ~ ~ ~ ~ ~ ~ ~ 1s.pl-2a-ins.hand-\textnormal{abandon}=ss 2a-\textnormal{go.there}=ss pv-2a-\textnormal{want}=ind.m\\
\glt `{``}Elder, we are having a good time, and you want to go and leave us just like that,{''} [the children said].'

\item\label{EJ41} \glll Kinúma'kshi: ``Ó'sh, téehą  óminitaa ~ ~ ~ ~ ~ ~ ~ ~ ~ ~ ~ ~ ~ ~ ~ mamáakahinito'sh.''\\
ki-ruwą'k\#shi o'sh teehą o-w-rį=taa ~ ~ ~ ~ ~ ~ ~ ~ ~ ~ ~ ~ ~ ~ ~ wa-wąąkah=rįt=o'sh\\
mid-\textnormal{man}\#\textnormal{be.good} \textnormal{gosh} \textnormal{be.far.away} pv.loc-1a-2s=loc ~ ~ ~ ~ ~ ~ ~ ~ ~ ~ ~ ~ ~ ~ ~ 1a-\textnormal{lie}.pos.aux.hab=2pl=ind.m\\
\glt `[Royal Chief replied,] ``Gosh, I have been staying with you for a long time.{''}'

\item\label{EJ42} \glll Súhkereseena rá'taxak, ~ ~ ~ ~ ~ ~ ~ ~ ~ ~ Kinúma'kshi kasí, máatah íwokahąą ~ ~ ~ ~ ~ ~ ~ ~ ~ kasímiroomako'sh.\\
suk=krE=s=ee=rą ra'-tax=ak ~ ~ ~ ~ ~ ~ ~ ~ ~ ~ ki-ruwą'k\#shi ka-si=$\varnothing$ wąątah i-woka=hąą ~ ~ ~ ~ ~ ~ ~ ~ ~ ka-si=awį=oowąk=o'sh\\
\textnormal{child}=3pl=def=dem.dist=top ins.heat-\textnormal{make.loud.noise}=ds ~ ~ ~ ~ ~ ~ ~ ~ ~ ~ mid-\textnormal{man}\#\textnormal{be.good} inch-\textnormal{travel}=cont \textnormal{river} pv.dir-\textnormal{edge}=ins ~ ~ ~ ~ ~ ~ ~ ~ ~ inch-\textnormal{travel}=cont=narr=ind.m\\
\glt `The children cried as Royal Chief set off traveling, and he kept right on traveling along the edge of the river.'

\item\label{EJ43} \glll Íwaaxani wáashiheres íkikiishkani ~ ~ ~ ~ ~ ~ ~ ~ ~ ~ éerehoomako'sh.\\
i-waaxE=rį waa-shi\#hrE=s i-kikiishkE=rį ~ ~ ~ ~ ~ ~ ~ ~ ~ ~ ee-reh=oowąk=o'sh\\
pv.ins-\textnormal{stop}=ss nom-\textnormal{be.good}\#caus=def pv.ins-\textnormal{consider}=ss ~ ~ ~ ~ ~ ~ ~ ~ ~ ~ pv-\textnormal{want}=narr=ind.m\\
\glt `He stopped and he wanted to try out his medicine.'

\item\label{EJ44} \glll Maná skiskika kaxtéka ték, híni ~ istámi kirúshani skiskika ~ ~ ~ ~ ~ ~ ~ ~ ~ ~ kaxtékaseet íkų'teroomako'sh.\\
wrą skiski=ka ka-xtE=ka tE=ak hi=rį ~ ista\#wį k-ru-shE=rį skiski=ka ~ ~ ~ ~ ~ ~ ~ ~ ~ ~ ka-xtE=ka=s=ee=t i-kų'tE=oowąk=o'sh\\
\textnormal{tree} \textnormal{diamond.willow}=hab agt-\textnormal{be.big}=hab \textnormal{stand}=ds \textnormal{arrive.there}=ss ~ \textnormal{face}\#\textnormal{orb} suus-ins.hand-\textnormal{grasp}=ss \textnormal{diamond.willow}=hab ~ ~ ~ ~ ~ ~ ~ ~ ~ ~ agt-\textnormal{be.big}=hab=def=dem.dist=loc pv.dir-\textnormal{throw}=narr=ind.m\\
\glt `There stood a big bunch of diamond willows and he arrived there, so he took his eyes out and threw them towards that bunch of willows.'

\item\label{EJ45} \glll Háktek, Kinúma'kshis: ``Mí'stami ~ ~ ~ ~ ~ ~ ~ ~ ~ ~ ~ ~ ~ ~ ~ kúha skóte; mí'stami ~ ~ ~ ~ ~ ~ ~ ~ ~ ~ ~ ~ ~ ~ ~ kúha skóte!''\\
ha=kte=ak ki-ruwą'k\#shi=s w'\~~-ista\#wį ~ ~ ~ ~ ~  ~ ~ ~ ~ ~ ~ ~ ~ ~ ~ kuh=E=$\varnothing$ skot=E w'\~~-ista\#wį ~ ~ ~ ~ ~ ~ ~ ~ ~ ~ ~ ~ ~ ~ ~ kuh=E=$\varnothing$ skot=E\\
prov=pot=ds mid-\textnormal{man}\#\textnormal{be.good}=def 1poss-\textnormal{face}\#\textnormal{orb} ~ ~ ~ ~ ~ ~ ~ ~ ~ ~ ~ ~ ~ ~ ~ \textnormal{come.here}.vert=sv=cont \textnormal{go.plop}=sv 1poss-\textnormal{face}\#\textnormal{orb} ~ ~ ~ ~ ~ ~ ~ ~ ~ ~ ~ ~ ~ ~ ~ \textnormal{come.here}.vert=sv=cont \textnormal{go.plop}=sv\\
\glt `So, Royal Chief was like, ``My eyes go plop coming back; my eyes go plop coming back!{''}'

\item\label{EJ46} \glll Istámis kúhini Kinúma'kshi istámis ~ ~ ~ ~ ~ sowókini \'{ı̨}'here karóopxeroomako'sh.\\
ista\#wį=s kuh=rį ki-ruwą'k\#shi ista\#wį=def ~ ~ ~ ~ ~ sowok=rį į'-hrE ka-roopxE=oowąk=o'sh\\
\textnormal{face}\#\textnormal{orb}=def \textnormal{come.here}.vert=ss mid-\textnormal{man}\#\textnormal{be.good} \textnormal{face}\#\textnormal{orb}=def ~ ~ ~ ~ ~ \textnormal{go.splash}=ss pv.rflx-caus ins.frce-\textnormal{enter}=narr=ind.m\\
\glt `His eyes came back to him and Royal Chief's eyes went in, kind of making a splash as they did.'

\item\label{EJ47} \glll Háktek, Kinúma'shi nátka ~ ~ ~ ~ ~ ~ ~ ~ ~ ~ ~ ~ ~ ~ ~ shixtéroomako'sh.\\
ha=kte=ak ki-ruwą'k\#shi rąt=ka ~ ~ ~ ~ ~ ~ ~ ~ ~ ~ ~ ~ ~ ~ ~ shi-xtE=oowąk=o'sh\\
prov=pot=ds mid-\textnormal{man}\#\textnormal{be.good} \textnormal{be.middle.of}=hab ~ ~ ~ ~ ~ ~ ~ ~ ~ ~ ~ ~ ~ ~ ~ \textnormal{be.good}-aug=narr=ind.m\\
\glt `So, Royal Chief was really happy.'

\item\label{EJ48} \glll ``Wáa'okipkaxanasho'sh; shíwahere'sh,'' ~ ~ ~ éerehoomako'sh.\\
waa-o-ki-k-pax=rąsh=o'sh shi\#wa-hrE=o'sh ~ ~ ~ ee-reh=oowąk=o'sh\\
nom-pv.irr-iter-mid-\textnormal{be.broken}=att=ind.m \textnormal{be.good}\#1a-caus=ind.m ~ ~ ~ pv-\textnormal{think}=narr=ind.m\\
\glt `{``}It is so pretty, what I know how to do,'' he thought.'

\item\label{EJ49} \glll Háktek, kixéeni kasí réehoomako'sh.\\
ha=kte=ak ki-xee=rį ka-si=$\varnothing$ rEEh=oowąk=o'sh\\
prov=pot=ds mid-\textnormal{slow}=ss inch-\textnormal{travel}=cont \textnormal{go.there}=narr=ind.m\\
\glt `So, he quit doing that and he set out traveling.' 

\item\label{EJ50} \glll Kinúma'kshi máareksuke wapáxirutini, ~ ~ ~ ~ ~ máama'sho máaxtihkshuk skéharani máareksuk ~ ~ ~ ~ ~ ka'ó'į'tiharani, ų́'shkami hą́p ~ ~ ~ ~ ~ ~ ~ ~ ~ ~ náamininashini íwaxaani skiskíka kaxtéka ték, wáashiheres kikíishkeroomako'sh.\\
ki-ruwą'k\#shi wąąreksuk=E wa-pa-xrut=rį ~ ~ ~ ~ ~ waa-wą'sho wąąxtik\#kshuk skE\#hrE=rį wąąreksuk ~ ~ ~ ~ ~ ka-o-į'-ti\#hrE=rį ų'sh=ka=awį hąp ~ ~ ~ ~ ~ ~ ~ ~ ~ ~ raawrį=rąsh=rį i-wa-xaa=rį skiski=ka ka-xtE=ka tE=ak waa-shi\#hrE=s kikiishkE=oowąk=o'sh\\
mid-\textnormal{man}\#\textnormal{be.good} \textnormal{bird}=sv unsp-ins.push-\textnormal{drive.a.herd}=ss ~ ~ ~ ~ ~ nom-\textnormal{be.infrequent} \textnormal{rabbit}\#\textnormal{be.narrow} \textnormal{jump}\#caus=ss \textnormal{bird} ~ ~ ~ ~ ~ agt-pv.irr-pv.rflx-\textnormal{be.afraid}\#caus=ss \textnormal{be.thus}=hab=cont \textnormal{day} ~ ~ ~ ~ ~ ~ ~ ~ ~ ~ \textnormal{three}=att=ss pv.ins-1a-\textnormal{stop}=ss \textnormal{diamond.willow}=hab agt-\textnormal{be.big}=hab \textnormal{stand}=ds nom-\textnormal{be.good}\#caus=def \textnormal{consider}=narr=ind.m\\
\glt `Royal Chief was making the birds scatter, and sometimes he made a cottontail jump, and he was a bird-scarer, continuing to do so for about three days, then he stopped and tried his medicine on a bunch of diamond willows standing there.'

\item\label{EJ51} \glll Ísekoomako'sh.\\
i-sek=oowąk=o'sh\\
pv.ins-\textnormal{make}=narr=ind.m\\
\glt `He did it.'

\largerpage
\item\label{EJ52} \glll Kinúma'kshis istámi manátaa ~ ~ ~ ~ ~ ~ ~ ~ ~ ~ ~ ~ ~ ~ ~ ~ ~ ókasasaara: ``Mí'stami ~ ~ ~ ~ ~ ~ ~ ~ ~ ~ ~ kúha skóte; mí'stami ~ ~ ~ ~ ~ ~ ~ ~ ~ ~ ~ ~ kúha skóte!''\\
ki-ruwą'k\#shi=s ista\#wį wrą=taa ~ ~ ~ ~ ~ ~ ~ ~ ~ ~ ~ ~ ~ ~ ~ ~ ~ o-ka-sa$\sim$saa=E=$\varnothing$ w'\~~-ista\#wį ~ ~ ~ ~ ~ ~ ~ ~ ~ ~ ~ kuh=E=$\varnothing$ skot=E w'\~~-ista\#wį ~ ~ ~ ~ ~ ~ ~ ~ ~ ~ ~ ~ kuh=E=$\varnothing$ skot=E\\
mid-\textnormal{man}\#\textnormal{be.good}=def \textnormal{face}\#\textnormal{orb} \textnormal{tree}=loc ~ ~ ~ ~ ~ ~ ~ ~ ~ ~ ~ ~ ~ ~ ~ ~ ~ pv.loc-ins.frce-dist$\sim$\textnormal{hang}=sv=cont 1poss-\textnormal{face}\#\textnormal{orb} ~ ~ ~ ~ ~ ~ ~ ~ ~ ~ ~ \textnormal{come.here}.vert=sv=cont \textnormal{go.plop}=sv 1poss-\textnormal{face}\#\textnormal{orb} ~ ~ ~ ~ ~ ~ ~ ~ ~ ~ ~ ~ \textnormal{come.here}.vert=sv=cont \textnormal{go.plop}=sv\\
\glt `Royal Chief was hanging his eyes around on a tree, going ``My eyes go plop coming back; my eyes go plop coming back!{''}'

\item\label{EJ53} \glll Háktek, Kinúma'kshi ``Ptawáa'iseks,'' ~ ~ ~ ~ ~ éerehoomako'sh.\\
ha=kte=ak ki-ruwą'k\#shi p-ta-waa-i-sek=s ~ ~ ~ ~ ~ ee-reh=oowąk=o'sh\\
prov=pot=ds mid-\textnormal{man}\#\textnormal{be.good} 1poss-al-nom-pv.ins-\textnormal{make}=def ~ ~ ~ ~ ~ pv-\textnormal{think}=narr=ind.m\\
\glt `So, Royal Chief thought, ``It is my medicine.{''}'

\item\label{EJ54} \glll Kinúma'kshi: ``Súhkereseet wakaráahini ~ ~ ptamáah tóops wakirúshekto'sh.''\\
ki-ruwą'k\#shi suk=krE=s=ee=t wa-k-rEEh=rį ~ ~ p-ta-wąąh toop=s wa-k-ru-shE=kt=o'sh\\
mid-\textnormal{man}\#\textnormal{be.good} \textnormal{child}=3pl=def=dem.dist=loc 1a-vert-\textnormal{go.there}=ss ~ ~ 1poss-al-\textnormal{arrow} \textnormal{four}=def 1a-vert-ins.hand-\textnormal{grasp}=pot=ind.m\\
\glt `Royal Chief [thought,] ``I'll go back to the children and get my four arrows back.{''}'

\item\label{EJ55} \glll Kinúma'kshi kiptáhini ráahami, ~ ~ ~ ~ ~ ~ ~ ~ ~ ~ ~ ~ ~ ~ ~ súhkereseet kí'hoomako'sh.\\
ki-ruwą'k\#shi ki-ptEh=rį rEEh=awį ~ ~ ~ ~ ~ ~ ~ ~ ~ ~ ~ ~ ~ ~ ~ suk=krE=s=ee=t ki'h=oowąk=o'sh\\
mid-\textnormal{man}\#\textnormal{be.good} vert-\textnormal{run}=ss \textnormal{go.there}=cont ~ ~ ~ ~ ~ ~ ~ ~ ~ ~ ~ ~ ~ ~ ~ \textnormal{child}=3pl=def=dem.dist=loc \textnormal{arrive.there}.vert=narr=ind.m\\
\glt `Royal Chief turned back running and got back there to the children.'

\item\label{EJ56} \glll ``Hée, ptúunite! Wáa, 
máa'ų'staa ~ ~ ~ ~ ~ ~ ~ ~ ~ ~ ptawáa'iseko'sh.''\\
hee p-tuu=rįt=E waa waa-ų't=taa ~ ~ ~ ~ ~ ~ ~ ~ ~ ~ p-ta-waa-i-sek=o'sh\\
\textnormal{hey} 1poss-\textnormal{sister's.child}=2pl=sv \textnormal{well} nom-\textnormal{be.far.away}=loc ~ ~ ~ ~ ~ ~ ~ ~ ~ ~ 1poss-al-nom-pv.ins-\textnormal{make}=ind.m\\
\glt `{``}Hey, nephews! Well, I did my medicine a long time ago,{''} [said Royal Chief].

\item\label{EJ57} \glll ``Teweteroo ptamáah tóops ó'ro'sha? ~ ~ ~ ~ ~ ~ ~ ~ ~ ~ Makíhkų'nista!''\\
t-we=t=roo p-ta-wąąh toop=s o'=o'sha ~ ~ ~ ~ ~ ~ ~ ~ ~ ~ wą-ki-k-kų'=rįt=ta\\
wh-indf=loc=dem.mid 1poss-al-\textnormal{arrow} \textnormal{four}=def \textnormal{be}=int.m ~ ~ ~ ~ ~ ~ ~ ~ ~ ~ 1s-vert-suus-\textnormal{give}=2pl=imp.m\\
\glt `[Royal Chief continued,] {``}Where are my four arrows? Give them back to me!{''}'

\largerpage
\item\label{EJ58} \glll Háktek, súhkeres ą́ąwe pá ókashųkanashini nátka xíhkereroomako'sh.\\
ha=kte=ak suk=krE=s ąąwe pa o-ka-shųk=rąsh=rį rąt=ka xik=krE=oowąk=o'sh\\
prov=pot=ds \textnormal{child}=3pl=def \textnormal{all} \textnormal{head} pv.loc-ins.frce-\textnormal{hang}=att=ss \textnormal{be.middle.of}=hab \textnormal{be.bad}=3pl=narr=ind.m\\
\glt `So, all the children were hanging their heads and they were sad.'

\item\label{EJ59} \glll Tamáahs súhkereseena ~ ~ ~ ~ ~ ~ ~ ~ ~ ~ ~ ~ ~ ~ ~ ~ ~ ~ ~ ~ kihkų́'kereroomako'sh.\\
ta-wąąh=s suk=krE=s=ee=rą ~ ~ ~ ~ ~ ~ ~ ~ ~ ~ ~ ~ ~ ~ ~ ~ ~ ~ ~ ~ ki-k-kų'=krE=oowąk=o'sh\\
al-\textnormal{arrow}=def \textnormal{child}=3pl=def=dem.dist=top ~ ~ ~ ~ ~ ~ ~ ~ ~ ~ ~ ~ ~ ~ ~ ~ ~ ~ ~ ~ vert-suus-\textnormal{give}=3pl=narr=ind.m\\
\glt `The children gave him his arrows back.'

\item\label{EJ60} \glll Súhkeres istámini óptik, tamáahs ~ ~ ~ ~ ~ ~ ~ ~ ~ ~ ~ ~ ~ ~ ~ kihkų́'kereroomako'sh.\\
suk=krE=s ista\#wrį o-ptik ta-wąąh=s ~ ~ ~ ~ ~ ~ ~ ~ ~ ~ ~ ~ ~ ~ ~ ki-k-kų'=krE=oowąk=o'sh\\
\textnormal{child}=3pl=def \textnormal{face}\#\textnormal{water} pv.loc-\textnormal{fall} al-\textnormal{arrow}=def ~ ~ ~ ~ ~ ~ ~ ~ ~ ~ ~ ~ ~ ~ ~ vert-suus-\textnormal{give}=3pl=narr=ind.m\\
\glt `The children's tears were falling as they gave his arrows back to him.'

\item\label{EJ61} \glll Háktek, Kinúma'kshi réehoomako'sh.\\
ha=kte=ak ki-ruwą'k\#shi rEEh=oowąk=o'sh\\
prov=pot=ds mid-\textnormal{man}\#\textnormal{be.good} \textnormal{go.there}=narr=ind.m\\
\glt `So, Royal Chief went off.'

\item\label{EJ62} \glll Tamáahs ké'ka'ni Kinúma'kshi ráahini ~ ~ ~ ~ ~ ~ ~ ~ ~ ~ máa'ąkeena ní'ni íkutaa ráahini ~ ~ ~ ~ ~ tamáah tóops patíkini réehoomako'sh.\\
ta-wąąh=s ke'ka'=rį ki-ruwą'k\#shi rEEh=rį ~ ~ ~ ~ ~ ~ ~ ~ ~ ~ waa'ąk=ee=rą rį'=rį i-ku=taa rEEh=rį ~ ~ ~ ~ ~ ta-wąąh toop=s pa-tik=rį rEEh=oowąk=o'sh\\
al-\textnormal{arrow}=def \textnormal{possess}=ss mid-\textnormal{man}\#\textnormal{be.good} \textnormal{go.there}=ss ~ ~ ~ ~ ~ ~ ~ ~ ~ ~ \textnormal{earth}=dem.dist=top \textnormal{climb}=ss pv.dir-\textnormal{be.opposite}=loc \textnormal{go.there}=ss ~ ~ ~ ~ ~ al-\textnormal{arrow} \textnormal{four}=def ins.push-\textnormal{throw}=ss \textnormal{go.there}=narr=ind.m\\
\glt `Having his arrows, Royal Chief went along, then he climbed a hill, then got to the other side and threw his four arrows away as he left.'

\item\label{EJ63} \glll Kinúma'kshi sími ítani ~ ~ ~ ~ ~ ~ ~ ~ ~ ~ ~ ~ ~ ~ ~  íwaxeeroomako'sh.\\
ki-ruwą'k\#shi si=awį i-tE=rį ~ ~ ~ ~ ~ ~ ~ ~ ~ ~ ~ ~ ~ ~ ~ i-wa-xee=oowąk=o'sh\\
mid-\textnormal{man}\#\textnormal{be.good} \textnormal{travel}=cont pv.ins-\textnormal{stand}=ss ~ ~ ~ ~ ~ ~ ~ ~ ~ ~ ~ ~ ~ ~ ~ pv.ins-1a-\textnormal{be.slow}=narr=ind.m\\
\glt `As Royal Chief was traveling, he got tired and he stopped.'

\item\label{EJ64} \glll Káni, Kinúma'kshi haná'ni, ~ ~ ~ ~ ~ ~ ~ ~ ~ ~ ~ ~ ~ ~ ~ ~ ~ ~ ~ ~ ``Wáa'okipkaxanasheena ~ ~ ~ ~ ~ ~ ~ ~ ~ ~ ~ ~ shíwaheres wakíkiishkekto'sh,'' éerehoomako'sh.\\
ka=rį ki-ruwą'k\#shi hrą'=rį ~ ~ ~ ~ ~ ~ ~ ~ ~ ~ ~ ~ ~ ~ ~ ~ ~ ~ ~ ~ waa-o-ki-k-pax=rąsh=ee=rą ~ ~ ~ ~ ~ ~ ~ ~ ~ ~ ~ ~ shi\#wa-hrE=s wa-kikiishkE=kt=o'sh ee-reh=oowąk=o'sh\\
prov=ss mid-\textnormal{man}\#\textnormal{be.good} \textnormal{sleep}=ss ~ ~ ~ ~ ~ ~ ~ ~ ~ ~ ~ ~ ~ ~ ~ ~ ~ ~ ~ ~ nom-pv.irr-iter-suus-\textnormal{be.broken}=att=dem.dist=top ~ ~ ~ ~ ~ ~ ~ ~ ~ ~ ~ ~ \textnormal{be.good}\#1a-caus=def 1a-\textnormal{consider}=pot=ind.m pv-\textnormal{think}=narr=ind.m\\
\glt `And, Royal Chief slept and thought, ``I'll try that pretty thing that I know how to do.{''}'

\item\label{EJ65} \glll Káni, skiskíka kaxték kitaaroomako'sh, ~ ~ ~  Kinúma'kshi.\\
ka=rį skiski=ka ka-xtE=ak kitaa=oowąk=o'sh ~ ~ ~ ki-ruwą'k\#shi\\
prov=ss \textnormal{diamond.willow}=hab agt-\textnormal{be.big}=ds \textnormal{wake.up}=narr=ind.m ~ ~ ~ mid-\textnormal{man}\#\textnormal{be.good}\\
\glt `And, there were diamond willows in a bunch there as he woke up, that Royal Chief.'

\item\label{EJ66} \glll Skiskíka kaxték híni ~ ~ ~ ~ ~ ~ ~ ~ ~ ~ wáashiheres ísekoomako'sh.\\
skiski=ka ka-xtE=k hi=rį ~ ~ ~ ~ ~ ~ ~ ~ ~ ~ waa-shi\#hrE=s i-sek=oowąk=o'sh\\
\textnormal{diamond.willow}=hab agt-\textnormal{be.big}=hab \textnormal{arrive.there}=ss ~ ~ ~ ~ ~ ~ ~ ~ ~ ~ nom-\textnormal{be.good}\#caus=def pv.ins-\textnormal{make}=narr=ind.m\\
\glt `He got to that bunch of diamond willows and he did his medicine.'

\item\label{EJ67} \glll Kinúma'kshi istámi kirusháni ~ ~ ~ ~ ~ ~ ~ ~ ~ ~ ~ ~ ~ skiskíka kaxtékseet istámi ~ ~ ~ ~ íkų'teroomako'sh.\\
ki-ruwą'k\#shi ista\#wį k-ru-shE=rį ~ ~ ~ ~ ~ ~ ~ ~ ~ ~ ~ ~ ~ skiski=ka ka+xtE=k=s=ee=t ista\#wį ~ ~ ~ ~ i-kų'tE=oowąk=o'sh\\
mid-\textnormal{man}\#\textnormal{be.good} \textnormal{face}\#\textnormal{orb} suus-ins.hand-\textnormal{grasp}=ss ~ ~ ~ ~ ~ ~ ~ ~ ~ ~ ~ ~ ~ \textnormal{diamond.willow}=hab agt-\textnormal{be.big}=hab=def=dem.dist=loc \textnormal{face}\#\textnormal{orb} ~ ~ ~ ~ pv.dir-\textnormal{throw}=narr=ind.m\\
\glt `Royal Chief took out his eyes and threw his eyes at the bunch of willows.'

\item\label{EJ68} \glll Káni, Kinúma'kshi ``Mí'stami kúha ~ ~ ~ skóte; mí'tami kúha skóte,'' ~ ~ ~ ~ ~ éeheeroomako'sh.\\
ka=rį ki-ruwą'k\#shi w'\~~-ista\#wį kuh=E=$\varnothing$ ~ ~ ~ skot=E w'\~~-ista\#wį kuh=E=$\varnothing$ skot=E ~ ~ ~ ~ ~ ee-hee=oowąk=o'sh\\
prov=ss mid-\textnormal{man}\#\textnormal{be.good} 1poss-\textnormal{face}\#\textnormal{orb} \textnormal{come.here}.vert=sv=cont ~ ~ ~ \textnormal{go.plop}=sv 1poss-\textnormal{face}\#\textnormal{orb} \textnormal{come.here}.vert=sv=cont \textnormal{go.plop}=sv ~ ~ ~ ~ ~ pv-\textnormal{say}=narr=ind.m\\
\glt `And, Royal Chief said, ``My eyes go plop coming back; my eyes go plop coming back.{''}'

\item\label{EJ69} \glll Háktek, istámis wáakuhinixoomako'sh.\\
ha=kte=ak ista\#wį=s waa-kuh=rįx=oowąk=o'sh\\
prov=pot=ds \textnormal{face}\#\textnormal{orb}=def neg-\textnormal{come.here}.vert=neg=narr=ind.m\\
\glt `So, his eyes did not come back.'

\item\label{EJ70} \glll Háktek, Kinúma'kshi hó ~ ~ ~ ~ ~ ~ ~ ~ ~ ~ ~ ~ ~ ~ ~ ~ ~ ~ ~ íxaxąąkereroomako'sh: ``Mí'stami ~ ~ ~ ~ ~ ~ ~ ~ ~ ~ ~ kúha skóte; mí'stami ~ ~ ~ ~ ~ ~ ~ ~ ~ ~ ~ ~ ~ kúha skóte!''\\
ha=kte=ak ki-ruwą'k\#shi ho ~ ~ ~ ~ ~ ~ ~ ~ ~ ~ ~ ~ ~ ~ ~ ~ ~ ~ ~ i-xa$\sim$xąą=krE=oowąk=o'sh w'\~~-ista\#wį ~ ~ ~ ~ ~ ~ ~ ~ ~ ~ ~ kuh=E=$\varnothing$ skot=E w'\~~-ista\#wį ~ ~ ~ ~ ~ ~ ~ ~ ~ ~ ~ ~ ~  kuh=E=$\varnothing$ skot=E\\
prov=pot=ds mid-\textnormal{man}\#\textnormal{be.good} \textnormal{voice} ~ ~ ~ ~ ~ ~ ~ ~ ~ ~ ~ ~ ~ ~ ~ ~ ~ ~ ~ pv.ins-aug$\sim$\textnormal{be.loud}=3pl=narr=ind.m 1poss-\textnormal{face}\#\textnormal{orb} ~ ~ ~ ~ ~ ~ ~ ~ ~ ~ ~ \textnormal{come.here}.vert=sv=cont \textnormal{go.plop}=sv 1poss-\textnormal{face}\#\textnormal{orb} ~ ~ ~ ~ ~ ~ ~ ~ ~ ~ ~ ~ ~  \textnormal{come.here}.vert=sv=cont \textnormal{go.plop}=sv\\
\glt `So, Royal Chief said in a loud voice, ``My eyes go plop coming back; my eyes go plop coming back!{''}'

\item\label{EJ71} \glll Istámis wáakuhinixoomako'sh.\\
ista\#wį waa-kuh=rįx=oowąk=o'sh\\
\textnormal{face}\#\textnormal{orb} neg-\textnormal{come.here}.vert=net=narr=ind.m\\
\glt `His eyes did not come back.'

\item\label{EJ72} \glll Tu'éshka: ``Mí'stami kúha skóte; ~ ~ ~ ~ ~ mí'stami kúha skóte!''\\
tu-eshka w'\~~-ista\#wį kuh=E=$\varnothing$ skot=E ~ ~ ~ ~ ~ w'\~~-ista\#wį kuh=E=$\varnothing$ skot=E\\
\textnormal{some}-smlt 1poss-\textnormal{face}\#\textnormal{orb} \textnormal{come.here}.vert=sv=cont \textnormal{go.plop}=sv ~ ~ ~ ~ ~ 1poss-\textnormal{face}\#\textnormal{orb} \textnormal{come.here}.vert=sv=cont \textnormal{go.plop}=sv\\
\glt `[With] some more [loud voice, he said,] ``My eyes go plop coming back; my eyes go plop coming back!''

\item\label{EJ73} \glll Kinúma'kshi istámis wáakuhinixoomako'sh.\\
ki-ruwa'k\#shi ista\#wį waa-kuh=rįx=oowąk=o'sh\\
mid-\textnormal{man}\#\textnormal{be.good} \textnormal{face}\#\textnormal{orb} neg-\textnormal{come.here}.vert=net=narr=ind.m\\
\glt `Royal Chief's eyes did not come back.'

\item\label{EJ74} \glll ``Wáashiwaheres wáaxte íwasekak, kashká'nik ~ ~ mi'óshka íwaseką't, ptamáah tóops ~ ~ ~ ~ ~ wakirúshekere'sh.''\\
waa-shi\#wa-hrE=s waa-xtE i-wa-sek=ak ka=shka'rįk ~ ~ wį-oshka i-wa-sek=ą't p-ta-wąąh toop=s ~ ~ ~ ~ ~ wa-k-ru-shE=krE=o'sh\\
nom-\textnormal{good}\#1a-caus=def nom-\textnormal{be.big} pv.ins-1a-\textnormal{make}=ds prov=dsj ~ ~ 1s-emph pv.ins-1a-\textnormal{make}=dem.anap 1poss-al-\textnormal{arrow} \textnormal{four}=def ~ ~ ~ ~ ~ 1a-vert-ins.hand-\textnormal{grasp}=3pl=ind.m\\
\glt `{``}I did my medicine poorly, but that was how I did it myself when I took my four arrows back,'' [thought Royal Chief].'

\item\label{EJ75} \glll Háktek, Kinúma'kshi rá'taxini ~ ~ ~ ~ ~ ~ ų́ųpat pakíshanasha níira maná ~ ~ ~ ~ ~ wáa'ikatąk, ``Maná té, nimátewe'sha?''\\
ha=kte=ak ki-ruwą'k\#shi ra'-tax=rį ~ ~ ~ ~ ~ ~ ųųpat pa-kish=nash=E=$\varnothing$ rįį=E=$\varnothing$ wrą ~ ~ ~ ~ ~ waa-i-ka-tąk wrą tE rį-wa-t-we=o'sha\\
prov=pot=ds mid-\textnormal{man}\#\textnormal{be.good} ins.heat-\textnormal{make.loud.noise}=ss ~ ~ ~ ~ ~ ~ \textnormal{be.different} ins.push-\textnormal{feel}=att=sv=cont \textnormal{walk}=sv=cont \textnormal{tree} ~ ~ ~ ~ ~ \textnormal{some}-pv.ins-ins.frce-\textnormal{bump} \textnormal{tree} \textnormal{stand} 2s-unsp-wh-indf=int.m\\
\glt `So, Royal Chief cried out and as he was walking, feeling his way around, he bumped into a tree, [asking,] ``Tree standing there, what are you?{''}'

\item\label{EJ76} \glll Háktek, manáseena: ``Tapsá manáko'sh.''\\
ha=kte=ak wrą=s=ee=rą tapsa wa-rąk=o'sh\\
prov=pot=ds \textnormal{tree}=def=dem.dist=top \textnormal{ash.tree} 1s-pos.sit=ind.m\\
\glt `So, the tree [replied,] ``I am an Ash.{''}'

\item\label{EJ77} \glll ``Hóo, wáaxte íwaseko'sh.''\\
hoo waa-xtE i-wa-sek=o'sh\\
\textnormal{yes} nom-\textnormal{be.big} pv.ins-1a-\textnormal{make}=ind.m\\
\glt `{``}Yes, I really messed up,{''} [Royal Chief said].'

\item\label{EJ78} \glll ``Wáawahe míko'sh.''\\
waa-wa-hE wįk=o'sh\\
nom-1a-\textnormal{see} \textnormal{be.none}=ind.m\\
\glt `{``}I cannot see anything,{''} [Royal Chief added].'

\item\label{EJ79} \glll ``Mamáhenashinista.''\\
wą$\sim$wą-hE=rąsh=rįt=ta\\
aug$\sim$1s-\textnormal{see}=att=2pl=imp.m\\
\glt `[Royal Chief implored,] ``You've got to try to see for me.{''}'

\item\label{EJ80} \glll ``Ímaataht waréeh íwateero'sh.''\\
i-wąątah=t wa-rEEh i-wa-tee=o'sh\\
pv.dir-\textnormal{river}=loc 1a-\textnormal{go.there} pv.ins-1a-\textnormal{die}=ind.m\\
\glt `{``}I would like to go to the river.{''}'

\item\label{EJ81} \glll Háktek, tapsáseena: ``Ímaataht ~ ~ ~ ~ ~ ~ ~ ~ ~ ~ órataaro'sh.''\\
ha=kte=ak tapsa=s=ee=rą i-wąątah=t ~ ~ ~ ~ ~ ~ ~ ~ ~ ~ o-ra=taa=o'sh\\
prov=pot=ds \textnormal{ash.tree}=def=dem.dist=top pv.dir-\textnormal{river}=loc ~ ~ ~ ~ ~ ~ ~ ~ ~ ~ pv.loc-2a=loc=ind.m\\
\glt `So, the Ash [said,] ``You are facing the river.{''}'

\item\label{EJ82} \glll ``Ą́'s shų́ųshuka ráahta!''\\
ą's shųųshu=ka rEEh=ta\\
\textnormal{be.near} \textnormal{be.straight.ahead}=hab \textnormal{go.there}=imp.m\\
\glt `[The Ash said,] {``}Go straight ahead this way!{''}'

\item\label{EJ83} \glll ``Wáashirehere'sh; ų́'shkakto'sh.''\\
waa-shi\#re-hrE=o'sh ų'sh=ka=kt=o'sh\\
nom-\textnormal{be.good}\#2a-caus=ind \textnormal{be.thus}=hab=pot=ind.m\\
\glt `{``}Thank you,'' [Royal Chief said,] ``That is what I will do.{''}'

\item\label{EJ84} \glll Háktek, Kinúma'kshi réehoomako'sh.\\
ha=kte=ak ki-ruwą'k\#shi rEEh=oowąk=o'sh\\
prov=pot=ds mid-\textnormal{man}\#\textnormal{be.good} \textnormal{go.there}=narr=ind.m\\
\glt `So, Royal Chief went off.'

\item\label{EJ85} \glll Káni, maná íkatąk, ``Maná matewé ~ ~ ~ ~ ~ ~ ~ ~ ~ ~ ni'ó'ro'sha?''\\
ka=rį wrą i-ka-tąk wrą wa-t-we ~ ~ ~ ~ ~ ~ ~ ~ ~ ~ rį-o'=o'sha\\
prov=ss \textnormal{tree} pv.ins-ins.frce-\textnormal{bump} \textnormal{tree} unsp-wh-indf ~ ~ ~ ~ ~ ~ ~ ~ ~ ~ 2s-\textnormal{be}=int.m\\
\glt `And, bumping into a [different] tree, [Royal Chief asked,] ``What kind of tree are you?{''}'

\item\label{EJ86} \glll ``Míihkatamanaka,'' éeheeroomako'sh.\\
wįįh=ka\#ta-wrą=ka ee-hee=oowąk=o'sh\\
\textnormal{woman}=hab\#al-\textnormal{tree}=hab pv-\textnormal{say}=narr=ind.m\\
\glt `{``}Box Elder,'' it said.'

\item\label{EJ87} \glll ``Hóo, wáashi'sh.''\\
hoo waa-shi=o'sh\\
\textnormal{yes} nom-\textnormal{be.good}=ind.m\\
\glt `[Royal Chief said,] ``Yes, that is good.{''}'

\item\label{EJ88} \glll ``Ítewetaa máataho'sha?''\\
i-t-we=taa wąątah=o'sha\\
pv.dir-wh-indf=loc \textnormal{river}=int.m\\
\glt `[Royal Chief asked,] ``Which way is the river?{''}'

\item\label{EJ89} \glll ``Shų́ųshuka ą́'s ráahta!''\\
shųųshu=ka ą's rEEh=ta\\
\textnormal{be.straight.ahead}=hab \textnormal{be.near} \textnormal{go.there}=imp.m\\
\glt `{``}Go this way straight ahead!{''} [said the Box Elder].'

\item\label{EJ90} \glll ``Wáashirehere'sh; ų́'shkakto'sh.''\\
waa-shi\#re-hrE=o'sh ų'sh=ka=kt=o'sh\\
nom-\textnormal{be.good}\#2a-caus=ind \textnormal{be.thus}=hab=pot=ind.m\\
\glt `{``}Thank you,'' [Royal Chief said,] ``That is what I will do.{''}'

\item\label{EJ91} \glll Háktek, Kinúma'kshi ráahami inák maná'na ~ ~ ~ ~ íkatąkoomako'sh.\\
ha=kte=ak ki-ruwą'k\#shi rEEh=awį irąk wrą=o'=rą ~ ~ ~ ~ i-ka-tąk=oowąk=o'sh\\
prov=pot=ds mid-\textnormal{man}\#\textnormal{be.good} \textnormal{go.there}=cont \textnormal{again} \textnormal{tree}=\textnormal{be}=top ~ ~ ~ ~ pv.ins-ins.frce-\textnormal{bump}=narr=ind.m\\
\glt `So, as Royal Chief was going, there was another tree that he bumped into.'

\item\label{EJ92} \glll Kinúma'kshiseena manáseet: ~ ~ ~ ~ ~  ``Nimátewe ó'ro'sha?''\\
ki-ruwą'k\#shi=s=ee=rą wrą=s=ee=t ~ ~ ~ ~ ~ rį-wa-t-we o'=o'sha\\
mid-\textnormal{man}\#\textnormal{be.good}=def=dem.dist=top \textnormal{tree}=def=dem.dist=loc ~ ~ ~ ~ ~ 2s-unsp-wh-indf \textnormal{be}=int.m\\
\glt `Royal Chief [said] to the tree, ``What kind are you?{''}'

\item\label{EJ93} \glll Manáseena: ``Mawáaxe'sh,'' éeheeroomako'sh.\\
wrą=s=ee=rą wą-waaxE=o'sh ee-hee=oowąk=o'sh\\
\textnormal{tree}=def=dem.dist=top 1s-\textnormal{cottonwood}=ind.m pv-\textnormal{say}=narr=ind.m\\
\glt `The tree said, ``I am a Cottonwood.{''}'

\item\label{EJ94} \glll Kinúma'kshiseena: ``Wáa'oshka máatahe ~ ~ ~ ~ ~ ~ ~ ~ ~ ~ wahíroote'sh.''\\
ki-ruwą'k\#shi=s=ee=rą waa-oshka wąątah=E ~ ~ ~ ~ ~ ~ ~ ~ ~ ~ wa-hi=ootE=o'sh\\
mid-\textnormal{man}\#\textnormal{be.good}=def=dem.dist=top nom-emph \textnormal{river}=sv ~ ~ ~ ~ ~ ~ ~ ~ ~ ~ 1a-\textnormal{arrive.there}=evid=ind.m\\
\glt `Royal Chief [said,] ``It's a good thing that I must have arrived at the river.{''}'

\item\label{EJ95} \glll Kinúma'kshi wáaxseeta: ``Ítewetaa máatahe ó'ro'sha?''\\
ki-ruwą'k\#shi waax=s=ee=taa i-t-we=taa wąątah=E o'=o'sha\\
mid-\textnormal{man}\#\textnormal{be.good} \textnormal{cottonwood}=def=dem.dist=loc pv.dir-wh-indf=loc \textnormal{river}=sv \textnormal{be}=int.m\\
\glt `Royal Chief [said] to the Cottonwood, ``Which way is it to the river?{''}'

\item\label{EJ96} \glll Wáaxseena: ``Máatahe órataaro'sh.''\\
waax=s=ee=rą wąątah=E o-ra=taa=o'sh\\
\textnormal{cottonwood}=def=dem.dist=top \textnormal{river}=sv pv.loc-2a=loc=ind.m\\
\glt `The Cottonwood [replied,] ``You are facing the river.{''}'

\item\label{EJ97} \glll  ``Ą́'s shų́ųshuka ráahta!''\\
ą's shųųshu=ka rEEh=ta\\
\textnormal{be.near} \textnormal{be.straight.ahead}=hab \textnormal{go.there}=imp.m\\
\glt `[The Cottonwood said,] {``}Go straight ahead this way!{''}'

\item\label{EJ98} \glll Háktek, Kinúma'kshi réehoomako'sh.\\
ha=kte=ak ki-ruwą'k\#shi rEEh=oowąk=o'sh\\
prov=pot=ds mid-\textnormal{man}\#\textnormal{be.good} \textnormal{go.there}=narr=ind.m\\
\glt `So, Royal Chief went off.'

\item\label{EJ99} \glll Maná xamáhąą íkatąka ~ ~ ~ ~ ~ ~ ~ ~ ~ ~ ~ ~ ~ ~ ~ Kinúma'kshi: ``Maná nimátewe ó'ro'sha?''\\
wrą xwąh=hąą i-ka-tąk=E=$\varnothing$ ~ ~ ~ ~ ~ ~ ~ ~ ~ ~ ~ ~ ~ ~ ~ ki-ruwą'k\#shi wrą rį-wa-t-we o'=o'sha\\
\textnormal{tree} \textnormal{be.small}=ins pv.ins-ins.frce-\textnormal{bump}=sv=cont ~ ~ ~ ~ ~ ~ ~ ~ ~ ~ ~ ~ ~ ~ ~ mid-\textnormal{man}\#\textnormal{be.good} \textnormal{tree} 2s-unsp-wh-indf \textnormal{be}=int.m\\
\glt `Bumping into a little tree, Royal Chief [asked,] ``What kind of tree are you?{''}'

\item\label{EJ100} \glll ``Mamanáseka'sh.''\\
wą-wrą\#se=ka=o'sh\\
1s-\textnormal{tree}\#\textnormal{be.red}=hab=ind.m\\
\glt `{``}I am a Red Willow.{''}'

\item\label{EJ101} \glll Háktek, Kinúma'kshi: ``Wáa'oshkanashe máatahe ~ ~ ~ ~ ~ wahíroote'sh.''\\
ha=kte=ak ki-ruwą'k\#shi waa-oshka=rąsh=E wąątah=E ~ ~ ~ ~ ~ wa-hi=ootE=o'sh\\
prov=pot=ds mid-\textnormal{man}\#\textnormal{be.good} nom-emph=att=sv \textnormal{river}=sv ~ ~ ~ ~ ~ 1a-\textnormal{arrive.there}=evid=ind.m\\
\glt `So, Royal Chief was like, ``It is kind of a good thing that I must have gotten to the river.{''}'

\item\label{EJ102} \glll ``Wáaxte íwaseke, wáawahe míko'sh.''\\
waa-xtE i-wa-sek=E waa-wa-hE wįk=o'sh\\
nom-\textnormal{be.big} pv.ins-1a-\textnormal{make}=sv nom-1a-\textnormal{see} \textnormal{be.none}=ind.m\\
\glt `[Royal Chief said,] {``}After I did something awful, I cannot see anything.{''}'

\item\label{EJ103} \glll``Mamáhenashinista.''\\
wą$\sim$wą-hE=rąsh=rįt=ta\\
aug$\sim$1s-\textnormal{see}=att=2pl=imp.m\\
\glt `[Royal Chief begged,] ``You've got to try to see for me.{''}'

\item\label{EJ104} \glll ``Máatahe, tewét ó'ro'sha?''\\
wąątah=E t-we=t o'=o'sha\\
\textnormal{river}=sv wh-indf=loc \textnormal{be}=int.m\\
\glt `{``}The river, where is it?'' [Royal Chief asked].'

\item\label{EJ105} \glll ``Ímarushani róoskata.''\\
i-wą-ru-shE=rį rooskE=ta\\
pv.ins-1s-ins.hand-\textnormal{grasp}=ss \textnormal{climb.down}=imp.m\\
\glt `{``}Grab hold of me and climb down,'' [the Red Willow said].'

\item\label{EJ106} \glll ``Máatahe ą́'teroo máko'sh.''\\
wąątah=E ą't=roo wąk=o'sh\\
\textnormal{river}=sv dem.anap=dem.mid pos.lie=ind.m\\
\glt `{``}The river is right there{''}, [said the Red Willow].'

\item\label{EJ107} \glll Háktek, Kinúma'kshi, áakereh, pakísha, ~ ~ ~ ~ máa'ąkanasha, máaptes íwokahąą híni ~ ~ ų́khąą minís rusháni istámis ~ ~ ~ ~ ~ ~ ~ ~ ~ ~ kirusá'roomako'sh.\\
ha=kte=ak ki-ruwą'k\#shi aakreh pa-kish=E=$\varnothing$ ~ ~ ~ ~ waa'ąk=rąsh=E=$\varnothing$ wąąpte=s i-woka=hąą hi=rį ~ ~ ųk=hąą wrį=s ru-shE=rį ista\#wį=s ~ ~ ~ ~ ~ ~ ~ ~ ~ ~ krusa'=oowąk=o'sh\\
prov=pot=ds mid-\textnormal{man}\#\textnormal{be.good} \textnormal{be.poor} ins.push-\textnormal{feel}=sv=cont ~ ~ ~ ~ \textnormal{earth}=att=sv=cont \textnormal{river.bank}=def pv.dir-\textnormal{edge}=ins \textnormal{arrive.there}=ss ~ ~ \textnormal{hand}=ins \textnormal{water}=def ins.hand-\textnormal{grasp}=ss \textnormal{face}\#\textnormal{orb}=def ~ ~ ~ ~ ~ ~ ~ ~ ~ ~ \textnormal{wash}=narr=ind.m\\
\glt `So, Royal Chief, poor thing, as he was feeling around, sort of on the ground, he got to the river bank edge and he took the water with his hands and washed his eyes.'

\item\label{EJ108} \glll Istámi kináakaroomako'sh.\\
ista\#wį ki-rąąka=oowąk=o'sh\\
\textnormal{face}\#\textnormal{orb} mid-\textnormal{be.new}=narr=ind.m\\
\glt `His eyes became like new.'

\item\label{EJ109} \glll Kinúma'kshi nátka shíroomako'sh.\\
ki-ruwą'k\#shi rąt=ka shi=oowąk=o'sh\\
mid-\textnormal{man}\#\textnormal{be.good} \textnormal{be.middle.of}=hab \textnormal{be.good}=narr=ind.m\\
\glt `Royal Chief was happy.'

\item\label{EJ110} \glll Kináatani máapte kaní'ni kasími ~ ~ ~ ~ ~ réehoomako'sh.\\
ki-rąątE=rį wąąpte ka-rį'=rį ka-si=awį ~ ~ ~ ~ ~ rEEh=oowąk=o'sh\\
iter-\textnormal{stand.up}=ss \textnormal{river.bank} ins.frce-\textnormal{climb}=ss inch-\textnormal{travel}=cont ~ ~ ~ ~ ~ \textnormal{go.there}=narr=ind.m\\
\glt `He got up again, climbed up the river bank, and he set off traveling.'

\end{exe}

\subsection{Free translation}\label{Ch6FreeTranslation}

As they tell it, Royal Chief was traveling along. While he was traveling along the edge of the river, he heard something: a shout! His ears perked up, and he headed over in the direction of what he had just heard.

When Royal Chief peeked through the trees, and he saw that there were many children playing there in the woods. The children spotted Royal Chief, causing them to all scatter and run away from him. When this happened, Royal Chief called after them, saying, ``O children! Don't run! Come back! It is rude to act this way when an elder is approaching.'' The children came back after hearing what Royal Chief said.

``You are always up to no good, so we figured that we would run away,'' the children said to Royal Chief. 

``People will laugh at you if they knew you were running away from your elders,'' warned Royal Chief. ``So,'' Royal Chief continued, ``children, what are you doing?''

The children explained: ``We are taking out our eyes and throwing them up to hang them on a tree.'' Royal Chief became curious.

``My nephews, I would very much like to see you do it,'' said Royal Chief. At his request, the children took out their eyes and threw them at the tree. Their eyes were hanging there in the tree, glistening. Seeing this, Royal Chief was impressed, saying, ``Children, that is really pretty.''

After showing Royal Chief what they did, the children called their eyes back, saying ``My eyes go plop coming back; my eyes go plop coming back!'' The elder really liked what he saw.

``Teach it to me,'' said Royal Chief, ``that thing that you did.'' The children said that they would teach him.

``What will you give us for it?'' asked the children.\footnote{It is customary to buy the rights to medicine from those who have knowledge of the medicine.}

Royal Chief thought and replied, ``I will give you my four arrows.'' The children were happy with this arrangement and the deal was struck. ``Nephews,'' said Royal Chief, ``show me how to do it.''

At this, the children started to explain to Royal Chief what to do. ``Whatever it is we do, you must do it in the exact same way. When we take our eyes out, you should do it, too.'' And so, the children grabbed all of their eyeballs, then threw their eyeballs towards the tree. Their eyes were just hanging all over that tree.

So, Royal Chief told the children, ``My nephews, I am going to head out traveling.'' The children forbade him from leaving.

``Elder,'' they said, ``we are having a good time, and you want to go and leave us just like that.''

``Gosh,'' mused Royal Chief, ``I have already been here a long time, staying with you all.''

The children cried as Royal Chief set off traveling, and he kept right on traveling along the edge of the river. After traveling, he stopped and wanted to try out what he had learned. There was a big bunch of diamond willows nearby, so he headed over there, took out his eyes, and threw them towards the willow trees. His eyes were there in the trees, and Royal Chief admired what he had done. So, Royal Chief said, ``My eyes go plop coming back; my eyes go plop coming back,'' and his eyes came back to him and went back in, making a kind of splashing sound when they did. Having done it, Royal Chief was really happy.

``It is so pretty, what I know how to do,'' Royal Chief thought. So, he stopped doing what he was doing and resumed his travels. In his travels, Royal Chief used this new medicine to do many mischievous things. He threw his eyes to make birds scatter. Sometimes, he used his medicine to make a cottontail jump in surprise, though he was mostly a bird-scarer, continuing with what he was doing for about three days. After about three days, he tried his medicine on a bunch of diamond willows standing nearby. Then, he threw his eyes at the willows.

Royal Chief was hanging his eyes around on a tree, going, ``My eyes go plop coming back; my eyes go plop coming back!'' After his eyes came back, Royal Chief thought, ``This is \textit{my} medicine. I should go back to the children and get my four arrows back.'' He then turned back and ran back to where the children were.

``Hey, nephews,'' Royal Chief called to the children. ``Well, the thing is, I did what I did a long time ago. Where are my four arrows? Give them back to me!''

Hearing this, the children hung their heads and were sad. They did give him his arrows back, but their tears were falling as they did so. After getting his arrows back, Royal Chief went off. 

Having his arrows, Royal Chief was going along. Eventually, he came to a hill, which he climbed. On the other side of the hill, he threw his four arrows away and left. After all that time traveling, Royal Chief became tired and stopped to take a rest. He slept, thinking, ``I'll try that pretty thing that I know how to do when I wake up.'' As he woke up, there were a bunch of diamond willows standing nearby. He headed over to that bunch of diamond willows and did what he did. 

He took out his eyes and threw them at the bunch of willows. After his eyes were hanging in the trees, Royal Chief said, ``My eyes go plop coming back; my eyes go plop coming back.'' However, his eyes did not come back.

In a loud voice, Royal Chief said again, ``My eyes go plop coming back; my eyes go plop coming back!'' Again, his eyes did not come back.

``I did not do it right, but it was just how I did it when I took my four arrows back,'' thought Royal Chief.\footnote{Royal Chief is unable to engage with the medicine as he did previously, because he has violated some aspect of the rules around using this medicine. Reneging on the agreement he had with the original bestowers of the medicine, the children, is likely what caused his eyes to no longer return when called.} He cried out and tried to find his eyes. As he was walking, he was feeling his way around. While he was feeling around, he bumped into a tree. Royal Chief asked, ``Hey, tree standing there, what are you?''

The tree responded, ``I am an Ash.''

Royal Chief sighed, saying, ``Yes, I really messed up. I cannot see anything. Ash, you've got to try to see for me. I would like to go to the river.''\footnote{Royal Chief is looking for the river, since he threw his eyes at a group of diamond willows. These trees would grow close to water, versus the native ash species of the upper Plains, which tend to be found in woodlands more generally. Thus, if Royal Chief can find his way to the river, he would be in a better position to find where his eyes were.}

Hearing Royal Chief's predicament, Ash said, ``You are actually facing the river. Go straight ahead this way!''

``Thank you,'' Royal Chief said. ``That is what I will do.'' So, Royal Chief went off. As he went along, he bumped into a different tree. He then asked, ``What kind of tree are you?''

``Box Elder,'' replied the tree.

``Yes,'' said Royal Chief. ``That is good.\footnote{Box elder trees are most commonly found near alluvial soils by streams, so Royal Chief knows that he is getting closer to the river by finding a box elder.} Which way is the river?''

``Go this way, straight ahead,'' said the Box Elder.

``Thank you,'' said Royal Chief. ``That is what I will do.'' So, Royal Chief kept going along until he bumped into yet another tree. ``What kind are you?'' he asked the tree.

The tree responded, ``I am a Cottonwood.''

Royal Chief said, ``It is a good thing that I must have arrived at the river.\footnote{Cottonwoods typically require nuterients provided by the flooding of rivers, so if Royal Chief has found a cottonwood, that means that he is near a river. He is now very close in his mind to finding his eyes.} Royal Chief asked the Cottonwood, ``Which way is it to the river?''

``You are facing the river,'' the Cottonwood replied. ``Go straight ahead this way!''

Royal Chief again went off. Bumping into a little tree, he asked, ``What kind of tree are you?''

``I am a Red Willow,'' replied the little tree.

Royal Chief was heartened, saying, ``It is kind of a good thing that I must have gotten to the river.''\footnote{Red willows are trees usually found on river banks, so Royal Chief is glad to finally be at the river.} Then, he explained to the tree, ``After I did something awful, I cannot see anything. You've got to try to see for me. The river, where is it?''

``Grab hold of me and climb down,'' said the Red Willow. ``The river is right there.''

Royal Chief, poor thing, as he was feeling around on the ground, he got himself to the edge of the river bank. He cupped some water in his hands and he washed where his eyes were. His eyes came back and were like new. Royal Chief was quite happy. He got up again, climbed up the river bank, and he set off traveling.

% hiree hok=ee=rą wa-kirą'=rį ee=w-reh=o'sh\\
% \textnormal{now} \textnormal{story}=dem.dist=top 1a-\textnormal{tell}=ss pv-1a-\textnormal{want}=ind.\\
% \glt `Now, I want to tell you a story.'

% \item \glll Réshka'eshka éeheero'sh, hókere.\\
%     réshka-eshka ee-hEE=o'sh hok=re\\
%     \textnormal{this.way}-sim pv-\textnormal{say}=ind.m \textnormal{story}=dem.prox\\
%     \glt `This is the way it goes, this story.'

% \item \glll Máxha, numá'keena ó'rak, Minítaari numá'koomako'sh.\\
%     wąx\#ha ruwą'k=ee=rą o'=ak wrį\#taari ruwą'k=oowąk=o'sh\\
%     \textnormal{one}\#\textnormal{time} \textnormal{man}=dem.dist=top \textnormal{be}=ds \textnormal{water}\#\textnormal{cross} \textnormal{man}=narr=ind.m\\
%     \glt `Once, there was a man, a Hidatsa man.'

% \item \glll Minítaari numá'kere kíikiniihisi xarékoomako'sh.\\
%     wrį\#taari ruwą'k=re kiikrįį-sįh xarek=oowąk=o'sh\\
%     \textnormal{water}\#\textnormal{cross} \textnormal{man}=dem.prox \textnormal{gamble}-ints \textnormal{love}=narr=ind.m\\
%     \glt `This Hidatsa man loved to really gamble.'

% \item \glll Wáaxtaani wáakiikiniinash íteeroomako'sh.\\
%     waa-xtE=rį waa-kiikrįį=rąsh i-tee=oowąk=o'sh\\
%     nom-\textnormal{big}=ss nom-\textnormal{gamble}=att pv.ins-\textnormal{like}=narr=ind.m\\
%     \glt `He liked gambling of all kinds.'

% \item \glll Ték, rá'skek óo, ta'íshahe, kotámaanuka kíikiniika hį, kirúxįhkereroomako'sh.\\
%     te=ak ra'ske=ak oo $\varnothing$-ta-i-shahe ko-ta-waarųka kiikrįį=ka hį ki-ru-xįk=krE=oowąk=o'sh\\
%     \textnormal{stand}=ds \textnormal{summer} dem.mid 3poss-al-pv.dir-\textnormal{across} 3poss.pers-al-\textnormal{man's.friend} \textnormal{gamble}=hab \textnormal{um} rflx-ins.hand-\textnormal{get.mad.at}=3pl=narr=ind.m\\
%     \glt `So, during the summer, his opponent, his friend who he would gamble with, um, they got mad at each other.'

% \item \glll Ték hį, he, Minítaari numa'kere wáaxtaani rá'kehini hį, ráahini róo kíikinii ímanakeres rusháani, ų́'sh-ų́'sh, rusháani ą́ąwe ka'úuxoomako'sh.\\
%     te=ak hį he wrį\#taari nuwą'k=re waa-xtE=rį ra'keh=rį hį rEEh=rį roo kiikrįį i-wrą=krE=s ru-shE=rį ų'sh-ų'sh ru-shE=rį ąąwe ka-uux=oowąk=o'sh\\
%     \textnormal{stand}=ds \textnormal{um} \textnormal{well} \textnormal{water}\#\textnormal{cross} \textnormal{man}=dem.prox nom-\textnormal{big}=ss \textnormal{angry}=ss \textnormal{um} \textnormal{go.there}=ss \textnormal{dem.mid} \textnormal{gamble} pv.ins-\textnormal{wood}=3pl=def ins.hand-\textnormal{grasp}=ss \textnormal{and.so.on} ins.hand-\textnormal{grasp}=ss \textnormal{all} ins.frce-\textnormal{broken}=narr=ind.m\\
%     \glt `So, um, well, this Hidatsa man got really mad, um, and he went there and and took the gambling sticks, taking them and breaking them all to pieces.'

% \item \glll Ímanakerere, tapsá ímanakereroomako'sh.\\
%     i-wrą=krE=re tapsa i-wrą=krE=oowąk=o'sh\\
%     pv.ins-\textnormal{wood}=3pl=dem.prox \textnormal{ashwood} pv.ins-\textnormal{wood}=3pl=narr=ind.m\\
%     \glt `These wood sticks, they were made of ashwood.'

% \item \glll Ó'aakiseke, túkereroomako'sh.\\
%     o-aaki\#i-sek=E tu=krE=oowąk=o'sh\\
%     pv.irr-\textnormal{on.top}\#pv.ins-\textnormal{make}=sv \textnormal{be.some}=3pl=narr=ind.m\\
%     \glt `They had some symbols on them.'

% \item  \glll Ték hį, ó'aakisehkerere hį, hį, wáakiruxka áakisehkereroomako'sh, ópusanasheena hų́ni, wáa'ų'shkerek.\\
%     te=ak hį o-aaki\#i-sek=krE=re hį hį waa-krux=ka aaki\#i-sek=krE=oowąk=o'sh o-pus=rąsh=ee=rą hų=rį waa-ų'sh=krE=ak\\
%     \textnormal{stand}=ds \textnormal{um} pv.irr-\textnormal{on.top}\#pv.ins-\textnormal{make}=3pl=dem.prox \textnormal{um} \textnormal{um} nom-\textnormal{snake}=hab \textnormal{on.top}\#pv.ins-\textnormal{make}=3pl=narr=ind.m pv.irr-\textnormal{striped}=att=dem.dist=top \textnormal{be.many}=ss nom-\textnormal{thus}=3pl=ds\\
%     \glt `So then, um, these symbols, um, uh, they were drawings of the back of a snake, there were lots of stripes, stuff like that.'

% \item \glll Ráahini ka'úuxini hį, patíkak, he, ų́'taaharaa hį, taníshkere'ishikara't, hį, taxópininashkaraa á'shka, tashká'eshkanashki, rusiríikereroomako'sh.\\
%     rEEh=rį ka-uux=rį hį pa-tik=ak he ų'=taa\#hrE=$\varnothing$ hį $\varnothing$-ta-rįshkrE=ishi=krE=ą't hį $\varnothing$-ta-xoprį=rąsh=krE=$\varnothing$ a'shka tashka'eshka=rąsh=ki ru-srii=krE=oowąk=o'sh\\
%     \textnormal{go.there}=ss ins.frce-\textnormal{be.broken}=ss \textnormal{um} ins.push-\textnormal{throw.away}=ds \textnormal{well} \textnormal{thus}=loc\#caus-$\varnothing$ \textnormal{um} 3poss-al-\textnormal{medicine}=vis=3pl=cond \textnormal{um} 3poss-al-\textnormal{holy}=att=3pl=cont psbl \textnormal{how}-sim=att=cond ins.hand-\textnormal{curse}=3pl=narr=ind.m\\
%     \glt `He went and broke them, um, and he threw them away, and for that reason, um, since that was evidently their medicine, um, however it was was maybe their holy thing or something, and he was cursed.'

% \item \glll Minítaarire, ó'haraani rusiríikerek, rókanashtaa hį, ná'ta'shkataahaa réshak, ų́'sh-ų'sh, wáaniini wáakirutiirishi mikák hį, he, wáa'ų'shkaso'nik, kikxų́hoomako'sh.\\
%     wrį\#taari=re o'\#hrE=rį ru-srii=krE=ak rok=rąsh=taa hį rą't=a'shka=taa=haa resh=ak ų'sh-ų'sh waa-rįį=rį waa-k-ru-tii=ishi wįk=ak hį he waa-ų'shka=so'rįk ki-kxųh=oowąk=o'sh\\
%     \textnormal{water}\#\textnormal{cross}=dem.prox \textnormal{be}\#caus=ss ins.hand-\textnormal{curse}=3pl=ds \textnormal{leg}=att=loc \textnormal{um} \textnormal{middle}=psbl=loc=sim \textnormal{like.this}=ds \textnormal{and.so.on} nom-\textnormal{walk}=ss nom-rflx-ins.hand-\textnormal{lift}=vis \textnormal{be.none}=ds \textnormal{um} \textnormal{well} nom-\textnormal{thus}=comp.caus mid-\textnormal{lie}=narr=ind.m\\
%     \glt `This Hidatsa man, he was cursed after that; his legs, um, like this on both sides and so forth, he could not walk or lift himself up, um, well, because of it was like this, he was laid up.'

% \item \glll Kikxų́hak hį, kikxų́hak, óohi įmáare, ikúhąą, ų́'sh, tewénak ó'eshka, wáa'irusiniinas...\\
%     ki-kxųh=ak hį ki-kxųh=ak oohi iwąą=E iku=hąą ų'sh t-we-rą=ak o'-eshka waa-i-ru-srįį=rą=s\\
%     mid-\textnormal{lie.down}=ds \textnormal{um} mid-\textnormal{lie.down} \textnormal{at.that} \textnormal{body}=sv \textnormal{be.all.over}=sim \textnormal{thus} wh-indef-top=ds \textnormal{be}-sim nom-pv.ins-ins.hand-\textnormal{curse}=top=def\\
%     \glt `Laying there, um, laying there, it overcame his body all overwhere, so, it was everywhere, the curse...'

% \item \glll Ráahini pánashini, watewé ó'ra'shka, ą́ąwe, ų́'sh, kiwáa'ikirutiirishi míhkanashaa, he, ų́'taaharaa hį...\\
%     rEEh=rį pa=rąsh=rį wa-t-we o'=a'shka ąąwe ų'sh ki-waa-i-ki-ru-tii=ishi mik=ka=rąsh=E=$\varnothing$ he ų'=taa=hrE=$\varnothing$ hį\\
%     \textnormal{go.there}=ss \textnormal{head}=att=ss unsp-wh-indef \textnormal{be}=psbl \textnormal{all} \textnormal{thus} mid-nom-rflx-ins.hand-\textnormal{lift}=vis \textnormal{be.none}=hab=att=sv=cont \textnormal{well} \textnormal{thus}=loc\#caus=cont \textnormal{um}\\
%     \glt `It went sort of to his head, or however it happened, and he could not move at all, so, not at all, well, for that reason, um...'

% \item \glll Hį, ptą́ąrak, kų́'here hį, kų́'hoona hį, kų́'he núpoomako'sh.\\
%     hį ptąą=ak k'-ųh=re hį k'-ųh=oo=rą hį k'-ųh=E rųp=oowąk=o'sh\\
%     \textnormal{um} \textnormal{autumn}=ds 3poss.pers-\textnormal{wife}=dem.prox \textnormal{um} 3poss.pers-\textnormal{wife}=dem.mid=top \textnormal{um} 3poss.pers-\textnormal{wife}=sv \textnormal{two}=narr=ind.m\\
%     \glt `Um, in the fall, this wife of his, um, that wife, um, he had two wives.' 

% \item \glll Numá'kere, Minítaarire, kų́'he ínupkerek hį, ímaxana hį, súkmiihoomako'sh.\\
%     ruwą'k=re wrį\#taari=re k'-ųh=E i-rųp=krE=k hį i-wąxną hį suk\#wįįh=oowąk=o'sh\\
%     \textnormal{man}=dem.prox \textnormal{water}\#\textnormal{cross}=dem.prox 3poss.pers-\textnormal{wife}=sv pv.ins-\textnormal{two}=3pl=ds \textnormal{um} pv.ins-\textnormal{one} \textnormal{one} \textnormal{child}\#\textnormal{woman}=narr=ind.m\\
%         \glt `This man, this Hidatsa, he had two wives and, um, one of them, um, she was a young woman.'

% \item \glll Súkmiih xamáhak hį, he, korátoore, íshahe, numá'k... numá'k, míih... míih ratóoroomako'sh.\\
%     suk\#wįįh xwąh=ak hį he ko-ratoo=E i-shahe ruwą'k ruwą'k wįįh wįįh ratoo=oowąk=o'sh\\
%         \textnormal{child}\#\textnormal{woman} \textnormal{small}=ds \textnormal{um} \textnormal{well} rel-\textnormal{old}=sv pv.ins-\textnormal{across} \textnormal{man} \textnormal{man} \textnormal{woman} \textnormal{woman} \textnormal{old}=narr=ind.m\\
%         \glt `The young one was really young and, um, well, the older one, on the other hand, the man...the man, his wife, she was an older woman.'

% \item \glll Míih ratóonitek, ték hį, kų́'h koxamáhs, waká're hį, pkaminíshak, wáa'oka're óksukeerak, ptą́ąrak, ą́ąwe xką́hkereroomako'sh.\\
%     wįįh ratoo-rįte=ak te=ak hį k'-ųh ko-xwąh=s wa-ka'=E hį ki-pa-wrįsh=ak waa-o-ka'=E o-ksukee=ak ptąą=ak ąąwe xkąh=krE=oowąk=o'sh\\
%     \textnormal{woman} \textnormal{old}-dim=ds \textnormal{stand}=ds \textnormal{um} 3poss.pers-\textnormal{wife} rel-\textnormal{small}=def unsp-\textnormal{possess}=sv \textnormal{um} suus-ins.push-\textnormal{be.rolled.up}=ds nom-ins.irr-\textnormal{possess}=sv pv.loc-\textnormal{put.inside}=ds \textnormal{autumn}=ds \textnormal{all} \textnormal{move.away}=3pl=narr=ind.m\\
%     \glt `The woman was a little bit older, and so the wife who was younger, her things, um, she packed them up, put her belongings away, and in the fall, they all [the younger wife and a new lover] moved away.'
