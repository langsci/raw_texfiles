\chapter{Nominal morphology}\label{chapter4}

This chapter focuses on the kind of morphology seen on nouns or associated with nominal elements. Unlike verbs, nominal elements in Mandan tend to bear minimal morphology. What morphology that does appear directly on nouns is typically that of possession marking, though there is a limited set of morphology that is shared between nouns and verbs.

The preponderance of grammatical descriptions of Mandan, like in other the Siouan languages for which we have extensive documentation, are investigations of the array of formatives that appear on verbs. This makes sense, given the fact that verbs carry a much greater information load in Siouan languages than nouns do. However, this fine-toothed comb approach to the analysis of verbal morphology often resulted in minimalistic treatments of the affixal and enclitic morphology associated with nouns, including that of elements associated with nouns, such as postpositions.

This chapter aims to rectify this gap in the literature by describing aspects of nouns and morphology related to nouns. First, the chapter presents the processes by which nouns are formed or modified in \sectref{SecNounDeriv}, then how possession is marked in \sectref{SecPossession}. Ways to refer to people directly are described in \sectref{SecDirectAddress}, as well as how free pronominal elements appear in \sectref{SecPronouns}. Quantifiers and numerals appear in \sectref{SecQuantifiers}. In \sectref{SecDeixis}, I give an overview of the postpositions that occur in Mandan in \sectref{SecPostpositions}.  I conclude this chapter by elaborating on how deictic elements express when or where something is, as well as verbs associated with deixis.

\section{Noun derivation}\label{SecNounDeriv}

There is often discussion at annual Siouan and Caddoan Languages Conferences regarding the status of nouns in Siouan languages. In many cases, what we might consider a noun in a Siouan language is ultimately derived from a verb or even is really a relative clause. Mandan is no exception to this generalization, as a large number of items that are treated as nouns in the syntax ultimately stem from some kind of nominalized verbal construction.

Putting more complicated morphology aside for the moment, there are certainly many lexical items in Mandan that are uncontroversially nouns. Such nouns synchronically consist of an unanalyzable root, typically of one or two syllables, as we see in (\ref{monosyllabicroots}) and (\ref{disyllabicroots}), respectively. Stems longer than two syllables are usually the result of some historical compounding or combination of multiple morphological items, but no longer have any analyzable internal structure to modern speakers. We can see examples of such stems in (\ref{trisyllabicroots}) below.

\begin{exe}

\item\label{monosyllabicroots} Monosyllabic nominal roots in Mandan

    \begin{xlist}
    
    \item \textit{á'p} `leaf'
    
    \item \textit{hą́p} `day'
    
    \item \textit{íi} `blood'
    
    \item \textit{múx} `cellar, basement, cache'
    
    \item \textit{xą́h} `grass, hay'
    
    \end{xlist}
    
\item\label{disyllabicroots} Disyllabic nominal roots in Mandan

    \begin{xlist}
    
    \item \textit{istų́h} `night'
    
    \item \textit{kahót} `prairie'
    
    \item \textit{kíipsąą} `painted turtle'
    
    \item \textit{náaku} `road, path, way'
    
    \item \textit{pą́ąhį'} `porcupine'
    
    \end{xlist}
    
\item\label{trisyllabicroots} Trisyllabic nominal roots in Mandan

    \begin{xlist}
    
    \item \textit{hą́axuraa} `bat (flying mammal)'
    
    \item \textit{Ihą́tu} `Lakota, Yankton Sioux'
    
    \item \textit{sháaxkuxke} `peppermint'
    
    \item \textit{wéehinuu} `spring, springtime'
    
    \item \textit{wíiratąą} `enemy'
    
    \end{xlist}

\end{exe}

Like other Siouan languages, the two principal manners of adding new items to the Mandan lexicon is through morphological derivation and compounding. Morphological derivation is treated in \sectref{SubSecNounDeriv}, and compounding is addressed in \sectref{SubSecCompounding}.

\subsection{Derivational morphology}\label{SubSecNounDeriv}

The most common method of deriving a noun is through the addition of morphology before or after its root. This subsection elaborates upon the various morphological items involved in this process.

\subsubsection{Prenominal prefixes}\label{SubSubSecPrenominalDomain}

The majority of nominal derivation takes place prefixally in Mandan. There is an overlap between these prefixes and those commonly employed in verbal morphology. Possessive prefixes have been omitted here, as they are discussed in greater depth in \sectref{SecPossession}. A list of derivational prefixes appears in (\ref{ListofPrenomMorph}) below.

\begin{exe}
\item\label{ListofPrenomMorph} List of prenominal morphology

\begin{xlist}
\item /i-/ instrumental nominalizer
\item /ka-/ agentive marker
\item /ko-/ relativizer
\item /o-\textsubscript{1}/ irrealis nominalizer
\item /o-\textsubscript{2}/ locative nominalizer
\item /waa-/ nominalizer
\end{xlist}
\end{exe}

The information that follows provides examples of how these particular morphological items are used in the corpus. Relevant morphology appears in bold in the data below.

\subsubsubsection{Instrumental nominalizer \textit{i}-}\label{SubSubSubNomInstrumental}

The instrumental nominalizer \textit{i}- is homophonous with the instrumental preverb \textit{i}- previously discussed in \sectref{SubParaInstrumental}. Both instrumental prefixes carry similar semantics, as they indicate that there is some argument that is being used to do an action. The main difference between these two \textit{i}- prefixes is that this nominalizer acts to form a kind of relative clause that involves an instrument.\footnote{Under the analysis that this nominalizer is really just another use of the instrumental preverb, the morphological glossing for such constructions will continue to use \textsc{pv} for preverb instead of making a separate abbreviation for its use as a nominalizer. These constructions are more accurately analyzed as headless relative clauses, i.e., the instrumental nominalizer + X is really `the thing with which one Xes or is Xed.'} We can see examples of nominals formed with the instrumental nominalizer in (\ref{INSnominals}) below.

\begin{exe}

\item\label{INSnominals} Examples of the instrumental nominalizer \textit{i}-

    \begin{xlist}
    
    \item\label{INSnominals1} \glll {\textbf{í}mashut}\\
        \textbf{i}-wąshut\\
        \textbf{pv.ins}-\textnormal{clothe}\\
        \glt `clothing'
    
    \item\label{INSnominals2} \glll \textbf{í}kate\\
    \textbf{i}-ka-te\\
    \textbf{pv.ins}-ins.frce-\textnormal{pound}\\
    \glt `cherry grinder pestle'
    
    % \item\label{INSnominals3} \glll t\textbf{íi}kakishka\\
    %     tV-\textbf{i}-ka-kish=ka\\
    %     al-\textbf{pv.ins}-ins.frce-\textnormal{brush}=hab\\
    %     \glt `[his/her] broom'
    
    \item\label{INSnominals3} \glll \textbf{í}'ah\\
        \textbf{i}-ah\\
        \textbf{pv.ins}-\textnormal{be.covered}\\
        \glt `skin, shell'
    
    \item\label{INSnominals4} \glll \textbf{í}hįįka\\
    \textbf{i}-hįį=ka\\
    \textbf{pv.ins}-\textnormal{drink}=hab\\
    \glt `pipe'\footnote{Mandan speakers `drink' smoke. There is no separate verb for the action of smoking tobacco.}
    
    \item\label{INSnominals5} \glll \textbf{í}katehere\\
    \textbf{i}-ka-te\#hrE\\
    \textbf{pv.ins}-ins.frce-\textnormal{pound}\#caus\\
    \glt `pemmican'
    
    \item\label{INSnominals6} \glll wa'\textbf{í}pteh\\
    wa-\textbf{i}-ptEh\\
    unsp-\textbf{pv.ins}-\textnormal{run}\\
    \glt `automobile'
    
    \end{xlist}

\end{exe}

In all of the examples above, we see that the noun is the result of some relativized clause. For example, the data in (\ref{INSnominals1}) is commonly just glossed as `clothing' by Mandan speakers, but its meaning is closer to `the thing with which one is clothed.' The other examples can likewise be broken down in this manner, i.e., (\ref{INSnominals2}) is `the thing with which one grinds', (\ref{INSnominals4}) is `the thing with which one smokes',  and so on. Many noun in Mandan are derived through this instrumental nominalizer, and it is fully productive. Novel words, such as the word for `automobile' [lit. `the thing that runs with something'] in (\ref{INSnominals6}), commonly make use of \textit{i}-.

Mandan is not unique in the Siouan family for employing this preverb in this way. \citet[48]{graczyk2007} describes the instrumental in Crow as having a similar behavior, we likewise see it in Lakota \citep[440]{ullrichblackbear2016}, and there are examples of nominalizations with \textit{i}- in Tutelo as well \citep[146f]{oliverio1997}.

\subsubsubsection{Agentive marker \textit{ka}-}\label{SubSubSubAgentive}

The agentive marker \textit{ka}- is used with verbs to indicate that someone is the doer of an action. This prefix is seen almost exclusively on active verbs, where the addition of \textit{ka}- is roughly equivalent to the English -\textit{er} suffix. We can see examples of this prefix in (\ref{AgentKA}) below.

\begin{exe}

\item\label{AgentKA} Examples of the agentive marker \textit{ka}-

    \begin{xlist}
    
    \item\label{AgentKA1} \glll \textbf{ka}róore\\
    \textbf{ka}-roo=E\\
    \textbf{agt}-\textnormal{speak}=sv\\
    \glt `a speaker' \citep[15]{kennard1936}
    
    \item\label{AgentKA2} \glll \textbf{ka}róokas\\
    \textbf{ka}-roo=ka=s\\
    \textbf{agt}-\textnormal{speak}=hab=def\\
    \glt `the speaker' \citep[15]{kennard1936}   
    
    \item\label{AgentKA3} \glll \textbf{ka}kíkų'teka\\
    \textbf{ka}-kikų'tE=ka\\
    \textbf{agt}-\textnormal{help}=hab\\
    \glt `helper' \citep[123]{hollow1973a}
    
	\item\label{AgentKA4} \glll \textbf{ka}{ró}kahashka\\
	\textbf{ka}-{ro}-ka-hash=ka\\
	\textbf{agt}-{1s.pl}-ins.frce-\textnormal{slaughter}=hab\\
	\glt `the one who slaughters us' [lit. `our slaughterer']\citep[146]{hollow1973a}
	
	\item\label{AgentKA5} \glll \textbf{ka}tánishkereka'na\\
	\textbf{ka}-ta-rįsh=krE\#ka'=rą\\
	\textbf{agt}-al-\textnormal{medicine}=3pl\#\textnormal{possess}=top\\
	\glt `owner of medicine' \citep[15]{kennard1936}
	
	\item\label{AgentKA6} \glll \textbf{ka}mánapkeres\\
	\textbf{ka}-wa-rąp=krE=s\\
	\textbf{agt}-unsp-\textnormal{dance}=3pl=def\\
	\glt `the dancers' \citep[15]{kennard1936}
	
	\item\label{AgentKA7} \glll \textbf{k}áakanake\\
	\textbf{k}-aaki\#rąk=E\\
	\textbf{agt}-\textnormal{be.on.top}\#sit.pos=sv\\
	\glt `a rider' \citep[15]{kennard1936}
	
	\item\label{AgentKA8} \glll \textbf{ka}'í'ųųtkas\\
	\textbf{ka}-i-ųųt=ka=s\\
	\textbf{agt}-pv.ord-\textnormal{be.first}=hab=def\\
	\glt `the head, leader' \citep[249]{trechter2012}
    
    \end{xlist}

\end{exe}

Many of these agentive constructions also feature the habitual marker =\textit{ka}. We can see a minimal pair in (\ref{AgentKA1}) and (\ref{AgentKA2}), where the first example is a very generalized `speaker', i.e., someone who is speaking. However, the use of the habitual =\textit{ka} in (\ref{AgentKA2}) indicates that this is an action this agent does habitually or regularly. \citet[15]{kennard1936} even qualifies this difference by adding that the data in (\ref{AgentKA2}) refers to ``one who holds that office.'' Also note that \textit{ka}- has an allophone /k-/ before stems beginning with long vowels, which we see in (\ref{AgentKA7}). This syncope does not occur when added to stems beginning with short vowel, as we see in (\ref{AgentKA8}).

\citet[441]{hollow1970} states that all instances of agentive \textit{ka}- must be accompanied by the habitual =\textit{ka}, which he calls a `nominal suffix.' The data in \citet{kennard1934, kennard1936}, as well as data he recorded in his own narratives later on \citep{hollow1973a,hollow1973b} show that most instances of \textit{ka}- do coincide with =\textit{ka}, but it is not a requirement, i.e., this is not a circumfix; it is a prefix and a possible enclitic that depends on the context of how regularly one does that action.

The origin of the agentive \textit{ka}- seems to align with the Proto-Siouan distal pronoun *ka `that one far away, there, then' It likely functioned as a free element in earlier stages of development, eventually becoming associated with describing `that one' person doing an action. Later, it became reanalyzed as being verbal morphology, only to eventually be associated with nominal morphology. %We see a reflex of this distal *ka in the Mandan conjunction \textit{káni}, which serves to connect two utterances in a manner similar to English phrases like `and after that' or `and then.'

\subsubsubsection{Relativizer \textit{ko}-}\label{SubSubSubRelativizerKO}

The relativizer has been described previously in \sectref{SubsubsecRelativized}, where its use in verbal morphology is on display. While the relativizer \textit{ko}- certainly co-occurs on verbs, it is treated syntactically as a noun and can take nominal morphology as well. It is common in Mandan to see the definite article on nouns bearing \textit{ko}-, and words with \textit{ko}- can also acts as nouns within a postpositional phrase. We can see examples of this verb-\textit{cum}-noun status in the examples in (\ref{NOMko}) below.

\begin{exe}

\item\label{NOMko} Examples of the relativizer \textit{ko}-

\begin{xlist}

\item\label{NOMko1} \glll \textbf{ko}'ųųtaahąkt ó'harani\\
    \textbf{ko}-ųųtaahąk=t o'\#hrE=rį\\
    \textbf{rel}-\textnormal{east}=loc \textnormal{be}\#caus=ss\\
    \glt `from one in the east' \citep[109]{hollow1973a}

\item\label{NOMko2} \glll \textbf{ko}túkerekas\\
    \textbf{ko}-tu=krE=ka=s\\
    \textbf{rel}-\textnormal{be.some}=3pl=hab=def\\
    \glt `the ones that usually have some' \citep[84]{hollow1973b}
    
\item\label{NOMko3} \glll \textbf{ko}márootkis\\
    \textbf{ko}-wą-rootki=s\\
    \textbf{rel}-1s-\textnormal{hit}=def\\
    \glt `the one who hit me' \citep[452]{hollow1970}

\item\label{NOMko4} \glll \textbf{ko}xamáhsįh\\
    \textbf{ko}-xwąh-sįh\\
    \textbf{rel}-\textnormal{be.small}-ints\\
    \glt `the really young one' \citep[31]{trechter2012}

\end{xlist}

\end{exe}

The examples above feature morphology we associate with verbs, such as in (\ref{NOMko3}) `the one who hit me', where there is a first person stative marker. However, there is also a definite marker, showing that there is a specific person to whom the speaker is referring. The data in (\ref{NOMko1}) provides further support for \textit{ko}- behaving as a nominalizer as the relativizer has converted a postpositional phrase \textit{ų́ųtaahąkt} `in the east' into a noun, which then can be part of another postpositional phrase that is headed by \textit{ó'harani} `from'. The fact that \textit{ko'ų́ųtaahąkt} is able to be the complement of a postpositional phrase is evidence that it is not a verb anymore, but a nominal element that is able to be selected as a complement of a postposition.

\subsubsubsection{Locative nominalizer: \textit{o-}\textsubscript{1}}
\label{SubSubSubLocNom}

Earlier in \sectref{SubParaLocative}, I describe the use of the locative preverb. Much as we see in \sectref{SubSubSubNomInstrumental} about the relationship between the instrumental preverb and the instrumental nominalizer in this chapter, so too can we connect the locative preverb to the locative nominalizer. Nominalized constructions involving \textit{o}- act as headless relative clauses that are treated as nouns, as we can see in (\ref{LocNomEx}) below.

\begin{exe}

\item\label{LocNomEx} Examples of the locative nominalizer \textit{o}-

    \begin{xlist}
    
    \item\label{LocNomEx1} \glll \textbf{ó}raxkaps\\
    \textbf{o}-ra-xkap=s\\
    \textbf{pv.loc}-ins.foot-\textnormal{be.stuck}=def\\
    \glt `where he was stuck' \citep[46]{hollow1973a}
    
    \item\label{LocNomEx2} \glll \textbf{ó}kiruskopka\\
    \textbf{o}-k-ru-skop=ka\\
    \textbf{pv.loc}-mid-ins.hand-\textnormal{be.crooked}=hab\\
    \glt `the place where the river bends' \citep[47]{hollow1973a}
    
    \item\label{LocNomEx3} \glll \textbf{ó}taht\\
    \textbf{o}-tah=t\\
    \textbf{pv.loc}-\textnormal{be.cool}=loc\\
    \glt `in the place where it is cool' \citep[53]{hollow1973a}   
    
    \item\label{LocNomEx4} \glll ta'\textbf{ó}sukanasha'shka\\
    ta-\textbf{o}-suk=rąsh=a'shka\\
    al-\textbf{pv.loc}-\textnormal{exit}=att=psbl\\
    \glt `his place where he can come out' \citep[98]{hollow1973a}
    
    \item\label{LocNomEx5} \glll \textbf{ó}maniitaa\\
    \textbf{o}-wa-rįį=taa\\
    \textbf{pv.loc}-1a-\textnormal{walk}=loc\\
    \glt `at the place where I walk' \citep[114]{hollow1973a}
    
    \item\label{LocNomEx6} \glll \textbf{ó}ranuunihinits\\
    \textbf{o}-ra-ruurįh=rįt=s\\
    \textbf{pv.loc}-2a-\textnormal{be.there}.pl.aux=2pl=def\\
    \glt `the place where you all are' \citep[130]{hollow1973a}
    
    \end{xlist}

\end{exe}

Translations of words involving this nominalizer are often translated as `where X happens' or `the place where so-and-so Xed.' We can see examples of this above. The data in (\ref{LocNomEx}) have the same behaviors as those previously seen for the instrumental nominalizer \textit{i}-, where any verbal morphology from the intended proposition has acquired nominal morphology on the edges of the word. For example, we can see in (\ref{LocNomEx1}) and (\ref{LocNomEx6}) that the speaker is indicating a specific place, so there is overt use of the definite article =\textit{s} on this word, indicating that it is being treated as a noun. Likewise, we see postpositions applied to the nominalized element in (\ref{LocNomEx3}) and (\ref{LocNomEx5}), which is something we expect to be possible for nouns and not verbs.

\subsubsubsection{Irrealis nominalizer: \textit{o}-\textsubscript{2}}\label{SubSubSubIrrealisNom}

The irrealis preverb \textit{o}- is described in detail in \sectref{ParaIrrealis}, and like other preverbs, there is a nominalizer counterpart to it. It is something of a catch-all nominalizer, as it is used in a wide variety of situations. This irrealis nominalizer is a very common element in the corpus. Certain constructions involving \textit{o}- make reference to a time (i.e., `the time when X happens or someone Xes'), but it appears to be a more generalized nominalizer, i.e., `that which Xes.' We can see examples of \textit{o}- in the data in (\ref{NOMirrealisEx}) below.

\begin{exe}

\item\label{NOMirrealisEx} Examples of the irrealis nominalizer \textit{o}-

    \begin{xlist}
    
    \item\label{NOMirrealisEx1} \glll \textbf{ó}kso\\
    \textbf{o}-kso\\
    \textbf{pv.irr}-\textnormal{spit}\\
    \glt `saliva' \citep[121]{hollow1970}
    
    \item\label{NOMirrealisEx2} \glll \textbf{ó}pųųse\\
    \textbf{o}-pųųs=E\\
    \textbf{pv.irr}-\textnormal{be.striped}=sv\\
    \glt `stripes, spots, pinto horse' \citep[160]{hollow1970}
    
    \item\label{NOMirrealisEx3} \glll wáa'\textbf{o}naaka\\
    waa-\textbf{o}-rąąka\\
    nom-\textbf{pv.irr}-\textnormal{be.new}\\
    \glt `something new, fresh' \citep[165]{hollow1970}
    
    \item\label{NOMirrealisEx4} \glll \textbf{ó}shikere\\
    \textbf{o}-shi=krE\\
    \textbf{pv.irr}-\textnormal{be.good}=3pl\\
    \glt `the best ones' \citep[252]{hollow1973b}
    
    \item\label{NOMirrealisEx5} \glll \textbf{ó}'ihekeshka\\
    \textbf{o}-i-hek-eshka\\
    \textbf{pv.irr}-pv.ins-\textnormal{know}-smlt\\
    \glt `the one that knew' \citep[257]{hollow1973b}
    
    \item\label{NOMirrealisEx6} \glll  Manápushek \textbf{Ó}ra'tak Mínaks\\
    wrą\#pushek \textbf{o}-ra'-tak wįrąk=s\\
    \textnormal{tree}\#\textnormal{juneberry} \textbf{pv.irr}-ins.heat-\textnormal{be.ripe} \textnormal{orb}=def\\
    \glt `June' [lit. `the month when juneberries are ripe'] \citep[161]{hollow1970}
    
    \end{xlist}

\end{exe}

This nominalizer has a wide range of uses. It is not restricted to temporal readings, like the one we see in (\ref{NOMirrealisEx6}). The \textit{o}- is able to create nouns out of verbs like in (\ref{NOMirrealisEx1}) where there is no locative or instrumental reading, i.e., the noun `saliva' is literally `what one spits.' We see that \textit{o}- can also co-occur with another nominalizer, the \textit{waa}-, which we see in (\ref{NOMirrealisEx3}). The semantics of this are not exactly redundant, as the word `something new' is better translated as `something that would be new' or `something that will be new.' See \sectref{SubSubSubNomWaa} for more information on the \textit{waa}- nominalizer.

This irrealis nominalizer is cognate with the Hidatsa relative marker \textit{aru}-, which has an allomorph \textit{oo}- before stems beginning with /ɾ/. Assuming that Mandan shares a more recent common ancestor with Hidatsa and Crow, it is likely the case that Hidatsa preserves an older system of nominalization that involves the irrealis marker, and that the Mandan irrealis marker has been influenced by the locative preverb \textit{o}-, collapsing the length distinction between the two and creating homophony. Mandan also shares a cognate with Hidatsa in the form of the relativizer \textit{ko}-, which is \textit{agu}- in Hidatsa. Hidatsa has yet another allomorph \textit{oo}- before /k/-initial stems, once again connecting Mandan \textit{o}- to being a more general relativizer that also results in the ensuing relativized clause being treated as a noun in the syntax.

\subsubsubsection{Nominalizer: \textit{waa}-}\label{SubSubSubNomWaa}

The nominalizer \textit{waa}- is related to the Proto-Siouan nominalizer *waa-. As previously discussed in \sectref{SubsubsUnspecifiedArgument}, this \textit{waa}- does have a non-prefixal form that acts as an unspecified argument marker of sorts, often being translated as `something', `someone', or `somewhere'. It seems to originate as an unbound formative, likely being a free pronominal in Proto-Siouan or Pre-Proto-Siouan. When used as a nominalizer, it can appear with either stative or active verbs, and it often behaves somewhat like a subject, i.e., `something that Xes' or `something with the quality X.' Examples of the nominalizer appear in (\ref{NOMexampleWAA}) below.

\begin{exe}

\item\label{NOMexampleWAA} Examples of the nominalizer \textit{waa}-

    \begin{xlist}
    
    \item\label{NOMexampleWAA1} \glll \textbf{wáa}xikxte\\
    \textbf{waa}-xik-xtE\\
    \textbf{nom}-\textnormal{be.bad}-aug\\
    \glt `something really bad' \citep[46]{hollow1973a}
    
    \item\label{NOMexampleWAA2} \glll \textbf{wáa}shooteena\\
    \textbf{waa}-shoot=ee=rą\\
    \textbf{nom}-\textnormal{be.white}=dem.dist=top\\
    \glt `something white' \citep[106]{hollow1973a}
    
    \item\label{NOMexampleWAA3} \glll \textbf{wáa}'isek\\
    \textbf{waa}-i-sek\\
    \textbf{nom}-pv.ins-\textnormal{do}\\
    \glt `a job' \citep[229]{trechter2012}
    
    \item\label{NOMexampleWAA4} \glll pta\textbf{wáa}'irokes\\
    p-ta-\textbf{waa}-i-roke=s\\
    1poss-al-\textbf{nom}-pv.ins-\textnormal{contain}=def\\
    \glt `my container' \citep[118]{trechter2012}
    
    \item\label{NOMexampleWAA5} \glll ta\textbf{wáa}'irukiriihs\\
    ta-\textbf{waa}-i-ru-kriih=s\\
    al-\textbf{nom}-pv.ins-ins.hand-\textnormal{be.lined.up}=def\\
    \glt `his staff' \citep[7]{hollow1973a}
    
    \item\label{NOMexampleWAA6} \glll \textbf{wíi}kapus\\
    \textbf{wV}-i-ka-pus\\
    \textbf{nom}-pv.ins-ins.frce-\textnormal{be.lined.up}\\
    \glt `pencil, pen' \citep[159]{hollow1970}
    
    \item\label{NOMexampleWAA7} \glll \textbf{wóo}pashe\\
    \textbf{wV}-o-pa-shE\\
    \textbf{nom}-pv.irr-ins.push-\textnormal{grasp}\\
    \glt `offering of corn' \citep[229]{hollow1970}
    
    \item\label{NOMexampleWAA8} \glll \textbf{wáa}'atxi'hs\\
    \textbf{waa}-at\#xi'h=s\\
    \textbf{nom}-\textnormal{father}\#\textnormal{be.old}=def\\
    \glt `the President' \citep[317]{hollow1970}
    
    \item\label{NOMexampleWAA9} \glll ósu $\sim$ \textbf{wóo}su\\
    o-su ~ wV-o-su\\
    pv.loc-\textnormal{be.a.hole} ~ \textbf{nom}-pv.loc-\textnormal{be.a.hole}\\
    \glt `hole' $\sim$ `post hole' \citep[131]{hollow1970}
    
    \end{xlist}

\end{exe}

Before preverbs, the nominalizer \textit{waa}- has an allomorph /wV-/, which has the effect of lengthening short vowels. There are no recorded instances of \textit{waa}- appearing before the transitivizer preverb \textit{aa}-, so it is unclear if \textit{waa}- would have an allomorph /w-/, similar to what have have seen previously in the description of the unspecified argument marker in the previous chapter.

In each of the words above in (\ref{NOMexampleWAA}), each word can be parsed as literally meaning `something does does the active verb' or `something that has this quality.' We can see this in (\ref{NOMexampleWAA4}), `my container', where the Mandan word means `my something with which one stores things.'

In addition to being used with verbs, \textit{waa}- can combine with other nouns. We see this in (\ref{NOMexampleWAA8}), where the word \textit{átxi'h} `grandfather' [lit. `old father'] takes the nominalizer and the definite article. This word is translated as `the President [of the United States]' in the context of \citeapos{hollow1970} dictionary, but this term is also used as a metonym for the United States government.\footnote{Other Indigenous people have a similar connection in their language between the term for `grandfather' and the government, specifically the United States government (cf. Lakota \textit{tȟuŋkášilayapi} `U.S. government, President of the U.S.' [lit. `grandfathers, ancestors']).} Thus, \textit{waa}- is not purely a nominalizer in the sense that it turns a verb or clause into a noun, but it can also turn nouns into nouns that often have a more abstract connection to their stems, as we see in the case of `grandfather' being the root of `government'. We see a similar process in (\ref{NOMexampleWAA9}), where the word for `hole' is \textit{ósu}, but a larger, deeper post hole is \textit{wóosu}, consisting of \textit{waa}- plus \textit{ósu}. Thus, while this nominalizer can turn a non-abtract noun into an abstract noun, the resulting noun does not inherently need to be abstract.


\subsubsection{Postnominal suffixes}\label{SubSubSecPostnominalDomain}

Much like the prenominal domain, the morphology found in the postnominal domain in Mandan has a considerable overlap with that of the postverbal domain. There are several suffixes and enclitics that appear on nouns that play similar roles on nouns. Most of the suffixes discussed in \sectref{SecSuffixField} are really nominal suffixes rather than verbal suffixes. Those suffixes are restated in (\ref{ListofPostnominalElements}) below, plus some additions that are exclusively found on nouns and other nominal elements.

\begin{exe}

\item\label{ListofPostnominalElements} List of postnominal morphology

    \begin{xlist}
\item \textit{-aaki} collective 1 (\textsc{coll})
\item \textit{-esh}	similitive 1 (\textsc{smlt})
\item \textit{-eshka} similitive 2 (\textsc{smlt})
\item \textit{-oshka} emphatic (\textsc{emph})
\item \textit{-sha}	collective 2 (\textsc{coll})
\item \textit{-sįh}	intensifier (\textsc{ints})
\item \textit{-xte}	augmentative (\textsc{aug})
    \end{xlist}

\end{exe}



\subsubsubsection{Collective suffixes: \textit{-aaki}, \textit{-sha}, \textit{-shka}}\label{suffixcollective}

There are really two competing collective suffixes in Mandan. Each of these suffixes is very restricted in where they can appear. These restrictions are described in \sectref{suffixcollective1} and \sectref{suffixcollective2} below.

\subsubsubsubsection{Collective 1: \textit{-aaki}}\label{suffixcollective1}

The first collective suffix \textit{-aaki} is attested with a single stem: \textit{numá'k} `person, man.' No other nouns permit the addition of this suffix. Its purpose is to indicate turn this noun from meaning an individual person or even a group of individuals into a collective. We can see examples of \textit{-aaki} in the data in (\ref{coll1}) below.

\begin{exe}

\item\label{coll1} Collective 1 examples

	\begin{xlist}
	
	\item \glll numá'k\textbf{aaki} máamikoomako'sh\\
	ruwą'k-\textbf{aaki} waa-wįk=oowąk=o'sh\\
	\textnormal{person}-\textbf{coll} \textnormal{some}-\textnormal{be.none}=narr=ind.m\\
	\glt `there were no people' \citep[178]{hollow1973a}
	
	\item \glll áakinuma'k\textbf{aaki}\\
	aaki\#ruwą'k-\textbf{aaki}\\
	\textnormal{be.above}\#\textnormal{person}-\textbf{coll}\\
	\glt `Native American(s)' \citep[220]{trechter2012b}\footnote{This word for `Native American' is a contraction of the term \textit{máa'ąk áaki numá'kaaki} `people on the land.' This term is cognate with the Hidatsa term for Native American \textit{(awa')áagaaruxbaaga} `people on the land', and is similar to terms for indigenous peoples found in nearby languages, e.g., Lakota \textit{ikčé wičáša} `ordinary people.'}
	
	\item \glll ómahą numá'k\textbf{aaki}\\
	owąhą ruwą'k-\textbf{aaki}\\
	\textnormal{Omaha} \textnormal{person}-\textbf{coll}\\
	\glt `Omaha tribe' \citep[431]{hollow1970}
		
	\end{xlist}

\end{exe}

This suffix \textit{-aaki} originates from the stative verb \textit{áaki} `be above.' This verb appears in compounds where it serves to intensify another stative verb, but this pattern does not seem to be productive in modern Mandan. Furthermore, \textit{áaki} serves as the initial element in all such compounds. We see an example of these compounds in (\ref{AAKIcompound}) below.

\begin{exe}
\item\label{AAKIcompound} Compound with \textit{áaki}

\glll	áakana'ro'sh\\
	aaki\#rą'=o'sh\\
	\textnormal{be.above}\#\textnormal{ache}=ind.m\\
	\glt `he is sick' \citep[168]{hollow1970}

\end{exe}

It is possible that the collective \textit{-aaki} could have been used metaphorically to describe a large number in the past. This suffix is not otherwise productive in modern Mandan.

\subsubsubsubsection{Collective 2: \textit{-sha}}\label{suffixcollective2}

This collective suffix comes from the Proto-Siouan collective *-sa. This suffix is restricted to numerals, as we see in (\ref{ExSHA}). It is often accompanied by an English translation along the lines of `X of them' or `all X of them', the latter especially being the translation used with the collective \textit{-sha} co-occurs with the ordinal preverb \textit{i}, as we see in (\ref{ExSHA2}) below.

\begin{exe}
\item\label{ExSHA} Examples of collective \textit{-sha}

	\begin{xlist}
	
	\item \glll núup\textbf{sha}\\
	rųųp-\textbf{sha}\\
	\textnormal{two}-\textbf{coll}\\
	\glt `both of them, two of them, twins' \citep[481]{hollow1970}
	
	\item \glll kixų́ųh\textbf{sha}\\
	kixųųh-\textbf{sha}\\
	\textnormal{five}-\textbf{coll}\\
	\glt `five of them' \citep[481]{hollow1970}
	
	\item \glll náamini\textbf{sha}\\
	raawrį-\textbf{sha}\\
	\textnormal{three}-\textbf{coll}\\
	\glt `three of them' \citep[481]{hollow1970}
	
	\item \glll nunáamini\textbf{sha} shí'sh\\
	rų-raawrį-\textbf{sha} shi=o'sh\\
	1a.pl-\textnormal{three}-\textbf{coll} \textnormal{be.good}=ind.m\\
	\glt `the three of us are good' \citep[481]{hollow1970}
	
	\end{xlist}

\end{exe}

%This collective suffix often co-occurs with the ordinal preverb \textit{i-}.

\begin{exe}
\item\label{ExSHA2} Collective suffix with ordinal preverb

	\begin{xlist}
	
	\item \glll \textbf{í}toop\textbf{sha} íhaa'aakit ~ ~ ~ ~ ~ ~ ~ ~ ~ ~ ~ ~ ~ ~ keréehkereroomako'sh\\
	\textbf{i}-toop-\textbf{sha} i-haa\#aaki=t ~ ~ ~ ~ ~ ~ ~ ~ ~ ~ ~ ~ ~ ~  krEEh=krE=oowąk=o'sh\\
	\textbf{pv.ord}-\textnormal{four}-\textbf{coll} pv.dir-\textnormal{cloud}\#\textnormal{be.above}=dir ~ ~ ~ ~ ~ ~ ~ ~ ~ ~ ~ ~ ~ ~ \textnormal{go.back.there}=3pl=narr=ind.m\\
	\glt `all four of them returned to heaven' \citep[175]{hollow1973a}
	
	\item \glll \textbf{í}nuup\textbf{sha} tasúke túkerek\\
	\textbf{i}-rųųp-\textbf{sha} ta-suk=E tu=krE=ak\\
	\textbf{pv.ord}-\textnormal{two}-\textbf{coll} 3poss.al-\textnormal{child}=sv \textnormal{be.some}=3pl=ds\\
	\glt `both of them had children' \citep[111]{hollow1973a}
	
	\item \glll \textbf{í}nuup\textbf{sha} ráse ísekwahere'sh\\
	\textbf{i}-rųųp-\textbf{sha} ras=E i-sek\#wa-hrE=o'sh\\
	\textbf{pv.ord}-\textnormal{two}-\textbf{coll} \textnormal{name}=sv pv.ins-\textnormal{make}\#1a-caus=ind.m\\
	\glt `I gave both of them their names' \citep[64]{hollow1973a} 
	
	\end{xlist}

\end{exe}

The collective \textit{-sha} is sometimes accompanied by the suffix \textit{-shka}. This suffix serves to emphasize the collective reading. This suffix comes from the Proto-Siouan suffix *-ska, which historically a similitive marker. Traces of this *-ska can be seen in other productive suffixes, like the emphatic \textit{-oshka} or the similitive suffix \textit{-eshka}, or on the interrogative word \textit{tashká} `how', where \textit{tá} is `what'. The collective suffix can appear with this emphatic \textit{-shka}, as we see in (\ref{ExSHKA1}) below.

\begin{exe}

\item\label{ExSHKA1} Examples of intensified collective suffixes

	\begin{xlist}
	
	\item \glll \textbf{í}nuup\textbf{shashka}na hų́pe ké'ka'rak ~ ~ ~ ~ ~ ~ ~ ~ kú'kerek\\
	\textbf{i}-rųųp-\textbf{sha-shka}=rą hųp=E ke'\#ka'=ak ~ ~ ~ ~ ~ ~ ~ ~  ku'=krE=ak\\
	pv.ord-\textnormal{two}-\textbf{coll-ints.coll}=top \textnormal{shoe}=sv \textnormal{keep}\#\textnormal{have}=ds ~ ~ ~ ~ ~ ~ ~ ~ \textnormal{give}=3pl=ds\\
	\glt `both of them kept shoes for him' \citep[109]{hollow1973a}
	
	\item \glll \textbf{í}toop\textbf{shashka} Máarepaaxu í'ų'taa ~  minípashų'ni réehkereroomako'sh\\
	\textbf{i}-toop-\textbf{sha-shka} wąą=E\#paaxu i-ų'=taa ~  wrį\#pa-shų'=rį rEEh=krE=oowąk=o'sh\\
	\textbf{pv.ord}-\textnormal{four}-\textbf{coll-ints.coll} \textnormal{eagle}=sv\#\textnormal{nose} pv.dir-\textnormal{be.closer}=loc ~  \textnormal{water}\#ins.push-\textnormal{thresh.with.feet}=ss \textnormal{go.there}=3pl=narr=ind.m\\
	\glt `the four of them were swimming toward Eagle Nose' \citep[295]{hollow1973b}
	
	\item \glll	óo ó'harani numá'kaaki hų́keres sheréekini ~  ~ tóop\textbf{shashka} kaháshkereroomako'sh\\
	oo o'\#hrE=rį ruwą'k-aaki hų=krE=s shreek=rį ~ ~   toop-\textbf{sha-shka} ka-hash=krE=oowąk=o'sh\\
	dem.mid \textnormal{be}\#caus=ss \textnormal{person}-coll \textnormal{many}=3pl=def \textnormal{war.whoop}=ss ~ ~  \textnormal{four}-\textbf{coll}-\textbf{ints.coll} ins.frce-\textnormal{be.disintegrated}=3pl=narr=ind.m\\
	\glt `from there, the whole lot of people war whooped and slaughtered all four of them' \citep[255]{hollow1973b}
	
	\end{xlist}

\end{exe}

The intensified collective /-sha-shka/ can appear with or without the ordinal preverb \textit{i-}. The presence of the preverb does not seem to alter the reading, though it seems to serve to intensify the collective meaning, with the English quantifier `all' often being added to the English translation given by consultants.

\subsubsubsection{Emphatic: -\textit{oshka}}\label{emphaticsuffix}

The emphatic suffix -\textit{oshka} is historically derived from the free adverbial \textit{óshka} `even', which itself seems to be composed of \textit{ó'} `be' plus the Proto-Siouan similitive *ska. It is added to nouns to indicate that there is some kind of emphasis on a noun. This differs from the topicalizer enclitic =\textit{na} in that this emphasis marker -\textit{oshka} is most often associated with information that conveys a contrast or surprise, while the =\textit{na} topicalizer serves to bring some entity into the foreground or to remind the speaker of a previous topic. We can see examples of this suffix in (\ref{EMPHsuffix}) below.

\begin{exe}

\item\label{EMPHsuffix} Examples of the emphatic suffix -\textit{oshka}

    \begin{xlist}
    
    \item\label{EMPHsuffix1} \glll kí'hini tewé'\textbf{oshka} wáaka'kina'nix\\
    ki'h=rį t-we-\textbf{oshka} waa-ka'\#kirą'=rįx\\
    \textnormal{arrive.back.here}=ss wh-indf-\textbf{emph} neg-\textnormal{possess}\#\textnormal{tell}=neg\\
    \glt `she got back and did not tell anyone' \citep[79]{hollow1973a}
    
    \item\label{EMPHsuffix2} \glll hi'\textbf{óshka} ímikshka\\
    hi-\textbf{oshka} i-wįk-shka\\
    \textnormal{tooth}-\textbf{emph} pv.ord-\textnormal{be.none}-ints.coll\\
    \glt `he had hardly any teeth' \citep[181]{hollow1973a}
    
    \item\label{EMPHsuffix3} \glll máaxtik\textbf{oshka} wáateerehererootiki, ~ ~ ~ ~ ~ ~ ~ ~ ~ ~ ~ ~ ~ ~ ~ rá'sitnuharani nurúusto'sh\\
    wąąxtik-\textbf{oshka} waa-tee\#re-hrE=ooti=ki ~ ~ ~ ~ ~ ~ ~ ~ ~ ~ ~ ~ ~ ~ ~ ra'-st\#rų-hrE=rį rų-ruut=t=o'sh\\
    \textnormal{jackrabbit}-\textbf{emph} \textnormal{some}-\textnormal{kill}\#2a-caus=evid=cond ~ ~ ~ ~ ~ ~ ~ ~ ~ ~ ~ ~ ~ ~ ~ ins.heat-\textnormal{be.roasted}\#1a.pl-caus=ss 1a.pl-\textnormal{eat}=pot=ind.m\\
    \glt `when you kill some jackrabbits, we will roast them and eat it' \citep[195]{hollow1973a}
    
    \item\label{EMPHsuffix4} \glll wáa'\textbf{oshka}nashe máatahe wahíroote'sh\\
    waa-\textbf{oshka}=rąsh=E wąątah=E wa-hi=ootE=o'sh\\
    nom-\textbf{emph}=att=sv \textnormal{river}=sv 1a-\textnormal{arrive.here}=evid=ind.m\\
    \glt `it is a good thing that I must have gotten to the river' \citep[36]{hollow1973a}
    
    \end{xlist}

\end{exe}

There is a difference between this suffix and the free word \textit{óshka} `even' in that the suffix triggers an epenthetic [ʔ] when attached to a vowel-final stem. The word \textit{óshka} `even', however, requires no such word-level phonological process and can occur after pauses, indicating that it is not a prefix but a word. We can see an example of this word in the example in (\ref{OSHKAasWord}) below.

\begin{exe}

\item\label{OSHKAasWord} Example of \textit{óshka} as a free word

    \glll kotúhiniikinus ų́'taa \textbf{óshka} ~ ~ ~ ~ ~ ~ ~ ~ ~ ~  kiná'roomako'sh\\
    ko-tuhrįį=rų=s ų'=taa \textbf{oshka} ~ ~ ~ ~ ~ ~ ~ ~ ~ ~  kirą'=oowąk=o'sh\\
    3poss.pers-\textnormal{mother-in-law}=anf=def \textnormal{be.close}=loc \textbf{\textnormal{even}} ~ ~ ~ ~ ~ ~ ~ ~ ~ ~ \textnormal{tell}=narr=ind.m\\
    \glt `he even spoke to his mother-in-law' \citep[182]{trechter2012}

\end{exe}

We can see the adverbial \textit{óshka} `even' in (\ref{OSHKAasWord}) above have semantic scope over the entire proposition, not just over the noun phrase. The speaker is expressing shock that this action would even happen, given the fact that it is a taboo in traditional Mandan customs for a man to speak directly to his mother-in-law. \citet[182]{trechter2012} remarks in a footnote that this is an indicator that the speaker is quite upset for him to be willing to flout this taboo to speak directly to his mother-in-law. Namely, the semantic scope of this sentence gives the reading `he \textit{even} spoke to his mother-in-law', not `he spoke \textit{even} to his mother-in-law' or `he spoke to \textit{even} his mother-in-law.'

This suffix also features heavily in a set of free pronominals in Mandan, but more on that will be addressed in the section on pronouns below in \sectref{SecPronouns}.

\subsubsubsection{Similitive suffixes: \textit{-esh} and \textit{-eshka}}\label{NominalSimilitives}

There are two similitive suffixes used on nouns in Mandan: -\textit{esh} and -\textit{eshka}. Of the two, -\textit{eshka} is better atteseted in the corpus. We can see these two similitives described in \sectref{suffixsimilitive1redux} and \sectref{suffixsimilitive2redux} below.

\subsubsubsubsection{Similitive 1: -\textit{esh}}\label{suffixsimilitive1redux}

As previously described in \sectref{suffixsimilitive1}, this term grants a similitive reading to a word. It can also have an almost diminutive reading, as it can be used in references to small amounts of things. We can see examples of \textit{-esh} in the data in (\ref{similitiveESHnom}) below.

\newpage

\begin{exe}

\item\label{similitiveESHnom} Examples of the similitive -\textit{esh} with nominals 


\begin{xlist}
	\item \glll íirapsi'\textbf{esh} máakeroomako'sh\\
	ii\#ra-psi-\textbf{esh} wąąkE=oowąk=o'sh\\
	\textnormal{blood}\#mut-\textnormal{be.black}-\textbf{smlt} \textnormal{be.lying}.aux=narr=ind.m\\
	\glt `a bit of blood that had gone black was there' \citep[132]{hollow1973a}
	
	\item \glll réshka'\textbf{esh} ká'nashki\\
	reshka-\textbf{esh} ka'=rąsh=ki\\
	\textnormal{this.way}-\textbf{smlt} \textnormal{possess}=att=cond\\
	\glt `whatever caused this' \citep[182]{trechter2012}

\end{xlist}

\end{exe}

This marker is most often found on verbs and adverbs, not nouns or other nominal elements. It is not clear if this limited distribution reflects the rarity of this formative in conversational speech, or if it is uncommon in all registers.

\subsubsubsubsection{Similitive 2: -\textit{eshka}}\label{suffixsimilitive2redux}

The verbal counterpart to the similitive -\textit{eshka} has been discussed previously in \sectref{suffixsimilitive2}. On verbs, this suffix conveys a sense that the action was done in a manner like that of the verb in question. For nouns, however, the similitive has an abstracting effect on the stems onto which they suffix. Examples of this suffix appear below in (\ref{NOMeshkaEX}).

\begin{exe}

\item\label{NOMeshkaEX} Examples of the similitive -\textit{eshka} with nominals

    \begin{xlist}
    
    \item\label{NOMeshkaEX1} \glll hą́pe tóop\textbf{eshka}k káare ímaatiht súkinista!\\
    hąp=E toop-\textbf{eshka}=ak kaare i-wąątih=t suk=rįt=ta\\
    \textnormal{day}=sv \textnormal{four}-\textbf{smlt}=ds neg.imp pv.dir-\textnormal{outside}=loc \textnormal{exit}=2pl=imp.m\\
    \glt `don't go outside for four days!' \citep[199]{hollow1973a}
    
    \item\label{NOMeshkaEX2} \glll wáamatawe'\textbf{eshka} éemanate'sh\\
    waa-wą-tawe-eshka ee-w-rą-tE=o'sh\\
    nom-1s-\textnormal{not.good}-smlt pv-1s-2a-\textnormal{say}.2sg=ind.m\\
    \glt `you said something bad about about me' \citep[88]{hollow1973b}
    
    \item\label{NOMeshkaEX3} \glll órut\textbf{eshka} áarupani ~ ~ ~ ~ ~ ~ ~ ~ ~ ~ ~ ~ ~ ~ ~  káakeropxekerektiki\\
    o-rut-\textbf{eshka} aa-ru-pE=rį ~ ~ ~ ~ ~ ~ ~ ~ ~ ~ ~ ~ ~ ~ ~  k-aa-k-ropxE=krE=kti=ki\\
    pv.irr-\textnormal{eat}-\textbf{smlt} pv.tr-ins.hand-\textnormal{grind}=ss ~ ~ ~ ~ ~ ~ ~ ~ ~ ~ ~ ~ ~ ~ ~  vert-pv.tr-suus-\textnormal{enter}=3pl=pot=cond\\
    \glt `they took only what they needed to eat and brought it back into the house' \citep[19]{trechter2012}
    
    \end{xlist}

\end{exe}

Much like the other similitive -\textit{esh}, the -\textit{eshka} suffix tends to be found more often on verbs. When applied to nouns, this suffix can indicate something reminiscent of the verb or noun in the root, but it can also convey a diminutive or limiting sense, as we see in (\ref{NOMeshkaEX1}), where the speaker is telling the listener to not go out for four days. Perhaps a more likely translation than the one given by \citeauthor{hollow1970} is to not go out for `about' four days or for `something like' four days. Likewise, we see -\textit{eshka} used (\ref{NOMeshkaEX3}) to describe a quantity of food that was `just enough.' 

\subsubsubsection{Intensifier: -\textit{sįh}}\label{suffixintensifierredux}

Like the verbal intensifier described in \sectref{suffixintensifier}, the nominal intensifier has the effect of creating nouns that have a intense or prototypical quality. It is relatively uncommon in the corpus, and seems to be associated with coloring a word to have some kind of emotional meaning on the part of the speaker, i.e., the speaker is judging something, expressing surprise or admiration at something, and so forth. We can see examples of \textit{-sįh} in (\ref{SIHintensifierEx}) below.

\begin{exe}

\item\label{SIHintensifierEx} Examples of the intensifier suffix -\textit{sįh}

\begin{xlist}

\item\label{SIHintensifierEX1} \glll shehék\textbf{sįh}\\
    shekek-\textbf{sįh}\\
    \textnormal{coyote}-\textbf{ints}\\
    \glt `a liar' \citep[206]{hollow1970}
    
\item\label{SIHintensifierEX2} \glll \'{ı̨}įsta\textbf{sįh}\\
    \'{ı̨}įsta-\textbf{sįh}\\
    \textnormal{middle.of.forehead}-\textbf{ints}\\
    \glt `the middle of the forehead' \citep[131]{hollow1973a}
    
\end{xlist}


\end{exe}

While the word \textit{shehék} `coyote' is often used synonymously with `liar' in Mandan, the use of the intesifier in (\ref{SIHintensifierEX1}) conveys more of the sense of `a real liar' or a liar in a stronger sense of the word. Likewise, we see -\textit{sįh} used in (\ref{SIHintensifierEX2}) after \textit{\'{ı̨}įsta} `middle of forehead' to convey that the speaker is referring to the very middle of the forehead or `smack-dab' in the middle of the forehead. In English, speakers rely on extraneous description to convey the sense of intensification that -\textit{sįh} does as a suffix.

\subsubsubsection{Augmentative: -\textit{xte}}\label{augmentativesuffixredux}

Like the nominal intensifier, the nominal augmentative is not as common as its verbal counterpart. It can be difficult at times to identify whether a speaker is using -\textit{xte} or is just using the stative verb \textit{xté} `be big.' Listening for intonational breaks, pauses, and word-level phonological processes like primary stress help to discern the augmentative from a word that is being described as being `big'. We can see several examples of the augmentative below, where the prosody of these words excludes the chance of them really being just \textit{xté}; the data in (\ref{XTEexamples}) are truly -\textit{xte}.

\begin{exe}

\item\label{XTEexamples} Examples of the augmentative suffix -\textit{xte}

    \begin{xlist}
    
    \item\label{XTEexamples1} \glll miníxu\textbf{xte}kere\\
    wrį\#xu-\textbf{xtE}=krE\\
    \textnormal{water}\#\textnormal{be.shallow}-\textbf{aug}=3pl\\
    \glt `large shallow lakes' \citep[11]{hollow1973a}
    
    \item\label{XTEexamples2} \glll mí'\textbf{xte}seena\\
    wį'-\textbf{xtE}=s=ee=rą\\
    \textnormal{rock}-\textbf{aug}=def=dem.dist=top\\
    \glt `that big rock' \citep[18]{hollow1973a}
    
    \item\label{XTEexamples3} \glll hó\textbf{xte} haráni\\
    ho-\textbf{xtE} hrE=rį\\
    \textnormal{voice}-\textbf{aug} caus=ss\\
    \glt `he really hollered' \citep[46]{hollow1973a}
    
    \item\label{XTEexamples4} \glll mí'ti\textbf{xte}s tanúma'kshikere kihkaráaroomako'sh\\
    wį'\#ti-\textbf{xtE}=s ta-ruwą'k\#shi=krE kihkraa=oowąk=o'sh\\
    \textnormal{rock}\#\textnormal{dwell}-\textbf{aug}=def al-\textnormal{man}\#\textnormal{be.good}=3pl \textnormal{look.for}=narr=ind.m\\
    \glt `he looked for just the chiefs of the big village' \citep[105]{hollow1973b}
    
    \end{xlist}

\end{exe}

The augmentative can appear on nouns without the English word `big' in the accompanying translations from Mandan consultants, like we see in (\ref{XTEexamples4}). There are also augmentatives used to convey a sense of entirity in (\ref{XTEexamples3}), where the translation of `he really hollered' would more accurately be translated as `he did it with his whole voice.' We can make the same generalization about the -\textit{xte} in (\ref{XTEexamples4}), where the protagonist of a narrative is looking for the chiefs for the entire village.

% \subsubsection{Postnominal enclitics}\label{SubSubNOMenclitics}

% We have seen that nominal elements in Mandan have only a few productive suffixes. We can draw a parallel between the distribution of suffixes in Mandan nouns with the paucity of suffixes we see in Mandan verbs. Like verbs, Mandan nouns show that they 

% \begin{exe}

% \item\label{nominalencliticslist} List of enclitics found on nouns

% \begin{tabular}{lll}
% /=s/& definite article (\rextsc{def})
%     &   (see \sectref{SecNOMdef})
% \end{tabular}

% \end{exe}

\subsection{Compounding}\label{SubSecCompounding}

Many words in Mandan are derived from the concatenation of affixal morphology. Compounding, however, is another common process for deriving new words. Nominal compounds fall into two general classes: noun-noun compounds and noun-verb compounds. Primary stress occurs on the initial noun in the compound, even if the result would be a single, light syllable bearing primary stress contrary to the expected stress pattern in Mandan as outlined in \sectref{primarystress}. However, this pattern is within the bounds of expected Mandan stress assignment once internal word boundaries are taken into account. Primary stress is thus one metric for determining a true compound versus a noun phrase that contains other lexical material.

\subsubsection{Noun-noun compounds}\label{SubSubSecNNcompounds}

Noun-noun compounds in Mandan typically are head-final, i.e., the defining element is the final noun in the compound. We can see examples of noun-noun compounds in the data in (\ref{NNcompoundEx}) below.

\begin{exe}

\item\label{NNcompoundEx} Examples of noun-noun compounds

    \begin{xlist}
    
    \item\label{NNcompoundEx1} \textit{pó'i'ahe} `fish scale'  $\leftarrow$ \textit{pó} `fish' + \textit{í'ahe} `skin, scale, covering'
    
    \item\label{NNcompoundEx2} \textit{hų́ųsiropxi}
    `cowboy' $\leftarrow$ \textit{hų́ųsi} `leggings' + \textit{ropxí} `leather, hide, skin'
    
    \item\label{NNcompoundEx3} \textit{pą́ąxekaruutka} `beetle, potato bug, ladybug' $\leftarrow$ \textit{pą́ąxe} `potato' or `sunchoke' + \textit{karúutka} `eater'
    
    \item\label{NNcompoundEx4} \textit{máanuxikpa} `skull' $\leftarrow$ \textit{máanuxik} `ghost + \textit{pá} `head'
    
    \item\label{NNcompoundEx5} \textit{manárokpuse} `lynx' $\leftarrow$ \textit{manárok} `forest' + \textit{púse} `cat'
    
    \item\label{NNcompoundEx6} \textit{minísaakanahka} `saddle' $\leftarrow$ \textit{minís} `horse' + \textit{áakanahka} `seat, chair'
    
    \item\label{NNcompoundEx7} \textit{wará'ireexikri} `kerosene' $\leftarrow$ \textit{wará'} `fire' + \textit{íreex} `light' + \textit{íkiri} `grease, oil'
    
    \item\label{NNcompoundEx8} \textit{Éexixtenuma'kaaki} `Arapaho' $\leftarrow$ \textit{éexi} `belly' + \textit{xté} `be big' + \textit{numá'kaaki} `people'
    
    \end{xlist}

\end{exe}

Most noun-noun compounds in Mandan consist of two parsable nouns. There are some words in Mandan, however, that appear to be older compounds than some of those listed above in (\ref{NNcompoundEx}). Several compounds that were formed prior to or just at the advent of the reservation period have slightly different phonological characteristics than many of the compounds seen today. In these older compounds, there is a predictable process whereby stems that end in a short vowel will undergo apocope when the following word begins with no onset and primary stress. Contemperary Mandan does not syncopate any vowel, but will insert an epenthetic glottal stop as described previously in \sectref{intrusiveglottal}. We can see examples of these older compounds in (\ref{oldcompounds}) below.

\newpage

\begin{exe}

\item\label{oldcompounds} Older Mandan noun-noun compounds

\begin{xlist}

\item\label{oldcompounds1} \textit{Mí'tųųtaahąhkas} `Mitutanka' (the southernmost Mandan settlement visited by Lewis and Clark)  $\leftarrow$ \textit{mí'ti} `village' + \textit{ų́ųtaahąk} `east' + =\textit{ka} habitual aspectual + =\textit{s} definite article

\item\label{oldcompounds2} \textit{istíkirus} `wash basin' $\leftarrow$ \textit{istá} `face' + \textit{íkirus} `tub', `shower', or `soap'

\item\label{oldcompounds3} \textit{maná'p} `leaves, tea' $\leftarrow$ \textit{maná} `tree' + \textit{á'p} `leaf'

\item\label{oldcompounds4} \textit{shúupikiri} `bone marrow' $\leftarrow$ \textit{shúupa} `shinbone' + \textit{íkiri} `grease', `lard', or `oil'

\end{xlist}


\end{exe}

These apocopated compounds are in the minority within the corpus. This pattern of deleting the final vowel to prevent hiatus in a compound or serial verb construction is observed in other Siouan languages, such as Crow \citep[50]{graczyk2007}, Hidatsa \citep[315]{park2012}, and Lakota \citep[533]{ullrichblackbear2016}, \textit{inter alios}. This lack of apocape in contemporary Mandan compounds appears to reflect a change in the phonology of Mandan from an earlier stage of development that happened sometime in the late nineteenth century, given the paucity of such compounds in neologisms that entered the language around that time that exhibit no evidence of apocape, e.g., `barrel' is \textit{miní íroke} [lit. `water container'], not *\textit{miníroke}. This lack of apocape suggests that this compound entered the lexicon after this apocape rule for compounds was lost in Mandan. Another possibility is that this rule was already in the process of being lost well before the Mandan people came into contact with Europeans and that only certain high-frequency compounds retained the older system of deleting final short vowels on the first noun to avoid hiatus.



\subsubsection{Noun-verb compounds}\label{SubSubSecNVcompounds}




We can contrast this with noun-verb compounds, where compounds are head-initial, i.e., the noun is qualified by the verb. In these kinds of compounds, the verb is stative and has a quasi-adjectival use. We see examples of these compounds in (\ref{NVcompoundEx}) below.

\begin{exe}
\item\label{NVcompoundEx} Examples of noun-verb compounds

\begin{xlist}

\item\label{NVcompoundEx1} \textit{áapxase} `red-winged blackbird' $\leftarrow$ \textit{áapxa} `wing' + \textit{sé} `be red'

\item\label{NVcompoundEx2} \textit{ą́sexaa} `moose' $\leftarrow$ \textit{ą́se} `horn' + \textit{xáa} `be spread out'

\item\label{NVcompoundEx3} \textit{wá'txi'hs} `my grandfather' $\leftarrow$ \textit{wá't} `my father' + \textit{xí'h} `be old' + =\textit{s} definite article

\item\label{NVcompoundEx4} \textit{hą́pxik} `storm' $\leftarrow$ \textit{hą́p} `day' + \textit{xík} `be bad'

\item\label{NVcompoundEx5} \textit{shų́thąshka} `mountain lion' $\leftarrow$ \textit{shų́t} `tail' + \textit{hą́shka} `be long'

\item\label{NVcompoundEx6} \textit{íkirisii} `butter' $\leftarrow$ \textit{íkiri} `grease, lard, oil' + \textit{síi} `be yellow'

\item\label{NVcompoundEx7} \textit{írupasanak} `pistol' $\leftarrow$ \textit{írupa} `gun, rifle, firearm' + \textit{sanák} `be short' or `be round'

\item\label{NVcompoundEx8} \textit{kóoxte} `pumpkin' $\leftarrow$ \textit{kóo} `squash' + \textit{xté} `be big'

\end{xlist}

\end{exe}

As previously discussed in \sectref{primarystress}, primary stress is used to disambiguate between a true compound and a noun with a stative verb used in an adjectival manner. For example, the word for `butter' in (\ref{NVcompoundEx6}) is literally `yellow lard', and there a single primary stress in the entire compound, i.e., \textit{íkirisii}. A speaker can disambiguate between this compound for butter and a description of some yellow-colored grease or oil with primary stress on both words, i.e., \textit{íkiri síi} `yellow grease' or `yellow oil.'

\section{Possession}\label{SecPossession}

Mandan does not have gender on nouns in the sense many languages of the Indo-European or Afro-Asiatic language families do. There is a lexical distinction, however, between whether a noun can be alienable or inaliable in terms of how it is possessed. This quality is lexically determined, and it is not always predictable. The possession class of every noun must be learned. \citet[561]{nichols1988} notes that these terms are standard usage among many Americanist linguists, but their definitions can be highly variable. For the purposes of this book, I define alienable as a noun that can in some what -- physically or metaphorically -- be separated from its possessor. An inalienable noun is one where a noun must be inherently possessed by someone.

Inalienable nouns tend to be body parts and kin. Alienable nouns tend to be all other kinds of nouns. These, however, are tendencies, as there are examples of nouns we would expect to be inalienable that bear alienable marking, and \textit{vice versa}, we also have nouns that we could conceive of as not being inherently possessed that do not take alienable marking. This system is inherited from Proto-Siouan, as \citet[108]{parksrankin2001} remark that all Siouan languages distinguish between alienable and inalienable possession.

\subsection{Inalienable possession}\label{SubSecInalienable}

Inalienable possession consists of a closed set of nouns that have a particular pattern for showing possession. These nouns consist almost entirely of body parts and kinship terms, but there are several items of material culture that also fall under this noun class. 

There are two inalienable possession paradigms for consonant-initial stems: one that occurs before stems beginning /w/, /r/, and /h/ followed by an oral front vowel (i.e., the resonant paradigm) and all others (i.e., the default paradigm). Vowel-initial stems also have two paradigms: one for those beginning with a high vowel (i.e., the high paradigm), and one for all others (i.e., the non-high paradigm). We can see how possession marking works in the words  \textit{shí} `foot, feet', \textit{réesik} `tongue', \textit{éekhuu} `sternum', and \textit{istá} `face' in \tabref{TableInalienable} below.

\begin{table}
\caption{Inalienable possession paradigms}\label{TableInalienable}
\begin{tabular}{lllll}
\lsptoprule
~&          \textbf{Default}&        \textbf{Resonant}&
   \textbf{Non-High}& \textbf{High}\\
\midrule
\textsc{1sg}&   \textit{mishí} & \textit{wiréesik}&
    \textit{wé'khuu}&   \textit{mí'sta}\\
\textsc{2sg}&   \textit{nishí} & \textit{riréesik}&
    \textit{ré'khuu}&   \textit{ní'sta}\\
\textsc{3sg}&   \textit{shí} & \textit{réesik}&
    \textit{éekhuu}&  \textit{istá}\\
\textsc{1du}&   \textit{nushí} & \textit{nuréesik}&
    \textit{réekhuu}&  \textit{núusta}\\
\textsc{1pl}&   \textit{nushínite} & \textit{nuréesikinite}&
    \textit{réekhuunite}&  \textit{núustanite}\\
\textsc{2pl}&   \textit{nishinite}&  \textit{riréesikinite}&
    \textit{ré'khuunite}&  \textit{ní'stanite}\\
\textsc{3pl}&   \textit{shíkere}& \textit{résihkere}&
    \textit{éekhuukere}&  \textit{istákere}\\
\hline\hline
\end{tabular}
\end{table}

We can see much similarity between the distribution of possessive prefixes on the table above and the verbal pronominal prefixes previously seen in \sectref{Ch2InflectionalPrefixes}. The default singular prefixes resemble stative prefixes, except for the first person plural possessive, \textit{nu}-, which is identical to the active prefix. As expected, given Mandan having no dedicated third person singular marking. The English translations of these kinds of words in a narrative are often given as general nouns, though the implication is that in Mandan, it is understood that these nouns belong to someone mentioned in the discourse.  \tabref{TableInalienablePrefixes} summarizes these possession markers appears below, with the possession markers depicted in underlying representation.


\begin{table}
        \caption{Inalienable possessive marking}\label{TableInalienablePrefixes}
    \begin{tabular}{lllll}
\lsptoprule
    ~&  \textbf{Default}&\textbf{Resonant}&\textbf{Non-High}&\textbf{High}\\
\midrule
    \textsc{1sg} &/wį-/&     /wi-/&      /w'-/&  /w'\~~-/\\
    \textsc{2sg} &/rį-/&     /ri-/&      /r'-/&  /r'\~~-/\\
    \textsc{3sg} &/$\varnothing$-/&     /$\varnothing$-/&      /$\varnothing$-/&  /$\varnothing$\~~-/\\
    \textsc{1du} &/rų-/&     /rų-/&      /rV-/$\sim$/r-/&  /rų-/\\
    \textsc{1pl} &/rų-~~~=rįt/&     /rų-~~~=rįt/&      /rV-~~~=rįt/$\sim$/r-~~~=rįt/&  /rų-~~~=rįt/\\
    \textsc{2pl} &/rį-~~~=rįt/&     /ri-~~~=rįt/&      /r'-~~~=rįt/&  /r'\~~-~~~=rįt/\\
    \textsc{3pl} &/$\varnothing$-~~~=rįt/&     /$\varnothing$-~~~=krE/&      /$\varnothing$-~~~=krE/&  /$\varnothing$-~~~=krE/\\\lspbottomrule
    \end{tabular}

\end{table}

The singular resonant prefixes in \tabref{TableInalienablePrefixes} all have oral vowels instead of nasal vowels, which preserves the underlying /w/ and /r/. Likewise, we see a contrast between /w'-/ \textsc{1poss},  /r'-/ \textsc{2poss}, and /r(V)-/ \textsc{1pl.poss} for nouns belonging to the non-high vowel paradigm, but underlying nasals for /w'\~~-/, /r'\~~-/, and /rų-/, respectively. The first person plural possessive for words in the high vowel paradigm have the further peculiarity of causing the initial vowel of the stem to undergo apherisis and lengthening the /ų/ of the prefix. This process results in a word like /rų-ista/ `our faces' becoming \textit{núusta} instead of *\textit{nu'ísta} or *\textit{ríista}. This replacement of the first syllable by /rų-/ even extends to nouns that begin with heavy open syllables, e.g., /rų-į'ta'ke/ `our foreheads' $\varnothing$ \textit{núuta'ke}. Vowel-initial nouns that start with a preverb will be declined in a manner identical to how first person active pronominals are, i.e., /rV-i-wąshut/ `our clothing' $\varnothing$ \textit{ríimashut}. representation.


The second and third person possessive forms all bear their respecting plural enclitics on the table above, i.e., =\textit{nit} for second person plural and =\textit{kere} for third person plural. Plural marking on nouns is quite uncommon and seen as unnecessary if the plurality of a noun is inferred or if it has been previously established. We can see plurality marked on the verb \textit{namáakahinitak} `you (pl.) reside, live' in the sentence below in (\ref{ExOptionalPlurals}), for example. The presence of the second person plural enclitic =\textit{nit} indicates that the speaker is taking the fact that he is addressing a group of men into account. In the second sentence, the noun \textit{nitásuknuma'k} `your young men' bears no plural marking whatsoever; neither the plurality of the noun itself (i.e., `young men') nor the plurality of the possessor (i.e., `your (pl.)') are reflected on the noun.

\begin{exe}

\item\label{ExOptionalPlurals} Lack of plural marking on nouns in a narrative context

\glll Róo róotaa namáakahi\textbf{nit}ak, shí'sh. ~ ~ ~ ~ ~ Tashká'nik, nitásuknuma'k, koxamáhanash máakahe, ~ ~ ~ ~ ~ óxkakere'sh.\\
    roo rV-o=taa ra-wąąkah=\textbf{rįt}=ak shi=o'sh ~ ~ ~ ~ ~  tashka=shka'rįk rį-ta-suk\#ruwą'k ko-xwąh=rąsh wąąkahE ~ ~ ~ ~ ~ o-xka=krE=o'sh\\
    dem.mid 1a.pl-pv.loc=loc 2a-\textnormal{sit}.aux.hab=\textbf{2pl}=ds \textnormal{be.good}=ind.m ~ ~ ~ ~ ~ \textnormal{how}=disj 2poss-al-\textnormal{child}\#\textnormal{man} rel-\textnormal{be.small}=att \textnormal{those} ~ ~ ~ ~ ~ pv.loc-\textnormal{be.foolish}=3pl=ind.m\\
    \glt `It is good that you (pl.) live together with us here. However, your (pl.) young men, especially those ones who are small, they are foolish.' \citep[205f]{trechter2012}

\end{exe}

No plural marking is typically realized on a noun bearing a first person plural possession prefix. Speakers confirm that one can do so, but it is superfluous in almost any context, given the fact that the first person plural prefix itself indicates plurality. One could add the =\textit{nit} enclitic to such nouns to disambiguate between a dual inclusive and a general first person plural reading, but again, speakers have indicated that such a situation is rare enough to warrant only producing forms without a plural enclitic in daily speech and in narratives. This is also the case with marking first person plurality on verbs with the =\textit{nit} enclitic; such plural marking may occur a few times within a narrative to reaffirm that the speaker is referring to a group of individuals rather than just a pair consisting of the speaker and addressee, but it is more common to see first person plural verbs without plural marking beyond the first person plural prefix itself. The same symmetry appears to be the case with possessive constructions.

\citet{hollow1970} makes note of whether a noun belongs to the inalienable class of nouns or not in his dictionary. I encourage the reader to examine that work to explore what kinds of nouns fall into which class. An itemization of all nouns that belong to the inalienable noun class is beyond the scope of the present book. However, the majority of nouns in Mandan fall under the class of alienable nouns. It is especially worth noting that words that are originally inalienable can have an alienable counterpart whose semantics differ. We can see an example of such a doublet in the data in (\ref{AlienableInalienableSameWord}) below.

\begin{exe}
\item\label{AlienableInalienableSameWord} Alienable-inalienable doublet featuring the same root word

    \begin{xlist}
    
    \item\label{AlienableInalienableSameWord1}  Inalienable: \textit{éexi} `his/her stomach, paunch'

    \item\label{AlienableInalienableSameWord2}  Alienable: \textit{ta'éexi} $\sim$ \textit{téexi} `his/her tripe'
    
    \end{xlist}
\end{exe}

The inclusion of the alienable \textit{ta}- prefix on the word for `stomach' changes it to become something that is no longer indelibly possessed by an individual (i.e., `one's own stomach'). Instead, the alienable prefix changes the meaning of the word to some kind of foodstuff. We see the same relationship between the word \textit{síipe} `his/her intestines', where the alienable version \textit{tasíipe} is `his/her sausage.'

A less common manifestation of inalienable possession is through the use of the possessive preverb \textit{i}-, which was described earlier in \sectref{SubParaPossessive}. Proto-Siouan had a third person possessive marker *i-, though it usually only shows up in compound-like constructions in Mandan. The possessed noun will follow the possessor, and the possessed noun bears the \textit{i-} prefix, as seen in the data in (\ref{InalienableI}) below.

\begin{exe} 

\item\label{InalienableI} Examples of \textit{i-} marking inalienable possession

    \begin{xlist}
    
    \item\label{InalienableI1} \glll imáa~\textbf{í}wahuu~ná're\\
    iwą\#\textbf{i}-wa-hu\#rą'=E\\
    \textnormal{body} \textbf{pv.poss}-unsp-\textnormal{bone} \textnormal{ache}\\
    \glt `rheumatism' [lit. `the body's bones ache'] \citep[96]{hollow1970}
    
    \item\label{InalienableI2} \glll ré'\textbf{i}hį\\
    rE\#\textbf{i}-hį\\
    \textnormal{penis}\#\textbf{pv.poss}-\textnormal{hair}\\
    \glt `male pubic hair' \citep[175]{hollow1970}
    
    \item\label{InalienableI3} \glll pó~\textbf{í}shųt\\
    po\#\textbf{i}-shųt\\
    \textnormal{fish}\#\textbf{pv.poss}-\textnormal{tail}\\
    \glt `fish tail' \citep[241]{hollow1970}
    
    \item\label{InalienableI4} \glll tí~\textbf{í}wasįh\\
    ti\#\textbf{i}-wa-sįh\\
    \textnormal{dwelling}\#\textbf{pv.ins}-unsp-\textnormal{be.strong}\\
    \glt `lodge pole' \citep[250]{hollow1970}
    
    \item\label{InalienableI5} \glll psh\'{ı̨}įxaa~\textit{í}miihka\\
    pshįįxaa\#\textit{i}-wįįh=ka\\
    \textnormal{sage}\#\textit{pv.poss}-\textnormal{woman}=hab\\
    \glt `female sage plant' \citep[286]{hollow1970}
    
    \item\label{InalienableI6} \glll matómiihka\\
    wąto\#wįįh=ka\\
    \textnormal{bear}\#\textnormal{woman}=hab\\
    \glt `female bear' \citep[286]{hollow1970}
    
    \end{xlist}

\end{exe}

Using the possessive preverb seems to not be obligatory, as we see in (\ref{InalienableI5}) and (\ref{InalienableI6}), where both nouns describe the female of a species, but the possessive preverb is only appears in (\ref{InalienableI5}). There is a slight difference in how these two constructions are read, i.e., the translation of (\ref{InalienableI5}) is `female sage plant', but a more faithful translation to what the \textit{i}- is doing here is `sage plant's female.' We can contrast this with the construction in (\ref{InalienableI6}), which is a true compound, i.e., `bear female.'

Finally, though there is no third person possessive prefix in most cases, for the set of words classified as kinship terms, there exists a specific third person possessive prefix, \textit{ko}-. This prefix is highly restricted, and it is not generalized to nouns referring to people. This prefix is reconstructable not just to Proto-Siouan, but it also has cognates in Catawban and Yuchi, indicating that this is a very old piece of morphology indeed. This *ko- is analyzed as a noun classifier of sorts, having reflexes in nouns having to do with people \citep[24ff]{kasak2016}. This classifier contrasted with Proto-Siouan *wi- and *wa-, which were used for non-human animals and all other nouns, respectively. In Mandan, it is no longer productive, and knowing whether a noun can take this \textit{ko}- possessive marker is lexically determined. See \sectref{SubSecKinshipTerms} in the following section of this chapter for more examples of its usage in Mandan.

\subsection{Alienable possession}\label{SubSecAlienable}

The vast majority of nouns are alienable. Functionally, this means that Mandan distinguishes from nouns that have a general reading versus those that are overtly possessed. When possessed, these nouns all bear the alienable possession marker \textit{ta}-, along with person marking to show who possesses the noun. This \textit{ta}- prefix is a direct descendant from Proto-Siouan *-hta, which was also a possession marker. This alienable possession marker in Proto-Siouan would combine with the other possession prefixes, resulting in a sequence that marked a noun as both possessed and also alienably possessed. 

Unlike the unalienable possession prefixes shown in the previous subsection, there is far less variability in how alienable possession marking manifests itself on Mandan nouns. The pattern is always a person marker, then the alienable possession marker \textit{ta}-, then the noun itself. The \textit{ta}- marker has an allomorph with vowel-initial stems, where the /a/ is replaced by the vowel onto which the prefix concatenates, resulting in either /tV-/ for stems beginning with short vowels or /t-/ for stems beginning with long vowels or those with /ʔ/ codas. In fast speech, speakers will refrain from blending the vowels on the \textit{ta}- with the following vowel, causing a glottal stop to be inserted between the vowels via the kind of epenthesis we saw previously in \sectref{glottalstopmetathesis}. We can see examples of alienable possession in \tabref{TableAlienableParadigm} below using the following words as exemplars: \textit{súk} `child', `\textit{óminik} `bean', and \textit{íire} `blood.'

\begin{table}
        \caption{Alienable possession paradigms}\label{TableAlienableParadigm}
    \begin{tabular}{llll}
\lsptoprule
    ~&  \textbf{Default}&   \textbf{Short Vowel}&   \textbf{Long Vowel}\\
\midrule
    \textsc{1sg}&   \textit{ptasúk}&  \textit{ptóominik}&
        \textit{ptíire}\\
    \textsc{2sg}&   \textit{nitásuk}&  \textit{nitóominik}&
        \textit{nitíire}\\
    \textsc{3sg}&   \textit{tasúk}&  \textit{tóominik}&
        \textit{tíire}\\
    \textsc{1du}&   \textit{nutásuk}&  \textit{nutóominik}&
        \textit{nutíire}\\
    \textsc{1pl}&   \textit{nutásukinite}&  \textit{nutóominikinite}&
        \textit{nutíinite}\\
    \textsc{2pl}&   \textit{nitásukinite}&  \textit{nitóominikinite}&
        \textit{nitíinite}\\
    \textsc{3pl}&   \textit{tasúhkere}&  \textit{tóominihkere}&
        \textit{tíikere}\\
    \lspbottomrule
    \end{tabular}

\end{table}

The pronominal prefixes for the second person and first person plural all mirror the distribution we saw in \sectref{SubSecInalienable}, where the second person possessor is identical to the stative \textit{ni}-, and the first person plural possessor is identical to the active \textit{nu}-. We again see the second person plural enclitic =\textit{nit} on the second person plural for alienably possessed nouns, and the third person plural =\textit{kere} on the third person plural. Like with the nouns that fall under the inalienable possession noun class, overtly marking the plurality of the possessor is rare in the corpus, especially in situations where the plurality of the possessor has been established in the discourse and is such understood by the interlocutors. A summary of the possessive markers from \tabref{TableAlienableParadigm} appears below in \tabref{TableAlienablePrefixes} in underlying notation, where X represents the nominal stem.

\begin{table}
    \caption{Alienable possessive marking}
    \label{TableAlienablePrefixes}
    \begin{tabular}{llll}
\lsptoprule
    ~&  \textbf{Default}&   \textbf{Short Vowel}&   \textbf{Long Vowel}\\
\midrule
    \textsc{1sg}&   /p-ta-X/&  /p-tV-X/&  /p-t-X/\\
    \textsc{2sg}&   /rį-ta-X/&  /rį-tV-/&  /rį-t-X/\\
    \textsc{3sg}&   /$\varnothing$-ta-X/&  /$\varnothing$-tV-X/&  /$\varnothing$-t-X/\\
    \textsc{1du}&   /rų-ta-X/&  /rų-tV-X/&  /rų-t-X/\\
    \textsc{1pl}&   /rų-ta-X=rįt/&  /rų-tV-X=rįt/&  /rų-t-X=rįt/\\
    \textsc{2pl}&   /rį-ta-X=rįt/&  /rį-tV-X=rįt/&  /rį-t-X=rįt/\\
    \textsc{3pl}&   /$\varnothing$-ta-X=krE/&  /$\varnothing$-tV-X=krE/&  /$\varnothing$-t-X=krE/\\
    \lspbottomrule
    \end{tabular}

\end{table}

We can see that the first person singular prefix for alienable possession is neither \textit{wa}- nor \textit{ma}- nor \textit{mi}-, and it does not conform to any first person singular marker we have encountered before in Mandan. This prefix is the result of a historical sound change, where some more familiar prefix, perhaps /wį-/ or /wą-/, routinely underwent syncope before the \textit{ta}- marker, resulting in a /w-ta-/ sequence. The /w/ assimilated the stop features of the following plosive, becoming a /p/. This is a very well-attested diachronic process in other Siouan langauges, e.g., Proto-Siouan *wi-htee `bison' > Mandan \textit{pt\'{ı̨}į} and Lakota \textit{pté} `bison.'

The third person singular personal possessive prefix \textit{ko}- likewise occurs with alienable possession marking, as there are certain kinship terms that are not treated as inalienable. See \sectref{SubSecKinshipTerms} for more information on how this prefix is used in Mandan.

\subsection{Summary of possession}\label{SubSecSummaryOfPossessioN}

Possession is predominantly prefixal in nature in Mandan. When number enclitics occur, they most often occur to disambiguate or to introduce the plurality of a possessor into the discourse. When there is a plural possessor and a plural possessee, it is more common to see plural marking for the possessor. However, plural marking for both can occur, as we see below. In (\ref{DoublePossessivePlural1}), we see double plural marking: once to mark the plurality of the possessor and again to mark the plurality of the possessee. Plurality of the possessee will always be closer to the stem than the plurality of the possessor. It is also possible to have only plurality of the possesee marked on the noun, as we see in (\ref{DoublePossessivePlural2}).

\begin{exe}

\item\label{DoublePossessivePlural} Multiple plural marking in possessive constructions

    \begin{xlist}
    
    \item\label{DoublePossessivePlural1} \glll nutúut\textbf{kerenit}s\\
    rų-tuut=\textbf{krE}=\textbf{rįt}=s\\
    1pl.poss-\textnormal{son-in-law}=\textbf{3pl}=\textbf{2pl}=def\\
    \glt `our (pl.) sons-in-law' \citep[477]{hollow1970}
    
    \item\label{DoublePossessivePlural2} \glll Ptasúh\textbf{kere}s wáaxte ísehkere'sh, ~ ~ ~ ~ ~ ~ ~ ~ numá'kaakinite!\\
    p-ta-suk=\textbf{krE}=s waa-xtE i-sek=krE=o'sh ~ ~ ~ ~ ~ ~ ~ ~ ruwą'k-aaki=\textbf{rįt}=E=$\varnothing$\\
    1poss-al-\textnormal{child}=\textbf{3pl}=def nom-\textnormal{be.big} pv.ins-\textnormal{do}=3pl=ind.m ~ ~ ~ ~ ~ ~ ~ ~ \textnormal{person}-coll=\textbf{2pl}=sv=voc\\
    \glt `My children did something terrible, people!' \citep[178]{hollow1973a}
    
    \end{xlist}

\end{exe}

The word \textit{ptasúhkeres} `my children' bears the third person plural =\textit{kere}, indicating that the possessee is plural. There are numerous instances of words like this in the corpus where the plural marking is omitted, especially if the possessee has been referenced several times and listeners are assumed to remember that the possessee is plural.

\section{Direct address}\label{SecDirectAddress}

When referring directly to an individual, Mandan employs a variety of strategies. These kinds of strategies are outlined in this section. We can see how Mandan treats personal names in \sectref{SubSecPersonalNames}, in particular what kinds of morphology we often see associated with personal names versus other nouns. I discuss how vocatives also factor into personal names and other nouns or noun phrases used as means of direct address. Lastly, I provide a list of kinship terminology in Mandan in \sectref{SubSecKinshipTerms} and discuss some aspects of how the Mandan people have traditionally referred to each other based on marriage and clan relationship.

\subsection{Personal names}\label{SubSecPersonalNames}

Mandan treats personal names as a separate class of nouns in the sense that they are often topicalized with the topic marker =\textit{na} or otherwise dislocated in the sentence as foregrounding or backgrounding of information. Such topicalized or shifted names are often parenthetical or placed into their own intonational phrases. These prosodic breaks cause many names to end in the stem vowel /=E/ to show that the name is a complete utterance. These nouns are often accompanied by the definite article =\textit{s}, given the fact that personal names are typically composed of one or more words, and a speaker wishes to disambiguate between some common noun and a person whose name happens to be a common noun. We see some Mandan personal names in (\ref{ExamplesOfProperNames}) below.

\begin{exe}
\item\label{ExamplesOfProperNames}

\begin{xlist}

    \item \glll Werók~Wáatashe\\
    wrok\#waatash=E\\
    \textnormal{male.buffalo}\#\textnormal{metal}=sv\\
    \glt `Iron Buffalo' (a.k.a. Edwin Benson)

    
    % \item \glll Kų́'sherehe\\
    % kų'sh\#sreh=E\\
    % \textnormal{be.very.close.to}\#\textnormal{come.apart}=sv\\
    % \glt `Just About Coming Apart'
    
    \item \glll Náaku~Hų́s\\
    rąąku\#hų=s\\
    \textnormal{road} \textnormal{be.many}=def\\
    \glt `Many Roads' (a.k.a. Mattie Grinnell)
    
    \item \glll Xaráte~Ptéhe\\
    xratE\#ptEh=E\\
    \textnormal{wolf}\#\textnormal{run}=sv\\
    \glt `Running Wolf' (a.k.a. Joseph Packineau)
    
    \item \glll Numá'k~Máxanas\\
    ruwą'k\#wąxrą=s\\
    \textnormal{man}\#\textnormal{one}=def\\
    \glt `Lone Man' (cultural figure)
    
    \item \glll Kók~Kí's\\
    kok\#ki'=s\\
    \textnormal{pronghorn}\#\textnormal{pack.on.back}=def\\
    \glt `Packs Antelope' (cultural figure)
    
\end{xlist}
    
\end{exe}

It is common for speakers to vacillate between adding the definite marker =\textit{s} and omitting it, especially if the name keep coming up in discourse. Speakers have commented that one should add the definite marker, as it sounds better to them, but the corpus reveals that proper names are marked with =\textit{s} less than half the time. It is unclear if this propensity to omit the =\textit{s} from personal names extended to most of the Mandan-speaking community prior to the loss of L1 speakers, or if this propensity to leave off the =\textit{s} is restricted to the few individuals who have contributed to the narratives that make up this corpus.

Treating personal names as definite is extended to other proper nouns, such as locations, social organizations, months, or holidays. We can see examples of definiteness marking in the data in (\ref{ExamplesProperNouns}) below, where these proper nouns always appear with =\textit{s} in the corpus.

\begin{exe}

\item\label{ExamplesProperNouns} Examples of place names in Mandan

    \begin{xlist}
     
    \item \glll Tíirupa Pshíi Wóoni's\\
    tV-i-rupa pshii wV-o-rį'=s\\
    al-pv.ins-\textnormal{fire.arrow} \textnormal{be.flat} unsp-pv.loc-\textnormal{shoot}=def\\
    \glt `Twin Buttes' (lit. `The Place Where He Shot His Gun Flat') \citep[92]{hollow1970}
    
    \item \glll Máatahkshuks\\
    wąątah\#kshuk=s\\
    \textnormal{river}\#\textnormal{be.narrow}=def\\
    \glt `Little Missouri River' (lit. `The Narrow River') \citep[123]{hollow1970} \footnote{The word \textit{máatahe} `river' used to refer exclusively to the Missouri River. Almost all other rivers were considered \textit{pasą́he} `creek.'}
    
    
    \item \glll Mí'oxats\\
    wį'\#o-xat=s\\
    \textnormal{stone}\#pv.loc-\textnormal{society}=def\\
    \glt `Stone Club Society' (lit. `The Stone Society') \citep[133]{hollow1970}
    
    \item \glll Wará'shųt~Pasą́hs\\
    wra'\#shųt\#pasąh=s\\
    \textnormal{fire}\#\textnormal{tail}\#\textnormal{creek}=def\\
    \glt `Grand River (in South Dakota)' (lit. `The Ashes Creek') \citep[137]{hollow1970} 
    
    \item \glll Súk Wáakapus Óreeh Mínaks\\
    suk waa-ka-pus o-rEEh wįrak=s\\
    \textnormal{child} nom-ins.frce-\textnormal{make.marks} pv.irr-\textnormal{go.there} \textnormal{month}=def\\
    \glt `September' (lit. `The Month When Children Go to School') \citep[159]{hollow1970}
    
    \item \glll Rá'ska~Hą́xopinixtes\\
    ra'ska\#hąp\#hoprį\#xtE=s\\
    \textnormal{summer}\#\textnormal{day}\#\textnormal{be.holy}\#\textnormal{be.big}=def\\
    \glt `Independence Day' (lit. `The Big Summer Holiday') \citep[169]{hollow1970}
    
    \end{xlist}
 
\end{exe}

Much like personal names, proper nouns also tend to end in the stem vowel /=E/ whenever they do not end in a definite =\textit{s}. The presence of the stem vowel indicates that speakers are topicalizing or focusing these proper names, a process described in greater detail in \sectref{Ch5TopicAndFocus}. For locations, there is a strong tendency to treat it as a separate intonational phrase and then follow it with some manner of resumptive deictic marker.


 
\subsection{Vocatives}\label{SubSecVocatives}

When talking about someone, it is considered proper for Mandan speakers to include the definite article =\textit{s}. Speaking directly to someone, however, necessitates an intonation phrase for a vocative element. Nouns treated as vocatives that do not end in a short vowel will have a stem vowel /=E/ added to the end of it. We can assume that the stem vowel is added to words that end in a short vowel, but the hiatus resolution of /VV/ sequences described in \sectref{shortvoweldeletion} will cause the stem vowel to be deleted. The final vowel in a vocative is lengthened and the intonation of this element involves a rising and then falling pitch on the right boundary of the final element of the noun phrase. This treatment of vocatives is similar to how Hidatsa treats the prosody of vocatives \citep[461]{park2012}.

Plural vocatives bear the second person plural marker =\textit{nit} to clarify if the speaker is addressing one person or multiple people. We can see examples of vocatives observed in the corpus in (\ref{VocativeExamples}) below.

\newpage

\begin{exe}

\item\label{VocativeExamples} Examples of vocatives in Mandan

    \begin{xlist}
    
    \item\label{VocativeExamples1} \glll Mú'KAAǃ Máamakų'ta!\\
    w'-ųųka=$\varnothing$ waa-wą-kų'=ta\\
    1poss-\textnormal{elder.brother}=voc \textnormal{some}-1s-\textnormal{give}=imp.m\\
    \glt `Older brother! Give me some!' \citep[139]{hollow1973a}
    
    \item\label{VocativeExamples2} \glll SúkiniTEE! Matewé írasekinito'sha?\\
    suk=rįt=E=$\varnothing$ wa-t-we i-ra-sek=rįt=o'sha\\
    \textnormal{child}=2pl=sv=voc unsp-wh-indf  pv.ins-2a-\textnormal{do}=2pl=int.m\\
    \glt `Children! What are you doing?' \citep[28]{hollow1973a}
    
    \end{xlist}

\end{exe}

Vocatives usually appear in the form of kinship terms, which are described below in \sectref{SubSecKinshipTerms}. It is more common to address someone by how you are related to them than to use their proper name when addressing them. The overt use of someone's proper name is more typical when talking about someone rather than talking to them.

\subsection{Kinship terminology}\label{SubSecKinshipTerms}

Family and clan relations are important features of traditional Mandan life. \citet[37ff]{bowers1950} gives a lengthy description of the different ways that family, moieties, and clans play a role in shaping the identity of a Mandan person. Kinship terms constitute a distinct class of nouns within Mandan in that they display some possession morphology that is unique to this class of nouns, i.e., the use of the third person personal possession prefix \textit{ko}-.

Many of these kinship terms are inherited from Proto-Siouan, though certain terms have been adapted to fit the cultural norms of how the Mandan people organize their family structure. The data in (\ref{ListOfKinshipTerms}) below are adapted from the information in \citet[40ff]{bowers1950} and \citet[45]{mixco1997a}. Certain terms are used by speakers of a particular sex, and as such, the table below identifies whether this term is used by a particular sex, and if so, how it might differ between sexes. Relationships to a man are denoted with the {\Male} symbol, relationships to a woman are denoted with the {\Female} symbol, and relationships that do not depend on the sex of a person are denoted with the {\Hermaphrodite} symbol. 


\begin{exe}
\item\label{ListOfKinshipTerms} List of kinship terms in Mandan

\begin{xlist}

\item\label{ListOfKinshipTerms1} `mother' (\Hermaphrodite)

    \begin{tabular}{llll}
    ~ &\textsc{sg}&\textsc{du}& \textsc{pl}\\
    1&  \textit{mihųųs}&\textit{nuhų́ųs}         &\textit{nuhųųnits}\\
    2&  \textit{nihų́ųs}&         &\textit{nihų́ųnits}\\
    3&  \textit{kohų́ųs} &         &\textit{kohų́ųkeres}
    \end{tabular}
    
\item\label{ListOfKinshipTerms2} `father' (\Hermaphrodite)

    \begin{tabular}{llll}
    ~ &\textsc{sg}&\textsc{du}&\textsc{pl}\\
    1&  \textit{wá'ts}&   \textit{ráatinits}                
                                    &\textit{ráats}\\
    2&  \textit{rá'ts}&            &\textit{rá'tinits}\\
    3&  \textit{kó'ts}&            &\textit{kó'tkeres}
    \end{tabular}

   
\item\label{ListOfKinshipTerms3} `older brother' (\Male)

    \begin{tabular}{llll}
    ~ &\textsc{sg}&\textsc{du}&\textsc{pl}\\
    1&  \textit{mú'kas}&   \textit{núukas}                
                                    &\textit{núukanits}\\
    2&  \textit{nú'kas}&            &\textit{nú'kanits}\\
    3&  \textit{kų́'kas}&            &\textit{kų́'kakeres}
    \end{tabular}
    
\item\label{ListOfKinshipTerms4} `younger brother' (\Male), `any brother' (\Female)

    \begin{tabular}{llll}
    ~ &\textsc{sg}&\textsc{du}&\textsc{pl}\\
    1&  \textit{mishų́ųkas}&   \textit{nushų́ųkas}                                         &\textit{nushų́ųkanits}\\
    2&  \textit{nishų́ųkas}&    &\textit{nishų́ųkanits}\\
    3&  \textit{koshų́ųkas}&    &\textit{koshų́ųkakeres}
    \end{tabular}
    

\item\label{ListOfKinshipTerms5} `older sister' (\Female)

    \begin{tabular}{llll}
    ~ &\textsc{sg}&\textsc{du}&\textsc{pl}\\
    1&  \textit{minúuks}&   \textit{nunúuks}                                         &\textit{nunúukinits}\\
    2&  \textit{ninúuks}&    &\textit{ninúukinits}\\
    3&  \textit{konúuks}&    &\textit{konúuhkeres}
    \end{tabular}
    
\item\label{ListOfKinshipTerms6} `younger sister' (\Female)

    \begin{tabular}{llll}
    ~ &\textsc{sg}&\textsc{du}&\textsc{pl}\\
    1&  \textit{ptą́ąkas}&   \textit{nutą́ąkas}                                         &\textit{nutą́ąkanits}\\
    2&  \textit{nitą́ąkas}&    &\textit{nitą́ąkanits}\\
    3&  \textit{kotą́ąkas}&    &\textit{kotą́ąkakeres}
    \end{tabular}

\item\label{ListOfKinshipTerms7} `any sister' (\Male)

    \begin{tabular}{llll}
    ~ &\textsc{sg}&\textsc{du}&\textsc{pl}\\
    1&  \textit{ptamíihs}&   \textit{nutámiihs}                                         &\textit{nutámiihinits}\\
    2&  \textit{nitámiihs}&    &\textit{nitámiihinits}\\
    3&  \textit{kotámiihs}&    &\textit{kotámiihkere}
    \end{tabular}

\item\label{ListOfKinshipTerms8} `grandmother' (\Hermaphrodite), `mother-in-law' (\Female)

    \begin{tabular}{llll}
    ~ &\textsc{sg}&\textsc{du}&\textsc{pl}\\
    1&  \textit{mihų́ųxi'hs}&   \textit{nuhų́ųxi'hs}                                         &\textit{nuhų́ųxi'hs}\\
    2&  \textit{nihų́ųxi'hs}&    &\textit{nihų́ųxi'hinits}\\
    3&  \textit{kohų́ųxi'hs}&    &\textit{kohų́ųxi'hkeres}
    \end{tabular}

\newpage
    
\item\label{ListOfKinshipTerms9} `grandfather' (\Hermaphrodite), `father-in-law' (\Female)

    \begin{tabular}{llll}
    ~ &\textsc{sg}&\textsc{du}&\textsc{pl}\\
    1&  \textit{wá'txi'hs}&   \textit{ráatxi'hs}                                         &\textit{ráatxi'hinits}\\
    2&  \textit{rá'txi'hs}&    &\textit{rá'txi'hinits}\\
    3&  \textit{kó'txi'hs}&    &\textit{kó'txi'hkeres}
    \end{tabular}

    
\item\label{ListOfKinshipTerms10} `daughter' (\Hermaphrodite)

    \begin{tabular}{llll}
    ~ &\textsc{sg}&\textsc{du}&\textsc{pl}\\
    1&  \textit{minúuhąks}&   \textit{nunúuhąks}                                         &\textit{nunúuhąkinits}\\
    2&  \textit{ninúuhąks}&    &\textit{ninúuhąkinits}\\
    3&  \textit{konúuhąks}&    &\textit{konúuhąhkeres}
    \end{tabular}

\item\label{ListOfKinshipTerms11} `son' (\Hermaphrodite)

    \begin{tabular}{llll}
    ~ &\textsc{sg}&\textsc{du}&\textsc{pl}\\
    1&  \textit{miníks}&   \textit{nuníks}                                         &\textit{nuníkinits}\\
    2&  \textit{niníks}&    &\textit{niníkinits}\\
    3&  \textit{koníks}&    &\textit{koníhkeres}
    \end{tabular}


\item\label{ListOfKinshipTerms12} `grandchild' or `daughter-in-law' (\Hermaphrodite)

    \begin{tabular}{llll}
    ~ &\textsc{sg}&\textsc{du}&\textsc{pl}\\
    1&  \textit{ptawíihąks}&   \textit{nutáwiihąks}                                         &\textit{nutáwiihąkinits}\\
    2&  \textit{nitáwiihąks}&    &\textit{nitáwiihąkinits}\\
    3&  \textit{kotáwiihąks}&    &\textit{kotáwiihąhkeres}
    \end{tabular}

\item\label{ListOfKinshipTerms13} `sister's child' or `sister's daughter's child' (\Male)

    \begin{tabular}{llll}
    ~ &\textsc{sg}&\textsc{du}&\textsc{pl}\\
    1&  \textit{ptúuhąks}&   \textit{nutúuhąks}                                         &\textit{nutúuhąkinits}\\
    2&  \textit{nitúuhąks}&    &\textit{nitúuhąkinits}\\
    3&  \textit{kotúuhąks}&    &\textit{kotúuhąhkeres}
    \end{tabular}
    
\item\label{ListOfKinshipTerms14} `wife' (\Hermaphrodite)

    \begin{tabular}{llll}
    ~ &\textsc{sg}&\textsc{du}&\textsc{pl}\\
    1&  \textit{mú'hs}&   \textit{núuhs}                                         &\textit{núuhinits}\\
    2&  \textit{nú'hs}&    &\textit{nú'hinits}\\
    3&  \textit{kų́'hs}&    &\textit{kų́'hkeres}
    \end{tabular}

\item\label{ListOfKinshipTerms15} `husband' (\Hermaphrodite)

    \begin{tabular}{llll}
    ~ &\textsc{sg}&\textsc{du}&\textsc{pl}\\
    1&  \textit{mí'wooroos}&   \textit{ríiwooroos}                                         &\textit{ríiwooroonits}\\
    2&  \textit{ní'wooroos}&    &\textit{ní'wooroonits}\\
    3&  \textit{kowóoroos}&    &\textit{kowóorookeres}
    \end{tabular}

\newpage

\item\label{ListOfKinshipTerms16} `husband's brother' (\Female)

    \begin{tabular}{llll}
    ~ &\textsc{sg}&\textsc{du}&\textsc{pl}\\
    1&  \textit{misíks}&   \textit{nusíks}                                         &\textit{nusíkinits}\\
    2&  \textit{nisíks}&    &\textit{nisíkinits}\\
    3&  \textit{kosíks}&    &\textit{kosíhkeres}
    \end{tabular}

    
\item\label{ListOfKinshipTerms17} `wife's brother' or `sister's husband' (\Male)

    \begin{tabular}{llll}
    ~ &\textsc{sg}&\textsc{du}&\textsc{pl}\\
    1&  \textit{wóowakihs}&   \textit{nuwóokihs}                                         &\textit{nuwóohihinits}\\
    2&  \textit{wóorakihs}&    &\textit{wóorakihinits}\\
    3&  \textit{kowóokihs}&    &\textit{kowóokihkeres}
    \end{tabular}

\item\label{ListOfKinshipTerms18} `son-in-law' or `father-in-law' (\Male)

    \begin{tabular}{llll}
    ~ &\textsc{sg}&\textsc{du}&\textsc{pl}\\
    1&  \textit{ptų́ųts}&   \textit{nutų́ųts}                                         &\textit{nutų́ųtinits}\\
    2&  \textit{nitų́ųts}&    &\textit{nitų́ųtinits}\\
    3&  \textit{kotų́ųts}&    &\textit{kotų́ųtkeres}
    \end{tabular}

\item\label{ListOfKinshipTerms19} `son-in-law' (\Hermaphrodite)

    \begin{tabular}{llll}
    ~ &\textsc{sg}&\textsc{du}&\textsc{pl}\\
    1&  \textit{ptaró'hąkas}&   \textit{nutáro'hąkas}                                         &\textit{nutáro'hąkanits}\\
    2&  \textit{nitáro'hąkas}&    &\textit{nitáro'hąkanits}\\
    3&  \textit{kotáro'hąkas}&    &\textit{kotáro'hąkakeres}
    \end{tabular}
    
\item\label{ListOfKinshipTerms20} `brother's wife' (\Male)

    \begin{tabular}{llll}
    ~ &\textsc{sg}&\textsc{du}&\textsc{pl}\\
    1&  \textit{ptaró'hąkamiihs}&   \textit{nutáro'hąkamiihs}                                         &\textit{nutáro'hąkamiihinits}\\
    2&  \textit{nitáro'hąkamiihs}&    &\textit{nitáro'hąkamiihinits}\\
    3&  \textit{kotáro'hąkamiihs}&    &\textit{kotáro'hąkamiihkeres}
    \end{tabular}
    
\item\label{ListOfKinshipTerms21} `husband's sister' or `brother's wife' (\Female)

    \begin{tabular}{llll}
    ~ &\textsc{sg}&\textsc{du}&\textsc{pl}\\
    1&  \textit{ptúus}&   \textit{nutúus}                                         &\textit{nutúunits}\\
    2&  \textit{nitúus}&    &\textit{nitúunits}\\
    3&  \textit{kotúus}&    &\textit{kotúukeres}
    \end{tabular}
    
\item\label{ListOfKinshipTerms22} `mother’s brother' (\Hermaphrodite)

    \begin{tabular}{llll}
    ~ &\textsc{sg}&\textsc{du}&\textsc{pl}\\
    1&  \textit{ptawáratoos}&   \textit{nutáwaratoos}                                         &\textit{nutáwaratoonits}\\
    2&  \textit{nitáwaratoos}&    &\textit{nitáwaratoonits}\\
    3&  \textit{kotáwaratoos}&    &\textit{kotáwaratookeres}
    \end{tabular}

\newpage

\item\label{ListOfKinshipTerms23} `father's sister' (\Hermaphrodite)

    \begin{tabular}{llll}
    ~ &\textsc{sg}&\textsc{du}&\textsc{pl}\\
    1&  \textit{ptúuminiks}&   \textit{nutúuminiks}                                         &\textit{nutúuminikinits}\\
    2&  \textit{nitúuminiks}&    &\textit{nitúuminikinits}\\
    3&  \textit{kotúuminiks}&    &\textit{kotúuminihkeres}
    \end{tabular}



\item\label{ListOfKinshipTerms24} `mother-in-law' (\Male)

    \begin{tabular}{llll}
    ~ &\textsc{sg}&\textsc{du}&\textsc{pl}\\
    1&  \textit{ptúuhiniks}&   \textit{nutúuhiniks}                                         &\textit{nutúuhinikinits}\\
    2&  \textit{nitúuhiniks}&    &\textit{nitúuhinikinits}\\
    3&  \textit{kotúuhiniks}&    &\textit{kotúuhinihkeres}
    \end{tabular}

\end{xlist}

\end{exe}

This list consists of terms given to \citet{bowers1950} by Mandan consultants during his trips to Fort Berthold between 1929 and 1931.\footnote{The Mandan individuals who contributed to \citeauthor{bowers1950}' (\citeyear[4]{bowers1950}) section on kinship terms in the Mandan language were White Calf, Crows Heart, Ben Benson, Bear on the Flat, Foolish Woman, Little Owl, Scattercorn, Mrs. Good Bear, Calf Woman, Mrs. Owen Baker, and Front Woman.} None of the same speakers worked with \citet{hollow1970} by the time he came to Fort Berthold in the late 1960s to compile a dictionary of the Mandan language as his doctoral dissertation at the University of California, Berkeley. There are some differences in how \citeauthor{hollow1970} interprets some kinship terms versus how \citeauthor{bowers1950} does, and it is not clear if these differences are due to certain terms having multiple uses or if there was a change in how the organization of the family unit in Mandan occurred for speakers born in the twentieth century versus those with whom \citeauthor{bowers1950} had worked, the majority of whom were born during the middle of the nineteenth century. The following terms in (\ref{DifferentKinship}) differ from those given above in (\ref{ListOfKinshipTerms}). 

\begin{exe}

\item\label{DifferentKinship} Differences in kinship terms observed in \citet{hollow1970}

\begin{xlist}

\item\label{DifferentKinship1} `husband' (\Hermaphrodite)

    \begin{tabular}{llll}
    ~ &\textsc{sg}&\textsc{du}&\textsc{pl}\\
    1&  \textit{miweróos}&   \textit{nuweróos}                                         &\textit{nuweróonits}\\
    2&  \textit{niweróos}&    &\textit{niweróonits}\\
    3&  \textit{koweróos}&    &\textit{koweróokeres}
    \end{tabular}
    
\item\label{DifferentKinship2} `sister-in-law' (\Male)

    \begin{tabular}{llll}
    ~ &\textsc{sg}&\textsc{du}&\textsc{pl}\\
    1&  \textit{ptaró'hąkas}&   \textit{nutáro'hąkas}                                         &\textit{nutáro'hąkanits}\\
    2&  \textit{nitáro'hąkas}&    &\textit{nitáro'hąkanits}\\
    3&  \textit{kotáro'hąkas}&    &\textit{kotáro'hąkakeres}
    \end{tabular}

\item\label{DifferentKinship3} `mother’s brother's wife' (\Male)

    \begin{tabular}{llll}
    ~ &\textsc{sg}&\textsc{du}&\textsc{pl}\\
    1&  \textit{ptaró'hąkamiihs}&   \textit{nutáro'hąkamiihs}                                         &\textit{nutáro'hąkamiihinits}\\
    2&  \textit{nitáro'hąkamiihs}&    &\textit{nitáro'hąkamiihinits}\\
    3&  \textit{kotáro'hąkamiihs}&    &\textit{kotáro'hąkamiihkeres}
    \end{tabular}
    
\item\label{DifferentKinship4} `father-in-law' (\Male)

    \begin{tabular}{llll}
    ~ &\textsc{sg}&\textsc{du}&\textsc{pl}\\
    1&  \textit{pta'íwaratookas}&   \textit{nutá'iwaratookas}                                         &\textit{nutá'iwaratookanits}\\
    2&  \textit{nitá'iwaratookas}&    &\textit{nitá'iwaratookanits}\\
    3&  \textit{kotá'iwaratookas}&    &\textit{kotá'iwaratookakeres}
    \end{tabular} 

% \item\label{DifferentKinship5} `old man' (\Hermaphrodite) or `husband' (\Female)

%     \begin{tabular}{llll}
%     ~ &\textsc{sg}&\textsc{du}&\textsc{pl}\\
%     1&  \textit{ptawáratookaxi'hs}&   \textit{nutáwaratookaxi'hs}                                         &\textit{nutáwaratookaxi'hinits}\\
%     2&  \textit{nitáwaratookaxi'hs}&    &\textit{nitáwaratookaxi'hinits}\\
%     3&  \textit{kotáwaratookaxi'hs}&    &\textit{kotáwaratookaxi'hkeres}
%     \end{tabular}
    
% \item\label{DifferentKinship6} `old woman' (\Hermaphrodite) or `wife' (\Male)

%     \begin{tabular}{llll}
%     ~ &\textsc{sg}&\textsc{du}&\textsc{pl}\\
%     1&  \textit{ptarókąąkas}&   \textit{nutárokąąkas}                                         &\textit{nutárokąąkanits}\\
%     2&  \textit{nitárokąąkas}&    &\textit{nitárokąąkanits}\\
%     3&  \textit{kotárokąąkas}&    &\textit{kotárokąąkakeres}
%     \end{tabular}

    
\end{xlist}

\end{exe}

The word for `husband' in (\ref{DifferentKinship1}) as given in \citet[306]{hollow1970} is supposedly the general Nuu'etaare form. He contrasts this word with `husband' in (\ref{ListOfKinshipTerms15}), where \textit{kowóoroos} is supposed to be the Nuptaare dialectal version of the word. This statement does not appear to be accurate, as the supposedly Nuptaare form is the term used for `husband' in the corpus in over 90 per cent of all instances of the term `husband'. It is likely that \citet{hollow1970} switched the two dialectal versions, given the fact that almost all consultants who worked with both \citet{bowers1950} and \citet{hollow1970} self-identified as Nuu'etaare speakers. Another possibility is that the Nuptaare version of the word `husband' is preferred for some other reason, such as an attempt to exoticize the events being described in some narratives as having happened long ago, when there was more dialectal variety in the Mandan language during the period of time when the different villages had their own speech varieties.

\citet[244]{hollow1970} identifies \textit{kotáro'hąkas} as `sister-in-law' in (\ref{DifferentKinship2}), but \citet[40]{bowers1950} equates it with `son-in-law'. \citeauthor{bowers1950}'s (\citeyear{bowers1950}) consultants note that the term \textit{ró'hąka} means `young person' or `young one.' As such, this term may not strictly be a kinship term in as much as it is an epithet for younger in-laws. Likewise, we see the term \textit{kotáro'hąkamiihs} listed as `brother's wife' in (\ref{ListOfKinshipTerms20}), but as `mother's brother's wife' in (\ref{DifferentKinship3}). There does not seem to be a conflict between the translations for these words, because they both appear to refer to a woman who has married a man who is part of one's mother's clan. As such, a more apt description of this term is any woman who marries into one's clan.

The term for `father-in-law' in (\ref{DifferentKinship4}) is the form most commonly seen in both \citet{hollow1970,hollow1973a} and \citet{trechter2012}. The word \textit{kotá'iwaratookas} is perhaps used to disambiguate between \textit{kotų́ųts}, which can mean both `father-in-law' and `son-in-law'. \citet[55]{bowers1950} reports that \textit{kotá'iwaratookas} should never be used to refer to a father-in-law whose clan is different than one's own. Therefore, it is likely that this term is more specific to refer to men who are part of one's own clan but who are also their father-in-law. The term \textit{kotų́ųts} may be more closely aligned with a more generalized word for men who marry into one's clan or men into whose clan one marries. Using \textit{kotá'iwaratookas} clears up any confusion over whether one is referring to their wife's father or her brother.
 
In addition to having different forms based on what relation the speaker has to the relative being discussed, certain kinship terms have a suppletive form when used as a vocative. These suppletive vocative forms appear in (\ref{KinshipVocatives}) below.

\begin{exe}

\item\label{KinshipVocatives} Vocative forms of kinship terms

    \begin{xlist}
    
    \item\label{KinshipVocatives1} \textit{ná'e} `mother!'
    
    \item\label{KinshipVocatives2} \textit{taté} `father!' 
    
    \item\label{KinshipVocatives3} \textit{náxi'hs} `grandmother!'
    
    \item\label{KinshipVocatives4} \textit{tatáxi'hs} `grandfather!'
    
    \item\label{KinshipVocatives5} \textit{ró'hąka} `son-in-law!' or `young one!'
    
    \item\label{KinshipVocative6} \textit{ratóore} `mother's brother!'
    \end{xlist}

\end{exe}

In addition to these vocative terms, it is also common for older women to use \textit{warátookaxi'he} `old man' when addressing their older husbands, and likewise, older men often call their older wives \textit{roką́ąka} `old woman.'

\section{Pronouns}\label{SecPronouns}

In \sectref{Ch2InflectionalPrefixes}, I describe
a variety of inflectional prefixes that I referred to as pronominals. These prefixes are not pronouns in the strictest sense, but agreement markers. It is not clear that Mandan has ``true'' pronouns that refer to individual persons in the sense that they are stand-alone elements that are neither personal pronouns nor proforms \citet{bhat2004}. We can divide pronouns and pronoun-like elements into two classes: what I shall call independent pronouns and interrogative pronouns. Independent pronouns are addressed in \S{SubSecIndependentPronouns}, and I describe interrogative pronouns in \sectref{SubSecInterrogativePronouns} below.


\subsection{Independent pronouns}\label{SubSecIndependentPronouns}

There are certainly elements in Mandan that speakers translate into English with an English pronoun, but these elements are doing more than indicating the person or number of the argument to which the speaker is referring. The so-called pronouns in Mandan really serve as discourse elements that can topicalize an argument, reinforce an argument, or remind the listener of an argument. Where pronoun-like words occur in Mandan, they are clause-level elements that affect the information structure of an utterance. For the sake of ease, I shall refer to them as independent pronouns in the sense that are free syntactic words that carry the semantics of identifying some individual or individuals in the discourse, but it is still worth reminding the reader that they are not pronouns in the same sense that English pronouns are pronouns. Mandan independent pronouns are never obligatory, nor are they predictable where in a sentence they will appear.

There are three words in Mandan that I refer to as independent pronouns. We can see these pronouns in \tabref{PronounsInMandan} below.

\begin{table}

\caption{Independent pronouns in Mandan}\label{PronounsInMandan} 

    \begin{tabular}{llll}
\lsptoprule
    ~   &\textbf{Emphatic}&  \textbf{Contrastive}&\textbf{Relativizing}\\
\midrule
    \textsc{1sg}&\textit{mí'shak}&\textit{mí'o'rak}&\textit{mí'o'na}\\
    \textsc{2sg/2pl}&\textit{ní'shak}&\textit{ní'o'rak}&\textit{ní'o'na}\\
    \textsc{3sg/3pl}&\textit{ishák}&\textit{í'o'rak}&\textit{í'o'na}\\    
    \textsc{1pl}&\textit{núushák}&\textit{núu'o'rak}&\textit{núu'o'na}\\    
    \lspbottomrule
    \end{tabular}

\end{table}



The bare forms of the pronouns refer to third person entities. Independent pronouns in Mandan are not marked for number, sex, or any other property of the referents aside from person. The only distinction for number in pronouns relates to first person marking, and this is grounded in the fact that there are unique pronominal prefixes for first personal singular and first person plural. This is not the case for second or third person, given that plurality is typically realized in the form of enclitics like =\textit{nit} and =\textit{kere}, respectively.

\subsubsection{Emphatic pronoun: \textit{ishák}}\label{SubSubSecEmphaticPronoun}

The most common form of independent pronoun is the emphatic pronoun. It is most often translated as just `X', `even X', or `X Xself', where X is the person, e.g., \textit{mí'shak} `I', `I myself', or `even I.' The emphatic pronoun can also be used as explicit possession, e.g., \textit{mí'shak} `mine' or `my'. We can see these emphatic pronouns in the data in (\ref{EmphaticPronounEx}) below.

\begin{exe}

\item\label{EmphaticPronounEx} Examples of the emphatic pronoun

    \begin{xlist}
    
    \item\label{EmphaticPronounEx1} \glll Numá'kshi Karáaha \textbf{ishák} rás ótu ~ ~ ~ ~ ~ ~ ~ ~ ~ kiná'mahere'sh\\
    ruwą'k\#shi kraah=E=$\varnothing$ \textbf{ishak} ras o-tu ~ ~ ~ ~ ~ ~ ~ ~ ~ kirą'\#wą-hrE=o'sh\\
    \textnormal{man}\#\textnormal{be.good} \textnormal{be.afraid}=sv=$\varnothing$ \textbf{pro} \textnormal{name} pv.irr-\textnormal{be.some} ~ ~ ~ ~ ~ ~ ~ ~ ~ \textnormal{tell}\#1s-caus=ind.m\\
    \glt `Afraid to be Chief himself told me to tell the name he has' \citep[64]{hollow1973a}
    
    \item\label{EmphaticPronounEx2} \glll weróohkeres réehkerek, \textbf{ishák}, ~ ~ ~ ~ ~ ~ ~ ~ ~ ~ ~ ~ ~ ~ ~ ~ kotų́ųts, kisúkini...\\
    wrook=krE=s rEEh=krE=ak \textbf{ishak} ~ ~ ~ ~ ~ ~ ~ ~ ~ ~ ~ ~ ~ ~ ~ ~  ko-tųųt=s ki-suk=rį\\
    \textnormal{buffalo.bull}=3pl=def \textnormal{go.there}=3pl=ds \textbf{pro} ~ ~ ~ ~ ~ ~ ~ ~ ~ ~ ~ ~ ~ ~ ~ ~ 3poss.pers-\textnormal{son-in-law}=def vert-\textnormal{exit}=ss\\
    \glt `With those buffalo bulls having gone, he, her son-in-law, went back out and...' \citep[119]{hollow1973a}
    
    \item\label{EmphaticPronounEx3} \glll Réeharata, \textbf{ní'shak}!\\
    rEEh\#hrE=ta \textbf{r'\~~-ishak}\\
    \textnormal{go.there}\#caus=imp.m \textbf{2poss-pro}\\
    \glt `It is your turn, you!' [lit. `you go, you!'] \citep[160]{hollow1973a}
    
    \item\label{EmphaticPronounEx4} \glll \textbf{Mí'shak} máa'ąk íwasek pshíiwahaani...\\
    \textbf{w'\~~-ishak} waa'ąk i-wa-sek pshii\#wa-hrE=rį\\
    \textbf{1sg.poss-pro} \textnormal{land} pv.ins-1a-\textnormal{make} \textnormal{be.flat}\#1a-caus=ss\\
    \glt `I myself, the land that I made, I made it flat and...' \citep[4]{hollow1973a}
    
    \item\label{EmphaticPronounEx5} \glll \textbf{Mí'shak} iną́k ptaną́ąku'sh\\
    \textbf{w'\~~-ishak} irąk p-ta-rąąku=o'sh\\
    \textbf{1sg.poss-pro} \textnormal{again} 1sg.poss-al-\textnormal{road}=ind.m\\
    \glt `It's \textit{my} road too' \citep[24]{hollow1973a}
    
    \item\label{EmphaticPronounEx6} \glll Ratóore, \textbf{ní'shak} inák éetekto'sh\\
    ratoo=E=$\varnothing$ \textbf{r'\~~-ishak} irąk ee-te=kt=o'sh\\
    \textnormal{be.mature}=sv=voc \textbf{2poss-pro} \textnormal{again} pv-\textnormal{say}.2sg=pot=ind.m\\
    \glt `Elder, even you should say say it' \citep[30]{hollow1973a}
    
    \item\label{EmphaticPronounEx7} \glll Manáhinii ret, \textbf{ní'shak} ~ ~ ~ ~ ~ ~ ~ ~ ~ ~ ~ ~ ~ ~ ~ ~ ~ ~ ~ ~ ~ ~ ~ ~ ~ nitáwookisita!\\
    wrąhrįį re=t \textbf{r'\~~-ishak} ~ ~ ~ ~ ~ ~ ~ ~ ~ ~ ~ ~ ~ ~ ~ ~ ~ ~ ~ ~ ~ ~ ~ ~ ~ rį-ta-wV-o-ki-si=ta\\
    \textnormal{spring} dem.prox=loc \textbf{2poss-pro} ~ ~ ~ ~ ~ ~ ~ ~ ~ ~ ~ ~ ~ ~ ~ ~ ~ ~ ~ ~ ~ ~ ~ ~ ~ 2poss-al-nom-pv.irr-mid-\textnormal{command}=imp.m\\
    \glt `the this one in the spring [water], \textit{you} take him as a slave' \citep[158]{hollow1973a}
    
    \end{xlist}

\end{exe}

The emphatic pronoun is typically used to reinforce a subject, as we see in (\ref{EmphaticPronounEx1}), (\ref{EmphaticPronounEx2}), (\ref{EmphaticPronounEx4}), and (\ref{EmphaticPronounEx6}). We can also see the emphatic pronoun used as a possessive in (\ref{EmphaticPronounEx5}), where the Mandan consultant said that \textit{mí'shak} means `mine' in this context .The emphatic pronoun in (\ref{EmphaticPronounEx3}) seems to be used in a vocative sense. Lastly, the \textit{ní'shak} in (\ref{EmphaticPronounEx7}) is ambiguous whether it is it truly indicating a subject or an indirect object, i.e., `\textit{you} take him as a slave' versus `take him as a slave for yourself.' Either possibility results in the emphatic pronoun being co-referential with the subject. No instances of emphatic pronouns being used as direct objects or playing other roles are attested in the corpus. As such, emphatic pronouns appear to be restricted to emphasizing the subject of a clause or a possessor that is co-referential with a clause-final noun that is possessed, as we see in (\ref{EmphaticPronounEx5}).

\subsubsection{Contrastive pronoun: \textit{í'o'rak}}\label{SubSubSecContrastivePronoun}

The contrastive pronoun by its very nature requires some contrast between the subject of one clause and the one in which the contrastive pronoun occurs. It appears to be composed of the possessive preverb \textit{i}-, plus the copula \textit{ó'} `be' with the different subject marker =\textit{ak}. As such, this is not a pronoun in the strictest sense, given the fact it is composed of a verb and other clausal morphology, but its distribution and treatment by speakers is pronominal in nature. 

Translations of this pronoun in English often appear as `X is the one', `as for X', or `X, however.' The contrastive pronoun can also be rendered into English with an independent pronoun alone, but there will be some accompanying prosody to act as a cue for contrast. Rather than relying on prosody to convey this contrastive reading, Mandan has overt pronouns, as we see in (\ref{ContrastivePronounEx}) below.

\begin{exe}

\item\label{ContrastivePronounEx} Examples of the contrastive pronoun

\begin{xlist}

\item\label{ContrastivePronounEx1} \glll Kiwáa'o'nixinash papshíikiniko're. ~ ~ ~ ~ ~ ~ ~ ~ ~ ~ ~ \textbf{Ní'orak}, rapaákxųhki ~ ~ ~ ~ ~ ~ ~ ~ ~ ~ ~ ~ miníreet ókipxe're.\\
    ki-waa-o'=rįx=rąsh pa-pshiik=rįk=o'sh ~ ~ ~ ~ ~ ~ ~ ~ ~ ~ ~ \textbf{r'\~~-i-o'=ak} ra-pa-kxųh=ki ~ ~ ~ ~ ~ ~ ~ ~ ~ ~ ~ ~ wrį=ee=t o-ki-pxE=o're\\
    mid-neg-\textnormal{be}=neg=att ins.push-\textnormal{push}=iter=ind.f ~ ~ ~ ~ ~ ~ ~ ~ ~ ~ ~ \textbf{2poss-pv.poss-\textnormal{\bfseries be}=ds} 2a-ins.push-\textnormal{lie.down}=cond ~ ~ ~ ~ ~ ~ ~ ~ ~ ~ ~ ~ \textnormal{water}=dem.dist=loc pv.irr-vert-\textnormal{stumble}=ind.f\\
    \glt `He will not be able to push you off. You, however, if you push him down on the water, he will land' \citep[105]{hollow1973a}
    
\item\label{ContrastivePronounEx2} \glll \textbf{Ní'o'rak}, wáa'ihųte résh ruháaro'sh\\
    \textbf{r'\~~-i-o'=ak} waa-i-hųt=E resh rų-haa=o'sh\\
    \textbf{2poss-pv.poss-\textnormal{\bfseries be}=ds} nom-pv.ins-\textnormal{burst}=sv \textnormal{like.this} 1a.pl-\textnormal{try}=ind.m\\
    \glt `you are the one, [it is your] fault we are like this' \citep[50]{trechter2012}
    
\item\label{ContrastivePronounEx3} \glll Káni \textbf{ní'o'rak} nipáaxu kshúkanashini ~ ~ ~ ~ ~ óraptehe tú ó'xere'sha?\\
    ka=rį \textbf{r'\~~-i-o'=ak} rį-paaxu kshuk=rąsh=rį ~ ~ ~ ~ ~  o-ra-ptEh=E tú o'xrE=o'sha\\
    \textnormal{prov}=ss \textbf{2poss-pv.poss-\textnormal{\bfseries be}=ds} 2poss-\textnormal{nose} \textnormal{be.narrow}=att=ss ~ ~ ~ ~ ~ pv.irr-2a-\textnormal{run}=sv \textnormal{be.some} dub=int.m\\
    \glt `And as for you, your nose is kind of narrow, so how could you even run at all?' \citep[59]{hollow1973b}
    
\item\label{ContrastivePronounEx4} \glll Kotewé nurátoora'shka? [...] ~ ~ ~ ~ ~ ~ ~ ~ ~ ~  ~ ~ ~ ~ ~ ~ ~ ~ ~ ~ \textbf{Komí'ma'o'rak}!\\
    ko-t-we rų-ratoo=a'shka ~ ~ ~ ~ ~ ~ ~ ~ ~ ~ ~ ~ ~ ~ ~ ~ ~ ~ ~ ~ ~ \textbf{ko-w'\~~-i-wą-o'=a}k\\
    rel-wh-indf 1a.pl-\textnormal{be.mature}=psbl ~ ~ ~ ~ ~ ~ ~ ~ ~ ~ ~ ~ ~ ~ ~ ~ ~ ~ ~ ~ ~ \textbf{rel-1poss-pv.poss-1s-\textnormal{\bfseries be}=ds}\\
    \glt `Who of us could possibly be older? [...] I am the one [who is]!' \citep[6]{hollow1973a}

\end{xlist}

\end{exe}

The contrastive pronoun has a very limited distribution, as it only arises in a back and forth between speakers. Such situations are not common in the corpus, which are ultimately comprised of traditional narratives where dialog is limited and the narrators often paraphrase what is said instead of giving full quotes of what the figures in the narrative are saying to each other. In conversation, these kinds of pronouns are more common than in the corpus. This pronoun is most often found for second person, as it is associated with interpersonal arguments over who did what, and it occurs in propositions where the speaker is assigning blame or accusing the listener. When used for the first person, this pronoun is directly contradicting something the listener has just said.

In (\ref{ContrastivePronounEx4}), we see double first person singular marking. One exponent of the first person singular feature is the \textit{ma}-, which is the subject of the verb \textit{ó'} `be'. The preverb \textit{i}- converts this verb into a nominal element, which then takes the first person singular possessive prefix /w'\~~-/. This is a rare manifestation of multiple exponence in Mandan. Other Siouan languages, such as Crow, have similar constructions where pronoun-like constructions can be added to words that already bear person marking. We can see examples of multiple instances of pronominal marking in Crow in (\ref{MultipleExponenceInCrow}) below.

\newpage

\begin{exe}

\item\label{MultipleExponenceInCrow} Multiple exponence in Crow pronouns

    \begin{xlist}
    
    \item\label{MultipleExponenceInCrow1} \glll hinne ahpaaxéesh koon Akbaatatdía iláak ~ ~ ~ ~ ~ ~ ~ ~ ~ ~ hilíasheek, ``hinne \textbf{biiwa}laakbacheék''\\
    hinne ahpaaxée-sh koon ak-baatách-día ilíi-ak ~ ~ ~ ~ ~ ~ ~ ~ ~ ~ hilía-shée-k hinne \textbf{bii-ba-}dáaka-bacheé-k\\
    \textnormal{this} \textnormal{cloud}-det \textnormal{source} rel-\textnormal{everything}-\textnormal{make} \textnormal{speak}-ss ~ ~ ~ ~ ~ ~ ~ ~ ~ ~ \textnormal{this}-\textnormal{say}-decl \textnormal{this} \textbf{1pro-1poss-}\textnormal{child}-\textnormal{man}-decl\\
    \glt `God spoke from this cloud and said this: ``this is my son{''}' \citep[61]{graczyk2007}
    
    \item\label{MultipleExponenceInCrow2} \glll éehk \textbf{biiwa}chuukák\\
    eehk \textbf{bii-ba-}chuuká-k\\
    dem.mid \textbf{1pro-1poss-}\textnormal{younger.brother}-decl\\
    \glt `that one is \textit{my} younger brother' \citep[61]{graczyk2007}
    
    \end{xlist}

\end{exe}

The Crow pronouns co-occur with other person marking in the data above, similar to how the possessive plus preverb combination works in Mandan in (\ref{ContrastivePronounEx4}). Looking at multiple narratives in Crow (e.g., \citealt{medicinehorse1986}), this phenomenon appears more often in Crow than it does in Mandan. Given the fact that the corpus consists of mostly traditional narratives rather than actual conversation, it is unclear how common this double marking of personhood in Mandan is outside of the domain of recounting of the domain of the narrative. Recordings of dialogs in Mandan may exist at the library at Nueta Hidatsa Sahnish College in New Town, ND, but access to such recordings has been restricted to outside availabilty as of this writing.

\subsubsection{Relativizing pronoun: \textit{í'o'na}}\label{SubSubSecRelativizingPronoun}

The relativizing pronoun is like the contrastive pronoun in that it is really a verbal construction consisting of the possessive preverb \textit{i}-, the verb \textit{ó'} `be', and the topic marker =\textit{na}. When this pronoun is used, it carries the meaning of `It is X who', where it precedes a verb or is immediately postposed after the matrix verb in a sentence. These pronouns are mostly attested in the \citet{hollow1970} dictionary and in the Mandan grammar by \citet[26]{kennard1936}, where \citeauthor{kennard1936} states that they are simply treated as relative clauses. We can see examples of these pronouns in (\ref{ExamplesOfRelativizingPronoun}) below.

\newpage

\begin{exe}

\item\label{ExamplesOfRelativizingPronoun} Examples of relativizing pronoun

    \begin{xlist}
    
    \item\label{ExamplesOfRelativizingPronoun1} \glll \textbf{mí'o'na} wáa'eepe mikó'sh\\
    \textbf{w'\~~-o'=ra}̨ waa-ee-pe wįk=o'sh\\
    \textbf{1sg.poss-pv.poss-\textnormal{\bfseries be}=top} nom-pv-\textnormal{say}.1sg \textnormal{be.none}=ind.m\\
    \glt `I am the one who did not say anything' \citep[244]{hollow1973b}
    
    \item\label{ExamplesOfRelativizingPronoun2} \glll hiré \textbf{núu'o'na} ą́'skanuhere'sh\\
    hire \textbf{nuu-o'=rą} ą'ska\#rų-hrE=o'sh\\
    \textnormal{now} \textbf{1pl.poss-pv.poss-\textnormal{\bfseries be}=top} \textnormal{that.way}\#1a.pl-caus=ind.m\\
    \glt `Now we are the ones who did it that way' \citep[176]{hollow1973b}
    
    \item\label{ExamplesOfRelativizingPronoun3} \glll \textbf{í'o'na} téeharani...\\
    \textbf{i-o'=rą} tee\#hrE=rį\\
    \textbf{pv.poss-\textnormal{\bfseries be}=top} \textnormal{die}\#caus=ss\\
    \glt `he was the one who killed her...' \citep[178]{hollow1973a}
    
    \item\label{ExamplesOfRelativizingPronoun4} \glll mí'shak, \textbf{maní'o'na} ą́'skarahara'shka ~  éewaharani...\\
    w'\~~-ishak \textbf{wa-r'\~~-i-o'=rą} ą'ska\#ra-hrE=a'shka ~  ee-wa-hrE=rį\\
    1sg.poss-pro \textbf{unsp-2poss-pv.poss-\textnormal{\bfseries be}=top} \textnormal{that.way}\#2a-caus=psbl ~ pv-1a-caus=ss\\
    \glt `Me, I thought you were the one who maybe did something...' \citep[238]{hollow1973b}

    \end{xlist}

\end{exe}

In each of the examples of the relativizing pronouns above, the subject is the element that is relativized. This pronoun is used sparingly in the corpus, and it seems to be more common in speakers born in the mid nineteenth century than in speakers born in the twentieth century. The data in (\ref{ExamplesOfRelativizingPronoun4}) contain a rare example of multiple pronouns being used within the same sentence. The emphatic pronoun is used first to topicalize the subject of the matrix clause, while the relativizing pronoun is the subject of a subordinate clause. The second person pronoun in (\ref{ExamplesOfRelativizingPronoun4}) also bears the unspecified argument marker /wa-/, which seems to amplify the uncertainty the speaker has over who was the one who did the action in question.

\subsection{Interrogative pronouns}\label{SubSecInterrogativePronouns}

While the personal pronouns described in \sectref{SubSecIndependentPronouns} above are somewhat marginal in the corpus, the interrogative pronouns in Mandan are quite common. Where English has \textit{wh}-words, Mandan has \textit{t}-words. These words all derive from the Proto-Siouan interrogative *ta(a). All interrogative pronouns in Mandan are derived from this same root, manifesting either was /ta/ or just /t/. A list a these interrogative pronouns appears in (\ref{ListOfInterrogativePronouns}) below.

\begin{exe}

\item\label{ListOfInterrogativePronouns} List of interrogative pronouns

    \begin{xlist}
    \item \textit{tá} `what' (said in surprise)
    \item \textit{tewé} `who' 
    \item \textit{kotewé} `who, what, which'
    \item \textit{watewé} `what' 
    \item \textit{tewét}$\sim$\textit{tewétaa} `where'
    \item \textit{tashká} `how' 
    \item \textit{tashká'eshka} `how (reason)'
    \item \textit{tashkáhąą} `how (means)'
    \item \textit{táhąą} `how (instrument)'
    \item \textit{tashkák} `why'
    \item \textit{ta'ą́ą} `how many'
    \item \textit{taxkó} `when'
    \item \textit{tákaki} `when'
    \end{xlist}

\end{exe}

The simple \textit{tá} is never used as part of a true question; it is more of an exclamation or reaction to something that someone has said or done. The remaining interrogative pronouns are used in asking questions or substituting for arguments or adjuncts in a sentence.

\subsubsection{Interrogative `who': \textit{tewé}}\label{SubSubSecTewe}

This interrogative pronoun is the most common one used for replacing an entire noun phrase. It is not used as a relativizer. It is underlyingly /t-we/, where the /-we/ is some element inherited from Proto-Siouan that may have indicated indefiniteness.\footnote{We see another instance of Proto-Siouan *-we in the reconstruction for the possessive *i-hta-we, where some Siouan languages use forms with reflexes of *-we when describing a nominalized version of a possessive, e.g., Lakota third person possessive prefix \textit{tȟa}-, which appears before the possessed noun, but the full possessive pronoun \textit{tȟáwa} is used to express something closer to `it is his/hers/theirs', e.g., \textit{tȟaólowaŋ waštéšte} `his songs are (always) good' versus \textit{olówaŋ tȟáwa kiŋ waštéšte}, which is closer to `the songs of his/hers/theirs are (always) good' \citep[188ff]{ullrichblackbear2016}.} We can see examples of this interrogative pronoun in the data in (\ref{ExamplesOfTewe}) below.

\begin{exe}

\item\label{ExamplesOfTewe} Examples of \textit{tewé} `who'

    \begin{xlist}
    
    \item\label{ExamplesOfTewe1} \glll \textbf{tewéna} ą́'teroo máatihtaa ~ ~ ~ ~ ~ ~ ~ ~ ~ ~ ~ ~ ~ ~ ~ ~ ~ réeho'xere'sha?\\
    \textbf{t-we=rą} ą't=roo wąątih=taa ~ ~ ~ ~ ~ ~ ~ ~ ~ ~ ~ ~ ~ ~ ~ ~ ~ rEEh=o'xrE=o'sha\\
    \textbf{wh-indf=top} dem.anap=dem.mid \textnormal{outside}=loc ~ ~ ~ ~ ~ ~ ~ ~ ~ ~ ~ ~ ~ ~ ~ ~ ~ \textnormal{go.there}=dub=int.m\\
    \glt `Who would even want to go there, outside?' \citep[108]{hollow1973b}
    
    \item\label{ExamplesOfTewe2} \glll \textbf{tewé} ó'rak ówokahąą...?\\
    \textbf{t-we} o'=ak o-woka=hąą\\
    \textbf{wh-indf} \textnormal{be}=ds pv.loc-\textnormal{follow}=sim\\
    \glt `who is  it that is following...?' \citep[66]{hollow1973a}
    
    \item\label{ExamplesOfTewe3} \glll \textbf{tewé} ų́ųte pask\'{ı̨}įhki   tapt\'{ı̨}įkto'sh\\
    \textbf{t-we} ųųtE pa-skįįh=ki ta-ptįį=kt=o'sh\\
    \textbf{wh-indf} \textnormal{be.first} ins.push-\textnormal{slash}=cond al-\textnormal{bison}=pot=ind.m\\
    \glt `when someone slashes it first, it will be his buffalo' \citep[6]{hollow1973b}
    
    \end{xlist}

\end{exe}

This pronoun is able to bear the topic marker =\textit{na}, as we see in (\ref{ExamplesOfTewe1}), emphasizing the person who would do such a thing, rather than the action itself. This interrogative pronoun, like all other interrogative pronouns can be used as a kind of indefinite pronoun, as we see in (\ref{ExamplesOfTewe3}). The \textit{tewé} in the aforementioned sentence is rendered into English as `someone' or `somebody', though the sentence can also be translated into English as `when whoever it is slashes it first, it will be his buffalo.' This is an alternative to using the alternative reading of the prefix \textit{waa}- or its free form \textit{wáa} as discussed in \sectref{SubSubSubSectionIndefiniteSubject}. It is unclear what the pragmatic difference is between these two elements, and the corpus does not yield enough instances of contrast between the two to draw adequate conclusions. One suspicion is that the interrogative pronouns, when used as indefinite pronouns, are even less definite than the bound \textit{waa-} or free \textit{wáa} pronominals, given the fact that the indefinite pronouns are often accompanied by adding `-ever' to the English translation, i.e., `someone' versus `whoever it is.'

\subsubsection{Interrogative `who, what, which': \textit{kotewé}}\label{SubSubSecKotewe}

This interrogative pronoun consists of the relativizer \textit{ko}-, plus \textit{tewé}. It can be used to replace any argument in a clause, allowing it to refer to human or non-human entities. This flexibility means that \textit{kotewé} is a very common interrogative pronoun in the corpus. We can see examples of \textit{kotewé} in (\ref{ExamplesOfKotewe}) below.

\begin{exe}
\item\label{ExamplesOfKotewe} Examples of \textit{kotewé}

    \begin{xlist}
    \item\label{ExamplesOfKotewe1} \glll \textbf{kotewé} ó're, íwahąąxiko'sh\\
    \textbf{ko-t-we} o'=re i-wa-hąąxik=o'sh\\
    \textbf{rel-wh-indf} \textnormal{be}=dem.prox pv.ins-1a-\textnormal{not.know}=ind.m\\
    \glt `I do not know which ones these are' \citep[153]{hollow1973a}
    
    \item\label{ExamplesOfKotewe2} \glll Numá'k Máxana éeheni Kinúma'kshi ~ ~ ~ ~ ~ ~ ~ ~ ~ ~ ~ ~ ~   íkirookereroomako'sh, \textbf{kotewé} ~ ~ ~ ~ ~ ~ ~ ~ ~ ~ ~ ~ óratoore.\\
    ruwą'k wąxrą ee-he=rį ki-ruwą'k\#shi ~ ~ ~ ~ ~ ~ ~ ~ ~ ~ ~ ~ ~ i-ki-roo=krE=oowąk=o'sh \textbf{ko-t-we} ~ ~ ~ ~ ~ ~ ~ ~ ~ ~  ~ ~ o-ratoo=E\\
    \textnormal{man} \textnormal{one} \textnormal{and} mid-\textnormal{man}\#\textnormal{be.good} ~ ~ ~ ~ ~ ~ ~ ~ ~ ~ ~ ~ ~ pv.ins-itr-\textnormal{speak}=3pl=narr=ind.m \textbf{rel-wh-indf} ~ ~ ~ ~ ~ ~ ~ ~ ~ ~ ~ ~ pv.irr-\textnormal{be.mature}=sv\\
    \glt `Lone Man and First Creator argued about it, which one was older' \citep[1]{hollow1973a}
    
    \item\label{ExamplesOfKotewe3} \glll \textbf{kotewé} nuk\'{ı̨}'kinitki, ~ ~ ~ ~ ~ ~ ~ ~ ~ ~ ~ ~ ~ ~ ~ ~ ~ ~ ~ ~ kanúwaka'nito'sh\\
    \textbf{ko-t-we} rų-kį'k=rįt=ki ~ ~ ~ ~ ~ ~ ~ ~ ~ ~ ~ ~ ~ ~ ~ ~ ~ ~ ~ ~  ka-rų-wa-ka'=rįt=o'sh\\
    \textbf{rel-wh-indf} 1a.pl-\textnormal{finish}=2pl=cond ~ ~ ~ ~ ~ ~ ~ ~ ~ ~ ~ ~ ~ ~ ~ ~ ~ ~ ~ ~ agt-1a.pl-unsp-\textnormal{have}=2pl=ind.m\\
    \glt `Whichever one of us gets through first, that is the one of us who will have it' \citep[43]{hollow1973a}
    
    \item\label{ExamplesOfKotewe4} \glll \textbf{kotewé} nurátoora'shka?\\
    \textbf{ko-t-we} rų-ratoo=a'shka\\
    \textbf{rel-wh-indf} 1a.pl-\textnormal{be.mature}=psbl\\
    \glt `who of us is older?' \citep[6]{hollow1973a}
    
    \item\label{ExamplesOfKotewe5} \glll máa'ąk ówashkat, \textbf{kotewé} hą́ąka, ~ ~ ~ ~ ~ ~ ~  ą́'teroo ó'ki: ``Káare ráahta.''\\
    waa'ąk o-washka=t \textbf{ko-t-we} hąąkE=$\varnothing$ ~ ~ ~ ~ ~ ~ ~   ą't=roo o'=ki kaare rEEh=ta\\
    \textnormal{land} pv.loc-\textnormal{be.high}=loc \textbf{rel-wh-indf} \textnormal{stand}.aux=cont ~ ~ ~ ~ ~ ~ ~ {dem.anap}=dem.mid
 \textnormal{be}=cond neg.imp \textnormal{go.there}=imp.m\\
    \glt `on the hill where it is high, whichever one he was standing on, when he was there, [he said]: ``Do not go there.{''}' \citep[93]{hollow1973a}
    
    \end{xlist}
\end{exe}

As the data above show, \textit{kotewé} can be used when the speaker wishes to constrain possible options. As such, it functions closer to the English word `which' than to either `who' or `what'. It is the case, however, that \textit{kotewé} occurs most often in the corpus with human arguments.

\subsubsection{Interrogative `what': \textit{watewé}}\label{SubSubSectionWatewe}

The interrogative pronoun combines the unspecified argument marker \textit{wa}- with the interrogative pronoun \textit{tewé}. This particular pronoun is used exclusively for non-human entities. Within the corpus, this pronoun often appears utterance-initially. This utterance-initial position often results in the initial /w/ being realized as [m], i.e., [ma.ˈtᵉwe], not *[mã.ˈtᵉwe]. We can see examples of this pronoun in (\ref{ExamplesOfWatewe}) below.

\begin{exe}

\item\label{ExamplesOfWatewe} Examples of \textit{watewé}

\begin{xlist}

\item\label{ExamplesOfWatewe1} \glll \textbf{matewé} órarukų'ro'sha?\\
    \textbf{wa-t-we} o-ra-ru-kų'=o'sha\\
    unsp-wh-indf pv.irr-2a-ins.hand-\textnormal{give}=int.m\\
    \glt `what will you give for it?' \citep[29]{hollow1973a}
    
\item\label{ExamplesOfWatewe2} \glll \textbf{matewék}, sháa manaháni éererehini...\\
    \textbf{wa-t-we=ak} shaa w-rą-hE=rį ee-re-reh=rį\\
    \textbf{unsp-wh-indf=ds} \textnormal{always} 1s-2a-\textnormal{see}=ss pv-2a-\textnormal{want}=ss\\
    \glt `whatever it is, you always want to look at me and...' \citep[95]{hollow1973a}
    
\item\label{ExamplesOfWatewe3} \glll \textbf{matewéna} rénak?!\\
    \textbf{wa-t-we=rą} re=rąk\\
    \textbf{unsp-wh-indf=top} dem.prox=pos.sit\\
    \glt `what is this?!' \citep[107]{hollow1973a}
    
\item\label{ExamplesOfWatewe4} \glll \textbf{watewé} ísekini éerehki, réesike ré ~ ~ ká'roomako'sh\\
    \textbf{wa-t-we} i-sek=rį ee-reh=ki reesik=E re ~ ~  ka'=oowąk=o'sh\\
    \textbf{unsp-wh-indf} pv.ins-\textnormal{do}=ss pv-\textnormal{want}=cond \textnormal{tongue}=sv dem.prox ~ ~ \textnormal{ask.for}=narr=ind.m\\
    \glt `when he wanted to do anything, he asked for this here tongue' \citep[187]{hollow1973a}

\newpage
    
\item\label{ExamplesOfWatewe5} \glll numá'k réehe, \textbf{watewé} éeniheni réeho'sh?\\
    ruwą'k rEEh=E \textbf{wa-t-we} ee-rį-he=rį rEEh=o'sh\\
    \textnormal{man} \textnormal{go.there}=sv \textbf{unsp-wh-indf} pv-2s-\textnormal{say}=ss \textnormal{go.there}=ind.m\\
    \glt `the man who went by, he went and said \textit{what} to you?' \citep[187]{hollow1973a}
    
\item\label{ExamplesOfWatewe6} \glll \textbf{matewé} túk, rútnuhere ó'xere'sha?\\
    \textbf{wa-t-we} tu=ak rut\#rų-hrE o'xrE=o'sha\\
    \textbf{unsp-wh-indf} \textnormal{be.some}=ds \textnormal{eat}\#1a.pl-caus dub=int.m\\
    \glt `I wonder if we can invite him to eat something we have here' \citep[127]{hollow1973a}

\item\label{ExamplesOfWatewe7} \glll \textbf{matewé} éeteki, rakína'ki, ó'ų'shka't\\
    \textbf{wa-t-we} ee-tE=ki ra-kirą'=ki o-ų'shka=ą't\\
    \textbf{unsp-wh-indf} pv-\textnormal{say}.sg=cond 2a-\textnormal{tell}=cond pv.irr-\textnormal{be.thus}=hyp\\
    \glt `if you were to say whatever [you want], if you were to tell it to me, then it would be so' \citep[241]{hollow1973b}

\end{xlist}

\end{exe}

Like other interrogative pronouns, \textit{watewé} is able to be used as an indefinite pronoun, as we see in (\ref{ExamplesOfWatewe2}) and (\ref{ExamplesOfWatewe4}) through (\ref{ExamplesOfWatewe7}). It is more common for indefinites to be marked via the unspecified argument marker \textit{wa}-, though these indefinite versions of the interrogative pronouns appear to be pragmatically used to place more emphasis on the indefinite nature of a particular argument.

Like \textit{tewé}, the pronoun \textit{watewé} is never used as a relativizer. Relativization is generally done using nominalizing preverbs, e.g., \textit{íraheke} `what you know' versus \textit{watewé íraheko'sha} `what do you know.'

\subsubsection{Interrogative `where': \textit{tewét} and \textit{tewétaa}}\label{SubSubSectionTewet}

Both \textit{tewét} and \textit{tewétaa} are composed of the interrogative pronoun \textit{tewé} plus a locative, i.e., /=t/ or /=taa/. Both pronouns are equivalent to `where', and like the other interrogatives, it cannot be used as a relativizer. There is functionally no difference between these two words. We can see examples of their use in the data in (\ref{ExamplesOfTewet}) below.

\begin{exe}

\item\label{ExamplesOfTewet} Examples of \textit{tewét} and \textit{tewétaa}

\begin{xlist}

\item\label{ExamplesOfTewet1} \glll \textbf{tewétaa} ó'raharani rahúuro'sha?\\
    \textbf{t-we=taa} o'\#ra-hrE=rį ra-huu=o'sha\\
    \textbf{wh-indf=loc} \textnormal{be}\#2a-caus=ss 2a-\textnormal{come.here}=int.m\\
    \glt `where do you come from?' \citep[299]{hollow1973b}

\item\label{ExamplesOfTewet2} \glll \textbf{tewéteroo} ptamáah tóops ó'ro'sha?\\
    \textbf{t-we=t=roo} p-ta-wąąh toop=s o'=o'sha\\
    \textbf{wh-indf=loc=dem.mid} 1sg.poss-al-\textnormal{arrow} \textnormal{four}=def \textnormal{be}=int.m\\
    \glt `where around here are my four arrows?' \citep[33]{hollow1973a}
    
\item\label{ExamplesOfTewet3} \glll máatahe, \textbf{tewét} ó'ro'sha?\\
    wąątah=\textbf{E t-we}=t o'=o'sha\\
    \textnormal{river}=sv \textbf{wh-indf=loc} \textnormal{be}=int.m\\
    \glt `the river, where is it?' \citep[37]{hollow1973a}
    
\item\label{ExamplesOfTewet4} \glll numá'koote're, \textbf{tewét} ó'harani húurak?\\
    ruwą'k=ootE=o're \textbf{t-we=t} o'\#hrE=rį huu=ak\\
    \textnormal{man}=evid=ind.f \textbf{wh-indf=loc} \textnormal{be}\#caus=ss \textnormal{come.here}=ds\\
    \glt `It is a boy, so where did he come from?' \citep[87]{hollow1973a}
    
\item\label{ExamplesOfTewet5} \glll náxi'he, \textbf{ítewetaa} ~ ~ ~ ~ ~ ~ ~ ~ ~ ~ raréeho'na?\\
    rą\#xi'h=E=$\varnothing$ \textbf{i-t-we=taa} ~ ~ ~ ~ ~ ~ ~ ~ ~ ~ ra-rEEh=o'rą\\
    \textnormal{mother}.voc\#\textnormal{be.old}=sv=voc \textbf{pv.dir-wh-indf=loc} ~ ~ ~ ~ ~ ~ ~ ~ ~ ~ 2a-\textnormal{go.there}=int.f\\
    \glt `grandmother, where are you going?' \citep[89]{hollow1973a}

\item\label{ExamplesOfTewet6} \glll súkxiknak \textbf{tewét} ó'harani húuni ~ ~ ~ ~ ~ wawáhere píiroomako'sh\\
    suk\#xik\#rąk \textbf{t-we=t} o'\#hrE=rį huu=rį ~ ~ ~ ~ ~  wa-wa-hrE pii=oowąk=o'sh\\
    \textnormal{child}\#\textnormal{be.bad}\#pos.sit \textbf{wh-indf=loc} \textnormal{be}\#caus=ss \textnormal{come.here}=ss ~ ~ ~ ~ ~ unsp-1a-caus \textnormal{devour}=narr=ind.m\\
    \glt `this bad kid came from somewhere and ate my food' \citep[120]{hollow1973b}
    
\item\label{ExamplesOfTewet7} \glll \textbf{tewétaa} rak\'{ı̨}haranitki, kóoxą'te óo ~ ~ ~ ~ ~ ~ ~ ~   iníiro'sh\\
    \textbf{t-we=taa} ra-kį\#hrE=rįt=ki kooxą'tE oo ~ ~ ~ ~ ~ ~ ~ ~  irįį=o'sh\\
    \textbf{wh-indf=loc} 2a-\textnormal{plant}\#caus=2pl=cond \textnormal{corn} dem.mid ~ ~ ~ ~ ~ ~ ~ ~  \textnormal{grow}=ind.m\\
    \glt `if you plant it wherever, corn will grow there' \citep[293]{hollow1973b}
    
\item\label{ExamplesOfTewet8} \glll waráahini wahekto'sh, \textbf{tewetaa}  ~ ~ ~ ~ ~ ~ ~ ~ ~ ~ ~ ~ ~ ~ ~ máakekereki\\
    wa-rEEh=rį wa-hE=kt=o'sh \textbf{t-we=taa} ~ ~ ~ ~ ~ ~ ~ ~ ~ ~ ~ ~ ~ ~ ~  wąąkE=krE=ki\\
    1a-\textnormal{go.there}=ss 1a-\textnormal{see}=pot=ind.m \textbf{wh-indf=loc} ~ ~ ~ ~ ~ ~ ~ ~ ~ ~ ~ ~ ~ ~ ~  \textnormal{lie}.pos.aux=3pl=cond\\
    \glt `I will go and see if they live somewhere' \citep[25]{trechter2012}
    
\end{xlist}

\end{exe}

The vast majority of instances of `where' pronouns in Mandan involve just \textit{tewét}, while \textit{tewétaa} has a tendency to be used when the following word in an intonational phrase begins with a consonant, especially a sonorant. These observations are merely tendencies, as there does not appear to be a viable rule for when one must be used versus the other.

These pronouns can take other nominal marking, often associated with directional or some other postpositional or deictic relationship. We can see in (\ref{ExamplesOfTewet5}) that \textit{tewét} takes the directional preverb \textit{i}- to become \textit{ítewet}, due to the fact that the clause features a verb expressing motion towards some place. Likewise, we see the postposition \textit{ó'harani} `from' used in (\ref{ExamplesOfTewet1}), (\ref{ExamplesOfTewet4}), and (\ref{ExamplesOfTewet6}) to indicate motion away from somewhere. 

\subsubsection{Interrogative `how': \textit{tashká}, \textit{tashká'eshka}, \textit{tashkáhąą}, \textit{táhąą}}\label{SubSubSecTashka}

Mandan employs a diverse set of interrogative pronouns to express how something occurs. The default word for `how' is \textit{tashká}, which is historically composed of the interrogative \textit{tá} plus the disjunctive marker =\textit{shka}. These two formatives do not appear to be decomposable in contemporary Mandan, however, as this word is analyzed as a single unit by speakers. 

While \textit{tashká} is the default interrogative pronoun for describing manner, there are other formatives that can occur with it to slightly modify its semantics. The similitive marker \textit{-eshka} often appears in conjunction with \textit{tashká} in the corpus. Instances of \textit{tashká'eshka} outnumber the default \textit{tashká} in the corpus. Another common item that accompanies \textit{tashká} is the instrumental postposition =\textit{hąą}. Examples of these three interrogative pronouns appear in (\ref{ExamplesOfTashka}) below.

% \newpage

\begin{exe}

\item\label{ExamplesOfTashka} Examples of \textit{tashká}, \textit{tashká'eshka}, and \textit{tashkáhąą}

\begin{xlist}

\item\label{ExamplesOfTashka1} \glll \textbf{tashká'sha}, waráhere?\\
    \textbf{tashka=o'sha} wa-ra-hrE\\
    \textbf{\textnormal{how}=int.m} unsp-2a-caus\\
    \glt `how it is, your food?' \citep[161]{hollow1973a}
    
\item\label{ExamplesOfTashka2} \glll \textbf{tashká} reheré'sh, wáa'ireseke?\\
    \textbf{tashka} re-hrE=o'sh waa-i-re-sek=E\\
    \textbf{\textnormal{how}} 2a-caus=ind.m nom-pv.ins-2a-\textnormal{make}=sv\\
    \glt `how did you do it, what you made?' \citep[3]{hollow1973a}
    
\item\label{ExamplesOfTashka3} \glll \textbf{tashká} ą́'t rushékere'sha?\\
    \textbf{tashka} ą't ru-shE=krE=o'sha\\
    \textbf{\textnormal{how}} dem.anap ins.hand-\textnormal{grasp}=3pl=int.m\\
    \glt `how can they take that?' \citep[267]{trechter2012}

\item\label{ExamplesOfTashka4} \glll \textbf{tashká'eshka}, ą́ąwe íseka, karáte ~ ~ ~ ~ ~ ~ ~ ~ ~ ~ ísekoote...\\
    \textbf{tashka-eshka} ąąwe i-sek=E=$\varnothing$ kratE ~ ~ ~ ~ ~ ~ ~ ~ ~ ~ i-sek=ootE\\
    \textbf{\textnormal{how}-smlt} \textnormal{all} pv.ins-\textnormal{make}=sv=cont \textnormal{right.there} ~ ~ ~ ~ ~ ~ ~ ~ ~ ~ pv.ins-\textnormal{make}=evid\\
    \glt `however it was, everything she made, she did it perfectly...' \citep[19]{trechter2012}

\item\label{ExamplesOfTashka5} \glll \textbf{tashká'eshka} íkahekoomako'sh\\
    \textbf{tashka-eshka} i-ka-hek=oowąk=o'sh\\
    \textbf{\textnormal{how}-smlt} pv.ins-incp-\textnormal{know}=narr=ind.m\\
    \glt `he somehow remembered' \citep[251]{trechter2012}
    
\item\label{ExamplesOfTashka6} \glll \textbf{tashká'eshkanash} ní'shak résh rahą́ąke'sha?\\
    \textbf{tashka-eshka=rąsh} r'\~~-ishak resh ra-hąąkE=o'sha\\
    \textbf{\textnormal{how}-smlt=att} 2poss-pro \textnormal{like.this} 2a-\textnormal{stand}.pos.aux=int.m\\
    \glt `how come you are like this?' \citep[6]{hollow1973a}
    
\item\label{ExamplesOfTashka7} \glll \textbf{tashká'eshka} rarátaxo'sha?\\
    \textbf{tashka-eshka} ra-ra-tax=o'sha\\
    \textbf{\textnormal{how}-smlt} 2a-ins.mth-\textnormal{make.loud.noise}=int.m\\
    \glt `how come you are crying?' \citep[42]{hollow1973a}
    
\item\label{ExamplesOfTashka8} \glll riráse, \textbf{tashkáhąą} éeheero'sha?\\
    ri-ras=E \textbf{tashka=hąą} ee-hee=o'sha\\
    2poss-\textnormal{name}=sv \textbf{\textnormal{how}=ins} pv-\textnormal{say}=int.m\\
    \glt `your name, how is it said?' \citep[20]{hollow1973a}

\end{xlist}

\end{exe}

These data show that \textit{tashká} gives the most generic reading of `how'. Questions involving this pronoun are very open-ended, and the speaker may not necessarily know what the potential answers to the question are. The addition of the similitive -\textit{eshka} tends to be used with questions dealing with motivation or reasoning. The range of uses for \textit{tashkáhąą} in the corpus are limited to asking how to say people's names. The use of the instrumental postposition on \textit{tashká} implies that the question is asking `using what' or `by what' is an action completed or a state accomplished. Personal fieldwork reveals that one consultant's father would always exclaim \textit{tashkáhąą} when he would see his relatives, expecting to hear how his family is doing.

\citet[483]{hollow1970} states that there is a fourth way to ask `how': \textit{táhąą}, which he translates as `what thing.' He gives no other examples of this pronoun in his dictionary, nor does it show up anywhere else in the corpus. It is unclear if this word is exclusively used for instruments, or if there are other situations in which a speaker might use this pronoun over others like \textit{tashkáhąą}.  %This word does, however, show us as part of a construction for the interrogative pronoun `when', which is explained below in \sectref{SubSubSecTahaaki}.

\subsubsection{Interrogative `why': \textit{tashkák}}\label{SubSubSecTashkak}

The interrogative `why' is clearly derived from \textit{tashká} `how' plus the different-subject switch-reference marker /=ak/. This word is very close semantically to `how' in Mandan, as there are instances of \textit{tashká} where the translation given is `how' in one sentence, and then repeated later on in a narrative and translated as `why'. Examples of this interrogative pronoun appear in (\ref{ExamplesOfTashkak}) below.

\begin{exe}

\item\label{ExamplesOfTashkak} Examples of \textit{tashkák}

    \begin{xlist}
    
    \item\label{ExamplesOfTashkak1} \glll numá'kaaki, \textbf{tashkák} wáa'ąąwe rakų́́'kiná'ka'na?\\
    ruwą'k-aaki \textbf{tashka=ak} waa-ąąwe ra-kų'\#kirą'=ka=o'rą\\
    \textnormal{person}-coll \textbf{\textnormal{\bfseries how}=ds} nom-\textnormal{all} 2a-\textnormal{give}\#\textnormal{tell}=hab=int.f\\
    \glt `to people, why do you always tell them everything?' \citep[213]{hollow1973a}
    
    \item\label{ExamplesOfTashkak2} \glll \textbf{tashkák} miníke, máamakaharaxi...\\
    \textbf{tashka=ak} wį-rįk=E waa-wąąkE\#hrE=xi\\
    \textbf{\textnormal{\bfseries how}=ds} 1poss-\textnormal{son}=sv neg-\textnormal{lie}.pos.aux\#caus=neg\\
    \glt `how come she did not leave my son...' \citep[82]{hollow1973a} 
    
    \item\label{ExamplesOfTashkak3} \glll \textbf{tashká'eshkak} ní'ąąwe raxirekanito'na?\\
    \textbf{tashka-eshka=ak} r'\~~-i-ąąwe ra-xireka=rįt=o'rą\\
    \textbf{\textnormal{\bfseries how}-smlt=ds} 2poss-pv.poss-\textnormal{all} 2a-\textnormal{be.skinny}=2pl=int.f\\
    \glt `why are you all so thin?' \citep[138]{hollow1973b}
    
    \item\label{ExamplesOfTashkak4} \glll téehą wáaraki'hinixak, \textbf{tashkák} ~ ~ ~ ~ ~ ~ ~ ~ ~ ~ éewereho'sh\\
    teehą waa-ra-ki'h=rįx=ak \textbf{tashka=}ak ~ ~ ~ ~ ~ ~ ~ ~ ~ ~ ee-we-reh=o'sh\\
    \textnormal{be.long.time} neg-2a-\textnormal{arrive.back.here}=neg=ds \textbf{\textnormal{\bfseries how}=ds} ~ ~ ~ ~ ~ ~ ~ ~ ~ ~ pv-1a-\textnormal{think}=ind.m\\
    \glt `With you not having returned after a long time, I wondered why' \citep[318]{hollow1973a}
    
    \item\label{ExamplesOfTashkak5} \glll \textbf{tashká'eshkanashki}, ishák í'eexinash ~ ~ ~ ~ ~ ~ ~ ~ ~ ~ ~ ~ ~  wóokųhkereroomako'sh\\
    \textbf{tashka-eshka=rąsh=ki} ishak i-eexi=rąsh ~ ~ ~ ~ ~ ~ ~ ~ ~ ~ ~ ~ ~  wV-o-kųh=krE=oowąk=o'sh\\
    \textbf{\textnormal{\bfseries how}-smlt=att=cond} pro pv.poss-\textnormal{tripe}=att ~ ~ ~ ~ ~ ~ ~ ~ ~ ~ ~ ~ ~  \textnormal{some}-pv.loc-\textnormal{want}=3pl=narr=ind.m\\
    \glt `for whatever reason, they wanted some of that tripe' \citep[267]{trechter2012}
    
    
    \end{xlist}

\end{exe}

Once again, this interrogative is never used as a relativizer in Mandan. We see a few unexpected form that are similar to the expected \textit{tashkák}, such as in (\ref{ExamplesOfTashkak3}), where we again see the similitive being used in a context where the speaker is assuming that there is some motivation behind an action taken by the listeners. The speaker is using \textit{tashká'eshkak} to ask why the listeners are skinny, because he is accusing them of doing something to cause themselves to become skinny. If the speaker had used \textit{tashkák}, the implication is that the speaker is asking what outside forces have caused the listeners to become so skinny.

Most instances of asking someone's motivation involves \textit{tashkák} in the corpus, and this interrogative uses the different-subject switch-reference marker =\textit{ak}. In \citet{trechter2012}, we see numerous occasions where Mr. Benson produces \textit{tashká'eshkanashki} as an indefinite pronoun, using the conditional =\textit{ki} instead of the expected =\textit{ak}. It is not clear if speakers have more flexibility to use other clause-final morphology instead of the different-subject switch-reference marker, or if constructions such as the one in (\ref{ExamplesOfTashkak5}) are a feature of Mr. Benson's idiolect, since no other consultant contributed similar constructions anywhere else in the corpus.

\subsubsection{Interrogative `how many': \textit{ta'ą́ą}}\label{SubSubSecTaAa}

Asking about quantities in Mandan involves the interrogative pronoun \textit{ta'ą́ą} `how many' or `how much.' The root of this pronoun is the interrogative \textit{tá}, plus the element /ąą/ that has some kind of quantitative reading.  There is no clear Proto-Siouan origin for the long nasal /ąą/ in this pronoun. However, it may be related to the /ąą/ in \textit{ą́ąwe}, especially if the \textit{-we} in \textit{ą́ąwe} is the same indefinite marker as the \textit{-we} in \textit{tewé}. Examples of this pronoun appear in (\ref{ExamplesOfTaAa}) below.

\largerpage

\begin{exe}

\item\label{ExamplesOfTaAa} Examples of \textit{ta'ą́ą}

\begin{xlist}

\item\label{ExamplesOfTaAa1} \glll \textbf{óta'ąa}̨s hi'sha?\\
    \textbf{o-ta'ąą}=s hi=o'sha\\
    \textbf{pv.loc-\textnormal{\bfseries how.many}=def} \textnormal{arrive.here}=int.m\\
    \glt `what time is it?' \citep[10]{kasak2014}
    
\item\label{ExamplesOfTaAa2} \glll míihe \textbf{ta'ą́ąroo} ká'kereroomako'sh\\
    wįįh=E \textbf{ta'ąą=roo} ka'=krE=oowąk=o'sh\\
    \textnormal{woman}=sv \textbf{\textnormal{\bfseries how.many}=dem.mid} \textnormal{possess}=3pl=narr=ind.m\\
    \glt `women, however many there, he had them' \citep[76]{hollow1973b}
    
\item\label{ExamplesOfTaAa3} \glll rapkakiriih \textbf{ta'ą́ąro'sha?}\\
    ra-ki-pa-kriih \textbf{ta'ąą=o'sha}\\
    2a-itr-ins.push-\textnormal{count} \textbf{\textnormal{\bfseries how.many}=int.m}\\
    \glt `how many did you count?' \citep[243]{trechter2012}
    
\item\label{ExamplesOfTaAa4} \glll wáahokshuke \textbf{ta'ą́ąki}, ą́ąwena ~ ~ ~ ~ ~ ~ ~ ~ ~ ~  rá'ke'ha kų́'karani...\\
    waa-ho-kshuk=E \textbf{ta'ąą=ki} ąąwe=rą ~ ~ ~ ~ ~ ~ ~ ~ ~ ~  ra'-ke'h=E=$\varnothing$ kų'=krE=rį\\
    nom-\textnormal{voice}\#\textnormal{be.narrow}=sv \textbf{\textnormal{\bfseries how.many}=cond} \textnormal{all}=top ~ ~ ~ ~ ~ ~ ~ ~ ~ ~  ins.heat-\textnormal{be.angry}=sv=cont \textnormal{give}=3pl=ss\\
    \glt `when there were however many animals, all of them got angry for him and...' \citep[43]{hollow1973a}


\end{xlist}

\end{exe}

This interrogative pronoun is exclusive to being used with quantification. We can see that in (\ref{ExamplesOfTaAa1}) that \textit{ta'ą́ą} is being used for asking what time it is. Literally, this question is asking `at how many has it arrived?' The arrival in question is that of the arms on a clock. This pronoun can also be used in an indefinite sense, as we see in (\ref{ExamplesOfTaAa2}) and (\ref{ExamplesOfTaAa4}), where the speakers are relating that they are unsure of a number, rendering the translation in English as `however many.'

\subsubsection{Interrogative `when': \textit{taxkó} and \textit{takáki}}\label{SubSubSecTahaaki}

The least common interrogative pronouns in the corpus are those that deal with time, i.e., `when'. \citet[483]{hollow1970} gives \textit{taxkó} as `when', but this word never appears anywhere else in the corpus. There are only two instances of interrogative pronouns for `when' present in the corpus, and both instances are the exact same sentence, repeated. The word used for `when' is \textit{takáki}. This word is composed of the interrogative root \textit{tá}, plus the habitual =\textit{ka} and the conditional =\textit{ki}. This example appears in (\ref{ExamplesOfTahaaki}) below.

\begin{exe}

\item\label{ExamplesOfTahaaki} Example of \textit{takáki}

% \begin{xlist}

% \item\label{ExamplesOfTahaaki1} 

    \glll \textbf{takáki} shą́te óraraahinito'sha?\\
    \textbf{ta=ka=ki} shąt=E o-ra-rEEh=rįt=o'sha\\
    \textbf{wh=hab=cond} \textnormal{hunt}=sv pv.irr-2a-\textnormal{go.there}=2pl=int.m\\
    \glt `when will you all go on a hunt?' \citep[140]{hollow1973b}

% \end{xlist}


\end{exe}

The conditional =\textit{ki} is a complementizer that is often translated as `when' or `if', so its semantics make sense with the interrogative pronoun for time-related questions. The paucity of occurrences in the data stems from the lack of every day conversation the corpus, since the small number of L1 speakers at the time of fieldwork was such that previous fieldworkers and consultants prioritized recording traditional narratives rather than minutiae of everyday life or casual conversations.

\section{Quantification}\label{SecQuantifiers}

Mandan has a small set of quantifiers. These quantifiers are all inflected as stative verbs. Cardinal numerals are also inflected as stative verbs. Ordinal numerals are derived from cardinal numerals. The behavior of these quantification-related lexical items is described below. I first describe numerals in \sectref{SubSecNumerals}, then discuss quantifiers in \sectref{SubSecQuantifiers}.

\subsection{Numerals}\label{SubSecNumerals}

Contemporary Mandan features a decimal (i.e., base-10) counting system. Proto-Siouan is posited to have had a quinary (i.e., base-5) counting system. Numbers above five in certain Siouan languages are compounds involving numbers that are below five. This older system of counting also has remnants in the Plains Sign Language that the Mandan traditionally use, where counting goes from the pinkie of the right hand to the thumb. To count beyond five, one extends the thumb of their left hand to touch the thumb of their right hand for six, and then count seven on their left index finger, eight on their middle finger, and so on until reaching ten at the left pinkie \citep[22]{tomkins1926}.

\subsubsection{Cardinal numerals}\label{SubSubSecCardinal}

The cardinal numerals for Mandan appear in \tabref{Tab1to10}.

\begin{table}
        \caption{Cardinal numerals from 1 to 10}\label{Tab1to10}
    \begin{tabular}{lllll}
\lsptoprule
    \textbf{Mandan}&\textbf{English}& ~&  \textbf{Mandan}&\textbf{English}\\
\midrule
    \textit{máxana}&`one'&~&  \textit{kíima}&`six'\\
    \textit{núp}&`two'&~&     \textit{kúupa}&`seven'\\
    \textit{náamini}&`three'&~&   \textit{tétoki}&`eight'\\
    \textit{tóop}&`four'&~&   \textit{máxpe}&`nine'\\
    \textit{kixų́ųh}&`five'&~&\textit{pirák}&`ten'\\\lspbottomrule
    \end{tabular}

\end{table}

The stems for the numerals that end in /p/ will often appear with a word-final /E/ to avoid ending a word in a bilabial stop.\footnote{See \sectref{geminatedissimilation} for the earlier discussion of this tendency against ending a word in /p/ in Mandan.} We can see examples of these cardinal numerals in the data in (\ref{ExamplesOfCardinalNumerals}).

\newpage
\begin{exe}

\item\label{ExamplesOfCardinalNumerals} Examples of cardinal numerals 1 through 10

\begin{xlist}
\item\label{ExamplesOfCardinalNumerals1} \glll koshų́ųka \textbf{\'{ı̨}'toopak} ínaamini ~ ~ ~ ~ ~ ~ ~   wáatashixikereroomako'sh\\
    ko-shųųka \textbf{į'-toop=ak} i-raawrį ~ ~ ~ ~ ~ ~ ~  waa-ta-shi=xi=krE=oowąk=o'sh\\
    3poss.pers-\textnormal{younger.brother} \textbf{pv.rflx-\textnormal{\bfseries four}=ds} pv.ord-\textnormal{three} ~ ~ ~ ~ ~ ~ ~  neg-al-\textnormal{be.good}=neg=3pl=narr=ind.m\\
    \glt `she had four younger brothers, three of whom did not like him' \citep[129]{hollow1973a}
    
\item\label{ExamplesOfCardinalNumerals2} \glll tamáana \textbf{kíimanashini} \textbf{kúupa} tewé ~ ~ ~ ~ ~ ~ ~ ~ ~ ~ ~ ~ ~ ~ ~ ~ ~ ~ ~ ~  hą́ąkerootekto'sh\\
    ta-waarą \textbf{kiiwą=rąsh=rį} \textbf{kuupa} t-we ~ ~ ~ ~ ~ ~ ~ ~ ~ ~ ~ ~ ~ ~ ~ ~ ~ ~ ~ ~   hąąkE=rootE=kt=o'sh\\
    al-\textnormal{winter} \textbf{\textnormal{\bfseries six}=att=ss} \textbf{\textnormal{\bfseries seven}} wh-indf ~ ~ ~ ~ ~ ~ ~ ~ ~ ~ ~ ~ ~ ~ ~ ~ ~ ~ ~ ~  \textnormal{stand}.pos.aux=evid=pot=ind.m\\
    \glt `[I do not know whether] he was six or seven years old' \citep[195]{hollow1973a}

\item\label{ExamplesOfCardinalNumerals3} \glll numá'kshiki, ráse \textbf{núpo'sh}\\
    ruwą'k\#shi=ki ras=E \textbf{rųp=o'sh}\\
    \textnormal{man}\#\textnormal{be.good}=cond \textnormal{name}=sv \textbf{\textnormal{\bfseries two}=ind.m}\\
    \glt `if he is a chief, then he has two names' \citep[14]{hollow1973a}
    
\end{xlist}

\end{exe}

Cardinal numerals above ten are all constructed by adding \textit{aak}-, a contrcted form of the postposition \textit{áaki} `above'. This prefix is then followed by a numeral one through nine. We can see these numerals in \tabref{Tab11to19}.

\begin{table}
        \caption{Cardinal numerals from 11 to 19}\label{Tab11to19}
    \begin{tabular}{lllll}
\lsptoprule
    \textbf{Mandan}&\textbf{English}& ~&  \textbf{Mandan}&\textbf{English}\\
\midrule
    \textit{aakmáxana}&`eleven'&~&  \textit{aahkíima}&`sixteen'\\
    \textit{aaknúp}&`twelve'&~&     \textit{aahkúupa}&`seventeen'\\
    \textit{aaknáamini}&`thirteen'&~&       
        \textit{aaktétoki}&`eighteen'\\
    \textit{aaktóop}&`fourteen'&~&   \textit{aakmáxpe}&`nineteen'\\
    \textit{aahkixų́ųh}&`fifteen'&~&~&\\\lspbottomrule
    \end{tabular}

\end{table}


Even though the prefix \textit{aak}- begins with a long vowel, speakers will typically leave this prefix unstressed when speaking, especially when recording vocabulary lists that are meant for pedagogical uses. However, speakers will vary in whether they stress the initial syllable or where primary stress typically falls on the cardinal numeral, e.g., \textit{aaknúp} and \textit{áaknup} both appear in the corpus for `twelve'. For some speakers, a Dorsey's Law vowel is inserted between the \textit{aak}- and the initial consonant of the numeral, i.e., \textit{aakmáxana} or \textit{aakamáxana} are both `eleven'. We can see examples of these numerals in (\ref{MoreExamplesOfCardinalNumerals}) below.

\begin{exe}

\item\label{MoreExamplesOfCardinalNumerals} Examples of cardinal numerals 11 through 19

\begin{xlist}

\item\label{MoreExamplesOfCardinalNumerals1} \glll \textbf{aahkíimanashini} \textbf{aahkixų́ųh} ų́'kerekto'sh, ~ ~ ~ ~ ~ ~ ~ ~ ~ tamáanakere\\
    \textbf{aak\#kiiwą=rąsh=rį} \textbf{aak\#kixųųh} ų'=krE=kt=o'sh ~ ~ ~ ~ ~ ~ ~ ~ ~ ta-waarą=krE\\
   \textbf{ \textnormal{\bfseries above}\#\textnormal{\bfseries five}=att=ss} \textbf{\textnormal{\bfseries above}\#\textnormal{\bfseries six}} \textnormal{be.near}=3pl=pot=ind.m ~ ~ ~ ~ ~ ~ ~ ~ ~ al-\textnormal{winter}=3pl\\
    \glt `they were maybe fifteen or sixteen, their age' \citep[41]{trechter2012}
    
\item\label{MoreExamplesOfCardinalNumerals2} \glll \textbf{ó'aakmaxanas} hí'sh\\
    \textbf{o-aak\#wąxrą=s} hi=o'sh\\
    \textbf{pv.loc-\textnormal{\bfseries above}\#\textnormal{\bfseries one}=def} \textnormal{arrive.here}=ind.m\\
    \glt `it is eleven o'clock' \citep[16]{kasak2014}

\end{xlist}

\end{exe}

Numerals above nineteen are formed by adding the base cardinal numeral to the `times' marker \textit{ha}, then adding the word \textit{pirák} `ten'. Thus, a numeral like `twenty' is literally `twice ten.' Multiplicative numerals in Mandan are described further in \sectref{SubSubSecOtherNums}. We can see these numerals in \tabref{Tab20to90} below.

\begin{table}
        \caption{Cardinal numerals from 20 to 90}\label{Tab20to90}
    \begin{tabular}{lllll}
\lsptoprule
    \textbf{Mandan}&\textbf{English}& ~&  \textbf{Mandan}&\textbf{English}\\
\midrule
    \textit{núphapirak}&`twenty'&~&                  
        \textit{kíimahapirak}&`sixty'\\
    \textit{náaminihapirak}&`thirty'&~&    
        \textit{kúupahapirak}&`seventy'\\
    \textit{tóophapirak}&`forty'&~&   
        \textit{tétokihapirak}&`eighty'\\
    \textit{kixų́ųhapirak}&`fifty'&~&
        \textit{máxpehapirak}&`ninety'\\\lspbottomrule
    \end{tabular}

\end{table}

Putting numerals together involves putting the highest number first, followed by the same-subject switch-reference marker and the next highest number. We can see this pattern in (\ref{NumbersAbove20}) below for numbers between twenty and thirty.

\newpage
\begin{exe}

\item\label{NumbersAbove20} Forming numbers above 20

    \begin{xlist}
    \item \textit{núphapirakini-máxana} `twenty-one'
    \item \textit{núphapirakini-núp} `twenty-two'
    \item \textit{núphapirakini-náamini} `twenty-three', etc.
    \end{xlist}


\end{exe}

Larger cardinal numerals in Mandan are \textit{h\'{ı̨}įsuk} `hundred' and \textit{h\'{ı̨}įsuk íkaakohi} `thousand'. The word for `thousand' translates as `hundred that is overly full.'

\subsubsection{Ordinal numerals}\label{SubSubSecOrdinal}

Ordinal numerals in Mandan are derived from their cardinal forms plus the ordinal preverb \textit{i}-. The one exception to this is the suppletive ordinal \textit{ų́ųte} `first'. This numeral will sometimes appear as \textit{í'ųųte} with the ordinal preverb. This pleonastic use of \textit{i}- is likely a regularization of the paradigm for forming ordinals, given that \textit{ų́ųte} is the only exception. \tabref{Tab1to10ordinal} below contains examples of the ordinal numerals in Mandan up to ten.

\begin{table}
        \caption{Ordinal numerals from 1 to 10}\label{Tab1to10ordinal}
    \begin{tabular}{lllll}
\lsptoprule
    \textbf{Mandan}&\textbf{English}& ~&  \textbf{Mandan}&\textbf{English}\\
\midrule
    \textit{ų́ųte}&`first'&~& 
        \textit{íkiima}&`sixth'\\
    \textit{ínup}&`second'&~&     
        \textit{íkuupa}&`seventh'\\
    \textit{ínaamini}&`third'&~&  
        \textit{ítetoki}&`eighth'\\
    \textit{ítoop}&`fourth'&~&   
        \textit{ímaxpe}&`nine'\\
    \textit{íkixųųh}&`fifth'&~&\textit{ípirak}&`tenth'\\\lspbottomrule
    \end{tabular}

\end{table}

In addition to these ordinal numerals, there is also the  quasi-ordinal \textit{híika} `last'. We can see examples of these numerals in the data in (\ref{ExamplesOfOrdinals}) below.

\begin{exe}

\item\label{ExamplesOfOrdinals} Examples of ordinal numerals

\begin{xlist}

\item\label{ExamplesOfOrdinals1} \glll \textbf{ų́ųte} kawáa'isehka ée ó'kere'sh\\
    \textbf{ųųt=E} ka-waa-i-sek=ka ee o'=krE=o'sh\\
    \textbf{\textnormal{\bfseries be.first}=sv} agt-nom-pv.ins-\textnormal{do}=hab dem.dist \textnormal{be}=3pl=ind.m\\
    \glt `at first, they were the workers' \citep[51]{hollow1973a}
    
\item\label{ExamplesOfOrdinals2} \glll ishtųhipak, \textbf{ítoophąk} ~ ~ ~ ~ ~ ~ ~ ~ ~ ~ ~ ~ ~ ~ ~ ~ ~ ~  kų́'kas manásh kih\'{ı̨}įra ~ ~ ~ ~ ~ ~ ~ ~ náakeroomako'sh\\
    ishtųh\#ip=ak \textbf{i-toop\#hąk} ~ ~ ~ ~ ~ ~ ~ ~ ~ ~ ~ ~ ~ ~ ~ ~ ~ ~ k'-ųųka=s wrąsh ki-hįį=E=$\varnothing$ ~ ~ ~ ~ ~ ~ ~ ~ rąąkE=oowąk=o'sh\\
    \textnormal{night}\#\textnormal{next.one}=ds \textbf{pv.ord-\textnormal{\bfseries four}\#stnd.pos} ~ ~ ~ ~ ~ ~ ~ ~ ~ ~ ~ ~ ~ ~ ~ ~ ~ ~ 3poss.pers-\textnormal{older.brother}=def \textnormal{tobacco} itr-\textnormal{drink}=sv=cont ~ ~ ~ ~ ~ ~ ~ ~ \textnormal{sit}.pos.aux=narr=ind.m\\
    \glt `the next night, the fourth one's older brother was sitting there smoking' \citep[245]{hollow1973b}

\item\label{ExamplesOfOrdinals3} \glll hiróo, \textbf{íkixųųhanaki}, ímaatiht ~ ~ ~ ~ ~ ~ ~ ~ ~ ~ ~ ~ ~ ~ ~  súhkereroomako'sh\\
    hiroo \textbf{i-kixųųh=ną=ki} i-wąątih=t ~ ~ ~ ~ ~ ~ ~ ~ ~ ~ ~ ~ ~ ~ ~  suk=krE=oowąk=o'sh\\
    \textnormal{now} pv.ord-\textnormal{\bfseries five}=top=cond pv.dir-\textnormal{outside}=loc ~ ~ ~ ~ ~ ~ ~ ~ ~ ~ ~ ~ ~ ~ ~ \textnormal{exit}=3pl=narr=ind.m\\
    \glt `now, when it was the fifth one, they went outside' \citep[201]{hollow1973a}

\end{xlist}

\end{exe}

\subsubsection{Multiplicative and other numerals}\label{SubSubSecOtherNums}

The most common kinds of numerals used in Mandan are the cardinal and ordinal numerals. However, there are other specific conditions where neither of these numerals is appropriate in Mandan. There are three additional kinds of numerals in Mandan described below. Multiplicative numerals, presented in \sectref{SubSubSubMultiplicative}, are used to explain how many times something has happened. In \sectref{SubSubSubDistributive}, the distributive numerals are used to describe the grouping in which an action has happened. Collective numerals are used to unite a grouping of people or things as a single unit, which are described in \sectref{SubSubSubCollective}.

\subsubsubsection{Multiplicative numerals}\label{SubSubSubMultiplicative}

When the occasion arises that one must explain the number of times something has happened, multiplicative numerals are used. These numerals are adverbial in nature, and are formed by adding the `times' marker \textit{ha} after a cardinal numeral. The one unexpected form is once again for the numeral `one', where instead of \textit{máxana}, a truncated form \textit{máx} take the `times' marker instead. We can see the multiplicative numerals from one to ten in \tabref{Tab1to10multiplicative} below.

\begin{table}
        \caption{Multiplicative numerals from 1 to 10}\label{Tab1to10multiplicative}
    \begin{tabular}{lllll}
\lsptoprule
    \textbf{Mandan}&\textbf{English}& ~&  \textbf{Mandan}&\textbf{English}\\
\midrule
    \textit{máxha}&`once, one time'&~&  \textit{kíimaha}&`six times'\\
    \textit{núpha}&`twice, two times'&~&     \textit{kúupaha}&`seven times'\\
    \textit{náaminiha}&`thrice, three times'&~&   \textit{tétokiha}&`eight times'\\
    \textit{tóopha}&`four times'&~&   \textit{máxpeha}&`nine times'\\
    \textit{kixų́ųha}&`five times'&~&\textit{pirákha}&`ten times'\\\lspbottomrule
    \end{tabular}

\end{table}

We can see examples of these multiplicative numerals in the data in (\ref{ExamplesOfMultiplicatives}) below.

\begin{exe}

\item\label{ExamplesOfMultiplicatives} Examples of multiplicative numerals

\begin{xlist}

\item\label{ExamplesOfMultiplicatives2} \glll káni máahe \textbf{núpha} ní'kereroomako'sh\\
    ka=rį wąąh=E \textbf{rųp\#ha} rį'=krE=oowąk=o'sh\\
    \textnormal{prov}=ss \textnormal{arrow}=sv \textbf{\textnormal{\bfseries two}\#\textnormal{\bfseries times}} \textnormal{shoot}=3pl=narr=ind.m\\
    \glt `and so they shot arrows at him twice' \citep[298]{hollow1973b} 

\item\label{ExamplesOfMultiplicatives1} \glll \textbf{tóopha} kiná'shka, wáapaksąhe míka ~ ~ ~ nákini...\\
    \textbf{toop\#ha} kirą'=shka waa-pa-ksąh=E wįk=E ~ ~ ~ rąk=rį\\
    \textbf{\textnormal{\bfseries four}\#\textnormal{\bfseries times}} \textnormal{tell}=dsj nom-ins.push-\textnormal{be.worried}=sv \textnormal{be.none}=sv ~ ~ ~  sit.pos=ss\\
    \glt `even though he told it to him four times, he sat there not paying any attention...' \citep[156]{hollow1973a}
    

\item\label{ExamplesOfMultiplicatives3} \glll \textbf{máxha} numá'keena ó'rak, Minítaari ~ ~ ~ ~ ~ ~ ~ ~ ~ ~ numá'koomako'sh\\
    \textbf{wąx\#ha} ruwą'k=ee=rą o'=ak wrį\#taari ~ ~ ~ ~ ~ ~ ~ ~ ~ ~  ruwą'k=oowąk=o'sh\\
    \textbf{\textnormal{\bfseries one}\#\textnormal{\bfseries times}} \textnormal{man}=dem.dist=top \textnormal{be}=ds \textnormal{water}\#\textnormal{cross} ~ ~ ~ ~ ~ ~ ~ ~ ~ ~  \textnormal{man}=narr=ind.m\\
    \glt `one time, there was a man, and he was a Hidatsa man' \citep[11]{trechter2012}

\end{xlist}

\end{exe}

As previously seen in \sectref{SubSubSecCardinal}, multiplicative numerals are used alongside cardinal numerals to form numbers above nineteen, e.g., \textit{núphapirak} `twenty', literally `twice ten.'

\subsubsubsection{Distributive numerals}\label{SubSubSubDistributive}

Distributive numerals serve as adverbials, describing how people or other nouns are arranged within a proposition. It is often translated as `X-by-X', `X of each', or `in groups of X.' Rather than having dedicated suffix or prefix for these numerals, Mandan has two different reduplicative processes to indicative distributivity. For monosyllabic numerals, its onset and a single mora of the syllable are prefixed onto the numeral. Polysyllabic numerals involve the onset and full moraic weight of the final syllable being suffixed onto the end of the word. If a numeral ends in a consonant, then the reduplication happens before that final vowel. We can see this process at work in \tabref{Tab1to10distributive} below.

\begin{table}
        \caption{Distributive numerals from 1 to 10}\label{Tab1to10distributive}
    \begin{tabular}{lllll}
\lsptoprule
    \textbf{Mandan}&\textbf{English}& ~&  \textbf{Mandan}&\textbf{English}\\
\midrule
    \textit{máxanana}&`one by one'&~&  \textit{kíimama}&`six by six'\\
    \textit{nunúp}&`two by two'&~&     \textit{kúupapa}&`seven by seven'\\
    \textit{náaminimini}&`three by three'&~&   \textit{tétokiki}&`eight by eight'\\
    \textit{totóop}&`four by four'&~&   \textit{máxpexpe}&`nine by nine'\\
    \textit{kixų́ųxųųh}&`five by five'&~&\textit{pirárak}&`ten by ten'\\\lspbottomrule
    \end{tabular}

\end{table}

In addition to these distributive numerals, there are other quasi numerals like \textit{xamámah} `little by little' that follow the same pattern of expressing a distributed amount. We can see examples of distributive numerals in (\ref{ExamplesOfDistributives}) below.

\begin{exe}

\item\label{ExamplesOfDistributives} Examples of distributive numerals

\begin{xlist}

\item\label{ExamplesOfDistributives1} \glll psh\'{ı̨}įxaare tóop kaskékerek, \textbf{nunúp} ká'ni...\\
    pshįįxaa=E toop ka-skE=krE=ak \textbf{rų$\sim$rųp} ka'=rį\\
    \textnormal{sage}=sv \textnormal{four} ins.frce-\textnormal{tie}=3pl=ds \textbf{dist$\varnothing$\textnormal{\bfseries two}} \textnormal{possess}=ss\\
    \glt `they tied up four sage plants, and they had two of each kind...' \citep[162]{hollow1973a}
    
\item\label{ExamplesOfDistributives2} \glll tkíni réehoomako'sh, \textbf{máxanana}\\
    tki=rį rEEh=oowąk=o'sh \textbf{wąxrą$\sim$}rą\\
    \textnormal{touch}=ss \textnormal{go.there}=narr=ind.m \textnormal{\bfseries one}$\sim$dist\\
    \glt `they touched him and they left, one by one' \citep[261]{hollow1973b}


\end{xlist}

\end{exe}

Distributive numerals are rare in the corpus, but they are present in list form in all previous descriptions of Mandan grammar \citep{kennard1936,hollow1970, mixco1997a}.


\subsubsubsection{Collective numerals}\label{SubSubSubCollective}

Collective numerals are used to indicate that a group is treated as a single unit that consists of a specific quantity. The translations for these numerals are often `X of them' or `all X of them', where X is the number of entities in question. These numerals are generally restricted to describing humans or anthropomorphicized animals in traditional narratives. The morphology of collective numerals has already been discussed previously in \sectref{suffixcollective}. The one exception to the typical -\textit{sha} and -\textit{shashka} marking is the suppletive \textit{máxanana} `one of them.' This suppletive form is homophonous with the distributive numeral, as we see in \tabref{Tab1to10collective} below.

\begin{table}
        \caption{Collective numerals from 1 to 10}\label{Tab1to10collective}
    \begin{tabular}{lllll}
\lsptoprule
    \textbf{Mandan}&\textbf{English}& ~&  \textbf{Mandan}&\textbf{English}\\
\midrule
    \textit{máxanana}&`one of them'&~&  \textit{kíimasha}&`six of them'\\
    \textit{núpsha}&`two of them, both of them'&~&     \textit{kúupasha}&`seven of them'\\
    \textit{náaminisha}&`three of them'&~&   \textit{tétokisha}&`eight of them'\\
    \textit{tóopsha}&`four of them'&~&   \textit{máxpesha}&`nine of them'\\
    \textit{kixų́ųhsha}&`five of them'&~&\textit{piráksha}&`ten of them'\\\lspbottomrule
    \end{tabular}

\end{table}

Collective numerals are rare in the corpus. We can see examples of them in (\ref{ExamplesOfCollectiveNums}) below.

\begin{exe}

\item\label{ExamplesOfCollectiveNums} Examples of collective numerals

\begin{xlist}

\item\label{ExamplesOfCollectiveNums1} \glll \textbf{máxanana} rusháni warátaa réeharani...\\
    \textbf{wąxrą$\sim$rą} ru-shE=rį wra'=taa rEEh\#hrE=rį\\
    \textbf{\textnormal{\bfseries one}$\sim$coll} ins.hand-\textnormal{grasp}=ss \textnormal{fire}=loc \textnormal{go.there}\#caus=ss\\
    \glt `one of them took it and put it into the fire...' \citep[175]{hollow1973b}

\item\label{ExamplesOfCollectiveNums2} \glll óo ó'harani numá'kaaki hų́keres sheréekini \textbf{tóopshashka} kahashkereroomako'sh\\
    oo o'\#hrE=rį ruwą'k-aaki hų=krE=s shreek=rį \textbf{toop-sha-shka} ka-hash=krE=oowąk=o'sh\\
    dem.mid \textnormal{be}\#caus=ss \textnormal{person}-coll \textnormal{be.many}=3pl=def \textnormal{war.whoop}=ss \textbf{\textnormal{\bfseries four}-coll-ints.coll} ins.frce-\textnormal{be.taken.apart}=3pl=narr=ind.m\\
    \glt `from there, the many people war whooped and slaughtered all four of them' \citep[255]{hollow1973b}
    
\item\label{ExamplesOfCollectiveNums3} \glll \textbf{ninúpshashka} raráahini éerehinitki, ~ ~ ~ ~ ~ ~ ~ ptúuhąhkere wawíiwak hékto'sh\\
    \textbf{rį-rųp-sha-shka} ra-rEEh=rį ee-reh=rįt=ki ~ ~ ~ ~ ~ ~ ~ p-tuuhąk=krE wa-wiiwa=ak hE=kt=o'sh\\
    \textbf{2s-\textnormal{\bfseries two}-coll-ints.coll} 2a-\textnormal{go.there}=ss pv-\textnormal{want}=2pl=cond ~ ~ ~ ~ ~ ~ ~ 1sg.poss-\textnormal{sister's.child}=3pl 1a-\textnormal{look.after}=ds \textnormal{see}=pot=ind.m\\
    \glt `if both of you want to go, then I will look after my nephews' \citep[64]{hollow1973b}
    
    
\item\label{ExamplesOfCollectiveNums4} \glll \textbf{ínupshashkana} hų́pe ké'ka'rak ~ ~ kų́'kerek...\\
    \textbf{i-rųp-sha-shka=rą} hųp=E ke'\#ka'=ak ~ ~   kų'=krE=ak\\
    \textbf{pv.ord-\textnormal{\bfseries two}-coll-ints.coll=top} \textnormal{moccasin}=sv \textnormal{keep}\#\textnormal{possess}=ds ~ ~  \textnormal{give}=3pl=ds\\
    \glt `both of them kept his shoes for him...' \citep[109]{hollow1973a}
    
\item\label{ExamplesOfCollectiveNums5} \glll Rápuseena, éeheni ~ ~ ~ ~ ~ ~ ~ ~ ~ ~ ~ ~ ~ ~ ~ kohų́ųkereseena, \textbf{ítoopsha} ~ ~ ~ ~ íhaa'aakit keréehkereroomako'sh\\
    ra'-pus=ee=rą eeherį ~ ~ ~ ~ ~ ~ ~ ~ ~ ~ ~ ~ ~ ~ ~  ko-hųų=krE=s=ee=rą \textbf{i-toop-sha} ~ ~ ~ ~ i-haa\#aaki=t k-rEEh=krE=oowąk=o'sh\\
    ins.heat-\textnormal{make.marks}=dem.dist=top \textnormal{and} ~ ~ ~ ~ ~ ~ ~ ~ ~ ~ ~ ~ ~ ~ ~  3poss.pers-\textnormal{mother}=3pl=def=dem.dist=top \textbf{pv.ord-\textnormal{\bfseries four}-coll} ~ ~ ~ ~ pv.dir-\textnormal{cloud}\#\textnormal{above}=loc vert-\textnormal{go.there}=3pl=narr=ind.m\\
    \glt `Charred in Streaks, along with their mother, all four of them returned to heaven' \citep[175]{hollow1973a}

\end{xlist}

\end{exe}

One peculiarity is that ordinal preverb \textit{i}- often accompanies the collective suffixes, as we can see in (\ref{ExamplesOfCollectiveNums3}) through (\ref{ExamplesOfCollectiveNums5}). It is not clear why the ordinal preverb is present on these numerals. \citet[27]{hollow1970}
even reports that this preverb can occur with distributive numerals, e.g., \textit{ínaaminimini} `three by three.'  One possibility is that speakers are generalizing \textit{i}- as an indicator that the numeral is simply not a cardinal numeral. Cardinal numerals never bear \textit{i}-, so the presence of this preverb on non-cardinal numerals could be symptomatic of a paradigmatic change in the treatment of numerals in Mandan. \citet{park2021} reports a similar phenomenon in Hidatsa, where \textit{ii}- is prefixed onto numerals referencing people, even if the numeral is not collective. 

It is unclear whether this variability between the presence of this preverbal element in Hidatsa is related, given the fact that virtually all L1 speakers of Mandan since the beginning of the twentieth century have also been fluent Hidatsa speakers. This mass bilingualism could have caused aspects of the Hidatsa number system to be mapped onto Mandan ones, but it is also possible that this system was in flux prior to the collapse of the tendency for children to learn both their parents' languages when individuals from different groups married. Unless other materials are discovered that might shed light on this situation, these hypotheses will have to remain in the realm of speculation.


\subsection{Quantifiers}\label{SubSecQuantifiers}

While Mandan does have several classes of numerals, it has relatively few quantifiers. Lie numerals, quantifiers are inflected like stative verbs whenever person features are needed. Most quantifiers are unbound words that bear their own primary stress, though the `some' allomorph of the unspecifed argument marker \textit{waa}- appears prefixed onto the verb. An unbound version, \textit{wáa}, does exist, but it is more rare in the corpus. The quantifier \textit{tú} `some' is often used in existential expressions, including in constructions of possession, e.g., \textit{tasúke túkereroomako'sh} `they had a child' is literally `some child was theirs.'

A table with the quantifiers in Mandan appears below in \tabref{TableQuantifiers}.

\begin{table}
        \caption{Quantifiers in Mandan}\label{TableQuantifiers}
    \begin{tabular}{ll}
\lsptoprule
    \textbf{Mandan}&\textbf{English}\\
\midrule
    \textit{ą́ąwe}&`all, every'\\
    \textit{hų́}&`many'\\
    \textit{íika}&`every'\\
    \textit{mík}&`none'\\
    \textit{są́ąka}&`few'\\
    \textit{tú}&`some'\\
    \textit{wáa$\sim$waa-}&`some'\\
    %\textit{wáaxte}&`a lot'\\
    \lspbottomrule
    \end{tabular}

\end{table}

Quantifiers appear postposed after the noun phrases they modify. We can see examples of quantifiers in the data in (\ref{ExamplesOfQuantifiers}) below.

\newpage

\begin{exe}

\item\label{ExamplesOfQuantifiers} Examples of quantifiers

\begin{xlist}

\item\label{ExamplesOfQuantifiers1} \glll máa'ų'st máana \textbf{hų́} núunihkereroomako'sh\\
    waa-ų't=t waarą \textbf{hų} ruurįh=krE=oowąk=o'sh\\
    nom-\textnormal{be.in.past}=loc \textnormal{winter} \textbf{\textnormal{\bfseries many}} \textnormal{be}.aux.pl=3pl=narr=ind.m\\
    \glt `they were already there for many years' \citep[220]{hollow1973a}
    
\item\label{ExamplesOfQuantifiers2} \glll  súknuma'hkeres pt\'{ı̨}į \textbf{są́ąkana} hékereroomako'sh\\
    suk\#rųwa'k=krE=s ptįį \textbf{sąąka=rą} hE=krE=oowąk=o'sh\\
    \textnormal{child}\#\textnormal{man}=3pl=def \textnormal{buffalo} \textbf{\textit{\bfseries few}=top} \textnormal{see}=3pl=narr=ind.m\\
    \glt `the young men saw a few buffalo' \citep[79]{hollow1973b}
    
\item\label{ExamplesOfQuantifiers3} \glll súhkeres \textbf{ą́ąwe} péeshi sanáake'sh\\
    suk=krE=s \textbf{ąąwe} peeshi srąąkE=o'sh\\
    \textnormal{child}=3pl=def \textbf{\textnormal{\bfseries all}} \textnormal{stomach} \textnormal{be.round}=ind.m\\
    \glt `all the children's stomachs were round' \citep[81]{hollow1973b}
    
\item\label{ExamplesOfQuantifiers4} \glll máaxtikokshuk \textbf{wáa}hektiki, ~ ~ ~ ~ ~ ~ ~ ~ ~ ~ ~ ~ ~ ~ ~ téeherekaroomako'sh\\
    wąąxtik\#o-kshuk \textbf{waa}-hE=kti=ki ~ ~ ~ ~ ~ ~ ~ ~ ~ ~ ~ ~ ~ ~ ~  tee\#hrE=ka=oowąk=o'sh\\
    \textnormal{rabbit}\#pv.irr\textnormal{be.narrow} \textbf{\textnormal{\bfseries some}}-\textnormal{see}=pot=cond ~ ~ ~ ~ ~ ~ ~ ~ ~ ~ ~ ~ ~ ~ ~ \textnormal{be.dead}\#caus=hab=narr=ind.m\\
    \glt `whenever he saw some cottontails, he would always kill them' \citep[196]{hollow1973a}

\end{xlist}

\end{exe}

The definite article and other demonstratives typically appear after the noun phrase and before the quantifier, but quantifiers are able to take other nominal morphology, like determiners, augmentatives, or topic markers. The placement of the determiners affects the readings of the phrase somewhat, e.g., \textit{súk xamáh hų́s} `the many small children' vers \textit{súk xamáhs hų́} `many [of] the small children.'

\subsection{Summary of quantification}\label{SubSubSubSecQuantificationSummary}

All words relating to quantification in Mandan take stative morphology when dealing with first or second person entities. Cardinal numerals are the most commonly encountered form of quantification, followed by quantifiers. Non-cardinal numerals can take the ordinal preverb \textit{i}-, given that there seems to be a leveling process where this \textit{i}- is no longer a marker or ordinality, but a marker of non-cardinality. This seemingly superfluous \textit{i}- may be due to language contact with Hidatsa, which will likewise add \textit{ii}- before certain numeral. Across all the kinds of numerals, the one numeral that has a suppletive form is `one', while all other numbers follow the expected pattern for any given numeral class.

\section{Postpositions}\label{SecPostpositions}

Typologically, languages that feature a default SOV word order tend to have postpositions instead of prepositions. Mandan conforms to the typology by placing adpositions after the noun phrase over which they have scope. Many concepts in Mandan are reducible to lower valency verbs that take only one or two arguments. This tendency to have oblique semantics encoded into a verb stems from the historical development of preverbs, in which Proto-Siouan or Pre-Proto-Siouan postpositions became reanalyzed as being integral parts of the following verb, rather than independent words or enclitics that pertained to the preceding noun phrase \citep{helmbrecht2006,helmbrecht2008,helmbrechtlehmann2008,kasak2019}.

Ignoring these preverbs, true postpositions in Mandan form two classes: postpositional enclitics and free postpositions. Postpositional enclitics are monosyllabic postpositions that do not have a primary stress of their own, which are addressed in \sectref{SubSecPostpositionalEnclitics}. Free postpositions are either polysyllabic or are compound postpositions, which are addressed in \sectref{SubSecFreePostpositions}.

\subsection{Postpositional enclitics}\label{SubSecPostpositionalEnclitics}

Postpositional enclitics adjoin to the rightmost edge of the noun phrase over which they scope. As enclitics, they can never exist as words unto themselves and always rely on another prosodic word to be pronounced. The these postpositional enclitics are shown on \tabref{TablePostpositionalEnclitics} below.

\begin{table}
        \caption{Postpositional enclitics in Mandan}\label{TablePostpositionalEnclitics}
    \begin{tabular}{ll}
\lsptoprule
    \textbf{Mandan}&\textbf{English}\\
\midrule
    /=t/$\sim$/=taa/& locative `in, at, to, toward'\\
    /=hąą/& instrumental `with, along, around'\\
    /=ku'sh/&  inessive `inside, within' (enclosed space)\\
    /=rok/&   inessive `inside, within' (area)  \\
    \lspbottomrule
    \end{tabular}
\end{table}

Some of these postpositional enclitics are able to be compounded. We can see examples of these enclitics in the following subsections.

\subsubsection{Locative: =\textit{t} and =\textit{taa}}

The most common postposition in the corpus is the locative =\textit{t} or =\textit{taa}. \citet[25]{kennard1936} explains that =\textit{taa} is the true locative, while =\textit{t} is ``essentially directional in function.'' He does, however, concede that these items generally overlap in usage. Throughout the corpus, both these locative markers are treated as equivalent by speakers, with one speaker producing an utterance that features =\textit{t}, only to repeat that same utterance later on in the narrative with =\textit{taa} instead. \citeauthor{kennard1936}'s observation about =\textit{t} tending to be used in directional expressions has some weight to it, but in the corpus, this difference is not systematic. In \citeapos{hollow1970} dictionary, locative =\textit{t} occurs frequently, but \citeauthor{hollow1970} consistently glosses each instance of =\textit{t} as having the underlying form of /=taa/. He makes no comment on when one should appear versus the other. \citet[39]{mixco1997a} makes no discussion of the difference between =\textit{t} and =\textit{taa} in his grammar, as only =\textit{taa} appears in his list of postpositions.

Given the propensity for Mandan speakers to decrease the overall intensity of phonation at the end of words, it is possible that =\textit{t} represents a truncated version of =\textit{taa}. In Proto-Siouan, *ta(a) is reconstructed as a general locative, with reflexives being found in all branches of the Siouan language family \citep{rankin2015}. As such, it is likely the case that this split between =\textit{t} and =\textit{taa} is an innovation in Mandan. In recordings conducted by Mr. Corey Spotted Bear with Mr. Edwin Benson in the mid 2000s and early 2010s, Mr. Benson produced only =\textit{taa} locatives in list elicitation. The apocape of /aa/ does not have any phonological conditioning, so this manifestation as =\textit{t} may be, in part, an artifact of rapid speech. There may be other factors involved in choosing =\textit{t} over =\textit{taa}, but my own fieldwork with Mr. Benson and with other Mandan speakers do not provide a conclusive motivation for choosing one over the other.

We can see examples of the locative, both as =\textit{t} and as =\textit{taa}, in the data in (\ref{ExamplesOfTaa}) below.

\begin{exe}

\item\label{ExamplesOfTaa} Examples of locative =\textit{t} and =\textit{taa}

\begin{xlist}

\item\label{ExamplesOfTaa1} \glll súk óniireena mákoomaksįh, \textbf{mústaa}\\
    suk o-rįį=ee=rą wąk=oowąk=sįh \textbf{wųt=taa}\\
    \textnormal{child} pv.irr-\textnormal{walk}=dem.dist=top sit.pos=narr=ints \textbf{\textnormal{\bfseries garden}=loc}\\
    \glt `a child's tracks were there, in the garden' \citep[84]{hollow1973a}
    
\item\label{ExamplesOfTaa2} \glll minísweeruts híroomako'sh, ~ ~ ~ ~ ~ ~ ~ ~ ~ ~ ~ ~ ~ ~ ~ ~ ~ ~ ~ \textbf{súhkereseetaa}\\
    wrįs\#wee\#rut=s hi=oowąk=o'sh ~ ~ ~ ~ ~ ~ ~ ~ ~ ~ ~ ~ ~ ~ ~ ~ ~ ~ ~  \textbf{suk=krE=s=ee=taa}\\
    \textnormal{horse}\#\textnormal{feces}\#\textnormal{eat}=def \textnormal{arrive.here}=narr=ind.m ~ ~ ~ ~ ~ ~ ~ ~ ~ ~ ~ ~ ~ ~ ~ ~ ~ ~ ~ \textbf{\textnormal{\bfseries child}=3pl=def=dem.dist=loc}\\
    \glt `the dog got there, to where the kids were' \citep[180]{hollow1973a}
    
\item\label{ExamplesOfTaa3} \glll máxanas rusháni \textbf{tíhų'sht} ~ ~ ~ ~ ~ ~ ~ ~ ~ ~ \'{ı̨}'tkika'sh\\ 
    wąxrą=s ru=shE=rį \textbf{ti\#hų'sh=t} ~ ~ ~ ~ ~ ~ ~ ~ ~ ~  į'-tkika=o'sh\\
    \textnormal{one}=def ins.hand-\textnormal{grasp}=ss \textbf{\textnormal{\bfseries house}\#\textnormal{\bfseries interior.edge}=loc} ~ ~ ~ ~ ~ ~ ~ ~ ~ ~ pv.rflx-\textnormal{hurl}=ind.m\\
    \glt `he took one of them and threw him at the wall' \citep[158]{hollow1973a}

\item\label{ExamplesOfTaa4} \glll \textbf{ókaraxt} kxų́hkereroomako'sh\\
    \textbf{o-krax=t} kxųh=krE=oowąk=o'sh\\
    \textbf{pv.loc-\textnormal{\bfseries be.low}=loc} \textnormal{lie.down}=3pl=narr=ind.m\\
    \glt `they went to bed in a coulee' \citep[171]{hollow1973a}
    
\item\label{ExamplesOfTaa5} \glll páaxu shishíhka \textbf{wará'nast}  ~ ~ ~ ~ ~ ~ ~ ~ ~ ~ ~ ~ ~ ~ ~ ~ íkų'tekereroomako'sh\\
    paaxu shi$\sim$shih=ka \textbf{wra'\#rąt=t}  ~ ~ ~ ~ ~ ~ ~ ~ ~ ~ ~ ~ ~ ~ ~ ~ i-kų'tE=krE=oowąk=o'sh\\
    \textnormal{nose} aug$\sim$\textnormal{be.sharp}=hab \textbf{\textnormal{\bfseries fire}\#\textnormal{\bfseries be.in.middle}=loc} ~ ~ ~ ~ ~ ~ ~ ~ ~ ~ ~ ~ ~ ~ ~ ~ pv.ins-\textnormal{throw}=3pl=narr=ind.m\\
    \glt `they threw the mosquito right into the middle of the fire' \citep[153]{hollow1973b}

\item\label{ExamplesOfTaa6} \glll \textbf{máapsitaarak}, miní ropxé nuréehto're\\
    \textbf{wąąpsi=taa=ak} wrį ropxE rų-rEEh=t=o're\\
    \textbf{\textnormal{\bfseries morning}=loc=ds} \textnormal{water} \textnormal{enter} 1a.pl-\textnormal{go.there}=pot=ind.f\\
    \glt `in the morning, we will go swimming' \citep[105]{hollow1973b}
    
\item\label{ExamplesOfTaa7} \glll roką́ąkaxi'heena tí \textbf{óhąkt} ~ ~ ~ ~ nákak...\\
    rokąąka\#xi'h=ee=rą ti \textbf{o-hąk=t} ~ ~ ~ ~  rąk=ak\\
    \textnormal{old.woman}\#\textnormal{be.old}=dem.dist=top \textnormal{house} \textbf{pv.loc-stnd.pos=loc} ~ ~ ~ ~  sit.pos=ds\\
    \glt `an old lady was sitting in the corner of the house and...' \citep[151]{hollow1973b}
    
\item\label{ExamplesOfTaa8} \glll \textbf{máataht}, máana kú' áahka wáa'ohe ~ ~ ~ ~ ~ ~ ~ ~ ~ ~ ómikak...\\
    \textbf{wąątah=t} waa=rą ku' aahka waa-o-hE ~ ~ ~ ~ ~ ~ ~ ~ ~ ~   o-wįk=ak\\
    \textbf{\textnormal{\bfseries river}=loc} \textnormal{some}=top \textnormal{be.beyond} rtro nom-pv.irr-\textnormal{see} ~ ~ ~ ~ ~ ~ ~ ~ ~ ~ pv.irr-\textnormal{be.none}=ds\\
    \glt `in the river, no one could see anything just a little ways away...' \citep[214]{hollow1973a}
    
\end{xlist}

\end{exe}

As the data above demonstrate, there can be no conclusive factor for a directional versus stationary use for =\textit{t} and =\textit{taa}. We see a stationary locative reading for =\textit{t} in (\ref{ExamplesOfTaa4}), (\ref{ExamplesOfTaa7}) and (\ref{ExamplesOfTaa8}). A directional locative reading for =\textit{taa} can be seen in (\ref{ExamplesOfTaa2}), as well as a temporal locative reading for `in the morning' in (\ref{ExamplesOfTaa6}).

Both =\textit{t} and =\textit{taa} are able to indicate a stationary location or express motion towards a location, but directional uses of these locative are often accompanied by the directional preverb \textit{i}-, as we can see in the data in (\ref{ExamplesOfTaaWithI}) below.

\begin{exe}

\item\label{ExamplesOfTaaWithI} Examples of locatives plus the directional preverb \textit{i}-

\begin{xlist}

\item\label{ExamplesOfTaaWithI1} \glll \textbf{ímaataht} waréeh íwateero'sh\\
    \textbf{i-wąątah=t} wa-rEEh i-wa-tee=o'sh\\
    \textbf{pv.dir-\textnormal{\bfseries river}=loc} 1a-\textnormal{go.there} pv.ins-1a-\textnormal{like}=ind.m\\
    \glt `I would like to go to the river' \citep[35]{hollow1973a}
    
\item\label{ExamplesOfTaaWithI2} \glll numá'ks \textbf{ími'tit} keréehoomako'sh\\
    ruwą'k=s \textbf{i-wį'\#ti=t} k-rEEh=oowąk=o'sh\\
    \textnormal{man}=def \textbf{pv.dir-\textnormal{\bfseries stone}\#\textnormal{house}=loc} vert-\textnormal{go.there}=narr=ind.m\\
    \glt `the man went back to the village' \citep[177]{hollow1973a}
    
\item\label{ExamplesOfTaaWithI3} \glll \textbf{ími'titaa} keréehį'herekereki, ~ ~   numá'kaaki máamikoomako'sh\\
    \textbf{i-wį'\#ti=taa} k-rEEh\#į'-hrE=krE=ki ~ ~   ruwą'k-aaki waa-wįk=oowąk=o'sh\\
    \textbf{pv.dir-\textnormal{\bfseries stone}\#\textnormal{\bfseries house}=loc} vert-\textnormal{go.there}\#pv.rflx-caus=3pl=cond ~ ~  \textnormal{person}-coll \textnormal{some}-\textnormal{be.none}=narr=ind.m\\
    \glt `when they got themselves back to the village, there were not any people there' \citep[182]{hollow1973a}

\item\label{ExamplesOfTaaWithI4} \glll \textbf{ípashahąkt} náaketaa máa'ąk ~ ~ ~ ~ ~ ~ ~ ~ ~ ~ ~ ~ ~ ~ ~  íwasekto'sh\\
    \textbf{i-pashahąk=t} rąąkE=taa waa'ąk ~ ~ ~ ~ ~ ~ ~ ~ ~ ~ ~ ~ ~ ~ ~  i-wa-sek=t=o'sh\\
    \textbf{pv.dir-\textnormal{\bfseries north}=loc} \textnormal{sit}.pos.aux=loc \textnormal{land} ~ ~ ~ ~ ~ ~ ~ ~ ~ ~ ~ ~ ~ ~ ~  pv.ins-1a-\textnormal{make}=pot=ind.m\\
    \glt `to the north, up that way, I will make land' \citep[9]{hollow1973a}
    
\item\label{ExamplesOfTaaWithI5} \glll kų́'rak keréeho'sh, \textbf{íXoshkataa}\\
    kų'=ak k-rEEh=o'sh \textbf{i-Xoshka=taa}\\
    \textnormal{give}=ds vert-\textnormal{go.there}=ind.m \textbf{pv.dir-\textnormal{\bfseries Xoshga}=loc}\\
    \glt `[after] giving it to him, he went home, to the Xoshga'\footnote{The Xoshga are a band of Hidatsa who separated from the main settlements of the Hidatsa at Like-a-Fishhook Village. These Hidatsa left with Crow Flies High (Hidtsa) and Bobtail Bull (Mandan) around 1870 to live off the reservation imposed upon their peoples by the federal government. The Xoshga wished to live traditional lifestyles. The U.S. Army forced the Xoshga band to abandon their village near the confluence of the Missouri and Yellowstone Rivers in 1894. While these people were mostly Hidatsas, several Mandans and at least one Arikara lived with them at the time that they were forced onto the Fort Berthold Reservation \citep[154ff]{malouf1963}. The families who trace their ancestry back to the Xoshga band are said by Hidatsa speakers of having their own distinct variety of Hidatsa, having lived apart from the main host of the Hidatsa population for over a generation. The name Xoshga comes from the Lakota and Dakota word \textit{ȟóški} `badlands' and the Hidatsa word \textit{adí} `village' \citep[6]{park2012}. While there were Mandan individuals who lived with the Xoshga, no Mandan speakers reported that there was anything unique about the Mandan language these people spoke. This lack of a difference in the Mandan spoken by those living with the Xoshga likely stems from the fact they represented a linguistic and cultural minority in the camp and were thus more likely to use the Hidatsa language. There was a Bull Head among the Xoshga, which is also the name of Mr. Ben Benson, Mr. Edwin Benson's grandfather, but it is not clear if these are two different individuals.} \citep[60]{hollow1973a}
    
\item\label{ExamplesOfTaaWithI6} \glll károotiki, róo numá'ks \textbf{íreextaa} ~ ~ ~ ~ ~ ~ ~ ~ ~ áareehkaroomako'sh\\
    ka=ooti=ki roo ruwą'k=s \textbf{i-reex=taa} ~ ~ ~ ~ ~ ~ ~ ~ ~  aa-rEEh=ka=oowąk=o'sh\\
    prov=evid=cond dem.mid \textnormal{man}=def \textbf{pv.ins-\textnormal{\bfseries glisten}=loc} ~ ~ ~ ~ ~ ~ ~ ~ ~  pv.tr-\textnormal{go.there}=hab=narr=ind.m\\
    \glt `and then, it kept taking this man here towards the light' \citep[95]{hollow1973b}

\end{xlist}

\end{exe}

The presence of a postposition in Mandan usually precludes the presence of definite marking. \citet{coen2021} finds that Siouan languages typically omit any kind of definiteness marking on nouns within a prospositional phrase, suggesting that postpositions carry an innate ambiguous definiteness feature, which appears to also be the case in Mandan. The definite article =\textit{s} appears extremely rarely within the corpus. In (\ref{DefPlusTaa}) below, we see both =\textit{s} and =\textit{taa} on `tree'. There is no real ambiguity over whether the children threw something at any tree or a specific tree, but the =\textit{s} appears here regardless.

\begin{exe}
	\item\label{DefPlusTaa} Co-occurance of definiteness with postposition =\textit{taa}
	
	\glll káni súh{kere}s istámi kirúshaani ~ ~ ~ ~ ~ ~ ~ ~ ~ ~ ~ ~  \textbf{ímanastaa} íkų'te{kere}roomako'sh\\
	ka=rį suk={krE}=s ista\#wį ki-ru-shE=rį ~ ~ ~ ~ ~ ~ ~ ~ ~ ~ ~ ~   \textbf{i-wrą=s=taa} i-kų'tE={krE}=oowąk=o'sh\\
	prov=ss \textnormal{child}={3pl}=def \textnormal{face}\#\textnormal{orb} suus-ins.hand-\textnormal{hold}=ss ~ ~ ~ ~ ~ ~ ~ ~ ~ ~ ~ ~  \textbf{pv.dir-\textnormal{\bfseries tree}=def=loc} pv.dir-\textnormal{throw}={3pl}=narr=ind.m\\
	\glt `and then, the children took their eyes and threw them toward the tree' \citep[29]{hollow1973a}
\end{exe}

Speakers are able to produce novel postpositional phrases that bear definite marking, but such examples are very rare in the corpus.

\subsubsection{Instrumental: =\textit{hąą}}

The second most common bound postposition is the instrumental =\textit{hąą}. This postposition often appears unnasalized as [haa]. This postposition is cognate the Proto-Siouan adverbializer *haa. The non-nasal version is more common in the corpus, but it is not clear if this postposition is simply lightly de-nasalized due to the fact it is word final and often follows a posttonic downstep and reduction in overall intensity. This can lead to word-final vowels having dubious nasal qualities when spoken at a natural pace. If this instrumental postposition is truly cognate with Proto-Siouan *haa, then its nasality is not completely explained. One possibility is that there has been interference between the Proto-Siouan adverbializer *haa and the distal particle *hą, of which we can see reflexes of in Mandan \textit{hą́hik} `time, occasion' and \textit{téehą} `to be far away.'

I follow \citet[25]{kennard1936} in calling this postposition an instrumental, given the fact that it plays that role with noun phrases. Examples of fragments using this postposition appear in (\ref{ExamplesOfInstrumentalHaa}) below.

\begin{exe}
\item\label{ExamplesOfInstrumentalHaa} Examples of instrumental use of =\textit{hąą} from \citet[25]{kennard1936}

\begin{xlist}
\item\label{ExamplesOfInstrumentalHaa1} \glll xé'hąkhąą\\
    xe'\#hąk=hąą\\
    \textnormal{drip}\#stnd.pos=ins\\
    \glt `with baskets'

\item\label{ExamplesOfInstrumentalHaa2} \glll ómanteeshąą\\
     o-wrą\#tee=s=hąą\\
     pv.loc-\textnormal{wood}\#\textnormal{be.dead}=def=ins\\
     \glt `with the axe'
     
\item\label{ExamplesOfInstrumentalHaa3} \glll máahąą\\
    wąąh=hąą\\
    \textnormal{arrow}=ins\\
    \glt `with arrows'

\item\label{ExamplesOfInstrumentalHaa4} \glll maná óksehąą\\
    wrą o-kse=hąą\\
    \textnormal{wood} pv.irr-\textnormal{be.hard}=ins\\
    \glt `with hard wood'
    
\item\label{ExamplesOfInstrumentalHaa5} \glll weréhųųptaahąą\\
    wreh\#ųųp=taa=hąą\\
    \textnormal{door}\#\textnormal{be.different}=loc=ins\\
    \glt `by means of the smoke hole'
    
\end{xlist}

\end{exe}

Within the corpus of narratives itself, there are only a few examples of the instrumental being used in this original sense. One possible reason for this lack of overt instrumental marking in Mandan likely stems from the fact that many verbs bear an instrumental preverb, which already indicates an applicative argument that bears the semantic role of instrument. Verbs that are not lexically marked with an instrumental preverb \textit{i}- will need to have an overt instrumental postposition on the semantic instrument of that clause. We can see examples of this behavior in (\ref{MoreExamplesOfInstrumentalHaa}) below.

\begin{exe}
\item\label{MoreExamplesOfInstrumentalHaa} Examples of =\textit{hąą} within clauses

\begin{xlist}

\item\label{MoreExamplesOfInstrumentalHaa1} \glll rúte óna'te karáxa úroomako'sh, \textbf{máahąą}\\
    rut=E o-rą'=tE krax=E=$\varnothing$ u=oowąk=o'sh \textbf{wąąh=hąą}\\
    \textnormal{rib}=sv pv.loc-\textnormal{be.in.middle}=sv \textnormal{be.low}=sv=cont \textnormal{shoot}=narr=ind.m \textbf{\textnormal{\bfseries arrow}=ins}\\
    \glt `he shot him low between the ribs, with arrows' \citep[41]{hollow1973b}

\item\label{MoreExamplesOfInstrumentalHaa2} \glll \textbf{ómanathąą} ké'nista!\\
    \textbf{o-wrąt=hąą} ke'=rįt=ta\\
    \textbf{pv.loc-\textnormal{\bfseries axe}=ins} \textnormal{dig}=2pl=imp.m\\
    \glt `dig it up with an axe!' \citep[151]{hollow1973b}
    
\item\label{MoreExamplesOfInstrumentalHaa3} \glll wará'taa, \textbf{wéxhąą} ~ ~ ~ ~ ~ ~ ~ ~ ~ ~ ~ ~ ~ ~ ~ ~ ~ ~ ~ ~ ~ ~ ~ ~ ~ ~ ~ ~ ~ ~  íkara'ptehereroomako'sh\\
    wra'=taa \textbf{wex=hąą} ~ ~ ~ ~ ~ ~ ~ ~ ~ ~ ~ ~ ~ ~ ~ ~ ~ ~ ~ ~ ~ ~ ~ ~ ~ ~ ~ ~ ~ ~  i-k-ra'-pte\#hrE=oowąk=o'sh\\
    \textnormal{fire}=loc \textbf{\textnormal{\bfseries coal}=ins} ~ ~ ~ ~ ~ ~ ~ ~ ~ ~ ~ ~ ~ ~ ~ ~ ~ ~ ~ ~ ~ ~ ~ ~ ~ ~ ~ ~ ~ ~  pv.ins-incp-ins.fire-\textnormal{be.burning}\#caus=narr=ind.m\\
    \glt `at the campfire, he lit it up with a lump of coal' \citep[185]{hollow1973b}

\item\label{MoreExamplesOfInstrumentalHaa4} \glll máareksuhkereseena \textbf{páaxuhąą} kashé, ~ ~ ~ ~ ~ ~ ~ ~ ~ ~ sháapa katiríikherek...\\
    wąąreksuk=krE=s=ee=rą \textbf{paaxu=hąą} ka-shE ~ ~ ~ ~ ~ ~ ~ ~ ~ ~ shaap=E=$\varnothing$ ka-triik\#hrE=ak\\
    \textnormal{bird}=3pl=dem.dist=top \textbf{\textnormal{\bfseries nose}=ins} pv.frce-\textnormal{grasp} ~ ~ ~ ~ ~ ~ ~ ~ ~ ~  \textnormal{be.chipped}=sv=cont pv.frce-\textnormal{be.fine}\#caus=ds\\
    \glt `the birds finely chopped it up with their beaks...' \citep[18]{hollow1973a}
    
\item\label{MoreExamplesOfInstrumentalHaa5} \glll numá'ką't \textbf{mí'hąą} waróotki'sh\\
    ruwą'k=ą't \textbf{wį'=hąą} wa-rootki=o'sh\\
    \textnormal{man}=dem.anap \textbf{\textnormal{\bfseries stone}=ins} 1a-\textnormal{hit}=ind.m\\
    \glt `I hit that man with a rock' \citep[432]{hollow1970}
    
\item\label{MoreExamplesOfInstrumentalHaa6} \glll \textbf{ų́khąą}, minís rusháni istámis ~ ~ ~ ~ ~ ~ ~ ~ ~ ~  kirusá'roomako'sh\\
    \textbf{ųk=hąą} wrį=s ru=shE=rį ista\#wį=s ~ ~ ~ ~ ~ ~ ~ ~ ~ ~ krusa'=oowąk=o'sh\\
    \textbf{\textnormal{\bfseries hand}=ind} \textnormal{water}=def ins.hand-\textnormal{grasp}=ss \textnormal{face}\#\textnormal{orb}=def ~ ~ ~ ~ ~ ~ ~ ~ ~ ~ \textnormal{wash}=narr=ind.m\\
    \glt `with his hands, he took the water and washed his eyes out' \citep[37]{hollow1973a}
    
\item\label{MoreExamplesOfInstrumentalHaa7} \glll áahiki, rútkeres \textbf{maná'p} ~ ~ ~ ~ ~ ~ ~ ~ ~ ~ ~ ~ ~ ~ ~  ípaminiishe...\\
    aa-hi=ki rut=krE=s \textbf{wrą\#a'p} ~ ~ ~ ~ ~ ~ ~ ~ ~ ~ ~ ~ ~ ~ ~  i-pa-wrįsh=E\\
    pv.tr-\textnormal{arrive.there}=cond \textnormal{rib}=3pl=def \textbf{\textnormal{\bfseries tree}\#\textnormal{\bfseries leaf}} ~ ~ ~ ~ ~ ~ ~ ~ ~ ~ ~ ~ ~ ~ ~  pv.ins-ins.push-\textnormal{wrap.up}=sv\\
    \glt `when he got them there, he had wrapped the ribs with leaves...' \citep[177]{hollow1973a}
    

\end{xlist}
\end{exe}

In the data above, we can see that the absence of the instrumental preverb \textit{i}- necessitates the presence of using the postposition =\textit{hąą} to mark what noun is playing the role of the instrument. It is not ungrammatical to mark the instrument with the instrumental postposition with a verb bearing the instrumental preverb, as we see in (\ref{MoreExamplesOfInstrumentalHaa3}). However, such constructions are rare in the corpus. It is more common to see constructions like the one in (\ref{MoreExamplesOfInstrumentalHaa7}), where the presence of the instrumental preverb \textit{i}- marks marks an argument as the instrument. The ordering of noun phrases within such constructions can vary, depending on what kind of elements the speaker wishes to topicalize and shift leftward in the clause, so context rather than canonical word order is often a bigger determiner in identifying which noun is the instrument in verbs with multiple overt nominal arguments of a verb marked with the instrumental preverb.

All Mandan speakers in the corpus are also fluent Hidatsa speakers, and Hidatsa has a postpositional element \textit{-haa} that marks some kind of path. Mandan can treat the instrumental =\textit{hąą} in an identical manner. It is unclear if this use is due to influence from Hidatsa or if there was some kind of polysemy in a language that is ancestral to Mandan and the two Missouri Valley Siouan languages, Hidatsa and Crow. When the instrumental is added to nouns that are destinations, it is often glossed as `around' or `through'. We can see examples of these non-instrumental readings in (\ref{NonInstrumentalHaa}) below.

\begin{exe}

\item\label{NonInstrumentalHaa} Examples of non-instrumental =\textit{hąą}

\begin{xlist}

\item\label{NonInstrumentalHaa1} \glll hų́ áaki'hkarani \textbf{tírokhąą} ~ ~ ~ ~ ~ ~ ~ ~ ~ ~ ~ ~ ~  rupiriha...\\
    hų aa-ki'h=krE=rį \textbf{ti=rok=hąą} ~ ~ ~ ~ ~ ~ ~ ~ ~ ~ ~ ~ ~ ru-prih=E=$\varnothing$\\
    \textnormal{all} pv.tr-\textnormal{arrive.there}.vert=3pl=ss \textbf{\textnormal{\bfseries house}=ines=ins} ~ ~ ~ ~ ~ ~ ~ ~ ~ ~ ~ ~ ~  ins.hand-\textnormal{be.spread.out}=sv=cont\\
    \glt `they brought a lot home and were spreading it out around inside the house...' \citep[184]{hollow1973a}

\item\label{NonInstrumentalHaa2} \glll inák, máatah \textbf{íwookąhąą}, ~ ~ ~ ~ ~ ~ ~ ~ ~ ~  Kinúma'kshi kasími réehoomako'sh\\
    irąk wąątah \textbf{i-wV-o-kąh=hąą} ~ ~ ~ ~ ~ ~ ~ ~ ~ ~  ki-ruwą'k\#shi ka-si=awį rEEh=oowąk=o'sh\\
    \textnormal{again} \textnormal{river} \textbf{pv.dir-unsp-pv.loc-\textnormal{\bfseries be.on.edge}=ins} ~ ~ ~ ~ ~ ~ ~ ~ ~ ~ mid-\textnormal{man}\#\textnormal{be.good} incp-\textnormal{travel}=cont \textnormal{go.there}=narr=ind.m\\
    \glt `again, along the river bank, Royal Chief was going along traveling' \citep[14]{hollow1973a}

\item\label{NonInstrumentalHaa3} \glll \textbf{tíhųshhąą} saráthara réehe, kánik ó'aaki kxų́hinik\\
    \textbf{ti\#hųsh=hąą} srat\#hrE=$\varnothing$ rEEh=E ka=rįk o-aaki kxųh=rįk\\
    \textbf{\textnormal{\bfseries house}\#\textnormal{\bfseries perimeter}=ins} \textnormal{be.thick}\#caus=cont  \textnormal{go.there}=sv prov=iter pv.loc-\textnormal{be.one.top} \textnormal{lie.down}=iter\\
    \glt `they put it in thick around the edge of the house and laid down on it' \citep[184]{hollow1973a}

\item\label{NonInstrumentalHaa4} \glll \textbf{réshkataahąą} tí ówaxastaa ítit róo \textbf{íshąąhtaahąą}\\
    \textbf{reshka=taa=hąą} ti o-waxat=taa i-ti=t roo \textbf{i-shąąh=taa=hąą}\\
    \textbf{\textnormal{\bfseries like.this}=loc=ins} \textnormal{house} pv.loc-\textnormal{form.a.point}=loc pv.dir-\textnormal{house}=loc dem.mid \textbf{pv.dir-\textnormal{\bfseries other.side}=loc=ins}\\
    \glt `along like that in the corner, towards the house here, along the other side of it' \citep[71]{trechter2012}
    
\item\label{NonInstrumentalHaa5} \glll \textbf{óhophąą} wáanuma'kaaki hų́ni síta ~ ~ ~   máakaho'sh\\
    \textbf{o-hop=hąą} waa-ruwą'k-aaki hų=rį sit=E=$\varnothing$ ~ ~ ~ wąąkah=o'sh\\
    \textbf{pv.loc-\textnormal{\bfseries hole}=ins} nom-\textnormal{person}-coll \textnormal{many}=ss \textnormal{be.noisy}=sv=cont ~ ~ ~  \textnormal{lie}.aux.hab=ind.m\\
    \glt `some people will come through a hole and they will be noisy' \citep[200]{hollow1973a}
    
\item\label{NonInstrumentalHaa6} \glll ó'harani, maná \textbf{ósisiphąą} ~ ~ ~ ~ ~ ~ ~ ~ ~ ~ ~ ~ ~ ~ ~   áareehkereroomako'sh\\
    o'\#hrE=rį wrą \textbf{o-si$\sim$sip=hąą} ~ ~ ~ ~ ~ ~ ~ ~ ~ ~ ~ ~ ~ ~ ~   aa-rEEh=krE=oowąk=o'sh\\
    \textnormal{be}\#caus=ss \textnormal{tree} \textbf{pv.loc-dist$\sim$\textnormal{\bfseries be.rough}=ins} ~ ~ ~ ~ ~ ~ ~ ~ ~ ~ ~ ~ ~ ~ ~ pv.tr-\textnormal{go.there}=3pl=narr=ind.m\\
    \glt `from there, she took him through the thick brush' \citep[49]{hollow1973b}

\end{xlist}

\end{exe}

Mandan and both Missouri Valley Siouan languages have a long vowel for this reflex of the Proto-Siouan adverbializer *haa, but nasality is only apparant in Mandan due to Missouri Valley's complete loss of phonemic nasality on vowels. While \citet{rankin2015} reconstructs this formative as having an oral vowel, under the notes for the demonstrative *ha, the aforementioned authors remark that both vowel qualities are observed in reflexes of the so-called adverbializer in Proto-Siouan. The fact that =\textit{hąą} in Mandan is often heard without nasality suggests that perhaps both versions of this formative were possible in Proto-Siouan as well, and that certain languages tended towards the oral version, while other languages adapted the nasal version. 

Both the oral and nasal versions of =\textit{hąą} are found on the same stems, suggesting some degree of interchangeability between them. Due to the lack of L1 speakers, it is not currently possible to confirm that both [haː] and [hãː] are acceptable in the same way that [iðəɹ] and [aɪ̯ðəɹ] are both acceptable pronunciations for the English word \textit{either}. The simultaneous enclitic =\textit{hąą} appears to be diachronically connected to this Proto-Siouan adverbial marker *haa$\sim$*hąą, especially seeing as how Mandan has semantically broadened its use to mark instruments. The simultaneous aspectual is an ablaut-triggering enclitic, as described in \sectref{ablaut}. Given that almost all ablaut-triggering enclitics contain a nasal vowel, I assume that the nasal version of the instrumental postposition =\textit{hąą} to be the original form and have transcribed all occurrences of this formative in the corpus as such. The non-nasal version may have arisen due to mass bilingualism between Mandan and Hidatsa speakers, and the cultural context of Hidatsa being the so-called default indigenous language used on Fort Berthold. As such, realizations of =\textit{hąą} as =\textit{haa} may be due to long-term contact with Hidatsa.

\subsubsection{Inessives: =\textit{ku'sh} and =\textit{rok}}

There are two postpositions that are typically translated as having inessive semantics: =\textit{ku'sh} and =\textit{rok}. These inessive postpositions can sometimes be used interchangeably, but their meanings are not equivalent. Of the two, =\textit{ku'sh} is the more common, and it has the added ability to exist as the stem when additional postpositional morphology is added, e.g., =\textit{taa} or =\textit{hąą}. The difference between these two inessives is that =\textit{ku'sh} has a more generalized meaning, while =\textit{rok} appears to convey the meaning that something is more centrally located within some boundary. 

\subsubsubsection{Inessive: =\textit{ku'sh} `inside'}

The postposition =\textit{ku'sh} is typically translated as `inside', and often appears on physical structures like one's home or some other building. This postposition is also found when discussing entrails or objects that are enclosed within the body of a human or animal. This postposition seems to be related to \textit{kú'} `be far away, beyond', with the /ʃ/ being cognate with the final segment of the male-directed indicative enclitic =\textit{o'sh}. The semantics do not seem to line up between being far away from something and being inside something, so the connection between these two lexical items is opaque, as there are no other obvious cognates in Proto-Siouan that have been constructed to date \citet{rankin2015}.\footnote{Another possibility is that the contamination went in the direction of `inside' to `beyond', as both forms in Mandan involve a coda /ʔ/, which should indicate a glottalized onset in Proto-Siouan, i.e., PSi *kʔu-sV for `inside'. The reconstruction for `beyond', however, has no glottalized onset in Proto-Siouan, *ku, but there is a glottal stop coda in Mandan \textit{kú'}. This discrepancy between the presence of an unaccounted for glottal stop in the Mandan `beyond' could indicate that there really is no directly shared reconstruction from Proto-Siouan between these two lexical items, but that their similarity may be the result of conflation of *ku `beyond' with some heretofore undocumented *kʔu that possibly had semantics similar to `within'. Further comparative work with other Siouan languages is needed to determine how this hypothesis bears fruit.}  We can see examples of its use in the data in (\ref{ExamplesOfKush}) below.

\begin{exe}

\item\label{ExamplesOfKush} Examples of =\textit{ku'sh}

\begin{xlist}

\item\label{ExamplesOfKush1} \glll wí'h\textbf{ku'sh}e xóoreshka'sh\\
    w'-iih=\textbf{ku'sh}=E xoo=eshka=o'sh\\
    1poss-\textnormal{mouth}=\textbf{\textnormal{\bfseries be.inside}}=sv \textnormal{ice}=smlt=ind.m\\
    \glt `inside of my mouth is just like ice' \citep[119]{hollow1973a}

\item\label{ExamplesOfKush2} \glll  ptaníshkere máa'ąh\textbf{ku'sh}taa reehwahere'sh\\
	p-ta-rįshkrE wąą'ąk=\textbf{ku'sh}=taa rEEh\#wa-hrE=o'sh\\
	1poss-al-\textnormal{medicine} \textnormal{earth}=\textbf{\textnormal{\bfseries be.inside}}=loc \textnormal{go.there}\#\textbf{1a}-caus-ind.m\\
	\glt `I put my medicine under [inside] the ground' \citep[48]{hollow1973a}    
	
\item\label{ExamplesOfKush3} \glll hų́ps tí\textbf{ku'sh}t áaropxeka'ehe\\
    hųp=s ti=\textbf{ku'sh}=t aa-ropxE=ka'ehe\\
    \textnormal{moccasin}=def \textnormal{house}=\textbf{\textnormal{\bfseries be.inside}}=loc pv.tr-\textnormal{enter}=quot\\
    \glt `she took the shoes inside the house, it is said' \citep[169]{hollow1973a}
    
\item\label{ExamplesOfKush4} \glll miní í\textbf{ku'sh}taa ropxáni réeho'sh\\
    wrį i-\textbf{ku'sh}=taa ropxE=rį rEEh=o'sh\\
    \textnormal{water} pv.dir-\textbf{\textnormal{\bfseries be.inside}}=loc \textnormal{enter}=ss \textnormal{go.there}=ind.m\\
    \glt `he went and entered under [inside] the water' \citep[10]{hollow1973b}

\item\label{ExamplesOfKush5} \glll ą́'teroo, \textbf{kú'sh}taa, rópxani máakana!\\
    ą't=roo \textbf{ku'sh}=taa ropxE=rį wąąkE=rą\\
    dem.anap=dem.mid \textbf{\textnormal{\bfseries be.inside}}=loc \textnormal{enter}=ss \textnormal{lie}.aux=imp.f\\
    \glt `there, inside, go in and stay there!' \citep[108]{hollow1973b}

\item\label{ExamplesOfKush6} \glll ímashute, \textbf{kúsh}taahąą íwa'tarakini...\\
    i-wąshut=E \textbf{ku'sh}=taa=hąą i-wa'-trEk=rį\\
    pv.ins-\textnormal{clothe}=sv \textbf{\textnormal{\bfseries be.inside}}=loc=ins pv.ins-ins.prce-\textnormal{sew}=ss\\
    \glt `it being inside her dress, she sewed it on...' \citep[106]{hollow1973a}
\end{xlist}
\end{exe}

This inessive most often appears as a bound element after the nominal over which it has semantic scope, as we see in (\ref{ExamplesOfKush1}) through (\ref{ExamplesOfKush3}). The bound version of =\textit{ku'sh} is often accompanied by the locative =\textit{taa} or =\textit{t}. A seemingly unbound version of =\textit{ku'sh} is possible when there is some other morphology present that is also bound, using /ku'sh/ as a stem, as we see in (\ref{ExamplesOfKush4}) through (\ref{ExamplesOfKush6}). The difference between the bound and unbound versions is that the bound =\textit{ku'sh} has an overt nominal that it seeks out as a host, while the unbound version has some null nominal it is referencing. Thus, while =\textit{ku'sh} appears to be unbound in instances like the ones we see in (\ref{ExamplesOfKush4}), the prosody provides a reading that is more like, `the water, he went and entered it.' Given that Mandan is such a prolific \textit{pro}-drop language that can omit any overt nominal, regardless of what role it is playing in clause. Thus, the lack of an overt nominal in some instances is what is giving the appearance of an unbound version of =\textit{ku'sh}, when the reality is that all instances of =\textit{ku'sh} are bound, but some are bound to phonologically null elements that have been dropped from the discourse but are still being tracked.

\subsubsubsection{Inessive: =\textit{rok} `within'}

This postposition is a reflex of Proto-Siouan *yooka$\sim$*rooka, which has a meaning of `be inside' or `be in the middle of' in modern Siouan languages. The Mandan reflex appears to be phonologically reduced, having undergone apocape and vowel truncation in its current state as an enclitic. The postposition =\textit{rok} is typically translated as `within', `in', or `among', and seems to be more restricted in use than the other inessive, =\textit{ku'sh}. The inessive =\textit{rok} is typically relegated to situations where something is within a boundary where it is near its middle. We can see examples of =\textit{rok} in (\ref{ExamplesOfRok}) below.

\begin{exe}

\item\label{ExamplesOfRok} Examples of =\textit{rok}

\begin{xlist}

\item\label{ExamplesOfRok1} \glll minís \'{ı̨}'sta\textbf{rok}e síireena ~ ~ ~ ~ ~ ~ ~ ~ ó'aakupak\\
    wrįs į'-ista=\textbf{rok}=E sii=ee=rą ~ ~ ~ ~ ~ ~ ~ ~  o-aakup=ak\\
    \textnormal{horse} pv.rflx-\textnormal{face}=\textbf{\textnormal{be.within}}=sv \textnormal{be.yellow}=dem.dist=top ~ ~ ~ ~ ~ ~ ~ ~  pv.irr-\textnormal{cover.the.head}=ds\\
    \glt `it was a horse that was yellow in the middle of the face that was broke to harness' \citep[59]{hollow1973a}
    
\item\label{ExamplesOfRok2} \glll kú't maná\textbf{rok} ósisiiptaa ~ ~ ~ ~ ~ ~ ~ ~ áaraahini...\\
    ku'=t wrą=\textbf{rok} o-si$\sim$siip=taa ~ ~ ~ ~ ~ ~ ~ ~ aa-rEEh=rį\\
    \textnormal{be.further}=loc \textnormal{tree}=\textbf{\textnormal{be.within}} pv.loc-aug$\sim$\textnormal{be.rough}=loc ~ ~ ~ ~ ~ ~ ~ ~ pv.tr-\textnormal{go.there}=ss\\
    \glt `he took her further into the brush where it is thick and...' \citep[176]{hollow1973a}


\item\label{ExamplesOfRok3} \glll tí\textbf{rok}shka shíhara íseka\\
    ti=\textbf{rok}=shka shi\#hrE=$\varnothing$ i-sek=E=$\varnothing$\\
    \textnormal{dwelling}=\textnormal{be.within}=dsj \textnormal{be.good}\#caus=cont pv.ins-\textnormal{make}=sv=cont\\
    \glt `they were making it nice even inside the house' \citep[202]{hollow1973a}
    
\item\label{ExamplesOfRok4} \glll hų́ áaki'hkarani ~ ~ ~ ~ ~ ~ ~ ~ ~ ~ ~ ~ ~ ~ ~ ~ ~ ~ ~ tí\textbf{rok}hąą rupiríha...\\
    hų aa-ki'h=krE=rį ~ ~ ~ ~ ~ ~ ~ ~ ~ ~ ~ ~ ~ ~ ~ ~ ~ ~ ~  ti=\textbf{rok}=hąą ru-prih=E=$\varnothing$\\
    \textnormal{many} pv.tr-\textnormal{arrive.back.here}=3pl=ss ~ ~ ~ ~ ~ ~ ~ ~ ~ ~ ~ ~ ~ ~ ~ ~ ~ ~ ~ \textnormal{dwelling}=\textbf{\textnormal{be.within}}=ins ins.hand-\textnormal{spread}=sv=cont\\
    \glt `they brought a lot of it home and were spreading it out all over inside the house...' \citep[184]{hollow1973a}


\item\label{ExamplesOfRok5} \glll k\'{ı̨}'kini, maná ósake tí\textbf{rok}taa, ~ ~ ~ ~ ~ ~ ~ ~ ~  óti'į'here xamáhshka'nik\\
    kį'k=rį wrą o-sak=E ti=\textbf{rok}=taa ~ ~ ~ ~ ~ ~ ~ ~ ~ o-ti\#į'-hrE xwąh=shka'rįk\\
    \textnormal{finish}=ss \textnormal{wood} pv.irr-\textnormal{be.dry}=sv \textnormal{dwelling}=\textbf{\textnormal{be.within}}=loc ~ ~ ~ ~ ~ ~ ~ ~ ~  pv.loc-\textnormal{dwell}\#pv.rflx-caus \textnormal{be.small}=dsj\\
    \glt `they finished it and the wood that was dry was in the house, but their house was small' \citep[197]{hollow1973a}

\end{xlist} 

\end{exe}

Much like =\textit{ku'sh}, =\textit{rok} can appear with additional postpositions. We can see both =\textit{taa} and =\textit{hąą} present in (\ref{ExamplesOfRok4}) and (\ref{ExamplesOfRok5}). The presence of the locative =\textit{taa} seems to be used to mark specific locations headed by =\textit{rok}, while the instrumental =\textit{hąą} indicates all throughout a location.

This postposition is relatively uncommon in the corpus, but it does appear to be quite productive in creating nominal elements that are then treated as nouns within the syntax. We can see some examples of =\textit{rok} constructions that have become lexicalized in (\ref{ExamplesOfNominalRok}) below.

\newpage

\begin{exe}

\item\label{ExamplesOfNominalRok} Examples of nouns created with =\textit{rok}

\begin{xlist}
 
\item\label{ExamplesOfNominalRok1} \glll manárok\\
    wrą=rok\\
    \textnormal{tree}=\textnormal{be.within}\\
    \glt `forest' [lit. `among the trees']

\item\label{ExamplesOfNominalRok2} \glll manáherok\\
    wrąh=rok\\
    \textnormal{few.trees}=\textnormal{be.within}\\
    \glt `meadow, clearing' [lit. `within where there are few trees']

\item\label{ExamplesOfNominalRok3} \glll róxerok\\
    rox=rok\\
    \textnormal{pelvis}=\textnormal{be.within}\\
    \glt `vulva' [lit. `within the hips']
    
\item\label{ExamplesOfNominalRok4} \glll ótirok\\
    o-ti=rok\\
    pv.irr-\textnormal{dwell}=\textnormal{be.within}\\
    \glt `ancestor' [lit. `those who live on within']

\item\label{ExamplesOfNominalRok5} \glll ų́kerok\\
    ųk=rok\\
    \textnormal{hand}=\textnormal{be.within}\\
    \glt `palm of hand' [lit. `inside of the hand']

\item\label{ExamplesOfNominalRok6} \glll íroke $\sim$ wíiroke\\
    i-rok=E ~ wV-i-rok=E\\
    pv.ins-\textnormal{be.within}=sv ~ nom-pv.ins-\textnormal{be.within}=sv\\
    \glt `bag, barrel, box, case container, pouch' [lit. `something that has it inside']

\item\label{ExamplesOfNominalRok7} \glll wáatirok\\
    waa-ti=rok\\
    nom-\textnormal{dwelling}=\textnormal{be.within}\\
    \glt `furniture' [lit. `something within the home']

\item\label{ExamplesOfNominalRok8} \glll tírok\\
    ti=rok\\
    \textnormal{dwelling}=\textnormal{be.within}\\
    \glt `family, household' [lit. `someone within our home']

\end{xlist}
 
\end{exe}

\newpage
While these nominals bear =\textit{rok}, they are treated as nouns synchronically, as they can take definite marking with =\textit{s} on the right edge of the word, alienable possession marking with \textit{ta}-, plural marking with =\textit{kere}, and even other postpositions, e.g., \textit{írokekeres} `the bags' or \textit{íroketaa} `in the bag.'
 
\subsection{Free postpositions}\label{SubSecFreePostpositions}

The majority of postpositions in Mandan exist as free formatives that can appear in isolation without having to be bound to the noun over which it has semantic scope. Free postposition are polysyllabic and bear a primary stress in contrast with bound postpositions, which manifest as enclitics that can never bear primary stress, even if doing so would produce a well-formed iambic foot as described in \sectref{stress}.

Mandan postpositions fall into two classes: some are morphologically simple whereas some take a particular preverb or are compound words. Morphologically simple postpositions are uniformly treated as stative verbs in the even that first or second person entities are involved, while the non-simple postpositions are divided by whether they take stative pronominals or active pronominals. Non-simple postpositions that contain a reflexive, instrumental, or directional preverb take stative marking, while all other non-simple postpositions take active marking. Some simple postpositions can co-occur with the locative and the instrumental optionally, which is indicated by including these enclitics after the stem, while some simple postpositions will always co-occur with the locative =\textit{taa}$\sim$=\textit{t}. We can see lists of both classes below; simple postpositions appear in (\ref{ListOfPostpositions}), and complex postpositions appear in (\ref{ListOfPostpositions2}).  

\begin{exe}

\item\label{ListOfPostpositions} List of postpositions with simple morphology

\begin{xlist}
\item \textit{áaki}, =\textit{taa}, =\textit{hąą} `above, over'

\item \textit{ą́'ska}, =\textit{taa}, =\textit{hąą} `near'

\item \textit{hų́ųtaa} `beside, next to'

\item \textit{kahų́h} `all around'

\item \textit{kótki} `across [horizontal]'

\item \textit{kútaa} `across from, opposite from'

\item \textit{máape} `beneath, under'

\item \textit{máatih}, =\textit{taa}, =\textit{hąą} `outside'

\item \textit{náashi}, =\textit{taa}, =\textit{hąą} `behind, in back of'

\item \textit{pahų́ųtaa} `beside, next to'

\item \textit{péexti} `in front of'

\item \textit{pkahų́ųtaa} `side by side with'

\item \textit{rohų́ųtaa} `beside, next to, by'

\item \textit{ruxeré}, =\textit{taa}, =\textit{hąą} `very far away from, distant from'

\item \textit{shą́ąh}, =\textit{taa} `across, other side of'

\item \textit{téehą} `far from'

\item \textit{ų́ųpa} `with'

\item \textit{ų́ųpataa} `nearby, around, in the vicinity of'

\item \textit{ų́'shka} `along with'

\item \textit{ų́'taa} `to, toward, near'

\item \textit{wáakutaa} `over, over the top of'

\end{xlist}

\item\label{ListOfPostpositions2} List of postpositions with non-simple morphology

\begin{xlist}
\item \textit{éeheni} `along with, in addition to'
\item \textit{í'ų'taa} `to, toward'
\item \textit{į́'sąąpe} `around'
\item \textit{óktiki} `to, toward'
\item \textit{ótaa} `facing towards, pointed at'
\item \textit{ó'hara} `through'
\item \textit{ó'harani} `from'
\end{xlist}

\end{exe}

All simple postpositions may also take the directional preverb \textit{i}-, even ones that are already expressing a directional relationship, e.g., \textit{ų́'taa} versus \textit{í'ų'taa} `towards'. The simple postpositions that feature the locative =\textit{taa} have not been attested in the corpus  without it, so it is not clear whether a bare postposition without the locative would be possible or if it would have slightly altered semantics.

Non-simple postpositions are typically composed of either a preverb plus another element, or they are a combination of the copula \textit{ó'} `be' plus causative morphology. Some of these non-simple postpositions have obvious etymologies, like \textit{ótaa} `facing towards, pointed at', which is a combination of the locative preverb \textit{o}- and the locative postposition =\textit{taa}. Others, like \textit{ó'harani} `from' have less obvious connections between the constituent formatives and their combined semantics, i.e., \textit{ó'harani} is composed of the copula \textit{ó'} `be', plus the causative \textit{here} and the same-subject switch-reference marker =\textit{ni}. Inflectional morphology on these non-simple postpositions will always appear after the preverb or the compounded copula, as we can see in the data  in (\ref{NonSimplePostpositionsExamples}) below.

\newpage

\begin{exe}

\item\label{NonSimplePostpositionsExamples} Examples of person marking on non-simple postpositions

\begin{xlist}

\item\label{NonSimplePostpositionsExamples1} \glll ímaataht \textbf{ó}ra\textbf{taa}ro'sh\\
    i-wąątah=t \textbf{o}-ra-\textbf{taa}=o'sh\\
    pv.dir-\textnormal{river}=loc \textbf{pv.loc}-2a-\textbf{\textnormal{be.facing}}=ind.m\\
    \glt `you are facing the river' \citep[35]{hollow1973a}

\item\label{NonSimplePostpositionsExamples2} \glll tewétaa \textbf{ó}'ra\textbf{harani} rahúuro'sha?\\
    t-we=taa \textbf{o'}\#ra-\textbf{hrE=rį} ra-huu=o'sha\\
    wh-indf=loc \textbf{\textnormal{be}}\#2a-\textbf{caus=ss} 2a-\textnormal{come.here}=int.m\\
    \glt `where did you come from?' \citep[299]{hollow1973a}

\end{xlist}

\end{exe}

It is possible that other postpositions existed, but no other candidates could be identified from the corpus. Furthermore, speakers seem to forego overt postpositions if the relationship between an entity and a proposition is understood from the context, as we can see from the many instances of nouns that are destinations that we would expect to bear some kind of directional postposition. One possible explanation for this discrepancy is that Mandan verbs may convey information pertaining to spatial or directional relationships to an object that might have otherwise been relegated to adpositions in other languages. If there are other recordings done by the Nueta Language Initiative or produced by families of other Mandan speakers in the late twentieth or early twenty-first centuries, such recordings may yield additional postpositions. However, those listed in (\ref{ListOfPostpositions}) and (\ref{ListOfPostpositions2}) represent the sum of postpositions attested in Mandan to date.
 
\section{Deixis and definiteness}\label{SecDeixis}

Mandan rigorously adds ancillary information to many propositions, especially nominal constructions. This ancillary information contextualizes or reinforces the spatial or temporal deixis of an entity. In addition to grounding the location of an entity in time and space, Mandan also can mark the position of said entity within the context of whether said entity is standing, sitting, or lying. Mandan can overtly mark definiteness on nouns, which is discussed in \sectref{SubSecArticles} below. In \sectref{SubSecDemonstratives}, I describe the behavior of demonstratives in Mandan with respect to their distance from the speaker and their positional status. This section concludes with a description of topic marking in \sectref{SubSecTopics}.

\subsection{Articles}\label{SubSecArticles}

Mandan has only one article, the definite article =\textit{s}. This article is homophonous and largely semantically consistent with the verbal definite evidential marker =\textit{s} described in \sectref{SubSubSecDefiniteEvidential}, which is found on verbs to express certainty on behalf of the speaker about the truth value of the proposition. It is not clear if the dual usage of =\textit{s} evolved along a single pathway and then spread from nominal to verbal morphology or vice versa. Both Crow and Hidatsa have remnants of a similar duality between marking definiteness on nouns and expressing certainty on verbs, as the definite article in Hidatsa -\textit{sh} is also used with forceful statements (\citealt[68]{boyle2007}; \citealt[231]{park2012}), while the Crow definite article -\textit{sh} manifests as -\textit{sht} for strong declarations \citep[394]{graczyk2007}.\footnote{Hidatsa also has strong declarative markers that are cognate with Crow, i.e., Crow -\textit{sht} is equivalent to Hidatsa -\textit{shd}. There is also a third ``definitive'' speech-act marker in Hidatsa, -\textit{shdaa'}, which \citet[231]{park2012} notes is the most common manifestation of how a forceful declarative statement is marked in careful speech, while -\textit{shd} is more common than -\textit{sh} otherwise. It is not clear if these other definitive markers in Crow and Hidatsa are related to reflexive of an š-grade version of the Proto-Siouan augmentative *xtE. Thus, it is possible that the homophony between the definite marker in all three languages and the strong or definitive declarative markers in Crow and Hidatsa is coincidental.}

Of all the nouns that exist in the corpus, a minority bear the definite article =\textit{s}, though in the majority of cases where the translation includes a definite article, a definite article also occurs in the original Mandan. There are no morphological or syntactic distinctions between a generic reading on a noun versus an indefinite one in Mandan. Since most nouns are not overtly marked for definiteness, definiteness can be implied by context. However, when =\textit{s} is present, such nouns are unambiguously definite. We can see examples of definite marking in the data in (\ref{ExamplesOfS}) below.

\begin{exe}

\item\label{ExamplesOfS} Examples of the definite article =\textit{s}

\begin{xlist}

\item\label{ExamplesOfS1} \glll maná teré\textbf{s} óruskani ~ ~ ~ ~ ~ ~ ~ ~ ~ ~ í'aakit áareehka\\
    wrą trE=\textbf{s} o-ru-skE=rį ~ ~ ~ ~ ~ ~ ~ ~ ~ ~ i-aaki=t aa-rEEh=ka\\
    \textnormal{tree} \textnormal{be.big.around}=\textbf{def} pv.loc-ins.hand-\textnormal{jump}=ss ~ ~ ~ ~ ~ ~ ~ ~ ~ ~ pv.dir-\textnormal{be.above}=loc pv.tr-\textnormal{go.there}=hab\\
    \glt `he pulled the big tree [out of the ground] and would take it upward' \citep[21]{hollow1973a}

\item\label{ExamplesOfS2} \glll numá'kshi\textbf{s} mí'ti óo manáshhįįra máakahkereroomako'sh\\
ruwą'kshi=\textbf{s} wį'\#ti oo wrąsh\#hįį=E=$\varnothing$ wąąkah=krE=oowąk=o'sh\\
    \textnormal{man}\#\textnormal{be.good}=\textbf{def} \textnormal{stone}\#\textnormal{dwelling} dem.mid \textnormal{tobacoo}\#\textnormal{drink}=sv=cont \textnormal{be.lying}.aux.hab=3pl=narr=ind.m\\
    \glt `they were always there smoking with the village chief there' \citep[230]{hollow1973b}
    
\item\label{ExamplesOfS3} \glll súk\textbf{s} í'ų'taa kų́'ta!\\
    suk=\textbf{s} i-ų'=taa kų'=ta\\
    \textnormal{child}=\textbf{def} pv.dir=\textnormal{be.near.to}=loc \textnormal{give}=imp.m\\
    \glt `give it to the child!' \citep[25]{kennard1936}

\item\label{ExamplesOfS4} \glll wáa'oshi\textbf{s} ą́ąwe íkatarakak, hį, ~ ~ ~ ~ ~ ~   íkihąąxikoomako'sh\\
    waa-o-shi=\textbf{s} ąąwe i-ka-trak=ak hį ~ ~ ~ ~ ~ ~   i-ki-hąąxik=oowąk=o'sh\\
    nom-pv.irr-\textnormal{be.good}=\textbf{def} \textnormal{all} pv.ins-incp-\textnormal{be.blocked}=ds \textnormal{um} ~ ~ ~ ~ ~ ~  pv.ins-mid-\textnormal{forget}=narr=ind.m\\
    \glt `all the good things suddenly happend to him and, well, he just forgot all about him' \citep[251]{trechter2012}
    
\item\label{ExamplesOfS5} \glll Núu'etaa tamí'tina nákoomako'sh\\
    rųų'etaa ta-wį'\#ti=rą rąk=oowąk=o'sh\\
    \textnormal{Mandan} al-\textnormal{stone}\#\textnormal{dwelling}=top pos.sit=narr=ind.m\\
    \glt `there once was a Mandan village' \citet[125]{hollow1973a}

% \item\label{ExamplesOfS6} \glll Kinúma'kshi máatah íwookahąą kasími\\
%     ki-ruwą'k\#shi wąątah i-wooka=hąą ka-si=awį\\
%     mid-\textnormal{man}\#\textnormal{be.good} \textnormal{river} pv.ins-\textnormal{be.at.edge.of}=ins ins.frce-\textnormal{travel}=cont\\
%     \glt `Royal Chief was traveling along the edge of the river' \citep[20]{hollow1973a}

\end{xlist}

\end{exe}

We see =\textit{s} in (\ref{ExamplesOfS1}) through (\ref{ExamplesOfS4}) on the noun phrases where we expect them, according to the context of the translations. Likewise, when no definite marking occurs, we see no =\textit{s}, like in (\ref{ExamplesOfS5}). In (\ref{ExamplesOfS5}), the narrator is describing some Mandan village that had existed in the past without making reference to a specific one, e.g., Like-A-Fishhook Village or Double Ditch Village. The example in (\ref{ExamplesOfS5}) likewise illustrates that there is no overt morphology for indefinite nouns in Mandan.

\citet[26]{kennard1936} reports that there is an indefinite article, \textit{-e}. As previously discussed in \sectref{stemvowel}, this formative does not represent indefinite marking on nouns, as it is the stem vowel /=E/. This stem vowel occupies the same position as a complementizer in a clause and serves to indicate the end of an intonational phrase. For that reason, it often appears when speakers are asked, ``how you do say X in Mandan?'' The presence of such morphology indicates a complete utterance rather than a fragment. \citet[39]{hollow1970} likewise does not consider this element an indefinite article, simply stating that it is part of an optional rule for adding a word-final vowel that is at the discretion of the speaker.\footnote{Refer back to \sectref{stemvowel} for a detailed explanation of when the stem vowel /=E/ must occur.} What is clear is that no one but \citeauthor{kennard1936} thought that there was an indefinite article in Mandan; all other published work on Mandan finds the contrary is true (\citealt{hollow1970}; \citealt{mixco1997a}; \citealt{kasak2019}; \textit{inter alios}).

\subsection{Demonstratives}\label{SubSecDemonstratives}

Mandan possesses a rich system of encoding both definiteness and deictic information alongside nouns. While the definite article =\textit{s} described above in \sectref{SubSecArticles} has the sole purpose of marking a nominal element as having the semantic property of being definite, there are two other kinds of formatives used in Mandan to express some kind of spatial context for a noun phrase. Some of these demonstratives or determiners are locative in origin, specifying the distance between the speaker and the nominal in questions. These particular demonstratives are described in \sectref{SubSubSecDeicticDeterminers} below. There is also an anaphoric demonstrative that refers back to some aforementioned entity, which is described in \sectref{SubSubSecAnaphoricDeterminers}. Other determiners in Mandan are demonstratives that are verbal in origin, detailing what physical position a nominal is in at a particular point of reference. These positionals are described in \sectref{SubSubSecPositionals}.

\subsubsection{Deictic determiners}\label{SubSubSecDeicticDeterminers}

While English possesses a two-way distance distinction in its demonstrative system (i.e., \textit{this} versus \textit{that}), Mandan has a four-way distance distinction. These deictic determiners are realized as enclitics, appearing the rightmost element in a noun phrase, even closer to the right edge of the determiner phrase than the definite article =\textit{s}. These deictic determiners can be seen in (\ref{ListOfDeicticDeterminers}) below.

\begin{exe}

\item\label{ListOfDeicticDeterminers} List of deictic determiners

\begin{xlist}

\item\label{ListOfDeicticDeterminers1} =\textit{re}: proximal determiner (nearest speaker)

\item\label{ListOfDeicticDeterminers2} =\textit{roo}: medial determiner (near to speaker and addressee)

\item\label{ListOfDeicticDeterminers3} =\textit{oo}: medial determiner (nearer to addressee than to speaker)

\item\label{ListOfDeicticDeterminers4} =\textit{ee}: distal determiner (far from both speaker and addressee)

\end{xlist}

\end{exe}

Historically, all of these elements are derived from Proto-Siouan determiners or deictic particles. The proximal determiner =\textit{re} is a reflex of PSi *re `this, here, now.' The medial =\textit{oo} likewise is a reflex of the Proto-Siouan general deictic particle *oo, which is also the origin of the locative or inessive preverb \textit{o-} in Mandan, as described previously in \sectref{SubParaLocative}. The other medial =\textit{roo} appears to be a portmanteau of =\textit{re} and =\textit{oo}. The distal =\textit{ee} is a reflex of PSi *ʔee, a demonstrative or pronoun meaning `that' or `the aforesaid.'

In the corpus, (\ref{ListOfDeicticDeterminers1}) is typically rendered into English as `this' or `here' while both (\ref{ListOfDeicticDeterminers2}) and (\ref{ListOfDeicticDeterminers3}) are translated as `that' or `there' by Mandan speakers. We can see examples of these deictic determiners used in context in (\ref{ExamplesOfDeicticDeterminers}) below.

\begin{exe}

\item\label{ExamplesOfDeicticDeterminers} Examples of deictic determiners

\begin{xlist}

\item\label{ExamplesOfDeicticDeterminers1} \glll ų́'sh ká'ni máaka't, roką́ąkaxi'he\textbf{re}na.\\
   ų'sh ka'=rį wąąkE=ą't rokąąka\#xi'h=\textbf{re}=rą\\
    \textnormal{thus} \textnormal{possess}=ss \textnormal{lie}.aux=hyp \textnormal{old.woman}\#\textnormal{be.old}=\textbf{dem.prox}=top\\
    \glt `she had him that way and would keep on living, this old lady did' \citep[88]{hollow1973a}
    
\item\label{ExamplesOfDeicticDeterminers2} \glll hiró, manáre ó'ikaniire ~ ~ ~ ~ ~ ~ ~ ~ ~ ~  réehtiki...\\
    hiro wrą=\textbf{re} o-i-ka-rįį=E ~ ~ ~ ~ ~ ~ ~ ~ ~ ~ rEEh=ti=ki\\
    \textnormal{then} \textnormal{tree}=\textbf{dem.prox} pv.irr-pv.dir-ins.frce-\textnormal{grow}=sv ~ ~ ~ ~ ~ ~ ~ ~ ~ ~ \textnormal{go.there}=pot=cond\\
    \glt `then, this tree, whenever he went to where it was growing...' \citep[99]{hollow1973a}
    
\item\label{ExamplesOfDeicticDeterminers3} \glll súkere ishák wakirútoomako'sh\\
    suk=\textbf{re} ishak wa-k-rut=oowąk=o'sh\\
    \textnormal{child}=\textbf{dem.prox} pro unsp-suus-\textnormal{eat}=narr=ind.m\\
    \glt `the boy ate everything up himself' \citep[189]{trechter2012}
    
\item\label{ExamplesOfDeicticDeterminers4} \glll kohų́ų\textbf{roo}na ~ ~ ~ ~ ~ ~ ~ ~ ~ ~ ~ ~ ~ ~ ~ ~ ~ ~ ~ ~  íkaniktakta\\
    ko-hųų=\textbf{roo}=rą ~ ~ ~ ~ ~ ~ ~ ~ ~ ~ ~ ~ ~ ~ ~ ~ ~ ~ ~ ~ i-ka-rįkta$\sim$kta\\
    3poss.pers-\textnormal{mother}=\textbf{dem.mid}=top ~ ~ ~ ~ ~ ~ ~ ~ ~ ~ ~ ~ ~ ~ ~ ~ ~ ~ ~ ~ pv.ins-pv.frce-\textnormal{think.about.someone}$\sim$aug\\
    \glt `that mother of his is always thinking about him' \citep[108]{hollow1973a}
    
\item\label{ExamplesOfDeicticDeterminers5} \glll ní'maare íkų'hąą ~ ~ ~ ~ ~ ~ ~ ~ ~ ~ ~ ~ ~ ~ ~ ~ ~ ~ tákraharaani nitáxaraxe\textbf{roo} ~ ~ manúuxikpa {r}á'kisekto'sh\\
	r'-iwąą=E i-kų'=hąą ~ ~ ~ ~ ~ ~ ~ ~ ~ ~ ~ ~ ~ ~ ~ ~ ~ ~ tak\#ra-hrE=rį rį-ta-xrax=\textbf{roo} ~ ~  wa-rųų\#xik\#pa {r'}-aaki\#isek=kt=o'sh\\
	2poss-\textnormal{body}=sv pv.dir-\textnormal{be.all.over}=loc ~ ~ ~ ~ ~ ~ ~ ~ ~ ~ ~ ~ ~ ~ ~ ~ ~ ~  \textnormal{be.painted.with.white.clay}\#2a-caus=ss 2poss-al-\textnormal{chest}=\textbf{dem.mid} ~ ~ unsp-\textnormal{be.fog}\#\textnormal{be.bad}\#\textnormal{head} {2a}-\textnormal{be.above}\#pv.ins-\textnormal{make}=pot=ind.m\\
	\glt `you should paint your body all over with white clay and paint a skull on your chest' \citep[98]{hollow1973b}

\item\label{ExamplesOfDeicticDeterminers6} \glll manákootkis\textbf{oo}\\
    wrą\#kootki=s=\textbf{oo}\\
    \textnormal{tree}\#\textnormal{be.horizontal.to}=def=\textbf{dem.mid}\\
    \glt `at the cross timber there' \citep[25]{kennard1936}
    
\item\label{ExamplesOfDeicticDeterminers7} \glll kapéhkanash\textbf{oo} ráahini mikték...\\
    ka-peh=ka=rąsh=\textbf{oo} rEEh=rį wįkte=ak\\
    agt-\textnormal{yell}=hab=att=\textbf{dem.mid} \textnormal{go.there}=ss \textnormal{walk.trail}=ds\\
    \glt `the town crier went walking along the trail...' \citep[31]{trechter2012}
    
    
    

\end{xlist}

\end{exe}

\subsubsection{Anaphoric determiners}\label{SubSubSecAnaphoricDeterminers}

In addition to the deictic determiners, Mandan possesses an anaphoric determiner \textit{ą́'t}, which functions as a kind of pronominal that relies on some antecedent to receive some kind of interpretation. \citet[61]{hollow1970} refers to this determiner as meaning `that one (furthest from the speaker)', but \textit{ą́'t} appears in non-distal contexts as well in narratives within the corpus. \citet[42]{mixco1997a}
groups this determiner with the deitic determiners but simply glosses \textit{ą́'t} as `that'. \citet[61]{hollow1970} glosses this item as `that one (furthest from the speaker),'
however \citeauthor{hollow1970} gives examples of this item with non-distal morphology, as we see below. In (\ref{ExNonDistalAt}), we see the anaphoric determiner \textit{ą't} combined with the medial deictic \textit{roo}, which is a blend of the proximal \textit{re} and true medial \textit{oo}.


\begin{exe}

\item\label{ExNonDistalAt} Non-distal \textit{ą́'teroo}

\glll ą́'teroo\\
    ą't=roo\\
    dem.anap=dem.mid\\
    \glt `that one' \citep[61]{hollow1970}

\end{exe}

We can therefore surmise that the \textit{ą't} does not have any innate locative deictic semantics, but rather refers back to an argument previously brought up in the discourse, i.e., personal referential deixis. This determiner appears to have evolved from the Proto-Siouan determiner *ʔa combined with the indefinite pronominal *tą. The vowel in *ʔa assimilated the nasality of the following syllable, after which, the final vowel syncopated and the glottal stop metathesized, resulting in Proto-Siouan *ʔa+tą > Mandan \textit{ą́'t}.\footnote{Metathesis of *ʔ with the following vowel in Proto-Siouan is a regular sound change in Mandan and Missouri Valley Siouan, e.g., *kʔu $\sim$ *kʔų `give' > Mandan \textit{kų́'}, Hidatsa \textit{gú'}, and Crow \textit{kuú} (< Proto-Missouri Valley *kuʔ).}

The anaphoric determiner can appear as a true determiner, i.e., encliticized onto a nominal element. In these situations, the anaphoric determiner has the effect of reinforcing the entity to which the speaker is referencing, akin to the use of the English adjective `very' before a noun, e.g., `the very person I was looking for.' The \textit{ą't} can also appear as an independent word with its own primary stress. In both cases, \textit{ą't} may bear additional deictics, such as those outlined above in \sectref{SubSubSecDeicticDeterminers}. We can see examples of both encliticized and free anaphoric determiners below. There is no restriction on the semantic role to which the anaphoric determiner can refer, i.e., the anaphoric determiner can refer to subjects, direct objects, locations, etc.

We can see in (\ref{ExAnaphoricDets1}) below that each anaphoric determiner refers back to the younger brother who is brought up in the previous sentence. In this case, each \textit{ą't} plays the role of the subject of the sentence. The entity in question is the subject in both sentences. This example likewise shows both a free and encliticized version of \textit{ą't}, both of which are coindexed with the same entity: the younger brother.

In (\ref{ExAnaphoricDets2}), we can see that the semantic roles of the conindexed entity differ. The cultural figure \textit{Kinúma'kshi} tricked the porcupine out of sharing in a buffalo carcass, so the porcupine is complaining to other animals that \textit{Kinúma'kshi} has wronged him. The first clause states that he, \textit{Kinúma'kshi}, is not good, and the second clause references that \textit{Kinúma'kshi} will end up eating all of the buffalo carcass. In this particular example, the \textit{ą́'t} is referring back to the carcass, not to \textit{Kinúma'kshi}.

\begin{exe}

\item\label{ExAnaphoricDets} Examples of the anaphoric determiner

\begin{xlist}

\item\label{ExAnaphoricDets1} \glll Mishų́ųka nimáaka ~ ~ ~ ~ ~ ~ ~ ~ ~ ~ máake'sh.  \textbf{Ą́'t}, mí'r\textbf{ą't}eena, ~ ~ ~  ``Páaxu Xtáni'' ~ ~ ~ ~ éeniheero'sh.\\
    wį-shųųka rį-wąąkE=$\varnothing$ ~ ~ ~ ~ ~ ~ ~ ~ ~ ~  wąąkE=o'sh \textbf{ą't} wį'=\textbf{ą't}=ee=rą ~ ~ ~ paaxu xtE=rį ~ ~ ~ ~  ee-rį-hEE=o'sh\\
    1poss-\textnormal{man's.younger.brother} 2s-\textnormal{be.lying}.aux=cont ~ ~ ~ ~ ~ ~ ~ ~ ~ ~ \textnormal{be.lying}.aux=ind.m \textbf{dem.anap} \textnormal{rock}=\textbf{dem.anap}=dem.dist=top ~ ~ ~ \textnormal{nose} \textnormal{be.big}=ss ~ ~ ~ ~ pv-2s-\textnormal{say}=ind.m\\
    \glt `My younger brother was saying bad things about you. That one, that very rock, he called you ``Big Nose{"}.' \citep[18]{hollow1973a}
    
\item\label{ExAnaphoricDets2} \glll Wáashinashso'nik, ée, \textbf{ą́'t} rústo'sh.\\
    waa-shi=rąsh=so'rįk ee \textbf{ą't} rut=t=o'sh\\
    neg-\textnormal{be.good}=att=dub dem.dist \textbf{dem.anap} \textnormal{eat}=pot=ind.m\\
    \glt Since [that one] is no good, him there, he will eat that one.' \citep[43]{hollow1973a}

\item\label{ExAnaphoricDets3} \glll Háki, nitúuminike áawereehki, ~ ~ ~ ~ ~  \textbf{ą́'t}eena ísekto'sh.\\
    ha=ki rį-tuuwįrįk=E aa-we-rEEh=ki ~ ~ ~ ~ ~ \textbf{ą't}=ee=rą i-sek=t=o'sh\\
    prov=cond 2poss-\textnormal{father's.sister}=sv pv.tr-1a-\textnormal{go.there}=cond ~ ~ ~ ~ ~ \textbf{dem.anap}=dem.dist=top pv.ins-\textnormal{do}=pot=ind.m\\
    \glt `So, if I take him to your aunt, that one should do it.' \citep[57]{hollow1973a}
    
\item\label{ExAnaphoricDets4} \glll Mí'nake warúwihto'sh. Waráahini \textbf{ą́'t}eetaa warúwihini ~ ~ ~ ~ ~ ~ ~ ~ ~ ~ wakúhki, nukí'niinisto'sh.\\
    wį'\#rąk=E wa-ru-wih=t=o'sh wa-rEEh=rį \textbf{ą't}=ee=taa wa-ru-wih=rį ~ ~ ~ ~ ~ ~ ~ ~ ~ ~ wa-kuh=ki rų-ki'\#rįį=rįt=t=o'sh\\
    \textnormal{stone}\#pos.stnd=sv 1a-ins.hand-\textnormal{display}=pot=ind.m 1a-\textnormal{go.there}=ss \textbf{dem.anap}=dem.dist=loc 1a-ins.hand-\textnormal{display}=ss ~ ~ ~ ~ ~ ~ ~ ~ ~ ~ 1a-\textnormal{come.back.here}=cond 1a.pl-\textnormal{haul}\#\textnormal{shoot}=2pl=ind.m\\
\glt `I will put up a monument. I will go and put it up over there and when I come back, we will race' \citep[39]{hollow1973a}

\end{xlist}

\end{exe}


Anaphoric demonstratives are able to be treated as topicalized elements, as we see in (\ref{ExAnaphoricDets3}). In the initial clause, the narrator is describing a situation where her younger brother is to be brought to his aunt. The same aunt is the one being referenced by the anaphoric demonstrative in the second clause. In this scenario, the destination of the first clause and the subject of the second clause are coindexed. The \textit{ą't} is topicalized here, reinforcing the fact that it is indeed the aunt who will be the one to do the requested action and not anyone else.

In addition to referring to specific entities referenced or alluded to in the discourse, the anaphoric determiner can also be used to reference spatial or temporal deixis. We see this above in (\ref{ExAnaphoricDets4}), where \textit{Kinúmakshi} challenges the buffalo to a race. First, \textit{Kinúmakshi} states that he will erect a monument outside, and then once the monument is up over there, they will race. In this context, the monument is the direct object of the first sentence, and the \textit{ą't} is the location of the activity in the following clause. Not only does this anaphoric demonstrative bear the distal demonstrative =\textit{ee}, but it also bears the locative postposition =\textit{taa}.

The anaphoric demonstrative has no unique marking for plural referents. As we see in (\ref{ExAnaphoricDemPl1}), each instance of \textit{ą't} refers to juneberries. In this example, we see multiple references to how the Mandan people traditionally prize juneberries and use them in other staples. Plurality is not marked on the \textit{ą't}, as we see in (\ref{ExAnaphoricDemPl1}) and (\ref{ExAnaphoricDemPl2}), where the anaphoric demonstrative is coindexed with a plural referent. In (\ref{ExAnaphoricDemPl2}), we see an overt quantifier with the referent \textit{ráse ínupshashka} `both of his names', which is coindexed with the anaphoric determiner \textit{ą't} without any other morphology or periphrastic information to indicate plurality even when the accompanying stative verb \textit{túkere'sh} `there are some' bears the third person plural marker =\textit{kere}.\footnote{In this case, the verb \textit{tú} is used to express possession of an inalienable noun, \textit{ráse} `name(s).'} In this example, the =\textit{kere} is the only overt morphological indicator of plurality; \textit{ą't} remains morphologically identical for both singular and plural referents.


\begin{exe}

\item\label{ExAnaphoricDemPl} Plurality and the anaphoric demonstrative

\begin{xlist}

\item\label{ExAnaphoricDemPl1} \glll Hiré, rá'skamak, wáawahe'sh: manápusheke. ~ ~ ~ ~ Nuwáahere koshí, \textbf{ą́'t} ó'ro'sh. ~ ~ ~ ~ ~  Wíipe ríisehki, ríirwerekini ~ ~ ~ ~ ~ ~ ~  órukuhe róokiharani ~ ~ ~ ~ ~ ~ ~ ~ ~ ~ ~ ~ ~ ~ ~ mapéshot ríiruhįhinik nurútka'sh.  Wóorut ~ ~ ~ ~ ~ shixté'sh, \textbf{ą́'t}, ~ ~ ~ ~ ~ ~ ~ ~ ~ ~ manápushek\textbf{ą't}\\
    hire ra'ska\#wąk waa-wa-hE=o'sh wrą\#pushek=E ~ ~ ~ ~ rų-waa-hrE ko-shi \textbf{ą't} o'=o'sh ~ ~ ~ ~ ~ wiipe rV-i-sek=ki rV-i-ru-wrek=rį ~ ~ ~ ~ ~  ~ ~  o-rukuhe rV-o-ki\#hrE=rį ~ ~ ~ ~ ~ ~ ~ ~ ~ ~ ~ ~ ~ ~ ~ wąpe\#shot rV-i-ru-hįh=rįk rų-rut=ka=o'sh wV-o-rut ~ ~ ~ ~ ~ shi-xtE=o'sh \textbf{ą't} ~ ~ ~ ~ ~ ~ ~ ~ ~ ~ wrą\#pushek=\textbf{ą't}\\
    \textnormal{now} \textnormal{summer}\#pos.lie part-1a-\textnormal{see}=ind.m \textnormal{wood}\#\textnormal{juneberries}=sv ~ ~ ~ ~ 1pl.poss-nom-caus rel-\textnormal{be.good} \textbf{dem.anap} \textnormal{be}=ind.m ~ ~ ~ ~ ~ \textnormal{cornball}  1a.pl-pv.ins-\textnormal{make}=cond 1a.pl-pv.ins-ins.hand-\textnormal{mix}=ss ~ ~ ~ ~ ~ ~ ~ pv.irr-\textnormal{be.by.self} 1a.pl-pv.loc-\textnormal{be.cooked}\#caus=ss ~ ~ ~ ~ ~ ~ ~ ~ ~ ~ ~ ~ ~ ~ ~ \textnormal{sunflower.plant}\#\textnormal{white} 1a.pl-pv.ins-\textnormal{mix.up}=itr 1a.pl-\textnormal{eat}=hab=ind.m nom-pv.irr-\textnormal{eat} ~ ~ ~ ~ ~  \textnormal{be.good}-aug=ind.m \textbf{dem.anap} ~ ~ ~ ~ ~ ~ ~ ~ ~ ~ \textnormal{wood}\#\textnormal{juneberry}=\textbf{dem.anap}\\
    \glt `Now, this summer, I saw some of them: juneberries. Our best food, those are it. When we make cornballs, we mix [juneberries] and cook them by themselves and we always eat them by mixing them up with flour. They are real good food, those are, those juneberries.' \citep[52]{hollow1973a}

\item\label{ExAnaphoricDemPl2} \glll Mishų́ųkak, koník ~ ~ ~ ~ ~ ~ ~ ~ ~ ~ ~ ~ ~ koxamáhere, ráse ínupshashka, ~ ~ ~ ~ ~  \textbf{ą́'t}, ráse ~ ~ ~ ~ ~ túkere'sh.\\
    wį-shųųka=ak ko-rįk ~ ~ ~ ~ ~ ~ ~ ~ ~  ~ ~ ~ ~   ko-xwąh=re ras=E i-rųp-sha-shka ~ ~ ~ ~ ~  ą't ras=E ~ ~ ~ ~ ~ tu=krE=o'sh\\
    1poss-\textnormal{man's.younger.brother}=ds 3poss.pers-\textnormal{son} ~ ~ ~ ~ ~ ~ ~ ~ ~ ~ ~ ~ ~  rel-\textnormal{be.small}=dem.prox \textnormal{name}=sv pv.num-\textnormal{two}-coll-ints.coll ~ ~ ~ ~ ~  \textbf{dem.anap} \textnormal{name}=sv ~ ~ ~  ~ ~ \textnormal{be.some}=3pl=ind.m\\
    \glt `My brother, that youngest son of his, both of his names, those ones, he has the names.' \citep[61]{hollow1973a}
    
\item\label{ExAnaphoricDemPl3} \glll Wóo'ipke ókapxiire ~ ~ ~ ~ ~ ~ ~ ~ ~  koxtémihka rarúshani ~ ~ ~ ~ ~ ~ ~ ~ ~ ~ ~ ~ ~  wóo'ipke, \textbf{ą́'te} \textbf{máakahe}, ~ ~ ~ ~ ~ ~ ~ ~ ~ ~ xamáhkerehara rupáaxini ~ ~ ~ ~ ~ ~ ~ ~ ~ ~ ~ wóo'ipke koxtés ~ ~ ~ ~ ~ ~ ~ ~ ~ ~ ~ ~ ~ ~ ~ ~ ~ ~ óreehraherekto'sh.\\
    wV-o-i-pke o-ka-pxii=E ~ ~ ~ ~ ~ ~ ~ ~ ~ ko-xtE\#wik=ka ra-ru-shE=rį ~ ~ ~ ~ ~ ~ ~ ~ ~ ~ ~ ~ ~  wV-o-i-pke \textbf{ą't=E} \textbf{wąąkahE} ~ ~ ~ ~ ~ ~ ~ ~ ~ ~ xwąh=krE\#hrE=$\varnothing$ ru-paax=rį ~ ~ ~ ~ ~ ~ ~ ~ ~ ~ ~  wV-o-i-pke ko-xtE=s ~ ~ ~ ~ ~ ~ ~ ~ ~ ~ ~ ~ ~ ~ ~ ~ ~ ~ o-rEEh\#ra-hrE=kt=o'sh\\
    nom-pv.irr-pv.ins-\textnormal{smell} pv.irr-ins.frce-\textnormal{be.wide}=sv ~ ~ ~ ~ ~ ~ ~ ~ ~ rel-\textnormal{be.big}\#\textnormal{be.none}=hab 2a-ins.hand-\textnormal{hold}=ss ~ ~ ~ ~ ~ ~ ~ ~ ~ ~ ~ ~ ~  nom-pv.irr-pv.ins-\textnormal{smell} \textbf{dem.anap=sv} {\bfseries\textnormal{those}} ~ ~ ~ ~ ~ ~ ~ ~ ~ ~ \textnormal{be.small}=3pl\#caus=cont ins.hand-\textnormal{be.broken}=ss ~ ~ ~ ~ ~ ~ ~ ~ ~ ~ ~  nom-pv.irr-pv.ins-\textnormal{smell} rel-\textnormal{be.big}=def  ~ ~ ~ ~ ~ ~ ~ ~ ~ ~ ~ ~ ~ ~ ~ ~ ~ ~ pv.loc-\textnormal{go.there}\#2a-caus=pot=ind.m\\
    \glt `You should take the biggest piece of sliced dried meat and put it, the biggest dried meat, with those ones, the smaller broken pieces.' \citep[223]{hollow1973b}

\end{xlist}

\end{exe}

One way that plurality can be indicated with the anaphoric demonstrative is through periphrastic information. In (\ref{ExAnaphoricDemPl3}) above, we see that the anaphoric demonstrative \textit{ą́'te} is referring back to pieces of dried meat from the previous clause. In this case, the anaphoric determiner bears a stem vowel, indicating that there is a prosodic break after it. Afterwards, the plural determiner \textit{máakahe} `those' appears, reinforcing the fact that the speaker is referring to those pieces of sliced meat. This periphrastic plural marking is completely optional, as the overwhelming majority of instances of \textit{ą't} referring to plural referents have no other indicators to cue the listener that the anaphoric demonstrative is coindexed with a singular or plural referent. This information is inferred from context.

\subsubsection{Positionals}\label{SubSubSecPositionals}

A very productive class of demonstratives in Mandan is the positionals. Positionals are  formatives that have orgins in the Proto-Siouan verbs `sit', `stand', and `lie'. \citet{rankin1977,rankin2004} identifies positionals as taking various paths of grammaticalization in different Siouan languages, but within Mandan, a common function of positionals is as a determiner that conveys information about the position a particular entity is in, as well as bestowing definite semantics upon the nominal in question. These positional determiners can be seen in (\ref{ListOfPositionals}) below.

\begin{exe}
    \item\label{ListOfPositionals} List of positional determiners

    \begin{xlist}
        \item\label{ListOfPositionalsA} \textit{hąk}: `standing' positional determiner
        \item\label{ListOfPositionalsB} \textit{mak}: `lying' positional determiner
        \item\label{ListOfPositionalsC} \textit{nak}: `sitting' positional determiner
        \item\label{ListOfPositionalsD} \textit{máakah}: plural determiner
    \end{xlist}
\end{exe}

We can see examples of the positional determiners in (\ref{ExamplesofPositionalDeterminers}) in the data below. The relevant determiner appears in bold. Note that the singular positional determiners are orthographically represented as being part of the nominal element over which they have semantic scope. This orthographic treatment of \textit{hąk}, \textit{mak}, and \textit{nak} follows \citet{hollow1970,hollow1973a,hollow1973b} and \citet{hollow1976}, and likely stems from the fact that these determiners do not have a primary stress of their own. Their phonoligical behavior is more like compounding than encliticization due to the fact that any consonant clusters created by the addition of these determiners do not result in the production of excrescent Dorsey's Law vowels. This fact is clear in the data in (\ref{ExamplesofPositionalDeterminers}) below.

\begin{exe}
    \item\label{ExamplesofPositionalDeterminers} Examples of positional determiners with lexical words

    \begin{xlist}
        \item\label{ExamplesofPositionalDeterminersA} \glll hą́pe ą́ąwe wáa'onatkoxik\textbf{hąk} ~ ~ ~ ~ ~ ~ ~ ~ ~ ~  waká'ni wahą́ąkaha'sh\\
        hąp(E) ąąwe waa-o-rątka\#o-xik\#\textbf{hąk} ~ ~ ~ ~ ~ ~ ~ ~ ~ ~  wa-ka'=rį wa-hąąkE=ka=o'sh\\
        \textnormal{day} \textnormal{all} nom-pv.irr-\textnormal{heart}\#pv.irr-\textnormal{be.bad}\#\textbf{pos.stnd} ~ ~ ~ ~ ~ ~ ~ ~ ~ ~ 1a-\textnormal{possess}=ss 1a-\textnormal{be.standing}.aux=hab=ind.m\\
        \glt `I always have this bad feeling [for them] every day.' \citep[56]{hollow1973a}
        

        \item\label{ExamplesofPositionalDeterminersB} \glll tashká'eshkák súkmiih\textbf{nak} húuro'na?\\
        tashka-eshka=ak suk\#wįįh\#\textbf{rąk} huu=o'rą\\
        \textnormal{how}-smlt=ds \textnormal{child}\#\textnormal{woman}\#\textbf{pos.sit} \textnormal{come.here}=int.f\\
        \glt `Why has this young woman come?' \citep[106]{hollow1973b}

        \item\label{ExamplesofPositionalDeterminersC} \glll máa'ąk\textbf{mak} rá'kxipwahere'sh\\
        waa'ąk\#\textbf{wąk} ra'-kxip\#wa-hrE=o'sh\\
        \textnormal{land}\#\textbf{pos.lie} ins.heat-\textnormal{shrivel}\#1a-caus=ind.m\\
        \glt `I made this land shrivel up.' \citep[217]{hollow1973a}

        \item\label{ExamplesofPositionalDeterminersD} \glll numá'kaaki ítąhąą \textbf{máakah} ~ ~ ~ ~ ~ ~ ~ ~ ~ ~ ~ ~ ~ ~ ~ ~  íhehkereki ríikasharatkere'sh\\
        ruwą'k-aaki i-tąą=hąą \textbf{wąąkah} ~ ~ ~ ~ ~ ~ ~ ~ ~ ~ ~ ~ ~ ~ ~ ~  i-hek=krE=ki rV-i-ka-shrat=krE=o'sh\\
        \textnormal{person}-coll pv.ins-\textnormal{be.different}=ins \textnormal{\bfseries these} ~ ~ ~ ~ ~ ~ ~ ~ ~ ~ ~ ~ ~ ~ ~ ~  pv.ins-\textnormal{know}=3pl=cond 1a.pl-pv.ins-ins.frce-\textnormal{\bfseries be.thick}=3pl=ind.m\\
        \glt `if these different people knew, they would all gather around' \citep[177]{hollow1973a}

    \end{xlist}
\end{exe}

We see that \textit{máakah} is not treated as part of the prosodic word in the same way that \textit{hąk}, \textit{mak}, and \textit{nak} are. This formative is typically given a primary stress, unlike its singular counterparts. Likewise, we see that any positional semantics are neutralized in plural contexts, and the habitual auxiliary \textit{máakah} `be lying' is used exclusively. It is ungrammatical to use the other habitual auxiliaries \textit{hą́ąkah} `be standing' or \textit{náakah} `be sitting' to indicate plurality \citep{trechter2013}.

Within the corpus, the proximal deictic determiner \textit{re} and the anaphoric determiner \textit{ą't} appear frequently followed by a positional determiner. These constructions have a quasi-pronominal functionality, serving to point out something immediately in view of the interlocutors with the \textit{re} constructions or something that was previously mentioned or no longer in view with the \textit{ą't} constructions. All the examples in (\ref{PositionalsWithDeterminers}) below come from \citet[28]{kennard1936}.

\begin{exe}
    \item\label{PositionalsWithDeterminers} Examples of positional determiners with other determiners

    \begin{xlist}

        \item\label{PositionalsWithDeterminersA} \glll réhąk\\
            re=hąk\\
            dem.prox=pos.stnd\\
            \glt `this one, standing'
        \item\label{PositionalsWithDeterminersB} \glll rémak\\
            re=wąk\\
            dem.prox=pos.lie\\
            \glt `this one, lying'
        \item\label{PositionalsWithDeterminersC} \glll rénak\\
            re=rąk\\
            dem.prox=pos.sit\\
            \glt `this one, sitting'
        \item\label{PositionalsWithDeterminersD} \glll ą́'tahąk\\
        ą't=hąk\\
        dem.anap=pos.stnd\\
        \glt `that one, standing'
        \item\label{PositionalsWithDeterminersE} \glll ą́'tamak\\
        ą't=wąk\\
        dem.anap=pos.lie\\
        \glt `that one, lying'
        \item\label{PositionalsWithDeterminersF} \glll ą́'tanak\\
        ą't=rąk\\
        dem.anap=pos.sit\\
        \glt `that one, sitting'
        
    \end{xlist}
    
\end{exe}

The plural positional determiner \textit{máakah} also occurs with \textit{re} and \textit{ą't}. These combined determiners are used pronominally, where such constructions are used instead of an aforementioned nominal. We can see examples of such constructions in (\ref{PluralPositionalDeterminers}) below.

\begin{exe}
    \item\label{PluralPositionalDeterminers} Examples of \textit{máakah} with other determiners

    \begin{xlist}
        \item\label{PluralPositionalDeterminersA} \glll \textbf{Rémaakahe}, ishák mashíkerekto'sh\\
        \textbf{re=wąąkah=E} ishak wąshi=krE=kt=o'sh\\
        \textbf{dem.prox=\textnormal{\bfseries these}=sv} pro \textnormal{white.person}=3l=pot=ind.m\\
        \glt `These ones, they will be white people.' \citep[13]{hollow1973a}

        \item\label{PluralPositionalDeterminersB} \glll Wóo'ipke, \textbf{ą́'tamaakahe}, ~ ~ ~ ~ ~ ~ ~ ~ ~ ~ ~ ~ xamáhkerehara rupáxini ~ ~ ~ ~ ~ ~ ~ ~ ~ ~ ~ wóo'ipke koxtés ~ ~ ~ ~ ~ ~ ~ ~ ~ ~ ~ ~ ~ ~ ~ ~ ~ óreehraherekto'sh\\
        wV-o-i-pke \textbf{ą't=wąąkah=E} ~ ~ ~ ~ ~ ~ ~ ~ ~ ~ ~ ~ xwąh=krE=hrE=$\varnothing$ ru-pax=rį ~ ~ ~ ~ ~ ~ ~ ~ ~ ~ ~ wV-o-i-pke ko-xtE=s ~ ~ ~ ~ ~ ~ ~ ~ ~ ~ ~ ~ ~ ~ ~ ~ ~ o-rEEh\#ra-hrE=kt=o'sh\\
        unsp-pv.irr-pv.ins-\textnormal{smell} \textbf{dem.anap=\textnormal{\bfseries these}=sv} ~ ~ ~ ~ ~ ~ ~ ~ ~ ~ ~ ~ \textnormal{be.small}=3pl=caus=cont ins.hand-\textnormal{be.broken}=ss ~ ~ ~ ~ ~ ~ ~ ~ ~ ~ ~ unsp-pv.irr-pv.ins-\textnormal{smell} rel-\textnormal{be.big}=def ~ ~ ~ ~ ~ ~ ~ ~ ~ ~ ~ ~ ~ ~ ~ ~ ~ pv.loc-\textnormal{go.there}\#2a-caus=pot=ind.m\\
        \glt `You should put the dry meat with them, those ones, the smaller broken pieces and the biggest [pieces of] dry meat.' \citep[223]{hollow1973b}
    \end{xlist}
\end{exe}

Most instances of \textit{re} and \textit{ą't} throughout the corpus appear without an accompanying positional determiner. The overwhelming majority of instances of positional determiners in the corpus is found at the right edge of a determiner phrase. These positional determiners always convey definite semantics upon the nominal over which they take semantic scope, and nominals bearing a positional determiner cannot also bear the definite article =\textit{s}.

These double determiner constructions are never encliticized onto an overt nominal, so their distribute is squarely pronominal in nature. There are no instances of positional determiners with other deictic determiners other than \textit{re} in the corpus, and there are no remaining L1 speakers to ask whether other combinations are possible.

The semantics of plurality are sometimes not be marked overtly on positional determiners. There are numerous instances of a morphologically singular positional determiner being used with obvious semantic plurality. In the data below in (\ref{UnexpectedPositionalDeterminers}), the plurality of the items marked with a positional determiner is established in the discourse. In (\ref{UnexpectedPositionalDeterminersA}), the speakers are exhorting their grandmother to go and eat the birds they have slaughtered. It has been established previously that there were multiple birds that were killed, so the plurality of the element in question is clear. However, the singular standing positional \textit{hąk} is used instead of the plural \textit{máakah}. In (\ref{UnexpectedPositionalDeterminersB}), we not only have the context of plurality from earlier in the discourse, but we have overt plural marking earlier in this very sentence. We see \textit{máakahe} `these' used to refer to the stars in question, and then the speaker harkens back to said stars with \textit{ą́'tamak}, with the anaphoric demonstrative and the singular lying positional.

\begin{exe}
    \item\label{UnexpectedPositionalDeterminers} Examples of unexpected number marking in positional determiners

    \begin{xlist}
        \item\label{UnexpectedPositionalDeterminersA} \glll máareksuk\textbf{hąk} rútini ráahana!\\
        wąąrek\#suk\#\textbf{hąk} rut=rį rEEh=rą\\
        \textnormal{bird}\#\textnormal{be.small}\#\textbf{pos.stnd} \textnormal{eat}=ss \textnormal{go.there}=imp.f\\
        \glt `go on and eat these birds!' \citep[148]{hollow1973a}

        \item\label{UnexpectedPositionalDeterminersB} \glll Karóotiki, xkék máakahe, wáateehą núuniha, ~ ~ ~ ~ ą́'ta\textbf{mak} ó'ro'sh\\
        ka=ooti=ki xkek wąąkah=E waa-teehą ruurįh=E=$\varnothing$ ~ ~ ~ ~ ą't\#\textbf{wąk} o'=o'sh\\
        prov=evid=cond \textnormal{star} \textnormal{these} nom-\textnormal{be.far} \textnormal{be}.pl=sv=cont ~ ~ ~ ~ dem.anap=\textbf{pos.lie} \textnormal{be}=ind.m\\
        \glt `And then, these stars, they are there for a long time, those ones are.' \citep[206]{hollow1973b}
    \end{xlist}
\end{exe}

The plurality of items like the ones we see above in (\ref{UnexpectedPositionalDeterminers}) is marked in the English translation. We have two possibilities to explain this discrepancy. One possibility is that Mandan speakers are able to forgo marking overt plurality with positional determiners if the positional information is more pertinent to the context. Plurality in these cases can therefore be inferred through the previous discourse. The other possibility is that the English translation is not entirely reflecting the nuance of what is being expressed in Mandan. For example, while there were multiple birds slaughtered in (\ref{UnexpectedPositionalDeterminersA}) and the speakers were alluding to these birds, perhaps the collective group of these birds is being treated more like a mass noun in this context than as count nouns. The same could be said for all the stars being referred to in (\ref{UnexpectedPositionalDeterminersB}), where \textit{ą́'tamak} is not truly describing the exact same thing as \textit{xkék máakahe}. It is unclear which possibility is most likely, given the fact that there are no longer any L1 speakers remaining who can provide judgments or insights into constructions like these.

Throughout the corpus, positional determiners are used after nominals as a kind of noun classifier. Some Mandan speakers report that certain nouns are intrinsically associated with \textit{hąk}, \textit{mak}, or \textit{nak}. This treatment of positional verbs as classifiers is common throughout the Siouan language family \citep[205]{rankin2004}. In the Dhegihan branch of Mississippi Valley Siouan, the path of grammaticalization has yielded a far more articulated system of classifiers that originate in these positional verbs. The shape of the nominal is often a clue to which positional classifier is required in Dhegihan languages. In Mandan, the physical shape of an item does not inherently dictate what positional determiner is required, though there are some strong tendencies. For example, flat, scattered, or distributed nominals are often used with \textit{mak}.
Thin or long nominals are often seen with \textit{hąk}. Short or round nominals usually take \textit{nak}. However, there are exceptions to these tendencies. We can find the same nominal used with differing positional determiners.

In (\ref{PositionalDeterminersNoSemantics}) below, we see three instances of the noun \textit{máa'ąk} `land, earth, ground, hill.' Each instance bears a different positional determiner. The choice to use a different positional determiner in each example below may arise from the narrator adding context to the shape of the land in the present context. At this point, the lack of L1 speakers renders this interpretation purely hypothetical. What we can determine from the triplet below is that Mandan has not developed an articulated noun class system in the same way that the Dhegihan languages have.  

\begin{exe}
    \item\label{PositionalDeterminersNoSemantics} Triplet with possible semantic differences

    \begin{xlist}
        \item\label{PositionalDeterminersNoSemanticsA} \glll \textbf{máa'ąkhąk}ero kisúkini éerehak ~ ~ ~ ~ ~ ~ ~ ~ ~ ~ ~ ókisuke xíkini\\
        \textbf{waa'ąk\#hąk}=ro ki-suk=rį ee-reh=ak ~ ~ ~ ~ ~ ~ ~ ~ ~ ~ ~ o-ki-suk=E xik=rį\\
        \textnormal{\bfseries land}\textbf{\#pos.stnd}=dem.mid vert-\textnormal{exit}=ss pv-\textnormal{think}=ds ~ ~ ~ ~ ~ ~ ~ ~ ~ ~ ~   pv.irr-vert-\textnormal{exit}=sv \textnormal{be.bad}=ss\\
        \glt `it wanted to get out from this ground, but it could not...' \citep[154]{hollow1973a}

        \item\label{PositionalDeterminersNoSemanticsB} \glll \textbf{máa'ąkmak} rá'kxipwahere'sh\\
        \textbf{waa'ąk\#wąk} ra'-kxip\#wa=hrE=o'sh\\
        \textnormal{\bfseries land}\textbf{\#pos.lie} ins.heat-\textnormal{shrivel}\#1a-caus=ind.m\\
        \glt `I made this land shrivel.' \citep[217]{hollow1973a}

        \item\label{PositionalDeterminersNoSemanticsC} \glll \textbf{Máa'ąknak}e íwasįhe, tóop óte ~ ~ ~ kotų́ųte're kosé ishí'sh\\
        \textbf{waa'ąk\#rąk}=E i-wa'-sįh=E toop o-tE ~ ~ ~ ko-tųųtE=o'=re ko-se ishi=o'sh\\
        \textnormal{\bfseries land}\textbf{\#pos.sit}=sv pv.ins-ins.prce-\textnormal{be.strong}=sv \textnormal{four} pv.irr-\textnormal{stand} ~ ~ ~ 3poss.pers-\textnormal{son-in-law}=\textnormal{be}=dem.prox rel-\textnormal{be.red} vis=ind.m\\
        \glt `At the things holding up the land, the four standing things, it is that son-in-law of hers, he must be the red one.' \citep[122]{hollow1973a}

        
    \end{xlist}
\end{exe}

\citet{hollow1976} treat \textit{mak} as the default deictic classifier, introducing vocabulary such as `what is this?' and `this is a...' with the compound form \textit{rémak}. There are some instances of different positional determiners being used with the same nominal to alter the semantics of that nominals. Edwin Benson provides a triplet where there are clear semantic differences between using each positional determiner using \textit{istų́h} `night' as an example \citep[42]{trechter2012}. This triplet appears in (\ref{PositionalDeterminersYesSemantics}) below.

\begin{exe}
    \item\label{PositionalDeterminersYesSemantics} Triplet with clear semantic differences

    \begin{xlist}
        \item\label{PositionalDeterminersYesSemanticsA} \glll istų́hąk\\
        istųh\#hąk\\
        \textnormal{night}\#pos.stnd\\
        \glt `that kind of night'

        \item\label{PositionalDeterminersYesSemanticsB} \glll istų́hmak\\
        istųh\#wąk\\
        \textnormal{night}\#pos.lie\\
        \glt `tonight'

        \item\label{PositionalDeterminersYesSemanticsC} \glll istų́hnak\\
        istųh\#rąk\\
        \textnormal{night}\#pos.sit\\
        \glt `in the night'
    \end{xlist}
    
    
\end{exe}

Given the lack of L1 speakers at the present time with whom we can test how widespread this pattern is, it is not clear if the semantics of each positional determiner carries the same meaning for all nominal constructions relating to temporality. We cannot therefore state that this three-way semantic split is uniform across Mandan temporal constructions. We do see some small degree of consistency with \textit{hą́p} `day', however. There are numerous instances throughout the corpus where the presence of \textit{mak} occurs after \textit{hą́p}, bearing the same semantics as \textit{istų́h} `night' does in (\ref{PositionalDeterminersYesSemanticsB}) above. We can see \textit{hą́p} with \textit{mak} in the example in (\ref{Mihapmak}) below.

\begin{exe}

\item\label{Mihapmak} Use of \textit{mak} to express temporal proximity

    \glll mihą́pmak\\
    wį\#hąp\#wąk\\ 
    \textnormal{orb}\#\textnormal{day}\#pos.lie\\
    \glt `today'

\end{exe}

There are no analogous constructions with \textit{hąk} and \textit{nak} in the corpus, so it is not possible to conclusively state how widespread this three-way semantic split is for other temporal nominal constructions. 


\subsection{Topics}\label{SubSecTopics}

Mandan employs several tactics when it comes to indicating that a nominal construction is a topic that the speaker wishes to emphasize in some way. \citet{kasak2022} outlines morpho-syntactic strategies employed in Mandan, most notably the difference between a topicalized and a focused element.\footnote{More on this aspect of topic marking can be seen in \sectref{Ch5TopicAndFocus}.} The description that follows highlights the morphological marking of topics in Mandan. There are two formatives that are found on nominal contructions in the corpus: the topic marker =\textit{na} and the aforementioned topic marker =\textit{nu}. An explanation of these two topic markers differ from each other follows below.

\subsubsection{Topic marker: =\textit{na}}\label{SubSubSubSecNa}

The topic marker =\textit{na} is cognate with the Proto-Siouan emphatic topic marker *ya $\sim$ *yą.\footnote{Proto-Siouan appears to have had an oral and nasal reflex of the emphatic topic marker, given the fact that reflexes of both vowels are found throughout the language family, e.g., Lakota =\textit{čha} and Tutelo =\textit{ya} have oral vowels for their topic markers, but Mandan =\textit{na}, Biloxi =\textit{yą}, and Assiniboine =\textit{(ȟtį)yą} have nasal vowels.} In Mandan, the topic marker is the last element to appear on a nominal construction, appearing after determiners and deictic markers, as illustrated in the example in (\ref{WhereNaGoes}) below.

\begin{exe}
    \item\label{WhereNaGoes} Example of the placement of =\textit{na} within the nominal complex

    \glll Húurąmi, súksee\textbf{na} áakaksheroomaksįh\\
    huu=awį suk=s=ee=\textbf{rą} aakakshe=oowąk=sįh\\
    \textnormal{come.here}=cont \textnormal{child}=def=dem.dist=\textbf{top} \textnormal{meet}=narr=ints\\
    \glt `the child met her while she was coming' \citep[89]{hollow1973a}
\end{exe}

Throughout the corpus, the topic marker enclitic =\textit{na} indicates nominal elements where the speaker wishes to convey some special salience. The topic marker can indicate a new topic or one that has previously been mentioned in the discourse, as we can see in (\ref{NAkindsoftopics}) below. In (\ref{NAkindsoftopicsA}) below, we see the first sentence of a narrative. The speaker uses =\textit{na} to mark that she is beginning a story, shifting the topic to the story itself. 

We see =\textit{na} used to juxtapose one argument with another in (\ref{NAkindsoftopicsB}) below, where two individuals are arguing and the topic marker indicates that it is Royal Chief, as opposed to Lone Man, who is ahead. This topic marker indicates a contrast between the element marked with =\textit{na} and another element in the discourse. This topic cannot be a shifting topic, as Royal Chief is already the subject of the previous clause, where he spoke about him be the oldest, so there is no change in topichood. The fact that Royal Chief is the salient figure instead of Lone Man is the reason for bestowing a morphological topic marker in this example, signaling one entity to the exclusion of others is the topic at hand.

There are also instances where there are neither shifts in aboutness or contrasts in one topic or another in elements bearing =\textit{na}. In (\ref{NAkindsoftopicsC}) below, a familiar topic appears with =\textit{na}, where the man at the center of the story bears the topic marker even after being introduced in the discourse. The presence of =\textit{na} even after establishing him as the topic serves to remind the listener that this is indeed the same man that has already been mentioned and not someone else.

\begin{exe}

\item\label{NAkindsoftopics} Kinds of topics marked by =\textit{na}

    \begin{xlist}
        \item\label{NAkindsoftopicsA} Shifting topic

        \glll \textbf{Hókeena} Kinúma'kshi íwarooni ~ ~ ~ ~ ~ ~ ~ ~ wakína'ni éewereho'sh\\ 
        \textbf{hok=ee=rą} ki-ruwą'k\#shi i-wa-roo=rį ~ ~ ~ ~ ~ ~ ~ ~ wa-kirą'=rį ee-we-reh=o'sh\\
        \textnormal{\bfseries story}\textbf{=dem.dist=top} mid-\textnormal{man}\#\textnormal{be.good} pv.ins-1a-\textnormal{talk}=ss ~ ~ ~ ~ ~ ~ ~ ~ 1a-\textnormal{tell}=ss pv-1a-\textnormal{want}=ind.m\\
        \glt `I want to tell \textbf{a story} and talk about Royal Chief.' \citep[20]{hollow1973a}

        \item\label{NAkindsoftopicsB} Contrastive topic

        \glll ``Ą́'skak mi'ó'ro'sh, korátoore. Éepe'sh,'' éeheni \textbf{Kinúma'kshiseena} Numá'k Máxanas pahų́hanashoomaks.\\ 
        ą's=ka=ak wį-o'=o'sh ko-ratoo=E ee-pE=o'sh ee=he=rį \textbf{ki-ruwą'k\#shi=s=ee=rą} ruwą'k wąxrą pa-hųh=rąsh=oowąk=s\\ 
        \textnormal{be.this.way}=hab=ds 1s-\textnormal{be}=ind.m rel-\textnormal{be.mature}=sv pv-\textnormal{say}.1a=ind.m pv-\textnormal{say}=ss \textbf{mid-\textnormal{\bfseries man}\#\textnormal{\bfseries be.good}=def=dem.dist=top} \textnormal{man} \textnormal{one} ins.push-\textnormal{get.ahead.of}=att=narr=def\\
        \glt `{``}That is why I am it, older. I said it,'' he said and \textbf{Royal Chief} got ahead of Lone Man.' \citep[9]{hollow1973a}

        \item\label{NAkindsoftopicsC} Familiar topic

        \glll Ų́'staa numá'keena ó'rak, ``Waréeho'sh,'' ~ ~ ~ ~ ~ ~ ~ éeheero'sh.  ``Máahsikųų waréhak, éet ~ ~ ~ ~ míishiihąktaa ~ watewétaa waraahini.'' ~ ~ ~ ~ ~ ~ \textbf{Numá'keena} ó'rak ~ ~ koshų́ųka ~ ki'ų́ųpani, inák numá'k ~ ~ ~ ~ íretaa máxana, ~ ~ ~ kų́'he ki'ų́ųpani, ~ ~ ~ ~ ~ ~ ~ ~ ~ ~ ~ ~ ~ ~ ~ ~ ~ ~ ~ ~ kotámaanukakere máxana. Ą́ąwe ki'órak ~ ~ ~ ~ ~ íkixųųhkereroomako'sh.\\ 
        ų't=taa ruwą'k=ee=rą o'=ak wa-rEEh=o'sh ~ ~ ~ ~ ~ ~ ~ ee-hee=o'sh wąąh\#si\#kųų wa-reh=ak ee=t ~ ~ ~ ~ wįįshii\#hąk=taa ~ wa-t-we=taa wa-rEEh=rį ~ ~ ~ ~ ~ ~ \textbf{ruwą'k=ee=rą} o'=ak ~ ~ ko-shųųka ~ ki-ųųpa=rį irąk ruwą'k ~ ~ ~ ~ i-retaa wąxrą ~ ~ ~ k'-ųh=E ki-ųųpa=rį ~ ~ ~ ~ ~ ~ ~ ~ ~ ~ ~ ~ ~ ~ ~ ~ ~ ~ ~ ~ ko-ta-waarųka=krE wąxrą ąąwe ki-o'=ak ~ ~ ~ ~ ~ i-kixųųh=krE=oowąk=o'sh \\ 
    \textnormal{be.in.past}=loc \textnormal{man}=dem=top \textnormal{be}=ds 1a-\textnormal{go.there}=ind.m ~ ~ ~ ~ ~ ~ ~ pv-\textnormal{say}=ind.m \textnormal{arrow}\#\textnormal{feather}\#\textnormal{trap} 1a-\textnormal{want}=ds dem.dist=loc ~ ~ ~ ~ \textnormal{west}\#pos.stnd=loc ~ unsp-wh-indef=loc 1a-\textnormal{go.there}=ss ~ ~ ~ ~ ~ ~ \textnormal{\bfseries man}\textbf{=dem.dist=top} \textnormal{be}=ds ~ ~ 3poss.pers-\textnormal{man's.younger.brother} ~ mid-\textnormal{with}=ss \textnormal{again} \textnormal{man} ~ ~ ~ ~ pv.ins-\textnormal{be.another} \textnormal{one} ~ ~ ~ 3poss.pers-\textnormal{wife}=sv mid-\textnormal{with}=ss ~ ~ ~ ~ ~ ~ ~ ~ ~ ~ ~ ~ ~ ~ ~ ~ ~ ~ ~ ~ 3poss.pers-al-\textnormal{man's friend}=3pl \textnormal{one} \textnormal{all} mid-\textnormal{be}=ds ~ ~ ~ ~ ~ pv.num-\textnormal{five}=3pl=narr=ind.m\\
        \glt `Long ago, there was a man and he said ``I am going. I want to trap eagles far away in the west, wherever it takes me.'' There was \textbf{the man} and he was with his younger brother, one other man also, he was with his wife and one of their friends. All together, there were five of them.' \citep[237]{trechter2012b}
    \end{xlist}

\end{exe}

In a survey of four Mandan narratives from \citet{hollow1973a}, \citet[586]{wolvengray1991} notes that the overwhelming majority of nominals bearing the topic marker =\textit{na} are active subjects. \citet{kasak2022} corroborates this tendency in Mandan to mark active subjects with the topic marker, elaborating that =\textit{na} can be found at the right edge of noun phrases that play almost any semantic role. Examples of different semantic roles bearing =\textit{na} marking appear in (\ref{NAexamples}) below, where the semantic role referenced in each example is displayed in bold.

\newpage
\begin{exe}

\item\label{NAexamples} Examples of =\textit{na} on nominals with differing roles

\begin{xlist}

\item\label{NAexamplesA} Active subject (agent)

    \glll \textbf{kowóorooreena} máah ~ ~ ~ ~ ~ ~ ~ ~ ~ ~ ~ ~ ~ ~ ~ íseksoomaksįh\\
    \textbf{ko-wooroo=ee=rą} wąąh ~ ~ ~ ~ ~ ~ ~ ~ ~ ~ ~ ~ ~ ~ ~ i-sek=s=oowąk=sįh\\
    \textbf{3poss.pers}-\textnormal{\bfseries husband}=\textbf{def}=\textbf{dem.dist}=\textbf{top} \textnormal{arrow} ~ ~ ~ ~ ~ ~ ~ ~ ~ ~ ~ ~ ~ ~ ~  pv.ins-\textnormal{make}=def=narr=ints\\
    \glt `\textbf{her husband} made an arrow' \citep[86]{hollow1973a}

\item\label{NAexamplesB} Stative subject (experiencer)

    \glll \textbf{súknuma'k} \textbf{shínasheena} ó'roomako'sh\\
        \textbf{\bfseries suk\#ruwą'k} \textbf{shi=rąsh=ee=rą} o'=oowąk=o'sh\\
        \textbf{\textnormal{\bfseries child}\#\textnormal{\bfseries man}} \textbf{\textnormal{\bfseries be.good}=att=dem.dist=top} \textnormal{be}=narr=ind.m\\
        \glt `it was \textbf{a nice young man}' \citep[125]{hollow1973a}

\item\label{NAexamplesC} Direct object (patient)\footnote{This verb `weep for' is not a transitive verb in English, but \textit{íratax} is transitive in Mandan. Another way of interpreting this situations would be `Corn Woman was mourning her child.'}

    \glll {Kóoxą'te Míihs} \textbf{tasúkseena} ~ ~ ~ ~ ~ ~ ~ ~ ~ ~ ~ ~ ~ ~ ~ írataxak\\
    kooxą'tE\#wįįh=s \textbf{ta-suk=ee=rą} ~ ~ ~ ~ ~ ~ ~ ~ ~ ~ ~ ~ ~ ~ ~ i-ra-tax=ak\\
    \textnormal{corn}\#\textnormal{woman}=def \textbf{al-\textnormal{\bfseries child}=dem.dist=top} ~ ~ ~ ~ ~ ~ ~ ~ ~ ~ ~ ~ ~ ~ ~ pv.ins-ins.mth-\textnormal{make.loud.noise}=ds\\
    \glt `Corn Woman was crying \textbf{for her child}' \citep[112]{hollow1973a}

%\newpage    
    
\item\label{NAexamplesD} Indirect object (goal/recipient)

    \glll \textbf{Wáaratookaxi'heena} ``hiré, ~ ~ ~ ~ ~ ~ ~ ~ ~ ~ rapéhini raréehto'sh, mí'ti ~ ~ ~ ~ ~ ~ ~ ~ ~ ~ ~ nata,'' éeheekereroomako'sh\\
    \textbf{waa-ratoo=ka\#xi'h=ee=rą} hire ~ ~ ~ ~ ~ ~ ~ ~ ~ ~ ra-peh=rį ra-rEEh=t=o'sh wį'\#ti ~ ~ ~ ~ ~ ~ ~ ~ ~ ~ ~ rąt=E=$\varnothing$ ee-hEE=krE=oowąk=o'sh\\
    \textbf{nom-\textnormal{\bfseries be.mature}=hab\#\textnormal{\bfseries be.old}=dem.dist=top} \textnormal{now} ~ ~ ~ ~ ~ ~ ~ ~ ~ ~ 2a-\textnormal{announce}=ss 2a-\textnormal{go.there}=ind.m \textnormal{stone}\#\textnormal{house} ~ ~ ~ ~ ~ ~ ~ ~ ~ ~ ~ \textnormal{be.in.middle}=sv=cont pv-\textnormal{say}=3pl=narr=ind.m\\
    \glt `they said \textbf{to the old man}, `now, you should go announce it while in the middle of the village.'{''} \citep[208]{hollow1973b}
    
\item\label{NAexamplesE} Oblique object of a postposition (instrument)

    \glll Rá'puseena \textbf{mí'} \textbf{réxeena} ~ ~ \textbf{ó'hara} pá róotkika'ehe\\
    ra'-pus=ee=rą \textbf{wį'} \textbf{rex}=\textbf{ee}=\textbf{rą} ~ ~ \textbf{o'hrE}=\textbf{$\varnothing$} pa rootki=ka'ehe\\
    ins.heat-\textnormal{be.spotted}=dem.dist=top \textnormal{\bfseries stone} \textnormal{\bfseries glisten}=\textbf{dem.dist}=\textbf{top} ~ ~ \textnormal{\bfseries with}=\textbf{cont} \textnormal{head} \textnormal{hit}=quot\\
    \glt `Charred-in-Streaks hit her head \textbf{with a translucent rock}, it is said' \citep[36]{kennard1936}
    
\item\label{NAexamplesF} Direct reference of a quoted speech

    \glll ``\textbf{Manákiniireena},'' éepeso'sh\\
    \textbf{wrą}\#\textbf{krįį}=\textbf{ee}=\textbf{rą} ee-pe=s=o'sh\\
    \textnormal{\bfseries wood}\#\textnormal{\bfseries be.stacked}=\textbf{dem.dist}=\textbf{top} pv-\textnormal{say}.1sg=def=ind.m\\
    \glt `{``}\textbf{An embankment},'' I said' \citep[37]{kennard1936}

\item\label{NAexamplesG} Adverbial adjunct (temporal)

    \glll Konúuke túk, \textbf{éena} háni ~ ~ ~ ~ ~ ~ ~ ~ ~ ~ tashíxteroomako'sh\\
    ko-rųųkE tu=ak \textbf{ee}=\textbf{rą} hE=rį ~ ~ ~ ~ ~ ~ ~ ~ ~ ~ ta-shi-xtE=oowąk=o'sh\\
    3poss.pers-\textnormal{sister} \textnormal{be.some}=ds \textbf{dem.dist}=\textbf{top} \textnormal{see}=ss ~ ~ ~ ~ ~ ~ ~ ~ ~ ~ al-\textnormal{be.good}-aug=narr=ind.m\\
    \glt `he had a sister, and she \textbf{then} saw him and really liked him' \citep[134]{hollow1973a} 

\end{xlist}

\end{exe}

The distribution of =\textit{na} is such that any nominal element can take it. The exception to this statement seems to be that enclitic postpositions inhibit the presence of topic marking. As we see in (\ref{NAexamplesE}), the nominal construction \textit{mí' réx} `translucent rock' under the scope of the postposition \textit{ó'hara} `with' is able to bear the topic marker =\textit{na}. However, we do not see enclitic postpositions like the locative =\textit{taa} and instrumental =\textit{hąą} occur with a topic marker in the corpus. It is unclear whether this absence of =\textit{na} with enclitic postpositions is due to a proscription against topic marking plus enclitic postpositions or just an incidental lack of such constructions in the corpus. A lack of L1 speakers indicates that this question of the grammaticality of =\textit{na} with remains open.

The topic marker is only used to refer to mark an entire nominal construction as being salient to the discourse. For example, in situations where a noun occurs with some descriptive adjunct, such as in \textit{súknuma'k shínasheena} `nice young man' in (\ref{NAexamplesB}) above, we only observe =\textit{na} at the rightmost edge of the overall nominal construction. We do not observe instances where an element within a nominal construction bears =\textit{na}: i.e, we do not see =\textit{na} on elements like \textit{súknuma'k} in the nominal above.

In situations where =\textit{na} occurs between a noun and some descriptive element, there is always a clear prosodic break that marks each element as being a separate construction rather than being part of a single nominal construction. In the example below, we see two different elements marked with =\textit{na}. A free translation of the sentence in (\ref{NAdoubleNA}) below could be `there was a big cave there', but the presence of two topic markers indicates that the speaker is intending to bring focus on two different aspects of the discourse, rendering a more accurate intended reading as `there was a cave, a big one, around there.'

\begin{exe}
    \item\label{NAdoubleNA} Instances of =\textit{na} on a noun with a following descriptive element

    \glll róo'ų'sh óhopee\textbf{na} xté\textbf{na} ~ ~ ~ ~ ~ ~ ~ ~ ~ ~ ~ nákoomaksįh\\ 
    roo-ų'sh o-hop=ee=\textbf{rą} xtE=\textbf{rą} ~ ~ ~ ~ ~ ~ ~ ~ ~ ~ ~ rąk=oowąk=sįh\\
    dem.mid-\textnormal{be.thus} pv.loc-\textnormal{hole}=dem.dist=\textbf{top} \textnormal{be.big}=\textbf{top} ~ ~ ~ ~ ~ ~ ~ ~ ~ ~ ~ pos.sit=narr=ints\\
    \glt `there was a cave, a big one, around there' \citep[93]{hollow1973a}
\end{exe}

We see =\textit{na} appear on deictic demonstratives, such as the distal demonstrative \textit{ée} in \textit{éena} in (\ref{NAexamplesG}) above. Deictic demonstratives are often used in an adverbial manner, either spatially or temporally, but they clearly pattern with nominals. True adverbials, such as \textit{inák} `again' or \textit{ą́'sh} `soon', never bear topic marking, so we can say that =\textit{na} is not simply any kind of topic marker, but a topic marker that is specific to nominals. Non-nominal elements can certainly be topicalized in Mandan discourse, but these elements will be marked through syntactic or prosodic means, not morphological ones.\footnote{See \sectref{Ch5TopicAndFocus} for further description of topicalization strategies in Mandan.}

\subsubsection{Aforementioned topic marker: =\textit{nu}}\label{SubSubSubNu}

A less common method of indicating a topic that is familiar is the aforementioned topic marker =\textit{nu}. It is likely a reflex of the Proto-Siouan *rų(-sa) `one', where the semantics have become referential instead of quantificational. In Mandan, the aforementioned topic marker differs from the topic marker =\textit{na} in that it appears within a different position within a nominal complex. Namely, =\textit{nu} appears before the definite article =\textit{s}. \citet[42]{mixco1997a} translates this marker as meaning `aforementioned' or `the former.' All instances of =\textit{nu} coincide with definite marking.

The context of the example below in (\ref{NUexampleA}) is that the speaker has already introduced some origin \textit{ó'harani} `from [there],'' so that location is already primed in the discourse. The speaker then clarifies where this location is by stating that it was from \textit{mí'ti xténus} `the big village', where =\textit{nu} is referencing the previously established origin of the shouting in this example. Likewise, in (\ref{NUexampleB}), we see that \textit{Hų́p Wará're} `Fiery Mocassin' is introduced, and then immediately referenced by the use of \textit{ée} `that one' plus a full complex of nominal morphology.

It is worth noting that (\ref{NUexampleB}) is the lone instance of both the aforementioned topic marker =\textit{nu} coinciding with the topic marker =\textit{na}. Clearly, both topic markers serve different purposes in the discourse, but employing both at the same time is possible under precise circumstances. In this case, it seems that the speaker wishes to refer back to a topic that she has just brought up while at the same time emphasizing a shift in aboutness to the same topic.


\begin{exe}
    \item\label{NUexample} Examples of the placement of =\textit{nu} within the nominal complex

    \begin{xlist}

    \item\label{NUexampleA} \glll Ó'harani, mí'ti xté\textbf{nu}s, péhkereroomako'sh.\\
    o'\#hrE=rį wį'\#ti xtE=\textbf{rų}=s peh=krE=oowąk=o'sh\\
    \textnormal{be}\#caus=ss \textnormal{stone}\#\textnormal{dwell} \textnormal{be.big}=\textbf{anf}=def \textnormal{shout}=3pl=narr=ind.m\\
    \glt `From there, \textbf{that big village}, they were shouting.' \citep[107]{hollow1973b}

    \item\label{NUexampleB} \glll Hų́p Wará're, \textbf{éenuseena}, ~ ~ ~ ~ ~ ~ ~ ~ ~ ~ mí'tis íkikisąąpek, ą́ąwe ~ ~ ~ ~ ~ ~ ~ ~ ~ ~ rá'pteroomako'sh\\ 
    hųp wra'=E \textbf{ee=rų=s=ee=rą} ~ ~ ~ ~ ~ ~ ~ ~ ~ ~ wį'\#ti=s i-ki-ki-sąąpE=ak ąąwe ~ ~ ~ ~ ~ ~ ~ ~ ~ ~ ra'-pte=oowąk=o'sh\\
    \textnormal{shoe} \textnormal{fire}=sv \textbf{dem.dist=anf=def=dem.dist=top} ~ ~ ~ ~ ~ ~ ~ ~ ~ ~ \textnormal{stone}\#\textnormal{dwell}=def  pv.ind-vert-mid-\textnormal{be.around}=ds \textnormal{all} ~ ~ ~ ~ ~ ~ ~ ~ ~ ~ ins.heat-\textnormal{burn}=narr=ind.m\\
    \glt Fiery Mocassin, \textbf{the one and the same}, went back around the village and it all burned.' \citep[155]{hollow1973a}

    \end{xlist}

\end{exe}


The aforementioned topic marker often appears immediately after its initial referent in the discourse, though it is able to appear later in the discourse to remind the listener that they have preexisting knowledge of the entity in question. We can see an example of =\textit{nu} in such a situation in (\ref{NUexample3}) below, where it appears at a later point in the discourse to refer back to its original referent.

\newpage

\begin{exe}
    \item\label{NUexample3} Use of =\textit{nu} later in the discourse

    \glll Óo ó'harani mí' pshíireena mákak, ~ ~ ~ ~ ~ ~ ~ ~ ~  warúshani warópxani éewereho're. Káni ~ ~ ~ ~ ~ ~ ~ ~ ~ ~ manáktetaa kihkanákmaherekere'sh. Káni óo ó'harani \textbf{mí'nus} warúsanahini wakíhkanako're.\\ 
    oo o'\#hrE=rį wį' pshii=ee=rą wąk=ak ~ ~ ~ ~ ~ ~ ~ ~ ~ wa-ru-shE=rį wa-ropxE=rį ee-we-reh=o're ka=rį ~ ~ ~ ~ ~ ~ ~ ~ ~ ~ wrąkte=taa kihkrąk\#wą-hrE=krE=o'sh ka=rį oo o'\#hrE=rį \textbf{wį'=rų=s} wa-ru-srąh=rį wa-kihkrąk=o're \\ 
    dem.mid \textnormal{be}\#caus=ss \textnormal{rock} \textnormal{be.flat}=dem.dist=top pos.lie=ds ~ ~ ~ ~ ~ ~ ~ ~ ~ 1a-ins.hand-\textnormal{hold}=ss 1a-\textnormal{enter}=ss pv-1a-\textnormal{want}=ind.f prov=ss ~ ~ ~ ~ ~ ~ ~ ~ ~ ~ \textnormal{altar}=loc \textnormal{be.seated}\#1s-caus=3pl=ind.m prov=ss dem.mid \textnormal{be}\#caus=ss \textbf{\textnormal{\bfseries rock}=anf=def} 1a-\textnormal{leave.behind}=ss 1a-\textnormal{be.seated}=ind.f\\
    \glt `I wanted to take a flat rock that was lying from there and go inside. And after that, they had me sit at the altar. And after that, I left \textbf{that same rock} [at the altar] and sat down.' \citep[318]{hollow1973b}
\end{exe}

In (\ref{NUexample3}) above, the flat rock is introduced into the discourse in the first sentence in the example. Several sentences later, the rock is re-introduced using =\textit{nu}, serving to remind the listener that this is the same rock that was present in the discourse earlier and not some new rock.

When compared to the topic marker =\textit{na}, the aforementioned topic marker =\textit{nu} is starkly less common in the corpus. It is more common in the texts gathered by \citet{kennard1934} that were re-elicited by \citet{hollow1973b} than in the narratives gathered by \citet{hollow1973a} in the later parts of the twentieth century or those narratives gathered in the twenty-first century by \citet{trechter2012b}. The fact that speakers born in the nineteenth century were more likely to use =\textit{nu} may suggest that =\textit{nu} was a more productive formative in the past but has fallen into disuse by later generations of Mandan speakers. 

Another possibility to explain the rarity of =\textit{nu} in the corpus is that there are stylistic reasons to use =\textit{nu} that are not being taken into consideration. It is certainly the case that there are register differences in English for speaking of `the former' or `the latter', so it is possible that the rarity of =\textit{nu} can come from a similar register difference in Mandan. We do not see =\textit{nu} in casual speech as often as we do in traditional narratives, but the casual speech for which we have recordings tend to be relatively short and do not track as many entities within the discourse. Without L1 speakers, however, there is no way to know whether the rarity of =\textit{nu} within the corpus is reflective of everyday speech or whether there are other nuances that are not immediately clear.
